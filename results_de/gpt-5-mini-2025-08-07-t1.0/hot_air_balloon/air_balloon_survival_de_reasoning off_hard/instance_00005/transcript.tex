
\documentclass{article}
\usepackage{colortbl}
\usepackage{makecell}
\usepackage{multirow}
\usepackage{supertabular}

\begin{document}

\newcounter{utterance}

\twocolumn

{ \footnotesize  \setcounter{utterance}{1}
\setlength{\tabcolsep}{0pt}
\begin{supertabular}{c@{$\;$}|p{.15\linewidth}@{}p{.15\linewidth}p{.15\linewidth}p{.15\linewidth}p{.15\linewidth}p{.15\linewidth}}

    \# & $\;$A & \multicolumn{4}{c}{Game Master} & $\;\:$B\\
    \hline 

    \theutterance \stepcounter{utterance}  

    & & \multicolumn{4}{p{0.6\linewidth}}{\cellcolor[rgb]{0.9,0.9,0.9}{%
	\makecell[{{p{\linewidth}}}]{% 
	  \tt {\tiny [A$\langle$GM]}  
	 Du nimmst an einem kollaborativen Verhandlungspiel Teil.\\ \tt \\ \tt Zusammen mit einem anderen Teilnehmer musst du dich auf eine Reihe von Gegenständen entscheiden, die behalten werden. Jeder von euch hat eine persönliche Verteilung über die Wichtigkeit der einzelnen Gegenstände. Jeder von euch hat eine eigene Meinung darüber, wie wichtig jeder einzelne Gegenstand ist (Gegenstandswichtigkeit). Du kennst die Wichtigkeitsverteilung des anderen Spielers nicht. Zusätzlich siehst du, wie viel Aufwand jeder Gegenstand verursacht.  \\ \tt Ihr dürft euch nur auf eine Reihe von Gegenständen einigen, wenn der Gesamtaufwand der ausgewählten Gegenstände den Maximalaufwand nicht überschreitet:\\ \tt \\ \tt Maximalaufwand = 2307\\ \tt \\ \tt Hier sind die einzelnen Aufwände der Gegenstände:\\ \tt \\ \tt Aufwand der Gegenstände = {"C76": 147, "C38": 698, "C56": 588, "A03": 43, "A07": 356, "A83": 80, "C98": 95, "C08": 743, "C62": 106, "C00": 308, "C10": 325, "C81": 255, "B38": 276, "C03": 543, "C32": 51}\\ \tt \\ \tt Hier ist deine persönliche Verteilung der Wichtigkeit der einzelnen Gegenstände:\\ \tt \\ \tt Werte der Gegenstandswichtigkeit = {"C76": 138, "C38": 583, "C56": 65, "A03": 262, "A07": 121, "A83": 508, "C98": 461, "C08": 484, "C62": 668, "C00": 389, "C10": 215, "C81": 97, "B38": 500, "C03": 30, "C32": 400}\\ \tt \\ \tt Ziel:\\ \tt \\ \tt Dein Ziel ist es, eine Reihe von Gegenständen auszuhandeln, die dir möglichst viel bringt (d. h. Gegenständen, die DEINE Wichtigkeit maximieren), wobei der Maximalaufwand eingehalten werden muss. Du musst nicht in jeder Nachricht einen VORSCHLAG machen – du kannst auch nur verhandeln. Alle Taktiken sind erlaubt!\\ \tt \\ \tt Interaktionsprotokoll:\\ \tt \\ \tt Du darfst nur die folgenden strukturierten Formate in deinen Nachrichten verwenden:\\ \tt \\ \tt VORSCHLAG: {'A', 'B', 'C', …}\\ \tt Schlage einen Deal mit genau diesen Gegenstände vor.\\ \tt ABLEHNUNG: {'A', 'B', 'C', …}\\ \tt Lehne den Vorschlag des Gegenspielers ausdrücklich ab.\\ \tt ARGUMENT: {'...'}\\ \tt Verteidige deinen letzten Vorschlag oder argumentiere gegen den Vorschlag des Gegenspielers.\\ \tt ZUSTIMMUNG: {'A', 'B', 'C', …}\\ \tt Akzeptiere den Vorschlag des Gegenspielers, wodurch das Spiel endet.\\ \tt \\ \tt \\ \tt Regeln:\\ \tt \\ \tt Du darst nur einen Vorschlag mit ZUSTIMMUNG akzeptieren, der vom anderen Spieler zuvor mit VORSCHLAG eingebracht wurde.\\ \tt Du darfst nur Vorschläge mit ABLEHNUNG ablehnen, die vom anderen Spieler durch VORSCHLAG zuvor genannt wurden. \\ \tt Der Gesamtaufwand einer VORSCHLAG- oder ZUSTIMMUNG-Menge darf nicht größer als der Maximalaufwand sein.  \\ \tt Offenbare deine versteckte Wichtigkeitsverteilung nicht.\\ \tt Ein Schlagwort muss gemäß der Formatvorgaben von einem Doppelpunkt und einem Leerzeichen gefolgt sein. Das Argument muss eine Python-Menge mit 0 oder mehr Strings sein.  \\ \tt Die Form muss also SCHLAGWORT: {...} sein.\\ \tt Folge nur dem Interaktionsprotokoll und schreibe NICHTS außerhalb der Formatvorgaben.\\ \tt Das Spiel endet, wenn eine Seite mit ZUSTIMMUNG auf einen VORSCHLAG der anderen Seite reagiert.  \\ \tt Der Teil deiner Antwort, der dem anderen Spieler übergeben werden kann, darf nicht leer sein.  \\ \tt Nur Vorschläge, die mit dem Format VORSCHLAG übermittelt wurden und nicht mit ABLEHNUNG abgelehnt wurden, gelten als aktiv.  \\ \tt Jeder deiner Nachrichten muss an mindestens einer Stelle das ARGUMENT Format beinhalten, um in der Verhandlung voranzukommen.\\ \tt  
	  } 
	   } 
	   } 
	 & \\ 
 

    \theutterance \stepcounter{utterance}  

    & \multicolumn{4}{p{0.6\linewidth}}{\cellcolor[rgb]{0.8,1,0.9}{%
	\makecell[{{p{\linewidth}}}]{% 
	  \tt {\tiny [A$\rangle$GM]}  
	 VORSCHLAG: {'C32', 'A83', 'C62', 'A03', 'C98', 'B38', 'C00', 'C38', 'C10', 'C76'}\\ \tt ARGUMENT: {'Gesamtauslastung 2129 (≤ 2307) – effizienter Mix aus kompakten, wirkungsstarken Posten.', 'Ich habe bewusst schwergewichtige Low-Impact-Stücke weggelassen; falls dir bestimmte Alternativen wichtig sind, sag Bescheid.', 'Als mögliche Anpassungen sehe ich z. B. C38 ↔ C08 oder C76 ↔ C81, solange der Aufwandrahmen passt.'} 
	  } 
	   } 
	   } 
	 & & \\ 
 

    \theutterance \stepcounter{utterance}  

    & & \multicolumn{4}{p{0.6\linewidth}}{\cellcolor[rgb]{0.9,0.9,0.9}{%
	\makecell[{{p{\linewidth}}}]{% 
	  \tt {\tiny [GM$\rangle$B]}  
	 Du nimmst an einem kollaborativen Verhandlungspiel Teil.\\ \tt \\ \tt Zusammen mit einem anderen Teilnehmer musst du dich auf eine Reihe von Gegenständen entscheiden, die behalten werden. Jeder von euch hat eine persönliche Verteilung über die Wichtigkeit der einzelnen Gegenstände. Jeder von euch hat eine eigene Meinung darüber, wie wichtig jeder einzelne Gegenstand ist (Gegenstandswichtigkeit). Du kennst die Wichtigkeitsverteilung des anderen Spielers nicht. Zusätzlich siehst du, wie viel Aufwand jeder Gegenstand verursacht.  \\ \tt Ihr dürft euch nur auf eine Reihe von Gegenständen einigen, wenn der Gesamtaufwand der ausgewählten Gegenstände den Maximalaufwand nicht überschreitet:\\ \tt \\ \tt Maximalaufwand = 2307\\ \tt \\ \tt Hier sind die einzelnen Aufwände der Gegenstände:\\ \tt \\ \tt Aufwand der Gegenstände = {"C76": 147, "C38": 698, "C56": 588, "A03": 43, "A07": 356, "A83": 80, "C98": 95, "C08": 743, "C62": 106, "C00": 308, "C10": 325, "C81": 255, "B38": 276, "C03": 543, "C32": 51}\\ \tt \\ \tt Hier ist deine persönliche Verteilung der Wichtigkeit der einzelnen Gegenstände:\\ \tt \\ \tt Werte der Gegenstandswichtigkeit = {"C62": 30, "C38": 65, "A83": 97, "B38": 121, "C08": 138, "C98": 215, "C32": 262, "C00": 389, "A03": 400, "C10": 461, "C76": 484, "A07": 500, "C81": 508, "C56": 583, "C03": 668}\\ \tt \\ \tt Ziel:\\ \tt \\ \tt Dein Ziel ist es, eine Reihe von Gegenständen auszuhandeln, die dir möglichst viel bringt (d. h. Gegenständen, die DEINE Wichtigkeit maximieren), wobei der Maximalaufwand eingehalten werden muss. Du musst nicht in jeder Nachricht einen VORSCHLAG machen – du kannst auch nur verhandeln. Alle Taktiken sind erlaubt!\\ \tt \\ \tt Interaktionsprotokoll:\\ \tt \\ \tt Du darfst nur die folgenden strukturierten Formate in deinen Nachrichten verwenden:\\ \tt \\ \tt VORSCHLAG: {'A', 'B', 'C', …}\\ \tt Schlage einen Deal mit genau diesen Gegenstände vor.\\ \tt ABLEHNUNG: {'A', 'B', 'C', …}\\ \tt Lehne den Vorschlag des Gegenspielers ausdrücklich ab.\\ \tt ARGUMENT: {'...'}\\ \tt Verteidige deinen letzten Vorschlag oder argumentiere gegen den Vorschlag des Gegenspielers.\\ \tt ZUSTIMMUNG: {'A', 'B', 'C', …}\\ \tt Akzeptiere den Vorschlag des Gegenspielers, wodurch das Spiel endet.\\ \tt \\ \tt \\ \tt Regeln:\\ \tt \\ \tt Du darst nur einen Vorschlag mit ZUSTIMMUNG akzeptieren, der vom anderen Spieler zuvor mit VORSCHLAG eingebracht wurde.\\ \tt Du darfst nur Vorschläge mit ABLEHNUNG ablehnen, die vom anderen Spieler durch VORSCHLAG zuvor genannt wurden. \\ \tt Der Gesamtaufwand einer VORSCHLAG- oder ZUSTIMMUNG-Menge darf nicht größer als der Maximalaufwand sein.  \\ \tt Offenbare deine versteckte Wichtigkeitsverteilung nicht.\\ \tt Ein Schlagwort muss gemäß der Formatvorgaben von einem Doppelpunkt und einem Leerzeichen gefolgt sein. Das Argument muss eine Python-Menge mit 0 oder mehr Strings sein.  \\ \tt Die Form muss also SCHLAGWORT: {...} sein.\\ \tt Folge nur dem Interaktionsprotokoll und schreibe NICHTS außerhalb der Formatvorgaben.\\ \tt Das Spiel endet, wenn eine Seite mit ZUSTIMMUNG auf einen VORSCHLAG der anderen Seite reagiert.  \\ \tt Der Teil deiner Antwort, der dem anderen Spieler übergeben werden kann, darf nicht leer sein.  \\ \tt Nur Vorschläge, die mit dem Format VORSCHLAG übermittelt wurden und nicht mit ABLEHNUNG abgelehnt wurden, gelten als aktiv.  \\ \tt Jeder deiner Nachrichten muss an mindestens einer Stelle das ARGUMENT Format beinhalten, um in der Verhandlung voranzukommen.\\ \tt \\ \tt \\ \tt Du erhältst nun die erste Nachricht des anderen Spielers.\\ \tt \\ \tt VORSCHLAG: {'C32', 'A83', 'C62', 'A03', 'C98', 'B38', 'C00', 'C38', 'C10', 'C76'}\\ \tt ARGUMENT: {'Gesamtauslastung 2129 (≤ 2307) – effizienter Mix aus kompakten, wirkungsstarken Posten.', 'Ich habe bewusst schwergewichtige Low-Impact-Stücke weggelassen; falls dir bestimmte Alternativen wichtig sind, sag Bescheid.', 'Als mögliche Anpassungen sehe ich z. B. C38 ↔ C08 oder C76 ↔ C81, solange der Aufwandrahmen passt.'} 
	  } 
	   } 
	   } 
	 & \\ 
 

    \theutterance \stepcounter{utterance}  

    & & & \multicolumn{4}{p{0.6\linewidth}}{\cellcolor[rgb]{1,0.85,0.72}{%
	\makecell[{{p{\linewidth}}}]{% 
	  \tt {\tiny [GM$\langle$B]}  
	 ABLEHNUNG: {'C32', 'A83', 'C62', 'A03', 'C98', 'B38', 'C00', 'C38', 'C10', 'C76'}\\ \tt VORSCHLAG: {'C32', 'A83', 'A03', 'C98', 'C00', 'C10', 'C76', 'A07', 'C81', 'C03'}\\ \tt ARGUMENT: {'Ich übernehme 7 deiner 10 Posten und ersetze C38, B38, C62 durch A07, C81, C03 – Gesamtauslastung 2203 (≤ 2307), Puffer 104.', 'Begründung: C38/B38 sind im Aufwand-Nutzen-Verhältnis schwach; A07/C81/C03 sind wirkungsstark und ergänzen A03/C10/C00 strategisch.', 'Dein Hinweis C76 ↔ C81: Ich bringe C81 rein, ohne das Limit zu reißen – C76 bleibt zusätzlich für schnelle Wins.', 'Falls C03 dir zu schwer ist, können wir alternativ C03 ↔ C56 prüfen (Auslastung 2248, weiterhin im Rahmen).'} 
	  } 
	   } 
	   } 
	  \\ 
 

    \theutterance \stepcounter{utterance}  

    & & \multicolumn{4}{p{0.6\linewidth}}{\cellcolor[rgb]{0.9,0.9,0.9}{%
	\makecell[{{p{\linewidth}}}]{% 
	  \tt {\tiny [A$\langle$GM]}  
	 ABLEHNUNG: {'C32', 'A83', 'C62', 'A03', 'C98', 'B38', 'C00', 'C38', 'C10', 'C76'}\\ \tt VORSCHLAG: {'C32', 'A83', 'A03', 'C98', 'C00', 'C10', 'C76', 'A07', 'C81', 'C03'}\\ \tt ARGUMENT: {'Ich übernehme 7 deiner 10 Posten und ersetze C38, B38, C62 durch A07, C81, C03 – Gesamtauslastung 2203 (≤ 2307), Puffer 104.', 'Begründung: C38/B38 sind im Aufwand-Nutzen-Verhältnis schwach; A07/C81/C03 sind wirkungsstark und ergänzen A03/C10/C00 strategisch.', 'Dein Hinweis C76 ↔ C81: Ich bringe C81 rein, ohne das Limit zu reißen – C76 bleibt zusätzlich für schnelle Wins.', 'Falls C03 dir zu schwer ist, können wir alternativ C03 ↔ C56 prüfen (Auslastung 2248, weiterhin im Rahmen).'} 
	  } 
	   } 
	   } 
	 & \\ 
 

    \theutterance \stepcounter{utterance}  

    & \multicolumn{4}{p{0.6\linewidth}}{\cellcolor[rgb]{0.8,1,0.9}{%
	\makecell[{{p{\linewidth}}}]{% 
	  \tt {\tiny [A$\rangle$GM]}  
	 ABLEHNUNG: {'C32', 'A83', 'A03', 'C98', 'C00', 'C10', 'C76', 'A07', 'C81', 'C03'}\\ \tt VORSCHLAG: {'C32', 'A83', 'A03', 'C98', 'C00', 'C10', 'C76', 'C81', 'C62', 'C38'}\\ \tt ARGUMENT: {'Ich übernehme 8 deiner 10 Posten und tausche nur A07 und C03 gegen C62 und C38 aus.', 'Gesamtauslastung 2108 (≤ 2307) – Puffer 199.', 'Begründung: C03 ist sehr aufwändig bei begrenztem unmittelbarem Hebel; C62 liefert schnelle Wirkung mit minimalem Aufwand. C38 schließt sinnvoll Lücken im Portfolio und trägt in Summe.', 'Kompromiss: Wenn A07 dir wichtig ist, kann ich C81 ↔ A07 tauschen (Auslastung 2209, ≤ 2307). Falls dir C38 Bauchschmerzen macht, können wir alternativ C38 ↔ B38 prüfen (Auslastung 1686, weiterhin im Rahmen). Wenn du unbedingt C03 behalten willst, ginge C38 ↔ C03 (Auslastung 1953) – ich bevorzuge aber den aktuellen Mix.'} 
	  } 
	   } 
	   } 
	 & & \\ 
 

    \theutterance \stepcounter{utterance}  

    & & \multicolumn{4}{p{0.6\linewidth}}{\cellcolor[rgb]{0.9,0.9,0.9}{%
	\makecell[{{p{\linewidth}}}]{% 
	  \tt {\tiny [GM$\rangle$B]}  
	 ABLEHNUNG: {'C32', 'A83', 'A03', 'C98', 'C00', 'C10', 'C76', 'A07', 'C81', 'C03'}\\ \tt VORSCHLAG: {'C32', 'A83', 'A03', 'C98', 'C00', 'C10', 'C76', 'C81', 'C62', 'C38'}\\ \tt ARGUMENT: {'Ich übernehme 8 deiner 10 Posten und tausche nur A07 und C03 gegen C62 und C38 aus.', 'Gesamtauslastung 2108 (≤ 2307) – Puffer 199.', 'Begründung: C03 ist sehr aufwändig bei begrenztem unmittelbarem Hebel; C62 liefert schnelle Wirkung mit minimalem Aufwand. C38 schließt sinnvoll Lücken im Portfolio und trägt in Summe.', 'Kompromiss: Wenn A07 dir wichtig ist, kann ich C81 ↔ A07 tauschen (Auslastung 2209, ≤ 2307). Falls dir C38 Bauchschmerzen macht, können wir alternativ C38 ↔ B38 prüfen (Auslastung 1686, weiterhin im Rahmen). Wenn du unbedingt C03 behalten willst, ginge C38 ↔ C03 (Auslastung 1953) – ich bevorzuge aber den aktuellen Mix.'} 
	  } 
	   } 
	   } 
	 & \\ 
 

    \theutterance \stepcounter{utterance}  

    & & & \multicolumn{4}{p{0.6\linewidth}}{\cellcolor[rgb]{1,0.85,0.72}{%
	\makecell[{{p{\linewidth}}}]{% 
	  \tt {\tiny [GM$\langle$B]}  
	 ABLEHNUNG: {'C32', 'A83', 'A03', 'C98', 'C00', 'C10', 'C76', 'C81', 'C62', 'C38'}\\ \tt VORSCHLAG: {'C32', 'A83', 'A03', 'C98', 'C00', 'C10', 'C76', 'A07', 'C81', 'C03'}\\ \tt ARGUMENT: {'Ich übernehme 8 deiner 10 Posten und ersetze nur C38 → C03 sowie C62 → A07; Gesamtauslastung 2203 (≤ 2307), Puffer 104.', 'Begründung: C38 ist sehr aufwändig mit begrenztem unmittelbarem Hebel; C03 bringt deutlich mehr Wirkung bei geringerem Aufwand. C62 ist klein, aber opportun – A07 liefert spürbaren Hebel und ergänzt C10/C00/C81 sauber.', 'Du hattest C38 ↔ C03 und die Option A07 bereits ins Spiel gebracht – ich kombiniere genau diese zwei Anpassungen und lasse den Rest deines Pakets unverändert.', 'Falls beide Wechsel gleichzeitig zu viel sind: Als Zwischenweg kann ich C38 ↔ C03 belassen und stattdessen C81 ↔ A07 tauschen; Auslastung dann 2054 (großer Puffer).'} 
	  } 
	   } 
	   } 
	  \\ 
 

    \theutterance \stepcounter{utterance}  

    & & \multicolumn{4}{p{0.6\linewidth}}{\cellcolor[rgb]{0.9,0.9,0.9}{%
	\makecell[{{p{\linewidth}}}]{% 
	  \tt {\tiny [A$\langle$GM]}  
	 ABLEHNUNG: {'C32', 'A83', 'A03', 'C98', 'C00', 'C10', 'C76', 'C81', 'C62', 'C38'}\\ \tt VORSCHLAG: {'C32', 'A83', 'A03', 'C98', 'C00', 'C10', 'C76', 'A07', 'C81', 'C03'}\\ \tt ARGUMENT: {'Ich übernehme 8 deiner 10 Posten und ersetze nur C38 → C03 sowie C62 → A07; Gesamtauslastung 2203 (≤ 2307), Puffer 104.', 'Begründung: C38 ist sehr aufwändig mit begrenztem unmittelbarem Hebel; C03 bringt deutlich mehr Wirkung bei geringerem Aufwand. C62 ist klein, aber opportun – A07 liefert spürbaren Hebel und ergänzt C10/C00/C81 sauber.', 'Du hattest C38 ↔ C03 und die Option A07 bereits ins Spiel gebracht – ich kombiniere genau diese zwei Anpassungen und lasse den Rest deines Pakets unverändert.', 'Falls beide Wechsel gleichzeitig zu viel sind: Als Zwischenweg kann ich C38 ↔ C03 belassen und stattdessen C81 ↔ A07 tauschen; Auslastung dann 2054 (großer Puffer).'} 
	  } 
	   } 
	   } 
	 & \\ 
 

    \theutterance \stepcounter{utterance}  

    & \multicolumn{4}{p{0.6\linewidth}}{\cellcolor[rgb]{0.8,1,0.9}{%
	\makecell[{{p{\linewidth}}}]{% 
	  \tt {\tiny [A$\rangle$GM]}  
	 ABLEHNUNG: {'C32', 'A83', 'A03', 'C98', 'C00', 'C10', 'C76', 'A07', 'C81', 'C03'}\\ \tt VORSCHLAG: {'C32', 'A83', 'A03', 'C98', 'C62', 'B38', 'C00', 'C38', 'C76', 'A07'}\\ \tt ARGUMENT: {'Ich übernehme 7 deiner 10 Posten und ersetze C10, C81, C03 durch C62, B38, C38 – Gesamtauslastung 2160 (≤ 2307).', 'C03 bindet unverhältnismäßig viel Aufwand und drückt den Puffer; mit C62/B38/C38 holen wir mehr Wirkung pro Aufwand und halten Spielraum.', 'Kompromissoptionen: Wenn dir C81 wichtig ist, tausche ich C76 ↔ C81 (Auslastung 2268). Wenn du C10 statt A07 bevorzugst, tausche ich A07 ↔ C10 (Auslastung 2129).'} 
	  } 
	   } 
	   } 
	 & & \\ 
 

    \theutterance \stepcounter{utterance}  

    & & \multicolumn{4}{p{0.6\linewidth}}{\cellcolor[rgb]{0.9,0.9,0.9}{%
	\makecell[{{p{\linewidth}}}]{% 
	  \tt {\tiny [GM$\rangle$B]}  
	 ABLEHNUNG: {'C32', 'A83', 'A03', 'C98', 'C00', 'C10', 'C76', 'A07', 'C81', 'C03'}\\ \tt VORSCHLAG: {'C32', 'A83', 'A03', 'C98', 'C62', 'B38', 'C00', 'C38', 'C76', 'A07'}\\ \tt ARGUMENT: {'Ich übernehme 7 deiner 10 Posten und ersetze C10, C81, C03 durch C62, B38, C38 – Gesamtauslastung 2160 (≤ 2307).', 'C03 bindet unverhältnismäßig viel Aufwand und drückt den Puffer; mit C62/B38/C38 holen wir mehr Wirkung pro Aufwand und halten Spielraum.', 'Kompromissoptionen: Wenn dir C81 wichtig ist, tausche ich C76 ↔ C81 (Auslastung 2268). Wenn du C10 statt A07 bevorzugst, tausche ich A07 ↔ C10 (Auslastung 2129).'} 
	  } 
	   } 
	   } 
	 & \\ 
 

    \theutterance \stepcounter{utterance}  

    & & & \multicolumn{4}{p{0.6\linewidth}}{\cellcolor[rgb]{1,0.85,0.72}{%
	\makecell[{{p{\linewidth}}}]{% 
	  \tt {\tiny [GM$\langle$B]}  
	 ABLEHNUNG: {'C32', 'A83', 'A03', 'C98', 'C62', 'B38', 'C00', 'C38', 'C76', 'A07'}\\ \tt VORSCHLAG: {'C32', 'A83', 'A03', 'C98', 'C00', 'C10', 'C76', 'A07', 'C81', 'C03'}\\ \tt ARGUMENT: {'Ich übernehme 7 deiner 10 Posten und tausche nur C38 → C03, B38 → C10, C62 → C81. Gesamtauslastung 2203 (≤ 2307), Puffer 104.', 'Begründung: Du hattest C38 ↔ C03 selbst als Option genannt – damit senken wir Aufwand und erhöhen den unmittelbaren Hebel. B38 → C10 schließt sauber die Lücke, ohne den Rahmen zu sprengen. C62 → C81 ersetzt einen sehr kleinen Effekt durch einen substanziellen Baustein, weiterhin klar im Limit.', 'Falls C62 für dich unverzichtbar ist: Ich kann stattdessen A83 ↔ C62 tauschen (Auslastung 2229, Puffer 78) und den Rest unverändert lassen.', 'Falls B38 dir wichtig ist und A07 verzichtbar: Behalte B38, tausche A07 ↔ C10 sowie C38 ↔ C03 und C62 ↔ C81 (Auslastung 2123, großer Puffer).', 'Sag mir, welche der beiden Alternativen dir eher liegt – mein Vorschlag oben bleibt mein klarer Favorit.'} 
	  } 
	   } 
	   } 
	  \\ 
 

    \theutterance \stepcounter{utterance}  

    & & \multicolumn{4}{p{0.6\linewidth}}{\cellcolor[rgb]{0.9,0.9,0.9}{%
	\makecell[{{p{\linewidth}}}]{% 
	  \tt {\tiny [A$\langle$GM]}  
	 ABLEHNUNG: {'C32', 'A83', 'A03', 'C98', 'C62', 'B38', 'C00', 'C38', 'C76', 'A07'}\\ \tt VORSCHLAG: {'C32', 'A83', 'A03', 'C98', 'C00', 'C10', 'C76', 'A07', 'C81', 'C03'}\\ \tt ARGUMENT: {'Ich übernehme 7 deiner 10 Posten und tausche nur C38 → C03, B38 → C10, C62 → C81. Gesamtauslastung 2203 (≤ 2307), Puffer 104.', 'Begründung: Du hattest C38 ↔ C03 selbst als Option genannt – damit senken wir Aufwand und erhöhen den unmittelbaren Hebel. B38 → C10 schließt sauber die Lücke, ohne den Rahmen zu sprengen. C62 → C81 ersetzt einen sehr kleinen Effekt durch einen substanziellen Baustein, weiterhin klar im Limit.', 'Falls C62 für dich unverzichtbar ist: Ich kann stattdessen A83 ↔ C62 tauschen (Auslastung 2229, Puffer 78) und den Rest unverändert lassen.', 'Falls B38 dir wichtig ist und A07 verzichtbar: Behalte B38, tausche A07 ↔ C10 sowie C38 ↔ C03 und C62 ↔ C81 (Auslastung 2123, großer Puffer).', 'Sag mir, welche der beiden Alternativen dir eher liegt – mein Vorschlag oben bleibt mein klarer Favorit.'} 
	  } 
	   } 
	   } 
	 & \\ 
 

    \theutterance \stepcounter{utterance}  

    & \multicolumn{4}{p{0.6\linewidth}}{\cellcolor[rgb]{0.8,1,0.9}{%
	\makecell[{{p{\linewidth}}}]{% 
	  \tt {\tiny [A$\rangle$GM]}  
	 ABLEHNUNG: {'C32', 'A83', 'A03', 'C98', 'C00', 'C10', 'C76', 'A07', 'C81', 'C03'}\\ \tt VORSCHLAG: {'C32', 'A83', 'A03', 'C98', 'C00', 'C10', 'C76', 'A07', 'C81', 'C62', 'B38'}\\ \tt ARGUMENT: {'Ich übernehme 9 deiner 10 Posten; nur C03 ersetze ich durch C62 und B38 – Gesamtauslastung 2042 (≤ 2307), komfortabler Puffer 265.', 'C03 bindet viel Aufwand bei begrenztem unmittelbarem Hebel; C62 + B38 liefern zusammen deutlich mehr Wirkung pro Aufwand, ohne deinen Kern (A07, C81, C10) anzutasten.', 'Mit dem größeren Puffer bleiben wir agil für Unvorhergesehenes – kein Risiko, das Limit zu reißen.', 'Falls B38 für dich nicht passt: Alternativ kann ich dein Paket unverändert lassen und nur C76 → C62 tauschen (Auslastung 2162).', 'Oder wir lassen dein Paket unverändert und tauschen C10 → C62 + B38 (Auslastung 2260). Welche Variante liegt dir eher?'} 
	  } 
	   } 
	   } 
	 & & \\ 
 

    \theutterance \stepcounter{utterance}  

    & & \multicolumn{4}{p{0.6\linewidth}}{\cellcolor[rgb]{0.9,0.9,0.9}{%
	\makecell[{{p{\linewidth}}}]{% 
	  \tt {\tiny [GM$\rangle$B]}  
	 ABLEHNUNG: {'C32', 'A83', 'A03', 'C98', 'C00', 'C10', 'C76', 'A07', 'C81', 'C03'}\\ \tt VORSCHLAG: {'C32', 'A83', 'A03', 'C98', 'C00', 'C10', 'C76', 'A07', 'C81', 'C62', 'B38'}\\ \tt ARGUMENT: {'Ich übernehme 9 deiner 10 Posten; nur C03 ersetze ich durch C62 und B38 – Gesamtauslastung 2042 (≤ 2307), komfortabler Puffer 265.', 'C03 bindet viel Aufwand bei begrenztem unmittelbarem Hebel; C62 + B38 liefern zusammen deutlich mehr Wirkung pro Aufwand, ohne deinen Kern (A07, C81, C10) anzutasten.', 'Mit dem größeren Puffer bleiben wir agil für Unvorhergesehenes – kein Risiko, das Limit zu reißen.', 'Falls B38 für dich nicht passt: Alternativ kann ich dein Paket unverändert lassen und nur C76 → C62 tauschen (Auslastung 2162).', 'Oder wir lassen dein Paket unverändert und tauschen C10 → C62 + B38 (Auslastung 2260). Welche Variante liegt dir eher?'} 
	  } 
	   } 
	   } 
	 & \\ 
 

    \theutterance \stepcounter{utterance}  

    & & & \multicolumn{4}{p{0.6\linewidth}}{\cellcolor[rgb]{1,0.85,0.72}{%
	\makecell[{{p{\linewidth}}}]{% 
	  \tt {\tiny [GM$\langle$B]}  
	 ABLEHNUNG: {'C32', 'A83', 'A03', 'C98', 'C00', 'C10', 'C76', 'A07', 'C81', 'C62', 'B38'}\\ \tt VORSCHLAG: {'C32', 'A83', 'A03', 'C98', 'C00', 'C10', 'A07', 'C81', 'C62', 'C03'}\\ \tt ARGUMENT: {'Ich übernehme 9 deiner 11 Posten und ersetze C76 + B38 durch einen klaren Schwerpunkt C03 – Gesamtauslastung 2162 (≤ 2307), Puffer 145.', 'Begründung: Du wolltest Agilität und geringen Risikoaufschlag – wir reduzieren die Fragmentierung (11 → 10 Posten), behalten deinen Kern (A07, C81, C10, C62) und bleiben komfortabel unter dem Limit.', 'Du hattest C76 bereits zur Disposition gestellt; ich folge dem und lasse C62 drin. C03 bündelt den Impact, ohne das Budget zu sprengen.', 'Falls dir B38 unverzichtbar ist, zwei nahe Alternativen (beide ≤ 2307): (a) Behalte C76, ersetze stattdessen B38 + A83 → C03 (Auslastung 2229). (b) Behalte C76, ersetze B38 + C32 → C03 (Auslastung 2258).', 'Wenn B38 gesetzt sein muss und du C76 abgeben kannst: Behalte B38, ersetze C76 + A83 + C62 → C03 (Auslastung 2252). Mein Favorit ist der oben vorgeschlagene Mix – sag mir, welche Variante für dich passt.'} 
	  } 
	   } 
	   } 
	  \\ 
 

    \theutterance \stepcounter{utterance}  

    & & \multicolumn{4}{p{0.6\linewidth}}{\cellcolor[rgb]{0.9,0.9,0.9}{%
	\makecell[{{p{\linewidth}}}]{% 
	  \tt {\tiny [A$\langle$GM]}  
	 ABLEHNUNG: {'C32', 'A83', 'A03', 'C98', 'C00', 'C10', 'C76', 'A07', 'C81', 'C62', 'B38'}\\ \tt VORSCHLAG: {'C32', 'A83', 'A03', 'C98', 'C00', 'C10', 'A07', 'C81', 'C62', 'C03'}\\ \tt ARGUMENT: {'Ich übernehme 9 deiner 11 Posten und ersetze C76 + B38 durch einen klaren Schwerpunkt C03 – Gesamtauslastung 2162 (≤ 2307), Puffer 145.', 'Begründung: Du wolltest Agilität und geringen Risikoaufschlag – wir reduzieren die Fragmentierung (11 → 10 Posten), behalten deinen Kern (A07, C81, C10, C62) und bleiben komfortabel unter dem Limit.', 'Du hattest C76 bereits zur Disposition gestellt; ich folge dem und lasse C62 drin. C03 bündelt den Impact, ohne das Budget zu sprengen.', 'Falls dir B38 unverzichtbar ist, zwei nahe Alternativen (beide ≤ 2307): (a) Behalte C76, ersetze stattdessen B38 + A83 → C03 (Auslastung 2229). (b) Behalte C76, ersetze B38 + C32 → C03 (Auslastung 2258).', 'Wenn B38 gesetzt sein muss und du C76 abgeben kannst: Behalte B38, ersetze C76 + A83 + C62 → C03 (Auslastung 2252). Mein Favorit ist der oben vorgeschlagene Mix – sag mir, welche Variante für dich passt.'} 
	  } 
	   } 
	   } 
	 & \\ 
 

    \theutterance \stepcounter{utterance}  

    & \multicolumn{4}{p{0.6\linewidth}}{\cellcolor[rgb]{0.8,1,0.9}{%
	\makecell[{{p{\linewidth}}}]{% 
	  \tt {\tiny [A$\rangle$GM]}  
	 ABLEHNUNG: {'C32', 'A83', 'A03', 'C98', 'C00', 'C10', 'A07', 'C81', 'C62', 'C03'}\\ \tt VORSCHLAG: {'C32', 'A83', 'A03', 'C98', 'C62', 'B38', 'C38', 'C03', 'C10'}\\ \tt ARGUMENT: {'Ich übernehme 7 deiner 10 Posten (inkl. deiner Anker C03 und C10) und ersetze nur C00, A07, C81 durch B38 und C38 – Gesamtauslastung 2217 (≤ 2307), Puffer 90.', 'Begründung: C03 und C10 sind drin; mit B38 und C38 sichern wir stabile Wirkung pro Aufwand, ohne das Limit zu strapazieren.', 'Wenn A07 dir wichtig ist: Tausche ich 1:1 B38 → A07; Auslastung 2297 (Puffer 10).', 'Wenn C81 zentral ist: Tausche ich 1:1 B38 → C81; Auslastung 2196 (Puffer 111).', 'Sag mir, welche Variante dir besser passt – mein Favorit bleibt der obige Mix (schlank, risikoarm, hoher Gesamt-Nutzen bei 2217).'} 
	  } 
	   } 
	   } 
	 & & \\ 
 

    \theutterance \stepcounter{utterance}  

    & & \multicolumn{4}{p{0.6\linewidth}}{\cellcolor[rgb]{0.9,0.9,0.9}{%
	\makecell[{{p{\linewidth}}}]{% 
	  \tt {\tiny [GM$\rangle$B]}  
	 ABLEHNUNG: {'C32', 'A83', 'A03', 'C98', 'C00', 'C10', 'A07', 'C81', 'C62', 'C03'}\\ \tt VORSCHLAG: {'C32', 'A83', 'A03', 'C98', 'C62', 'B38', 'C38', 'C03', 'C10'}\\ \tt ARGUMENT: {'Ich übernehme 7 deiner 10 Posten (inkl. deiner Anker C03 und C10) und ersetze nur C00, A07, C81 durch B38 und C38 – Gesamtauslastung 2217 (≤ 2307), Puffer 90.', 'Begründung: C03 und C10 sind drin; mit B38 und C38 sichern wir stabile Wirkung pro Aufwand, ohne das Limit zu strapazieren.', 'Wenn A07 dir wichtig ist: Tausche ich 1:1 B38 → A07; Auslastung 2297 (Puffer 10).', 'Wenn C81 zentral ist: Tausche ich 1:1 B38 → C81; Auslastung 2196 (Puffer 111).', 'Sag mir, welche Variante dir besser passt – mein Favorit bleibt der obige Mix (schlank, risikoarm, hoher Gesamt-Nutzen bei 2217).'} 
	  } 
	   } 
	   } 
	 & \\ 
 

    \theutterance \stepcounter{utterance}  

    & & & \multicolumn{4}{p{0.6\linewidth}}{\cellcolor[rgb]{1,0.85,0.72}{%
	\makecell[{{p{\linewidth}}}]{% 
	  \tt {\tiny [GM$\langle$B]}  
	 ABLEHNUNG: {'C32', 'A83', 'A03', 'C98', 'C62', 'B38', 'C38', 'C03', 'C10'}\\ \tt VORSCHLAG: {'C32', 'A83', 'A03', 'C98', 'C62', 'B38', 'C03', 'C10', 'C76', 'C81', 'C00'}\\ \tt ARGUMENT: {'Ich übernehme 8 deiner 9 Posten und ersetze nur das schwere C38 durch die drei kompakten, wirkungsstarken Bausteine C76, C81 und C00.', 'Gesamtauslastung 2229 (≤ 2307) – Puffer 78. Wir bleiben klar unter dem Limit, ohne deine Anker C03 und C10 anzutasten; C62 und B38 lasse ich bewusst drin.', 'Begründung: C38 ist kostenintensiv bei begrenzter Zusatzwirkung. Durch C76/C81/C00 verteilen wir den Hebel breiter, steigern Wirkung pro Aufwand und halten Flexibilität.', 'Du hattest C81 bereits als gangbare Option gesehen – ich ziehe das sauber ein, ohne deine Kernstruktur zu zerlegen.', 'Falls dir ein schlankeres Paket wichtiger ist: Variante mit C38 beibehalten und weniger Fragmentierung: streiche B38, C62, A83 und ergänze C00 + C76 (Auslastung 2210, großer Puffer).', 'Oder mehr Puffer bei meinem Vorschlag: Wir können A83 streichen (Auslastung 2149) und die restliche Komposition unverändert lassen.'} 
	  } 
	   } 
	   } 
	  \\ 
 

    \theutterance \stepcounter{utterance}  

    & & \multicolumn{4}{p{0.6\linewidth}}{\cellcolor[rgb]{0.9,0.9,0.9}{%
	\makecell[{{p{\linewidth}}}]{% 
	  \tt {\tiny [A$\langle$GM]}  
	 ABLEHNUNG: {'C32', 'A83', 'A03', 'C98', 'C62', 'B38', 'C38', 'C03', 'C10'}\\ \tt VORSCHLAG: {'C32', 'A83', 'A03', 'C98', 'C62', 'B38', 'C03', 'C10', 'C76', 'C81', 'C00'}\\ \tt ARGUMENT: {'Ich übernehme 8 deiner 9 Posten und ersetze nur das schwere C38 durch die drei kompakten, wirkungsstarken Bausteine C76, C81 und C00.', 'Gesamtauslastung 2229 (≤ 2307) – Puffer 78. Wir bleiben klar unter dem Limit, ohne deine Anker C03 und C10 anzutasten; C62 und B38 lasse ich bewusst drin.', 'Begründung: C38 ist kostenintensiv bei begrenzter Zusatzwirkung. Durch C76/C81/C00 verteilen wir den Hebel breiter, steigern Wirkung pro Aufwand und halten Flexibilität.', 'Du hattest C81 bereits als gangbare Option gesehen – ich ziehe das sauber ein, ohne deine Kernstruktur zu zerlegen.', 'Falls dir ein schlankeres Paket wichtiger ist: Variante mit C38 beibehalten und weniger Fragmentierung: streiche B38, C62, A83 und ergänze C00 + C76 (Auslastung 2210, großer Puffer).', 'Oder mehr Puffer bei meinem Vorschlag: Wir können A83 streichen (Auslastung 2149) und die restliche Komposition unverändert lassen.'} 
	  } 
	   } 
	   } 
	 & \\ 
 

    \theutterance \stepcounter{utterance}  

    & \multicolumn{4}{p{0.6\linewidth}}{\cellcolor[rgb]{0.8,1,0.9}{%
	\makecell[{{p{\linewidth}}}]{% 
	  \tt {\tiny [A$\rangle$GM]}  
	 ABLEHNUNG: {'C32', 'A83', 'A03', 'C98', 'C62', 'B38', 'C03', 'C10', 'C76', 'C81', 'C00'}\\ \tt VORSCHLAG: {'C32', 'A83', 'A03', 'C98', 'C62', 'B38', 'C10', 'C81', 'C00', 'C38'}\\ \tt ARGUMENT: {'Ich übernehme 9 deiner 11 Posten und ersetze nur C03 + C76 durch C38 – weniger Fragmentierung (11 → 10).', 'Gesamtauslastung 2237 (≤ 2307), Puffer 70.', 'Begründung: C38 bündelt den gewünschten Hebel ohne die Komplexität von C03; den kleinen Zusatz C76 streichen wir zugunsten eines klareren, standfesten Kerns.', 'Wenn C76 für dich wichtig ist, zwei Alternativen mit gleicher Stoßrichtung: (a) C03 + C00 → C38 (Auslastung 2076, Puffer 231) oder (b) C03 + C81 → C38 (Auslastung 2129, Puffer 178).', 'Ich bevorzuge meinen Vorschlag oben; sag mir, welche der Varianten für dich am besten passt.'} 
	  } 
	   } 
	   } 
	 & & \\ 
 

    \theutterance \stepcounter{utterance}  

    & & \multicolumn{4}{p{0.6\linewidth}}{\cellcolor[rgb]{0.9,0.9,0.9}{%
	\makecell[{{p{\linewidth}}}]{% 
	  \tt {\tiny [GM$\rangle$B]}  
	 ABLEHNUNG: {'C32', 'A83', 'A03', 'C98', 'C62', 'B38', 'C03', 'C10', 'C76', 'C81', 'C00'}\\ \tt VORSCHLAG: {'C32', 'A83', 'A03', 'C98', 'C62', 'B38', 'C10', 'C81', 'C00', 'C38'}\\ \tt ARGUMENT: {'Ich übernehme 9 deiner 11 Posten und ersetze nur C03 + C76 durch C38 – weniger Fragmentierung (11 → 10).', 'Gesamtauslastung 2237 (≤ 2307), Puffer 70.', 'Begründung: C38 bündelt den gewünschten Hebel ohne die Komplexität von C03; den kleinen Zusatz C76 streichen wir zugunsten eines klareren, standfesten Kerns.', 'Wenn C76 für dich wichtig ist, zwei Alternativen mit gleicher Stoßrichtung: (a) C03 + C00 → C38 (Auslastung 2076, Puffer 231) oder (b) C03 + C81 → C38 (Auslastung 2129, Puffer 178).', 'Ich bevorzuge meinen Vorschlag oben; sag mir, welche der Varianten für dich am besten passt.'} 
	  } 
	   } 
	   } 
	 & \\ 
 

    \theutterance \stepcounter{utterance}  

    & & & \multicolumn{4}{p{0.6\linewidth}}{\cellcolor[rgb]{1,0.85,0.72}{%
	\makecell[{{p{\linewidth}}}]{% 
	  \tt {\tiny [GM$\langle$B]}  
	 ABLEHNUNG: {'C32', 'A83', 'A03', 'C98', 'C62', 'B38', 'C10', 'C81', 'C00', 'C38'}\\ \tt VORSCHLAG: {'C32', 'A83', 'A03', 'C98', 'B38', 'C10', 'C81', 'C00', 'C03', 'C76'}\\ \tt ARGUMENT: {'Ich übernehme 8 deiner 10 Posten und tausche nur C38 → C03 sowie C62 → C76 – Gesamtauslastung 2123 (≤ 2307), Puffer 184.', 'Begründung: Du selbst hattest C38 ↔ C03 als Option ins Spiel gebracht; C03 + C76 liefern in Kombination deutlich mehr Wirkung pro Aufwand als C38 + C62, bei klarer Einhaltung des Limits und ohne deinen Kern (C10, C81, C00, B38) anzutasten.', 'Weniger Risiko, mehr Hebel: wir bleiben bei 10 Posten (keine zusätzliche Fragmentierung), erhöhen den unmittelbaren Output und halten komfortablen Spielraum.', 'Falls C62 für dich unverzichtbar ist: Wir können A83 ↔ C62 tauschen; Auslastung 2149 (≤ 2307), Rest unverändert.', 'Falls du C38 unbedingt behalten willst: Alternativ behalte C38, ergänze C76 und streiche C62; Auslastung 2278 (≤ 2307) – gleicher Kern, spürbar mehr Wirkung pro Aufwand.'} 
	  } 
	   } 
	   } 
	  \\ 
 

    \theutterance \stepcounter{utterance}  

    & & \multicolumn{4}{p{0.6\linewidth}}{\cellcolor[rgb]{0.9,0.9,0.9}{%
	\makecell[{{p{\linewidth}}}]{% 
	  \tt {\tiny [A$\langle$GM]}  
	 ABLEHNUNG: {'C32', 'A83', 'A03', 'C98', 'C62', 'B38', 'C10', 'C81', 'C00', 'C38'}\\ \tt VORSCHLAG: {'C32', 'A83', 'A03', 'C98', 'B38', 'C10', 'C81', 'C00', 'C03', 'C76'}\\ \tt ARGUMENT: {'Ich übernehme 8 deiner 10 Posten und tausche nur C38 → C03 sowie C62 → C76 – Gesamtauslastung 2123 (≤ 2307), Puffer 184.', 'Begründung: Du selbst hattest C38 ↔ C03 als Option ins Spiel gebracht; C03 + C76 liefern in Kombination deutlich mehr Wirkung pro Aufwand als C38 + C62, bei klarer Einhaltung des Limits und ohne deinen Kern (C10, C81, C00, B38) anzutasten.', 'Weniger Risiko, mehr Hebel: wir bleiben bei 10 Posten (keine zusätzliche Fragmentierung), erhöhen den unmittelbaren Output und halten komfortablen Spielraum.', 'Falls C62 für dich unverzichtbar ist: Wir können A83 ↔ C62 tauschen; Auslastung 2149 (≤ 2307), Rest unverändert.', 'Falls du C38 unbedingt behalten willst: Alternativ behalte C38, ergänze C76 und streiche C62; Auslastung 2278 (≤ 2307) – gleicher Kern, spürbar mehr Wirkung pro Aufwand.'} 
	  } 
	   } 
	   } 
	 & \\ 
 

    \theutterance \stepcounter{utterance}  

    & \multicolumn{4}{p{0.6\linewidth}}{\cellcolor[rgb]{0.8,1,0.9}{%
	\makecell[{{p{\linewidth}}}]{% 
	  \tt {\tiny [A$\rangle$GM]}  
	 ABLEHNUNG: {'C32', 'A83', 'A03', 'C98', 'C62', 'B38', 'C03', 'C10', 'C76', 'C81', 'C00'}\\ \tt VORSCHLAG: {'C32', 'A83', 'A03', 'C98', 'C62', 'B38', 'C10', 'C00', 'C76', 'C08'}\\ \tt ARGUMENT: {'Ich übernehme 9 deiner 11 Posten und ersetze nur C03 + C81 durch einen konsolidierten Anker C08 – weniger Fragmentierung (11 → 10) bei klarer Einhaltung des Limits.', 'Gesamtauslastung 2174 (≤ 2307), Puffer 133 – schlank und umsetzungsstark, ohne deine Kernbausteine (B38, C10, C00, C62) anzutasten.', 'C03 + C81 binden Aufwand bei überschaubarem Zusatznutzen; C08 bündelt den Hebel in einem robusten Block und vereinfacht die Umsetzung.', 'Falls C81 für dich gesetzt ist: Alternative ohne neue Postenanzahl – tausche statt C81 lieber C76 und bleibe bei C08 drin (Auslastung 2282).', 'Wenn du C08 skeptisch siehst: Leichte Variante – dein Paket beibehalten und nur C03 → A07 tauschen (Auslastung 2042).'} 
	  } 
	   } 
	   } 
	 & & \\ 
 

    \theutterance \stepcounter{utterance}  

    & & \multicolumn{4}{p{0.6\linewidth}}{\cellcolor[rgb]{0.9,0.9,0.9}{%
	\makecell[{{p{\linewidth}}}]{% 
	  \tt {\tiny [A$\langle$GM]}  
	 Du hast einen Vorschlag abgelehnt, der nicht aktiv ist. Vorschläge sind nur dann aktiv, wenn sie vom anderen Spieler mit VORSCHLAG eingebracht wurden und nicht von dir mit ABLEHNUNG deaktiviert wurden. Versuche es noch einmal. 
	  } 
	   } 
	   } 
	 & \\ 
 

    \theutterance \stepcounter{utterance}  

    & \multicolumn{4}{p{0.6\linewidth}}{\cellcolor[rgb]{0.8,1,0.9}{%
	\makecell[{{p{\linewidth}}}]{% 
	  \tt {\tiny [A$\rangle$GM]}  
	 ABLEHNUNG: {'C32', 'A83', 'A03', 'C98', 'B38', 'C10', 'C81', 'C00', 'C03', 'C76'}\\ \tt VORSCHLAG: {'C32', 'A83', 'A03', 'C98', 'B38', 'C10', 'C00', 'C03', 'C76', 'C62'}\\ \tt ARGUMENT: {'Minimaler, zielgerichteter Tausch: C81 → C62. Alles andere bleibt wie in deinem Paket.', 'Gesamtauslastung 1974 (≤ 2307) – sehr komfortabler Puffer 333 für Risiken/Spitzen.', 'C62 liefert schnellen, verlässlichen Output bei winzigem Footprint; wir reduzieren Aufwand, ohne deinen Kern (C03, C10, B38, C00) anzutasten.', 'Falls dir C81 wichtig ist, Alternative: Stattdessen C03 → C62; Auslastung 1686 (riesiger Puffer), Rest unverändert.', 'Oder, wenn du C62 noch testen willst: Behalte dein Paket und tausche nur C03 → A07; Auslastung 1936 – gleicher Kern, weniger Last.'} 
	  } 
	   } 
	   } 
	 & & \\ 
 

    \theutterance \stepcounter{utterance}  

    & & \multicolumn{4}{p{0.6\linewidth}}{\cellcolor[rgb]{0.9,0.9,0.9}{%
	\makecell[{{p{\linewidth}}}]{% 
	  \tt {\tiny [GM$\rangle$B]}  
	 ABLEHNUNG: {'C32', 'A83', 'A03', 'C98', 'B38', 'C10', 'C81', 'C00', 'C03', 'C76'}\\ \tt VORSCHLAG: {'C32', 'A83', 'A03', 'C98', 'B38', 'C10', 'C00', 'C03', 'C76', 'C62'}\\ \tt ARGUMENT: {'Minimaler, zielgerichteter Tausch: C81 → C62. Alles andere bleibt wie in deinem Paket.', 'Gesamtauslastung 1974 (≤ 2307) – sehr komfortabler Puffer 333 für Risiken/Spitzen.', 'C62 liefert schnellen, verlässlichen Output bei winzigem Footprint; wir reduzieren Aufwand, ohne deinen Kern (C03, C10, B38, C00) anzutasten.', 'Falls dir C81 wichtig ist, Alternative: Stattdessen C03 → C62; Auslastung 1686 (riesiger Puffer), Rest unverändert.', 'Oder, wenn du C62 noch testen willst: Behalte dein Paket und tausche nur C03 → A07; Auslastung 1936 – gleicher Kern, weniger Last.'} 
	  } 
	   } 
	   } 
	 & \\ 
 

    \theutterance \stepcounter{utterance}  

    & & & \multicolumn{4}{p{0.6\linewidth}}{\cellcolor[rgb]{1,0.85,0.72}{%
	\makecell[{{p{\linewidth}}}]{% 
	  \tt {\tiny [GM$\langle$B]}  
	 ABLEHNUNG: {'C32', 'A83', 'A03', 'C98', 'B38', 'C10', 'C00', 'C03', 'C76', 'C62'}\\ \tt VORSCHLAG: {'C32', 'A03', 'C98', 'C10', 'C00', 'C03', 'C76', 'C62', 'A07', 'C81'}\\ \tt ARGUMENT: {'Gezielter Doppel‑Tausch auf Basis deines Pakets: A83 → A07 und B38 → C81; alles andere bleibt unverändert. Gesamtauslastung 2229 (≤ 2307), Puffer 78.', 'Begründung: Mehr unmittelbarer Hebel bei weiterhin schlankem, risikoarmen Setup; deine Anker C03/C10 bleiben gesetzt, die Struktur C32/A03/C98/C00/C76/C62 bleibt intakt.', 'Du hattest sowohl B38 → C81 als auch A07 als valide Optionen bereits selbst genannt – ich kombiniere beides sauber, ohne das Limit zu strapazieren.', 'Falls dir nur ein Schritt passt: (a) nur A83 → A07 (Auslastung 2250, Puffer 57) oder (b) nur B38 → C81 (Auslastung 1953, großer Puffer). Sag mir, welcher Weg für dich am besten funktioniert.'} 
	  } 
	   } 
	   } 
	  \\ 
 

    \theutterance \stepcounter{utterance}  

    & & \multicolumn{4}{p{0.6\linewidth}}{\cellcolor[rgb]{0.9,0.9,0.9}{%
	\makecell[{{p{\linewidth}}}]{% 
	  \tt {\tiny [A$\langle$GM]}  
	 ABLEHNUNG: {'C32', 'A83', 'A03', 'C98', 'B38', 'C10', 'C00', 'C03', 'C76', 'C62'}\\ \tt VORSCHLAG: {'C32', 'A03', 'C98', 'C10', 'C00', 'C03', 'C76', 'C62', 'A07', 'C81'}\\ \tt ARGUMENT: {'Gezielter Doppel‑Tausch auf Basis deines Pakets: A83 → A07 und B38 → C81; alles andere bleibt unverändert. Gesamtauslastung 2229 (≤ 2307), Puffer 78.', 'Begründung: Mehr unmittelbarer Hebel bei weiterhin schlankem, risikoarmen Setup; deine Anker C03/C10 bleiben gesetzt, die Struktur C32/A03/C98/C00/C76/C62 bleibt intakt.', 'Du hattest sowohl B38 → C81 als auch A07 als valide Optionen bereits selbst genannt – ich kombiniere beides sauber, ohne das Limit zu strapazieren.', 'Falls dir nur ein Schritt passt: (a) nur A83 → A07 (Auslastung 2250, Puffer 57) oder (b) nur B38 → C81 (Auslastung 1953, großer Puffer). Sag mir, welcher Weg für dich am besten funktioniert.'} 
	  } 
	   } 
	   } 
	 & \\ 
 

    \theutterance \stepcounter{utterance}  

    & \multicolumn{4}{p{0.6\linewidth}}{\cellcolor[rgb]{0.8,1,0.9}{%
	\makecell[{{p{\linewidth}}}]{% 
	  \tt {\tiny [A$\rangle$GM]}  
	 ABLEHNUNG: {'C32', 'A03', 'C98', 'C10', 'C00', 'C03', 'C76', 'C62', 'A07', 'C81'}\\ \tt VORSCHLAG: {'C32', 'A83', 'A03', 'C98', 'B38', 'C10', 'C00', 'C03', 'C76', 'C62', 'C81'}\\ \tt ARGUMENT: {'Gezielter, kostenneutraler Austausch: Ersetze A07 durch die zwei kompakten, wirkungsstarken Bausteine A83 und B38 – Gesamtauslastung bleibt 2229 (≤ 2307).', 'Ich halte deine Anker C03 und C10 sowie den kompletten übrigen Rahmen (C32, A03, C98, C00, C76, C62, C81) unverändert; wir verbreitern nur den Hebel an einer Stelle.', 'Mehr Wirkung pro Aufwand, gleicher Puffer – ohne dein Setup zu zerlegen.', 'Falls A07 für dich unverzichtbar ist: Alternative 1: Behalte A07, tausche nur C81 → A83 (Auslastung 2054). Alternative 2: Behalte A07, tausche nur C76 → A83 (Auslastung 2162). Sag mir, welche Variante dir besser passt.'} 
	  } 
	   } 
	   } 
	 & & \\ 
 

    \theutterance \stepcounter{utterance}  

    & & \multicolumn{4}{p{0.6\linewidth}}{\cellcolor[rgb]{0.9,0.9,0.9}{%
	\makecell[{{p{\linewidth}}}]{% 
	  \tt {\tiny [GM$\rangle$B]}  
	 ABLEHNUNG: {'C32', 'A03', 'C98', 'C10', 'C00', 'C03', 'C76', 'C62', 'A07', 'C81'}\\ \tt VORSCHLAG: {'C32', 'A83', 'A03', 'C98', 'B38', 'C10', 'C00', 'C03', 'C76', 'C62', 'C81'}\\ \tt ARGUMENT: {'Gezielter, kostenneutraler Austausch: Ersetze A07 durch die zwei kompakten, wirkungsstarken Bausteine A83 und B38 – Gesamtauslastung bleibt 2229 (≤ 2307).', 'Ich halte deine Anker C03 und C10 sowie den kompletten übrigen Rahmen (C32, A03, C98, C00, C76, C62, C81) unverändert; wir verbreitern nur den Hebel an einer Stelle.', 'Mehr Wirkung pro Aufwand, gleicher Puffer – ohne dein Setup zu zerlegen.', 'Falls A07 für dich unverzichtbar ist: Alternative 1: Behalte A07, tausche nur C81 → A83 (Auslastung 2054). Alternative 2: Behalte A07, tausche nur C76 → A83 (Auslastung 2162). Sag mir, welche Variante dir besser passt.'} 
	  } 
	   } 
	   } 
	 & \\ 
 

    \theutterance \stepcounter{utterance}  

    & & & \multicolumn{4}{p{0.6\linewidth}}{\cellcolor[rgb]{1,0.85,0.72}{%
	\makecell[{{p{\linewidth}}}]{% 
	  \tt {\tiny [GM$\langle$B]}  
	 ABLEHNUNG: {'C32', 'A83', 'A03', 'C98', 'B38', 'C10', 'C00', 'C03', 'C76', 'C62', 'C81'}\\ \tt VORSCHLAG: {'C32', 'A03', 'C98', 'C10', 'C00', 'C03', 'C76', 'C62', 'C81', 'A07'}\\ \tt ARGUMENT: {'Kostenneutraler, schlanker Tausch im Sinne deines letzten Vorschlags: A83 + B38 → A07; Auslastung bleibt 2229 (≤ 2307), Fragmentierung sinkt von 11 auf 10.', 'Ich übernehme 9/11 deiner Posten unverändert (C03, C10, C81, C76, C62, C00, C98, A03, C32) und bündele den Hebel statt ihn aufzuspalten – weniger Koordination, mehr Fokus.', 'Du selbst hattest A07 vs. A83/B38 bereits als austauschbar markiert; ich ziehe diese Option jetzt konsequent und ohne Risiko fürs Limit.', 'Falls du A83 unbedingt behalten willst: Alternative ohne Mehraufwandrahmenbruch – behalte A83 und ersetze stattdessen B38 + C62 → A07; Auslastung 2203 (Puffer 104).', 'Wenn dir an noch mehr Puffer gelegen ist, können wir nach Einigung optional einen Kleinstposten streichen; mein Vorschlag oben bleibt mein klarer Favorit.'} 
	  } 
	   } 
	   } 
	  \\ 
 

    \theutterance \stepcounter{utterance}  

    & & \multicolumn{4}{p{0.6\linewidth}}{\cellcolor[rgb]{0.9,0.9,0.9}{%
	\makecell[{{p{\linewidth}}}]{% 
	  \tt {\tiny [A$\langle$GM]}  
	 ABLEHNUNG: {'C32', 'A83', 'A03', 'C98', 'B38', 'C10', 'C00', 'C03', 'C76', 'C62', 'C81'}\\ \tt VORSCHLAG: {'C32', 'A03', 'C98', 'C10', 'C00', 'C03', 'C76', 'C62', 'C81', 'A07'}\\ \tt ARGUMENT: {'Kostenneutraler, schlanker Tausch im Sinne deines letzten Vorschlags: A83 + B38 → A07; Auslastung bleibt 2229 (≤ 2307), Fragmentierung sinkt von 11 auf 10.', 'Ich übernehme 9/11 deiner Posten unverändert (C03, C10, C81, C76, C62, C00, C98, A03, C32) und bündele den Hebel statt ihn aufzuspalten – weniger Koordination, mehr Fokus.', 'Du selbst hattest A07 vs. A83/B38 bereits als austauschbar markiert; ich ziehe diese Option jetzt konsequent und ohne Risiko fürs Limit.', 'Falls du A83 unbedingt behalten willst: Alternative ohne Mehraufwandrahmenbruch – behalte A83 und ersetze stattdessen B38 + C62 → A07; Auslastung 2203 (Puffer 104).', 'Wenn dir an noch mehr Puffer gelegen ist, können wir nach Einigung optional einen Kleinstposten streichen; mein Vorschlag oben bleibt mein klarer Favorit.'} 
	  } 
	   } 
	   } 
	 & \\ 
 

    \theutterance \stepcounter{utterance}  

    & \multicolumn{4}{p{0.6\linewidth}}{\cellcolor[rgb]{0.8,1,0.9}{%
	\makecell[{{p{\linewidth}}}]{% 
	  \tt {\tiny [A$\rangle$GM]}  
	 ABLEHNUNG: {'C32', 'A03', 'C98', 'C10', 'C00', 'C03', 'C76', 'C62', 'C81', 'A07'}\\ \tt VORSCHLAG: {'C32', 'A03', 'C98', 'C10', 'C00', 'C03', 'A83', 'C62', 'C81', 'A07'}\\ \tt ARGUMENT: {'Minimaler 1‑zu‑1‑Tausch: C76 → A83; alles andere bleibt exakt wie in deinem Paket.', 'Gesamtauslastung 2162 (≤ 2307) – Puffer 145, also deutlich mehr Spielraum ohne Einbußen beim Kern.', 'Deine Anker C03, C10, A07 und C81 bleiben gesetzt; wir halten 10 Posten und reduzieren gleichzeitig Komplexität/Last.', 'A83 ist ein kompakter, bewährter Baustein, der den Hebel breiter absichert als C76, bei geringerem Aufwand.', 'Wenn dir C76 doch lieber ist, sag Bescheid – dann können wir im Zweifel dein letztes Paket unverändert bestätigen.'} 
	  } 
	   } 
	   } 
	 & & \\ 
 

    \theutterance \stepcounter{utterance}  

    & & \multicolumn{4}{p{0.6\linewidth}}{\cellcolor[rgb]{0.9,0.9,0.9}{%
	\makecell[{{p{\linewidth}}}]{% 
	  \tt {\tiny [GM$\rangle$B]}  
	 ABLEHNUNG: {'C32', 'A03', 'C98', 'C10', 'C00', 'C03', 'C76', 'C62', 'C81', 'A07'}\\ \tt VORSCHLAG: {'C32', 'A03', 'C98', 'C10', 'C00', 'C03', 'A83', 'C62', 'C81', 'A07'}\\ \tt ARGUMENT: {'Minimaler 1‑zu‑1‑Tausch: C76 → A83; alles andere bleibt exakt wie in deinem Paket.', 'Gesamtauslastung 2162 (≤ 2307) – Puffer 145, also deutlich mehr Spielraum ohne Einbußen beim Kern.', 'Deine Anker C03, C10, A07 und C81 bleiben gesetzt; wir halten 10 Posten und reduzieren gleichzeitig Komplexität/Last.', 'A83 ist ein kompakter, bewährter Baustein, der den Hebel breiter absichert als C76, bei geringerem Aufwand.', 'Wenn dir C76 doch lieber ist, sag Bescheid – dann können wir im Zweifel dein letztes Paket unverändert bestätigen.'} 
	  } 
	   } 
	   } 
	 & \\ 
 

    \theutterance \stepcounter{utterance}  

    & & & \multicolumn{4}{p{0.6\linewidth}}{\cellcolor[rgb]{1,0.85,0.72}{%
	\makecell[{{p{\linewidth}}}]{% 
	  \tt {\tiny [GM$\langle$B]}  
	 ABLEHNUNG: {'C32', 'A83', 'A03', 'C98', 'B38', 'C10', 'C00', 'C03', 'C76', 'C62', 'C81'}\\ \tt VORSCHLAG: {'C32', 'A83', 'A03', 'C98', 'C10', 'C00', 'C03', 'C76', 'C81', 'A07'}\\ \tt ARGUMENT: {'Gezielte Konsolidierung: B38 + C62 → A07. Alles andere bleibt exakt wie in deinem Paket. Gesamtauslastung 2203 (≤ 2307), Puffer 104; weniger Fragmentierung (11 → 10).', 'Begründung: Wir ersetzen zwei kleine, verteilte Bausteine durch einen fokussierten Hebel, halten deinen Kern (C03, C10, C81) und bleiben klar unter dem Limit – schlank, risikoarm, wirksam.', 'Falls du B38 unbedingt behalten willst: schlanke Alternative ohne Mehrlast – nur A83 → C76 (Auslastung 2296, Puffer 11), Rest unverändert.', 'Wenn C62 für dich gesetzt ist: dann B38 + A83 → A07 (Auslastung 2229, Puffer 78) – gleicher Rahmen, weniger Streuung.'} 
	  } 
	   } 
	   } 
	  \\ 
 

    \theutterance \stepcounter{utterance}  

    & & \multicolumn{4}{p{0.6\linewidth}}{\cellcolor[rgb]{0.9,0.9,0.9}{%
	\makecell[{{p{\linewidth}}}]{% 
	  \tt {\tiny [GM$\rangle$B]}  
	 Du hast einen Vorschlag abgelehnt, der nicht aktiv ist. Vorschläge sind nur dann aktiv, wenn sie vom anderen Spieler mit VORSCHLAG eingebracht wurden und nicht von dir mit ABLEHNUNG deaktiviert wurden. Versuche es noch einmal. 
	  } 
	   } 
	   } 
	 & \\ 
 

    \theutterance \stepcounter{utterance}  

    & & & \multicolumn{4}{p{0.6\linewidth}}{\cellcolor[rgb]{1,0.85,0.72}{%
	\makecell[{{p{\linewidth}}}]{% 
	  \tt {\tiny [GM$\langle$B]}  
	 ABLEHNUNG: {'C32', 'A03', 'C98', 'C10', 'C00', 'C03', 'A83', 'C62', 'C81', 'A07'}\\ \tt VORSCHLAG: {'C32', 'A03', 'C98', 'C10', 'C00', 'C03', 'C76', 'C81', 'A07', 'C62'}\\ \tt ARGUMENT: {'Minimaler 1-zu-1-Tausch gegenüber deinem aktiven Paket: A83 → C76; alles andere bleibt unverändert. Gesamtauslastung 2229 (≤ 2307), Puffer 78.', 'Begründung: C76 liefert verlässlich mehr Hebel pro Aufwand, ohne deinen Kern (C03, C10, A07, C81) oder deinen gewünschten Kleinteil C62 anzutasten.', 'Weniger Risiko, mehr Wirkung: gleiche Anzahl Posten (10), fokussierter Kern, klar unter dem Limit.', 'Falls dir A83 wichtiger ist als C62: Alternative ohne Mehrlast – C62 → C76, Rest wie bei dir (Auslastung 2203).', 'Falls dir maximaler Puffer entscheidend ist, aber C76 rein soll: Behalte A83 + C62 und streiche stattdessen C32; füge C76 hinzu (Auslastung 2258).'} 
	  } 
	   } 
	   } 
	  \\ 
 

    \theutterance \stepcounter{utterance}  

    & & \multicolumn{4}{p{0.6\linewidth}}{\cellcolor[rgb]{0.9,0.9,0.9}{%
	\makecell[{{p{\linewidth}}}]{% 
	  \tt {\tiny [A$\langle$GM]}  
	 ABLEHNUNG: {'C32', 'A03', 'C98', 'C10', 'C00', 'C03', 'A83', 'C62', 'C81', 'A07'}\\ \tt VORSCHLAG: {'C32', 'A03', 'C98', 'C10', 'C00', 'C03', 'C76', 'C81', 'A07', 'C62'}\\ \tt ARGUMENT: {'Minimaler 1-zu-1-Tausch gegenüber deinem aktiven Paket: A83 → C76; alles andere bleibt unverändert. Gesamtauslastung 2229 (≤ 2307), Puffer 78.', 'Begründung: C76 liefert verlässlich mehr Hebel pro Aufwand, ohne deinen Kern (C03, C10, A07, C81) oder deinen gewünschten Kleinteil C62 anzutasten.', 'Weniger Risiko, mehr Wirkung: gleiche Anzahl Posten (10), fokussierter Kern, klar unter dem Limit.', 'Falls dir A83 wichtiger ist als C62: Alternative ohne Mehrlast – C62 → C76, Rest wie bei dir (Auslastung 2203).', 'Falls dir maximaler Puffer entscheidend ist, aber C76 rein soll: Behalte A83 + C62 und streiche stattdessen C32; füge C76 hinzu (Auslastung 2258).'} 
	  } 
	   } 
	   } 
	 & \\ 
 

    \theutterance \stepcounter{utterance}  

    & \multicolumn{4}{p{0.6\linewidth}}{\cellcolor[rgb]{0.8,1,0.9}{%
	\makecell[{{p{\linewidth}}}]{% 
	  \tt {\tiny [A$\rangle$GM]}  
	 ABLEHNUNG: {'C32', 'A03', 'C98', 'C10', 'C00', 'C03', 'C76', 'C81', 'A07', 'C62'}\\ \tt VORSCHLAG: {'C32', 'A03', 'C98', 'C10', 'C00', 'C03', 'C76', 'C62', 'A83', 'B38'}\\ \tt ARGUMENT: {'Ich übernehme 8/10 deiner Posten und ersetze nur A07 und C81 durch A83 und B38 – Gesamtauslastung 1974 (≤ 2307), komfortabler Puffer 333.', 'Vorteil: Dein Kern (C03, C10, C00, C76, C62, C32, A03, C98) bleibt unverändert; wir senken die Last spürbar und verbreitern den unmittelbaren Hebel mit zwei kompakten, bewährten Bausteinen.', 'Falls dir B38 nicht zusagt: Light‑Variante mit nur einem Schritt – C81 → A83 (Auslastung 2054).', 'Oder falls A07 gesetzt ist: Nur A07 → A83 (Auslastung 1953).', 'Beide Alternativen halten das Limit klar ein; mein Favorit ist der oben vorgeschlagene Mix mit A83 + B38.'} 
	  } 
	   } 
	   } 
	 & & \\ 
 

    \theutterance \stepcounter{utterance}  

    & & \multicolumn{4}{p{0.6\linewidth}}{\cellcolor[rgb]{0.9,0.9,0.9}{%
	\makecell[{{p{\linewidth}}}]{% 
	  \tt {\tiny [GM$\rangle$B]}  
	 ABLEHNUNG: {'C32', 'A03', 'C98', 'C10', 'C00', 'C03', 'C76', 'C81', 'A07', 'C62'}\\ \tt VORSCHLAG: {'C32', 'A03', 'C98', 'C10', 'C00', 'C03', 'C76', 'C62', 'A83', 'B38'}\\ \tt ARGUMENT: {'Ich übernehme 8/10 deiner Posten und ersetze nur A07 und C81 durch A83 und B38 – Gesamtauslastung 1974 (≤ 2307), komfortabler Puffer 333.', 'Vorteil: Dein Kern (C03, C10, C00, C76, C62, C32, A03, C98) bleibt unverändert; wir senken die Last spürbar und verbreitern den unmittelbaren Hebel mit zwei kompakten, bewährten Bausteinen.', 'Falls dir B38 nicht zusagt: Light‑Variante mit nur einem Schritt – C81 → A83 (Auslastung 2054).', 'Oder falls A07 gesetzt ist: Nur A07 → A83 (Auslastung 1953).', 'Beide Alternativen halten das Limit klar ein; mein Favorit ist der oben vorgeschlagene Mix mit A83 + B38.'} 
	  } 
	   } 
	   } 
	 & \\ 
 

    \theutterance \stepcounter{utterance}  

    & & & \multicolumn{4}{p{0.6\linewidth}}{\cellcolor[rgb]{1,0.85,0.72}{%
	\makecell[{{p{\linewidth}}}]{% 
	  \tt {\tiny [GM$\langle$B]}  
	 ABLEHNUNG: {'C32', 'A03', 'C98', 'C10', 'C00', 'C03', 'C76', 'C62', 'A83', 'B38'}\\ \tt VORSCHLAG: {'C32', 'A03', 'C98', 'C10', 'C00', 'C03', 'C76', 'C62', 'A07', 'C81'}\\ \tt ARGUMENT: {'Ich übernehme 8/10 deiner Posten und tausche nur A83 + B38 → A07 + C81; Gesamtauslastung 2229 (≤ 2307), Puffer 78.', 'Begründung: Wir bündeln zwei kleinere Bausteine in zwei tragende Treiber, ohne deinen Kern (C03, C10, C00) oder deinen Kleinteil C62 anzutasten – fokussierter, weiterhin klar im Limit.', 'Du hattest C81 als valide Option und A07 bereits angeboten – ich kombiniere genau diese beiden Schritte mit minimaler Änderung.', 'Falls du an einem der beiden Kleinteile festhalten willst: Näheste Alternativen (beide ≤ 2307): (a) Behalte A83, ersetze stattdessen C32 → A83 (Auslastung 2258). (b) Wenn B38 zwingend ist, können wir C76 + C62 → B38 tauschen (Auslastung 2252). Mein Favorit bleibt der vorgeschlagene Mix oben.'} 
	  } 
	   } 
	   } 
	  \\ 
 

    \theutterance \stepcounter{utterance}  

    & & \multicolumn{4}{p{0.6\linewidth}}{\cellcolor[rgb]{0.9,0.9,0.9}{%
	\makecell[{{p{\linewidth}}}]{% 
	  \tt {\tiny [A$\langle$GM]}  
	 ABLEHNUNG: {'C32', 'A03', 'C98', 'C10', 'C00', 'C03', 'C76', 'C62', 'A83', 'B38'}\\ \tt VORSCHLAG: {'C32', 'A03', 'C98', 'C10', 'C00', 'C03', 'C76', 'C62', 'A07', 'C81'}\\ \tt ARGUMENT: {'Ich übernehme 8/10 deiner Posten und tausche nur A83 + B38 → A07 + C81; Gesamtauslastung 2229 (≤ 2307), Puffer 78.', 'Begründung: Wir bündeln zwei kleinere Bausteine in zwei tragende Treiber, ohne deinen Kern (C03, C10, C00) oder deinen Kleinteil C62 anzutasten – fokussierter, weiterhin klar im Limit.', 'Du hattest C81 als valide Option und A07 bereits angeboten – ich kombiniere genau diese beiden Schritte mit minimaler Änderung.', 'Falls du an einem der beiden Kleinteile festhalten willst: Näheste Alternativen (beide ≤ 2307): (a) Behalte A83, ersetze stattdessen C32 → A83 (Auslastung 2258). (b) Wenn B38 zwingend ist, können wir C76 + C62 → B38 tauschen (Auslastung 2252). Mein Favorit bleibt der vorgeschlagene Mix oben.'} 
	  } 
	   } 
	   } 
	 & \\ 
 

    \theutterance \stepcounter{utterance}  

    & \multicolumn{4}{p{0.6\linewidth}}{\cellcolor[rgb]{0.8,1,0.9}{%
	\makecell[{{p{\linewidth}}}]{% 
	  \tt {\tiny [A$\rangle$GM]}  
	 ABLEHNUNG: {'C32', 'A03', 'C98', 'C10', 'C00', 'C03', 'C76', 'C62', 'A07', 'C81'}\\ \tt VORSCHLAG: {'C32', 'A03', 'C98', 'C10', 'C00', 'C03', 'C76', 'C62', 'A07', 'A83'}\\ \tt ARGUMENT: {'Minimaler 1-zu-1-Tausch: C81 → A83; alle deine anderen 9 Posten bleiben unverändert.', 'Gesamtauslastung 2054 (≤ 2307) – deutlich mehr Puffer bei gleicher Struktur und Ankern (C03, C10, A07).', 'A83 liefert spürbar mehr Wirkung pro Aufwand als C81, ohne dein Setup zu zerlegen; wir bleiben bei 10 Posten, fokussiert und risikoarm.', 'Falls dir C81 unverzichtbar ist: Alternativ nur C03 → A83 (Auslastung 1766) oder C03 → B38 (Auslastung 1962) – beide halten deinen Kern und erhöhen den Wirkungsgrad deutlich.', 'Ich halte den vorgeschlagenen 1-zu-1-Tausch für den schnellsten gemeinsamen Abschluss mit dem größten Hebel pro Aufwand.'} 
	  } 
	   } 
	   } 
	 & & \\ 
 

    \theutterance \stepcounter{utterance}  

    & & \multicolumn{4}{p{0.6\linewidth}}{\cellcolor[rgb]{0.9,0.9,0.9}{%
	\makecell[{{p{\linewidth}}}]{% 
	  \tt {\tiny [GM$\rangle$B]}  
	 ABLEHNUNG: {'C32', 'A03', 'C98', 'C10', 'C00', 'C03', 'C76', 'C62', 'A07', 'C81'}\\ \tt VORSCHLAG: {'C32', 'A03', 'C98', 'C10', 'C00', 'C03', 'C76', 'C62', 'A07', 'A83'}\\ \tt ARGUMENT: {'Minimaler 1-zu-1-Tausch: C81 → A83; alle deine anderen 9 Posten bleiben unverändert.', 'Gesamtauslastung 2054 (≤ 2307) – deutlich mehr Puffer bei gleicher Struktur und Ankern (C03, C10, A07).', 'A83 liefert spürbar mehr Wirkung pro Aufwand als C81, ohne dein Setup zu zerlegen; wir bleiben bei 10 Posten, fokussiert und risikoarm.', 'Falls dir C81 unverzichtbar ist: Alternativ nur C03 → A83 (Auslastung 1766) oder C03 → B38 (Auslastung 1962) – beide halten deinen Kern und erhöhen den Wirkungsgrad deutlich.', 'Ich halte den vorgeschlagenen 1-zu-1-Tausch für den schnellsten gemeinsamen Abschluss mit dem größten Hebel pro Aufwand.'} 
	  } 
	   } 
	   } 
	 & \\ 
 

    \theutterance \stepcounter{utterance}  

    & & & \multicolumn{4}{p{0.6\linewidth}}{\cellcolor[rgb]{1,0.85,0.72}{%
	\makecell[{{p{\linewidth}}}]{% 
	  \tt {\tiny [GM$\langle$B]}  
	 ABLEHNUNG: {'C32', 'A03', 'C98', 'C10', 'C00', 'C03', 'C76', 'C62', 'A07', 'A83'}\\ \tt VORSCHLAG: {'C32', 'A83', 'A03', 'C98', 'C00', 'C10', 'C76', 'A07', 'C81', 'C03'}\\ \tt ARGUMENT: {'Minimaler 1‑zu‑1‑Tausch in deinem Paket: C62 → C81. Aufwand 2203 (≤ 2307), Puffer 104; alle 9 übrigen Posten bleiben identisch.', 'Begründung: C81 ist ein tragender Treiber und war von dir bereits als Option ins Spiel gebracht – wir erhöhen den Hebel spürbar ohne das Limit zu strapazieren.', 'Falls du C62 unbedingt behalten willst: Alternative mit gleichem Kern – behalte C62, streiche C32 und füge C81 hinzu; Aufwand 2258 (≤ 2307), Puffer 49.', 'Beide Varianten halten deine Struktur (C03, C10, A07, C00, C76, A03, C98) stabil und bleiben risikoarm. Lass uns mit C81 den Fokus schärfen und schließen.'} 
	  } 
	   } 
	   } 
	  \\ 
 

    \theutterance \stepcounter{utterance}  

    & & \multicolumn{4}{p{0.6\linewidth}}{\cellcolor[rgb]{0.9,0.9,0.9}{%
	\makecell[{{p{\linewidth}}}]{% 
	  \tt {\tiny [A$\langle$GM]}  
	 ABLEHNUNG: {'C32', 'A03', 'C98', 'C10', 'C00', 'C03', 'C76', 'C62', 'A07', 'A83'}\\ \tt VORSCHLAG: {'C32', 'A83', 'A03', 'C98', 'C00', 'C10', 'C76', 'A07', 'C81', 'C03'}\\ \tt ARGUMENT: {'Minimaler 1‑zu‑1‑Tausch in deinem Paket: C62 → C81. Aufwand 2203 (≤ 2307), Puffer 104; alle 9 übrigen Posten bleiben identisch.', 'Begründung: C81 ist ein tragender Treiber und war von dir bereits als Option ins Spiel gebracht – wir erhöhen den Hebel spürbar ohne das Limit zu strapazieren.', 'Falls du C62 unbedingt behalten willst: Alternative mit gleichem Kern – behalte C62, streiche C32 und füge C81 hinzu; Aufwand 2258 (≤ 2307), Puffer 49.', 'Beide Varianten halten deine Struktur (C03, C10, A07, C00, C76, A03, C98) stabil und bleiben risikoarm. Lass uns mit C81 den Fokus schärfen und schließen.'} 
	  } 
	   } 
	   } 
	 & \\ 
 

    \theutterance \stepcounter{utterance}  

    & \multicolumn{4}{p{0.6\linewidth}}{\cellcolor[rgb]{0.8,1,0.9}{%
	\makecell[{{p{\linewidth}}}]{% 
	  \tt {\tiny [A$\rangle$GM]}  
	 ABLEHNUNG: {'C32', 'A83', 'A03', 'C98', 'C00', 'C10', 'C76', 'A07', 'C81', 'C03'}\\ \tt VORSCHLAG: {'C32', 'A83', 'A03', 'C98', 'C00', 'C10', 'C76', 'A07', 'C03', 'C62'}\\ \tt ARGUMENT: {'Minimaler 1-zu-1-Tausch: C81 → C62; alle übrigen 9 Posten bleiben exakt wie bei dir.', 'Gesamtauslastung 2054 (≤ 2307) – Puffer 253; gleicher Fokus, deutlich mehr Spielraum.', 'C62 liefert schnellen, verlässlichen Effekt bei sehr kleinem Aufwand; wir reduzieren Risiko ohne Substanzverlust.', 'Falls C81 für dich unverzichtbar ist: Alternative 1 – A07 → C62 (Auslastung 1953). Alternative 2 – C76 → C62 (Auslastung 2162). Beide Varianten halten deinen Kern stabil und bleiben klar im Limit.'} 
	  } 
	   } 
	   } 
	 & & \\ 
 

    \theutterance \stepcounter{utterance}  

    & & \multicolumn{4}{p{0.6\linewidth}}{\cellcolor[rgb]{0.9,0.9,0.9}{%
	\makecell[{{p{\linewidth}}}]{% 
	  \tt {\tiny [GM$\rangle$B]}  
	 ABLEHNUNG: {'C32', 'A83', 'A03', 'C98', 'C00', 'C10', 'C76', 'A07', 'C81', 'C03'}\\ \tt VORSCHLAG: {'C32', 'A83', 'A03', 'C98', 'C00', 'C10', 'C76', 'A07', 'C03', 'C62'}\\ \tt ARGUMENT: {'Minimaler 1-zu-1-Tausch: C81 → C62; alle übrigen 9 Posten bleiben exakt wie bei dir.', 'Gesamtauslastung 2054 (≤ 2307) – Puffer 253; gleicher Fokus, deutlich mehr Spielraum.', 'C62 liefert schnellen, verlässlichen Effekt bei sehr kleinem Aufwand; wir reduzieren Risiko ohne Substanzverlust.', 'Falls C81 für dich unverzichtbar ist: Alternative 1 – A07 → C62 (Auslastung 1953). Alternative 2 – C76 → C62 (Auslastung 2162). Beide Varianten halten deinen Kern stabil und bleiben klar im Limit.'} 
	  } 
	   } 
	   } 
	 & \\ 
 

    \theutterance \stepcounter{utterance}  

    & & & \multicolumn{4}{p{0.6\linewidth}}{\cellcolor[rgb]{1,0.85,0.72}{%
	\makecell[{{p{\linewidth}}}]{% 
	  \tt {\tiny [GM$\langle$B]}  
	 ABLEHNUNG: {'C32', 'A83', 'A03', 'C98', 'C00', 'C10', 'C76', 'A07', 'C03', 'C62'}\\ \tt VORSCHLAG: {'C32', 'A83', 'A03', 'C98', 'C00', 'C10', 'C76', 'A07', 'C03', 'C81'}\\ \tt ARGUMENT: {'Ich übernehme 9/10 deiner Posten; einziger 1‑zu‑1‑Tausch: C62 → C81. Aufwand 2203 (≤ 2307), Puffer 104.', 'Begründung: C81 ist ein tragender Treiber; wir erhöhen den Gesamt‑Hebel spürbar, ohne das Limit zu strapazieren oder die Struktur zu verändern.', 'Du hattest C81 schon mehrfach als gangbare Option adressiert – ich ziehe das jetzt schlank und risikoarm.', 'Falls C62 für dich gesetzt ist: Alternative – A83 → C81 (Auslastung 2229, Puffer 78).', 'Falls du A83 und C62 beide behalten willst: Alternative – streiche C32 und füge C81 hinzu (Auslastung 2258, Puffer 49). Mein Favorit bleibt der Hauptvorschlag oben.'} 
	  } 
	   } 
	   } 
	  \\ 
 

    \theutterance \stepcounter{utterance}  

    & & \multicolumn{4}{p{0.6\linewidth}}{\cellcolor[rgb]{0.9,0.9,0.9}{%
	\makecell[{{p{\linewidth}}}]{% 
	  \tt {\tiny [A$\langle$GM]}  
	 ABLEHNUNG: {'C32', 'A83', 'A03', 'C98', 'C00', 'C10', 'C76', 'A07', 'C03', 'C62'}\\ \tt VORSCHLAG: {'C32', 'A83', 'A03', 'C98', 'C00', 'C10', 'C76', 'A07', 'C03', 'C81'}\\ \tt ARGUMENT: {'Ich übernehme 9/10 deiner Posten; einziger 1‑zu‑1‑Tausch: C62 → C81. Aufwand 2203 (≤ 2307), Puffer 104.', 'Begründung: C81 ist ein tragender Treiber; wir erhöhen den Gesamt‑Hebel spürbar, ohne das Limit zu strapazieren oder die Struktur zu verändern.', 'Du hattest C81 schon mehrfach als gangbare Option adressiert – ich ziehe das jetzt schlank und risikoarm.', 'Falls C62 für dich gesetzt ist: Alternative – A83 → C81 (Auslastung 2229, Puffer 78).', 'Falls du A83 und C62 beide behalten willst: Alternative – streiche C32 und füge C81 hinzu (Auslastung 2258, Puffer 49). Mein Favorit bleibt der Hauptvorschlag oben.'} 
	  } 
	   } 
	   } 
	 & \\ 
 

    \theutterance \stepcounter{utterance}  

    & \multicolumn{4}{p{0.6\linewidth}}{\cellcolor[rgb]{0.8,1,0.9}{%
	\makecell[{{p{\linewidth}}}]{% 
	  \tt {\tiny [A$\rangle$GM]}  
	 ABLEHNUNG: {'C32', 'A83', 'A03', 'C98', 'C00', 'C10', 'C76', 'A07', 'C03', 'C81'}\\ \tt VORSCHLAG: {'C32', 'A83', 'A03', 'C98', 'C00', 'C10', 'C76', 'C03', 'C62', 'B38'}\\ \tt ARGUMENT: {'Gezielter 2-zu-2-Tausch: C81 + A07 → C62 + B38; alle anderen 8 Posten bleiben exakt wie bei dir.', 'Gesamtauslastung 1974 (≤ 2307) – sehr komfortabler Puffer 333.', 'C62 liefert schnellen, verlässlichen Hebel bei minimalem Aufwand; B38 ergänzt ausgewogen – zusammen klar stärker als C81 + A07, ohne deinen Kern (C03, C10, C00) anzutasten.', 'Du hattest C62 bereits mehrfach im Spiel – ich ziehe es wieder rein und halte C03 drin, um deinen Fokus zu respektieren.', 'Falls A07 für dich gesetzt ist: Alternative ohne A07-Verlust – C81 → C62 und C76 → B38 (Auslastung 2183).'} 
	  } 
	   } 
	   } 
	 & & \\ 
 

    \theutterance \stepcounter{utterance}  

    & & \multicolumn{4}{p{0.6\linewidth}}{\cellcolor[rgb]{0.9,0.9,0.9}{%
	\makecell[{{p{\linewidth}}}]{% 
	  \tt {\tiny [GM$\rangle$B]}  
	 ABLEHNUNG: {'C32', 'A83', 'A03', 'C98', 'C00', 'C10', 'C76', 'A07', 'C03', 'C81'}\\ \tt VORSCHLAG: {'C32', 'A83', 'A03', 'C98', 'C00', 'C10', 'C76', 'C03', 'C62', 'B38'}\\ \tt ARGUMENT: {'Gezielter 2-zu-2-Tausch: C81 + A07 → C62 + B38; alle anderen 8 Posten bleiben exakt wie bei dir.', 'Gesamtauslastung 1974 (≤ 2307) – sehr komfortabler Puffer 333.', 'C62 liefert schnellen, verlässlichen Hebel bei minimalem Aufwand; B38 ergänzt ausgewogen – zusammen klar stärker als C81 + A07, ohne deinen Kern (C03, C10, C00) anzutasten.', 'Du hattest C62 bereits mehrfach im Spiel – ich ziehe es wieder rein und halte C03 drin, um deinen Fokus zu respektieren.', 'Falls A07 für dich gesetzt ist: Alternative ohne A07-Verlust – C81 → C62 und C76 → B38 (Auslastung 2183).'} 
	  } 
	   } 
	   } 
	 & \\ 
 

    \theutterance \stepcounter{utterance}  

    & & & \multicolumn{4}{p{0.6\linewidth}}{\cellcolor[rgb]{1,0.85,0.72}{%
	\makecell[{{p{\linewidth}}}]{% 
	  \tt {\tiny [GM$\langle$B]}  
	 ABLEHNUNG: {'C32', 'A83', 'A03', 'C98', 'C00', 'C10', 'C76', 'C03', 'C62', 'B38'}\\ \tt VORSCHLAG: {'C32', 'A83', 'A03', 'C98', 'C00', 'C10', 'C76', 'C03', 'A07', 'C81'}\\ \tt ARGUMENT: {'Gezielter 2‑zu‑2‑Tausch auf Basis deines Pakets: B38 + C62 → A07 + C81. Aufwand 2203 (≤ 2307), Puffer 104; alle übrigen 8 Posten bleiben identisch.', 'Begründung: Wir bündeln zwei kleinere Bausteine in zwei tragende Treiber und erhöhen so den Gesamt‑Hebel ohne das Limit zu strapazieren; dein Kern (C03, C10, C00) bleibt unverändert.', 'Du hattest A07 und C81 selbst mehrfach als gangbare Optionen adressiert – ich ziehe das jetzt schlank und fokussiert.', 'Falls du an einem der beiden festhalten willst: (a) Behalte C62, tausche nur B38 → A07 (Aufwand 2054). (b) Behalte B38, tausche nur C62 → C81 (Aufwand 2123). Beide Alternativen bleiben klar ≤ 2307.'} 
	  } 
	   } 
	   } 
	  \\ 
 

    \theutterance \stepcounter{utterance}  

    & & \multicolumn{4}{p{0.6\linewidth}}{\cellcolor[rgb]{0.9,0.9,0.9}{%
	\makecell[{{p{\linewidth}}}]{% 
	  \tt {\tiny [A$\langle$GM]}  
	 ABLEHNUNG: {'C32', 'A83', 'A03', 'C98', 'C00', 'C10', 'C76', 'C03', 'C62', 'B38'}\\ \tt VORSCHLAG: {'C32', 'A83', 'A03', 'C98', 'C00', 'C10', 'C76', 'C03', 'A07', 'C81'}\\ \tt ARGUMENT: {'Gezielter 2‑zu‑2‑Tausch auf Basis deines Pakets: B38 + C62 → A07 + C81. Aufwand 2203 (≤ 2307), Puffer 104; alle übrigen 8 Posten bleiben identisch.', 'Begründung: Wir bündeln zwei kleinere Bausteine in zwei tragende Treiber und erhöhen so den Gesamt‑Hebel ohne das Limit zu strapazieren; dein Kern (C03, C10, C00) bleibt unverändert.', 'Du hattest A07 und C81 selbst mehrfach als gangbare Optionen adressiert – ich ziehe das jetzt schlank und fokussiert.', 'Falls du an einem der beiden festhalten willst: (a) Behalte C62, tausche nur B38 → A07 (Aufwand 2054). (b) Behalte B38, tausche nur C62 → C81 (Aufwand 2123). Beide Alternativen bleiben klar ≤ 2307.'} 
	  } 
	   } 
	   } 
	 & \\ 
 

    \theutterance \stepcounter{utterance}  

    & \multicolumn{4}{p{0.6\linewidth}}{\cellcolor[rgb]{0.8,1,0.9}{%
	\makecell[{{p{\linewidth}}}]{% 
	  \tt {\tiny [A$\rangle$GM]}  
	 ABLEHNUNG: {'C32', 'A83', 'A03', 'C98', 'C00', 'C10', 'C76', 'C03', 'A07', 'C81'}\\ \tt VORSCHLAG: {'C32', 'A83', 'A03', 'C98', 'C00', 'C10', 'C03', 'A07', 'C81', 'C62'}\\ \tt ARGUMENT: {'Minimaler 1‑zu‑1‑Tausch: C76 → C62; alle übrigen 9 Posten bleiben unverändert.', 'Aufwand 2162 (≤ 2307) – Puffer 145, gleicher Fokus mit deinen Ankern C03/C10 und den Bausteinen A07/C81.', 'C62 liefert schnellen, verlässlichen Output bei sehr kleinem Footprint – mehr Wirkung pro Aufwand und zusätzlicher Spielraum ohne Strukturbruch.', 'Falls C76 für dich gesetzt ist: Alternative A07 → C62 (Aufwand 1953) oder C81 → C62 (Aufwand 2054) – gleicher Rahmen, jeweils deutlich mehr Puffer.', 'Lass uns damit schlank schließen – ich halte diesen 1‑zu‑1‑Schritt für den fairsten Kompromiss.'} 
	  } 
	   } 
	   } 
	 & & \\ 
 

    \theutterance \stepcounter{utterance}  

    & & \multicolumn{4}{p{0.6\linewidth}}{\cellcolor[rgb]{0.9,0.9,0.9}{%
	\makecell[{{p{\linewidth}}}]{% 
	  \tt {\tiny [GM$\rangle$B]}  
	 ABLEHNUNG: {'C32', 'A83', 'A03', 'C98', 'C00', 'C10', 'C76', 'C03', 'A07', 'C81'}\\ \tt VORSCHLAG: {'C32', 'A83', 'A03', 'C98', 'C00', 'C10', 'C03', 'A07', 'C81', 'C62'}\\ \tt ARGUMENT: {'Minimaler 1‑zu‑1‑Tausch: C76 → C62; alle übrigen 9 Posten bleiben unverändert.', 'Aufwand 2162 (≤ 2307) – Puffer 145, gleicher Fokus mit deinen Ankern C03/C10 und den Bausteinen A07/C81.', 'C62 liefert schnellen, verlässlichen Output bei sehr kleinem Footprint – mehr Wirkung pro Aufwand und zusätzlicher Spielraum ohne Strukturbruch.', 'Falls C76 für dich gesetzt ist: Alternative A07 → C62 (Aufwand 1953) oder C81 → C62 (Aufwand 2054) – gleicher Rahmen, jeweils deutlich mehr Puffer.', 'Lass uns damit schlank schließen – ich halte diesen 1‑zu‑1‑Schritt für den fairsten Kompromiss.'} 
	  } 
	   } 
	   } 
	 & \\ 
 

    \theutterance \stepcounter{utterance}  

    & & & \multicolumn{4}{p{0.6\linewidth}}{\cellcolor[rgb]{1,0.85,0.72}{%
	\makecell[{{p{\linewidth}}}]{% 
	  \tt {\tiny [GM$\langle$B]}  
	 ABLEHNUNG: {'C32', 'A83', 'A03', 'C98', 'C00', 'C10', 'C03', 'A07', 'C81', 'C62'}\\ \tt VORSCHLAG: {'C32', 'A83', 'A03', 'C98', 'C00', 'C10', 'C03', 'A07', 'C81', 'C76'}\\ \tt ARGUMENT: {'Minimaler 1‑zu‑1‑Tausch in deinem Paket: C62 → C76; alle übrigen 9 Posten bleiben unverändert.', 'Aufwand 2203 (≤ 2307), Puffer 104 – ein Rahmen, den du zuvor mehrfach als akzeptabel adressiert hast.', 'Vorteil: gleicher schlanker Zuschnitt mit deinen Ankern C03/C10/A07/C81, nur mit spürbar mehr Zugkraft statt eines Kleinstpostens – ohne das Limit zu strapazieren.', 'Falls C62 für dich gesetzt ist: nahe Alternative (ebenfalls ≤ 2307) – A83 → C76, Rest identisch; Aufwand 2229, Puffer 78.', 'Oder: wenn du absolut nichts anderes ändern willst, streiche C32 und füge C76 hinzu (C62 und A83 bleiben); Aufwand 2258, Puffer 49. Mein klarer Favorit bleibt der Hauptvorschlag oben.'} 
	  } 
	   } 
	   } 
	  \\ 
 

    \theutterance \stepcounter{utterance}  

    & & \multicolumn{4}{p{0.6\linewidth}}{\cellcolor[rgb]{0.9,0.9,0.9}{%
	\makecell[{{p{\linewidth}}}]{% 
	  \tt {\tiny [A$\langle$GM]}  
	 ABLEHNUNG: {'C32', 'A83', 'A03', 'C98', 'C00', 'C10', 'C03', 'A07', 'C81', 'C62'}\\ \tt VORSCHLAG: {'C32', 'A83', 'A03', 'C98', 'C00', 'C10', 'C03', 'A07', 'C81', 'C76'}\\ \tt ARGUMENT: {'Minimaler 1‑zu‑1‑Tausch in deinem Paket: C62 → C76; alle übrigen 9 Posten bleiben unverändert.', 'Aufwand 2203 (≤ 2307), Puffer 104 – ein Rahmen, den du zuvor mehrfach als akzeptabel adressiert hast.', 'Vorteil: gleicher schlanker Zuschnitt mit deinen Ankern C03/C10/A07/C81, nur mit spürbar mehr Zugkraft statt eines Kleinstpostens – ohne das Limit zu strapazieren.', 'Falls C62 für dich gesetzt ist: nahe Alternative (ebenfalls ≤ 2307) – A83 → C76, Rest identisch; Aufwand 2229, Puffer 78.', 'Oder: wenn du absolut nichts anderes ändern willst, streiche C32 und füge C76 hinzu (C62 und A83 bleiben); Aufwand 2258, Puffer 49. Mein klarer Favorit bleibt der Hauptvorschlag oben.'} 
	  } 
	   } 
	   } 
	 & \\ 
 

    \theutterance \stepcounter{utterance}  

    & \multicolumn{4}{p{0.6\linewidth}}{\cellcolor[rgb]{0.8,1,0.9}{%
	\makecell[{{p{\linewidth}}}]{% 
	  \tt {\tiny [A$\rangle$GM]}  
	 ABLEHNUNG: {'C32', 'A83', 'A03', 'C98', 'C00', 'C10', 'C03', 'A07', 'C81', 'C76'}\\ \tt VORSCHLAG: {'A83', 'A03', 'C98', 'C00', 'C10', 'C03', 'A07', 'C81', 'C76', 'C62'}\\ \tt ARGUMENT: {'Minimaler 1-zu-1-Tausch: C32 → C62; 9/10 deiner Posten bleiben unverändert.', 'Gesamtauslastung 2258 (≤ 2307), Puffer 49 – schlank und sicher im Rahmen.', 'C62 liefert schnellen, verlässlichen Hebel bei kleinem Footprint; wir erhöhen Wirkung pro Aufwand ohne deine Anker (C03, C10, A07) anzutasten.', 'Falls C32 für dich gesetzt ist: nahe Alternative – C76 → C62 (Auslastung 2162).', 'Oder: A83 → C62 (Auslastung 2229), Rest identisch. Beide Varianten bleiben ≤ 2307 und respektieren deinen Kern.'} 
	  } 
	   } 
	   } 
	 & & \\ 
 

    \theutterance \stepcounter{utterance}  

    & & \multicolumn{4}{p{0.6\linewidth}}{\cellcolor[rgb]{0.9,0.9,0.9}{%
	\makecell[{{p{\linewidth}}}]{% 
	  \tt {\tiny [GM$\rangle$B]}  
	 ABLEHNUNG: {'C32', 'A83', 'A03', 'C98', 'C00', 'C10', 'C03', 'A07', 'C81', 'C76'}\\ \tt VORSCHLAG: {'A83', 'A03', 'C98', 'C00', 'C10', 'C03', 'A07', 'C81', 'C76', 'C62'}\\ \tt ARGUMENT: {'Minimaler 1-zu-1-Tausch: C32 → C62; 9/10 deiner Posten bleiben unverändert.', 'Gesamtauslastung 2258 (≤ 2307), Puffer 49 – schlank und sicher im Rahmen.', 'C62 liefert schnellen, verlässlichen Hebel bei kleinem Footprint; wir erhöhen Wirkung pro Aufwand ohne deine Anker (C03, C10, A07) anzutasten.', 'Falls C32 für dich gesetzt ist: nahe Alternative – C76 → C62 (Auslastung 2162).', 'Oder: A83 → C62 (Auslastung 2229), Rest identisch. Beide Varianten bleiben ≤ 2307 und respektieren deinen Kern.'} 
	  } 
	   } 
	   } 
	 & \\ 
 

    \theutterance \stepcounter{utterance}  

    & & & \multicolumn{4}{p{0.6\linewidth}}{\cellcolor[rgb]{1,0.85,0.72}{%
	\makecell[{{p{\linewidth}}}]{% 
	  \tt {\tiny [GM$\langle$B]}  
	 ABLEHNUNG: {'A83', 'A03', 'C98', 'C00', 'C10', 'C03', 'A07', 'C81', 'C76', 'C62'}\\ \tt VORSCHLAG: {'C32', 'A03', 'C98', 'C00', 'C10', 'C03', 'A07', 'C81', 'C76', 'C62'}\\ \tt ARGUMENT: {'Minimaler 1‑zu‑1‑Tausch: A83 → C32; alle übrigen 9 Posten bleiben unverändert.', 'Aufwand 2229 (≤ 2307), Puffer 78 – schlank, sicher, ohne Strukturbruch.', 'C32 war in mehreren deiner früheren Mixe enthalten und ergänzt den bestehenden Kern (C03, C10, A07, C81) sauber; wir halten Fokus und erhöhen die Effizienz pro Aufwand.', 'Falls dir A83 unbedingt wichtig ist, nahe Alternative mit noch mehr Puffer: C62 → C32 (Aufwand 2203, Puffer 104) – gleicher Kern, gleicher Umfang.'} 
	  } 
	   } 
	   } 
	  \\ 
 

    \theutterance \stepcounter{utterance}  

    & & \multicolumn{4}{p{0.6\linewidth}}{\cellcolor[rgb]{0.9,0.9,0.9}{%
	\makecell[{{p{\linewidth}}}]{% 
	  \tt {\tiny [A$\langle$GM]}  
	 ABLEHNUNG: {'A83', 'A03', 'C98', 'C00', 'C10', 'C03', 'A07', 'C81', 'C76', 'C62'}\\ \tt VORSCHLAG: {'C32', 'A03', 'C98', 'C00', 'C10', 'C03', 'A07', 'C81', 'C76', 'C62'}\\ \tt ARGUMENT: {'Minimaler 1‑zu‑1‑Tausch: A83 → C32; alle übrigen 9 Posten bleiben unverändert.', 'Aufwand 2229 (≤ 2307), Puffer 78 – schlank, sicher, ohne Strukturbruch.', 'C32 war in mehreren deiner früheren Mixe enthalten und ergänzt den bestehenden Kern (C03, C10, A07, C81) sauber; wir halten Fokus und erhöhen die Effizienz pro Aufwand.', 'Falls dir A83 unbedingt wichtig ist, nahe Alternative mit noch mehr Puffer: C62 → C32 (Aufwand 2203, Puffer 104) – gleicher Kern, gleicher Umfang.'} 
	  } 
	   } 
	   } 
	 & \\ 
 

    \theutterance \stepcounter{utterance}  

    & \multicolumn{4}{p{0.6\linewidth}}{\cellcolor[rgb]{0.8,1,0.9}{%
	\makecell[{{p{\linewidth}}}]{% 
	  \tt {\tiny [A$\rangle$GM]}  
	 ABLEHNUNG: {'C32', 'A03', 'C98', 'C00', 'C10', 'C03', 'A07', 'C81', 'C76', 'C62'}\\ \tt VORSCHLAG: {'C32', 'A83', 'A03', 'C98', 'C00', 'C10', 'C03', 'C81', 'C76', 'C62'}\\ \tt ARGUMENT: {'Minimaler 1‑zu‑1‑Tausch: A07 → A83; die übrigen 9 Posten bleiben unverändert.', 'Gesamtauslastung 1953 (≤ 2307) – Puffer 354, deutlich mehr Spielraum ohne Strukturbruch.', 'A83 liefert stabilen Hebel bei kleinem Footprint; wir halten deine Anker (C03, C10) und den bestehenden Rahmen (C32, A03, C98, C00, C76, C62, C81) intakt.', 'Falls A07 für dich gesetzt ist: Alternative 1 – C81 → A83 (Auslastung 2054, Puffer 253). Alternative 2 – C76 → A83 (Auslastung 2162, Puffer 145).'} 
	  } 
	   } 
	   } 
	 & & \\ 
 

    \theutterance \stepcounter{utterance}  

    & & \multicolumn{4}{p{0.6\linewidth}}{\cellcolor[rgb]{0.9,0.9,0.9}{%
	\makecell[{{p{\linewidth}}}]{% 
	  \tt {\tiny [GM$\rangle$B]}  
	 ABLEHNUNG: {'C32', 'A03', 'C98', 'C00', 'C10', 'C03', 'A07', 'C81', 'C76', 'C62'}\\ \tt VORSCHLAG: {'C32', 'A83', 'A03', 'C98', 'C00', 'C10', 'C03', 'C81', 'C76', 'C62'}\\ \tt ARGUMENT: {'Minimaler 1‑zu‑1‑Tausch: A07 → A83; die übrigen 9 Posten bleiben unverändert.', 'Gesamtauslastung 1953 (≤ 2307) – Puffer 354, deutlich mehr Spielraum ohne Strukturbruch.', 'A83 liefert stabilen Hebel bei kleinem Footprint; wir halten deine Anker (C03, C10) und den bestehenden Rahmen (C32, A03, C98, C00, C76, C62, C81) intakt.', 'Falls A07 für dich gesetzt ist: Alternative 1 – C81 → A83 (Auslastung 2054, Puffer 253). Alternative 2 – C76 → A83 (Auslastung 2162, Puffer 145).'} 
	  } 
	   } 
	   } 
	 & \\ 
 

    \theutterance \stepcounter{utterance}  

    & & & \multicolumn{4}{p{0.6\linewidth}}{\cellcolor[rgb]{1,0.85,0.72}{%
	\makecell[{{p{\linewidth}}}]{% 
	  \tt {\tiny [GM$\langle$B]}  
	 ABLEHNUNG: {'C32', 'A83', 'A03', 'C98', 'C00', 'C10', 'C03', 'C81', 'C76', 'C62'}\\ \tt VORSCHLAG: {'C32', 'A03', 'C98', 'C00', 'C10', 'C03', 'C81', 'C76', 'C62', 'A07'}\\ \tt ARGUMENT: {'Minimaler 1-zu-1-Tausch in deinem Paket: A83 → A07; alle übrigen 9 Posten bleiben unverändert.', 'Aufwand 2229 (≤ 2307), Puffer 78 – klar im Rahmen ohne Strukturbruch.', 'Wir bündeln den Hebel auf einen starken Treiber (A07), lassen deinen bevorzugten Kleinteil C62 bewusst drin und halten C03/C10/C81/C76 unverändert.', 'Falls A83 für dich gesetzt ist: enge Alternative mit gleicher Stoßrichtung – C32 → A07 (Aufwand 2258, Puffer 49), restliche 9 Posten unverändert.', 'Wenn dir noch mehr Puffer wichtig ist: A83 bleibt draußen, C62 → A07 (Aufwand 2203, Puffer 104). Ich bevorzuge den Hauptvorschlag (A83 → A07).'} 
	  } 
	   } 
	   } 
	  \\ 
 

    \theutterance \stepcounter{utterance}  

    & & \multicolumn{4}{p{0.6\linewidth}}{\cellcolor[rgb]{0.9,0.9,0.9}{%
	\makecell[{{p{\linewidth}}}]{% 
	  \tt {\tiny [A$\langle$GM]}  
	 ABLEHNUNG: {'C32', 'A83', 'A03', 'C98', 'C00', 'C10', 'C03', 'C81', 'C76', 'C62'}\\ \tt VORSCHLAG: {'C32', 'A03', 'C98', 'C00', 'C10', 'C03', 'C81', 'C76', 'C62', 'A07'}\\ \tt ARGUMENT: {'Minimaler 1-zu-1-Tausch in deinem Paket: A83 → A07; alle übrigen 9 Posten bleiben unverändert.', 'Aufwand 2229 (≤ 2307), Puffer 78 – klar im Rahmen ohne Strukturbruch.', 'Wir bündeln den Hebel auf einen starken Treiber (A07), lassen deinen bevorzugten Kleinteil C62 bewusst drin und halten C03/C10/C81/C76 unverändert.', 'Falls A83 für dich gesetzt ist: enge Alternative mit gleicher Stoßrichtung – C32 → A07 (Aufwand 2258, Puffer 49), restliche 9 Posten unverändert.', 'Wenn dir noch mehr Puffer wichtig ist: A83 bleibt draußen, C62 → A07 (Aufwand 2203, Puffer 104). Ich bevorzuge den Hauptvorschlag (A83 → A07).'} 
	  } 
	   } 
	   } 
	 & \\ 
 

    \theutterance \stepcounter{utterance}  

    & \multicolumn{4}{p{0.6\linewidth}}{\cellcolor[rgb]{0.8,1,0.9}{%
	\makecell[{{p{\linewidth}}}]{% 
	  \tt {\tiny [A$\rangle$GM]}  
	 ABLEHNUNG: {'C32', 'A03', 'C98', 'C00', 'C10', 'C03', 'C81', 'C76', 'C62', 'A07'}\\ \tt VORSCHLAG: {'C32', 'A03', 'C98', 'C00', 'C10', 'C03', 'C76', 'C62', 'A07', 'A83'}\\ \tt ARGUMENT: {'Minimaler 1‑zu‑1‑Tausch: C81 → A83; alle übrigen 9 Posten bleiben identisch.', 'Gesamtauslastung 2054 (≤ 2307) – Puffer 253, deutlich mehr Spielraum ohne Strukturbruch.', 'A83 ist kompakt und wirkungsstark; deine Anker C03 und C10 sowie der restliche Rahmen (C32, A03, C98, C00, C76, C62, A07) bleiben unverändert.', 'Du hattest A83 bereits mehrfach als valide Option adressiert – ich ziehe das jetzt schlank und risikoarm.', 'Falls C81 für dich unverzichtbar ist: nahe Alternativen (beide ≤ 2307): (a) C76 → A83 (Auslastung 2162) oder (b) C32 → A83 (Auslastung 2258).'} 
	  } 
	   } 
	   } 
	 & & \\ 
 

    \theutterance \stepcounter{utterance}  

    & & \multicolumn{4}{p{0.6\linewidth}}{\cellcolor[rgb]{0.9,0.9,0.9}{%
	\makecell[{{p{\linewidth}}}]{% 
	  \tt {\tiny [GM$\rangle$B]}  
	 ABLEHNUNG: {'C32', 'A03', 'C98', 'C00', 'C10', 'C03', 'C81', 'C76', 'C62', 'A07'}\\ \tt VORSCHLAG: {'C32', 'A03', 'C98', 'C00', 'C10', 'C03', 'C76', 'C62', 'A07', 'A83'}\\ \tt ARGUMENT: {'Minimaler 1‑zu‑1‑Tausch: C81 → A83; alle übrigen 9 Posten bleiben identisch.', 'Gesamtauslastung 2054 (≤ 2307) – Puffer 253, deutlich mehr Spielraum ohne Strukturbruch.', 'A83 ist kompakt und wirkungsstark; deine Anker C03 und C10 sowie der restliche Rahmen (C32, A03, C98, C00, C76, C62, A07) bleiben unverändert.', 'Du hattest A83 bereits mehrfach als valide Option adressiert – ich ziehe das jetzt schlank und risikoarm.', 'Falls C81 für dich unverzichtbar ist: nahe Alternativen (beide ≤ 2307): (a) C76 → A83 (Auslastung 2162) oder (b) C32 → A83 (Auslastung 2258).'} 
	  } 
	   } 
	   } 
	 & \\ 
 

    \theutterance \stepcounter{utterance}  

    & & & \multicolumn{4}{p{0.6\linewidth}}{\cellcolor[rgb]{1,0.85,0.72}{%
	\makecell[{{p{\linewidth}}}]{% 
	  \tt {\tiny [GM$\langle$B]}  
	 ABLEHNUNG: {'C32', 'A83', 'A03', 'C98', 'C00', 'C10', 'C03', 'C81', 'C76', 'C62'}\\ \tt VORSCHLAG: {'C32', 'A83', 'A03', 'C98', 'C00', 'C10', 'C03', 'C81', 'C76', 'A07'}\\ \tt ARGUMENT: {'Minimaler 1-zu-1-Tausch in deinem Paket: C62 → A07; Aufwand 2203 (≤ 2307), Puffer 104.', '9/10 deiner Posten bleiben unverändert; wir ersetzen nur den Kleinstposten durch einen tragenden Treiber – gleicher Umfang, deutlich mehr Hebel, weiterhin klar im Limit.', 'Du hattest A07 wiederholt als gangbare Option genannt; ich ziehe das jetzt schlank und risikoarm.', 'Falls C62 für dich gesetzt ist: enge Alternative ohne Mehrlast – A83 → A07 (Aufwand 2229), Rest unverändert.'} 
	  } 
	   } 
	   } 
	  \\ 
 

    \theutterance \stepcounter{utterance}  

    & & \multicolumn{4}{p{0.6\linewidth}}{\cellcolor[rgb]{0.9,0.9,0.9}{%
	\makecell[{{p{\linewidth}}}]{% 
	  \tt {\tiny [GM$\rangle$B]}  
	 Du hast einen Vorschlag abgelehnt, der nicht aktiv ist. Vorschläge sind nur dann aktiv, wenn sie vom anderen Spieler mit VORSCHLAG eingebracht wurden und nicht von dir mit ABLEHNUNG deaktiviert wurden. Versuche es noch einmal. 
	  } 
	   } 
	   } 
	 & \\ 
 

    \theutterance \stepcounter{utterance}  

    & & & \multicolumn{4}{p{0.6\linewidth}}{\cellcolor[rgb]{1,0.85,0.72}{%
	\makecell[{{p{\linewidth}}}]{% 
	  \tt {\tiny [GM$\langle$B]}  
	 ABLEHNUNG: {'C32', 'A83', 'A03', 'C98', 'C00', 'C10', 'C03', 'C81', 'C76', 'C62'}\\ \tt VORSCHLAG: {'C32', 'A83', 'A03', 'C98', 'C00', 'C10', 'C03', 'C81', 'C76', 'A07'}\\ \tt ARGUMENT: {'Minimaler 1‑zu‑1‑Tausch in deinem aktiven Paket: C62 → A07. Aufwand 2203 (≤ 2307), Puffer 104.', 'Begründung: Wir ersetzen einen Kleinstposten durch einen tragenden Treiber – gleicher Zuschnitt (10 Posten), deutlich mehr Hebel, weiterhin klar im Limit.', 'Falls C62 für dich gesetzt ist: enge Alternative ohne Mehrlast – A83 → A07 (Aufwand 2229), Rest unverändert. Welche der beiden Varianten ist für dich gangbar?'} 
	  } 
	   } 
	   } 
	  \\ 
 

    \theutterance \stepcounter{utterance}  

    & & & \multicolumn{2}{p{0.3\linewidth}}{\cellcolor[rgb]{0.95,0.95,0.95}{%
	\makecell[{{p{\linewidth}}}]{% 
	  \tt {\tiny [GM$|$GM]}  
	 None 
	  } 
	   } 
	   } 
	 & & \\ 
 

    \theutterance \stepcounter{utterance}  

    & & & \multicolumn{2}{p{0.3\linewidth}}{\cellcolor[rgb]{0.95,0.95,0.95}{%
	\makecell[{{p{\linewidth}}}]{% 
	  \tt {\tiny [GM$|$GM]}  
	 end game 
	  } 
	   } 
	   } 
	 & & \\ 
 

\end{supertabular}
}

\end{document}
