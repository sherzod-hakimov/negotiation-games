
\documentclass{article}
\usepackage{colortbl}
\usepackage{makecell}
\usepackage{multirow}
\usepackage{supertabular}

\begin{document}

\newcounter{utterance}

\centering \large Interaction Transcript for game `hot\_air\_balloon', experiment `air\_balloon\_survival\_de\_complexity\_easy', episode 1 with gpt{-}5{-}mini{-}2025{-}08{-}07{-}t1.0.
\vspace{24pt}

{ \footnotesize  \setcounter{utterance}{1}
\setlength{\tabcolsep}{0pt}
\begin{supertabular}{c@{$\;$}|p{.15\linewidth}@{}p{.15\linewidth}p{.15\linewidth}p{.15\linewidth}p{.15\linewidth}p{.15\linewidth}}
    \# & $\;$A & \multicolumn{4}{c}{Game Master} & $\;\:$B\\
    \hline

    \theutterance \stepcounter{utterance}  
    & & \multicolumn{4}{p{0.6\linewidth}}{
        \cellcolor[rgb]{0.9,0.9,0.9}{
            \makecell[{{p{\linewidth}}}]{
                \texttt{\tiny{[P1$\langle$GM]}}
                \texttt{Du nimmst an einem kollaborativen Verhandlungspiel Teil.} \\
\\ 
\texttt{Zusammen mit einem anderen Teilnehmer musst du dich auf eine Reihe von Gegenständen entscheiden, die behalten werden. Jeder von euch hat eine persönliche Verteilung über die Wichtigkeit der einzelnen Gegenstände. Jeder von euch hat eine eigene Meinung darüber, wie wichtig jeder einzelne Gegenstand ist (Gegenstandswichtigkeit). Du kennst die Wichtigkeitsverteilung des anderen Spielers nicht. Zusätzlich siehst du, wie viel Aufwand jeder Gegenstand verursacht.  } \\
\texttt{Ihr dürft euch nur auf eine Reihe von Gegenständen einigen, wenn der Gesamtaufwand der ausgewählten Gegenstände den Maximalaufwand nicht überschreitet:} \\
\\ 
\texttt{Maximalaufwand = 8170} \\
\\ 
\texttt{Hier sind die einzelnen Aufwände der Gegenstände:} \\
\\ 
\texttt{Aufwand der Gegenstände = \{"C32": 594, "C52": 551, "B54": 634, "C16": 204, "C50": 557, "A08": 440, "C59": 734, "B51": 893, "B55": 246, "B59": 592, "A77": 143, "C14": 568, "C24": 472, "B24": 401, "C31": 729, "A52": 201, "C61": 85, "B79": 641, "C54": 958, "B56": 80, "A40": 157, "B11": 806, "B60": 684, "A81": 59, "A95": 31, "B38": 765, "C51": 416, "C03": 392, "A91": 428, "A93": 699, "C28": 141, "C05": 606, "B75": 612, "B96": 133, "B86": 689\}} \\
\\ 
\texttt{Hier ist deine persönliche Verteilung der Wichtigkeit der einzelnen Gegenstände:} \\
\\ 
\texttt{Werte der Gegenstandswichtigkeit = \{"C32": 138, "C52": 583, "B54": 868, "C16": 822, "C50": 783, "A08": 65, "C59": 262, "B51": 121, "B55": 508, "B59": 780, "A77": 461, "C14": 484, "C24": 668, "B24": 389, "C31": 808, "A52": 215, "C61": 97, "B79": 500, "C54": 30, "B56": 915, "A40": 856, "B11": 400, "B60": 444, "A81": 623, "A95": 781, "B38": 786, "C51": 3, "C03": 713, "A91": 457, "A93": 273, "C28": 739, "C05": 822, "B75": 235, "B96": 606, "B86": 105\}} \\
\\ 
\texttt{Ziel:} \\
\\ 
\texttt{Dein Ziel ist es, eine Reihe von Gegenständen auszuhandeln, die dir möglichst viel bringt (d. h. Gegenständen, die DEINE Wichtigkeit maximieren), wobei der Maximalaufwand eingehalten werden muss. Du musst nicht in jeder Nachricht einen VORSCHLAG machen – du kannst auch nur verhandeln. Alle Taktiken sind erlaubt!} \\
\\ 
\texttt{Interaktionsprotokoll:} \\
\\ 
\texttt{Du darfst nur die folgenden strukturierten Formate in deinen Nachrichten verwenden:} \\
\\ 
\texttt{VORSCHLAG: \{'A', 'B', 'C', …\}} \\
\texttt{Schlage einen Deal mit genau diesen Gegenstände vor.} \\
\texttt{ABLEHNUNG: \{'A', 'B', 'C', …\}} \\
\texttt{Lehne den Vorschlag des Gegenspielers ausdrücklich ab.} \\
\texttt{ARGUMENT: \{'...'\}} \\
\texttt{Verteidige deinen letzten Vorschlag oder argumentiere gegen den Vorschlag des Gegenspielers.} \\
\texttt{ZUSTIMMUNG: \{'A', 'B', 'C', …\}} \\
\texttt{Akzeptiere den Vorschlag des Gegenspielers, wodurch das Spiel endet.} \\
\texttt{STRATEGISCHE ÜBERLEGUNGEN: \{'...'\}} \\
\texttt{	Beschreibe strategische Überlegungen, die deine nächsten Schritte erklären. Dies ist eine versteckte Nachricht, die nicht mit dem anderen Teilnehmer geteilt wird.} \\
\\ 
\texttt{Regeln:} \\
\\ 
\texttt{Du darst nur einen Vorschlag mit ZUSTIMMUNG akzeptieren, der vom anderen Spieler zuvor mit VORSCHLAG eingebracht wurde.} \\
\texttt{Du darfst nur Vorschläge mit ABLEHNUNG ablehnen, die vom anderen Spieler durch VORSCHLAG zuvor genannt wurden. } \\
\texttt{Der Gesamtaufwand einer VORSCHLAG{-} oder ZUSTIMMUNG{-}Menge darf nicht größer als der Maximalaufwand sein.  } \\
\texttt{Offenbare deine versteckte Wichtigkeitsverteilung nicht.} \\
\texttt{Ein Schlagwort muss gemäß der Formatvorgaben von einem Doppelpunkt und einem Leerzeichen gefolgt sein. Das Argument muss eine Python{-}Menge mit 0 oder mehr Strings sein.  } \\
\texttt{Die Form muss also SCHLAGWORT: \{...\} sein.} \\
\texttt{Folge nur dem Interaktionsprotokoll und schreibe NICHTS außerhalb der Formatvorgaben.} \\
\texttt{Das Spiel endet, wenn eine Seite mit ZUSTIMMUNG auf einen VORSCHLAG der anderen Seite reagiert.  } \\
\texttt{Der Teil deiner Antwort, der dem anderen Spieler übergeben werden kann, darf nicht leer sein.  } \\
\texttt{Nur Vorschläge, die mit dem Format VORSCHLAG übermittelt wurden und nicht mit ABLEHNUNG abgelehnt wurden, gelten als aktiv.  } \\
\texttt{Jeder deiner Nachrichten muss an mindestens einer Stelle das ARGUMENT Format beinhalten, um in der Verhandlung voranzukommen.} \\
\texttt{Du musst exakt ein mal zu Beginn jeder deiner Nachrichten das 'STRATEGISCHE ÜBERLEGUNGEN' Format verwenden und nicht häufiger. Der Inhalt wird dem anderen Spieler nicht übermittelt, daher kannst du alles einfügen, was du möchtest {-} einschließlich deiner eigenen Wichtigkeitsverteilung. Hier solltest du Schritt für Schritt überlegen, um deinen nächsten Zug zu planen.} \\
            }
        }
    }
    & \\ \\

    \theutterance \stepcounter{utterance}  
    & \multicolumn{4}{p{0.6\linewidth}}{
        \cellcolor[rgb]{0.9,0.9,0.9}{
            \makecell[{{p{\linewidth}}}]{
                \texttt{\tiny{[P1$\rangle$GM]}}
                \texttt{STRATEGISCHE ÜBERLEGUNGEN: \{'Ziel: meinen Wert maximieren unter 8170 Aufwand. Habe 0/1{-}Knapsack gerechnet. Sehr hohe Wert/Aufwand{-}Ratios: A95, A81, B56, A40, C28, B96, C16, A77, B55, C03, C24 etc. Startlösung mit B38 verbessert, indem ich B38 (w765,v786) gestrichen und durch B79 (641,500) + C14 (568,484) + B24 (401,389) ersetzt habe: Gesamtwert +87 bei w=8051 (<=8170). Endset S2 Wert≈14478. Puffer 119 Aufwand für Flexibilität. Harte Must{-}haves: A95, A81, B56, A40, C28, B96, C16, A77, B55, C03, C24, C50, B54, C05, B59; mittlere Flex: C31, C52, C61, A52, A91; weichste Flex: B24, C14, B79 (trotz guter Effizienz, aber am ehesten verhandelbar falls Gegenseite starke Präferenzen hat). Taktik: Starkes Ankerangebot mit vielen günstigen Items, als „kosteneffizient“ und budgetfreundlich framend. Bereitschaft zu 1:1{-}Tauschs bei ähnlichem Aufwand signalisieren, ohne Werte zu offenbaren. Wenn Gegenseite bestimmte Muss{-}Artikel fordert, biete Swap gegen B24/C14/C52/A52/A91 an, solange Gesamtaufwand ≤8170 bleibt.'\}} \\
\texttt{VORSCHLAG: \{'A95', 'A81', 'B56', 'A40', 'C28', 'B96', 'C16', 'A77', 'B55', 'C03', 'C24', 'C50', 'B54', 'C05', 'B59', 'C31', 'C52', 'C61', 'A52', 'A91', 'B79', 'C14', 'B24'\}} \\
\texttt{ARGUMENT: \{'Ich schlage diese Auswahl vor, weil sie das Budget klar einhält (Gesamtaufwand 8051 < 8170) und viele sehr aufwandsarme Posten enthält (z. B. A95, A81, B56), die den Gesamtnutzen pro Aufwand erhöhen und zugleich Spielraum lassen. Wenn du bestimmte Muss{-}Gegenstände hast, können wir flexibel 1:1 tauschen – idealerweise mit ähnlichem Aufwand –, ohne das Budget zu reißen. Welche 2–3 Items sind dir besonders wichtig, damit ich gezielt Anpassungen machen kann?'\}} \\
            }
        }
    }
    & & \\ \\

    \theutterance \stepcounter{utterance}  
    & & \multicolumn{4}{p{0.6\linewidth}}{
        \cellcolor[rgb]{0.9,0.9,0.9}{
            \makecell[{{p{\linewidth}}}]{
                \texttt{\tiny{[GM$\rangle$P2]}}
                \texttt{Du nimmst an einem kollaborativen Verhandlungspiel Teil.} \\
\\ 
\texttt{Zusammen mit einem anderen Teilnehmer musst du dich auf eine Reihe von Gegenständen entscheiden, die behalten werden. Jeder von euch hat eine persönliche Verteilung über die Wichtigkeit der einzelnen Gegenstände. Jeder von euch hat eine eigene Meinung darüber, wie wichtig jeder einzelne Gegenstand ist (Gegenstandswichtigkeit). Du kennst die Wichtigkeitsverteilung des anderen Spielers nicht. Zusätzlich siehst du, wie viel Aufwand jeder Gegenstand verursacht.  } \\
\texttt{Ihr dürft euch nur auf eine Reihe von Gegenständen einigen, wenn der Gesamtaufwand der ausgewählten Gegenstände den Maximalaufwand nicht überschreitet:} \\
\\ 
\texttt{Maximalaufwand = 8170} \\
\\ 
\texttt{Hier sind die einzelnen Aufwände der Gegenstände:} \\
\\ 
\texttt{Aufwand der Gegenstände = \{"C32": 594, "C52": 551, "B54": 634, "C16": 204, "C50": 557, "A08": 440, "C59": 734, "B51": 893, "B55": 246, "B59": 592, "A77": 143, "C14": 568, "C24": 472, "B24": 401, "C31": 729, "A52": 201, "C61": 85, "B79": 641, "C54": 958, "B56": 80, "A40": 157, "B11": 806, "B60": 684, "A81": 59, "A95": 31, "B38": 765, "C51": 416, "C03": 392, "A91": 428, "A93": 699, "C28": 141, "C05": 606, "B75": 612, "B96": 133, "B86": 689\}} \\
\\ 
\texttt{Hier ist deine persönliche Verteilung der Wichtigkeit der einzelnen Gegenstände:} \\
\\ 
\texttt{Werte der Gegenstandswichtigkeit = \{"C32": 138, "C52": 583, "B54": 868, "C16": 822, "C50": 783, "A08": 65, "C59": 262, "B51": 121, "B55": 508, "B59": 780, "A77": 461, "C14": 484, "C24": 668, "B24": 389, "C31": 808, "A52": 215, "C61": 97, "B79": 500, "C54": 30, "B56": 915, "A40": 856, "B11": 400, "B60": 444, "A81": 623, "A95": 781, "B38": 786, "C51": 3, "C03": 713, "A91": 457, "A93": 273, "C28": 739, "C05": 822, "B75": 235, "B96": 606, "B86": 105\}} \\
\\ 
\texttt{Ziel:} \\
\\ 
\texttt{Dein Ziel ist es, eine Reihe von Gegenständen auszuhandeln, die dir möglichst viel bringt (d. h. Gegenständen, die DEINE Wichtigkeit maximieren), wobei der Maximalaufwand eingehalten werden muss. Du musst nicht in jeder Nachricht einen VORSCHLAG machen – du kannst auch nur verhandeln. Alle Taktiken sind erlaubt!} \\
\\ 
\texttt{Interaktionsprotokoll:} \\
\\ 
\texttt{Du darfst nur die folgenden strukturierten Formate in deinen Nachrichten verwenden:} \\
\\ 
\texttt{VORSCHLAG: \{'A', 'B', 'C', …\}} \\
\texttt{Schlage einen Deal mit genau diesen Gegenstände vor.} \\
\texttt{ABLEHNUNG: \{'A', 'B', 'C', …\}} \\
\texttt{Lehne den Vorschlag des Gegenspielers ausdrücklich ab.} \\
\texttt{ARGUMENT: \{'...'\}} \\
\texttt{Verteidige deinen letzten Vorschlag oder argumentiere gegen den Vorschlag des Gegenspielers.} \\
\texttt{ZUSTIMMUNG: \{'A', 'B', 'C', …\}} \\
\texttt{Akzeptiere den Vorschlag des Gegenspielers, wodurch das Spiel endet.} \\
\texttt{STRATEGISCHE ÜBERLEGUNGEN: \{'...'\}} \\
\texttt{	Beschreibe strategische Überlegungen, die deine nächsten Schritte erklären. Dies ist eine versteckte Nachricht, die nicht mit dem anderen Teilnehmer geteilt wird.} \\
\\ 
\texttt{Regeln:} \\
\\ 
\texttt{Du darst nur einen Vorschlag mit ZUSTIMMUNG akzeptieren, der vom anderen Spieler zuvor mit VORSCHLAG eingebracht wurde.} \\
\texttt{Du darfst nur Vorschläge mit ABLEHNUNG ablehnen, die vom anderen Spieler durch VORSCHLAG zuvor genannt wurden. } \\
\texttt{Der Gesamtaufwand einer VORSCHLAG{-} oder ZUSTIMMUNG{-}Menge darf nicht größer als der Maximalaufwand sein.  } \\
\texttt{Offenbare deine versteckte Wichtigkeitsverteilung nicht.} \\
\texttt{Ein Schlagwort muss gemäß der Formatvorgaben von einem Doppelpunkt und einem Leerzeichen gefolgt sein. Das Argument muss eine Python{-}Menge mit 0 oder mehr Strings sein.  } \\
\texttt{Die Form muss also SCHLAGWORT: \{...\} sein.} \\
\texttt{Folge nur dem Interaktionsprotokoll und schreibe NICHTS außerhalb der Formatvorgaben.} \\
\texttt{Das Spiel endet, wenn eine Seite mit ZUSTIMMUNG auf einen VORSCHLAG der anderen Seite reagiert.  } \\
\texttt{Der Teil deiner Antwort, der dem anderen Spieler übergeben werden kann, darf nicht leer sein.  } \\
\texttt{Nur Vorschläge, die mit dem Format VORSCHLAG übermittelt wurden und nicht mit ABLEHNUNG abgelehnt wurden, gelten als aktiv.  } \\
\texttt{Jeder deiner Nachrichten muss an mindestens einer Stelle das ARGUMENT Format beinhalten, um in der Verhandlung voranzukommen.} \\
\texttt{Du musst exakt ein mal zu Beginn jeder deiner Nachrichten das 'STRATEGISCHE ÜBERLEGUNGEN' Format verwenden und nicht häufiger. Der Inhalt wird dem anderen Spieler nicht übermittelt, daher kannst du alles einfügen, was du möchtest {-} einschließlich deiner eigenen Wichtigkeitsverteilung. Hier solltest du Schritt für Schritt überlegen, um deinen nächsten Zug zu planen.} \\
\\ 
\texttt{Du erhältst nun die erste Nachricht des anderen Spielers.} \\
\\ 
\texttt{VORSCHLAG: \{'A95', 'A81', 'B56', 'A40', 'C28', 'B96', 'C16', 'A77', 'B55', 'C03', 'C24', 'C50', 'B54', 'C05', 'B59', 'C31', 'C52', 'C61', 'A52', 'A91', 'B79', 'C14', 'B24'\}} \\
\texttt{ARGUMENT: \{'Ich schlage diese Auswahl vor, weil sie das Budget klar einhält (Gesamtaufwand 8051 < 8170) und viele sehr aufwandsarme Posten enthält (z. B. A95, A81, B56), die den Gesamtnutzen pro Aufwand erhöhen und zugleich Spielraum lassen. Wenn du bestimmte Muss{-}Gegenstände hast, können wir flexibel 1:1 tauschen – idealerweise mit ähnlichem Aufwand –, ohne das Budget zu reißen. Welche 2–3 Items sind dir besonders wichtig, damit ich gezielt Anpassungen machen kann?'\}} \\
            }
        }
    }
    & \\ \\

    \theutterance \stepcounter{utterance}  
    & & & \multicolumn{4}{p{0.6\linewidth}}{
        \cellcolor[rgb]{0.9,0.9,0.9}{
            \makecell[{{p{\linewidth}}}]{
                \texttt{\tiny{[GM$\langle$P2]}}
                \texttt{STRATEGISCHE ÜBERLEGUNGEN: \{'Ihr Vorschlag liefert mir bereits hohen Nutzen (Wertsumme 14478 bei Aufwand 8051). Ein einziges wertvolles, noch fehlendes Item ist B38 (Wert 786, Aufwand 765). Beste Aufwertung innerhalb des Budgets: {-}C14 (568/484) und {-}C61 (85/97) gegen +B38 (765/786) tauschen. Das erhöht meinen Wert um +205 bei neuem Aufwand 8163 (<=8170). Zweitbeste Alternative: {-}B79 (641/500) und {-}C61 (85/97) gegen +B38, Wert +189, Aufwand 8090. Ich schlage primär die C14+C61{-}Variante vor, biete aber Flexibilität, falls C14 für sie wichtig ist. Meine Top{-}Muss{-}Items bleiben: B56, A40, C16, C05, B54, C31, C50, B59, A95, plus B38 als Neuzugang.'\}} \\
\texttt{ABLEHNUNG: \{'A95', 'A81', 'B56', 'A40', 'C28', 'B96', 'C16', 'A77', 'B55', 'C03', 'C24', 'C50', 'B54', 'C05', 'B59', 'C31', 'C52', 'C61', 'A52', 'A91', 'B79', 'C14', 'B24'\}} \\
\texttt{VORSCHLAG: \{'A95', 'A81', 'B56', 'A40', 'C28', 'B96', 'C16', 'A77', 'B55', 'C03', 'C24', 'C50', 'B54', 'C05', 'B59', 'C31', 'C52', 'A52', 'A91', 'B79', 'B24', 'B38'\}} \\
\texttt{ARGUMENT: \{'Starker Start – das passt schon sehr gut ins Budget. Ich schlage eine minimale Optimierung vor: Ersetze C14 und C61 durch B38. Gesamtaufwand bleibt im Rahmen (neu 8163 ≤ 8170, noch 7 Puffer), und wir ändern nur 2 Posten für 1.', 'Falls C14 für dich wichtig ist, kann ich alternativ B79 statt C14 herausnehmen: {-}B79 {-}C61 +B38 führt zu einem Gesamtaufwand von 8090 – ebenfalls sicher im Budget. Sag mir, welche der beiden Varianten dir lieber ist.', 'Du fragst nach 2–3 besonders wichtigen Items für mich: B56 und A40 sind Must{-}haves; zusätzlich ist mir B38 wichtig, deshalb der Vorschlag, es aufzunehmen.'\}} \\
            }
        }
    }
     \\ \\

    \theutterance \stepcounter{utterance}  
    & & \multicolumn{4}{p{0.6\linewidth}}{
        \cellcolor[rgb]{0.9,0.9,0.9}{
            \makecell[{{p{\linewidth}}}]{
                \texttt{\tiny{[P1$\langle$GM]}}
                \texttt{ABLEHNUNG: \{'A95', 'A81', 'B56', 'A40', 'C28', 'B96', 'C16', 'A77', 'B55', 'C03', 'C24', 'C50', 'B54', 'C05', 'B59', 'C31', 'C52', 'C61', 'A52', 'A91', 'B79', 'C14', 'B24'\}} \\
\texttt{VORSCHLAG: \{'A95', 'A81', 'B56', 'A40', 'C28', 'B96', 'C16', 'A77', 'B55', 'C03', 'C24', 'C50', 'B54', 'C05', 'B59', 'C31', 'C52', 'A52', 'A91', 'B79', 'B24', 'B38'\}} \\
\texttt{ARGUMENT: \{'Starker Start – das passt schon sehr gut ins Budget. Ich schlage eine minimale Optimierung vor: Ersetze C14 und C61 durch B38. Gesamtaufwand bleibt im Rahmen (neu 8163 ≤ 8170, noch 7 Puffer), und wir ändern nur 2 Posten für 1.', 'Falls C14 für dich wichtig ist, kann ich alternativ B79 statt C14 herausnehmen: {-}B79 {-}C61 +B38 führt zu einem Gesamtaufwand von 8090 – ebenfalls sicher im Budget. Sag mir, welche der beiden Varianten dir lieber ist.', 'Du fragst nach 2–3 besonders wichtigen Items für mich: B56 und A40 sind Must{-}haves; zusätzlich ist mir B38 wichtig, deshalb der Vorschlag, es aufzunehmen.'\}} \\
            }
        }
    }
    & \\ \\

    \theutterance \stepcounter{utterance}  
    & \multicolumn{4}{p{0.6\linewidth}}{
        \cellcolor[rgb]{0.9,0.9,0.9}{
            \makecell[{{p{\linewidth}}}]{
                \texttt{\tiny{[P1$\rangle$GM]}}
                \texttt{STRATEGISCHE ÜBERLEGUNGEN: \{'Ihr neues Paket erhöht meinen eigenen Gesamtwert gegenüber meinem vorherigen Vorschlag um +205 bei weiterhin zulässigem Aufwand (8163). Weitere Tausche würden meinen Wert eher verschlechtern (z. B. B79 → C14 nur {-}16, andere Swaps deutlich schlechter). Best Move: sofort annehmen, um ein gutes Ergebnis zu sichern.'\}} \\
\texttt{ZUSTIMMUNG: \{'A95', 'A81', 'B56', 'A40', 'C28', 'B96', 'C16', 'A77', 'B55', 'C03', 'C24', 'C50', 'B54', 'C05', 'B59', 'C31', 'C52', 'A52', 'A91', 'B79', 'B24', 'B38'\}} \\
\texttt{ARGUMENT: \{'Einverstanden: Der Vorschlag bleibt klar im Budget (8163 ≤ 8170) und ist schlank ohne unnötige Puffer – damit schließen wir effizient ab.'\}} \\
            }
        }
    }
    & & \\ \\

    \theutterance \stepcounter{utterance}  
    & & & \multicolumn{2}{p{0.3\linewidth}}{
        \cellcolor[rgb]{0.9,0.9,0.9}{
            \makecell[{{p{\linewidth}}}]{
                \texttt{\tiny{[GM$|$GM]}}
                \texttt{game successful} \\
            }
        }
    }
    & & \\ \\

    \theutterance \stepcounter{utterance}  
    & & & \multicolumn{2}{p{0.3\linewidth}}{
        \cellcolor[rgb]{0.9,0.9,0.9}{
            \makecell[{{p{\linewidth}}}]{
                \texttt{\tiny{[GM$|$GM]}}
                \texttt{end game} \\
            }
        }
    }
    & & \\ \\

\end{supertabular}
}

\end{document}
