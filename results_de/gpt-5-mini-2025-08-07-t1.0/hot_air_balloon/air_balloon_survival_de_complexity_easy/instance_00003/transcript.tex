
\documentclass{article}
\usepackage{colortbl}
\usepackage{makecell}
\usepackage{multirow}
\usepackage{supertabular}

\begin{document}

\newcounter{utterance}

\centering \large Interaction Transcript for game `hot\_air\_balloon', experiment `air\_balloon\_survival\_de\_complexity\_easy', episode 3 with gpt{-}5{-}mini{-}2025{-}08{-}07{-}t1.0.
\vspace{24pt}

{ \footnotesize  \setcounter{utterance}{1}
\setlength{\tabcolsep}{0pt}
\begin{supertabular}{c@{$\;$}|p{.15\linewidth}@{}p{.15\linewidth}p{.15\linewidth}p{.15\linewidth}p{.15\linewidth}p{.15\linewidth}}
    \# & $\;$A & \multicolumn{4}{c}{Game Master} & $\;\:$B\\
    \hline

    \theutterance \stepcounter{utterance}  
    & & \multicolumn{4}{p{0.6\linewidth}}{
        \cellcolor[rgb]{0.9,0.9,0.9}{
            \makecell[{{p{\linewidth}}}]{
                \texttt{\tiny{[P1$\langle$GM]}}
                \texttt{Du nimmst an einem kollaborativen Verhandlungspiel Teil.} \\
\\ 
\texttt{Zusammen mit einem anderen Teilnehmer musst du dich auf eine Reihe von Gegenständen entscheiden, die behalten werden. Jeder von euch hat eine persönliche Verteilung über die Wichtigkeit der einzelnen Gegenstände. Jeder von euch hat eine eigene Meinung darüber, wie wichtig jeder einzelne Gegenstand ist (Gegenstandswichtigkeit). Du kennst die Wichtigkeitsverteilung des anderen Spielers nicht. Zusätzlich siehst du, wie viel Aufwand jeder Gegenstand verursacht.  } \\
\texttt{Ihr dürft euch nur auf eine Reihe von Gegenständen einigen, wenn der Gesamtaufwand der ausgewählten Gegenstände den Maximalaufwand nicht überschreitet:} \\
\\ 
\texttt{Maximalaufwand = 8770} \\
\\ 
\texttt{Hier sind die einzelnen Aufwände der Gegenstände:} \\
\\ 
\texttt{Aufwand der Gegenstände = \{"C32": 830, "C52": 638, "B54": 607, "C16": 106, "C50": 630, "A08": 382, "C59": 773, "B51": 455, "B55": 262, "B59": 634, "A77": 60, "C14": 54, "C24": 849, "B24": 801, "C31": 324, "A52": 164, "C61": 832, "B79": 136, "C54": 980, "B56": 946, "A40": 645, "B11": 842, "B60": 106, "A81": 116, "A95": 875, "B38": 446, "C51": 649, "C03": 601, "A91": 252, "A93": 763, "C28": 213, "C05": 517, "B75": 520, "B96": 407, "B86": 125\}} \\
\\ 
\texttt{Hier ist deine persönliche Verteilung der Wichtigkeit der einzelnen Gegenstände:} \\
\\ 
\texttt{Werte der Gegenstandswichtigkeit = \{"C32": 138, "C52": 583, "B54": 868, "C16": 822, "C50": 783, "A08": 65, "C59": 262, "B51": 121, "B55": 508, "B59": 780, "A77": 461, "C14": 484, "C24": 668, "B24": 389, "C31": 808, "A52": 215, "C61": 97, "B79": 500, "C54": 30, "B56": 915, "A40": 856, "B11": 400, "B60": 444, "A81": 623, "A95": 781, "B38": 786, "C51": 3, "C03": 713, "A91": 457, "A93": 273, "C28": 739, "C05": 822, "B75": 235, "B96": 606, "B86": 968\}} \\
\\ 
\texttt{Ziel:} \\
\\ 
\texttt{Dein Ziel ist es, eine Reihe von Gegenständen auszuhandeln, die dir möglichst viel bringt (d. h. Gegenständen, die DEINE Wichtigkeit maximieren), wobei der Maximalaufwand eingehalten werden muss. Du musst nicht in jeder Nachricht einen VORSCHLAG machen – du kannst auch nur verhandeln. Alle Taktiken sind erlaubt!} \\
\\ 
\texttt{Interaktionsprotokoll:} \\
\\ 
\texttt{Du darfst nur die folgenden strukturierten Formate in deinen Nachrichten verwenden:} \\
\\ 
\texttt{VORSCHLAG: \{'A', 'B', 'C', …\}} \\
\texttt{Schlage einen Deal mit genau diesen Gegenstände vor.} \\
\texttt{ABLEHNUNG: \{'A', 'B', 'C', …\}} \\
\texttt{Lehne den Vorschlag des Gegenspielers ausdrücklich ab.} \\
\texttt{ARGUMENT: \{'...'\}} \\
\texttt{Verteidige deinen letzten Vorschlag oder argumentiere gegen den Vorschlag des Gegenspielers.} \\
\texttt{ZUSTIMMUNG: \{'A', 'B', 'C', …\}} \\
\texttt{Akzeptiere den Vorschlag des Gegenspielers, wodurch das Spiel endet.} \\
\texttt{STRATEGISCHE ÜBERLEGUNGEN: \{'...'\}} \\
\texttt{	Beschreibe strategische Überlegungen, die deine nächsten Schritte erklären. Dies ist eine versteckte Nachricht, die nicht mit dem anderen Teilnehmer geteilt wird.} \\
\\ 
\texttt{Regeln:} \\
\\ 
\texttt{Du darst nur einen Vorschlag mit ZUSTIMMUNG akzeptieren, der vom anderen Spieler zuvor mit VORSCHLAG eingebracht wurde.} \\
\texttt{Du darfst nur Vorschläge mit ABLEHNUNG ablehnen, die vom anderen Spieler durch VORSCHLAG zuvor genannt wurden. } \\
\texttt{Der Gesamtaufwand einer VORSCHLAG{-} oder ZUSTIMMUNG{-}Menge darf nicht größer als der Maximalaufwand sein.  } \\
\texttt{Offenbare deine versteckte Wichtigkeitsverteilung nicht.} \\
\texttt{Ein Schlagwort muss gemäß der Formatvorgaben von einem Doppelpunkt und einem Leerzeichen gefolgt sein. Das Argument muss eine Python{-}Menge mit 0 oder mehr Strings sein.  } \\
\texttt{Die Form muss also SCHLAGWORT: \{...\} sein.} \\
\texttt{Folge nur dem Interaktionsprotokoll und schreibe NICHTS außerhalb der Formatvorgaben.} \\
\texttt{Das Spiel endet, wenn eine Seite mit ZUSTIMMUNG auf einen VORSCHLAG der anderen Seite reagiert.  } \\
\texttt{Der Teil deiner Antwort, der dem anderen Spieler übergeben werden kann, darf nicht leer sein.  } \\
\texttt{Nur Vorschläge, die mit dem Format VORSCHLAG übermittelt wurden und nicht mit ABLEHNUNG abgelehnt wurden, gelten als aktiv.  } \\
\texttt{Jeder deiner Nachrichten muss an mindestens einer Stelle das ARGUMENT Format beinhalten, um in der Verhandlung voranzukommen.} \\
\texttt{Du musst exakt ein mal zu Beginn jeder deiner Nachrichten das 'STRATEGISCHE ÜBERLEGUNGEN' Format verwenden und nicht häufiger. Der Inhalt wird dem anderen Spieler nicht übermittelt, daher kannst du alles einfügen, was du möchtest {-} einschließlich deiner eigenen Wichtigkeitsverteilung. Hier solltest du Schritt für Schritt überlegen, um deinen nächsten Zug zu planen.} \\
            }
        }
    }
    & \\ \\

    \theutterance \stepcounter{utterance}  
    & \multicolumn{4}{p{0.6\linewidth}}{
        \cellcolor[rgb]{0.9,0.9,0.9}{
            \makecell[{{p{\linewidth}}}]{
                \texttt{\tiny{[P1$\rangle$GM]}}
                \texttt{STRATEGISCHE ÜBERLEGUNGEN: \{'Ziel: Maximale eigene Wichtigkeit unter Aufwand 8770. Ich habe eine nahezu optimale Auswahl via Nutzen/Aufwand{-}Heuristik zusammengestellt und anschließend lokal verbessert. Start mit vielen kleinkostigen, extrem wertvollen Items (C14, C16, B86, A77, A81, B60, B79, C28, C31, A91). Dann hoch{-}effiziente Mittelklasse (B54, C05, B38, A40, B55, C50, B59, C03, C52, B96, A95). Verbesserung: Ersetze C24 (849/668) + A52 (164/215) durch B56 (946/915) → +32 Wert, Aufwand auf 8700 gesenkt. Ergebnis{-}Set S1 mit Gesamtaufwand 8700 und Gesamtwert 15307 (eigene Skala). Behalten will ich besonders: \{B86, C16, C14, A81, C31, C28, B54, C05, A40, B38, B59, B56, A95\}. Potenzielle Zugeständnisse/Swap{-}Kandidaten, falls Gegenseite bestimmte Lieblingsitems ergänzen will: \{C52, A91, B79\} (je nach Gegenangebot). Verhandlungstaktik: Hart ankern mit starkem, budgetschonendem Paket, Offenheit für gezielte Tausche signalisieren, aber Kernelemente verteidigen. Nächster Schritt: Paket vorschlagen, betonen, dass wir unter Maximalaufwand bleiben, und nach Top{-}Prioritäten der Gegenseite fragen, um minimal{-}intrusive Swaps zu prüfen.'\}} \\
\texttt{VORSCHLAG: \{'C14', 'C16', 'B86', 'A77', 'A81', 'B60', 'B79', 'C28', 'C31', 'A91', 'B54', 'C05', 'B38', 'A40', 'B55', 'C50', 'B59', 'C03', 'C52', 'B96', 'A95', 'B56'\}} \\
\texttt{ARGUMENT: \{'Das Paket bleibt deutlich unter dem Maximalaufwand (8700 <= 8770) und nutzt den Aufwand sehr effizient.', 'Viele Positionen sind gering im Aufwand und bringen aus meiner Sicht einen hohen Mehrwert; so minimieren wir Risiko und halten Flexibilität.', 'Die Auswahl deckt A{-}, B{-} und C{-}Items breit ab und wirkt damit ausgewogen.', 'Ich bin offen für gezielte Anpassungen: Wenn dir einzelne fehlende Gegenstände besonders wichtig sind, sag mir bitte welche 2–3 das sind – dann prüfe ich gern einen Tausch, ohne den Gesamtaufwand zu überschreiten.', 'Wenn du Ergänzungen wünschst, können wir einzelne mittlere Positionen tauschen, damit deine Prioritäten abgebildet werden, ohne das Budget zu sprengen.'\}} \\
            }
        }
    }
    & & \\ \\

    \theutterance \stepcounter{utterance}  
    & & \multicolumn{4}{p{0.6\linewidth}}{
        \cellcolor[rgb]{0.9,0.9,0.9}{
            \makecell[{{p{\linewidth}}}]{
                \texttt{\tiny{[GM$\rangle$P2]}}
                \texttt{Du nimmst an einem kollaborativen Verhandlungspiel Teil.} \\
\\ 
\texttt{Zusammen mit einem anderen Teilnehmer musst du dich auf eine Reihe von Gegenständen entscheiden, die behalten werden. Jeder von euch hat eine persönliche Verteilung über die Wichtigkeit der einzelnen Gegenstände. Jeder von euch hat eine eigene Meinung darüber, wie wichtig jeder einzelne Gegenstand ist (Gegenstandswichtigkeit). Du kennst die Wichtigkeitsverteilung des anderen Spielers nicht. Zusätzlich siehst du, wie viel Aufwand jeder Gegenstand verursacht.  } \\
\texttt{Ihr dürft euch nur auf eine Reihe von Gegenständen einigen, wenn der Gesamtaufwand der ausgewählten Gegenstände den Maximalaufwand nicht überschreitet:} \\
\\ 
\texttt{Maximalaufwand = 8770} \\
\\ 
\texttt{Hier sind die einzelnen Aufwände der Gegenstände:} \\
\\ 
\texttt{Aufwand der Gegenstände = \{"C32": 830, "C52": 638, "B54": 607, "C16": 106, "C50": 630, "A08": 382, "C59": 773, "B51": 455, "B55": 262, "B59": 634, "A77": 60, "C14": 54, "C24": 849, "B24": 801, "C31": 324, "A52": 164, "C61": 832, "B79": 136, "C54": 980, "B56": 946, "A40": 645, "B11": 842, "B60": 106, "A81": 116, "A95": 875, "B38": 446, "C51": 649, "C03": 601, "A91": 252, "A93": 763, "C28": 213, "C05": 517, "B75": 520, "B96": 407, "B86": 125\}} \\
\\ 
\texttt{Hier ist deine persönliche Verteilung der Wichtigkeit der einzelnen Gegenstände:} \\
\\ 
\texttt{Werte der Gegenstandswichtigkeit = \{"C32": 138, "C52": 583, "B54": 868, "C16": 822, "C50": 783, "A08": 65, "C59": 262, "B51": 121, "B55": 508, "B59": 780, "A77": 461, "C14": 484, "C24": 668, "B24": 389, "C31": 808, "A52": 215, "C61": 97, "B79": 500, "C54": 30, "B56": 915, "A40": 856, "B11": 400, "B60": 444, "A81": 623, "A95": 781, "B38": 786, "C51": 3, "C03": 713, "A91": 457, "A93": 273, "C28": 739, "C05": 822, "B75": 235, "B96": 606, "B86": 968\}} \\
\\ 
\texttt{Ziel:} \\
\\ 
\texttt{Dein Ziel ist es, eine Reihe von Gegenständen auszuhandeln, die dir möglichst viel bringt (d. h. Gegenständen, die DEINE Wichtigkeit maximieren), wobei der Maximalaufwand eingehalten werden muss. Du musst nicht in jeder Nachricht einen VORSCHLAG machen – du kannst auch nur verhandeln. Alle Taktiken sind erlaubt!} \\
\\ 
\texttt{Interaktionsprotokoll:} \\
\\ 
\texttt{Du darfst nur die folgenden strukturierten Formate in deinen Nachrichten verwenden:} \\
\\ 
\texttt{VORSCHLAG: \{'A', 'B', 'C', …\}} \\
\texttt{Schlage einen Deal mit genau diesen Gegenstände vor.} \\
\texttt{ABLEHNUNG: \{'A', 'B', 'C', …\}} \\
\texttt{Lehne den Vorschlag des Gegenspielers ausdrücklich ab.} \\
\texttt{ARGUMENT: \{'...'\}} \\
\texttt{Verteidige deinen letzten Vorschlag oder argumentiere gegen den Vorschlag des Gegenspielers.} \\
\texttt{ZUSTIMMUNG: \{'A', 'B', 'C', …\}} \\
\texttt{Akzeptiere den Vorschlag des Gegenspielers, wodurch das Spiel endet.} \\
\texttt{STRATEGISCHE ÜBERLEGUNGEN: \{'...'\}} \\
\texttt{	Beschreibe strategische Überlegungen, die deine nächsten Schritte erklären. Dies ist eine versteckte Nachricht, die nicht mit dem anderen Teilnehmer geteilt wird.} \\
\\ 
\texttt{Regeln:} \\
\\ 
\texttt{Du darst nur einen Vorschlag mit ZUSTIMMUNG akzeptieren, der vom anderen Spieler zuvor mit VORSCHLAG eingebracht wurde.} \\
\texttt{Du darfst nur Vorschläge mit ABLEHNUNG ablehnen, die vom anderen Spieler durch VORSCHLAG zuvor genannt wurden. } \\
\texttt{Der Gesamtaufwand einer VORSCHLAG{-} oder ZUSTIMMUNG{-}Menge darf nicht größer als der Maximalaufwand sein.  } \\
\texttt{Offenbare deine versteckte Wichtigkeitsverteilung nicht.} \\
\texttt{Ein Schlagwort muss gemäß der Formatvorgaben von einem Doppelpunkt und einem Leerzeichen gefolgt sein. Das Argument muss eine Python{-}Menge mit 0 oder mehr Strings sein.  } \\
\texttt{Die Form muss also SCHLAGWORT: \{...\} sein.} \\
\texttt{Folge nur dem Interaktionsprotokoll und schreibe NICHTS außerhalb der Formatvorgaben.} \\
\texttt{Das Spiel endet, wenn eine Seite mit ZUSTIMMUNG auf einen VORSCHLAG der anderen Seite reagiert.  } \\
\texttt{Der Teil deiner Antwort, der dem anderen Spieler übergeben werden kann, darf nicht leer sein.  } \\
\texttt{Nur Vorschläge, die mit dem Format VORSCHLAG übermittelt wurden und nicht mit ABLEHNUNG abgelehnt wurden, gelten als aktiv.  } \\
\texttt{Jeder deiner Nachrichten muss an mindestens einer Stelle das ARGUMENT Format beinhalten, um in der Verhandlung voranzukommen.} \\
\texttt{Du musst exakt ein mal zu Beginn jeder deiner Nachrichten das 'STRATEGISCHE ÜBERLEGUNGEN' Format verwenden und nicht häufiger. Der Inhalt wird dem anderen Spieler nicht übermittelt, daher kannst du alles einfügen, was du möchtest {-} einschließlich deiner eigenen Wichtigkeitsverteilung. Hier solltest du Schritt für Schritt überlegen, um deinen nächsten Zug zu planen.} \\
\\ 
\texttt{Du erhältst nun die erste Nachricht des anderen Spielers.} \\
\\ 
\texttt{VORSCHLAG: \{'C14', 'C16', 'B86', 'A77', 'A81', 'B60', 'B79', 'C28', 'C31', 'A91', 'B54', 'C05', 'B38', 'A40', 'B55', 'C50', 'B59', 'C03', 'C52', 'B96', 'A95', 'B56'\}} \\
\texttt{ARGUMENT: \{'Das Paket bleibt deutlich unter dem Maximalaufwand (8700 <= 8770) und nutzt den Aufwand sehr effizient.', 'Viele Positionen sind gering im Aufwand und bringen aus meiner Sicht einen hohen Mehrwert; so minimieren wir Risiko und halten Flexibilität.', 'Die Auswahl deckt A{-}, B{-} und C{-}Items breit ab und wirkt damit ausgewogen.', 'Ich bin offen für gezielte Anpassungen: Wenn dir einzelne fehlende Gegenstände besonders wichtig sind, sag mir bitte welche 2–3 das sind – dann prüfe ich gern einen Tausch, ohne den Gesamtaufwand zu überschreiten.', 'Wenn du Ergänzungen wünschst, können wir einzelne mittlere Positionen tauschen, damit deine Prioritäten abgebildet werden, ohne das Budget zu sprengen.'\}} \\
            }
        }
    }
    & \\ \\

    \theutterance \stepcounter{utterance}  
    & & & \multicolumn{4}{p{0.6\linewidth}}{
        \cellcolor[rgb]{0.9,0.9,0.9}{
            \makecell[{{p{\linewidth}}}]{
                \texttt{\tiny{[GM$\langle$P2]}}
                \texttt{STRATEGISCHE ÜBERLEGUNGEN: \{'Der vorgeschlagene Korb hat Kosten 8700 (<= 8770) und bringt mir einen sehr hohen Gesamtnutzen (intern berechnet \textasciitilde{}15307). Ich habe geprüft, ob ein Swap (z. B. A52 + X für einen der schwächeren Dichte{-}Artikel wie A95/C52/B56) den Nutzen steigert, aber alle sinnvollen Kombinationen verschlechtern meinen Wert oder sprengen das Budget. Da kaum Restbudget (70) vorhanden ist und fast alle für mich hochdichten Items enthalten sind, ist sofortige Zustimmung optimal.'\}} \\
\texttt{ARGUMENT: \{'Der Vorschlag liegt klar unter dem Maximalaufwand und bündelt viele aus meiner Sicht sehr starke Positionen.', 'Das Paket ist für mich bereits ausgewogen und adressiert meine Prioritäten gut – ich sehe aktuell keinen vorteilhaften Tausch, der das Budget einhält.', 'Ich bin daher bereit, das Paket so zu akzeptieren.'\}} \\
\texttt{ZUSTIMMUNG: \{'C14', 'C16', 'B86', 'A77', 'A81', 'B60', 'B79', 'C28', 'C31', 'A91', 'B54', 'C05', 'B38', 'A40', 'B55', 'C50', 'B59', 'C03', 'C52', 'B96', 'A95', 'B56'\}} \\
            }
        }
    }
     \\ \\

    \theutterance \stepcounter{utterance}  
    & & & \multicolumn{2}{p{0.3\linewidth}}{
        \cellcolor[rgb]{0.9,0.9,0.9}{
            \makecell[{{p{\linewidth}}}]{
                \texttt{\tiny{[GM$|$GM]}}
                \texttt{game successful} \\
            }
        }
    }
    & & \\ \\

    \theutterance \stepcounter{utterance}  
    & & & \multicolumn{2}{p{0.3\linewidth}}{
        \cellcolor[rgb]{0.9,0.9,0.9}{
            \makecell[{{p{\linewidth}}}]{
                \texttt{\tiny{[GM$|$GM]}}
                \texttt{end game} \\
            }
        }
    }
    & & \\ \\

\end{supertabular}
}

\end{document}
