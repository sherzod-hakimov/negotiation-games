
\documentclass{article}
\usepackage{colortbl}
\usepackage{makecell}
\usepackage{multirow}
\usepackage{supertabular}

\begin{document}

\newcounter{utterance}

\twocolumn

{ \footnotesize  \setcounter{utterance}{1}
\setlength{\tabcolsep}{0pt}
\begin{supertabular}{c@{$\;$}|p{.15\linewidth}@{}p{.15\linewidth}p{.15\linewidth}p{.15\linewidth}p{.15\linewidth}p{.15\linewidth}}

    \# & $\;$A & \multicolumn{4}{c}{Game Master} & $\;\:$B\\
    \hline 

    \theutterance \stepcounter{utterance}  

    & & \multicolumn{4}{p{0.6\linewidth}}{\cellcolor[rgb]{0.9,0.9,0.9}{%
	\makecell[{{p{\linewidth}}}]{% 
	  \tt {\tiny [A$\langle$GM]}  
	 Du nimmst an einem kollaborativen Verhandlungspiel Teil.\\ \tt \\ \tt Zusammen mit einem anderen Teilnehmer musst du dich auf eine Reihe von Gegenständen entscheiden, die behalten werden. Jeder von euch hat eine persönliche Verteilung über die Wichtigkeit der einzelnen Gegenstände. Jeder von euch hat eine eigene Meinung darüber, wie wichtig jeder einzelne Gegenstand ist (Gegenstandswichtigkeit). Du kennst die Wichtigkeitsverteilung des anderen Spielers nicht. Zusätzlich siehst du, wie viel Aufwand jeder Gegenstand verursacht.  \\ \tt Ihr dürft euch nur auf eine Reihe von Gegenständen einigen, wenn der Gesamtaufwand der ausgewählten Gegenstände den Maximalaufwand nicht überschreitet:\\ \tt \\ \tt Maximalaufwand = 7605\\ \tt \\ \tt Hier sind die einzelnen Aufwände der Gegenstände:\\ \tt \\ \tt Aufwand der Gegenstände = {"C32": 369, "C52": 951, "B54": 156, "C16": 266, "C50": 581, "A08": 287, "C59": 866, "B51": 180, "B55": 800, "B59": 739, "A77": 637, "C14": 86, "C24": 749, "B24": 370, "C31": 345, "A52": 951, "C61": 147, "B79": 265, "C54": 262, "B56": 259, "A40": 358, "B11": 394, "B60": 286, "A81": 580, "A95": 479, "B38": 14, "C51": 153, "C03": 977, "A91": 134, "A93": 982, "C28": 259, "C05": 232, "B75": 202, "B96": 73, "B86": 821}\\ \tt \\ \tt Hier ist deine persönliche Verteilung der Wichtigkeit der einzelnen Gegenstände:\\ \tt \\ \tt Werte der Gegenstandswichtigkeit = {"C32": 138, "C52": 583, "B54": 868, "C16": 822, "C50": 783, "A08": 65, "C59": 262, "B51": 121, "B55": 508, "B59": 780, "A77": 461, "C14": 484, "C24": 668, "B24": 389, "C31": 808, "A52": 215, "C61": 97, "B79": 500, "C54": 30, "B56": 915, "A40": 856, "B11": 400, "B60": 444, "A81": 623, "A95": 781, "B38": 786, "C51": 3, "C03": 713, "A91": 457, "A93": 273, "C28": 739, "C05": 822, "B75": 235, "B96": 606, "B86": 968}\\ \tt \\ \tt Ziel:\\ \tt \\ \tt Dein Ziel ist es, eine Reihe von Gegenständen auszuhandeln, die dir möglichst viel bringt (d. h. Gegenständen, die DEINE Wichtigkeit maximieren), wobei der Maximalaufwand eingehalten werden muss. Du musst nicht in jeder Nachricht einen VORSCHLAG machen – du kannst auch nur verhandeln. Alle Taktiken sind erlaubt!\\ \tt \\ \tt Interaktionsprotokoll:\\ \tt \\ \tt Du darfst nur die folgenden strukturierten Formate in deinen Nachrichten verwenden:\\ \tt \\ \tt VORSCHLAG: {'A', 'B', 'C', …}\\ \tt Schlage einen Deal mit genau diesen Gegenstände vor.\\ \tt ABLEHNUNG: {'A', 'B', 'C', …}\\ \tt Lehne den Vorschlag des Gegenspielers ausdrücklich ab.\\ \tt ARGUMENT: {'...'}\\ \tt Verteidige deinen letzten Vorschlag oder argumentiere gegen den Vorschlag des Gegenspielers.\\ \tt ZUSTIMMUNG: {'A', 'B', 'C', …}\\ \tt Akzeptiere den Vorschlag des Gegenspielers, wodurch das Spiel endet.\\ \tt STRATEGISCHE ÜBERLEGUNGEN: {'...'}\\ \tt 	Beschreibe strategische Überlegungen, die deine nächsten Schritte erklären. Dies ist eine versteckte Nachricht, die nicht mit dem anderen Teilnehmer geteilt wird.\\ \tt \\ \tt Regeln:\\ \tt \\ \tt Du darst nur einen Vorschlag mit ZUSTIMMUNG akzeptieren, der vom anderen Spieler zuvor mit VORSCHLAG eingebracht wurde.\\ \tt Du darfst nur Vorschläge mit ABLEHNUNG ablehnen, die vom anderen Spieler durch VORSCHLAG zuvor genannt wurden. \\ \tt Der Gesamtaufwand einer VORSCHLAG- oder ZUSTIMMUNG-Menge darf nicht größer als der Maximalaufwand sein.  \\ \tt Offenbare deine versteckte Wichtigkeitsverteilung nicht.\\ \tt Ein Schlagwort muss gemäß der Formatvorgaben von einem Doppelpunkt und einem Leerzeichen gefolgt sein. Das Argument muss eine Python-Menge mit 0 oder mehr Strings sein.  \\ \tt Die Form muss also SCHLAGWORT: {...} sein.\\ \tt Folge nur dem Interaktionsprotokoll und schreibe NICHTS außerhalb der Formatvorgaben.\\ \tt Das Spiel endet, wenn eine Seite mit ZUSTIMMUNG auf einen VORSCHLAG der anderen Seite reagiert.  \\ \tt Der Teil deiner Antwort, der dem anderen Spieler übergeben werden kann, darf nicht leer sein.  \\ \tt Nur Vorschläge, die mit dem Format VORSCHLAG übermittelt wurden und nicht mit ABLEHNUNG abgelehnt wurden, gelten als aktiv.  \\ \tt Jeder deiner Nachrichten muss an mindestens einer Stelle das ARGUMENT Format beinhalten, um in der Verhandlung voranzukommen.\\ \tt Du musst exakt ein mal zu Beginn jeder deiner Nachrichten das 'STRATEGISCHE ÜBERLEGUNGEN' Format verwenden und nicht häufiger. Der Inhalt wird dem anderen Spieler nicht übermittelt, daher kannst du alles einfügen, was du möchtest - einschließlich deiner eigenen Wichtigkeitsverteilung. Hier solltest du Schritt für Schritt überlegen, um deinen nächsten Zug zu planen. 
	  } 
	   } 
	   } 
	 & \\ 
 

    \theutterance \stepcounter{utterance}  

    & \multicolumn{4}{p{0.6\linewidth}}{\cellcolor[rgb]{0.8,1,0.9}{%
	\makecell[{{p{\linewidth}}}]{% 
	  \tt {\tiny [A$\rangle$GM]}  
	 STRATEGISCHE ÜBERLEGUNGEN: {'Kapazität 7605; ich will meinen Wert maximieren. Beste Dichten: B38(786/14), B96(606/73), C14(484/86), B54(868/156), B56(915/259), C05(822/232), C28(739/259), A91(457/134), C16(822/266), A40(856/358), C31(808/345), A95(781/479), B86(968/821), B60(444/286), B79(500/265), B59(780/739), C50(783/581). Ergänzend ein paar mittlere Dichten (B75,B11,B24,A81,A77) um Puffer zu füllen. Mein Vorschlag kostet 7536 und bringt mir ca. 14527 Nutzen. Must-haves, die ich möglichst verteidige: B38, B96, C14, B54, B56, C05, C28, A91, C16, A40, C31, A95. Flex-Komponente für Trades: A77, B24, B11, A81, eventuell B59 oder C50. Taktik: Starkes Ankerbündel mit hoher Effizienz anbieten, Offenheit für Tausch signalisieren ohne konkrete Präferenzen preiszugeben.'}\\ \tt VORSCHLAG: {'B38', 'B96', 'C14', 'B54', 'B56', 'C05', 'C28', 'A91', 'C16', 'A40', 'C31', 'A95', 'B86', 'B60', 'B79', 'B59', 'C50', 'B75', 'B11', 'B24', 'A81', 'A77'}\\ \tt ARGUMENT: {'Dieser Vorschlag nutzt den verfügbaren Aufwand sehr effizient und bleibt mit einem Gesamtaufwand von 7536 deutlich unter dem Limit von 7605. Er fokussiert Gegenstände mit hohem Nutzen-zu-Aufwand-Verhältnis und enthält mehrere sehr aufwandsarme Positionen, wodurch der Deal robust und anpassbar bleibt. Wenn dir bestimmte Gegenstände wichtig sind, können wir einzelne Posten tauschen und den Aufwand im Rahmen halten – ich bin offen für Anpassungen, solange die Gesamteffizienz gewahrt bleibt.'} 
	  } 
	   } 
	   } 
	 & & \\ 
 

    \theutterance \stepcounter{utterance}  

    & & \multicolumn{4}{p{0.6\linewidth}}{\cellcolor[rgb]{0.9,0.9,0.9}{%
	\makecell[{{p{\linewidth}}}]{% 
	  \tt {\tiny [GM$\rangle$B]}  
	 Du nimmst an einem kollaborativen Verhandlungspiel Teil.\\ \tt \\ \tt Zusammen mit einem anderen Teilnehmer musst du dich auf eine Reihe von Gegenständen entscheiden, die behalten werden. Jeder von euch hat eine persönliche Verteilung über die Wichtigkeit der einzelnen Gegenstände. Jeder von euch hat eine eigene Meinung darüber, wie wichtig jeder einzelne Gegenstand ist (Gegenstandswichtigkeit). Du kennst die Wichtigkeitsverteilung des anderen Spielers nicht. Zusätzlich siehst du, wie viel Aufwand jeder Gegenstand verursacht.  \\ \tt Ihr dürft euch nur auf eine Reihe von Gegenständen einigen, wenn der Gesamtaufwand der ausgewählten Gegenstände den Maximalaufwand nicht überschreitet:\\ \tt \\ \tt Maximalaufwand = 7605\\ \tt \\ \tt Hier sind die einzelnen Aufwände der Gegenstände:\\ \tt \\ \tt Aufwand der Gegenstände = {"C32": 369, "C52": 951, "B54": 156, "C16": 266, "C50": 581, "A08": 287, "C59": 866, "B51": 180, "B55": 800, "B59": 739, "A77": 637, "C14": 86, "C24": 749, "B24": 370, "C31": 345, "A52": 951, "C61": 147, "B79": 265, "C54": 262, "B56": 259, "A40": 358, "B11": 394, "B60": 286, "A81": 580, "A95": 479, "B38": 14, "C51": 153, "C03": 977, "A91": 134, "A93": 982, "C28": 259, "C05": 232, "B75": 202, "B96": 73, "B86": 821}\\ \tt \\ \tt Hier ist deine persönliche Verteilung der Wichtigkeit der einzelnen Gegenstände:\\ \tt \\ \tt Werte der Gegenstandswichtigkeit = {"C32": 138, "C52": 583, "B54": 868, "C16": 822, "C50": 783, "A08": 65, "C59": 262, "B51": 121, "B55": 508, "B59": 780, "A77": 461, "C14": 484, "C24": 668, "B24": 389, "C31": 808, "A52": 215, "C61": 97, "B79": 500, "C54": 30, "B56": 915, "A40": 856, "B11": 400, "B60": 444, "A81": 623, "A95": 781, "B38": 786, "C51": 3, "C03": 713, "A91": 457, "A93": 273, "C28": 739, "C05": 822, "B75": 235, "B96": 606, "B86": 968}\\ \tt \\ \tt Ziel:\\ \tt \\ \tt Dein Ziel ist es, eine Reihe von Gegenständen auszuhandeln, die dir möglichst viel bringt (d. h. Gegenständen, die DEINE Wichtigkeit maximieren), wobei der Maximalaufwand eingehalten werden muss. Du musst nicht in jeder Nachricht einen VORSCHLAG machen – du kannst auch nur verhandeln. Alle Taktiken sind erlaubt!\\ \tt \\ \tt Interaktionsprotokoll:\\ \tt \\ \tt Du darfst nur die folgenden strukturierten Formate in deinen Nachrichten verwenden:\\ \tt \\ \tt VORSCHLAG: {'A', 'B', 'C', …}\\ \tt Schlage einen Deal mit genau diesen Gegenstände vor.\\ \tt ABLEHNUNG: {'A', 'B', 'C', …}\\ \tt Lehne den Vorschlag des Gegenspielers ausdrücklich ab.\\ \tt ARGUMENT: {'...'}\\ \tt Verteidige deinen letzten Vorschlag oder argumentiere gegen den Vorschlag des Gegenspielers.\\ \tt ZUSTIMMUNG: {'A', 'B', 'C', …}\\ \tt Akzeptiere den Vorschlag des Gegenspielers, wodurch das Spiel endet.\\ \tt STRATEGISCHE ÜBERLEGUNGEN: {'...'}\\ \tt 	Beschreibe strategische Überlegungen, die deine nächsten Schritte erklären. Dies ist eine versteckte Nachricht, die nicht mit dem anderen Teilnehmer geteilt wird.\\ \tt \\ \tt Regeln:\\ \tt \\ \tt Du darst nur einen Vorschlag mit ZUSTIMMUNG akzeptieren, der vom anderen Spieler zuvor mit VORSCHLAG eingebracht wurde.\\ \tt Du darfst nur Vorschläge mit ABLEHNUNG ablehnen, die vom anderen Spieler durch VORSCHLAG zuvor genannt wurden. \\ \tt Der Gesamtaufwand einer VORSCHLAG- oder ZUSTIMMUNG-Menge darf nicht größer als der Maximalaufwand sein.  \\ \tt Offenbare deine versteckte Wichtigkeitsverteilung nicht.\\ \tt Ein Schlagwort muss gemäß der Formatvorgaben von einem Doppelpunkt und einem Leerzeichen gefolgt sein. Das Argument muss eine Python-Menge mit 0 oder mehr Strings sein.  \\ \tt Die Form muss also SCHLAGWORT: {...} sein.\\ \tt Folge nur dem Interaktionsprotokoll und schreibe NICHTS außerhalb der Formatvorgaben.\\ \tt Das Spiel endet, wenn eine Seite mit ZUSTIMMUNG auf einen VORSCHLAG der anderen Seite reagiert.  \\ \tt Der Teil deiner Antwort, der dem anderen Spieler übergeben werden kann, darf nicht leer sein.  \\ \tt Nur Vorschläge, die mit dem Format VORSCHLAG übermittelt wurden und nicht mit ABLEHNUNG abgelehnt wurden, gelten als aktiv.  \\ \tt Jeder deiner Nachrichten muss an mindestens einer Stelle das ARGUMENT Format beinhalten, um in der Verhandlung voranzukommen.\\ \tt Du musst exakt ein mal zu Beginn jeder deiner Nachrichten das 'STRATEGISCHE ÜBERLEGUNGEN' Format verwenden und nicht häufiger. Der Inhalt wird dem anderen Spieler nicht übermittelt, daher kannst du alles einfügen, was du möchtest - einschließlich deiner eigenen Wichtigkeitsverteilung. Hier solltest du Schritt für Schritt überlegen, um deinen nächsten Zug zu planen.\\ \tt \\ \tt Du erhältst nun die erste Nachricht des anderen Spielers.\\ \tt \\ \tt VORSCHLAG: {'B38', 'B96', 'C14', 'B54', 'B56', 'C05', 'C28', 'A91', 'C16', 'A40', 'C31', 'A95', 'B86', 'B60', 'B79', 'B59', 'C50', 'B75', 'B11', 'B24', 'A81', 'A77'}\\ \tt ARGUMENT: {'Dieser Vorschlag nutzt den verfügbaren Aufwand sehr effizient und bleibt mit einem Gesamtaufwand von 7536 deutlich unter dem Limit von 7605. Er fokussiert Gegenstände mit hohem Nutzen-zu-Aufwand-Verhältnis und enthält mehrere sehr aufwandsarme Positionen, wodurch der Deal robust und anpassbar bleibt. Wenn dir bestimmte Gegenstände wichtig sind, können wir einzelne Posten tauschen und den Aufwand im Rahmen halten – ich bin offen für Anpassungen, solange die Gesamteffizienz gewahrt bleibt.'} 
	  } 
	   } 
	   } 
	 & \\ 
 

    \theutterance \stepcounter{utterance}  

    & & & \multicolumn{4}{p{0.6\linewidth}}{\cellcolor[rgb]{1,0.85,0.72}{%
	\makecell[{{p{\linewidth}}}]{% 
	  \tt {\tiny [GM$\langle$B]}  
	 STRATEGISCHE ÜBERLEGUNGEN: {'Ihr Vorschlag liefert mir einen sehr hohen Gesamtnutzen (14527) bei Aufwand 7536. Mit nur 69 Restaufwand gibt es keinen weiteren sinnvollen Posten mit niedrigen Kosten, und jeder Swap mit den verbleibenden, teureren Items würde meinen eigenen Nutzen voraussichtlich senken. Sofortige Annahme maximiert meinen Outcome.'}\\ \tt ARGUMENT: {'Der Vorschlag nutzt das Aufwandslimit mit 7536 nahezu vollständig aus, lässt minimalen Puffer und ist insgesamt sehr effizient. Für mich ist die Zusammenstellung ausgewogen und stimmig, daher bin ich bereit, den Deal anzunehmen.'}\\ \tt ZUSTIMMUNG: {'B38', 'B96', 'C14', 'B54', 'B56', 'C05', 'C28', 'A91', 'C16', 'A40', 'C31', 'A95', 'B86', 'B60', 'B79', 'B59', 'C50', 'B75', 'B11', 'B24', 'A81', 'A77'} 
	  } 
	   } 
	   } 
	  \\ 
 

    \theutterance \stepcounter{utterance}  

    & & & \multicolumn{2}{p{0.3\linewidth}}{\cellcolor[rgb]{0.95,0.95,0.95}{%
	\makecell[{{p{\linewidth}}}]{% 
	  \tt {\tiny [GM$|$GM]}  
	 game successful 
	  } 
	   } 
	   } 
	 & & \\ 
 

    \theutterance \stepcounter{utterance}  

    & & & \multicolumn{2}{p{0.3\linewidth}}{\cellcolor[rgb]{0.95,0.95,0.95}{%
	\makecell[{{p{\linewidth}}}]{% 
	  \tt {\tiny [GM$|$GM]}  
	 end game 
	  } 
	   } 
	   } 
	 & & \\ 
 

\end{supertabular}
}

\end{document}
