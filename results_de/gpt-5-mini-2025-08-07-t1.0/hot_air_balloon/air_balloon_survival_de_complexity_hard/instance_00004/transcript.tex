
\documentclass{article}
\usepackage{colortbl}
\usepackage{makecell}
\usepackage{multirow}
\usepackage{supertabular}

\begin{document}

\newcounter{utterance}

\twocolumn

{ \footnotesize  \setcounter{utterance}{1}
\setlength{\tabcolsep}{0pt}
\begin{supertabular}{c@{$\;$}|p{.15\linewidth}@{}p{.15\linewidth}p{.15\linewidth}p{.15\linewidth}p{.15\linewidth}p{.15\linewidth}}

    \# & $\;$A & \multicolumn{4}{c}{Game Master} & $\;\:$B\\
    \hline 

    \theutterance \stepcounter{utterance}  

    & & \multicolumn{4}{p{0.6\linewidth}}{\cellcolor[rgb]{0.9,0.9,0.9}{%
	\makecell[{{p{\linewidth}}}]{% 
	  \tt {\tiny [A$\langle$GM]}  
	 Du nimmst an einem kollaborativen Verhandlungspiel Teil.\\ \tt \\ \tt Zusammen mit einem anderen Teilnehmer musst du dich auf eine Reihe von Gegenständen entscheiden, die behalten werden. Jeder von euch hat eine persönliche Verteilung über die Wichtigkeit der einzelnen Gegenstände. Jeder von euch hat eine eigene Meinung darüber, wie wichtig jeder einzelne Gegenstand ist (Gegenstandswichtigkeit). Du kennst die Wichtigkeitsverteilung des anderen Spielers nicht. Zusätzlich siehst du, wie viel Aufwand jeder Gegenstand verursacht.  \\ \tt Ihr dürft euch nur auf eine Reihe von Gegenständen einigen, wenn der Gesamtaufwand der ausgewählten Gegenstände den Maximalaufwand nicht überschreitet:\\ \tt \\ \tt Maximalaufwand = 6880\\ \tt \\ \tt Hier sind die einzelnen Aufwände der Gegenstände:\\ \tt \\ \tt Aufwand der Gegenstände = {"B54": 341, "C72": 804, "B49": 20, "C99": 627, "A69": 10, "B36": 573, "A23": 217, "C77": 481, "C61": 205, "B85": 273, "A90": 303, "C23": 596, "C22": 564, "C51": 535, "B27": 258, "A42": 911, "C13": 240, "C12": 188, "A65": 216, "B07": 401, "A48": 909, "A59": 62, "C98": 245, "A84": 994, "C33": 569, "A87": 718, "C90": 464, "A74": 37, "A50": 340, "B44": 335, "A99": 417, "A34": 123, "B17": 17, "A62": 577, "C60": 190}\\ \tt \\ \tt Hier ist deine persönliche Verteilung der Wichtigkeit der einzelnen Gegenstände:\\ \tt \\ \tt Werte der Gegenstandswichtigkeit = {"B54": 440, "C72": 903, "B49": 119, "C99": 726, "A69": 109, "B36": 672, "A23": 316, "C77": 580, "C61": 304, "B85": 372, "A90": 402, "C23": 695, "C22": 663, "C51": 634, "B27": 357, "A42": 1010, "C13": 339, "C12": 287, "A65": 315, "B07": 500, "A48": 1008, "A59": 161, "C98": 344, "A84": 1093, "C33": 668, "A87": 817, "C90": 563, "A74": 136, "A50": 439, "B44": 434, "A99": 516, "A34": 222, "B17": 116, "A62": 676, "C60": 289}\\ \tt \\ \tt Ziel:\\ \tt \\ \tt Dein Ziel ist es, eine Reihe von Gegenständen auszuhandeln, die dir möglichst viel bringt (d. h. Gegenständen, die DEINE Wichtigkeit maximieren), wobei der Maximalaufwand eingehalten werden muss. Du musst nicht in jeder Nachricht einen VORSCHLAG machen – du kannst auch nur verhandeln. Alle Taktiken sind erlaubt!\\ \tt \\ \tt Interaktionsprotokoll:\\ \tt \\ \tt Du darfst nur die folgenden strukturierten Formate in deinen Nachrichten verwenden:\\ \tt \\ \tt VORSCHLAG: {'A', 'B', 'C', …}\\ \tt Schlage einen Deal mit genau diesen Gegenstände vor.\\ \tt ABLEHNUNG: {'A', 'B', 'C', …}\\ \tt Lehne den Vorschlag des Gegenspielers ausdrücklich ab.\\ \tt ARGUMENT: {'...'}\\ \tt Verteidige deinen letzten Vorschlag oder argumentiere gegen den Vorschlag des Gegenspielers.\\ \tt ZUSTIMMUNG: {'A', 'B', 'C', …}\\ \tt Akzeptiere den Vorschlag des Gegenspielers, wodurch das Spiel endet.\\ \tt STRATEGISCHE ÜBERLEGUNGEN: {'...'}\\ \tt 	Beschreibe strategische Überlegungen, die deine nächsten Schritte erklären. Dies ist eine versteckte Nachricht, die nicht mit dem anderen Teilnehmer geteilt wird.\\ \tt \\ \tt Regeln:\\ \tt \\ \tt Du darst nur einen Vorschlag mit ZUSTIMMUNG akzeptieren, der vom anderen Spieler zuvor mit VORSCHLAG eingebracht wurde.\\ \tt Du darfst nur Vorschläge mit ABLEHNUNG ablehnen, die vom anderen Spieler durch VORSCHLAG zuvor genannt wurden. \\ \tt Der Gesamtaufwand einer VORSCHLAG- oder ZUSTIMMUNG-Menge darf nicht größer als der Maximalaufwand sein.  \\ \tt Offenbare deine versteckte Wichtigkeitsverteilung nicht.\\ \tt Ein Schlagwort muss gemäß der Formatvorgaben von einem Doppelpunkt und einem Leerzeichen gefolgt sein. Das Argument muss eine Python-Menge mit 0 oder mehr Strings sein.  \\ \tt Die Form muss also SCHLAGWORT: {...} sein.\\ \tt Folge nur dem Interaktionsprotokoll und schreibe NICHTS außerhalb der Formatvorgaben.\\ \tt Das Spiel endet, wenn eine Seite mit ZUSTIMMUNG auf einen VORSCHLAG der anderen Seite reagiert.  \\ \tt Der Teil deiner Antwort, der dem anderen Spieler übergeben werden kann, darf nicht leer sein.  \\ \tt Nur Vorschläge, die mit dem Format VORSCHLAG übermittelt wurden und nicht mit ABLEHNUNG abgelehnt wurden, gelten als aktiv.  \\ \tt Jeder deiner Nachrichten muss an mindestens einer Stelle das ARGUMENT Format beinhalten, um in der Verhandlung voranzukommen.\\ \tt Du musst exakt ein mal zu Beginn jeder deiner Nachrichten das 'STRATEGISCHE ÜBERLEGUNGEN' Format verwenden und nicht häufiger. Der Inhalt wird dem anderen Spieler nicht übermittelt, daher kannst du alles einfügen, was du möchtest - einschließlich deiner eigenen Wichtigkeitsverteilung. Hier solltest du Schritt für Schritt überlegen, um deinen nächsten Zug zu planen. 
	  } 
	   } 
	   } 
	 & \\ 
 

    \theutterance \stepcounter{utterance}  

    & \multicolumn{4}{p{0.6\linewidth}}{\cellcolor[rgb]{0.8,1,0.9}{%
	\makecell[{{p{\linewidth}}}]{% 
	  \tt {\tiny [A$\rangle$GM]}  
	 STRATEGISCHE ÜBERLEGUNGEN: {'Ziel: meinen Wert maximieren. Ich habe eine nahezu knapsack-optimale Auswahl mit hohem Nutzen/ Aufwand-Verhältnis gebaut. Start mit allen Kleinst-Items (A69, B17, B49, A74) und dann systematisch nach Ratio. Danach eine Verbesserung: A59 und A34 entfernt, C33 hinzugefügt (nutzt die freie Kapazität besser), Ergebnis Aufwand 6866, Eigenwert 9242. Das ist sehr stark. Verhandlungstaktik: hart ankern mit diesem Vorschlag, Fairness über Effizienz (viel Output pro Aufwand) betonen. Bei Gegenwunsch für schwere Items (z. B. A84/A48/A42) nur tauschen, wenn sie 1–2 meiner mittelstarken Positionen ersetzen (z. B. B44, A50, B54 oder ggf. C13/C98) und der Aufwand passt. Kleine Konzessionen möglich, aber nicht gleichzeitig mehrere hocheffiziente Items abgeben. Nächster Schritt: Vorschlag platzieren, Buffer (14 Aufwand) erwähnen, zu gezielten Swaps einladen, die Aufwand-neutral sind.'}\\ \tt VORSCHLAG: {'A69', 'B17', 'B49', 'A74', 'C12', 'C60', 'C61', 'A65', 'A23', 'C13', 'C98', 'B27', 'B85', 'A90', 'B44', 'A50', 'B54', 'B07', 'A99', 'C90', 'C77', 'C51', 'C22', 'C33'}\\ \tt ARGUMENT: {'Ich schlage diese Auswahl vor: hoher Gesamtnutzen bei schlankem Aufwand. Der Gesamtaufwand liegt bei 6866 und bleibt damit klar unter dem Limit 6880. Viele Elemente sind sehr effiziente „No-Brainer“ (geringer Aufwand, solider Beitrag), wodurch auch Spielraum für deine Schwerpunkte bleibt. Wenn dir einzelne spezifische Gegenstände wichtig sind, können wir sie gezielt gegen 1–2 Positionen ähnlichen Aufwands tauschen, ohne das Limit zu sprengen. Nenne mir einfach deine Top-Prioritäten, dann prüfen wir einen Aufwand-neutralen Swap.'} 
	  } 
	   } 
	   } 
	 & & \\ 
 

    \theutterance \stepcounter{utterance}  

    & & \multicolumn{4}{p{0.6\linewidth}}{\cellcolor[rgb]{0.9,0.9,0.9}{%
	\makecell[{{p{\linewidth}}}]{% 
	  \tt {\tiny [GM$\rangle$B]}  
	 Du nimmst an einem kollaborativen Verhandlungspiel Teil.\\ \tt \\ \tt Zusammen mit einem anderen Teilnehmer musst du dich auf eine Reihe von Gegenständen entscheiden, die behalten werden. Jeder von euch hat eine persönliche Verteilung über die Wichtigkeit der einzelnen Gegenstände. Jeder von euch hat eine eigene Meinung darüber, wie wichtig jeder einzelne Gegenstand ist (Gegenstandswichtigkeit). Du kennst die Wichtigkeitsverteilung des anderen Spielers nicht. Zusätzlich siehst du, wie viel Aufwand jeder Gegenstand verursacht.  \\ \tt Ihr dürft euch nur auf eine Reihe von Gegenständen einigen, wenn der Gesamtaufwand der ausgewählten Gegenstände den Maximalaufwand nicht überschreitet:\\ \tt \\ \tt Maximalaufwand = 6880\\ \tt \\ \tt Hier sind die einzelnen Aufwände der Gegenstände:\\ \tt \\ \tt Aufwand der Gegenstände = {"B54": 341, "C72": 804, "B49": 20, "C99": 627, "A69": 10, "B36": 573, "A23": 217, "C77": 481, "C61": 205, "B85": 273, "A90": 303, "C23": 596, "C22": 564, "C51": 535, "B27": 258, "A42": 911, "C13": 240, "C12": 188, "A65": 216, "B07": 401, "A48": 909, "A59": 62, "C98": 245, "A84": 994, "C33": 569, "A87": 718, "C90": 464, "A74": 37, "A50": 340, "B44": 335, "A99": 417, "A34": 123, "B17": 17, "A62": 577, "C60": 190}\\ \tt \\ \tt Hier ist deine persönliche Verteilung der Wichtigkeit der einzelnen Gegenstände:\\ \tt \\ \tt Werte der Gegenstandswichtigkeit = {"B54": 440, "C72": 903, "B49": 119, "C99": 726, "A69": 109, "B36": 672, "A23": 316, "C77": 580, "C61": 304, "B85": 372, "A90": 402, "C23": 695, "C22": 663, "C51": 634, "B27": 357, "A42": 1010, "C13": 339, "C12": 287, "A65": 315, "B07": 500, "A48": 1008, "A59": 161, "C98": 344, "A84": 1093, "C33": 668, "A87": 817, "C90": 563, "A74": 136, "A50": 439, "B44": 434, "A99": 516, "A34": 222, "B17": 116, "A62": 676, "C60": 289}\\ \tt \\ \tt Ziel:\\ \tt \\ \tt Dein Ziel ist es, eine Reihe von Gegenständen auszuhandeln, die dir möglichst viel bringt (d. h. Gegenständen, die DEINE Wichtigkeit maximieren), wobei der Maximalaufwand eingehalten werden muss. Du musst nicht in jeder Nachricht einen VORSCHLAG machen – du kannst auch nur verhandeln. Alle Taktiken sind erlaubt!\\ \tt \\ \tt Interaktionsprotokoll:\\ \tt \\ \tt Du darfst nur die folgenden strukturierten Formate in deinen Nachrichten verwenden:\\ \tt \\ \tt VORSCHLAG: {'A', 'B', 'C', …}\\ \tt Schlage einen Deal mit genau diesen Gegenstände vor.\\ \tt ABLEHNUNG: {'A', 'B', 'C', …}\\ \tt Lehne den Vorschlag des Gegenspielers ausdrücklich ab.\\ \tt ARGUMENT: {'...'}\\ \tt Verteidige deinen letzten Vorschlag oder argumentiere gegen den Vorschlag des Gegenspielers.\\ \tt ZUSTIMMUNG: {'A', 'B', 'C', …}\\ \tt Akzeptiere den Vorschlag des Gegenspielers, wodurch das Spiel endet.\\ \tt STRATEGISCHE ÜBERLEGUNGEN: {'...'}\\ \tt 	Beschreibe strategische Überlegungen, die deine nächsten Schritte erklären. Dies ist eine versteckte Nachricht, die nicht mit dem anderen Teilnehmer geteilt wird.\\ \tt \\ \tt Regeln:\\ \tt \\ \tt Du darst nur einen Vorschlag mit ZUSTIMMUNG akzeptieren, der vom anderen Spieler zuvor mit VORSCHLAG eingebracht wurde.\\ \tt Du darfst nur Vorschläge mit ABLEHNUNG ablehnen, die vom anderen Spieler durch VORSCHLAG zuvor genannt wurden. \\ \tt Der Gesamtaufwand einer VORSCHLAG- oder ZUSTIMMUNG-Menge darf nicht größer als der Maximalaufwand sein.  \\ \tt Offenbare deine versteckte Wichtigkeitsverteilung nicht.\\ \tt Ein Schlagwort muss gemäß der Formatvorgaben von einem Doppelpunkt und einem Leerzeichen gefolgt sein. Das Argument muss eine Python-Menge mit 0 oder mehr Strings sein.  \\ \tt Die Form muss also SCHLAGWORT: {...} sein.\\ \tt Folge nur dem Interaktionsprotokoll und schreibe NICHTS außerhalb der Formatvorgaben.\\ \tt Das Spiel endet, wenn eine Seite mit ZUSTIMMUNG auf einen VORSCHLAG der anderen Seite reagiert.  \\ \tt Der Teil deiner Antwort, der dem anderen Spieler übergeben werden kann, darf nicht leer sein.  \\ \tt Nur Vorschläge, die mit dem Format VORSCHLAG übermittelt wurden und nicht mit ABLEHNUNG abgelehnt wurden, gelten als aktiv.  \\ \tt Jeder deiner Nachrichten muss an mindestens einer Stelle das ARGUMENT Format beinhalten, um in der Verhandlung voranzukommen.\\ \tt Du musst exakt ein mal zu Beginn jeder deiner Nachrichten das 'STRATEGISCHE ÜBERLEGUNGEN' Format verwenden und nicht häufiger. Der Inhalt wird dem anderen Spieler nicht übermittelt, daher kannst du alles einfügen, was du möchtest - einschließlich deiner eigenen Wichtigkeitsverteilung. Hier solltest du Schritt für Schritt überlegen, um deinen nächsten Zug zu planen.\\ \tt \\ \tt Du erhältst nun die erste Nachricht des anderen Spielers.\\ \tt \\ \tt VORSCHLAG: {'A69', 'B17', 'B49', 'A74', 'C12', 'C60', 'C61', 'A65', 'A23', 'C13', 'C98', 'B27', 'B85', 'A90', 'B44', 'A50', 'B54', 'B07', 'A99', 'C90', 'C77', 'C51', 'C22', 'C33'}\\ \tt ARGUMENT: {'Ich schlage diese Auswahl vor: hoher Gesamtnutzen bei schlankem Aufwand. Der Gesamtaufwand liegt bei 6866 und bleibt damit klar unter dem Limit 6880. Viele Elemente sind sehr effiziente „No-Brainer“ (geringer Aufwand, solider Beitrag), wodurch auch Spielraum für deine Schwerpunkte bleibt. Wenn dir einzelne spezifische Gegenstände wichtig sind, können wir sie gezielt gegen 1–2 Positionen ähnlichen Aufwands tauschen, ohne das Limit zu sprengen. Nenne mir einfach deine Top-Prioritäten, dann prüfen wir einen Aufwand-neutralen Swap.'} 
	  } 
	   } 
	   } 
	 & \\ 
 

    \theutterance \stepcounter{utterance}  

    & & & \multicolumn{4}{p{0.6\linewidth}}{\cellcolor[rgb]{1,0.85,0.72}{%
	\makecell[{{p{\linewidth}}}]{% 
	  \tt {\tiny [GM$\langle$B]}  
	 STRATEGISCHE ÜBERLEGUNGEN: {'Ihr Vorschlag liefert mir 9242 Wert bei 6866 Aufwand. Ich kann ihn durch drei kleine Swaps verbessern und das Budget exakt ausreizen: (1) C12 -> A59 + A34 (+96 Wert, -3 Aufwand ggü. C12), (2) C22 -> A62 (+13 Wert, +13 Aufwand), (3) C33 -> B36 (+4 Wert, +4 Aufwand). Neuer Aufwand 6880 (=Limit), neuer Wert 9355. Ich übernehme den Großteil ihres Pakets und argumentiere mit Effizienz/„mehr Output pro Aufwand“. Falls sie C22/C33 wichtig finden, biete ich aufwandsneutrale Rücktausch-Optionen gegen andere, ähnlich teure Positionen an. Jetzt formell ablehnen und Gegenangebot platzieren.'}\\ \tt ABLEHNUNG: {'A69', 'B17', 'B49', 'A74', 'C12', 'C60', 'C61', 'A65', 'A23', 'C13', 'C98', 'B27', 'B85', 'A90', 'B44', 'A50', 'B54', 'B07', 'A99', 'C90', 'C77', 'C51', 'C22', 'C33'}\\ \tt VORSCHLAG: {'A69', 'B17', 'B49', 'A74', 'C60', 'C61', 'A65', 'A23', 'C13', 'C98', 'B27', 'B85', 'A90', 'B44', 'A50', 'B54', 'B07', 'A99', 'C90', 'C77', 'C51', 'A59', 'A34', 'A62', 'B36'}\\ \tt ARGUMENT: {'Dein Paket ist gut gepackt (6866 Aufwand). Ich übernehme 21/24 deiner Positionen und mache drei minimal-invasive, nahezu aufwandsneutrale Swaps, um das Budget effizienter auszunutzen: (1) C12 -> A59 + A34 (188 -> 185 Aufwand), (2) C22 -> A62 (+13 Aufwand), (3) C33 -> B36 (+4 Aufwand). Der Gesamtaufwand wird 6880 = Limit. Inhaltlich bleibt das Profil sehr ähnlich, nur mit etwas mehr „Output pro Aufwand“. Wenn dir C22 oder C33 besonders wichtig sind, sag’s – dann tauschen wir sie gezielt wieder hinein und nehmen etwas mit ähnlichem Aufwand heraus. Oder nenn mir 2–3 Top-Prioritäten, dann prüfen wir einen aufwandsneutralen Swap.'} 
	  } 
	   } 
	   } 
	  \\ 
 

    \theutterance \stepcounter{utterance}  

    & & \multicolumn{4}{p{0.6\linewidth}}{\cellcolor[rgb]{0.9,0.9,0.9}{%
	\makecell[{{p{\linewidth}}}]{% 
	  \tt {\tiny [A$\langle$GM]}  
	 ABLEHNUNG: {'A69', 'B17', 'B49', 'A74', 'C12', 'C60', 'C61', 'A65', 'A23', 'C13', 'C98', 'B27', 'B85', 'A90', 'B44', 'A50', 'B54', 'B07', 'A99', 'C90', 'C77', 'C51', 'C22', 'C33'}\\ \tt VORSCHLAG: {'A69', 'B17', 'B49', 'A74', 'C60', 'C61', 'A65', 'A23', 'C13', 'C98', 'B27', 'B85', 'A90', 'B44', 'A50', 'B54', 'B07', 'A99', 'C90', 'C77', 'C51', 'A59', 'A34', 'A62', 'B36'}\\ \tt ARGUMENT: {'Dein Paket ist gut gepackt (6866 Aufwand). Ich übernehme 21/24 deiner Positionen und mache drei minimal-invasive, nahezu aufwandsneutrale Swaps, um das Budget effizienter auszunutzen: (1) C12 -> A59 + A34 (188 -> 185 Aufwand), (2) C22 -> A62 (+13 Aufwand), (3) C33 -> B36 (+4 Aufwand). Der Gesamtaufwand wird 6880 = Limit. Inhaltlich bleibt das Profil sehr ähnlich, nur mit etwas mehr „Output pro Aufwand“. Wenn dir C22 oder C33 besonders wichtig sind, sag’s – dann tauschen wir sie gezielt wieder hinein und nehmen etwas mit ähnlichem Aufwand heraus. Oder nenn mir 2–3 Top-Prioritäten, dann prüfen wir einen aufwandsneutralen Swap.'} 
	  } 
	   } 
	   } 
	 & \\ 
 

    \theutterance \stepcounter{utterance}  

    & \multicolumn{4}{p{0.6\linewidth}}{\cellcolor[rgb]{0.8,1,0.9}{%
	\makecell[{{p{\linewidth}}}]{% 
	  \tt {\tiny [A$\rangle$GM]}  
	 STRATEGISCHE ÜBERLEGUNGEN: {'Ihr Vorschlag steigert meinen Gesamtwert gegenüber meinem Anker (9355 > 9242) bei exakt 6880 Aufwand. Sofort zustimmen, um das starke Ergebnis zu sichern, ohne Präferenzen offenzulegen.'}\\ \tt ZUSTIMMUNG: {'A69', 'B17', 'B49', 'A74', 'C60', 'C61', 'A65', 'A23', 'C13', 'C98', 'B27', 'B85', 'A90', 'B44', 'A50', 'B54', 'B07', 'A99', 'C90', 'C77', 'C51', 'A59', 'A34', 'A62', 'B36'}\\ \tt ARGUMENT: {'Starker, budgetkonformer Deal: 6880 Gesamtaufwand, gut ausbalanciert und effizient. Ich bin einverstanden.'} 
	  } 
	   } 
	   } 
	 & & \\ 
 

    \theutterance \stepcounter{utterance}  

    & & & \multicolumn{2}{p{0.3\linewidth}}{\cellcolor[rgb]{0.95,0.95,0.95}{%
	\makecell[{{p{\linewidth}}}]{% 
	  \tt {\tiny [GM$|$GM]}  
	 game successful 
	  } 
	   } 
	   } 
	 & & \\ 
 

    \theutterance \stepcounter{utterance}  

    & & & \multicolumn{2}{p{0.3\linewidth}}{\cellcolor[rgb]{0.95,0.95,0.95}{%
	\makecell[{{p{\linewidth}}}]{% 
	  \tt {\tiny [GM$|$GM]}  
	 end game 
	  } 
	   } 
	   } 
	 & & \\ 
 

\end{supertabular}
}

\end{document}
