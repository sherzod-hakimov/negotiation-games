
\documentclass{article}
\usepackage{colortbl}
\usepackage{makecell}
\usepackage{multirow}
\usepackage{supertabular}

\begin{document}

\newcounter{utterance}

\twocolumn

{ \footnotesize  \setcounter{utterance}{1}
\setlength{\tabcolsep}{0pt}
\begin{supertabular}{c@{$\;$}|p{.15\linewidth}@{}p{.15\linewidth}p{.15\linewidth}p{.15\linewidth}p{.15\linewidth}p{.15\linewidth}}

    \# & $\;$A & \multicolumn{4}{c}{Game Master} & $\;\:$B\\
    \hline 

    \theutterance \stepcounter{utterance}  

    & & \multicolumn{4}{p{0.6\linewidth}}{\cellcolor[rgb]{0.9,0.9,0.9}{%
	\makecell[{{p{\linewidth}}}]{% 
	  \tt {\tiny [A$\langle$GM]}  
	 Du nimmst an einem kollaborativen Verhandlungspiel Teil.\\ \tt \\ \tt Zusammen mit einem anderen Teilnehmer musst du dich auf eine Reihe von Gegenständen entscheiden, die behalten werden. Jeder von euch hat eine persönliche Verteilung über die Wichtigkeit der einzelnen Gegenstände. Jeder von euch hat eine eigene Meinung darüber, wie wichtig jeder einzelne Gegenstand ist (Gegenstandswichtigkeit). Du kennst die Wichtigkeitsverteilung des anderen Spielers nicht. Zusätzlich siehst du, wie viel Aufwand jeder Gegenstand verursacht.  \\ \tt Ihr dürft euch nur auf eine Reihe von Gegenständen einigen, wenn der Gesamtaufwand der ausgewählten Gegenstände den Maximalaufwand nicht überschreitet:\\ \tt \\ \tt Maximalaufwand = 10073\\ \tt \\ \tt Hier sind die einzelnen Aufwände der Gegenstände:\\ \tt \\ \tt Aufwand der Gegenstände = {"B54": 132, "C72": 800, "B49": 708, "C99": 302, "A69": 983, "B36": 420, "A23": 623, "C77": 349, "C61": 239, "B85": 28, "A90": 986, "C23": 660, "C22": 718, "C51": 707, "B27": 188, "A42": 774, "C13": 785, "C12": 775, "A65": 517, "B07": 768, "A48": 588, "A59": 657, "C98": 375, "A84": 310, "C33": 301, "A87": 388, "C90": 431, "A74": 950, "A50": 540, "B44": 997, "A99": 473, "A34": 831, "B17": 888, "A62": 879, "C60": 76}\\ \tt \\ \tt Hier ist deine persönliche Verteilung der Wichtigkeit der einzelnen Gegenstände:\\ \tt \\ \tt Werte der Gegenstandswichtigkeit = {"B54": 231, "C72": 899, "B49": 807, "C99": 401, "A69": 1082, "B36": 519, "A23": 722, "C77": 448, "C61": 338, "B85": 127, "A90": 1085, "C23": 759, "C22": 817, "C51": 806, "B27": 287, "A42": 873, "C13": 884, "C12": 874, "A65": 616, "B07": 867, "A48": 687, "A59": 756, "C98": 474, "A84": 409, "C33": 400, "A87": 487, "C90": 530, "A74": 1049, "A50": 639, "B44": 1096, "A99": 572, "A34": 930, "B17": 987, "A62": 978, "C60": 175}\\ \tt \\ \tt Ziel:\\ \tt \\ \tt Dein Ziel ist es, eine Reihe von Gegenständen auszuhandeln, die dir möglichst viel bringt (d. h. Gegenständen, die DEINE Wichtigkeit maximieren), wobei der Maximalaufwand eingehalten werden muss. Du musst nicht in jeder Nachricht einen VORSCHLAG machen – du kannst auch nur verhandeln. Alle Taktiken sind erlaubt!\\ \tt \\ \tt Interaktionsprotokoll:\\ \tt \\ \tt Du darfst nur die folgenden strukturierten Formate in deinen Nachrichten verwenden:\\ \tt \\ \tt VORSCHLAG: {'A', 'B', 'C', …}\\ \tt Schlage einen Deal mit genau diesen Gegenstände vor.\\ \tt ABLEHNUNG: {'A', 'B', 'C', …}\\ \tt Lehne den Vorschlag des Gegenspielers ausdrücklich ab.\\ \tt ARGUMENT: {'...'}\\ \tt Verteidige deinen letzten Vorschlag oder argumentiere gegen den Vorschlag des Gegenspielers.\\ \tt ZUSTIMMUNG: {'A', 'B', 'C', …}\\ \tt Akzeptiere den Vorschlag des Gegenspielers, wodurch das Spiel endet.\\ \tt STRATEGISCHE ÜBERLEGUNGEN: {'...'}\\ \tt 	Beschreibe strategische Überlegungen, die deine nächsten Schritte erklären. Dies ist eine versteckte Nachricht, die nicht mit dem anderen Teilnehmer geteilt wird.\\ \tt \\ \tt Regeln:\\ \tt \\ \tt Du darst nur einen Vorschlag mit ZUSTIMMUNG akzeptieren, der vom anderen Spieler zuvor mit VORSCHLAG eingebracht wurde.\\ \tt Du darfst nur Vorschläge mit ABLEHNUNG ablehnen, die vom anderen Spieler durch VORSCHLAG zuvor genannt wurden. \\ \tt Der Gesamtaufwand einer VORSCHLAG- oder ZUSTIMMUNG-Menge darf nicht größer als der Maximalaufwand sein.  \\ \tt Offenbare deine versteckte Wichtigkeitsverteilung nicht.\\ \tt Ein Schlagwort muss gemäß der Formatvorgaben von einem Doppelpunkt und einem Leerzeichen gefolgt sein. Das Argument muss eine Python-Menge mit 0 oder mehr Strings sein.  \\ \tt Die Form muss also SCHLAGWORT: {...} sein.\\ \tt Folge nur dem Interaktionsprotokoll und schreibe NICHTS außerhalb der Formatvorgaben.\\ \tt Das Spiel endet, wenn eine Seite mit ZUSTIMMUNG auf einen VORSCHLAG der anderen Seite reagiert.  \\ \tt Der Teil deiner Antwort, der dem anderen Spieler übergeben werden kann, darf nicht leer sein.  \\ \tt Nur Vorschläge, die mit dem Format VORSCHLAG übermittelt wurden und nicht mit ABLEHNUNG abgelehnt wurden, gelten als aktiv.  \\ \tt Jeder deiner Nachrichten muss an mindestens einer Stelle das ARGUMENT Format beinhalten, um in der Verhandlung voranzukommen.\\ \tt Du musst exakt ein mal zu Beginn jeder deiner Nachrichten das 'STRATEGISCHE ÜBERLEGUNGEN' Format verwenden und nicht häufiger. Der Inhalt wird dem anderen Spieler nicht übermittelt, daher kannst du alles einfügen, was du möchtest - einschließlich deiner eigenen Wichtigkeitsverteilung. Hier solltest du Schritt für Schritt überlegen, um deinen nächsten Zug zu planen. 
	  } 
	   } 
	   } 
	 & \\ 
 

    \theutterance \stepcounter{utterance}  

    & \multicolumn{4}{p{0.6\linewidth}}{\cellcolor[rgb]{0.8,1,0.9}{%
	\makecell[{{p{\linewidth}}}]{% 
	  \tt {\tiny [A$\rangle$GM]}  
	 STRATEGISCHE ÜBERLEGUNGEN: {'Ziel: mein Nutzen maximieren unter Aufwand 10073. Ich habe eine knapsack-Optimierung gemacht: Start mit allen sehr hohen Nutzen/Kosten-Ratio-Items (B85, C60, B27, C61, C33, C99, A84, C77, C98, A87, B36, C90, A99, A50, A48) und danach die besten 1.19–1.16er (A65, A23, A59, C23) sowie 1.14er (B49, C22, C51) und B54. Das ergab Kosten 9730. Dann lokaler Tausch: C90 (431) gegen A42 (774) benutzt die Rest-343 und steigert meinen Wert um +343, Kosten exakt 10073. Ergebnis-Set S hat Kosten 10073 und meinen Gesamtwert 12350. Must-haves für mich (sehr hohe Ratio/kleine Kosten): B85, C60, B27, C61, C33, C99, A84, B54; außerdem C77, C98, A87, B36. Verhandlungsstrategie: hart ankern mit voller Kapazitätsauslastung und vielen kleinen/mittleren Posten. Wenn Gegenseite große A-Items (A90, A69, A74) oder B44 will, bin ich bereit, 1–2 mittlere Posten mit niedrigeren Ratios (A59, A50, A23, C23, A99, ggf. einer der 1.14er B49/C22/C51) zu tauschen, solange mein Nettoverlust klein bleibt. Erst eine kleine Konzession anbieten (max 1 Tausch), dafür gezielt um Aufnahme eines meiner mittelguten Items bitten, falls sie etwas entfernen wollen. Keine Offenlegung meiner Werte.'}\\ \tt VORSCHLAG: {'B85', 'C60', 'B27', 'C61', 'C33', 'C99', 'A84', 'C77', 'C98', 'A87', 'B36', 'A42', 'A99', 'A50', 'A48', 'A65', 'A23', 'A59', 'C23', 'B49', 'C22', 'C51', 'B54'}\\ \tt ARGUMENT: {'Dieser Vorschlag nutzt den Maximalaufwand exakt aus (10073) und setzt stark auf effiziente, eher kleinere bis mittlere Positionen. So bleiben wir flexibel und vermeiden Klumpenrisiken. Wenn du 1–2 große Items (z. B. A90, A69 oder A74) unbedingt drin haben möchtest, können wir gezielt 1–2 mittelgroße Posten tauschen, solange der Gesamtaufwand eingehalten wird. Sag mir bitte deine Top-Prioritäten, dann schlage ich einen passenden Tausch vor.'} 
	  } 
	   } 
	   } 
	 & & \\ 
 

    \theutterance \stepcounter{utterance}  

    & & \multicolumn{4}{p{0.6\linewidth}}{\cellcolor[rgb]{0.9,0.9,0.9}{%
	\makecell[{{p{\linewidth}}}]{% 
	  \tt {\tiny [GM$\rangle$B]}  
	 Du nimmst an einem kollaborativen Verhandlungspiel Teil.\\ \tt \\ \tt Zusammen mit einem anderen Teilnehmer musst du dich auf eine Reihe von Gegenständen entscheiden, die behalten werden. Jeder von euch hat eine persönliche Verteilung über die Wichtigkeit der einzelnen Gegenstände. Jeder von euch hat eine eigene Meinung darüber, wie wichtig jeder einzelne Gegenstand ist (Gegenstandswichtigkeit). Du kennst die Wichtigkeitsverteilung des anderen Spielers nicht. Zusätzlich siehst du, wie viel Aufwand jeder Gegenstand verursacht.  \\ \tt Ihr dürft euch nur auf eine Reihe von Gegenständen einigen, wenn der Gesamtaufwand der ausgewählten Gegenstände den Maximalaufwand nicht überschreitet:\\ \tt \\ \tt Maximalaufwand = 10073\\ \tt \\ \tt Hier sind die einzelnen Aufwände der Gegenstände:\\ \tt \\ \tt Aufwand der Gegenstände = {"B54": 132, "C72": 800, "B49": 708, "C99": 302, "A69": 983, "B36": 420, "A23": 623, "C77": 349, "C61": 239, "B85": 28, "A90": 986, "C23": 660, "C22": 718, "C51": 707, "B27": 188, "A42": 774, "C13": 785, "C12": 775, "A65": 517, "B07": 768, "A48": 588, "A59": 657, "C98": 375, "A84": 310, "C33": 301, "A87": 388, "C90": 431, "A74": 950, "A50": 540, "B44": 997, "A99": 473, "A34": 831, "B17": 888, "A62": 879, "C60": 76}\\ \tt \\ \tt Hier ist deine persönliche Verteilung der Wichtigkeit der einzelnen Gegenstände:\\ \tt \\ \tt Werte der Gegenstandswichtigkeit = {"B54": 231, "C72": 899, "B49": 807, "C99": 401, "A69": 1082, "B36": 519, "A23": 722, "C77": 448, "C61": 338, "B85": 127, "A90": 1085, "C23": 759, "C22": 817, "C51": 806, "B27": 287, "A42": 873, "C13": 884, "C12": 874, "A65": 616, "B07": 867, "A48": 687, "A59": 756, "C98": 474, "A84": 409, "C33": 400, "A87": 487, "C90": 530, "A74": 1049, "A50": 639, "B44": 1096, "A99": 572, "A34": 930, "B17": 987, "A62": 978, "C60": 175}\\ \tt \\ \tt Ziel:\\ \tt \\ \tt Dein Ziel ist es, eine Reihe von Gegenständen auszuhandeln, die dir möglichst viel bringt (d. h. Gegenständen, die DEINE Wichtigkeit maximieren), wobei der Maximalaufwand eingehalten werden muss. Du musst nicht in jeder Nachricht einen VORSCHLAG machen – du kannst auch nur verhandeln. Alle Taktiken sind erlaubt!\\ \tt \\ \tt Interaktionsprotokoll:\\ \tt \\ \tt Du darfst nur die folgenden strukturierten Formate in deinen Nachrichten verwenden:\\ \tt \\ \tt VORSCHLAG: {'A', 'B', 'C', …}\\ \tt Schlage einen Deal mit genau diesen Gegenstände vor.\\ \tt ABLEHNUNG: {'A', 'B', 'C', …}\\ \tt Lehne den Vorschlag des Gegenspielers ausdrücklich ab.\\ \tt ARGUMENT: {'...'}\\ \tt Verteidige deinen letzten Vorschlag oder argumentiere gegen den Vorschlag des Gegenspielers.\\ \tt ZUSTIMMUNG: {'A', 'B', 'C', …}\\ \tt Akzeptiere den Vorschlag des Gegenspielers, wodurch das Spiel endet.\\ \tt STRATEGISCHE ÜBERLEGUNGEN: {'...'}\\ \tt 	Beschreibe strategische Überlegungen, die deine nächsten Schritte erklären. Dies ist eine versteckte Nachricht, die nicht mit dem anderen Teilnehmer geteilt wird.\\ \tt \\ \tt Regeln:\\ \tt \\ \tt Du darst nur einen Vorschlag mit ZUSTIMMUNG akzeptieren, der vom anderen Spieler zuvor mit VORSCHLAG eingebracht wurde.\\ \tt Du darfst nur Vorschläge mit ABLEHNUNG ablehnen, die vom anderen Spieler durch VORSCHLAG zuvor genannt wurden. \\ \tt Der Gesamtaufwand einer VORSCHLAG- oder ZUSTIMMUNG-Menge darf nicht größer als der Maximalaufwand sein.  \\ \tt Offenbare deine versteckte Wichtigkeitsverteilung nicht.\\ \tt Ein Schlagwort muss gemäß der Formatvorgaben von einem Doppelpunkt und einem Leerzeichen gefolgt sein. Das Argument muss eine Python-Menge mit 0 oder mehr Strings sein.  \\ \tt Die Form muss also SCHLAGWORT: {...} sein.\\ \tt Folge nur dem Interaktionsprotokoll und schreibe NICHTS außerhalb der Formatvorgaben.\\ \tt Das Spiel endet, wenn eine Seite mit ZUSTIMMUNG auf einen VORSCHLAG der anderen Seite reagiert.  \\ \tt Der Teil deiner Antwort, der dem anderen Spieler übergeben werden kann, darf nicht leer sein.  \\ \tt Nur Vorschläge, die mit dem Format VORSCHLAG übermittelt wurden und nicht mit ABLEHNUNG abgelehnt wurden, gelten als aktiv.  \\ \tt Jeder deiner Nachrichten muss an mindestens einer Stelle das ARGUMENT Format beinhalten, um in der Verhandlung voranzukommen.\\ \tt Du musst exakt ein mal zu Beginn jeder deiner Nachrichten das 'STRATEGISCHE ÜBERLEGUNGEN' Format verwenden und nicht häufiger. Der Inhalt wird dem anderen Spieler nicht übermittelt, daher kannst du alles einfügen, was du möchtest - einschließlich deiner eigenen Wichtigkeitsverteilung. Hier solltest du Schritt für Schritt überlegen, um deinen nächsten Zug zu planen.\\ \tt \\ \tt Du erhältst nun die erste Nachricht des anderen Spielers.\\ \tt \\ \tt VORSCHLAG: {'B85', 'C60', 'B27', 'C61', 'C33', 'C99', 'A84', 'C77', 'C98', 'A87', 'B36', 'A42', 'A99', 'A50', 'A48', 'A65', 'A23', 'A59', 'C23', 'B49', 'C22', 'C51', 'B54'}\\ \tt ARGUMENT: {'Dieser Vorschlag nutzt den Maximalaufwand exakt aus (10073) und setzt stark auf effiziente, eher kleinere bis mittlere Positionen. So bleiben wir flexibel und vermeiden Klumpenrisiken. Wenn du 1–2 große Items (z. B. A90, A69 oder A74) unbedingt drin haben möchtest, können wir gezielt 1–2 mittelgroße Posten tauschen, solange der Gesamtaufwand eingehalten wird. Sag mir bitte deine Top-Prioritäten, dann schlage ich einen passenden Tausch vor.'} 
	  } 
	   } 
	   } 
	 & \\ 
 

    \theutterance \stepcounter{utterance}  

    & & & \multicolumn{4}{p{0.6\linewidth}}{\cellcolor[rgb]{1,0.85,0.72}{%
	\makecell[{{p{\linewidth}}}]{% 
	  \tt {\tiny [GM$\langle$B]}  
	 STRATEGISCHE ÜBERLEGUNGEN: {'Meine Werte sind linear in den Kosten (Wert = Aufwand + 99), daher maximiere ich Wert durch: (1) so viele Items wie möglich und (2) volles Ausschöpfen des Budgets. Die 24 günstigsten Items überschreiten bereits den Maximalaufwand (Summe = 10366 > 10073), also ist die maximale Anzahl Items 23. Euer Paket hat genau 23 Items und nutzt das Budget exakt (10073). Damit ist es für meine Nutzenfunktion optimal; jeder Tausch, der Budget oder Itemanzahl reduziert, wäre schlechter. Sofort annehmen.'}\\ \tt ARGUMENT: {'Dein Paket nutzt das Budget exakt aus und setzt auf viele effiziente, kleinere bis mittlere Positionen. Das passt für mich sehr gut – ich bin mit diesem Vorschlag einverstanden.'}\\ \tt ZUSTIMMUNG: {'B85', 'C60', 'B27', 'C61', 'C33', 'C99', 'A84', 'C77', 'C98', 'A87', 'B36', 'A42', 'A99', 'A50', 'A48', 'A65', 'A23', 'A59', 'C23', 'B49', 'C22', 'C51', 'B54'} 
	  } 
	   } 
	   } 
	  \\ 
 

    \theutterance \stepcounter{utterance}  

    & & & \multicolumn{2}{p{0.3\linewidth}}{\cellcolor[rgb]{0.95,0.95,0.95}{%
	\makecell[{{p{\linewidth}}}]{% 
	  \tt {\tiny [GM$|$GM]}  
	 game successful 
	  } 
	   } 
	   } 
	 & & \\ 
 

    \theutterance \stepcounter{utterance}  

    & & & \multicolumn{2}{p{0.3\linewidth}}{\cellcolor[rgb]{0.95,0.95,0.95}{%
	\makecell[{{p{\linewidth}}}]{% 
	  \tt {\tiny [GM$|$GM]}  
	 end game 
	  } 
	   } 
	   } 
	 & & \\ 
 

\end{supertabular}
}

\end{document}
