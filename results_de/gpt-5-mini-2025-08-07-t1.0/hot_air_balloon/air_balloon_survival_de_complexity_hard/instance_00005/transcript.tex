
\documentclass{article}
\usepackage{colortbl}
\usepackage{makecell}
\usepackage{multirow}
\usepackage{supertabular}

\begin{document}

\newcounter{utterance}

\centering \large Interaction Transcript for game `hot\_air\_balloon', experiment `air\_balloon\_survival\_de\_complexity\_hard', episode 5 with gpt{-}5{-}mini{-}2025{-}08{-}07{-}t1.0.
\vspace{24pt}

{ \footnotesize  \setcounter{utterance}{1}
\setlength{\tabcolsep}{0pt}
\begin{supertabular}{c@{$\;$}|p{.15\linewidth}@{}p{.15\linewidth}p{.15\linewidth}p{.15\linewidth}p{.15\linewidth}p{.15\linewidth}}
    \# & $\;$A & \multicolumn{4}{c}{Game Master} & $\;\:$B\\
    \hline

    \theutterance \stepcounter{utterance}  
    & & \multicolumn{4}{p{0.6\linewidth}}{
        \cellcolor[rgb]{0.9,0.9,0.9}{
            \makecell[{{p{\linewidth}}}]{
                \texttt{\tiny{[P1$\langle$GM]}}
                \texttt{Du nimmst an einem kollaborativen Verhandlungspiel Teil.} \\
\\ 
\texttt{Zusammen mit einem anderen Teilnehmer musst du dich auf eine Reihe von Gegenständen entscheiden, die behalten werden. Jeder von euch hat eine persönliche Verteilung über die Wichtigkeit der einzelnen Gegenstände. Jeder von euch hat eine eigene Meinung darüber, wie wichtig jeder einzelne Gegenstand ist (Gegenstandswichtigkeit). Du kennst die Wichtigkeitsverteilung des anderen Spielers nicht. Zusätzlich siehst du, wie viel Aufwand jeder Gegenstand verursacht.  } \\
\texttt{Ihr dürft euch nur auf eine Reihe von Gegenständen einigen, wenn der Gesamtaufwand der ausgewählten Gegenstände den Maximalaufwand nicht überschreitet:} \\
\\ 
\texttt{Maximalaufwand = 7351} \\
\\ 
\texttt{Hier sind die einzelnen Aufwände der Gegenstände:} \\
\\ 
\texttt{Aufwand der Gegenstände = \{"B54": 518, "C72": 656, "B49": 96, "C99": 780, "A69": 190, "B36": 224, "A23": 231, "C77": 181, "C61": 312, "B85": 928, "A90": 826, "C23": 101, "C22": 61, "C51": 814, "B27": 322, "A42": 900, "C13": 744, "C12": 150, "A65": 65, "B07": 853, "A48": 454, "A59": 154, "C98": 237, "A84": 45, "C33": 766, "A87": 294, "C90": 985, "A74": 353, "A50": 60, "B44": 604, "A99": 92, "A34": 454, "B17": 206, "A62": 813, "C60": 234\}} \\
\\ 
\texttt{Hier ist deine persönliche Verteilung der Wichtigkeit der einzelnen Gegenstände:} \\
\\ 
\texttt{Werte der Gegenstandswichtigkeit = \{"B54": 616, "C72": 754, "B49": 194, "C99": 878, "A69": 288, "B36": 322, "A23": 329, "C77": 279, "C61": 410, "B85": 1026, "A90": 924, "C23": 199, "C22": 159, "C51": 912, "B27": 420, "A42": 998, "C13": 842, "C12": 248, "A65": 163, "B07": 951, "A48": 552, "A59": 252, "C98": 335, "A84": 143, "C33": 864, "A87": 392, "C90": 1083, "A74": 451, "A50": 158, "B44": 702, "A99": 190, "A34": 552, "B17": 304, "A62": 911, "C60": 332\}} \\
\\ 
\texttt{Ziel:} \\
\\ 
\texttt{Dein Ziel ist es, eine Reihe von Gegenständen auszuhandeln, die dir möglichst viel bringt (d. h. Gegenständen, die DEINE Wichtigkeit maximieren), wobei der Maximalaufwand eingehalten werden muss. Du musst nicht in jeder Nachricht einen VORSCHLAG machen – du kannst auch nur verhandeln. Alle Taktiken sind erlaubt!} \\
\\ 
\texttt{Interaktionsprotokoll:} \\
\\ 
\texttt{Du darfst nur die folgenden strukturierten Formate in deinen Nachrichten verwenden:} \\
\\ 
\texttt{VORSCHLAG: \{'A', 'B', 'C', …\}} \\
\texttt{Schlage einen Deal mit genau diesen Gegenstände vor.} \\
\texttt{ABLEHNUNG: \{'A', 'B', 'C', …\}} \\
\texttt{Lehne den Vorschlag des Gegenspielers ausdrücklich ab.} \\
\texttt{ARGUMENT: \{'...'\}} \\
\texttt{Verteidige deinen letzten Vorschlag oder argumentiere gegen den Vorschlag des Gegenspielers.} \\
\texttt{ZUSTIMMUNG: \{'A', 'B', 'C', …\}} \\
\texttt{Akzeptiere den Vorschlag des Gegenspielers, wodurch das Spiel endet.} \\
\texttt{STRATEGISCHE ÜBERLEGUNGEN: \{'...'\}} \\
\texttt{	Beschreibe strategische Überlegungen, die deine nächsten Schritte erklären. Dies ist eine versteckte Nachricht, die nicht mit dem anderen Teilnehmer geteilt wird.} \\
\\ 
\texttt{Regeln:} \\
\\ 
\texttt{Du darst nur einen Vorschlag mit ZUSTIMMUNG akzeptieren, der vom anderen Spieler zuvor mit VORSCHLAG eingebracht wurde.} \\
\texttt{Du darfst nur Vorschläge mit ABLEHNUNG ablehnen, die vom anderen Spieler durch VORSCHLAG zuvor genannt wurden. } \\
\texttt{Der Gesamtaufwand einer VORSCHLAG{-} oder ZUSTIMMUNG{-}Menge darf nicht größer als der Maximalaufwand sein.  } \\
\texttt{Offenbare deine versteckte Wichtigkeitsverteilung nicht.} \\
\texttt{Ein Schlagwort muss gemäß der Formatvorgaben von einem Doppelpunkt und einem Leerzeichen gefolgt sein. Das Argument muss eine Python{-}Menge mit 0 oder mehr Strings sein.  } \\
\texttt{Die Form muss also SCHLAGWORT: \{...\} sein.} \\
\texttt{Folge nur dem Interaktionsprotokoll und schreibe NICHTS außerhalb der Formatvorgaben.} \\
\texttt{Das Spiel endet, wenn eine Seite mit ZUSTIMMUNG auf einen VORSCHLAG der anderen Seite reagiert.  } \\
\texttt{Der Teil deiner Antwort, der dem anderen Spieler übergeben werden kann, darf nicht leer sein.  } \\
\texttt{Nur Vorschläge, die mit dem Format VORSCHLAG übermittelt wurden und nicht mit ABLEHNUNG abgelehnt wurden, gelten als aktiv.  } \\
\texttt{Jeder deiner Nachrichten muss an mindestens einer Stelle das ARGUMENT Format beinhalten, um in der Verhandlung voranzukommen.} \\
\texttt{Du musst exakt ein mal zu Beginn jeder deiner Nachrichten das 'STRATEGISCHE ÜBERLEGUNGEN' Format verwenden und nicht häufiger. Der Inhalt wird dem anderen Spieler nicht übermittelt, daher kannst du alles einfügen, was du möchtest {-} einschließlich deiner eigenen Wichtigkeitsverteilung. Hier solltest du Schritt für Schritt überlegen, um deinen nächsten Zug zu planen.} \\
            }
        }
    }
    & \\ \\

    \theutterance \stepcounter{utterance}  
    & \multicolumn{4}{p{0.6\linewidth}}{
        \cellcolor[rgb]{0.9,0.9,0.9}{
            \makecell[{{p{\linewidth}}}]{
                \texttt{\tiny{[P1$\rangle$GM]}}
                \texttt{STRATEGISCHE ÜBERLEGUNGEN: \{'Ziel: meine Wichtigkeit maximieren bei Aufwand ≤ 7351. Vorgehen: Items nach Nutzen/Aufwand sortiert (Greedy), dann lokale Swaps getestet. Beste gefundene Kombi (Wert 9898, Aufwand 7350): Ausgangsset S26 und A48→C33 getauscht. Starke Alternativen als spätere Zugeständnisse: (1) Basis{-}S26 ohne C33 (Wert 9586, Aufwand 7038); (2) B44→A42 (Wert 9882, Aufwand 7334); (3) B54→A90 (Wert 9894, Aufwand 7346); (4) A34→C33 statt A48→C33 (gleichwertig). Taktik: mit meinem Optimum eröffnen, Effizienz argumentieren, nach Gegenüber{-}Prioritäten fragen. Falls C33 strittig, biete Rücktausch C33↔A48/A34 an; falls A42 später Thema, kann ich auf Tausch B44→A42 verzichten. Werte bleiben geheim.'\}} \\
\texttt{VORSCHLAG: \{'A84', 'A50', 'C22', 'A65', 'A99', 'B49', 'C23', 'C12', 'A59', 'C77', 'A69', 'B17', 'B36', 'A23', 'C60', 'C98', 'A87', 'C61', 'B27', 'A74', 'C33', 'A34', 'B54', 'B44', 'C72', 'C13'\}} \\
\texttt{ARGUMENT: \{'Ich schlage diese Auswahl vor, weil sie viele sehr effiziente, kleine Posten bündelt und mit Gesamtaufwand 7350 sicher unter dem Limit 7351 bleibt (1 Punkt Puffer). Teure, weniger effiziente Brocken wurden bewusst vermieden; C33 ist der gezielte „Booster“, der uns noch unter dem Limit hält. Wenn dir bestimmte Gegenstände besonders wichtig sind, nenne mir bitte deine Top{-}3, dann kann ich das Set gezielt anpassen, ohne das Limit zu reißen.'\}} \\
            }
        }
    }
    & & \\ \\

    \theutterance \stepcounter{utterance}  
    & & \multicolumn{4}{p{0.6\linewidth}}{
        \cellcolor[rgb]{0.9,0.9,0.9}{
            \makecell[{{p{\linewidth}}}]{
                \texttt{\tiny{[GM$\rangle$P2]}}
                \texttt{Du nimmst an einem kollaborativen Verhandlungspiel Teil.} \\
\\ 
\texttt{Zusammen mit einem anderen Teilnehmer musst du dich auf eine Reihe von Gegenständen entscheiden, die behalten werden. Jeder von euch hat eine persönliche Verteilung über die Wichtigkeit der einzelnen Gegenstände. Jeder von euch hat eine eigene Meinung darüber, wie wichtig jeder einzelne Gegenstand ist (Gegenstandswichtigkeit). Du kennst die Wichtigkeitsverteilung des anderen Spielers nicht. Zusätzlich siehst du, wie viel Aufwand jeder Gegenstand verursacht.  } \\
\texttt{Ihr dürft euch nur auf eine Reihe von Gegenständen einigen, wenn der Gesamtaufwand der ausgewählten Gegenstände den Maximalaufwand nicht überschreitet:} \\
\\ 
\texttt{Maximalaufwand = 7351} \\
\\ 
\texttt{Hier sind die einzelnen Aufwände der Gegenstände:} \\
\\ 
\texttt{Aufwand der Gegenstände = \{"B54": 518, "C72": 656, "B49": 96, "C99": 780, "A69": 190, "B36": 224, "A23": 231, "C77": 181, "C61": 312, "B85": 928, "A90": 826, "C23": 101, "C22": 61, "C51": 814, "B27": 322, "A42": 900, "C13": 744, "C12": 150, "A65": 65, "B07": 853, "A48": 454, "A59": 154, "C98": 237, "A84": 45, "C33": 766, "A87": 294, "C90": 985, "A74": 353, "A50": 60, "B44": 604, "A99": 92, "A34": 454, "B17": 206, "A62": 813, "C60": 234\}} \\
\\ 
\texttt{Hier ist deine persönliche Verteilung der Wichtigkeit der einzelnen Gegenstände:} \\
\\ 
\texttt{Werte der Gegenstandswichtigkeit = \{"B54": 616, "C72": 754, "B49": 194, "C99": 878, "A69": 288, "B36": 322, "A23": 329, "C77": 279, "C61": 410, "B85": 1026, "A90": 924, "C23": 199, "C22": 159, "C51": 912, "B27": 420, "A42": 998, "C13": 842, "C12": 248, "A65": 163, "B07": 951, "A48": 552, "A59": 252, "C98": 335, "A84": 143, "C33": 864, "A87": 392, "C90": 1083, "A74": 451, "A50": 158, "B44": 702, "A99": 190, "A34": 552, "B17": 304, "A62": 911, "C60": 332\}} \\
\\ 
\texttt{Ziel:} \\
\\ 
\texttt{Dein Ziel ist es, eine Reihe von Gegenständen auszuhandeln, die dir möglichst viel bringt (d. h. Gegenständen, die DEINE Wichtigkeit maximieren), wobei der Maximalaufwand eingehalten werden muss. Du musst nicht in jeder Nachricht einen VORSCHLAG machen – du kannst auch nur verhandeln. Alle Taktiken sind erlaubt!} \\
\\ 
\texttt{Interaktionsprotokoll:} \\
\\ 
\texttt{Du darfst nur die folgenden strukturierten Formate in deinen Nachrichten verwenden:} \\
\\ 
\texttt{VORSCHLAG: \{'A', 'B', 'C', …\}} \\
\texttt{Schlage einen Deal mit genau diesen Gegenstände vor.} \\
\texttt{ABLEHNUNG: \{'A', 'B', 'C', …\}} \\
\texttt{Lehne den Vorschlag des Gegenspielers ausdrücklich ab.} \\
\texttt{ARGUMENT: \{'...'\}} \\
\texttt{Verteidige deinen letzten Vorschlag oder argumentiere gegen den Vorschlag des Gegenspielers.} \\
\texttt{ZUSTIMMUNG: \{'A', 'B', 'C', …\}} \\
\texttt{Akzeptiere den Vorschlag des Gegenspielers, wodurch das Spiel endet.} \\
\texttt{STRATEGISCHE ÜBERLEGUNGEN: \{'...'\}} \\
\texttt{	Beschreibe strategische Überlegungen, die deine nächsten Schritte erklären. Dies ist eine versteckte Nachricht, die nicht mit dem anderen Teilnehmer geteilt wird.} \\
\\ 
\texttt{Regeln:} \\
\\ 
\texttt{Du darst nur einen Vorschlag mit ZUSTIMMUNG akzeptieren, der vom anderen Spieler zuvor mit VORSCHLAG eingebracht wurde.} \\
\texttt{Du darfst nur Vorschläge mit ABLEHNUNG ablehnen, die vom anderen Spieler durch VORSCHLAG zuvor genannt wurden. } \\
\texttt{Der Gesamtaufwand einer VORSCHLAG{-} oder ZUSTIMMUNG{-}Menge darf nicht größer als der Maximalaufwand sein.  } \\
\texttt{Offenbare deine versteckte Wichtigkeitsverteilung nicht.} \\
\texttt{Ein Schlagwort muss gemäß der Formatvorgaben von einem Doppelpunkt und einem Leerzeichen gefolgt sein. Das Argument muss eine Python{-}Menge mit 0 oder mehr Strings sein.  } \\
\texttt{Die Form muss also SCHLAGWORT: \{...\} sein.} \\
\texttt{Folge nur dem Interaktionsprotokoll und schreibe NICHTS außerhalb der Formatvorgaben.} \\
\texttt{Das Spiel endet, wenn eine Seite mit ZUSTIMMUNG auf einen VORSCHLAG der anderen Seite reagiert.  } \\
\texttt{Der Teil deiner Antwort, der dem anderen Spieler übergeben werden kann, darf nicht leer sein.  } \\
\texttt{Nur Vorschläge, die mit dem Format VORSCHLAG übermittelt wurden und nicht mit ABLEHNUNG abgelehnt wurden, gelten als aktiv.  } \\
\texttt{Jeder deiner Nachrichten muss an mindestens einer Stelle das ARGUMENT Format beinhalten, um in der Verhandlung voranzukommen.} \\
\texttt{Du musst exakt ein mal zu Beginn jeder deiner Nachrichten das 'STRATEGISCHE ÜBERLEGUNGEN' Format verwenden und nicht häufiger. Der Inhalt wird dem anderen Spieler nicht übermittelt, daher kannst du alles einfügen, was du möchtest {-} einschließlich deiner eigenen Wichtigkeitsverteilung. Hier solltest du Schritt für Schritt überlegen, um deinen nächsten Zug zu planen.} \\
\\ 
\texttt{Du erhältst nun die erste Nachricht des anderen Spielers.} \\
\\ 
\texttt{VORSCHLAG: \{'A84', 'A50', 'C22', 'A65', 'A99', 'B49', 'C23', 'C12', 'A59', 'C77', 'A69', 'B17', 'B36', 'A23', 'C60', 'C98', 'A87', 'C61', 'B27', 'A74', 'C33', 'A34', 'B54', 'B44', 'C72', 'C13'\}} \\
\texttt{ARGUMENT: \{'Ich schlage diese Auswahl vor, weil sie viele sehr effiziente, kleine Posten bündelt und mit Gesamtaufwand 7350 sicher unter dem Limit 7351 bleibt (1 Punkt Puffer). Teure, weniger effiziente Brocken wurden bewusst vermieden; C33 ist der gezielte „Booster“, der uns noch unter dem Limit hält. Wenn dir bestimmte Gegenstände besonders wichtig sind, nenne mir bitte deine Top{-}3, dann kann ich das Set gezielt anpassen, ohne das Limit zu reißen.'\}} \\
            }
        }
    }
    & \\ \\

    \theutterance \stepcounter{utterance}  
    & & & \multicolumn{4}{p{0.6\linewidth}}{
        \cellcolor[rgb]{0.9,0.9,0.9}{
            \makecell[{{p{\linewidth}}}]{
                \texttt{\tiny{[GM$\langle$P2]}}
                \texttt{STRATEGISCHE ÜBERLEGUNGEN: \{'Ihr Vorschlag nutzt das Budget extrem effizient (7350/7351) und stapelt viele hoch{-}dichte Kleinteile, die in meiner Bewertung sehr gut abschneiden. Ein Versuch, große Brocken wie C90/B85/A42 einzubauen, würde mich zwingen, mehrere dieser effizienten Posten zu streichen und senkt meinen Gesamtnutzen. Gegen mein Profil ist dieses Paket nahezu optimal; daher akzeptiere ich sofort, um den sicheren hohen Nutzen zu sichern.'\}} \\
\texttt{ARGUMENT: \{'Der Korb ist mit 7350 von 7351 nahezu perfekt ausgelastet und kombiniert viele effiziente Kleinteile mit wenigen Boostern – für mich eine sehr starke Balance. Ich bin bereit, das so zu fixieren.'\}} \\
\texttt{ZUSTIMMUNG: \{'A84', 'A50', 'C22', 'A65', 'A99', 'B49', 'C23', 'C12', 'A59', 'C77', 'A69', 'B17', 'B36', 'A23', 'C60', 'C98', 'A87', 'C61', 'B27', 'A74', 'C33', 'A34', 'B54', 'B44', 'C72', 'C13'\}} \\
            }
        }
    }
     \\ \\

    \theutterance \stepcounter{utterance}  
    & & & \multicolumn{2}{p{0.3\linewidth}}{
        \cellcolor[rgb]{0.9,0.9,0.9}{
            \makecell[{{p{\linewidth}}}]{
                \texttt{\tiny{[GM$|$GM]}}
                \texttt{game successful} \\
            }
        }
    }
    & & \\ \\

    \theutterance \stepcounter{utterance}  
    & & & \multicolumn{2}{p{0.3\linewidth}}{
        \cellcolor[rgb]{0.9,0.9,0.9}{
            \makecell[{{p{\linewidth}}}]{
                \texttt{\tiny{[GM$|$GM]}}
                \texttt{end game} \\
            }
        }
    }
    & & \\ \\

\end{supertabular}
}

\end{document}
