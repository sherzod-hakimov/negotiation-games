
\documentclass{article}
\usepackage{colortbl}
\usepackage{makecell}
\usepackage{multirow}
\usepackage{supertabular}

\begin{document}

\newcounter{utterance}

\centering \large Interaction Transcript for game `hot\_air\_balloon', experiment `air\_balloon\_survival\_de\_complexity\_hard', episode 3 with gpt{-}5{-}mini{-}2025{-}08{-}07{-}t1.0.
\vspace{24pt}

{ \footnotesize  \setcounter{utterance}{1}
\setlength{\tabcolsep}{0pt}
\begin{supertabular}{c@{$\;$}|p{.15\linewidth}@{}p{.15\linewidth}p{.15\linewidth}p{.15\linewidth}p{.15\linewidth}p{.15\linewidth}}
    \# & $\;$A & \multicolumn{4}{c}{Game Master} & $\;\:$B\\
    \hline

    \theutterance \stepcounter{utterance}  
    & & \multicolumn{4}{p{0.6\linewidth}}{
        \cellcolor[rgb]{0.9,0.9,0.9}{
            \makecell[{{p{\linewidth}}}]{
                \texttt{\tiny{[P1$\langle$GM]}}
                \texttt{Du nimmst an einem kollaborativen Verhandlungspiel Teil.} \\
\\ 
\texttt{Zusammen mit einem anderen Teilnehmer musst du dich auf eine Reihe von Gegenständen entscheiden, die behalten werden. Jeder von euch hat eine persönliche Verteilung über die Wichtigkeit der einzelnen Gegenstände. Jeder von euch hat eine eigene Meinung darüber, wie wichtig jeder einzelne Gegenstand ist (Gegenstandswichtigkeit). Du kennst die Wichtigkeitsverteilung des anderen Spielers nicht. Zusätzlich siehst du, wie viel Aufwand jeder Gegenstand verursacht.  } \\
\texttt{Ihr dürft euch nur auf eine Reihe von Gegenständen einigen, wenn der Gesamtaufwand der ausgewählten Gegenstände den Maximalaufwand nicht überschreitet:} \\
\\ 
\texttt{Maximalaufwand = 6938} \\
\\ 
\texttt{Hier sind die einzelnen Aufwände der Gegenstände:} \\
\\ 
\texttt{Aufwand der Gegenstände = \{"B54": 997, "C72": 950, "B49": 272, "C99": 448, "A69": 382, "B36": 804, "A23": 237, "C77": 56, "C61": 105, "B85": 612, "A90": 528, "C23": 527, "C22": 524, "C51": 167, "B27": 133, "A42": 300, "C13": 945, "C12": 50, "A65": 906, "B07": 70, "A48": 224, "A59": 4, "C98": 689, "A84": 64, "C33": 434, "A87": 750, "C90": 733, "A74": 873, "A50": 22, "B44": 68, "A99": 57, "A34": 10, "B17": 36, "A62": 551, "C60": 348\}} \\
\\ 
\texttt{Hier ist deine persönliche Verteilung der Wichtigkeit der einzelnen Gegenstände:} \\
\\ 
\texttt{Werte der Gegenstandswichtigkeit = \{"B54": 1096, "C72": 1049, "B49": 371, "C99": 547, "A69": 481, "B36": 903, "A23": 336, "C77": 155, "C61": 204, "B85": 711, "A90": 627, "C23": 626, "C22": 623, "C51": 266, "B27": 232, "A42": 399, "C13": 1044, "C12": 149, "A65": 1005, "B07": 169, "A48": 323, "A59": 103, "C98": 788, "A84": 163, "C33": 533, "A87": 849, "C90": 832, "A74": 972, "A50": 121, "B44": 167, "A99": 156, "A34": 109, "B17": 135, "A62": 650, "C60": 447\}} \\
\\ 
\texttt{Ziel:} \\
\\ 
\texttt{Dein Ziel ist es, eine Reihe von Gegenständen auszuhandeln, die dir möglichst viel bringt (d. h. Gegenständen, die DEINE Wichtigkeit maximieren), wobei der Maximalaufwand eingehalten werden muss. Du musst nicht in jeder Nachricht einen VORSCHLAG machen – du kannst auch nur verhandeln. Alle Taktiken sind erlaubt!} \\
\\ 
\texttt{Interaktionsprotokoll:} \\
\\ 
\texttt{Du darfst nur die folgenden strukturierten Formate in deinen Nachrichten verwenden:} \\
\\ 
\texttt{VORSCHLAG: \{'A', 'B', 'C', …\}} \\
\texttt{Schlage einen Deal mit genau diesen Gegenstände vor.} \\
\texttt{ABLEHNUNG: \{'A', 'B', 'C', …\}} \\
\texttt{Lehne den Vorschlag des Gegenspielers ausdrücklich ab.} \\
\texttt{ARGUMENT: \{'...'\}} \\
\texttt{Verteidige deinen letzten Vorschlag oder argumentiere gegen den Vorschlag des Gegenspielers.} \\
\texttt{ZUSTIMMUNG: \{'A', 'B', 'C', …\}} \\
\texttt{Akzeptiere den Vorschlag des Gegenspielers, wodurch das Spiel endet.} \\
\texttt{STRATEGISCHE ÜBERLEGUNGEN: \{'...'\}} \\
\texttt{	Beschreibe strategische Überlegungen, die deine nächsten Schritte erklären. Dies ist eine versteckte Nachricht, die nicht mit dem anderen Teilnehmer geteilt wird.} \\
\\ 
\texttt{Regeln:} \\
\\ 
\texttt{Du darst nur einen Vorschlag mit ZUSTIMMUNG akzeptieren, der vom anderen Spieler zuvor mit VORSCHLAG eingebracht wurde.} \\
\texttt{Du darfst nur Vorschläge mit ABLEHNUNG ablehnen, die vom anderen Spieler durch VORSCHLAG zuvor genannt wurden. } \\
\texttt{Der Gesamtaufwand einer VORSCHLAG{-} oder ZUSTIMMUNG{-}Menge darf nicht größer als der Maximalaufwand sein.  } \\
\texttt{Offenbare deine versteckte Wichtigkeitsverteilung nicht.} \\
\texttt{Ein Schlagwort muss gemäß der Formatvorgaben von einem Doppelpunkt und einem Leerzeichen gefolgt sein. Das Argument muss eine Python{-}Menge mit 0 oder mehr Strings sein.  } \\
\texttt{Die Form muss also SCHLAGWORT: \{...\} sein.} \\
\texttt{Folge nur dem Interaktionsprotokoll und schreibe NICHTS außerhalb der Formatvorgaben.} \\
\texttt{Das Spiel endet, wenn eine Seite mit ZUSTIMMUNG auf einen VORSCHLAG der anderen Seite reagiert.  } \\
\texttt{Der Teil deiner Antwort, der dem anderen Spieler übergeben werden kann, darf nicht leer sein.  } \\
\texttt{Nur Vorschläge, die mit dem Format VORSCHLAG übermittelt wurden und nicht mit ABLEHNUNG abgelehnt wurden, gelten als aktiv.  } \\
\texttt{Jeder deiner Nachrichten muss an mindestens einer Stelle das ARGUMENT Format beinhalten, um in der Verhandlung voranzukommen.} \\
\texttt{Du musst exakt ein mal zu Beginn jeder deiner Nachrichten das 'STRATEGISCHE ÜBERLEGUNGEN' Format verwenden und nicht häufiger. Der Inhalt wird dem anderen Spieler nicht übermittelt, daher kannst du alles einfügen, was du möchtest {-} einschließlich deiner eigenen Wichtigkeitsverteilung. Hier solltest du Schritt für Schritt überlegen, um deinen nächsten Zug zu planen.} \\
            }
        }
    }
    & \\ \\

    \theutterance \stepcounter{utterance}  
    & \multicolumn{4}{p{0.6\linewidth}}{
        \cellcolor[rgb]{0.9,0.9,0.9}{
            \makecell[{{p{\linewidth}}}]{
                \texttt{\tiny{[P1$\rangle$GM]}}
                \texttt{STRATEGISCHE ÜBERLEGUNGEN: \{'Ziel: sofort ein sehr starkes, nahe{-}optimales Paket platzieren, das fast das ganze Budget nutzt und viele meiner Top{-}Ratio{-}Items enthält, damit wenig Spielraum für teure, für mich schlechtere Tauschgeschäfte bleibt. Rechenweg (interne Werte): Ich habe eine knapsack{-}nahe Auswahl per Nutzen/Aufwand{-}Ratio aufgebaut. Enthaltene Items und Summen: Gesamtaufwand 6918/6938, Gesamtwert 9591. Enthalten: A59(4,103), A34(10,109), A50(22,121), B17(36,135), C12(50,149), C77(56,155), A99(57,156), A84(64,163), B44(68,167), B07(70,169), C61(105,204), B27(133,232), C51(167,266), B49(272,371), A42(300,399), C60(348,447), A69(382,481), C33(434,533), C99(448,547), A90(528,627), C22(524,623), C23(527,626), A62(551,650), B85(612,711), C98(689,788), A48(224,323), A23(237,336). Freier Puffer 20. Potenzielle Tausch{-}Blöcke, falls nötig: eher mittelgroße wie B85(612,711), C98(689,788), A62(551,650) – aber nur gegen klaren Gegenwert, weil die kleinen Items extrem hohe Effizienz haben und ich sie ungern aufgebe. Ich werde argumentieren mit Budgettreue, Effizienz und Puffer; zugleich Offenheit signalisieren, gezielt 1–2 mittelgroße Items zu verschieben, falls der/die andere dafür etwas ebenso effizientes einbringt. Keine Offenlegung meiner genauen Bewertungen.'\}} \\
\texttt{VORSCHLAG: \{'A59', 'A34', 'A50', 'B17', 'C12', 'C77', 'A99', 'A84', 'B44', 'B07', 'C61', 'B27', 'C51', 'B49', 'A42', 'C60', 'A69', 'C33', 'C99', 'A90', 'C22', 'C23', 'A62', 'B85', 'C98', 'A48', 'A23'\}} \\
\texttt{ARGUMENT: \{'Dieser Vorschlag bleibt mit Gesamtaufwand 6918 unter dem Limit 6938 (20 Puffer) und setzt stark auf besonders effiziente Gegenstände, sodass wir für jeden Aufwandsschritt viel Nutzen bekommen. Falls dir einzelne Schlüsselposten fehlen, sag mir bitte konkret, welche dir wichtig sind – ich kann punktuell ein oder zwei mittelgroße Positionen umschichten, wenn der Gegenwert in Aufwand und Nutzen passt, ohne die Effizienz des Gesamtpakets zu stark zu verwässern.'\}} \\
            }
        }
    }
    & & \\ \\

    \theutterance \stepcounter{utterance}  
    & & \multicolumn{4}{p{0.6\linewidth}}{
        \cellcolor[rgb]{0.9,0.9,0.9}{
            \makecell[{{p{\linewidth}}}]{
                \texttt{\tiny{[GM$\rangle$P2]}}
                \texttt{Du nimmst an einem kollaborativen Verhandlungspiel Teil.} \\
\\ 
\texttt{Zusammen mit einem anderen Teilnehmer musst du dich auf eine Reihe von Gegenständen entscheiden, die behalten werden. Jeder von euch hat eine persönliche Verteilung über die Wichtigkeit der einzelnen Gegenstände. Jeder von euch hat eine eigene Meinung darüber, wie wichtig jeder einzelne Gegenstand ist (Gegenstandswichtigkeit). Du kennst die Wichtigkeitsverteilung des anderen Spielers nicht. Zusätzlich siehst du, wie viel Aufwand jeder Gegenstand verursacht.  } \\
\texttt{Ihr dürft euch nur auf eine Reihe von Gegenständen einigen, wenn der Gesamtaufwand der ausgewählten Gegenstände den Maximalaufwand nicht überschreitet:} \\
\\ 
\texttt{Maximalaufwand = 6938} \\
\\ 
\texttt{Hier sind die einzelnen Aufwände der Gegenstände:} \\
\\ 
\texttt{Aufwand der Gegenstände = \{"B54": 997, "C72": 950, "B49": 272, "C99": 448, "A69": 382, "B36": 804, "A23": 237, "C77": 56, "C61": 105, "B85": 612, "A90": 528, "C23": 527, "C22": 524, "C51": 167, "B27": 133, "A42": 300, "C13": 945, "C12": 50, "A65": 906, "B07": 70, "A48": 224, "A59": 4, "C98": 689, "A84": 64, "C33": 434, "A87": 750, "C90": 733, "A74": 873, "A50": 22, "B44": 68, "A99": 57, "A34": 10, "B17": 36, "A62": 551, "C60": 348\}} \\
\\ 
\texttt{Hier ist deine persönliche Verteilung der Wichtigkeit der einzelnen Gegenstände:} \\
\\ 
\texttt{Werte der Gegenstandswichtigkeit = \{"B54": 1096, "C72": 1049, "B49": 371, "C99": 547, "A69": 481, "B36": 903, "A23": 336, "C77": 155, "C61": 204, "B85": 711, "A90": 627, "C23": 626, "C22": 623, "C51": 266, "B27": 232, "A42": 399, "C13": 1044, "C12": 149, "A65": 1005, "B07": 169, "A48": 323, "A59": 103, "C98": 788, "A84": 163, "C33": 533, "A87": 849, "C90": 832, "A74": 972, "A50": 121, "B44": 167, "A99": 156, "A34": 109, "B17": 135, "A62": 650, "C60": 447\}} \\
\\ 
\texttt{Ziel:} \\
\\ 
\texttt{Dein Ziel ist es, eine Reihe von Gegenständen auszuhandeln, die dir möglichst viel bringt (d. h. Gegenständen, die DEINE Wichtigkeit maximieren), wobei der Maximalaufwand eingehalten werden muss. Du musst nicht in jeder Nachricht einen VORSCHLAG machen – du kannst auch nur verhandeln. Alle Taktiken sind erlaubt!} \\
\\ 
\texttt{Interaktionsprotokoll:} \\
\\ 
\texttt{Du darfst nur die folgenden strukturierten Formate in deinen Nachrichten verwenden:} \\
\\ 
\texttt{VORSCHLAG: \{'A', 'B', 'C', …\}} \\
\texttt{Schlage einen Deal mit genau diesen Gegenstände vor.} \\
\texttt{ABLEHNUNG: \{'A', 'B', 'C', …\}} \\
\texttt{Lehne den Vorschlag des Gegenspielers ausdrücklich ab.} \\
\texttt{ARGUMENT: \{'...'\}} \\
\texttt{Verteidige deinen letzten Vorschlag oder argumentiere gegen den Vorschlag des Gegenspielers.} \\
\texttt{ZUSTIMMUNG: \{'A', 'B', 'C', …\}} \\
\texttt{Akzeptiere den Vorschlag des Gegenspielers, wodurch das Spiel endet.} \\
\texttt{STRATEGISCHE ÜBERLEGUNGEN: \{'...'\}} \\
\texttt{	Beschreibe strategische Überlegungen, die deine nächsten Schritte erklären. Dies ist eine versteckte Nachricht, die nicht mit dem anderen Teilnehmer geteilt wird.} \\
\\ 
\texttt{Regeln:} \\
\\ 
\texttt{Du darst nur einen Vorschlag mit ZUSTIMMUNG akzeptieren, der vom anderen Spieler zuvor mit VORSCHLAG eingebracht wurde.} \\
\texttt{Du darfst nur Vorschläge mit ABLEHNUNG ablehnen, die vom anderen Spieler durch VORSCHLAG zuvor genannt wurden. } \\
\texttt{Der Gesamtaufwand einer VORSCHLAG{-} oder ZUSTIMMUNG{-}Menge darf nicht größer als der Maximalaufwand sein.  } \\
\texttt{Offenbare deine versteckte Wichtigkeitsverteilung nicht.} \\
\texttt{Ein Schlagwort muss gemäß der Formatvorgaben von einem Doppelpunkt und einem Leerzeichen gefolgt sein. Das Argument muss eine Python{-}Menge mit 0 oder mehr Strings sein.  } \\
\texttt{Die Form muss also SCHLAGWORT: \{...\} sein.} \\
\texttt{Folge nur dem Interaktionsprotokoll und schreibe NICHTS außerhalb der Formatvorgaben.} \\
\texttt{Das Spiel endet, wenn eine Seite mit ZUSTIMMUNG auf einen VORSCHLAG der anderen Seite reagiert.  } \\
\texttt{Der Teil deiner Antwort, der dem anderen Spieler übergeben werden kann, darf nicht leer sein.  } \\
\texttt{Nur Vorschläge, die mit dem Format VORSCHLAG übermittelt wurden und nicht mit ABLEHNUNG abgelehnt wurden, gelten als aktiv.  } \\
\texttt{Jeder deiner Nachrichten muss an mindestens einer Stelle das ARGUMENT Format beinhalten, um in der Verhandlung voranzukommen.} \\
\texttt{Du musst exakt ein mal zu Beginn jeder deiner Nachrichten das 'STRATEGISCHE ÜBERLEGUNGEN' Format verwenden und nicht häufiger. Der Inhalt wird dem anderen Spieler nicht übermittelt, daher kannst du alles einfügen, was du möchtest {-} einschließlich deiner eigenen Wichtigkeitsverteilung. Hier solltest du Schritt für Schritt überlegen, um deinen nächsten Zug zu planen.} \\
\\ 
\texttt{Du erhältst nun die erste Nachricht des anderen Spielers.} \\
\\ 
\texttt{VORSCHLAG: \{'A59', 'A34', 'A50', 'B17', 'C12', 'C77', 'A99', 'A84', 'B44', 'B07', 'C61', 'B27', 'C51', 'B49', 'A42', 'C60', 'A69', 'C33', 'C99', 'A90', 'C22', 'C23', 'A62', 'B85', 'C98', 'A48', 'A23'\}} \\
\texttt{ARGUMENT: \{'Dieser Vorschlag bleibt mit Gesamtaufwand 6918 unter dem Limit 6938 (20 Puffer) und setzt stark auf besonders effiziente Gegenstände, sodass wir für jeden Aufwandsschritt viel Nutzen bekommen. Falls dir einzelne Schlüsselposten fehlen, sag mir bitte konkret, welche dir wichtig sind – ich kann punktuell ein oder zwei mittelgroße Positionen umschichten, wenn der Gegenwert in Aufwand und Nutzen passt, ohne die Effizienz des Gesamtpakets zu stark zu verwässern.'\}} \\
            }
        }
    }
    & \\ \\

    \theutterance \stepcounter{utterance}  
    & & & \multicolumn{4}{p{0.6\linewidth}}{
        \cellcolor[rgb]{0.9,0.9,0.9}{
            \makecell[{{p{\linewidth}}}]{
                \texttt{\tiny{[GM$\langle$P2]}}
                \texttt{STRATEGISCHE ÜBERLEGUNGEN: \{'Ihr Vorschlag erzielt nach meiner Bewertung einen sehr hohen Gesamtnutzen für mich (deutlich besser als Varianten, in denen ich einzelne große, aber weniger dichte Posten wie B54/C72/C13/A65/A74/A87 hineinreklamiere). Durch die vielen sehr effizienten Klein{-} und Mittelposten ist das Paket für meine Präferenzen quasi nahe am Optimum; Austausch gegen große Einzelstücke würde meine Summe senken. Deshalb akzeptiere ich sofort, um keinen Wert durch unnötige Umschichtungen zu riskieren.'\}} \\
\texttt{ARGUMENT: \{'Das Paket ist sehr ausgewogen und nutzt das Aufwandslimit hocheffizient – ich sehe keinen besseren Tausch ohne Wertverlust.'\}} \\
\texttt{ZUSTIMMUNG: \{'A59', 'A34', 'A50', 'B17', 'C12', 'C77', 'A99', 'A84', 'B44', 'B07', 'C61', 'B27', 'C51', 'B49', 'A42', 'C60', 'A69', 'C33', 'C99', 'A90', 'C22', 'C23', 'A62', 'B85', 'C98', 'A48', 'A23'\}} \\
            }
        }
    }
     \\ \\

    \theutterance \stepcounter{utterance}  
    & & & \multicolumn{2}{p{0.3\linewidth}}{
        \cellcolor[rgb]{0.9,0.9,0.9}{
            \makecell[{{p{\linewidth}}}]{
                \texttt{\tiny{[GM$|$GM]}}
                \texttt{game successful} \\
            }
        }
    }
    & & \\ \\

    \theutterance \stepcounter{utterance}  
    & & & \multicolumn{2}{p{0.3\linewidth}}{
        \cellcolor[rgb]{0.9,0.9,0.9}{
            \makecell[{{p{\linewidth}}}]{
                \texttt{\tiny{[GM$|$GM]}}
                \texttt{end game} \\
            }
        }
    }
    & & \\ \\

\end{supertabular}
}

\end{document}
