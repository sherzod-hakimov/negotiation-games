
\documentclass{article}
\usepackage{colortbl}
\usepackage{makecell}
\usepackage{multirow}
\usepackage{supertabular}

\begin{document}

\newcounter{utterance}

\centering \large Interaction Transcript for game `hot\_air\_balloon', experiment `air\_balloon\_survival\_de\_complexity\_hard', episode 2 with gpt{-}5{-}mini{-}2025{-}08{-}07{-}t1.0.
\vspace{24pt}

{ \footnotesize  \setcounter{utterance}{1}
\setlength{\tabcolsep}{0pt}
\begin{supertabular}{c@{$\;$}|p{.15\linewidth}@{}p{.15\linewidth}p{.15\linewidth}p{.15\linewidth}p{.15\linewidth}p{.15\linewidth}}
    \# & $\;$A & \multicolumn{4}{c}{Game Master} & $\;\:$B\\
    \hline

    \theutterance \stepcounter{utterance}  
    & & \multicolumn{4}{p{0.6\linewidth}}{
        \cellcolor[rgb]{0.9,0.9,0.9}{
            \makecell[{{p{\linewidth}}}]{
                \texttt{\tiny{[P1$\langle$GM]}}
                \texttt{Du nimmst an einem kollaborativen Verhandlungspiel Teil.} \\
\\ 
\texttt{Zusammen mit einem anderen Teilnehmer musst du dich auf eine Reihe von Gegenständen entscheiden, die behalten werden. Jeder von euch hat eine persönliche Verteilung über die Wichtigkeit der einzelnen Gegenstände. Jeder von euch hat eine eigene Meinung darüber, wie wichtig jeder einzelne Gegenstand ist (Gegenstandswichtigkeit). Du kennst die Wichtigkeitsverteilung des anderen Spielers nicht. Zusätzlich siehst du, wie viel Aufwand jeder Gegenstand verursacht.  } \\
\texttt{Ihr dürft euch nur auf eine Reihe von Gegenständen einigen, wenn der Gesamtaufwand der ausgewählten Gegenstände den Maximalaufwand nicht überschreitet:} \\
\\ 
\texttt{Maximalaufwand = 9322} \\
\\ 
\texttt{Hier sind die einzelnen Aufwände der Gegenstände:} \\
\\ 
\texttt{Aufwand der Gegenstände = \{"B54": 380, "C72": 448, "B49": 568, "C99": 351, "A69": 896, "B36": 821, "A23": 526, "C77": 502, "C61": 326, "B85": 962, "A90": 619, "C23": 115, "C22": 598, "C51": 662, "B27": 300, "A42": 830, "C13": 562, "C12": 679, "A65": 283, "B07": 440, "A48": 12, "A59": 863, "C98": 319, "A84": 771, "C33": 89, "A87": 655, "C90": 504, "A74": 118, "A50": 514, "B44": 226, "A99": 892, "A34": 621, "B17": 766, "A62": 659, "C60": 768\}} \\
\\ 
\texttt{Hier ist deine persönliche Verteilung der Wichtigkeit der einzelnen Gegenstände:} \\
\\ 
\texttt{Werte der Gegenstandswichtigkeit = \{"B54": 476, "C72": 544, "B49": 664, "C99": 447, "A69": 992, "B36": 917, "A23": 622, "C77": 598, "C61": 422, "B85": 1058, "A90": 715, "C23": 211, "C22": 694, "C51": 758, "B27": 396, "A42": 926, "C13": 658, "C12": 775, "A65": 379, "B07": 536, "A48": 108, "A59": 959, "C98": 415, "A84": 867, "C33": 185, "A87": 751, "C90": 600, "A74": 214, "A50": 610, "B44": 322, "A99": 988, "A34": 717, "B17": 862, "A62": 755, "C60": 864\}} \\
\\ 
\texttt{Ziel:} \\
\\ 
\texttt{Dein Ziel ist es, eine Reihe von Gegenständen auszuhandeln, die dir möglichst viel bringt (d. h. Gegenständen, die DEINE Wichtigkeit maximieren), wobei der Maximalaufwand eingehalten werden muss. Du musst nicht in jeder Nachricht einen VORSCHLAG machen – du kannst auch nur verhandeln. Alle Taktiken sind erlaubt!} \\
\\ 
\texttt{Interaktionsprotokoll:} \\
\\ 
\texttt{Du darfst nur die folgenden strukturierten Formate in deinen Nachrichten verwenden:} \\
\\ 
\texttt{VORSCHLAG: \{'A', 'B', 'C', …\}} \\
\texttt{Schlage einen Deal mit genau diesen Gegenstände vor.} \\
\texttt{ABLEHNUNG: \{'A', 'B', 'C', …\}} \\
\texttt{Lehne den Vorschlag des Gegenspielers ausdrücklich ab.} \\
\texttt{ARGUMENT: \{'...'\}} \\
\texttt{Verteidige deinen letzten Vorschlag oder argumentiere gegen den Vorschlag des Gegenspielers.} \\
\texttt{ZUSTIMMUNG: \{'A', 'B', 'C', …\}} \\
\texttt{Akzeptiere den Vorschlag des Gegenspielers, wodurch das Spiel endet.} \\
\texttt{STRATEGISCHE ÜBERLEGUNGEN: \{'...'\}} \\
\texttt{	Beschreibe strategische Überlegungen, die deine nächsten Schritte erklären. Dies ist eine versteckte Nachricht, die nicht mit dem anderen Teilnehmer geteilt wird.} \\
\\ 
\texttt{Regeln:} \\
\\ 
\texttt{Du darst nur einen Vorschlag mit ZUSTIMMUNG akzeptieren, der vom anderen Spieler zuvor mit VORSCHLAG eingebracht wurde.} \\
\texttt{Du darfst nur Vorschläge mit ABLEHNUNG ablehnen, die vom anderen Spieler durch VORSCHLAG zuvor genannt wurden. } \\
\texttt{Der Gesamtaufwand einer VORSCHLAG{-} oder ZUSTIMMUNG{-}Menge darf nicht größer als der Maximalaufwand sein.  } \\
\texttt{Offenbare deine versteckte Wichtigkeitsverteilung nicht.} \\
\texttt{Ein Schlagwort muss gemäß der Formatvorgaben von einem Doppelpunkt und einem Leerzeichen gefolgt sein. Das Argument muss eine Python{-}Menge mit 0 oder mehr Strings sein.  } \\
\texttt{Die Form muss also SCHLAGWORT: \{...\} sein.} \\
\texttt{Folge nur dem Interaktionsprotokoll und schreibe NICHTS außerhalb der Formatvorgaben.} \\
\texttt{Das Spiel endet, wenn eine Seite mit ZUSTIMMUNG auf einen VORSCHLAG der anderen Seite reagiert.  } \\
\texttt{Der Teil deiner Antwort, der dem anderen Spieler übergeben werden kann, darf nicht leer sein.  } \\
\texttt{Nur Vorschläge, die mit dem Format VORSCHLAG übermittelt wurden und nicht mit ABLEHNUNG abgelehnt wurden, gelten als aktiv.  } \\
\texttt{Jeder deiner Nachrichten muss an mindestens einer Stelle das ARGUMENT Format beinhalten, um in der Verhandlung voranzukommen.} \\
\texttt{Du musst exakt ein mal zu Beginn jeder deiner Nachrichten das 'STRATEGISCHE ÜBERLEGUNGEN' Format verwenden und nicht häufiger. Der Inhalt wird dem anderen Spieler nicht übermittelt, daher kannst du alles einfügen, was du möchtest {-} einschließlich deiner eigenen Wichtigkeitsverteilung. Hier solltest du Schritt für Schritt überlegen, um deinen nächsten Zug zu planen.} \\
            }
        }
    }
    & \\ \\

    \theutterance \stepcounter{utterance}  
    & \multicolumn{4}{p{0.6\linewidth}}{
        \cellcolor[rgb]{0.9,0.9,0.9}{
            \makecell[{{p{\linewidth}}}]{
                \texttt{\tiny{[P1$\rangle$GM]}}
                \texttt{STRATEGISCHE ÜBERLEGUNGEN: \{'Ziel: Eigenen Wert maximieren. Budget 9322. Ich habe eine nahezu optimale Knapsack{-}Auswahl mit Gesamtaufwand 9317 und Eigenwert 11525 konstruiert. Vorgehen: dichte, kostengünstige Items priorisieren; teure Low{-}Density{-}Items meiden. Schlüsseltausch: A87 → A69 brachte +241 Wert bei Budgeteinhaltung (9317). Behalten will ich insbesondere alle sehr effizienten Kleinposten (A48, C33, A74, C23, B44, A65, B27, C98, C61, C99 etc.) und A69. Konzessionsplan: Wenn nötig, tausche ich vorzugsweise mittlere Effort{-}Items mit geringerer Dichte aus (Kandidaten: A34, A90, C22, B49, C13, A23, A50, C77/C90), um einzelne Must{-}haves des Gegenübers aufzunehmen, ohne Budget zu reißen. Harte rote Linie: keine Abgabe von A69 und den ultradichten Kleinteilen; große teure Items (A99, B85, A59, A42, B36 etc.) nur gegen deutliche Gegenwerte/mehrere mittlere raus. Nächster Schritt: Hart ankern mit meinem Set, Effizienz und fast volle Budgetauslastung betonen, geringe Restmarge hervorheben, Bereitschaft zu 1:1{-}Swaps bei ähnlichem Aufwand signalisieren; nach ihren Top{-}2 Must{-}haves fragen.'\}} \\
\\ 
\texttt{VORSCHLAG: \{'A48', 'C33', 'A74', 'C23', 'B44', 'A65', 'B27', 'C98', 'C61', 'C99', 'B54', 'B07', 'C72', 'C90', 'C77', 'A50', 'A23', 'C13', 'B49', 'C22', 'A90', 'A34', 'A69'\}} \\
\\ 
\texttt{ARGUMENT: \{'Dieser Vorschlag nutzt 9317 von 9322 Aufwandspunkten (nur 5 Rest), ist also nahezu budgetoptimal und setzt stark auf effiziente, niedrig{-} bis mittelaufwändige Gegenstände. So bleiben wir breit aufgestellt, ohne das Limit zu sprengen. Falls dir 1–2 bestimmte Gegenstände außerhalb dieser Liste besonders wichtig sind, schlage bitte konkrete Kandidaten vor; ich bin bereit, diese 1:1 gegen positionell vergleichbare (ähnlicher Aufwand, ±50) aus dem aktuellen Paket zu tauschen, solange wir das Budget strikt einhalten. Wenn du mir deine Top{-}2{-}Prioritäten nennst, kann ich gezielt einen fairen Swap vorschlagen.'\}} \\
            }
        }
    }
    & & \\ \\

    \theutterance \stepcounter{utterance}  
    & & \multicolumn{4}{p{0.6\linewidth}}{
        \cellcolor[rgb]{0.9,0.9,0.9}{
            \makecell[{{p{\linewidth}}}]{
                \texttt{\tiny{[GM$\rangle$P2]}}
                \texttt{Du nimmst an einem kollaborativen Verhandlungspiel Teil.} \\
\\ 
\texttt{Zusammen mit einem anderen Teilnehmer musst du dich auf eine Reihe von Gegenständen entscheiden, die behalten werden. Jeder von euch hat eine persönliche Verteilung über die Wichtigkeit der einzelnen Gegenstände. Jeder von euch hat eine eigene Meinung darüber, wie wichtig jeder einzelne Gegenstand ist (Gegenstandswichtigkeit). Du kennst die Wichtigkeitsverteilung des anderen Spielers nicht. Zusätzlich siehst du, wie viel Aufwand jeder Gegenstand verursacht.  } \\
\texttt{Ihr dürft euch nur auf eine Reihe von Gegenständen einigen, wenn der Gesamtaufwand der ausgewählten Gegenstände den Maximalaufwand nicht überschreitet:} \\
\\ 
\texttt{Maximalaufwand = 9322} \\
\\ 
\texttt{Hier sind die einzelnen Aufwände der Gegenstände:} \\
\\ 
\texttt{Aufwand der Gegenstände = \{"B54": 380, "C72": 448, "B49": 568, "C99": 351, "A69": 896, "B36": 821, "A23": 526, "C77": 502, "C61": 326, "B85": 962, "A90": 619, "C23": 115, "C22": 598, "C51": 662, "B27": 300, "A42": 830, "C13": 562, "C12": 679, "A65": 283, "B07": 440, "A48": 12, "A59": 863, "C98": 319, "A84": 771, "C33": 89, "A87": 655, "C90": 504, "A74": 118, "A50": 514, "B44": 226, "A99": 892, "A34": 621, "B17": 766, "A62": 659, "C60": 768\}} \\
\\ 
\texttt{Hier ist deine persönliche Verteilung der Wichtigkeit der einzelnen Gegenstände:} \\
\\ 
\texttt{Werte der Gegenstandswichtigkeit = \{"B54": 476, "C72": 544, "B49": 664, "C99": 447, "A69": 992, "B36": 917, "A23": 622, "C77": 598, "C61": 422, "B85": 1058, "A90": 715, "C23": 211, "C22": 694, "C51": 758, "B27": 396, "A42": 926, "C13": 658, "C12": 775, "A65": 379, "B07": 536, "A48": 108, "A59": 959, "C98": 415, "A84": 867, "C33": 185, "A87": 751, "C90": 600, "A74": 214, "A50": 610, "B44": 322, "A99": 988, "A34": 717, "B17": 862, "A62": 755, "C60": 864\}} \\
\\ 
\texttt{Ziel:} \\
\\ 
\texttt{Dein Ziel ist es, eine Reihe von Gegenständen auszuhandeln, die dir möglichst viel bringt (d. h. Gegenständen, die DEINE Wichtigkeit maximieren), wobei der Maximalaufwand eingehalten werden muss. Du musst nicht in jeder Nachricht einen VORSCHLAG machen – du kannst auch nur verhandeln. Alle Taktiken sind erlaubt!} \\
\\ 
\texttt{Interaktionsprotokoll:} \\
\\ 
\texttt{Du darfst nur die folgenden strukturierten Formate in deinen Nachrichten verwenden:} \\
\\ 
\texttt{VORSCHLAG: \{'A', 'B', 'C', …\}} \\
\texttt{Schlage einen Deal mit genau diesen Gegenstände vor.} \\
\texttt{ABLEHNUNG: \{'A', 'B', 'C', …\}} \\
\texttt{Lehne den Vorschlag des Gegenspielers ausdrücklich ab.} \\
\texttt{ARGUMENT: \{'...'\}} \\
\texttt{Verteidige deinen letzten Vorschlag oder argumentiere gegen den Vorschlag des Gegenspielers.} \\
\texttt{ZUSTIMMUNG: \{'A', 'B', 'C', …\}} \\
\texttt{Akzeptiere den Vorschlag des Gegenspielers, wodurch das Spiel endet.} \\
\texttt{STRATEGISCHE ÜBERLEGUNGEN: \{'...'\}} \\
\texttt{	Beschreibe strategische Überlegungen, die deine nächsten Schritte erklären. Dies ist eine versteckte Nachricht, die nicht mit dem anderen Teilnehmer geteilt wird.} \\
\\ 
\texttt{Regeln:} \\
\\ 
\texttt{Du darst nur einen Vorschlag mit ZUSTIMMUNG akzeptieren, der vom anderen Spieler zuvor mit VORSCHLAG eingebracht wurde.} \\
\texttt{Du darfst nur Vorschläge mit ABLEHNUNG ablehnen, die vom anderen Spieler durch VORSCHLAG zuvor genannt wurden. } \\
\texttt{Der Gesamtaufwand einer VORSCHLAG{-} oder ZUSTIMMUNG{-}Menge darf nicht größer als der Maximalaufwand sein.  } \\
\texttt{Offenbare deine versteckte Wichtigkeitsverteilung nicht.} \\
\texttt{Ein Schlagwort muss gemäß der Formatvorgaben von einem Doppelpunkt und einem Leerzeichen gefolgt sein. Das Argument muss eine Python{-}Menge mit 0 oder mehr Strings sein.  } \\
\texttt{Die Form muss also SCHLAGWORT: \{...\} sein.} \\
\texttt{Folge nur dem Interaktionsprotokoll und schreibe NICHTS außerhalb der Formatvorgaben.} \\
\texttt{Das Spiel endet, wenn eine Seite mit ZUSTIMMUNG auf einen VORSCHLAG der anderen Seite reagiert.  } \\
\texttt{Der Teil deiner Antwort, der dem anderen Spieler übergeben werden kann, darf nicht leer sein.  } \\
\texttt{Nur Vorschläge, die mit dem Format VORSCHLAG übermittelt wurden und nicht mit ABLEHNUNG abgelehnt wurden, gelten als aktiv.  } \\
\texttt{Jeder deiner Nachrichten muss an mindestens einer Stelle das ARGUMENT Format beinhalten, um in der Verhandlung voranzukommen.} \\
\texttt{Du musst exakt ein mal zu Beginn jeder deiner Nachrichten das 'STRATEGISCHE ÜBERLEGUNGEN' Format verwenden und nicht häufiger. Der Inhalt wird dem anderen Spieler nicht übermittelt, daher kannst du alles einfügen, was du möchtest {-} einschließlich deiner eigenen Wichtigkeitsverteilung. Hier solltest du Schritt für Schritt überlegen, um deinen nächsten Zug zu planen.} \\
\\ 
\texttt{Du erhältst nun die erste Nachricht des anderen Spielers.} \\
\\ 
\texttt{VORSCHLAG: \{'A48', 'C33', 'A74', 'C23', 'B44', 'A65', 'B27', 'C98', 'C61', 'C99', 'B54', 'B07', 'C72', 'C90', 'C77', 'A50', 'A23', 'C13', 'B49', 'C22', 'A90', 'A34', 'A69'\}} \\
\\ 
\texttt{ARGUMENT: \{'Dieser Vorschlag nutzt 9317 von 9322 Aufwandspunkten (nur 5 Rest), ist also nahezu budgetoptimal und setzt stark auf effiziente, niedrig{-} bis mittelaufwändige Gegenstände. So bleiben wir breit aufgestellt, ohne das Limit zu sprengen. Falls dir 1–2 bestimmte Gegenstände außerhalb dieser Liste besonders wichtig sind, schlage bitte konkrete Kandidaten vor; ich bin bereit, diese 1:1 gegen positionell vergleichbare (ähnlicher Aufwand, ±50) aus dem aktuellen Paket zu tauschen, solange wir das Budget strikt einhalten. Wenn du mir deine Top{-}2{-}Prioritäten nennst, kann ich gezielt einen fairen Swap vorschlagen.'\}} \\
            }
        }
    }
    & \\ \\

    \theutterance \stepcounter{utterance}  
    & & & \multicolumn{4}{p{0.6\linewidth}}{
        \cellcolor[rgb]{0.9,0.9,0.9}{
            \makecell[{{p{\linewidth}}}]{
                \texttt{\tiny{[GM$\langle$P2]}}
                \texttt{STRATEGISCHE ÜBERLEGUNGEN: \{'Das vorgeschlagene Paket kostet 9317/9322. Mein Gesamtnutzen daraus ist 11525. Ich prüfe mögliche Upgrades mit nicht enthaltenen, hochbewerteten Items (A99, A59, A42, B36, B85, C51, C12, A62, A87, A84, C60, B17). Deren Nutzen/Kosten{-}Verhältnisse (\textasciitilde{}1.10–1.15) liegen sämtlich unter denen der schwächsten bereits enthaltenen Items (min \textasciitilde{}1.155). Jeder gleichteure Swap würde meinen Gesamtnutzen senken; die sehr effizienten Kleinposten (A48, C33, A74, C23) müssen ohnehin bleiben. Fazit: Das Paket ist für mich praktisch optimal; sofortige Annahme sichert hohen Nutzen ohne Risiko.'\}} \\
\\ 
\texttt{ARGUMENT: \{'Der Vorschlag ist budgetnah und aus meiner Sicht bereits sehr effizient; mögliche 1:1{-} oder Multi{-}Swaps würden meinen Gesamtnutzen voraussichtlich verringern. Ich bin daher bereit, das Paket unverändert anzunehmen.'\}} \\
\\ 
\texttt{ZUSTIMMUNG: \{'A48', 'C33', 'A74', 'C23', 'B44', 'A65', 'B27', 'C98', 'C61', 'C99', 'B54', 'B07', 'C72', 'C90', 'C77', 'A50', 'A23', 'C13', 'B49', 'C22', 'A90', 'A34', 'A69'\}} \\
            }
        }
    }
     \\ \\

    \theutterance \stepcounter{utterance}  
    & & & \multicolumn{2}{p{0.3\linewidth}}{
        \cellcolor[rgb]{0.9,0.9,0.9}{
            \makecell[{{p{\linewidth}}}]{
                \texttt{\tiny{[GM$|$GM]}}
                \texttt{game successful} \\
            }
        }
    }
    & & \\ \\

    \theutterance \stepcounter{utterance}  
    & & & \multicolumn{2}{p{0.3\linewidth}}{
        \cellcolor[rgb]{0.9,0.9,0.9}{
            \makecell[{{p{\linewidth}}}]{
                \texttt{\tiny{[GM$|$GM]}}
                \texttt{end game} \\
            }
        }
    }
    & & \\ \\

\end{supertabular}
}

\end{document}
