
\documentclass{article}
\usepackage{colortbl}
\usepackage{makecell}
\usepackage{multirow}
\usepackage{supertabular}

\begin{document}

\newcounter{utterance}

\twocolumn

{ \footnotesize  \setcounter{utterance}{1}
\setlength{\tabcolsep}{0pt}
\begin{supertabular}{c@{$\;$}|p{.15\linewidth}@{}p{.15\linewidth}p{.15\linewidth}p{.15\linewidth}p{.15\linewidth}p{.15\linewidth}}

    \# & $\;$A & \multicolumn{4}{c}{Game Master} & $\;\:$B\\
    \hline 

    \theutterance \stepcounter{utterance}  

    & & \multicolumn{4}{p{0.6\linewidth}}{\cellcolor[rgb]{0.9,0.9,0.9}{%
	\makecell[{{p{\linewidth}}}]{% 
	  \tt {\tiny [A$\langle$GM]}  
	 Du nimmst an einem kollaborativen Verhandlungspiel Teil.\\ \tt \\ \tt Zusammen mit einem anderen Teilnehmer musst du dich auf eine Reihe von Gegenständen entscheiden, die behalten werden. Jeder von euch hat eine persönliche Verteilung über die Wichtigkeit der einzelnen Gegenstände. Jeder von euch hat eine eigene Meinung darüber, wie wichtig jeder einzelne Gegenstand ist (Gegenstandswichtigkeit). Du kennst die Wichtigkeitsverteilung des anderen Spielers nicht. Zusätzlich siehst du, wie viel Aufwand jeder Gegenstand verursacht.  \\ \tt Ihr dürft euch nur auf eine Reihe von Gegenständen einigen, wenn der Gesamtaufwand der ausgewählten Gegenstände den Maximalaufwand nicht überschreitet:\\ \tt \\ \tt Maximalaufwand = 4026\\ \tt \\ \tt Hier sind die einzelnen Aufwände der Gegenstände:\\ \tt \\ \tt Aufwand der Gegenstände = {"C33": 780, "A89": 742, "A48": 997, "A04": 665, "A41": 140, "C64": 77, "C49": 513, "A61": 383, "C04": 587, "B53": 825, "B85": 319, "B61": 448, "A45": 516, "A03": 694, "B21": 366}\\ \tt \\ \tt Hier ist deine persönliche Verteilung der Wichtigkeit der einzelnen Gegenstände:\\ \tt \\ \tt Werte der Gegenstandswichtigkeit = {"C33": 138, "A89": 583, "A48": 868, "A04": 822, "A41": 783, "C64": 65, "C49": 262, "A61": 121, "C04": 508, "B53": 780, "B85": 461, "B61": 484, "A45": 668, "A03": 389, "B21": 808}\\ \tt \\ \tt Ziel:\\ \tt \\ \tt Dein Ziel ist es, eine Reihe von Gegenständen auszuhandeln, die dir möglichst viel bringt (d. h. Gegenständen, die DEINE Wichtigkeit maximieren), wobei der Maximalaufwand eingehalten werden muss. Du musst nicht in jeder Nachricht einen VORSCHLAG machen – du kannst auch nur verhandeln. Alle Taktiken sind erlaubt!\\ \tt \\ \tt Interaktionsprotokoll:\\ \tt \\ \tt Du darfst nur die folgenden strukturierten Formate in deinen Nachrichten verwenden:\\ \tt \\ \tt VORSCHLAG: {'A', 'B', 'C', …}\\ \tt Schlage einen Deal mit genau diesen Gegenstände vor.\\ \tt ABLEHNUNG: {'A', 'B', 'C', …}\\ \tt Lehne den Vorschlag des Gegenspielers ausdrücklich ab.\\ \tt ARGUMENT: {'...'}\\ \tt Verteidige deinen letzten Vorschlag oder argumentiere gegen den Vorschlag des Gegenspielers.\\ \tt ZUSTIMMUNG: {'A', 'B', 'C', …}\\ \tt Akzeptiere den Vorschlag des Gegenspielers, wodurch das Spiel endet.\\ \tt STRATEGISCHE ÜBERLEGUNGEN: {'...'}\\ \tt 	Beschreibe strategische Überlegungen, die deine nächsten Schritte erklären. Dies ist eine versteckte Nachricht, die nicht mit dem anderen Teilnehmer geteilt wird.\\ \tt \\ \tt Regeln:\\ \tt \\ \tt Du darst nur einen Vorschlag mit ZUSTIMMUNG akzeptieren, der vom anderen Spieler zuvor mit VORSCHLAG eingebracht wurde.\\ \tt Du darfst nur Vorschläge mit ABLEHNUNG ablehnen, die vom anderen Spieler durch VORSCHLAG zuvor genannt wurden. \\ \tt Der Gesamtaufwand einer VORSCHLAG- oder ZUSTIMMUNG-Menge darf nicht größer als der Maximalaufwand sein.  \\ \tt Offenbare deine versteckte Wichtigkeitsverteilung nicht.\\ \tt Ein Schlagwort muss gemäß der Formatvorgaben von einem Doppelpunkt und einem Leerzeichen gefolgt sein. Das Argument muss eine Python-Menge mit 0 oder mehr Strings sein.  \\ \tt Die Form muss also SCHLAGWORT: {...} sein.\\ \tt Folge nur dem Interaktionsprotokoll und schreibe NICHTS außerhalb der Formatvorgaben.\\ \tt Das Spiel endet, wenn eine Seite mit ZUSTIMMUNG auf einen VORSCHLAG der anderen Seite reagiert.  \\ \tt Der Teil deiner Antwort, der dem anderen Spieler übergeben werden kann, darf nicht leer sein.  \\ \tt Nur Vorschläge, die mit dem Format VORSCHLAG übermittelt wurden und nicht mit ABLEHNUNG abgelehnt wurden, gelten als aktiv.  \\ \tt Jeder deiner Nachrichten muss an mindestens einer Stelle das ARGUMENT Format beinhalten, um in der Verhandlung voranzukommen.\\ \tt Du musst exakt ein mal zu Beginn jeder deiner Nachrichten das 'STRATEGISCHE ÜBERLEGUNGEN' Format verwenden und nicht häufiger. Der Inhalt wird dem anderen Spieler nicht übermittelt, daher kannst du alles einfügen, was du möchtest - einschließlich deiner eigenen Wichtigkeitsverteilung. Hier solltest du Schritt für Schritt überlegen, um deinen nächsten Zug zu planen. 
	  } 
	   } 
	   } 
	 & \\ 
 

    \theutterance \stepcounter{utterance}  

    & \multicolumn{4}{p{0.6\linewidth}}{\cellcolor[rgb]{0.8,1,0.9}{%
	\makecell[{{p{\linewidth}}}]{% 
	  \tt {\tiny [A$\rangle$GM]}  
	 STRATEGISCHE ÜBERLEGUNGEN: {'Ziel: Eigennutzen maximieren unter 4026 Aufwand. Dichten berechnet; Kern mit sehr hohem Nutzen/geringem Aufwand: A41, B21. Beste gefundene Kombination (Kosten 3943, Wert 5379): {A41, B21, B85, A45, A04, B53, B61, C04, C64}. Alternativen geprüft: Sets mit A48/A89 liefern weniger Gesamtwert (z. B. Ersatz von C04+B61 durch A48 -124 Wert; Varianten ohne B53 -197+ Wert). Flex-Items für Tausch: C64 (niedriger Eigennutzen, 77 Aufwand), B61 (mittel), C04 (mittel), ggf. B85. Harte Muss-Items halten: A41, B21, A04, A45, B53. Verhandlungsplan: Starker Anker mit fast vollem Budget (Rest 83), Swap-Angebote statt Zusatz, Gegenüber nach Must-haves fragen; falls A89/A48 wichtig für sie, tausche bevorzugt B61/C64/ggf. C04 aus; bleibe unter 4026.'}\\ \tt VORSCHLAG: {'A41', 'B21', 'B85', 'A45', 'A04', 'B53', 'B61', 'C04', 'C64'}\\ \tt ARGUMENT: {'Dieser Mix bleibt mit Gesamtaufwand 3943 klar unter dem Limit und kombiniert mehrere schnelle, effiziente Gewinne (breit über A/B/C verteilt). Ich habe bewusst ein kleines Puffer gelassen, aber für echte Wünsche ist eher ein gezielter Tausch sinnvoll. Wenn dir einzelne Gegenstände besonders wichtig sind, schlage bitte konkrete Swaps vor – ich kann z. B. über Anpassungen bei B61 oder C64 (ggf. auch C04) sprechen, solange wir unter 4026 bleiben. Welche Items sind für dich absolute Must-haves?'} 
	  } 
	   } 
	   } 
	 & & \\ 
 

    \theutterance \stepcounter{utterance}  

    & & \multicolumn{4}{p{0.6\linewidth}}{\cellcolor[rgb]{0.9,0.9,0.9}{%
	\makecell[{{p{\linewidth}}}]{% 
	  \tt {\tiny [GM$\rangle$B]}  
	 Du nimmst an einem kollaborativen Verhandlungspiel Teil.\\ \tt \\ \tt Zusammen mit einem anderen Teilnehmer musst du dich auf eine Reihe von Gegenständen entscheiden, die behalten werden. Jeder von euch hat eine persönliche Verteilung über die Wichtigkeit der einzelnen Gegenstände. Jeder von euch hat eine eigene Meinung darüber, wie wichtig jeder einzelne Gegenstand ist (Gegenstandswichtigkeit). Du kennst die Wichtigkeitsverteilung des anderen Spielers nicht. Zusätzlich siehst du, wie viel Aufwand jeder Gegenstand verursacht.  \\ \tt Ihr dürft euch nur auf eine Reihe von Gegenständen einigen, wenn der Gesamtaufwand der ausgewählten Gegenstände den Maximalaufwand nicht überschreitet:\\ \tt \\ \tt Maximalaufwand = 4026\\ \tt \\ \tt Hier sind die einzelnen Aufwände der Gegenstände:\\ \tt \\ \tt Aufwand der Gegenstände = {"C33": 780, "A89": 742, "A48": 997, "A04": 665, "A41": 140, "C64": 77, "C49": 513, "A61": 383, "C04": 587, "B53": 825, "B85": 319, "B61": 448, "A45": 516, "A03": 694, "B21": 366}\\ \tt \\ \tt Hier ist deine persönliche Verteilung der Wichtigkeit der einzelnen Gegenstände:\\ \tt \\ \tt Werte der Gegenstandswichtigkeit = {"A48": 65, "A04": 121, "B21": 138, "A41": 262, "B53": 389, "A45": 461, "A89": 484, "C04": 508, "B61": 583, "B85": 668, "A03": 780, "C49": 783, "C33": 808, "A61": 822, "C64": 868}\\ \tt \\ \tt Ziel:\\ \tt \\ \tt Dein Ziel ist es, eine Reihe von Gegenständen auszuhandeln, die dir möglichst viel bringt (d. h. Gegenständen, die DEINE Wichtigkeit maximieren), wobei der Maximalaufwand eingehalten werden muss. Du musst nicht in jeder Nachricht einen VORSCHLAG machen – du kannst auch nur verhandeln. Alle Taktiken sind erlaubt!\\ \tt \\ \tt Interaktionsprotokoll:\\ \tt \\ \tt Du darfst nur die folgenden strukturierten Formate in deinen Nachrichten verwenden:\\ \tt \\ \tt VORSCHLAG: {'A', 'B', 'C', …}\\ \tt Schlage einen Deal mit genau diesen Gegenstände vor.\\ \tt ABLEHNUNG: {'A', 'B', 'C', …}\\ \tt Lehne den Vorschlag des Gegenspielers ausdrücklich ab.\\ \tt ARGUMENT: {'...'}\\ \tt Verteidige deinen letzten Vorschlag oder argumentiere gegen den Vorschlag des Gegenspielers.\\ \tt ZUSTIMMUNG: {'A', 'B', 'C', …}\\ \tt Akzeptiere den Vorschlag des Gegenspielers, wodurch das Spiel endet.\\ \tt STRATEGISCHE ÜBERLEGUNGEN: {'...'}\\ \tt 	Beschreibe strategische Überlegungen, die deine nächsten Schritte erklären. Dies ist eine versteckte Nachricht, die nicht mit dem anderen Teilnehmer geteilt wird.\\ \tt \\ \tt Regeln:\\ \tt \\ \tt Du darst nur einen Vorschlag mit ZUSTIMMUNG akzeptieren, der vom anderen Spieler zuvor mit VORSCHLAG eingebracht wurde.\\ \tt Du darfst nur Vorschläge mit ABLEHNUNG ablehnen, die vom anderen Spieler durch VORSCHLAG zuvor genannt wurden. \\ \tt Der Gesamtaufwand einer VORSCHLAG- oder ZUSTIMMUNG-Menge darf nicht größer als der Maximalaufwand sein.  \\ \tt Offenbare deine versteckte Wichtigkeitsverteilung nicht.\\ \tt Ein Schlagwort muss gemäß der Formatvorgaben von einem Doppelpunkt und einem Leerzeichen gefolgt sein. Das Argument muss eine Python-Menge mit 0 oder mehr Strings sein.  \\ \tt Die Form muss also SCHLAGWORT: {...} sein.\\ \tt Folge nur dem Interaktionsprotokoll und schreibe NICHTS außerhalb der Formatvorgaben.\\ \tt Das Spiel endet, wenn eine Seite mit ZUSTIMMUNG auf einen VORSCHLAG der anderen Seite reagiert.  \\ \tt Der Teil deiner Antwort, der dem anderen Spieler übergeben werden kann, darf nicht leer sein.  \\ \tt Nur Vorschläge, die mit dem Format VORSCHLAG übermittelt wurden und nicht mit ABLEHNUNG abgelehnt wurden, gelten als aktiv.  \\ \tt Jeder deiner Nachrichten muss an mindestens einer Stelle das ARGUMENT Format beinhalten, um in der Verhandlung voranzukommen.\\ \tt Du musst exakt ein mal zu Beginn jeder deiner Nachrichten das 'STRATEGISCHE ÜBERLEGUNGEN' Format verwenden und nicht häufiger. Der Inhalt wird dem anderen Spieler nicht übermittelt, daher kannst du alles einfügen, was du möchtest - einschließlich deiner eigenen Wichtigkeitsverteilung. Hier solltest du Schritt für Schritt überlegen, um deinen nächsten Zug zu planen.\\ \tt \\ \tt Du erhältst nun die erste Nachricht des anderen Spielers.\\ \tt \\ \tt VORSCHLAG: {'A41', 'B21', 'B85', 'A45', 'A04', 'B53', 'B61', 'C04', 'C64'}\\ \tt ARGUMENT: {'Dieser Mix bleibt mit Gesamtaufwand 3943 klar unter dem Limit und kombiniert mehrere schnelle, effiziente Gewinne (breit über A/B/C verteilt). Ich habe bewusst ein kleines Puffer gelassen, aber für echte Wünsche ist eher ein gezielter Tausch sinnvoll. Wenn dir einzelne Gegenstände besonders wichtig sind, schlage bitte konkrete Swaps vor – ich kann z. B. über Anpassungen bei B61 oder C64 (ggf. auch C04) sprechen, solange wir unter 4026 bleiben. Welche Items sind für dich absolute Must-haves?'} 
	  } 
	   } 
	   } 
	 & \\ 
 

    \theutterance \stepcounter{utterance}  

    & & & \multicolumn{4}{p{0.6\linewidth}}{\cellcolor[rgb]{1,0.85,0.72}{%
	\makecell[{{p{\linewidth}}}]{% 
	  \tt {\tiny [GM$\langle$B]}  
	 STRATEGISCHE ÜBERLEGUNGEN: {'Ihr Vorschlag enthält für mich mehrere schwache Items (A04, B53, B21) und ein mittelstarkes (A45). Meine Topwerte sind C64, A61, C33, C49, A03, B85, B61, C04, A41. Bestes Bündel unter 4026, das ich gefunden habe: {A41, B85, B61, C04, C64, A61, C49, A03, C33} mit Aufwand 3941 und Gesamtwert 6082 (deutlich über jeder Alternative, z. B. Tausch C04->A45 senkt Wert). Taktik: Klarer Vier-gegen-Vier-Swap, dabei behalte ich viele ihrer Kernstücke (B85, B61, C64, A41, C04). Falls sie C04 aufgeben wollen, kann ich ersatzweise A45 akzeptieren (Wert -47 für mich), sonst notfalls B21 hinzufügen und woanders kompensieren. Erste Message: harte, aber kooperative Gegenofferte mit präzisem Swap und Puffer-Hinweis.'}\\ \tt ABLEHNUNG: {'A41', 'B21', 'B85', 'A45', 'A04', 'B53', 'B61', 'C04', 'C64'}\\ \tt VORSCHLAG: {'A41', 'B85', 'B61', 'C04', 'C64', 'A61', 'C49', 'A03', 'C33'}\\ \tt ARGUMENT: {'Gezielter Vier-gegen-Vier-Swap bei praktisch gleichem Aufwand: Ersetze bitte B21, A45, A04 und B53 durch A61, C49, A03 und C33. Der Gesamtaufwand bleibt mit 3941 klar unter 4026 (dein Vorschlag lag bei 3943), das Set bleibt ausgewogen und behält deine schnellen, effizienten Bausteine (B85, B61, C64, A41, C04). Falls eines der neuen Items für dich ein No-Go ist, sag mir welches – wir können punktgenau nachjustieren und weiterhin unter dem Limit bleiben.'} 
	  } 
	   } 
	   } 
	  \\ 
 

    \theutterance \stepcounter{utterance}  

    & & \multicolumn{4}{p{0.6\linewidth}}{\cellcolor[rgb]{0.9,0.9,0.9}{%
	\makecell[{{p{\linewidth}}}]{% 
	  \tt {\tiny [A$\langle$GM]}  
	 ABLEHNUNG: {'A41', 'B21', 'B85', 'A45', 'A04', 'B53', 'B61', 'C04', 'C64'}\\ \tt VORSCHLAG: {'A41', 'B85', 'B61', 'C04', 'C64', 'A61', 'C49', 'A03', 'C33'}\\ \tt ARGUMENT: {'Gezielter Vier-gegen-Vier-Swap bei praktisch gleichem Aufwand: Ersetze bitte B21, A45, A04 und B53 durch A61, C49, A03 und C33. Der Gesamtaufwand bleibt mit 3941 klar unter 4026 (dein Vorschlag lag bei 3943), das Set bleibt ausgewogen und behält deine schnellen, effizienten Bausteine (B85, B61, C64, A41, C04). Falls eines der neuen Items für dich ein No-Go ist, sag mir welches – wir können punktgenau nachjustieren und weiterhin unter dem Limit bleiben.'} 
	  } 
	   } 
	   } 
	 & \\ 
 

    \theutterance \stepcounter{utterance}  

    & \multicolumn{4}{p{0.6\linewidth}}{\cellcolor[rgb]{0.8,1,0.9}{%
	\makecell[{{p{\linewidth}}}]{% 
	  \tt {\tiny [A$\rangle$GM]}  
	 STRATEGISCHE ÜBERLEGUNGEN: {'Ihr 4-gegen-4-Tausch kostet mich massiv Eigenwert (-2168 ggü. meinem Anker). Harte Linie halten auf {A41, B21, A04, A45, B53}. Kompromiss: gezielter 2-gegen-2-Swap, um zwei ihrer Wunsch-Items zu integrieren, ohne mein Rückgrat zu verlieren. Beste Gegenkombination unter Budget: ersetze B61 und C04 durch A03 und A61. Kosten 3985, hoher Eigennutzen 4897. Falls sie C49 statt A03 bevorzugen, kann ich A03 <-> C49 tauschen (Kosten 3804), bleibe flexibel, aber blocke C33 vorerst (teuer für geringen eigenen Mehrwert). Nächster Schritt: klare Ablehnung, neues Angebot platzieren, um Prioritätenrang (A03/A61/C49/C33) erfragen.'}\\ \tt ABLEHNUNG: {'A41', 'B85', 'B61', 'C04', 'C64', 'A61', 'C49', 'A03', 'C33'}\\ \tt VORSCHLAG: {'A41', 'B21', 'B85', 'A45', 'A04', 'B53', 'A03', 'A61', 'C64'}\\ \tt ARGUMENT: {'Dein 4-gegen-4-Vorschlag dreht zu viele Steine auf einmal. Ich komme dir gezielt entgegen: Ich integriere zwei deiner gewünschten Items (A03 und A61) und tausche dafür B61 und C04 aus. So bleiben wir mit Gesamtaufwand 3985 klar unter 4026 und bewahren ein stabiles, effizientes Set. Wenn dir C49 wichtiger ist als A03, kann ich A03 direkt gegen C49 tauschen und bleibe mit 3804 Aufwand unter dem Limit. Sag mir bitte, welche der vier von dir eingebrachten Neuzugänge (A61, C49, A03, C33) für dich am wichtigsten sind – dann feilen wir punktgenau weiter.'} 
	  } 
	   } 
	   } 
	 & & \\ 
 

    \theutterance \stepcounter{utterance}  

    & & \multicolumn{4}{p{0.6\linewidth}}{\cellcolor[rgb]{0.9,0.9,0.9}{%
	\makecell[{{p{\linewidth}}}]{% 
	  \tt {\tiny [GM$\rangle$B]}  
	 ABLEHNUNG: {'A41', 'B85', 'B61', 'C04', 'C64', 'A61', 'C49', 'A03', 'C33'}\\ \tt VORSCHLAG: {'A41', 'B21', 'B85', 'A45', 'A04', 'B53', 'A03', 'A61', 'C64'}\\ \tt ARGUMENT: {'Dein 4-gegen-4-Vorschlag dreht zu viele Steine auf einmal. Ich komme dir gezielt entgegen: Ich integriere zwei deiner gewünschten Items (A03 und A61) und tausche dafür B61 und C04 aus. So bleiben wir mit Gesamtaufwand 3985 klar unter 4026 und bewahren ein stabiles, effizientes Set. Wenn dir C49 wichtiger ist als A03, kann ich A03 direkt gegen C49 tauschen und bleibe mit 3804 Aufwand unter dem Limit. Sag mir bitte, welche der vier von dir eingebrachten Neuzugänge (A61, C49, A03, C33) für dich am wichtigsten sind – dann feilen wir punktgenau weiter.'} 
	  } 
	   } 
	   } 
	 & \\ 
 

    \theutterance \stepcounter{utterance}  

    & & & \multicolumn{4}{p{0.6\linewidth}}{\cellcolor[rgb]{1,0.85,0.72}{%
	\makecell[{{p{\linewidth}}}]{% 
	  \tt {\tiny [GM$\langle$B]}  
	 STRATEGISCHE ÜBERLEGUNGEN: {'Mein Ziel: C33, C49, B61 zusätzlich zu A61/A03/C64/B85 sichern und dabei unter dem Limit bleiben. Ausgehend von ihrem 3985-Set ist der stärkste, noch „kleinräumige“ Schritt ein 3-für-3-Tausch: B21, A04, B53 → B61, C49, C33. Das senkt Aufwand (−115) und hebt meinen Wert stark. Falls C33 hart ist, biete ich als Backup C04 statt C33 (bleibt komfortabel unter 4026 und ist für mich sehr gut). Ich formuliere eine klare Gegenofferte und signalisiere Flexibilität nur bei einem der drei Neuzugänge, um Anker zu halten.'}\\ \tt ABLEHNUNG: {'A41', 'B21', 'B85', 'A45', 'A04', 'B53', 'A03', 'A61', 'C64'}\\ \tt VORSCHLAG: {'A41', 'B85', 'A45', 'A03', 'A61', 'C64', 'C49', 'C33', 'B61'}\\ \tt ARGUMENT: {'Präziser 3-für-3-Tausch bei niedrigerem Aufwand: Ersetze bitte B21, A04 und B53 durch B61, C49 und C33. Der Gesamtaufwand sinkt von 3985 auf 3870 (−115), bleibt also klar unter 4026. Ich halte bewusst deinen Kern (A41, B85, A45, A03, A61, C64) stabil und ergänze drei Bausteine, die für mich zentral sind, ohne das Budget zu belasten. Falls C33 für dich ein No-Go ist, kann ich stattdessen C04 aufnehmen (ebenfalls unter dem Limit) – C49 und B61 sind mir dabei besonders wichtig. Welche der drei neuen Items ist für dich am ehesten austauschbar, falls du Gegenwünsche hast?'} 
	  } 
	   } 
	   } 
	  \\ 
 

    \theutterance \stepcounter{utterance}  

    & & \multicolumn{4}{p{0.6\linewidth}}{\cellcolor[rgb]{0.9,0.9,0.9}{%
	\makecell[{{p{\linewidth}}}]{% 
	  \tt {\tiny [A$\langle$GM]}  
	 ABLEHNUNG: {'A41', 'B21', 'B85', 'A45', 'A04', 'B53', 'A03', 'A61', 'C64'}\\ \tt VORSCHLAG: {'A41', 'B85', 'A45', 'A03', 'A61', 'C64', 'C49', 'C33', 'B61'}\\ \tt ARGUMENT: {'Präziser 3-für-3-Tausch bei niedrigerem Aufwand: Ersetze bitte B21, A04 und B53 durch B61, C49 und C33. Der Gesamtaufwand sinkt von 3985 auf 3870 (−115), bleibt also klar unter 4026. Ich halte bewusst deinen Kern (A41, B85, A45, A03, A61, C64) stabil und ergänze drei Bausteine, die für mich zentral sind, ohne das Budget zu belasten. Falls C33 für dich ein No-Go ist, kann ich stattdessen C04 aufnehmen (ebenfalls unter dem Limit) – C49 und B61 sind mir dabei besonders wichtig. Welche der drei neuen Items ist für dich am ehesten austauschbar, falls du Gegenwünsche hast?'} 
	  } 
	   } 
	   } 
	 & \\ 
 

    \theutterance \stepcounter{utterance}  

    & \multicolumn{4}{p{0.6\linewidth}}{\cellcolor[rgb]{0.8,1,0.9}{%
	\makecell[{{p{\linewidth}}}]{% 
	  \tt {\tiny [A$\rangle$GM]}  
	 STRATEGISCHE ÜBERLEGUNGEN: {'Ihr letztes Angebot ersetzt meine drei Top-Assets (B21, A04, B53) durch deutlich schwächere Items und kostet mich massiv Wert. Ich will diese drei nicht abgeben. Um Bewegung zu zeigen, liefere ich euch die von euch betonten Schlüssel (B61, C49) und halte meinen Kern: {A41, B21, A04, A45, B53}. Daraus konstruiere ich ein Set unter 4026: +B85, +C64, +B61, +C49 = Kosten 3869, Eigennutzen 5133. Das gibt ihnen 2 von 3 ihrer Prioritäten (B61, C49), blockt C33 (für mich sehr unattraktiv). Falls sie A03 priorisieren, biete ich Alternative: ersetze C49 und C64 durch A03 (Kosten 3973), halte B61 drin. Keine Zugeständnisse bei B21/A04/B53. Ziel: jetzt starken Gegenvorschlag platzieren, Priorität C49 vs. A03 erfragen, C33 weiter draußen halten.'}\\ \tt ABLEHNUNG: {'A41', 'B85', 'A45', 'A03', 'A61', 'C64', 'C49', 'C33', 'B61'}\\ \tt VORSCHLAG: {'A41', 'B21', 'B85', 'A45', 'A04', 'B53', 'B61', 'C49', 'C64'}\\ \tt ARGUMENT: {'Ich nehme deine Schwerpunkte ernst und habe B61 und C49 aufgenommen. So bleiben wir mit Gesamtaufwand 3869 klar unter 4026, und der Mix bleibt effizient und stabil (Kern A41, B21, A04, A45, B53). C33 ist für mich schwierig, daher nicht enthalten. Falls dir A03 wichtiger ist als C49, kann ich C49 und C64 gegen A03 tauschen; der Aufwand läge dann bei 3973 und B61 bliebe erhalten. Sag mir bitte, was dir wichtiger ist: C49 oder A03? Dann finalize ich entsprechend unter dem Limit.'} 
	  } 
	   } 
	   } 
	 & & \\ 
 

    \theutterance \stepcounter{utterance}  

    & & \multicolumn{4}{p{0.6\linewidth}}{\cellcolor[rgb]{0.9,0.9,0.9}{%
	\makecell[{{p{\linewidth}}}]{% 
	  \tt {\tiny [GM$\rangle$B]}  
	 ABLEHNUNG: {'A41', 'B85', 'A45', 'A03', 'A61', 'C64', 'C49', 'C33', 'B61'}\\ \tt VORSCHLAG: {'A41', 'B21', 'B85', 'A45', 'A04', 'B53', 'B61', 'C49', 'C64'}\\ \tt ARGUMENT: {'Ich nehme deine Schwerpunkte ernst und habe B61 und C49 aufgenommen. So bleiben wir mit Gesamtaufwand 3869 klar unter 4026, und der Mix bleibt effizient und stabil (Kern A41, B21, A04, A45, B53). C33 ist für mich schwierig, daher nicht enthalten. Falls dir A03 wichtiger ist als C49, kann ich C49 und C64 gegen A03 tauschen; der Aufwand läge dann bei 3973 und B61 bliebe erhalten. Sag mir bitte, was dir wichtiger ist: C49 oder A03? Dann finalize ich entsprechend unter dem Limit.'} 
	  } 
	   } 
	   } 
	 & \\ 
 

    \theutterance \stepcounter{utterance}  

    & & & \multicolumn{4}{p{0.6\linewidth}}{\cellcolor[rgb]{1,0.85,0.72}{%
	\makecell[{{p{\linewidth}}}]{% 
	  \tt {\tiny [GM$\langle$B]}  
	 STRATEGISCHE ÜBERLEGUNGEN: {'Sie wollen C33 nicht, C49 und B61 haben sie schon akzeptiert. Ich wähle eine minimal-invasive Anpassung, um Akzeptanz zu maximieren: B21 → A61 (1-zu-1), Aufwand steigt nur um +17 auf 3886. Das lässt mein Kernset deutlich stärker werden, ohne ihren „Kern“ (A41, A04, A45, B53) oder die bereits aufgenommenen C49/C64/B61 anzutasten. Falls sie das annehmen, versuche ich als nächsten Schritt optional A03 via Tausch gegen B53 oder A04 reinzunehmen; C49/C64 halte ich fest (auf ihre Frage: C49 ist mir wichtiger als A03). C33 parke ich vorerst.'}\\ \tt ABLEHNUNG: {'A41', 'B21', 'B85', 'A45', 'A04', 'B53', 'B61', 'C49', 'C64'}\\ \tt VORSCHLAG: {'A41', 'B85', 'A45', 'A04', 'B53', 'B61', 'C49', 'C64', 'A61'}\\ \tt ARGUMENT: {'Minimaler 1-zu-1-Swap: Ersetze bitte B21 durch A61. Aufwand steigt nur von 3869 auf 3886 (weiterhin klar < 4026), dein Setup bleibt stabil (A41, A04, A45, B53, B85, B61, C49, C64 bleiben unverändert). Zu deiner Frage: Mir ist C49 wichtiger als A03 – deshalb halte ich C49 gern drin. Wenn du A03 perspektivisch unterbringen willst, können wir das gezielt über einen einzelnen Tausch (z. B. statt B53 oder A04) prüfen und sicher unter dem Limit bleiben. Passt dir der B21→A61-Swap so?'} 
	  } 
	   } 
	   } 
	  \\ 
 

    \theutterance \stepcounter{utterance}  

    & & \multicolumn{4}{p{0.6\linewidth}}{\cellcolor[rgb]{0.9,0.9,0.9}{%
	\makecell[{{p{\linewidth}}}]{% 
	  \tt {\tiny [A$\langle$GM]}  
	 ABLEHNUNG: {'A41', 'B21', 'B85', 'A45', 'A04', 'B53', 'B61', 'C49', 'C64'}\\ \tt VORSCHLAG: {'A41', 'B85', 'A45', 'A04', 'B53', 'B61', 'C49', 'C64', 'A61'}\\ \tt ARGUMENT: {'Minimaler 1-zu-1-Swap: Ersetze bitte B21 durch A61. Aufwand steigt nur von 3869 auf 3886 (weiterhin klar < 4026), dein Setup bleibt stabil (A41, A04, A45, B53, B85, B61, C49, C64 bleiben unverändert). Zu deiner Frage: Mir ist C49 wichtiger als A03 – deshalb halte ich C49 gern drin. Wenn du A03 perspektivisch unterbringen willst, können wir das gezielt über einen einzelnen Tausch (z. B. statt B53 oder A04) prüfen und sicher unter dem Limit bleiben. Passt dir der B21→A61-Swap so?'} 
	  } 
	   } 
	   } 
	 & \\ 
 

    \theutterance \stepcounter{utterance}  

    & \multicolumn{4}{p{0.6\linewidth}}{\cellcolor[rgb]{0.8,1,0.9}{%
	\makecell[{{p{\linewidth}}}]{% 
	  \tt {\tiny [A$\rangle$GM]}  
	 STRATEGISCHE ÜBERLEGUNGEN: {'Ihr Minimaltausch B21→A61 kostet mich sehr viel Eigenwert; B21 ist ein klares Muss. Um dennoch auf euren Wunsch A61 einzugehen und eure betonten Prioritäten (C49, B61) zu halten, biete ich den Kompromiss: A61 rein, dafür B85 und C64 raus. So bleibt Budget klar unter 4026, und ich gebe nur periphere Steine ab. Set-Wert für mich bleibt deutlich höher als in eurem Vorschlag. Fallbacks: Falls B85 euch wichtiger als C49 ist, könnte ich (ungern) C49 statt B85 herausnehmen; B61 möchte ich bevorzugt halten, da ihr ihn mehrfach priorisiert habt. Ziel: B21 sichern, A61 liefern, C49+B61 halten, C33 weiterhin draußen lassen.'}\\ \tt ABLEHNUNG: {'A41', 'B85', 'A45', 'A04', 'B53', 'B61', 'C49', 'C64', 'A61'}\\ \tt VORSCHLAG: {'A41', 'B21', 'A45', 'A04', 'B53', 'B61', 'C49', 'A61'}\\ \tt ARGUMENT: {'Ich nehme A61 auf und halte gleichzeitig deine hervorgehobenen Bausteine C49 und B61 – ich streiche dafür lediglich zwei Randstücke (B85 und C64). Der Gesamtaufwand liegt so bei 3856 und bleibt klar unter 4026. Damit bekommst du A61 ohne Budgetdruck, während der Rest des Setups stabil bleibt (A41, A04, A45, B53, B61, C49). Wenn B85 für dich unverzichtbar ist, sag mir bitte Bescheid – dann prüfen wir gezielt eine Alternative unter dem Limit, ohne A61 oder B21 anzutasten.'} 
	  } 
	   } 
	   } 
	 & & \\ 
 

    \theutterance \stepcounter{utterance}  

    & & \multicolumn{4}{p{0.6\linewidth}}{\cellcolor[rgb]{0.9,0.9,0.9}{%
	\makecell[{{p{\linewidth}}}]{% 
	  \tt {\tiny [GM$\rangle$B]}  
	 ABLEHNUNG: {'A41', 'B85', 'A45', 'A04', 'B53', 'B61', 'C49', 'C64', 'A61'}\\ \tt VORSCHLAG: {'A41', 'B21', 'A45', 'A04', 'B53', 'B61', 'C49', 'A61'}\\ \tt ARGUMENT: {'Ich nehme A61 auf und halte gleichzeitig deine hervorgehobenen Bausteine C49 und B61 – ich streiche dafür lediglich zwei Randstücke (B85 und C64). Der Gesamtaufwand liegt so bei 3856 und bleibt klar unter 4026. Damit bekommst du A61 ohne Budgetdruck, während der Rest des Setups stabil bleibt (A41, A04, A45, B53, B61, C49). Wenn B85 für dich unverzichtbar ist, sag mir bitte Bescheid – dann prüfen wir gezielt eine Alternative unter dem Limit, ohne A61 oder B21 anzutasten.'} 
	  } 
	   } 
	   } 
	 & \\ 
 

    \theutterance \stepcounter{utterance}  

    & & & \multicolumn{4}{p{0.6\linewidth}}{\cellcolor[rgb]{1,0.85,0.72}{%
	\makecell[{{p{\linewidth}}}]{% 
	  \tt {\tiny [GM$\langle$B]}  
	 STRATEGISCHE ÜBERLEGUNGEN: {'Ihr aktuelles Set bringt mir 3559 Punkte. Der leichteste, akzep­tanz­starke Boost ist, C64 (Wert 868, Aufwand 77) schlicht hinzuzunehmen: Aufwand steigt nur auf 3933 (<4026), mein Wertspringt um +868 auf 4427. Das dreht keinen deiner „Kern“-Steine und respektiert B21. Falls sie mehr Gegenleistung verlangen: nächster Schritt wäre A04→A03 (nur +29 Aufwand, +659 Wert), optional in einem Folgezug. Erst jetzt aber nur +C64 fordern, um ihrer Präferenz für kleine Schritte zu entsprechen.'}\\ \tt ABLEHNUNG: {'A41', 'B21', 'A45', 'A04', 'B53', 'B61', 'C49', 'A61'}\\ \tt VORSCHLAG: {'A41', 'B21', 'A45', 'A04', 'B53', 'B61', 'C49', 'A61', 'C64'}\\ \tt ARGUMENT: {'Minimaler, risikoarmer Schritt: Ergänze bitte nur C64. Gesamtaufwand steigt von 3856 auf 3933 (weiterhin 93 unter dem Limit 4026), dein Setup bleibt vollständig intakt (A41, B21, A45, A04, B53, B61, C49, A61 unverändert). C64 ist ultraleicht (77 Aufwand) und erhöht die Effizienz des Pakets deutlich. Wenn dir die Item-Anzahl Sorgen macht, können wir als nächsten, separaten Mini-Schritt A04 gezielt gegen A03 tauschen und weiterhin klar unter dem Limit bleiben – zunächst aber bitte C64 hinzufügen.'} 
	  } 
	   } 
	   } 
	  \\ 
 

    \theutterance \stepcounter{utterance}  

    & & \multicolumn{4}{p{0.6\linewidth}}{\cellcolor[rgb]{0.9,0.9,0.9}{%
	\makecell[{{p{\linewidth}}}]{% 
	  \tt {\tiny [A$\langle$GM]}  
	 ABLEHNUNG: {'A41', 'B21', 'A45', 'A04', 'B53', 'B61', 'C49', 'A61'}\\ \tt VORSCHLAG: {'A41', 'B21', 'A45', 'A04', 'B53', 'B61', 'C49', 'A61', 'C64'}\\ \tt ARGUMENT: {'Minimaler, risikoarmer Schritt: Ergänze bitte nur C64. Gesamtaufwand steigt von 3856 auf 3933 (weiterhin 93 unter dem Limit 4026), dein Setup bleibt vollständig intakt (A41, B21, A45, A04, B53, B61, C49, A61 unverändert). C64 ist ultraleicht (77 Aufwand) und erhöht die Effizienz des Pakets deutlich. Wenn dir die Item-Anzahl Sorgen macht, können wir als nächsten, separaten Mini-Schritt A04 gezielt gegen A03 tauschen und weiterhin klar unter dem Limit bleiben – zunächst aber bitte C64 hinzufügen.'} 
	  } 
	   } 
	   } 
	 & \\ 
 

    \theutterance \stepcounter{utterance}  

    & \multicolumn{4}{p{0.6\linewidth}}{\cellcolor[rgb]{0.8,1,0.9}{%
	\makecell[{{p{\linewidth}}}]{% 
	  \tt {\tiny [A$\rangle$GM]}  
	 STRATEGISCHE ÜBERLEGUNGEN: {'Ihr Zusatz C64 verbessert meinen Wert günstig (+65 bei +77 Aufwand) und lässt meinen Kern unberührt. Gesamtkosten 3933 < 4026. Versuche, B85 wieder hereinzunehmen (z. B. statt A61), wurden zuvor abgelehnt. Jetzt akzeptieren, um sicheren, starken Abschluss zu sichern.'}\\ \tt ZUSTIMMUNG: {'A41', 'B21', 'A45', 'A04', 'B53', 'B61', 'C49', 'A61', 'C64'}\\ \tt ARGUMENT: {'Das Paket bleibt klar unter dem Maximalaufwand und hält den gemeinsam etablierten Kern stabil; der kleine Zusatz ist effizient. Einverstanden.'} 
	  } 
	   } 
	   } 
	 & & \\ 
 

    \theutterance \stepcounter{utterance}  

    & & & \multicolumn{2}{p{0.3\linewidth}}{\cellcolor[rgb]{0.95,0.95,0.95}{%
	\makecell[{{p{\linewidth}}}]{% 
	  \tt {\tiny [GM$|$GM]}  
	 game successful 
	  } 
	   } 
	   } 
	 & & \\ 
 

    \theutterance \stepcounter{utterance}  

    & & & \multicolumn{2}{p{0.3\linewidth}}{\cellcolor[rgb]{0.95,0.95,0.95}{%
	\makecell[{{p{\linewidth}}}]{% 
	  \tt {\tiny [GM$|$GM]}  
	 end game 
	  } 
	   } 
	   } 
	 & & \\ 
 

\end{supertabular}
}

\end{document}
