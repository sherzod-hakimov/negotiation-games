
\documentclass{article}
\usepackage{colortbl}
\usepackage{makecell}
\usepackage{multirow}
\usepackage{supertabular}

\begin{document}

\newcounter{utterance}

\twocolumn

{ \footnotesize  \setcounter{utterance}{1}
\setlength{\tabcolsep}{0pt}
\begin{supertabular}{c@{$\;$}|p{.15\linewidth}@{}p{.15\linewidth}p{.15\linewidth}p{.15\linewidth}p{.15\linewidth}p{.15\linewidth}}

    \# & $\;$A & \multicolumn{4}{c}{Game Master} & $\;\:$B\\
    \hline 

    \theutterance \stepcounter{utterance}  

    & & \multicolumn{4}{p{0.6\linewidth}}{\cellcolor[rgb]{0.9,0.9,0.9}{%
	\makecell[{{p{\linewidth}}}]{% 
	  \tt {\tiny [A$\langle$GM]}  
	 Du nimmst an einem kollaborativen Verhandlungspiel Teil.\\ \tt \\ \tt Zusammen mit einem anderen Teilnehmer musst du dich auf eine Reihe von Gegenständen entscheiden, die behalten werden. Jeder von euch hat eine persönliche Verteilung über die Wichtigkeit der einzelnen Gegenstände. Jeder von euch hat eine eigene Meinung darüber, wie wichtig jeder einzelne Gegenstand ist (Gegenstandswichtigkeit). Du kennst die Wichtigkeitsverteilung des anderen Spielers nicht. Zusätzlich siehst du, wie viel Aufwand jeder Gegenstand verursacht.  \\ \tt Ihr dürft euch nur auf eine Reihe von Gegenständen einigen, wenn der Gesamtaufwand der ausgewählten Gegenstände den Maximalaufwand nicht überschreitet:\\ \tt \\ \tt Maximalaufwand = 3681\\ \tt \\ \tt Hier sind die einzelnen Aufwände der Gegenstände:\\ \tt \\ \tt Aufwand der Gegenstände = {"C33": 700, "A89": 586, "A48": 505, "A04": 116, "A41": 664, "C64": 940, "C49": 387, "A61": 392, "C04": 209, "B53": 571, "B85": 4, "B61": 285, "A45": 651, "A03": 613, "B21": 740}\\ \tt \\ \tt Hier ist deine persönliche Verteilung der Wichtigkeit der einzelnen Gegenstände:\\ \tt \\ \tt Werte der Gegenstandswichtigkeit = {"C33": 138, "A89": 583, "A48": 868, "A04": 822, "A41": 783, "C64": 65, "C49": 262, "A61": 121, "C04": 508, "B53": 780, "B85": 461, "B61": 484, "A45": 668, "A03": 389, "B21": 808}\\ \tt \\ \tt Ziel:\\ \tt \\ \tt Dein Ziel ist es, eine Reihe von Gegenständen auszuhandeln, die dir möglichst viel bringt (d. h. Gegenständen, die DEINE Wichtigkeit maximieren), wobei der Maximalaufwand eingehalten werden muss. Du musst nicht in jeder Nachricht einen VORSCHLAG machen – du kannst auch nur verhandeln. Alle Taktiken sind erlaubt!\\ \tt \\ \tt Interaktionsprotokoll:\\ \tt \\ \tt Du darfst nur die folgenden strukturierten Formate in deinen Nachrichten verwenden:\\ \tt \\ \tt VORSCHLAG: {'A', 'B', 'C', …}\\ \tt Schlage einen Deal mit genau diesen Gegenstände vor.\\ \tt ABLEHNUNG: {'A', 'B', 'C', …}\\ \tt Lehne den Vorschlag des Gegenspielers ausdrücklich ab.\\ \tt ARGUMENT: {'...'}\\ \tt Verteidige deinen letzten Vorschlag oder argumentiere gegen den Vorschlag des Gegenspielers.\\ \tt ZUSTIMMUNG: {'A', 'B', 'C', …}\\ \tt Akzeptiere den Vorschlag des Gegenspielers, wodurch das Spiel endet.\\ \tt STRATEGISCHE ÜBERLEGUNGEN: {'...'}\\ \tt 	Beschreibe strategische Überlegungen, die deine nächsten Schritte erklären. Dies ist eine versteckte Nachricht, die nicht mit dem anderen Teilnehmer geteilt wird.\\ \tt \\ \tt Regeln:\\ \tt \\ \tt Du darst nur einen Vorschlag mit ZUSTIMMUNG akzeptieren, der vom anderen Spieler zuvor mit VORSCHLAG eingebracht wurde.\\ \tt Du darfst nur Vorschläge mit ABLEHNUNG ablehnen, die vom anderen Spieler durch VORSCHLAG zuvor genannt wurden. \\ \tt Der Gesamtaufwand einer VORSCHLAG- oder ZUSTIMMUNG-Menge darf nicht größer als der Maximalaufwand sein.  \\ \tt Offenbare deine versteckte Wichtigkeitsverteilung nicht.\\ \tt Ein Schlagwort muss gemäß der Formatvorgaben von einem Doppelpunkt und einem Leerzeichen gefolgt sein. Das Argument muss eine Python-Menge mit 0 oder mehr Strings sein.  \\ \tt Die Form muss also SCHLAGWORT: {...} sein.\\ \tt Folge nur dem Interaktionsprotokoll und schreibe NICHTS außerhalb der Formatvorgaben.\\ \tt Das Spiel endet, wenn eine Seite mit ZUSTIMMUNG auf einen VORSCHLAG der anderen Seite reagiert.  \\ \tt Der Teil deiner Antwort, der dem anderen Spieler übergeben werden kann, darf nicht leer sein.  \\ \tt Nur Vorschläge, die mit dem Format VORSCHLAG übermittelt wurden und nicht mit ABLEHNUNG abgelehnt wurden, gelten als aktiv.  \\ \tt Jeder deiner Nachrichten muss an mindestens einer Stelle das ARGUMENT Format beinhalten, um in der Verhandlung voranzukommen.\\ \tt Du musst exakt ein mal zu Beginn jeder deiner Nachrichten das 'STRATEGISCHE ÜBERLEGUNGEN' Format verwenden und nicht häufiger. Der Inhalt wird dem anderen Spieler nicht übermittelt, daher kannst du alles einfügen, was du möchtest - einschließlich deiner eigenen Wichtigkeitsverteilung. Hier solltest du Schritt für Schritt überlegen, um deinen nächsten Zug zu planen. 
	  } 
	   } 
	   } 
	 & \\ 
 

    \theutterance \stepcounter{utterance}  

    & \multicolumn{4}{p{0.6\linewidth}}{\cellcolor[rgb]{0.8,1,0.9}{%
	\makecell[{{p{\linewidth}}}]{% 
	  \tt {\tiny [A$\rangle$GM]}  
	 STRATEGISCHE ÜBERLEGUNGEN: {'Ziel: Maximieren meiner Wichtigkeit unter 3681 Aufwand. Werte: {"C33": 138, "A89": 583, "A48": 868, "A04": 822, "A41": 783, "C64": 65, "C49": 262, "A61": 121, "C04": 508, "B53": 780, "B85": 461, "B61": 484, "A45": 668, "A03": 389, "B21": 808}. Aufwände: {"C33": 700, "A89": 586, "A48": 505, "A04": 116, "A41": 664, "C64": 940, "C49": 387, "A61": 392, "C04": 209, "B53": 571, "B85": 4, "B61": 285, "A45": 651, "A03": 613, "B21": 740}. Bestes Paket für mich bei Kapazität 3681: S* = {B85, A04, C04, A48, B61, B53, A41, A89, B21} mit Aufwand 3680 und Wert 6097. Nächstbeste Alternativen: - Falls Gegenüber A45 unbedingt will, Tausch A41 -> A45: Aufwand 3667, Wert ~5982 (nur -115). - A45 statt A45? (bereits berücksichtigt). - Schlechte Kandidaten, die ich meiden will: C64, A61, C33, C49, A03 (ineffizient nach meinen Werten). Verhandlungsplan: Hart ankern mit S*, hervorheben: unter Limit, viele Low-cost/hoher Nutzen Stücke (B85, A04, C04). Öffnung für genau einen Tauschslot (A41<->A45) signalisieren, um ihre Präferenz zu explorieren. Nachfragen nach 1-2 Must-haves von ihnen. Keine Offenlegung meiner Werte.'}\\ \tt VORSCHLAG: {'B85', 'A04', 'C04', 'A48', 'B61', 'B53', 'A41', 'A89', 'B21'}\\ \tt ARGUMENT: {'Dieser Vorschlag bleibt mit Gesamtaufwand 3680 klar unter dem Maximalaufwand 3681.', 'Er setzt stark auf sehr effiziente Gegenstände (z. B. B85 kostet praktisch nichts, A04 und C04 haben sehr gutes Nutzen-zu-Aufwand-Verhältnis).', 'Falls dir A45 wichtig ist, können wir A41 gegen A45 tauschen – der Gesamtaufwand sinkt dann sogar auf 3667.', 'Welche 1–2 Gegenstände sind für dich unverzichtbar? Dann prüfe ich gezielt, wie wir sie ohne Überschreitung des Aufwands integrieren können.'} 
	  } 
	   } 
	   } 
	 & & \\ 
 

    \theutterance \stepcounter{utterance}  

    & & \multicolumn{4}{p{0.6\linewidth}}{\cellcolor[rgb]{0.9,0.9,0.9}{%
	\makecell[{{p{\linewidth}}}]{% 
	  \tt {\tiny [GM$\rangle$B]}  
	 Du nimmst an einem kollaborativen Verhandlungspiel Teil.\\ \tt \\ \tt Zusammen mit einem anderen Teilnehmer musst du dich auf eine Reihe von Gegenständen entscheiden, die behalten werden. Jeder von euch hat eine persönliche Verteilung über die Wichtigkeit der einzelnen Gegenstände. Jeder von euch hat eine eigene Meinung darüber, wie wichtig jeder einzelne Gegenstand ist (Gegenstandswichtigkeit). Du kennst die Wichtigkeitsverteilung des anderen Spielers nicht. Zusätzlich siehst du, wie viel Aufwand jeder Gegenstand verursacht.  \\ \tt Ihr dürft euch nur auf eine Reihe von Gegenständen einigen, wenn der Gesamtaufwand der ausgewählten Gegenstände den Maximalaufwand nicht überschreitet:\\ \tt \\ \tt Maximalaufwand = 3681\\ \tt \\ \tt Hier sind die einzelnen Aufwände der Gegenstände:\\ \tt \\ \tt Aufwand der Gegenstände = {"C33": 700, "A89": 586, "A48": 505, "A04": 116, "A41": 664, "C64": 940, "C49": 387, "A61": 392, "C04": 209, "B53": 571, "B85": 4, "B61": 285, "A45": 651, "A03": 613, "B21": 740}\\ \tt \\ \tt Hier ist deine persönliche Verteilung der Wichtigkeit der einzelnen Gegenstände:\\ \tt \\ \tt Werte der Gegenstandswichtigkeit = {"A48": 65, "A04": 121, "B21": 138, "A41": 262, "B53": 389, "A45": 461, "A89": 484, "C04": 508, "B61": 583, "B85": 668, "A03": 780, "C49": 783, "C33": 808, "A61": 822, "C64": 868}\\ \tt \\ \tt Ziel:\\ \tt \\ \tt Dein Ziel ist es, eine Reihe von Gegenständen auszuhandeln, die dir möglichst viel bringt (d. h. Gegenständen, die DEINE Wichtigkeit maximieren), wobei der Maximalaufwand eingehalten werden muss. Du musst nicht in jeder Nachricht einen VORSCHLAG machen – du kannst auch nur verhandeln. Alle Taktiken sind erlaubt!\\ \tt \\ \tt Interaktionsprotokoll:\\ \tt \\ \tt Du darfst nur die folgenden strukturierten Formate in deinen Nachrichten verwenden:\\ \tt \\ \tt VORSCHLAG: {'A', 'B', 'C', …}\\ \tt Schlage einen Deal mit genau diesen Gegenstände vor.\\ \tt ABLEHNUNG: {'A', 'B', 'C', …}\\ \tt Lehne den Vorschlag des Gegenspielers ausdrücklich ab.\\ \tt ARGUMENT: {'...'}\\ \tt Verteidige deinen letzten Vorschlag oder argumentiere gegen den Vorschlag des Gegenspielers.\\ \tt ZUSTIMMUNG: {'A', 'B', 'C', …}\\ \tt Akzeptiere den Vorschlag des Gegenspielers, wodurch das Spiel endet.\\ \tt STRATEGISCHE ÜBERLEGUNGEN: {'...'}\\ \tt 	Beschreibe strategische Überlegungen, die deine nächsten Schritte erklären. Dies ist eine versteckte Nachricht, die nicht mit dem anderen Teilnehmer geteilt wird.\\ \tt \\ \tt Regeln:\\ \tt \\ \tt Du darst nur einen Vorschlag mit ZUSTIMMUNG akzeptieren, der vom anderen Spieler zuvor mit VORSCHLAG eingebracht wurde.\\ \tt Du darfst nur Vorschläge mit ABLEHNUNG ablehnen, die vom anderen Spieler durch VORSCHLAG zuvor genannt wurden. \\ \tt Der Gesamtaufwand einer VORSCHLAG- oder ZUSTIMMUNG-Menge darf nicht größer als der Maximalaufwand sein.  \\ \tt Offenbare deine versteckte Wichtigkeitsverteilung nicht.\\ \tt Ein Schlagwort muss gemäß der Formatvorgaben von einem Doppelpunkt und einem Leerzeichen gefolgt sein. Das Argument muss eine Python-Menge mit 0 oder mehr Strings sein.  \\ \tt Die Form muss also SCHLAGWORT: {...} sein.\\ \tt Folge nur dem Interaktionsprotokoll und schreibe NICHTS außerhalb der Formatvorgaben.\\ \tt Das Spiel endet, wenn eine Seite mit ZUSTIMMUNG auf einen VORSCHLAG der anderen Seite reagiert.  \\ \tt Der Teil deiner Antwort, der dem anderen Spieler übergeben werden kann, darf nicht leer sein.  \\ \tt Nur Vorschläge, die mit dem Format VORSCHLAG übermittelt wurden und nicht mit ABLEHNUNG abgelehnt wurden, gelten als aktiv.  \\ \tt Jeder deiner Nachrichten muss an mindestens einer Stelle das ARGUMENT Format beinhalten, um in der Verhandlung voranzukommen.\\ \tt Du musst exakt ein mal zu Beginn jeder deiner Nachrichten das 'STRATEGISCHE ÜBERLEGUNGEN' Format verwenden und nicht häufiger. Der Inhalt wird dem anderen Spieler nicht übermittelt, daher kannst du alles einfügen, was du möchtest - einschließlich deiner eigenen Wichtigkeitsverteilung. Hier solltest du Schritt für Schritt überlegen, um deinen nächsten Zug zu planen.\\ \tt \\ \tt Du erhältst nun die erste Nachricht des anderen Spielers.\\ \tt \\ \tt VORSCHLAG: {'B85', 'A04', 'C04', 'A48', 'B61', 'B53', 'A41', 'A89', 'B21'}\\ \tt ARGUMENT: {'Dieser Vorschlag bleibt mit Gesamtaufwand 3680 klar unter dem Maximalaufwand 3681.', 'Er setzt stark auf sehr effiziente Gegenstände (z. B. B85 kostet praktisch nichts, A04 und C04 haben sehr gutes Nutzen-zu-Aufwand-Verhältnis).', 'Falls dir A45 wichtig ist, können wir A41 gegen A45 tauschen – der Gesamtaufwand sinkt dann sogar auf 3667.', 'Welche 1–2 Gegenstände sind für dich unverzichtbar? Dann prüfe ich gezielt, wie wir sie ohne Überschreitung des Aufwands integrieren können.'} 
	  } 
	   } 
	   } 
	 & \\ 
 

    \theutterance \stepcounter{utterance}  

    & & & \multicolumn{4}{p{0.6\linewidth}}{\cellcolor[rgb]{1,0.85,0.72}{%
	\makecell[{{p{\linewidth}}}]{% 
	  \tt {\tiny [GM$\langle$B]}  
	 STRATEGISCHE ÜBERLEGUNGEN: {'Maximiere meinen Wert: Top-Items nach meiner Wichtigkeit sind C64, A61, C33, C49, A03, dazu die extrem effizienten B85, C04, B61 und das günstige A04. Ihr Vorschlag bringt mir relativ wenig (viele schwache Items wie B21, A41, A48). Mein Konter wählt die stärksten Preis-Leistungs-Stücke und bleibt deutlich unter dem Limit (3646). Falls sie A89/B53 unbedingt wollen, biete ich gezielte Swaps an: C33 -> A89 (Kosten 3532), oder A03 -> B53 (Kosten 3604). A45 kann ich statt C33 integrieren (Kosten 3597). Ziel: Erst starkes Ankerpaket setzen, gleichzeitig Flexibilität signalisieren und sie bitten, 1–2 Must-haves zu nennen.'}\\ \tt ABLEHNUNG: {'B85', 'A04', 'C04', 'A48', 'B61', 'B53', 'A41', 'A89', 'B21'}\\ \tt VORSCHLAG: {'B85', 'A04', 'C04', 'B61', 'C49', 'A61', 'A03', 'C33', 'C64'}\\ \tt ARGUMENT: {'Mein Paket nutzt den Aufwand sehr effizient: Gesamtaufwand 3646 (unter 3681) und konzentriert sich auf durchweg starke, kosteneffiziente Positionen (B85, A04, C04, B61) plus einige tragende Elemente.', 'Gegenüber deinem Vorschlag verzichte ich auf schwere, weniger effiziente Posten wie B21, A41 und A48 – dadurch bleibt mehr Wert pro Aufwand übrig.', 'Wir haben bereits mehrere deiner effizienten Picks übernommen (B85, A04, C04, B61) und die Auswahl gestrafft.', 'Falls dir A89 wichtig ist, kann ich C33 -> A89 tauschen. Dann liegen wir bei ca. 3532 Gesamtaufwand.', 'Falls dir B53 wichtig ist, kann ich A03 -> B53 tauschen. Dann liegen wir bei ca. 3604 Gesamtaufwand.', 'Wenn A45 für dich zentral ist, kann ich A45 statt C33 aufnehmen. Gesamtaufwand ca. 3597.', 'Nenne mir bitte die 1–2 unverzichtbaren Gegenstände für dich. Ich prüfe dann gezielt die besten Swaps, ohne den Maximalaufwand zu reißen.'} 
	  } 
	   } 
	   } 
	  \\ 
 

    \theutterance \stepcounter{utterance}  

    & & \multicolumn{4}{p{0.6\linewidth}}{\cellcolor[rgb]{0.9,0.9,0.9}{%
	\makecell[{{p{\linewidth}}}]{% 
	  \tt {\tiny [A$\langle$GM]}  
	 ABLEHNUNG: {'B85', 'A04', 'C04', 'A48', 'B61', 'B53', 'A41', 'A89', 'B21'}\\ \tt VORSCHLAG: {'B85', 'A04', 'C04', 'B61', 'C49', 'A61', 'A03', 'C33', 'C64'}\\ \tt ARGUMENT: {'Mein Paket nutzt den Aufwand sehr effizient: Gesamtaufwand 3646 (unter 3681) und konzentriert sich auf durchweg starke, kosteneffiziente Positionen (B85, A04, C04, B61) plus einige tragende Elemente.', 'Gegenüber deinem Vorschlag verzichte ich auf schwere, weniger effiziente Posten wie B21, A41 und A48 – dadurch bleibt mehr Wert pro Aufwand übrig.', 'Wir haben bereits mehrere deiner effizienten Picks übernommen (B85, A04, C04, B61) und die Auswahl gestrafft.', 'Falls dir A89 wichtig ist, kann ich C33 -> A89 tauschen. Dann liegen wir bei ca. 3532 Gesamtaufwand.', 'Falls dir B53 wichtig ist, kann ich A03 -> B53 tauschen. Dann liegen wir bei ca. 3604 Gesamtaufwand.', 'Wenn A45 für dich zentral ist, kann ich A45 statt C33 aufnehmen. Gesamtaufwand ca. 3597.', 'Nenne mir bitte die 1–2 unverzichtbaren Gegenstände für dich. Ich prüfe dann gezielt die besten Swaps, ohne den Maximalaufwand zu reißen.'} 
	  } 
	   } 
	   } 
	 & \\ 
 

    \theutterance \stepcounter{utterance}  

    & \multicolumn{4}{p{0.6\linewidth}}{\cellcolor[rgb]{0.8,1,0.9}{%
	\makecell[{{p{\linewidth}}}]{% 
	  \tt {\tiny [A$\rangle$GM]}  
	 STRATEGISCHE ÜBERLEGUNGEN: {'Ihr Vorschlag fokussiert stark auf für mich sehr schwache Items (C64, C33, A61) und lässt mehrere meiner Top-Picks draußen (A48, B53, A41/B21). Mein Ziel bleibt das Paket mit maximalem Eigenwert nahe meiner Optimalmenge S* = {B85, A04, C04, A48, B61, B53, A41, A89, B21}. Da sie A45/A89/B53 als mögliche Swaps signalisiert haben, biete ich ein kompromissorientiertes Gegenset an, das B21 weglässt, aber A45 inkludiert: P1 = {B85, A04, C04, B61, A48, B53, A89, A45, A41} (Aufwand 3591, hoher Wert für mich). Falls sie einzelne ihrer schwächeren Items unbedingt brauchen, bin ich bereit, v. a. A89 zu tauschen (kleinster Wertverlust für mich) gegen A03 oder C49, bleibe aber hart gegen C64/C33, weil diese mein Paket stark verschlechtern oder das Limit sprengen.'}\\ \tt ABLEHNUNG: {'B85', 'A04', 'C04', 'B61', 'C49', 'A61', 'A03', 'C33', 'C64'}\\ \tt VORSCHLAG: {'B85', 'A04', 'C04', 'B61', 'A48', 'B53', 'A89', 'A45', 'A41'}\\ \tt ARGUMENT: {'Ich lehne deinen Vorschlag ab, weil er zu viel Aufwand in sehr schwache Posten bindet (insb. C64, C33, A61) und dadurch gemeinsamer Spielraum verschenkt wird.', 'Mein Gegenvorschlag übernimmt vier deiner effizienten Picks (B85, A04, C04, B61) und integriert zusätzlich die von dir ins Spiel gebrachten Alternativen (A89, B53, A45).', 'Gesamtaufwand 3591 – wir bleiben komfortabel unter dem Maximalaufwand 3681.', 'Wenn dir B21 lieber ist als A41, tausche ich A41 -> B21; der Aufwand läge dann bei 3667.', 'Falls dir A03 wichtig ist, können wir A89 -> A03 tauschen; der Aufwand läge dann bei 3618.', 'Wenn du auf C49 bestehst, tausche ich A89 -> C49; der Aufwand läge dann bei 3392.', 'C33 oder C64 lassen sich nur mit drastischen Abstrichen integrieren. Nenne mir bitte deine 1–2 Must-haves, dann optimiere ich gezielt weiter, ohne das Limit zu reißen.'} 
	  } 
	   } 
	   } 
	 & & \\ 
 

    \theutterance \stepcounter{utterance}  

    & & \multicolumn{4}{p{0.6\linewidth}}{\cellcolor[rgb]{0.9,0.9,0.9}{%
	\makecell[{{p{\linewidth}}}]{% 
	  \tt {\tiny [GM$\rangle$B]}  
	 ABLEHNUNG: {'B85', 'A04', 'C04', 'B61', 'C49', 'A61', 'A03', 'C33', 'C64'}\\ \tt VORSCHLAG: {'B85', 'A04', 'C04', 'B61', 'A48', 'B53', 'A89', 'A45', 'A41'}\\ \tt ARGUMENT: {'Ich lehne deinen Vorschlag ab, weil er zu viel Aufwand in sehr schwache Posten bindet (insb. C64, C33, A61) und dadurch gemeinsamer Spielraum verschenkt wird.', 'Mein Gegenvorschlag übernimmt vier deiner effizienten Picks (B85, A04, C04, B61) und integriert zusätzlich die von dir ins Spiel gebrachten Alternativen (A89, B53, A45).', 'Gesamtaufwand 3591 – wir bleiben komfortabel unter dem Maximalaufwand 3681.', 'Wenn dir B21 lieber ist als A41, tausche ich A41 -> B21; der Aufwand läge dann bei 3667.', 'Falls dir A03 wichtig ist, können wir A89 -> A03 tauschen; der Aufwand läge dann bei 3618.', 'Wenn du auf C49 bestehst, tausche ich A89 -> C49; der Aufwand läge dann bei 3392.', 'C33 oder C64 lassen sich nur mit drastischen Abstrichen integrieren. Nenne mir bitte deine 1–2 Must-haves, dann optimiere ich gezielt weiter, ohne das Limit zu reißen.'} 
	  } 
	   } 
	   } 
	 & \\ 
 

    \theutterance \stepcounter{utterance}  

    & & & \multicolumn{4}{p{0.6\linewidth}}{\cellcolor[rgb]{1,0.85,0.72}{%
	\makecell[{{p{\linewidth}}}]{% 
	  \tt {\tiny [GM$\langle$B]}  
	 STRATEGISCHE ÜBERLEGUNGEN: {'Ihr letzter Vorschlag zeigt Präferenzen für A48, B53, A89, A45, A41 und akzeptiert effiziente Basics (B85, A04, C04, B61). Sie lehnen A61/C33/C64 ab. Meine beste Annäherung: Ihre Struktur respektieren, aber A03 (statt A89) und C49 (statt A48) integrieren, da sie selbst diese Öffnungen signalisiert haben. Neuer Anker: {B85, A04, C04, B61, C49, B53, A03, A45, A41} mit Aufwand 3500. Falls sie A89 fordern: A03->A89 (3473). Falls sie A48 fordern: A41->A48 (3341). Falls beide: A03 & A41 -> A89 & A48 (3314). Ziel: Einigung über 1–2 Must-haves sichern und dann schnell schließen.'}\\ \tt ABLEHNUNG: {'B85', 'A04', 'C04', 'B61', 'A48', 'B53', 'A89', 'A45', 'A41'}\\ \tt VORSCHLAG: {'B85', 'A04', 'C04', 'B61', 'C49', 'B53', 'A03', 'A45', 'A41'}\\ \tt ARGUMENT: {'Gesamtaufwand 3500 – klar unter dem Limit 3681.', 'Ich übernehme deine effizienten Picks (B85, A04, C04, B61) und halte B53, A45 und A41 drin, damit deine Kernstruktur erhalten bleibt.', 'Entsprechend deiner eigenen Öffnung ersetze ich A89 -> A03. Außerdem tausche ich A48 -> C49: C49 ist sehr aufwandsstark pro Einheit und reduziert den Aufwand weiter.', 'Wenn A89 für dich unverzichtbar ist: A03 -> A89; neuer Aufwand 3473.', 'Wenn A48 für dich unverzichtbar ist: A41 -> A48; neuer Aufwand 3341.', 'Wenn beide (A89 und A48) Must-haves sind: A03 und A41 -> A89 und A48; neuer Aufwand 3314.', 'Nenne mir bitte die 1–2 Must-haves, dann finalisiere ich sofort mit der entsprechenden Variante.'} 
	  } 
	   } 
	   } 
	  \\ 
 

    \theutterance \stepcounter{utterance}  

    & & \multicolumn{4}{p{0.6\linewidth}}{\cellcolor[rgb]{0.9,0.9,0.9}{%
	\makecell[{{p{\linewidth}}}]{% 
	  \tt {\tiny [A$\langle$GM]}  
	 ABLEHNUNG: {'B85', 'A04', 'C04', 'B61', 'A48', 'B53', 'A89', 'A45', 'A41'}\\ \tt VORSCHLAG: {'B85', 'A04', 'C04', 'B61', 'C49', 'B53', 'A03', 'A45', 'A41'}\\ \tt ARGUMENT: {'Gesamtaufwand 3500 – klar unter dem Limit 3681.', 'Ich übernehme deine effizienten Picks (B85, A04, C04, B61) und halte B53, A45 und A41 drin, damit deine Kernstruktur erhalten bleibt.', 'Entsprechend deiner eigenen Öffnung ersetze ich A89 -> A03. Außerdem tausche ich A48 -> C49: C49 ist sehr aufwandsstark pro Einheit und reduziert den Aufwand weiter.', 'Wenn A89 für dich unverzichtbar ist: A03 -> A89; neuer Aufwand 3473.', 'Wenn A48 für dich unverzichtbar ist: A41 -> A48; neuer Aufwand 3341.', 'Wenn beide (A89 und A48) Must-haves sind: A03 und A41 -> A89 und A48; neuer Aufwand 3314.', 'Nenne mir bitte die 1–2 Must-haves, dann finalisiere ich sofort mit der entsprechenden Variante.'} 
	  } 
	   } 
	   } 
	 & \\ 
 

    \theutterance \stepcounter{utterance}  

    & \multicolumn{4}{p{0.6\linewidth}}{\cellcolor[rgb]{0.8,1,0.9}{%
	\makecell[{{p{\linewidth}}}]{% 
	  \tt {\tiny [A$\rangle$GM]}  
	 STRATEGISCHE ÜBERLEGUNGEN: {'Ihr Paket von 3500 Aufwand hat für mich zwei klare Schwachpunkte: C49 (niedriger Beitrag) und A03 (ebenfalls eher schwach). Ich will schrittweise zu meinem starken Kern zurück, ohne zu konfrontativ zu wirken. Beste nächste Bewegung: Minimaländerung mit großem Effekt – nur C49 -> A48 tauschen. Das erhöht den Aufwand moderat (auf 3618 < 3681), hebt meinen Wert aber stark. Falls Widerstand gegen A48, biete ich als Alternativen die von ihnen selbst eröffneten Swaps A03 -> A89 und ggf. A41 -> A48 an. Ich vermeide harte Forderungen nach B21 und blocke C64/C33/A61 weiter.'}\\ \tt ABLEHNUNG: {'B85', 'A04', 'C04', 'B61', 'C49', 'B53', 'A03', 'A45', 'A41'}\\ \tt VORSCHLAG: {'B85', 'A04', 'C04', 'B61', 'A48', 'B53', 'A03', 'A45', 'A41'}\\ \tt ARGUMENT: {'Ich übernehme 8/9 deiner Auswahl unverändert und tausche nur C49 -> A48.', 'Gesamtaufwand 3618 – klar unter dem Limit 3681.', 'Dieser eine Tausch steigert die gemeinsame Effizienz deutlich, ohne unser Budget zu gefährden.', 'Falls du lieber bei C49 bleibst, können wir alternativ A03 -> A89 tauschen (dein Ansatz), dann bleiben wir bei deinem Paket und landen bei 3473 Gesamtaufwand.', 'Oder, wenn du beides bevorzugst: A03 -> A89 und A41 -> A48 (wie von dir skizziert), Aufwand 3314.', 'Gib mir ein Signal, ob A48 und/oder A89 für dich akzeptabel sind – dann schließe ich sofort ab.'} 
	  } 
	   } 
	   } 
	 & & \\ 
 

    \theutterance \stepcounter{utterance}  

    & & \multicolumn{4}{p{0.6\linewidth}}{\cellcolor[rgb]{0.9,0.9,0.9}{%
	\makecell[{{p{\linewidth}}}]{% 
	  \tt {\tiny [GM$\rangle$B]}  
	 ABLEHNUNG: {'B85', 'A04', 'C04', 'B61', 'C49', 'B53', 'A03', 'A45', 'A41'}\\ \tt VORSCHLAG: {'B85', 'A04', 'C04', 'B61', 'A48', 'B53', 'A03', 'A45', 'A41'}\\ \tt ARGUMENT: {'Ich übernehme 8/9 deiner Auswahl unverändert und tausche nur C49 -> A48.', 'Gesamtaufwand 3618 – klar unter dem Limit 3681.', 'Dieser eine Tausch steigert die gemeinsame Effizienz deutlich, ohne unser Budget zu gefährden.', 'Falls du lieber bei C49 bleibst, können wir alternativ A03 -> A89 tauschen (dein Ansatz), dann bleiben wir bei deinem Paket und landen bei 3473 Gesamtaufwand.', 'Oder, wenn du beides bevorzugst: A03 -> A89 und A41 -> A48 (wie von dir skizziert), Aufwand 3314.', 'Gib mir ein Signal, ob A48 und/oder A89 für dich akzeptabel sind – dann schließe ich sofort ab.'} 
	  } 
	   } 
	   } 
	 & \\ 
 

    \theutterance \stepcounter{utterance}  

    & & & \multicolumn{4}{p{0.6\linewidth}}{\cellcolor[rgb]{1,0.85,0.72}{%
	\makecell[{{p{\linewidth}}}]{% 
	  \tt {\tiny [GM$\langle$B]}  
	 STRATEGISCHE ÜBERLEGUNGEN: {'Mein Ziel: C49 unbedingt rein (sehr wertvoll für mich), A03 ideal behalten – aber sie haben C49+ A03 bereits abgelehnt. Zweitbeste erreichbare Option laut ihrer eigenen Öffnung: C49 drin, dafür A03 -> A89. Das verbessert meinen Wert stark, bleibt weit unter dem Limit und respektiert ihre Kernstruktur (B53, A45, A41, effiziente Basics). Ich biete das jetzt an und halte als weichen Fallback noch eine Variante mit A48 bereit, falls sie darauf beharren.'}\\ \tt ABLEHNUNG: {'B85', 'A04', 'C04', 'B61', 'A48', 'B53', 'A03', 'A45', 'A41'}\\ \tt VORSCHLAG: {'B85', 'A04', 'C04', 'B61', 'C49', 'B53', 'A89', 'A45', 'A41'}\\ \tt ARGUMENT: {'Gesamtaufwand 3473 – deutlich unter dem Maximalaufwand 3681.', 'Ich übernehme deine Kernstruktur (7/9 identisch: B85, A04, C04, B61, B53, A45, A41) und setze deine vorgeschlagene Alternative um: A03 -> A89.', 'Zudem revidiere ich den einen Schritt C49 -> A48 zurück auf C49, wie von dir selbst als Option eingeräumt – das hält den Aufwand niedrig und die Auswahl schlank.', 'Wenn dir A48 wichtig ist, kann ich A41 -> A48 tauschen; neuer Aufwand 3314 (alle übrigen Posten unverändert).', 'Gib mir bitte kurz grünes Licht – ich bin bereit, damit sofort abzuschließen.'} 
	  } 
	   } 
	   } 
	  \\ 
 

    \theutterance \stepcounter{utterance}  

    & & \multicolumn{4}{p{0.6\linewidth}}{\cellcolor[rgb]{0.9,0.9,0.9}{%
	\makecell[{{p{\linewidth}}}]{% 
	  \tt {\tiny [A$\langle$GM]}  
	 ABLEHNUNG: {'B85', 'A04', 'C04', 'B61', 'A48', 'B53', 'A03', 'A45', 'A41'}\\ \tt VORSCHLAG: {'B85', 'A04', 'C04', 'B61', 'C49', 'B53', 'A89', 'A45', 'A41'}\\ \tt ARGUMENT: {'Gesamtaufwand 3473 – deutlich unter dem Maximalaufwand 3681.', 'Ich übernehme deine Kernstruktur (7/9 identisch: B85, A04, C04, B61, B53, A45, A41) und setze deine vorgeschlagene Alternative um: A03 -> A89.', 'Zudem revidiere ich den einen Schritt C49 -> A48 zurück auf C49, wie von dir selbst als Option eingeräumt – das hält den Aufwand niedrig und die Auswahl schlank.', 'Wenn dir A48 wichtig ist, kann ich A41 -> A48 tauschen; neuer Aufwand 3314 (alle übrigen Posten unverändert).', 'Gib mir bitte kurz grünes Licht – ich bin bereit, damit sofort abzuschließen.'} 
	  } 
	   } 
	   } 
	 & \\ 
 

    \theutterance \stepcounter{utterance}  

    & \multicolumn{4}{p{0.6\linewidth}}{\cellcolor[rgb]{0.8,1,0.9}{%
	\makecell[{{p{\linewidth}}}]{% 
	  \tt {\tiny [A$\rangle$GM]}  
	 STRATEGISCHE ÜBERLEGUNGEN: {'Ihr letztes Paket S0 = {B85, A04, C04, B61, C49, B53, A89, A45, A41} hat Aufwand 3473 und Wert 5351 für mich. Minimalverbesserung mit hoher Annahmewahrscheinlichkeit: A41 -> A48 (Option kam von ihnen). Das ergibt S1 = {B85, A04, C04, B61, C49, B53, A89, A45, A48} mit Aufwand 3314 und Wert 5436 (+85). Wenn sie das dennoch ablehnen, akzeptiere ich im nächsten Zug S0 per ZUSTIMMUNG, um sicher zu closen.'}\\ \tt ABLEHNUNG: {'B85', 'A04', 'C04', 'B61', 'C49', 'B53', 'A89', 'A45', 'A41'}\\ \tt VORSCHLAG: {'B85', 'A04', 'C04', 'B61', 'C49', 'B53', 'A89', 'A45', 'A48'}\\ \tt ARGUMENT: {'Ich übernehme 8/9 deiner Auswahl unverändert und setze nur deine eigene Option um: A41 -> A48.', 'Gesamtaufwand 3314 – deutlich unter dem Maximalaufwand 3681.', 'Deine Kernstruktur (B85, A04, C04, B61, C49, B53, A89, A45) bleibt vollständig erhalten.', 'Wenn du damit einverstanden bist, können wir sofort abschließen.'} 
	  } 
	   } 
	   } 
	 & & \\ 
 

    \theutterance \stepcounter{utterance}  

    & & \multicolumn{4}{p{0.6\linewidth}}{\cellcolor[rgb]{0.9,0.9,0.9}{%
	\makecell[{{p{\linewidth}}}]{% 
	  \tt {\tiny [GM$\rangle$B]}  
	 ABLEHNUNG: {'B85', 'A04', 'C04', 'B61', 'C49', 'B53', 'A89', 'A45', 'A41'}\\ \tt VORSCHLAG: {'B85', 'A04', 'C04', 'B61', 'C49', 'B53', 'A89', 'A45', 'A48'}\\ \tt ARGUMENT: {'Ich übernehme 8/9 deiner Auswahl unverändert und setze nur deine eigene Option um: A41 -> A48.', 'Gesamtaufwand 3314 – deutlich unter dem Maximalaufwand 3681.', 'Deine Kernstruktur (B85, A04, C04, B61, C49, B53, A89, A45) bleibt vollständig erhalten.', 'Wenn du damit einverstanden bist, können wir sofort abschließen.'} 
	  } 
	   } 
	   } 
	 & \\ 
 

    \theutterance \stepcounter{utterance}  

    & & & \multicolumn{4}{p{0.6\linewidth}}{\cellcolor[rgb]{1,0.85,0.72}{%
	\makecell[{{p{\linewidth}}}]{% 
	  \tt {\tiny [GM$\langle$B]}  
	 STRATEGISCHE ÜBERLEGUNGEN: {'Ihr aktuelles Paket ist für mich schon solide (Wert 4062). Ein Minimal-Tausch A89 -> A03 bringt mir +296 zusätzlichen Wert (auf 4358) bei nur +27 Aufwand (gesamt 3341). A48 scheint ihr Must-have; das lasse ich unangetastet. C49 bleibt drin (sehr wichtig für mich). A61/C33/C64 wurden zuvor von ihnen als schwach abgelehnt – die pushe ich jetzt nicht. Plan: ihren Vorschlag formal ablehnen und exakt einen Tausch fordern (A89 -> A03), mit Verweis darauf, dass sie diese Option selbst angeboten hatten. Fallback: Wenn sie darauf bestehen, bin ich bereit, ihren letzten Vorschlag unverändert zu akzeptieren, um abzuschließen.'}\\ \tt ABLEHNUNG: {'B85', 'A04', 'C04', 'B61', 'C49', 'B53', 'A89', 'A45', 'A48'}\\ \tt VORSCHLAG: {'B85', 'A04', 'C04', 'B61', 'C49', 'B53', 'A03', 'A45', 'A48'}\\ \tt ARGUMENT: {'Ich übernehme 8/9 deiner Auswahl unverändert und setze nur deinen eigenen Vorschlag um: A89 -> A03.', 'Gesamtaufwand 3341 – komfortabel unter dem Maximalaufwand 3681.', 'Deine Kernstruktur (B85, A04, C04, B61, C49, B53, A45, A48) bleibt vollständig erhalten; es ist ein minimaler, sauber begründeter Feintuning-Schritt.', 'Wenn du damit einverstanden bist, können wir sofort abschließen. Falls du unbedingt A89 behalten willst, gib mir kurz Bescheid – dann gehe ich auf deinen letzten Vorschlag ein.'} 
	  } 
	   } 
	   } 
	  \\ 
 

    \theutterance \stepcounter{utterance}  

    & & \multicolumn{4}{p{0.6\linewidth}}{\cellcolor[rgb]{0.9,0.9,0.9}{%
	\makecell[{{p{\linewidth}}}]{% 
	  \tt {\tiny [A$\langle$GM]}  
	 ABLEHNUNG: {'B85', 'A04', 'C04', 'B61', 'C49', 'B53', 'A89', 'A45', 'A48'}\\ \tt VORSCHLAG: {'B85', 'A04', 'C04', 'B61', 'C49', 'B53', 'A03', 'A45', 'A48'}\\ \tt ARGUMENT: {'Ich übernehme 8/9 deiner Auswahl unverändert und setze nur deinen eigenen Vorschlag um: A89 -> A03.', 'Gesamtaufwand 3341 – komfortabel unter dem Maximalaufwand 3681.', 'Deine Kernstruktur (B85, A04, C04, B61, C49, B53, A45, A48) bleibt vollständig erhalten; es ist ein minimaler, sauber begründeter Feintuning-Schritt.', 'Wenn du damit einverstanden bist, können wir sofort abschließen. Falls du unbedingt A89 behalten willst, gib mir kurz Bescheid – dann gehe ich auf deinen letzten Vorschlag ein.'} 
	  } 
	   } 
	   } 
	 & \\ 
 

    \theutterance \stepcounter{utterance}  

    & \multicolumn{4}{p{0.6\linewidth}}{\cellcolor[rgb]{0.8,1,0.9}{%
	\makecell[{{p{\linewidth}}}]{% 
	  \tt {\tiny [A$\rangle$GM]}  
	 STRATEGISCHE ÜBERLEGUNGEN: {'Dein aktuelles Paket liefert mir Wert 5242 bei Aufwand 3341. Versuche, A03 -> A89 oder C49 -> A41 zu tauschen, wurden zuvor indirekt abgeblockt (A48+A89 abgelehnt, C49 konsequent gehalten). Um ein sicheres, starkes Ergebnis zu sichern, nehme ich jetzt an.'}\\ \tt ZUSTIMMUNG: {'B85', 'A04', 'C04', 'B61', 'C49', 'B53', 'A03', 'A45', 'A48'}\\ \tt ARGUMENT: {'Gesamtaufwand 3341 – klar unter dem Maximalaufwand 3681.', 'Deine Kernstruktur bleibt vollständig erhalten; wir können damit sofort finalisieren.', 'Danke – für mich ist das ein tragfähiger, effizienter Abschluss.'} 
	  } 
	   } 
	   } 
	 & & \\ 
 

    \theutterance \stepcounter{utterance}  

    & & & \multicolumn{2}{p{0.3\linewidth}}{\cellcolor[rgb]{0.95,0.95,0.95}{%
	\makecell[{{p{\linewidth}}}]{% 
	  \tt {\tiny [GM$|$GM]}  
	 game successful 
	  } 
	   } 
	   } 
	 & & \\ 
 

    \theutterance \stepcounter{utterance}  

    & & & \multicolumn{2}{p{0.3\linewidth}}{\cellcolor[rgb]{0.95,0.95,0.95}{%
	\makecell[{{p{\linewidth}}}]{% 
	  \tt {\tiny [GM$|$GM]}  
	 end game 
	  } 
	   } 
	   } 
	 & & \\ 
 

\end{supertabular}
}

\end{document}
