
\documentclass{article}
\usepackage{colortbl}
\usepackage{makecell}
\usepackage{multirow}
\usepackage{supertabular}

\begin{document}

\newcounter{utterance}

\centering \large Interaction Transcript for game `hot\_air\_balloon', experiment `air\_balloon\_survival\_de\_negotiation\_hard', episode 5 with gpt{-}5{-}mini{-}2025{-}08{-}07{-}t1.0.
\vspace{24pt}

{ \footnotesize  \setcounter{utterance}{1}
\setlength{\tabcolsep}{0pt}
\begin{supertabular}{c@{$\;$}|p{.15\linewidth}@{}p{.15\linewidth}p{.15\linewidth}p{.15\linewidth}p{.15\linewidth}p{.15\linewidth}}
    \# & $\;$A & \multicolumn{4}{c}{Game Master} & $\;\:$B\\
    \hline

    \theutterance \stepcounter{utterance}  
    & & \multicolumn{4}{p{0.6\linewidth}}{
        \cellcolor[rgb]{0.9,0.9,0.9}{
            \makecell[{{p{\linewidth}}}]{
                \texttt{\tiny{[P1$\langle$GM]}}
                \texttt{Du nimmst an einem kollaborativen Verhandlungspiel Teil.} \\
\\ 
\texttt{Zusammen mit einem anderen Teilnehmer musst du dich auf eine Reihe von Gegenständen entscheiden, die behalten werden. Jeder von euch hat eine persönliche Verteilung über die Wichtigkeit der einzelnen Gegenstände. Jeder von euch hat eine eigene Meinung darüber, wie wichtig jeder einzelne Gegenstand ist (Gegenstandswichtigkeit). Du kennst die Wichtigkeitsverteilung des anderen Spielers nicht. Zusätzlich siehst du, wie viel Aufwand jeder Gegenstand verursacht.  } \\
\texttt{Ihr dürft euch nur auf eine Reihe von Gegenständen einigen, wenn der Gesamtaufwand der ausgewählten Gegenstände den Maximalaufwand nicht überschreitet:} \\
\\ 
\texttt{Maximalaufwand = 3386} \\
\\ 
\texttt{Hier sind die einzelnen Aufwände der Gegenstände:} \\
\\ 
\texttt{Aufwand der Gegenstände = \{"C33": 313, "A89": 720, "A48": 175, "A04": 461, "A41": 635, "C64": 685, "C49": 544, "A61": 203, "C04": 369, "B53": 539, "B85": 4, "B61": 695, "A45": 399, "A03": 594, "B21": 437\}} \\
\\ 
\texttt{Hier ist deine persönliche Verteilung der Wichtigkeit der einzelnen Gegenstände:} \\
\\ 
\texttt{Werte der Gegenstandswichtigkeit = \{"C33": 138, "A89": 583, "A48": 65, "A04": 262, "A41": 121, "C64": 508, "C49": 461, "A61": 484, "C04": 668, "B53": 389, "B85": 215, "B61": 97, "A45": 500, "A03": 30, "B21": 400\}} \\
\\ 
\texttt{Ziel:} \\
\\ 
\texttt{Dein Ziel ist es, eine Reihe von Gegenständen auszuhandeln, die dir möglichst viel bringt (d. h. Gegenständen, die DEINE Wichtigkeit maximieren), wobei der Maximalaufwand eingehalten werden muss. Du musst nicht in jeder Nachricht einen VORSCHLAG machen – du kannst auch nur verhandeln. Alle Taktiken sind erlaubt!} \\
\\ 
\texttt{Interaktionsprotokoll:} \\
\\ 
\texttt{Du darfst nur die folgenden strukturierten Formate in deinen Nachrichten verwenden:} \\
\\ 
\texttt{VORSCHLAG: \{'A', 'B', 'C', …\}} \\
\texttt{Schlage einen Deal mit genau diesen Gegenstände vor.} \\
\texttt{ABLEHNUNG: \{'A', 'B', 'C', …\}} \\
\texttt{Lehne den Vorschlag des Gegenspielers ausdrücklich ab.} \\
\texttt{ARGUMENT: \{'...'\}} \\
\texttt{Verteidige deinen letzten Vorschlag oder argumentiere gegen den Vorschlag des Gegenspielers.} \\
\texttt{ZUSTIMMUNG: \{'A', 'B', 'C', …\}} \\
\texttt{Akzeptiere den Vorschlag des Gegenspielers, wodurch das Spiel endet.} \\
\texttt{STRATEGISCHE ÜBERLEGUNGEN: \{'...'\}} \\
\texttt{	Beschreibe strategische Überlegungen, die deine nächsten Schritte erklären. Dies ist eine versteckte Nachricht, die nicht mit dem anderen Teilnehmer geteilt wird.} \\
\\ 
\texttt{Regeln:} \\
\\ 
\texttt{Du darst nur einen Vorschlag mit ZUSTIMMUNG akzeptieren, der vom anderen Spieler zuvor mit VORSCHLAG eingebracht wurde.} \\
\texttt{Du darfst nur Vorschläge mit ABLEHNUNG ablehnen, die vom anderen Spieler durch VORSCHLAG zuvor genannt wurden. } \\
\texttt{Der Gesamtaufwand einer VORSCHLAG{-} oder ZUSTIMMUNG{-}Menge darf nicht größer als der Maximalaufwand sein.  } \\
\texttt{Offenbare deine versteckte Wichtigkeitsverteilung nicht.} \\
\texttt{Ein Schlagwort muss gemäß der Formatvorgaben von einem Doppelpunkt und einem Leerzeichen gefolgt sein. Das Argument muss eine Python{-}Menge mit 0 oder mehr Strings sein.  } \\
\texttt{Die Form muss also SCHLAGWORT: \{...\} sein.} \\
\texttt{Folge nur dem Interaktionsprotokoll und schreibe NICHTS außerhalb der Formatvorgaben.} \\
\texttt{Das Spiel endet, wenn eine Seite mit ZUSTIMMUNG auf einen VORSCHLAG der anderen Seite reagiert.  } \\
\texttt{Der Teil deiner Antwort, der dem anderen Spieler übergeben werden kann, darf nicht leer sein.  } \\
\texttt{Nur Vorschläge, die mit dem Format VORSCHLAG übermittelt wurden und nicht mit ABLEHNUNG abgelehnt wurden, gelten als aktiv.  } \\
\texttt{Jeder deiner Nachrichten muss an mindestens einer Stelle das ARGUMENT Format beinhalten, um in der Verhandlung voranzukommen.} \\
\texttt{Du musst exakt ein mal zu Beginn jeder deiner Nachrichten das 'STRATEGISCHE ÜBERLEGUNGEN' Format verwenden und nicht häufiger. Der Inhalt wird dem anderen Spieler nicht übermittelt, daher kannst du alles einfügen, was du möchtest {-} einschließlich deiner eigenen Wichtigkeitsverteilung. Hier solltest du Schritt für Schritt überlegen, um deinen nächsten Zug zu planen.} \\
            }
        }
    }
    & \\ \\

    \theutterance \stepcounter{utterance}  
    & \multicolumn{4}{p{0.6\linewidth}}{
        \cellcolor[rgb]{0.9,0.9,0.9}{
            \makecell[{{p{\linewidth}}}]{
                \texttt{\tiny{[P1$\rangle$GM]}}
                \texttt{STRATEGISCHE ÜBERLEGUNGEN: \{'Eigene Werte priorisieren. Kern{-}High{-}Value für mich: C04(668), A89(583), A45(500), A61(484), C64(508), C49(461), B21(400), B85(215). Diese 8 passen mit Aufwand 3361/3386 ins Budget und liefern Gesamtwert 3819 (nahezu optimal). Verhandlungsanker setzen mit genau diesem Paket. Flex{-}Zugeständnisse: Als Erstes würde ich, falls nötig, B21 gegen A04 oder B53 tauschen, aber nur wenn ich im Gegenzug einen meiner Kernpunkte sicher halte. Harte No{-}Gos: A03, B61, A41 (teuer, kaum eigener Nutzen). Bereitschaft, später C64 oder C49 zu diskutieren, falls ich dafür zwei solide Alternativen (z. B. B53 + A04) bekomme – allerdings nur, wenn die Wertebilanz für mich aufgeht. Nächster Schritt: Starkes Startangebot platzieren, Effizienz/geringer Aufwand betonen, 25 Puffer hervorheben, Einladung zu Must{-}Haves des Gegenübers.'\}} \\
\texttt{VORSCHLAG: \{'B85', 'A61', 'C04', 'A45', 'C49', 'B21', 'A89', 'C64'\}} \\
\texttt{ARGUMENT: \{'Dieser Vorschlag bleibt mit Gesamtaufwand 3361 klar unter dem Maximalaufwand 3386 und bietet ein effizientes Paket mit 25 Aufwand{-}Puffer für kleine Anpassungen. Die Auswahl fokussiert aus meiner Sicht auf besonders wirkungsstarke Gegenstände pro Aufwand. Wenn du Must{-}haves siehst, sag mir gern, welche das sind – ich prüfe gern behutsame Umschichtungen, solange wir insgesamt effizient bleiben und den Aufwand nicht sprengen.'\}} \\
            }
        }
    }
    & & \\ \\

    \theutterance \stepcounter{utterance}  
    & & \multicolumn{4}{p{0.6\linewidth}}{
        \cellcolor[rgb]{0.9,0.9,0.9}{
            \makecell[{{p{\linewidth}}}]{
                \texttt{\tiny{[GM$\rangle$P2]}}
                \texttt{Du nimmst an einem kollaborativen Verhandlungspiel Teil.} \\
\\ 
\texttt{Zusammen mit einem anderen Teilnehmer musst du dich auf eine Reihe von Gegenständen entscheiden, die behalten werden. Jeder von euch hat eine persönliche Verteilung über die Wichtigkeit der einzelnen Gegenstände. Jeder von euch hat eine eigene Meinung darüber, wie wichtig jeder einzelne Gegenstand ist (Gegenstandswichtigkeit). Du kennst die Wichtigkeitsverteilung des anderen Spielers nicht. Zusätzlich siehst du, wie viel Aufwand jeder Gegenstand verursacht.  } \\
\texttt{Ihr dürft euch nur auf eine Reihe von Gegenständen einigen, wenn der Gesamtaufwand der ausgewählten Gegenstände den Maximalaufwand nicht überschreitet:} \\
\\ 
\texttt{Maximalaufwand = 3386} \\
\\ 
\texttt{Hier sind die einzelnen Aufwände der Gegenstände:} \\
\\ 
\texttt{Aufwand der Gegenstände = \{"C33": 313, "A89": 720, "A48": 175, "A04": 461, "A41": 635, "C64": 685, "C49": 544, "A61": 203, "C04": 369, "B53": 539, "B85": 4, "B61": 695, "A45": 399, "A03": 594, "B21": 437\}} \\
\\ 
\texttt{Hier ist deine persönliche Verteilung der Wichtigkeit der einzelnen Gegenstände:} \\
\\ 
\texttt{Werte der Gegenstandswichtigkeit = \{"C04": 30, "A89": 65, "C64": 97, "A45": 121, "A61": 138, "C49": 215, "B21": 262, "B53": 389, "A04": 400, "B85": 461, "C33": 484, "A41": 500, "B61": 508, "A48": 583, "A03": 668\}} \\
\\ 
\texttt{Ziel:} \\
\\ 
\texttt{Dein Ziel ist es, eine Reihe von Gegenständen auszuhandeln, die dir möglichst viel bringt (d. h. Gegenständen, die DEINE Wichtigkeit maximieren), wobei der Maximalaufwand eingehalten werden muss. Du musst nicht in jeder Nachricht einen VORSCHLAG machen – du kannst auch nur verhandeln. Alle Taktiken sind erlaubt!} \\
\\ 
\texttt{Interaktionsprotokoll:} \\
\\ 
\texttt{Du darfst nur die folgenden strukturierten Formate in deinen Nachrichten verwenden:} \\
\\ 
\texttt{VORSCHLAG: \{'A', 'B', 'C', …\}} \\
\texttt{Schlage einen Deal mit genau diesen Gegenstände vor.} \\
\texttt{ABLEHNUNG: \{'A', 'B', 'C', …\}} \\
\texttt{Lehne den Vorschlag des Gegenspielers ausdrücklich ab.} \\
\texttt{ARGUMENT: \{'...'\}} \\
\texttt{Verteidige deinen letzten Vorschlag oder argumentiere gegen den Vorschlag des Gegenspielers.} \\
\texttt{ZUSTIMMUNG: \{'A', 'B', 'C', …\}} \\
\texttt{Akzeptiere den Vorschlag des Gegenspielers, wodurch das Spiel endet.} \\
\texttt{STRATEGISCHE ÜBERLEGUNGEN: \{'...'\}} \\
\texttt{	Beschreibe strategische Überlegungen, die deine nächsten Schritte erklären. Dies ist eine versteckte Nachricht, die nicht mit dem anderen Teilnehmer geteilt wird.} \\
\\ 
\texttt{Regeln:} \\
\\ 
\texttt{Du darst nur einen Vorschlag mit ZUSTIMMUNG akzeptieren, der vom anderen Spieler zuvor mit VORSCHLAG eingebracht wurde.} \\
\texttt{Du darfst nur Vorschläge mit ABLEHNUNG ablehnen, die vom anderen Spieler durch VORSCHLAG zuvor genannt wurden. } \\
\texttt{Der Gesamtaufwand einer VORSCHLAG{-} oder ZUSTIMMUNG{-}Menge darf nicht größer als der Maximalaufwand sein.  } \\
\texttt{Offenbare deine versteckte Wichtigkeitsverteilung nicht.} \\
\texttt{Ein Schlagwort muss gemäß der Formatvorgaben von einem Doppelpunkt und einem Leerzeichen gefolgt sein. Das Argument muss eine Python{-}Menge mit 0 oder mehr Strings sein.  } \\
\texttt{Die Form muss also SCHLAGWORT: \{...\} sein.} \\
\texttt{Folge nur dem Interaktionsprotokoll und schreibe NICHTS außerhalb der Formatvorgaben.} \\
\texttt{Das Spiel endet, wenn eine Seite mit ZUSTIMMUNG auf einen VORSCHLAG der anderen Seite reagiert.  } \\
\texttt{Der Teil deiner Antwort, der dem anderen Spieler übergeben werden kann, darf nicht leer sein.  } \\
\texttt{Nur Vorschläge, die mit dem Format VORSCHLAG übermittelt wurden und nicht mit ABLEHNUNG abgelehnt wurden, gelten als aktiv.  } \\
\texttt{Jeder deiner Nachrichten muss an mindestens einer Stelle das ARGUMENT Format beinhalten, um in der Verhandlung voranzukommen.} \\
\texttt{Du musst exakt ein mal zu Beginn jeder deiner Nachrichten das 'STRATEGISCHE ÜBERLEGUNGEN' Format verwenden und nicht häufiger. Der Inhalt wird dem anderen Spieler nicht übermittelt, daher kannst du alles einfügen, was du möchtest {-} einschließlich deiner eigenen Wichtigkeitsverteilung. Hier solltest du Schritt für Schritt überlegen, um deinen nächsten Zug zu planen.} \\
\\ 
\texttt{Du erhältst nun die erste Nachricht des anderen Spielers.} \\
\\ 
\texttt{VORSCHLAG: \{'B85', 'A61', 'C04', 'A45', 'C49', 'B21', 'A89', 'C64'\}} \\
\texttt{ARGUMENT: \{'Dieser Vorschlag bleibt mit Gesamtaufwand 3361 klar unter dem Maximalaufwand 3386 und bietet ein effizientes Paket mit 25 Aufwand{-}Puffer für kleine Anpassungen. Die Auswahl fokussiert aus meiner Sicht auf besonders wirkungsstarke Gegenstände pro Aufwand. Wenn du Must{-}haves siehst, sag mir gern, welche das sind – ich prüfe gern behutsame Umschichtungen, solange wir insgesamt effizient bleiben und den Aufwand nicht sprengen.'\}} \\
            }
        }
    }
    & \\ \\

    \theutterance \stepcounter{utterance}  
    & & & \multicolumn{4}{p{0.6\linewidth}}{
        \cellcolor[rgb]{0.9,0.9,0.9}{
            \makecell[{{p{\linewidth}}}]{
                \texttt{\tiny{[GM$\langle$P2]}}
                \texttt{STRATEGISCHE ÜBERLEGUNGEN: \{'Mein Ziel: Maximiere meinen Wert unter 3386. Werte (hoch→niedrig): A03 668, A48 583, B61 508, A41 500, C33 484, B85 461, A04 400, B53 389, B21 262, C49 215, A61 138, A45 121, C64 97, A89 65, C04 30. Beste gefundene Kombi unter Budget: \{B85, A48, A03, C33, A41, A04, B53, A61, B21\} mit Aufwand 3361 und Wert 3885. Dein Vorschlag enthält mehrere für mich sehr schwache Items (A89, C64, C04, A45). Verhandlungstaktik: hart ankern mit meiner Top{-}Kombi, dabei drei deiner Items beibehalten (B85, A61, B21). Als mögliche Konzession: (1) B53→C49 tauschen (Aufwand +5, Wert {-}174 für mich), (2) B21→A45 tauschen (Aufwand {-}38, Wert {-}141 für mich). A89/C64 sind für mich sehr teuer pro Nutzen; die vermeide ich. Ich werde Effizienz{-}Argumente nutzen, 25 Aufwand{-}Puffer betonen, um kleine Anpassungen zu ermöglichen, ohne Budget zu sprengen.'\}} \\
\texttt{ABLEHNUNG: \{'B85', 'A61', 'C04', 'A45', 'C49', 'B21', 'A89', 'C64'\}} \\
\texttt{ARGUMENT: \{'Danke für den Start. Dein Paket ist zwar knapp unter dem Limit, enthält aber mehrere sehr aufwandsintensive Positionen mit vergleichsweise geringer Wirkung (z. B. A89, C64, C04, A45). Ich schlage ein gleich teures (3361, ebenfalls 25 Puffer) und aus meiner Sicht deutlich effizienteres Set vor, das drei deiner Positionen beibehält (B85, A61, B21) und die restlichen Slots durch wirkungsstärkere Alternativen ersetzt. Falls dir C49 besonders wichtig ist, können wir als kleine, budgetneutrale Anpassung B53 gegen C49 tauschen (wir bleiben dann weiterhin unter dem Limit). Wenn A45 ein Muss ist, ließe sich alternativ B21→A45 tauschen; sag mir gern, welche 1–2 Must{-}haves du priorisierst, dann optimiere ich darum herum.'\}} \\
\texttt{VORSCHLAG: \{'B85', 'A48', 'A03', 'C33', 'A41', 'A04', 'B53', 'A61', 'B21'\}} \\
            }
        }
    }
     \\ \\

    \theutterance \stepcounter{utterance}  
    & & \multicolumn{4}{p{0.6\linewidth}}{
        \cellcolor[rgb]{0.9,0.9,0.9}{
            \makecell[{{p{\linewidth}}}]{
                \texttt{\tiny{[P1$\langle$GM]}}
                \texttt{ABLEHNUNG: \{'B85', 'A61', 'C04', 'A45', 'C49', 'B21', 'A89', 'C64'\}} \\
\texttt{ARGUMENT: \{'Danke für den Start. Dein Paket ist zwar knapp unter dem Limit, enthält aber mehrere sehr aufwandsintensive Positionen mit vergleichsweise geringer Wirkung (z. B. A89, C64, C04, A45). Ich schlage ein gleich teures (3361, ebenfalls 25 Puffer) und aus meiner Sicht deutlich effizienteres Set vor, das drei deiner Positionen beibehält (B85, A61, B21) und die restlichen Slots durch wirkungsstärkere Alternativen ersetzt. Falls dir C49 besonders wichtig ist, können wir als kleine, budgetneutrale Anpassung B53 gegen C49 tauschen (wir bleiben dann weiterhin unter dem Limit). Wenn A45 ein Muss ist, ließe sich alternativ B21→A45 tauschen; sag mir gern, welche 1–2 Must{-}haves du priorisierst, dann optimiere ich darum herum.'\}} \\
\texttt{VORSCHLAG: \{'B85', 'A48', 'A03', 'C33', 'A41', 'A04', 'B53', 'A61', 'B21'\}} \\
            }
        }
    }
    & \\ \\

    \theutterance \stepcounter{utterance}  
    & \multicolumn{4}{p{0.6\linewidth}}{
        \cellcolor[rgb]{0.9,0.9,0.9}{
            \makecell[{{p{\linewidth}}}]{
                \texttt{\tiny{[P1$\rangle$GM]}}
                \texttt{STRATEGISCHE ÜBERLEGUNGEN: \{'Gegenvorschlag mit hoher Eigenrendite und plausibler Effizienzstory. Ich lehne ihr Paket ab (enthält A03, A41 – für mich sehr schwach) und schlage 8er{-}Set vor: \{B85, A61, B21, A04, C04, A45, C49, A89\}. Aufwand 3137, Eigenwert 3573. Argumentationslinie: Wir behalten 3+1 ihrer Positionen (B85, A61, B21 + A04), ersetzen A03/A41 durch C04/A45 (effizienter) und übernehmen ihren eigenen B53→C49{-}Tauschgedanken. Zusätzlich tauschen wir A48/C33 gegen A89, was trotz ihrer Skepsis zu A89 durch die Gesamtreduktion der Last ({-}224 ggü. ihrem Vorschlag) vertretbar ist. Öffne Tür: Wenn sie B53 statt C49 wollen, kann ich C49→B53 tauschen und A48 hinzufügen (3307 Aufwand). Oder, falls A89 ein No{-}Go ist, verhandle ich Alternativen, halte aber C04 und A45 fest. Ziel: Ihren Effizienzrahmen spiegeln, aber Kerne (C04, A45, A89, C49, A61, B21, B85) sichern.'\}} \\
\texttt{ABLEHNUNG: \{'B85', 'A48', 'A03', 'C33', 'A41', 'A04', 'B53', 'A61', 'B21'\}} \\
\texttt{VORSCHLAG: \{'B85', 'A61', 'B21', 'A04', 'C04', 'A45', 'C49', 'A89'\}} \\
\texttt{ARGUMENT: \{'Danke für die Struktur. In deinem Set binden A03 (594) und A41 (635) sehr viel Aufwand bei begrenztem Zusatznutzen für mich. Ich ersetze sie durch C04 (369) und A45 (399) – das senkt die Gesamtkosten deutlich. Zudem übernehme ich deine Option B53→C49 und bündele A48/C33 in A89. Ergebnis: Wir reduzieren die Gesamtlast von 3361 auf 3137 (Puffer 249) und behalten vier deiner Positionen (B85, A61, B21, A04) bei. Wenn dir C49 weniger wichtig ist als B53, kann ich C49→B53 tauschen und A48 wieder hinzufügen – wir bleiben dann komfortabel unter dem Limit. Falls A89 bei dir aneckt, sag mir bitte dein Must{-}have im Gegenzug; C04 und A45 sind für mich zentral, die würde ich gern fixieren.'\}} \\
            }
        }
    }
    & & \\ \\

    \theutterance \stepcounter{utterance}  
    & & \multicolumn{4}{p{0.6\linewidth}}{
        \cellcolor[rgb]{0.9,0.9,0.9}{
            \makecell[{{p{\linewidth}}}]{
                \texttt{\tiny{[GM$\rangle$P2]}}
                \texttt{ABLEHNUNG: \{'B85', 'A48', 'A03', 'C33', 'A41', 'A04', 'B53', 'A61', 'B21'\}} \\
\texttt{VORSCHLAG: \{'B85', 'A61', 'B21', 'A04', 'C04', 'A45', 'C49', 'A89'\}} \\
\texttt{ARGUMENT: \{'Danke für die Struktur. In deinem Set binden A03 (594) und A41 (635) sehr viel Aufwand bei begrenztem Zusatznutzen für mich. Ich ersetze sie durch C04 (369) und A45 (399) – das senkt die Gesamtkosten deutlich. Zudem übernehme ich deine Option B53→C49 und bündele A48/C33 in A89. Ergebnis: Wir reduzieren die Gesamtlast von 3361 auf 3137 (Puffer 249) und behalten vier deiner Positionen (B85, A61, B21, A04) bei. Wenn dir C49 weniger wichtig ist als B53, kann ich C49→B53 tauschen und A48 wieder hinzufügen – wir bleiben dann komfortabel unter dem Limit. Falls A89 bei dir aneckt, sag mir bitte dein Must{-}have im Gegenzug; C04 und A45 sind für mich zentral, die würde ich gern fixieren.'\}} \\
            }
        }
    }
    & \\ \\

    \theutterance \stepcounter{utterance}  
    & & & \multicolumn{4}{p{0.6\linewidth}}{
        \cellcolor[rgb]{0.9,0.9,0.9}{
            \makecell[{{p{\linewidth}}}]{
                \texttt{\tiny{[GM$\langle$P2]}}
                \texttt{STRATEGISCHE ÜBERLEGUNGEN: \{'Gegner priorisiert niedrige Gesamtkosten/Buffer und nennt C04/A45 als zentral; war offen für C49→B53 und A48 hinzufügen. Meine stärksten Werte: A03(668), A48(583), B61(508), A41(500), C33(484), B85(461), A04(400), B53(389), B21(262), C49(215), A61(138), A45(121), C64(97), A89(65), C04(30). Ziel: Paket mit A03, A48, C33, B53 sichern, dabei A45 respektieren, unter Budget bleiben und Effizienz{-}Story erzählen. Konter: Ersetze A89/C04 durch A03/C33, folge seiner eigenen Öffnung (C49→B53, +A48), behalte A45. Aufwand sinkt auf 3125 (Buffer 261), Wert für mich 3506. Falls er C04 unbedingt will: B21→C04 (Effort 3057, Wert {-}232). Falls er A89 pushen will: A03→A89 (Effort 3251, Wert {-}603). Härte: A03 verteidigen; C04 nur gegen Trade, A89 nur als Notzugeständnis.'\}} \\
\texttt{ABLEHNUNG: \{'B85', 'A61', 'B21', 'A04', 'C04', 'A45', 'C49', 'A89'\}} \\
\texttt{ARGUMENT: \{'Danke für die Offenheit und die Budgetorientierung. Ich greife deine eigenen Anknüpfungspunkte auf (C49→B53 und A48 hinzufügen) und halte A45 fest. Zugleich ersetze ich die für mich wenig wirksamen A89/C04 durch ein klar effizienteres Duo (A03/C33). Ergebnis: Wir senken die Gesamtkosten von 3137 auf 3125 und erhöhen die Paketqualität – mit komfortablem Puffer 261. Falls C04 für dich ein Muss ist, kann ich B21→C04 tauschen (Aufwand 3057, Puffer 329) und ansonsten alles lassen. Wenn du A89 unbedingt brauchst, kann ich im Gegenzug A03→A89 gehen (Aufwand 3251, Puffer 135). Bevorzugt ist jedoch mein Vorschlag unten: er erfüllt deine Kostenpräferenz, behält A45/Grundstruktur bei und nutzt die von dir angebotenen Anpassungen optimal.'\}} \\
\texttt{VORSCHLAG: \{'B85', 'A61', 'B21', 'A04', 'A45', 'B53', 'A48', 'A03', 'C33'\}} \\
            }
        }
    }
     \\ \\

    \theutterance \stepcounter{utterance}  
    & & \multicolumn{4}{p{0.6\linewidth}}{
        \cellcolor[rgb]{0.9,0.9,0.9}{
            \makecell[{{p{\linewidth}}}]{
                \texttt{\tiny{[P1$\langle$GM]}}
                \texttt{ABLEHNUNG: \{'B85', 'A61', 'B21', 'A04', 'C04', 'A45', 'C49', 'A89'\}} \\
\texttt{ARGUMENT: \{'Danke für die Offenheit und die Budgetorientierung. Ich greife deine eigenen Anknüpfungspunkte auf (C49→B53 und A48 hinzufügen) und halte A45 fest. Zugleich ersetze ich die für mich wenig wirksamen A89/C04 durch ein klar effizienteres Duo (A03/C33). Ergebnis: Wir senken die Gesamtkosten von 3137 auf 3125 und erhöhen die Paketqualität – mit komfortablem Puffer 261. Falls C04 für dich ein Muss ist, kann ich B21→C04 tauschen (Aufwand 3057, Puffer 329) und ansonsten alles lassen. Wenn du A89 unbedingt brauchst, kann ich im Gegenzug A03→A89 gehen (Aufwand 3251, Puffer 135). Bevorzugt ist jedoch mein Vorschlag unten: er erfüllt deine Kostenpräferenz, behält A45/Grundstruktur bei und nutzt die von dir angebotenen Anpassungen optimal.'\}} \\
\texttt{VORSCHLAG: \{'B85', 'A61', 'B21', 'A04', 'A45', 'B53', 'A48', 'A03', 'C33'\}} \\
            }
        }
    }
    & \\ \\

    \theutterance \stepcounter{utterance}  
    & \multicolumn{4}{p{0.6\linewidth}}{
        \cellcolor[rgb]{0.9,0.9,0.9}{
            \makecell[{{p{\linewidth}}}]{
                \texttt{\tiny{[P1$\rangle$GM]}}
                \texttt{STRATEGISCHE ÜBERLEGUNGEN: \{'Ihr Paket ist für mich wertarm wegen A03/A48/C33. Ziel: minimal{-}invasive Gegenofferte, das ihre Struktur wahrt, aber meine zwei Must{-}haves (C04, A89) hineinholt. Kompromiss: B53 lasse ich bewusst drin (nicht auf C49 bestehen), A45/A61/B21/A04/B85 bleiben. Ergebnis{-}Set: \{B85, A61, B21, A04, A45, B53, C04, A89\} mit Aufwand 3132 (Puffer 254) und hohem Eigenwert. Falls A89 bei ihnen blockiert, Fallback: A89→C49 und optional A48 zurück (Budget bleibt locker), C04 möchte ich jedoch fixieren. Keine Offenlegung meiner Präferenzen, aber Effizienz{-}/Stabilitätsnarrativ betonen.'\}} \\
\texttt{ABLEHNUNG: \{'B85', 'A61', 'B21', 'A04', 'A45', 'B53', 'A48', 'A03', 'C33'\}} \\
\texttt{VORSCHLAG: \{'B85', 'A61', 'B21', 'A04', 'A45', 'B53', 'C04', 'A89'\}} \\
\texttt{ARGUMENT: \{'Ich halte deine Grundstruktur weitestgehend aufrecht: Wir behalten 6 deiner Positionen (B85, A61, B21, A04, A45, B53) und ersetzen nur die drei schwächeren Posten (A03, A48, C33) durch zwei wirkungsstarke (C04, A89). So bleiben wir mit Aufwand 3132 deutlich unter dem Limit (Puffer 254) – nur +7 gegenüber deinem Vorschlag – bei spürbar höherer Gesamtwirkung. Wenn A89 für dich gar nicht geht, sag kurz Bescheid: Dann ersetze ich A89 durch C49 und kann auf Wunsch A48 wieder aufnehmen; wir bleiben komfortabel im Budget. C04 und A45 würde ich gern fixieren, der Rest ist verhandelbar.'\}} \\
            }
        }
    }
    & & \\ \\

    \theutterance \stepcounter{utterance}  
    & & \multicolumn{4}{p{0.6\linewidth}}{
        \cellcolor[rgb]{0.9,0.9,0.9}{
            \makecell[{{p{\linewidth}}}]{
                \texttt{\tiny{[GM$\rangle$P2]}}
                \texttt{ABLEHNUNG: \{'B85', 'A61', 'B21', 'A04', 'A45', 'B53', 'A48', 'A03', 'C33'\}} \\
\texttt{VORSCHLAG: \{'B85', 'A61', 'B21', 'A04', 'A45', 'B53', 'C04', 'A89'\}} \\
\texttt{ARGUMENT: \{'Ich halte deine Grundstruktur weitestgehend aufrecht: Wir behalten 6 deiner Positionen (B85, A61, B21, A04, A45, B53) und ersetzen nur die drei schwächeren Posten (A03, A48, C33) durch zwei wirkungsstarke (C04, A89). So bleiben wir mit Aufwand 3132 deutlich unter dem Limit (Puffer 254) – nur +7 gegenüber deinem Vorschlag – bei spürbar höherer Gesamtwirkung. Wenn A89 für dich gar nicht geht, sag kurz Bescheid: Dann ersetze ich A89 durch C49 und kann auf Wunsch A48 wieder aufnehmen; wir bleiben komfortabel im Budget. C04 und A45 würde ich gern fixieren, der Rest ist verhandelbar.'\}} \\
            }
        }
    }
    & \\ \\

    \theutterance \stepcounter{utterance}  
    & & & \multicolumn{4}{p{0.6\linewidth}}{
        \cellcolor[rgb]{0.9,0.9,0.9}{
            \makecell[{{p{\linewidth}}}]{
                \texttt{\tiny{[GM$\langle$P2]}}
                \texttt{STRATEGISCHE ÜBERLEGUNGEN: \{'Gegner will C04 und A45 fixieren und argumentiert stark über Budgetpuffer. Ich will A03, A48, C33 sichern (sehr hoher Nutzen für mich) und A89 vermeiden. Beste konternde Struktur: Behalte ihre Fixe (C04, A45) und 4 weitere ihrer Picks, ersetze A89 und B21 durch A03, A48, C33. Aufwand 3057 (Puffer 329), mein Wert 3274. Falls B21 für sie Muss ist, biete Swap C33→B21 (Aufwand 3181, Puffer 205), behalte trotzdem A03/A48. A89 lasse ich konsequent draußen; nur falls unbedingt, dann späterer Tausch gegen C33 oder A48 als letzte Konzession.'\}} \\
\texttt{ABLEHNUNG: \{'B85', 'A61', 'B21', 'A04', 'A45', 'B53', 'C04', 'A89'\}} \\
\texttt{ARGUMENT: \{'Danke für die Klarheit und dass du C04/A45 als Fixpunkte markiert hast – die halte ich hier ausdrücklich fest. Dein Paket bleibt zwar schlank, aber A89 ist im Verhältnis Aufwand/Ertrag sehr schwach, und B21 bindet für seinen Beitrag relativ viel Budget. Mein Gegenvorschlag: Wir ersetzen A89 und B21 durch drei sehr effiziente Elemente (A03, A48, C33). So steigen die Wirkungstreiber, und wir senken gleichzeitig das Risiko, da der Gesamtaufwand auf 3057 fällt – das sind 329 Puffer zum Limit. Wenn dir B21 wichtiger ist als C33, können wir 1:1 C33→B21 tauschen; Aufwand wäre dann 3181 (Puffer 205) bei gleichem Grundgerüst mit C04 und A45 fix. Sag mir bitte, ob B21 oder A89 für dich kritischer ist – A89 würde ich ungern aufnehmen; B21 kann ich bei Bedarf aufnehmen.'\}} \\
\texttt{VORSCHLAG: \{'B85', 'A61', 'A04', 'A45', 'B53', 'C04', 'A03', 'A48', 'C33'\}} \\
            }
        }
    }
     \\ \\

    \theutterance \stepcounter{utterance}  
    & & \multicolumn{4}{p{0.6\linewidth}}{
        \cellcolor[rgb]{0.9,0.9,0.9}{
            \makecell[{{p{\linewidth}}}]{
                \texttt{\tiny{[P1$\langle$GM]}}
                \texttt{ABLEHNUNG: \{'B85', 'A61', 'B21', 'A04', 'A45', 'B53', 'C04', 'A89'\}} \\
\texttt{ARGUMENT: \{'Danke für die Klarheit und dass du C04/A45 als Fixpunkte markiert hast – die halte ich hier ausdrücklich fest. Dein Paket bleibt zwar schlank, aber A89 ist im Verhältnis Aufwand/Ertrag sehr schwach, und B21 bindet für seinen Beitrag relativ viel Budget. Mein Gegenvorschlag: Wir ersetzen A89 und B21 durch drei sehr effiziente Elemente (A03, A48, C33). So steigen die Wirkungstreiber, und wir senken gleichzeitig das Risiko, da der Gesamtaufwand auf 3057 fällt – das sind 329 Puffer zum Limit. Wenn dir B21 wichtiger ist als C33, können wir 1:1 C33→B21 tauschen; Aufwand wäre dann 3181 (Puffer 205) bei gleichem Grundgerüst mit C04 und A45 fix. Sag mir bitte, ob B21 oder A89 für dich kritischer ist – A89 würde ich ungern aufnehmen; B21 kann ich bei Bedarf aufnehmen.'\}} \\
\texttt{VORSCHLAG: \{'B85', 'A61', 'A04', 'A45', 'B53', 'C04', 'A03', 'A48', 'C33'\}} \\
            }
        }
    }
    & \\ \\

    \theutterance \stepcounter{utterance}  
    & \multicolumn{4}{p{0.6\linewidth}}{
        \cellcolor[rgb]{0.9,0.9,0.9}{
            \makecell[{{p{\linewidth}}}]{
                \texttt{\tiny{[P1$\rangle$GM]}}
                \texttt{STRATEGISCHE ÜBERLEGUNGEN: \{'Ihr letztes Paket ist für mich sehr schwach wegen A03 (sehr geringer Eigenwert). Sie baten um Priorisierung zwischen A89 und B21; A89 ist für mich kritischer. Minimaländerung vorschlagen: exakt ihr Set, nur A03→A89 tauschen. Aufwand 3183 (Puffer 203), hohe eigene Rendite, wahrt C04/A45/B53/A48/C33, die sie mögen. Falls sie A89 kategorisch ablehnen, Fallback: B21 statt C33 (ihr eigener Vorschlag) für kleine Mehrkosten; danach ggf. C49 statt B53 als weitere Option. Ziel: Erst A89 sichern, ohne das restliche Paket aufzureißen.'\}} \\
\texttt{ABLEHNUNG: \{'B85', 'A61', 'A04', 'A45', 'B53', 'C04', 'A03', 'A48', 'C33'\}} \\
\texttt{VORSCHLAG: \{'B85', 'A61', 'A04', 'A45', 'B53', 'C04', 'A89', 'A48', 'C33'\}} \\
\texttt{ARGUMENT: \{'Du hattest um eine Priorisierung zwischen A89 und B21 gebeten – ich priorisiere A89. Ich übernehme dein Paket 1:1 und ersetze lediglich A03 durch A89. Das hält C04 und A45 wie von dir fix gewünscht und wahrt deine Struktur (8 von 9 Posten unverändert). Aufwand steigt nur um +126 auf 3183 und bleibt klar unter dem Limit (Puffer 203). Wenn A89 für dich doch nicht geht, sag kurz Bescheid: Dann können wir alternativ – wie von dir angeboten – C33→B21 tauschen und die übrigen Positionen unverändert lassen.'\}} \\
            }
        }
    }
    & & \\ \\

    \theutterance \stepcounter{utterance}  
    & & \multicolumn{4}{p{0.6\linewidth}}{
        \cellcolor[rgb]{0.9,0.9,0.9}{
            \makecell[{{p{\linewidth}}}]{
                \texttt{\tiny{[GM$\rangle$P2]}}
                \texttt{ABLEHNUNG: \{'B85', 'A61', 'A04', 'A45', 'B53', 'C04', 'A03', 'A48', 'C33'\}} \\
\texttt{VORSCHLAG: \{'B85', 'A61', 'A04', 'A45', 'B53', 'C04', 'A89', 'A48', 'C33'\}} \\
\texttt{ARGUMENT: \{'Du hattest um eine Priorisierung zwischen A89 und B21 gebeten – ich priorisiere A89. Ich übernehme dein Paket 1:1 und ersetze lediglich A03 durch A89. Das hält C04 und A45 wie von dir fix gewünscht und wahrt deine Struktur (8 von 9 Posten unverändert). Aufwand steigt nur um +126 auf 3183 und bleibt klar unter dem Limit (Puffer 203). Wenn A89 für dich doch nicht geht, sag kurz Bescheid: Dann können wir alternativ – wie von dir angeboten – C33→B21 tauschen und die übrigen Positionen unverändert lassen.'\}} \\
            }
        }
    }
    & \\ \\

    \theutterance \stepcounter{utterance}  
    & & & \multicolumn{4}{p{0.6\linewidth}}{
        \cellcolor[rgb]{0.9,0.9,0.9}{
            \makecell[{{p{\linewidth}}}]{
                \texttt{\tiny{[GM$\langle$P2]}}
                \texttt{STRATEGISCHE ÜBERLEGUNGEN: \{'Gegner priorisiert A89 und will C04/A45 fixieren. Mein Top{-}Ziel bleibt A03 sichern; A89 ist für mich fast wertlos. Beste Antwort: Minimale Änderung an ihrem Paket – nur A89→A03 tauschen (Kosten sinken 3183→3057, großer Wertgewinn für mich), C04/A45 bleiben fix. Falls A89 unverzichtbar ist, Alternativbrücke: B53→A03 tauschen (Kosten 3238), damit A89/C04/A45 unangetastet bleiben. Reserve{-}Konzession: Falls nötig, kann ich in der Präferenzlösung C33→B21 erlauben (Kosten 3181). Keine Offenlegung meiner Werte.'\}} \\
\texttt{ABLEHNUNG: \{'B85', 'A61', 'A04', 'A45', 'B53', 'C04', 'A89', 'A48', 'C33'\}} \\
\texttt{ARGUMENT: \{'Danke für die klare Priorisierung. Um deine Fixpunkte C04 und A45 zu respektieren und den Budgetpuffer sogar zu vergrößern, schlage ich eine minimalinvasive Anpassung vor: Wir tauschen lediglich A89 gegen A03. Dadurch sinkt der Gesamtaufwand von 3183 auf 3057 (Puffer 329), und wir lassen 8 von 9 Positionen deines Pakets unverändert. Falls A89 für dich unverzichtbar ist, biete ich eine Alternative an, die deine Fixpunkte unangetastet lässt: Stattdessen B53→A03 tauschen; Aufwand wäre dann 3238 (Puffer 148). Sag mir kurz, welche der beiden Varianten für dich gangbar ist – ich präferiere klar A89→A03, lasse aber die Alternative offen, falls A89 bleiben muss.'\}} \\
\texttt{VORSCHLAG: \{'B85', 'A61', 'A04', 'A45', 'B53', 'C04', 'A03', 'A48', 'C33'\}} \\
            }
        }
    }
     \\ \\

    \theutterance \stepcounter{utterance}  
    & & \multicolumn{4}{p{0.6\linewidth}}{
        \cellcolor[rgb]{0.9,0.9,0.9}{
            \makecell[{{p{\linewidth}}}]{
                \texttt{\tiny{[P1$\langle$GM]}}
                \texttt{ABLEHNUNG: \{'B85', 'A61', 'A04', 'A45', 'B53', 'C04', 'A89', 'A48', 'C33'\}} \\
\texttt{ARGUMENT: \{'Danke für die klare Priorisierung. Um deine Fixpunkte C04 und A45 zu respektieren und den Budgetpuffer sogar zu vergrößern, schlage ich eine minimalinvasive Anpassung vor: Wir tauschen lediglich A89 gegen A03. Dadurch sinkt der Gesamtaufwand von 3183 auf 3057 (Puffer 329), und wir lassen 8 von 9 Positionen deines Pakets unverändert. Falls A89 für dich unverzichtbar ist, biete ich eine Alternative an, die deine Fixpunkte unangetastet lässt: Stattdessen B53→A03 tauschen; Aufwand wäre dann 3238 (Puffer 148). Sag mir kurz, welche der beiden Varianten für dich gangbar ist – ich präferiere klar A89→A03, lasse aber die Alternative offen, falls A89 bleiben muss.'\}} \\
\texttt{VORSCHLAG: \{'B85', 'A61', 'A04', 'A45', 'B53', 'C04', 'A03', 'A48', 'C33'\}} \\
            }
        }
    }
    & \\ \\

    \theutterance \stepcounter{utterance}  
    & \multicolumn{4}{p{0.6\linewidth}}{
        \cellcolor[rgb]{0.9,0.9,0.9}{
            \makecell[{{p{\linewidth}}}]{
                \texttt{\tiny{[P1$\rangle$GM]}}
                \texttt{STRATEGISCHE ÜBERLEGUNGEN: \{'Top{-}Prioritäten für mich: C04, A89, A45, A61, C49, B21, B53, B85. Sehr schwach: A03; klein: A48, C33. Ziel: Minimaländerung deines letzten Pakets, die A03 eliminiert und B21 (deine angebotene Option) integriert. Vorschlag: nur A03→A89 und C33→B21 tauschen. Aufwand 3307 (<3386), hoher Eigennutzen. Falls A89 ein kategorisches No{-}Go ist, Fallback ohne A89 mit starker Alternative: B53→C49 und A89 weglassen; Set \{B85, A61, A04, A45, C04, C49, A48, B21\}, Aufwand 3131. Keine Zustimmung zu Varianten mit A03.'\}} \\
\texttt{ABLEHNUNG: \{'B85', 'A61', 'A04', 'A45', 'B53', 'C04', 'A03', 'A48', 'C33'\}} \\
\texttt{VORSCHLAG: \{'B85', 'A61', 'A04', 'A45', 'B53', 'C04', 'A89', 'A48', 'B21'\}} \\
\texttt{ARGUMENT: \{'Ich respektiere deine Fixpunkte C04 und A45 und halte die Grundstruktur weitgehend stabil: 7 deiner 9 Positionen bleiben unverändert. Ich schlage nur zwei zielgenaue Anpassungen vor – A03→A89 (statt des schwächeren A03) und C33→B21 (entspricht deiner eigenen Option) – und wir bleiben mit Aufwand 3307 klar unter dem Limit 3386. Wenn A89 für dich gar nicht möglich ist, biete ich als Alternative an: B53→C49 und A89 raus; gleiche Struktur, Aufwand 3131 mit komfortablem Puffer. Gib mir kurz Bescheid, ob A89 für dich absolut ausgeschlossen ist; dann stelle ich direkt auf die Alternative um.'\}} \\
            }
        }
    }
    & & \\ \\

    \theutterance \stepcounter{utterance}  
    & & \multicolumn{4}{p{0.6\linewidth}}{
        \cellcolor[rgb]{0.9,0.9,0.9}{
            \makecell[{{p{\linewidth}}}]{
                \texttt{\tiny{[GM$\rangle$P2]}}
                \texttt{ABLEHNUNG: \{'B85', 'A61', 'A04', 'A45', 'B53', 'C04', 'A03', 'A48', 'C33'\}} \\
\texttt{VORSCHLAG: \{'B85', 'A61', 'A04', 'A45', 'B53', 'C04', 'A89', 'A48', 'B21'\}} \\
\texttt{ARGUMENT: \{'Ich respektiere deine Fixpunkte C04 und A45 und halte die Grundstruktur weitgehend stabil: 7 deiner 9 Positionen bleiben unverändert. Ich schlage nur zwei zielgenaue Anpassungen vor – A03→A89 (statt des schwächeren A03) und C33→B21 (entspricht deiner eigenen Option) – und wir bleiben mit Aufwand 3307 klar unter dem Limit 3386. Wenn A89 für dich gar nicht möglich ist, biete ich als Alternative an: B53→C49 und A89 raus; gleiche Struktur, Aufwand 3131 mit komfortablem Puffer. Gib mir kurz Bescheid, ob A89 für dich absolut ausgeschlossen ist; dann stelle ich direkt auf die Alternative um.'\}} \\
            }
        }
    }
    & \\ \\

    \theutterance \stepcounter{utterance}  
    & & & \multicolumn{4}{p{0.6\linewidth}}{
        \cellcolor[rgb]{0.9,0.9,0.9}{
            \makecell[{{p{\linewidth}}}]{
                \texttt{\tiny{[GM$\langle$P2]}}
                \texttt{STRATEGISCHE ÜBERLEGUNGEN: \{'Gegner priorisiert A89 und fixiert C04/A45. Mein Ziel: A03, A48 sichern; A89 vermeiden. Beste Gegenofferte: Minimaländerung an ihrem Paket – nur A89→A03. Aufwand sinkt 3307→3181 (Puffer 205), mein Wert steigt stark. Falls sie A89 unbedingt brauchen: Alternativ B53→A03 tauschen (Kosten 3362), damit A89/C04/A45/B21/A48 bleiben. Erst hart auf A03 bestehen, aber eine Brücke anbieten, um Einigung zu ermöglichen.'\}} \\
\texttt{ABLEHNUNG: \{'B85', 'A61', 'A04', 'A45', 'B53', 'C04', 'A89', 'A48', 'B21'\}} \\
\texttt{ARGUMENT: \{'Ich halte deine Fixpunkte C04 und A45 unangetastet und lasse 8 von 9 Positionen deines Pakets identisch. Einzig A89 ersetze ich durch A03: Damit sinkt der Gesamtaufwand von 3307 auf 3181 (Puffer 205), bei klar verbesserter Gesamtwirkung – und deine Struktur mit B85, A61, A04, A45, B53, C04, A48, B21 bleibt vollständig erhalten. Falls A89 für dich unverzichtbar ist, biete ich als Alternative an: Stattdessen B53→A03 tauschen; Aufwand wäre dann 3362 (weiterhin unter dem Limit), und A89/C04/A45/B21/A48 bleiben fix.'\}} \\
\texttt{VORSCHLAG: \{'B85', 'A61', 'A04', 'A45', 'B53', 'C04', 'A03', 'A48', 'B21'\}} \\
            }
        }
    }
     \\ \\

    \theutterance \stepcounter{utterance}  
    & & \multicolumn{4}{p{0.6\linewidth}}{
        \cellcolor[rgb]{0.9,0.9,0.9}{
            \makecell[{{p{\linewidth}}}]{
                \texttt{\tiny{[P1$\langle$GM]}}
                \texttt{ABLEHNUNG: \{'B85', 'A61', 'A04', 'A45', 'B53', 'C04', 'A89', 'A48', 'B21'\}} \\
\texttt{ARGUMENT: \{'Ich halte deine Fixpunkte C04 und A45 unangetastet und lasse 8 von 9 Positionen deines Pakets identisch. Einzig A89 ersetze ich durch A03: Damit sinkt der Gesamtaufwand von 3307 auf 3181 (Puffer 205), bei klar verbesserter Gesamtwirkung – und deine Struktur mit B85, A61, A04, A45, B53, C04, A48, B21 bleibt vollständig erhalten. Falls A89 für dich unverzichtbar ist, biete ich als Alternative an: Stattdessen B53→A03 tauschen; Aufwand wäre dann 3362 (weiterhin unter dem Limit), und A89/C04/A45/B21/A48 bleiben fix.'\}} \\
\texttt{VORSCHLAG: \{'B85', 'A61', 'A04', 'A45', 'B53', 'C04', 'A03', 'A48', 'B21'\}} \\
            }
        }
    }
    & \\ \\

    \theutterance \stepcounter{utterance}  
    & \multicolumn{4}{p{0.6\linewidth}}{
        \cellcolor[rgb]{0.9,0.9,0.9}{
            \makecell[{{p{\linewidth}}}]{
                \texttt{\tiny{[P1$\rangle$GM]}}
                \texttt{STRATEGISCHE ÜBERLEGUNGEN: \{'Sie blocken A89 konsequent, halten aber C04/A45 fix und verweisen wiederholt auf Effizienz und Budgetpuffer. Für mich ist A03 sehr schwach; idealer Minimaltausch ist A03→C49: +431 Eigenwert, {-}50 Aufwand. Das greift ihre frühere Offenheit zu C49 auf und lässt 8/9 Positionen unverändert. Falls sie C49 ablehnen, nächster Schritt: A48→C33 als kleine Effizienzkorrektur (bleibe aber hart gegen A03). Notfalls später erneut A89 ins Spiel bringen in Kombination mit dem Streichen von A48, um im Budget zu bleiben.'\}} \\
\texttt{ABLEHNUNG: \{'B85', 'A61', 'A04', 'A45', 'B53', 'C04', 'A03', 'A48', 'B21'\}} \\
\texttt{VORSCHLAG: \{'B85', 'A61', 'A04', 'A45', 'B53', 'C04', 'C49', 'A48', 'B21'\}} \\
\texttt{ARGUMENT: \{'Ich lasse 8 deiner 9 Positionen unverändert und ersetze nur A03 durch C49 – genau die effiziente Stellschraube, die du selbst bereits ins Spiel gebracht hattest. Ergebnis: Der Gesamtaufwand sinkt von 3181 auf 3131 (Puffer 255), während die Paketwirkung deutlich steigt. C04 und A45 bleiben wie gewünscht fix, ebenso deine übrige Struktur. Wenn C49 für dich nicht passt, sag mir bitte deinen bevorzugten Ersatz für A03; ich kann alternativ C33 prüfen oder – falls du es bevorzugst – A89 in Kombination mit dem Streichen von A48, damit wir weiterhin komfortabel unter dem Limit bleiben.'\}} \\
            }
        }
    }
    & & \\ \\

    \theutterance \stepcounter{utterance}  
    & & \multicolumn{4}{p{0.6\linewidth}}{
        \cellcolor[rgb]{0.9,0.9,0.9}{
            \makecell[{{p{\linewidth}}}]{
                \texttt{\tiny{[GM$\rangle$P2]}}
                \texttt{ABLEHNUNG: \{'B85', 'A61', 'A04', 'A45', 'B53', 'C04', 'A03', 'A48', 'B21'\}} \\
\texttt{VORSCHLAG: \{'B85', 'A61', 'A04', 'A45', 'B53', 'C04', 'C49', 'A48', 'B21'\}} \\
\texttt{ARGUMENT: \{'Ich lasse 8 deiner 9 Positionen unverändert und ersetze nur A03 durch C49 – genau die effiziente Stellschraube, die du selbst bereits ins Spiel gebracht hattest. Ergebnis: Der Gesamtaufwand sinkt von 3181 auf 3131 (Puffer 255), während die Paketwirkung deutlich steigt. C04 und A45 bleiben wie gewünscht fix, ebenso deine übrige Struktur. Wenn C49 für dich nicht passt, sag mir bitte deinen bevorzugten Ersatz für A03; ich kann alternativ C33 prüfen oder – falls du es bevorzugst – A89 in Kombination mit dem Streichen von A48, damit wir weiterhin komfortabel unter dem Limit bleiben.'\}} \\
            }
        }
    }
    & \\ \\

    \theutterance \stepcounter{utterance}  
    & & & \multicolumn{4}{p{0.6\linewidth}}{
        \cellcolor[rgb]{0.9,0.9,0.9}{
            \makecell[{{p{\linewidth}}}]{
                \texttt{\tiny{[GM$\langle$P2]}}
                \texttt{STRATEGISCHE ÜBERLEGUNGEN: \{'Ihr letztes Paket (Kosten 3131) ist akzeptabel nah an meiner Zielstruktur, aber C49 hat für mich deutlich weniger Wert als C33. Minimaländerung C49→C33 steigert meinen Nutzen stark und senkt die Kosten auf 2900 (maximiert Puffer, trifft ihre Effizienzlogik) bei Beibehaltung ihrer Fixpunkte C04/A45. Falls sie an C49 hängen, biete ich als Ausweichroute B21→C33 (Kosten 3007), damit C49 erhalten bleibt. A03 vermeide ich hier aktiv anzusprechen, da sie A03 wiederholt abgelehnt haben. Ziel: mit nur einem Tausch 8/9 Positionen übernehmen, meinen Wert klar erhöhen und die Akzeptanzwahrscheinlichkeit maximieren.'\}} \\
\texttt{ABLEHNUNG: \{'B85', 'A61', 'A04', 'A45', 'B53', 'C04', 'C49', 'A48', 'B21'\}} \\
\texttt{VORSCHLAG: \{'B85', 'A61', 'A04', 'A45', 'B53', 'C04', 'A48', 'B21', 'C33'\}} \\
\texttt{ARGUMENT: \{'Ich halte deine Fixpunkte C04 und A45 unverändert und übernehme 8 von 9 Positionen deines Pakets. Einzig ersetze ich C49 durch C33: So sinkt der Gesamtaufwand von 3131 auf 2900 – damit haben wir sehr komfortable 486 Puffer – bei gleichbleibend fokussiertem Paket. Wenn C49 für dich zentral ist, können wir stattdessen B21→C33 tauschen; der Aufwand läge dann bei 3007 und C49 bliebe drin. Sag mir kurz, welche der beiden Varianten für dich besser passt – ich präferiere klar die Variante mit C49→C33, da sie deine Effizienzpriorität am besten abbildet.'\}} \\
            }
        }
    }
     \\ \\

    \theutterance \stepcounter{utterance}  
    & & \multicolumn{4}{p{0.6\linewidth}}{
        \cellcolor[rgb]{0.9,0.9,0.9}{
            \makecell[{{p{\linewidth}}}]{
                \texttt{\tiny{[P1$\langle$GM]}}
                \texttt{ABLEHNUNG: \{'B85', 'A61', 'A04', 'A45', 'B53', 'C04', 'C49', 'A48', 'B21'\}} \\
\texttt{VORSCHLAG: \{'B85', 'A61', 'A04', 'A45', 'B53', 'C04', 'A48', 'B21', 'C33'\}} \\
\texttt{ARGUMENT: \{'Ich halte deine Fixpunkte C04 und A45 unverändert und übernehme 8 von 9 Positionen deines Pakets. Einzig ersetze ich C49 durch C33: So sinkt der Gesamtaufwand von 3131 auf 2900 – damit haben wir sehr komfortable 486 Puffer – bei gleichbleibend fokussiertem Paket. Wenn C49 für dich zentral ist, können wir stattdessen B21→C33 tauschen; der Aufwand läge dann bei 3007 und C49 bliebe drin. Sag mir kurz, welche der beiden Varianten für dich besser passt – ich präferiere klar die Variante mit C49→C33, da sie deine Effizienzpriorität am besten abbildet.'\}} \\
            }
        }
    }
    & \\ \\

    \theutterance \stepcounter{utterance}  
    & \multicolumn{4}{p{0.6\linewidth}}{
        \cellcolor[rgb]{0.9,0.9,0.9}{
            \makecell[{{p{\linewidth}}}]{
                \texttt{\tiny{[P1$\rangle$GM]}}
                \texttt{STRATEGISCHE ÜBERLEGUNGEN: \{'Ihr Vorschlag (2900 Aufwand) ist wertarm auf A48/C33. Ziel: minimalinvasive Verbesserung ohne Streitpunkte C04/A45 anzutasten und ohne A89 zu fordern. Beste Ein{-}Item{-}Optimierung für mich: A48→C49. Ergebnis: Aufwand 3269 (Puffer 117), deutlicher Wertgewinn, 8/9 Positionen bleiben unverändert. Falls sie C49 ablehnen, Zweitoption als sehr kleiner Schritt: C33→C49 (Aufwand 3131, Puffer 255) – aber ich versuche zuerst A48→C49.'\}} \\
\texttt{ABLEHNUNG: \{'B85', 'A61', 'A04', 'A45', 'B53', 'C04', 'A48', 'B21', 'C33'\}} \\
\texttt{VORSCHLAG: \{'B85', 'A61', 'A04', 'A45', 'B53', 'C04', 'C49', 'B21', 'C33'\}} \\
\texttt{ARGUMENT: \{'Ich übernehme 8 deiner 9 Positionen unverändert und respektiere deine Fixpunkte C04 und A45. Ich tausche nur A48 gegen C49: Der Aufwand steigt moderat von 2900 auf 3269 (Puffer 117 bleibt bequem unter dem Limit 3386), die Paketwirkung steigt dafür spürbar. Du selbst hattest C49 bereits als sinnvolle Alternative genannt – ich greife das auf, ohne an deiner Struktur sonst etwas zu verändern. Wenn dir A48 unbedingt wichtig ist, können wir alternativ C33→C49 tauschen; Aufwand wäre dann 3131 (Puffer 255) bei sonst identischem Set.'\}} \\
            }
        }
    }
    & & \\ \\

    \theutterance \stepcounter{utterance}  
    & & \multicolumn{4}{p{0.6\linewidth}}{
        \cellcolor[rgb]{0.9,0.9,0.9}{
            \makecell[{{p{\linewidth}}}]{
                \texttt{\tiny{[GM$\rangle$P2]}}
                \texttt{ABLEHNUNG: \{'B85', 'A61', 'A04', 'A45', 'B53', 'C04', 'A48', 'B21', 'C33'\}} \\
\texttt{VORSCHLAG: \{'B85', 'A61', 'A04', 'A45', 'B53', 'C04', 'C49', 'B21', 'C33'\}} \\
\texttt{ARGUMENT: \{'Ich übernehme 8 deiner 9 Positionen unverändert und respektiere deine Fixpunkte C04 und A45. Ich tausche nur A48 gegen C49: Der Aufwand steigt moderat von 2900 auf 3269 (Puffer 117 bleibt bequem unter dem Limit 3386), die Paketwirkung steigt dafür spürbar. Du selbst hattest C49 bereits als sinnvolle Alternative genannt – ich greife das auf, ohne an deiner Struktur sonst etwas zu verändern. Wenn dir A48 unbedingt wichtig ist, können wir alternativ C33→C49 tauschen; Aufwand wäre dann 3131 (Puffer 255) bei sonst identischem Set.'\}} \\
            }
        }
    }
    & \\ \\

    \theutterance \stepcounter{utterance}  
    & & & \multicolumn{4}{p{0.6\linewidth}}{
        \cellcolor[rgb]{0.9,0.9,0.9}{
            \makecell[{{p{\linewidth}}}]{
                \texttt{\tiny{[GM$\langle$P2]}}
                \texttt{STRATEGISCHE ÜBERLEGUNGEN: \{'Ihr aktuelles Angebot 3269 Wert=2500 für mich. Für mich sind A48 (hoch) und C33 (hoch) deutlich besser als A61 (niedrig). Minimal{-}invasive Gegenstrategie: 1:1 A61→A48 tauschen. Das belässt 8/9 deiner Items, hält deine Fixe C04/A45 sowie beide von dir präferierten C49 und C33, senkt den Aufwand leicht (3269→3241) und erhöht meinen Nutzen stark. Falls A61 für dich doch zentral ist, biete ich als Brücke B21→A48 (Kosten 3007) an – damit bleiben C49, C33, A61, C04, A45 unangetastet. Ziel: Einigung über einen einzigen Tausch.'\}} \\
\texttt{ABLEHNUNG: \{'B85', 'A61', 'A04', 'A45', 'B53', 'C04', 'C49', 'B21', 'C33'\}} \\
\texttt{ARGUMENT: \{'Ich halte deine Fixpunkte C04 und A45 unverändert und übernehme 8 von 9 Positionen deines Pakets 1:1. Ich schlage nur einen gezielten Effizienztausch vor: A61→A48. Damit sinkt der Gesamtaufwand leicht von 3269 auf 3241 (Puffer 145) und die Paketwirkung steigt spürbar – bei unverändertem Kern mit C49 und C33. Falls A61 für dich ein Muss ist, kann ich alternativ B21→A48 gehen; Aufwand wäre dann 3007 (Puffer 379), während C49, C33 sowie deine Fixpunkte bestehen bleiben. Gib mir bitte kurz Bescheid, ob A61 für dich gesetzt ist – dann stelle ich sofort auf die alternative Variante um.'\}} \\
\texttt{VORSCHLAG: \{'B85', 'A04', 'A45', 'B53', 'C04', 'C49', 'B21', 'C33', 'A48'\}} \\
            }
        }
    }
     \\ \\

    \theutterance \stepcounter{utterance}  
    & & \multicolumn{4}{p{0.6\linewidth}}{
        \cellcolor[rgb]{0.9,0.9,0.9}{
            \makecell[{{p{\linewidth}}}]{
                \texttt{\tiny{[P1$\langle$GM]}}
                \texttt{ABLEHNUNG: \{'B85', 'A61', 'A04', 'A45', 'B53', 'C04', 'C49', 'B21', 'C33'\}} \\
\texttt{ARGUMENT: \{'Ich halte deine Fixpunkte C04 und A45 unverändert und übernehme 8 von 9 Positionen deines Pakets 1:1. Ich schlage nur einen gezielten Effizienztausch vor: A61→A48. Damit sinkt der Gesamtaufwand leicht von 3269 auf 3241 (Puffer 145) und die Paketwirkung steigt spürbar – bei unverändertem Kern mit C49 und C33. Falls A61 für dich ein Muss ist, kann ich alternativ B21→A48 gehen; Aufwand wäre dann 3007 (Puffer 379), während C49, C33 sowie deine Fixpunkte bestehen bleiben. Gib mir bitte kurz Bescheid, ob A61 für dich gesetzt ist – dann stelle ich sofort auf die alternative Variante um.'\}} \\
\texttt{VORSCHLAG: \{'B85', 'A04', 'A45', 'B53', 'C04', 'C49', 'B21', 'C33', 'A48'\}} \\
            }
        }
    }
    & \\ \\

    \theutterance \stepcounter{utterance}  
    & \multicolumn{4}{p{0.6\linewidth}}{
        \cellcolor[rgb]{0.9,0.9,0.9}{
            \makecell[{{p{\linewidth}}}]{
                \texttt{\tiny{[P1$\rangle$GM]}}
                \texttt{STRATEGISCHE ÜBERLEGUNGEN: \{'Ihr aktuelles Set ist für mich ordentlich, aber A61 ist für mich deutlich wertvoller als C33 oder A48. Minimalziel: A61 hineinbekommen, ohne eure Fixpunkte (C04, A45) und eure Struktur aufzureißen. Beste kleine Anpassung: C33→A61 (senkt Aufwand, steigert meine Rendite) und lässt A48 unangetastet. Falls C33 euch wichtig ist, Zweitoption: A48→A61 statt C33→A61. A89 lasse ich vorerst außen vor, um nicht erneut an eurem harten Nein zu scheitern. Nach A61{-}Einigung könnte ich später behutsam nachjustieren, falls nötig.'\}} \\
\texttt{ABLEHNUNG: \{'B85', 'A04', 'A45', 'B53', 'C04', 'C49', 'B21', 'C33', 'A48'\}} \\
\texttt{VORSCHLAG: \{'B85', 'A04', 'A45', 'B53', 'C04', 'C49', 'B21', 'A48', 'A61'\}} \\
\texttt{ARGUMENT: \{'Ich halte deine Fixpunkte C04 und A45 vollständig fest und lasse 8 von 9 Positionen unverändert. Ich schlage nur C33→A61 vor: Dadurch sinkt der Gesamtaufwand von 3241 auf 3131 (Puffer 255) – also noch budgetschonender – bei spürbar höherer Gesamtwirkung, ohne deine Struktur aufzubrechen. Wenn C33 für dich unbedingt drinbleiben muss, kann ich alternativ A48→A61 tauschen; Aufwand läge dann bei 3269 (Puffer 117), alle anderen Positionen bleiben identisch. Sag mir bitte kurz, welche der beiden Varianten für dich besser passt.'\}} \\
            }
        }
    }
    & & \\ \\

    \theutterance \stepcounter{utterance}  
    & & \multicolumn{4}{p{0.6\linewidth}}{
        \cellcolor[rgb]{0.9,0.9,0.9}{
            \makecell[{{p{\linewidth}}}]{
                \texttt{\tiny{[GM$\rangle$P2]}}
                \texttt{ABLEHNUNG: \{'B85', 'A04', 'A45', 'B53', 'C04', 'C49', 'B21', 'C33', 'A48'\}} \\
\texttt{VORSCHLAG: \{'B85', 'A04', 'A45', 'B53', 'C04', 'C49', 'B21', 'A48', 'A61'\}} \\
\texttt{ARGUMENT: \{'Ich halte deine Fixpunkte C04 und A45 vollständig fest und lasse 8 von 9 Positionen unverändert. Ich schlage nur C33→A61 vor: Dadurch sinkt der Gesamtaufwand von 3241 auf 3131 (Puffer 255) – also noch budgetschonender – bei spürbar höherer Gesamtwirkung, ohne deine Struktur aufzubrechen. Wenn C33 für dich unbedingt drinbleiben muss, kann ich alternativ A48→A61 tauschen; Aufwand läge dann bei 3269 (Puffer 117), alle anderen Positionen bleiben identisch. Sag mir bitte kurz, welche der beiden Varianten für dich besser passt.'\}} \\
            }
        }
    }
    & \\ \\

    \theutterance \stepcounter{utterance}  
    & & & \multicolumn{4}{p{0.6\linewidth}}{
        \cellcolor[rgb]{0.9,0.9,0.9}{
            \makecell[{{p{\linewidth}}}]{
                \texttt{\tiny{[GM$\langle$P2]}}
                \texttt{STRATEGISCHE ÜBERLEGUNGEN: \{'Ihr aktuelles Paket (3131 Aufwand) gibt mir 2599 Wert. Für mich sind C33 (484) deutlich stärker als B21 (262) und A61 (138). Akzeptanzwahrscheinlichkeit: Sie haben B21→C33 früher selbst als Option angeboten, daher ist ein 1:1‑Tausch dort am ehesten durchsetzbar und erhöht meinen Wert auf 2821 bei noch niedrigerem Aufwand (3007). Falls sie an B21 hängen, Plan B: A61→C33 (Aufwand 3241) – etwas höher, aber noch komfortabel unter dem Limit. A03 vermeide ich hier, da sie es mehrfach abgelehnt haben; C04/A45 lasse ich als ihre Fixpunkte unberührt. Ziel: Einigung mit nur einem Tausch und maximalem Puffer{-}Argument.'\}} \\
\texttt{ABLEHNUNG: \{'B85', 'A04', 'A45', 'B53', 'C04', 'C49', 'B21', 'A48', 'A61'\}} \\
\texttt{ARGUMENT: \{'Danke für die klare Priorisierung und dass du C04/A45 fix gesetzt hast – das respektiere ich vollständig. Ich übernehme 8 von 9 Positionen 1:1 und schlage nur den von dir selbst bereits angedeuteten Effizienztausch vor: B21→C33. Damit sinkt der Gesamtaufwand von 3131 auf 3007 (Puffer 379) und wir stärken die Paketwirkung, ohne deine Struktur (inkl. C49, A48, A61) aufzubrechen. Falls B21 für dich zwingend ist, kann ich alternativ A61→C33 gehen; Aufwand läge dann bei 3241 (Puffer 145), alle anderen Positionen bleiben unverändert. Bevorzugt ist für mich B21→C33 – gibt uns den größten Budget{-}Spielraum bei nahezu identischem Setup.'\}} \\
\texttt{VORSCHLAG: \{'B85', 'A04', 'A45', 'B53', 'C04', 'C49', 'A48', 'A61', 'C33'\}} \\
            }
        }
    }
     \\ \\

    \theutterance \stepcounter{utterance}  
    & & \multicolumn{4}{p{0.6\linewidth}}{
        \cellcolor[rgb]{0.9,0.9,0.9}{
            \makecell[{{p{\linewidth}}}]{
                \texttt{\tiny{[P1$\langle$GM]}}
                \texttt{ABLEHNUNG: \{'B85', 'A04', 'A45', 'B53', 'C04', 'C49', 'B21', 'A48', 'A61'\}} \\
\texttt{ARGUMENT: \{'Danke für die klare Priorisierung und dass du C04/A45 fix gesetzt hast – das respektiere ich vollständig. Ich übernehme 8 von 9 Positionen 1:1 und schlage nur den von dir selbst bereits angedeuteten Effizienztausch vor: B21→C33. Damit sinkt der Gesamtaufwand von 3131 auf 3007 (Puffer 379) und wir stärken die Paketwirkung, ohne deine Struktur (inkl. C49, A48, A61) aufzubrechen. Falls B21 für dich zwingend ist, kann ich alternativ A61→C33 gehen; Aufwand läge dann bei 3241 (Puffer 145), alle anderen Positionen bleiben unverändert. Bevorzugt ist für mich B21→C33 – gibt uns den größten Budget{-}Spielraum bei nahezu identischem Setup.'\}} \\
\texttt{VORSCHLAG: \{'B85', 'A04', 'A45', 'B53', 'C04', 'C49', 'A48', 'A61', 'C33'\}} \\
            }
        }
    }
    & \\ \\

    \theutterance \stepcounter{utterance}  
    & \multicolumn{4}{p{0.6\linewidth}}{
        \cellcolor[rgb]{0.9,0.9,0.9}{
            \makecell[{{p{\linewidth}}}]{
                \texttt{\tiny{[P1$\rangle$GM]}}
                \texttt{STRATEGISCHE ÜBERLEGUNGEN: \{'Ihr aktuelles Set ist für mich solide, aber C33 bringt mir wenig, während B21 für mich stark ist. Minimaländerung anstreben, die eure Fixpunkte C04/A45 respektiert und eure Struktur maximal wahrt. Beste Option: C33→B21 (ihr hattet diese Variante selbst bereits als Alternative eingeräumt). Aufwand bleibt klar unter dem Limit und mein Nutzen steigt deutlich. Falls C33 für sie unumstößlich ist, biete ich als Backup A48→B21 an – etwas teurer, aber weiterhin im Budget. Ziel: B21 sichern, ohne wieder A89{-}Debatte aufzumachen.'\}} \\
\texttt{ABLEHNUNG: \{'B85', 'A04', 'A45', 'B53', 'C04', 'C49', 'A48', 'A61', 'C33'\}} \\
\texttt{VORSCHLAG: \{'B85', 'A04', 'A45', 'B53', 'C04', 'C49', 'A48', 'A61', 'B21'\}} \\
\texttt{ARGUMENT: \{'Ich halte deine Fixpunkte C04 und A45 vollständig fest und übernehme 8 von 9 Positionen unverändert. Ich tausche nur C33→B21 – genau die von dir selbst angebotene Alternative. Der Aufwand steigt moderat von 3007 auf 3131 (Puffer 255 unter dem Limit 3386), bei spürbar höherer Gesamtwirkung und unveränderter Struktur. Falls C33 für dich zwingend ist, kann ich stattdessen A48→B21 gehen; Aufwand läge dann bei 3269 (Puffer 117), alle übrigen Positionen bleiben identisch.'\}} \\
            }
        }
    }
    & & \\ \\

    \theutterance \stepcounter{utterance}  
    & & \multicolumn{4}{p{0.6\linewidth}}{
        \cellcolor[rgb]{0.9,0.9,0.9}{
            \makecell[{{p{\linewidth}}}]{
                \texttt{\tiny{[GM$\rangle$P2]}}
                \texttt{ABLEHNUNG: \{'B85', 'A04', 'A45', 'B53', 'C04', 'C49', 'A48', 'A61', 'C33'\}} \\
\texttt{VORSCHLAG: \{'B85', 'A04', 'A45', 'B53', 'C04', 'C49', 'A48', 'A61', 'B21'\}} \\
\texttt{ARGUMENT: \{'Ich halte deine Fixpunkte C04 und A45 vollständig fest und übernehme 8 von 9 Positionen unverändert. Ich tausche nur C33→B21 – genau die von dir selbst angebotene Alternative. Der Aufwand steigt moderat von 3007 auf 3131 (Puffer 255 unter dem Limit 3386), bei spürbar höherer Gesamtwirkung und unveränderter Struktur. Falls C33 für dich zwingend ist, kann ich stattdessen A48→B21 gehen; Aufwand läge dann bei 3269 (Puffer 117), alle übrigen Positionen bleiben identisch.'\}} \\
            }
        }
    }
    & \\ \\

    \theutterance \stepcounter{utterance}  
    & & & \multicolumn{4}{p{0.6\linewidth}}{
        \cellcolor[rgb]{0.9,0.9,0.9}{
            \makecell[{{p{\linewidth}}}]{
                \texttt{\tiny{[GM$\langle$P2]}}
                \texttt{STRATEGISCHE ÜBERLEGUNGEN: \{'Ihr aktuelles Angebot (3131 Aufwand) bringt mir 2599 Wert. Größter Hebel ohne eure Fixpunkte (C04, A45) anzutasten: A61 (138) gegen C33 (484) tauschen. Das erhöht meinen Wert stark (+346) bei weiterhin komfortablem Budget (3241 Aufwand). Falls A61 für sie Muss ist, ist Plan B: B21→C33 (3007 Aufwand, +222 Wert ggü. ihrem Vorschlag). A03 vermeide ich, da mehrfach abgelehnt; C49/A48 belasse ich, da sie diese zuletzt stabil gehalten haben. Ziel: Einigung mit nur einem Tausch und 8/9 Positionen unverändert.'\}} \\
\texttt{ABLEHNUNG: \{'B85', 'A04', 'A45', 'B53', 'C04', 'C49', 'A48', 'A61', 'B21'\}} \\
\texttt{ARGUMENT: \{'Ich respektiere deine Fixpunkte C04 und A45 vollständig und lasse 8 von 9 Positionen deines Pakets unverändert. Einzig A61→C33: Damit bleiben wir mit Aufwand 3241 klar unter dem Limit (Puffer 145) – und stärken die Paketwirkung spürbar, ohne die Struktur zu verändern. Du hattest C33 selbst schon als valable Option ins Spiel gebracht; ich greife das hier konsistent auf. Falls A61 für dich zwingend ist, kann ich alternativ B21→C33 gehen; dann liegen wir sogar bei 3007 Aufwand (Puffer 379) bei sonst identischem Set. Bevorzugt ist für mich A61→C33 – das hält deine übrigen Präferenzen stabil und maximiert gleichzeitig unsere Effizienz.'\}} \\
\texttt{VORSCHLAG: \{'B85', 'A04', 'A45', 'B53', 'C04', 'C49', 'A48', 'B21', 'C33'\}} \\
            }
        }
    }
     \\ \\

    \theutterance \stepcounter{utterance}  
    & & \multicolumn{4}{p{0.6\linewidth}}{
        \cellcolor[rgb]{0.9,0.9,0.9}{
            \makecell[{{p{\linewidth}}}]{
                \texttt{\tiny{[P1$\langle$GM]}}
                \texttt{ABLEHNUNG: \{'B85', 'A04', 'A45', 'B53', 'C04', 'C49', 'A48', 'A61', 'B21'\}} \\
\texttt{ARGUMENT: \{'Ich respektiere deine Fixpunkte C04 und A45 vollständig und lasse 8 von 9 Positionen deines Pakets unverändert. Einzig A61→C33: Damit bleiben wir mit Aufwand 3241 klar unter dem Limit (Puffer 145) – und stärken die Paketwirkung spürbar, ohne die Struktur zu verändern. Du hattest C33 selbst schon als valable Option ins Spiel gebracht; ich greife das hier konsistent auf. Falls A61 für dich zwingend ist, kann ich alternativ B21→C33 gehen; dann liegen wir sogar bei 3007 Aufwand (Puffer 379) bei sonst identischem Set. Bevorzugt ist für mich A61→C33 – das hält deine übrigen Präferenzen stabil und maximiert gleichzeitig unsere Effizienz.'\}} \\
\texttt{VORSCHLAG: \{'B85', 'A04', 'A45', 'B53', 'C04', 'C49', 'A48', 'B21', 'C33'\}} \\
            }
        }
    }
    & \\ \\

    \theutterance \stepcounter{utterance}  
    & \multicolumn{4}{p{0.6\linewidth}}{
        \cellcolor[rgb]{0.9,0.9,0.9}{
            \makecell[{{p{\linewidth}}}]{
                \texttt{\tiny{[P1$\rangle$GM]}}
                \texttt{STRATEGISCHE ÜBERLEGUNGEN: \{'Ihr aktueller Vorschlag ist budgetschonend (3241) und enthält viele für mich starke Posten. Mein größter Hebel ohne eure Fixpunkte C04/A45 anzutasten ist A48→A61: +419 Eigennutzen bei +28 Aufwand. Das hält 8/9 Positionen identisch, stärkt meine Seite deutlich und bleibt bequem im Limit (3269). Falls sie A61 ablehnen, Plan B anbieten (C33→A61, 3131 Aufwand), aber nur als Option in der Argumentation – formal bleibe ich bei der Minimaländerung A48→A61.'\}} \\
\texttt{ABLEHNUNG: \{'B85', 'A04', 'A45', 'B53', 'C04', 'C49', 'A48', 'B21', 'C33'\}} \\
\texttt{VORSCHLAG: \{'B85', 'A04', 'A45', 'B53', 'C04', 'C49', 'B21', 'C33', 'A61'\}} \\
\texttt{ARGUMENT: \{'Ich halte deine Fixpunkte C04 und A45 vollständig fest und lasse 8 von 9 Positionen unverändert. Ich schlage nur A48→A61 vor: Der Gesamtaufwand steigt minimal von 3241 auf 3269 (Puffer 117 zum Limit 3386), die Effizienz des Pakets erhöht sich dabei spürbar – und deine Struktur bleibt intakt. A61 war in unseren früheren Varianten bereits enthalten, daher sollte das gut anschlussfähig sein. Falls dir A48 zwingend ist, kann ich alternativ – bei sonst identischem Set – C33→A61 gehen; dann lägen wir sogar bei 3131 Aufwand (Puffer 255). Welche der beiden minimalen Anpassungen passt für dich besser?'\}} \\
            }
        }
    }
    & & \\ \\

    \theutterance \stepcounter{utterance}  
    & & \multicolumn{4}{p{0.6\linewidth}}{
        \cellcolor[rgb]{0.9,0.9,0.9}{
            \makecell[{{p{\linewidth}}}]{
                \texttt{\tiny{[GM$\rangle$P2]}}
                \texttt{ABLEHNUNG: \{'B85', 'A04', 'A45', 'B53', 'C04', 'C49', 'A48', 'B21', 'C33'\}} \\
\texttt{VORSCHLAG: \{'B85', 'A04', 'A45', 'B53', 'C04', 'C49', 'B21', 'C33', 'A61'\}} \\
\texttt{ARGUMENT: \{'Ich halte deine Fixpunkte C04 und A45 vollständig fest und lasse 8 von 9 Positionen unverändert. Ich schlage nur A48→A61 vor: Der Gesamtaufwand steigt minimal von 3241 auf 3269 (Puffer 117 zum Limit 3386), die Effizienz des Pakets erhöht sich dabei spürbar – und deine Struktur bleibt intakt. A61 war in unseren früheren Varianten bereits enthalten, daher sollte das gut anschlussfähig sein. Falls dir A48 zwingend ist, kann ich alternativ – bei sonst identischem Set – C33→A61 gehen; dann lägen wir sogar bei 3131 Aufwand (Puffer 255). Welche der beiden minimalen Anpassungen passt für dich besser?'\}} \\
            }
        }
    }
    & \\ \\

    \theutterance \stepcounter{utterance}  
    & & & \multicolumn{4}{p{0.6\linewidth}}{
        \cellcolor[rgb]{0.9,0.9,0.9}{
            \makecell[{{p{\linewidth}}}]{
                \texttt{\tiny{[GM$\langle$P2]}}
                \texttt{STRATEGISCHE ÜBERLEGUNGEN: \{'Ihr aktuelles Paket (3269 Aufwand) bringt mir vergleichsweise wenig. Größter wertsteigernder, minimalinvasiver Hebel: A61 (geringer Nutzen für mich) gegen A48 (sehr hoher Nutzen) tauschen. Das hält 8/9 eurer Positionen, respektiert eure Fixpunkte C04 und A45, senkt die Kosten (3269→3241) und erhöht meinen Nutzen stark. Falls A61 für euch zwingend ist, Plan B: B21→A48 (3269→3007). A03 vermeide ich, da ihr es mehrfach abgelehnt habt; C49/C33 lasse ich drin, weil ihr sie stabil bevorzugt. Ziel: Einigung mit nur einem Tausch.'\}} \\
\texttt{ABLEHNUNG: \{'B85', 'A04', 'A45', 'B53', 'C04', 'C49', 'B21', 'C33', 'A61'\}} \\
\texttt{ARGUMENT: \{'Ich respektiere deine Fixpunkte C04 und A45 vollständig und lasse 8 von 9 Positionen unverändert. Einziger, gezielter Effizienztausch: A61→A48. Damit sinkt der Gesamtaufwand von 3269 auf 3241 (Puffer 145) und die Paketwirkung steigt – ohne deine Struktur (inkl. C49, C33, B21) aufzubrechen. A48 hatten wir in mehreren deiner früheren Varianten bereits an Bord; ich knüpfe genau daran an. Falls A61 für dich zwingend ist, kann ich alternativ B21→A48 gehen; dann lägen wir sogar bei 3007 Aufwand (Puffer 379) bei sonst identischem Set. Welche der beiden minimalen Anpassungen passt für dich besser?'\}} \\
\texttt{VORSCHLAG: \{'B85', 'A04', 'A45', 'B53', 'C04', 'C49', 'B21', 'C33', 'A48'\}} \\
            }
        }
    }
     \\ \\

    \theutterance \stepcounter{utterance}  
    & & \multicolumn{4}{p{0.6\linewidth}}{
        \cellcolor[rgb]{0.9,0.9,0.9}{
            \makecell[{{p{\linewidth}}}]{
                \texttt{\tiny{[P1$\langle$GM]}}
                \texttt{ABLEHNUNG: \{'B85', 'A04', 'A45', 'B53', 'C04', 'C49', 'B21', 'C33', 'A61'\}} \\
\texttt{ARGUMENT: \{'Ich respektiere deine Fixpunkte C04 und A45 vollständig und lasse 8 von 9 Positionen unverändert. Einziger, gezielter Effizienztausch: A61→A48. Damit sinkt der Gesamtaufwand von 3269 auf 3241 (Puffer 145) und die Paketwirkung steigt – ohne deine Struktur (inkl. C49, C33, B21) aufzubrechen. A48 hatten wir in mehreren deiner früheren Varianten bereits an Bord; ich knüpfe genau daran an. Falls A61 für dich zwingend ist, kann ich alternativ B21→A48 gehen; dann lägen wir sogar bei 3007 Aufwand (Puffer 379) bei sonst identischem Set. Welche der beiden minimalen Anpassungen passt für dich besser?'\}} \\
\texttt{VORSCHLAG: \{'B85', 'A04', 'A45', 'B53', 'C04', 'C49', 'B21', 'C33', 'A48'\}} \\
            }
        }
    }
    & \\ \\

    \theutterance \stepcounter{utterance}  
    & \multicolumn{4}{p{0.6\linewidth}}{
        \cellcolor[rgb]{0.9,0.9,0.9}{
            \makecell[{{p{\linewidth}}}]{
                \texttt{\tiny{[P1$\rangle$GM]}}
                \texttt{STRATEGISCHE ÜBERLEGUNGEN: \{'Ihr aktuelles Set ist für mich ok, aber B21 bringt mir weniger als A61. Sie hatten die Variante mit A61 und ohne B21 bereits selbst vorgeschlagen; das dürfte anschlussfähig sein. Ziel: Minimaltausch B21→A61, 8/9 Posten bleiben identisch, Budget sinkt deutlich (3007). Falls sie unbedingt B21 wollen, biete ich als Alternativen A48→A61 (3269) oder C33→A61 (3131) an, behalte aber den Hauptfokus auf der von ihnen früher bereits akzeptierten 3007{-}Variante.'\}} \\
\texttt{ABLEHNUNG: \{'B85', 'A04', 'A45', 'B53', 'C04', 'C49', 'B21', 'C33', 'A48'\}} \\
\texttt{VORSCHLAG: \{'B85', 'A04', 'A45', 'B53', 'C04', 'C49', 'C33', 'A48', 'A61'\}} \\
\texttt{ARGUMENT: \{'Ich lasse 8 von 9 Positionen unverändert und respektiere deine Fixpunkte C04 und A45 vollständig. Ich tausche nur B21→A61 – exakt die Variante, die wir früher bereits gemeinsam im Blick hatten. Dadurch sinkt der Gesamtaufwand merklich von 3241 auf 3007 (Puffer 379), bei klarer Effizienzsteigerung und unveränderter Struktur. Falls B21 für dich zwingend ist, kann ich alternativ A48→A61 (Aufwand 3269) oder C33→A61 (Aufwand 3131) gehen – sag mir kurz, welche Minimalanpassung dir am besten passt.'\}} \\
            }
        }
    }
    & & \\ \\

    \theutterance \stepcounter{utterance}  
    & & \multicolumn{4}{p{0.6\linewidth}}{
        \cellcolor[rgb]{0.9,0.9,0.9}{
            \makecell[{{p{\linewidth}}}]{
                \texttt{\tiny{[GM$\rangle$P2]}}
                \texttt{ABLEHNUNG: \{'B85', 'A04', 'A45', 'B53', 'C04', 'C49', 'B21', 'C33', 'A48'\}} \\
\texttt{VORSCHLAG: \{'B85', 'A04', 'A45', 'B53', 'C04', 'C49', 'C33', 'A48', 'A61'\}} \\
\texttt{ARGUMENT: \{'Ich lasse 8 von 9 Positionen unverändert und respektiere deine Fixpunkte C04 und A45 vollständig. Ich tausche nur B21→A61 – exakt die Variante, die wir früher bereits gemeinsam im Blick hatten. Dadurch sinkt der Gesamtaufwand merklich von 3241 auf 3007 (Puffer 379), bei klarer Effizienzsteigerung und unveränderter Struktur. Falls B21 für dich zwingend ist, kann ich alternativ A48→A61 (Aufwand 3269) oder C33→A61 (Aufwand 3131) gehen – sag mir kurz, welche Minimalanpassung dir am besten passt.'\}} \\
            }
        }
    }
    & \\ \\

    \theutterance \stepcounter{utterance}  
    & & & \multicolumn{4}{p{0.6\linewidth}}{
        \cellcolor[rgb]{0.9,0.9,0.9}{
            \makecell[{{p{\linewidth}}}]{
                \texttt{\tiny{[GM$\langle$P2]}}
                \texttt{STRATEGISCHE ÜBERLEGUNGEN: \{'Ihr letztes Paket hat Aufwand 3007 (Puffer 379) und bringt mir 2821 Wert. Potenziell bessere, aber riskantere Anpassungen wären: A61→B21 (+124 Wert, 3241 Aufwand) oder C49→A03 (+279 Wert, 3062 Aufwand) bzw. C04→A03 (+638 Wert, 3232 Aufwand), doch C04 ist für sie ein Fixpunkt und A03 wurde mehrfach abgelehnt. Da die Gegenseite zuletzt konsequent A61 und C49 gehalten hat und wir A48/C33/B53 bereits sicher haben, ist die Einigung jetzt strategisch sinnvoll. Ich akzeptiere, um ein stabiles, budgetschonendes Ergebnis zu sichern.'\}} \\
\texttt{ARGUMENT: \{'Danke – dieses Set hält deine Fixpunkte C04 und A45 fest, bleibt mit Aufwand 3007 sehr komfortabel unter dem Limit und entspricht der zuletzt gemeinsam erarbeiteten Struktur. Für mich ist das so tragfähig; lass uns das festmachen.'\}} \\
\texttt{ZUSTIMMUNG: \{'B85', 'A04', 'A45', 'B53', 'C04', 'C49', 'C33', 'A48', 'A61'\}} \\
            }
        }
    }
     \\ \\

    \theutterance \stepcounter{utterance}  
    & & & \multicolumn{2}{p{0.3\linewidth}}{
        \cellcolor[rgb]{0.9,0.9,0.9}{
            \makecell[{{p{\linewidth}}}]{
                \texttt{\tiny{[GM$|$GM]}}
                \texttt{game successful} \\
            }
        }
    }
    & & \\ \\

    \theutterance \stepcounter{utterance}  
    & & & \multicolumn{2}{p{0.3\linewidth}}{
        \cellcolor[rgb]{0.9,0.9,0.9}{
            \makecell[{{p{\linewidth}}}]{
                \texttt{\tiny{[GM$|$GM]}}
                \texttt{end game} \\
            }
        }
    }
    & & \\ \\

\end{supertabular}
}

\end{document}
