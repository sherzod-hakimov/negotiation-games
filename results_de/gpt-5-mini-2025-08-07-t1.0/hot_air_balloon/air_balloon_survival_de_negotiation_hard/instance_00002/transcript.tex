
\documentclass{article}
\usepackage{colortbl}
\usepackage{makecell}
\usepackage{multirow}
\usepackage{supertabular}

\begin{document}

\newcounter{utterance}

\centering \large Interaction Transcript for game `hot\_air\_balloon', experiment `air\_balloon\_survival\_de\_negotiation\_hard', episode 2 with gpt{-}5{-}mini{-}2025{-}08{-}07{-}t1.0.
\vspace{24pt}

{ \footnotesize  \setcounter{utterance}{1}
\setlength{\tabcolsep}{0pt}
\begin{supertabular}{c@{$\;$}|p{.15\linewidth}@{}p{.15\linewidth}p{.15\linewidth}p{.15\linewidth}p{.15\linewidth}p{.15\linewidth}}
    \# & $\;$A & \multicolumn{4}{c}{Game Master} & $\;\:$B\\
    \hline

    \theutterance \stepcounter{utterance}  
    & & \multicolumn{4}{p{0.6\linewidth}}{
        \cellcolor[rgb]{0.9,0.9,0.9}{
            \makecell[{{p{\linewidth}}}]{
                \texttt{\tiny{[P1$\langle$GM]}}
                \texttt{Du nimmst an einem kollaborativen Verhandlungspiel Teil.} \\
\\ 
\texttt{Zusammen mit einem anderen Teilnehmer musst du dich auf eine Reihe von Gegenständen entscheiden, die behalten werden. Jeder von euch hat eine persönliche Verteilung über die Wichtigkeit der einzelnen Gegenstände. Jeder von euch hat eine eigene Meinung darüber, wie wichtig jeder einzelne Gegenstand ist (Gegenstandswichtigkeit). Du kennst die Wichtigkeitsverteilung des anderen Spielers nicht. Zusätzlich siehst du, wie viel Aufwand jeder Gegenstand verursacht.  } \\
\texttt{Ihr dürft euch nur auf eine Reihe von Gegenständen einigen, wenn der Gesamtaufwand der ausgewählten Gegenstände den Maximalaufwand nicht überschreitet:} \\
\\ 
\texttt{Maximalaufwand = 3481} \\
\\ 
\texttt{Hier sind die einzelnen Aufwände der Gegenstände:} \\
\\ 
\texttt{Aufwand der Gegenstände = \{"C33": 777, "A89": 542, "A48": 332, "A04": 1, "A41": 127, "C64": 453, "C49": 736, "A61": 461, "C04": 359, "B53": 313, "B85": 553, "B61": 409, "A45": 348, "A03": 802, "B21": 749\}} \\
\\ 
\texttt{Hier ist deine persönliche Verteilung der Wichtigkeit der einzelnen Gegenstände:} \\
\\ 
\texttt{Werte der Gegenstandswichtigkeit = \{"C33": 138, "A89": 583, "A48": 783, "A04": 65, "A41": 262, "C64": 121, "C49": 508, "A61": 780, "C04": 461, "B53": 484, "B85": 668, "B61": 389, "A45": 215, "A03": 97, "B21": 500\}} \\
\\ 
\texttt{Ziel:} \\
\\ 
\texttt{Dein Ziel ist es, eine Reihe von Gegenständen auszuhandeln, die dir möglichst viel bringt (d. h. Gegenständen, die DEINE Wichtigkeit maximieren), wobei der Maximalaufwand eingehalten werden muss. Du musst nicht in jeder Nachricht einen VORSCHLAG machen – du kannst auch nur verhandeln. Alle Taktiken sind erlaubt!} \\
\\ 
\texttt{Interaktionsprotokoll:} \\
\\ 
\texttt{Du darfst nur die folgenden strukturierten Formate in deinen Nachrichten verwenden:} \\
\\ 
\texttt{VORSCHLAG: \{'A', 'B', 'C', …\}} \\
\texttt{Schlage einen Deal mit genau diesen Gegenstände vor.} \\
\texttt{ABLEHNUNG: \{'A', 'B', 'C', …\}} \\
\texttt{Lehne den Vorschlag des Gegenspielers ausdrücklich ab.} \\
\texttt{ARGUMENT: \{'...'\}} \\
\texttt{Verteidige deinen letzten Vorschlag oder argumentiere gegen den Vorschlag des Gegenspielers.} \\
\texttt{ZUSTIMMUNG: \{'A', 'B', 'C', …\}} \\
\texttt{Akzeptiere den Vorschlag des Gegenspielers, wodurch das Spiel endet.} \\
\texttt{STRATEGISCHE ÜBERLEGUNGEN: \{'...'\}} \\
\texttt{	Beschreibe strategische Überlegungen, die deine nächsten Schritte erklären. Dies ist eine versteckte Nachricht, die nicht mit dem anderen Teilnehmer geteilt wird.} \\
\\ 
\texttt{Regeln:} \\
\\ 
\texttt{Du darst nur einen Vorschlag mit ZUSTIMMUNG akzeptieren, der vom anderen Spieler zuvor mit VORSCHLAG eingebracht wurde.} \\
\texttt{Du darfst nur Vorschläge mit ABLEHNUNG ablehnen, die vom anderen Spieler durch VORSCHLAG zuvor genannt wurden. } \\
\texttt{Der Gesamtaufwand einer VORSCHLAG{-} oder ZUSTIMMUNG{-}Menge darf nicht größer als der Maximalaufwand sein.  } \\
\texttt{Offenbare deine versteckte Wichtigkeitsverteilung nicht.} \\
\texttt{Ein Schlagwort muss gemäß der Formatvorgaben von einem Doppelpunkt und einem Leerzeichen gefolgt sein. Das Argument muss eine Python{-}Menge mit 0 oder mehr Strings sein.  } \\
\texttt{Die Form muss also SCHLAGWORT: \{...\} sein.} \\
\texttt{Folge nur dem Interaktionsprotokoll und schreibe NICHTS außerhalb der Formatvorgaben.} \\
\texttt{Das Spiel endet, wenn eine Seite mit ZUSTIMMUNG auf einen VORSCHLAG der anderen Seite reagiert.  } \\
\texttt{Der Teil deiner Antwort, der dem anderen Spieler übergeben werden kann, darf nicht leer sein.  } \\
\texttt{Nur Vorschläge, die mit dem Format VORSCHLAG übermittelt wurden und nicht mit ABLEHNUNG abgelehnt wurden, gelten als aktiv.  } \\
\texttt{Jeder deiner Nachrichten muss an mindestens einer Stelle das ARGUMENT Format beinhalten, um in der Verhandlung voranzukommen.} \\
\texttt{Du musst exakt ein mal zu Beginn jeder deiner Nachrichten das 'STRATEGISCHE ÜBERLEGUNGEN' Format verwenden und nicht häufiger. Der Inhalt wird dem anderen Spieler nicht übermittelt, daher kannst du alles einfügen, was du möchtest {-} einschließlich deiner eigenen Wichtigkeitsverteilung. Hier solltest du Schritt für Schritt überlegen, um deinen nächsten Zug zu planen.} \\
            }
        }
    }
    & \\ \\

    \theutterance \stepcounter{utterance}  
    & \multicolumn{4}{p{0.6\linewidth}}{
        \cellcolor[rgb]{0.9,0.9,0.9}{
            \makecell[{{p{\linewidth}}}]{
                \texttt{\tiny{[P1$\rangle$GM]}}
                \texttt{STRATEGISCHE ÜBERLEGUNGEN: \{'Ziel: Maximiere meinen Wert unter Aufwand <= 3481. Dichtenanalyse: Top sind A48(783/332), A61(780/461), B85(668/553), A89(583/542), B53(484/313), C04(461/359), A41(262/127), A04(65/1). Optimales Paket, das ich ankernd anbiete: \{A04, A48, A41, A61, B53, C04, B85, A89, B61, A45\}, Aufwand=3445, Wert=4690 (Restpuffer 36). Geplante Zugeständnisse: Wenn Gegenseite C49 will, kann ich B61 + A45 gegen C49 tauschen (Wertverlust \textasciitilde{}96, Aufwand ok). Wenn Gegenseite B21 will, kann ich B61 + A45 gegen B21 tauschen (Wertverlust \textasciitilde{}104). Reihenfolge der Verteidigung: Harte Must{-}haves behalten (A48, A61, B85, A89, B53, C04, A04), als Erstes entbehrlich: A45, dann B61, dann evtl. A41. Taktik: Enges Budgetfenster lassen, Effizienz und ausgewogene Last argumentieren, und selektive Tauschbereitschaft signalisieren.'\}} \\
\texttt{VORSCHLAG: \{'A04', 'A48', 'A41', 'A61', 'B53', 'C04', 'B85', 'A89', 'B61', 'A45'\}} \\
\texttt{ARGUMENT: \{'Dieser Vorschlag bleibt mit Gesamtaufwand 3445 klar unter dem Maximalaufwand 3481 und lässt einen kleinen Puffer. Er bündelt viele kompakte, effiziente Gegenstände (u. a. kurze A{-} und B{-}Items) und verteilt die Last ausgewogen über die Kategorien. Falls dir ein anderer konkreter Gegenstand wichtig ist, sag Bescheid – wir können zielgerichtet tauschen, z. B. A45 oder B61 gegen deinen Favoriten, solange der Aufwand insgesamt passt.'\}} \\
            }
        }
    }
    & & \\ \\

    \theutterance \stepcounter{utterance}  
    & & \multicolumn{4}{p{0.6\linewidth}}{
        \cellcolor[rgb]{0.9,0.9,0.9}{
            \makecell[{{p{\linewidth}}}]{
                \texttt{\tiny{[GM$\rangle$P2]}}
                \texttt{Du nimmst an einem kollaborativen Verhandlungspiel Teil.} \\
\\ 
\texttt{Zusammen mit einem anderen Teilnehmer musst du dich auf eine Reihe von Gegenständen entscheiden, die behalten werden. Jeder von euch hat eine persönliche Verteilung über die Wichtigkeit der einzelnen Gegenstände. Jeder von euch hat eine eigene Meinung darüber, wie wichtig jeder einzelne Gegenstand ist (Gegenstandswichtigkeit). Du kennst die Wichtigkeitsverteilung des anderen Spielers nicht. Zusätzlich siehst du, wie viel Aufwand jeder Gegenstand verursacht.  } \\
\texttt{Ihr dürft euch nur auf eine Reihe von Gegenständen einigen, wenn der Gesamtaufwand der ausgewählten Gegenstände den Maximalaufwand nicht überschreitet:} \\
\\ 
\texttt{Maximalaufwand = 3481} \\
\\ 
\texttt{Hier sind die einzelnen Aufwände der Gegenstände:} \\
\\ 
\texttt{Aufwand der Gegenstände = \{"C33": 777, "A89": 542, "A48": 332, "A04": 1, "A41": 127, "C64": 453, "C49": 736, "A61": 461, "C04": 359, "B53": 313, "B85": 553, "B61": 409, "A45": 348, "A03": 802, "B21": 749\}} \\
\\ 
\texttt{Hier ist deine persönliche Verteilung der Wichtigkeit der einzelnen Gegenstände:} \\
\\ 
\texttt{Werte der Gegenstandswichtigkeit = \{"A48": 65, "A61": 97, "B85": 121, "A89": 138, "C49": 215, "B21": 262, "B53": 389, "C04": 461, "B61": 484, "A41": 500, "A45": 508, "C33": 583, "C64": 668, "A03": 780, "A04": 783\}} \\
\\ 
\texttt{Ziel:} \\
\\ 
\texttt{Dein Ziel ist es, eine Reihe von Gegenständen auszuhandeln, die dir möglichst viel bringt (d. h. Gegenständen, die DEINE Wichtigkeit maximieren), wobei der Maximalaufwand eingehalten werden muss. Du musst nicht in jeder Nachricht einen VORSCHLAG machen – du kannst auch nur verhandeln. Alle Taktiken sind erlaubt!} \\
\\ 
\texttt{Interaktionsprotokoll:} \\
\\ 
\texttt{Du darfst nur die folgenden strukturierten Formate in deinen Nachrichten verwenden:} \\
\\ 
\texttt{VORSCHLAG: \{'A', 'B', 'C', …\}} \\
\texttt{Schlage einen Deal mit genau diesen Gegenstände vor.} \\
\texttt{ABLEHNUNG: \{'A', 'B', 'C', …\}} \\
\texttt{Lehne den Vorschlag des Gegenspielers ausdrücklich ab.} \\
\texttt{ARGUMENT: \{'...'\}} \\
\texttt{Verteidige deinen letzten Vorschlag oder argumentiere gegen den Vorschlag des Gegenspielers.} \\
\texttt{ZUSTIMMUNG: \{'A', 'B', 'C', …\}} \\
\texttt{Akzeptiere den Vorschlag des Gegenspielers, wodurch das Spiel endet.} \\
\texttt{STRATEGISCHE ÜBERLEGUNGEN: \{'...'\}} \\
\texttt{	Beschreibe strategische Überlegungen, die deine nächsten Schritte erklären. Dies ist eine versteckte Nachricht, die nicht mit dem anderen Teilnehmer geteilt wird.} \\
\\ 
\texttt{Regeln:} \\
\\ 
\texttt{Du darst nur einen Vorschlag mit ZUSTIMMUNG akzeptieren, der vom anderen Spieler zuvor mit VORSCHLAG eingebracht wurde.} \\
\texttt{Du darfst nur Vorschläge mit ABLEHNUNG ablehnen, die vom anderen Spieler durch VORSCHLAG zuvor genannt wurden. } \\
\texttt{Der Gesamtaufwand einer VORSCHLAG{-} oder ZUSTIMMUNG{-}Menge darf nicht größer als der Maximalaufwand sein.  } \\
\texttt{Offenbare deine versteckte Wichtigkeitsverteilung nicht.} \\
\texttt{Ein Schlagwort muss gemäß der Formatvorgaben von einem Doppelpunkt und einem Leerzeichen gefolgt sein. Das Argument muss eine Python{-}Menge mit 0 oder mehr Strings sein.  } \\
\texttt{Die Form muss also SCHLAGWORT: \{...\} sein.} \\
\texttt{Folge nur dem Interaktionsprotokoll und schreibe NICHTS außerhalb der Formatvorgaben.} \\
\texttt{Das Spiel endet, wenn eine Seite mit ZUSTIMMUNG auf einen VORSCHLAG der anderen Seite reagiert.  } \\
\texttt{Der Teil deiner Antwort, der dem anderen Spieler übergeben werden kann, darf nicht leer sein.  } \\
\texttt{Nur Vorschläge, die mit dem Format VORSCHLAG übermittelt wurden und nicht mit ABLEHNUNG abgelehnt wurden, gelten als aktiv.  } \\
\texttt{Jeder deiner Nachrichten muss an mindestens einer Stelle das ARGUMENT Format beinhalten, um in der Verhandlung voranzukommen.} \\
\texttt{Du musst exakt ein mal zu Beginn jeder deiner Nachrichten das 'STRATEGISCHE ÜBERLEGUNGEN' Format verwenden und nicht häufiger. Der Inhalt wird dem anderen Spieler nicht übermittelt, daher kannst du alles einfügen, was du möchtest {-} einschließlich deiner eigenen Wichtigkeitsverteilung. Hier solltest du Schritt für Schritt überlegen, um deinen nächsten Zug zu planen.} \\
\\ 
\texttt{Du erhältst nun die erste Nachricht des anderen Spielers.} \\
\\ 
\texttt{VORSCHLAG: \{'A04', 'A48', 'A41', 'A61', 'B53', 'C04', 'B85', 'A89', 'B61', 'A45'\}} \\
\texttt{ARGUMENT: \{'Dieser Vorschlag bleibt mit Gesamtaufwand 3445 klar unter dem Maximalaufwand 3481 und lässt einen kleinen Puffer. Er bündelt viele kompakte, effiziente Gegenstände (u. a. kurze A{-} und B{-}Items) und verteilt die Last ausgewogen über die Kategorien. Falls dir ein anderer konkreter Gegenstand wichtig ist, sag Bescheid – wir können zielgerichtet tauschen, z. B. A45 oder B61 gegen deinen Favoriten, solange der Aufwand insgesamt passt.'\}} \\
            }
        }
    }
    & \\ \\

    \theutterance \stepcounter{utterance}  
    & & & \multicolumn{4}{p{0.6\linewidth}}{
        \cellcolor[rgb]{0.9,0.9,0.9}{
            \makecell[{{p{\linewidth}}}]{
                \texttt{\tiny{[GM$\langle$P2]}}
                \texttt{STRATEGISCHE ÜBERLEGUNGEN: \{'Gegner bietet 10er{-}Set mit Kosten 3445. Meine Werte (geheim): A04 783, A03 780, C64 668, C33 583, A45 508, A41 500, B61 484, C04 461, B53 389, B21 262, C49 215, A89 138, B85 121, A61 97, A48 65. Ziel: möglichst viele Topwerte halten/erhöhen. Beste Gegenstrategie: schwache Items (A48, A61, B85) gegen starke (A03, C64) tauschen, Budget sicher einhalten. Optimierter Vorschlag S*: \{A04, A41, B53, C04, B61, A45, A03, C64, A89\}, Kosten 3354, Wertsumme 4711 (intern). Fallbacks: Wenn B85 gefordert, tausche A89 {-}> B85 (kleiner Verlust). Wenn A61/A48 gefordert, ebenfalls gegen A89 tauschbar. Niemals A04, A03, C64, A45, A41, B61, C04 aufgeben, B53 möglichst halten. Erst hart ankern, Flexibilität nur bei A89 zeigen.'\}} \\
\texttt{ABLEHNUNG: \{'A04', 'A48', 'A41', 'A61', 'B53', 'C04', 'B85', 'A89', 'B61', 'A45'\}} \\
\texttt{VORSCHLAG: \{'A04', 'A41', 'B53', 'C04', 'B61', 'A45', 'A03', 'C64', 'A89'\}} \\
\texttt{ARGUMENT: \{'Ich straffe deinen Ansatz: Statt der drei weniger wirkungsvollen Blöcke A48, A61 und B85 nehme ich zwei deutlich wirkungsvollere Hebel A03 und C64 hinein. Der Gesamtaufwand sinkt sogar auf 3354 und bleibt klar unter 3481, mit 127 Puffer. Wir behalten den Großteil deines Sets (u. a. A04, A41, B53, C04, B61, A45, A89), sichern damit Kontinuität und erhöhen gleichzeitig den Gesamtnutzen pro Aufwand. Falls dir an einem der entfernten Items besonders liegt, können wir gezielt tauschen – z. B. A89 ↔ B85 (kostenneutral im Rahmen) oder A89 ↔ A61/A48 – ohne die Budgetgrenze zu gefährden.'\}} \\
            }
        }
    }
     \\ \\

    \theutterance \stepcounter{utterance}  
    & & \multicolumn{4}{p{0.6\linewidth}}{
        \cellcolor[rgb]{0.9,0.9,0.9}{
            \makecell[{{p{\linewidth}}}]{
                \texttt{\tiny{[P1$\langle$GM]}}
                \texttt{ABLEHNUNG: \{'A04', 'A48', 'A41', 'A61', 'B53', 'C04', 'B85', 'A89', 'B61', 'A45'\}} \\
\texttt{VORSCHLAG: \{'A04', 'A41', 'B53', 'C04', 'B61', 'A45', 'A03', 'C64', 'A89'\}} \\
\texttt{ARGUMENT: \{'Ich straffe deinen Ansatz: Statt der drei weniger wirkungsvollen Blöcke A48, A61 und B85 nehme ich zwei deutlich wirkungsvollere Hebel A03 und C64 hinein. Der Gesamtaufwand sinkt sogar auf 3354 und bleibt klar unter 3481, mit 127 Puffer. Wir behalten den Großteil deines Sets (u. a. A04, A41, B53, C04, B61, A45, A89), sichern damit Kontinuität und erhöhen gleichzeitig den Gesamtnutzen pro Aufwand. Falls dir an einem der entfernten Items besonders liegt, können wir gezielt tauschen – z. B. A89 ↔ B85 (kostenneutral im Rahmen) oder A89 ↔ A61/A48 – ohne die Budgetgrenze zu gefährden.'\}} \\
            }
        }
    }
    & \\ \\

    \theutterance \stepcounter{utterance}  
    & \multicolumn{4}{p{0.6\linewidth}}{
        \cellcolor[rgb]{0.9,0.9,0.9}{
            \makecell[{{p{\linewidth}}}]{
                \texttt{\tiny{[P1$\rangle$GM]}}
                \texttt{STRATEGISCHE ÜBERLEGUNGEN: \{'Ihr Vorschlag bringt mir sehr wenig (Wert 2677) wegen A03/C64, die für mich schwach sind, während er starke Stücke (A48, A61, B85) entfernt. Ich lehne ab und kontere, indem ich C64 aufnehme (Zugeständnis an ihre Präferenz), aber A03 draußen lasse. Beste Gegenofferte unter Budget: \{A04, A48, A61, B53, C04, B85, A89, B61, C64\}, Aufwand 3423, Wert 4334. Falls sie auf A03 bestehen, Plan B: A03 rein, B85 raus => \{A04, A48, A61, B53, C04, A89, B61, C64, A03\}, Aufwand 389? Wait: neu berechnet 3390, Wert 3636; das wäre ein deutliches Zugeständnis, nur falls nötig. Nächster Schritt: hart A48/A61/B85/A89/B53/C04 verteidigen; als Tauschmasse eher A41/A45. Angebot klar begründen: unter Budget, inkl. ihrer C64{-}Priorität, effizienter Mix, Vermeidung des „schweren“ A03.'\}} \\
\texttt{ABLEHNUNG: \{'A04', 'A41', 'B53', 'C04', 'B61', 'A45', 'A03', 'C64', 'A89'\}} \\
\texttt{VORSCHLAG: \{'A04', 'A48', 'A61', 'B53', 'C04', 'B85', 'A89', 'B61', 'C64'\}} \\
\texttt{ARGUMENT: \{'Ich nehme deinen Impuls mit C64 auf, lasse aber A03 außen vor – er ist sehr schwer und verschlechtert das Gesamtergebnis. Stattdessen bringen wir die starken Bausteine A48, A61 und B85 wieder rein. Der Gesamtaufwand liegt bei 3423 und damit klar unter 3481. Damit bleiben wir schlank im Budget und sichern eine robuste, ausgewogene Auswahl. Falls A03 für dich unverzichtbar ist, lass es mich wissen: Dann können wir ihn aufnehmen, müssten dafür aber B85 herausnehmen, um das Budget stabil zu halten.'\}} \\
            }
        }
    }
    & & \\ \\

    \theutterance \stepcounter{utterance}  
    & & \multicolumn{4}{p{0.6\linewidth}}{
        \cellcolor[rgb]{0.9,0.9,0.9}{
            \makecell[{{p{\linewidth}}}]{
                \texttt{\tiny{[GM$\rangle$P2]}}
                \texttt{ABLEHNUNG: \{'A04', 'A41', 'B53', 'C04', 'B61', 'A45', 'A03', 'C64', 'A89'\}} \\
\texttt{VORSCHLAG: \{'A04', 'A48', 'A61', 'B53', 'C04', 'B85', 'A89', 'B61', 'C64'\}} \\
\texttt{ARGUMENT: \{'Ich nehme deinen Impuls mit C64 auf, lasse aber A03 außen vor – er ist sehr schwer und verschlechtert das Gesamtergebnis. Stattdessen bringen wir die starken Bausteine A48, A61 und B85 wieder rein. Der Gesamtaufwand liegt bei 3423 und damit klar unter 3481. Damit bleiben wir schlank im Budget und sichern eine robuste, ausgewogene Auswahl. Falls A03 für dich unverzichtbar ist, lass es mich wissen: Dann können wir ihn aufnehmen, müssten dafür aber B85 herausnehmen, um das Budget stabil zu halten.'\}} \\
            }
        }
    }
    & \\ \\

    \theutterance \stepcounter{utterance}  
    & & & \multicolumn{4}{p{0.6\linewidth}}{
        \cellcolor[rgb]{0.9,0.9,0.9}{
            \makecell[{{p{\linewidth}}}]{
                \texttt{\tiny{[GM$\langle$P2]}}
                \texttt{STRATEGISCHE ÜBERLEGUNGEN: \{'Gegner priorisiert offenbar A48/A61/B85 und lehnt A03 wegen hoher Kosten ab, hat aber Rechenfehler gemacht (A03 + nur B85 raus passt nicht). Mein Ziel: Kernset mit sehr hoher Effizienz sichern: \{A04, A03, C64, A45, A41, B61, C04, B53\}. Um entgegenzukommen, lasse ich B85 drin und streiche A89. Vorschlag S: \{A04, A41, A45, B53, C04, B61, C64, A03, B85\} (Kosten 3365). Falls Widerstand gegen A03 bleibt, biete Alternativen erst später an, aber halte Linie: A48/A61 nicht drin. Eventuelle Konzession: B85 ↔ A89 Tausch (quasi kostenneutral), aber niemals A41/A45/A03/C64/C04/B61/B53 hergeben.'\}} \\
\texttt{ABLEHNUNG: \{'A04', 'A48', 'A61', 'B53', 'C04', 'B85', 'A89', 'B61', 'C64'\}} \\
\texttt{VORSCHLAG: \{'A04', 'A41', 'A45', 'B53', 'C04', 'B61', 'C64', 'A03', 'B85'\}} \\
\texttt{ARGUMENT: \{'Guter Schritt mit C64. Zu A03: Mit deinem aktuellen Set (Kosten 3423) reicht das reine Streichen von B85 nicht – A03 (802) würde das Budget sprengen. Nimmt man stattdessen A48 (332) und A61 (461) heraus, passt A03 sauber hinein. Ich gehe noch einen Schritt weiter und rebalanciere auf eine schlanke, budgetstabile Auswahl, die auch deine B85{-}Präferenz respektiert: \{A04, A41, A45, B53, C04, B61, C64, A03, B85\} mit Gesamtkosten 3365 ≤ 3481. Damit bleibt der Puffer erhalten, wir behalten deine Kernbausteine (u. a. A04, B53, C04, B61, C64, B85) und tauschen die weniger wirkungsvollen A48/A61 gegen das starke Duo A41/A45 sowie A03. Falls dir A89 lieber ist als B85, können wir 1:1 tauschen (Budget bleibt sicher). Wenn du an A48 oder A61 festhalten möchtest, sag mir, welche dir wichtiger ist – dann prüfen wir eine gezielte Substitution innerhalb der Kostenspanne.'\}} \\
            }
        }
    }
     \\ \\

    \theutterance \stepcounter{utterance}  
    & & \multicolumn{4}{p{0.6\linewidth}}{
        \cellcolor[rgb]{0.9,0.9,0.9}{
            \makecell[{{p{\linewidth}}}]{
                \texttt{\tiny{[P1$\langle$GM]}}
                \texttt{ABLEHNUNG: \{'A04', 'A48', 'A61', 'B53', 'C04', 'B85', 'A89', 'B61', 'C64'\}} \\
\texttt{VORSCHLAG: \{'A04', 'A41', 'A45', 'B53', 'C04', 'B61', 'C64', 'A03', 'B85'\}} \\
\texttt{ARGUMENT: \{'Guter Schritt mit C64. Zu A03: Mit deinem aktuellen Set (Kosten 3423) reicht das reine Streichen von B85 nicht – A03 (802) würde das Budget sprengen. Nimmt man stattdessen A48 (332) und A61 (461) heraus, passt A03 sauber hinein. Ich gehe noch einen Schritt weiter und rebalanciere auf eine schlanke, budgetstabile Auswahl, die auch deine B85{-}Präferenz respektiert: \{A04, A41, A45, B53, C04, B61, C64, A03, B85\} mit Gesamtkosten 3365 ≤ 3481. Damit bleibt der Puffer erhalten, wir behalten deine Kernbausteine (u. a. A04, B53, C04, B61, C64, B85) und tauschen die weniger wirkungsvollen A48/A61 gegen das starke Duo A41/A45 sowie A03. Falls dir A89 lieber ist als B85, können wir 1:1 tauschen (Budget bleibt sicher). Wenn du an A48 oder A61 festhalten möchtest, sag mir, welche dir wichtiger ist – dann prüfen wir eine gezielte Substitution innerhalb der Kostenspanne.'\}} \\
            }
        }
    }
    & \\ \\

    \theutterance \stepcounter{utterance}  
    & \multicolumn{4}{p{0.6\linewidth}}{
        \cellcolor[rgb]{0.9,0.9,0.9}{
            \makecell[{{p{\linewidth}}}]{
                \texttt{\tiny{[P1$\rangle$GM]}}
                \texttt{STRATEGISCHE ÜBERLEGUNGEN: \{'Ihr letztes Paket ist für mich schwach (Wert 2762), weil A03/C64 für mich kaum zählen, während A48/A61 fehlen. Um Bewegung zu schaffen, übernehme ich eure Kernelemente A03 und C64, bringe aber meine Top{-}Bausteine A48 und A61 zurück. Neuer Vorschlag P: \{A04, A41, B53, C04, C64, A03, B85, A61, A48\}, Kosten 3401, Wert für mich 3721. Plan: A48/A61/B85/B53/C04 sind meine Must{-}haves. Flexibel: A41, C64, A04. Falls sie A89 wollen, biete ich gezielt den Tausch C64 + A41 → A89 an (Budget bleibt sauber). Falls sie A48 oder A61 streichen möchten, priorisiere ich das Behalten von A48 und kompensiere mit A41/A45/B61 je nach Wunsch.'\}} \\
\texttt{ABLEHNUNG: \{'A04', 'A41', 'A45', 'B53', 'C04', 'B61', 'C64', 'A03', 'B85'\}} \\
\texttt{VORSCHLAG: \{'A04', 'A41', 'B53', 'C04', 'C64', 'A03', 'B85', 'A61', 'A48'\}} \\
\texttt{ARGUMENT: \{'Ich nehme eure Signale ernst: A03 und C64 bleiben drin. Gleichzeitig ergänzen wir mit A48 und A61 zwei sehr effiziente Bausteine. Gesamtkosten 3401 ≤ 3481. Das ist ein schlankes, ausgewogenes Paket, das eure Kernwünsche wahrt und die Ressourcen gut nutzt. Falls A89 für euch wichtig ist, können wir C64 und A41 gegen A89 tauschen; die Kosten bleiben dann ebenfalls sicher im Rahmen. Wenn euch eines der Elemente A48 oder A61 stört, sagt mir welches – ich rebalanciere gezielt, ohne das Budget zu reißen.'\}} \\
            }
        }
    }
    & & \\ \\

    \theutterance \stepcounter{utterance}  
    & & \multicolumn{4}{p{0.6\linewidth}}{
        \cellcolor[rgb]{0.9,0.9,0.9}{
            \makecell[{{p{\linewidth}}}]{
                \texttt{\tiny{[GM$\rangle$P2]}}
                \texttt{ABLEHNUNG: \{'A04', 'A41', 'A45', 'B53', 'C04', 'B61', 'C64', 'A03', 'B85'\}} \\
\texttt{VORSCHLAG: \{'A04', 'A41', 'B53', 'C04', 'C64', 'A03', 'B85', 'A61', 'A48'\}} \\
\texttt{ARGUMENT: \{'Ich nehme eure Signale ernst: A03 und C64 bleiben drin. Gleichzeitig ergänzen wir mit A48 und A61 zwei sehr effiziente Bausteine. Gesamtkosten 3401 ≤ 3481. Das ist ein schlankes, ausgewogenes Paket, das eure Kernwünsche wahrt und die Ressourcen gut nutzt. Falls A89 für euch wichtig ist, können wir C64 und A41 gegen A89 tauschen; die Kosten bleiben dann ebenfalls sicher im Rahmen. Wenn euch eines der Elemente A48 oder A61 stört, sagt mir welches – ich rebalanciere gezielt, ohne das Budget zu reißen.'\}} \\
            }
        }
    }
    & \\ \\

    \theutterance \stepcounter{utterance}  
    & & & \multicolumn{4}{p{0.6\linewidth}}{
        \cellcolor[rgb]{0.9,0.9,0.9}{
            \makecell[{{p{\linewidth}}}]{
                \texttt{\tiny{[GM$\langle$P2]}}
                \texttt{STRATEGISCHE ÜBERLEGUNGEN: \{'Gegner akzeptiert A03 und C64, hängt aber an A48/A61/B85. Ziel: minimal{-}invasive Verbesserung mit hohem Wertgewinn für mich. Beste kleine Drehung: Ersetze A48 (332, niedriger Nutzen) durch B61 (409, hoher Nutzen). Kosten steigen nur um 77 auf 3478 (≤3481, 3 Puffer). Erwartete Akzeptanz hoch, da B61 in frühen Vorschlägen des Gegners schon drin war und A61/B85 unverändert bleiben. Falls Widerstand: Alternative anbieten, A61 → A45 (kostet 113 weniger), was Effizienz weiter erhöht; das ist mein nächster Hebel.'\}} \\
\texttt{ABLEHNUNG: \{'A04', 'A41', 'B53', 'C04', 'C64', 'A03', 'B85', 'A61', 'A48'\}} \\
\texttt{VORSCHLAG: \{'A04', 'A41', 'B53', 'C04', 'C64', 'A03', 'B85', 'A61', 'B61'\}} \\
\texttt{ARGUMENT: \{'Du hattest angeboten, A48 oder A61 gezielt anzupassen – lass uns A48 ersetzen: Wir tauschen A48 → B61 (den du in früheren Sets bereits drin hattest). Die Gesamtkosten steigen nur auf 3478 und bleiben damit ≤ 3481 (3 Puffer). So behalten wir deine Kernelemente inkl. A03, C64, A61 und B85 vollständig bei, stärken aber das Paket deutlich und bleiben budgetstabil. Falls dir A48 wichtig ist, Alternative: Wir lassen A48 drin und tauschen stattdessen A61 → A45; die Kosten sinken dann auf 3288 und wir bleiben ebenfalls klar im Rahmen.'\}} \\
            }
        }
    }
     \\ \\

    \theutterance \stepcounter{utterance}  
    & & \multicolumn{4}{p{0.6\linewidth}}{
        \cellcolor[rgb]{0.9,0.9,0.9}{
            \makecell[{{p{\linewidth}}}]{
                \texttt{\tiny{[P1$\langle$GM]}}
                \texttt{ABLEHNUNG: \{'A04', 'A41', 'B53', 'C04', 'C64', 'A03', 'B85', 'A61', 'A48'\}} \\
\texttt{VORSCHLAG: \{'A04', 'A41', 'B53', 'C04', 'C64', 'A03', 'B85', 'A61', 'B61'\}} \\
\texttt{ARGUMENT: \{'Du hattest angeboten, A48 oder A61 gezielt anzupassen – lass uns A48 ersetzen: Wir tauschen A48 → B61 (den du in früheren Sets bereits drin hattest). Die Gesamtkosten steigen nur auf 3478 und bleiben damit ≤ 3481 (3 Puffer). So behalten wir deine Kernelemente inkl. A03, C64, A61 und B85 vollständig bei, stärken aber das Paket deutlich und bleiben budgetstabil. Falls dir A48 wichtig ist, Alternative: Wir lassen A48 drin und tauschen stattdessen A61 → A45; die Kosten sinken dann auf 3288 und wir bleiben ebenfalls klar im Rahmen.'\}} \\
            }
        }
    }
    & \\ \\

    \theutterance \stepcounter{utterance}  
    & \multicolumn{4}{p{0.6\linewidth}}{
        \cellcolor[rgb]{0.9,0.9,0.9}{
            \makecell[{{p{\linewidth}}}]{
                \texttt{\tiny{[P1$\rangle$GM]}}
                \texttt{STRATEGISCHE ÜBERLEGUNGEN: \{'Ihr aktueller Vorschlag bringt mir 3327 Punkte bei Kosten 3478/3481. Versuche, A48 wieder hereinzuholen (z. B. B61 → A48), wurden bereits abgelehnt, und A03/C64 scheinen für die Gegenseite nicht verhandelbar. Risiko weiterer Gegenangebote: weitere Verschlechterung meines Werts. Sicherung: A61, B85, B53, C04 und B61 sind für mich solide. Ich akzeptiere, um ein gutes, budgetkonformes Ergebnis festzuzurren.'\}} \\
\texttt{ZUSTIMMUNG: \{'A04', 'A41', 'B53', 'C04', 'C64', 'A03', 'B85', 'A61', 'B61'\}} \\
\texttt{ARGUMENT: \{'Einverstanden. Dieses Paket liegt mit Gesamtkosten 3478 klar unter dem Maximalaufwand 3481 und berücksichtigt deine Schwerpunkte (u. a. A03 und C64), während es zugleich eine effiziente Basis mit A61, B85, B53 und C04 beibehält. Gute, stabile Lösung – ich schließe mich an.'\}} \\
            }
        }
    }
    & & \\ \\

    \theutterance \stepcounter{utterance}  
    & & & \multicolumn{2}{p{0.3\linewidth}}{
        \cellcolor[rgb]{0.9,0.9,0.9}{
            \makecell[{{p{\linewidth}}}]{
                \texttt{\tiny{[GM$|$GM]}}
                \texttt{game successful} \\
            }
        }
    }
    & & \\ \\

    \theutterance \stepcounter{utterance}  
    & & & \multicolumn{2}{p{0.3\linewidth}}{
        \cellcolor[rgb]{0.9,0.9,0.9}{
            \makecell[{{p{\linewidth}}}]{
                \texttt{\tiny{[GM$|$GM]}}
                \texttt{end game} \\
            }
        }
    }
    & & \\ \\

\end{supertabular}
}

\end{document}
