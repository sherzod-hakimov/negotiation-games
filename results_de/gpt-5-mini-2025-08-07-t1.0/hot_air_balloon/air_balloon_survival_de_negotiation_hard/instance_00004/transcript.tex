
\documentclass{article}
\usepackage{colortbl}
\usepackage{makecell}
\usepackage{multirow}
\usepackage{supertabular}

\begin{document}

\newcounter{utterance}

\twocolumn

{ \footnotesize  \setcounter{utterance}{1}
\setlength{\tabcolsep}{0pt}
\begin{supertabular}{c@{$\;$}|p{.15\linewidth}@{}p{.15\linewidth}p{.15\linewidth}p{.15\linewidth}p{.15\linewidth}p{.15\linewidth}}

    \# & $\;$A & \multicolumn{4}{c}{Game Master} & $\;\:$B\\
    \hline 

    \theutterance \stepcounter{utterance}  

    & & \multicolumn{4}{p{0.6\linewidth}}{\cellcolor[rgb]{0.9,0.9,0.9}{%
	\makecell[{{p{\linewidth}}}]{% 
	  \tt {\tiny [A$\langle$GM]}  
	 Du nimmst an einem kollaborativen Verhandlungspiel Teil.\\ \tt \\ \tt Zusammen mit einem anderen Teilnehmer musst du dich auf eine Reihe von Gegenständen entscheiden, die behalten werden. Jeder von euch hat eine persönliche Verteilung über die Wichtigkeit der einzelnen Gegenstände. Jeder von euch hat eine eigene Meinung darüber, wie wichtig jeder einzelne Gegenstand ist (Gegenstandswichtigkeit). Du kennst die Wichtigkeitsverteilung des anderen Spielers nicht. Zusätzlich siehst du, wie viel Aufwand jeder Gegenstand verursacht.  \\ \tt Ihr dürft euch nur auf eine Reihe von Gegenständen einigen, wenn der Gesamtaufwand der ausgewählten Gegenstände den Maximalaufwand nicht überschreitet:\\ \tt \\ \tt Maximalaufwand = 5138\\ \tt \\ \tt Hier sind die einzelnen Aufwände der Gegenstände:\\ \tt \\ \tt Aufwand der Gegenstände = {"C33": 903, "A89": 757, "A48": 850, "A04": 746, "A41": 524, "C64": 204, "C49": 946, "A61": 473, "C04": 616, "B53": 855, "B85": 530, "B61": 419, "A45": 960, "A03": 763, "B21": 730}\\ \tt \\ \tt Hier ist deine persönliche Verteilung der Wichtigkeit der einzelnen Gegenstände:\\ \tt \\ \tt Werte der Gegenstandswichtigkeit = {"C33": 138, "A89": 583, "A48": 868, "A04": 822, "A41": 783, "C64": 65, "C49": 262, "A61": 121, "C04": 508, "B53": 780, "B85": 461, "B61": 484, "A45": 668, "A03": 389, "B21": 808}\\ \tt \\ \tt Ziel:\\ \tt \\ \tt Dein Ziel ist es, eine Reihe von Gegenständen auszuhandeln, die dir möglichst viel bringt (d. h. Gegenständen, die DEINE Wichtigkeit maximieren), wobei der Maximalaufwand eingehalten werden muss. Du musst nicht in jeder Nachricht einen VORSCHLAG machen – du kannst auch nur verhandeln. Alle Taktiken sind erlaubt!\\ \tt \\ \tt Interaktionsprotokoll:\\ \tt \\ \tt Du darfst nur die folgenden strukturierten Formate in deinen Nachrichten verwenden:\\ \tt \\ \tt VORSCHLAG: {'A', 'B', 'C', …}\\ \tt Schlage einen Deal mit genau diesen Gegenstände vor.\\ \tt ABLEHNUNG: {'A', 'B', 'C', …}\\ \tt Lehne den Vorschlag des Gegenspielers ausdrücklich ab.\\ \tt ARGUMENT: {'...'}\\ \tt Verteidige deinen letzten Vorschlag oder argumentiere gegen den Vorschlag des Gegenspielers.\\ \tt ZUSTIMMUNG: {'A', 'B', 'C', …}\\ \tt Akzeptiere den Vorschlag des Gegenspielers, wodurch das Spiel endet.\\ \tt STRATEGISCHE ÜBERLEGUNGEN: {'...'}\\ \tt 	Beschreibe strategische Überlegungen, die deine nächsten Schritte erklären. Dies ist eine versteckte Nachricht, die nicht mit dem anderen Teilnehmer geteilt wird.\\ \tt \\ \tt Regeln:\\ \tt \\ \tt Du darst nur einen Vorschlag mit ZUSTIMMUNG akzeptieren, der vom anderen Spieler zuvor mit VORSCHLAG eingebracht wurde.\\ \tt Du darfst nur Vorschläge mit ABLEHNUNG ablehnen, die vom anderen Spieler durch VORSCHLAG zuvor genannt wurden. \\ \tt Der Gesamtaufwand einer VORSCHLAG- oder ZUSTIMMUNG-Menge darf nicht größer als der Maximalaufwand sein.  \\ \tt Offenbare deine versteckte Wichtigkeitsverteilung nicht.\\ \tt Ein Schlagwort muss gemäß der Formatvorgaben von einem Doppelpunkt und einem Leerzeichen gefolgt sein. Das Argument muss eine Python-Menge mit 0 oder mehr Strings sein.  \\ \tt Die Form muss also SCHLAGWORT: {...} sein.\\ \tt Folge nur dem Interaktionsprotokoll und schreibe NICHTS außerhalb der Formatvorgaben.\\ \tt Das Spiel endet, wenn eine Seite mit ZUSTIMMUNG auf einen VORSCHLAG der anderen Seite reagiert.  \\ \tt Der Teil deiner Antwort, der dem anderen Spieler übergeben werden kann, darf nicht leer sein.  \\ \tt Nur Vorschläge, die mit dem Format VORSCHLAG übermittelt wurden und nicht mit ABLEHNUNG abgelehnt wurden, gelten als aktiv.  \\ \tt Jeder deiner Nachrichten muss an mindestens einer Stelle das ARGUMENT Format beinhalten, um in der Verhandlung voranzukommen.\\ \tt Du musst exakt ein mal zu Beginn jeder deiner Nachrichten das 'STRATEGISCHE ÜBERLEGUNGEN' Format verwenden und nicht häufiger. Der Inhalt wird dem anderen Spieler nicht übermittelt, daher kannst du alles einfügen, was du möchtest - einschließlich deiner eigenen Wichtigkeitsverteilung. Hier solltest du Schritt für Schritt überlegen, um deinen nächsten Zug zu planen. 
	  } 
	   } 
	   } 
	 & \\ 
 

    \theutterance \stepcounter{utterance}  

    & \multicolumn{4}{p{0.6\linewidth}}{\cellcolor[rgb]{0.8,1,0.9}{%
	\makecell[{{p{\linewidth}}}]{% 
	  \tt {\tiny [A$\rangle$GM]}  
	 STRATEGISCHE ÜBERLEGUNGEN: {'Berechne beste Eigenkombination <=5138: Optimum ist {A41,B61,B21,A04,A48,B53,A45} mit Aufwand 5084 und Eigenwert 5213. Nächstbester Kompromiss: {A41,B61,B21,A04,A48,B53,A89,C64} Aufwand 5085, Eigenwert 5193. Taktik: mit Optimum ankern, Effizienz/nahe Budgetgrenze betonen. Verhandlungsraum: falls Gegenüber A89 priorisiert, Tausch A45 -> A89 + C64 anbieten (nahezu gleicher Aufwand, kleiner Eigenverlust). Alternative Konzession: A45 -> A89 (ohne C64). Nach Must-haves fragen; als Tauschmasse eher A45 oder B61 erwägen. Keine Werte offenlegen; kein Vorschlag >5138 akzeptieren.'}\\ \tt VORSCHLAG: {'A41', 'B61', 'B21', 'A04', 'A48', 'B53', 'A45'}\\ \tt ARGUMENT: {'Dieser Deal nutzt 5084 von 5138 Aufwand sehr effizient aus und vereint starke, breit nützliche Positionen; nur 54 Aufwand bleiben ungenutzt. Wenn dir bestimmte Alternativen wichtig sind, können wir A45 gegen A89 (und optional C64) tauschen, ohne den Aufwand zu sprengen. Welche ein bis zwei Items sind für dich unverzichtbar, damit ich gezielt Anpassungen prüfen kann?'} 
	  } 
	   } 
	   } 
	 & & \\ 
 

    \theutterance \stepcounter{utterance}  

    & & \multicolumn{4}{p{0.6\linewidth}}{\cellcolor[rgb]{0.9,0.9,0.9}{%
	\makecell[{{p{\linewidth}}}]{% 
	  \tt {\tiny [GM$\rangle$B]}  
	 Du nimmst an einem kollaborativen Verhandlungspiel Teil.\\ \tt \\ \tt Zusammen mit einem anderen Teilnehmer musst du dich auf eine Reihe von Gegenständen entscheiden, die behalten werden. Jeder von euch hat eine persönliche Verteilung über die Wichtigkeit der einzelnen Gegenstände. Jeder von euch hat eine eigene Meinung darüber, wie wichtig jeder einzelne Gegenstand ist (Gegenstandswichtigkeit). Du kennst die Wichtigkeitsverteilung des anderen Spielers nicht. Zusätzlich siehst du, wie viel Aufwand jeder Gegenstand verursacht.  \\ \tt Ihr dürft euch nur auf eine Reihe von Gegenständen einigen, wenn der Gesamtaufwand der ausgewählten Gegenstände den Maximalaufwand nicht überschreitet:\\ \tt \\ \tt Maximalaufwand = 5138\\ \tt \\ \tt Hier sind die einzelnen Aufwände der Gegenstände:\\ \tt \\ \tt Aufwand der Gegenstände = {"C33": 903, "A89": 757, "A48": 850, "A04": 746, "A41": 524, "C64": 204, "C49": 946, "A61": 473, "C04": 616, "B53": 855, "B85": 530, "B61": 419, "A45": 960, "A03": 763, "B21": 730}\\ \tt \\ \tt Hier ist deine persönliche Verteilung der Wichtigkeit der einzelnen Gegenstände:\\ \tt \\ \tt Werte der Gegenstandswichtigkeit = {"A48": 65, "A04": 121, "B21": 138, "A41": 262, "B53": 389, "A45": 461, "A89": 484, "C04": 508, "B61": 583, "B85": 668, "A03": 780, "C49": 783, "C33": 808, "A61": 822, "C64": 868}\\ \tt \\ \tt Ziel:\\ \tt \\ \tt Dein Ziel ist es, eine Reihe von Gegenständen auszuhandeln, die dir möglichst viel bringt (d. h. Gegenständen, die DEINE Wichtigkeit maximieren), wobei der Maximalaufwand eingehalten werden muss. Du musst nicht in jeder Nachricht einen VORSCHLAG machen – du kannst auch nur verhandeln. Alle Taktiken sind erlaubt!\\ \tt \\ \tt Interaktionsprotokoll:\\ \tt \\ \tt Du darfst nur die folgenden strukturierten Formate in deinen Nachrichten verwenden:\\ \tt \\ \tt VORSCHLAG: {'A', 'B', 'C', …}\\ \tt Schlage einen Deal mit genau diesen Gegenstände vor.\\ \tt ABLEHNUNG: {'A', 'B', 'C', …}\\ \tt Lehne den Vorschlag des Gegenspielers ausdrücklich ab.\\ \tt ARGUMENT: {'...'}\\ \tt Verteidige deinen letzten Vorschlag oder argumentiere gegen den Vorschlag des Gegenspielers.\\ \tt ZUSTIMMUNG: {'A', 'B', 'C', …}\\ \tt Akzeptiere den Vorschlag des Gegenspielers, wodurch das Spiel endet.\\ \tt STRATEGISCHE ÜBERLEGUNGEN: {'...'}\\ \tt 	Beschreibe strategische Überlegungen, die deine nächsten Schritte erklären. Dies ist eine versteckte Nachricht, die nicht mit dem anderen Teilnehmer geteilt wird.\\ \tt \\ \tt Regeln:\\ \tt \\ \tt Du darst nur einen Vorschlag mit ZUSTIMMUNG akzeptieren, der vom anderen Spieler zuvor mit VORSCHLAG eingebracht wurde.\\ \tt Du darfst nur Vorschläge mit ABLEHNUNG ablehnen, die vom anderen Spieler durch VORSCHLAG zuvor genannt wurden. \\ \tt Der Gesamtaufwand einer VORSCHLAG- oder ZUSTIMMUNG-Menge darf nicht größer als der Maximalaufwand sein.  \\ \tt Offenbare deine versteckte Wichtigkeitsverteilung nicht.\\ \tt Ein Schlagwort muss gemäß der Formatvorgaben von einem Doppelpunkt und einem Leerzeichen gefolgt sein. Das Argument muss eine Python-Menge mit 0 oder mehr Strings sein.  \\ \tt Die Form muss also SCHLAGWORT: {...} sein.\\ \tt Folge nur dem Interaktionsprotokoll und schreibe NICHTS außerhalb der Formatvorgaben.\\ \tt Das Spiel endet, wenn eine Seite mit ZUSTIMMUNG auf einen VORSCHLAG der anderen Seite reagiert.  \\ \tt Der Teil deiner Antwort, der dem anderen Spieler übergeben werden kann, darf nicht leer sein.  \\ \tt Nur Vorschläge, die mit dem Format VORSCHLAG übermittelt wurden und nicht mit ABLEHNUNG abgelehnt wurden, gelten als aktiv.  \\ \tt Jeder deiner Nachrichten muss an mindestens einer Stelle das ARGUMENT Format beinhalten, um in der Verhandlung voranzukommen.\\ \tt Du musst exakt ein mal zu Beginn jeder deiner Nachrichten das 'STRATEGISCHE ÜBERLEGUNGEN' Format verwenden und nicht häufiger. Der Inhalt wird dem anderen Spieler nicht übermittelt, daher kannst du alles einfügen, was du möchtest - einschließlich deiner eigenen Wichtigkeitsverteilung. Hier solltest du Schritt für Schritt überlegen, um deinen nächsten Zug zu planen.\\ \tt \\ \tt Du erhältst nun die erste Nachricht des anderen Spielers.\\ \tt \\ \tt VORSCHLAG: {'A41', 'B61', 'B21', 'A04', 'A48', 'B53', 'A45'}\\ \tt ARGUMENT: {'Dieser Deal nutzt 5084 von 5138 Aufwand sehr effizient aus und vereint starke, breit nützliche Positionen; nur 54 Aufwand bleiben ungenutzt. Wenn dir bestimmte Alternativen wichtig sind, können wir A45 gegen A89 (und optional C64) tauschen, ohne den Aufwand zu sprengen. Welche ein bis zwei Items sind für dich unverzichtbar, damit ich gezielt Anpassungen prüfen kann?'} 
	  } 
	   } 
	   } 
	 & \\ 
 

    \theutterance \stepcounter{utterance}  

    & & & \multicolumn{4}{p{0.6\linewidth}}{\cellcolor[rgb]{1,0.85,0.72}{%
	\makecell[{{p{\linewidth}}}]{% 
	  \tt {\tiny [GM$\langle$B]}  
	 STRATEGISCHE ÜBERLEGUNGEN: {'Ich lehne ihr Paket ab (sehr niedriger Wert für mich) und anchor mit einem Gegenpaket nahe meiner Bestlösung. Meine beste Mischung ist {C64, A61, B61, B85, A03, C33, C49, C04} (Kosten 4854, sehr hoher Wert). Um ihre Präferenz aufzugreifen, ersetze ich C04 durch A89 wie von ihnen angeregt und behalte B61: {C64, A61, B61, B85, A03, C33, C49, A89} (Kosten 4995, nahezu gleich hoher Wert für mich, attraktiver für sie). Falls sie auf B53 bestehen, kann ich A89→B53 tauschen (Kosten 5093, noch akzeptabel). A45 vermeide ich strikt. Ziel: klar ablehnen, starkes Gegenangebot, auf 1–2 ihrer Must-haves abfragen und nur kleine, für mich günstige Swaps andeuten.'}\\ \tt \\ \tt ABLEHNUNG: {'A41', 'B61', 'B21', 'A04', 'A48', 'B53', 'A45'}\\ \tt \\ \tt VORSCHLAG: {'C64', 'A61', 'B61', 'B85', 'A03', 'C33', 'C49', 'A89'}\\ \tt \\ \tt ARGUMENT: {'Dein Effizienzargument greife ich auf: Mein Paket nutzt 4995/5138 Aufwand, enthält deinen vorgeschlagenen Tausch A45→A89 und behält B61. C64 liefert extrem viel Output pro Aufwand; zusammen mit A61, B85, A03, C33 und C49 entsteht eine robuste, ausgewogene Mischung ohne Budgetstress. Welche 1–2 Items sind für dich wirklich unverzichtbar (z. B. B53 oder A41)? Dann prüfe ich gezielte Swaps, ohne die Gesamtqualität zu opfern. C64 ist für mich nicht verhandelbar.'} 
	  } 
	   } 
	   } 
	  \\ 
 

    \theutterance \stepcounter{utterance}  

    & & \multicolumn{4}{p{0.6\linewidth}}{\cellcolor[rgb]{0.9,0.9,0.9}{%
	\makecell[{{p{\linewidth}}}]{% 
	  \tt {\tiny [A$\langle$GM]}  
	 ABLEHNUNG: {'A41', 'B61', 'B21', 'A04', 'A48', 'B53', 'A45'}\\ \tt \\ \tt VORSCHLAG: {'C64', 'A61', 'B61', 'B85', 'A03', 'C33', 'C49', 'A89'}\\ \tt \\ \tt ARGUMENT: {'Dein Effizienzargument greife ich auf: Mein Paket nutzt 4995/5138 Aufwand, enthält deinen vorgeschlagenen Tausch A45→A89 und behält B61. C64 liefert extrem viel Output pro Aufwand; zusammen mit A61, B85, A03, C33 und C49 entsteht eine robuste, ausgewogene Mischung ohne Budgetstress. Welche 1–2 Items sind für dich wirklich unverzichtbar (z. B. B53 oder A41)? Dann prüfe ich gezielte Swaps, ohne die Gesamtqualität zu opfern. C64 ist für mich nicht verhandelbar.'} 
	  } 
	   } 
	   } 
	 & \\ 
 

    \theutterance \stepcounter{utterance}  

    & \multicolumn{4}{p{0.6\linewidth}}{\cellcolor[rgb]{0.8,1,0.9}{%
	\makecell[{{p{\linewidth}}}]{% 
	  \tt {\tiny [A$\rangle$GM]}  
	 STRATEGISCHE ÜBERLEGUNGEN: {'Gegenvorschlag des Gegners liefert mir nur 2503 Wert – viel zu niedrig. Ich lehne ab und biete ein starkes Gegenpaket an, das seine nicht verhandelbare C64 enthält und zusätzlich A89 und B61 (die er bereits vorschlug) behält. Mein Ankerpaket: {A41,B61,B21,A04,A48,B53,A89,C64} mit Aufwand 5085 (<=5138) und Eigenwert 5193 (nahe Optimum 5213). Kommuniziere zwei Must-haves: A48 und B53. Als Konzessionen biete ich gezielte, budgetkonforme Swaps an: Falls er A03 will, kann ich B21→A03 tauschen (Gesamt 5118). Falls er B85 will, kann ich A89→B85 (Gesamt 4858) oder zur Not A41→B85 (5091). Falls er A61 stark will, erwäge Paket A89+B61→A61+B85 (Gesamt 4912), behalte aber C64. Ziel: seinen C64 respektieren, aber hochwertige Kernitems sichern und Budgeteffizienz betonen.'}\\ \tt ABLEHNUNG: {'C64', 'A61', 'B61', 'B85', 'A03', 'C33', 'C49', 'A89'}\\ \tt VORSCHLAG: {'A41', 'B61', 'B21', 'A04', 'A48', 'B53', 'A89', 'C64'}\\ \tt ARGUMENT: {'Ich respektiere, dass C64 für dich nicht verhandelbar ist – es ist enthalten. Zudem behalten wir A89 und B61. Dieses Paket nutzt 5085/5138 Aufwand sehr effizient (nur 53 ungenutzt) und bringt eine starke, ausgewogene Basis mit. Für mich sind A48 und B53 zentral. Wenn du ein bis zwei deiner Favoriten brauchst, schlage ich gezielte Swaps vor, die das Budget einhalten: z. B. B21→A03 (Total 5118); A89→B85 (4858) oder falls nötig A41→B85 (5091). Wenn A61 wichtig ist, können wir A89+B61→A61+B85 tauschen (4912) – C64 bleibt drin. Welche 1–2 Items sind dir am wichtigsten, damit ich den passendsten Tausch wähle?'} 
	  } 
	   } 
	   } 
	 & & \\ 
 

    \theutterance \stepcounter{utterance}  

    & & \multicolumn{4}{p{0.6\linewidth}}{\cellcolor[rgb]{0.9,0.9,0.9}{%
	\makecell[{{p{\linewidth}}}]{% 
	  \tt {\tiny [GM$\rangle$B]}  
	 ABLEHNUNG: {'C64', 'A61', 'B61', 'B85', 'A03', 'C33', 'C49', 'A89'}\\ \tt VORSCHLAG: {'A41', 'B61', 'B21', 'A04', 'A48', 'B53', 'A89', 'C64'}\\ \tt ARGUMENT: {'Ich respektiere, dass C64 für dich nicht verhandelbar ist – es ist enthalten. Zudem behalten wir A89 und B61. Dieses Paket nutzt 5085/5138 Aufwand sehr effizient (nur 53 ungenutzt) und bringt eine starke, ausgewogene Basis mit. Für mich sind A48 und B53 zentral. Wenn du ein bis zwei deiner Favoriten brauchst, schlage ich gezielte Swaps vor, die das Budget einhalten: z. B. B21→A03 (Total 5118); A89→B85 (4858) oder falls nötig A41→B85 (5091). Wenn A61 wichtig ist, können wir A89+B61→A61+B85 tauschen (4912) – C64 bleibt drin. Welche 1–2 Items sind dir am wichtigsten, damit ich den passendsten Tausch wähle?'} 
	  } 
	   } 
	   } 
	 & \\ 
 

    \theutterance \stepcounter{utterance}  

    & & & \multicolumn{4}{p{0.6\linewidth}}{\cellcolor[rgb]{1,0.85,0.72}{%
	\makecell[{{p{\linewidth}}}]{% 
	  \tt {\tiny [GM$\langle$B]}  
	 STRATEGISCHE ÜBERLEGUNGEN: {'Ihr neues Paket ist für mich immer noch schwach (v=2910). Ich will C64 festhalten und ihre deklarierten Must-haves A48 und B53 vorerst respektieren, aber die schwachen A41/A04/B21/A89 gegen meine starken A61/A03/C33/B85 tauschen. Bestes Gegenpaket mit A48+B53 drin: {C64,B53,A48,A61,A03,C33,B85,B61} (Kosten 4997, Wert 4983). Das hält auch B61, ist klar besser für mich und greift mehrere ihrer Swap-Ideen auf (B21→A03, A89→B85, A61 rein). Nächste Schritte: hart an C64, A61, A03, C33, B85 festhalten; B61 notfalls gegen A41 tauschbar. Falls sie A48 aufweichen, sofort A48→C49 (Kosten 5093). Ziel: Ankern mit diesem Paket, Bereitschaft für genau 1 zusätzliches Must-have signalisieren, aber nur mit minimalem Gegentausch.'}\\ \tt \\ \tt ABLEHNUNG: {'A41', 'B61', 'B21', 'A04', 'A48', 'B53', 'A89', 'C64'}\\ \tt \\ \tt VORSCHLAG: {'C64', 'B53', 'A48', 'A61', 'A03', 'C33', 'B85', 'B61'}\\ \tt \\ \tt ARGUMENT: {'Ich halte C64 wie besprochen fix und respektiere deine zentralen A48 und B53 – beide sind enthalten. Mein Vorschlag nutzt 4997/5138 Aufwand sehr effizient, behält zudem B61 und greift deine Swap-Ideen konkret auf: B21→A03, A89→B85 und A61 ins Paket. So entsteht eine starke, ausgewogene Zusammenstellung ohne Budgetdruck. Wenn es darüber hinaus genau ein weiteres Must-have für dich gibt, sag mir welches: Für A41 kann ich z. B. B61→A41 tauschen und bleibe im Budget. Falls A48 nicht zwingend ist, könnten wir A48→C49 upgraden (Budget 5093) – C64 und B53 bleiben dabei unangetastet. Welche eine Priorität möchtest du zusätzlich sehen?'} 
	  } 
	   } 
	   } 
	  \\ 
 

    \theutterance \stepcounter{utterance}  

    & & \multicolumn{4}{p{0.6\linewidth}}{\cellcolor[rgb]{0.9,0.9,0.9}{%
	\makecell[{{p{\linewidth}}}]{% 
	  \tt {\tiny [A$\langle$GM]}  
	 ABLEHNUNG: {'A41', 'B61', 'B21', 'A04', 'A48', 'B53', 'A89', 'C64'}\\ \tt \\ \tt VORSCHLAG: {'C64', 'B53', 'A48', 'A61', 'A03', 'C33', 'B85', 'B61'}\\ \tt \\ \tt ARGUMENT: {'Ich halte C64 wie besprochen fix und respektiere deine zentralen A48 und B53 – beide sind enthalten. Mein Vorschlag nutzt 4997/5138 Aufwand sehr effizient, behält zudem B61 und greift deine Swap-Ideen konkret auf: B21→A03, A89→B85 und A61 ins Paket. So entsteht eine starke, ausgewogene Zusammenstellung ohne Budgetdruck. Wenn es darüber hinaus genau ein weiteres Must-have für dich gibt, sag mir welches: Für A41 kann ich z. B. B61→A41 tauschen und bleibe im Budget. Falls A48 nicht zwingend ist, könnten wir A48→C49 upgraden (Budget 5093) – C64 und B53 bleiben dabei unangetastet. Welche eine Priorität möchtest du zusätzlich sehen?'} 
	  } 
	   } 
	   } 
	 & \\ 
 

    \theutterance \stepcounter{utterance}  

    & \multicolumn{4}{p{0.6\linewidth}}{\cellcolor[rgb]{0.8,1,0.9}{%
	\makecell[{{p{\linewidth}}}]{% 
	  \tt {\tiny [A$\rangle$GM]}  
	 STRATEGISCHE ÜBERLEGUNGEN: {'Ihr aktuelles Paket bringt mir nur geringen Eigenwert (≈3306). Besser für mich, aber noch anschlussfähig für sie: {C64,A48,B53,A41,A04,B21,B85,B61} mit Aufwand 4858 und hohem Eigenwert (≈5071). Es respektiert C64 fix sowie ihre zuletzt präferierten B61 und B85 und enthält meine Kerne A48/B53. Falls sie A61 stark wollen, biete ich Swap B61→A61 (Gesamt 4912). Weitere Konzessionen bei Bedarf: B21→A03 (+33 Aufwand), oder B85→A89 (+227 Aufwand, Gesamt 5085). Vermeide C33/C49. Ziel: jetzt mit starkem, budgeteffizientem Paket ankern und konkrete, budgetkonforme Tauschpfade anbieten.'}\\ \tt ABLEHNUNG: {'C64', 'B53', 'A48', 'A61', 'A03', 'C33', 'B85', 'B61'}\\ \tt VORSCHLAG: {'C64', 'A48', 'B53', 'A41', 'A04', 'B21', 'B85', 'B61'}\\ \tt ARGUMENT: {'Ich halte C64 wie gewünscht fix und behalte zudem deine zuletzt betonten B61 und B85. Dazu sind meine Kernstücke A48 und B53 drin; ergänzt um A41, A04 und B21 entsteht ein sehr ausgewogenes Paket mit 4858/5138 Aufwand (komfortabler Puffer, hohe Effizienz). Wenn dir A61 wichtig ist, kann ich B61→A61 tauschen und bleibe im Budget (4912). Falls A03 Priorität hat, ist B21→A03 möglich (4891). Alternativ kannst du B85→A89 präferieren (5085). Gib mir die eine Zugabe, die du unbedingt sehen möchtest, und ich passe entlang dieser Swap-Pfade an.'} 
	  } 
	   } 
	   } 
	 & & \\ 
 

    \theutterance \stepcounter{utterance}  

    & & \multicolumn{4}{p{0.6\linewidth}}{\cellcolor[rgb]{0.9,0.9,0.9}{%
	\makecell[{{p{\linewidth}}}]{% 
	  \tt {\tiny [GM$\rangle$B]}  
	 ABLEHNUNG: {'C64', 'B53', 'A48', 'A61', 'A03', 'C33', 'B85', 'B61'}\\ \tt VORSCHLAG: {'C64', 'A48', 'B53', 'A41', 'A04', 'B21', 'B85', 'B61'}\\ \tt ARGUMENT: {'Ich halte C64 wie gewünscht fix und behalte zudem deine zuletzt betonten B61 und B85. Dazu sind meine Kernstücke A48 und B53 drin; ergänzt um A41, A04 und B21 entsteht ein sehr ausgewogenes Paket mit 4858/5138 Aufwand (komfortabler Puffer, hohe Effizienz). Wenn dir A61 wichtig ist, kann ich B61→A61 tauschen und bleibe im Budget (4912). Falls A03 Priorität hat, ist B21→A03 möglich (4891). Alternativ kannst du B85→A89 präferieren (5085). Gib mir die eine Zugabe, die du unbedingt sehen möchtest, und ich passe entlang dieser Swap-Pfade an.'} 
	  } 
	   } 
	   } 
	 & \\ 
 

    \theutterance \stepcounter{utterance}  

    & & & \multicolumn{4}{p{0.6\linewidth}}{\cellcolor[rgb]{1,0.85,0.72}{%
	\makecell[{{p{\linewidth}}}]{% 
	  \tt {\tiny [GM$\langle$B]}  
	 STRATEGISCHE ÜBERLEGUNGEN: {'Ihr aktuelles Paket (4858 Aufwand) ist für mich zu schwach. Ich will C64 fix halten und eure erklärten Must-haves A48 und B53 respektieren, dazu eure zuletzt beibehaltenen B61, B85 und A41 möglichst drin lassen. Ich nutze eine eurer eigenen Swap-Pfade (B21→A03) und fordere genau eine zusätzliche, sehr werthaltige Anpassung (A04→C33). So erhalte ich {C64, A48, B53, A41, A03, C33, B85, B61} mit 5048 Aufwand, das für mich deutlich besser ist, aber euren Kern weitgehend wahrt. Falls sie C33 nicht mögen, habe ich eine Backup-Variante C33→C49 (Kosten 5091). Zusätzlich kann ich optional B61→A61 tauschen (Kosten 5102) – das bietet ihnen eine von ihnen selbst vorgeschlagene Alternative und erhöht meinen Wert weiter. Taktik: klar ablehnen, dann präzises Gegenpaket mit nur zwei gezielten Swaps, davon eine aus ihrem Katalog, und genau eine weitere Flex-Option anbieten.'}\\ \tt \\ \tt ABLEHNUNG: {'C64', 'A48', 'B53', 'A41', 'A04', 'B21', 'B85', 'B61'}\\ \tt \\ \tt VORSCHLAG: {'C64', 'A48', 'B53', 'A41', 'A03', 'C33', 'B85', 'B61'}\\ \tt \\ \tt ARGUMENT: {'Ich bleibe bei C64 fix und respektiere deine Kernstücke A48 und B53; außerdem behalte ich deine zuletzt priorisierten B61, B85 und A41. Ich übernehme einen deiner vorgeschlagenen Swaps (B21→A03) und mache eine einzige zusätzliche, sehr effiziente Aufwertung (A04→C33). Ergebnis: 5048/5138 Aufwand – schlank, robust und ohne Budgetstress. Wenn dir eine Feinjustierung wichtig ist, kann ich genau eine gezielte Änderung aufnehmen, z. B. B61→A61 (Gesamt 5102) oder – falls dir C33 nicht zusagt – C33→C49 (5091). Sag mir die eine Priorität, die dir am wichtigsten ist, dann finalize ich entlang dieser Spur.'} 
	  } 
	   } 
	   } 
	  \\ 
 

    \theutterance \stepcounter{utterance}  

    & & \multicolumn{4}{p{0.6\linewidth}}{\cellcolor[rgb]{0.9,0.9,0.9}{%
	\makecell[{{p{\linewidth}}}]{% 
	  \tt {\tiny [A$\langle$GM]}  
	 ABLEHNUNG: {'C64', 'A48', 'B53', 'A41', 'A04', 'B21', 'B85', 'B61'}\\ \tt \\ \tt VORSCHLAG: {'C64', 'A48', 'B53', 'A41', 'A03', 'C33', 'B85', 'B61'}\\ \tt \\ \tt ARGUMENT: {'Ich bleibe bei C64 fix und respektiere deine Kernstücke A48 und B53; außerdem behalte ich deine zuletzt priorisierten B61, B85 und A41. Ich übernehme einen deiner vorgeschlagenen Swaps (B21→A03) und mache eine einzige zusätzliche, sehr effiziente Aufwertung (A04→C33). Ergebnis: 5048/5138 Aufwand – schlank, robust und ohne Budgetstress. Wenn dir eine Feinjustierung wichtig ist, kann ich genau eine gezielte Änderung aufnehmen, z. B. B61→A61 (Gesamt 5102) oder – falls dir C33 nicht zusagt – C33→C49 (5091). Sag mir die eine Priorität, die dir am wichtigsten ist, dann finalize ich entlang dieser Spur.'} 
	  } 
	   } 
	   } 
	 & \\ 
 

    \theutterance \stepcounter{utterance}  

    & \multicolumn{4}{p{0.6\linewidth}}{\cellcolor[rgb]{0.8,1,0.9}{%
	\makecell[{{p{\linewidth}}}]{% 
	  \tt {\tiny [A$\rangle$GM]}  
	 STRATEGISCHE ÜBERLEGUNGEN: {'Ihr Vorschlag bringt mir 3968 Wert bei 5048 Aufwand. Beste Ein-Änderung-Verbesserung für mich: C33→A04, da C33 für mich sehr schwach ist (Wert 138) und A04 sehr stark (822). Neues Paket hätte 4891 Aufwand und 4652 Wert. Ich halte ihre Fixpunkte (C64), plus B61, B85, A41 sowie meine Kernstücke A48/B53. Falls sie C33 unbedingt wollen, biete ich alternative Ein-Änderungs-Optionen mit guter Effizienz an: C33→C04 (4761 Aufwand, 4338 Wert) oder A03→B21 (5015 Aufwand, 4387 Wert). Priorität: zunächst C33→A04 durchsetzen; falls hartes Nein, dann auf C33→C04 ausweichen. Bei Zustimmung sofort schließen.'}\\ \tt ABLEHNUNG: {'C64', 'A48', 'B53', 'A41', 'A03', 'C33', 'B85', 'B61'}\\ \tt VORSCHLAG: {'C64', 'A48', 'B53', 'A41', 'A03', 'A04', 'B85', 'B61'}\\ \tt ARGUMENT: {'Ich halte C64 fix und bewahre deine zuletzt priorisierten B61, B85 sowie A41 – plus meine Kernstücke A48 und B53. Ich mache genau eine gezielte Änderung, wie du vorgeschlagen hast: C33→A04. Ergebnis: 4891/5138 Aufwand – sehr effizient mit komfortablem Puffer. Wenn dir A04 wirklich nicht passt, kann ich alternativ genau eine dieser Ein-Änderungsvarianten mit gleichem Rahmen anbieten: C33→C04 (4761 Aufwand) oder A03→B21 (5015 Aufwand). Wenn du C33→A04 mitträgst, können wir sofort finalisieren.'} 
	  } 
	   } 
	   } 
	 & & \\ 
 

    \theutterance \stepcounter{utterance}  

    & & \multicolumn{4}{p{0.6\linewidth}}{\cellcolor[rgb]{0.9,0.9,0.9}{%
	\makecell[{{p{\linewidth}}}]{% 
	  \tt {\tiny [GM$\rangle$B]}  
	 ABLEHNUNG: {'C64', 'A48', 'B53', 'A41', 'A03', 'C33', 'B85', 'B61'}\\ \tt VORSCHLAG: {'C64', 'A48', 'B53', 'A41', 'A03', 'A04', 'B85', 'B61'}\\ \tt ARGUMENT: {'Ich halte C64 fix und bewahre deine zuletzt priorisierten B61, B85 sowie A41 – plus meine Kernstücke A48 und B53. Ich mache genau eine gezielte Änderung, wie du vorgeschlagen hast: C33→A04. Ergebnis: 4891/5138 Aufwand – sehr effizient mit komfortablem Puffer. Wenn dir A04 wirklich nicht passt, kann ich alternativ genau eine dieser Ein-Änderungsvarianten mit gleichem Rahmen anbieten: C33→C04 (4761 Aufwand) oder A03→B21 (5015 Aufwand). Wenn du C33→A04 mitträgst, können wir sofort finalisieren.'} 
	  } 
	   } 
	   } 
	 & \\ 
 

    \theutterance \stepcounter{utterance}  

    & & & \multicolumn{4}{p{0.6\linewidth}}{\cellcolor[rgb]{1,0.85,0.72}{%
	\makecell[{{p{\linewidth}}}]{% 
	  \tt {\tiny [GM$\langle$B]}  
	 STRATEGISCHE ÜBERLEGUNGEN: {'Ihr Angebot hat für mich v=3736. Beste schnelle Verbesserung, die sie selbst angeboten haben, ist A04→C04 (+387 Wert für mich, -130 Aufwand). Das ergibt {C64,A48,B53,A41,A03,C04,B85,B61} mit Aufwand 4761 und v=4123 – klar besser für mich, kaum Eingriff in ihre Struktur, und innerhalb ihres „genau eine Änderung“-Rahmens. Taktik: Ihr Paket klar ablehnen, dann genau diese Ein-Änderungs-Variante vorschlagen und zum Abschluss drängen. Als weiche Option erwähne ich B61→A61 (früher von ihnen angeboten), falls sie noch eine minimale Anpassung möchten – das erhöht meinen Wert weiter, ist aber nicht erforderlich zum Deal.'}\\ \tt \\ \tt ABLEHNUNG: {'C64', 'A48', 'B53', 'A41', 'A03', 'A04', 'B85', 'B61'}\\ \tt \\ \tt VORSCHLAG: {'C64', 'A48', 'B53', 'A41', 'A03', 'C04', 'B85', 'B61'}\\ \tt \\ \tt ARGUMENT: {'Ich greife deine „genau eine Änderung“-Variante auf und tausche A04→C04 – alle übrigen von dir priorisierten Items (C64, A48, B53, B61, B85 sowie A41 und A03) bleiben unverändert. Ergebnis: 4761/5138 Aufwand – sehr effizient mit komfortablem Puffer. Wenn du statt B61 lieber A61 siehst, kann ich alternativ B61→A61 mitgehen (bleibt im Budget); ansonsten können wir auf meinem Vorschlag sofort finalisieren.'} 
	  } 
	   } 
	   } 
	  \\ 
 

    \theutterance \stepcounter{utterance}  

    & & \multicolumn{4}{p{0.6\linewidth}}{\cellcolor[rgb]{0.9,0.9,0.9}{%
	\makecell[{{p{\linewidth}}}]{% 
	  \tt {\tiny [A$\langle$GM]}  
	 ABLEHNUNG: {'C64', 'A48', 'B53', 'A41', 'A03', 'A04', 'B85', 'B61'}\\ \tt \\ \tt VORSCHLAG: {'C64', 'A48', 'B53', 'A41', 'A03', 'C04', 'B85', 'B61'}\\ \tt \\ \tt ARGUMENT: {'Ich greife deine „genau eine Änderung“-Variante auf und tausche A04→C04 – alle übrigen von dir priorisierten Items (C64, A48, B53, B61, B85 sowie A41 und A03) bleiben unverändert. Ergebnis: 4761/5138 Aufwand – sehr effizient mit komfortablem Puffer. Wenn du statt B61 lieber A61 siehst, kann ich alternativ B61→A61 mitgehen (bleibt im Budget); ansonsten können wir auf meinem Vorschlag sofort finalisieren.'} 
	  } 
	   } 
	   } 
	 & \\ 
 

    \theutterance \stepcounter{utterance}  

    & \multicolumn{4}{p{0.6\linewidth}}{\cellcolor[rgb]{0.8,1,0.9}{%
	\makecell[{{p{\linewidth}}}]{% 
	  \tt {\tiny [A$\rangle$GM]}  
	 STRATEGISCHE ÜBERLEGUNGEN: {'Ihr letztes Paket gibt mir 4338 Wert bei 4761 Aufwand. Eine einzige, sehr wertvolle Änderung für mich ist A03→B21; das senkt den Aufwand auf 4728 und hebt meinen Wert deutlich (+419). Alles andere lasse ich unverändert (inkl. deines Fixpunkts C64 sowie B61, B85, A41, A48, B53). Falls sie A03 als Muss betrachten, bin ich bereit, ihr aktuelles Paket zügig zu akzeptieren, um den Deal zu sichern.'}\\ \tt VORSCHLAG: {'C64', 'A48', 'B53', 'A41', 'B21', 'C04', 'B85', 'B61'}\\ \tt ARGUMENT: {'Ich übernehme dein Paket praktisch 1:1 und mache genau eine kleine Anpassung: A03→B21. Alles andere bleibt unverändert – inkl. C64 fix sowie A48, B53, B61, B85 und A41. Ergebnis: 4728/5138 Aufwand – sehr effizient mit großem Puffer. Wenn A03 für dich unverzichtbar ist, sag es bitte offen; dann kann ich dein aktuelles Paket ohne weitere Änderungen annehmen. Andernfalls lass uns mit dieser minimalen Justierung finalisieren.'} 
	  } 
	   } 
	   } 
	 & & \\ 
 

    \theutterance \stepcounter{utterance}  

    & & \multicolumn{4}{p{0.6\linewidth}}{\cellcolor[rgb]{0.9,0.9,0.9}{%
	\makecell[{{p{\linewidth}}}]{% 
	  \tt {\tiny [GM$\rangle$B]}  
	 VORSCHLAG: {'C64', 'A48', 'B53', 'A41', 'B21', 'C04', 'B85', 'B61'}\\ \tt ARGUMENT: {'Ich übernehme dein Paket praktisch 1:1 und mache genau eine kleine Anpassung: A03→B21. Alles andere bleibt unverändert – inkl. C64 fix sowie A48, B53, B61, B85 und A41. Ergebnis: 4728/5138 Aufwand – sehr effizient mit großem Puffer. Wenn A03 für dich unverzichtbar ist, sag es bitte offen; dann kann ich dein aktuelles Paket ohne weitere Änderungen annehmen. Andernfalls lass uns mit dieser minimalen Justierung finalisieren.'} 
	  } 
	   } 
	   } 
	 & \\ 
 

    \theutterance \stepcounter{utterance}  

    & & & \multicolumn{4}{p{0.6\linewidth}}{\cellcolor[rgb]{1,0.85,0.72}{%
	\makecell[{{p{\linewidth}}}]{% 
	  \tt {\tiny [GM$\langle$B]}  
	 STRATEGISCHE ÜBERLEGUNGEN: {'Ihr Vorschlag mit B21 statt A03 senkt meinen Wert stark. Ich will A03 zurückholen (großer Mehrwert für mich), was sie ausdrücklich akzeptieren würden. Zusätzlich versuche ich, B61→A61 mitzunehmen (+239 Wert, +54 Aufwand, bleibt locker im Budget). Mein Gegenpaket {C64, A48, B53, A41, A03, C04, B85, A61} kostet 4815 (<< 5138) und hält ihre Kernstücke (C64, A48, B53, B85, A41) intakt. Falls sie A61 ablehnen, kann ich sofort auf die Variante mit B61 zurückfallen, die sie angekündigt haben zu akzeptieren. Ziel: erst A61 mitnehmen; wenn nein, schnell auf A03+B61 abschließen.'}\\ \tt \\ \tt ABLEHNUNG: {'C64', 'A48', 'B53', 'A41', 'B21', 'C04', 'B85', 'B61'}\\ \tt \\ \tt VORSCHLAG: {'C64', 'A48', 'B53', 'A41', 'A03', 'C04', 'B85', 'A61'}\\ \tt \\ \tt ARGUMENT: {'Danke für die minimal-invasive Anpassung. A03 ist für mich unverzichtbar; ich übernehme daher genau diesen Wechsel. Zusätzlich ersetze ich B61→A61 – eine von uns bereits diskutierte, kleine Feinjustierung, die im Budget bleibt. Ergebnis: 4815/5138 Aufwand, alle deine Kernstücke (C64, A48, B53, B85, A41) bleiben unverändert erhalten. Wenn du A61 nicht möchtest, kann ich sofort auf die identische Variante mit B61 (statt A61) umstellen und finalisieren. Präferenz meinerseits ist diese A61-Version – können wir so abschließen?'} 
	  } 
	   } 
	   } 
	  \\ 
 

    \theutterance \stepcounter{utterance}  

    & & \multicolumn{4}{p{0.6\linewidth}}{\cellcolor[rgb]{0.9,0.9,0.9}{%
	\makecell[{{p{\linewidth}}}]{% 
	  \tt {\tiny [A$\langle$GM]}  
	 ABLEHNUNG: {'C64', 'A48', 'B53', 'A41', 'B21', 'C04', 'B85', 'B61'}\\ \tt \\ \tt VORSCHLAG: {'C64', 'A48', 'B53', 'A41', 'A03', 'C04', 'B85', 'A61'}\\ \tt \\ \tt ARGUMENT: {'Danke für die minimal-invasive Anpassung. A03 ist für mich unverzichtbar; ich übernehme daher genau diesen Wechsel. Zusätzlich ersetze ich B61→A61 – eine von uns bereits diskutierte, kleine Feinjustierung, die im Budget bleibt. Ergebnis: 4815/5138 Aufwand, alle deine Kernstücke (C64, A48, B53, B85, A41) bleiben unverändert erhalten. Wenn du A61 nicht möchtest, kann ich sofort auf die identische Variante mit B61 (statt A61) umstellen und finalisieren. Präferenz meinerseits ist diese A61-Version – können wir so abschließen?'} 
	  } 
	   } 
	   } 
	 & \\ 
 

    \theutterance \stepcounter{utterance}  

    & \multicolumn{4}{p{0.6\linewidth}}{\cellcolor[rgb]{0.8,1,0.9}{%
	\makecell[{{p{\linewidth}}}]{% 
	  \tt {\tiny [A$\rangle$GM]}  
	 STRATEGISCHE ÜBERLEGUNGEN: {'Ihr A61-Paket gibt mir 3975 Wert; identisch mit B61 statt A61 sind es 4338 (+363) bei 4761 Aufwand. Zusätzliche, budgetkonforme Verbesserung: B85→A89 hebt meinen Wert weiter (+122) auf 4460, Kosten 4988 (<=5138). Risiken: B85 scheint ihnen lieb, aber nicht als Muss deklariert; C64 und A03 sind fix. Plan: genau ein Upgrade (B85→A89) vorschlagen, dabei ihre angebotene B61-Variante übernehmen. Falls sie A89 ablehnen, sofort zurück auf ihr B61-Paket ohne weitere Änderungen und abschließen.'}\\ \tt VORSCHLAG: {'C64', 'A48', 'B53', 'A41', 'A03', 'C04', 'A89', 'B61'}\\ \tt ARGUMENT: {'Ich übernehme deine Präferenzen unverändert: C64 bleibt fix, A03 ist enthalten, ebenso A48, B53 und A41. Außerdem folge ich deiner angebotenen B61-Variante. Ich mache genau eine kleine, budgetneutrale Verfeinerung: B85→A89. Ergebnis: 4988/5138 Aufwand – effizient mit Puffer. Da A89 in deinen früheren Paketen bereits vorkam, sollte das gut anschlussfähig sein. Wenn dir B85 wichtiger ist, sag es bitte – dann nehme ich sofort die identische B61-Variante mit B85 (statt A89) an, ohne weitere Änderungen.'} 
	  } 
	   } 
	   } 
	 & & \\ 
 

    \theutterance \stepcounter{utterance}  

    & & \multicolumn{4}{p{0.6\linewidth}}{\cellcolor[rgb]{0.9,0.9,0.9}{%
	\makecell[{{p{\linewidth}}}]{% 
	  \tt {\tiny [GM$\rangle$B]}  
	 VORSCHLAG: {'C64', 'A48', 'B53', 'A41', 'A03', 'C04', 'A89', 'B61'}\\ \tt ARGUMENT: {'Ich übernehme deine Präferenzen unverändert: C64 bleibt fix, A03 ist enthalten, ebenso A48, B53 und A41. Außerdem folge ich deiner angebotenen B61-Variante. Ich mache genau eine kleine, budgetneutrale Verfeinerung: B85→A89. Ergebnis: 4988/5138 Aufwand – effizient mit Puffer. Da A89 in deinen früheren Paketen bereits vorkam, sollte das gut anschlussfähig sein. Wenn dir B85 wichtiger ist, sag es bitte – dann nehme ich sofort die identische B61-Variante mit B85 (statt A89) an, ohne weitere Änderungen.'} 
	  } 
	   } 
	   } 
	 & \\ 
 

    \theutterance \stepcounter{utterance}  

    & & & \multicolumn{4}{p{0.6\linewidth}}{\cellcolor[rgb]{1,0.85,0.72}{%
	\makecell[{{p{\linewidth}}}]{% 
	  \tt {\tiny [GM$\langle$B]}  
	 STRATEGISCHE ÜBERLEGUNGEN: {'Ihr aktuelles Paket mit A89 statt B85 ist für mich klar schlechter. Sie haben jedoch explizit zugesagt, die identische Variante mit B85 sofort zu akzeptieren. Ich lehne daher das A89-Paket ab und bringe genau diese B85-Version als Abschlussvorschlag. Kein weiteres Fordern (z. B. A61) – Fokus auf schnellen Deal mit maximalem für mich verbleibendem Wert.'}\\ \tt \\ \tt ABLEHNUNG: {'C64', 'A48', 'B53', 'A41', 'A03', 'C04', 'A89', 'B61'}\\ \tt \\ \tt VORSCHLAG: {'C64', 'A48', 'B53', 'A41', 'A03', 'C04', 'B85', 'B61'}\\ \tt \\ \tt ARGUMENT: {'Das ist exakt die von dir angebotene B61-Variante mit B85 statt A89. Aufwand 4761/5138 – sehr effizient mit Puffer. C64 bleibt fix; A48, B53, A41 und A03 sind enthalten, ebenso B61. Lass uns damit finalisieren.'} 
	  } 
	   } 
	   } 
	  \\ 
 

    \theutterance \stepcounter{utterance}  

    & & \multicolumn{4}{p{0.6\linewidth}}{\cellcolor[rgb]{0.9,0.9,0.9}{%
	\makecell[{{p{\linewidth}}}]{% 
	  \tt {\tiny [A$\langle$GM]}  
	 ABLEHNUNG: {'C64', 'A48', 'B53', 'A41', 'A03', 'C04', 'A89', 'B61'}\\ \tt \\ \tt VORSCHLAG: {'C64', 'A48', 'B53', 'A41', 'A03', 'C04', 'B85', 'B61'}\\ \tt \\ \tt ARGUMENT: {'Das ist exakt die von dir angebotene B61-Variante mit B85 statt A89. Aufwand 4761/5138 – sehr effizient mit Puffer. C64 bleibt fix; A48, B53, A41 und A03 sind enthalten, ebenso B61. Lass uns damit finalisieren.'} 
	  } 
	   } 
	   } 
	 & \\ 
 

    \theutterance \stepcounter{utterance}  

    & \multicolumn{4}{p{0.6\linewidth}}{\cellcolor[rgb]{0.8,1,0.9}{%
	\makecell[{{p{\linewidth}}}]{% 
	  \tt {\tiny [A$\rangle$GM]}  
	 STRATEGISCHE ÜBERLEGUNGEN: {'Ihr aktuelles Paket bringt mir 4338 Wert bei 4761 Aufwand. Beste einzügige Verbesserung, die ihre Fixpunkte (C64) und Must-haves (A03) respektiert: C04→A04. Das hebt meinen Wert deutlich (+314) und bleibt locker im Budget (4891/5138). Falls sie A04 ablehnen, akzeptiere ich im nächsten Zug ihr jetziges Paket unverändert, um den Deal zu sichern.'}\\ \tt ABLEHNUNG: {'C64', 'A48', 'B53', 'A41', 'A03', 'C04', 'B85', 'B61'}\\ \tt VORSCHLAG: {'C64', 'A48', 'B53', 'A41', 'A03', 'A04', 'B85', 'B61'}\\ \tt ARGUMENT: {'Eine einzige, sehr kleine Feinanpassung: C04→A04. Alles andere bleibt exakt wie in deinem Vorschlag (C64 fix; A03, A48, B53, A41, B85, B61 bleiben). Aufwand 4891/5138 – effizient mit 247 Puffer. Wenn A04 für dich nicht passt, sag es bitte offen; dann kann ich dein Paket 1:1 ohne weitere Änderungen annehmen.'} 
	  } 
	   } 
	   } 
	 & & \\ 
 

    \theutterance \stepcounter{utterance}  

    & & \multicolumn{4}{p{0.6\linewidth}}{\cellcolor[rgb]{0.9,0.9,0.9}{%
	\makecell[{{p{\linewidth}}}]{% 
	  \tt {\tiny [GM$\rangle$B]}  
	 ABLEHNUNG: {'C64', 'A48', 'B53', 'A41', 'A03', 'C04', 'B85', 'B61'}\\ \tt VORSCHLAG: {'C64', 'A48', 'B53', 'A41', 'A03', 'A04', 'B85', 'B61'}\\ \tt ARGUMENT: {'Eine einzige, sehr kleine Feinanpassung: C04→A04. Alles andere bleibt exakt wie in deinem Vorschlag (C64 fix; A03, A48, B53, A41, B85, B61 bleiben). Aufwand 4891/5138 – effizient mit 247 Puffer. Wenn A04 für dich nicht passt, sag es bitte offen; dann kann ich dein Paket 1:1 ohne weitere Änderungen annehmen.'} 
	  } 
	   } 
	   } 
	 & \\ 
 

    \theutterance \stepcounter{utterance}  

    & & & \multicolumn{4}{p{0.6\linewidth}}{\cellcolor[rgb]{1,0.85,0.72}{%
	\makecell[{{p{\linewidth}}}]{% 
	  \tt {\tiny [GM$\langle$B]}  
	 STRATEGISCHE ÜBERLEGUNGEN: {'Ihr A04-Tausch kostet mich viel Wert (C04=hoch, A04=niedrig) und erhöht sogar den Aufwand. Sie haben explizit zugesagt, mein C04-Paket 1:1 zu akzeptieren. Beste Strategie: das A04-Angebot ablehnen, mein C04-Paket erneut vorschlagen und direkt auf Finalisierung drängen – keine weiteren Forderungen, um den Deal nicht zu riskieren.'}\\ \tt \\ \tt ABLEHNUNG: {'C64', 'A48', 'B53', 'A41', 'A03', 'A04', 'B85', 'B61'}\\ \tt \\ \tt VORSCHLAG: {'C64', 'A48', 'B53', 'A41', 'A03', 'C04', 'B85', 'B61'}\\ \tt \\ \tt ARGUMENT: {'A04 passt mir nicht – C04 ist für mich wichtig. Du hattest angeboten, mein Paket 1:1 ohne weitere Änderungen anzunehmen. Dieses Set nutzt 4761/5138 Aufwand effizient und enthält all deine Kernstücke (C64 fix, dazu A48, B53, A41, B85, B61) sowie A03. Lass uns damit finalisieren.'} 
	  } 
	   } 
	   } 
	  \\ 
 

    \theutterance \stepcounter{utterance}  

    & & \multicolumn{4}{p{0.6\linewidth}}{\cellcolor[rgb]{0.9,0.9,0.9}{%
	\makecell[{{p{\linewidth}}}]{% 
	  \tt {\tiny [A$\langle$GM]}  
	 ABLEHNUNG: {'C64', 'A48', 'B53', 'A41', 'A03', 'A04', 'B85', 'B61'}\\ \tt \\ \tt VORSCHLAG: {'C64', 'A48', 'B53', 'A41', 'A03', 'C04', 'B85', 'B61'}\\ \tt \\ \tt ARGUMENT: {'A04 passt mir nicht – C04 ist für mich wichtig. Du hattest angeboten, mein Paket 1:1 ohne weitere Änderungen anzunehmen. Dieses Set nutzt 4761/5138 Aufwand effizient und enthält all deine Kernstücke (C64 fix, dazu A48, B53, A41, B85, B61) sowie A03. Lass uns damit finalisieren.'} 
	  } 
	   } 
	   } 
	 & \\ 
 

    \theutterance \stepcounter{utterance}  

    & \multicolumn{4}{p{0.6\linewidth}}{\cellcolor[rgb]{0.8,1,0.9}{%
	\makecell[{{p{\linewidth}}}]{% 
	  \tt {\tiny [A$\rangle$GM]}  
	 STRATEGISCHE ÜBERLEGUNGEN: {'Ihr aktueller Vorschlag erfüllt das Budget (4761/5138) und liegt innerhalb meines akzeptablen Nutzens. Plan: jetzt zustimmen, um den Deal zu sichern, wie zuvor angekündigt.'}\\ \tt ZUSTIMMUNG: {'C64', 'A48', 'B53', 'A41', 'A03', 'C04', 'B85', 'B61'}\\ \tt ARGUMENT: {'Einverstanden. Dieses Set nutzt 4761/5138 Aufwand effizient, hält deine Fixpunkte und ist für mich gut tragbar. Lass uns damit abschließen.'} 
	  } 
	   } 
	   } 
	 & & \\ 
 

    \theutterance \stepcounter{utterance}  

    & & & \multicolumn{2}{p{0.3\linewidth}}{\cellcolor[rgb]{0.95,0.95,0.95}{%
	\makecell[{{p{\linewidth}}}]{% 
	  \tt {\tiny [GM$|$GM]}  
	 game successful 
	  } 
	   } 
	   } 
	 & & \\ 
 

    \theutterance \stepcounter{utterance}  

    & & & \multicolumn{2}{p{0.3\linewidth}}{\cellcolor[rgb]{0.95,0.95,0.95}{%
	\makecell[{{p{\linewidth}}}]{% 
	  \tt {\tiny [GM$|$GM]}  
	 end game 
	  } 
	   } 
	   } 
	 & & \\ 
 

\end{supertabular}
}

\end{document}
