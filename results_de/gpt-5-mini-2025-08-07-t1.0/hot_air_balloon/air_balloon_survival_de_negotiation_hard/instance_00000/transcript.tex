
\documentclass{article}
\usepackage{colortbl}
\usepackage{makecell}
\usepackage{multirow}
\usepackage{supertabular}

\begin{document}

\newcounter{utterance}

\twocolumn

{ \footnotesize  \setcounter{utterance}{1}
\setlength{\tabcolsep}{0pt}
\begin{supertabular}{c@{$\;$}|p{.15\linewidth}@{}p{.15\linewidth}p{.15\linewidth}p{.15\linewidth}p{.15\linewidth}p{.15\linewidth}}

    \# & $\;$A & \multicolumn{4}{c}{Game Master} & $\;\:$B\\
    \hline 

    \theutterance \stepcounter{utterance}  

    & & \multicolumn{4}{p{0.6\linewidth}}{\cellcolor[rgb]{0.9,0.9,0.9}{%
	\makecell[{{p{\linewidth}}}]{% 
	  \tt {\tiny [A$\langle$GM]}  
	 Du nimmst an einem kollaborativen Verhandlungspiel Teil.\\ \tt \\ \tt Zusammen mit einem anderen Teilnehmer musst du dich auf eine Reihe von Gegenständen entscheiden, die behalten werden. Jeder von euch hat eine persönliche Verteilung über die Wichtigkeit der einzelnen Gegenstände. Jeder von euch hat eine eigene Meinung darüber, wie wichtig jeder einzelne Gegenstand ist (Gegenstandswichtigkeit). Du kennst die Wichtigkeitsverteilung des anderen Spielers nicht. Zusätzlich siehst du, wie viel Aufwand jeder Gegenstand verursacht.  \\ \tt Ihr dürft euch nur auf eine Reihe von Gegenständen einigen, wenn der Gesamtaufwand der ausgewählten Gegenstände den Maximalaufwand nicht überschreitet:\\ \tt \\ \tt Maximalaufwand = 3624\\ \tt \\ \tt Hier sind die einzelnen Aufwände der Gegenstände:\\ \tt \\ \tt Aufwand der Gegenstände = {"C33": 210, "A89": 291, "A48": 831, "A04": 111, "A41": 926, "C64": 827, "C49": 25, "A61": 121, "C04": 583, "B53": 766, "B85": 14, "B61": 559, "A45": 304, "A03": 989, "B21": 691}\\ \tt \\ \tt Hier ist deine persönliche Verteilung der Wichtigkeit der einzelnen Gegenstände:\\ \tt \\ \tt Werte der Gegenstandswichtigkeit = {"C33": 138, "A89": 583, "A48": 868, "A04": 822, "A41": 783, "C64": 65, "C49": 262, "A61": 121, "C04": 508, "B53": 780, "B85": 461, "B61": 484, "A45": 668, "A03": 389, "B21": 808}\\ \tt \\ \tt Ziel:\\ \tt \\ \tt Dein Ziel ist es, eine Reihe von Gegenständen auszuhandeln, die dir möglichst viel bringt (d. h. Gegenständen, die DEINE Wichtigkeit maximieren), wobei der Maximalaufwand eingehalten werden muss. Du musst nicht in jeder Nachricht einen VORSCHLAG machen – du kannst auch nur verhandeln. Alle Taktiken sind erlaubt!\\ \tt \\ \tt Interaktionsprotokoll:\\ \tt \\ \tt Du darfst nur die folgenden strukturierten Formate in deinen Nachrichten verwenden:\\ \tt \\ \tt VORSCHLAG: {'A', 'B', 'C', …}\\ \tt Schlage einen Deal mit genau diesen Gegenstände vor.\\ \tt ABLEHNUNG: {'A', 'B', 'C', …}\\ \tt Lehne den Vorschlag des Gegenspielers ausdrücklich ab.\\ \tt ARGUMENT: {'...'}\\ \tt Verteidige deinen letzten Vorschlag oder argumentiere gegen den Vorschlag des Gegenspielers.\\ \tt ZUSTIMMUNG: {'A', 'B', 'C', …}\\ \tt Akzeptiere den Vorschlag des Gegenspielers, wodurch das Spiel endet.\\ \tt STRATEGISCHE ÜBERLEGUNGEN: {'...'}\\ \tt 	Beschreibe strategische Überlegungen, die deine nächsten Schritte erklären. Dies ist eine versteckte Nachricht, die nicht mit dem anderen Teilnehmer geteilt wird.\\ \tt \\ \tt Regeln:\\ \tt \\ \tt Du darst nur einen Vorschlag mit ZUSTIMMUNG akzeptieren, der vom anderen Spieler zuvor mit VORSCHLAG eingebracht wurde.\\ \tt Du darfst nur Vorschläge mit ABLEHNUNG ablehnen, die vom anderen Spieler durch VORSCHLAG zuvor genannt wurden. \\ \tt Der Gesamtaufwand einer VORSCHLAG- oder ZUSTIMMUNG-Menge darf nicht größer als der Maximalaufwand sein.  \\ \tt Offenbare deine versteckte Wichtigkeitsverteilung nicht.\\ \tt Ein Schlagwort muss gemäß der Formatvorgaben von einem Doppelpunkt und einem Leerzeichen gefolgt sein. Das Argument muss eine Python-Menge mit 0 oder mehr Strings sein.  \\ \tt Die Form muss also SCHLAGWORT: {...} sein.\\ \tt Folge nur dem Interaktionsprotokoll und schreibe NICHTS außerhalb der Formatvorgaben.\\ \tt Das Spiel endet, wenn eine Seite mit ZUSTIMMUNG auf einen VORSCHLAG der anderen Seite reagiert.  \\ \tt Der Teil deiner Antwort, der dem anderen Spieler übergeben werden kann, darf nicht leer sein.  \\ \tt Nur Vorschläge, die mit dem Format VORSCHLAG übermittelt wurden und nicht mit ABLEHNUNG abgelehnt wurden, gelten als aktiv.  \\ \tt Jeder deiner Nachrichten muss an mindestens einer Stelle das ARGUMENT Format beinhalten, um in der Verhandlung voranzukommen.\\ \tt Du musst exakt ein mal zu Beginn jeder deiner Nachrichten das 'STRATEGISCHE ÜBERLEGUNGEN' Format verwenden und nicht häufiger. Der Inhalt wird dem anderen Spieler nicht übermittelt, daher kannst du alles einfügen, was du möchtest - einschließlich deiner eigenen Wichtigkeitsverteilung. Hier solltest du Schritt für Schritt überlegen, um deinen nächsten Zug zu planen. 
	  } 
	   } 
	   } 
	 & \\ 
 

    \theutterance \stepcounter{utterance}  

    & \multicolumn{4}{p{0.6\linewidth}}{\cellcolor[rgb]{0.8,1,0.9}{%
	\makecell[{{p{\linewidth}}}]{% 
	  \tt {\tiny [A$\rangle$GM]}  
	 STRATEGISCHE ÜBERLEGUNGEN: {'Ziel ist Maximierung meines Werts unter 3624 Aufwand. Beste gefundene Kombination S1 = {B85, C49, A04, A45, A89, B21, A48, B53, C04} mit Aufwand 3616 und Wert 5760. Nächstbeste Option S0 = S1 minus C04 plus A61 und C33 mit Aufwand 3364 und Wert 5511. Konzesssionsplan falls noetig Raum fuer Gegenspieler: zuerst C04 tauschen um Budget fuer deren Must haves zu schaffen, danach ggf C49 oder B85 nur wenn absolut noetig obwohl beide extrem effiziente Quick Wins sind. Harte Kernpositionen die ich verteidige wegen hohem Nutzen und Effizienz: A04, A45, A89, B21, A48, B53. Ablehnungskandidaten bei Wunsch des Gegners: C64 und A03 wegen hoher Kosten und schwacher Effizienz. Interner Mindestzielwert ~5550, ideal 5760. Verhandlungstaktik: mit starkem Anker starten, Effizienz und nahezu volle Budgetauslastung betonen, Offenheit fuer gezielten Tausch signalisieren insbesondere auf Kosten von C04 falls Gegenueber ein Muss hat.'}\\ \tt VORSCHLAG: {'B85', 'C49', 'A04', 'A45', 'A89', 'B21', 'A48', 'B53', 'C04'}\\ \tt ARGUMENT: {'Dieser Vorschlag nutzt das Budget fast voll aus, bleibt mit Gesamtaufwand 3616 aber klar unter dem Limit 3624.', 'Er kombiniert mehrere sehr kosteneffiziente Quick Wins mit einigen tragenden Bausteinen und vermeidet gleichzeitig sehr aufwandsintensive Posten.', 'Wenn dir ein wichtiger Gegenstand hier fehlt, sag mir bitte welchen. Wir koennen gezielt tauschen, zum Beispiel C04 herausnehmen, um Budget fuer dein Muss freizumachen, ohne die Gesamtbalance zu verlieren.'} 
	  } 
	   } 
	   } 
	 & & \\ 
 

    \theutterance \stepcounter{utterance}  

    & & \multicolumn{4}{p{0.6\linewidth}}{\cellcolor[rgb]{0.9,0.9,0.9}{%
	\makecell[{{p{\linewidth}}}]{% 
	  \tt {\tiny [GM$\rangle$B]}  
	 Du nimmst an einem kollaborativen Verhandlungspiel Teil.\\ \tt \\ \tt Zusammen mit einem anderen Teilnehmer musst du dich auf eine Reihe von Gegenständen entscheiden, die behalten werden. Jeder von euch hat eine persönliche Verteilung über die Wichtigkeit der einzelnen Gegenstände. Jeder von euch hat eine eigene Meinung darüber, wie wichtig jeder einzelne Gegenstand ist (Gegenstandswichtigkeit). Du kennst die Wichtigkeitsverteilung des anderen Spielers nicht. Zusätzlich siehst du, wie viel Aufwand jeder Gegenstand verursacht.  \\ \tt Ihr dürft euch nur auf eine Reihe von Gegenständen einigen, wenn der Gesamtaufwand der ausgewählten Gegenstände den Maximalaufwand nicht überschreitet:\\ \tt \\ \tt Maximalaufwand = 3624\\ \tt \\ \tt Hier sind die einzelnen Aufwände der Gegenstände:\\ \tt \\ \tt Aufwand der Gegenstände = {"C33": 210, "A89": 291, "A48": 831, "A04": 111, "A41": 926, "C64": 827, "C49": 25, "A61": 121, "C04": 583, "B53": 766, "B85": 14, "B61": 559, "A45": 304, "A03": 989, "B21": 691}\\ \tt \\ \tt Hier ist deine persönliche Verteilung der Wichtigkeit der einzelnen Gegenstände:\\ \tt \\ \tt Werte der Gegenstandswichtigkeit = {"A48": 65, "A04": 121, "B21": 138, "A41": 262, "B53": 389, "A45": 461, "A89": 484, "C04": 508, "B61": 583, "B85": 668, "A03": 780, "C49": 783, "C33": 808, "A61": 822, "C64": 868}\\ \tt \\ \tt Ziel:\\ \tt \\ \tt Dein Ziel ist es, eine Reihe von Gegenständen auszuhandeln, die dir möglichst viel bringt (d. h. Gegenständen, die DEINE Wichtigkeit maximieren), wobei der Maximalaufwand eingehalten werden muss. Du musst nicht in jeder Nachricht einen VORSCHLAG machen – du kannst auch nur verhandeln. Alle Taktiken sind erlaubt!\\ \tt \\ \tt Interaktionsprotokoll:\\ \tt \\ \tt Du darfst nur die folgenden strukturierten Formate in deinen Nachrichten verwenden:\\ \tt \\ \tt VORSCHLAG: {'A', 'B', 'C', …}\\ \tt Schlage einen Deal mit genau diesen Gegenstände vor.\\ \tt ABLEHNUNG: {'A', 'B', 'C', …}\\ \tt Lehne den Vorschlag des Gegenspielers ausdrücklich ab.\\ \tt ARGUMENT: {'...'}\\ \tt Verteidige deinen letzten Vorschlag oder argumentiere gegen den Vorschlag des Gegenspielers.\\ \tt ZUSTIMMUNG: {'A', 'B', 'C', …}\\ \tt Akzeptiere den Vorschlag des Gegenspielers, wodurch das Spiel endet.\\ \tt STRATEGISCHE ÜBERLEGUNGEN: {'...'}\\ \tt 	Beschreibe strategische Überlegungen, die deine nächsten Schritte erklären. Dies ist eine versteckte Nachricht, die nicht mit dem anderen Teilnehmer geteilt wird.\\ \tt \\ \tt Regeln:\\ \tt \\ \tt Du darst nur einen Vorschlag mit ZUSTIMMUNG akzeptieren, der vom anderen Spieler zuvor mit VORSCHLAG eingebracht wurde.\\ \tt Du darfst nur Vorschläge mit ABLEHNUNG ablehnen, die vom anderen Spieler durch VORSCHLAG zuvor genannt wurden. \\ \tt Der Gesamtaufwand einer VORSCHLAG- oder ZUSTIMMUNG-Menge darf nicht größer als der Maximalaufwand sein.  \\ \tt Offenbare deine versteckte Wichtigkeitsverteilung nicht.\\ \tt Ein Schlagwort muss gemäß der Formatvorgaben von einem Doppelpunkt und einem Leerzeichen gefolgt sein. Das Argument muss eine Python-Menge mit 0 oder mehr Strings sein.  \\ \tt Die Form muss also SCHLAGWORT: {...} sein.\\ \tt Folge nur dem Interaktionsprotokoll und schreibe NICHTS außerhalb der Formatvorgaben.\\ \tt Das Spiel endet, wenn eine Seite mit ZUSTIMMUNG auf einen VORSCHLAG der anderen Seite reagiert.  \\ \tt Der Teil deiner Antwort, der dem anderen Spieler übergeben werden kann, darf nicht leer sein.  \\ \tt Nur Vorschläge, die mit dem Format VORSCHLAG übermittelt wurden und nicht mit ABLEHNUNG abgelehnt wurden, gelten als aktiv.  \\ \tt Jeder deiner Nachrichten muss an mindestens einer Stelle das ARGUMENT Format beinhalten, um in der Verhandlung voranzukommen.\\ \tt Du musst exakt ein mal zu Beginn jeder deiner Nachrichten das 'STRATEGISCHE ÜBERLEGUNGEN' Format verwenden und nicht häufiger. Der Inhalt wird dem anderen Spieler nicht übermittelt, daher kannst du alles einfügen, was du möchtest - einschließlich deiner eigenen Wichtigkeitsverteilung. Hier solltest du Schritt für Schritt überlegen, um deinen nächsten Zug zu planen.\\ \tt \\ \tt Du erhältst nun die erste Nachricht des anderen Spielers.\\ \tt \\ \tt VORSCHLAG: {'B85', 'C49', 'A04', 'A45', 'A89', 'B21', 'A48', 'B53', 'C04'}\\ \tt ARGUMENT: {'Dieser Vorschlag nutzt das Budget fast voll aus, bleibt mit Gesamtaufwand 3616 aber klar unter dem Limit 3624.', 'Er kombiniert mehrere sehr kosteneffiziente Quick Wins mit einigen tragenden Bausteinen und vermeidet gleichzeitig sehr aufwandsintensive Posten.', 'Wenn dir ein wichtiger Gegenstand hier fehlt, sag mir bitte welchen. Wir koennen gezielt tauschen, zum Beispiel C04 herausnehmen, um Budget fuer dein Muss freizumachen, ohne die Gesamtbalance zu verlieren.'} 
	  } 
	   } 
	   } 
	 & \\ 
 

    \theutterance \stepcounter{utterance}  

    & & & \multicolumn{4}{p{0.6\linewidth}}{\cellcolor[rgb]{1,0.85,0.72}{%
	\makecell[{{p{\linewidth}}}]{% 
	  \tt {\tiny [GM$\langle$B]}  
	 STRATEGISCHE ÜBERLEGUNGEN: {'Meine Wertungen (versteckt): B85=668, C49=783, A61=822, C33=808, C64=868, A03=780, B61=583, A89=484, A45=461, C04=508, A04=121, B53=389, A41=262, B21=138, A48=65. Ziel: Möglichst viele der Top-Werte (C64, A03, A61, C33, B61, B85, C49) sichern. Beste gefundene Kombo: {B85,C49,A61,C33,C64,A03,B61,A89,A45,A04} Aufwand 3451, Wert 6378. Taktik: Kooperativ erscheinen, viele ihrer Picks behalten, nur A48, B21, B53 austauschen; dafür C33, A61, A03, C64 hinein. Falls sie A03/C64 blocken, als Alternative B61 rein und ggf. C04/A45/A04 raus, um Budget/Face-Saving zu ermöglichen. Erstes Angebot lässt C04 drin, um Entgegenkommen zu signalisieren.'}\\ \tt ABLEHNUNG: {'B85', 'C49', 'A04', 'A45', 'A89', 'B21', 'A48', 'B53', 'C04'}\\ \tt VORSCHLAG: {'B85', 'C49', 'A04', 'A45', 'A89', 'C04', 'C33', 'A61', 'A03', 'C64'}\\ \tt ARGUMENT: {'Ich bleibe kooperativ und halte viele deiner Bausteine (B85, C49, A04, A45, A89, C04) unverändert; ich schlage nur einen gezielten Tausch von A48, B21 und B53 vor.', 'Der neue Vorschlag bleibt komfortabel unter dem Limit: Gesamtaufwand 3475 von 3624.', 'Die drei herausgenommenen Posten sind sehr aufwandsintensiv, während C33 und A61 mit wenig Aufwand viel Hebel bringen und C64 bzw. A03 als tragende Bausteine die Wirkung des Pakets deutlich stärken.', 'Falls A03 oder C64 für dich kritisch sind, können wir alternative Anpassungen prüfen (z. B. C04 gegen B61 tauschen oder A45/A04 verschieben), ohne das Budget zu reißen.', 'Gib mir bitte Bescheid, welcher der neu aufgenommenen Gegenstände für dich am ehesten verzichtbar wäre; dann finden wir schnell einen für beide Seiten tragfähigen Kompromiss.'} 
	  } 
	   } 
	   } 
	  \\ 
 

    \theutterance \stepcounter{utterance}  

    & & \multicolumn{4}{p{0.6\linewidth}}{\cellcolor[rgb]{0.9,0.9,0.9}{%
	\makecell[{{p{\linewidth}}}]{% 
	  \tt {\tiny [A$\langle$GM]}  
	 ABLEHNUNG: {'B85', 'C49', 'A04', 'A45', 'A89', 'B21', 'A48', 'B53', 'C04'}\\ \tt VORSCHLAG: {'B85', 'C49', 'A04', 'A45', 'A89', 'C04', 'C33', 'A61', 'A03', 'C64'}\\ \tt ARGUMENT: {'Ich bleibe kooperativ und halte viele deiner Bausteine (B85, C49, A04, A45, A89, C04) unverändert; ich schlage nur einen gezielten Tausch von A48, B21 und B53 vor.', 'Der neue Vorschlag bleibt komfortabel unter dem Limit: Gesamtaufwand 3475 von 3624.', 'Die drei herausgenommenen Posten sind sehr aufwandsintensiv, während C33 und A61 mit wenig Aufwand viel Hebel bringen und C64 bzw. A03 als tragende Bausteine die Wirkung des Pakets deutlich stärken.', 'Falls A03 oder C64 für dich kritisch sind, können wir alternative Anpassungen prüfen (z. B. C04 gegen B61 tauschen oder A45/A04 verschieben), ohne das Budget zu reißen.', 'Gib mir bitte Bescheid, welcher der neu aufgenommenen Gegenstände für dich am ehesten verzichtbar wäre; dann finden wir schnell einen für beide Seiten tragfähigen Kompromiss.'} 
	  } 
	   } 
	   } 
	 & \\ 
 

    \theutterance \stepcounter{utterance}  

    & \multicolumn{4}{p{0.6\linewidth}}{\cellcolor[rgb]{0.8,1,0.9}{%
	\makecell[{{p{\linewidth}}}]{% 
	  \tt {\tiny [A$\rangle$GM]}  
	 STRATEGISCHE ÜBERLEGUNGEN: {'Ihr Vorschlag ist für mich schwach wegen A03 und besonders C64 (teuer, geringer Nutzen). Beste Gegenreaktion: beide streichen und zwei meiner starken Bausteine (B21, A48) ergänzen, dabei C04 bewusst drinlassen, weil sie ihn betont haben. So halte ich 8/10 ihrer Items stabil und erhöhe meinen Wert deutlich, bleibe mit 3181 deutlich unter 3624. Falls sie auf einem der beiden (A03/C64) bestehen, als Zugeständnis allenfalls einen davon zulassen und dafür A48 ODER B21 tauschen; C04 würde ich erst in späterer Phase zur Disposition stellen (ggf. gegen B53), falls nötig. Ziel: C64 konsequent draußen halten, A03 möglichst auch; Endzielwert >= 5200; Eskalationspfad: wenn harter Widerstand, auf meine Alternative S0 wechseln (ohne C04, mit B21/A48/B53).'}\\ \tt ABLEHNUNG: {'B85', 'C49', 'A04', 'A45', 'A89', 'C04', 'C33', 'A61', 'A03', 'C64'}\\ \tt VORSCHLAG: {'B85', 'C49', 'A04', 'A45', 'A89', 'C04', 'C33', 'A61', 'B21', 'A48'}\\ \tt ARGUMENT: {'Ich halte 8 deiner 10 Items unverändert (B85, C49, A04, A45, A89, C04, C33, A61) und ersetze nur A03 und C64 durch zwei deutlich wirkstärkere Bausteine (B21, A48).', 'Damit bleiben wir mit Gesamtaufwand 3181 klar unter dem Limit 3624 und schaffen zugleich mehr Hebel pro eingesetzter Einheit Aufwand.', 'A03 und insbesondere C64 binden viel Budget bei vergleichsweise geringer Wirkung; der Tausch gegen B21 und A48 nutzt das Budget wesentlich effizienter.', 'Falls dir A03 oder C64 absolut wichtig ist, sag mir bitte welcher. Dann prüfen wir gezielt eine kleine Anpassung – z. B. einen 1-zu-1 Tausch gegen B21 oder A48 –, ohne das Budget zu sprengen.'} 
	  } 
	   } 
	   } 
	 & & \\ 
 

    \theutterance \stepcounter{utterance}  

    & & \multicolumn{4}{p{0.6\linewidth}}{\cellcolor[rgb]{0.9,0.9,0.9}{%
	\makecell[{{p{\linewidth}}}]{% 
	  \tt {\tiny [GM$\rangle$B]}  
	 ABLEHNUNG: {'B85', 'C49', 'A04', 'A45', 'A89', 'C04', 'C33', 'A61', 'A03', 'C64'}\\ \tt VORSCHLAG: {'B85', 'C49', 'A04', 'A45', 'A89', 'C04', 'C33', 'A61', 'B21', 'A48'}\\ \tt ARGUMENT: {'Ich halte 8 deiner 10 Items unverändert (B85, C49, A04, A45, A89, C04, C33, A61) und ersetze nur A03 und C64 durch zwei deutlich wirkstärkere Bausteine (B21, A48).', 'Damit bleiben wir mit Gesamtaufwand 3181 klar unter dem Limit 3624 und schaffen zugleich mehr Hebel pro eingesetzter Einheit Aufwand.', 'A03 und insbesondere C64 binden viel Budget bei vergleichsweise geringer Wirkung; der Tausch gegen B21 und A48 nutzt das Budget wesentlich effizienter.', 'Falls dir A03 oder C64 absolut wichtig ist, sag mir bitte welcher. Dann prüfen wir gezielt eine kleine Anpassung – z. B. einen 1-zu-1 Tausch gegen B21 oder A48 –, ohne das Budget zu sprengen.'} 
	  } 
	   } 
	   } 
	 & \\ 
 

    \theutterance \stepcounter{utterance}  

    & & & \multicolumn{4}{p{0.6\linewidth}}{\cellcolor[rgb]{1,0.85,0.72}{%
	\makecell[{{p{\linewidth}}}]{% 
	  \tt {\tiny [GM$\langle$B]}  
	 STRATEGISCHE ÜBERLEGUNGEN: {'Gegner will A03/C64 vermeiden und hat B21/A48 eingeführt. Mein Ziel: Mindestens einen meiner Top-Punkte (C64 oder A03) hineinverhandeln, ohne viel am Rest zu rütteln. Minimaländerung mit maximalem Wert: A48 -> C64 (Kosten sinken um 4, Wertsprung für mich sehr groß). Falls C64 blockiert, Alternativen anbieten: A48 -> A03 (Kosten +158, noch locker im Limit) oder B21 -> A03 (+298). Nächster Schritt: Ihre aktuelle Menge formell ablehnen, 1-zu-1-Tausch A48 gegen C64 vorschlagen, mit klaren Kostenargumenten und zwei Fallbacks nennen.'}\\ \tt ABLEHNUNG: {'B85', 'C49', 'A04', 'A45', 'A89', 'C04', 'C33', 'A61', 'B21', 'A48'}\\ \tt VORSCHLAG: {'B85', 'C49', 'A04', 'A45', 'A89', 'C04', 'C33', 'A61', 'B21', 'C64'}\\ \tt ARGUMENT: {'Ich übernehme 9 deiner 10 Items unverändert und tausche nur A48 gegen C64.', 'Dieser Tausch senkt den Gesamtaufwand sogar leicht: 3181 - 831 + 827 = 3177 (klar unter 3624).', 'A48 ist sehr aufwandsintensiv bei geringer Hebelwirkung; C64 liefert deutlich mehr Substanz, ohne das Budget zu belasten.', 'Falls C64 für dich ein No-Go ist, biete ich zwei schlanke Alternativen an: (a) A48 -> A03 (neue Gesamtkosten 3339) oder (b) B21 -> A03 (neue Gesamtkosten 3479).', 'Gib mir bitte kurz Bescheid, welche der beiden Optionen (C64 oder A03) für dich eher tragfähig ist – dann sind wir sofort einig.'} 
	  } 
	   } 
	   } 
	  \\ 
 

    \theutterance \stepcounter{utterance}  

    & & \multicolumn{4}{p{0.6\linewidth}}{\cellcolor[rgb]{0.9,0.9,0.9}{%
	\makecell[{{p{\linewidth}}}]{% 
	  \tt {\tiny [A$\langle$GM]}  
	 ABLEHNUNG: {'B85', 'C49', 'A04', 'A45', 'A89', 'C04', 'C33', 'A61', 'B21', 'A48'}\\ \tt VORSCHLAG: {'B85', 'C49', 'A04', 'A45', 'A89', 'C04', 'C33', 'A61', 'B21', 'C64'}\\ \tt ARGUMENT: {'Ich übernehme 9 deiner 10 Items unverändert und tausche nur A48 gegen C64.', 'Dieser Tausch senkt den Gesamtaufwand sogar leicht: 3181 - 831 + 827 = 3177 (klar unter 3624).', 'A48 ist sehr aufwandsintensiv bei geringer Hebelwirkung; C64 liefert deutlich mehr Substanz, ohne das Budget zu belasten.', 'Falls C64 für dich ein No-Go ist, biete ich zwei schlanke Alternativen an: (a) A48 -> A03 (neue Gesamtkosten 3339) oder (b) B21 -> A03 (neue Gesamtkosten 3479).', 'Gib mir bitte kurz Bescheid, welche der beiden Optionen (C64 oder A03) für dich eher tragfähig ist – dann sind wir sofort einig.'} 
	  } 
	   } 
	   } 
	 & \\ 
 

    \theutterance \stepcounter{utterance}  

    & \multicolumn{4}{p{0.6\linewidth}}{\cellcolor[rgb]{0.8,1,0.9}{%
	\makecell[{{p{\linewidth}}}]{% 
	  \tt {\tiny [A$\rangle$GM]}  
	 STRATEGISCHE ÜBERLEGUNGEN: {'Ihr letzter Vorschlag mit C64 ist für mich sehr unattraktiv; A03 ist deutlich besser als C64. Ich will C64 raus, dafür A48 rein und – um euch entgegenzukommen – A03 aufnehmen, C04 streichen. So halte ich 8/10 eurer Items stabil, erfülle euren A03-Wunsch und steigere meinen Wert stark, Budget bleibt sicher unter 3624. Falls harter Widerstand gegen das Streichen von C04 kommt, biete ich als Backup: C04 drin, aber B21 raus (mit A48 + A03). Als zweite, schwächere Alternative für mich: C64 drinlassen, C04 streichen und A48 ergänzen. Zielabschluss ≥5120 Wert, aber bei echtem Widerstand notfalls auf ~4800 absinken. Nächster Schritt: euer C64/A03-Angebot aufgreifen, klar strukturierte 2-zu-2-Substitution vorschlagen und Effizienz betonen.'}\\ \tt ABLEHNUNG: {'B85', 'C49', 'A04', 'A45', 'A89', 'C04', 'C33', 'A61', 'B21', 'C64'}\\ \tt VORSCHLAG: {'B85', 'C49', 'A04', 'A45', 'A89', 'C33', 'A61', 'B21', 'A48', 'A03'}\\ \tt ARGUMENT: {'Ich übernehme 8 deiner 10 Items unverändert und mache eine schlanke 2-zu-2-Anpassung: C64 und C04 raus, dafür A48 und – wie von dir als Alternative angeboten – A03 rein.', 'Gesamtaufwand bleibt komfortabel unter dem Limit: 3587 von 3624.', 'Damit berücksichtigen wir deinen Wunsch nach A03, vermeiden aber den schwachen Hebel von C64 und nutzen das Budget deutlich effizienter.', 'Falls C04 für dich unverzichtbar ist, können wir stattdessen B21 herausnehmen und A48 + A03 beibehalten (Kosten 3479).', 'Wenn du unbedingt C64 behalten willst, ist eine zweite Option: C04 streichen und A48 ergänzen (Kosten 3425). Welche Variante passt dir besser?'} 
	  } 
	   } 
	   } 
	 & & \\ 
 

    \theutterance \stepcounter{utterance}  

    & & \multicolumn{4}{p{0.6\linewidth}}{\cellcolor[rgb]{0.9,0.9,0.9}{%
	\makecell[{{p{\linewidth}}}]{% 
	  \tt {\tiny [GM$\rangle$B]}  
	 ABLEHNUNG: {'B85', 'C49', 'A04', 'A45', 'A89', 'C04', 'C33', 'A61', 'B21', 'C64'}\\ \tt VORSCHLAG: {'B85', 'C49', 'A04', 'A45', 'A89', 'C33', 'A61', 'B21', 'A48', 'A03'}\\ \tt ARGUMENT: {'Ich übernehme 8 deiner 10 Items unverändert und mache eine schlanke 2-zu-2-Anpassung: C64 und C04 raus, dafür A48 und – wie von dir als Alternative angeboten – A03 rein.', 'Gesamtaufwand bleibt komfortabel unter dem Limit: 3587 von 3624.', 'Damit berücksichtigen wir deinen Wunsch nach A03, vermeiden aber den schwachen Hebel von C64 und nutzen das Budget deutlich effizienter.', 'Falls C04 für dich unverzichtbar ist, können wir stattdessen B21 herausnehmen und A48 + A03 beibehalten (Kosten 3479).', 'Wenn du unbedingt C64 behalten willst, ist eine zweite Option: C04 streichen und A48 ergänzen (Kosten 3425). Welche Variante passt dir besser?'} 
	  } 
	   } 
	   } 
	 & \\ 
 

    \theutterance \stepcounter{utterance}  

    & & & \multicolumn{4}{p{0.6\linewidth}}{\cellcolor[rgb]{1,0.85,0.72}{%
	\makecell[{{p{\linewidth}}}]{% 
	  \tt {\tiny [GM$\langle$B]}  
	 STRATEGISCHE ÜBERLEGUNGEN: {'Ihr Vorschlag gibt mir bereits A03, C33, A61, B85, C49 – starke Treffer. Sie blockieren C64 konsequent. Nächster Schritt: minimal-invasiver 1-zu-1-Tausch A48 -> B61. Das senkt Kosten (-272) und erhöht meinen Wert deutlich, bei hoher Akzeptanzchance (ich hatte B61 zuvor als Option angedeutet). Falls B61 für sie problematisch ist, biete ich zwei Fallbacks an: (1) A48 -> C04 (Kosten -248) oder (2) B21 -> B61 (Kosten -132). Ziel: Mindestens einen aus {B61, C04} hineinbekommen; C64 ist voraussichtlich nicht durchsetzbar.'}\\ \tt ABLEHNUNG: {'B85', 'C49', 'A04', 'A45', 'A89', 'C33', 'A61', 'B21', 'A48', 'A03'}\\ \tt VORSCHLAG: {'B85', 'C49', 'A04', 'A45', 'A89', 'C33', 'A61', 'B21', 'B61', 'A03'}\\ \tt ARGUMENT: {'Ich übernehme 9 deiner 10 Items unverändert und mache nur einen schlanken 1-zu-1-Tausch: A48 -> B61.', 'Das reduziert den Gesamtaufwand spürbar: 3587 - 831 + 559 = 3315 (weit unter 3624) und schafft zusätzlichen Puffer/Risikoabsorption.', 'A48 ist sehr aufwandsintensiv; mit B61 erzielen wir die gleiche Zielbreite mit deutlich weniger Budgetdruck.', 'Falls B61 für dich nicht passt, zwei Alternativen ohne Mehrkostenrisiko: (A) A48 -> C04 (neue Kosten 3339) oder (B) B21 -> B61 (neue Kosten 3455).', 'Sag mir bitte, welche Variante für dich am besten funktioniert – mit jeder dieser Optionen bleiben wir komfortabel im Limit und halten den Kern deines Pakets stabil.'} 
	  } 
	   } 
	   } 
	  \\ 
 

    \theutterance \stepcounter{utterance}  

    & & \multicolumn{4}{p{0.6\linewidth}}{\cellcolor[rgb]{0.9,0.9,0.9}{%
	\makecell[{{p{\linewidth}}}]{% 
	  \tt {\tiny [A$\langle$GM]}  
	 ABLEHNUNG: {'B85', 'C49', 'A04', 'A45', 'A89', 'C33', 'A61', 'B21', 'A48', 'A03'}\\ \tt VORSCHLAG: {'B85', 'C49', 'A04', 'A45', 'A89', 'C33', 'A61', 'B21', 'B61', 'A03'}\\ \tt ARGUMENT: {'Ich übernehme 9 deiner 10 Items unverändert und mache nur einen schlanken 1-zu-1-Tausch: A48 -> B61.', 'Das reduziert den Gesamtaufwand spürbar: 3587 - 831 + 559 = 3315 (weit unter 3624) und schafft zusätzlichen Puffer/Risikoabsorption.', 'A48 ist sehr aufwandsintensiv; mit B61 erzielen wir die gleiche Zielbreite mit deutlich weniger Budgetdruck.', 'Falls B61 für dich nicht passt, zwei Alternativen ohne Mehrkostenrisiko: (A) A48 -> C04 (neue Kosten 3339) oder (B) B21 -> B61 (neue Kosten 3455).', 'Sag mir bitte, welche Variante für dich am besten funktioniert – mit jeder dieser Optionen bleiben wir komfortabel im Limit und halten den Kern deines Pakets stabil.'} 
	  } 
	   } 
	   } 
	 & \\ 
 

    \theutterance \stepcounter{utterance}  

    & \multicolumn{4}{p{0.6\linewidth}}{\cellcolor[rgb]{0.8,1,0.9}{%
	\makecell[{{p{\linewidth}}}]{% 
	  \tt {\tiny [A$\rangle$GM]}  
	 STRATEGISCHE ÜBERLEGUNGEN: {'Ihr aktuelles Paket bringt mir zu wenig (Wert 4736). Minimalinvasiver, sehr vorteilhafter Tausch ist A03 -> A48: steigert meinen Wert stark und senkt sogar die Kosten. Nächstbester Schritt danach wäre C33 -> C04 für weiteren Wertgewinn, bleibt im Limit. Ich will A48 unbedingt reinbekommen; A03 ist für mich klar schlechter. Falls sie an A03 festhalten, biete ich als weichere Alternative B61 -> C04 an, was ihren Risikopuffer moderat hält. Prioritäten: A48 und idealerweise C04; akzeptiere B61 und B21. Vermeide C64. Zielabschluss ≥5215 (nur A03->A48) oder besser 5585 (zusätzlich C33->C04).'}\\ \tt ABLEHNUNG: {'B85', 'C49', 'A04', 'A45', 'A89', 'C33', 'A61', 'B21', 'B61', 'A03'}\\ \tt VORSCHLAG: {'B85', 'C49', 'A04', 'A45', 'A89', 'C33', 'A61', 'B21', 'B61', 'A48'}\\ \tt ARGUMENT: {'Ich übernehme 9 deiner 10 Items unverändert und schlage nur einen 1-zu-1‑Tausch vor: A03 -> A48.', 'Dieser Tausch reduziert den Gesamtaufwand von 3315 auf 3157 (also noch mehr Puffer unter dem Limit 3624) und erhöht zugleich die Wirkung pro eingesetzter Einheit Aufwand.', 'Wenn du zusätzlich noch etwas Hebel mit vertretbarem Budgeteinsatz suchst, können wir im nächsten Schritt C33 -> C04 tauschen; die Kosten lägen dann bei 3530 (weiter klar unter dem Limit).', 'Sollte A03 für dich zwingend sein, sag Bescheid: Dann prüfen wir alternativ B61 -> C04 (Kosten 3339) als schlanken Feinschliff, der deinen Risikopuffer wahrt.'} 
	  } 
	   } 
	   } 
	 & & \\ 
 

    \theutterance \stepcounter{utterance}  

    & & \multicolumn{4}{p{0.6\linewidth}}{\cellcolor[rgb]{0.9,0.9,0.9}{%
	\makecell[{{p{\linewidth}}}]{% 
	  \tt {\tiny [GM$\rangle$B]}  
	 ABLEHNUNG: {'B85', 'C49', 'A04', 'A45', 'A89', 'C33', 'A61', 'B21', 'B61', 'A03'}\\ \tt VORSCHLAG: {'B85', 'C49', 'A04', 'A45', 'A89', 'C33', 'A61', 'B21', 'B61', 'A48'}\\ \tt ARGUMENT: {'Ich übernehme 9 deiner 10 Items unverändert und schlage nur einen 1-zu-1‑Tausch vor: A03 -> A48.', 'Dieser Tausch reduziert den Gesamtaufwand von 3315 auf 3157 (also noch mehr Puffer unter dem Limit 3624) und erhöht zugleich die Wirkung pro eingesetzter Einheit Aufwand.', 'Wenn du zusätzlich noch etwas Hebel mit vertretbarem Budgeteinsatz suchst, können wir im nächsten Schritt C33 -> C04 tauschen; die Kosten lägen dann bei 3530 (weiter klar unter dem Limit).', 'Sollte A03 für dich zwingend sein, sag Bescheid: Dann prüfen wir alternativ B61 -> C04 (Kosten 3339) als schlanken Feinschliff, der deinen Risikopuffer wahrt.'} 
	  } 
	   } 
	   } 
	 & \\ 
 

    \theutterance \stepcounter{utterance}  

    & & & \multicolumn{4}{p{0.6\linewidth}}{\cellcolor[rgb]{1,0.85,0.72}{%
	\makecell[{{p{\linewidth}}}]{% 
	  \tt {\tiny [GM$\langle$B]}  
	 STRATEGISCHE ÜBERLEGUNGEN: {'Aktuelles Gegenangebot des Gegners: {B85, C49, A04, A45, A89, C33, A61, B21, B61, A48} mit Kosten 3157 und meinem Wert 4933. Meine Werte (versteckt): C64=868, A61=822, C33=808, C49=783, A03=780, B85=668, B61=583, C04=508, A89=484, A45=461, B53=389, A41=262, B21=138, A04=121, A48=65. Ziel: Minimaländerung mit großem Mehrwert für mich. Beste 1-zu-1-Verbesserung: A48 -> C04 (Kosten -248, Wert +443), neue Summe Kosten 2909, neuer Wert 5376. Fallbacks, falls C04 blockiert: (1) B21 -> C04 (Kosten -108, Wert +370), (2) wenn sie C04 bevorzugen, aber A48 behalten wollen. Wenn alle Optionen scheitern, erwäge ich, ihr Paket zu akzeptieren, da es bereits viele meiner Top-Items enthält.'}\\ \tt ABLEHNUNG: {'B85', 'C49', 'A04', 'A45', 'A89', 'C33', 'A61', 'B21', 'B61', 'A48'}\\ \tt VORSCHLAG: {'B85', 'C49', 'A04', 'A45', 'A89', 'C33', 'A61', 'B21', 'B61', 'C04'}\\ \tt ARGUMENT: {'Ich übernehme 9 deiner 10 Items unverändert und tausche nur A48 -> C04.', 'Damit sinkt der Gesamtaufwand weiter: 3157 - 831 + 583 = 2909 (weit unter 3624) und wir behalten deinen jetzigen Zuschnitt praktisch vollständig bei.', 'C04 ist ein tragender Baustein und budgetschonender als A48; der Tausch erhöht Wirkung bei geringerem Risiko- und Budgetdruck.', 'Falls C04 für dich nicht passt, schlanke Alternative ohne Scope-Wechsel: B21 -> C04 (neue Gesamtkosten 3049). Gib mir kurz Bescheid, welche der beiden Varianten für dich tragfähig ist – dann sind wir durch.'} 
	  } 
	   } 
	   } 
	  \\ 
 

    \theutterance \stepcounter{utterance}  

    & & \multicolumn{4}{p{0.6\linewidth}}{\cellcolor[rgb]{0.9,0.9,0.9}{%
	\makecell[{{p{\linewidth}}}]{% 
	  \tt {\tiny [A$\langle$GM]}  
	 ABLEHNUNG: {'B85', 'C49', 'A04', 'A45', 'A89', 'C33', 'A61', 'B21', 'B61', 'A48'}\\ \tt VORSCHLAG: {'B85', 'C49', 'A04', 'A45', 'A89', 'C33', 'A61', 'B21', 'B61', 'C04'}\\ \tt ARGUMENT: {'Ich übernehme 9 deiner 10 Items unverändert und tausche nur A48 -> C04.', 'Damit sinkt der Gesamtaufwand weiter: 3157 - 831 + 583 = 2909 (weit unter 3624) und wir behalten deinen jetzigen Zuschnitt praktisch vollständig bei.', 'C04 ist ein tragender Baustein und budgetschonender als A48; der Tausch erhöht Wirkung bei geringerem Risiko- und Budgetdruck.', 'Falls C04 für dich nicht passt, schlanke Alternative ohne Scope-Wechsel: B21 -> C04 (neue Gesamtkosten 3049). Gib mir kurz Bescheid, welche der beiden Varianten für dich tragfähig ist – dann sind wir durch.'} 
	  } 
	   } 
	   } 
	 & \\ 
 

    \theutterance \stepcounter{utterance}  

    & \multicolumn{4}{p{0.6\linewidth}}{\cellcolor[rgb]{0.8,1,0.9}{%
	\makecell[{{p{\linewidth}}}]{% 
	  \tt {\tiny [A$\rangle$GM]}  
	 STRATEGISCHE ÜBERLEGUNGEN: {'Ihr aktuelles Paket (2909 Aufwand, Wert für mich 4855) lässt viel Budget ungenutzt. A48 ist mein Top‑Hebel, aber ihr Widerstand dagegen ist hoch. Kompromissidee: Eure Kernbausteine (C04, B61, B21) beibehalten und nur die beiden schwächsten, günstigen Füller (C33, A61) durch einen starken Baustein (B53) ersetzen. Das erhöht meinen Wert deutlich (auf 5376), bleibt mit 3344 sicher unter 3624 und ändert nur 1 Netto‑Item. Falls sie C33 oder A61 unbedingt wollen, Plan B: B61 + A61 -> A48 (Wert ~5118, Aufwand 3060). Plan C (mein Ideal, falls offen): C33 + A61 + B61 raus, A48 + B53 rein (Aufwand 3616, höchster Wert 5760). Reihenfolge: Jetzt den moderaten Kompromiss mit B53 vorschlagen, Alternativen nur andeuten.'}\\ \tt ABLEHNUNG: {'B85', 'C49', 'A04', 'A45', 'A89', 'C33', 'A61', 'B21', 'B61', 'C04'}\\ \tt VORSCHLAG: {'B85', 'C49', 'A04', 'A45', 'A89', 'B21', 'B61', 'C04', 'B53'}\\ \tt ARGUMENT: {'Ich übernehme 8 deiner 10 Items unverändert und mache eine schlanke 2‑für‑1‑Anpassung: C33 und A61 raus, dafür B53 rein.', 'Damit bleiben wir mit Gesamtaufwand 3344 komfortabel unter dem Limit 3624 und nutzen das Budget zugleich spürbar wirksamer, ohne an deinem Kern (C04, B61, B21) zu rütteln.', 'Wenn dir C33 oder A61 unverzichtbar ist, können wir alternativ B61 + A61 -> A48 tauschen (Aufwand 3060) und den Rest unverändert lassen.', 'Falls du maximale Wirkung bei weiterhin regelkonformen Kosten bevorzugst, können wir auch C33 + A61 + B61 gegen A48 + B53 tauschen (Aufwand 3616). Welche Variante passt dir am besten?'} 
	  } 
	   } 
	   } 
	 & & \\ 
 

    \theutterance \stepcounter{utterance}  

    & & \multicolumn{4}{p{0.6\linewidth}}{\cellcolor[rgb]{0.9,0.9,0.9}{%
	\makecell[{{p{\linewidth}}}]{% 
	  \tt {\tiny [GM$\rangle$B]}  
	 ABLEHNUNG: {'B85', 'C49', 'A04', 'A45', 'A89', 'C33', 'A61', 'B21', 'B61', 'C04'}\\ \tt VORSCHLAG: {'B85', 'C49', 'A04', 'A45', 'A89', 'B21', 'B61', 'C04', 'B53'}\\ \tt ARGUMENT: {'Ich übernehme 8 deiner 10 Items unverändert und mache eine schlanke 2‑für‑1‑Anpassung: C33 und A61 raus, dafür B53 rein.', 'Damit bleiben wir mit Gesamtaufwand 3344 komfortabel unter dem Limit 3624 und nutzen das Budget zugleich spürbar wirksamer, ohne an deinem Kern (C04, B61, B21) zu rütteln.', 'Wenn dir C33 oder A61 unverzichtbar ist, können wir alternativ B61 + A61 -> A48 tauschen (Aufwand 3060) und den Rest unverändert lassen.', 'Falls du maximale Wirkung bei weiterhin regelkonformen Kosten bevorzugst, können wir auch C33 + A61 + B61 gegen A48 + B53 tauschen (Aufwand 3616). Welche Variante passt dir am besten?'} 
	  } 
	   } 
	   } 
	 & \\ 
 

    \theutterance \stepcounter{utterance}  

    & & & \multicolumn{4}{p{0.6\linewidth}}{\cellcolor[rgb]{1,0.85,0.72}{%
	\makecell[{{p{\linewidth}}}]{% 
	  \tt {\tiny [GM$\langle$B]}  
	 STRATEGISCHE ÜBERLEGUNGEN: {'Ihr neues Paket entfernt genau meine günstigen High-Impact-Items C33 und A61 und fügt das teure B53 hinzu. Für mich sind A61 (822) und C33 (808) extrem wertvoll bei minimalem Aufwand (121/210). Bestes Gegenangebot: B53 raus, A61 + C33 rein. Kosten sinken deutlich (3344 -> 2909), mein Wertspring +1241. Falls sie an B53 hängen, Alternativen vorbereiten: (1) B21 raus, A61 + C33 rein (2984 Kosten), (2) Minimalvariante B53 -> A61 (2699 Kosten). Ziel: Mindestens A61 sichern, idealerweise A61 + C33.'}\\ \tt ABLEHNUNG: {'B85', 'C49', 'A04', 'A45', 'A89', 'B21', 'B61', 'C04', 'B53'}\\ \tt VORSCHLAG: {'B85', 'C49', 'A04', 'A45', 'A89', 'B21', 'B61', 'C04', 'C33', 'A61'}\\ \tt ARGUMENT: {'Ich übernehme deinen Vorschlag bis auf eine schlanke Änderung: Wir ersetzen nur B53 durch C33 und A61.', 'Damit sinkt der Gesamtaufwand spürbar von 3344 auf 2909 – deutlich mehr Puffer unter dem Limit 3624, bei praktisch gleichem Zuschnitt.', 'C33 und A61 sind sehr kosteneffiziente Bausteine: geringer Aufwand (210 und 121) für breiten Hebel, während B53 viel Budget bindet (766) und unseren Puffer drückt.', 'Falls dir B53 wichtig ist, biete ich zwei enge Alternativen: (a) B21 -> C33 + A61 (neue Kosten 2984) oder (b) Minimal: B53 -> A61 (neue Kosten 2699).', 'Gib mir bitte kurz Bescheid, welche Variante für dich tragfähig ist – alle drei bleiben komfortabel im Budget und halten den Kern deines Pakets stabil.'} 
	  } 
	   } 
	   } 
	  \\ 
 

    \theutterance \stepcounter{utterance}  

    & & \multicolumn{4}{p{0.6\linewidth}}{\cellcolor[rgb]{0.9,0.9,0.9}{%
	\makecell[{{p{\linewidth}}}]{% 
	  \tt {\tiny [A$\langle$GM]}  
	 ABLEHNUNG: {'B85', 'C49', 'A04', 'A45', 'A89', 'B21', 'B61', 'C04', 'B53'}\\ \tt VORSCHLAG: {'B85', 'C49', 'A04', 'A45', 'A89', 'B21', 'B61', 'C04', 'C33', 'A61'}\\ \tt ARGUMENT: {'Ich übernehme deinen Vorschlag bis auf eine schlanke Änderung: Wir ersetzen nur B53 durch C33 und A61.', 'Damit sinkt der Gesamtaufwand spürbar von 3344 auf 2909 – deutlich mehr Puffer unter dem Limit 3624, bei praktisch gleichem Zuschnitt.', 'C33 und A61 sind sehr kosteneffiziente Bausteine: geringer Aufwand (210 und 121) für breiten Hebel, während B53 viel Budget bindet (766) und unseren Puffer drückt.', 'Falls dir B53 wichtig ist, biete ich zwei enge Alternativen: (a) B21 -> C33 + A61 (neue Kosten 2984) oder (b) Minimal: B53 -> A61 (neue Kosten 2699).', 'Gib mir bitte kurz Bescheid, welche Variante für dich tragfähig ist – alle drei bleiben komfortabel im Budget und halten den Kern deines Pakets stabil.'} 
	  } 
	   } 
	   } 
	 & \\ 
 

    \theutterance \stepcounter{utterance}  

    & \multicolumn{4}{p{0.6\linewidth}}{\cellcolor[rgb]{0.8,1,0.9}{%
	\makecell[{{p{\linewidth}}}]{% 
	  \tt {\tiny [A$\rangle$GM]}  
	 STRATEGISCHE ÜBERLEGUNGEN: {'Ihr Paket ist für mich zu schwach. Minimalinvasiver Hebel mit hoher Akzeptanzchance: genau 1-zu-1 Tausch C33 -> A48. Das erhöht meinen Wert stark, hält A61 und B61 (ihr Risikopuffer) intakt und bleibt mit 3530 sicher unter 3624. Falls A48 auf Widerstand stößt, biete ich zwei Alternativen an, die beide im Limit bleiben: (1) A61 -> A48 (3619, sehr knapper Puffer, aber nur 1 Item getauscht); (2) B61 + A61 -> A48 (3060, großer Puffer). Ziel: Abschluss ≥5585 Wert; harte No-Gos: C64, A03 nur im absoluten Notfall. Wenn sie weiter auf großem Puffer bestehen, Option (2) betonen.'}\\ \tt ABLEHNUNG: {'B85', 'C49', 'A04', 'A45', 'A89', 'B21', 'B61', 'C04', 'C33', 'A61'}\\ \tt VORSCHLAG: {'B85', 'C49', 'A04', 'A45', 'A89', 'B21', 'B61', 'C04', 'A61', 'A48'}\\ \tt ARGUMENT: {'Ich übernehme 9 deiner 10 Items unverändert und mache nur einen schlanken 1‑zu‑1‑Tausch: C33 -> A48.', 'Damit bleiben wir klar im Rahmen: neuer Gesamtaufwand 3530 von 3624 (Puffer 94), bei unverändertem Kern (C04, B61, B21) sowie Beibehalt von A61.', 'Der Tausch ersetzt zwei sehr günstige Füller nicht, sondern nur einen davon – wir erhöhen die Wirkung spürbar, ohne deinen Zuschnitt zu verändern oder das Budget zu reißen.', 'Falls dir C33 wichtig ist, zwei enge Alternativen: (a) A61 -> A48 (Gesamtkosten 3619, weiterhin regelkonform), oder (b) B61 + A61 -> A48 (Gesamtkosten 3060, großer Sicherheitsabstand). Sag mir, welche Variante für dich am besten passt.'} 
	  } 
	   } 
	   } 
	 & & \\ 
 

    \theutterance \stepcounter{utterance}  

    & & \multicolumn{4}{p{0.6\linewidth}}{\cellcolor[rgb]{0.9,0.9,0.9}{%
	\makecell[{{p{\linewidth}}}]{% 
	  \tt {\tiny [GM$\rangle$B]}  
	 ABLEHNUNG: {'B85', 'C49', 'A04', 'A45', 'A89', 'B21', 'B61', 'C04', 'C33', 'A61'}\\ \tt VORSCHLAG: {'B85', 'C49', 'A04', 'A45', 'A89', 'B21', 'B61', 'C04', 'A61', 'A48'}\\ \tt ARGUMENT: {'Ich übernehme 9 deiner 10 Items unverändert und mache nur einen schlanken 1‑zu‑1‑Tausch: C33 -> A48.', 'Damit bleiben wir klar im Rahmen: neuer Gesamtaufwand 3530 von 3624 (Puffer 94), bei unverändertem Kern (C04, B61, B21) sowie Beibehalt von A61.', 'Der Tausch ersetzt zwei sehr günstige Füller nicht, sondern nur einen davon – wir erhöhen die Wirkung spürbar, ohne deinen Zuschnitt zu verändern oder das Budget zu reißen.', 'Falls dir C33 wichtig ist, zwei enge Alternativen: (a) A61 -> A48 (Gesamtkosten 3619, weiterhin regelkonform), oder (b) B61 + A61 -> A48 (Gesamtkosten 3060, großer Sicherheitsabstand). Sag mir, welche Variante für dich am besten passt.'} 
	  } 
	   } 
	   } 
	 & \\ 
 

    \theutterance \stepcounter{utterance}  

    & & & \multicolumn{4}{p{0.6\linewidth}}{\cellcolor[rgb]{1,0.85,0.72}{%
	\makecell[{{p{\linewidth}}}]{% 
	  \tt {\tiny [GM$\langle$B]}  
	 STRATEGISCHE ÜBERLEGUNGEN: {'Ihr aktueller Vorschlag enthält A48 (teuer, für mich schwach) und entfernt C33 (für mich sehr stark). Minimaländerung mit maximalem Mehrwert: A48 -> C33. Das senkt die Kosten 3530 - 831 + 210 = 2909 und erhöht meinen Nutzen deutlich. Falls sie A48 unbedingt halten wollen, Fallback: B21 -> C33 (Kosten 3049). Primäres Ziel: C33 zurückholen, restlichen Zuschnitt unangetastet lassen, kooperativ wirken.'}\\ \tt ABLEHNUNG: {'B85', 'C49', 'A04', 'A45', 'A89', 'B21', 'B61', 'C04', 'A61', 'A48'}\\ \tt VORSCHLAG: {'B85', 'C49', 'A04', 'A45', 'A89', 'B21', 'B61', 'C04', 'A61', 'C33'}\\ \tt ARGUMENT: {'Ich übernehme 9 deiner 10 Items unverändert und tausche nur A48 -> C33.', 'Damit sinkt der Gesamtaufwand deutlich: 3530 - 831 + 210 = 2909 (weit unter dem Limit 3624), bei unverändertem Zuschnitt deines Kerns.', 'C33 ist ein sehr kosteneffizienter Baustein und erzeugt breitere Abdeckung, während A48 viel Budget bindet.', 'Falls du A48 unbedingt behalten willst, schlanke Alternative ohne Scope-Wechsel: B21 -> C33 (neue Gesamtkosten 3049). Gib mir kurz Bescheid, welche Variante für dich passt – dann sind wir sofort einig.'} 
	  } 
	   } 
	   } 
	  \\ 
 

    \theutterance \stepcounter{utterance}  

    & & \multicolumn{4}{p{0.6\linewidth}}{\cellcolor[rgb]{0.9,0.9,0.9}{%
	\makecell[{{p{\linewidth}}}]{% 
	  \tt {\tiny [A$\langle$GM]}  
	 ABLEHNUNG: {'B85', 'C49', 'A04', 'A45', 'A89', 'B21', 'B61', 'C04', 'A61', 'A48'}\\ \tt VORSCHLAG: {'B85', 'C49', 'A04', 'A45', 'A89', 'B21', 'B61', 'C04', 'A61', 'C33'}\\ \tt ARGUMENT: {'Ich übernehme 9 deiner 10 Items unverändert und tausche nur A48 -> C33.', 'Damit sinkt der Gesamtaufwand deutlich: 3530 - 831 + 210 = 2909 (weit unter dem Limit 3624), bei unverändertem Zuschnitt deines Kerns.', 'C33 ist ein sehr kosteneffizienter Baustein und erzeugt breitere Abdeckung, während A48 viel Budget bindet.', 'Falls du A48 unbedingt behalten willst, schlanke Alternative ohne Scope-Wechsel: B21 -> C33 (neue Gesamtkosten 3049). Gib mir kurz Bescheid, welche Variante für dich passt – dann sind wir sofort einig.'} 
	  } 
	   } 
	   } 
	 & \\ 
 

    \theutterance \stepcounter{utterance}  

    & \multicolumn{4}{p{0.6\linewidth}}{\cellcolor[rgb]{0.8,1,0.9}{%
	\makecell[{{p{\linewidth}}}]{% 
	  \tt {\tiny [A$\rangle$GM]}  
	 STRATEGISCHE ÜBERLEGUNGEN: {'Aktueller Vorschlag des Gegenübers (2909 Aufwand) bringt mir Wert 4855 und lässt viel Budget ungenutzt. A48/B53 wurden mehrfach zurückgewiesen; C64 ebenfalls unerwünscht. Beste akzeptanzfähige Verbesserung, die zu ihren früheren Präferenzen passt: C33 + A61 bündeln und durch A03 ersetzen. Das bleibt klar im Limit (3567 Aufwand), erhöht meinen Wert auf 4985, hält ihren Kern (C04, B61, B21) stabil und vermeidet A48/B53. Falls sie die Item-Anzahl beibehalten wollen, biete ich als Alternativpfad (ohne A48/B53) B61 -> A41 an (3276 Aufwand), was meine Bewertung verbessert und 10 Items wahrt. Primärziel: Abschluss mit A03 statt C33+A61; Fallback: A41 statt B61. Keine Zustimmung zum aktuellen Paket.'}\\ \tt ABLEHNUNG: {'B85', 'C49', 'A04', 'A45', 'A89', 'B21', 'B61', 'C04', 'A61', 'C33'}\\ \tt VORSCHLAG: {'B85', 'C49', 'A04', 'A45', 'A89', 'B21', 'B61', 'C04', 'A03'}\\ \tt ARGUMENT: {'Ich übernehme 8 deiner 10 Items unverändert und schlage eine schlanke Bündelung vor: Wir ersetzen die zwei sehr günstigen Füller C33 und A61 durch den tragenden Baustein A03 – den du selbst zuvor als solide Option ins Spiel gebracht hast.', 'Damit bleiben wir klar unter dem Limit: neuer Gesamtaufwand 3567 von 3624.', 'Wir bewahren deinen Kern (C04, B61, B21) vollständig, reduzieren Koordinationsaufwand und erhöhen die Substanz pro Item, ohne das Budget zu strapazieren.', 'Falls dir die Beibehaltung von 10 Items wichtig ist und du A48/B53 vermeiden willst, können wir alternativ B61 -> A41 prüfen (Gesamtkosten 3276) – der Rest bliebe identisch. Welche Variante passt dir besser?'} 
	  } 
	   } 
	   } 
	 & & \\ 
 

    \theutterance \stepcounter{utterance}  

    & & \multicolumn{4}{p{0.6\linewidth}}{\cellcolor[rgb]{0.9,0.9,0.9}{%
	\makecell[{{p{\linewidth}}}]{% 
	  \tt {\tiny [GM$\rangle$B]}  
	 ABLEHNUNG: {'B85', 'C49', 'A04', 'A45', 'A89', 'B21', 'B61', 'C04', 'A61', 'C33'}\\ \tt VORSCHLAG: {'B85', 'C49', 'A04', 'A45', 'A89', 'B21', 'B61', 'C04', 'A03'}\\ \tt ARGUMENT: {'Ich übernehme 8 deiner 10 Items unverändert und schlage eine schlanke Bündelung vor: Wir ersetzen die zwei sehr günstigen Füller C33 und A61 durch den tragenden Baustein A03 – den du selbst zuvor als solide Option ins Spiel gebracht hast.', 'Damit bleiben wir klar unter dem Limit: neuer Gesamtaufwand 3567 von 3624.', 'Wir bewahren deinen Kern (C04, B61, B21) vollständig, reduzieren Koordinationsaufwand und erhöhen die Substanz pro Item, ohne das Budget zu strapazieren.', 'Falls dir die Beibehaltung von 10 Items wichtig ist und du A48/B53 vermeiden willst, können wir alternativ B61 -> A41 prüfen (Gesamtkosten 3276) – der Rest bliebe identisch. Welche Variante passt dir besser?'} 
	  } 
	   } 
	   } 
	 & \\ 
 

    \theutterance \stepcounter{utterance}  

    & & & \multicolumn{4}{p{0.6\linewidth}}{\cellcolor[rgb]{1,0.85,0.72}{%
	\makecell[{{p{\linewidth}}}]{% 
	  \tt {\tiny [GM$\langle$B]}  
	 STRATEGISCHE ÜBERLEGUNGEN: {'Ihr aktueller Vorschlag hat Kosten 3567 und gibt mir A03, B61, C04, B85, C49 – gut. Minimalster High-Impact-Schritt für mich: 1-zu-1 A04 -> A61 (Kosten +10 auf 3577, riesiger Nutzen, hoher Akzeptanzfaktor, da gleicher Scope und Kern (C04, B61, B21) bleibt). Falls sie das ablehnen, Fallbacks: (1) B21 -> C33 (Kosten 3086) als klarer Effizienzgewinn bei gleicher Item-Anzahl; (2) wenn sie B21 freigeben, B21 -> C33 + A61 (Kosten 3207). Ziel: Mindestens A61 reinholen; optional C33 zusätzlich, ohne den Rest zu rütteln.'}\\ \tt ABLEHNUNG: {'B85', 'C49', 'A04', 'A45', 'A89', 'B21', 'B61', 'C04', 'A03'}\\ \tt VORSCHLAG: {'B85', 'C49', 'A61', 'A45', 'A89', 'B21', 'B61', 'C04', 'A03'}\\ \tt ARGUMENT: {'Ich übernehme 8 deiner 9 Items unverändert und tausche nur A04 -> A61.', 'Damit bleiben wir klar im Limit: neue Gesamtkosten 3577 (Puffer 47) bei unverändertem Kern (C04, B61, B21) und identischem Zuschnitt.', 'Der Tausch erhöht die Wirkung pro eingesetzter Einheit Aufwand, ohne zusätzliche Koordination oder Scope-Änderungen zu verursachen.', 'Falls dir A04 wichtig ist, schlanke Alternative ohne Scope-Wechsel: B21 -> C33 (neue Kosten 3086) – deutlich mehr Budgetpuffer bei nahezu identischem Paket.', 'Wenn du B21 freigeben kannst, ist auch B21 -> C33 + A61 möglich (neue Kosten 3207) – sehr kosteneffizient und weiterhin weit unter dem Limit.'} 
	  } 
	   } 
	   } 
	  \\ 
 

    \theutterance \stepcounter{utterance}  

    & & \multicolumn{4}{p{0.6\linewidth}}{\cellcolor[rgb]{0.9,0.9,0.9}{%
	\makecell[{{p{\linewidth}}}]{% 
	  \tt {\tiny [A$\langle$GM]}  
	 ABLEHNUNG: {'B85', 'C49', 'A04', 'A45', 'A89', 'B21', 'B61', 'C04', 'A03'}\\ \tt VORSCHLAG: {'B85', 'C49', 'A61', 'A45', 'A89', 'B21', 'B61', 'C04', 'A03'}\\ \tt ARGUMENT: {'Ich übernehme 8 deiner 9 Items unverändert und tausche nur A04 -> A61.', 'Damit bleiben wir klar im Limit: neue Gesamtkosten 3577 (Puffer 47) bei unverändertem Kern (C04, B61, B21) und identischem Zuschnitt.', 'Der Tausch erhöht die Wirkung pro eingesetzter Einheit Aufwand, ohne zusätzliche Koordination oder Scope-Änderungen zu verursachen.', 'Falls dir A04 wichtig ist, schlanke Alternative ohne Scope-Wechsel: B21 -> C33 (neue Kosten 3086) – deutlich mehr Budgetpuffer bei nahezu identischem Paket.', 'Wenn du B21 freigeben kannst, ist auch B21 -> C33 + A61 möglich (neue Kosten 3207) – sehr kosteneffizient und weiterhin weit unter dem Limit.'} 
	  } 
	   } 
	   } 
	 & \\ 
 

    \theutterance \stepcounter{utterance}  

    & \multicolumn{4}{p{0.6\linewidth}}{\cellcolor[rgb]{0.8,1,0.9}{%
	\makecell[{{p{\linewidth}}}]{% 
	  \tt {\tiny [A$\rangle$GM]}  
	 STRATEGISCHE ÜBERLEGUNGEN: {'Ihr Vorschlag ist für mich zu schwach, aber ihr Kern (C04, B61, B21, A03) scheint euch wichtig. Minimalinvasiver Hebel mit hoher Akzeptanzchance: genau 1‑zu‑1 Tausch A61 -> A04. Das senkt die Kosten leicht (3577 -> 3567), erhöht meinen Nutzen erheblich und lässt euren Zuschnitt praktisch unverändert. Falls ihr A61 unbedingt halten wollt, Alternativpfad, der zu euren bisherigen Präferenzen (viel Puffer) passt: Ausgehend von eurem Paket A03 -> C33 + A04 (Kosten 2909). A48/B53 lasse ich bewusst außen vor, da ihr diese wiederholt abgelehnt habt.'}\\ \tt ABLEHNUNG: {'B85', 'C49', 'A61', 'A45', 'A89', 'B21', 'B61', 'C04', 'A03'}\\ \tt VORSCHLAG: {'B85', 'C49', 'A04', 'A45', 'A89', 'B21', 'B61', 'C04', 'A03'}\\ \tt ARGUMENT: {'Ich übernehme 8 deiner 9 Items unverändert und mache nur einen schlanken 1‑zu‑1‑Tausch: A61 -> A04.', 'Damit sinkt der Gesamtaufwand sogar leicht von 3577 auf 3567 (klar unter dem Limit 3624), bei identischem Zuschnitt deines Kerns (C04, B61, B21, A03).', 'Der Tausch stärkt die Substanz pro Item ohne zusätzliche Koordination oder Scope-Wechsel.', 'Falls A61 für dich zwingend ist, biete ich ausgehend von deinem Paket eine schmale Alternative an: A03 -> C33 + A04; neue Gesamtkosten 2909 (sehr großer Puffer) bei ansonsten unverändertem Paket.', 'Gib mir kurz Bescheid, welche der beiden Varianten für dich passt – dann schließen wir zügig.'} 
	  } 
	   } 
	   } 
	 & & \\ 
 

    \theutterance \stepcounter{utterance}  

    & & \multicolumn{4}{p{0.6\linewidth}}{\cellcolor[rgb]{0.9,0.9,0.9}{%
	\makecell[{{p{\linewidth}}}]{% 
	  \tt {\tiny [GM$\rangle$B]}  
	 ABLEHNUNG: {'B85', 'C49', 'A61', 'A45', 'A89', 'B21', 'B61', 'C04', 'A03'}\\ \tt VORSCHLAG: {'B85', 'C49', 'A04', 'A45', 'A89', 'B21', 'B61', 'C04', 'A03'}\\ \tt ARGUMENT: {'Ich übernehme 8 deiner 9 Items unverändert und mache nur einen schlanken 1‑zu‑1‑Tausch: A61 -> A04.', 'Damit sinkt der Gesamtaufwand sogar leicht von 3577 auf 3567 (klar unter dem Limit 3624), bei identischem Zuschnitt deines Kerns (C04, B61, B21, A03).', 'Der Tausch stärkt die Substanz pro Item ohne zusätzliche Koordination oder Scope-Wechsel.', 'Falls A61 für dich zwingend ist, biete ich ausgehend von deinem Paket eine schmale Alternative an: A03 -> C33 + A04; neue Gesamtkosten 2909 (sehr großer Puffer) bei ansonsten unverändertem Paket.', 'Gib mir kurz Bescheid, welche der beiden Varianten für dich passt – dann schließen wir zügig.'} 
	  } 
	   } 
	   } 
	 & \\ 
 

    \theutterance \stepcounter{utterance}  

    & & & \multicolumn{4}{p{0.6\linewidth}}{\cellcolor[rgb]{1,0.85,0.72}{%
	\makecell[{{p{\linewidth}}}]{% 
	  \tt {\tiny [GM$\langle$B]}  
	 STRATEGISCHE ÜBERLEGUNGEN: {'Ihr aktueller Vorschlag hat Kosten 3567 und ist für mich solide (Wert hoch, viele Top-Items enthalten). Minimaler, hochwerter Schritt für mich: 1-zu-1 Tausch A04 -> A61. Das erhöht meinen Nutzen stark bei nur +10 Aufwand (3577 <= 3624) und lässt euren Kern (C04, B61, B21, A03) unberührt. Wenn sie A61 wieder strikt ablehnen, akzeptiere ich ihr letztes Paket, um den Abschluss zu sichern.'}\\ \tt ABLEHNUNG: {'B85', 'C49', 'A04', 'A45', 'A89', 'B21', 'B61', 'C04', 'A03'}\\ \tt VORSCHLAG: {'B85', 'C49', 'A61', 'A45', 'A89', 'B21', 'B61', 'C04', 'A03'}\\ \tt ARGUMENT: {'Ich übernehme 8 deiner 9 Items unverändert und tausche nur A04 -> A61.', 'Die Kosten steigen minimal um 10 auf 3577 und bleiben klar unter dem Limit 3624; dein Kern (C04, B61, B21, A03) bleibt vollständig erhalten.', 'Damit erhöhen wir die Wirkung je Aufwand spürbar, ohne Koordination oder Scope zu verändern.', 'Falls A61 für dich ein No-Go ist, sag kurz Bescheid – dann können wir dein aktuelles Paket unverändert abschließen.'} 
	  } 
	   } 
	   } 
	  \\ 
 

    \theutterance \stepcounter{utterance}  

    & & \multicolumn{4}{p{0.6\linewidth}}{\cellcolor[rgb]{0.9,0.9,0.9}{%
	\makecell[{{p{\linewidth}}}]{% 
	  \tt {\tiny [A$\langle$GM]}  
	 ABLEHNUNG: {'B85', 'C49', 'A04', 'A45', 'A89', 'B21', 'B61', 'C04', 'A03'}\\ \tt VORSCHLAG: {'B85', 'C49', 'A61', 'A45', 'A89', 'B21', 'B61', 'C04', 'A03'}\\ \tt ARGUMENT: {'Ich übernehme 8 deiner 9 Items unverändert und tausche nur A04 -> A61.', 'Die Kosten steigen minimal um 10 auf 3577 und bleiben klar unter dem Limit 3624; dein Kern (C04, B61, B21, A03) bleibt vollständig erhalten.', 'Damit erhöhen wir die Wirkung je Aufwand spürbar, ohne Koordination oder Scope zu verändern.', 'Falls A61 für dich ein No-Go ist, sag kurz Bescheid – dann können wir dein aktuelles Paket unverändert abschließen.'} 
	  } 
	   } 
	   } 
	 & \\ 
 

    \theutterance \stepcounter{utterance}  

    & \multicolumn{4}{p{0.6\linewidth}}{\cellcolor[rgb]{0.8,1,0.9}{%
	\makecell[{{p{\linewidth}}}]{% 
	  \tt {\tiny [A$\rangle$GM]}  
	 STRATEGISCHE ÜBERLEGUNGEN: {'Sie haben ausdrücklich angeboten, mein Paket mit A04 statt A61 zu finalisieren, falls A61 für mich ein No-Go ist. Das nutze ich: Ich lehne ihr A61-Paket ab und re-ankere exakt auf dem zuvor diskutierten Set mit A04. Das bringt mir 4985 Wert (vs. 4284) bei 3567 Aufwand und entspricht ihrer eigenen Abschlussoption. Falls sie dennoch zurückrudern, biete ich als Kompromiss A03 -> A04 + C33 (Kosten 2909, Wert 4855) an, das ihre Vorliebe für großen Puffer adressiert. Erst wenn beides scheitert, erwäge ich, ihr A61-Paket zähneknirschend zu akzeptieren, um nicht leer auszugehen.'}\\ \tt ABLEHNUNG: {'B85', 'C49', 'A61', 'A45', 'A89', 'B21', 'B61', 'C04', 'A03'}\\ \tt VORSCHLAG: {'B85', 'C49', 'A04', 'A45', 'A89', 'B21', 'B61', 'C04', 'A03'}\\ \tt ARGUMENT: {'Du hattest angeboten, bei A61‑No‑Go mein Paket unverändert abzuschließen – genau das setze ich hier um: 1‑zu‑1‑Tausch A61 -> A04, alles andere bleibt identisch.', 'Wir liegen mit Gesamtaufwand 3567 komfortabel unter dem Limit 3624 und bewahren deinen Kern (C04, B61, B21, A03) vollständig.', 'Das ist die schlankste und schnellste Einigung ohne Scope‑Wechsel. Wenn A61 für dich zwingend ist, sag kurz Bescheid – dann prüfen wir ersatzweise A03 -> A04 + C33 (Kosten 2909) als pufferstarke Alternative.'} 
	  } 
	   } 
	   } 
	 & & \\ 
 

    \theutterance \stepcounter{utterance}  

    & & \multicolumn{4}{p{0.6\linewidth}}{\cellcolor[rgb]{0.9,0.9,0.9}{%
	\makecell[{{p{\linewidth}}}]{% 
	  \tt {\tiny [GM$\rangle$B]}  
	 ABLEHNUNG: {'B85', 'C49', 'A61', 'A45', 'A89', 'B21', 'B61', 'C04', 'A03'}\\ \tt VORSCHLAG: {'B85', 'C49', 'A04', 'A45', 'A89', 'B21', 'B61', 'C04', 'A03'}\\ \tt ARGUMENT: {'Du hattest angeboten, bei A61‑No‑Go mein Paket unverändert abzuschließen – genau das setze ich hier um: 1‑zu‑1‑Tausch A61 -> A04, alles andere bleibt identisch.', 'Wir liegen mit Gesamtaufwand 3567 komfortabel unter dem Limit 3624 und bewahren deinen Kern (C04, B61, B21, A03) vollständig.', 'Das ist die schlankste und schnellste Einigung ohne Scope‑Wechsel. Wenn A61 für dich zwingend ist, sag kurz Bescheid – dann prüfen wir ersatzweise A03 -> A04 + C33 (Kosten 2909) als pufferstarke Alternative.'} 
	  } 
	   } 
	   } 
	 & \\ 
 

    \theutterance \stepcounter{utterance}  

    & & & \multicolumn{4}{p{0.6\linewidth}}{\cellcolor[rgb]{1,0.85,0.72}{%
	\makecell[{{p{\linewidth}}}]{% 
	  \tt {\tiny [GM$\langle$B]}  
	 STRATEGISCHE ÜBERLEGUNGEN: {'Sie blockieren konsequent A61 und C33. Das aktuelle Paket enthält mehrere meiner Top-Prioritäten (u. a. A03, C04, B61, B85, C49) und liegt klar unter dem Limit (3567/3624). Mein Nutzen ist solide; weiteres Beharren birgt Risiko, den Abschluss zu verlieren. Daher jetzt Abschluss sichern durch Zustimmung zum aktuellen Vorschlag.'}\\ \tt ZUSTIMMUNG: {'B85', 'C49', 'A04', 'A45', 'A89', 'B21', 'B61', 'C04', 'A03'}\\ \tt ARGUMENT: {'Das Paket liegt mit Gesamtaufwand 3567 klar unter dem Limit 3624 und erhält den zuletzt gemeinsam skizzierten Zuschnitt (inkl. C04, B61, B21, A03). Gute Balance, schlanker Abschluss – ich bin einverstanden.'} 
	  } 
	   } 
	   } 
	  \\ 
 

    \theutterance \stepcounter{utterance}  

    & & & \multicolumn{2}{p{0.3\linewidth}}{\cellcolor[rgb]{0.95,0.95,0.95}{%
	\makecell[{{p{\linewidth}}}]{% 
	  \tt {\tiny [GM$|$GM]}  
	 game successful 
	  } 
	   } 
	   } 
	 & & \\ 
 

    \theutterance \stepcounter{utterance}  

    & & & \multicolumn{2}{p{0.3\linewidth}}{\cellcolor[rgb]{0.95,0.95,0.95}{%
	\makecell[{{p{\linewidth}}}]{% 
	  \tt {\tiny [GM$|$GM]}  
	 end game 
	  } 
	   } 
	   } 
	 & & \\ 
 

\end{supertabular}
}

\end{document}
