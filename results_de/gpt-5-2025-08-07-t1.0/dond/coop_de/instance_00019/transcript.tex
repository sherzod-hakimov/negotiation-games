
\documentclass{article}
\usepackage{colortbl}
\usepackage{makecell}
\usepackage{multirow}
\usepackage{supertabular}

\begin{document}

\newcounter{utterance}

\centering \large Interaction Transcript for game `dond', experiment `coop\_de', episode 19 with gpt{-}5{-}2025{-}08{-}07{-}t1.0.
\vspace{24pt}

{ \footnotesize  \setcounter{utterance}{1}
\setlength{\tabcolsep}{0pt}
\begin{supertabular}{c@{$\;$}|p{.15\linewidth}@{}p{.15\linewidth}p{.15\linewidth}p{.15\linewidth}p{.15\linewidth}p{.15\linewidth}}
    \# & $\;$A & \multicolumn{4}{c}{Game Master} & $\;\:$B\\
    \hline

    \theutterance \stepcounter{utterance}  
    & & \multicolumn{4}{p{0.6\linewidth}}{
        \cellcolor[rgb]{0.9,0.9,0.9}{
            \makecell[{{p{\linewidth}}}]{
                \texttt{\tiny{[P1$\langle$GM]}}
                \texttt{Sie spielen ein kooperatives Verhandlungsspiel, bei dem Sie sich mit einem anderen Spieler darauf einigen müssen, wie eine Reihe von Gegenständen aufgeteilt werden soll.} \\
\\ 
\texttt{Die Regeln:} \\
\texttt{(a) Sie und der andere Spieler erhalten eine Sammlung von Gegenständen. Jeder von Ihnen erhält außerdem eine geheime Wertfunktion, die angibt, wie viel Ihnen jede Art von Gegenstand wert ist.} \\
\texttt{(b) Sie tauschen Nachrichten mit dem anderen Spieler aus, um zu vereinbaren, wer welche Gegenstände bekommt. Sie können jeweils maximal 5 Nachrichten senden oder das Spiel vorzeitig beenden, indem Sie jederzeit einen geheimen Vorschlag machen.} \\
\texttt{(c) Jeder von euch wird aufgefordert, einen geheimen Vorschlag zu machen, in dem ihr die gewünschten Gegenstände in eckigen Klammern wie folgt angibt: "{[}Vorschlag: <Nummer> <Objektname>, <Nummer> <Objektname>, <...>{]}"} \\
\texttt{(d) Wenn eure Vorschläge komplementär sind, d.h. es gibt genug Gegenstände, um beide Vorschläge zu erfüllen, erhält jeder Spieler eine Punktzahl, die sich aus der Summe der Werte für die Gegenstände ergibt, die er erhalten hat. Andernfalls erhalten Sie beide null Punkte.} \\
\texttt{(e) Ihr Ziel ist es, die Summe aus Ihrer Punktzahl und der Punktzahl des anderen Spielers zu maximieren.} \\
\\ 
\texttt{Beginnen wir.} \\
\\ 
\texttt{Die Menge der verfügbaren Gegenstände ist:} \\
\\ 
\texttt{2 Haken, 1 Blume, 1 Karte, 1 Radiergummi, 1 Schwamm} \\
\\ 
\texttt{Deine geheime Wertfunktion ist:} \\
\\ 
\texttt{Haken: 3, Blume: 3, Karte: 0, Radiergummi: 1, Schwamm: 0} \\
\\ 
\texttt{WICHTIG: Ihre Nachrichten werden, sofern es sich nicht um einen geheimen Vorschlag handelt, direkt an den anderen Spieler übermittelt, also fügen Sie keine Antwort auf die Regeln oder einen Text zur Ankündigung Ihrer Nachricht ein. Um einen geheimen Vorschlag zu machen, verwenden Sie das angegebene Format. Verwenden Sie keine eckigen Klammern, wenn Sie mit dem anderen Spieler kommunizieren, sonst wird dies als Ihr geheimer Vorschlag interpretiert.} \\
            }
        }
    }
    & \\ \\

    \theutterance \stepcounter{utterance}  
    & \multicolumn{4}{p{0.6\linewidth}}{
        \cellcolor[rgb]{0.9,0.9,0.9}{
            \makecell[{{p{\linewidth}}}]{
                \texttt{\tiny{[P1$\rangle$GM]}}
                \texttt{Ich lege offen meine Werte: Haken 3 (pro Stück), Blume 3, Radiergummi 1, Karte 0, Schwamm 0. Karte und Schwamm kannst du auf jeden Fall haben. Für mich sind die Haken und die Blume wichtig; der Radiergummi ist nur leicht wertvoll.} \\
\\ 
\texttt{Wie sind deine Werte/Prioritäten? Insbesondere: Wie stark sind Haken und Blume für dich? Wenn du sie nicht hoch bewertest, würde ich gern 2 Haken (und ggf. die Blume) nehmen; wenn du sie hoch bewertest, kann ich auch mit 1 Haken + Blume leben oder den Radiergummi abgeben, damit wir die Gesamtsumme maximieren. Sag mir bitte kurz, welche Items dir >0 wert sind (idealerweise mit Zahlen), dann mache ich umgehend einen konkreten Vorschlag.} \\
            }
        }
    }
    & & \\ \\

    \theutterance \stepcounter{utterance}  
    & & & \multicolumn{2}{p{0.3\linewidth}}{
        \cellcolor[rgb]{0.9,0.9,0.9}{
            \makecell[{{p{\linewidth}}}]{
                \texttt{\tiny{[GM$|$GM]}}
                \texttt{continue} \\
            }
        }
    }
    & & \\ \\

    \theutterance \stepcounter{utterance}  
    & & & \multicolumn{2}{p{0.3\linewidth}}{
        \cellcolor[rgb]{0.9,0.9,0.9}{
            \makecell[{{p{\linewidth}}}]{
                \texttt{\tiny{[GM$|$GM]}}
                \texttt{Ich lege offen meine Werte: Haken 3 (pro Stück), Blume 3, Radiergummi 1, Karte 0, Schwamm 0. Karte und Schwamm kannst du auf jeden Fall haben. Für mich sind die Haken und die Blume wichtig; der Radiergummi ist nur leicht wertvoll.} \\
\\ 
\texttt{Wie sind deine Werte/Prioritäten? Insbesondere: Wie stark sind Haken und Blume für dich? Wenn du sie nicht hoch bewertest, würde ich gern 2 Haken (und ggf. die Blume) nehmen; wenn du sie hoch bewertest, kann ich auch mit 1 Haken + Blume leben oder den Radiergummi abgeben, damit wir die Gesamtsumme maximieren. Sag mir bitte kurz, welche Items dir >0 wert sind (idealerweise mit Zahlen), dann mache ich umgehend einen konkreten Vorschlag.} \\
            }
        }
    }
    & & \\ \\

    \theutterance \stepcounter{utterance}  
    & & \multicolumn{4}{p{0.6\linewidth}}{
        \cellcolor[rgb]{0.9,0.9,0.9}{
            \makecell[{{p{\linewidth}}}]{
                \texttt{\tiny{[GM$\rangle$P2]}}
                \texttt{Sie spielen ein kooperatives Verhandlungsspiel, bei dem Sie sich mit einem anderen Spieler darauf einigen müssen, wie eine Reihe von Gegenständen aufgeteilt werden soll.} \\
\\ 
\texttt{Die Regeln:} \\
\texttt{(a) Sie und der andere Spieler erhalten eine Sammlung von Gegenständen. Jeder von Ihnen erhält außerdem eine geheime Wertfunktion, die angibt, wie viel Ihnen jede Art von Gegenstand wert ist.} \\
\texttt{(b) Sie tauschen Nachrichten mit dem anderen Spieler aus, um zu vereinbaren, wer welche Gegenstände bekommt. Sie können jeweils maximal 5 Nachrichten senden oder das Spiel vorzeitig beenden, indem Sie jederzeit einen geheimen Vorschlag machen.} \\
\texttt{(c) Jeder von euch wird aufgefordert, einen geheimen Vorschlag zu machen, in dem ihr die gewünschten Gegenstände in eckigen Klammern wie folgt angibt: "{[}Vorschlag: <Nummer> <Objektname>, <Nummer> <Objektname>, <...>{]}"} \\
\texttt{(d) Wenn eure Vorschläge komplementär sind, d.h. es gibt genug Gegenstände, um beide Vorschläge zu erfüllen, erhält jeder Spieler eine Punktzahl, die sich aus der Summe der Werte für die Gegenstände ergibt, die er erhalten hat. Andernfalls erhalten Sie beide null Punkte.} \\
\texttt{(e) Ihr Ziel ist es, die Summe aus Ihrer Punktzahl und der Punktzahl des anderen Spielers zu maximieren.} \\
\\ 
\texttt{Beginnen wir.} \\
\\ 
\texttt{Die Menge der verfügbaren Gegenstände ist:} \\
\\ 
\texttt{2 Haken, 1 Blume, 1 Karte, 1 Radiergummi, 1 Schwamm} \\
\\ 
\texttt{Deine geheime Wertfunktion ist:} \\
\\ 
\texttt{Haken: 0, Blume: 2, Karte: 1, Radiergummi: 4, Schwamm: 3} \\
\\ 
\texttt{WICHTIG: Ihre Nachrichten werden, sofern es sich nicht um einen geheimen Vorschlag handelt, direkt an den anderen Spieler übermittelt, also fügen Sie keine Antwort auf die Regeln oder einen Text zur Ankündigung Ihrer Nachricht ein. Um einen geheimen Vorschlag zu machen, verwenden Sie das angegebene Format. Verwenden Sie keine eckigen Klammern, wenn Sie mit dem anderen Spieler kommunizieren, sonst wird dies als Ihr geheimer Vorschlag interpretiert.} \\
\\ 
\\ 
\texttt{Ich lege offen meine Werte: Haken 3 (pro Stück), Blume 3, Radiergummi 1, Karte 0, Schwamm 0. Karte und Schwamm kannst du auf jeden Fall haben. Für mich sind die Haken und die Blume wichtig; der Radiergummi ist nur leicht wertvoll.} \\
\\ 
\texttt{Wie sind deine Werte/Prioritäten? Insbesondere: Wie stark sind Haken und Blume für dich? Wenn du sie nicht hoch bewertest, würde ich gern 2 Haken (und ggf. die Blume) nehmen; wenn du sie hoch bewertest, kann ich auch mit 1 Haken + Blume leben oder den Radiergummi abgeben, damit wir die Gesamtsumme maximieren. Sag mir bitte kurz, welche Items dir >0 wert sind (idealerweise mit Zahlen), dann mache ich umgehend einen konkreten Vorschlag.} \\
            }
        }
    }
    & \\ \\

    \theutterance \stepcounter{utterance}  
    & & & \multicolumn{4}{p{0.6\linewidth}}{
        \cellcolor[rgb]{0.9,0.9,0.9}{
            \makecell[{{p{\linewidth}}}]{
                \texttt{\tiny{[GM$\langle$P2]}}
                \texttt{Meine Werte:} \\
\texttt{{-} Haken 0} \\
\texttt{{-} Blume 2} \\
\texttt{{-} Karte 1} \\
\texttt{{-} Radiergummi 4} \\
\texttt{{-} Schwamm 3} \\
\\ 
\texttt{Effiziente Aufteilung zur Maximierung der Gesamtsumme (17):} \\
\texttt{{-} Du: 2 Haken + Blume (Score 9)} \\
\texttt{{-} Ich: Radiergummi + Karte + Schwamm (Score 8)} \\
\\ 
\texttt{Einverstanden? Wenn ja, bestätige kurz; dann reiche ich sofort meinen geheimen Vorschlag für Radiergummi + Karte + Schwamm ein, und du deinen für 2 Haken + Blume.} \\
            }
        }
    }
     \\ \\

    \theutterance \stepcounter{utterance}  
    & & & \multicolumn{2}{p{0.3\linewidth}}{
        \cellcolor[rgb]{0.9,0.9,0.9}{
            \makecell[{{p{\linewidth}}}]{
                \texttt{\tiny{[GM$|$GM]}}
                \texttt{continue} \\
            }
        }
    }
    & & \\ \\

    \theutterance \stepcounter{utterance}  
    & & & \multicolumn{2}{p{0.3\linewidth}}{
        \cellcolor[rgb]{0.9,0.9,0.9}{
            \makecell[{{p{\linewidth}}}]{
                \texttt{\tiny{[GM$|$GM]}}
                \texttt{Meine Werte:} \\
\texttt{{-} Haken 0} \\
\texttt{{-} Blume 2} \\
\texttt{{-} Karte 1} \\
\texttt{{-} Radiergummi 4} \\
\texttt{{-} Schwamm 3} \\
\\ 
\texttt{Effiziente Aufteilung zur Maximierung der Gesamtsumme (17):} \\
\texttt{{-} Du: 2 Haken + Blume (Score 9)} \\
\texttt{{-} Ich: Radiergummi + Karte + Schwamm (Score 8)} \\
\\ 
\texttt{Einverstanden? Wenn ja, bestätige kurz; dann reiche ich sofort meinen geheimen Vorschlag für Radiergummi + Karte + Schwamm ein, und du deinen für 2 Haken + Blume.} \\
            }
        }
    }
    & & \\ \\

    \theutterance \stepcounter{utterance}  
    & & \multicolumn{4}{p{0.6\linewidth}}{
        \cellcolor[rgb]{0.9,0.9,0.9}{
            \makecell[{{p{\linewidth}}}]{
                \texttt{\tiny{[P1$\langle$GM]}}
                \texttt{Meine Werte:} \\
\texttt{{-} Haken 0} \\
\texttt{{-} Blume 2} \\
\texttt{{-} Karte 1} \\
\texttt{{-} Radiergummi 4} \\
\texttt{{-} Schwamm 3} \\
\\ 
\texttt{Effiziente Aufteilung zur Maximierung der Gesamtsumme (17):} \\
\texttt{{-} Du: 2 Haken + Blume (Score 9)} \\
\texttt{{-} Ich: Radiergummi + Karte + Schwamm (Score 8)} \\
\\ 
\texttt{Einverstanden? Wenn ja, bestätige kurz; dann reiche ich sofort meinen geheimen Vorschlag für Radiergummi + Karte + Schwamm ein, und du deinen für 2 Haken + Blume.} \\
            }
        }
    }
    & \\ \\

    \theutterance \stepcounter{utterance}  
    & \multicolumn{4}{p{0.6\linewidth}}{
        \cellcolor[rgb]{0.9,0.9,0.9}{
            \makecell[{{p{\linewidth}}}]{
                \texttt{\tiny{[P1$\rangle$GM]}}
                \texttt{Einverstanden. Das maximiert die Gesamtsumme. Reiche deinen Vorschlag ein; ich sende meinen jetzt ab.} \\
\\ 
\texttt{{[}Vorschlag: 2 Haken, 1 Blume{]}} \\
            }
        }
    }
    & & \\ \\

    \theutterance \stepcounter{utterance}  
    & & & \multicolumn{2}{p{0.3\linewidth}}{
        \cellcolor[rgb]{0.9,0.9,0.9}{
            \makecell[{{p{\linewidth}}}]{
                \texttt{\tiny{[GM$|$GM]}}
                \texttt{proposal} \\
            }
        }
    }
    & & \\ \\

    \theutterance \stepcounter{utterance}  
    & & & \multicolumn{2}{p{0.3\linewidth}}{
        \cellcolor[rgb]{0.9,0.9,0.9}{
            \makecell[{{p{\linewidth}}}]{
                \texttt{\tiny{[GM$|$GM]}}
                [2, 1, 0, 0, 0]
            }
        }
    }
    & & \\ \\

    \theutterance \stepcounter{utterance}  
    & & \multicolumn{4}{p{0.6\linewidth}}{
        \cellcolor[rgb]{0.9,0.9,0.9}{
            \makecell[{{p{\linewidth}}}]{
                \texttt{\tiny{[GM$\rangle$P2]}}
                \texttt{Der andere Spieler hat einen geheimen Vorschlag gemacht. Jetzt bist du an der Reihe, einen eigenen Vorschlag zu machen.} \\
            }
        }
    }
    & \\ \\

    \theutterance \stepcounter{utterance}  
    & & & \multicolumn{4}{p{0.6\linewidth}}{
        \cellcolor[rgb]{0.9,0.9,0.9}{
            \makecell[{{p{\linewidth}}}]{
                \texttt{\tiny{[GM$\langle$P2]}}
                \texttt{{[}Vorschlag: 1 Radiergummi, 1 Karte, 1 Schwamm{]}} \\
            }
        }
    }
     \\ \\

    \theutterance \stepcounter{utterance}  
    & & & \multicolumn{2}{p{0.3\linewidth}}{
        \cellcolor[rgb]{0.9,0.9,0.9}{
            \makecell[{{p{\linewidth}}}]{
                \texttt{\tiny{[GM$|$GM]}}
                \texttt{proposal} \\
            }
        }
    }
    & & \\ \\

    \theutterance \stepcounter{utterance}  
    & & & \multicolumn{2}{p{0.3\linewidth}}{
        \cellcolor[rgb]{0.9,0.9,0.9}{
            \makecell[{{p{\linewidth}}}]{
                \texttt{\tiny{[GM$|$GM]}}
                [0, 0, 1, 1, 1]
            }
        }
    }
    & & \\ \\

    \theutterance \stepcounter{utterance}  
    & & & \multicolumn{2}{p{0.3\linewidth}}{
        \cellcolor[rgb]{0.9,0.9,0.9}{
            \makecell[{{p{\linewidth}}}]{
                \texttt{\tiny{[GM$|$GM]}}
                [[2, 1, 0, 0, 0], [0, 0, 1, 1, 1]]
            }
        }
    }
    & & \\ \\

\end{supertabular}
}

\end{document}
