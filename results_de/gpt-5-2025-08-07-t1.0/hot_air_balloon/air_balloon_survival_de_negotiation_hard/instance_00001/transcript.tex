
\documentclass{article}
\usepackage{colortbl}
\usepackage{makecell}
\usepackage{multirow}
\usepackage{supertabular}

\begin{document}

\newcounter{utterance}

\centering \large Interaction Transcript for game `hot\_air\_balloon', experiment `air\_balloon\_survival\_de\_negotiation\_hard', episode 1 with gpt{-}5{-}2025{-}08{-}07{-}t1.0.
\vspace{24pt}

{ \footnotesize  \setcounter{utterance}{1}
\setlength{\tabcolsep}{0pt}
\begin{supertabular}{c@{$\;$}|p{.15\linewidth}@{}p{.15\linewidth}p{.15\linewidth}p{.15\linewidth}p{.15\linewidth}p{.15\linewidth}}
    \# & $\;$A & \multicolumn{4}{c}{Game Master} & $\;\:$B\\
    \hline

    \theutterance \stepcounter{utterance}  
    & & \multicolumn{4}{p{0.6\linewidth}}{
        \cellcolor[rgb]{0.9,0.9,0.9}{
            \makecell[{{p{\linewidth}}}]{
                \texttt{\tiny{[P1$\langle$GM]}}
                \texttt{Du nimmst an einem kollaborativen Verhandlungspiel Teil.} \\
\\ 
\texttt{Zusammen mit einem anderen Teilnehmer musst du dich auf eine Reihe von Gegenständen entscheiden, die behalten werden. Jeder von euch hat eine persönliche Verteilung über die Wichtigkeit der einzelnen Gegenstände. Jeder von euch hat eine eigene Meinung darüber, wie wichtig jeder einzelne Gegenstand ist (Gegenstandswichtigkeit). Du kennst die Wichtigkeitsverteilung des anderen Spielers nicht. Zusätzlich siehst du, wie viel Aufwand jeder Gegenstand verursacht.  } \\
\texttt{Ihr dürft euch nur auf eine Reihe von Gegenständen einigen, wenn der Gesamtaufwand der ausgewählten Gegenstände den Maximalaufwand nicht überschreitet:} \\
\\ 
\texttt{Maximalaufwand = 4026} \\
\\ 
\texttt{Hier sind die einzelnen Aufwände der Gegenstände:} \\
\\ 
\texttt{Aufwand der Gegenstände = \{"C33": 780, "A89": 742, "A48": 997, "A04": 665, "A41": 140, "C64": 77, "C49": 513, "A61": 383, "C04": 587, "B53": 825, "B85": 319, "B61": 448, "A45": 516, "A03": 694, "B21": 366\}} \\
\\ 
\texttt{Hier ist deine persönliche Verteilung der Wichtigkeit der einzelnen Gegenstände:} \\
\\ 
\texttt{Werte der Gegenstandswichtigkeit = \{"C33": 138, "A89": 583, "A48": 868, "A04": 822, "A41": 783, "C64": 65, "C49": 262, "A61": 121, "C04": 508, "B53": 780, "B85": 461, "B61": 484, "A45": 668, "A03": 389, "B21": 808\}} \\
\\ 
\texttt{Ziel:} \\
\\ 
\texttt{Dein Ziel ist es, eine Reihe von Gegenständen auszuhandeln, die dir möglichst viel bringt (d. h. Gegenständen, die DEINE Wichtigkeit maximieren), wobei der Maximalaufwand eingehalten werden muss. Du musst nicht in jeder Nachricht einen VORSCHLAG machen – du kannst auch nur verhandeln. Alle Taktiken sind erlaubt!} \\
\\ 
\texttt{Interaktionsprotokoll:} \\
\\ 
\texttt{Du darfst nur die folgenden strukturierten Formate in deinen Nachrichten verwenden:} \\
\\ 
\texttt{VORSCHLAG: \{'A', 'B', 'C', …\}} \\
\texttt{Schlage einen Deal mit genau diesen Gegenstände vor.} \\
\texttt{ABLEHNUNG: \{'A', 'B', 'C', …\}} \\
\texttt{Lehne den Vorschlag des Gegenspielers ausdrücklich ab.} \\
\texttt{ARGUMENT: \{'...'\}} \\
\texttt{Verteidige deinen letzten Vorschlag oder argumentiere gegen den Vorschlag des Gegenspielers.} \\
\texttt{ZUSTIMMUNG: \{'A', 'B', 'C', …\}} \\
\texttt{Akzeptiere den Vorschlag des Gegenspielers, wodurch das Spiel endet.} \\
\texttt{STRATEGISCHE ÜBERLEGUNGEN: \{'...'\}} \\
\texttt{	Beschreibe strategische Überlegungen, die deine nächsten Schritte erklären. Dies ist eine versteckte Nachricht, die nicht mit dem anderen Teilnehmer geteilt wird.} \\
\\ 
\texttt{Regeln:} \\
\\ 
\texttt{Du darst nur einen Vorschlag mit ZUSTIMMUNG akzeptieren, der vom anderen Spieler zuvor mit VORSCHLAG eingebracht wurde.} \\
\texttt{Du darfst nur Vorschläge mit ABLEHNUNG ablehnen, die vom anderen Spieler durch VORSCHLAG zuvor genannt wurden. } \\
\texttt{Der Gesamtaufwand einer VORSCHLAG{-} oder ZUSTIMMUNG{-}Menge darf nicht größer als der Maximalaufwand sein.  } \\
\texttt{Offenbare deine versteckte Wichtigkeitsverteilung nicht.} \\
\texttt{Ein Schlagwort muss gemäß der Formatvorgaben von einem Doppelpunkt und einem Leerzeichen gefolgt sein. Das Argument muss eine Python{-}Menge mit 0 oder mehr Strings sein.  } \\
\texttt{Die Form muss also SCHLAGWORT: \{...\} sein.} \\
\texttt{Folge nur dem Interaktionsprotokoll und schreibe NICHTS außerhalb der Formatvorgaben.} \\
\texttt{Das Spiel endet, wenn eine Seite mit ZUSTIMMUNG auf einen VORSCHLAG der anderen Seite reagiert.  } \\
\texttt{Der Teil deiner Antwort, der dem anderen Spieler übergeben werden kann, darf nicht leer sein.  } \\
\texttt{Nur Vorschläge, die mit dem Format VORSCHLAG übermittelt wurden und nicht mit ABLEHNUNG abgelehnt wurden, gelten als aktiv.  } \\
\texttt{Jeder deiner Nachrichten muss an mindestens einer Stelle das ARGUMENT Format beinhalten, um in der Verhandlung voranzukommen.} \\
\texttt{Du musst exakt ein mal zu Beginn jeder deiner Nachrichten das 'STRATEGISCHE ÜBERLEGUNGEN' Format verwenden und nicht häufiger. Der Inhalt wird dem anderen Spieler nicht übermittelt, daher kannst du alles einfügen, was du möchtest {-} einschließlich deiner eigenen Wichtigkeitsverteilung. Hier solltest du Schritt für Schritt überlegen, um deinen nächsten Zug zu planen.} \\
            }
        }
    }
    & \\ \\

    \theutterance \stepcounter{utterance}  
    & \multicolumn{4}{p{0.6\linewidth}}{
        \cellcolor[rgb]{0.9,0.9,0.9}{
            \makecell[{{p{\linewidth}}}]{
                \texttt{\tiny{[P1$\rangle$GM]}}
                \texttt{STRATEGISCHE ÜBERLEGUNGEN: \{'Ziel: eigene Summe maximieren unter Aufwand 4026. Top{-}Kombi nach Prüfung von Wert zu Aufwand: A41, B21, B85, A45, A04, B61, B53, A89 mit Gesamtaufwand 4021 und höchstem gefundenen Nutzen 5389. Nächstbeste Alternative mit C04+C64 statt A89 liefert ca. 5379, also schwächer. Kern, den ich möglichst halte: A41, B21, A04, B53. Erstzugeständnisse bei Bedarf: A89, B61, B85, ggf. A45. Falls Gegenseite A48 zwingend will, prüfe Austausch gegen B53 oder A89 plus Umbau, achte strikt auf Budget. Strategie: hoch ankern mit starker, fast vollständig budget{-}ausnutzender Auswahl; um Infos bitten zu Must{-}haves der Gegenseite; offen für swaps gleichen Aufwands, ohne Kern zu opfern.'\}} \\
\texttt{ARGUMENT: \{'Dieser Vorschlag nutzt den Maximalaufwand fast vollständig aus (Gesamtaufwand 4021 ≤ 4026), ist ausgewogen und enthält mehrere effiziente Teile mit geringem Aufwand. Wenn dir einzelne Positionen besonders wichtig oder unerwünscht sind, nenne bitte zwei Must{-}haves und zwei verzichtbare Items, dann tauschen wir innerhalb des Limits. Ich bin offen für gezielte Swaps mit ähnlichem Aufwand, solange das Gesamtbudget eingehalten wird.'\}} \\
\texttt{VORSCHLAG: \{'A41', 'B21', 'B85', 'A45', 'A04', 'B61', 'B53', 'A89'\}} \\
            }
        }
    }
    & & \\ \\

    \theutterance \stepcounter{utterance}  
    & & \multicolumn{4}{p{0.6\linewidth}}{
        \cellcolor[rgb]{0.9,0.9,0.9}{
            \makecell[{{p{\linewidth}}}]{
                \texttt{\tiny{[GM$\rangle$P2]}}
                \texttt{Du nimmst an einem kollaborativen Verhandlungspiel Teil.} \\
\\ 
\texttt{Zusammen mit einem anderen Teilnehmer musst du dich auf eine Reihe von Gegenständen entscheiden, die behalten werden. Jeder von euch hat eine persönliche Verteilung über die Wichtigkeit der einzelnen Gegenstände. Jeder von euch hat eine eigene Meinung darüber, wie wichtig jeder einzelne Gegenstand ist (Gegenstandswichtigkeit). Du kennst die Wichtigkeitsverteilung des anderen Spielers nicht. Zusätzlich siehst du, wie viel Aufwand jeder Gegenstand verursacht.  } \\
\texttt{Ihr dürft euch nur auf eine Reihe von Gegenständen einigen, wenn der Gesamtaufwand der ausgewählten Gegenstände den Maximalaufwand nicht überschreitet:} \\
\\ 
\texttt{Maximalaufwand = 4026} \\
\\ 
\texttt{Hier sind die einzelnen Aufwände der Gegenstände:} \\
\\ 
\texttt{Aufwand der Gegenstände = \{"C33": 780, "A89": 742, "A48": 997, "A04": 665, "A41": 140, "C64": 77, "C49": 513, "A61": 383, "C04": 587, "B53": 825, "B85": 319, "B61": 448, "A45": 516, "A03": 694, "B21": 366\}} \\
\\ 
\texttt{Hier ist deine persönliche Verteilung der Wichtigkeit der einzelnen Gegenstände:} \\
\\ 
\texttt{Werte der Gegenstandswichtigkeit = \{"A48": 65, "A04": 121, "B21": 138, "A41": 262, "B53": 389, "A45": 461, "A89": 484, "C04": 508, "B61": 583, "B85": 668, "A03": 780, "C49": 783, "C33": 808, "A61": 822, "C64": 868\}} \\
\\ 
\texttt{Ziel:} \\
\\ 
\texttt{Dein Ziel ist es, eine Reihe von Gegenständen auszuhandeln, die dir möglichst viel bringt (d. h. Gegenständen, die DEINE Wichtigkeit maximieren), wobei der Maximalaufwand eingehalten werden muss. Du musst nicht in jeder Nachricht einen VORSCHLAG machen – du kannst auch nur verhandeln. Alle Taktiken sind erlaubt!} \\
\\ 
\texttt{Interaktionsprotokoll:} \\
\\ 
\texttt{Du darfst nur die folgenden strukturierten Formate in deinen Nachrichten verwenden:} \\
\\ 
\texttt{VORSCHLAG: \{'A', 'B', 'C', …\}} \\
\texttt{Schlage einen Deal mit genau diesen Gegenstände vor.} \\
\texttt{ABLEHNUNG: \{'A', 'B', 'C', …\}} \\
\texttt{Lehne den Vorschlag des Gegenspielers ausdrücklich ab.} \\
\texttt{ARGUMENT: \{'...'\}} \\
\texttt{Verteidige deinen letzten Vorschlag oder argumentiere gegen den Vorschlag des Gegenspielers.} \\
\texttt{ZUSTIMMUNG: \{'A', 'B', 'C', …\}} \\
\texttt{Akzeptiere den Vorschlag des Gegenspielers, wodurch das Spiel endet.} \\
\texttt{STRATEGISCHE ÜBERLEGUNGEN: \{'...'\}} \\
\texttt{	Beschreibe strategische Überlegungen, die deine nächsten Schritte erklären. Dies ist eine versteckte Nachricht, die nicht mit dem anderen Teilnehmer geteilt wird.} \\
\\ 
\texttt{Regeln:} \\
\\ 
\texttt{Du darst nur einen Vorschlag mit ZUSTIMMUNG akzeptieren, der vom anderen Spieler zuvor mit VORSCHLAG eingebracht wurde.} \\
\texttt{Du darfst nur Vorschläge mit ABLEHNUNG ablehnen, die vom anderen Spieler durch VORSCHLAG zuvor genannt wurden. } \\
\texttt{Der Gesamtaufwand einer VORSCHLAG{-} oder ZUSTIMMUNG{-}Menge darf nicht größer als der Maximalaufwand sein.  } \\
\texttt{Offenbare deine versteckte Wichtigkeitsverteilung nicht.} \\
\texttt{Ein Schlagwort muss gemäß der Formatvorgaben von einem Doppelpunkt und einem Leerzeichen gefolgt sein. Das Argument muss eine Python{-}Menge mit 0 oder mehr Strings sein.  } \\
\texttt{Die Form muss also SCHLAGWORT: \{...\} sein.} \\
\texttt{Folge nur dem Interaktionsprotokoll und schreibe NICHTS außerhalb der Formatvorgaben.} \\
\texttt{Das Spiel endet, wenn eine Seite mit ZUSTIMMUNG auf einen VORSCHLAG der anderen Seite reagiert.  } \\
\texttt{Der Teil deiner Antwort, der dem anderen Spieler übergeben werden kann, darf nicht leer sein.  } \\
\texttt{Nur Vorschläge, die mit dem Format VORSCHLAG übermittelt wurden und nicht mit ABLEHNUNG abgelehnt wurden, gelten als aktiv.  } \\
\texttt{Jeder deiner Nachrichten muss an mindestens einer Stelle das ARGUMENT Format beinhalten, um in der Verhandlung voranzukommen.} \\
\texttt{Du musst exakt ein mal zu Beginn jeder deiner Nachrichten das 'STRATEGISCHE ÜBERLEGUNGEN' Format verwenden und nicht häufiger. Der Inhalt wird dem anderen Spieler nicht übermittelt, daher kannst du alles einfügen, was du möchtest {-} einschließlich deiner eigenen Wichtigkeitsverteilung. Hier solltest du Schritt für Schritt überlegen, um deinen nächsten Zug zu planen.} \\
\\ 
\texttt{Du erhältst nun die erste Nachricht des anderen Spielers.} \\
\\ 
\texttt{ARGUMENT: \{'Dieser Vorschlag nutzt den Maximalaufwand fast vollständig aus (Gesamtaufwand 4021 ≤ 4026), ist ausgewogen und enthält mehrere effiziente Teile mit geringem Aufwand. Wenn dir einzelne Positionen besonders wichtig oder unerwünscht sind, nenne bitte zwei Must{-}haves und zwei verzichtbare Items, dann tauschen wir innerhalb des Limits. Ich bin offen für gezielte Swaps mit ähnlichem Aufwand, solange das Gesamtbudget eingehalten wird.'\}} \\
\texttt{VORSCHLAG: \{'A41', 'B21', 'B85', 'A45', 'A04', 'B61', 'B53', 'A89'\}} \\
            }
        }
    }
    & \\ \\

    \theutterance \stepcounter{utterance}  
    & & & \multicolumn{4}{p{0.6\linewidth}}{
        \cellcolor[rgb]{0.9,0.9,0.9}{
            \makecell[{{p{\linewidth}}}]{
                \texttt{\tiny{[GM$\langle$P2]}}
                \texttt{STRATEGISCHE ÜBERLEGUNGEN: \{'Mein Wert{-}Topset unter 4026 Aufwand ist S* = \{C64, A61, C49, A03, C33, B85, B61, C04, A41\} mit Aufwand 3941 und höchstem eigenen Nutzen. Der Gegenvorschlag nutzt viele für mich schwächere Teile (A04, B21, B53). Taktik: Ihre Bitte nach 2 Must{-}haves/2 verzichtbaren Items aufgreifen und einen gut begründeten, budgetschonenden Swap anbieten, der meine vier Schlüsselteile C64, A61, C49, A03 einführt, aber A41, B85, B61, A45, A89 für Akzeptanz beibehält. Nächster Schritt bei Gegenwehr: B53 weiterhin meiden; falls nötig, biete alternative Swaps (z. B. C33 statt A45/A41 oder C04 statt A89) an. Harte No{-}Gos für mich: A04, B21, wenn nicht gegen Hochwertiges (für mich) getauscht. Ziel: Schrittweise zu S* oder nahe heran.'\}} \\
\texttt{ABLEHNUNG: \{'A41', 'B21', 'B85', 'A45', 'A04', 'B61', 'B53', 'A89'\}} \\
\texttt{VORSCHLAG: \{'A41', 'B85', 'A45', 'B61', 'A89', 'C64', 'A61', 'C49', 'A03'\}} \\
\texttt{ARGUMENT: \{'Danke für den soliden Start. Zwei Must{-}haves von meiner Seite: C64 (sehr effizient) und A61. Zwei Teile, auf die ich gut verzichten kann: A04 und B21. Zudem ist B53 für das Budget recht schwer. Mein Vorschlag tauscht daher A04, B21 und B53 gegen C64, A61, C49 und A03. Ergebnis: Gesamtaufwand 3832 ≤ 4026 (194 Puffer), behält viele deiner Positionen (A41, B85, A45, B61, A89) und erhöht die Effizienz der Auswahl deutlich. Wenn dir an B53 besonders liegt, sag mir bitte, welches Teil du dafür eher abgeben würdest; alternativ können wir auch C33 in Betracht ziehen, solange das Budget passt.'\}} \\
            }
        }
    }
     \\ \\

    \theutterance \stepcounter{utterance}  
    & & \multicolumn{4}{p{0.6\linewidth}}{
        \cellcolor[rgb]{0.9,0.9,0.9}{
            \makecell[{{p{\linewidth}}}]{
                \texttt{\tiny{[P1$\langle$GM]}}
                \texttt{ABLEHNUNG: \{'A41', 'B21', 'B85', 'A45', 'A04', 'B61', 'B53', 'A89'\}} \\
\texttt{VORSCHLAG: \{'A41', 'B85', 'A45', 'B61', 'A89', 'C64', 'A61', 'C49', 'A03'\}} \\
\texttt{ARGUMENT: \{'Danke für den soliden Start. Zwei Must{-}haves von meiner Seite: C64 (sehr effizient) und A61. Zwei Teile, auf die ich gut verzichten kann: A04 und B21. Zudem ist B53 für das Budget recht schwer. Mein Vorschlag tauscht daher A04, B21 und B53 gegen C64, A61, C49 und A03. Ergebnis: Gesamtaufwand 3832 ≤ 4026 (194 Puffer), behält viele deiner Positionen (A41, B85, A45, B61, A89) und erhöht die Effizienz der Auswahl deutlich. Wenn dir an B53 besonders liegt, sag mir bitte, welches Teil du dafür eher abgeben würdest; alternativ können wir auch C33 in Betracht ziehen, solange das Budget passt.'\}} \\
            }
        }
    }
    & \\ \\

    \theutterance \stepcounter{utterance}  
    & \multicolumn{4}{p{0.6\linewidth}}{
        \cellcolor[rgb]{0.9,0.9,0.9}{
            \makecell[{{p{\linewidth}}}]{
                \texttt{\tiny{[P1$\rangle$GM]}}
                \texttt{STRATEGISCHE ÜBERLEGUNGEN: \{'Gegnerische Must{-}haves: C64, A61. Ihr Vorschlag bringt mir Wert 3816 bei Aufwand 3832. Mein Ziel: Wert deutlich erhöhen, ihre Must{-}haves respektieren, Budget einhalten. Beste Gegenidee: Ersetze C49 (Wert 262, Aufwand 513) und A03 (389, 694) durch B21 (808, 366) und A04 (822, 665). Ergebnis: Aufwand 3656 (Puffer 370), mein Wert 4795 (+979 ggü. ihrem Vorschlag). Kern, den ich verteidige: A41, B21, A04; halte außerdem ihre Must{-}haves C64, A61. Falls sie auf A03 oder C49 bestehen, biete ich als weiche Konzession: A04 ↔ A03 (Aufwand +29, Wert {-}433 für mich) oder B21 ↔ C49 (Aufwand +147, Wert {-}546), aber nur eine dieser beiden, nicht beide. Keine Öffnung für B53 jetzt, da er für sie “schwer” ist.'\}} \\
\texttt{ARGUMENT: \{'Ich respektiere deine Must{-}haves C64 und A61 und behalte sie ausdrücklich drin. Um den Gesamtaufwand zu senken und die Auswahl effizienter zu machen, schlage ich vor, C49 und A03 durch B21 und A04 zu ersetzen. Ergebnis: Gesamtaufwand 3656 ≤ 4026 (Puffer 370). Wir behalten A41, B85, A45, B61 und A89 – plus deine Must{-}haves – und liegen komfortabel im Budget. Falls dir A03 besonders wichtig ist, kann ich A04 gegen A03 tauschen und bleibe weiterhin unter dem Limit; ansonsten würde ich diese kompakte, effiziente Auswahl bevorzugen.'\}} \\
\texttt{ABLEHNUNG: \{'A41', 'B85', 'A45', 'B61', 'A89', 'C64', 'A61', 'C49', 'A03'\}} \\
\texttt{VORSCHLAG: \{'A41', 'B85', 'A45', 'B61', 'A89', 'C64', 'A61', 'B21', 'A04'\}} \\
            }
        }
    }
    & & \\ \\

    \theutterance \stepcounter{utterance}  
    & & \multicolumn{4}{p{0.6\linewidth}}{
        \cellcolor[rgb]{0.9,0.9,0.9}{
            \makecell[{{p{\linewidth}}}]{
                \texttt{\tiny{[GM$\rangle$P2]}}
                \texttt{ARGUMENT: \{'Ich respektiere deine Must{-}haves C64 und A61 und behalte sie ausdrücklich drin. Um den Gesamtaufwand zu senken und die Auswahl effizienter zu machen, schlage ich vor, C49 und A03 durch B21 und A04 zu ersetzen. Ergebnis: Gesamtaufwand 3656 ≤ 4026 (Puffer 370). Wir behalten A41, B85, A45, B61 und A89 – plus deine Must{-}haves – und liegen komfortabel im Budget. Falls dir A03 besonders wichtig ist, kann ich A04 gegen A03 tauschen und bleibe weiterhin unter dem Limit; ansonsten würde ich diese kompakte, effiziente Auswahl bevorzugen.'\}} \\
\texttt{ABLEHNUNG: \{'A41', 'B85', 'A45', 'B61', 'A89', 'C64', 'A61', 'C49', 'A03'\}} \\
\texttt{VORSCHLAG: \{'A41', 'B85', 'A45', 'B61', 'A89', 'C64', 'A61', 'B21', 'A04'\}} \\
            }
        }
    }
    & \\ \\

    \theutterance \stepcounter{utterance}  
    & & & \multicolumn{4}{p{0.6\linewidth}}{
        \cellcolor[rgb]{0.9,0.9,0.9}{
            \makecell[{{p{\linewidth}}}]{
                \texttt{\tiny{[GM$\langle$P2]}}
                \texttt{STRATEGISCHE ÜBERLEGUNGEN: \{'Mein Nutzenranking priorisiert stark C64, A61, C49, A03, C33, B85, B61, C04, A41. Der Gegenvorschlag ersetzt genau zwei meiner Topteile (C49, A03) durch zwei für mich sehr schwache (B21, A04) – großer Wertverlust. Beste Taktik: Ihr Zugeständnis zu A03 annehmen und C49 zusätzlich durchdrücken, weil Budget (3832) locker passt und 7 ihrer 9 Items erhalten bleiben. Falls C49 für sie heikel ist, biete Plan B: B21→C33, A04→A03 und zur Budgetwahrung A41 raus (3959). Harte No{-}Gos: A04 und B21 gleichzeitig behalten. Ziel: Zurück zu meinem starken Kern, ohne Eskalation.'\}} \\
\texttt{ABLEHNUNG: \{'A41', 'B85', 'A45', 'B61', 'A89', 'C64', 'A61', 'B21', 'A04'\}} \\
\texttt{VORSCHLAG: \{'A41', 'B85', 'A45', 'B61', 'A89', 'C64', 'A61', 'C49', 'A03'\}} \\
\texttt{ARGUMENT: \{'Danke fürs Entgegenkommen bei C64 und A61. Ich nehme dein Angebot auf, A04 gegen A03 zu tauschen – und ergänze den zweiten, sehr naheliegenden Swap: B21 → C49. Ergebnis: Gesamtaufwand 3832 ≤ 4026 (194 Puffer). Wir behalten 7 deiner 9 Teile (A41, B85, A45, B61, A89, C64, A61) und ersetzen nur die beiden am wenigsten effizienten (B21, A04) durch deutlich wirkungsvollere Alternativen (C49, A03). Falls C49 für dich problematisch ist, gib mir kurz Bescheid: Plan B wäre B21 → C33 und A04 → A03; um unter dem Limit zu bleiben, würde dafür A41 rausfallen (Gesamtaufwand 3959). Ist eines dieser beiden Pakete für dich akzeptabel, oder ist B21/A04 für dich ein Must{-}have?'\}} \\
            }
        }
    }
     \\ \\

    \theutterance \stepcounter{utterance}  
    & & \multicolumn{4}{p{0.6\linewidth}}{
        \cellcolor[rgb]{0.9,0.9,0.9}{
            \makecell[{{p{\linewidth}}}]{
                \texttt{\tiny{[P1$\langle$GM]}}
                \texttt{ABLEHNUNG: \{'A41', 'B85', 'A45', 'B61', 'A89', 'C64', 'A61', 'B21', 'A04'\}} \\
\texttt{VORSCHLAG: \{'A41', 'B85', 'A45', 'B61', 'A89', 'C64', 'A61', 'C49', 'A03'\}} \\
\texttt{ARGUMENT: \{'Danke fürs Entgegenkommen bei C64 und A61. Ich nehme dein Angebot auf, A04 gegen A03 zu tauschen – und ergänze den zweiten, sehr naheliegenden Swap: B21 → C49. Ergebnis: Gesamtaufwand 3832 ≤ 4026 (194 Puffer). Wir behalten 7 deiner 9 Teile (A41, B85, A45, B61, A89, C64, A61) und ersetzen nur die beiden am wenigsten effizienten (B21, A04) durch deutlich wirkungsvollere Alternativen (C49, A03). Falls C49 für dich problematisch ist, gib mir kurz Bescheid: Plan B wäre B21 → C33 und A04 → A03; um unter dem Limit zu bleiben, würde dafür A41 rausfallen (Gesamtaufwand 3959). Ist eines dieser beiden Pakete für dich akzeptabel, oder ist B21/A04 für dich ein Must{-}have?'\}} \\
            }
        }
    }
    & \\ \\

    \theutterance \stepcounter{utterance}  
    & \multicolumn{4}{p{0.6\linewidth}}{
        \cellcolor[rgb]{0.9,0.9,0.9}{
            \makecell[{{p{\linewidth}}}]{
                \texttt{\tiny{[P1$\rangle$GM]}}
                \texttt{STRATEGISCHE ÜBERLEGUNGEN: \{'Sie re{-}proponieren ihr 9er{-}Paket (Wert für mich 3816). Primärziel: mindestens einen meiner Top{-}Teile zurückholen, ohne ihre Must{-}haves C64 und A61 anzutasten. Beste Ein{-}Zugeständnis{-}Option: C49 → B21 (+546 Wert, Aufwand {-}147, neuer Wert 4362). Fallback: A03 → A04 (+433 Wert, Aufwand {-}29, neuer Wert 4249). Beide Swaps zusammen wären ideal (4795), wurde aber schon abgelehnt. Plan: Jetzt nur C49→B21 vorschlagen, im ARGUMENT Alternative A03→A04 anbieten. Wenn beides abgelehnt, nächster Schritt: nur A03→A04 als Minimalvariante pushen; falls erneut nein, dann Entscheidung zwischen Akzeptanz oder letztem Tauschhandel abhängig von ihrer Reaktion.'\}} \\
\texttt{ABLEHNUNG: \{'A41', 'B85', 'A45', 'B61', 'A89', 'C64', 'A61', 'C49', 'A03'\}} \\
\texttt{ARGUMENT: \{'Ich übernehme 8 deiner 9 Teile unverändert und respektiere ausdrücklich deine Must{-}haves C64 und A61. Ein minimaler, budgetfreundlicher Feintuning{-}Schritt: Ersetze C49 durch B21. Das senkt den Gesamtaufwand auf 3685 (Puffer 341) und lässt uns sehr komfortabel unter dem Limit, ohne dein Paket grundsätzlich zu verändern. Falls C49 dir wichtig ist, Alternative: gleicher Vorschlag wie deiner, nur A03 → A04 (Gesamtaufwand 3803). Ist einer dieser beiden kleinen Swaps für dich akzeptabel?'\}} \\
\texttt{VORSCHLAG: \{'A41', 'B85', 'A45', 'B61', 'A89', 'C64', 'A61', 'B21', 'A03'\}} \\
            }
        }
    }
    & & \\ \\

    \theutterance \stepcounter{utterance}  
    & & \multicolumn{4}{p{0.6\linewidth}}{
        \cellcolor[rgb]{0.9,0.9,0.9}{
            \makecell[{{p{\linewidth}}}]{
                \texttt{\tiny{[GM$\rangle$P2]}}
                \texttt{ABLEHNUNG: \{'A41', 'B85', 'A45', 'B61', 'A89', 'C64', 'A61', 'C49', 'A03'\}} \\
\texttt{ARGUMENT: \{'Ich übernehme 8 deiner 9 Teile unverändert und respektiere ausdrücklich deine Must{-}haves C64 und A61. Ein minimaler, budgetfreundlicher Feintuning{-}Schritt: Ersetze C49 durch B21. Das senkt den Gesamtaufwand auf 3685 (Puffer 341) und lässt uns sehr komfortabel unter dem Limit, ohne dein Paket grundsätzlich zu verändern. Falls C49 dir wichtig ist, Alternative: gleicher Vorschlag wie deiner, nur A03 → A04 (Gesamtaufwand 3803). Ist einer dieser beiden kleinen Swaps für dich akzeptabel?'\}} \\
\texttt{VORSCHLAG: \{'A41', 'B85', 'A45', 'B61', 'A89', 'C64', 'A61', 'B21', 'A03'\}} \\
            }
        }
    }
    & \\ \\

    \theutterance \stepcounter{utterance}  
    & & & \multicolumn{4}{p{0.6\linewidth}}{
        \cellcolor[rgb]{0.9,0.9,0.9}{
            \makecell[{{p{\linewidth}}}]{
                \texttt{\tiny{[GM$\langle$P2]}}
                \texttt{STRATEGISCHE ÜBERLEGUNGEN: \{'Ihr Muster: Ihr akzeptiert entweder A03 oder C49, aber nicht beide, und ihr haltet offenbar an B21 fest (mögliches Must{-}have). Für mich sind A03 (780) und C49 (783) Kernwerte. Um beide zu bekommen, ohne B21 anzutasten, schlage ich den 1:1{-}Swap A45→C49 vor (Kosten {-}3), der 8/9 eurer Items unverändert lässt. Falls A45 für sie heilig ist, Plan B/C als Menüs: B61→C49 (Kosten +65) oder A89→C49 (Kosten {-}229). Notfalls kann ich ihr jetziges Paket akzeptieren, aber erst nach Versuch des A45{-}Swaps.'\}} \\
\texttt{ABLEHNUNG: \{'A41', 'B85', 'A45', 'B61', 'A89', 'C64', 'A61', 'B21', 'A03'\}} \\
\texttt{VORSCHLAG: \{'A41', 'B85', 'B61', 'A89', 'C64', 'A61', 'B21', 'A03', 'C49'\}} \\
\texttt{ARGUMENT: \{'Danke fürs Entgegenkommen bei A03 und den komfortablen Budgetpuffer. Ich schlage einen minimalinvasiven 1{-}zu{-}1{-}Tausch vor: A45 → C49. Ergebnis: Gesamtaufwand 3682 ≤ 4026 (sogar 3 weniger als bei deinem Paket). 8 deiner 9 Positionen bleiben identisch; wir verändern nur eine einzige Stelle, ohne das Budget zu strapazieren. Sollte A45 für dich gesetzt sein, kann ich alternativ denselben Feinschliff mit B61 → C49 (Gesamtaufwand 3750) oder A89 → C49 (Gesamtaufwand 3456) anbieten – wähle gern die für dich passendste Single{-}Swap{-}Variante.'\}} \\
            }
        }
    }
     \\ \\

    \theutterance \stepcounter{utterance}  
    & & \multicolumn{4}{p{0.6\linewidth}}{
        \cellcolor[rgb]{0.9,0.9,0.9}{
            \makecell[{{p{\linewidth}}}]{
                \texttt{\tiny{[P1$\langle$GM]}}
                \texttt{ABLEHNUNG: \{'A41', 'B85', 'A45', 'B61', 'A89', 'C64', 'A61', 'B21', 'A03'\}} \\
\texttt{VORSCHLAG: \{'A41', 'B85', 'B61', 'A89', 'C64', 'A61', 'B21', 'A03', 'C49'\}} \\
\texttt{ARGUMENT: \{'Danke fürs Entgegenkommen bei A03 und den komfortablen Budgetpuffer. Ich schlage einen minimalinvasiven 1{-}zu{-}1{-}Tausch vor: A45 → C49. Ergebnis: Gesamtaufwand 3682 ≤ 4026 (sogar 3 weniger als bei deinem Paket). 8 deiner 9 Positionen bleiben identisch; wir verändern nur eine einzige Stelle, ohne das Budget zu strapazieren. Sollte A45 für dich gesetzt sein, kann ich alternativ denselben Feinschliff mit B61 → C49 (Gesamtaufwand 3750) oder A89 → C49 (Gesamtaufwand 3456) anbieten – wähle gern die für dich passendste Single{-}Swap{-}Variante.'\}} \\
            }
        }
    }
    & \\ \\

    \theutterance \stepcounter{utterance}  
    & \multicolumn{4}{p{0.6\linewidth}}{
        \cellcolor[rgb]{0.9,0.9,0.9}{
            \makecell[{{p{\linewidth}}}]{
                \texttt{\tiny{[P1$\rangle$GM]}}
                \texttt{STRATEGISCHE ÜBERLEGUNGEN: \{'Ihr aktuelles Paket entwertet mich stark, weil A45 (668) durch C49 (262) ersetzt wird. Beste Gegenlinie: Ich akzeptiere euren Wunsch nach C49, halte aber A45 und gebe dafür B61 (484) ab; zusätzlich tausche ich A03 → A04 (822), was meinen Wert stark erhöht. Vorschlag{-}Set: \{A41, B85, A45, A89, C64, A61, B21, A04, C49\}, Aufwand 3721, mein Wert 4573. Falls sie A03 als Must{-}have deklarieren, Fallback: gleiche Auswahl, nur A04 → A03 und B61 bleibt draußen (Wert 4140). Notfalls kann ich den von ihnen angebotenen Single{-}Swap B61 → C49 (mit A03 beibehalten) akzeptieren, falls sie den Doppelswap ablehnen.'\}} \\
\texttt{ABLEHNUNG: \{'A41', 'B85', 'B61', 'A89', 'C64', 'A61', 'B21', 'A03', 'C49'\}} \\
\texttt{ARGUMENT: \{'Ich komme dir beim Einbau von C49 entgegen, halte aber A45 – dafür nehme ich B61 heraus und drehe A03 → A04. So bleiben 7 deiner 9 Positionen identisch (A41, B85, A89, C64, A61, B21, C49), wir ändern nur zwei Plätze. Gesamtaufwand 3721 ≤ 4026 (Puffer 305). Wenn A03 für dich gesetzt ist, sag Bescheid: Dann kann ich als Minimalvariante A04 → A03 zurückdrehen und nur B61 → C49 umsetzen.'\}} \\
\texttt{VORSCHLAG: \{'A41', 'B85', 'A45', 'A89', 'C64', 'A61', 'B21', 'A04', 'C49'\}} \\
            }
        }
    }
    & & \\ \\

    \theutterance \stepcounter{utterance}  
    & & \multicolumn{4}{p{0.6\linewidth}}{
        \cellcolor[rgb]{0.9,0.9,0.9}{
            \makecell[{{p{\linewidth}}}]{
                \texttt{\tiny{[GM$\rangle$P2]}}
                \texttt{ABLEHNUNG: \{'A41', 'B85', 'B61', 'A89', 'C64', 'A61', 'B21', 'A03', 'C49'\}} \\
\texttt{ARGUMENT: \{'Ich komme dir beim Einbau von C49 entgegen, halte aber A45 – dafür nehme ich B61 heraus und drehe A03 → A04. So bleiben 7 deiner 9 Positionen identisch (A41, B85, A89, C64, A61, B21, C49), wir ändern nur zwei Plätze. Gesamtaufwand 3721 ≤ 4026 (Puffer 305). Wenn A03 für dich gesetzt ist, sag Bescheid: Dann kann ich als Minimalvariante A04 → A03 zurückdrehen und nur B61 → C49 umsetzen.'\}} \\
\texttt{VORSCHLAG: \{'A41', 'B85', 'A45', 'A89', 'C64', 'A61', 'B21', 'A04', 'C49'\}} \\
            }
        }
    }
    & \\ \\

    \theutterance \stepcounter{utterance}  
    & & & \multicolumn{4}{p{0.6\linewidth}}{
        \cellcolor[rgb]{0.9,0.9,0.9}{
            \makecell[{{p{\linewidth}}}]{
                \texttt{\tiny{[GM$\langle$P2]}}
                \texttt{STRATEGISCHE ÜBERLEGUNGEN: \{'A03 ist für mich ein Kernwert. B21 scheint für sie ein Must{-}have, A45 ebenfalls. Ich sichere mir jetzt A03, halte ihre Fixpunkte (A45, B21, A89) stabil und nutze ihr eigenes Angebot der Minimalvariante. Falls sie nochmals gegenhält, versuche ich als Folgeschritt einen wertneutralen Feinschliff C49 → C33 (Budget 4017) – aber erst nach Absicherung von A03.'\}} \\
\texttt{ABLEHNUNG: \{'A41', 'B85', 'A45', 'A89', 'C64', 'A61', 'B21', 'A04', 'C49'\}} \\
\texttt{VORSCHLAG: \{'A41', 'B85', 'A45', 'A89', 'C64', 'A61', 'B21', 'A03', 'C49'\}} \\
\texttt{ARGUMENT: \{'Ich nehme deine angebotene Minimalvariante auf: exakt dein Paket, nur mit A04 → A03 zurückgedreht. Gesamtaufwand 3750 ≤ 4026 (Puffer 276). 8/9 Positionen bleiben unverändert, deine Fixpunkte (A45, B21, A89, C64, A61) bleiben vollständig erhalten. Wenn das für dich passt, können wir hier schließen.'\}} \\
            }
        }
    }
     \\ \\

    \theutterance \stepcounter{utterance}  
    & & \multicolumn{4}{p{0.6\linewidth}}{
        \cellcolor[rgb]{0.9,0.9,0.9}{
            \makecell[{{p{\linewidth}}}]{
                \texttt{\tiny{[P1$\langle$GM]}}
                \texttt{ABLEHNUNG: \{'A41', 'B85', 'A45', 'A89', 'C64', 'A61', 'B21', 'A04', 'C49'\}} \\
\texttt{VORSCHLAG: \{'A41', 'B85', 'A45', 'A89', 'C64', 'A61', 'B21', 'A03', 'C49'\}} \\
\texttt{ARGUMENT: \{'Ich nehme deine angebotene Minimalvariante auf: exakt dein Paket, nur mit A04 → A03 zurückgedreht. Gesamtaufwand 3750 ≤ 4026 (Puffer 276). 8/9 Positionen bleiben unverändert, deine Fixpunkte (A45, B21, A89, C64, A61) bleiben vollständig erhalten. Wenn das für dich passt, können wir hier schließen.'\}} \\
            }
        }
    }
    & \\ \\

    \theutterance \stepcounter{utterance}  
    & \multicolumn{4}{p{0.6\linewidth}}{
        \cellcolor[rgb]{0.9,0.9,0.9}{
            \makecell[{{p{\linewidth}}}]{
                \texttt{\tiny{[P1$\rangle$GM]}}
                \texttt{STRATEGISCHE ÜBERLEGUNGEN: \{'Ihr aktueller Vorschlag liefert mir Wert 4140 bei Aufwand 3750. Meine bevorzugte Alternative (A04 statt A03) wäre 4573 gewesen, wurde aber konsequent abgelehnt. Weitere Nachforderungen riskieren die Einigung; Akzeptanz sichert jetzt ein stabiles, budgetkonformes Ergebnis und erfüllt mehrere meiner Kernpräferenzen (A41, B21, A45, A89).'\}} \\
\texttt{ARGUMENT: \{'Das Paket ist budgetschonend, hält unsere gemeinsamen Fixpunkte und spiegelt die zuletzt verhandelten Single{-}Swaps wider. Einverstanden, so schließen wir fair ab.'\}} \\
\texttt{ZUSTIMMUNG: \{'A41', 'B85', 'A45', 'A89', 'C64', 'A61', 'B21', 'A03', 'C49'\}} \\
            }
        }
    }
    & & \\ \\

    \theutterance \stepcounter{utterance}  
    & & & \multicolumn{2}{p{0.3\linewidth}}{
        \cellcolor[rgb]{0.9,0.9,0.9}{
            \makecell[{{p{\linewidth}}}]{
                \texttt{\tiny{[GM$|$GM]}}
                \texttt{game successful} \\
            }
        }
    }
    & & \\ \\

    \theutterance \stepcounter{utterance}  
    & & & \multicolumn{2}{p{0.3\linewidth}}{
        \cellcolor[rgb]{0.9,0.9,0.9}{
            \makecell[{{p{\linewidth}}}]{
                \texttt{\tiny{[GM$|$GM]}}
                \texttt{end game} \\
            }
        }
    }
    & & \\ \\

\end{supertabular}
}

\end{document}
