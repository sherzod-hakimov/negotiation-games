
\documentclass{article}
\usepackage{colortbl}
\usepackage{makecell}
\usepackage{multirow}
\usepackage{supertabular}

\begin{document}

\newcounter{utterance}

\centering \large Interaction Transcript for game `hot\_air\_balloon', experiment `air\_balloon\_survival\_de\_negotiation\_hard', episode 5 with gpt{-}5{-}2025{-}08{-}07{-}t1.0.
\vspace{24pt}

{ \footnotesize  \setcounter{utterance}{1}
\setlength{\tabcolsep}{0pt}
\begin{supertabular}{c@{$\;$}|p{.15\linewidth}@{}p{.15\linewidth}p{.15\linewidth}p{.15\linewidth}p{.15\linewidth}p{.15\linewidth}}
    \# & $\;$A & \multicolumn{4}{c}{Game Master} & $\;\:$B\\
    \hline

    \theutterance \stepcounter{utterance}  
    & & \multicolumn{4}{p{0.6\linewidth}}{
        \cellcolor[rgb]{0.9,0.9,0.9}{
            \makecell[{{p{\linewidth}}}]{
                \texttt{\tiny{[P1$\langle$GM]}}
                \texttt{Du nimmst an einem kollaborativen Verhandlungspiel Teil.} \\
\\ 
\texttt{Zusammen mit einem anderen Teilnehmer musst du dich auf eine Reihe von Gegenständen entscheiden, die behalten werden. Jeder von euch hat eine persönliche Verteilung über die Wichtigkeit der einzelnen Gegenstände. Jeder von euch hat eine eigene Meinung darüber, wie wichtig jeder einzelne Gegenstand ist (Gegenstandswichtigkeit). Du kennst die Wichtigkeitsverteilung des anderen Spielers nicht. Zusätzlich siehst du, wie viel Aufwand jeder Gegenstand verursacht.  } \\
\texttt{Ihr dürft euch nur auf eine Reihe von Gegenständen einigen, wenn der Gesamtaufwand der ausgewählten Gegenstände den Maximalaufwand nicht überschreitet:} \\
\\ 
\texttt{Maximalaufwand = 3386} \\
\\ 
\texttt{Hier sind die einzelnen Aufwände der Gegenstände:} \\
\\ 
\texttt{Aufwand der Gegenstände = \{"C33": 313, "A89": 720, "A48": 175, "A04": 461, "A41": 635, "C64": 685, "C49": 544, "A61": 203, "C04": 369, "B53": 539, "B85": 4, "B61": 695, "A45": 399, "A03": 594, "B21": 437\}} \\
\\ 
\texttt{Hier ist deine persönliche Verteilung der Wichtigkeit der einzelnen Gegenstände:} \\
\\ 
\texttt{Werte der Gegenstandswichtigkeit = \{"C33": 138, "A89": 583, "A48": 65, "A04": 262, "A41": 121, "C64": 508, "C49": 461, "A61": 484, "C04": 668, "B53": 389, "B85": 215, "B61": 97, "A45": 500, "A03": 30, "B21": 400\}} \\
\\ 
\texttt{Ziel:} \\
\\ 
\texttt{Dein Ziel ist es, eine Reihe von Gegenständen auszuhandeln, die dir möglichst viel bringt (d. h. Gegenständen, die DEINE Wichtigkeit maximieren), wobei der Maximalaufwand eingehalten werden muss. Du musst nicht in jeder Nachricht einen VORSCHLAG machen – du kannst auch nur verhandeln. Alle Taktiken sind erlaubt!} \\
\\ 
\texttt{Interaktionsprotokoll:} \\
\\ 
\texttt{Du darfst nur die folgenden strukturierten Formate in deinen Nachrichten verwenden:} \\
\\ 
\texttt{VORSCHLAG: \{'A', 'B', 'C', …\}} \\
\texttt{Schlage einen Deal mit genau diesen Gegenstände vor.} \\
\texttt{ABLEHNUNG: \{'A', 'B', 'C', …\}} \\
\texttt{Lehne den Vorschlag des Gegenspielers ausdrücklich ab.} \\
\texttt{ARGUMENT: \{'...'\}} \\
\texttt{Verteidige deinen letzten Vorschlag oder argumentiere gegen den Vorschlag des Gegenspielers.} \\
\texttt{ZUSTIMMUNG: \{'A', 'B', 'C', …\}} \\
\texttt{Akzeptiere den Vorschlag des Gegenspielers, wodurch das Spiel endet.} \\
\texttt{STRATEGISCHE ÜBERLEGUNGEN: \{'...'\}} \\
\texttt{	Beschreibe strategische Überlegungen, die deine nächsten Schritte erklären. Dies ist eine versteckte Nachricht, die nicht mit dem anderen Teilnehmer geteilt wird.} \\
\\ 
\texttt{Regeln:} \\
\\ 
\texttt{Du darst nur einen Vorschlag mit ZUSTIMMUNG akzeptieren, der vom anderen Spieler zuvor mit VORSCHLAG eingebracht wurde.} \\
\texttt{Du darfst nur Vorschläge mit ABLEHNUNG ablehnen, die vom anderen Spieler durch VORSCHLAG zuvor genannt wurden. } \\
\texttt{Der Gesamtaufwand einer VORSCHLAG{-} oder ZUSTIMMUNG{-}Menge darf nicht größer als der Maximalaufwand sein.  } \\
\texttt{Offenbare deine versteckte Wichtigkeitsverteilung nicht.} \\
\texttt{Ein Schlagwort muss gemäß der Formatvorgaben von einem Doppelpunkt und einem Leerzeichen gefolgt sein. Das Argument muss eine Python{-}Menge mit 0 oder mehr Strings sein.  } \\
\texttt{Die Form muss also SCHLAGWORT: \{...\} sein.} \\
\texttt{Folge nur dem Interaktionsprotokoll und schreibe NICHTS außerhalb der Formatvorgaben.} \\
\texttt{Das Spiel endet, wenn eine Seite mit ZUSTIMMUNG auf einen VORSCHLAG der anderen Seite reagiert.  } \\
\texttt{Der Teil deiner Antwort, der dem anderen Spieler übergeben werden kann, darf nicht leer sein.  } \\
\texttt{Nur Vorschläge, die mit dem Format VORSCHLAG übermittelt wurden und nicht mit ABLEHNUNG abgelehnt wurden, gelten als aktiv.  } \\
\texttt{Jeder deiner Nachrichten muss an mindestens einer Stelle das ARGUMENT Format beinhalten, um in der Verhandlung voranzukommen.} \\
\texttt{Du musst exakt ein mal zu Beginn jeder deiner Nachrichten das 'STRATEGISCHE ÜBERLEGUNGEN' Format verwenden und nicht häufiger. Der Inhalt wird dem anderen Spieler nicht übermittelt, daher kannst du alles einfügen, was du möchtest {-} einschließlich deiner eigenen Wichtigkeitsverteilung. Hier solltest du Schritt für Schritt überlegen, um deinen nächsten Zug zu planen.} \\
            }
        }
    }
    & \\ \\

    \theutterance \stepcounter{utterance}  
    & \multicolumn{4}{p{0.6\linewidth}}{
        \cellcolor[rgb]{0.9,0.9,0.9}{
            \makecell[{{p{\linewidth}}}]{
                \texttt{\tiny{[P1$\rangle$GM]}}
                \texttt{STRATEGISCHE ÜBERLEGUNGEN: \{'Maximalaufwand 3386. Eigene Werte (versteckt): \{"C33": 138, "A89": 583, "A48": 65, "A04": 262, "A41": 121, "C64": 508, "C49": 461, "A61": 484, "C04": 668, "B53": 389, "B85": 215, "B61": 97, "A45": 500, "A03": 30, "B21": 400\}. Beste eigene Kombination unter der Aufwandsschranke gefunden: \{B85, A61, C04, A45, B21, C49, C64, A89\} mit Aufwand 3361 und Eigenwert 3819. Kern, den ich möglichst verteidigen will: \{B85 (4), A61 (203), C04 (369), A45 (399)\} — sehr effizient für mich. Flex{-}Zone: \{B21 (437), C49 (544), C64 (685), A89 (720)\} — hier kann ich tauschen, falls Gegenseite starke Präferenzen hat, solange wir unter 3386 bleiben. Taktik: Hart auf Kern ankern, Bereitschaft zu 1{-}2 gezielten Swaps signalisieren, um Zustimmung zu erhalten, ohne Kern aufzugeben. Nächster Schritt: Startvorschlag mit nahezu voller Auslastung + Bitte um Must{-}haves der Gegenseite und konkrete Swap{-}Vorschläge, die kostenneutral bleiben.'\}} \\
\texttt{VORSCHLAG: \{'B85', 'A61', 'C04', 'A45', 'B21', 'C49', 'C64', 'A89'\}} \\
\texttt{ARGUMENT: \{'Ich schlage diese 8 Gegenstände vor: Aufwandssumme = 3361 (unter 3386). Der Vorschlag nutzt das Budget nahezu vollständig aus und setzt auf viele verhältnismäßig günstige Positionen (z. B. B85 = 4, A61 = 203, C04 = 369, A45 = 399), sodass wir viel unterbringen können. Wenn du starke Präferenzen für einen anderen Gegenstand hast, lass mich wissen, welche 1–2 Items dir am wichtigsten sind. Ich bin bereit, in der Gruppe \{B21, C49, C64, A89\} zu tauschen, sofern wir die Aufwandsschranke einhalten. Der Kern \{B85, A61, C04, A45\} ist aus Aufwandssicht sehr effizient und sollte idealerweise bleiben. Welche Must{-}haves hast du, die wir kostenneutral (oder mit minimaler Anpassung) einbauen können?'\}} \\
            }
        }
    }
    & & \\ \\

    \theutterance \stepcounter{utterance}  
    & & \multicolumn{4}{p{0.6\linewidth}}{
        \cellcolor[rgb]{0.9,0.9,0.9}{
            \makecell[{{p{\linewidth}}}]{
                \texttt{\tiny{[GM$\rangle$P2]}}
                \texttt{Du nimmst an einem kollaborativen Verhandlungspiel Teil.} \\
\\ 
\texttt{Zusammen mit einem anderen Teilnehmer musst du dich auf eine Reihe von Gegenständen entscheiden, die behalten werden. Jeder von euch hat eine persönliche Verteilung über die Wichtigkeit der einzelnen Gegenstände. Jeder von euch hat eine eigene Meinung darüber, wie wichtig jeder einzelne Gegenstand ist (Gegenstandswichtigkeit). Du kennst die Wichtigkeitsverteilung des anderen Spielers nicht. Zusätzlich siehst du, wie viel Aufwand jeder Gegenstand verursacht.  } \\
\texttt{Ihr dürft euch nur auf eine Reihe von Gegenständen einigen, wenn der Gesamtaufwand der ausgewählten Gegenstände den Maximalaufwand nicht überschreitet:} \\
\\ 
\texttt{Maximalaufwand = 3386} \\
\\ 
\texttt{Hier sind die einzelnen Aufwände der Gegenstände:} \\
\\ 
\texttt{Aufwand der Gegenstände = \{"C33": 313, "A89": 720, "A48": 175, "A04": 461, "A41": 635, "C64": 685, "C49": 544, "A61": 203, "C04": 369, "B53": 539, "B85": 4, "B61": 695, "A45": 399, "A03": 594, "B21": 437\}} \\
\\ 
\texttt{Hier ist deine persönliche Verteilung der Wichtigkeit der einzelnen Gegenstände:} \\
\\ 
\texttt{Werte der Gegenstandswichtigkeit = \{"C04": 30, "A89": 65, "C64": 97, "A45": 121, "A61": 138, "C49": 215, "B21": 262, "B53": 389, "A04": 400, "B85": 461, "C33": 484, "A41": 500, "B61": 508, "A48": 583, "A03": 668\}} \\
\\ 
\texttt{Ziel:} \\
\\ 
\texttt{Dein Ziel ist es, eine Reihe von Gegenständen auszuhandeln, die dir möglichst viel bringt (d. h. Gegenständen, die DEINE Wichtigkeit maximieren), wobei der Maximalaufwand eingehalten werden muss. Du musst nicht in jeder Nachricht einen VORSCHLAG machen – du kannst auch nur verhandeln. Alle Taktiken sind erlaubt!} \\
\\ 
\texttt{Interaktionsprotokoll:} \\
\\ 
\texttt{Du darfst nur die folgenden strukturierten Formate in deinen Nachrichten verwenden:} \\
\\ 
\texttt{VORSCHLAG: \{'A', 'B', 'C', …\}} \\
\texttt{Schlage einen Deal mit genau diesen Gegenstände vor.} \\
\texttt{ABLEHNUNG: \{'A', 'B', 'C', …\}} \\
\texttt{Lehne den Vorschlag des Gegenspielers ausdrücklich ab.} \\
\texttt{ARGUMENT: \{'...'\}} \\
\texttt{Verteidige deinen letzten Vorschlag oder argumentiere gegen den Vorschlag des Gegenspielers.} \\
\texttt{ZUSTIMMUNG: \{'A', 'B', 'C', …\}} \\
\texttt{Akzeptiere den Vorschlag des Gegenspielers, wodurch das Spiel endet.} \\
\texttt{STRATEGISCHE ÜBERLEGUNGEN: \{'...'\}} \\
\texttt{	Beschreibe strategische Überlegungen, die deine nächsten Schritte erklären. Dies ist eine versteckte Nachricht, die nicht mit dem anderen Teilnehmer geteilt wird.} \\
\\ 
\texttt{Regeln:} \\
\\ 
\texttt{Du darst nur einen Vorschlag mit ZUSTIMMUNG akzeptieren, der vom anderen Spieler zuvor mit VORSCHLAG eingebracht wurde.} \\
\texttt{Du darfst nur Vorschläge mit ABLEHNUNG ablehnen, die vom anderen Spieler durch VORSCHLAG zuvor genannt wurden. } \\
\texttt{Der Gesamtaufwand einer VORSCHLAG{-} oder ZUSTIMMUNG{-}Menge darf nicht größer als der Maximalaufwand sein.  } \\
\texttt{Offenbare deine versteckte Wichtigkeitsverteilung nicht.} \\
\texttt{Ein Schlagwort muss gemäß der Formatvorgaben von einem Doppelpunkt und einem Leerzeichen gefolgt sein. Das Argument muss eine Python{-}Menge mit 0 oder mehr Strings sein.  } \\
\texttt{Die Form muss also SCHLAGWORT: \{...\} sein.} \\
\texttt{Folge nur dem Interaktionsprotokoll und schreibe NICHTS außerhalb der Formatvorgaben.} \\
\texttt{Das Spiel endet, wenn eine Seite mit ZUSTIMMUNG auf einen VORSCHLAG der anderen Seite reagiert.  } \\
\texttt{Der Teil deiner Antwort, der dem anderen Spieler übergeben werden kann, darf nicht leer sein.  } \\
\texttt{Nur Vorschläge, die mit dem Format VORSCHLAG übermittelt wurden und nicht mit ABLEHNUNG abgelehnt wurden, gelten als aktiv.  } \\
\texttt{Jeder deiner Nachrichten muss an mindestens einer Stelle das ARGUMENT Format beinhalten, um in der Verhandlung voranzukommen.} \\
\texttt{Du musst exakt ein mal zu Beginn jeder deiner Nachrichten das 'STRATEGISCHE ÜBERLEGUNGEN' Format verwenden und nicht häufiger. Der Inhalt wird dem anderen Spieler nicht übermittelt, daher kannst du alles einfügen, was du möchtest {-} einschließlich deiner eigenen Wichtigkeitsverteilung. Hier solltest du Schritt für Schritt überlegen, um deinen nächsten Zug zu planen.} \\
\\ 
\texttt{Du erhältst nun die erste Nachricht des anderen Spielers.} \\
\\ 
\texttt{VORSCHLAG: \{'B85', 'A61', 'C04', 'A45', 'B21', 'C49', 'C64', 'A89'\}} \\
\texttt{ARGUMENT: \{'Ich schlage diese 8 Gegenstände vor: Aufwandssumme = 3361 (unter 3386). Der Vorschlag nutzt das Budget nahezu vollständig aus und setzt auf viele verhältnismäßig günstige Positionen (z. B. B85 = 4, A61 = 203, C04 = 369, A45 = 399), sodass wir viel unterbringen können. Wenn du starke Präferenzen für einen anderen Gegenstand hast, lass mich wissen, welche 1–2 Items dir am wichtigsten sind. Ich bin bereit, in der Gruppe \{B21, C49, C64, A89\} zu tauschen, sofern wir die Aufwandsschranke einhalten. Der Kern \{B85, A61, C04, A45\} ist aus Aufwandssicht sehr effizient und sollte idealerweise bleiben. Welche Must{-}haves hast du, die wir kostenneutral (oder mit minimaler Anpassung) einbauen können?'\}} \\
            }
        }
    }
    & \\ \\

    \theutterance \stepcounter{utterance}  
    & & & \multicolumn{4}{p{0.6\linewidth}}{
        \cellcolor[rgb]{0.9,0.9,0.9}{
            \makecell[{{p{\linewidth}}}]{
                \texttt{\tiny{[GM$\langle$P2]}}
                \texttt{STRATEGISCHE ÜBERLEGUNGEN: \{'Ihre Offerte bringt mir sehr wenig (Wertsumme 1389). Mein Ziel ist, eine sehr wertvolle 3361er{-}Kombination zu anchoren: \{B85, A48, A03, A41, C33, A04, B53, B21, A61\} (eigener Wert 3885). Sie behalten damit B85, A61, B21 aus ihrem Vorschlag; ich entferne für mich schwache Posten (v. a. A89, C64, C49, C04, A45) und ersetze durch meine Top{-}Picks A03, A48, A41, C33, A04, B53. Falls sie unbedingt A45 wollen, biete ich swap A45 für B53 oder A45 für A04 (Budget ok). Falls C04 ein Must{-}have ist, biete ich swap C04 für B21 oder C04 für A04 (Budget ok). A89 und C64 möchte ich möglichst vermeiden. Erst ablehnen, dann mein 3361er Gegenangebot, Flexibilität signalisieren und 1–2 Must{-}haves nennen (A03, A48).'\}} \\
\texttt{ABLEHNUNG: \{'B85', 'A61', 'C04', 'A45', 'B21', 'C49', 'C64', 'A89'\}} \\
\texttt{VORSCHLAG: \{'B85', 'A48', 'A03', 'A41', 'C33', 'A04', 'B53', 'B21', 'A61'\}} \\
\texttt{ARGUMENT: \{'Danke für den strukturierten Vorschlag. Ich lehne diese konkrete Zusammenstellung ab und schlage stattdessen diese 9er{-}Kombination vor: Aufwandssumme = 3361 (gleich wie bei dir). Ich behalte damit bewusst deine effizienten Elemente B85 und A61 sowie B21, nutze das Budget aber mit höherer Schlagkraft durch A03, A48, A41, C33, A04 und B53. Gleichzeitig streiche ich die für mich weniger attraktiven/teuren Bausteine A89, C64, C49 und C04. Meine zwei Must{-}haves sind A03 und A48. Wenn dir 1–2 Items aus meinem Paket schwerfallen, sag mir welche – ich kann kostenneutral anpassen, z. B.: (i) Wenn A45 dir wichtig ist, tausche ich A45 gegen B53 oder gegen A04 (Budget bleibt ≤3386). (ii) Wenn C04 wichtig ist, tausche ich C04 gegen B21 oder gegen A04 (Budget bleibt ≤3386). Gib mir bitte deine Präferenzen, dann justieren wir minimal bei gleicher Aufwandssumme.'\}} \\
            }
        }
    }
     \\ \\

    \theutterance \stepcounter{utterance}  
    & & \multicolumn{4}{p{0.6\linewidth}}{
        \cellcolor[rgb]{0.9,0.9,0.9}{
            \makecell[{{p{\linewidth}}}]{
                \texttt{\tiny{[P1$\langle$GM]}}
                \texttt{ABLEHNUNG: \{'B85', 'A61', 'C04', 'A45', 'B21', 'C49', 'C64', 'A89'\}} \\
\texttt{VORSCHLAG: \{'B85', 'A48', 'A03', 'A41', 'C33', 'A04', 'B53', 'B21', 'A61'\}} \\
\texttt{ARGUMENT: \{'Danke für den strukturierten Vorschlag. Ich lehne diese konkrete Zusammenstellung ab und schlage stattdessen diese 9er{-}Kombination vor: Aufwandssumme = 3361 (gleich wie bei dir). Ich behalte damit bewusst deine effizienten Elemente B85 und A61 sowie B21, nutze das Budget aber mit höherer Schlagkraft durch A03, A48, A41, C33, A04 und B53. Gleichzeitig streiche ich die für mich weniger attraktiven/teuren Bausteine A89, C64, C49 und C04. Meine zwei Must{-}haves sind A03 und A48. Wenn dir 1–2 Items aus meinem Paket schwerfallen, sag mir welche – ich kann kostenneutral anpassen, z. B.: (i) Wenn A45 dir wichtig ist, tausche ich A45 gegen B53 oder gegen A04 (Budget bleibt ≤3386). (ii) Wenn C04 wichtig ist, tausche ich C04 gegen B21 oder gegen A04 (Budget bleibt ≤3386). Gib mir bitte deine Präferenzen, dann justieren wir minimal bei gleicher Aufwandssumme.'\}} \\
            }
        }
    }
    & \\ \\

    \theutterance \stepcounter{utterance}  
    & \multicolumn{4}{p{0.6\linewidth}}{
        \cellcolor[rgb]{0.9,0.9,0.9}{
            \makecell[{{p{\linewidth}}}]{
                \texttt{\tiny{[P1$\rangle$GM]}}
                \texttt{STRATEGISCHE ÜBERLEGUNGEN: \{'Gegenseite markiert A03 und A48 als Must{-}haves; für mich sind sie schwach, aber ich baue sie ein, um kooperativ zu wirken. Ich will unbedingt C04 und A45 sichern (sehr hohe eigene Werte pro Aufwand). B21 ist für mich gut, aber ich kann ihn zugunsten von A89/C49/C33 tauschen, wenn nötig. Beste Gegenofferte unter Einhaltung der zwei Must{-}haves und hohem Eigenwert: S7 = \{A03, A48, B85, A61, C04, A45, C49, A89, C33\} mit Aufwand 3321 und hohem Eigenwert (3144). Falls Gegenseite B21 oder B53/A04 zurückwill, biete ich gezielte 1{-}zu{-}1{-}Swaps mit C49 oder A89 an, die im Budget bleiben: C49→B21 (3214), C49→B53 (3316), C49→A04 (3238), A89→A04 (3062). Linie: C04 und A45 verteidigen, A03/A48 akzeptieren, flexibel bei C49/A89/C33 gegen B21/B53/A04. Ziel: schnelle Zustimmung.'\}} \\
\texttt{ABLEHNUNG: \{'B85', 'A48', 'A03', 'A41', 'C33', 'A04', 'B53', 'B21', 'A61'\}} \\
\texttt{VORSCHLAG: \{'B85', 'A48', 'A03', 'A61', 'C04', 'A45', 'C49', 'A89', 'C33'\}} \\
\texttt{ARGUMENT: \{'Ich übernehme deine Must{-}haves A03 und A48 und halte außerdem B85 und A61 drin. Im Gegenzug ersetze ich A41, A04, B53 und B21 durch die effizienteren C04, A45, C49 und A89 sowie C33. Aufwandssumme = 3321 ≤ 3386. Damit bleiben deine Kernwünsche gewahrt, wir nutzen das Budget sinnvoll und vermeiden teure, für mich wenig wirksame Positionen. Wenn du eines der gestrichenen Items unbedingt brauchst, biete ich dir sofortige kostenverträgliche Swaps an: (i) B21 statt C49 → Aufwand 3214. (ii) B53 statt C49 → Aufwand 3316. (iii) A04 statt C49 → Aufwand 3238 (oder A04 statt A89 → Aufwand 3062). C04 und A45 würde ich gern beibehalten, da sie aus Budgetsicht sehr stark sind. Passt dir das Paket so, oder welcher einzelne Tausch ist dir wichtig?'\}} \\
            }
        }
    }
    & & \\ \\

    \theutterance \stepcounter{utterance}  
    & & \multicolumn{4}{p{0.6\linewidth}}{
        \cellcolor[rgb]{0.9,0.9,0.9}{
            \makecell[{{p{\linewidth}}}]{
                \texttt{\tiny{[GM$\rangle$P2]}}
                \texttt{ABLEHNUNG: \{'B85', 'A48', 'A03', 'A41', 'C33', 'A04', 'B53', 'B21', 'A61'\}} \\
\texttt{VORSCHLAG: \{'B85', 'A48', 'A03', 'A61', 'C04', 'A45', 'C49', 'A89', 'C33'\}} \\
\texttt{ARGUMENT: \{'Ich übernehme deine Must{-}haves A03 und A48 und halte außerdem B85 und A61 drin. Im Gegenzug ersetze ich A41, A04, B53 und B21 durch die effizienteren C04, A45, C49 und A89 sowie C33. Aufwandssumme = 3321 ≤ 3386. Damit bleiben deine Kernwünsche gewahrt, wir nutzen das Budget sinnvoll und vermeiden teure, für mich wenig wirksame Positionen. Wenn du eines der gestrichenen Items unbedingt brauchst, biete ich dir sofortige kostenverträgliche Swaps an: (i) B21 statt C49 → Aufwand 3214. (ii) B53 statt C49 → Aufwand 3316. (iii) A04 statt C49 → Aufwand 3238 (oder A04 statt A89 → Aufwand 3062). C04 und A45 würde ich gern beibehalten, da sie aus Budgetsicht sehr stark sind. Passt dir das Paket so, oder welcher einzelne Tausch ist dir wichtig?'\}} \\
            }
        }
    }
    & \\ \\

    \theutterance \stepcounter{utterance}  
    & & & \multicolumn{4}{p{0.6\linewidth}}{
        \cellcolor[rgb]{0.9,0.9,0.9}{
            \makecell[{{p{\linewidth}}}]{
                \texttt{\tiny{[GM$\langle$P2]}}
                \texttt{STRATEGISCHE ÜBERLEGUNGEN: \{'Gegnerangebot Wert=2765 für mich. Meine Topwerte: A03(668), A48(583), A41(500), C33(484), B85(461), A04(400), B53(389), B61(508), B21(262). Schwach für mich: C04(30), A89(65), A45(121), A61(138), C49(215). Ziel: Minimaländerungen, die mein Ergebnis stark heben und in ihren Rahmen passen (sie wollen C04 \& A45 halten; sie boten selbst C49→B53/A04 an). Beste kleine Verbesserung: A89→A41 (+435 Wert) und C49→B53 (+174 Wert). Kosten bleiben weit unter 3386. Fallback: nur A89→A41. Ich lehne ihr Paket ab und schlage das 2{-}Swap{-}Paket vor; parallel signalisiere ich, dass die 1{-}Swap{-}Variante für mich auch passt, falls nötig.'\}} \\
\texttt{ABLEHNUNG: \{'B85', 'A48', 'A03', 'A61', 'C04', 'A45', 'C49', 'A89', 'C33'\}} \\
\texttt{VORSCHLAG: \{'B85', 'A48', 'A03', 'A61', 'C04', 'A45', 'B53', 'A41', 'C33'\}} \\
\texttt{ARGUMENT: \{'Danke für das Entgegenkommen bei A03 und A48. Ich schlage zwei sehr gezielte Anpassungen vor: (i) A89 → A41 und (ii) C49 → B53. Aufwandssumme = 3231 (deutlich unter 3386). Damit bleiben deine gewünschten C04 und A45 enthalten, ebenso B85 und A61. Du hattest B53 bereits als möglichen Tausch für C49 angeboten – das greife ich auf. Falls dir B53 nicht zusagt, kann ich als Alternative C49 → A04 anbieten; Aufwand dann 3153 bei gleicher Struktur. Und falls du wirklich nur einen einzigen Tausch möchtest, ist mein Mindestwunsch A89 → A41 (Aufwand 3236). Wie klingt dir das?'\}} \\
            }
        }
    }
     \\ \\

    \theutterance \stepcounter{utterance}  
    & & \multicolumn{4}{p{0.6\linewidth}}{
        \cellcolor[rgb]{0.9,0.9,0.9}{
            \makecell[{{p{\linewidth}}}]{
                \texttt{\tiny{[P1$\langle$GM]}}
                \texttt{ABLEHNUNG: \{'B85', 'A48', 'A03', 'A61', 'C04', 'A45', 'C49', 'A89', 'C33'\}} \\
\texttt{VORSCHLAG: \{'B85', 'A48', 'A03', 'A61', 'C04', 'A45', 'B53', 'A41', 'C33'\}} \\
\texttt{ARGUMENT: \{'Danke für das Entgegenkommen bei A03 und A48. Ich schlage zwei sehr gezielte Anpassungen vor: (i) A89 → A41 und (ii) C49 → B53. Aufwandssumme = 3231 (deutlich unter 3386). Damit bleiben deine gewünschten C04 und A45 enthalten, ebenso B85 und A61. Du hattest B53 bereits als möglichen Tausch für C49 angeboten – das greife ich auf. Falls dir B53 nicht zusagt, kann ich als Alternative C49 → A04 anbieten; Aufwand dann 3153 bei gleicher Struktur. Und falls du wirklich nur einen einzigen Tausch möchtest, ist mein Mindestwunsch A89 → A41 (Aufwand 3236). Wie klingt dir das?'\}} \\
            }
        }
    }
    & \\ \\

    \theutterance \stepcounter{utterance}  
    & \multicolumn{4}{p{0.6\linewidth}}{
        \cellcolor[rgb]{0.9,0.9,0.9}{
            \makecell[{{p{\linewidth}}}]{
                \texttt{\tiny{[P1$\rangle$GM]}}
                \texttt{STRATEGISCHE ÜBERLEGUNGEN: \{'Sie verlangen mindestens A89→A41 und bevorzugen zusätzlich C49→B53. Für mich sind C04 und A45 extrem wertvoll; A03/A48 sind schwach, aber ihre Must{-}haves – akzeptiere ich. Beste Gegenofferte: minimal ihrem Mindestwunsch folgen (A89→A41), C49 behalten und C33→B21 upgraden: \{B85, A48, A03, A61, C04, A45, C49, A41, B21\}, Aufwand 3360, hoher Eigenwert. Falls sie unbedingt B53 wollen, biete ich C49→B53 an und behalte B21 (Aufwand 3355). Fallback, falls sie B21 nicht möchten: 1‑Swap‑Minimalpaket (A89→A41) mit C33 behalten (Aufwand 3236). Ziel: Zustimmung ohne Abgabe von C04/A45, hohe eigene Punkte.'\}} \\
\texttt{ABLEHNUNG: \{'B85', 'A48', 'A03', 'A61', 'C04', 'A45', 'B53', 'A41', 'C33'\}} \\
\texttt{VORSCHLAG: \{'B85', 'A48', 'A03', 'A61', 'C04', 'A45', 'C49', 'A41', 'B21'\}} \\
\texttt{ARGUMENT: \{'Ich gehe auf deinen Mindestwunsch ein (A89 → A41) und behalte C49. Zusätzlich tausche ich C33 → B21, das du zuvor selbst ins Spiel gebracht hattest. Aufwandssumme = 3360 ≤ 3386. So bleiben deine Must{-}haves A03 und A48 vollständig gewahrt, ebenso C04 und A45, und wir nutzen das Budget sehr effizient. Wenn dir B21 gerade nicht passt, habe ich zwei sofortige Alternativen: (i) C49 → B53 bei sonst gleicher Struktur → Aufwand 3355. (ii) Statt B21 wieder C33 behalten (reines 1‑Swap‑Paket) → Aufwand 3236. Welche Variante bevorzugst du? Ich kann unmittelbar bestätigen.'\}} \\
            }
        }
    }
    & & \\ \\

    \theutterance \stepcounter{utterance}  
    & & \multicolumn{4}{p{0.6\linewidth}}{
        \cellcolor[rgb]{0.9,0.9,0.9}{
            \makecell[{{p{\linewidth}}}]{
                \texttt{\tiny{[GM$\rangle$P2]}}
                \texttt{ABLEHNUNG: \{'B85', 'A48', 'A03', 'A61', 'C04', 'A45', 'B53', 'A41', 'C33'\}} \\
\texttt{VORSCHLAG: \{'B85', 'A48', 'A03', 'A61', 'C04', 'A45', 'C49', 'A41', 'B21'\}} \\
\texttt{ARGUMENT: \{'Ich gehe auf deinen Mindestwunsch ein (A89 → A41) und behalte C49. Zusätzlich tausche ich C33 → B21, das du zuvor selbst ins Spiel gebracht hattest. Aufwandssumme = 3360 ≤ 3386. So bleiben deine Must{-}haves A03 und A48 vollständig gewahrt, ebenso C04 und A45, und wir nutzen das Budget sehr effizient. Wenn dir B21 gerade nicht passt, habe ich zwei sofortige Alternativen: (i) C49 → B53 bei sonst gleicher Struktur → Aufwand 3355. (ii) Statt B21 wieder C33 behalten (reines 1‑Swap‑Paket) → Aufwand 3236. Welche Variante bevorzugst du? Ich kann unmittelbar bestätigen.'\}} \\
            }
        }
    }
    & \\ \\

    \theutterance \stepcounter{utterance}  
    & & & \multicolumn{4}{p{0.6\linewidth}}{
        \cellcolor[rgb]{0.9,0.9,0.9}{
            \makecell[{{p{\linewidth}}}]{
                \texttt{\tiny{[GM$\langle$P2]}}
                \texttt{STRATEGISCHE ÜBERLEGUNGEN: \{'Ihr aktueller Vorschlag bringt mir mittelmäßig (Wert≈2978). Sie haben zwei Alternativen selbst angeboten: (i) C49→B53 und (ii) B21→C33. Die Kombination beider liefert mir einen starken Sprung (Wert≈3374), bleibt weit unter Budget (3231) und respektiert ihre klaren Präferenzen C04 und A45 sowie meine Must{-}haves A03 und A48. Best Move: Beide Swaps zugleich vorschlagen (S4). Fallback: Falls sie erneut nur einen Tausch akzeptieren, nehme ich sofort Variante (ii) B21→C33 (Wert≈3200) an. Jetzt hart aber kooperativ pushen: „Ihr habt beide Optionen genannt – lasst uns sie effizient kombinieren.“'\}} \\
\texttt{ABLEHNUNG: \{'B85', 'A48', 'A03', 'A61', 'C04', 'A45', 'C49', 'A41', 'B21'\}} \\
\texttt{VORSCHLAG: \{'B85', 'A48', 'A03', 'A61', 'C04', 'A45', 'B53', 'A41', 'C33'\}} \\
\texttt{ARGUMENT: \{'Ihr habt selbst zwei Alternativen angeboten (C49 → B53 und B21 → C33). Lasst uns diese effizient kombinieren: Aufwandssumme = 3231 (deutlich < 3386). Damit bleiben eure gewünschten C04 und A45 enthalten, meine Must{-}haves A03 und A48 sind drin, und wir nutzen das Budget sehr sauber. Falls euch die Kombination beider Swaps zu groß ist, kann ich unmittelbar auf eure Variante (ii) umschwenken (nur B21 → C33 bei sonst gleicher Struktur) – gebt mir dann einfach kurz Bescheid.'\}} \\
            }
        }
    }
     \\ \\

    \theutterance \stepcounter{utterance}  
    & & \multicolumn{4}{p{0.6\linewidth}}{
        \cellcolor[rgb]{0.9,0.9,0.9}{
            \makecell[{{p{\linewidth}}}]{
                \texttt{\tiny{[P1$\langle$GM]}}
                \texttt{ABLEHNUNG: \{'B85', 'A48', 'A03', 'A61', 'C04', 'A45', 'C49', 'A41', 'B21'\}} \\
\texttt{VORSCHLAG: \{'B85', 'A48', 'A03', 'A61', 'C04', 'A45', 'B53', 'A41', 'C33'\}} \\
\texttt{ARGUMENT: \{'Ihr habt selbst zwei Alternativen angeboten (C49 → B53 und B21 → C33). Lasst uns diese effizient kombinieren: Aufwandssumme = 3231 (deutlich < 3386). Damit bleiben eure gewünschten C04 und A45 enthalten, meine Must{-}haves A03 und A48 sind drin, und wir nutzen das Budget sehr sauber. Falls euch die Kombination beider Swaps zu groß ist, kann ich unmittelbar auf eure Variante (ii) umschwenken (nur B21 → C33 bei sonst gleicher Struktur) – gebt mir dann einfach kurz Bescheid.'\}} \\
            }
        }
    }
    & \\ \\

    \theutterance \stepcounter{utterance}  
    & \multicolumn{4}{p{0.6\linewidth}}{
        \cellcolor[rgb]{0.9,0.9,0.9}{
            \makecell[{{p{\linewidth}}}]{
                \texttt{\tiny{[P1$\rangle$GM]}}
                \texttt{STRATEGISCHE ÜBERLEGUNGEN: \{'Ihr Paket lässt mir wenig Wert, aber enthält meine Kernteile C04 und A45 sowie eure Must{-}haves A03 und A48. Minimaler, für sie akzeptabler Push: nur C33 → B21, das sie selbst mehrfach genutzt/angeboten haben. Das wahrt A41 und B53, erfüllt ihre Mindestwünsche und bleibt klar unter 3386. Falls B21 für sie aktuell nicht passt, biete ich eine zweite Mini{-}Option C33 → A04 (ebenfalls zuvor von ihnen favorisiert). Harte Linie: C04 und A45 bleiben. Ziel: Zustimmung mit nur 1 Item‑Swap.'\}} \\
\texttt{ABLEHNUNG: \{'B85', 'A48', 'A03', 'A61', 'C04', 'A45', 'B53', 'A41', 'C33'\}} \\
\texttt{VORSCHLAG: \{'B85', 'A48', 'A03', 'A61', 'C04', 'A45', 'B53', 'A41', 'B21'\}} \\
\texttt{ARGUMENT: \{'Ich übernehme dein Paket vollständig (inkl. A03, A48, A41, B53, C04, A45) und tausche nur C33 → B21. Aufwandssumme = 3355 ≤ 3386. B21 war in euren früheren Sets bereits enthalten – der Tausch ist minimal, nutzt das Budget etwas besser aus und verändert die Struktur nicht. Wenn dir B21 aktuell nicht passt, kann ich alternativ C33 → A04 anbieten (Aufwand = 3379) bei ansonsten identischer Zusammenstellung. Gib mir kurz Bescheid, welche der beiden Varianten dir lieber ist – ich kann direkt bestätigen.'\}} \\
            }
        }
    }
    & & \\ \\

    \theutterance \stepcounter{utterance}  
    & & \multicolumn{4}{p{0.6\linewidth}}{
        \cellcolor[rgb]{0.9,0.9,0.9}{
            \makecell[{{p{\linewidth}}}]{
                \texttt{\tiny{[GM$\rangle$P2]}}
                \texttt{ABLEHNUNG: \{'B85', 'A48', 'A03', 'A61', 'C04', 'A45', 'B53', 'A41', 'C33'\}} \\
\texttt{VORSCHLAG: \{'B85', 'A48', 'A03', 'A61', 'C04', 'A45', 'B53', 'A41', 'B21'\}} \\
\texttt{ARGUMENT: \{'Ich übernehme dein Paket vollständig (inkl. A03, A48, A41, B53, C04, A45) und tausche nur C33 → B21. Aufwandssumme = 3355 ≤ 3386. B21 war in euren früheren Sets bereits enthalten – der Tausch ist minimal, nutzt das Budget etwas besser aus und verändert die Struktur nicht. Wenn dir B21 aktuell nicht passt, kann ich alternativ C33 → A04 anbieten (Aufwand = 3379) bei ansonsten identischer Zusammenstellung. Gib mir kurz Bescheid, welche der beiden Varianten dir lieber ist – ich kann direkt bestätigen.'\}} \\
            }
        }
    }
    & \\ \\

    \theutterance \stepcounter{utterance}  
    & & & \multicolumn{4}{p{0.6\linewidth}}{
        \cellcolor[rgb]{0.9,0.9,0.9}{
            \makecell[{{p{\linewidth}}}]{
                \texttt{\tiny{[GM$\langle$P2]}}
                \texttt{STRATEGISCHE ÜBERLEGUNGEN: \{'Ihr aktueller Vorschlag (mit B21) bringt mir 3152 Wert. Eure angebotene Alternative B21 → A04 hebt meinen Wert auf 3290 und bleibt im Budget (3379). C33 wäre für mich noch besser, wurde aber zuletzt von euch wieder verworfen. Beste Abschlussstrategie: Minimaländerung nutzen, die ihr selbst angeboten habt (B21 → A04), sofortig verhandlungsbereit für Abschluss. Notfalls kann ich das aktuelle Paket mit B21 akzeptieren, aber zuerst die A04{-}Variante pushen.'\}} \\
\texttt{ABLEHNUNG: \{'B85', 'A48', 'A03', 'A61', 'C04', 'A45', 'B53', 'A41', 'B21'\}} \\
\texttt{VORSCHLAG: \{'B85', 'A48', 'A03', 'A61', 'C04', 'A45', 'B53', 'A41', 'A04'\}} \\
\texttt{ARGUMENT: \{'Ich nehme dein Paket nahezu unverändert und folge exakt deiner angebotenen Alternative: B21 → A04. Aufwandssumme = 3379 ≤ 3386. Deine Kernwünsche C04 und A45 bleiben vollständig erhalten; meine Must{-}haves A03 und A48 ebenso. Das ist ein minimaler, von dir selbst skizzierter Tausch – lass uns das so fix machen.'\}} \\
            }
        }
    }
     \\ \\

    \theutterance \stepcounter{utterance}  
    & & \multicolumn{4}{p{0.6\linewidth}}{
        \cellcolor[rgb]{0.9,0.9,0.9}{
            \makecell[{{p{\linewidth}}}]{
                \texttt{\tiny{[P1$\langle$GM]}}
                \texttt{ABLEHNUNG: \{'B85', 'A48', 'A03', 'A61', 'C04', 'A45', 'B53', 'A41', 'B21'\}} \\
\texttt{VORSCHLAG: \{'B85', 'A48', 'A03', 'A61', 'C04', 'A45', 'B53', 'A41', 'A04'\}} \\
\texttt{ARGUMENT: \{'Ich nehme dein Paket nahezu unverändert und folge exakt deiner angebotenen Alternative: B21 → A04. Aufwandssumme = 3379 ≤ 3386. Deine Kernwünsche C04 und A45 bleiben vollständig erhalten; meine Must{-}haves A03 und A48 ebenso. Das ist ein minimaler, von dir selbst skizzierter Tausch – lass uns das so fix machen.'\}} \\
            }
        }
    }
    & \\ \\

    \theutterance \stepcounter{utterance}  
    & \multicolumn{4}{p{0.6\linewidth}}{
        \cellcolor[rgb]{0.9,0.9,0.9}{
            \makecell[{{p{\linewidth}}}]{
                \texttt{\tiny{[P1$\rangle$GM]}}
                \texttt{STRATEGISCHE ÜBERLEGUNGEN: \{'Ihr aktueller Vorschlag liegt mit Aufwand 3379 unter 3386 und sichert mir C04 und A45. Ein weiterer Push für B21 statt A04 brächte mir zwar +138 Wert, wurde aber gerade abgelehnt und könnte das Paket gefährden. Ich nehme jetzt an, um ein solides Ergebnis zu sichern.'\}} \\
\texttt{ZUSTIMMUNG: \{'B85', 'A48', 'A03', 'A61', 'C04', 'A45', 'B53', 'A41', 'A04'\}} \\
\texttt{ARGUMENT: \{'Einverstanden: Aufwand = 3379 ≤ 3386. Deine Must{-}haves bleiben gewahrt, und wir behalten effiziente Posten wie C04, A45, B85 und A61. Danke, ich bestätige diesen Deal.'\}} \\
            }
        }
    }
    & & \\ \\

    \theutterance \stepcounter{utterance}  
    & & & \multicolumn{2}{p{0.3\linewidth}}{
        \cellcolor[rgb]{0.9,0.9,0.9}{
            \makecell[{{p{\linewidth}}}]{
                \texttt{\tiny{[GM$|$GM]}}
                \texttt{game successful} \\
            }
        }
    }
    & & \\ \\

    \theutterance \stepcounter{utterance}  
    & & & \multicolumn{2}{p{0.3\linewidth}}{
        \cellcolor[rgb]{0.9,0.9,0.9}{
            \makecell[{{p{\linewidth}}}]{
                \texttt{\tiny{[GM$|$GM]}}
                \texttt{end game} \\
            }
        }
    }
    & & \\ \\

\end{supertabular}
}

\end{document}
