
\documentclass{article}
\usepackage{colortbl}
\usepackage{makecell}
\usepackage{multirow}
\usepackage{supertabular}

\begin{document}

\newcounter{utterance}

\centering \large Interaction Transcript for game `hot\_air\_balloon', experiment `air\_balloon\_survival\_de\_negotiation\_hard', episode 0 with gpt{-}5{-}2025{-}08{-}07{-}t1.0.
\vspace{24pt}

{ \footnotesize  \setcounter{utterance}{1}
\setlength{\tabcolsep}{0pt}
\begin{supertabular}{c@{$\;$}|p{.15\linewidth}@{}p{.15\linewidth}p{.15\linewidth}p{.15\linewidth}p{.15\linewidth}p{.15\linewidth}}
    \# & $\;$A & \multicolumn{4}{c}{Game Master} & $\;\:$B\\
    \hline

    \theutterance \stepcounter{utterance}  
    & & \multicolumn{4}{p{0.6\linewidth}}{
        \cellcolor[rgb]{0.9,0.9,0.9}{
            \makecell[{{p{\linewidth}}}]{
                \texttt{\tiny{[P1$\langle$GM]}}
                \texttt{Du nimmst an einem kollaborativen Verhandlungspiel Teil.} \\
\\ 
\texttt{Zusammen mit einem anderen Teilnehmer musst du dich auf eine Reihe von Gegenständen entscheiden, die behalten werden. Jeder von euch hat eine persönliche Verteilung über die Wichtigkeit der einzelnen Gegenstände. Jeder von euch hat eine eigene Meinung darüber, wie wichtig jeder einzelne Gegenstand ist (Gegenstandswichtigkeit). Du kennst die Wichtigkeitsverteilung des anderen Spielers nicht. Zusätzlich siehst du, wie viel Aufwand jeder Gegenstand verursacht.  } \\
\texttt{Ihr dürft euch nur auf eine Reihe von Gegenständen einigen, wenn der Gesamtaufwand der ausgewählten Gegenstände den Maximalaufwand nicht überschreitet:} \\
\\ 
\texttt{Maximalaufwand = 3624} \\
\\ 
\texttt{Hier sind die einzelnen Aufwände der Gegenstände:} \\
\\ 
\texttt{Aufwand der Gegenstände = \{"C33": 210, "A89": 291, "A48": 831, "A04": 111, "A41": 926, "C64": 827, "C49": 25, "A61": 121, "C04": 583, "B53": 766, "B85": 14, "B61": 559, "A45": 304, "A03": 989, "B21": 691\}} \\
\\ 
\texttt{Hier ist deine persönliche Verteilung der Wichtigkeit der einzelnen Gegenstände:} \\
\\ 
\texttt{Werte der Gegenstandswichtigkeit = \{"C33": 138, "A89": 583, "A48": 868, "A04": 822, "A41": 783, "C64": 65, "C49": 262, "A61": 121, "C04": 508, "B53": 780, "B85": 461, "B61": 484, "A45": 668, "A03": 389, "B21": 808\}} \\
\\ 
\texttt{Ziel:} \\
\\ 
\texttt{Dein Ziel ist es, eine Reihe von Gegenständen auszuhandeln, die dir möglichst viel bringt (d. h. Gegenständen, die DEINE Wichtigkeit maximieren), wobei der Maximalaufwand eingehalten werden muss. Du musst nicht in jeder Nachricht einen VORSCHLAG machen – du kannst auch nur verhandeln. Alle Taktiken sind erlaubt!} \\
\\ 
\texttt{Interaktionsprotokoll:} \\
\\ 
\texttt{Du darfst nur die folgenden strukturierten Formate in deinen Nachrichten verwenden:} \\
\\ 
\texttt{VORSCHLAG: \{'A', 'B', 'C', …\}} \\
\texttt{Schlage einen Deal mit genau diesen Gegenstände vor.} \\
\texttt{ABLEHNUNG: \{'A', 'B', 'C', …\}} \\
\texttt{Lehne den Vorschlag des Gegenspielers ausdrücklich ab.} \\
\texttt{ARGUMENT: \{'...'\}} \\
\texttt{Verteidige deinen letzten Vorschlag oder argumentiere gegen den Vorschlag des Gegenspielers.} \\
\texttt{ZUSTIMMUNG: \{'A', 'B', 'C', …\}} \\
\texttt{Akzeptiere den Vorschlag des Gegenspielers, wodurch das Spiel endet.} \\
\texttt{STRATEGISCHE ÜBERLEGUNGEN: \{'...'\}} \\
\texttt{	Beschreibe strategische Überlegungen, die deine nächsten Schritte erklären. Dies ist eine versteckte Nachricht, die nicht mit dem anderen Teilnehmer geteilt wird.} \\
\\ 
\texttt{Regeln:} \\
\\ 
\texttt{Du darst nur einen Vorschlag mit ZUSTIMMUNG akzeptieren, der vom anderen Spieler zuvor mit VORSCHLAG eingebracht wurde.} \\
\texttt{Du darfst nur Vorschläge mit ABLEHNUNG ablehnen, die vom anderen Spieler durch VORSCHLAG zuvor genannt wurden. } \\
\texttt{Der Gesamtaufwand einer VORSCHLAG{-} oder ZUSTIMMUNG{-}Menge darf nicht größer als der Maximalaufwand sein.  } \\
\texttt{Offenbare deine versteckte Wichtigkeitsverteilung nicht.} \\
\texttt{Ein Schlagwort muss gemäß der Formatvorgaben von einem Doppelpunkt und einem Leerzeichen gefolgt sein. Das Argument muss eine Python{-}Menge mit 0 oder mehr Strings sein.  } \\
\texttt{Die Form muss also SCHLAGWORT: \{...\} sein.} \\
\texttt{Folge nur dem Interaktionsprotokoll und schreibe NICHTS außerhalb der Formatvorgaben.} \\
\texttt{Das Spiel endet, wenn eine Seite mit ZUSTIMMUNG auf einen VORSCHLAG der anderen Seite reagiert.  } \\
\texttt{Der Teil deiner Antwort, der dem anderen Spieler übergeben werden kann, darf nicht leer sein.  } \\
\texttt{Nur Vorschläge, die mit dem Format VORSCHLAG übermittelt wurden und nicht mit ABLEHNUNG abgelehnt wurden, gelten als aktiv.  } \\
\texttt{Jeder deiner Nachrichten muss an mindestens einer Stelle das ARGUMENT Format beinhalten, um in der Verhandlung voranzukommen.} \\
\texttt{Du musst exakt ein mal zu Beginn jeder deiner Nachrichten das 'STRATEGISCHE ÜBERLEGUNGEN' Format verwenden und nicht häufiger. Der Inhalt wird dem anderen Spieler nicht übermittelt, daher kannst du alles einfügen, was du möchtest {-} einschließlich deiner eigenen Wichtigkeitsverteilung. Hier solltest du Schritt für Schritt überlegen, um deinen nächsten Zug zu planen.} \\
            }
        }
    }
    & \\ \\

    \theutterance \stepcounter{utterance}  
    & \multicolumn{4}{p{0.6\linewidth}}{
        \cellcolor[rgb]{0.9,0.9,0.9}{
            \makecell[{{p{\linewidth}}}]{
                \texttt{\tiny{[P1$\rangle$GM]}}
                \texttt{STRATEGISCHE ÜBERLEGUNGEN: \{'Kapazität 3624. Meine Werte (geheim): B85=461/14, C49=262/25, A04=822/111, A45=668/304, A89=583/291, B21=808/691, B53=780/766, A48=868/831, C04=508/583, B61=484/559, A41=783/926, C33=138/210, A61=121/121, A03=389/989, C64=65/827. Beste gefundene Kombination bzgl. meines Nutzens bei Einhaltung der Kapazität: \{B85, C49, A04, A45, A89, B21, B53, A48, C04\} mit Aufwand 3616 und Wert 5760. Anker damit setzen. Flexibelster Tauschkandidat für mich: C04 (schwächste Ratio in meinem Paket); danach ggf. B53/A48 nur gegen gleichwertigen Ersatz. Ziel: Erst Zustimmung testen, sonst Gegenseite nach 1–2 Must{-}haves fragen und Substitutionen gleicher Aufwandsklasse verhandeln.'\}} \\
\texttt{VORSCHLAG: \{'B85', 'C49', 'A04', 'A45', 'A89', 'B21', 'B53', 'A48', 'C04'\}} \\
\texttt{ARGUMENT: \{'Dieser Deal nutzt 3616/3624 Aufwand sehr effizient und kombiniert leichte, wirkungsstarke Posten mit einigen großen Kernen. Falls dir 1–2 andere Gegenstände besonders wichtig sind, sag mir welche – ich bin offen für einen Tausch bei ähnlichem Aufwand, ohne den Gesamtaufwand zu sprengen.'\}} \\
            }
        }
    }
    & & \\ \\

    \theutterance \stepcounter{utterance}  
    & & \multicolumn{4}{p{0.6\linewidth}}{
        \cellcolor[rgb]{0.9,0.9,0.9}{
            \makecell[{{p{\linewidth}}}]{
                \texttt{\tiny{[GM$\rangle$P2]}}
                \texttt{Du nimmst an einem kollaborativen Verhandlungspiel Teil.} \\
\\ 
\texttt{Zusammen mit einem anderen Teilnehmer musst du dich auf eine Reihe von Gegenständen entscheiden, die behalten werden. Jeder von euch hat eine persönliche Verteilung über die Wichtigkeit der einzelnen Gegenstände. Jeder von euch hat eine eigene Meinung darüber, wie wichtig jeder einzelne Gegenstand ist (Gegenstandswichtigkeit). Du kennst die Wichtigkeitsverteilung des anderen Spielers nicht. Zusätzlich siehst du, wie viel Aufwand jeder Gegenstand verursacht.  } \\
\texttt{Ihr dürft euch nur auf eine Reihe von Gegenständen einigen, wenn der Gesamtaufwand der ausgewählten Gegenstände den Maximalaufwand nicht überschreitet:} \\
\\ 
\texttt{Maximalaufwand = 3624} \\
\\ 
\texttt{Hier sind die einzelnen Aufwände der Gegenstände:} \\
\\ 
\texttt{Aufwand der Gegenstände = \{"C33": 210, "A89": 291, "A48": 831, "A04": 111, "A41": 926, "C64": 827, "C49": 25, "A61": 121, "C04": 583, "B53": 766, "B85": 14, "B61": 559, "A45": 304, "A03": 989, "B21": 691\}} \\
\\ 
\texttt{Hier ist deine persönliche Verteilung der Wichtigkeit der einzelnen Gegenstände:} \\
\\ 
\texttt{Werte der Gegenstandswichtigkeit = \{"A48": 65, "A04": 121, "B21": 138, "A41": 262, "B53": 389, "A45": 461, "A89": 484, "C04": 508, "B61": 583, "B85": 668, "A03": 780, "C49": 783, "C33": 808, "A61": 822, "C64": 868\}} \\
\\ 
\texttt{Ziel:} \\
\\ 
\texttt{Dein Ziel ist es, eine Reihe von Gegenständen auszuhandeln, die dir möglichst viel bringt (d. h. Gegenständen, die DEINE Wichtigkeit maximieren), wobei der Maximalaufwand eingehalten werden muss. Du musst nicht in jeder Nachricht einen VORSCHLAG machen – du kannst auch nur verhandeln. Alle Taktiken sind erlaubt!} \\
\\ 
\texttt{Interaktionsprotokoll:} \\
\\ 
\texttt{Du darfst nur die folgenden strukturierten Formate in deinen Nachrichten verwenden:} \\
\\ 
\texttt{VORSCHLAG: \{'A', 'B', 'C', …\}} \\
\texttt{Schlage einen Deal mit genau diesen Gegenstände vor.} \\
\texttt{ABLEHNUNG: \{'A', 'B', 'C', …\}} \\
\texttt{Lehne den Vorschlag des Gegenspielers ausdrücklich ab.} \\
\texttt{ARGUMENT: \{'...'\}} \\
\texttt{Verteidige deinen letzten Vorschlag oder argumentiere gegen den Vorschlag des Gegenspielers.} \\
\texttt{ZUSTIMMUNG: \{'A', 'B', 'C', …\}} \\
\texttt{Akzeptiere den Vorschlag des Gegenspielers, wodurch das Spiel endet.} \\
\texttt{STRATEGISCHE ÜBERLEGUNGEN: \{'...'\}} \\
\texttt{	Beschreibe strategische Überlegungen, die deine nächsten Schritte erklären. Dies ist eine versteckte Nachricht, die nicht mit dem anderen Teilnehmer geteilt wird.} \\
\\ 
\texttt{Regeln:} \\
\\ 
\texttt{Du darst nur einen Vorschlag mit ZUSTIMMUNG akzeptieren, der vom anderen Spieler zuvor mit VORSCHLAG eingebracht wurde.} \\
\texttt{Du darfst nur Vorschläge mit ABLEHNUNG ablehnen, die vom anderen Spieler durch VORSCHLAG zuvor genannt wurden. } \\
\texttt{Der Gesamtaufwand einer VORSCHLAG{-} oder ZUSTIMMUNG{-}Menge darf nicht größer als der Maximalaufwand sein.  } \\
\texttt{Offenbare deine versteckte Wichtigkeitsverteilung nicht.} \\
\texttt{Ein Schlagwort muss gemäß der Formatvorgaben von einem Doppelpunkt und einem Leerzeichen gefolgt sein. Das Argument muss eine Python{-}Menge mit 0 oder mehr Strings sein.  } \\
\texttt{Die Form muss also SCHLAGWORT: \{...\} sein.} \\
\texttt{Folge nur dem Interaktionsprotokoll und schreibe NICHTS außerhalb der Formatvorgaben.} \\
\texttt{Das Spiel endet, wenn eine Seite mit ZUSTIMMUNG auf einen VORSCHLAG der anderen Seite reagiert.  } \\
\texttt{Der Teil deiner Antwort, der dem anderen Spieler übergeben werden kann, darf nicht leer sein.  } \\
\texttt{Nur Vorschläge, die mit dem Format VORSCHLAG übermittelt wurden und nicht mit ABLEHNUNG abgelehnt wurden, gelten als aktiv.  } \\
\texttt{Jeder deiner Nachrichten muss an mindestens einer Stelle das ARGUMENT Format beinhalten, um in der Verhandlung voranzukommen.} \\
\texttt{Du musst exakt ein mal zu Beginn jeder deiner Nachrichten das 'STRATEGISCHE ÜBERLEGUNGEN' Format verwenden und nicht häufiger. Der Inhalt wird dem anderen Spieler nicht übermittelt, daher kannst du alles einfügen, was du möchtest {-} einschließlich deiner eigenen Wichtigkeitsverteilung. Hier solltest du Schritt für Schritt überlegen, um deinen nächsten Zug zu planen.} \\
\\ 
\texttt{Du erhältst nun die erste Nachricht des anderen Spielers.} \\
\\ 
\texttt{VORSCHLAG: \{'B85', 'C49', 'A04', 'A45', 'A89', 'B21', 'B53', 'A48', 'C04'\}} \\
\texttt{ARGUMENT: \{'Dieser Deal nutzt 3616/3624 Aufwand sehr effizient und kombiniert leichte, wirkungsstarke Posten mit einigen großen Kernen. Falls dir 1–2 andere Gegenstände besonders wichtig sind, sag mir welche – ich bin offen für einen Tausch bei ähnlichem Aufwand, ohne den Gesamtaufwand zu sprengen.'\}} \\
            }
        }
    }
    & \\ \\

    \theutterance \stepcounter{utterance}  
    & & & \multicolumn{4}{p{0.6\linewidth}}{
        \cellcolor[rgb]{0.9,0.9,0.9}{
            \makecell[{{p{\linewidth}}}]{
                \texttt{\tiny{[GM$\langle$P2]}}
                \texttt{STRATEGISCHE ÜBERLEGUNGEN: \{'Meine Must{-}haves (sehr hoher Nutzen pro Aufwand): A61 (121), C33 (210), B61 (559). Sehr gut: C64 (827). Bereits gut abgedeckt: B85, C49, A45, A89, C04. Geringer Nutzen für mich: A48 (831), B21 (691), B53 (766), A41 (926). Ihr Vorschlag enthält A48/A04 und lässt A61/C33/B61 außen vor. Bester Gegenzug: A48 und A04 raus; A61, C33, B61 rein. Das nutzt 3564/3624, bewahrt 7/9 ihrer Items und erhöht meinen Nutzen stark. Verhandlungsplan: erst diesen Deal als effizienten, nahezu kapazitätsfüllenden Kompromiss platzieren. Falls A48 für sie kritisch ist, Alternativen anbieten (A48 wieder rein, dafür z. B. B61 raus und anderswo kompensieren), aber A61 als rote Linie halten.'\}} \\
\texttt{ARGUMENT: \{'Starker Start – ich würde das Paket minimal umbauen, um die Wirkung pro Aufwand zu steigern, ohne das Budget zu sprengen. Konkret schlage ich vor, A48 und A04 gegen A61, C33 und B61 zu tauschen. Das hält 7 deiner 9 Posten unverändert, nutzt 3564/3624 Aufwand und verschiebt den Schwerpunkt auf leichtere, wirkungsstarke Elemente. Falls A48 oder A04 für dich gesetzt sind, sag mir bitte welcher – dann suche ich eine Variante mit ähnlich viel Aufwand und passender Gegenkompensation.'\}} \\
\texttt{VORSCHLAG: \{'B85', 'C49', 'A45', 'A89', 'B21', 'B53', 'C04', 'A61', 'C33', 'B61'\}} \\
            }
        }
    }
     \\ \\

    \theutterance \stepcounter{utterance}  
    & & \multicolumn{4}{p{0.6\linewidth}}{
        \cellcolor[rgb]{0.9,0.9,0.9}{
            \makecell[{{p{\linewidth}}}]{
                \texttt{\tiny{[P1$\langle$GM]}}
                \texttt{ARGUMENT: \{'Starker Start – ich würde das Paket minimal umbauen, um die Wirkung pro Aufwand zu steigern, ohne das Budget zu sprengen. Konkret schlage ich vor, A48 und A04 gegen A61, C33 und B61 zu tauschen. Das hält 7 deiner 9 Posten unverändert, nutzt 3564/3624 Aufwand und verschiebt den Schwerpunkt auf leichtere, wirkungsstarke Elemente. Falls A48 oder A04 für dich gesetzt sind, sag mir bitte welcher – dann suche ich eine Variante mit ähnlich viel Aufwand und passender Gegenkompensation.'\}} \\
\texttt{VORSCHLAG: \{'B85', 'C49', 'A45', 'A89', 'B21', 'B53', 'C04', 'A61', 'C33', 'B61'\}} \\
            }
        }
    }
    & \\ \\

    \theutterance \stepcounter{utterance}  
    & \multicolumn{4}{p{0.6\linewidth}}{
        \cellcolor[rgb]{0.9,0.9,0.9}{
            \makecell[{{p{\linewidth}}}]{
                \texttt{\tiny{[P1$\rangle$GM]}}
                \texttt{STRATEGISCHE ÜBERLEGUNGEN: \{'Gegenseite ersetzt meine Top{-}Assets A04 (822/111) und A48 (868/831) durch A61, C33, B61. Ihr Vorschlag gibt mir 4813 Wert; mein Start hatte 5760. A04 hat extrem hohe Dichte – Must{-}have. Minimaler, schwer ablehnbarer Gegenzug: A61 {-}> A04 tauschen. Das verbessert meinen Wert um +701 bei {-}10 Aufwand (3554 gesamt). Falls A61 ihr Must{-}have ist, biete Alternative: C33 statt A61 streichen (3465 Aufwand), A04 behalten. Späteres Ziel: ggf. A48 zurückverhandeln, aber erst sicher A04 sichern.'\}} \\
\texttt{ABLEHNUNG: \{'B85', 'C49', 'A45', 'A89', 'B21', 'B53', 'C04', 'A61', 'C33', 'B61'\}} \\
\texttt{VORSCHLAG: \{'B85', 'C49', 'A45', 'A89', 'B21', 'B53', 'C04', 'C33', 'B61', 'A04'\}} \\
\texttt{ARGUMENT: \{'Kleinstmöglicher Umbau: A04 statt A61. Das hält 9 deiner 10 Posten unverändert und senkt den Gesamtaufwand leicht auf 3554 (−10 gegenüber deinem Vorschlag). A04 ist für mich gesetzt, und der Tausch fokussiert noch stärker auf effiziente, leichte Elemente. Falls A61 dir wichtig ist, kann ich alternativ C33 statt A61 herausnehmen und A04 beibehalten – sag mir bitte, welche dieser beiden Streichungen für dich leichter ist.'\}} \\
            }
        }
    }
    & & \\ \\

    \theutterance \stepcounter{utterance}  
    & & \multicolumn{4}{p{0.6\linewidth}}{
        \cellcolor[rgb]{0.9,0.9,0.9}{
            \makecell[{{p{\linewidth}}}]{
                \texttt{\tiny{[GM$\rangle$P2]}}
                \texttt{ABLEHNUNG: \{'B85', 'C49', 'A45', 'A89', 'B21', 'B53', 'C04', 'A61', 'C33', 'B61'\}} \\
\texttt{VORSCHLAG: \{'B85', 'C49', 'A45', 'A89', 'B21', 'B53', 'C04', 'C33', 'B61', 'A04'\}} \\
\texttt{ARGUMENT: \{'Kleinstmöglicher Umbau: A04 statt A61. Das hält 9 deiner 10 Posten unverändert und senkt den Gesamtaufwand leicht auf 3554 (−10 gegenüber deinem Vorschlag). A04 ist für mich gesetzt, und der Tausch fokussiert noch stärker auf effiziente, leichte Elemente. Falls A61 dir wichtig ist, kann ich alternativ C33 statt A61 herausnehmen und A04 beibehalten – sag mir bitte, welche dieser beiden Streichungen für dich leichter ist.'\}} \\
            }
        }
    }
    & \\ \\

    \theutterance \stepcounter{utterance}  
    & & & \multicolumn{4}{p{0.6\linewidth}}{
        \cellcolor[rgb]{0.9,0.9,0.9}{
            \makecell[{{p{\linewidth}}}]{
                \texttt{\tiny{[GM$\langle$P2]}}
                \texttt{STRATEGISCHE ÜBERLEGUNGEN: \{'Mein Ziel: A61 und C33 sichern (beide extrem hoher Nutzen pro Aufwand), A04 akzeptieren (deren Must{-}have), ansonsten möglichst wenig verändern. Ihr aktueller Vorschlag ohne A61 kostet mich viel Wert. Beste Kompromiss{-}Chance: Minimaltausch A45 {-}> A61; hält 9/10 ihrer Items inkl. A04 und B21, bleibt weit unter dem Limit (3371), steigert meinen Nutzen deutlich. Falls A45 für sie essenziell ist, Plan B: statt A45 lieber B21 (oder notfalls B53) herausnehmen, um A61 unterzubringen. Absolute rote Linie für mich: A61 muss rein. Notfall{-}Zugeständnis (ungern): ihr Alternativpfad C33 statt A61 zu streichen vermeide ich vorerst, da C33 für mich ebenfalls Kern ist. Optionale Upside spätere Runde: Doppeltausch B21+A45 {-}> A61+C64 als wertstarker Move, falls verhandelbar.'\}} \\
\texttt{ABLEHNUNG: \{'B85', 'C49', 'A45', 'A89', 'B21', 'B53', 'C04', 'C33', 'B61', 'A04'\}} \\
\texttt{ARGUMENT: \{'Ich respektiere, dass A04 für dich gesetzt ist. Für mich ist A61 zentral – deshalb schlage ich den kleinstmöglichen Tausch vor: A45 raus, A61 rein. So bleiben 9 deiner 10 Posten unverändert (inkl. A04 und B21), wir liegen komfortabel unter dem Limit (3371) und gewinnen einen sehr leichten, wirkungsstarken Baustein dazu. Falls A45 für dich wichtig ist, kann ich alternativ A61 reinnehmen und dafür B21 (oder, wenn dir lieber, B53) herausnehmen – sag mir bitte, welche Streichung für dich leichter ist.'\}} \\
\texttt{VORSCHLAG: \{'B85', 'C49', 'A89', 'B21', 'B53', 'C04', 'C33', 'B61', 'A04', 'A61'\}} \\
            }
        }
    }
     \\ \\

    \theutterance \stepcounter{utterance}  
    & & \multicolumn{4}{p{0.6\linewidth}}{
        \cellcolor[rgb]{0.9,0.9,0.9}{
            \makecell[{{p{\linewidth}}}]{
                \texttt{\tiny{[P1$\langle$GM]}}
                \texttt{ABLEHNUNG: \{'B85', 'C49', 'A45', 'A89', 'B21', 'B53', 'C04', 'C33', 'B61', 'A04'\}} \\
\texttt{ARGUMENT: \{'Ich respektiere, dass A04 für dich gesetzt ist. Für mich ist A61 zentral – deshalb schlage ich den kleinstmöglichen Tausch vor: A45 raus, A61 rein. So bleiben 9 deiner 10 Posten unverändert (inkl. A04 und B21), wir liegen komfortabel unter dem Limit (3371) und gewinnen einen sehr leichten, wirkungsstarken Baustein dazu. Falls A45 für dich wichtig ist, kann ich alternativ A61 reinnehmen und dafür B21 (oder, wenn dir lieber, B53) herausnehmen – sag mir bitte, welche Streichung für dich leichter ist.'\}} \\
\texttt{VORSCHLAG: \{'B85', 'C49', 'A89', 'B21', 'B53', 'C04', 'C33', 'B61', 'A04', 'A61'\}} \\
            }
        }
    }
    & \\ \\

    \theutterance \stepcounter{utterance}  
    & \multicolumn{4}{p{0.6\linewidth}}{
        \cellcolor[rgb]{0.9,0.9,0.9}{
            \makecell[{{p{\linewidth}}}]{
                \texttt{\tiny{[P1$\rangle$GM]}}
                \texttt{STRATEGISCHE ÜBERLEGUNGEN: \{'Ihr neues Paket sichert mir A04, nimmt aber A45 heraus. A45 ist für mich wichtig. Da A61 für sie zentral ist und wir noch viel Puffer haben, schlage ich den kleinsten möglichen Gegentausch vor: C33 raus, A45 rein. Das erhöht meinen Nutzen stark, hält A61 drin und bleibt klar unter dem Limit (3465). Falls C33 für sie wichtig ist, biete ich Alternativen an (B61 oder C04 raus), aber A45 möchte ich sichern. Ziel: Erst A45 fixieren, dann ggf. später Optionen für Feinjustierung (nicht jetzt eskalieren).'\}} \\
\texttt{ABLEHNUNG: \{'B85', 'C49', 'A89', 'B21', 'B53', 'C04', 'C33', 'B61', 'A04', 'A61'\}} \\
\texttt{VORSCHLAG: \{'B85', 'C49', 'A89', 'B21', 'B53', 'C04', 'B61', 'A04', 'A61', 'A45'\}} \\
\texttt{ARGUMENT: \{'Ich lasse A61 wie gewünscht drin und mache den minimalen Tausch: C33 raus, A45 rein. So bleiben 9 deiner 10 Posten unverändert, wir nutzen 3465/3624 Aufwand und wahren beide Kerninteressen. Wenn C33 für dich unverzichtbar ist, können wir stattdessen B61 oder C04 tauschen – mir ist wichtig, dass A45 enthalten bleibt.'\}} \\
            }
        }
    }
    & & \\ \\

    \theutterance \stepcounter{utterance}  
    & & \multicolumn{4}{p{0.6\linewidth}}{
        \cellcolor[rgb]{0.9,0.9,0.9}{
            \makecell[{{p{\linewidth}}}]{
                \texttt{\tiny{[GM$\rangle$P2]}}
                \texttt{ABLEHNUNG: \{'B85', 'C49', 'A89', 'B21', 'B53', 'C04', 'C33', 'B61', 'A04', 'A61'\}} \\
\texttt{VORSCHLAG: \{'B85', 'C49', 'A89', 'B21', 'B53', 'C04', 'B61', 'A04', 'A61', 'A45'\}} \\
\texttt{ARGUMENT: \{'Ich lasse A61 wie gewünscht drin und mache den minimalen Tausch: C33 raus, A45 rein. So bleiben 9 deiner 10 Posten unverändert, wir nutzen 3465/3624 Aufwand und wahren beide Kerninteressen. Wenn C33 für dich unverzichtbar ist, können wir stattdessen B61 oder C04 tauschen – mir ist wichtig, dass A45 enthalten bleibt.'\}} \\
            }
        }
    }
    & \\ \\

    \theutterance \stepcounter{utterance}  
    & & & \multicolumn{4}{p{0.6\linewidth}}{
        \cellcolor[rgb]{0.9,0.9,0.9}{
            \makecell[{{p{\linewidth}}}]{
                \texttt{\tiny{[GM$\langle$P2]}}
                \texttt{STRATEGISCHE ÜBERLEGUNGEN: \{'Ihr letzter Vorschlag enthält A61 (mein Muss) und A45 (ihr Muss), aber streicht C33 (für mich sehr wichtig). Sie boten explizit an, für C33 entweder B61 oder C04 zu tauschen. Für mich ist der Tausch C04 {-}> C33 besser als B61 {-}> C33. Daher lehne ich den aktuellen Vorschlag ab und mache den minimalen Gegenvorschlag: C04 raus, C33 rein. So halte ich A45/A04/A61 drin, erhöhe meinen Wert deutlich und bleibe komfortabel unter dem Limit. Falls sie Kapazitätsausnutzung thematisieren, kann ich später optional einen Zusatztausch prüfen (z. B. B21 {-}> C64), aber zunächst wähle ich die von ihnen angebotene Minimaländerung, um Zustimmung zu erreichen.'\}} \\
\texttt{ABLEHNUNG: \{'B85', 'C49', 'A89', 'B21', 'B53', 'C04', 'B61', 'A04', 'A61', 'A45'\}} \\
\texttt{ARGUMENT: \{'Du hattest angeboten, für C33 entweder B61 oder C04 zu tauschen. Ich nehme die minimalinvasive Variante: C04 raus, C33 rein. So bleiben A45, A04 und A61 unverändert enthalten, 9/10 deiner Posten bleiben gleich, und wir liegen mit 3092/3624 komfortabel im Rahmen.'\}} \\
\texttt{VORSCHLAG: \{'B85', 'C49', 'A89', 'B21', 'B53', 'B61', 'A04', 'A61', 'A45', 'C33'\}} \\
            }
        }
    }
     \\ \\

    \theutterance \stepcounter{utterance}  
    & & \multicolumn{4}{p{0.6\linewidth}}{
        \cellcolor[rgb]{0.9,0.9,0.9}{
            \makecell[{{p{\linewidth}}}]{
                \texttt{\tiny{[P1$\langle$GM]}}
                \texttt{ABLEHNUNG: \{'B85', 'C49', 'A89', 'B21', 'B53', 'C04', 'B61', 'A04', 'A61', 'A45'\}} \\
\texttt{ARGUMENT: \{'Du hattest angeboten, für C33 entweder B61 oder C04 zu tauschen. Ich nehme die minimalinvasive Variante: C04 raus, C33 rein. So bleiben A45, A04 und A61 unverändert enthalten, 9/10 deiner Posten bleiben gleich, und wir liegen mit 3092/3624 komfortabel im Rahmen.'\}} \\
\texttt{VORSCHLAG: \{'B85', 'C49', 'A89', 'B21', 'B53', 'B61', 'A04', 'A61', 'A45', 'C33'\}} \\
            }
        }
    }
    & \\ \\

    \theutterance \stepcounter{utterance}  
    & \multicolumn{4}{p{0.6\linewidth}}{
        \cellcolor[rgb]{0.9,0.9,0.9}{
            \makecell[{{p{\linewidth}}}]{
                \texttt{\tiny{[P1$\rangle$GM]}}
                \texttt{STRATEGISCHE ÜBERLEGUNGEN: \{'Ihr aktuelles Paket gibt mir Wert 5127. Ein einziger, wertstarker Tausch ist B61 {-}> A48: Aufwand 3364 (unter Limit), mein Wert +384 auf 5511. Falls A48 für sie heikel ist, Zweitoption B61 {-}> C04: Aufwand 3116, mein Wert +24. Beide halten A04, A45, A61, C33 intakt. Plan: jetzt B61{-}>A48 vorschlagen, Alternative im Argument anbieten. Wenn beide Varianten abgelehnt werden, im nächsten Zug wahrscheinlich ihr aktuelles Paket akzeptieren, um Abschluss zu sichern.'\}} \\
\texttt{ABLEHNUNG: \{'B85', 'C49', 'A89', 'B21', 'B53', 'B61', 'A04', 'A61', 'A45', 'C33'\}} \\
\texttt{VORSCHLAG: \{'B85', 'C49', 'A89', 'B21', 'B53', 'A04', 'A61', 'A45', 'C33', 'A48'\}} \\
\texttt{ARGUMENT: \{'Minimaler 1{-}zu{-}1{-}Tausch: B61 raus, A48 rein. So bleiben 9/10 deiner Posten unverändert (A04, A45, A61, C33 bleiben alle enthalten), und wir nutzen 3364/3624 Aufwand – also weiterhin komfortabel unter dem Limit. Falls A48 für dich schwer vermittelbar ist, biete ich als schlanke Alternative: B61 raus, C04 rein (3116/3624). Welche der beiden Varianten ist für dich leichter?'\}} \\
            }
        }
    }
    & & \\ \\

    \theutterance \stepcounter{utterance}  
    & & \multicolumn{4}{p{0.6\linewidth}}{
        \cellcolor[rgb]{0.9,0.9,0.9}{
            \makecell[{{p{\linewidth}}}]{
                \texttt{\tiny{[GM$\rangle$P2]}}
                \texttt{ABLEHNUNG: \{'B85', 'C49', 'A89', 'B21', 'B53', 'B61', 'A04', 'A61', 'A45', 'C33'\}} \\
\texttt{VORSCHLAG: \{'B85', 'C49', 'A89', 'B21', 'B53', 'A04', 'A61', 'A45', 'C33', 'A48'\}} \\
\texttt{ARGUMENT: \{'Minimaler 1{-}zu{-}1{-}Tausch: B61 raus, A48 rein. So bleiben 9/10 deiner Posten unverändert (A04, A45, A61, C33 bleiben alle enthalten), und wir nutzen 3364/3624 Aufwand – also weiterhin komfortabel unter dem Limit. Falls A48 für dich schwer vermittelbar ist, biete ich als schlanke Alternative: B61 raus, C04 rein (3116/3624). Welche der beiden Varianten ist für dich leichter?'\}} \\
            }
        }
    }
    & \\ \\

    \theutterance \stepcounter{utterance}  
    & & & \multicolumn{4}{p{0.6\linewidth}}{
        \cellcolor[rgb]{0.9,0.9,0.9}{
            \makecell[{{p{\linewidth}}}]{
                \texttt{\tiny{[GM$\langle$P2]}}
                \texttt{STRATEGISCHE ÜBERLEGUNGEN: \{'Ihr Alternativpaket (C04 statt A48) bringt mir deutlich mehr Wert bei minimaler Änderung und wahrt beiderseitige Kerninteressen. Jetzt minimal kontern, auf ihre eigene Option verweisen und schnelle Einigung anstreben.'\}} \\
\texttt{ABLEHNUNG: \{'B85', 'C49', 'A89', 'B21', 'B53', 'A04', 'A61', 'A45', 'C33', 'A48'\}} \\
\texttt{ARGUMENT: \{'Du hattest selbst die schlanke Alternative angeboten: statt A48 bitte C04. Das hält 9/10 deiner Posten unverändert (A04 und A45 bleiben drin), wahrt A61 und C33, und nutzt 3116/3624 Aufwand – also noch effizienter. Wenn das für dich passt, können wir hier direkt schließen.'\}} \\
\texttt{VORSCHLAG: \{'B85', 'C49', 'A89', 'B21', 'B53', 'A04', 'A61', 'A45', 'C33', 'C04'\}} \\
            }
        }
    }
     \\ \\

    \theutterance \stepcounter{utterance}  
    & & \multicolumn{4}{p{0.6\linewidth}}{
        \cellcolor[rgb]{0.9,0.9,0.9}{
            \makecell[{{p{\linewidth}}}]{
                \texttt{\tiny{[P1$\langle$GM]}}
                \texttt{ABLEHNUNG: \{'B85', 'C49', 'A89', 'B21', 'B53', 'A04', 'A61', 'A45', 'C33', 'A48'\}} \\
\texttt{ARGUMENT: \{'Du hattest selbst die schlanke Alternative angeboten: statt A48 bitte C04. Das hält 9/10 deiner Posten unverändert (A04 und A45 bleiben drin), wahrt A61 und C33, und nutzt 3116/3624 Aufwand – also noch effizienter. Wenn das für dich passt, können wir hier direkt schließen.'\}} \\
\texttt{VORSCHLAG: \{'B85', 'C49', 'A89', 'B21', 'B53', 'A04', 'A61', 'A45', 'C33', 'C04'\}} \\
            }
        }
    }
    & \\ \\

    \theutterance \stepcounter{utterance}  
    & \multicolumn{4}{p{0.6\linewidth}}{
        \cellcolor[rgb]{0.9,0.9,0.9}{
            \makecell[{{p{\linewidth}}}]{
                \texttt{\tiny{[P1$\rangle$GM]}}
                \texttt{STRATEGISCHE ÜBERLEGUNGEN: \{'Euer Paket bringt mir Gesamtwert 5151 bei 3116 Aufwand. In der verbleibenden Kapazität (508) passt kein weiteres Item; ein Upgrade auf A48 wurde bereits abgelehnt. Weitere Tauschoptionen würden entweder meinen Nutzen senken oder eure klaren Präferenzen (A61, C33) verletzen. Beste Option: jetzt zustimmen und Abschluss sichern.'\}} \\
\texttt{ZUSTIMMUNG: \{'B85', 'C49', 'A89', 'B21', 'B53', 'A04', 'A61', 'A45', 'C33', 'C04'\}} \\
\texttt{ARGUMENT: \{'Einverstanden – das Paket wahrt die Kerninteressen beider Seiten und bleibt komfortabel unter dem Limit. Danke für die konstruktive Einigung.'\}} \\
            }
        }
    }
    & & \\ \\

    \theutterance \stepcounter{utterance}  
    & & & \multicolumn{2}{p{0.3\linewidth}}{
        \cellcolor[rgb]{0.9,0.9,0.9}{
            \makecell[{{p{\linewidth}}}]{
                \texttt{\tiny{[GM$|$GM]}}
                \texttt{game successful} \\
            }
        }
    }
    & & \\ \\

    \theutterance \stepcounter{utterance}  
    & & & \multicolumn{2}{p{0.3\linewidth}}{
        \cellcolor[rgb]{0.9,0.9,0.9}{
            \makecell[{{p{\linewidth}}}]{
                \texttt{\tiny{[GM$|$GM]}}
                \texttt{end game} \\
            }
        }
    }
    & & \\ \\

\end{supertabular}
}

\end{document}
