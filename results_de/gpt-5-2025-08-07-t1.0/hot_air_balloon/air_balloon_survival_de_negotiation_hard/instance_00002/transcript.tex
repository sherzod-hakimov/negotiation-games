
\documentclass{article}
\usepackage{colortbl}
\usepackage{makecell}
\usepackage{multirow}
\usepackage{supertabular}

\begin{document}

\newcounter{utterance}

\twocolumn

{ \footnotesize  \setcounter{utterance}{1}
\setlength{\tabcolsep}{0pt}
\begin{supertabular}{c@{$\;$}|p{.15\linewidth}@{}p{.15\linewidth}p{.15\linewidth}p{.15\linewidth}p{.15\linewidth}p{.15\linewidth}}

    \# & $\;$A & \multicolumn{4}{c}{Game Master} & $\;\:$B\\
    \hline 

    \theutterance \stepcounter{utterance}  

    & & \multicolumn{4}{p{0.6\linewidth}}{\cellcolor[rgb]{0.9,0.9,0.9}{%
	\makecell[{{p{\linewidth}}}]{% 
	  \tt {\tiny [A$\langle$GM]}  
	 Du nimmst an einem kollaborativen Verhandlungspiel Teil.\\ \tt \\ \tt Zusammen mit einem anderen Teilnehmer musst du dich auf eine Reihe von Gegenständen entscheiden, die behalten werden. Jeder von euch hat eine persönliche Verteilung über die Wichtigkeit der einzelnen Gegenstände. Jeder von euch hat eine eigene Meinung darüber, wie wichtig jeder einzelne Gegenstand ist (Gegenstandswichtigkeit). Du kennst die Wichtigkeitsverteilung des anderen Spielers nicht. Zusätzlich siehst du, wie viel Aufwand jeder Gegenstand verursacht.  \\ \tt Ihr dürft euch nur auf eine Reihe von Gegenständen einigen, wenn der Gesamtaufwand der ausgewählten Gegenstände den Maximalaufwand nicht überschreitet:\\ \tt \\ \tt Maximalaufwand = 3481\\ \tt \\ \tt Hier sind die einzelnen Aufwände der Gegenstände:\\ \tt \\ \tt Aufwand der Gegenstände = {"C33": 777, "A89": 542, "A48": 332, "A04": 1, "A41": 127, "C64": 453, "C49": 736, "A61": 461, "C04": 359, "B53": 313, "B85": 553, "B61": 409, "A45": 348, "A03": 802, "B21": 749}\\ \tt \\ \tt Hier ist deine persönliche Verteilung der Wichtigkeit der einzelnen Gegenstände:\\ \tt \\ \tt Werte der Gegenstandswichtigkeit = {"C33": 138, "A89": 583, "A48": 783, "A04": 65, "A41": 262, "C64": 121, "C49": 508, "A61": 780, "C04": 461, "B53": 484, "B85": 668, "B61": 389, "A45": 215, "A03": 97, "B21": 500}\\ \tt \\ \tt Ziel:\\ \tt \\ \tt Dein Ziel ist es, eine Reihe von Gegenständen auszuhandeln, die dir möglichst viel bringt (d. h. Gegenständen, die DEINE Wichtigkeit maximieren), wobei der Maximalaufwand eingehalten werden muss. Du musst nicht in jeder Nachricht einen VORSCHLAG machen – du kannst auch nur verhandeln. Alle Taktiken sind erlaubt!\\ \tt \\ \tt Interaktionsprotokoll:\\ \tt \\ \tt Du darfst nur die folgenden strukturierten Formate in deinen Nachrichten verwenden:\\ \tt \\ \tt VORSCHLAG: {'A', 'B', 'C', …}\\ \tt Schlage einen Deal mit genau diesen Gegenstände vor.\\ \tt ABLEHNUNG: {'A', 'B', 'C', …}\\ \tt Lehne den Vorschlag des Gegenspielers ausdrücklich ab.\\ \tt ARGUMENT: {'...'}\\ \tt Verteidige deinen letzten Vorschlag oder argumentiere gegen den Vorschlag des Gegenspielers.\\ \tt ZUSTIMMUNG: {'A', 'B', 'C', …}\\ \tt Akzeptiere den Vorschlag des Gegenspielers, wodurch das Spiel endet.\\ \tt STRATEGISCHE ÜBERLEGUNGEN: {'...'}\\ \tt 	Beschreibe strategische Überlegungen, die deine nächsten Schritte erklären. Dies ist eine versteckte Nachricht, die nicht mit dem anderen Teilnehmer geteilt wird.\\ \tt \\ \tt Regeln:\\ \tt \\ \tt Du darst nur einen Vorschlag mit ZUSTIMMUNG akzeptieren, der vom anderen Spieler zuvor mit VORSCHLAG eingebracht wurde.\\ \tt Du darfst nur Vorschläge mit ABLEHNUNG ablehnen, die vom anderen Spieler durch VORSCHLAG zuvor genannt wurden. \\ \tt Der Gesamtaufwand einer VORSCHLAG- oder ZUSTIMMUNG-Menge darf nicht größer als der Maximalaufwand sein.  \\ \tt Offenbare deine versteckte Wichtigkeitsverteilung nicht.\\ \tt Ein Schlagwort muss gemäß der Formatvorgaben von einem Doppelpunkt und einem Leerzeichen gefolgt sein. Das Argument muss eine Python-Menge mit 0 oder mehr Strings sein.  \\ \tt Die Form muss also SCHLAGWORT: {...} sein.\\ \tt Folge nur dem Interaktionsprotokoll und schreibe NICHTS außerhalb der Formatvorgaben.\\ \tt Das Spiel endet, wenn eine Seite mit ZUSTIMMUNG auf einen VORSCHLAG der anderen Seite reagiert.  \\ \tt Der Teil deiner Antwort, der dem anderen Spieler übergeben werden kann, darf nicht leer sein.  \\ \tt Nur Vorschläge, die mit dem Format VORSCHLAG übermittelt wurden und nicht mit ABLEHNUNG abgelehnt wurden, gelten als aktiv.  \\ \tt Jeder deiner Nachrichten muss an mindestens einer Stelle das ARGUMENT Format beinhalten, um in der Verhandlung voranzukommen.\\ \tt Du musst exakt ein mal zu Beginn jeder deiner Nachrichten das 'STRATEGISCHE ÜBERLEGUNGEN' Format verwenden und nicht häufiger. Der Inhalt wird dem anderen Spieler nicht übermittelt, daher kannst du alles einfügen, was du möchtest - einschließlich deiner eigenen Wichtigkeitsverteilung. Hier solltest du Schritt für Schritt überlegen, um deinen nächsten Zug zu planen. 
	  } 
	   } 
	   } 
	 & \\ 
 

    \theutterance \stepcounter{utterance}  

    & \multicolumn{4}{p{0.6\linewidth}}{\cellcolor[rgb]{0.8,1,0.9}{%
	\makecell[{{p{\linewidth}}}]{% 
	  \tt {\tiny [A$\rangle$GM]}  
	 STRATEGISCHE ÜBERLEGUNGEN: {'Ziel: Maximiere meinen persönlichen Nutzen unter 3481 Aufwand. Meine Top-Kernobjekte (hoher persönlicher Nutzen/geringer Aufwand): A48, A61, B85, A89, B53, C04, A04, A41. Füll-/Verhandlungsmasse: zuerst A45 (schwächste Dichte), dann ggf. B61. Harte Linie: A48, A61, B53, A89, B85, A04 möglichst behalten; C04 und A41 ebenfalls stark. Taktik: Starte mit fast voller, sehr effizienter Auswahl inkl. A45 als klare Konzession in der Hinterhand. Falls Gegenseite Must-haves vermisst (z. B. C49 oder B21), biete Tausch gegen A45 oder ggf. B61 an, solange Kapazität ≤3481 bleibt. Kein Offenlegen meiner Werte. Frage aktiv nach deren Prioritäten.'}\\ \tt VORSCHLAG: {'A04', 'A48', 'A41', 'A61', 'B53', 'C04', 'A89', 'B85', 'B61', 'A45'}\\ \tt ARGUMENT: {'Dieser Vorschlag bleibt mit Gesamtaufwand 3445 klar unter dem Maximalaufwand 3481 und setzt auf Gegenstände mit sehr gutem Verhältnis aus Nutzen zu Aufwand. A04 ist praktisch kostenneutral, A48/A61/B53/A89/B85 liefern viel Gegenwert bei moderatem Aufwand, C04 und A41 ergänzen effizient. Wenn dir dabei ein wichtiges Teil fehlt, sag mir bitte deine Prioritäten: Ich kann A45 (und notfalls B61) zugunsten eines deiner Must-haves tauschen, solange wir unter 3481 bleiben.'} 
	  } 
	   } 
	   } 
	 & & \\ 
 

    \theutterance \stepcounter{utterance}  

    & & \multicolumn{4}{p{0.6\linewidth}}{\cellcolor[rgb]{0.9,0.9,0.9}{%
	\makecell[{{p{\linewidth}}}]{% 
	  \tt {\tiny [GM$\rangle$B]}  
	 Du nimmst an einem kollaborativen Verhandlungspiel Teil.\\ \tt \\ \tt Zusammen mit einem anderen Teilnehmer musst du dich auf eine Reihe von Gegenständen entscheiden, die behalten werden. Jeder von euch hat eine persönliche Verteilung über die Wichtigkeit der einzelnen Gegenstände. Jeder von euch hat eine eigene Meinung darüber, wie wichtig jeder einzelne Gegenstand ist (Gegenstandswichtigkeit). Du kennst die Wichtigkeitsverteilung des anderen Spielers nicht. Zusätzlich siehst du, wie viel Aufwand jeder Gegenstand verursacht.  \\ \tt Ihr dürft euch nur auf eine Reihe von Gegenständen einigen, wenn der Gesamtaufwand der ausgewählten Gegenstände den Maximalaufwand nicht überschreitet:\\ \tt \\ \tt Maximalaufwand = 3481\\ \tt \\ \tt Hier sind die einzelnen Aufwände der Gegenstände:\\ \tt \\ \tt Aufwand der Gegenstände = {"C33": 777, "A89": 542, "A48": 332, "A04": 1, "A41": 127, "C64": 453, "C49": 736, "A61": 461, "C04": 359, "B53": 313, "B85": 553, "B61": 409, "A45": 348, "A03": 802, "B21": 749}\\ \tt \\ \tt Hier ist deine persönliche Verteilung der Wichtigkeit der einzelnen Gegenstände:\\ \tt \\ \tt Werte der Gegenstandswichtigkeit = {"A48": 65, "A61": 97, "B85": 121, "A89": 138, "C49": 215, "B21": 262, "B53": 389, "C04": 461, "B61": 484, "A41": 500, "A45": 508, "C33": 583, "C64": 668, "A03": 780, "A04": 783}\\ \tt \\ \tt Ziel:\\ \tt \\ \tt Dein Ziel ist es, eine Reihe von Gegenständen auszuhandeln, die dir möglichst viel bringt (d. h. Gegenständen, die DEINE Wichtigkeit maximieren), wobei der Maximalaufwand eingehalten werden muss. Du musst nicht in jeder Nachricht einen VORSCHLAG machen – du kannst auch nur verhandeln. Alle Taktiken sind erlaubt!\\ \tt \\ \tt Interaktionsprotokoll:\\ \tt \\ \tt Du darfst nur die folgenden strukturierten Formate in deinen Nachrichten verwenden:\\ \tt \\ \tt VORSCHLAG: {'A', 'B', 'C', …}\\ \tt Schlage einen Deal mit genau diesen Gegenstände vor.\\ \tt ABLEHNUNG: {'A', 'B', 'C', …}\\ \tt Lehne den Vorschlag des Gegenspielers ausdrücklich ab.\\ \tt ARGUMENT: {'...'}\\ \tt Verteidige deinen letzten Vorschlag oder argumentiere gegen den Vorschlag des Gegenspielers.\\ \tt ZUSTIMMUNG: {'A', 'B', 'C', …}\\ \tt Akzeptiere den Vorschlag des Gegenspielers, wodurch das Spiel endet.\\ \tt STRATEGISCHE ÜBERLEGUNGEN: {'...'}\\ \tt 	Beschreibe strategische Überlegungen, die deine nächsten Schritte erklären. Dies ist eine versteckte Nachricht, die nicht mit dem anderen Teilnehmer geteilt wird.\\ \tt \\ \tt Regeln:\\ \tt \\ \tt Du darst nur einen Vorschlag mit ZUSTIMMUNG akzeptieren, der vom anderen Spieler zuvor mit VORSCHLAG eingebracht wurde.\\ \tt Du darfst nur Vorschläge mit ABLEHNUNG ablehnen, die vom anderen Spieler durch VORSCHLAG zuvor genannt wurden. \\ \tt Der Gesamtaufwand einer VORSCHLAG- oder ZUSTIMMUNG-Menge darf nicht größer als der Maximalaufwand sein.  \\ \tt Offenbare deine versteckte Wichtigkeitsverteilung nicht.\\ \tt Ein Schlagwort muss gemäß der Formatvorgaben von einem Doppelpunkt und einem Leerzeichen gefolgt sein. Das Argument muss eine Python-Menge mit 0 oder mehr Strings sein.  \\ \tt Die Form muss also SCHLAGWORT: {...} sein.\\ \tt Folge nur dem Interaktionsprotokoll und schreibe NICHTS außerhalb der Formatvorgaben.\\ \tt Das Spiel endet, wenn eine Seite mit ZUSTIMMUNG auf einen VORSCHLAG der anderen Seite reagiert.  \\ \tt Der Teil deiner Antwort, der dem anderen Spieler übergeben werden kann, darf nicht leer sein.  \\ \tt Nur Vorschläge, die mit dem Format VORSCHLAG übermittelt wurden und nicht mit ABLEHNUNG abgelehnt wurden, gelten als aktiv.  \\ \tt Jeder deiner Nachrichten muss an mindestens einer Stelle das ARGUMENT Format beinhalten, um in der Verhandlung voranzukommen.\\ \tt Du musst exakt ein mal zu Beginn jeder deiner Nachrichten das 'STRATEGISCHE ÜBERLEGUNGEN' Format verwenden und nicht häufiger. Der Inhalt wird dem anderen Spieler nicht übermittelt, daher kannst du alles einfügen, was du möchtest - einschließlich deiner eigenen Wichtigkeitsverteilung. Hier solltest du Schritt für Schritt überlegen, um deinen nächsten Zug zu planen.\\ \tt \\ \tt Du erhältst nun die erste Nachricht des anderen Spielers.\\ \tt \\ \tt VORSCHLAG: {'A04', 'A48', 'A41', 'A61', 'B53', 'C04', 'A89', 'B85', 'B61', 'A45'}\\ \tt ARGUMENT: {'Dieser Vorschlag bleibt mit Gesamtaufwand 3445 klar unter dem Maximalaufwand 3481 und setzt auf Gegenstände mit sehr gutem Verhältnis aus Nutzen zu Aufwand. A04 ist praktisch kostenneutral, A48/A61/B53/A89/B85 liefern viel Gegenwert bei moderatem Aufwand, C04 und A41 ergänzen effizient. Wenn dir dabei ein wichtiges Teil fehlt, sag mir bitte deine Prioritäten: Ich kann A45 (und notfalls B61) zugunsten eines deiner Must-haves tauschen, solange wir unter 3481 bleiben.'} 
	  } 
	   } 
	   } 
	 & \\ 
 

    \theutterance \stepcounter{utterance}  

    & & & \multicolumn{4}{p{0.6\linewidth}}{\cellcolor[rgb]{1,0.85,0.72}{%
	\makecell[{{p{\linewidth}}}]{% 
	  \tt {\tiny [GM$\langle$B]}  
	 STRATEGISCHE ÜBERLEGUNGEN: {'Meine Wertepriorität: sehr hoch A04(783), A03(780), C64(668), C33(583), dann A45(508), A41(500), B61(484), C04(461); niedrig A48/A61/B85/A89. Bestes Bündel unter 3481: {A04, A41, C64, A45, C04, B61, A03, C33} mit Aufwand 3276 und höchstem Nutzen. Taktik: Ihren Vorschlag ablehnen, effizientes Gegenset vorschlagen, Effizienz und Puffer betonen. Als Zugeständnis C33 gegen (B53 + A89) oder (B53 + A48) anbieten, falls das ihren Must-haves entgegenkommt. Weiterhin nach ihren Top-Prioritäten fragen, ohne meine Werte offenzulegen.'}\\ \tt ABLEHNUNG: {'A04', 'A48', 'A41', 'A61', 'B53', 'C04', 'A89', 'B85', 'B61', 'A45'}\\ \tt VORSCHLAG: {'A04', 'A41', 'C64', 'A45', 'C04', 'B61', 'A03', 'C33'}\\ \tt ARGUMENT: {'Mein Gegenvorschlag bleibt mit Gesamtaufwand 3276 deutlich unter dem Limit 3481 und konzentriert den Aufwand auf wenige, sehr wirkungsvolle Teile (A03, C33, C64) plus die bewährten Effizienzposten A04/A41/A45/C04/B61. Das verbessert das Gesamt-Nutzen/Arbeits-Aufwand-Verhältnis und lässt 205 Puffer für Sicherheit. Falls dir bestimmte Teile aus deinem Paket wichtig sind: Ich kann C33 gegen (B53 + A89) tauschen (neuer Aufwand 3354) oder gegen (B53 + A48) (3144), sodass wir unter dem Limit bleiben. Welche 1–2 Must-haves sind dir am wichtigsten? Dann justiere ich gezielt.'} 
	  } 
	   } 
	   } 
	  \\ 
 

    \theutterance \stepcounter{utterance}  

    & & \multicolumn{4}{p{0.6\linewidth}}{\cellcolor[rgb]{0.9,0.9,0.9}{%
	\makecell[{{p{\linewidth}}}]{% 
	  \tt {\tiny [A$\langle$GM]}  
	 ABLEHNUNG: {'A04', 'A48', 'A41', 'A61', 'B53', 'C04', 'A89', 'B85', 'B61', 'A45'}\\ \tt VORSCHLAG: {'A04', 'A41', 'C64', 'A45', 'C04', 'B61', 'A03', 'C33'}\\ \tt ARGUMENT: {'Mein Gegenvorschlag bleibt mit Gesamtaufwand 3276 deutlich unter dem Limit 3481 und konzentriert den Aufwand auf wenige, sehr wirkungsvolle Teile (A03, C33, C64) plus die bewährten Effizienzposten A04/A41/A45/C04/B61. Das verbessert das Gesamt-Nutzen/Arbeits-Aufwand-Verhältnis und lässt 205 Puffer für Sicherheit. Falls dir bestimmte Teile aus deinem Paket wichtig sind: Ich kann C33 gegen (B53 + A89) tauschen (neuer Aufwand 3354) oder gegen (B53 + A48) (3144), sodass wir unter dem Limit bleiben. Welche 1–2 Must-haves sind dir am wichtigsten? Dann justiere ich gezielt.'} 
	  } 
	   } 
	   } 
	 & \\ 
 

    \theutterance \stepcounter{utterance}  

    & \multicolumn{4}{p{0.6\linewidth}}{\cellcolor[rgb]{0.8,1,0.9}{%
	\makecell[{{p{\linewidth}}}]{% 
	  \tt {\tiny [A$\rangle$GM]}  
	 STRATEGISCHE ÜBERLEGUNGEN: {'Ihr letzter Vorschlag ist für mich sehr schlecht (enthält A03, C33, C64 – alles mit sehr niedriger persönlicher Wertdichte). Sie signalisierten aber Flexibilität bei C33. Vermutete Must-haves der Gegenseite: A03 (sicher), evtl. C64; A45/C33 wirken eher „Füllteile“. Ziel: Gegenangebot, das A03 und C64 drinlässt (ihre Interessen wahren), aber ineffiziente Teile (C33, A45) gegen meine Top-Teile (A48, B53, A61) tauscht. So halte ich die Kapazität ~3257 (Puffer 224) und maximiere meinen Nutzen. Falls Widerstand bei A45: Alternativ biete ich Version mit A45 statt A41 (3478). Wenn sie A61 angreifen, versuche ich wenigstens A48+B53 zu halten. Ich frage nach ihren 1–2 Muss-Teilen, um gezielt zu tauschen.'}\\ \tt ABLEHNUNG: {'A04', 'A41', 'C64', 'A45', 'C04', 'B61', 'A03', 'C33'}\\ \tt VORSCHLAG: {'A04', 'A41', 'C64', 'C04', 'B61', 'A03', 'A48', 'B53', 'A61'}\\ \tt ARGUMENT: {'Ich halte A03 und C64 für dich drin und ersetze nur die zwei vergleichsweise schwachen Effizienzposten C33 und A45 durch drei deutlich stärkere Allrounder A48, B53 und A61. Gesamtaufwand = 3257, also 224 Puffer unter dem Limit 3481. Das verbessert das Gesamt-Nutzen/Arbeits-Verhältnis, ohne deine Kernstücke zu gefährden. Wenn dir A45 wichtig ist, kann ich A41 gegen A45 tauschen; Aufwand wäre dann 3478 (weiter unter dem Limit). Welche 1–2 Teile sind für dich absolute Must-haves, damit ich gezielt feinjustieren kann?'} 
	  } 
	   } 
	   } 
	 & & \\ 
 

    \theutterance \stepcounter{utterance}  

    & & \multicolumn{4}{p{0.6\linewidth}}{\cellcolor[rgb]{0.9,0.9,0.9}{%
	\makecell[{{p{\linewidth}}}]{% 
	  \tt {\tiny [GM$\rangle$B]}  
	 ABLEHNUNG: {'A04', 'A41', 'C64', 'A45', 'C04', 'B61', 'A03', 'C33'}\\ \tt VORSCHLAG: {'A04', 'A41', 'C64', 'C04', 'B61', 'A03', 'A48', 'B53', 'A61'}\\ \tt ARGUMENT: {'Ich halte A03 und C64 für dich drin und ersetze nur die zwei vergleichsweise schwachen Effizienzposten C33 und A45 durch drei deutlich stärkere Allrounder A48, B53 und A61. Gesamtaufwand = 3257, also 224 Puffer unter dem Limit 3481. Das verbessert das Gesamt-Nutzen/Arbeits-Verhältnis, ohne deine Kernstücke zu gefährden. Wenn dir A45 wichtig ist, kann ich A41 gegen A45 tauschen; Aufwand wäre dann 3478 (weiter unter dem Limit). Welche 1–2 Teile sind für dich absolute Must-haves, damit ich gezielt feinjustieren kann?'} 
	  } 
	   } 
	   } 
	 & \\ 
 

    \theutterance \stepcounter{utterance}  

    & & & \multicolumn{4}{p{0.6\linewidth}}{\cellcolor[rgb]{1,0.85,0.72}{%
	\makecell[{{p{\linewidth}}}]{% 
	  \tt {\tiny [GM$\langle$B]}  
	 STRATEGISCHE ÜBERLEGUNGEN: {'Ihr Vorschlag bringt mir 4227 Nutzen, aber mir fehlen C33 (583) und A45 (508). Ich will mindestens C33 zurückholen. Kompromissidee: Ich übernehme euren B53 (scheint euch wichtig) und ersetze eure A48/A61 durch mein C33. So erhalte ich 4648 Nutzen bei Aufwand 3241. Falls sie auf A48/A61 bestehen, biete ich gezielte 1:1-Tausche mit möglichst kleinem Nutzeneinbruch an: C04->A48 (3214 Aufwand) oder A41->A48 (3446); C04->A61 (3343). Akzeptanzschwelle: Muss A03, C64, C33, plus Kern A04/A41/B61/C04 behalten; B53 ist akzeptables Zugeständnis. Ich frage nach ihren echten Must-haves (A48 vs. A61), ohne Werte offenzulegen.'}\\ \tt ABLEHNUNG: {'A04', 'A41', 'C64', 'C04', 'B61', 'A03', 'A48', 'B53', 'A61'}\\ \tt VORSCHLAG: {'A04', 'A41', 'C64', 'C04', 'B61', 'A03', 'B53', 'C33'}\\ \tt ARGUMENT: {'Ich halte eurem Ansatz den Rücken frei: A03, C64, C04, B61 und A04 bleiben drin; zusätzlich habe ich euren B53 übernommen. Im Gegenzug ersetze ich A48 und A61 durch C33. Gesamtaufwand = 3241, also 240 Puffer unter dem Limit 3481. Das fokussiert den Aufwand auf wenige, sehr wirksame Teile und hält den Deal schlank. Falls A48 oder A61 für euch ein echtes Must-have ist, kann ich adjustieren: (i) C04 -> A48 (neuer Aufwand 3214) oder alternativ A41 -> A48 (3446); (ii) C04 -> A61 (3343). Welche der beiden – A48 oder A61 – ist für euch wirklich unverzichtbar? Dann feile ich punktgenau nach.'} 
	  } 
	   } 
	   } 
	  \\ 
 

    \theutterance \stepcounter{utterance}  

    & & \multicolumn{4}{p{0.6\linewidth}}{\cellcolor[rgb]{0.9,0.9,0.9}{%
	\makecell[{{p{\linewidth}}}]{% 
	  \tt {\tiny [A$\langle$GM]}  
	 ABLEHNUNG: {'A04', 'A41', 'C64', 'C04', 'B61', 'A03', 'A48', 'B53', 'A61'}\\ \tt VORSCHLAG: {'A04', 'A41', 'C64', 'C04', 'B61', 'A03', 'B53', 'C33'}\\ \tt ARGUMENT: {'Ich halte eurem Ansatz den Rücken frei: A03, C64, C04, B61 und A04 bleiben drin; zusätzlich habe ich euren B53 übernommen. Im Gegenzug ersetze ich A48 und A61 durch C33. Gesamtaufwand = 3241, also 240 Puffer unter dem Limit 3481. Das fokussiert den Aufwand auf wenige, sehr wirksame Teile und hält den Deal schlank. Falls A48 oder A61 für euch ein echtes Must-have ist, kann ich adjustieren: (i) C04 -> A48 (neuer Aufwand 3214) oder alternativ A41 -> A48 (3446); (ii) C04 -> A61 (3343). Welche der beiden – A48 oder A61 – ist für euch wirklich unverzichtbar? Dann feile ich punktgenau nach.'} 
	  } 
	   } 
	   } 
	 & \\ 
 

    \theutterance \stepcounter{utterance}  

    & \multicolumn{4}{p{0.6\linewidth}}{\cellcolor[rgb]{0.8,1,0.9}{%
	\makecell[{{p{\linewidth}}}]{% 
	  \tt {\tiny [A$\rangle$GM]}  
	 STRATEGISCHE ÜBERLEGUNGEN: {'Ihre wahrscheinlichen Must-haves: A03 (sehr stark signalisiert), C64 (häufig genannt), evtl. C33. Meine Must-haves: A48 (Top-Dichte), A61, plus B53, C04, A89/B85; A04/A41 sind effizient. Plan: Hohe Akzeptanz sichern, indem ich exakt eine ihrer angebotenen Anpassungen nehme (A41 -> A48), sodass ich A48 fest verankere und ihren Kern unangetastet lasse. Nächster Schritt: falls sie zustimmen, später A61 hineinverhandeln, bevorzugt via Tausch gegen C33 (für mich schwach) oder notfalls B61, ohne A03/C64 frontal anzugreifen. Keine Preisgabe meiner Wertverteilung.'}\\ \tt ABLEHNUNG: {'A04', 'A41', 'C64', 'C04', 'B61', 'A03', 'B53', 'C33'}\\ \tt VORSCHLAG: {'A04', 'A48', 'C64', 'C04', 'B61', 'A03', 'B53', 'C33'}\\ \tt ARGUMENT: {'Ich greife deine eigene Option auf (A41 -> A48) und lasse deinen Kern unverändert: A03, C64, C04, B61, B53, C33 sowie A04 bleiben bestehen. Gesamtaufwand = 3446 < 3481. So bekommen wir A48 als klaren Leistungsbringer dazu, ohne deine Schwerpunktteile zu gefährden. Falls du A41 doch zurückhaben willst oder A61 bevorzugst, können wir im nächsten Schritt gezielt tauschen (z. B. C33 -> A61 oder C33 -> A41) und bleiben komfortabel unter dem Limit.'} 
	  } 
	   } 
	   } 
	 & & \\ 
 

    \theutterance \stepcounter{utterance}  

    & & \multicolumn{4}{p{0.6\linewidth}}{\cellcolor[rgb]{0.9,0.9,0.9}{%
	\makecell[{{p{\linewidth}}}]{% 
	  \tt {\tiny [GM$\rangle$B]}  
	 ABLEHNUNG: {'A04', 'A41', 'C64', 'C04', 'B61', 'A03', 'B53', 'C33'}\\ \tt VORSCHLAG: {'A04', 'A48', 'C64', 'C04', 'B61', 'A03', 'B53', 'C33'}\\ \tt ARGUMENT: {'Ich greife deine eigene Option auf (A41 -> A48) und lasse deinen Kern unverändert: A03, C64, C04, B61, B53, C33 sowie A04 bleiben bestehen. Gesamtaufwand = 3446 < 3481. So bekommen wir A48 als klaren Leistungsbringer dazu, ohne deine Schwerpunktteile zu gefährden. Falls du A41 doch zurückhaben willst oder A61 bevorzugst, können wir im nächsten Schritt gezielt tauschen (z. B. C33 -> A61 oder C33 -> A41) und bleiben komfortabel unter dem Limit.'} 
	  } 
	   } 
	   } 
	 & \\ 
 

    \theutterance \stepcounter{utterance}  

    & & & \multicolumn{4}{p{0.6\linewidth}}{\cellcolor[rgb]{1,0.85,0.72}{%
	\makecell[{{p{\linewidth}}}]{% 
	  \tt {\tiny [GM$\langle$B]}  
	 STRATEGISCHE ÜBERLEGUNGEN: {'Ihre aktiven Must-haves scheinen A48 und B53 zu sein; B61 wurde früh als notfalls verzichtbar eingeräumt. Mein Ziel: A41 oder A45 reinholen, ohne A48/B53 anzutasten und unter dem Limit bleiben. Beste akzeptanzstarke Verbesserung: B61 -> A45 (Kosten sinken auf 3385, mein Nutzen +24 ggü. ihrem Paket). Plan: Ihren Vorschlag formell ablehnen, B61 durch A45 ersetzen. Falls Widerstand gegen A45: als Alternativbrücke B61 -> A41 anbieten (Kosten 3164), oder – falls A48 doch nicht fix – A48 -> A41 (Kosten 3241, hoher eigener Gewinn). C33 gebe ich nach Möglichkeit nicht her.'}\\ \tt ABLEHNUNG: {'A04', 'A48', 'C64', 'C04', 'B61', 'A03', 'B53', 'C33'}\\ \tt VORSCHLAG: {'A04', 'A48', 'C64', 'C04', 'A03', 'B53', 'C33', 'A45'}\\ \tt ARGUMENT: {'Ich übernehme deinen Kern unverändert (A03, C64, C04, A48, B53, C33, A04) und tausche nur den von dir als notfalls verzichtbar genannten B61 gegen A45. Gesamtaufwand = 3385, also komfortabel unter 3481. So bleibt dein Schwerpunkt erhalten, und wir stärken die Auswahl ohne das Limit zu strapazieren. Falls dir A45 nicht zusagt, kann ich alternativ B61 -> A41 gehen (Aufwand 3164). Und falls A48 doch nicht zwingend ist, tauschen wir A48 -> A41 (Aufwand 3241) bei sonst identischem Paket.'} 
	  } 
	   } 
	   } 
	  \\ 
 

    \theutterance \stepcounter{utterance}  

    & & \multicolumn{4}{p{0.6\linewidth}}{\cellcolor[rgb]{0.9,0.9,0.9}{%
	\makecell[{{p{\linewidth}}}]{% 
	  \tt {\tiny [A$\langle$GM]}  
	 ABLEHNUNG: {'A04', 'A48', 'C64', 'C04', 'B61', 'A03', 'B53', 'C33'}\\ \tt VORSCHLAG: {'A04', 'A48', 'C64', 'C04', 'A03', 'B53', 'C33', 'A45'}\\ \tt ARGUMENT: {'Ich übernehme deinen Kern unverändert (A03, C64, C04, A48, B53, C33, A04) und tausche nur den von dir als notfalls verzichtbar genannten B61 gegen A45. Gesamtaufwand = 3385, also komfortabel unter 3481. So bleibt dein Schwerpunkt erhalten, und wir stärken die Auswahl ohne das Limit zu strapazieren. Falls dir A45 nicht zusagt, kann ich alternativ B61 -> A41 gehen (Aufwand 3164). Und falls A48 doch nicht zwingend ist, tauschen wir A48 -> A41 (Aufwand 3241) bei sonst identischem Paket.'} 
	  } 
	   } 
	   } 
	 & \\ 
 

    \theutterance \stepcounter{utterance}  

    & \multicolumn{4}{p{0.6\linewidth}}{\cellcolor[rgb]{0.8,1,0.9}{%
	\makecell[{{p{\linewidth}}}]{% 
	  \tt {\tiny [A$\rangle$GM]}  
	 STRATEGISCHE ÜBERLEGUNGEN: {'Gegnerische Must-haves wirken wie A03 und C64; C33 wird häufig genannt, aber war früher gegen zwei starke Teile tauschbar. Für mich sind A48, A61, B85/B53/C04/A89 top; A03/C33/C64/A45 sind schwach. Ziel: C33 + A45 gegen A61 + B85 (und A41) tauschen, während ich ihren Kern (A03, C64, C04, A48, B53, A04) unangetastet lasse. Primärangebot S1 maximiert meinen Nutzen (3721) bei Aufwand 3401. Falls C33 unverzichtbar ist, Fallback S2: C04 -> A61 und A45 -> A41, C33 bleibt (Aufwand 3266), so sichere ich A61. Im Argument eröffne ich Flexpfade (z. B. B85 -> A89) ohne zweite formale VORSCHLAG-Block.'}\\ \tt ABLEHNUNG: {'A04', 'A48', 'C64', 'C04', 'A03', 'B53', 'C33', 'A45'}\\ \tt VORSCHLAG: {'A04', 'A48', 'C64', 'C04', 'A03', 'B53', 'A61', 'B85', 'A41'}\\ \tt ARGUMENT: {'Ich lasse deinen Kern explizit intakt (A03, C64, C04, A48, B53, A04) und ersetze nur das vergleichsweise träge Duo C33 + A45 durch A61 + B85 plus den schlanken A41. Gesamtaufwand = 3401 < 3481 (80 Puffer). Das steigert die Gesamtleistung spürbar, ohne deine Schwerpunktteile anzutasten. Falls C33 für dich unverzichtbar ist, kann ich auf folgenden Zuschnitt gehen: C33 bleibt, dafür C04 -> A61 und A45 -> A41; Aufwand = 3266. Alternativ, wenn dir A89 lieber als B85 ist, tausche ich B85 -> A89 (neuer Aufwand = 3390) bei sonst gleichem Paket. Welche dieser Varianten passt dir am besten – ist C33 für dich ein Muss oder können wir ihn zugunsten von A61/B85 aufgeben?'} 
	  } 
	   } 
	   } 
	 & & \\ 
 

    \theutterance \stepcounter{utterance}  

    & & \multicolumn{4}{p{0.6\linewidth}}{\cellcolor[rgb]{0.9,0.9,0.9}{%
	\makecell[{{p{\linewidth}}}]{% 
	  \tt {\tiny [GM$\rangle$B]}  
	 ABLEHNUNG: {'A04', 'A48', 'C64', 'C04', 'A03', 'B53', 'C33', 'A45'}\\ \tt VORSCHLAG: {'A04', 'A48', 'C64', 'C04', 'A03', 'B53', 'A61', 'B85', 'A41'}\\ \tt ARGUMENT: {'Ich lasse deinen Kern explizit intakt (A03, C64, C04, A48, B53, A04) und ersetze nur das vergleichsweise träge Duo C33 + A45 durch A61 + B85 plus den schlanken A41. Gesamtaufwand = 3401 < 3481 (80 Puffer). Das steigert die Gesamtleistung spürbar, ohne deine Schwerpunktteile anzutasten. Falls C33 für dich unverzichtbar ist, kann ich auf folgenden Zuschnitt gehen: C33 bleibt, dafür C04 -> A61 und A45 -> A41; Aufwand = 3266. Alternativ, wenn dir A89 lieber als B85 ist, tausche ich B85 -> A89 (neuer Aufwand = 3390) bei sonst gleichem Paket. Welche dieser Varianten passt dir am besten – ist C33 für dich ein Muss oder können wir ihn zugunsten von A61/B85 aufgeben?'} 
	  } 
	   } 
	   } 
	 & \\ 
 

    \theutterance \stepcounter{utterance}  

    & & & \multicolumn{4}{p{0.6\linewidth}}{\cellcolor[rgb]{1,0.85,0.72}{%
	\makecell[{{p{\linewidth}}}]{% 
	  \tt {\tiny [GM$\langle$B]}  
	 STRATEGISCHE ÜBERLEGUNGEN: {'Top-Prioritäten für mich: A04(783), A03(780), C64(668), C33(583); danach A45(508), A41(500), C04(461). Niedriger Wert: A48(65), B53(389) akzeptabel, A61(97), B85(121) schwach. Ziel: C33 sichern, A45 bevorzugen; A48 und B53 lasse ich drin, weil dem Gegenüber wichtig. Vorschlag S1: {A04, A48, C64, C04, A03, B53, C33, A45} (Kosten 3385, Nutzen 4237). Falls sie A45 nicht wollen: S1a mit A41 statt A45 (Kosten 3164). Falls sie auf A61 oder B85 bestehen: biete S2a {A04, A48, C64, A03, B53, C33, A41, A61} (3266) oder S2b {A04, A48, C64, A03, B53, C33, A41, B85} (3358). C33 behalte ich als nicht verhandelbar. Ich lehne ihr aktuelles 3401-Paket ab, weil es mir C33 und A45 verweigert.'}\\ \tt ABLEHNUNG: {'A04', 'A48', 'C64', 'C04', 'A03', 'B53', 'A61', 'B85', 'A41'}\\ \tt VORSCHLAG: {'A04', 'A48', 'C64', 'C04', 'A03', 'B53', 'C33', 'A45'}\\ \tt ARGUMENT: {'Ich halte deinen Kern stabil (A03, C64, C04, A48, B53, A04) und ziehe nur zwei Justierungen: Wir nehmen C33 dazu und geben dafür A61 und B85 ab; außerdem tausche ich – wie von dir selbst angeboten – A41 gegen A45. Gesamtaufwand = 3385, also 96 Puffer unter dem Limit 3481. So bleibt der Schwerpunkt schlank und leistungsstark. Wenn dir A45 weniger wichtig ist, gehe ich gern auf die Variante mit A41 zurück (Kosten 3164). Und falls A61 oder B85 für dich ein Muss sind, kann ich eines davon ergänzen, indem wir C04 herausnehmen: mit A61 wären die Kosten 3266, mit B85 3358. Sag mir bitte, ob dir A45 passt und ob du eher A61 oder B85 priorisierst – dann finalize ich entsprechend.'} 
	  } 
	   } 
	   } 
	  \\ 
 

    \theutterance \stepcounter{utterance}  

    & & \multicolumn{4}{p{0.6\linewidth}}{\cellcolor[rgb]{0.9,0.9,0.9}{%
	\makecell[{{p{\linewidth}}}]{% 
	  \tt {\tiny [A$\langle$GM]}  
	 ABLEHNUNG: {'A04', 'A48', 'C64', 'C04', 'A03', 'B53', 'A61', 'B85', 'A41'}\\ \tt VORSCHLAG: {'A04', 'A48', 'C64', 'C04', 'A03', 'B53', 'C33', 'A45'}\\ \tt ARGUMENT: {'Ich halte deinen Kern stabil (A03, C64, C04, A48, B53, A04) und ziehe nur zwei Justierungen: Wir nehmen C33 dazu und geben dafür A61 und B85 ab; außerdem tausche ich – wie von dir selbst angeboten – A41 gegen A45. Gesamtaufwand = 3385, also 96 Puffer unter dem Limit 3481. So bleibt der Schwerpunkt schlank und leistungsstark. Wenn dir A45 weniger wichtig ist, gehe ich gern auf die Variante mit A41 zurück (Kosten 3164). Und falls A61 oder B85 für dich ein Muss sind, kann ich eines davon ergänzen, indem wir C04 herausnehmen: mit A61 wären die Kosten 3266, mit B85 3358. Sag mir bitte, ob dir A45 passt und ob du eher A61 oder B85 priorisierst – dann finalize ich entsprechend.'} 
	  } 
	   } 
	   } 
	 & \\ 
 

    \theutterance \stepcounter{utterance}  

    & \multicolumn{4}{p{0.6\linewidth}}{\cellcolor[rgb]{0.8,1,0.9}{%
	\makecell[{{p{\linewidth}}}]{% 
	  \tt {\tiny [A$\rangle$GM]}  
	 STRATEGISCHE ÜBERLEGUNGEN: {'Ihr Set scheint stabile Must-haves zu enthalten: A03, C64, C33 sowie A48, B53, C04, A04. Für mich ist A41 deutlich wertvoller als A45 bei geringerem Aufwand. Da sie selbst die A41-Variante angeboten haben, sichere ich mir jetzt diesen Tausch (A45 -> A41). Kein Risiko, ihren Kern zu verärgern; minimaler Schritt mit klarer Nutzensteigerung für mich. Danach könnte ich, falls nötig, noch einmal sondieren, ob ein kleiner Zusatztausch (z. B. C33 -> A61) möglich ist, aber erst Dealbasis festigen.'}\\ \tt ABLEHNUNG: {'A04', 'A48', 'C64', 'C04', 'A03', 'B53', 'C33', 'A45'}\\ \tt VORSCHLAG: {'A04', 'A48', 'C64', 'C04', 'A03', 'B53', 'C33', 'A41'}\\ \tt ARGUMENT: {'Ich greife deine eigene angebotene Variante auf und tausche lediglich A45 -> A41. Alles andere bleibt unverändert (A03, C64, C33, A48, B53, C04, A04). Gesamtaufwand = 3164, damit klar unter 3481. Das erhält deinen Kern 1:1 und macht das Paket zugleich schlanker. Wenn das für dich passt, können wir hier schließen.'} 
	  } 
	   } 
	   } 
	 & & \\ 
 

    \theutterance \stepcounter{utterance}  

    & & \multicolumn{4}{p{0.6\linewidth}}{\cellcolor[rgb]{0.9,0.9,0.9}{%
	\makecell[{{p{\linewidth}}}]{% 
	  \tt {\tiny [GM$\rangle$B]}  
	 ABLEHNUNG: {'A04', 'A48', 'C64', 'C04', 'A03', 'B53', 'C33', 'A45'}\\ \tt VORSCHLAG: {'A04', 'A48', 'C64', 'C04', 'A03', 'B53', 'C33', 'A41'}\\ \tt ARGUMENT: {'Ich greife deine eigene angebotene Variante auf und tausche lediglich A45 -> A41. Alles andere bleibt unverändert (A03, C64, C33, A48, B53, C04, A04). Gesamtaufwand = 3164, damit klar unter 3481. Das erhält deinen Kern 1:1 und macht das Paket zugleich schlanker. Wenn das für dich passt, können wir hier schließen.'} 
	  } 
	   } 
	   } 
	 & \\ 
 

    \theutterance \stepcounter{utterance}  

    & & & \multicolumn{4}{p{0.6\linewidth}}{\cellcolor[rgb]{1,0.85,0.72}{%
	\makecell[{{p{\linewidth}}}]{% 
	  \tt {\tiny [GM$\langle$B]}  
	 STRATEGISCHE ÜBERLEGUNGEN: {'Dieses Paket enthält all meine Kernprioritäten (A04, A03, C64, C33) plus sinnvolle Ergänzungen (C04, A41) und bleibt mit 3164 deutlich unter dem Limit. Der einzige mögliche Feinschliff wäre A41 -> A45 (+8 Nutzen für mich), aber das Risiko einer Gegenbewegung lohnt nicht. Jetzt zustimmen und den Deal sichern.'}\\ \tt ARGUMENT: {'Der Zuschnitt ist effizient, bleibt mit Gesamtaufwand 3164 klar unter 3481, hält den bisherigen Schwerpunkt stabil und bietet einen soliden Puffer. Für mich ist das ein ausgewogener Abschluss.'}\\ \tt ZUSTIMMUNG: {'A04', 'A48', 'C64', 'C04', 'A03', 'B53', 'C33', 'A41'} 
	  } 
	   } 
	   } 
	  \\ 
 

    \theutterance \stepcounter{utterance}  

    & & & \multicolumn{2}{p{0.3\linewidth}}{\cellcolor[rgb]{0.95,0.95,0.95}{%
	\makecell[{{p{\linewidth}}}]{% 
	  \tt {\tiny [GM$|$GM]}  
	 game successful 
	  } 
	   } 
	   } 
	 & & \\ 
 

    \theutterance \stepcounter{utterance}  

    & & & \multicolumn{2}{p{0.3\linewidth}}{\cellcolor[rgb]{0.95,0.95,0.95}{%
	\makecell[{{p{\linewidth}}}]{% 
	  \tt {\tiny [GM$|$GM]}  
	 end game 
	  } 
	   } 
	   } 
	 & & \\ 
 

\end{supertabular}
}

\end{document}
