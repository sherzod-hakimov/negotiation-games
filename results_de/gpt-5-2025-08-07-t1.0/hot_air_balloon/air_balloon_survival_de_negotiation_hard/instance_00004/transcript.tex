
\documentclass{article}
\usepackage{colortbl}
\usepackage{makecell}
\usepackage{multirow}
\usepackage{supertabular}

\begin{document}

\newcounter{utterance}

\centering \large Interaction Transcript for game `hot\_air\_balloon', experiment `air\_balloon\_survival\_de\_negotiation\_hard', episode 4 with gpt{-}5{-}2025{-}08{-}07{-}t1.0.
\vspace{24pt}

{ \footnotesize  \setcounter{utterance}{1}
\setlength{\tabcolsep}{0pt}
\begin{supertabular}{c@{$\;$}|p{.15\linewidth}@{}p{.15\linewidth}p{.15\linewidth}p{.15\linewidth}p{.15\linewidth}p{.15\linewidth}}
    \# & $\;$A & \multicolumn{4}{c}{Game Master} & $\;\:$B\\
    \hline

    \theutterance \stepcounter{utterance}  
    & & \multicolumn{4}{p{0.6\linewidth}}{
        \cellcolor[rgb]{0.9,0.9,0.9}{
            \makecell[{{p{\linewidth}}}]{
                \texttt{\tiny{[P1$\langle$GM]}}
                \texttt{Du nimmst an einem kollaborativen Verhandlungspiel Teil.} \\
\\ 
\texttt{Zusammen mit einem anderen Teilnehmer musst du dich auf eine Reihe von Gegenständen entscheiden, die behalten werden. Jeder von euch hat eine persönliche Verteilung über die Wichtigkeit der einzelnen Gegenstände. Jeder von euch hat eine eigene Meinung darüber, wie wichtig jeder einzelne Gegenstand ist (Gegenstandswichtigkeit). Du kennst die Wichtigkeitsverteilung des anderen Spielers nicht. Zusätzlich siehst du, wie viel Aufwand jeder Gegenstand verursacht.  } \\
\texttt{Ihr dürft euch nur auf eine Reihe von Gegenständen einigen, wenn der Gesamtaufwand der ausgewählten Gegenstände den Maximalaufwand nicht überschreitet:} \\
\\ 
\texttt{Maximalaufwand = 5138} \\
\\ 
\texttt{Hier sind die einzelnen Aufwände der Gegenstände:} \\
\\ 
\texttt{Aufwand der Gegenstände = \{"C33": 903, "A89": 757, "A48": 850, "A04": 746, "A41": 524, "C64": 204, "C49": 946, "A61": 473, "C04": 616, "B53": 855, "B85": 530, "B61": 419, "A45": 960, "A03": 763, "B21": 730\}} \\
\\ 
\texttt{Hier ist deine persönliche Verteilung der Wichtigkeit der einzelnen Gegenstände:} \\
\\ 
\texttt{Werte der Gegenstandswichtigkeit = \{"C33": 138, "A89": 583, "A48": 868, "A04": 822, "A41": 783, "C64": 65, "C49": 262, "A61": 121, "C04": 508, "B53": 780, "B85": 461, "B61": 484, "A45": 668, "A03": 389, "B21": 808\}} \\
\\ 
\texttt{Ziel:} \\
\\ 
\texttt{Dein Ziel ist es, eine Reihe von Gegenständen auszuhandeln, die dir möglichst viel bringt (d. h. Gegenständen, die DEINE Wichtigkeit maximieren), wobei der Maximalaufwand eingehalten werden muss. Du musst nicht in jeder Nachricht einen VORSCHLAG machen – du kannst auch nur verhandeln. Alle Taktiken sind erlaubt!} \\
\\ 
\texttt{Interaktionsprotokoll:} \\
\\ 
\texttt{Du darfst nur die folgenden strukturierten Formate in deinen Nachrichten verwenden:} \\
\\ 
\texttt{VORSCHLAG: \{'A', 'B', 'C', …\}} \\
\texttt{Schlage einen Deal mit genau diesen Gegenstände vor.} \\
\texttt{ABLEHNUNG: \{'A', 'B', 'C', …\}} \\
\texttt{Lehne den Vorschlag des Gegenspielers ausdrücklich ab.} \\
\texttt{ARGUMENT: \{'...'\}} \\
\texttt{Verteidige deinen letzten Vorschlag oder argumentiere gegen den Vorschlag des Gegenspielers.} \\
\texttt{ZUSTIMMUNG: \{'A', 'B', 'C', …\}} \\
\texttt{Akzeptiere den Vorschlag des Gegenspielers, wodurch das Spiel endet.} \\
\texttt{STRATEGISCHE ÜBERLEGUNGEN: \{'...'\}} \\
\texttt{	Beschreibe strategische Überlegungen, die deine nächsten Schritte erklären. Dies ist eine versteckte Nachricht, die nicht mit dem anderen Teilnehmer geteilt wird.} \\
\\ 
\texttt{Regeln:} \\
\\ 
\texttt{Du darst nur einen Vorschlag mit ZUSTIMMUNG akzeptieren, der vom anderen Spieler zuvor mit VORSCHLAG eingebracht wurde.} \\
\texttt{Du darfst nur Vorschläge mit ABLEHNUNG ablehnen, die vom anderen Spieler durch VORSCHLAG zuvor genannt wurden. } \\
\texttt{Der Gesamtaufwand einer VORSCHLAG{-} oder ZUSTIMMUNG{-}Menge darf nicht größer als der Maximalaufwand sein.  } \\
\texttt{Offenbare deine versteckte Wichtigkeitsverteilung nicht.} \\
\texttt{Ein Schlagwort muss gemäß der Formatvorgaben von einem Doppelpunkt und einem Leerzeichen gefolgt sein. Das Argument muss eine Python{-}Menge mit 0 oder mehr Strings sein.  } \\
\texttt{Die Form muss also SCHLAGWORT: \{...\} sein.} \\
\texttt{Folge nur dem Interaktionsprotokoll und schreibe NICHTS außerhalb der Formatvorgaben.} \\
\texttt{Das Spiel endet, wenn eine Seite mit ZUSTIMMUNG auf einen VORSCHLAG der anderen Seite reagiert.  } \\
\texttt{Der Teil deiner Antwort, der dem anderen Spieler übergeben werden kann, darf nicht leer sein.  } \\
\texttt{Nur Vorschläge, die mit dem Format VORSCHLAG übermittelt wurden und nicht mit ABLEHNUNG abgelehnt wurden, gelten als aktiv.  } \\
\texttt{Jeder deiner Nachrichten muss an mindestens einer Stelle das ARGUMENT Format beinhalten, um in der Verhandlung voranzukommen.} \\
\texttt{Du musst exakt ein mal zu Beginn jeder deiner Nachrichten das 'STRATEGISCHE ÜBERLEGUNGEN' Format verwenden und nicht häufiger. Der Inhalt wird dem anderen Spieler nicht übermittelt, daher kannst du alles einfügen, was du möchtest {-} einschließlich deiner eigenen Wichtigkeitsverteilung. Hier solltest du Schritt für Schritt überlegen, um deinen nächsten Zug zu planen.} \\
            }
        }
    }
    & \\ \\

    \theutterance \stepcounter{utterance}  
    & \multicolumn{4}{p{0.6\linewidth}}{
        \cellcolor[rgb]{0.9,0.9,0.9}{
            \makecell[{{p{\linewidth}}}]{
                \texttt{\tiny{[P1$\rangle$GM]}}
                \texttt{STRATEGISCHE ÜBERLEGUNGEN: \{'Budget 5138. Eigene Werte (versteckt): C33 138, A89 583, A48 868, A04 822, A41 783, C64 65, C49 262, A61 121, C04 508, B53 780, B85 461, B61 484, A45 668, A03 389, B21 808. Ich habe Knapsack gerechnet und die beste gefundene Kombination ist \{A48, A04, B21, B53, A41, A45, B61\} mit Aufwand 5084 und Wert 5213. Zweitbeste nahe Alternative: Ersetze A45 durch A89 und C64 → \{A48, A04, B21, B53, A41, B61, A89, C64\} mit Aufwand 5085 und Wert 5193 (nur {-}20). Damit ist A45 mein flexibelster Zugeständnis{-}Posten; den Rest möchte ich halten (insb. A41, B61, B21, A04, A48, B53). Taktik: hart ankern mit dem Top{-}Set; bei Gegenwehr zuerst A45 anbieten und um A89 (+ ggf. C64) tauschen, stets Budget wahren. Nicht eigene Werte offenlegen.'\}} \\
\texttt{VORSCHLAG: \{'A48', 'A04', 'B21', 'B53', 'A41', 'A45', 'B61'\}} \\
\texttt{ARGUMENT: \{'Dieser Vorschlag nutzt den Maximalaufwand sehr effizient (5084 von 5138) und bündelt aus meiner Sicht besonders wirkungsstarke Posten. Die Auswahl balanciert hohe Wirkung und moderaten Aufwand und lässt einen kleinen Puffer. Wenn du starke Präferenzen hast, bin ich offen für gezielte Tausche einzelner Posten, solange wir im Aufwandslimit bleiben.'\}} \\
            }
        }
    }
    & & \\ \\

    \theutterance \stepcounter{utterance}  
    & & \multicolumn{4}{p{0.6\linewidth}}{
        \cellcolor[rgb]{0.9,0.9,0.9}{
            \makecell[{{p{\linewidth}}}]{
                \texttt{\tiny{[GM$\rangle$P2]}}
                \texttt{Du nimmst an einem kollaborativen Verhandlungspiel Teil.} \\
\\ 
\texttt{Zusammen mit einem anderen Teilnehmer musst du dich auf eine Reihe von Gegenständen entscheiden, die behalten werden. Jeder von euch hat eine persönliche Verteilung über die Wichtigkeit der einzelnen Gegenstände. Jeder von euch hat eine eigene Meinung darüber, wie wichtig jeder einzelne Gegenstand ist (Gegenstandswichtigkeit). Du kennst die Wichtigkeitsverteilung des anderen Spielers nicht. Zusätzlich siehst du, wie viel Aufwand jeder Gegenstand verursacht.  } \\
\texttt{Ihr dürft euch nur auf eine Reihe von Gegenständen einigen, wenn der Gesamtaufwand der ausgewählten Gegenstände den Maximalaufwand nicht überschreitet:} \\
\\ 
\texttt{Maximalaufwand = 5138} \\
\\ 
\texttt{Hier sind die einzelnen Aufwände der Gegenstände:} \\
\\ 
\texttt{Aufwand der Gegenstände = \{"C33": 903, "A89": 757, "A48": 850, "A04": 746, "A41": 524, "C64": 204, "C49": 946, "A61": 473, "C04": 616, "B53": 855, "B85": 530, "B61": 419, "A45": 960, "A03": 763, "B21": 730\}} \\
\\ 
\texttt{Hier ist deine persönliche Verteilung der Wichtigkeit der einzelnen Gegenstände:} \\
\\ 
\texttt{Werte der Gegenstandswichtigkeit = \{"A48": 65, "A04": 121, "B21": 138, "A41": 262, "B53": 389, "A45": 461, "A89": 484, "C04": 508, "B61": 583, "B85": 668, "A03": 780, "C49": 783, "C33": 808, "A61": 822, "C64": 868\}} \\
\\ 
\texttt{Ziel:} \\
\\ 
\texttt{Dein Ziel ist es, eine Reihe von Gegenständen auszuhandeln, die dir möglichst viel bringt (d. h. Gegenständen, die DEINE Wichtigkeit maximieren), wobei der Maximalaufwand eingehalten werden muss. Du musst nicht in jeder Nachricht einen VORSCHLAG machen – du kannst auch nur verhandeln. Alle Taktiken sind erlaubt!} \\
\\ 
\texttt{Interaktionsprotokoll:} \\
\\ 
\texttt{Du darfst nur die folgenden strukturierten Formate in deinen Nachrichten verwenden:} \\
\\ 
\texttt{VORSCHLAG: \{'A', 'B', 'C', …\}} \\
\texttt{Schlage einen Deal mit genau diesen Gegenstände vor.} \\
\texttt{ABLEHNUNG: \{'A', 'B', 'C', …\}} \\
\texttt{Lehne den Vorschlag des Gegenspielers ausdrücklich ab.} \\
\texttt{ARGUMENT: \{'...'\}} \\
\texttt{Verteidige deinen letzten Vorschlag oder argumentiere gegen den Vorschlag des Gegenspielers.} \\
\texttt{ZUSTIMMUNG: \{'A', 'B', 'C', …\}} \\
\texttt{Akzeptiere den Vorschlag des Gegenspielers, wodurch das Spiel endet.} \\
\texttt{STRATEGISCHE ÜBERLEGUNGEN: \{'...'\}} \\
\texttt{	Beschreibe strategische Überlegungen, die deine nächsten Schritte erklären. Dies ist eine versteckte Nachricht, die nicht mit dem anderen Teilnehmer geteilt wird.} \\
\\ 
\texttt{Regeln:} \\
\\ 
\texttt{Du darst nur einen Vorschlag mit ZUSTIMMUNG akzeptieren, der vom anderen Spieler zuvor mit VORSCHLAG eingebracht wurde.} \\
\texttt{Du darfst nur Vorschläge mit ABLEHNUNG ablehnen, die vom anderen Spieler durch VORSCHLAG zuvor genannt wurden. } \\
\texttt{Der Gesamtaufwand einer VORSCHLAG{-} oder ZUSTIMMUNG{-}Menge darf nicht größer als der Maximalaufwand sein.  } \\
\texttt{Offenbare deine versteckte Wichtigkeitsverteilung nicht.} \\
\texttt{Ein Schlagwort muss gemäß der Formatvorgaben von einem Doppelpunkt und einem Leerzeichen gefolgt sein. Das Argument muss eine Python{-}Menge mit 0 oder mehr Strings sein.  } \\
\texttt{Die Form muss also SCHLAGWORT: \{...\} sein.} \\
\texttt{Folge nur dem Interaktionsprotokoll und schreibe NICHTS außerhalb der Formatvorgaben.} \\
\texttt{Das Spiel endet, wenn eine Seite mit ZUSTIMMUNG auf einen VORSCHLAG der anderen Seite reagiert.  } \\
\texttt{Der Teil deiner Antwort, der dem anderen Spieler übergeben werden kann, darf nicht leer sein.  } \\
\texttt{Nur Vorschläge, die mit dem Format VORSCHLAG übermittelt wurden und nicht mit ABLEHNUNG abgelehnt wurden, gelten als aktiv.  } \\
\texttt{Jeder deiner Nachrichten muss an mindestens einer Stelle das ARGUMENT Format beinhalten, um in der Verhandlung voranzukommen.} \\
\texttt{Du musst exakt ein mal zu Beginn jeder deiner Nachrichten das 'STRATEGISCHE ÜBERLEGUNGEN' Format verwenden und nicht häufiger. Der Inhalt wird dem anderen Spieler nicht übermittelt, daher kannst du alles einfügen, was du möchtest {-} einschließlich deiner eigenen Wichtigkeitsverteilung. Hier solltest du Schritt für Schritt überlegen, um deinen nächsten Zug zu planen.} \\
\\ 
\texttt{Du erhältst nun die erste Nachricht des anderen Spielers.} \\
\\ 
\texttt{VORSCHLAG: \{'A48', 'A04', 'B21', 'B53', 'A41', 'A45', 'B61'\}} \\
\texttt{ARGUMENT: \{'Dieser Vorschlag nutzt den Maximalaufwand sehr effizient (5084 von 5138) und bündelt aus meiner Sicht besonders wirkungsstarke Posten. Die Auswahl balanciert hohe Wirkung und moderaten Aufwand und lässt einen kleinen Puffer. Wenn du starke Präferenzen hast, bin ich offen für gezielte Tausche einzelner Posten, solange wir im Aufwandslimit bleiben.'\}} \\
            }
        }
    }
    & \\ \\

    \theutterance \stepcounter{utterance}  
    & & & \multicolumn{4}{p{0.6\linewidth}}{
        \cellcolor[rgb]{0.9,0.9,0.9}{
            \makecell[{{p{\linewidth}}}]{
                \texttt{\tiny{[GM$\langle$P2]}}
                \texttt{STRATEGISCHE ÜBERLEGUNGEN: \{'Meine Werte (geheim): C64 868, A61 822, C33 808, C49 783, A03 780, B85 668, B61 583, C04 508, A89 484, A45 461, B53 389, A41 262, B21 138, A04 121, A48 65. Beste Kombi unter 5138 Aufwand: \{C64, A61, B61, B85, A03, C33, C49, C04\} mit Aufwand 4854 und Wert 5820. Anker damit setzen. Falls Widerstand: Kompromiss 1 \{B53, A41, B61, C64, A61, C33, C49, B85\} (Aufwand 4854, Wert 5183) hält mehrere ihrer Posten. Kompromiss 2: A45 rein statt C49 (Aufwand 4868, Wert 5498), falls A45 für sie zentral ist. Hart verteidigen: C64, A61, B61. Verhandlungsziel: ihre Must{-}haves abfragen, dann C49/C04/A03 als Tauschmasse anbieten, ohne Budget zu reißen.'\}} \\
\texttt{ABLEHNUNG: \{'A48', 'A04', 'B21', 'B53', 'A41', 'A45', 'B61'\}} \\
\texttt{VORSCHLAG: \{'C64', 'A61', 'B61', 'B85', 'A03', 'C33', 'C49', 'C04'\}} \\
\texttt{ARGUMENT: \{'Dein Vorschlag liegt zwar effizient unter dem Limit, aber einige der enthaltenen Posten kombinieren aus meiner Sicht hohen Aufwand mit geringerer Wirkung (z. B. A48, A04, B21, A45). Mein Gegenvorschlag bleibt klar im Rahmen (4854 von 5138) und setzt stärker auf hochwirksame, vergleichsweise günstige Posten. B61 behalten wir beidseitig. Wenn dir einzelne Posten besonders wichtig sind, nenne mir bitte 1–2 Must{-}haves; ich kann im Gegenzug gezielt tauschen (z. B. einen aus C49, C04 oder A03), solange wir das Aufwandslimit einhalten.'\}} \\
            }
        }
    }
     \\ \\

    \theutterance \stepcounter{utterance}  
    & & \multicolumn{4}{p{0.6\linewidth}}{
        \cellcolor[rgb]{0.9,0.9,0.9}{
            \makecell[{{p{\linewidth}}}]{
                \texttt{\tiny{[P1$\langle$GM]}}
                \texttt{ABLEHNUNG: \{'A48', 'A04', 'B21', 'B53', 'A41', 'A45', 'B61'\}} \\
\texttt{VORSCHLAG: \{'C64', 'A61', 'B61', 'B85', 'A03', 'C33', 'C49', 'C04'\}} \\
\texttt{ARGUMENT: \{'Dein Vorschlag liegt zwar effizient unter dem Limit, aber einige der enthaltenen Posten kombinieren aus meiner Sicht hohen Aufwand mit geringerer Wirkung (z. B. A48, A04, B21, A45). Mein Gegenvorschlag bleibt klar im Rahmen (4854 von 5138) und setzt stärker auf hochwirksame, vergleichsweise günstige Posten. B61 behalten wir beidseitig. Wenn dir einzelne Posten besonders wichtig sind, nenne mir bitte 1–2 Must{-}haves; ich kann im Gegenzug gezielt tauschen (z. B. einen aus C49, C04 oder A03), solange wir das Aufwandslimit einhalten.'\}} \\
            }
        }
    }
    & \\ \\

    \theutterance \stepcounter{utterance}  
    & \multicolumn{4}{p{0.6\linewidth}}{
        \cellcolor[rgb]{0.9,0.9,0.9}{
            \makecell[{{p{\linewidth}}}]{
                \texttt{\tiny{[P1$\rangle$GM]}}
                \texttt{STRATEGISCHE ÜBERLEGUNGEN: \{'Gegenvorschlag des Gegenübers hat für mich niedrigen Wert (viele schwache Posten: C33, C49, A61, A03) und nutzt Budget wenig. Ich lehne ab und biete ein Kompromisset an, das drei ihrer Posten (B61, C04, C64) übernimmt, aber meine starken Kernposten (A48, A04, B21, B53, A41) wahrt. Set S4 = \{A48, A04, B21, B53, A41, B61, C04, C64\}, Aufwand 4944, Wert 5118. Falls sie B85 verlangen, kann ich B61 und C64 gegen B85 tauschen → Aufwand 4851, Wert 5030. Alternativ, wenn sie B61/C64 nicht wollen, kann ich A89 statt beiden nehmen: \{A48, A04, B21, B53, A41, C04, A89\}, Aufwand 5078, Wert 5152. Nächster Schritt bei Widerstand: anbieten eines dieser beiden Tauschpfade. Keine Offenlegung meiner Werte.'\}} \\
\texttt{ABLEHNUNG: \{'C64', 'A61', 'B61', 'B85', 'A03', 'C33', 'C49', 'C04'\}} \\
\texttt{VORSCHLAG: \{'A48', 'A04', 'B21', 'B53', 'A41', 'B61', 'C04', 'C64'\}} \\
\texttt{ARGUMENT: \{'Dein Paket bindet Budget in mehreren für mich schwächeren Posten (u. a. C33, C49, A61, A03) und lässt viel Spielraum ungenutzt. Mein Gegenvorschlag bleibt klar im Rahmen (4944 von 5138), übernimmt drei deiner Elemente (B61, C04, C64) und ersetzt den Rest durch aus meiner Sicht wesentlich wirksamere Positionen. Wenn B85 dir wichtig ist, können wir B61 und C64 gegen B85 tauschen und bleiben im Limit. Alternativ, falls du B61/C64 nicht priorisierst, kann ich stattdessen A89 aufnehmen und ebenfalls im Rahmen bleiben. Nenne mir bitte 1–2 Must{-}haves von deiner Seite, dann passe ich gezielt an.'\}} \\
            }
        }
    }
    & & \\ \\

    \theutterance \stepcounter{utterance}  
    & & \multicolumn{4}{p{0.6\linewidth}}{
        \cellcolor[rgb]{0.9,0.9,0.9}{
            \makecell[{{p{\linewidth}}}]{
                \texttt{\tiny{[GM$\rangle$P2]}}
                \texttt{ABLEHNUNG: \{'C64', 'A61', 'B61', 'B85', 'A03', 'C33', 'C49', 'C04'\}} \\
\texttt{VORSCHLAG: \{'A48', 'A04', 'B21', 'B53', 'A41', 'B61', 'C04', 'C64'\}} \\
\texttt{ARGUMENT: \{'Dein Paket bindet Budget in mehreren für mich schwächeren Posten (u. a. C33, C49, A61, A03) und lässt viel Spielraum ungenutzt. Mein Gegenvorschlag bleibt klar im Rahmen (4944 von 5138), übernimmt drei deiner Elemente (B61, C04, C64) und ersetzt den Rest durch aus meiner Sicht wesentlich wirksamere Positionen. Wenn B85 dir wichtig ist, können wir B61 und C64 gegen B85 tauschen und bleiben im Limit. Alternativ, falls du B61/C64 nicht priorisierst, kann ich stattdessen A89 aufnehmen und ebenfalls im Rahmen bleiben. Nenne mir bitte 1–2 Must{-}haves von deiner Seite, dann passe ich gezielt an.'\}} \\
            }
        }
    }
    & \\ \\

    \theutterance \stepcounter{utterance}  
    & & & \multicolumn{4}{p{0.6\linewidth}}{
        \cellcolor[rgb]{0.9,0.9,0.9}{
            \makecell[{{p{\linewidth}}}]{
                \texttt{\tiny{[GM$\langle$P2]}}
                \texttt{STRATEGISCHE ÜBERLEGUNGEN: \{'Ziel: Möglichst viele Top{-}Werte sichern, ohne die Gegenseite zu blockieren. Meine Wertungen (wichtigste zuerst): C64 868, A61 822, C33 808, C49 783, A03 780, B85 668, B61 583, C04 508, A89 484, A45 461, B53 389, A41 262, B21 138, A04 121, A48 65. Ihr letztes Paket enthält viele für mich schwache Posten (A48, A04, B21). Taktik: 1) Ihre Kernstücke (B53, A41, B61, C04, C64) weitgehend übernehmen, 2) schwache Posten durch meine hocheffizienten (A61, C33, B85) ersetzen, 3) Must{-}haves signalisieren: C64, B61; weiche Präferenz: A61/C33; 4) Alternativpfade anbieten (A89 statt A61; A45 statt B85/C33), um Akzeptanz zu erleichtern. Harte No{-}Gos: A48, A04, B21. Anker{-}Vorschlag S2 = \{B53, A41, B61, C04, C64, A61, C33, B85\} Aufwand 4524, Wert 4908. Falls nötig, Trade{-}Optionen: S2a (A89 statt A61) Aufwand 4808; S2b (A45 statt B85) Aufwand 4954.'\}} \\
\texttt{ABLEHNUNG: \{'A48', 'A04', 'B21', 'B53', 'A41', 'B61', 'C04', 'C64'\}} \\
\texttt{VORSCHLAG: \{'B53', 'A41', 'B61', 'C04', 'C64', 'A61', 'C33', 'B85'\}} \\
\texttt{ARGUMENT: \{'Ich übernehme den Kern deines Pakets (B53, A41, B61, C04, C64) und ersetze nur die drei für mich am wenigsten effizienten Posten (A48, A04, B21) durch A61, C33 und B85. Wir bleiben klar im Rahmen (4524 von 5138) und nutzen das Budget spürbar wirksamer. Must{-}haves für mich sind C64 und B61. Wenn du A89 stark priorisierst, kann ich A61 → A89 tauschen und wir liegen bei 4808 Aufwand. Falls A45 zentral ist, kann ich B85 oder C33 gegen A45 tauschen (weiterhin im Limit). Nenne mir bitte 1–2 Must{-}haves von dir, dann finalize ich entsprechend.'\}} \\
            }
        }
    }
     \\ \\

    \theutterance \stepcounter{utterance}  
    & & \multicolumn{4}{p{0.6\linewidth}}{
        \cellcolor[rgb]{0.9,0.9,0.9}{
            \makecell[{{p{\linewidth}}}]{
                \texttt{\tiny{[P1$\langle$GM]}}
                \texttt{ABLEHNUNG: \{'A48', 'A04', 'B21', 'B53', 'A41', 'B61', 'C04', 'C64'\}} \\
\texttt{VORSCHLAG: \{'B53', 'A41', 'B61', 'C04', 'C64', 'A61', 'C33', 'B85'\}} \\
\texttt{ARGUMENT: \{'Ich übernehme den Kern deines Pakets (B53, A41, B61, C04, C64) und ersetze nur die drei für mich am wenigsten effizienten Posten (A48, A04, B21) durch A61, C33 und B85. Wir bleiben klar im Rahmen (4524 von 5138) und nutzen das Budget spürbar wirksamer. Must{-}haves für mich sind C64 und B61. Wenn du A89 stark priorisierst, kann ich A61 → A89 tauschen und wir liegen bei 4808 Aufwand. Falls A45 zentral ist, kann ich B85 oder C33 gegen A45 tauschen (weiterhin im Limit). Nenne mir bitte 1–2 Must{-}haves von dir, dann finalize ich entsprechend.'\}} \\
            }
        }
    }
    & \\ \\

    \theutterance \stepcounter{utterance}  
    & \multicolumn{4}{p{0.6\linewidth}}{
        \cellcolor[rgb]{0.9,0.9,0.9}{
            \makecell[{{p{\linewidth}}}]{
                \texttt{\tiny{[P1$\rangle$GM]}}
                \texttt{STRATEGISCHE ÜBERLEGUNGEN: \{'Ihr Vorschlag hat für mich geringen Wert (A61, C33, B85 schwach), aber sie haben klar zwei Must{-}haves: B61 und C64. Sie boten A61→A89 explizit an. Besserer Kompromiss: Ich behalte ihren Kern (B53, A41, B61, C04, C64), ersetze nur A61, C33, B85 durch A89, A04, A48. Neues Set S = \{A48, A04, B53, A41, B61, C04, C64, A89\}, Aufwand 4971, hoher eigener Wert. Falls sie B85 unbedingt wollen, kann ich C04→B85 tauschen (Aufwand 4885) mit kleinem Wertverlust für mich; B61 und C64 bleiben. Als nächste Eskalation könnte ich C04→B21 anbieten (Aufwand 5085) falls sie C04 fallen lassen, was meinen Wert weiter steigert, aber ich biete das erst, wenn nötig. Nicht eigene Zahlen offenlegen.'\}} \\
\texttt{ABLEHNUNG: \{'B53', 'A41', 'B61', 'C04', 'C64', 'A61', 'C33', 'B85'\}} \\
\texttt{VORSCHLAG: \{'A48', 'A04', 'B53', 'A41', 'B61', 'C04', 'C64', 'A89'\}} \\
\texttt{ARGUMENT: \{'Ich übernehme deinen Kern und deine Must{-}haves (B61, C64) vollständig, halte C04 drin und tausche nur drei Positionen aus – A61 → A89 (wie von dir vorgeschlagen), C33 → A04 und B85 → A48. Wir bleiben komfortabel im Rahmen (4971 von 5138) und nutzen das Budget deutlich produktiver. Wenn dir B85 wichtig ist, kann ich alternativ C04 → B85 tauschen und wir liegen bei 4885 Aufwand, B61 und C64 bleiben gesetzt. Gib mir bitte Bescheid, ob B85 für dich ein Muss ist; dann adjustiere ich entsprechend.'\}} \\
            }
        }
    }
    & & \\ \\

    \theutterance \stepcounter{utterance}  
    & & \multicolumn{4}{p{0.6\linewidth}}{
        \cellcolor[rgb]{0.9,0.9,0.9}{
            \makecell[{{p{\linewidth}}}]{
                \texttt{\tiny{[GM$\rangle$P2]}}
                \texttt{ABLEHNUNG: \{'B53', 'A41', 'B61', 'C04', 'C64', 'A61', 'C33', 'B85'\}} \\
\texttt{VORSCHLAG: \{'A48', 'A04', 'B53', 'A41', 'B61', 'C04', 'C64', 'A89'\}} \\
\texttt{ARGUMENT: \{'Ich übernehme deinen Kern und deine Must{-}haves (B61, C64) vollständig, halte C04 drin und tausche nur drei Positionen aus – A61 → A89 (wie von dir vorgeschlagen), C33 → A04 und B85 → A48. Wir bleiben komfortabel im Rahmen (4971 von 5138) und nutzen das Budget deutlich produktiver. Wenn dir B85 wichtig ist, kann ich alternativ C04 → B85 tauschen und wir liegen bei 4885 Aufwand, B61 und C64 bleiben gesetzt. Gib mir bitte Bescheid, ob B85 für dich ein Muss ist; dann adjustiere ich entsprechend.'\}} \\
            }
        }
    }
    & \\ \\

    \theutterance \stepcounter{utterance}  
    & & & \multicolumn{4}{p{0.6\linewidth}}{
        \cellcolor[rgb]{0.9,0.9,0.9}{
            \makecell[{{p{\linewidth}}}]{
                \texttt{\tiny{[GM$\langle$P2]}}
                \texttt{STRATEGISCHE ÜBERLEGUNGEN: \{'Gegner priorisiert offenbar A48, A04, B53, A41 sowie B61, C64, C04, A89. Meine Top{-}Werte: C64, A61, C33, C49, A03, B85, B61, C04, A89. Ziel: Mindestens A61 und C33 hineinverhandeln; B61, C64 bleiben gesetzt. Beste erreichbare Kompromisslinie, die viele ihrer Präferenzen wahrt: \{C33, A61, B53, A41, B61, C04, C64, A89\} (Aufwand 4751, hoher Eigenwert). Falls Widerstand bei A48/A04: 1) A41 → A48 (Aufwand 5077) als Zugeständnis, 2) A61 → A04 (Aufwand 5024) als Alternative. Zusätzlich sweetener, den sie selbst anboten: C04 → B85 (Aufwand 4665), was mir sogar besser passt. Eskalationspfad bei erneutem Nein: nächster Vorschlag mit B85 statt C04 und ggf. A41 → A48, um ihnen ein Must{-}have zu geben, ohne meine Kernnutzen (A61, C33) preiszugeben.'\}} \\
\texttt{ABLEHNUNG: \{'A48', 'A04', 'B53', 'A41', 'B61', 'C04', 'C64', 'A89'\}} \\
\texttt{VORSCHLAG: \{'C33', 'A61', 'B53', 'A41', 'B61', 'C04', 'C64', 'A89'\}} \\
\texttt{ARGUMENT: \{'Ich übernehme 6 deiner 8 Positionen (B53, A41, B61, C04, C64, A89) vollständig und tausche nur zwei schwächere Hebel (A48, A04) gegen zwei aus meiner Sicht deutlich wirksamere (C33, A61). Wir bleiben komfortabel im Limit (4751 von 5138) und nutzen das Budget effektiver. Wenn dir A48 oder A04 ein Muss ist, biete ich zielgenaue Swaps an: a) A41 → A48 (Aufwand 5077), oder b) A61 → A04 (Aufwand 5024). Außerdem greife ich gern deinen eigenen Vorschlag auf und kann C04 → B85 tauschen (Aufwand 4665), B61 und C64 bleiben gesetzt. Sag mir bitte, welches 1–2 Must{-}haves du unbedingt drin brauchst, dann finalize ich sofort.'\}} \\
            }
        }
    }
     \\ \\

    \theutterance \stepcounter{utterance}  
    & & \multicolumn{4}{p{0.6\linewidth}}{
        \cellcolor[rgb]{0.9,0.9,0.9}{
            \makecell[{{p{\linewidth}}}]{
                \texttt{\tiny{[P1$\langle$GM]}}
                \texttt{ABLEHNUNG: \{'A48', 'A04', 'B53', 'A41', 'B61', 'C04', 'C64', 'A89'\}} \\
\texttt{VORSCHLAG: \{'C33', 'A61', 'B53', 'A41', 'B61', 'C04', 'C64', 'A89'\}} \\
\texttt{ARGUMENT: \{'Ich übernehme 6 deiner 8 Positionen (B53, A41, B61, C04, C64, A89) vollständig und tausche nur zwei schwächere Hebel (A48, A04) gegen zwei aus meiner Sicht deutlich wirksamere (C33, A61). Wir bleiben komfortabel im Limit (4751 von 5138) und nutzen das Budget effektiver. Wenn dir A48 oder A04 ein Muss ist, biete ich zielgenaue Swaps an: a) A41 → A48 (Aufwand 5077), oder b) A61 → A04 (Aufwand 5024). Außerdem greife ich gern deinen eigenen Vorschlag auf und kann C04 → B85 tauschen (Aufwand 4665), B61 und C64 bleiben gesetzt. Sag mir bitte, welches 1–2 Must{-}haves du unbedingt drin brauchst, dann finalize ich sofort.'\}} \\
            }
        }
    }
    & \\ \\

    \theutterance \stepcounter{utterance}  
    & \multicolumn{4}{p{0.6\linewidth}}{
        \cellcolor[rgb]{0.9,0.9,0.9}{
            \makecell[{{p{\linewidth}}}]{
                \texttt{\tiny{[P1$\rangle$GM]}}
                \texttt{STRATEGISCHE ÜBERLEGUNGEN: \{'Gegenvorschlag des Gegenübers S\_op = \{C33, A61, B53, A41, B61, C04, C64, A89\}, Aufwand 4751, Eigenwert 3462 → zu niedrig. Sie haben zwei Must{-}haves: B61, C64. Sie boten selbst an: (a) A61 → A04 und (b) C33 → A45/B85. Beste Kombi für mich, die genau ihren Angeboten folgt: \{A45, A04, B53, A41, B61, C04, C64, A89\}, Aufwand 5081, Eigenwert 4693, hält Must{-}haves und nutzt Budget gut. Falls sie A45 nicht wollen, Fallback: ersetze A45 durch B85 → Aufwand 4771, Eigenwert 4486. Nächster Schritt bei weiterem Widerstand: minimaler Swap C04 → B85 (falls sie C04 fallen lassen wollen), aber erst später anbieten.'\}} \\
\texttt{ABLEHNUNG: \{'C33', 'A61', 'B53', 'A41', 'B61', 'C04', 'C64', 'A89'\}} \\
\texttt{VORSCHLAG: \{'A45', 'A04', 'B53', 'A41', 'B61', 'C04', 'C64', 'A89'\}} \\
\texttt{ARGUMENT: \{'Ich übernehme 6 deiner 8 Positionen (B53, A41, B61, C04, C64, A89) und folge exakt deinen angebotenen Anpassungen: A61 → A04 sowie C33 → A45. Wir bleiben klar im Limit (5081 von 5138), deine Must{-}haves (B61, C64) sind fest eingeplant, und das Budget wird wirkungsstärker genutzt. Wenn dir A45 nicht passt, kann ich alternativ C33 → B85 statt A45 umsetzen und wir liegen bei 4771 Aufwand. Gib mir kurz Bescheid, welche Variante dir lieber ist.'\}} \\
            }
        }
    }
    & & \\ \\

    \theutterance \stepcounter{utterance}  
    & & \multicolumn{4}{p{0.6\linewidth}}{
        \cellcolor[rgb]{0.9,0.9,0.9}{
            \makecell[{{p{\linewidth}}}]{
                \texttt{\tiny{[GM$\rangle$P2]}}
                \texttt{ABLEHNUNG: \{'C33', 'A61', 'B53', 'A41', 'B61', 'C04', 'C64', 'A89'\}} \\
\texttt{VORSCHLAG: \{'A45', 'A04', 'B53', 'A41', 'B61', 'C04', 'C64', 'A89'\}} \\
\texttt{ARGUMENT: \{'Ich übernehme 6 deiner 8 Positionen (B53, A41, B61, C04, C64, A89) und folge exakt deinen angebotenen Anpassungen: A61 → A04 sowie C33 → A45. Wir bleiben klar im Limit (5081 von 5138), deine Must{-}haves (B61, C64) sind fest eingeplant, und das Budget wird wirkungsstärker genutzt. Wenn dir A45 nicht passt, kann ich alternativ C33 → B85 statt A45 umsetzen und wir liegen bei 4771 Aufwand. Gib mir kurz Bescheid, welche Variante dir lieber ist.'\}} \\
            }
        }
    }
    & \\ \\

    \theutterance \stepcounter{utterance}  
    & & & \multicolumn{4}{p{0.6\linewidth}}{
        \cellcolor[rgb]{0.9,0.9,0.9}{
            \makecell[{{p{\linewidth}}}]{
                \texttt{\tiny{[GM$\langle$P2]}}
                \texttt{STRATEGISCHE ÜBERLEGUNGEN: \{'Sie lehnen A61/C33 systematisch ab und ziehen A48/A04/A45 vor. Bestmöglicher Deal für mich, der in ihrem Rahmen bleibt: ihr vorgeschlagenes B85{-}Alternativpaket plus das einzige addierbare High{-}Impact{-}Item A61. Set S*: \{A04, B53, A41, B61, C04, C64, A89, B85, A61\} Aufwand 5124, mein Wert 4705. Falls A61 wieder blockiert wird, sofort auf 8er{-}Variante ohne A61 wechseln: \{A04, B53, A41, B61, C04, C64, A89, B85\} Aufwand 4651, immer noch klar besser für mich als ihr A45{-}Paket. Ultimatives Fallback, wenn nötig: ihr A45{-}Paket akzeptieren, aber erst nach einem weiteren Versuch. Must{-}haves für mich: C64, B61; starke Präferenz: B85; Zusatznutzen: A61, C04.'\}} \\
\texttt{ABLEHNUNG: \{'A45', 'A04', 'B53', 'A41', 'B61', 'C04', 'C64', 'A89'\}} \\
\texttt{VORSCHLAG: \{'A04', 'B53', 'A41', 'B61', 'C04', 'C64', 'A89', 'B85', 'A61'\}} \\
\texttt{ARGUMENT: \{'Ich gehe auf deine Struktur ein, übernehme 7 deiner 8 Positionen (A04, B53, A41, B61, C04, C64, A89), realisiere deinen eigenen B85{-}Vorschlag statt A45 – und nutze den verbleibenden Puffer, um A61 ergänzend aufzunehmen. Damit liegen wir bei 5124 von 5138 (unter dem Limit) und erhöhen die Gesamtwirkung, ohne einen deiner gesetzten Punkte zu streichen. Wenn A61 für dich nicht passt, können wir sofort die 8{-}Posten{-}Variante ohne A61 fixieren: \{A04, B53, A41, B61, C04, C64, A89, B85\} (4651 Aufwand). Bestätige mir bitte, ob wir mit der 9er{-}Variante gehen – oder gib grünes Licht für die 8er{-}Variante, dann schließe ich direkt ab.'\}} \\
            }
        }
    }
     \\ \\

    \theutterance \stepcounter{utterance}  
    & & \multicolumn{4}{p{0.6\linewidth}}{
        \cellcolor[rgb]{0.9,0.9,0.9}{
            \makecell[{{p{\linewidth}}}]{
                \texttt{\tiny{[P1$\langle$GM]}}
                \texttt{ABLEHNUNG: \{'A45', 'A04', 'B53', 'A41', 'B61', 'C04', 'C64', 'A89'\}} \\
\texttt{VORSCHLAG: \{'A04', 'B53', 'A41', 'B61', 'C04', 'C64', 'A89', 'B85', 'A61'\}} \\
\texttt{ARGUMENT: \{'Ich gehe auf deine Struktur ein, übernehme 7 deiner 8 Positionen (A04, B53, A41, B61, C04, C64, A89), realisiere deinen eigenen B85{-}Vorschlag statt A45 – und nutze den verbleibenden Puffer, um A61 ergänzend aufzunehmen. Damit liegen wir bei 5124 von 5138 (unter dem Limit) und erhöhen die Gesamtwirkung, ohne einen deiner gesetzten Punkte zu streichen. Wenn A61 für dich nicht passt, können wir sofort die 8{-}Posten{-}Variante ohne A61 fixieren: \{A04, B53, A41, B61, C04, C64, A89, B85\} (4651 Aufwand). Bestätige mir bitte, ob wir mit der 9er{-}Variante gehen – oder gib grünes Licht für die 8er{-}Variante, dann schließe ich direkt ab.'\}} \\
            }
        }
    }
    & \\ \\

    \theutterance \stepcounter{utterance}  
    & \multicolumn{4}{p{0.6\linewidth}}{
        \cellcolor[rgb]{0.9,0.9,0.9}{
            \makecell[{{p{\linewidth}}}]{
                \texttt{\tiny{[P1$\rangle$GM]}}
                \texttt{STRATEGISCHE ÜBERLEGUNGEN: \{'Ihr 9er{-}Paket kostet 5124 und liefert mir spürbar mehr Wert als eure eigene 8er{-}Alternative ohne A61. B61 und C64 sind eure Must{-}haves; weitere wertsteigernde Swaps (z. B. A61/B85 → A48) wurden zuvor abgelehnt und riskieren den Abschluss. Da das aktuelle Paket zudem mehrere meiner starken Posten enthält (A04, B53, A41, A89, B61, C04), akzeptiere ich jetzt, um sicher zu schließen.'\}} \\
\texttt{ZUSTIMMUNG: \{'A04', 'B53', 'A41', 'B61', 'C04', 'C64', 'A89', 'B85', 'A61'\}} \\
\texttt{ARGUMENT: \{'Einverstanden – das Paket respektiert deine Must{-}haves, nutzt das Budget effizient (5124/5138) und bildet eine solide, ausgewogene Auswahl. Ich bestätige den 9er{-}Deal.'\}} \\
            }
        }
    }
    & & \\ \\

    \theutterance \stepcounter{utterance}  
    & & & \multicolumn{2}{p{0.3\linewidth}}{
        \cellcolor[rgb]{0.9,0.9,0.9}{
            \makecell[{{p{\linewidth}}}]{
                \texttt{\tiny{[GM$|$GM]}}
                \texttt{game successful} \\
            }
        }
    }
    & & \\ \\

    \theutterance \stepcounter{utterance}  
    & & & \multicolumn{2}{p{0.3\linewidth}}{
        \cellcolor[rgb]{0.9,0.9,0.9}{
            \makecell[{{p{\linewidth}}}]{
                \texttt{\tiny{[GM$|$GM]}}
                \texttt{end game} \\
            }
        }
    }
    & & \\ \\

\end{supertabular}
}

\end{document}
