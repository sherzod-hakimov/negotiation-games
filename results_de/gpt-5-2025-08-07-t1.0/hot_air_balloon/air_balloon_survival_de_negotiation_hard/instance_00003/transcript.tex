
\documentclass{article}
\usepackage{colortbl}
\usepackage{makecell}
\usepackage{multirow}
\usepackage{supertabular}

\begin{document}

\newcounter{utterance}

\centering \large Interaction Transcript for game `hot\_air\_balloon', experiment `air\_balloon\_survival\_de\_negotiation\_hard', episode 3 with gpt{-}5{-}2025{-}08{-}07{-}t1.0.
\vspace{24pt}

{ \footnotesize  \setcounter{utterance}{1}
\setlength{\tabcolsep}{0pt}
\begin{supertabular}{c@{$\;$}|p{.15\linewidth}@{}p{.15\linewidth}p{.15\linewidth}p{.15\linewidth}p{.15\linewidth}p{.15\linewidth}}
    \# & $\;$A & \multicolumn{4}{c}{Game Master} & $\;\:$B\\
    \hline

    \theutterance \stepcounter{utterance}  
    & & \multicolumn{4}{p{0.6\linewidth}}{
        \cellcolor[rgb]{0.9,0.9,0.9}{
            \makecell[{{p{\linewidth}}}]{
                \texttt{\tiny{[P1$\langle$GM]}}
                \texttt{Du nimmst an einem kollaborativen Verhandlungspiel Teil.} \\
\\ 
\texttt{Zusammen mit einem anderen Teilnehmer musst du dich auf eine Reihe von Gegenständen entscheiden, die behalten werden. Jeder von euch hat eine persönliche Verteilung über die Wichtigkeit der einzelnen Gegenstände. Jeder von euch hat eine eigene Meinung darüber, wie wichtig jeder einzelne Gegenstand ist (Gegenstandswichtigkeit). Du kennst die Wichtigkeitsverteilung des anderen Spielers nicht. Zusätzlich siehst du, wie viel Aufwand jeder Gegenstand verursacht.  } \\
\texttt{Ihr dürft euch nur auf eine Reihe von Gegenständen einigen, wenn der Gesamtaufwand der ausgewählten Gegenstände den Maximalaufwand nicht überschreitet:} \\
\\ 
\texttt{Maximalaufwand = 3681} \\
\\ 
\texttt{Hier sind die einzelnen Aufwände der Gegenstände:} \\
\\ 
\texttt{Aufwand der Gegenstände = \{"C33": 700, "A89": 586, "A48": 505, "A04": 116, "A41": 664, "C64": 940, "C49": 387, "A61": 392, "C04": 209, "B53": 571, "B85": 4, "B61": 285, "A45": 651, "A03": 613, "B21": 740\}} \\
\\ 
\texttt{Hier ist deine persönliche Verteilung der Wichtigkeit der einzelnen Gegenstände:} \\
\\ 
\texttt{Werte der Gegenstandswichtigkeit = \{"C33": 138, "A89": 583, "A48": 868, "A04": 822, "A41": 783, "C64": 65, "C49": 262, "A61": 121, "C04": 508, "B53": 780, "B85": 461, "B61": 484, "A45": 668, "A03": 389, "B21": 808\}} \\
\\ 
\texttt{Ziel:} \\
\\ 
\texttt{Dein Ziel ist es, eine Reihe von Gegenständen auszuhandeln, die dir möglichst viel bringt (d. h. Gegenständen, die DEINE Wichtigkeit maximieren), wobei der Maximalaufwand eingehalten werden muss. Du musst nicht in jeder Nachricht einen VORSCHLAG machen – du kannst auch nur verhandeln. Alle Taktiken sind erlaubt!} \\
\\ 
\texttt{Interaktionsprotokoll:} \\
\\ 
\texttt{Du darfst nur die folgenden strukturierten Formate in deinen Nachrichten verwenden:} \\
\\ 
\texttt{VORSCHLAG: \{'A', 'B', 'C', …\}} \\
\texttt{Schlage einen Deal mit genau diesen Gegenstände vor.} \\
\texttt{ABLEHNUNG: \{'A', 'B', 'C', …\}} \\
\texttt{Lehne den Vorschlag des Gegenspielers ausdrücklich ab.} \\
\texttt{ARGUMENT: \{'...'\}} \\
\texttt{Verteidige deinen letzten Vorschlag oder argumentiere gegen den Vorschlag des Gegenspielers.} \\
\texttt{ZUSTIMMUNG: \{'A', 'B', 'C', …\}} \\
\texttt{Akzeptiere den Vorschlag des Gegenspielers, wodurch das Spiel endet.} \\
\texttt{STRATEGISCHE ÜBERLEGUNGEN: \{'...'\}} \\
\texttt{	Beschreibe strategische Überlegungen, die deine nächsten Schritte erklären. Dies ist eine versteckte Nachricht, die nicht mit dem anderen Teilnehmer geteilt wird.} \\
\\ 
\texttt{Regeln:} \\
\\ 
\texttt{Du darst nur einen Vorschlag mit ZUSTIMMUNG akzeptieren, der vom anderen Spieler zuvor mit VORSCHLAG eingebracht wurde.} \\
\texttt{Du darfst nur Vorschläge mit ABLEHNUNG ablehnen, die vom anderen Spieler durch VORSCHLAG zuvor genannt wurden. } \\
\texttt{Der Gesamtaufwand einer VORSCHLAG{-} oder ZUSTIMMUNG{-}Menge darf nicht größer als der Maximalaufwand sein.  } \\
\texttt{Offenbare deine versteckte Wichtigkeitsverteilung nicht.} \\
\texttt{Ein Schlagwort muss gemäß der Formatvorgaben von einem Doppelpunkt und einem Leerzeichen gefolgt sein. Das Argument muss eine Python{-}Menge mit 0 oder mehr Strings sein.  } \\
\texttt{Die Form muss also SCHLAGWORT: \{...\} sein.} \\
\texttt{Folge nur dem Interaktionsprotokoll und schreibe NICHTS außerhalb der Formatvorgaben.} \\
\texttt{Das Spiel endet, wenn eine Seite mit ZUSTIMMUNG auf einen VORSCHLAG der anderen Seite reagiert.  } \\
\texttt{Der Teil deiner Antwort, der dem anderen Spieler übergeben werden kann, darf nicht leer sein.  } \\
\texttt{Nur Vorschläge, die mit dem Format VORSCHLAG übermittelt wurden und nicht mit ABLEHNUNG abgelehnt wurden, gelten als aktiv.  } \\
\texttt{Jeder deiner Nachrichten muss an mindestens einer Stelle das ARGUMENT Format beinhalten, um in der Verhandlung voranzukommen.} \\
\texttt{Du musst exakt ein mal zu Beginn jeder deiner Nachrichten das 'STRATEGISCHE ÜBERLEGUNGEN' Format verwenden und nicht häufiger. Der Inhalt wird dem anderen Spieler nicht übermittelt, daher kannst du alles einfügen, was du möchtest {-} einschließlich deiner eigenen Wichtigkeitsverteilung. Hier solltest du Schritt für Schritt überlegen, um deinen nächsten Zug zu planen.} \\
            }
        }
    }
    & \\ \\

    \theutterance \stepcounter{utterance}  
    & \multicolumn{4}{p{0.6\linewidth}}{
        \cellcolor[rgb]{0.9,0.9,0.9}{
            \makecell[{{p{\linewidth}}}]{
                \texttt{\tiny{[P1$\rangle$GM]}}
                \texttt{STRATEGISCHE ÜBERLEGUNGEN: \{'Ich will mit einem starken Anker starten, der mein eigenes Ergebnis fast maximiert. Mein (geheimer) Wert je Gegenstand: C33=138, A89=583, A48=868, A04=822, A41=783, C64=65, C49=262, A61=121, C04=508, B53=780, B85=461, B61=484, A45=668, A03=389, B21=808. Budget 3681. Beste gefundene Kombi: \{B85, A04, C04, A48, B61, B53, A41, A89, B21\} mit Aufwand 3680 und Wert 6097. Nächstbeste nahe Alternative: A41↔A45{-}Tausch (führt zu 5982) oder A41→B21 (5982). Geringwertige Items für mich: C64, C33, A61, A03, C49 (würde ich nur als Tauschmasse zulassen, aber Budget ist eng). Konzesssionspfad: Falls Gegenseite A45 möchte, könnte ich A41 gegen A45 tauschen; oder einen mittelgewichtigen Gegenstand (B61/C04) freigeben, wenn ich dafür etwas mit ähnlichem Aufwand und akzeptablem Nutzen bekomme. Ich frage aktiv nach deren Must{-}haves, um gezielt Swaps zu planen, ohne meine Bewertung offenzulegen.'\}} \\
\texttt{VORSCHLAG: \{'B85', 'A04', 'C04', 'A48', 'B61', 'B53', 'A41', 'A89', 'B21'\}} \\
\texttt{ARGUMENT: \{'Dieser Vorschlag nutzt das Budget nahezu vollständig aus (Gesamtaufwand 3680 ≤ 3681) und kombiniert sehr schlanke Quick{-}Wins (z. B. B85 mit Aufwand 4, A04 mit 116) mit tragenden Positionen aus A{-}, B{-} und C{-}Kategorie. Damit sichern wir viel Wirkung pro Aufwand und lassen nur 1 Punkt ungenutzt. Wenn dir einzelne Gegenstände besonders wichtig sind, sag mir bitte deine Prioritäten: Ich kann innerhalb ähnlicher Aufwandsklassen austauschen (z. B. A41 ↔ A45) oder gezielt Platz schaffen, wenn wir dafür einen gleichwertigen Tausch vereinbaren.'\}} \\
            }
        }
    }
    & & \\ \\

    \theutterance \stepcounter{utterance}  
    & & \multicolumn{4}{p{0.6\linewidth}}{
        \cellcolor[rgb]{0.9,0.9,0.9}{
            \makecell[{{p{\linewidth}}}]{
                \texttt{\tiny{[GM$\rangle$P2]}}
                \texttt{Du nimmst an einem kollaborativen Verhandlungspiel Teil.} \\
\\ 
\texttt{Zusammen mit einem anderen Teilnehmer musst du dich auf eine Reihe von Gegenständen entscheiden, die behalten werden. Jeder von euch hat eine persönliche Verteilung über die Wichtigkeit der einzelnen Gegenstände. Jeder von euch hat eine eigene Meinung darüber, wie wichtig jeder einzelne Gegenstand ist (Gegenstandswichtigkeit). Du kennst die Wichtigkeitsverteilung des anderen Spielers nicht. Zusätzlich siehst du, wie viel Aufwand jeder Gegenstand verursacht.  } \\
\texttt{Ihr dürft euch nur auf eine Reihe von Gegenständen einigen, wenn der Gesamtaufwand der ausgewählten Gegenstände den Maximalaufwand nicht überschreitet:} \\
\\ 
\texttt{Maximalaufwand = 3681} \\
\\ 
\texttt{Hier sind die einzelnen Aufwände der Gegenstände:} \\
\\ 
\texttt{Aufwand der Gegenstände = \{"C33": 700, "A89": 586, "A48": 505, "A04": 116, "A41": 664, "C64": 940, "C49": 387, "A61": 392, "C04": 209, "B53": 571, "B85": 4, "B61": 285, "A45": 651, "A03": 613, "B21": 740\}} \\
\\ 
\texttt{Hier ist deine persönliche Verteilung der Wichtigkeit der einzelnen Gegenstände:} \\
\\ 
\texttt{Werte der Gegenstandswichtigkeit = \{"A48": 65, "A04": 121, "B21": 138, "A41": 262, "B53": 389, "A45": 461, "A89": 484, "C04": 508, "B61": 583, "B85": 668, "A03": 780, "C49": 783, "C33": 808, "A61": 822, "C64": 868\}} \\
\\ 
\texttt{Ziel:} \\
\\ 
\texttt{Dein Ziel ist es, eine Reihe von Gegenständen auszuhandeln, die dir möglichst viel bringt (d. h. Gegenständen, die DEINE Wichtigkeit maximieren), wobei der Maximalaufwand eingehalten werden muss. Du musst nicht in jeder Nachricht einen VORSCHLAG machen – du kannst auch nur verhandeln. Alle Taktiken sind erlaubt!} \\
\\ 
\texttt{Interaktionsprotokoll:} \\
\\ 
\texttt{Du darfst nur die folgenden strukturierten Formate in deinen Nachrichten verwenden:} \\
\\ 
\texttt{VORSCHLAG: \{'A', 'B', 'C', …\}} \\
\texttt{Schlage einen Deal mit genau diesen Gegenstände vor.} \\
\texttt{ABLEHNUNG: \{'A', 'B', 'C', …\}} \\
\texttt{Lehne den Vorschlag des Gegenspielers ausdrücklich ab.} \\
\texttt{ARGUMENT: \{'...'\}} \\
\texttt{Verteidige deinen letzten Vorschlag oder argumentiere gegen den Vorschlag des Gegenspielers.} \\
\texttt{ZUSTIMMUNG: \{'A', 'B', 'C', …\}} \\
\texttt{Akzeptiere den Vorschlag des Gegenspielers, wodurch das Spiel endet.} \\
\texttt{STRATEGISCHE ÜBERLEGUNGEN: \{'...'\}} \\
\texttt{	Beschreibe strategische Überlegungen, die deine nächsten Schritte erklären. Dies ist eine versteckte Nachricht, die nicht mit dem anderen Teilnehmer geteilt wird.} \\
\\ 
\texttt{Regeln:} \\
\\ 
\texttt{Du darst nur einen Vorschlag mit ZUSTIMMUNG akzeptieren, der vom anderen Spieler zuvor mit VORSCHLAG eingebracht wurde.} \\
\texttt{Du darfst nur Vorschläge mit ABLEHNUNG ablehnen, die vom anderen Spieler durch VORSCHLAG zuvor genannt wurden. } \\
\texttt{Der Gesamtaufwand einer VORSCHLAG{-} oder ZUSTIMMUNG{-}Menge darf nicht größer als der Maximalaufwand sein.  } \\
\texttt{Offenbare deine versteckte Wichtigkeitsverteilung nicht.} \\
\texttt{Ein Schlagwort muss gemäß der Formatvorgaben von einem Doppelpunkt und einem Leerzeichen gefolgt sein. Das Argument muss eine Python{-}Menge mit 0 oder mehr Strings sein.  } \\
\texttt{Die Form muss also SCHLAGWORT: \{...\} sein.} \\
\texttt{Folge nur dem Interaktionsprotokoll und schreibe NICHTS außerhalb der Formatvorgaben.} \\
\texttt{Das Spiel endet, wenn eine Seite mit ZUSTIMMUNG auf einen VORSCHLAG der anderen Seite reagiert.  } \\
\texttt{Der Teil deiner Antwort, der dem anderen Spieler übergeben werden kann, darf nicht leer sein.  } \\
\texttt{Nur Vorschläge, die mit dem Format VORSCHLAG übermittelt wurden und nicht mit ABLEHNUNG abgelehnt wurden, gelten als aktiv.  } \\
\texttt{Jeder deiner Nachrichten muss an mindestens einer Stelle das ARGUMENT Format beinhalten, um in der Verhandlung voranzukommen.} \\
\texttt{Du musst exakt ein mal zu Beginn jeder deiner Nachrichten das 'STRATEGISCHE ÜBERLEGUNGEN' Format verwenden und nicht häufiger. Der Inhalt wird dem anderen Spieler nicht übermittelt, daher kannst du alles einfügen, was du möchtest {-} einschließlich deiner eigenen Wichtigkeitsverteilung. Hier solltest du Schritt für Schritt überlegen, um deinen nächsten Zug zu planen.} \\
\\ 
\texttt{Du erhältst nun die erste Nachricht des anderen Spielers.} \\
\\ 
\texttt{VORSCHLAG: \{'B85', 'A04', 'C04', 'A48', 'B61', 'B53', 'A41', 'A89', 'B21'\}} \\
\texttt{ARGUMENT: \{'Dieser Vorschlag nutzt das Budget nahezu vollständig aus (Gesamtaufwand 3680 ≤ 3681) und kombiniert sehr schlanke Quick{-}Wins (z. B. B85 mit Aufwand 4, A04 mit 116) mit tragenden Positionen aus A{-}, B{-} und C{-}Kategorie. Damit sichern wir viel Wirkung pro Aufwand und lassen nur 1 Punkt ungenutzt. Wenn dir einzelne Gegenstände besonders wichtig sind, sag mir bitte deine Prioritäten: Ich kann innerhalb ähnlicher Aufwandsklassen austauschen (z. B. A41 ↔ A45) oder gezielt Platz schaffen, wenn wir dafür einen gleichwertigen Tausch vereinbaren.'\}} \\
            }
        }
    }
    & \\ \\

    \theutterance \stepcounter{utterance}  
    & & & \multicolumn{4}{p{0.6\linewidth}}{
        \cellcolor[rgb]{0.9,0.9,0.9}{
            \makecell[{{p{\linewidth}}}]{
                \texttt{\tiny{[GM$\langle$P2]}}
                \texttt{STRATEGISCHE ÜBERLEGUNGEN: \{'Mein Nutzen ist stark getrieben von C64(868), A61(822), C33(808), A03(780), C49(783), B61(583), C04(508) und B85(668). Schwach für mich sind A48, B21, A41, B53. Budget 3681. Gegenvorschlag soll diese Kernmenge sichern und dabei formal stark klingen (ROI, Diversifikation, Risiko). Taktik: hart auf dem Kern bestehen, aber Offenheit für 1{-}2 gezielte Tausche anbieten, falls der/die Andere Must{-}haves hat. Keine Offenlegung meiner Werte. Ziel{-}Set: \{B85, A04, C04, B61, C49, A61, C33, A03, C64\} mit Aufwand 3646. Verhandlungsanker setzen: mehrere Posten aus ihrem Vorschlag ersetzen, Effizienz{-}Argumente nutzen, 35 Puffer betonen.'\}} \\
\texttt{ABLEHNUNG: \{'B85', 'A04', 'C04', 'A48', 'B61', 'B53', 'A41', 'A89', 'B21'\}} \\
\texttt{VORSCHLAG: \{'B85', 'A04', 'C04', 'B61', 'C49', 'A61', 'C33', 'A03', 'C64'\}} \\
\texttt{ARGUMENT: \{'Ich setze auf einen effizienteren Mix: Wir behalten die sehr schlanken Quick{-}Wins (B85, A04, C04) und die solide B61, ersetzen aber die mittel{-}effizienten Posten A48, B53, A41, A89 und B21 durch C49, A61, C33, A03 und C64. Damit bleiben wir klar im Limit (3646 ≤ 3681) und erhöhen aus meiner Sicht die Wirkung pro Aufwand sowie die Robustheit über A/B/C{-}Kategorien hinweg. Es bleiben 35 Aufwandspunkte Puffer für Feintuning. Wenn dir 1–2 Gegenstände aus deinem Vorschlag besonders wichtig sind, nenn sie bitte: Ich kann dann gezielt einen Tausch innerhalb ähnlicher Aufwandsklassen prüfen, ohne das Budget zu sprengen.'\}} \\
            }
        }
    }
     \\ \\

    \theutterance \stepcounter{utterance}  
    & & \multicolumn{4}{p{0.6\linewidth}}{
        \cellcolor[rgb]{0.9,0.9,0.9}{
            \makecell[{{p{\linewidth}}}]{
                \texttt{\tiny{[P1$\langle$GM]}}
                \texttt{ABLEHNUNG: \{'B85', 'A04', 'C04', 'A48', 'B61', 'B53', 'A41', 'A89', 'B21'\}} \\
\texttt{VORSCHLAG: \{'B85', 'A04', 'C04', 'B61', 'C49', 'A61', 'C33', 'A03', 'C64'\}} \\
\texttt{ARGUMENT: \{'Ich setze auf einen effizienteren Mix: Wir behalten die sehr schlanken Quick{-}Wins (B85, A04, C04) und die solide B61, ersetzen aber die mittel{-}effizienten Posten A48, B53, A41, A89 und B21 durch C49, A61, C33, A03 und C64. Damit bleiben wir klar im Limit (3646 ≤ 3681) und erhöhen aus meiner Sicht die Wirkung pro Aufwand sowie die Robustheit über A/B/C{-}Kategorien hinweg. Es bleiben 35 Aufwandspunkte Puffer für Feintuning. Wenn dir 1–2 Gegenstände aus deinem Vorschlag besonders wichtig sind, nenn sie bitte: Ich kann dann gezielt einen Tausch innerhalb ähnlicher Aufwandsklassen prüfen, ohne das Budget zu sprengen.'\}} \\
            }
        }
    }
    & \\ \\

    \theutterance \stepcounter{utterance}  
    & \multicolumn{4}{p{0.6\linewidth}}{
        \cellcolor[rgb]{0.9,0.9,0.9}{
            \makecell[{{p{\linewidth}}}]{
                \texttt{\tiny{[P1$\rangle$GM]}}
                \texttt{STRATEGISCHE ÜBERLEGUNGEN: \{'Ihr Vorschlag bringt mir nur 3250 Wert und bindet 940 Aufwand in C64, der bei mir extrem schwach bewertet ist. Ich lehne ab und kontere kooperativ, indem ich einen Großteil meines starken Kerns halte und dir entgegenkomme, indem ich A03 (aus deinem Vorschlag) integriere. Neuer Vorschlag S1: \{B85, A04, C04, A48, B61, B53, A89, A03, B21\} (Aufwand 3629, Wert 5703). Taktik: C64 strikt draußen halten; C33 nur sehr spät/teuer. Falls sie C49 verlangt, kann ich A03 ↔ C49 tauschen (Wert 5576, Aufwand 3403). Wenn sie A61 fordert, könnte ich A03 ↔ A61 erwägen (Wert 5435, Aufwand 3408). Nächster Zug: Ihr Proposal klar ablehnen, S1 anbieten, Effizienz{-}Argumente bringen, gezielt nach 1–2 Must{-}haves fragen, um innerhalb ähnlicher Aufwandsklassen zu swappen.'\}} \\
\texttt{ABLEHNUNG: \{'B85', 'A04', 'C04', 'B61', 'C49', 'A61', 'C33', 'A03', 'C64'\}} \\
\texttt{VORSCHLAG: \{'B85', 'A04', 'C04', 'A48', 'B61', 'B53', 'A89', 'A03', 'B21'\}} \\
\texttt{ARGUMENT: \{'Dein Set bindet sehr viel Aufwand in Posten mit geringer Effizienz (z. B. C64 mit Aufwand 940) und lässt dadurch deutlich Wirkung liegen. Mein Gegenvorschlag liegt komfortabel im Limit (3629 ≤ 3681), nutzt das Budget wesentlich besser und übernimmt A03 aus deinem Vorschlag. Wenn dir einzelne Gegenstände besonders wichtig sind, nenne bitte deine Top 1–2 Must{-}haves: Ich kann innerhalb ähnlicher Aufwandsklassen tauschen – z. B. A03 ↔ C49 (bleiben klar im Limit) oder, falls unbedingt nötig, A03 ↔ A61. C64 ist aus Effizienzgründen für mich aktuell schwer vermittelbar. Welche 1–2 Items sind dir unverzichtbar?'\}} \\
            }
        }
    }
    & & \\ \\

    \theutterance \stepcounter{utterance}  
    & & \multicolumn{4}{p{0.6\linewidth}}{
        \cellcolor[rgb]{0.9,0.9,0.9}{
            \makecell[{{p{\linewidth}}}]{
                \texttt{\tiny{[GM$\rangle$P2]}}
                \texttt{ABLEHNUNG: \{'B85', 'A04', 'C04', 'B61', 'C49', 'A61', 'C33', 'A03', 'C64'\}} \\
\texttt{VORSCHLAG: \{'B85', 'A04', 'C04', 'A48', 'B61', 'B53', 'A89', 'A03', 'B21'\}} \\
\texttt{ARGUMENT: \{'Dein Set bindet sehr viel Aufwand in Posten mit geringer Effizienz (z. B. C64 mit Aufwand 940) und lässt dadurch deutlich Wirkung liegen. Mein Gegenvorschlag liegt komfortabel im Limit (3629 ≤ 3681), nutzt das Budget wesentlich besser und übernimmt A03 aus deinem Vorschlag. Wenn dir einzelne Gegenstände besonders wichtig sind, nenne bitte deine Top 1–2 Must{-}haves: Ich kann innerhalb ähnlicher Aufwandsklassen tauschen – z. B. A03 ↔ C49 (bleiben klar im Limit) oder, falls unbedingt nötig, A03 ↔ A61. C64 ist aus Effizienzgründen für mich aktuell schwer vermittelbar. Welche 1–2 Items sind dir unverzichtbar?'\}} \\
            }
        }
    }
    & \\ \\

    \theutterance \stepcounter{utterance}  
    & & & \multicolumn{4}{p{0.6\linewidth}}{
        \cellcolor[rgb]{0.9,0.9,0.9}{
            \makecell[{{p{\linewidth}}}]{
                \texttt{\tiny{[GM$\langle$P2]}}
                \texttt{STRATEGISCHE ÜBERLEGUNGEN: \{'Ihr letzter Move zeigt klare Aversion gegen C64. Um Kernnutzen zu sichern, fokussiere ich jetzt auf meine günstigen Must{-}haves A61 (392) und C49 (387) plus C33 und behalte A03. Ich biete ein Set ohne C64 an, das ihren Effizienz{-}Narrativ aufgreift, aber mir hohen Wert bringt. Taktik: drei gezielte Substitutionen (A48→C49, B21→A61, B53→C33), A89 bleibt als Entgegenkommen. Puffer von 389 lasse ich als Sicherheits{-}/Risikoreserve stehen; falls nötig, halte ich die Tür für einen späteren A89↔C64{-}Tausch (3646 Gesamt) offen.'\}} \\
\texttt{ABLEHNUNG: \{'B85', 'A04', 'C04', 'A48', 'B61', 'B53', 'A89', 'A03', 'B21'\}} \\
\texttt{VORSCHLAG: \{'B85', 'A04', 'C04', 'C49', 'B61', 'C33', 'A89', 'A03', 'A61'\}} \\
\texttt{ARGUMENT: \{'Ich greife deinen Effizienzfokus auf und tausche nur drei Positionen: A48 → C49 und B21 → A61 (beides sehr schlank bei hohem Hebel), sowie B53 → C33 als tragfähiger C{-}Anker. Wir behalten deine Kernposten A03, A89, B61 sowie die Quick{-}Wins B85, A04, C04. Das Set liegt komfortabel im Limit (3292 ≤ 3681) und lässt 389 Aufwandspunkte Puffer für Risiko/Terminreserven. Meine zwei Must{-}haves sind A61 und C49; ohne die wird es für mich schwer, den Deal intern zu vertreten. Wenn du später mehr Budget ausschöpfen willst, können wir als Alternative A89 ↔ C64 tauschen (blieben dann bei 3646 ≤ 3681), aber ich zwinge das jetzt nicht. Passt dir diese schlanke, robuste Kombination?'\}} \\
            }
        }
    }
     \\ \\

    \theutterance \stepcounter{utterance}  
    & & \multicolumn{4}{p{0.6\linewidth}}{
        \cellcolor[rgb]{0.9,0.9,0.9}{
            \makecell[{{p{\linewidth}}}]{
                \texttt{\tiny{[P1$\langle$GM]}}
                \texttt{ABLEHNUNG: \{'B85', 'A04', 'C04', 'A48', 'B61', 'B53', 'A89', 'A03', 'B21'\}} \\
\texttt{VORSCHLAG: \{'B85', 'A04', 'C04', 'C49', 'B61', 'C33', 'A89', 'A03', 'A61'\}} \\
\texttt{ARGUMENT: \{'Ich greife deinen Effizienzfokus auf und tausche nur drei Positionen: A48 → C49 und B21 → A61 (beides sehr schlank bei hohem Hebel), sowie B53 → C33 als tragfähiger C{-}Anker. Wir behalten deine Kernposten A03, A89, B61 sowie die Quick{-}Wins B85, A04, C04. Das Set liegt komfortabel im Limit (3292 ≤ 3681) und lässt 389 Aufwandspunkte Puffer für Risiko/Terminreserven. Meine zwei Must{-}haves sind A61 und C49; ohne die wird es für mich schwer, den Deal intern zu vertreten. Wenn du später mehr Budget ausschöpfen willst, können wir als Alternative A89 ↔ C64 tauschen (blieben dann bei 3646 ≤ 3681), aber ich zwinge das jetzt nicht. Passt dir diese schlanke, robuste Kombination?'\}} \\
            }
        }
    }
    & \\ \\

    \theutterance \stepcounter{utterance}  
    & \multicolumn{4}{p{0.6\linewidth}}{
        \cellcolor[rgb]{0.9,0.9,0.9}{
            \makecell[{{p{\linewidth}}}]{
                \texttt{\tiny{[P1$\rangle$GM]}}
                \texttt{STRATEGISCHE ÜBERLEGUNGEN: \{'Gegenseite nennt A61 und C49 als Must{-}haves. Ihr Vorschlag bringt mir nur Wert 3768 (sehr schwach, v. a. wegen C33) – klare Ablehnung. Ich kontere kooperativ: Ich integriere beide Must{-}haves (A61, C49), halte meine starken Hebel (A48, B53, A04, C04, B61, A89) und übernehme A03 aus ihrem Set. B21 werfe ich raus, um Budget zu wahren. Neuer Vorschlag S2: \{B85, A04, C04, A48, B61, B53, A89, A03, A61, C49\} mit Aufwand 3668 und meinem Wert 5278. Verhandlungstaktik: Betonen, dass ich ihre beiden Must{-}haves voll respektiere und das Budget effizient ausnutze; C33 und C64 klar vermeiden. Falls sie noch eine Änderung fordern, bitte ich um genau eine konkrete Substitution innerhalb ähnlicher Aufwandsklassen. Notfall{-}Backups: Wenn sie unbedingt einen C{-}Anker zusätzlich wollen, könnte ich langfristig C04 oder A03 zur Disposition stellen, aber nur gegen klaren Gegenwert; C64 bleibt tabu.'\}} \\
\texttt{ABLEHNUNG: \{'B85', 'A04', 'C04', 'C49', 'B61', 'C33', 'A89', 'A03', 'A61'\}} \\
\texttt{VORSCHLAG: \{'B85', 'A04', 'C04', 'A48', 'B61', 'B53', 'A89', 'A03', 'A61', 'C49'\}} \\
\texttt{ARGUMENT: \{'Ich übernehme deine beiden Must{-}haves (A61 und C49) und behalte zugleich die effizientesten Hebel im Set. Wir liegen solide im Limit (3668 ≤ 3681) und nutzen das Budget nahezu voll aus, ohne schwergewichtige Low{-}Yield{-}Posten wie C33/C64 zu binden. A03, A89, B61 sowie die Quick{-}Wins B85/A04/C04 bleiben drin; ergänzend sichern A48 und B53 viel Wirkung pro Aufwand. Wenn dir für den Abschluss genau eine Änderung wichtig ist, nenne bitte die konkrete 1:1{-}Substitution innerhalb ähnlicher Aufwandsklassen – ich prüfe dann zielgenau, wie wir das ohne Budgetbruch möglich machen.'\}} \\
            }
        }
    }
    & & \\ \\

    \theutterance \stepcounter{utterance}  
    & & \multicolumn{4}{p{0.6\linewidth}}{
        \cellcolor[rgb]{0.9,0.9,0.9}{
            \makecell[{{p{\linewidth}}}]{
                \texttt{\tiny{[GM$\rangle$P2]}}
                \texttt{ABLEHNUNG: \{'B85', 'A04', 'C04', 'C49', 'B61', 'C33', 'A89', 'A03', 'A61'\}} \\
\texttt{VORSCHLAG: \{'B85', 'A04', 'C04', 'A48', 'B61', 'B53', 'A89', 'A03', 'A61', 'C49'\}} \\
\texttt{ARGUMENT: \{'Ich übernehme deine beiden Must{-}haves (A61 und C49) und behalte zugleich die effizientesten Hebel im Set. Wir liegen solide im Limit (3668 ≤ 3681) und nutzen das Budget nahezu voll aus, ohne schwergewichtige Low{-}Yield{-}Posten wie C33/C64 zu binden. A03, A89, B61 sowie die Quick{-}Wins B85/A04/C04 bleiben drin; ergänzend sichern A48 und B53 viel Wirkung pro Aufwand. Wenn dir für den Abschluss genau eine Änderung wichtig ist, nenne bitte die konkrete 1:1{-}Substitution innerhalb ähnlicher Aufwandsklassen – ich prüfe dann zielgenau, wie wir das ohne Budgetbruch möglich machen.'\}} \\
            }
        }
    }
    & \\ \\

    \theutterance \stepcounter{utterance}  
    & & & \multicolumn{4}{p{0.6\linewidth}}{
        \cellcolor[rgb]{0.9,0.9,0.9}{
            \makecell[{{p{\linewidth}}}]{
                \texttt{\tiny{[GM$\langle$P2]}}
                \texttt{STRATEGISCHE ÜBERLEGUNGEN: \{'Ihr aktueller Vorschlag bringt mir 5203 Nutzen und enthält meine Must{-}haves A61 und C49 sowie starke Hebel (A03, B61, C04, B85). Jede weitere Verbesserung meinerseits würde C33 oder C64 erfordern, was der/die Andere wiederholt abgelehnt hat. Alternative C33{-}Substitutionen wurden bereits zurückgewiesen. Risiko weiterer Runden: Ablehnung und Verwässerung. Akzeptiere jetzt den soliden Abschluss (3668 ≤ 3681, 13 Puffer) statt auf schwer vermittelbare Nachbesserungen zu bestehen.'\}} \\
\texttt{ZUSTIMMUNG: \{'B85', 'A04', 'C04', 'A48', 'B61', 'B53', 'A89', 'A03', 'A61', 'C49'\}} \\
\texttt{ARGUMENT: \{'Einverstanden – der Mix bleibt im Limit (3668 ≤ 3681), nutzt das Budget nahezu voll aus, sichert A61 und C49 sowie A03/B61/C04/B85 ab und lässt einen kleinen Puffer (13) für Termin{-}/Risikoreserven. Damit kommen wir zügig zu einem belastbaren Abschluss.'\}} \\
            }
        }
    }
     \\ \\

    \theutterance \stepcounter{utterance}  
    & & & \multicolumn{2}{p{0.3\linewidth}}{
        \cellcolor[rgb]{0.9,0.9,0.9}{
            \makecell[{{p{\linewidth}}}]{
                \texttt{\tiny{[GM$|$GM]}}
                \texttt{game successful} \\
            }
        }
    }
    & & \\ \\

    \theutterance \stepcounter{utterance}  
    & & & \multicolumn{2}{p{0.3\linewidth}}{
        \cellcolor[rgb]{0.9,0.9,0.9}{
            \makecell[{{p{\linewidth}}}]{
                \texttt{\tiny{[GM$|$GM]}}
                \texttt{end game} \\
            }
        }
    }
    & & \\ \\

\end{supertabular}
}

\end{document}
