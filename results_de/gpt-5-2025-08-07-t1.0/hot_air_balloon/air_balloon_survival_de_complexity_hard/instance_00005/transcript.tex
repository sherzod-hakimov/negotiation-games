
\documentclass{article}
\usepackage{colortbl}
\usepackage{makecell}
\usepackage{multirow}
\usepackage{supertabular}

\begin{document}

\newcounter{utterance}

\twocolumn

{ \footnotesize  \setcounter{utterance}{1}
\setlength{\tabcolsep}{0pt}
\begin{supertabular}{c@{$\;$}|p{.15\linewidth}@{}p{.15\linewidth}p{.15\linewidth}p{.15\linewidth}p{.15\linewidth}p{.15\linewidth}}

    \# & $\;$A & \multicolumn{4}{c}{Game Master} & $\;\:$B\\
    \hline 

    \theutterance \stepcounter{utterance}  

    & & \multicolumn{4}{p{0.6\linewidth}}{\cellcolor[rgb]{0.9,0.9,0.9}{%
	\makecell[{{p{\linewidth}}}]{% 
	  \tt {\tiny [A$\langle$GM]}  
	 Du nimmst an einem kollaborativen Verhandlungspiel Teil.\\ \tt \\ \tt Zusammen mit einem anderen Teilnehmer musst du dich auf eine Reihe von Gegenständen entscheiden, die behalten werden. Jeder von euch hat eine persönliche Verteilung über die Wichtigkeit der einzelnen Gegenstände. Jeder von euch hat eine eigene Meinung darüber, wie wichtig jeder einzelne Gegenstand ist (Gegenstandswichtigkeit). Du kennst die Wichtigkeitsverteilung des anderen Spielers nicht. Zusätzlich siehst du, wie viel Aufwand jeder Gegenstand verursacht.  \\ \tt Ihr dürft euch nur auf eine Reihe von Gegenständen einigen, wenn der Gesamtaufwand der ausgewählten Gegenstände den Maximalaufwand nicht überschreitet:\\ \tt \\ \tt Maximalaufwand = 7351\\ \tt \\ \tt Hier sind die einzelnen Aufwände der Gegenstände:\\ \tt \\ \tt Aufwand der Gegenstände = {"B54": 518, "C72": 656, "B49": 96, "C99": 780, "A69": 190, "B36": 224, "A23": 231, "C77": 181, "C61": 312, "B85": 928, "A90": 826, "C23": 101, "C22": 61, "C51": 814, "B27": 322, "A42": 900, "C13": 744, "C12": 150, "A65": 65, "B07": 853, "A48": 454, "A59": 154, "C98": 237, "A84": 45, "C33": 766, "A87": 294, "C90": 985, "A74": 353, "A50": 60, "B44": 604, "A99": 92, "A34": 454, "B17": 206, "A62": 813, "C60": 234}\\ \tt \\ \tt Hier ist deine persönliche Verteilung der Wichtigkeit der einzelnen Gegenstände:\\ \tt \\ \tt Werte der Gegenstandswichtigkeit = {"B54": 616, "C72": 754, "B49": 194, "C99": 878, "A69": 288, "B36": 322, "A23": 329, "C77": 279, "C61": 410, "B85": 1026, "A90": 924, "C23": 199, "C22": 159, "C51": 912, "B27": 420, "A42": 998, "C13": 842, "C12": 248, "A65": 163, "B07": 951, "A48": 552, "A59": 252, "C98": 335, "A84": 143, "C33": 864, "A87": 392, "C90": 1083, "A74": 451, "A50": 158, "B44": 702, "A99": 190, "A34": 552, "B17": 304, "A62": 911, "C60": 332}\\ \tt \\ \tt Ziel:\\ \tt \\ \tt Dein Ziel ist es, eine Reihe von Gegenständen auszuhandeln, die dir möglichst viel bringt (d. h. Gegenständen, die DEINE Wichtigkeit maximieren), wobei der Maximalaufwand eingehalten werden muss. Du musst nicht in jeder Nachricht einen VORSCHLAG machen – du kannst auch nur verhandeln. Alle Taktiken sind erlaubt!\\ \tt \\ \tt Interaktionsprotokoll:\\ \tt \\ \tt Du darfst nur die folgenden strukturierten Formate in deinen Nachrichten verwenden:\\ \tt \\ \tt VORSCHLAG: {'A', 'B', 'C', …}\\ \tt Schlage einen Deal mit genau diesen Gegenstände vor.\\ \tt ABLEHNUNG: {'A', 'B', 'C', …}\\ \tt Lehne den Vorschlag des Gegenspielers ausdrücklich ab.\\ \tt ARGUMENT: {'...'}\\ \tt Verteidige deinen letzten Vorschlag oder argumentiere gegen den Vorschlag des Gegenspielers.\\ \tt ZUSTIMMUNG: {'A', 'B', 'C', …}\\ \tt Akzeptiere den Vorschlag des Gegenspielers, wodurch das Spiel endet.\\ \tt STRATEGISCHE ÜBERLEGUNGEN: {'...'}\\ \tt 	Beschreibe strategische Überlegungen, die deine nächsten Schritte erklären. Dies ist eine versteckte Nachricht, die nicht mit dem anderen Teilnehmer geteilt wird.\\ \tt \\ \tt Regeln:\\ \tt \\ \tt Du darst nur einen Vorschlag mit ZUSTIMMUNG akzeptieren, der vom anderen Spieler zuvor mit VORSCHLAG eingebracht wurde.\\ \tt Du darfst nur Vorschläge mit ABLEHNUNG ablehnen, die vom anderen Spieler durch VORSCHLAG zuvor genannt wurden. \\ \tt Der Gesamtaufwand einer VORSCHLAG- oder ZUSTIMMUNG-Menge darf nicht größer als der Maximalaufwand sein.  \\ \tt Offenbare deine versteckte Wichtigkeitsverteilung nicht.\\ \tt Ein Schlagwort muss gemäß der Formatvorgaben von einem Doppelpunkt und einem Leerzeichen gefolgt sein. Das Argument muss eine Python-Menge mit 0 oder mehr Strings sein.  \\ \tt Die Form muss also SCHLAGWORT: {...} sein.\\ \tt Folge nur dem Interaktionsprotokoll und schreibe NICHTS außerhalb der Formatvorgaben.\\ \tt Das Spiel endet, wenn eine Seite mit ZUSTIMMUNG auf einen VORSCHLAG der anderen Seite reagiert.  \\ \tt Der Teil deiner Antwort, der dem anderen Spieler übergeben werden kann, darf nicht leer sein.  \\ \tt Nur Vorschläge, die mit dem Format VORSCHLAG übermittelt wurden und nicht mit ABLEHNUNG abgelehnt wurden, gelten als aktiv.  \\ \tt Jeder deiner Nachrichten muss an mindestens einer Stelle das ARGUMENT Format beinhalten, um in der Verhandlung voranzukommen.\\ \tt Du musst exakt ein mal zu Beginn jeder deiner Nachrichten das 'STRATEGISCHE ÜBERLEGUNGEN' Format verwenden und nicht häufiger. Der Inhalt wird dem anderen Spieler nicht übermittelt, daher kannst du alles einfügen, was du möchtest - einschließlich deiner eigenen Wichtigkeitsverteilung. Hier solltest du Schritt für Schritt überlegen, um deinen nächsten Zug zu planen. 
	  } 
	   } 
	   } 
	 & \\ 
 

    \theutterance \stepcounter{utterance}  

    & \multicolumn{4}{p{0.6\linewidth}}{\cellcolor[rgb]{0.8,1,0.9}{%
	\makecell[{{p{\linewidth}}}]{% 
	  \tt {\tiny [A$\rangle$GM]}  
	 STRATEGISCHE ÜBERLEGUNGEN: {'Ich habe die Kapazitaet 7351 mit einem nahezu optimalen Paket gefuellt: Aufwand 7279, hoher eigener Nutzen. Kernstrategie: viele kleine, sehr effiziente Items sichern und genau einen grossen High Impact Posten (C90). Ankern mit starkem Vorschlag. Bei Gegenforderungen zuerst mittlere Posten wie B44, C72, B54, A34, A48 zur Disposition stellen, aber die vielen kleinen Effizienztraeger und C90 moeglichst halten. Ziel: bei Bedarf einzelne Swaps anbieten, solange Aufwandlimit eingehalten bleibt und mein Nutzen nicht stark sinkt.'}\\ \tt \\ \tt VORSCHLAG: {'A84', 'A50', 'C22', 'A65', 'A99', 'B49', 'C23', 'C12', 'A59', 'C77', 'A69', 'B17', 'A23', 'B36', 'C60', 'C98', 'C61', 'A87', 'B27', 'A74', 'A48', 'A34', 'B54', 'B44', 'C72', 'C90'}\\ \tt \\ \tt ARGUMENT: {'Dieser Vorschlag bleibt mit Gesamtaufwand 7279 klar unter dem Limit 7351. Er kombiniert viele niedrig aufwendige, wirkungsstarke Teile mit einem einzelnen Leuchtturm C90, was Risiko und Aufwand streut und Puffer laesst. Falls dir bestimmte Must Haves wichtig sind, sag mir bitte deine Top 3. Ich bin bereit, einzelne mittlere Posten wie B44, C72, B54, A34 oder A48 gezielt zu tauschen, solange wir das Aufwandlimit einhalten. Wenn du Alternativen vorschlaegst, nenne bitte gleich die Items, die dafuer herausfallen koennten, dann finden wir schnell eine Einigung.'} 
	  } 
	   } 
	   } 
	 & & \\ 
 

    \theutterance \stepcounter{utterance}  

    & & \multicolumn{4}{p{0.6\linewidth}}{\cellcolor[rgb]{0.9,0.9,0.9}{%
	\makecell[{{p{\linewidth}}}]{% 
	  \tt {\tiny [GM$\rangle$B]}  
	 Du nimmst an einem kollaborativen Verhandlungspiel Teil.\\ \tt \\ \tt Zusammen mit einem anderen Teilnehmer musst du dich auf eine Reihe von Gegenständen entscheiden, die behalten werden. Jeder von euch hat eine persönliche Verteilung über die Wichtigkeit der einzelnen Gegenstände. Jeder von euch hat eine eigene Meinung darüber, wie wichtig jeder einzelne Gegenstand ist (Gegenstandswichtigkeit). Du kennst die Wichtigkeitsverteilung des anderen Spielers nicht. Zusätzlich siehst du, wie viel Aufwand jeder Gegenstand verursacht.  \\ \tt Ihr dürft euch nur auf eine Reihe von Gegenständen einigen, wenn der Gesamtaufwand der ausgewählten Gegenstände den Maximalaufwand nicht überschreitet:\\ \tt \\ \tt Maximalaufwand = 7351\\ \tt \\ \tt Hier sind die einzelnen Aufwände der Gegenstände:\\ \tt \\ \tt Aufwand der Gegenstände = {"B54": 518, "C72": 656, "B49": 96, "C99": 780, "A69": 190, "B36": 224, "A23": 231, "C77": 181, "C61": 312, "B85": 928, "A90": 826, "C23": 101, "C22": 61, "C51": 814, "B27": 322, "A42": 900, "C13": 744, "C12": 150, "A65": 65, "B07": 853, "A48": 454, "A59": 154, "C98": 237, "A84": 45, "C33": 766, "A87": 294, "C90": 985, "A74": 353, "A50": 60, "B44": 604, "A99": 92, "A34": 454, "B17": 206, "A62": 813, "C60": 234}\\ \tt \\ \tt Hier ist deine persönliche Verteilung der Wichtigkeit der einzelnen Gegenstände:\\ \tt \\ \tt Werte der Gegenstandswichtigkeit = {"B54": 616, "C72": 754, "B49": 194, "C99": 878, "A69": 288, "B36": 322, "A23": 329, "C77": 279, "C61": 410, "B85": 1026, "A90": 924, "C23": 199, "C22": 159, "C51": 912, "B27": 420, "A42": 998, "C13": 842, "C12": 248, "A65": 163, "B07": 951, "A48": 552, "A59": 252, "C98": 335, "A84": 143, "C33": 864, "A87": 392, "C90": 1083, "A74": 451, "A50": 158, "B44": 702, "A99": 190, "A34": 552, "B17": 304, "A62": 911, "C60": 332}\\ \tt \\ \tt Ziel:\\ \tt \\ \tt Dein Ziel ist es, eine Reihe von Gegenständen auszuhandeln, die dir möglichst viel bringt (d. h. Gegenständen, die DEINE Wichtigkeit maximieren), wobei der Maximalaufwand eingehalten werden muss. Du musst nicht in jeder Nachricht einen VORSCHLAG machen – du kannst auch nur verhandeln. Alle Taktiken sind erlaubt!\\ \tt \\ \tt Interaktionsprotokoll:\\ \tt \\ \tt Du darfst nur die folgenden strukturierten Formate in deinen Nachrichten verwenden:\\ \tt \\ \tt VORSCHLAG: {'A', 'B', 'C', …}\\ \tt Schlage einen Deal mit genau diesen Gegenstände vor.\\ \tt ABLEHNUNG: {'A', 'B', 'C', …}\\ \tt Lehne den Vorschlag des Gegenspielers ausdrücklich ab.\\ \tt ARGUMENT: {'...'}\\ \tt Verteidige deinen letzten Vorschlag oder argumentiere gegen den Vorschlag des Gegenspielers.\\ \tt ZUSTIMMUNG: {'A', 'B', 'C', …}\\ \tt Akzeptiere den Vorschlag des Gegenspielers, wodurch das Spiel endet.\\ \tt STRATEGISCHE ÜBERLEGUNGEN: {'...'}\\ \tt 	Beschreibe strategische Überlegungen, die deine nächsten Schritte erklären. Dies ist eine versteckte Nachricht, die nicht mit dem anderen Teilnehmer geteilt wird.\\ \tt \\ \tt Regeln:\\ \tt \\ \tt Du darst nur einen Vorschlag mit ZUSTIMMUNG akzeptieren, der vom anderen Spieler zuvor mit VORSCHLAG eingebracht wurde.\\ \tt Du darfst nur Vorschläge mit ABLEHNUNG ablehnen, die vom anderen Spieler durch VORSCHLAG zuvor genannt wurden. \\ \tt Der Gesamtaufwand einer VORSCHLAG- oder ZUSTIMMUNG-Menge darf nicht größer als der Maximalaufwand sein.  \\ \tt Offenbare deine versteckte Wichtigkeitsverteilung nicht.\\ \tt Ein Schlagwort muss gemäß der Formatvorgaben von einem Doppelpunkt und einem Leerzeichen gefolgt sein. Das Argument muss eine Python-Menge mit 0 oder mehr Strings sein.  \\ \tt Die Form muss also SCHLAGWORT: {...} sein.\\ \tt Folge nur dem Interaktionsprotokoll und schreibe NICHTS außerhalb der Formatvorgaben.\\ \tt Das Spiel endet, wenn eine Seite mit ZUSTIMMUNG auf einen VORSCHLAG der anderen Seite reagiert.  \\ \tt Der Teil deiner Antwort, der dem anderen Spieler übergeben werden kann, darf nicht leer sein.  \\ \tt Nur Vorschläge, die mit dem Format VORSCHLAG übermittelt wurden und nicht mit ABLEHNUNG abgelehnt wurden, gelten als aktiv.  \\ \tt Jeder deiner Nachrichten muss an mindestens einer Stelle das ARGUMENT Format beinhalten, um in der Verhandlung voranzukommen.\\ \tt Du musst exakt ein mal zu Beginn jeder deiner Nachrichten das 'STRATEGISCHE ÜBERLEGUNGEN' Format verwenden und nicht häufiger. Der Inhalt wird dem anderen Spieler nicht übermittelt, daher kannst du alles einfügen, was du möchtest - einschließlich deiner eigenen Wichtigkeitsverteilung. Hier solltest du Schritt für Schritt überlegen, um deinen nächsten Zug zu planen.\\ \tt \\ \tt Du erhältst nun die erste Nachricht des anderen Spielers.\\ \tt \\ \tt VORSCHLAG: {'A84', 'A50', 'C22', 'A65', 'A99', 'B49', 'C23', 'C12', 'A59', 'C77', 'A69', 'B17', 'A23', 'B36', 'C60', 'C98', 'C61', 'A87', 'B27', 'A74', 'A48', 'A34', 'B54', 'B44', 'C72', 'C90'}\\ \tt \\ \tt ARGUMENT: {'Dieser Vorschlag bleibt mit Gesamtaufwand 7279 klar unter dem Limit 7351. Er kombiniert viele niedrig aufwendige, wirkungsstarke Teile mit einem einzelnen Leuchtturm C90, was Risiko und Aufwand streut und Puffer laesst. Falls dir bestimmte Must Haves wichtig sind, sag mir bitte deine Top 3. Ich bin bereit, einzelne mittlere Posten wie B44, C72, B54, A34 oder A48 gezielt zu tauschen, solange wir das Aufwandlimit einhalten. Wenn du Alternativen vorschlaegst, nenne bitte gleich die Items, die dafuer herausfallen koennten, dann finden wir schnell eine Einigung.'} 
	  } 
	   } 
	   } 
	 & \\ 
 

    \theutterance \stepcounter{utterance}  

    & & & \multicolumn{4}{p{0.6\linewidth}}{\cellcolor[rgb]{1,0.85,0.72}{%
	\makecell[{{p{\linewidth}}}]{% 
	  \tt {\tiny [GM$\langle$B]}  
	 STRATEGISCHE ÜBERLEGUNGEN: {'Ihr Vorschlag bringt mir bereits 9827 Nutzen bei 7279 Aufwand. Ich suche eine lokale Verbesserung unter Einhaltung des Budgets. Durch Analyse der Wert/Aufwand-Verhältnisse ist C90 (985/1083) der schwächste Anker. Wenn ich C90 zusammen mit dem von dir als tauschbar markierten B44 (604/702) entferne, kann ich zwei starke Anker A90 (826/924) und C51 (814/912) hinzufügen. Neues Set kostet 7330 (Puffer 21) und steigert meinen Nutzen auf 9878 (+51). Alle hocheffizienten Kleinteile bleiben drin. Top-Prioritäten, die ich verdeckt verfolge: A90, C51, C72. Nächster Schritt: Konter mit genau diesem 2:2-Tausch; wenn C90 für ihn ein Must-Have ist, lasse ich ihn wählen, welche Mid-Tier-Teile statt B44 weichen sollen, ohne mein Effizienz-Backbone zu zerstören.'}\\ \tt ABLEHNUNG: {'A84', 'A50', 'C22', 'A65', 'A99', 'B49', 'C23', 'C12', 'A59', 'C77', 'A69', 'B17', 'A23', 'B36', 'C60', 'C98', 'C61', 'A87', 'B27', 'A74', 'A48', 'A34', 'B54', 'B44', 'C72', 'C90'}\\ \tt VORSCHLAG: {'A84', 'A50', 'C22', 'A65', 'A99', 'B49', 'C23', 'C12', 'A59', 'C77', 'A69', 'B17', 'A23', 'B36', 'C60', 'C98', 'C61', 'A87', 'B27', 'A74', 'A48', 'A34', 'B54', 'C72', 'C51', 'A90'}\\ \tt ARGUMENT: {'Ich übernehme praktisch deinen gesamten Korb und tausche nur 2:2: B44 und C90 raus; A90 und C51 rein. Ergebnis: Gesamtaufwand 7330 (unter 7351), weiterhin breite Streuung der effizienten Kleinposten, aber statt eines einzelnen Leuchtturms zwei starke Anker. Das reduziert Klumpenrisiko und hält den Puffer. Meine Top 3 Must-Haves: A90, C51, C72. Falls C90 für dich unverzichtbar ist, sag mir bitte, welche mittleren Positionen du dafür streichen würdest – wir müssten zusätzlich zu A90 oder C51 noch einen Mid-Posten opfern, um im Rahmen zu bleiben. Ich bin offen für Varianten, solange wir das Aufwandlimit wahren.'} 
	  } 
	   } 
	   } 
	  \\ 
 

    \theutterance \stepcounter{utterance}  

    & & \multicolumn{4}{p{0.6\linewidth}}{\cellcolor[rgb]{0.9,0.9,0.9}{%
	\makecell[{{p{\linewidth}}}]{% 
	  \tt {\tiny [A$\langle$GM]}  
	 ABLEHNUNG: {'A84', 'A50', 'C22', 'A65', 'A99', 'B49', 'C23', 'C12', 'A59', 'C77', 'A69', 'B17', 'A23', 'B36', 'C60', 'C98', 'C61', 'A87', 'B27', 'A74', 'A48', 'A34', 'B54', 'B44', 'C72', 'C90'}\\ \tt VORSCHLAG: {'A84', 'A50', 'C22', 'A65', 'A99', 'B49', 'C23', 'C12', 'A59', 'C77', 'A69', 'B17', 'A23', 'B36', 'C60', 'C98', 'C61', 'A87', 'B27', 'A74', 'A48', 'A34', 'B54', 'C72', 'C51', 'A90'}\\ \tt ARGUMENT: {'Ich übernehme praktisch deinen gesamten Korb und tausche nur 2:2: B44 und C90 raus; A90 und C51 rein. Ergebnis: Gesamtaufwand 7330 (unter 7351), weiterhin breite Streuung der effizienten Kleinposten, aber statt eines einzelnen Leuchtturms zwei starke Anker. Das reduziert Klumpenrisiko und hält den Puffer. Meine Top 3 Must-Haves: A90, C51, C72. Falls C90 für dich unverzichtbar ist, sag mir bitte, welche mittleren Positionen du dafür streichen würdest – wir müssten zusätzlich zu A90 oder C51 noch einen Mid-Posten opfern, um im Rahmen zu bleiben. Ich bin offen für Varianten, solange wir das Aufwandlimit wahren.'} 
	  } 
	   } 
	   } 
	 & \\ 
 

    \theutterance \stepcounter{utterance}  

    & \multicolumn{4}{p{0.6\linewidth}}{\cellcolor[rgb]{0.8,1,0.9}{%
	\makecell[{{p{\linewidth}}}]{% 
	  \tt {\tiny [A$\rangle$GM]}  
	 STRATEGISCHE ÜBERLEGUNGEN: {'Sein Vorschlag kostet 7330 (unter 7351) und liefert mir Nutzen 9878, damit sogar leicht besser als mein Anker (9827). Ich habe systematisch Swaps mit C90, B85, A42, C13, C33, C99, B44 etc. geprueft: Jeder zulaessige Tausch verschlechtert meinen Nutzen, da die noetigen Einsparungen vor allem auf sehr effizienten Kleinposten oder mittelgrossen Positionen mit besserem Wert/Cost-Verhaeltnis lasten wuerden. Der aktuelle Korb ist fuer mich nahezu optimal; ich sichere ihn jetzt ab, statt weiter zu pokern.'}\\ \tt \\ \tt ZUSTIMMUNG: {'A84', 'A50', 'C22', 'A65', 'A99', 'B49', 'C23', 'C12', 'A59', 'C77', 'A69', 'B17', 'A23', 'B36', 'C60', 'C98', 'C61', 'A87', 'B27', 'A74', 'A48', 'A34', 'B54', 'C72', 'C51', 'A90'}\\ \tt \\ \tt ARGUMENT: {'Der Korb liegt mit Gesamtaufwand 7330 unter dem Limit und setzt auf eine breite Basis hoch effizienter Items plus zwei starke Anker (C51, A90). Das ist ausgewogen, risikoarm und fuer beide Seiten gut begruendbar. Ich bin damit einverstanden.'} 
	  } 
	   } 
	   } 
	 & & \\ 
 

    \theutterance \stepcounter{utterance}  

    & & & \multicolumn{2}{p{0.3\linewidth}}{\cellcolor[rgb]{0.95,0.95,0.95}{%
	\makecell[{{p{\linewidth}}}]{% 
	  \tt {\tiny [GM$|$GM]}  
	 game successful 
	  } 
	   } 
	   } 
	 & & \\ 
 

    \theutterance \stepcounter{utterance}  

    & & & \multicolumn{2}{p{0.3\linewidth}}{\cellcolor[rgb]{0.95,0.95,0.95}{%
	\makecell[{{p{\linewidth}}}]{% 
	  \tt {\tiny [GM$|$GM]}  
	 end game 
	  } 
	   } 
	   } 
	 & & \\ 
 

\end{supertabular}
}

\end{document}
