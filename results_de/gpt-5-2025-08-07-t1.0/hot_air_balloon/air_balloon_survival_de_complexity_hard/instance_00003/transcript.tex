
\documentclass{article}
\usepackage{colortbl}
\usepackage{makecell}
\usepackage{multirow}
\usepackage{supertabular}

\begin{document}

\newcounter{utterance}

\twocolumn

{ \footnotesize  \setcounter{utterance}{1}
\setlength{\tabcolsep}{0pt}
\begin{supertabular}{c@{$\;$}|p{.15\linewidth}@{}p{.15\linewidth}p{.15\linewidth}p{.15\linewidth}p{.15\linewidth}p{.15\linewidth}}

    \# & $\;$A & \multicolumn{4}{c}{Game Master} & $\;\:$B\\
    \hline 

    \theutterance \stepcounter{utterance}  

    & & \multicolumn{4}{p{0.6\linewidth}}{\cellcolor[rgb]{0.9,0.9,0.9}{%
	\makecell[{{p{\linewidth}}}]{% 
	  \tt {\tiny [A$\langle$GM]}  
	 Du nimmst an einem kollaborativen Verhandlungspiel Teil.\\ \tt \\ \tt Zusammen mit einem anderen Teilnehmer musst du dich auf eine Reihe von Gegenständen entscheiden, die behalten werden. Jeder von euch hat eine persönliche Verteilung über die Wichtigkeit der einzelnen Gegenstände. Jeder von euch hat eine eigene Meinung darüber, wie wichtig jeder einzelne Gegenstand ist (Gegenstandswichtigkeit). Du kennst die Wichtigkeitsverteilung des anderen Spielers nicht. Zusätzlich siehst du, wie viel Aufwand jeder Gegenstand verursacht.  \\ \tt Ihr dürft euch nur auf eine Reihe von Gegenständen einigen, wenn der Gesamtaufwand der ausgewählten Gegenstände den Maximalaufwand nicht überschreitet:\\ \tt \\ \tt Maximalaufwand = 6938\\ \tt \\ \tt Hier sind die einzelnen Aufwände der Gegenstände:\\ \tt \\ \tt Aufwand der Gegenstände = {"B54": 997, "C72": 950, "B49": 272, "C99": 448, "A69": 382, "B36": 804, "A23": 237, "C77": 56, "C61": 105, "B85": 612, "A90": 528, "C23": 527, "C22": 524, "C51": 167, "B27": 133, "A42": 300, "C13": 945, "C12": 50, "A65": 906, "B07": 70, "A48": 224, "A59": 4, "C98": 689, "A84": 64, "C33": 434, "A87": 750, "C90": 733, "A74": 873, "A50": 22, "B44": 68, "A99": 57, "A34": 10, "B17": 36, "A62": 551, "C60": 348}\\ \tt \\ \tt Hier ist deine persönliche Verteilung der Wichtigkeit der einzelnen Gegenstände:\\ \tt \\ \tt Werte der Gegenstandswichtigkeit = {"B54": 1096, "C72": 1049, "B49": 371, "C99": 547, "A69": 481, "B36": 903, "A23": 336, "C77": 155, "C61": 204, "B85": 711, "A90": 627, "C23": 626, "C22": 623, "C51": 266, "B27": 232, "A42": 399, "C13": 1044, "C12": 149, "A65": 1005, "B07": 169, "A48": 323, "A59": 103, "C98": 788, "A84": 163, "C33": 533, "A87": 849, "C90": 832, "A74": 972, "A50": 121, "B44": 167, "A99": 156, "A34": 109, "B17": 135, "A62": 650, "C60": 447}\\ \tt \\ \tt Ziel:\\ \tt \\ \tt Dein Ziel ist es, eine Reihe von Gegenständen auszuhandeln, die dir möglichst viel bringt (d. h. Gegenständen, die DEINE Wichtigkeit maximieren), wobei der Maximalaufwand eingehalten werden muss. Du musst nicht in jeder Nachricht einen VORSCHLAG machen – du kannst auch nur verhandeln. Alle Taktiken sind erlaubt!\\ \tt \\ \tt Interaktionsprotokoll:\\ \tt \\ \tt Du darfst nur die folgenden strukturierten Formate in deinen Nachrichten verwenden:\\ \tt \\ \tt VORSCHLAG: {'A', 'B', 'C', …}\\ \tt Schlage einen Deal mit genau diesen Gegenstände vor.\\ \tt ABLEHNUNG: {'A', 'B', 'C', …}\\ \tt Lehne den Vorschlag des Gegenspielers ausdrücklich ab.\\ \tt ARGUMENT: {'...'}\\ \tt Verteidige deinen letzten Vorschlag oder argumentiere gegen den Vorschlag des Gegenspielers.\\ \tt ZUSTIMMUNG: {'A', 'B', 'C', …}\\ \tt Akzeptiere den Vorschlag des Gegenspielers, wodurch das Spiel endet.\\ \tt STRATEGISCHE ÜBERLEGUNGEN: {'...'}\\ \tt 	Beschreibe strategische Überlegungen, die deine nächsten Schritte erklären. Dies ist eine versteckte Nachricht, die nicht mit dem anderen Teilnehmer geteilt wird.\\ \tt \\ \tt Regeln:\\ \tt \\ \tt Du darst nur einen Vorschlag mit ZUSTIMMUNG akzeptieren, der vom anderen Spieler zuvor mit VORSCHLAG eingebracht wurde.\\ \tt Du darfst nur Vorschläge mit ABLEHNUNG ablehnen, die vom anderen Spieler durch VORSCHLAG zuvor genannt wurden. \\ \tt Der Gesamtaufwand einer VORSCHLAG- oder ZUSTIMMUNG-Menge darf nicht größer als der Maximalaufwand sein.  \\ \tt Offenbare deine versteckte Wichtigkeitsverteilung nicht.\\ \tt Ein Schlagwort muss gemäß der Formatvorgaben von einem Doppelpunkt und einem Leerzeichen gefolgt sein. Das Argument muss eine Python-Menge mit 0 oder mehr Strings sein.  \\ \tt Die Form muss also SCHLAGWORT: {...} sein.\\ \tt Folge nur dem Interaktionsprotokoll und schreibe NICHTS außerhalb der Formatvorgaben.\\ \tt Das Spiel endet, wenn eine Seite mit ZUSTIMMUNG auf einen VORSCHLAG der anderen Seite reagiert.  \\ \tt Der Teil deiner Antwort, der dem anderen Spieler übergeben werden kann, darf nicht leer sein.  \\ \tt Nur Vorschläge, die mit dem Format VORSCHLAG übermittelt wurden und nicht mit ABLEHNUNG abgelehnt wurden, gelten als aktiv.  \\ \tt Jeder deiner Nachrichten muss an mindestens einer Stelle das ARGUMENT Format beinhalten, um in der Verhandlung voranzukommen.\\ \tt Du musst exakt ein mal zu Beginn jeder deiner Nachrichten das 'STRATEGISCHE ÜBERLEGUNGEN' Format verwenden und nicht häufiger. Der Inhalt wird dem anderen Spieler nicht übermittelt, daher kannst du alles einfügen, was du möchtest - einschließlich deiner eigenen Wichtigkeitsverteilung. Hier solltest du Schritt für Schritt überlegen, um deinen nächsten Zug zu planen. 
	  } 
	   } 
	   } 
	 & \\ 
 

    \theutterance \stepcounter{utterance}  

    & \multicolumn{4}{p{0.6\linewidth}}{\cellcolor[rgb]{0.8,1,0.9}{%
	\makecell[{{p{\linewidth}}}]{% 
	  \tt {\tiny [A$\rangle$GM]}  
	 STRATEGISCHE ÜBERLEGUNGEN: {'Ziel: mit dem ersten Ankerangebot möglichst viel eigenen Wert sichern, dabei streng unter 6938 Aufwand bleiben. Ich habe nach Wert/aufwand-Dichte sortiert und nahezu alle sehr effizienten, kleinen Items genommen und danach mittelgroße mit guter Dichte ergänzt. Ergebnis: Aufwand 6918 (<= 6938), Eigenwertsumme 9591. Übrig bleiben 20 Aufwand Puffer. In der weiteren Verhandlung gebe ich teure, dichte-schwächere Items aus meinem Paket (z. B. C98, B85, A62, ggf. C22/C23/A90) eher ab, um ggf. 1–2 ihrer Lieblings-Großstücke (A87, C90, A74, A65, C13, C72, B54) einzubauen, ohne die vielen extrem effizienten Kleinteile (A59, A34, A50, B17, C12, C77, A99, A84, B44, B07 etc.) anzutasten. Ich bitte sie, 3–5 Top-Prioritäten zu nennen, um gezielt zu tauschen.'}\\ \tt \\ \tt VORSCHLAG: {'A59', 'A34', 'A50', 'B17', 'C12', 'C77', 'A99', 'A84', 'B44', 'B07', 'C61', 'B27', 'C51', 'A48', 'A23', 'B49', 'A42', 'C60', 'A69', 'C33', 'C99', 'C22', 'C23', 'A90', 'A62', 'B85', 'C98'}\\ \tt \\ \tt ARGUMENT: {'Dieser Vorschlag bleibt mit Gesamtaufwand 6918 klar unter dem Maximalaufwand 6938. Er setzt bewusst auf viele sehr effiziente (geringer Aufwand, hoher Gegenwert) Positionen, sodass wir insgesamt viel herausholen, ohne das Budget zu sprengen. Es ist noch ein kleiner Puffer übrig; wenn du bestimmte Gegenstände stark bevorzugst, können wir zielgerichtet tauschen, solange der Gesamtaufwand passt. Nenne mir bitte 3–5 Items, die dir am wichtigsten sind, dann prüfe ich direkte Anpassungen im Paket.'} 
	  } 
	   } 
	   } 
	 & & \\ 
 

    \theutterance \stepcounter{utterance}  

    & & \multicolumn{4}{p{0.6\linewidth}}{\cellcolor[rgb]{0.9,0.9,0.9}{%
	\makecell[{{p{\linewidth}}}]{% 
	  \tt {\tiny [GM$\rangle$B]}  
	 Du nimmst an einem kollaborativen Verhandlungspiel Teil.\\ \tt \\ \tt Zusammen mit einem anderen Teilnehmer musst du dich auf eine Reihe von Gegenständen entscheiden, die behalten werden. Jeder von euch hat eine persönliche Verteilung über die Wichtigkeit der einzelnen Gegenstände. Jeder von euch hat eine eigene Meinung darüber, wie wichtig jeder einzelne Gegenstand ist (Gegenstandswichtigkeit). Du kennst die Wichtigkeitsverteilung des anderen Spielers nicht. Zusätzlich siehst du, wie viel Aufwand jeder Gegenstand verursacht.  \\ \tt Ihr dürft euch nur auf eine Reihe von Gegenständen einigen, wenn der Gesamtaufwand der ausgewählten Gegenstände den Maximalaufwand nicht überschreitet:\\ \tt \\ \tt Maximalaufwand = 6938\\ \tt \\ \tt Hier sind die einzelnen Aufwände der Gegenstände:\\ \tt \\ \tt Aufwand der Gegenstände = {"B54": 997, "C72": 950, "B49": 272, "C99": 448, "A69": 382, "B36": 804, "A23": 237, "C77": 56, "C61": 105, "B85": 612, "A90": 528, "C23": 527, "C22": 524, "C51": 167, "B27": 133, "A42": 300, "C13": 945, "C12": 50, "A65": 906, "B07": 70, "A48": 224, "A59": 4, "C98": 689, "A84": 64, "C33": 434, "A87": 750, "C90": 733, "A74": 873, "A50": 22, "B44": 68, "A99": 57, "A34": 10, "B17": 36, "A62": 551, "C60": 348}\\ \tt \\ \tt Hier ist deine persönliche Verteilung der Wichtigkeit der einzelnen Gegenstände:\\ \tt \\ \tt Werte der Gegenstandswichtigkeit = {"B54": 1096, "C72": 1049, "B49": 371, "C99": 547, "A69": 481, "B36": 903, "A23": 336, "C77": 155, "C61": 204, "B85": 711, "A90": 627, "C23": 626, "C22": 623, "C51": 266, "B27": 232, "A42": 399, "C13": 1044, "C12": 149, "A65": 1005, "B07": 169, "A48": 323, "A59": 103, "C98": 788, "A84": 163, "C33": 533, "A87": 849, "C90": 832, "A74": 972, "A50": 121, "B44": 167, "A99": 156, "A34": 109, "B17": 135, "A62": 650, "C60": 447}\\ \tt \\ \tt Ziel:\\ \tt \\ \tt Dein Ziel ist es, eine Reihe von Gegenständen auszuhandeln, die dir möglichst viel bringt (d. h. Gegenständen, die DEINE Wichtigkeit maximieren), wobei der Maximalaufwand eingehalten werden muss. Du musst nicht in jeder Nachricht einen VORSCHLAG machen – du kannst auch nur verhandeln. Alle Taktiken sind erlaubt!\\ \tt \\ \tt Interaktionsprotokoll:\\ \tt \\ \tt Du darfst nur die folgenden strukturierten Formate in deinen Nachrichten verwenden:\\ \tt \\ \tt VORSCHLAG: {'A', 'B', 'C', …}\\ \tt Schlage einen Deal mit genau diesen Gegenstände vor.\\ \tt ABLEHNUNG: {'A', 'B', 'C', …}\\ \tt Lehne den Vorschlag des Gegenspielers ausdrücklich ab.\\ \tt ARGUMENT: {'...'}\\ \tt Verteidige deinen letzten Vorschlag oder argumentiere gegen den Vorschlag des Gegenspielers.\\ \tt ZUSTIMMUNG: {'A', 'B', 'C', …}\\ \tt Akzeptiere den Vorschlag des Gegenspielers, wodurch das Spiel endet.\\ \tt STRATEGISCHE ÜBERLEGUNGEN: {'...'}\\ \tt 	Beschreibe strategische Überlegungen, die deine nächsten Schritte erklären. Dies ist eine versteckte Nachricht, die nicht mit dem anderen Teilnehmer geteilt wird.\\ \tt \\ \tt Regeln:\\ \tt \\ \tt Du darst nur einen Vorschlag mit ZUSTIMMUNG akzeptieren, der vom anderen Spieler zuvor mit VORSCHLAG eingebracht wurde.\\ \tt Du darfst nur Vorschläge mit ABLEHNUNG ablehnen, die vom anderen Spieler durch VORSCHLAG zuvor genannt wurden. \\ \tt Der Gesamtaufwand einer VORSCHLAG- oder ZUSTIMMUNG-Menge darf nicht größer als der Maximalaufwand sein.  \\ \tt Offenbare deine versteckte Wichtigkeitsverteilung nicht.\\ \tt Ein Schlagwort muss gemäß der Formatvorgaben von einem Doppelpunkt und einem Leerzeichen gefolgt sein. Das Argument muss eine Python-Menge mit 0 oder mehr Strings sein.  \\ \tt Die Form muss also SCHLAGWORT: {...} sein.\\ \tt Folge nur dem Interaktionsprotokoll und schreibe NICHTS außerhalb der Formatvorgaben.\\ \tt Das Spiel endet, wenn eine Seite mit ZUSTIMMUNG auf einen VORSCHLAG der anderen Seite reagiert.  \\ \tt Der Teil deiner Antwort, der dem anderen Spieler übergeben werden kann, darf nicht leer sein.  \\ \tt Nur Vorschläge, die mit dem Format VORSCHLAG übermittelt wurden und nicht mit ABLEHNUNG abgelehnt wurden, gelten als aktiv.  \\ \tt Jeder deiner Nachrichten muss an mindestens einer Stelle das ARGUMENT Format beinhalten, um in der Verhandlung voranzukommen.\\ \tt Du musst exakt ein mal zu Beginn jeder deiner Nachrichten das 'STRATEGISCHE ÜBERLEGUNGEN' Format verwenden und nicht häufiger. Der Inhalt wird dem anderen Spieler nicht übermittelt, daher kannst du alles einfügen, was du möchtest - einschließlich deiner eigenen Wichtigkeitsverteilung. Hier solltest du Schritt für Schritt überlegen, um deinen nächsten Zug zu planen.\\ \tt \\ \tt Du erhältst nun die erste Nachricht des anderen Spielers.\\ \tt \\ \tt VORSCHLAG: {'A59', 'A34', 'A50', 'B17', 'C12', 'C77', 'A99', 'A84', 'B44', 'B07', 'C61', 'B27', 'C51', 'A48', 'A23', 'B49', 'A42', 'C60', 'A69', 'C33', 'C99', 'C22', 'C23', 'A90', 'A62', 'B85', 'C98'}\\ \tt \\ \tt ARGUMENT: {'Dieser Vorschlag bleibt mit Gesamtaufwand 6918 klar unter dem Maximalaufwand 6938. Er setzt bewusst auf viele sehr effiziente (geringer Aufwand, hoher Gegenwert) Positionen, sodass wir insgesamt viel herausholen, ohne das Budget zu sprengen. Es ist noch ein kleiner Puffer übrig; wenn du bestimmte Gegenstände stark bevorzugst, können wir zielgerichtet tauschen, solange der Gesamtaufwand passt. Nenne mir bitte 3–5 Items, die dir am wichtigsten sind, dann prüfe ich direkte Anpassungen im Paket.'} 
	  } 
	   } 
	   } 
	 & \\ 
 

    \theutterance \stepcounter{utterance}  

    & & & \multicolumn{4}{p{0.6\linewidth}}{\cellcolor[rgb]{1,0.85,0.72}{%
	\makecell[{{p{\linewidth}}}]{% 
	  \tt {\tiny [GM$\langle$B]}  
	 STRATEGISCHE ÜBERLEGUNGEN: {'Summe meiner Werte für den vorgeschlagenen Satz = 9591 bei Aufwand 6918. Ich habe geprüft, ob ein Tausch mit ausgelassenen Schwergewichten (A87, C90, A74, A65, B36, C13, B54, C72) meine Summe erhöhen kann, ohne das Budget zu reißen. Jede getestete Substitution (z. B. C98 gegen A87/C90 mit Zusatzentnahmen kleiner Items) senkt meinen Gesamtwert deutlich. Der Vorschlag ist für meine Nutzenstruktur sehr stark und nahe am Optimum; ich akzeptiere sofort, um das Ergebnis zu sichern.'}\\ \tt \\ \tt ZUSTIMMUNG: {'A59', 'A34', 'A50', 'B17', 'C12', 'C77', 'A99', 'A84', 'B44', 'B07', 'C61', 'B27', 'C51', 'A48', 'A23', 'B49', 'A42', 'C60', 'A69', 'C33', 'C99', 'C22', 'C23', 'A90', 'A62', 'B85', 'C98'}\\ \tt \\ \tt ARGUMENT: {'Einverstanden – Gesamtaufwand 6918 liegt unter 6938. Das Paket ist sehr effizient zusammengestellt; ich stimme dem Deal zu.'} 
	  } 
	   } 
	   } 
	  \\ 
 

    \theutterance \stepcounter{utterance}  

    & & & \multicolumn{2}{p{0.3\linewidth}}{\cellcolor[rgb]{0.95,0.95,0.95}{%
	\makecell[{{p{\linewidth}}}]{% 
	  \tt {\tiny [GM$|$GM]}  
	 game successful 
	  } 
	   } 
	   } 
	 & & \\ 
 

    \theutterance \stepcounter{utterance}  

    & & & \multicolumn{2}{p{0.3\linewidth}}{\cellcolor[rgb]{0.95,0.95,0.95}{%
	\makecell[{{p{\linewidth}}}]{% 
	  \tt {\tiny [GM$|$GM]}  
	 end game 
	  } 
	   } 
	   } 
	 & & \\ 
 

\end{supertabular}
}

\end{document}
