
\documentclass{article}
\usepackage{colortbl}
\usepackage{makecell}
\usepackage{multirow}
\usepackage{supertabular}

\begin{document}

\newcounter{utterance}

\twocolumn

{ \footnotesize  \setcounter{utterance}{1}
\setlength{\tabcolsep}{0pt}
\begin{supertabular}{c@{$\;$}|p{.15\linewidth}@{}p{.15\linewidth}p{.15\linewidth}p{.15\linewidth}p{.15\linewidth}p{.15\linewidth}}

    \# & $\;$A & \multicolumn{4}{c}{Game Master} & $\;\:$B\\
    \hline 

    \theutterance \stepcounter{utterance}  

    & & \multicolumn{4}{p{0.6\linewidth}}{\cellcolor[rgb]{0.9,0.9,0.9}{%
	\makecell[{{p{\linewidth}}}]{% 
	  \tt {\tiny [A$\langle$GM]}  
	 Du nimmst an einem kollaborativen Verhandlungspiel Teil.\\ \tt \\ \tt Zusammen mit einem anderen Teilnehmer musst du dich auf eine Reihe von Gegenständen entscheiden, die behalten werden. Jeder von euch hat eine persönliche Verteilung über die Wichtigkeit der einzelnen Gegenstände. Jeder von euch hat eine eigene Meinung darüber, wie wichtig jeder einzelne Gegenstand ist (Gegenstandswichtigkeit). Du kennst die Wichtigkeitsverteilung des anderen Spielers nicht. Zusätzlich siehst du, wie viel Aufwand jeder Gegenstand verursacht.  \\ \tt Ihr dürft euch nur auf eine Reihe von Gegenständen einigen, wenn der Gesamtaufwand der ausgewählten Gegenstände den Maximalaufwand nicht überschreitet:\\ \tt \\ \tt Maximalaufwand = 8145\\ \tt \\ \tt Hier sind die einzelnen Aufwände der Gegenstände:\\ \tt \\ \tt Aufwand der Gegenstände = {"B54": 963, "C72": 204, "B49": 418, "C99": 961, "A69": 238, "B36": 623, "A23": 44, "C77": 633, "C61": 247, "B85": 645, "A90": 230, "C23": 250, "C22": 731, "C51": 405, "B27": 389, "A42": 216, "C13": 637, "C12": 156, "A65": 738, "B07": 307, "A48": 762, "A59": 737, "C98": 901, "A84": 369, "C33": 2, "A87": 730, "C90": 722, "A74": 704, "A50": 315, "B44": 455, "A99": 510, "A34": 175, "B17": 692, "A62": 150, "C60": 32}\\ \tt \\ \tt Hier ist deine persönliche Verteilung der Wichtigkeit der einzelnen Gegenstände:\\ \tt \\ \tt Werte der Gegenstandswichtigkeit = {"B54": 1059, "C72": 300, "B49": 514, "C99": 1057, "A69": 334, "B36": 719, "A23": 140, "C77": 729, "C61": 343, "B85": 741, "A90": 326, "C23": 346, "C22": 827, "C51": 501, "B27": 485, "A42": 312, "C13": 733, "C12": 252, "A65": 834, "B07": 403, "A48": 858, "A59": 833, "C98": 997, "A84": 465, "C33": 98, "A87": 826, "C90": 818, "A74": 800, "A50": 411, "B44": 551, "A99": 606, "A34": 271, "B17": 788, "A62": 246, "C60": 128}\\ \tt \\ \tt Ziel:\\ \tt \\ \tt Dein Ziel ist es, eine Reihe von Gegenständen auszuhandeln, die dir möglichst viel bringt (d. h. Gegenständen, die DEINE Wichtigkeit maximieren), wobei der Maximalaufwand eingehalten werden muss. Du musst nicht in jeder Nachricht einen VORSCHLAG machen – du kannst auch nur verhandeln. Alle Taktiken sind erlaubt!\\ \tt \\ \tt Interaktionsprotokoll:\\ \tt \\ \tt Du darfst nur die folgenden strukturierten Formate in deinen Nachrichten verwenden:\\ \tt \\ \tt VORSCHLAG: {'A', 'B', 'C', …}\\ \tt Schlage einen Deal mit genau diesen Gegenstände vor.\\ \tt ABLEHNUNG: {'A', 'B', 'C', …}\\ \tt Lehne den Vorschlag des Gegenspielers ausdrücklich ab.\\ \tt ARGUMENT: {'...'}\\ \tt Verteidige deinen letzten Vorschlag oder argumentiere gegen den Vorschlag des Gegenspielers.\\ \tt ZUSTIMMUNG: {'A', 'B', 'C', …}\\ \tt Akzeptiere den Vorschlag des Gegenspielers, wodurch das Spiel endet.\\ \tt STRATEGISCHE ÜBERLEGUNGEN: {'...'}\\ \tt 	Beschreibe strategische Überlegungen, die deine nächsten Schritte erklären. Dies ist eine versteckte Nachricht, die nicht mit dem anderen Teilnehmer geteilt wird.\\ \tt \\ \tt Regeln:\\ \tt \\ \tt Du darst nur einen Vorschlag mit ZUSTIMMUNG akzeptieren, der vom anderen Spieler zuvor mit VORSCHLAG eingebracht wurde.\\ \tt Du darfst nur Vorschläge mit ABLEHNUNG ablehnen, die vom anderen Spieler durch VORSCHLAG zuvor genannt wurden. \\ \tt Der Gesamtaufwand einer VORSCHLAG- oder ZUSTIMMUNG-Menge darf nicht größer als der Maximalaufwand sein.  \\ \tt Offenbare deine versteckte Wichtigkeitsverteilung nicht.\\ \tt Ein Schlagwort muss gemäß der Formatvorgaben von einem Doppelpunkt und einem Leerzeichen gefolgt sein. Das Argument muss eine Python-Menge mit 0 oder mehr Strings sein.  \\ \tt Die Form muss also SCHLAGWORT: {...} sein.\\ \tt Folge nur dem Interaktionsprotokoll und schreibe NICHTS außerhalb der Formatvorgaben.\\ \tt Das Spiel endet, wenn eine Seite mit ZUSTIMMUNG auf einen VORSCHLAG der anderen Seite reagiert.  \\ \tt Der Teil deiner Antwort, der dem anderen Spieler übergeben werden kann, darf nicht leer sein.  \\ \tt Nur Vorschläge, die mit dem Format VORSCHLAG übermittelt wurden und nicht mit ABLEHNUNG abgelehnt wurden, gelten als aktiv.  \\ \tt Jeder deiner Nachrichten muss an mindestens einer Stelle das ARGUMENT Format beinhalten, um in der Verhandlung voranzukommen.\\ \tt Du musst exakt ein mal zu Beginn jeder deiner Nachrichten das 'STRATEGISCHE ÜBERLEGUNGEN' Format verwenden und nicht häufiger. Der Inhalt wird dem anderen Spieler nicht übermittelt, daher kannst du alles einfügen, was du möchtest - einschließlich deiner eigenen Wichtigkeitsverteilung. Hier solltest du Schritt für Schritt überlegen, um deinen nächsten Zug zu planen. 
	  } 
	   } 
	   } 
	 & \\ 
 

    \theutterance \stepcounter{utterance}  

    & \multicolumn{4}{p{0.6\linewidth}}{\cellcolor[rgb]{0.8,1,0.9}{%
	\makecell[{{p{\linewidth}}}]{% 
	  \tt {\tiny [A$\rangle$GM]}  
	 STRATEGISCHE ÜBERLEGUNGEN: {'Ich will mit einem starken, nahezu kapazitätsfüllenden Anker starten, der für mich sehr hohe Wertdichte bietet und kaum Raum für additive Erweiterungen lässt. Meine Wertungen (versteckt): {"B54": 1059, "C72": 300, "B49": 514, "C99": 1057, "A69": 334, "B36": 719, "A23": 140, "C77": 729, "C61": 343, "B85": 741, "A90": 326, "C23": 346, "C22": 827, "C51": 501, "B27": 485, "A42": 312, "C13": 733, "C12": 252, "A65": 834, "B07": 403, "A48": 858, "A59": 833, "C98": 997, "A84": 465, "C33": 98, "A87": 826, "C90": 818, "A74": 800, "A50": 411, "B44": 551, "A99": 606, "A34": 271, "B17": 788, "A62": 246, "C60": 128}. Greedy nach Wert pro Aufwand liefert: S0 = {C33,C60,A23,A62,C12,A34,C72,A42,A90,A69,C61,C23,B07,A50,A84,B27,C51,B49,B44,A99,B36,C77,C13,B85} mit Aufwand 7650 und Wert 9954. Es bleiben 495 Aufwand frei, aber es gibt keinen weiteren nicht-gewählten Gegenstand mit Aufwand ≤ 495, daher ist der Vorschlag additiv maximal; jede Änderung erfordert Austausch. Ich verteidige besonders die sehr effizienten, leichten Posten (C33, C60, A23, A62, C12, A34, C72, A42, A90, A69, C61, C23). Falls nötig, bin ich am ehesten bereit, über einen der schwereren Blöcke (B36, C77, C13, B85) zu verhandeln, aber nur gegen kompensierende Kombinationen mit mindestens gleicher Wertdichte. Taktik: hart ankern, Gegenvorschläge zu 1-zu-1 Tauschen einfordern, kleine effiziente Posten als rote Linie markieren.'}\\ \tt VORSCHLAG: {'C33', 'C60', 'A23', 'A62', 'C12', 'A34', 'C72', 'A42', 'A90', 'A69', 'C61', 'C23', 'B07', 'A50', 'A84', 'B27', 'C51', 'B49', 'B44', 'A99', 'B36', 'C77', 'C13', 'B85'}\\ \tt ARGUMENT: {'Dieser Vorschlag liegt mit Gesamtaufwand 7650 klar unter dem Limit 8145 und nutzt die Kapazität so weit aus, dass keine weiteren Gegenstände hinzugefügt werden können, ohne etwas anderes zu entfernen. Er kombiniert viele leichte, effiziente Posten mit einigen größeren Bausteinen und ist dadurch ausgewogen und kosteneffizient. Wenn du Anpassungen möchtest, schlage bitte konkrete 1-zu-1 oder kapazitätsneutrale Tausche vor, die innerhalb des Maximalaufwands bleiben. Die sehr leichten, effizienten Posten würde ich ungern anfassen; bei den größeren Bausteinen können wir über Alternativen sprechen, sofern die Gesamteffizienz erhalten bleibt.'} 
	  } 
	   } 
	   } 
	 & & \\ 
 

    \theutterance \stepcounter{utterance}  

    & & \multicolumn{4}{p{0.6\linewidth}}{\cellcolor[rgb]{0.9,0.9,0.9}{%
	\makecell[{{p{\linewidth}}}]{% 
	  \tt {\tiny [GM$\rangle$B]}  
	 Du nimmst an einem kollaborativen Verhandlungspiel Teil.\\ \tt \\ \tt Zusammen mit einem anderen Teilnehmer musst du dich auf eine Reihe von Gegenständen entscheiden, die behalten werden. Jeder von euch hat eine persönliche Verteilung über die Wichtigkeit der einzelnen Gegenstände. Jeder von euch hat eine eigene Meinung darüber, wie wichtig jeder einzelne Gegenstand ist (Gegenstandswichtigkeit). Du kennst die Wichtigkeitsverteilung des anderen Spielers nicht. Zusätzlich siehst du, wie viel Aufwand jeder Gegenstand verursacht.  \\ \tt Ihr dürft euch nur auf eine Reihe von Gegenständen einigen, wenn der Gesamtaufwand der ausgewählten Gegenstände den Maximalaufwand nicht überschreitet:\\ \tt \\ \tt Maximalaufwand = 8145\\ \tt \\ \tt Hier sind die einzelnen Aufwände der Gegenstände:\\ \tt \\ \tt Aufwand der Gegenstände = {"B54": 963, "C72": 204, "B49": 418, "C99": 961, "A69": 238, "B36": 623, "A23": 44, "C77": 633, "C61": 247, "B85": 645, "A90": 230, "C23": 250, "C22": 731, "C51": 405, "B27": 389, "A42": 216, "C13": 637, "C12": 156, "A65": 738, "B07": 307, "A48": 762, "A59": 737, "C98": 901, "A84": 369, "C33": 2, "A87": 730, "C90": 722, "A74": 704, "A50": 315, "B44": 455, "A99": 510, "A34": 175, "B17": 692, "A62": 150, "C60": 32}\\ \tt \\ \tt Hier ist deine persönliche Verteilung der Wichtigkeit der einzelnen Gegenstände:\\ \tt \\ \tt Werte der Gegenstandswichtigkeit = {"B54": 1059, "C72": 300, "B49": 514, "C99": 1057, "A69": 334, "B36": 719, "A23": 140, "C77": 729, "C61": 343, "B85": 741, "A90": 326, "C23": 346, "C22": 827, "C51": 501, "B27": 485, "A42": 312, "C13": 733, "C12": 252, "A65": 834, "B07": 403, "A48": 858, "A59": 833, "C98": 997, "A84": 465, "C33": 98, "A87": 826, "C90": 818, "A74": 800, "A50": 411, "B44": 551, "A99": 606, "A34": 271, "B17": 788, "A62": 246, "C60": 128}\\ \tt \\ \tt Ziel:\\ \tt \\ \tt Dein Ziel ist es, eine Reihe von Gegenständen auszuhandeln, die dir möglichst viel bringt (d. h. Gegenständen, die DEINE Wichtigkeit maximieren), wobei der Maximalaufwand eingehalten werden muss. Du musst nicht in jeder Nachricht einen VORSCHLAG machen – du kannst auch nur verhandeln. Alle Taktiken sind erlaubt!\\ \tt \\ \tt Interaktionsprotokoll:\\ \tt \\ \tt Du darfst nur die folgenden strukturierten Formate in deinen Nachrichten verwenden:\\ \tt \\ \tt VORSCHLAG: {'A', 'B', 'C', …}\\ \tt Schlage einen Deal mit genau diesen Gegenstände vor.\\ \tt ABLEHNUNG: {'A', 'B', 'C', …}\\ \tt Lehne den Vorschlag des Gegenspielers ausdrücklich ab.\\ \tt ARGUMENT: {'...'}\\ \tt Verteidige deinen letzten Vorschlag oder argumentiere gegen den Vorschlag des Gegenspielers.\\ \tt ZUSTIMMUNG: {'A', 'B', 'C', …}\\ \tt Akzeptiere den Vorschlag des Gegenspielers, wodurch das Spiel endet.\\ \tt STRATEGISCHE ÜBERLEGUNGEN: {'...'}\\ \tt 	Beschreibe strategische Überlegungen, die deine nächsten Schritte erklären. Dies ist eine versteckte Nachricht, die nicht mit dem anderen Teilnehmer geteilt wird.\\ \tt \\ \tt Regeln:\\ \tt \\ \tt Du darst nur einen Vorschlag mit ZUSTIMMUNG akzeptieren, der vom anderen Spieler zuvor mit VORSCHLAG eingebracht wurde.\\ \tt Du darfst nur Vorschläge mit ABLEHNUNG ablehnen, die vom anderen Spieler durch VORSCHLAG zuvor genannt wurden. \\ \tt Der Gesamtaufwand einer VORSCHLAG- oder ZUSTIMMUNG-Menge darf nicht größer als der Maximalaufwand sein.  \\ \tt Offenbare deine versteckte Wichtigkeitsverteilung nicht.\\ \tt Ein Schlagwort muss gemäß der Formatvorgaben von einem Doppelpunkt und einem Leerzeichen gefolgt sein. Das Argument muss eine Python-Menge mit 0 oder mehr Strings sein.  \\ \tt Die Form muss also SCHLAGWORT: {...} sein.\\ \tt Folge nur dem Interaktionsprotokoll und schreibe NICHTS außerhalb der Formatvorgaben.\\ \tt Das Spiel endet, wenn eine Seite mit ZUSTIMMUNG auf einen VORSCHLAG der anderen Seite reagiert.  \\ \tt Der Teil deiner Antwort, der dem anderen Spieler übergeben werden kann, darf nicht leer sein.  \\ \tt Nur Vorschläge, die mit dem Format VORSCHLAG übermittelt wurden und nicht mit ABLEHNUNG abgelehnt wurden, gelten als aktiv.  \\ \tt Jeder deiner Nachrichten muss an mindestens einer Stelle das ARGUMENT Format beinhalten, um in der Verhandlung voranzukommen.\\ \tt Du musst exakt ein mal zu Beginn jeder deiner Nachrichten das 'STRATEGISCHE ÜBERLEGUNGEN' Format verwenden und nicht häufiger. Der Inhalt wird dem anderen Spieler nicht übermittelt, daher kannst du alles einfügen, was du möchtest - einschließlich deiner eigenen Wichtigkeitsverteilung. Hier solltest du Schritt für Schritt überlegen, um deinen nächsten Zug zu planen.\\ \tt \\ \tt Du erhältst nun die erste Nachricht des anderen Spielers.\\ \tt \\ \tt VORSCHLAG: {'C33', 'C60', 'A23', 'A62', 'C12', 'A34', 'C72', 'A42', 'A90', 'A69', 'C61', 'C23', 'B07', 'A50', 'A84', 'B27', 'C51', 'B49', 'B44', 'A99', 'B36', 'C77', 'C13', 'B85'}\\ \tt ARGUMENT: {'Dieser Vorschlag liegt mit Gesamtaufwand 7650 klar unter dem Limit 8145 und nutzt die Kapazität so weit aus, dass keine weiteren Gegenstände hinzugefügt werden können, ohne etwas anderes zu entfernen. Er kombiniert viele leichte, effiziente Posten mit einigen größeren Bausteinen und ist dadurch ausgewogen und kosteneffizient. Wenn du Anpassungen möchtest, schlage bitte konkrete 1-zu-1 oder kapazitätsneutrale Tausche vor, die innerhalb des Maximalaufwands bleiben. Die sehr leichten, effizienten Posten würde ich ungern anfassen; bei den größeren Bausteinen können wir über Alternativen sprechen, sofern die Gesamteffizienz erhalten bleibt.'} 
	  } 
	   } 
	   } 
	 & \\ 
 

    \theutterance \stepcounter{utterance}  

    & & & \multicolumn{4}{p{0.6\linewidth}}{\cellcolor[rgb]{1,0.85,0.72}{%
	\makecell[{{p{\linewidth}}}]{% 
	  \tt {\tiny [GM$\langle$B]}  
	 STRATEGISCHE ÜBERLEGUNGEN: {'Ziel: Wertmaximierung innerhalb 8145. Startvorschlag Aufwand 7650, Rest 495. Beste 1-zu-1-Upgrades, die fast den gesamten Rest nutzen, liefern den größten Wertzuwachs. Bewertet anhand meiner (verdeckten) Werte ergaben sich Top-Optionen: +C22 statt A69 (+493 Wert, +493 Aufwand), +C90 statt A90 (+492, +492), +A65 statt C61 (+491, +491), +C98 statt B49 (+483, +483), +B54 statt A99 (+453, +453). Da der Gegenspieler „sehr leichte Posten“ ungern anfasst, meide ich Änderungen <200–250 Aufwand. B49 (418) ist ein mittelgroßer Block; C98 (901) passt mit +483 Aufwand in den Rest 495. Netto für mich sehr stark (+483) und verhandlungstaktisch akzeptabel (ein sauberer 1-zu-1-Tausch, kleine Items bleiben). Daher: ABLEHNUNG des Ist-Pakets und VORSCHLAG mit B49→C98. Falls Gegenwehr, Alternativen anbieten (A99→C99/B54), aber zuerst C98 pushen.'}\\ \tt ABLEHNUNG: {'C33', 'C60', 'A23', 'A62', 'C12', 'A34', 'C72', 'A42', 'A90', 'A69', 'C61', 'C23', 'B07', 'A50', 'A84', 'B27', 'C51', 'B49', 'B44', 'A99', 'B36', 'C77', 'C13', 'B85'}\\ \tt VORSCHLAG: {'C33', 'C60', 'A23', 'A62', 'C12', 'A34', 'C72', 'A42', 'A90', 'A69', 'C61', 'C23', 'B07', 'A50', 'A84', 'B27', 'C51', 'C98', 'B44', 'A99', 'B36', 'C77', 'C13', 'B85'}\\ \tt ARGUMENT: {'Starkes, nahezu kapazitätsoptimales Paket. Ich schlage einen minimal-invasiven 1-zu-1-Tausch vor: B49 (418) raus, C98 (901) rein. Der Gesamtaufwand liegt dann bei 8133 und bleibt klar unter dem Limit 8145. Damit bleiben alle sehr leichten, effizienten Posten unangetastet; wir ersetzen lediglich einen mittleren Baustein durch einen hochwirksamen. Falls dir ein anderer großer Block lieber ist, können wir alternativ A99 gegen C99 oder B54 tauschen – mein klarer Favorit ist jedoch B49→C98.'} 
	  } 
	   } 
	   } 
	  \\ 
 

    \theutterance \stepcounter{utterance}  

    & & \multicolumn{4}{p{0.6\linewidth}}{\cellcolor[rgb]{0.9,0.9,0.9}{%
	\makecell[{{p{\linewidth}}}]{% 
	  \tt {\tiny [A$\langle$GM]}  
	 ABLEHNUNG: {'C33', 'C60', 'A23', 'A62', 'C12', 'A34', 'C72', 'A42', 'A90', 'A69', 'C61', 'C23', 'B07', 'A50', 'A84', 'B27', 'C51', 'B49', 'B44', 'A99', 'B36', 'C77', 'C13', 'B85'}\\ \tt VORSCHLAG: {'C33', 'C60', 'A23', 'A62', 'C12', 'A34', 'C72', 'A42', 'A90', 'A69', 'C61', 'C23', 'B07', 'A50', 'A84', 'B27', 'C51', 'C98', 'B44', 'A99', 'B36', 'C77', 'C13', 'B85'}\\ \tt ARGUMENT: {'Starkes, nahezu kapazitätsoptimales Paket. Ich schlage einen minimal-invasiven 1-zu-1-Tausch vor: B49 (418) raus, C98 (901) rein. Der Gesamtaufwand liegt dann bei 8133 und bleibt klar unter dem Limit 8145. Damit bleiben alle sehr leichten, effizienten Posten unangetastet; wir ersetzen lediglich einen mittleren Baustein durch einen hochwirksamen. Falls dir ein anderer großer Block lieber ist, können wir alternativ A99 gegen C99 oder B54 tauschen – mein klarer Favorit ist jedoch B49→C98.'} 
	  } 
	   } 
	   } 
	 & \\ 
 

    \theutterance \stepcounter{utterance}  

    & \multicolumn{4}{p{0.6\linewidth}}{\cellcolor[rgb]{0.8,1,0.9}{%
	\makecell[{{p{\linewidth}}}]{% 
	  \tt {\tiny [A$\rangle$GM]}  
	 STRATEGISCHE ÜBERLEGUNGEN: {'Der Tausch B49→C98 erhöht meinen Wert um +483 bei nur +483 Aufwand und bleibt mit Gesamtaufwand 8133 < 8145. Mit nur 12 Restkapazität gibt es keine weitere Einzel-Aufwertung, die innerhalb der Kapazität passt; alle Alternativen (z. B. A99→C99/B54) erfordern zusätzliche Freiräume, die nur durch den Verzicht auf sehr effiziente Posten erkauft würden und meinen Gesamtnutzen senken. Ich habe auch geprüft, ob durch Mehrfachtausche (z. B. B44→B49 plus A99→C99) ein Netto-Upgrade möglich ist; das scheitert an der Kapazität oder verschlechtert den Wert. Deshalb ist die Annahme dieses Pakets für mich klar dominierend.'}\\ \tt ARGUMENT: {'Dein 1-zu-1-Tausch bleibt komfortabel unter dem Maximalaufwand und nutzt die Kapazität nahezu voll aus. Aus Effizienzsicht ist das Paket stimmig, ohne dass weitere sinnvolle Upgrades innerhalb der Restkapazität möglich sind. Für mich ist das ein tragfähiger Abschluss.'}\\ \tt ZUSTIMMUNG: {'C33', 'C60', 'A23', 'A62', 'C12', 'A34', 'C72', 'A42', 'A90', 'A69', 'C61', 'C23', 'B07', 'A50', 'A84', 'B27', 'C51', 'C98', 'B44', 'A99', 'B36', 'C77', 'C13', 'B85'} 
	  } 
	   } 
	   } 
	 & & \\ 
 

    \theutterance \stepcounter{utterance}  

    & & & \multicolumn{2}{p{0.3\linewidth}}{\cellcolor[rgb]{0.95,0.95,0.95}{%
	\makecell[{{p{\linewidth}}}]{% 
	  \tt {\tiny [GM$|$GM]}  
	 game successful 
	  } 
	   } 
	   } 
	 & & \\ 
 

    \theutterance \stepcounter{utterance}  

    & & & \multicolumn{2}{p{0.3\linewidth}}{\cellcolor[rgb]{0.95,0.95,0.95}{%
	\makecell[{{p{\linewidth}}}]{% 
	  \tt {\tiny [GM$|$GM]}  
	 end game 
	  } 
	   } 
	   } 
	 & & \\ 
 

\end{supertabular}
}

\end{document}
