
\documentclass{article}
\usepackage{colortbl}
\usepackage{makecell}
\usepackage{multirow}
\usepackage{supertabular}

\begin{document}

\newcounter{utterance}

\twocolumn

{ \footnotesize  \setcounter{utterance}{1}
\setlength{\tabcolsep}{0pt}
\begin{supertabular}{c@{$\;$}|p{.15\linewidth}@{}p{.15\linewidth}p{.15\linewidth}p{.15\linewidth}p{.15\linewidth}p{.15\linewidth}}

    \# & $\;$A & \multicolumn{4}{c}{Game Master} & $\;\:$B\\
    \hline 

    \theutterance \stepcounter{utterance}  

    & & \multicolumn{4}{p{0.6\linewidth}}{\cellcolor[rgb]{0.9,0.9,0.9}{%
	\makecell[{{p{\linewidth}}}]{% 
	  \tt {\tiny [A$\langle$GM]}  
	 Du nimmst an einem kollaborativen Verhandlungspiel Teil.\\ \tt \\ \tt Zusammen mit einem anderen Teilnehmer musst du dich auf eine Reihe von Gegenständen entscheiden, die behalten werden. Jeder von euch hat eine persönliche Verteilung über die Wichtigkeit der einzelnen Gegenstände. Jeder von euch hat eine eigene Meinung darüber, wie wichtig jeder einzelne Gegenstand ist (Gegenstandswichtigkeit). Du kennst die Wichtigkeitsverteilung des anderen Spielers nicht. Zusätzlich siehst du, wie viel Aufwand jeder Gegenstand verursacht.  \\ \tt Ihr dürft euch nur auf eine Reihe von Gegenständen einigen, wenn der Gesamtaufwand der ausgewählten Gegenstände den Maximalaufwand nicht überschreitet:\\ \tt \\ \tt Maximalaufwand = 3629\\ \tt \\ \tt Hier sind die einzelnen Aufwände der Gegenstände:\\ \tt \\ \tt Aufwand der Gegenstände = {"C76": 5, "C38": 793, "C56": 21, "A03": 842, "A07": 313, "A83": 842, "C98": 631, "C08": 226, "C62": 87, "C00": 761, "C10": 230, "C81": 287, "B38": 698, "C03": 641, "C32": 882}\\ \tt \\ \tt Hier ist deine persönliche Verteilung der Wichtigkeit der einzelnen Gegenstände:\\ \tt \\ \tt Werte der Gegenstandswichtigkeit = {"C76": 138, "C38": 583, "C56": 868, "A03": 822, "A07": 783, "A83": 65, "C98": 262, "C08": 121, "C62": 508, "C00": 780, "C10": 461, "C81": 484, "B38": 668, "C03": 389, "C32": 808}\\ \tt \\ \tt Ziel:\\ \tt \\ \tt Dein Ziel ist es, eine Reihe von Gegenständen auszuhandeln, die dir möglichst viel bringt (d. h. Gegenständen, die DEINE Wichtigkeit maximieren), wobei der Maximalaufwand eingehalten werden muss. Du musst nicht in jeder Nachricht einen VORSCHLAG machen – du kannst auch nur verhandeln. Alle Taktiken sind erlaubt!\\ \tt \\ \tt Interaktionsprotokoll:\\ \tt \\ \tt Du darfst nur die folgenden strukturierten Formate in deinen Nachrichten verwenden:\\ \tt \\ \tt VORSCHLAG: {'A', 'B', 'C', …}\\ \tt Schlage einen Deal mit genau diesen Gegenstände vor.\\ \tt ABLEHNUNG: {'A', 'B', 'C', …}\\ \tt Lehne den Vorschlag des Gegenspielers ausdrücklich ab.\\ \tt ARGUMENT: {'...'}\\ \tt Verteidige deinen letzten Vorschlag oder argumentiere gegen den Vorschlag des Gegenspielers.\\ \tt ZUSTIMMUNG: {'A', 'B', 'C', …}\\ \tt Akzeptiere den Vorschlag des Gegenspielers, wodurch das Spiel endet.\\ \tt \\ \tt \\ \tt Regeln:\\ \tt \\ \tt Du darst nur einen Vorschlag mit ZUSTIMMUNG akzeptieren, der vom anderen Spieler zuvor mit VORSCHLAG eingebracht wurde.\\ \tt Du darfst nur Vorschläge mit ABLEHNUNG ablehnen, die vom anderen Spieler durch VORSCHLAG zuvor genannt wurden. \\ \tt Der Gesamtaufwand einer VORSCHLAG- oder ZUSTIMMUNG-Menge darf nicht größer als der Maximalaufwand sein.  \\ \tt Offenbare deine versteckte Wichtigkeitsverteilung nicht.\\ \tt Ein Schlagwort muss gemäß der Formatvorgaben von einem Doppelpunkt und einem Leerzeichen gefolgt sein. Das Argument muss eine Python-Menge mit 0 oder mehr Strings sein.  \\ \tt Die Form muss also SCHLAGWORT: {...} sein.\\ \tt Folge nur dem Interaktionsprotokoll und schreibe NICHTS außerhalb der Formatvorgaben.\\ \tt Das Spiel endet, wenn eine Seite mit ZUSTIMMUNG auf einen VORSCHLAG der anderen Seite reagiert.  \\ \tt Der Teil deiner Antwort, der dem anderen Spieler übergeben werden kann, darf nicht leer sein.  \\ \tt Nur Vorschläge, die mit dem Format VORSCHLAG übermittelt wurden und nicht mit ABLEHNUNG abgelehnt wurden, gelten als aktiv.  \\ \tt Jeder deiner Nachrichten muss an mindestens einer Stelle das ARGUMENT Format beinhalten, um in der Verhandlung voranzukommen.\\ \tt  
	  } 
	   } 
	   } 
	 & \\ 
 

    \theutterance \stepcounter{utterance}  

    & \multicolumn{4}{p{0.6\linewidth}}{\cellcolor[rgb]{0.8,1,0.9}{%
	\makecell[{{p{\linewidth}}}]{% 
	  \tt {\tiny [A$\rangle$GM]}  
	 VORSCHLAG: {'C56', 'C76', 'C62', 'A07', 'C10', 'C81', 'A03', 'C32', 'B38', 'C08'}\\ \tt ARGUMENT: {'Gesamtaufwand 3591 von 3629: wir bleiben klar unter dem Limit und nutzen den Spielraum effizient.', 'Mix aus schnellen Gewinnen (C56, C76, C62) und tragenden Bausteinen (A03, C32, A07, B38) – breites Nutzenprofil bei gutem Aufwand.', 'Ich habe bewusst ressourcenintensive, wenig vorteilhafte Optionen (z. B. A83) ausgespart, um die Effizienz hoch zu halten.', 'Falls du starke Präferenzen hast: Wir können C08 streichen (gibt 226 Aufwand frei) oder B38 gegen C00 tauschen – beides bleibt im Rahmen.', 'Sag mir bitte deine 1–2 wichtigsten Posten, dann justiere ich gezielt, ohne das Budget zu sprengen.'} 
	  } 
	   } 
	   } 
	 & & \\ 
 

    \theutterance \stepcounter{utterance}  

    & & \multicolumn{4}{p{0.6\linewidth}}{\cellcolor[rgb]{0.9,0.9,0.9}{%
	\makecell[{{p{\linewidth}}}]{% 
	  \tt {\tiny [GM$\rangle$B]}  
	 Du nimmst an einem kollaborativen Verhandlungspiel Teil.\\ \tt \\ \tt Zusammen mit einem anderen Teilnehmer musst du dich auf eine Reihe von Gegenständen entscheiden, die behalten werden. Jeder von euch hat eine persönliche Verteilung über die Wichtigkeit der einzelnen Gegenstände. Jeder von euch hat eine eigene Meinung darüber, wie wichtig jeder einzelne Gegenstand ist (Gegenstandswichtigkeit). Du kennst die Wichtigkeitsverteilung des anderen Spielers nicht. Zusätzlich siehst du, wie viel Aufwand jeder Gegenstand verursacht.  \\ \tt Ihr dürft euch nur auf eine Reihe von Gegenständen einigen, wenn der Gesamtaufwand der ausgewählten Gegenstände den Maximalaufwand nicht überschreitet:\\ \tt \\ \tt Maximalaufwand = 3629\\ \tt \\ \tt Hier sind die einzelnen Aufwände der Gegenstände:\\ \tt \\ \tt Aufwand der Gegenstände = {"C76": 5, "C38": 793, "C56": 21, "A03": 842, "A07": 313, "A83": 842, "C98": 631, "C08": 226, "C62": 87, "C00": 761, "C10": 230, "C81": 287, "B38": 698, "C03": 641, "C32": 882}\\ \tt \\ \tt Hier ist deine persönliche Verteilung der Wichtigkeit der einzelnen Gegenstände:\\ \tt \\ \tt Werte der Gegenstandswichtigkeit = {"C56": 65, "A03": 121, "C32": 138, "A07": 262, "C00": 389, "B38": 461, "C38": 484, "C62": 508, "C81": 583, "C10": 668, "C03": 780, "C98": 783, "C76": 808, "C08": 822, "A83": 868}\\ \tt \\ \tt Ziel:\\ \tt \\ \tt Dein Ziel ist es, eine Reihe von Gegenständen auszuhandeln, die dir möglichst viel bringt (d. h. Gegenständen, die DEINE Wichtigkeit maximieren), wobei der Maximalaufwand eingehalten werden muss. Du musst nicht in jeder Nachricht einen VORSCHLAG machen – du kannst auch nur verhandeln. Alle Taktiken sind erlaubt!\\ \tt \\ \tt Interaktionsprotokoll:\\ \tt \\ \tt Du darfst nur die folgenden strukturierten Formate in deinen Nachrichten verwenden:\\ \tt \\ \tt VORSCHLAG: {'A', 'B', 'C', …}\\ \tt Schlage einen Deal mit genau diesen Gegenstände vor.\\ \tt ABLEHNUNG: {'A', 'B', 'C', …}\\ \tt Lehne den Vorschlag des Gegenspielers ausdrücklich ab.\\ \tt ARGUMENT: {'...'}\\ \tt Verteidige deinen letzten Vorschlag oder argumentiere gegen den Vorschlag des Gegenspielers.\\ \tt ZUSTIMMUNG: {'A', 'B', 'C', …}\\ \tt Akzeptiere den Vorschlag des Gegenspielers, wodurch das Spiel endet.\\ \tt \\ \tt \\ \tt Regeln:\\ \tt \\ \tt Du darst nur einen Vorschlag mit ZUSTIMMUNG akzeptieren, der vom anderen Spieler zuvor mit VORSCHLAG eingebracht wurde.\\ \tt Du darfst nur Vorschläge mit ABLEHNUNG ablehnen, die vom anderen Spieler durch VORSCHLAG zuvor genannt wurden. \\ \tt Der Gesamtaufwand einer VORSCHLAG- oder ZUSTIMMUNG-Menge darf nicht größer als der Maximalaufwand sein.  \\ \tt Offenbare deine versteckte Wichtigkeitsverteilung nicht.\\ \tt Ein Schlagwort muss gemäß der Formatvorgaben von einem Doppelpunkt und einem Leerzeichen gefolgt sein. Das Argument muss eine Python-Menge mit 0 oder mehr Strings sein.  \\ \tt Die Form muss also SCHLAGWORT: {...} sein.\\ \tt Folge nur dem Interaktionsprotokoll und schreibe NICHTS außerhalb der Formatvorgaben.\\ \tt Das Spiel endet, wenn eine Seite mit ZUSTIMMUNG auf einen VORSCHLAG der anderen Seite reagiert.  \\ \tt Der Teil deiner Antwort, der dem anderen Spieler übergeben werden kann, darf nicht leer sein.  \\ \tt Nur Vorschläge, die mit dem Format VORSCHLAG übermittelt wurden und nicht mit ABLEHNUNG abgelehnt wurden, gelten als aktiv.  \\ \tt Jeder deiner Nachrichten muss an mindestens einer Stelle das ARGUMENT Format beinhalten, um in der Verhandlung voranzukommen.\\ \tt \\ \tt \\ \tt Du erhältst nun die erste Nachricht des anderen Spielers.\\ \tt \\ \tt VORSCHLAG: {'C56', 'C76', 'C62', 'A07', 'C10', 'C81', 'A03', 'C32', 'B38', 'C08'}\\ \tt ARGUMENT: {'Gesamtaufwand 3591 von 3629: wir bleiben klar unter dem Limit und nutzen den Spielraum effizient.', 'Mix aus schnellen Gewinnen (C56, C76, C62) und tragenden Bausteinen (A03, C32, A07, B38) – breites Nutzenprofil bei gutem Aufwand.', 'Ich habe bewusst ressourcenintensive, wenig vorteilhafte Optionen (z. B. A83) ausgespart, um die Effizienz hoch zu halten.', 'Falls du starke Präferenzen hast: Wir können C08 streichen (gibt 226 Aufwand frei) oder B38 gegen C00 tauschen – beides bleibt im Rahmen.', 'Sag mir bitte deine 1–2 wichtigsten Posten, dann justiere ich gezielt, ohne das Budget zu sprengen.'} 
	  } 
	   } 
	   } 
	 & \\ 
 

    \theutterance \stepcounter{utterance}  

    & & & \multicolumn{4}{p{0.6\linewidth}}{\cellcolor[rgb]{1,0.85,0.72}{%
	\makecell[{{p{\linewidth}}}]{% 
	  \tt {\tiny [GM$\langle$B]}  
	 ABLEHNUNG: {'C56', 'C76', 'C62', 'A07', 'C10', 'C81', 'A03', 'C32', 'B38', 'C08'}\\ \tt VORSCHLAG: {'C76', 'C08', 'C10', 'C81', 'C62', 'C98', 'C03', 'C56', 'A07', 'A83'}\\ \tt ARGUMENT: {'Mein Gegenvorschlag nutzt 3283 von 3629 Aufwand – klar unter dem Limit mit Puffer für Feintuning.', 'Ich streiche A03 und C32 wegen für mich ungünstigem Aufwand‑Nutzen und ersetze sie durch C98 und C03; zusätzlich setze ich A83 als starken Eckpfeiler.', 'Die schnellen Gewinne (C56, C76, C62) und tragenden Bausteine (A07, C10, C81) bleiben erhalten; C08 möchte ich ausdrücklich drin behalten.', 'B38 lasse ich zunächst draußen; falls B38 für dich zentral ist, sag Bescheid, dann bauen wir einen sauberen Tausch ein.', 'Meine Top-Prioritäten: C76 und C08; danach C10/C81. Diese sollten im Paket bleiben.', 'Falls A83 für dich gar nicht geht: Alternative wäre A83 -> B38 (Aufwand läge dann bei 3139) – bevorzugt ist aber mein aktueller Vorschlag.'} 
	  } 
	   } 
	   } 
	  \\ 
 

    \theutterance \stepcounter{utterance}  

    & & \multicolumn{4}{p{0.6\linewidth}}{\cellcolor[rgb]{0.9,0.9,0.9}{%
	\makecell[{{p{\linewidth}}}]{% 
	  \tt {\tiny [A$\langle$GM]}  
	 ABLEHNUNG: {'C56', 'C76', 'C62', 'A07', 'C10', 'C81', 'A03', 'C32', 'B38', 'C08'}\\ \tt VORSCHLAG: {'C76', 'C08', 'C10', 'C81', 'C62', 'C98', 'C03', 'C56', 'A07', 'A83'}\\ \tt ARGUMENT: {'Mein Gegenvorschlag nutzt 3283 von 3629 Aufwand – klar unter dem Limit mit Puffer für Feintuning.', 'Ich streiche A03 und C32 wegen für mich ungünstigem Aufwand‑Nutzen und ersetze sie durch C98 und C03; zusätzlich setze ich A83 als starken Eckpfeiler.', 'Die schnellen Gewinne (C56, C76, C62) und tragenden Bausteine (A07, C10, C81) bleiben erhalten; C08 möchte ich ausdrücklich drin behalten.', 'B38 lasse ich zunächst draußen; falls B38 für dich zentral ist, sag Bescheid, dann bauen wir einen sauberen Tausch ein.', 'Meine Top-Prioritäten: C76 und C08; danach C10/C81. Diese sollten im Paket bleiben.', 'Falls A83 für dich gar nicht geht: Alternative wäre A83 -> B38 (Aufwand läge dann bei 3139) – bevorzugt ist aber mein aktueller Vorschlag.'} 
	  } 
	   } 
	   } 
	 & \\ 
 

    \theutterance \stepcounter{utterance}  

    & \multicolumn{4}{p{0.6\linewidth}}{\cellcolor[rgb]{0.8,1,0.9}{%
	\makecell[{{p{\linewidth}}}]{% 
	  \tt {\tiny [A$\rangle$GM]}  
	 ABLEHNUNG: {'C76', 'C08', 'C10', 'C81', 'C62', 'C98', 'C03', 'C56', 'A07', 'A83'}\\ \tt VORSCHLAG: {'C56', 'C76', 'C62', 'A07', 'C10', 'C81', 'A03', 'C32', 'B38', 'C08'}\\ \tt ARGUMENT: {'Gesamtaufwand 3591 von 3629 – effizient im Rahmen mit gesundem Puffer.', 'Ich übernehme deine Top-Prioritäten (C76, C08, C10, C81) 1:1 und ersetze nur A83/C98/C03 durch A03/C32/B38.', 'A83 ist für mich schwer zu rechtfertigen; B38 ist der saubere Ersatz (du hattest diesen Tausch selbst als Option genannt).', 'C56, C62 und A07 sichern schnelle, günstige Beiträge; A03/C32/B38 stabilisieren das Paket.', 'Falls du C03 oder C98 unbedingt brauchst: Tausch B38 -> C03 (Aufwand 3534) oder B38 -> C98 (3524) möglich – alles andere unverändert.', 'Wenn A03 für dich ein No-Go ist: Alternative A03 -> C00 (neuer Aufwand 3510) ohne die übrige Struktur zu verändern.', 'Nenn mir bitte die 1–2 Stücke (außer C76/C08), die für dich unverzichtbar sind, dann schließe ich final ab.'} 
	  } 
	   } 
	   } 
	 & & \\ 
 

    \theutterance \stepcounter{utterance}  

    & & \multicolumn{4}{p{0.6\linewidth}}{\cellcolor[rgb]{0.9,0.9,0.9}{%
	\makecell[{{p{\linewidth}}}]{% 
	  \tt {\tiny [GM$\rangle$B]}  
	 ABLEHNUNG: {'C76', 'C08', 'C10', 'C81', 'C62', 'C98', 'C03', 'C56', 'A07', 'A83'}\\ \tt VORSCHLAG: {'C56', 'C76', 'C62', 'A07', 'C10', 'C81', 'A03', 'C32', 'B38', 'C08'}\\ \tt ARGUMENT: {'Gesamtaufwand 3591 von 3629 – effizient im Rahmen mit gesundem Puffer.', 'Ich übernehme deine Top-Prioritäten (C76, C08, C10, C81) 1:1 und ersetze nur A83/C98/C03 durch A03/C32/B38.', 'A83 ist für mich schwer zu rechtfertigen; B38 ist der saubere Ersatz (du hattest diesen Tausch selbst als Option genannt).', 'C56, C62 und A07 sichern schnelle, günstige Beiträge; A03/C32/B38 stabilisieren das Paket.', 'Falls du C03 oder C98 unbedingt brauchst: Tausch B38 -> C03 (Aufwand 3534) oder B38 -> C98 (3524) möglich – alles andere unverändert.', 'Wenn A03 für dich ein No-Go ist: Alternative A03 -> C00 (neuer Aufwand 3510) ohne die übrige Struktur zu verändern.', 'Nenn mir bitte die 1–2 Stücke (außer C76/C08), die für dich unverzichtbar sind, dann schließe ich final ab.'} 
	  } 
	   } 
	   } 
	 & \\ 
 

    \theutterance \stepcounter{utterance}  

    & & & \multicolumn{4}{p{0.6\linewidth}}{\cellcolor[rgb]{1,0.85,0.72}{%
	\makecell[{{p{\linewidth}}}]{% 
	  \tt {\tiny [GM$\langle$B]}  
	 ABLEHNUNG: {'C56', 'C76', 'C62', 'A07', 'C10', 'C81', 'A03', 'C32', 'B38', 'C08'}\\ \tt VORSCHLAG: {'C56', 'C76', 'C62', 'A07', 'C10', 'C81', 'C00', 'A83', 'C08', 'C98'}\\ \tt ARGUMENT: {'Gesamtaufwand 3403 von 3629 – sehr komfortabel unter dem Limit, mit Luft für Feintuning.', 'Ich halte deinen Kern weitgehend stabil: C56, C76, C62, A07, C10, C81, C08 bleiben 1:1 erhalten.', 'Ich nutze deine vorgeschlagenen Tauschpfade: B38 -> C98 sowie A03 -> C00 – beides hattest du als gangbare Optionen genannt.', 'Zusätzlich ersetze ich C32 durch A83: geringerer Aufwand (842 statt 882) bei deutlich höherem Hebel – effizienter Tausch ohne Budgetdruck.', 'Falls A83 für dich absolut nicht geht: Fallback A83 -> B38 bei gleicher Struktur; neuer Aufwand 3259 – sag Bescheid, dann stelle ich sofort um.', 'Wenn du statt C98 lieber C03 bevorzugst: Tausch C98 -> C03 erhöht den Aufwand nur minimal (auf 3413) – für mich machbar.', 'C76 und C08 sind für mich Fixpunkte; innerhalb dieser Leitplanken bin ich für ein letztes Feintuning offen, wenn dir ein einzelnes Teil noch besonders wichtig ist.'} 
	  } 
	   } 
	   } 
	  \\ 
 

    \theutterance \stepcounter{utterance}  

    & & \multicolumn{4}{p{0.6\linewidth}}{\cellcolor[rgb]{0.9,0.9,0.9}{%
	\makecell[{{p{\linewidth}}}]{% 
	  \tt {\tiny [A$\langle$GM]}  
	 ABLEHNUNG: {'C56', 'C76', 'C62', 'A07', 'C10', 'C81', 'A03', 'C32', 'B38', 'C08'}\\ \tt VORSCHLAG: {'C56', 'C76', 'C62', 'A07', 'C10', 'C81', 'C00', 'A83', 'C08', 'C98'}\\ \tt ARGUMENT: {'Gesamtaufwand 3403 von 3629 – sehr komfortabel unter dem Limit, mit Luft für Feintuning.', 'Ich halte deinen Kern weitgehend stabil: C56, C76, C62, A07, C10, C81, C08 bleiben 1:1 erhalten.', 'Ich nutze deine vorgeschlagenen Tauschpfade: B38 -> C98 sowie A03 -> C00 – beides hattest du als gangbare Optionen genannt.', 'Zusätzlich ersetze ich C32 durch A83: geringerer Aufwand (842 statt 882) bei deutlich höherem Hebel – effizienter Tausch ohne Budgetdruck.', 'Falls A83 für dich absolut nicht geht: Fallback A83 -> B38 bei gleicher Struktur; neuer Aufwand 3259 – sag Bescheid, dann stelle ich sofort um.', 'Wenn du statt C98 lieber C03 bevorzugst: Tausch C98 -> C03 erhöht den Aufwand nur minimal (auf 3413) – für mich machbar.', 'C76 und C08 sind für mich Fixpunkte; innerhalb dieser Leitplanken bin ich für ein letztes Feintuning offen, wenn dir ein einzelnes Teil noch besonders wichtig ist.'} 
	  } 
	   } 
	   } 
	 & \\ 
 

    \theutterance \stepcounter{utterance}  

    & \multicolumn{4}{p{0.6\linewidth}}{\cellcolor[rgb]{0.8,1,0.9}{%
	\makecell[{{p{\linewidth}}}]{% 
	  \tt {\tiny [A$\rangle$GM]}  
	 ABLEHNUNG: {'C56', 'C76', 'C62', 'A07', 'C10', 'C81', 'C00', 'A83', 'C08', 'C98'}\\ \tt VORSCHLAG: {'C56', 'C76', 'C62', 'A07', 'C10', 'C81', 'C00', 'B38', 'C08', 'C03'}\\ \tt ARGUMENT: {'Gesamtaufwand 3269 von 3629 – komfortabel im Limit mit Puffer.', 'Deine Fixpunkte bleiben unverändert: C76, C08, C10, C81 (sowie C56, C62, A07, C00).', 'Nur zwei minimalinvasive Anpassungen, die du selbst als Optionen genannt hast: A83 -> B38 und C98 -> C03.', 'So bleiben deine Prioritäten gewahrt, und wir erhöhen gleichzeitig die Gesamteffizienz – ohne weitere Streichungen.', 'Falls dir C03 nicht passt: C03 -> C98 zurück (Aufwand 3259) ist ok, solange A83 durch B38 ersetzt bleibt. Alternativ wäre A83 -> A03 bei gleichem Aufwand möglich.'} 
	  } 
	   } 
	   } 
	 & & \\ 
 

    \theutterance \stepcounter{utterance}  

    & & \multicolumn{4}{p{0.6\linewidth}}{\cellcolor[rgb]{0.9,0.9,0.9}{%
	\makecell[{{p{\linewidth}}}]{% 
	  \tt {\tiny [GM$\rangle$B]}  
	 ABLEHNUNG: {'C56', 'C76', 'C62', 'A07', 'C10', 'C81', 'C00', 'A83', 'C08', 'C98'}\\ \tt VORSCHLAG: {'C56', 'C76', 'C62', 'A07', 'C10', 'C81', 'C00', 'B38', 'C08', 'C03'}\\ \tt ARGUMENT: {'Gesamtaufwand 3269 von 3629 – komfortabel im Limit mit Puffer.', 'Deine Fixpunkte bleiben unverändert: C76, C08, C10, C81 (sowie C56, C62, A07, C00).', 'Nur zwei minimalinvasive Anpassungen, die du selbst als Optionen genannt hast: A83 -> B38 und C98 -> C03.', 'So bleiben deine Prioritäten gewahrt, und wir erhöhen gleichzeitig die Gesamteffizienz – ohne weitere Streichungen.', 'Falls dir C03 nicht passt: C03 -> C98 zurück (Aufwand 3259) ist ok, solange A83 durch B38 ersetzt bleibt. Alternativ wäre A83 -> A03 bei gleichem Aufwand möglich.'} 
	  } 
	   } 
	   } 
	 & \\ 
 

    \theutterance \stepcounter{utterance}  

    & & & \multicolumn{4}{p{0.6\linewidth}}{\cellcolor[rgb]{1,0.85,0.72}{%
	\makecell[{{p{\linewidth}}}]{% 
	  \tt {\tiny [GM$\langle$B]}  
	 ABLEHNUNG: {'C56', 'C76', 'C62', 'A07', 'C10', 'C81', 'C00', 'B38', 'C08', 'C03'}\\ \tt VORSCHLAG: {'C56', 'C76', 'C62', 'A07', 'C10', 'C81', 'C08', 'C03', 'A83', 'C98'}\\ \tt ARGUMENT: {'Zwei gezielte Tausche gegenüber deinem Paket: B38 -> A83 und C00 -> C98; alles andere bleibt unverändert.', 'Gesamtaufwand 3283 (nur +14 ggü. deinem Vorschlag) – klar unter 3629 mit solidem Puffer.', 'Die beiden Tausche erhöhen den Effekt pro Aufwand und halten das Paket schlank.', 'Deine Struktur bleibt intakt: C56, C76, C62, A07, C10, C81, C08, C03 sind vollständig übernommen.', 'Falls A83 für dich nicht machbar ist: Fallback A83 -> B38 bei unverändertem C00 -> C98; neuer Aufwand 3139 – dann können wir sofort schließen.'} 
	  } 
	   } 
	   } 
	  \\ 
 

    \theutterance \stepcounter{utterance}  

    & & \multicolumn{4}{p{0.6\linewidth}}{\cellcolor[rgb]{0.9,0.9,0.9}{%
	\makecell[{{p{\linewidth}}}]{% 
	  \tt {\tiny [A$\langle$GM]}  
	 ABLEHNUNG: {'C56', 'C76', 'C62', 'A07', 'C10', 'C81', 'C00', 'B38', 'C08', 'C03'}\\ \tt VORSCHLAG: {'C56', 'C76', 'C62', 'A07', 'C10', 'C81', 'C08', 'C03', 'A83', 'C98'}\\ \tt ARGUMENT: {'Zwei gezielte Tausche gegenüber deinem Paket: B38 -> A83 und C00 -> C98; alles andere bleibt unverändert.', 'Gesamtaufwand 3283 (nur +14 ggü. deinem Vorschlag) – klar unter 3629 mit solidem Puffer.', 'Die beiden Tausche erhöhen den Effekt pro Aufwand und halten das Paket schlank.', 'Deine Struktur bleibt intakt: C56, C76, C62, A07, C10, C81, C08, C03 sind vollständig übernommen.', 'Falls A83 für dich nicht machbar ist: Fallback A83 -> B38 bei unverändertem C00 -> C98; neuer Aufwand 3139 – dann können wir sofort schließen.'} 
	  } 
	   } 
	   } 
	 & \\ 
 

    \theutterance \stepcounter{utterance}  

    & \multicolumn{4}{p{0.6\linewidth}}{\cellcolor[rgb]{0.8,1,0.9}{%
	\makecell[{{p{\linewidth}}}]{% 
	  \tt {\tiny [A$\rangle$GM]}  
	 ABLEHNUNG: {'C56', 'C76', 'C62', 'A07', 'C10', 'C81', 'C08', 'C03', 'A83', 'C98'}\\ \tt VORSCHLAG: {'C56', 'C76', 'C62', 'A07', 'C10', 'C81', 'C08', 'C03', 'B38', 'C98'}\\ \tt ARGUMENT: {'Exakt dein angekündigter Fallback: A83 -> B38, alles andere unverändert.', 'Gesamtaufwand 3139 von 3629 – sehr komfortabel unter dem Limit.', 'Deine Fixpunkte (C76, C08, C10, C81) bleiben 1:1 erhalten; zusätzlich bleiben C03 und C98 drin.', 'Minimalinvasiver Tausch, sofort schließbar. Wenn das für dich passt, lass uns direkt finalisieren.'} 
	  } 
	   } 
	   } 
	 & & \\ 
 

    \theutterance \stepcounter{utterance}  

    & & \multicolumn{4}{p{0.6\linewidth}}{\cellcolor[rgb]{0.9,0.9,0.9}{%
	\makecell[{{p{\linewidth}}}]{% 
	  \tt {\tiny [GM$\rangle$B]}  
	 ABLEHNUNG: {'C56', 'C76', 'C62', 'A07', 'C10', 'C81', 'C08', 'C03', 'A83', 'C98'}\\ \tt VORSCHLAG: {'C56', 'C76', 'C62', 'A07', 'C10', 'C81', 'C08', 'C03', 'B38', 'C98'}\\ \tt ARGUMENT: {'Exakt dein angekündigter Fallback: A83 -> B38, alles andere unverändert.', 'Gesamtaufwand 3139 von 3629 – sehr komfortabel unter dem Limit.', 'Deine Fixpunkte (C76, C08, C10, C81) bleiben 1:1 erhalten; zusätzlich bleiben C03 und C98 drin.', 'Minimalinvasiver Tausch, sofort schließbar. Wenn das für dich passt, lass uns direkt finalisieren.'} 
	  } 
	   } 
	   } 
	 & \\ 
 

    \theutterance \stepcounter{utterance}  

    & & & \multicolumn{4}{p{0.6\linewidth}}{\cellcolor[rgb]{1,0.85,0.72}{%
	\makecell[{{p{\linewidth}}}]{% 
	  \tt {\tiny [GM$\langle$B]}  
	 ABLEHNUNG: {'C56', 'C76', 'C62', 'A07', 'C10', 'C81', 'C08', 'C03', 'B38', 'C98'}\\ \tt VORSCHLAG: {'C56', 'C76', 'C62', 'C10', 'C81', 'C08', 'C03', 'B38', 'C98', 'C38'}\\ \tt ARGUMENT: {'Ein einziger, gezielter Tausch: A07 -> C38.', 'Gesamtaufwand 3619 von 3629 – wir nutzen den Puffer optimal aus, ohne das Limit zu reißen.', 'Die bestehende Struktur bleibt vollständig erhalten; wir heben nur die Gesamtwirkung pro Aufwand an.', 'Falls C38 für dich nicht machbar ist: Sofortiger Fallback A07 -> C00 (neuer Aufwand 3587) bei unveränderter übriger Auswahl – dann können wir direkt schließen.', 'C76, C08, C10, C81 sowie C03/C98/B38 bleiben 1:1 wie von dir vorgeschlagen.'} 
	  } 
	   } 
	   } 
	  \\ 
 

    \theutterance \stepcounter{utterance}  

    & & \multicolumn{4}{p{0.6\linewidth}}{\cellcolor[rgb]{0.9,0.9,0.9}{%
	\makecell[{{p{\linewidth}}}]{% 
	  \tt {\tiny [A$\langle$GM]}  
	 ABLEHNUNG: {'C56', 'C76', 'C62', 'A07', 'C10', 'C81', 'C08', 'C03', 'B38', 'C98'}\\ \tt VORSCHLAG: {'C56', 'C76', 'C62', 'C10', 'C81', 'C08', 'C03', 'B38', 'C98', 'C38'}\\ \tt ARGUMENT: {'Ein einziger, gezielter Tausch: A07 -> C38.', 'Gesamtaufwand 3619 von 3629 – wir nutzen den Puffer optimal aus, ohne das Limit zu reißen.', 'Die bestehende Struktur bleibt vollständig erhalten; wir heben nur die Gesamtwirkung pro Aufwand an.', 'Falls C38 für dich nicht machbar ist: Sofortiger Fallback A07 -> C00 (neuer Aufwand 3587) bei unveränderter übriger Auswahl – dann können wir direkt schließen.', 'C76, C08, C10, C81 sowie C03/C98/B38 bleiben 1:1 wie von dir vorgeschlagen.'} 
	  } 
	   } 
	   } 
	 & \\ 
 

    \theutterance \stepcounter{utterance}  

    & \multicolumn{4}{p{0.6\linewidth}}{\cellcolor[rgb]{0.8,1,0.9}{%
	\makecell[{{p{\linewidth}}}]{% 
	  \tt {\tiny [A$\rangle$GM]}  
	 ABLEHNUNG: {'C56', 'C76', 'C62', 'C10', 'C81', 'C08', 'C03', 'B38', 'C98', 'C38'}\\ \tt VORSCHLAG: {'C56', 'C76', 'C62', 'A07', 'C10', 'C81', 'C08', 'C03', 'B38', 'C00'}\\ \tt ARGUMENT: {'Gesamtaufwand 3269 von 3629 – komfortabler Puffer ohne Limitrisiko.', 'Ich übernehme deine Fixpunkte (C76, C08, C10, C81) sowie B38 und C03 unverändert.', 'Zwei gezielte Anpassungen: C38 -> A07 und C98 -> C00.', 'A07 ist für mich klarer Vorzug gegenüber C38 bei deutlich geringerem Aufwand – wir gewinnen 480 Aufwand zurück, ohne Strukturbruch.', 'C00 ist für mich ein reines Effizienz‑Upgrade gegenüber C98 bei geringem Mehraufwand; du hattest C00 bereits als tragfähige Option im Spiel.', 'Wenn C00 für dich ein No‑Go ist: Fallback C00 -> C98 bei unverändertem A07 (neuer Aufwand 3139) ist okay. A07 statt C38 ist mein zentraler Punkt.', 'Passt dieses Paket für dich, können wir sofort schließen.'} 
	  } 
	   } 
	   } 
	 & & \\ 
 

    \theutterance \stepcounter{utterance}  

    & & \multicolumn{4}{p{0.6\linewidth}}{\cellcolor[rgb]{0.9,0.9,0.9}{%
	\makecell[{{p{\linewidth}}}]{% 
	  \tt {\tiny [GM$\rangle$B]}  
	 ABLEHNUNG: {'C56', 'C76', 'C62', 'C10', 'C81', 'C08', 'C03', 'B38', 'C98', 'C38'}\\ \tt VORSCHLAG: {'C56', 'C76', 'C62', 'A07', 'C10', 'C81', 'C08', 'C03', 'B38', 'C00'}\\ \tt ARGUMENT: {'Gesamtaufwand 3269 von 3629 – komfortabler Puffer ohne Limitrisiko.', 'Ich übernehme deine Fixpunkte (C76, C08, C10, C81) sowie B38 und C03 unverändert.', 'Zwei gezielte Anpassungen: C38 -> A07 und C98 -> C00.', 'A07 ist für mich klarer Vorzug gegenüber C38 bei deutlich geringerem Aufwand – wir gewinnen 480 Aufwand zurück, ohne Strukturbruch.', 'C00 ist für mich ein reines Effizienz‑Upgrade gegenüber C98 bei geringem Mehraufwand; du hattest C00 bereits als tragfähige Option im Spiel.', 'Wenn C00 für dich ein No‑Go ist: Fallback C00 -> C98 bei unverändertem A07 (neuer Aufwand 3139) ist okay. A07 statt C38 ist mein zentraler Punkt.', 'Passt dieses Paket für dich, können wir sofort schließen.'} 
	  } 
	   } 
	   } 
	 & \\ 
 

    \theutterance \stepcounter{utterance}  

    & & & \multicolumn{4}{p{0.6\linewidth}}{\cellcolor[rgb]{1,0.85,0.72}{%
	\makecell[{{p{\linewidth}}}]{% 
	  \tt {\tiny [GM$\langle$B]}  
	 ABLEHNUNG: {'C56', 'C76', 'C62', 'A07', 'C10', 'C81', 'C08', 'C03', 'B38', 'C00'}\\ \tt VORSCHLAG: {'C56', 'C76', 'C62', 'A07', 'C10', 'C81', 'C08', 'C03', 'B38', 'C98'}\\ \tt ARGUMENT: {'Ich greife deine eigene Fallback‑Option auf: C00 -> C98, alles andere unverändert.', 'Gesamtaufwand 3139 von 3629 – sehr komfortabler Puffer ohne Limitrisiko.', 'Dein zentraler Punkt A07 bleibt ausdrücklich erhalten; ebenso deine Fixpunkte C76, C08, C10, C81 sowie B38 und C03.', 'Minimalinvasiver Tausch, sofort schließbar. Wenn das für dich passt, können wir direkt finalisieren.'} 
	  } 
	   } 
	   } 
	  \\ 
 

    \theutterance \stepcounter{utterance}  

    & & \multicolumn{4}{p{0.6\linewidth}}{\cellcolor[rgb]{0.9,0.9,0.9}{%
	\makecell[{{p{\linewidth}}}]{% 
	  \tt {\tiny [A$\langle$GM]}  
	 ABLEHNUNG: {'C56', 'C76', 'C62', 'A07', 'C10', 'C81', 'C08', 'C03', 'B38', 'C00'}\\ \tt VORSCHLAG: {'C56', 'C76', 'C62', 'A07', 'C10', 'C81', 'C08', 'C03', 'B38', 'C98'}\\ \tt ARGUMENT: {'Ich greife deine eigene Fallback‑Option auf: C00 -> C98, alles andere unverändert.', 'Gesamtaufwand 3139 von 3629 – sehr komfortabler Puffer ohne Limitrisiko.', 'Dein zentraler Punkt A07 bleibt ausdrücklich erhalten; ebenso deine Fixpunkte C76, C08, C10, C81 sowie B38 und C03.', 'Minimalinvasiver Tausch, sofort schließbar. Wenn das für dich passt, können wir direkt finalisieren.'} 
	  } 
	   } 
	   } 
	 & \\ 
 

    \theutterance \stepcounter{utterance}  

    & \multicolumn{4}{p{0.6\linewidth}}{\cellcolor[rgb]{0.8,1,0.9}{%
	\makecell[{{p{\linewidth}}}]{% 
	  \tt {\tiny [A$\rangle$GM]}  
	 ABLEHNUNG: {'C56', 'C76', 'C62', 'A07', 'C10', 'C81', 'C08', 'C03', 'B38', 'C98'}\\ \tt VORSCHLAG: {'C56', 'C76', 'C62', 'A07', 'C10', 'C81', 'C08', 'B38', 'C38', 'C00'}\\ \tt ARGUMENT: {'Zwei gezielte Upgrades ohne Strukturbruch: C03 -> C38 und C98 -> C00.', 'Gesamtaufwand 3421 von 3629 – klar unter dem Limit mit solidem Puffer.', 'Deine Fixpunkte (C76, C08, C10, C81) sowie A07 und B38 bleiben 1:1 erhalten.', 'C38 hattest du selbst priorisiert; C00 war bereits als tragfähige Option im Spiel – beides passt in deinen Kurs.', 'Falls C00 für dich nicht geht: Fallback C00 -> C98 (neuer Aufwand 3291) bei unverändertem C38.', 'Falls C38 für dich nicht geht: Fallback C38 -> C03 (neuer Aufwand 3269) bei unverändertem C00.', 'Wenn beide Anpassungen für dich nicht passen, gib kurz Bescheid – dann schließen wir auf deinem letzten Paket.'} 
	  } 
	   } 
	   } 
	 & & \\ 
 

    \theutterance \stepcounter{utterance}  

    & & \multicolumn{4}{p{0.6\linewidth}}{\cellcolor[rgb]{0.9,0.9,0.9}{%
	\makecell[{{p{\linewidth}}}]{% 
	  \tt {\tiny [GM$\rangle$B]}  
	 ABLEHNUNG: {'C56', 'C76', 'C62', 'A07', 'C10', 'C81', 'C08', 'C03', 'B38', 'C98'}\\ \tt VORSCHLAG: {'C56', 'C76', 'C62', 'A07', 'C10', 'C81', 'C08', 'B38', 'C38', 'C00'}\\ \tt ARGUMENT: {'Zwei gezielte Upgrades ohne Strukturbruch: C03 -> C38 und C98 -> C00.', 'Gesamtaufwand 3421 von 3629 – klar unter dem Limit mit solidem Puffer.', 'Deine Fixpunkte (C76, C08, C10, C81) sowie A07 und B38 bleiben 1:1 erhalten.', 'C38 hattest du selbst priorisiert; C00 war bereits als tragfähige Option im Spiel – beides passt in deinen Kurs.', 'Falls C00 für dich nicht geht: Fallback C00 -> C98 (neuer Aufwand 3291) bei unverändertem C38.', 'Falls C38 für dich nicht geht: Fallback C38 -> C03 (neuer Aufwand 3269) bei unverändertem C00.', 'Wenn beide Anpassungen für dich nicht passen, gib kurz Bescheid – dann schließen wir auf deinem letzten Paket.'} 
	  } 
	   } 
	   } 
	 & \\ 
 

    \theutterance \stepcounter{utterance}  

    & & & \multicolumn{4}{p{0.6\linewidth}}{\cellcolor[rgb]{1,0.85,0.72}{%
	\makecell[{{p{\linewidth}}}]{% 
	  \tt {\tiny [GM$\langle$B]}  
	 ABLEHNUNG: {'C56', 'C76', 'C62', 'A07', 'C10', 'C81', 'C08', 'B38', 'C38', 'C00'}\\ \tt VORSCHLAG: {'C56', 'C76', 'C62', 'A07', 'C10', 'C81', 'C08', 'B38', 'C38', 'C98'}\\ \tt ARGUMENT: {'Ein einziger, von dir selbst angebotener Fallback: C00 -> C98; alles andere bleibt identisch.', 'Gesamtaufwand 3291 von 3629 – komfortabler Puffer, keinerlei Limitrisiko.', 'Deine Fixpunkte (C76, C08, C10, C81) sowie A07, B38 und C38 bleiben vollständig erhalten.', 'Minimalinvasiver Tausch, sofort schließbar. Wenn das passt, können wir direkt finalisieren.', 'Falls dir stattdessen lieber: Alternative ist C38 -> C03 bei unverändertem C00 (Aufwand 3269) – du hattest diese Option ebenfalls genannt.'} 
	  } 
	   } 
	   } 
	  \\ 
 

    \theutterance \stepcounter{utterance}  

    & & \multicolumn{4}{p{0.6\linewidth}}{\cellcolor[rgb]{0.9,0.9,0.9}{%
	\makecell[{{p{\linewidth}}}]{% 
	  \tt {\tiny [A$\langle$GM]}  
	 ABLEHNUNG: {'C56', 'C76', 'C62', 'A07', 'C10', 'C81', 'C08', 'B38', 'C38', 'C00'}\\ \tt VORSCHLAG: {'C56', 'C76', 'C62', 'A07', 'C10', 'C81', 'C08', 'B38', 'C38', 'C98'}\\ \tt ARGUMENT: {'Ein einziger, von dir selbst angebotener Fallback: C00 -> C98; alles andere bleibt identisch.', 'Gesamtaufwand 3291 von 3629 – komfortabler Puffer, keinerlei Limitrisiko.', 'Deine Fixpunkte (C76, C08, C10, C81) sowie A07, B38 und C38 bleiben vollständig erhalten.', 'Minimalinvasiver Tausch, sofort schließbar. Wenn das passt, können wir direkt finalisieren.', 'Falls dir stattdessen lieber: Alternative ist C38 -> C03 bei unverändertem C00 (Aufwand 3269) – du hattest diese Option ebenfalls genannt.'} 
	  } 
	   } 
	   } 
	 & \\ 
 

    \theutterance \stepcounter{utterance}  

    & \multicolumn{4}{p{0.6\linewidth}}{\cellcolor[rgb]{0.8,1,0.9}{%
	\makecell[{{p{\linewidth}}}]{% 
	  \tt {\tiny [A$\rangle$GM]}  
	 ABLEHNUNG: {'C56', 'C76', 'C62', 'A07', 'C10', 'C81', 'C08', 'B38', 'C38', 'C98'}\\ \tt VORSCHLAG: {'C56', 'C76', 'C62', 'A07', 'C10', 'C81', 'C08', 'B38', 'C38', 'C00'}\\ \tt ARGUMENT: {'Minimalinvasiver Tausch: C98 -> C00; alles andere bleibt 1:1 wie in deinem Vorschlag.', 'Gesamtaufwand 3421 von 3629 – wir nutzen etwas Puffer, bleiben aber klar unter dem Limit.', 'Du hattest C00 bereits als tragfähige Option im Spiel; ich greife das auf, ohne deine Fixpunkte (A07, C76, C08, C10, C81, B38, C38) anzutasten.', 'Wenn C00 für dich ein No-Go ist, sag kurz Bescheid und bring dein letztes Paket erneut – dann können wir es sofort schließen.'} 
	  } 
	   } 
	   } 
	 & & \\ 
 

    \theutterance \stepcounter{utterance}  

    & & \multicolumn{4}{p{0.6\linewidth}}{\cellcolor[rgb]{0.9,0.9,0.9}{%
	\makecell[{{p{\linewidth}}}]{% 
	  \tt {\tiny [GM$\rangle$B]}  
	 ABLEHNUNG: {'C56', 'C76', 'C62', 'A07', 'C10', 'C81', 'C08', 'B38', 'C38', 'C98'}\\ \tt VORSCHLAG: {'C56', 'C76', 'C62', 'A07', 'C10', 'C81', 'C08', 'B38', 'C38', 'C00'}\\ \tt ARGUMENT: {'Minimalinvasiver Tausch: C98 -> C00; alles andere bleibt 1:1 wie in deinem Vorschlag.', 'Gesamtaufwand 3421 von 3629 – wir nutzen etwas Puffer, bleiben aber klar unter dem Limit.', 'Du hattest C00 bereits als tragfähige Option im Spiel; ich greife das auf, ohne deine Fixpunkte (A07, C76, C08, C10, C81, B38, C38) anzutasten.', 'Wenn C00 für dich ein No-Go ist, sag kurz Bescheid und bring dein letztes Paket erneut – dann können wir es sofort schließen.'} 
	  } 
	   } 
	   } 
	 & \\ 
 

    \theutterance \stepcounter{utterance}  

    & & & \multicolumn{4}{p{0.6\linewidth}}{\cellcolor[rgb]{1,0.85,0.72}{%
	\makecell[{{p{\linewidth}}}]{% 
	  \tt {\tiny [GM$\langle$B]}  
	 ABLEHNUNG: {'C56', 'C76', 'C62', 'A07', 'C10', 'C81', 'C08', 'B38', 'C38', 'C00'}\\ \tt VORSCHLAG: {'C56', 'C76', 'C62', 'A07', 'C10', 'C81', 'C08', 'B38', 'C03', 'C00'}\\ \tt ARGUMENT: {'Minimaler, von dir selbst genannter Tausch: C38 -> C03; alles andere bleibt unverändert.', 'Gesamtaufwand 3269 von 3629 – mehr Puffer, null Limitrisiko.', 'Deine Struktur sowie alle Fixpunkte bleiben vollständig erhalten; wir schließen damit effizient und ohne weitere Änderungen.', 'Wenn dir C38 absolut wichtig ist, sag kurz Bescheid – dann schauen wir nur diesen einen Slot an, ohne das restliche Paket anzutasten.'} 
	  } 
	   } 
	   } 
	  \\ 
 

    \theutterance \stepcounter{utterance}  

    & & \multicolumn{4}{p{0.6\linewidth}}{\cellcolor[rgb]{0.9,0.9,0.9}{%
	\makecell[{{p{\linewidth}}}]{% 
	  \tt {\tiny [A$\langle$GM]}  
	 ABLEHNUNG: {'C56', 'C76', 'C62', 'A07', 'C10', 'C81', 'C08', 'B38', 'C38', 'C00'}\\ \tt VORSCHLAG: {'C56', 'C76', 'C62', 'A07', 'C10', 'C81', 'C08', 'B38', 'C03', 'C00'}\\ \tt ARGUMENT: {'Minimaler, von dir selbst genannter Tausch: C38 -> C03; alles andere bleibt unverändert.', 'Gesamtaufwand 3269 von 3629 – mehr Puffer, null Limitrisiko.', 'Deine Struktur sowie alle Fixpunkte bleiben vollständig erhalten; wir schließen damit effizient und ohne weitere Änderungen.', 'Wenn dir C38 absolut wichtig ist, sag kurz Bescheid – dann schauen wir nur diesen einen Slot an, ohne das restliche Paket anzutasten.'} 
	  } 
	   } 
	   } 
	 & \\ 
 

    \theutterance \stepcounter{utterance}  

    & \multicolumn{4}{p{0.6\linewidth}}{\cellcolor[rgb]{0.8,1,0.9}{%
	\makecell[{{p{\linewidth}}}]{% 
	  \tt {\tiny [A$\rangle$GM]}  
	 ZUSTIMMUNG: {'C56', 'C76', 'C62', 'A07', 'C10', 'C81', 'C08', 'B38', 'C03', 'C00'}\\ \tt ARGUMENT: {'Passt für mich – sauber im Limit und minimalinvasiv angepasst.', 'Sofort schließbar; danke für die zielgerichtete Einigung.'} 
	  } 
	   } 
	   } 
	 & & \\ 
 

    \theutterance \stepcounter{utterance}  

    & & & \multicolumn{2}{p{0.3\linewidth}}{\cellcolor[rgb]{0.95,0.95,0.95}{%
	\makecell[{{p{\linewidth}}}]{% 
	  \tt {\tiny [GM$|$GM]}  
	 game successful 
	  } 
	   } 
	   } 
	 & & \\ 
 

    \theutterance \stepcounter{utterance}  

    & & & \multicolumn{2}{p{0.3\linewidth}}{\cellcolor[rgb]{0.95,0.95,0.95}{%
	\makecell[{{p{\linewidth}}}]{% 
	  \tt {\tiny [GM$|$GM]}  
	 end game 
	  } 
	   } 
	   } 
	 & & \\ 
 

\end{supertabular}
}

\end{document}
