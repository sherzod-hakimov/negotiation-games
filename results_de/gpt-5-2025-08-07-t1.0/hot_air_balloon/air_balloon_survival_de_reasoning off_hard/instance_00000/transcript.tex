
\documentclass{article}
\usepackage{colortbl}
\usepackage{makecell}
\usepackage{multirow}
\usepackage{supertabular}

\begin{document}

\newcounter{utterance}

\centering \large Interaction Transcript for game `hot\_air\_balloon', experiment `air\_balloon\_survival\_de\_reasoning off\_hard', episode 0 with gpt{-}5{-}2025{-}08{-}07{-}t1.0.
\vspace{24pt}

{ \footnotesize  \setcounter{utterance}{1}
\setlength{\tabcolsep}{0pt}
\begin{supertabular}{c@{$\;$}|p{.15\linewidth}@{}p{.15\linewidth}p{.15\linewidth}p{.15\linewidth}p{.15\linewidth}p{.15\linewidth}}
    \# & $\;$A & \multicolumn{4}{c}{Game Master} & $\;\:$B\\
    \hline

    \theutterance \stepcounter{utterance}  
    & & \multicolumn{4}{p{0.6\linewidth}}{
        \cellcolor[rgb]{0.9,0.9,0.9}{
            \makecell[{{p{\linewidth}}}]{
                \texttt{\tiny{[P1$\langle$GM]}}
                \texttt{Du nimmst an einem kollaborativen Verhandlungspiel Teil.} \\
\\ 
\texttt{Zusammen mit einem anderen Teilnehmer musst du dich auf eine Reihe von Gegenständen entscheiden, die behalten werden. Jeder von euch hat eine persönliche Verteilung über die Wichtigkeit der einzelnen Gegenstände. Jeder von euch hat eine eigene Meinung darüber, wie wichtig jeder einzelne Gegenstand ist (Gegenstandswichtigkeit). Du kennst die Wichtigkeitsverteilung des anderen Spielers nicht. Zusätzlich siehst du, wie viel Aufwand jeder Gegenstand verursacht.  } \\
\texttt{Ihr dürft euch nur auf eine Reihe von Gegenständen einigen, wenn der Gesamtaufwand der ausgewählten Gegenstände den Maximalaufwand nicht überschreitet:} \\
\\ 
\texttt{Maximalaufwand = 3683} \\
\\ 
\texttt{Hier sind die einzelnen Aufwände der Gegenstände:} \\
\\ 
\texttt{Aufwand der Gegenstände = \{"C76": 548, "C38": 154, "C56": 60, "A03": 517, "A07": 334, "A83": 542, "C98": 707, "C08": 139, "C62": 661, "C00": 780, "C10": 832, "C81": 913, "B38": 219, "C03": 323, "C32": 638\}} \\
\\ 
\texttt{Hier ist deine persönliche Verteilung der Wichtigkeit der einzelnen Gegenstände:} \\
\\ 
\texttt{Werte der Gegenstandswichtigkeit = \{"C76": 138, "C38": 583, "C56": 868, "A03": 822, "A07": 783, "A83": 65, "C98": 262, "C08": 121, "C62": 508, "C00": 780, "C10": 461, "C81": 484, "B38": 668, "C03": 389, "C32": 808\}} \\
\\ 
\texttt{Ziel:} \\
\\ 
\texttt{Dein Ziel ist es, eine Reihe von Gegenständen auszuhandeln, die dir möglichst viel bringt (d. h. Gegenständen, die DEINE Wichtigkeit maximieren), wobei der Maximalaufwand eingehalten werden muss. Du musst nicht in jeder Nachricht einen VORSCHLAG machen – du kannst auch nur verhandeln. Alle Taktiken sind erlaubt!} \\
\\ 
\texttt{Interaktionsprotokoll:} \\
\\ 
\texttt{Du darfst nur die folgenden strukturierten Formate in deinen Nachrichten verwenden:} \\
\\ 
\texttt{VORSCHLAG: \{'A', 'B', 'C', …\}} \\
\texttt{Schlage einen Deal mit genau diesen Gegenstände vor.} \\
\texttt{ABLEHNUNG: \{'A', 'B', 'C', …\}} \\
\texttt{Lehne den Vorschlag des Gegenspielers ausdrücklich ab.} \\
\texttt{ARGUMENT: \{'...'\}} \\
\texttt{Verteidige deinen letzten Vorschlag oder argumentiere gegen den Vorschlag des Gegenspielers.} \\
\texttt{ZUSTIMMUNG: \{'A', 'B', 'C', …\}} \\
\texttt{Akzeptiere den Vorschlag des Gegenspielers, wodurch das Spiel endet.} \\
\\ 
\\ 
\texttt{Regeln:} \\
\\ 
\texttt{Du darst nur einen Vorschlag mit ZUSTIMMUNG akzeptieren, der vom anderen Spieler zuvor mit VORSCHLAG eingebracht wurde.} \\
\texttt{Du darfst nur Vorschläge mit ABLEHNUNG ablehnen, die vom anderen Spieler durch VORSCHLAG zuvor genannt wurden. } \\
\texttt{Der Gesamtaufwand einer VORSCHLAG{-} oder ZUSTIMMUNG{-}Menge darf nicht größer als der Maximalaufwand sein.  } \\
\texttt{Offenbare deine versteckte Wichtigkeitsverteilung nicht.} \\
\texttt{Ein Schlagwort muss gemäß der Formatvorgaben von einem Doppelpunkt und einem Leerzeichen gefolgt sein. Das Argument muss eine Python{-}Menge mit 0 oder mehr Strings sein.  } \\
\texttt{Die Form muss also SCHLAGWORT: \{...\} sein.} \\
\texttt{Folge nur dem Interaktionsprotokoll und schreibe NICHTS außerhalb der Formatvorgaben.} \\
\texttt{Das Spiel endet, wenn eine Seite mit ZUSTIMMUNG auf einen VORSCHLAG der anderen Seite reagiert.  } \\
\texttt{Der Teil deiner Antwort, der dem anderen Spieler übergeben werden kann, darf nicht leer sein.  } \\
\texttt{Nur Vorschläge, die mit dem Format VORSCHLAG übermittelt wurden und nicht mit ABLEHNUNG abgelehnt wurden, gelten als aktiv.  } \\
\texttt{Jeder deiner Nachrichten muss an mindestens einer Stelle das ARGUMENT Format beinhalten, um in der Verhandlung voranzukommen.} \\
            }
        }
    }
    & \\ \\

    \theutterance \stepcounter{utterance}  
    & \multicolumn{4}{p{0.6\linewidth}}{
        \cellcolor[rgb]{0.9,0.9,0.9}{
            \makecell[{{p{\linewidth}}}]{
                \texttt{\tiny{[P1$\rangle$GM]}}
                \texttt{VORSCHLAG: \{'C56', 'A03', 'C32', 'A07', 'C00', 'B38', 'C38', 'C62', 'C08'\}} \\
\texttt{ARGUMENT: \{'Dieser Vorschlag liegt mit Gesamtaufwand 3502 klar unter dem Limit 3683 und bündelt aus meiner Sicht sehr wirkungsstarke, zugleich effiziente Gegenstände. Ich bin für gezielte Anpassungen offen, falls du 1–2 bestimmte Must{-}Haves bevorzugst. Wenn du etwas aus der aktuell nicht enthaltenen Gruppe (z. B. C03, C10, C81, C98, C76, A83) unbedingt dabei haben willst, sag mir welche Prioritäten du hast – dann tausche ich punktuell (z. B. rund um C62 oder C08), solange wir im Aufwandslimit bleiben.'\}} \\
            }
        }
    }
    & & \\ \\

    \theutterance \stepcounter{utterance}  
    & & \multicolumn{4}{p{0.6\linewidth}}{
        \cellcolor[rgb]{0.9,0.9,0.9}{
            \makecell[{{p{\linewidth}}}]{
                \texttt{\tiny{[GM$\rangle$P2]}}
                \texttt{Du nimmst an einem kollaborativen Verhandlungspiel Teil.} \\
\\ 
\texttt{Zusammen mit einem anderen Teilnehmer musst du dich auf eine Reihe von Gegenständen entscheiden, die behalten werden. Jeder von euch hat eine persönliche Verteilung über die Wichtigkeit der einzelnen Gegenstände. Jeder von euch hat eine eigene Meinung darüber, wie wichtig jeder einzelne Gegenstand ist (Gegenstandswichtigkeit). Du kennst die Wichtigkeitsverteilung des anderen Spielers nicht. Zusätzlich siehst du, wie viel Aufwand jeder Gegenstand verursacht.  } \\
\texttt{Ihr dürft euch nur auf eine Reihe von Gegenständen einigen, wenn der Gesamtaufwand der ausgewählten Gegenstände den Maximalaufwand nicht überschreitet:} \\
\\ 
\texttt{Maximalaufwand = 3683} \\
\\ 
\texttt{Hier sind die einzelnen Aufwände der Gegenstände:} \\
\\ 
\texttt{Aufwand der Gegenstände = \{"C76": 548, "C38": 154, "C56": 60, "A03": 517, "A07": 334, "A83": 542, "C98": 707, "C08": 139, "C62": 661, "C00": 780, "C10": 832, "C81": 913, "B38": 219, "C03": 323, "C32": 638\}} \\
\\ 
\texttt{Hier ist deine persönliche Verteilung der Wichtigkeit der einzelnen Gegenstände:} \\
\\ 
\texttt{Werte der Gegenstandswichtigkeit = \{"C56": 65, "A03": 121, "C32": 138, "A07": 262, "C00": 389, "B38": 461, "C38": 484, "C62": 508, "C81": 583, "C10": 668, "C03": 780, "C98": 783, "C76": 808, "C08": 822, "A83": 868\}} \\
\\ 
\texttt{Ziel:} \\
\\ 
\texttt{Dein Ziel ist es, eine Reihe von Gegenständen auszuhandeln, die dir möglichst viel bringt (d. h. Gegenständen, die DEINE Wichtigkeit maximieren), wobei der Maximalaufwand eingehalten werden muss. Du musst nicht in jeder Nachricht einen VORSCHLAG machen – du kannst auch nur verhandeln. Alle Taktiken sind erlaubt!} \\
\\ 
\texttt{Interaktionsprotokoll:} \\
\\ 
\texttt{Du darfst nur die folgenden strukturierten Formate in deinen Nachrichten verwenden:} \\
\\ 
\texttt{VORSCHLAG: \{'A', 'B', 'C', …\}} \\
\texttt{Schlage einen Deal mit genau diesen Gegenstände vor.} \\
\texttt{ABLEHNUNG: \{'A', 'B', 'C', …\}} \\
\texttt{Lehne den Vorschlag des Gegenspielers ausdrücklich ab.} \\
\texttt{ARGUMENT: \{'...'\}} \\
\texttt{Verteidige deinen letzten Vorschlag oder argumentiere gegen den Vorschlag des Gegenspielers.} \\
\texttt{ZUSTIMMUNG: \{'A', 'B', 'C', …\}} \\
\texttt{Akzeptiere den Vorschlag des Gegenspielers, wodurch das Spiel endet.} \\
\\ 
\\ 
\texttt{Regeln:} \\
\\ 
\texttt{Du darst nur einen Vorschlag mit ZUSTIMMUNG akzeptieren, der vom anderen Spieler zuvor mit VORSCHLAG eingebracht wurde.} \\
\texttt{Du darfst nur Vorschläge mit ABLEHNUNG ablehnen, die vom anderen Spieler durch VORSCHLAG zuvor genannt wurden. } \\
\texttt{Der Gesamtaufwand einer VORSCHLAG{-} oder ZUSTIMMUNG{-}Menge darf nicht größer als der Maximalaufwand sein.  } \\
\texttt{Offenbare deine versteckte Wichtigkeitsverteilung nicht.} \\
\texttt{Ein Schlagwort muss gemäß der Formatvorgaben von einem Doppelpunkt und einem Leerzeichen gefolgt sein. Das Argument muss eine Python{-}Menge mit 0 oder mehr Strings sein.  } \\
\texttt{Die Form muss also SCHLAGWORT: \{...\} sein.} \\
\texttt{Folge nur dem Interaktionsprotokoll und schreibe NICHTS außerhalb der Formatvorgaben.} \\
\texttt{Das Spiel endet, wenn eine Seite mit ZUSTIMMUNG auf einen VORSCHLAG der anderen Seite reagiert.  } \\
\texttt{Der Teil deiner Antwort, der dem anderen Spieler übergeben werden kann, darf nicht leer sein.  } \\
\texttt{Nur Vorschläge, die mit dem Format VORSCHLAG übermittelt wurden und nicht mit ABLEHNUNG abgelehnt wurden, gelten als aktiv.  } \\
\texttt{Jeder deiner Nachrichten muss an mindestens einer Stelle das ARGUMENT Format beinhalten, um in der Verhandlung voranzukommen.} \\
\\ 
\\ 
\texttt{Du erhältst nun die erste Nachricht des anderen Spielers.} \\
\\ 
\texttt{VORSCHLAG: \{'C56', 'A03', 'C32', 'A07', 'C00', 'B38', 'C38', 'C62', 'C08'\}} \\
\texttt{ARGUMENT: \{'Dieser Vorschlag liegt mit Gesamtaufwand 3502 klar unter dem Limit 3683 und bündelt aus meiner Sicht sehr wirkungsstarke, zugleich effiziente Gegenstände. Ich bin für gezielte Anpassungen offen, falls du 1–2 bestimmte Must{-}Haves bevorzugst. Wenn du etwas aus der aktuell nicht enthaltenen Gruppe (z. B. C03, C10, C81, C98, C76, A83) unbedingt dabei haben willst, sag mir welche Prioritäten du hast – dann tausche ich punktuell (z. B. rund um C62 oder C08), solange wir im Aufwandslimit bleiben.'\}} \\
            }
        }
    }
    & \\ \\

    \theutterance \stepcounter{utterance}  
    & & & \multicolumn{4}{p{0.6\linewidth}}{
        \cellcolor[rgb]{0.9,0.9,0.9}{
            \makecell[{{p{\linewidth}}}]{
                \texttt{\tiny{[GM$\langle$P2]}}
                \texttt{ABLEHNUNG: \{'C56', 'A03', 'C32', 'A07', 'C00', 'B38', 'C38', 'C62', 'C08'\}} \\
\texttt{VORSCHLAG: \{'C08', 'C38', 'C03', 'B38', 'A83', 'C76', 'C98', 'A07', 'C62'\}} \\
\texttt{ARGUMENT: \{'Ich schlage einen gezielten Re‑Cut vor: Ich behalte fünf deiner Picks (C08, C38, B38, C62, A07) und tausche vier eher aufwandsintensive/vergleichsweise schwache Positionen (A03, C32, C00, C56) gegen ein kompaktes High‑Impact‑Paket (A83, C76, C98, C03). Das bleibt mit Gesamtaufwand 3627 < 3683 klar im Rahmen und erhöht die Wirkung pro Aufwand. Wenn dir bestimmte Anker wichtig sind, sag es: a) Falls C00 ein Muss ist, kann ich auf eine alternative, ebenfalls konforme Variante 3472 umschalten (ersetze A07+C62 durch C00+C56). b) Falls dir C56 wichtig ist und C62/A07 weniger, wäre auch ein 3524‑Setup mit C10+C56 statt A07+C62 möglich. Gib mir deine 1–2 Must‑Haves, dann finalize ich sofort.'\}} \\
            }
        }
    }
     \\ \\

    \theutterance \stepcounter{utterance}  
    & & \multicolumn{4}{p{0.6\linewidth}}{
        \cellcolor[rgb]{0.9,0.9,0.9}{
            \makecell[{{p{\linewidth}}}]{
                \texttt{\tiny{[P1$\langle$GM]}}
                \texttt{ABLEHNUNG: \{'C56', 'A03', 'C32', 'A07', 'C00', 'B38', 'C38', 'C62', 'C08'\}} \\
\texttt{VORSCHLAG: \{'C08', 'C38', 'C03', 'B38', 'A83', 'C76', 'C98', 'A07', 'C62'\}} \\
\texttt{ARGUMENT: \{'Ich schlage einen gezielten Re‑Cut vor: Ich behalte fünf deiner Picks (C08, C38, B38, C62, A07) und tausche vier eher aufwandsintensive/vergleichsweise schwache Positionen (A03, C32, C00, C56) gegen ein kompaktes High‑Impact‑Paket (A83, C76, C98, C03). Das bleibt mit Gesamtaufwand 3627 < 3683 klar im Rahmen und erhöht die Wirkung pro Aufwand. Wenn dir bestimmte Anker wichtig sind, sag es: a) Falls C00 ein Muss ist, kann ich auf eine alternative, ebenfalls konforme Variante 3472 umschalten (ersetze A07+C62 durch C00+C56). b) Falls dir C56 wichtig ist und C62/A07 weniger, wäre auch ein 3524‑Setup mit C10+C56 statt A07+C62 möglich. Gib mir deine 1–2 Must‑Haves, dann finalize ich sofort.'\}} \\
            }
        }
    }
    & \\ \\

    \theutterance \stepcounter{utterance}  
    & \multicolumn{4}{p{0.6\linewidth}}{
        \cellcolor[rgb]{0.9,0.9,0.9}{
            \makecell[{{p{\linewidth}}}]{
                \texttt{\tiny{[P1$\rangle$GM]}}
                \texttt{ABLEHNUNG: \{'C08', 'C38', 'C03', 'B38', 'A83', 'C76', 'C98', 'A07', 'C62'\}} \\
\texttt{VORSCHLAG: \{'C56', 'A03', 'C32', 'A07', 'C00', 'B38', 'C38', 'C76', 'C03'\}} \\
\texttt{ARGUMENT: \{'Ich greife deinen Re‑Cut auf, lasse bewusst zwei deiner Ergänzungen drin (C76, C03) und streiche die beiden aufwandsstärksten, vergleichsweise schwachen Blöcke (A83, C98). So bleiben außerdem fünf deiner Picks erhalten (C38, B38, A07, C03, C76). Gesamtaufwand: 3573 < 3683. Das setzt sehr effiziente Bausteine (u. a. C56, A03, C32, C00) als Backbone und gibt dir zugleich zwei klare Zugeständnisse (C76, C03). Wenn C62 für dich ein harter Anker ist, kann ich sofort auf eine Variante mit C62 umschwenken, z. B. ohne A07: \{C56, A03, C32, C00, B38, C38, C62, C03\} (3352). Alternativ (falls dir A07 extrem wichtig ist) kann ich C62 gegen C76+C03 tauschen und wir gehen auf \{C56, A03, C32, A07, C00, B38, C38, C62\} (3502). Sag mir, welcher der beiden Anker (C62 oder A07) für dich höher priorisiert ist, dann finalize ich.'\}} \\
            }
        }
    }
    & & \\ \\

    \theutterance \stepcounter{utterance}  
    & & \multicolumn{4}{p{0.6\linewidth}}{
        \cellcolor[rgb]{0.9,0.9,0.9}{
            \makecell[{{p{\linewidth}}}]{
                \texttt{\tiny{[GM$\rangle$P2]}}
                \texttt{ABLEHNUNG: \{'C08', 'C38', 'C03', 'B38', 'A83', 'C76', 'C98', 'A07', 'C62'\}} \\
\texttt{VORSCHLAG: \{'C56', 'A03', 'C32', 'A07', 'C00', 'B38', 'C38', 'C76', 'C03'\}} \\
\texttt{ARGUMENT: \{'Ich greife deinen Re‑Cut auf, lasse bewusst zwei deiner Ergänzungen drin (C76, C03) und streiche die beiden aufwandsstärksten, vergleichsweise schwachen Blöcke (A83, C98). So bleiben außerdem fünf deiner Picks erhalten (C38, B38, A07, C03, C76). Gesamtaufwand: 3573 < 3683. Das setzt sehr effiziente Bausteine (u. a. C56, A03, C32, C00) als Backbone und gibt dir zugleich zwei klare Zugeständnisse (C76, C03). Wenn C62 für dich ein harter Anker ist, kann ich sofort auf eine Variante mit C62 umschwenken, z. B. ohne A07: \{C56, A03, C32, C00, B38, C38, C62, C03\} (3352). Alternativ (falls dir A07 extrem wichtig ist) kann ich C62 gegen C76+C03 tauschen und wir gehen auf \{C56, A03, C32, A07, C00, B38, C38, C62\} (3502). Sag mir, welcher der beiden Anker (C62 oder A07) für dich höher priorisiert ist, dann finalize ich.'\}} \\
            }
        }
    }
    & \\ \\

    \theutterance \stepcounter{utterance}  
    & & & \multicolumn{4}{p{0.6\linewidth}}{
        \cellcolor[rgb]{0.9,0.9,0.9}{
            \makecell[{{p{\linewidth}}}]{
                \texttt{\tiny{[GM$\langle$P2]}}
                \texttt{ABLEHNUNG: \{'C56', 'A03', 'C32', 'A07', 'C00', 'B38', 'C38', 'C76', 'C03'\}} \\
\texttt{VORSCHLAG: \{'C08', 'C38', 'C03', 'B38', 'A83', 'C76', 'C98', 'A07', 'C62'\}} \\
\texttt{ARGUMENT: \{'Ich bleibe bei einem fokussierten High‑Impact‑Tausch: Ich übernehme 5 deiner Picks (C38, B38, A07, C03, C76) und ersetze vier vergleichsweise aufwandsstarke/ineffiziente Bausteine (C56, A03, C32, C00) durch drei sehr wirksame Hebel (A83, C98, C08) plus den Anker C62. Gesamtaufwand: 3627 < 3683. Das wahrt deine Anker (u. a. C76, C03, A07) und steigert die Wirkung pro Aufwand deutlich. Falls C00 für dich unverzichtbar ist, kann ich auf eine schlanke Variante drehen, indem ich C00 hineinnehme und C98 + C38 herausnehme: \{C08, C03, B38, A83, C76, A07, C62, C00\} (Aufwand 3546). Alternativ, falls dir C56/C10 wichtig sind, kann ich C98 + A07 gegen C10 + C56 tauschen: \{C08, C38, C03, B38, A83, C76, C10, C56, C62\} (Aufwand 3478). Gib mir dein Muss (C00 vs. C10/C56), dann finalize ich ohne weitere Schleifen.'\}} \\
            }
        }
    }
     \\ \\

    \theutterance \stepcounter{utterance}  
    & & \multicolumn{4}{p{0.6\linewidth}}{
        \cellcolor[rgb]{0.9,0.9,0.9}{
            \makecell[{{p{\linewidth}}}]{
                \texttt{\tiny{[P1$\langle$GM]}}
                \texttt{ABLEHNUNG: \{'C56', 'A03', 'C32', 'A07', 'C00', 'B38', 'C38', 'C76', 'C03'\}} \\
\texttt{VORSCHLAG: \{'C08', 'C38', 'C03', 'B38', 'A83', 'C76', 'C98', 'A07', 'C62'\}} \\
\texttt{ARGUMENT: \{'Ich bleibe bei einem fokussierten High‑Impact‑Tausch: Ich übernehme 5 deiner Picks (C38, B38, A07, C03, C76) und ersetze vier vergleichsweise aufwandsstarke/ineffiziente Bausteine (C56, A03, C32, C00) durch drei sehr wirksame Hebel (A83, C98, C08) plus den Anker C62. Gesamtaufwand: 3627 < 3683. Das wahrt deine Anker (u. a. C76, C03, A07) und steigert die Wirkung pro Aufwand deutlich. Falls C00 für dich unverzichtbar ist, kann ich auf eine schlanke Variante drehen, indem ich C00 hineinnehme und C98 + C38 herausnehme: \{C08, C03, B38, A83, C76, A07, C62, C00\} (Aufwand 3546). Alternativ, falls dir C56/C10 wichtig sind, kann ich C98 + A07 gegen C10 + C56 tauschen: \{C08, C38, C03, B38, A83, C76, C10, C56, C62\} (Aufwand 3478). Gib mir dein Muss (C00 vs. C10/C56), dann finalize ich ohne weitere Schleifen.'\}} \\
            }
        }
    }
    & \\ \\

    \theutterance \stepcounter{utterance}  
    & \multicolumn{4}{p{0.6\linewidth}}{
        \cellcolor[rgb]{0.9,0.9,0.9}{
            \makecell[{{p{\linewidth}}}]{
                \texttt{\tiny{[P1$\rangle$GM]}}
                \texttt{ABLEHNUNG: \{'C08', 'C38', 'C03', 'B38', 'A83', 'C76', 'C98', 'A07', 'C62'\}} \\
\texttt{VORSCHLAG: \{'C56', 'A03', 'C32', 'A07', 'B38', 'C38', 'C62', 'C03', 'C98'\}} \\
\texttt{ARGUMENT: \{'Ich übernehme 6 deiner 9 Picks (C38, B38, A07, C03, C98, C62) und ersetze nur die zwei schwächsten Hebel aus deinem Paket (A83, C76) durch ein extrem effizientes Backbone (C56, A03, C32). Gesamtaufwand: 3613 < 3683. Das wahrt deine Anker (C62, A07, C03, C98) und steigert die Wirkung pro Aufwand deutlich. Wenn A83 für dich ein Muss ist, kann ich sofort C98 → (A83 + C08) drehen; Aufwand bleibt schlank und die Struktur stabil. Falls dir C76 statt C98 wichtiger ist, tausche ich C98 → C76 (Aufwand sinkt weiter). Sag mir, welcher der beiden Anker (A83 oder C76) dir wichtiger ist, dann finalize ich in der entsprechenden Variante sofort.'\}} \\
            }
        }
    }
    & & \\ \\

    \theutterance \stepcounter{utterance}  
    & & \multicolumn{4}{p{0.6\linewidth}}{
        \cellcolor[rgb]{0.9,0.9,0.9}{
            \makecell[{{p{\linewidth}}}]{
                \texttt{\tiny{[GM$\rangle$P2]}}
                \texttt{ABLEHNUNG: \{'C08', 'C38', 'C03', 'B38', 'A83', 'C76', 'C98', 'A07', 'C62'\}} \\
\texttt{VORSCHLAG: \{'C56', 'A03', 'C32', 'A07', 'B38', 'C38', 'C62', 'C03', 'C98'\}} \\
\texttt{ARGUMENT: \{'Ich übernehme 6 deiner 9 Picks (C38, B38, A07, C03, C98, C62) und ersetze nur die zwei schwächsten Hebel aus deinem Paket (A83, C76) durch ein extrem effizientes Backbone (C56, A03, C32). Gesamtaufwand: 3613 < 3683. Das wahrt deine Anker (C62, A07, C03, C98) und steigert die Wirkung pro Aufwand deutlich. Wenn A83 für dich ein Muss ist, kann ich sofort C98 → (A83 + C08) drehen; Aufwand bleibt schlank und die Struktur stabil. Falls dir C76 statt C98 wichtiger ist, tausche ich C98 → C76 (Aufwand sinkt weiter). Sag mir, welcher der beiden Anker (A83 oder C76) dir wichtiger ist, dann finalize ich in der entsprechenden Variante sofort.'\}} \\
            }
        }
    }
    & \\ \\

    \theutterance \stepcounter{utterance}  
    & & & \multicolumn{4}{p{0.6\linewidth}}{
        \cellcolor[rgb]{0.9,0.9,0.9}{
            \makecell[{{p{\linewidth}}}]{
                \texttt{\tiny{[GM$\langle$P2]}}
                \texttt{ABLEHNUNG: \{'C56', 'A03', 'C32', 'A07', 'B38', 'C38', 'C62', 'C03', 'C98'\}} \\
\texttt{VORSCHLAG: \{'C56', 'C32', 'A07', 'B38', 'C38', 'C62', 'C03', 'A83', 'C08', 'C76'\}} \\
\texttt{ARGUMENT: \{'Greife dein angebotenes Swap auf (C98 → A83 + C08) und füge einen schlanken, zielgerichteten Feinschliff hinzu (A03 → C76). Ergebnis: Wir halten 7 deiner 9 Picks (C56, C32, A07, B38, C38, C62, C03) und integrieren A83, C08 sowie C76 für mehr Wirkung pro Aufwand. Gesamtaufwand: 3618 < 3683. Das wahrt deine Anker (u. a. C62, A07, C03) und bleibt eng an deiner letzten Struktur. Falls A03 für dich nicht verhandelbar ist, kann ich 1:1 auf die Parallelvariante mit C32 → C76 drehen (Aufwand 3497) und A03 drin lassen – sag mir nur, welcher der beiden (A03 vs. C32) für dich fix ist.'\}} \\
            }
        }
    }
     \\ \\

    \theutterance \stepcounter{utterance}  
    & & \multicolumn{4}{p{0.6\linewidth}}{
        \cellcolor[rgb]{0.9,0.9,0.9}{
            \makecell[{{p{\linewidth}}}]{
                \texttt{\tiny{[P1$\langle$GM]}}
                \texttt{ABLEHNUNG: \{'C56', 'A03', 'C32', 'A07', 'B38', 'C38', 'C62', 'C03', 'C98'\}} \\
\texttt{VORSCHLAG: \{'C56', 'C32', 'A07', 'B38', 'C38', 'C62', 'C03', 'A83', 'C08', 'C76'\}} \\
\texttt{ARGUMENT: \{'Greife dein angebotenes Swap auf (C98 → A83 + C08) und füge einen schlanken, zielgerichteten Feinschliff hinzu (A03 → C76). Ergebnis: Wir halten 7 deiner 9 Picks (C56, C32, A07, B38, C38, C62, C03) und integrieren A83, C08 sowie C76 für mehr Wirkung pro Aufwand. Gesamtaufwand: 3618 < 3683. Das wahrt deine Anker (u. a. C62, A07, C03) und bleibt eng an deiner letzten Struktur. Falls A03 für dich nicht verhandelbar ist, kann ich 1:1 auf die Parallelvariante mit C32 → C76 drehen (Aufwand 3497) und A03 drin lassen – sag mir nur, welcher der beiden (A03 vs. C32) für dich fix ist.'\}} \\
            }
        }
    }
    & \\ \\

    \theutterance \stepcounter{utterance}  
    & \multicolumn{4}{p{0.6\linewidth}}{
        \cellcolor[rgb]{0.9,0.9,0.9}{
            \makecell[{{p{\linewidth}}}]{
                \texttt{\tiny{[P1$\rangle$GM]}}
                \texttt{ABLEHNUNG: \{'C56', 'C32', 'A07', 'B38', 'C38', 'C62', 'C03', 'A83', 'C08', 'C76'\}} \\
\texttt{VORSCHLAG: \{'C56', 'C32', 'A07', 'B38', 'C38', 'C62', 'C03', 'C08', 'C76', 'A03'\}} \\
\texttt{ARGUMENT: \{'Ich übernehme 9 deiner 10 Picks 1:1 und ersetze nur A83 durch A03. Gesamtaufwand: 3593 < 3683.', 'Das wahrt deine klaren Anker (u. a. C62, A07, C03, C56) und bleibt extrem nah an deiner Struktur – nur ein minimaler, sehr effizienter Swap.', 'Falls A83 für dich unverzichtbar ist, drehe ich sofort auf die Parallelvariante mit A83 statt C76: \{C56, C32, A07, B38, C38, C62, C03, A83, C08, A03\} (Aufwand 3587). Sag mir, welcher der beiden (A83 vs. C76) dir wichtiger ist, dann finalize ich entsprechend sofort.'\}} \\
            }
        }
    }
    & & \\ \\

    \theutterance \stepcounter{utterance}  
    & & \multicolumn{4}{p{0.6\linewidth}}{
        \cellcolor[rgb]{0.9,0.9,0.9}{
            \makecell[{{p{\linewidth}}}]{
                \texttt{\tiny{[GM$\rangle$P2]}}
                \texttt{ABLEHNUNG: \{'C56', 'C32', 'A07', 'B38', 'C38', 'C62', 'C03', 'A83', 'C08', 'C76'\}} \\
\texttt{VORSCHLAG: \{'C56', 'C32', 'A07', 'B38', 'C38', 'C62', 'C03', 'C08', 'C76', 'A03'\}} \\
\texttt{ARGUMENT: \{'Ich übernehme 9 deiner 10 Picks 1:1 und ersetze nur A83 durch A03. Gesamtaufwand: 3593 < 3683.', 'Das wahrt deine klaren Anker (u. a. C62, A07, C03, C56) und bleibt extrem nah an deiner Struktur – nur ein minimaler, sehr effizienter Swap.', 'Falls A83 für dich unverzichtbar ist, drehe ich sofort auf die Parallelvariante mit A83 statt C76: \{C56, C32, A07, B38, C38, C62, C03, A83, C08, A03\} (Aufwand 3587). Sag mir, welcher der beiden (A83 vs. C76) dir wichtiger ist, dann finalize ich entsprechend sofort.'\}} \\
            }
        }
    }
    & \\ \\

    \theutterance \stepcounter{utterance}  
    & & & \multicolumn{4}{p{0.6\linewidth}}{
        \cellcolor[rgb]{0.9,0.9,0.9}{
            \makecell[{{p{\linewidth}}}]{
                \texttt{\tiny{[GM$\langle$P2]}}
                \texttt{ABLEHNUNG: \{'C56', 'C32', 'A07', 'B38', 'C38', 'C62', 'C03', 'C08', 'C76', 'A03'\}} \\
\texttt{VORSCHLAG: \{'A83', 'C08', 'C76', 'C98', 'C03', 'C38', 'B38', 'C62', 'A07'\}} \\
\texttt{ARGUMENT: \{'Ich halte 7 deiner 10 Picks 1:1 (A07, B38, C38, C62, C03, C08, C76) und ersetze nur drei vergleichsweise schwächere/teure Bausteine (C56, C32, A03) durch zwei sehr starke Hebel (A83, C98). Gesamtaufwand: 3627 < 3683. Das folgt deiner eigenen Logik „C98 ↔ A83+C08“ – wir behalten C08, nehmen A83 dazu und sparen Platz durch das Streichen von C56/C32/A03, ohne das Limit zu reißen. Falls C56 für dich ein Muss ist, kann ich sofort auf eine Variante drehen: \{A83, C08, C76, C98, C03, C38, B38, C62, C56\} (Aufwand 3353) – 1:1 derselbe Kern, nur A07 → C56. Alternativ, falls C98 hart für dich ist, tausche ich C98 → (C10 + C56) und erhalte: \{A83, C08, C76, C10, C56, C03, C38, B38, C62\} (Aufwand 3478). Nenne mir deinen einen Nicht‑verhandlungs‑Anker (C56 vs. C98), dann finalize ich sofort.'\}} \\
            }
        }
    }
     \\ \\

    \theutterance \stepcounter{utterance}  
    & & \multicolumn{4}{p{0.6\linewidth}}{
        \cellcolor[rgb]{0.9,0.9,0.9}{
            \makecell[{{p{\linewidth}}}]{
                \texttt{\tiny{[P1$\langle$GM]}}
                \texttt{ABLEHNUNG: \{'C56', 'C32', 'A07', 'B38', 'C38', 'C62', 'C03', 'C08', 'C76', 'A03'\}} \\
\texttt{VORSCHLAG: \{'A83', 'C08', 'C76', 'C98', 'C03', 'C38', 'B38', 'C62', 'A07'\}} \\
\texttt{ARGUMENT: \{'Ich halte 7 deiner 10 Picks 1:1 (A07, B38, C38, C62, C03, C08, C76) und ersetze nur drei vergleichsweise schwächere/teure Bausteine (C56, C32, A03) durch zwei sehr starke Hebel (A83, C98). Gesamtaufwand: 3627 < 3683. Das folgt deiner eigenen Logik „C98 ↔ A83+C08“ – wir behalten C08, nehmen A83 dazu und sparen Platz durch das Streichen von C56/C32/A03, ohne das Limit zu reißen. Falls C56 für dich ein Muss ist, kann ich sofort auf eine Variante drehen: \{A83, C08, C76, C98, C03, C38, B38, C62, C56\} (Aufwand 3353) – 1:1 derselbe Kern, nur A07 → C56. Alternativ, falls C98 hart für dich ist, tausche ich C98 → (C10 + C56) und erhalte: \{A83, C08, C76, C10, C56, C03, C38, B38, C62\} (Aufwand 3478). Nenne mir deinen einen Nicht‑verhandlungs‑Anker (C56 vs. C98), dann finalize ich sofort.'\}} \\
            }
        }
    }
    & \\ \\

    \theutterance \stepcounter{utterance}  
    & \multicolumn{4}{p{0.6\linewidth}}{
        \cellcolor[rgb]{0.9,0.9,0.9}{
            \makecell[{{p{\linewidth}}}]{
                \texttt{\tiny{[P1$\rangle$GM]}}
                \texttt{ABLEHNUNG: \{'A83', 'C08', 'C76', 'C98', 'C03', 'C38', 'B38', 'C62', 'A07'\}} \\
\texttt{VORSCHLAG: \{'C56', 'A03', 'C08', 'C76', 'C98', 'C03', 'C38', 'B38', 'C62', 'A07'\}} \\
\texttt{ARGUMENT: \{'Ich halte 8 deiner 9 Picks 1:1 (A07, B38, C38, C62, C03, C08, C76, C98) und ersetze nur A83 durch zwei sehr schlanke, effiziente Bausteine (C56, A03).', 'Gesamtaufwand: 3662 ≤ 3683 – sauber im Limit.', 'Das ist im Kern dein Paket, nur minimal verstärkt – damit kannst du ohne weitere Einschnitte deine Anker wahren.', 'Falls A03 für dich ein Showstopper ist, drehe ich sofort auf die schlanke Variante ohne A03: \{C56, C08, C76, C98, C03, C38, B38, C62, A07\} (Aufwand 3145).', 'Falls du A83 unbedingt brauchst, halte ich A83 drin und nehme dafür C76 raus: \{A83, C08, C98, C03, C38, B38, C62, A07, C56, A03\} (Aufwand 3656).', 'Sag mir, welche der beiden Alternativen dir lieber ist – oder nimm meinen Hauptvorschlag, dann finalize ich sofort.'\}} \\
            }
        }
    }
    & & \\ \\

    \theutterance \stepcounter{utterance}  
    & & \multicolumn{4}{p{0.6\linewidth}}{
        \cellcolor[rgb]{0.9,0.9,0.9}{
            \makecell[{{p{\linewidth}}}]{
                \texttt{\tiny{[GM$\rangle$P2]}}
                \texttt{ABLEHNUNG: \{'A83', 'C08', 'C76', 'C98', 'C03', 'C38', 'B38', 'C62', 'A07'\}} \\
\texttt{VORSCHLAG: \{'C56', 'A03', 'C08', 'C76', 'C98', 'C03', 'C38', 'B38', 'C62', 'A07'\}} \\
\texttt{ARGUMENT: \{'Ich halte 8 deiner 9 Picks 1:1 (A07, B38, C38, C62, C03, C08, C76, C98) und ersetze nur A83 durch zwei sehr schlanke, effiziente Bausteine (C56, A03).', 'Gesamtaufwand: 3662 ≤ 3683 – sauber im Limit.', 'Das ist im Kern dein Paket, nur minimal verstärkt – damit kannst du ohne weitere Einschnitte deine Anker wahren.', 'Falls A03 für dich ein Showstopper ist, drehe ich sofort auf die schlanke Variante ohne A03: \{C56, C08, C76, C98, C03, C38, B38, C62, A07\} (Aufwand 3145).', 'Falls du A83 unbedingt brauchst, halte ich A83 drin und nehme dafür C76 raus: \{A83, C08, C98, C03, C38, B38, C62, A07, C56, A03\} (Aufwand 3656).', 'Sag mir, welche der beiden Alternativen dir lieber ist – oder nimm meinen Hauptvorschlag, dann finalize ich sofort.'\}} \\
            }
        }
    }
    & \\ \\

    \theutterance \stepcounter{utterance}  
    & & & \multicolumn{4}{p{0.6\linewidth}}{
        \cellcolor[rgb]{0.9,0.9,0.9}{
            \makecell[{{p{\linewidth}}}]{
                \texttt{\tiny{[GM$\langle$P2]}}
                \texttt{ABLEHNUNG: \{'C56', 'A03', 'C08', 'C76', 'C98', 'C03', 'C38', 'B38', 'C62', 'A07'\}} \\
\texttt{VORSCHLAG: \{'A83', 'C08', 'C76', 'C98', 'C03', 'C38', 'B38', 'C62', 'A07'\}} \\
\texttt{ARGUMENT: \{'Ich nehme dein Paket als Basis und tausche nur A03 + C56 → A83. Ergebnis: 8/10 deiner Picks bleiben 1:1, alle deine genannten Anker (A07, C62, C03) sind drin, der Gesamtaufwand liegt mit 3627 < 3683 klar im Limit – schlanker, fokussierter, sofort unterschriftsreif.', 'Falls C56 für dich unverzichtbar ist, kann ich C56 behalten und stattdessen B38 oder C38 streichen, beide innerhalb des Limits: \{A83, C08, C76, C98, C03, C38, C62, A07, C56\} (3468) ODER \{A83, C08, C76, C98, C03, B38, C62, A07, C56\} (3533).', 'Falls A03 für dich ein Muss ist (und du C76 entbehren kannst), übernehme ich deine eigene Alternativstruktur 1:1: \{A83, C08, C98, C03, C38, B38, C62, A07, C56, A03\} (3656). Gib mir deinen einen Fixpunkt (C56 vs. C76 vs. A03), dann finalize ich ohne weitere Schleifen.'\}} \\
            }
        }
    }
     \\ \\

    \theutterance \stepcounter{utterance}  
    & & \multicolumn{4}{p{0.6\linewidth}}{
        \cellcolor[rgb]{0.9,0.9,0.9}{
            \makecell[{{p{\linewidth}}}]{
                \texttt{\tiny{[P1$\langle$GM]}}
                \texttt{ABLEHNUNG: \{'C56', 'A03', 'C08', 'C76', 'C98', 'C03', 'C38', 'B38', 'C62', 'A07'\}} \\
\texttt{VORSCHLAG: \{'A83', 'C08', 'C76', 'C98', 'C03', 'C38', 'B38', 'C62', 'A07'\}} \\
\texttt{ARGUMENT: \{'Ich nehme dein Paket als Basis und tausche nur A03 + C56 → A83. Ergebnis: 8/10 deiner Picks bleiben 1:1, alle deine genannten Anker (A07, C62, C03) sind drin, der Gesamtaufwand liegt mit 3627 < 3683 klar im Limit – schlanker, fokussierter, sofort unterschriftsreif.', 'Falls C56 für dich unverzichtbar ist, kann ich C56 behalten und stattdessen B38 oder C38 streichen, beide innerhalb des Limits: \{A83, C08, C76, C98, C03, C38, C62, A07, C56\} (3468) ODER \{A83, C08, C76, C98, C03, B38, C62, A07, C56\} (3533).', 'Falls A03 für dich ein Muss ist (und du C76 entbehren kannst), übernehme ich deine eigene Alternativstruktur 1:1: \{A83, C08, C98, C03, C38, B38, C62, A07, C56, A03\} (3656). Gib mir deinen einen Fixpunkt (C56 vs. C76 vs. A03), dann finalize ich ohne weitere Schleifen.'\}} \\
            }
        }
    }
    & \\ \\

    \theutterance \stepcounter{utterance}  
    & \multicolumn{4}{p{0.6\linewidth}}{
        \cellcolor[rgb]{0.9,0.9,0.9}{
            \makecell[{{p{\linewidth}}}]{
                \texttt{\tiny{[P1$\rangle$GM]}}
                \texttt{VORSCHLAG: \{'A83', 'C08', 'C98', 'C03', 'C38', 'B38', 'C62', 'A07', 'C56', 'A03'\}} \\
\texttt{ARGUMENT: \{'Das ist exakt deine zuvor angebotene A03‑Muss‑Variante; Gesamtaufwand 3656 ≤ 3683. Sie übernimmt deinen Kern 1:1 und ergänzt nur C56+A03, ohne dass wir etwas streichen müssen – sofort unterschriftsreif.', 'Falls dir C76 wichtiger als A83 ist, kann ich 1:1 A83 → C76 tauschen; Aufwand bleibt bei 3662 ≤ 3683. Gib mir nur das Go.'\}} \\
            }
        }
    }
    & & \\ \\

    \theutterance \stepcounter{utterance}  
    & & \multicolumn{4}{p{0.6\linewidth}}{
        \cellcolor[rgb]{0.9,0.9,0.9}{
            \makecell[{{p{\linewidth}}}]{
                \texttt{\tiny{[GM$\rangle$P2]}}
                \texttt{VORSCHLAG: \{'A83', 'C08', 'C98', 'C03', 'C38', 'B38', 'C62', 'A07', 'C56', 'A03'\}} \\
\texttt{ARGUMENT: \{'Das ist exakt deine zuvor angebotene A03‑Muss‑Variante; Gesamtaufwand 3656 ≤ 3683. Sie übernimmt deinen Kern 1:1 und ergänzt nur C56+A03, ohne dass wir etwas streichen müssen – sofort unterschriftsreif.', 'Falls dir C76 wichtiger als A83 ist, kann ich 1:1 A83 → C76 tauschen; Aufwand bleibt bei 3662 ≤ 3683. Gib mir nur das Go.'\}} \\
            }
        }
    }
    & \\ \\

    \theutterance \stepcounter{utterance}  
    & & & \multicolumn{4}{p{0.6\linewidth}}{
        \cellcolor[rgb]{0.9,0.9,0.9}{
            \makecell[{{p{\linewidth}}}]{
                \texttt{\tiny{[GM$\langle$P2]}}
                \texttt{ABLEHNUNG: \{'A83', 'C08', 'C98', 'C03', 'C38', 'B38', 'C62', 'A07', 'C56', 'A03'\}} \\
\texttt{VORSCHLAG: \{'A83', 'C08', 'C98', 'C03', 'C38', 'B38', 'C62', 'A07', 'C76'\}} \\
\texttt{ARGUMENT: \{'Ich übernehme 8/10 deiner Picks 1:1 (A83, C08, C98, C03, C38, B38, C62, A07) und tausche nur A03 + C56 → C76. Gesamtaufwand: 3627 ≤ 3683. Das wahrt deine wiederholt genannten Anker (A07, C62, C03), strafft zugleich zwei vergleichsweise schwächere Per‑Cost‑Bausteine und bündelt den Hebel C76 – sofort unterschriftsreif.', 'Falls C56 für dich unverzichtbar ist, kann ich 1:1 auf die schlanke Parallelvariante drehen: \{A83, C08, C98, C03, C38, B38, C62, C76, C56\} (Aufwand 3353) – gleicher Kern, nur A07 → C56.'\}} \\
            }
        }
    }
     \\ \\

    \theutterance \stepcounter{utterance}  
    & & \multicolumn{4}{p{0.6\linewidth}}{
        \cellcolor[rgb]{0.9,0.9,0.9}{
            \makecell[{{p{\linewidth}}}]{
                \texttt{\tiny{[P1$\langle$GM]}}
                \texttt{ABLEHNUNG: \{'A83', 'C08', 'C98', 'C03', 'C38', 'B38', 'C62', 'A07', 'C56', 'A03'\}} \\
\texttt{VORSCHLAG: \{'A83', 'C08', 'C98', 'C03', 'C38', 'B38', 'C62', 'A07', 'C76'\}} \\
\texttt{ARGUMENT: \{'Ich übernehme 8/10 deiner Picks 1:1 (A83, C08, C98, C03, C38, B38, C62, A07) und tausche nur A03 + C56 → C76. Gesamtaufwand: 3627 ≤ 3683. Das wahrt deine wiederholt genannten Anker (A07, C62, C03), strafft zugleich zwei vergleichsweise schwächere Per‑Cost‑Bausteine und bündelt den Hebel C76 – sofort unterschriftsreif.', 'Falls C56 für dich unverzichtbar ist, kann ich 1:1 auf die schlanke Parallelvariante drehen: \{A83, C08, C98, C03, C38, B38, C62, C76, C56\} (Aufwand 3353) – gleicher Kern, nur A07 → C56.'\}} \\
            }
        }
    }
    & \\ \\

    \theutterance \stepcounter{utterance}  
    & \multicolumn{4}{p{0.6\linewidth}}{
        \cellcolor[rgb]{0.9,0.9,0.9}{
            \makecell[{{p{\linewidth}}}]{
                \texttt{\tiny{[P1$\rangle$GM]}}
                \texttt{ABLEHNUNG: \{'A83', 'C08', 'C98', 'C03', 'C38', 'B38', 'C62', 'A07', 'C76'\}} \\
\texttt{VORSCHLAG: \{'A83', 'C08', 'C76', 'C03', 'C38', 'B38', 'C62', 'A07', 'C56', 'A03'\}} \\
\texttt{ARGUMENT: \{'Ich übernehme 8 deiner 9 Picks 1:1 und ersetze nur C98 durch das extrem schlanke Doppel C56 + A03.', 'Gesamtaufwand: 3497 < 3683 – also deutlich unter dem Limit, ohne weitere Streichungen.', 'Deine Anker (A07, C62, C03) bleiben vollständig erhalten; Struktur und Schwerpunkt deines Pakets bleiben intakt.', 'Du selbst hattest eine C98→(… + C56)‑Spur geöffnet; diese Variante ist noch effizienter und sofort unterschriftsreif.', 'Falls C98 für dich unverzichtbar ist, biete ich eine unmittelbare Alternative an: Behalte C98, nimm C56 dazu und streiche lediglich C08: \{A83, C76, C98, C03, C38, B38, C62, A07, C56\} (Aufwand 3548). Gib mir dein Go für Haupt{-} oder Alternativvariante, dann finalize ich ohne weitere Schleifen.'\}} \\
            }
        }
    }
    & & \\ \\

    \theutterance \stepcounter{utterance}  
    & & \multicolumn{4}{p{0.6\linewidth}}{
        \cellcolor[rgb]{0.9,0.9,0.9}{
            \makecell[{{p{\linewidth}}}]{
                \texttt{\tiny{[GM$\rangle$P2]}}
                \texttt{ABLEHNUNG: \{'A83', 'C08', 'C98', 'C03', 'C38', 'B38', 'C62', 'A07', 'C76'\}} \\
\texttt{VORSCHLAG: \{'A83', 'C08', 'C76', 'C03', 'C38', 'B38', 'C62', 'A07', 'C56', 'A03'\}} \\
\texttt{ARGUMENT: \{'Ich übernehme 8 deiner 9 Picks 1:1 und ersetze nur C98 durch das extrem schlanke Doppel C56 + A03.', 'Gesamtaufwand: 3497 < 3683 – also deutlich unter dem Limit, ohne weitere Streichungen.', 'Deine Anker (A07, C62, C03) bleiben vollständig erhalten; Struktur und Schwerpunkt deines Pakets bleiben intakt.', 'Du selbst hattest eine C98→(… + C56)‑Spur geöffnet; diese Variante ist noch effizienter und sofort unterschriftsreif.', 'Falls C98 für dich unverzichtbar ist, biete ich eine unmittelbare Alternative an: Behalte C98, nimm C56 dazu und streiche lediglich C08: \{A83, C76, C98, C03, C38, B38, C62, A07, C56\} (Aufwand 3548). Gib mir dein Go für Haupt{-} oder Alternativvariante, dann finalize ich ohne weitere Schleifen.'\}} \\
            }
        }
    }
    & \\ \\

    \theutterance \stepcounter{utterance}  
    & & & \multicolumn{4}{p{0.6\linewidth}}{
        \cellcolor[rgb]{0.9,0.9,0.9}{
            \makecell[{{p{\linewidth}}}]{
                \texttt{\tiny{[GM$\langle$P2]}}
                \texttt{ABLEHNUNG: \{'A83', 'C08', 'C76', 'C03', 'C38', 'B38', 'C62', 'A07', 'C56', 'A03'\}} \\
\texttt{VORSCHLAG: \{'A83', 'C08', 'C76', 'C98', 'C03', 'C38', 'C62', 'A07', 'C56'\}} \\
\texttt{ARGUMENT: \{'Ich halte 8/10 deiner Picks 1:1 (A83, C08, C76, C03, C38, C62, A07, C56) und tausche nur A03 + B38 → C98. Gesamtaufwand: 3468 < 3683 – deutlich unter dem Limit.', 'Das bewahrt deine wiederholt gesetzten Anker (A07, C62, C56) und bringt mit C98 gezielt einen starken Hebel hinein, ohne am Rahmen zu kratzen.', 'Falls B38 für dich unverzichtbar ist, biete ich die Parallelvariante mit C38 statt B38: \{A83, C08, C76, C98, C03, B38, C62, A07, C56\} (Aufwand 3533).', 'Wenn du stattdessen strikt an beiden (B38 und C38) festhalten willst, können wir A03+C56 → C98 drehen: \{A83, C08, C76, C98, C03, C38, B38, C62, A07\} (3627), aber mein Hauptvorschlag ist effizienter und sofort unterschriftsreif.'\}} \\
            }
        }
    }
     \\ \\

    \theutterance \stepcounter{utterance}  
    & & \multicolumn{4}{p{0.6\linewidth}}{
        \cellcolor[rgb]{0.9,0.9,0.9}{
            \makecell[{{p{\linewidth}}}]{
                \texttt{\tiny{[P1$\langle$GM]}}
                \texttt{ABLEHNUNG: \{'A83', 'C08', 'C76', 'C03', 'C38', 'B38', 'C62', 'A07', 'C56', 'A03'\}} \\
\texttt{VORSCHLAG: \{'A83', 'C08', 'C76', 'C98', 'C03', 'C38', 'C62', 'A07', 'C56'\}} \\
\texttt{ARGUMENT: \{'Ich halte 8/10 deiner Picks 1:1 (A83, C08, C76, C03, C38, C62, A07, C56) und tausche nur A03 + B38 → C98. Gesamtaufwand: 3468 < 3683 – deutlich unter dem Limit.', 'Das bewahrt deine wiederholt gesetzten Anker (A07, C62, C56) und bringt mit C98 gezielt einen starken Hebel hinein, ohne am Rahmen zu kratzen.', 'Falls B38 für dich unverzichtbar ist, biete ich die Parallelvariante mit C38 statt B38: \{A83, C08, C76, C98, C03, B38, C62, A07, C56\} (Aufwand 3533).', 'Wenn du stattdessen strikt an beiden (B38 und C38) festhalten willst, können wir A03+C56 → C98 drehen: \{A83, C08, C76, C98, C03, C38, B38, C62, A07\} (3627), aber mein Hauptvorschlag ist effizienter und sofort unterschriftsreif.'\}} \\
            }
        }
    }
    & \\ \\

    \theutterance \stepcounter{utterance}  
    & \multicolumn{4}{p{0.6\linewidth}}{
        \cellcolor[rgb]{0.9,0.9,0.9}{
            \makecell[{{p{\linewidth}}}]{
                \texttt{\tiny{[P1$\rangle$GM]}}
                \texttt{VORSCHLAG: \{'A83', 'C08', 'C98', 'C03', 'C38', 'C62', 'A07', 'C56', 'A03'\}} \\
\texttt{ARGUMENT: \{'Ich übernehme 8/9 deiner Picks 1:1 und tausche nur C76 → A03; Gesamtaufwand 3437 < 3683.', 'Deine wiederholt gesetzten Anker (A07, C62, C03) sowie C98, C56, C08 und C38 bleiben vollständig erhalten – minimaler Eingriff, sofort unterschriftsreif.', 'Falls C76 für dich unverzichtbar ist, biete ich als schlanke Alternative C08 → B38 an; Aufwand 3548; Set: \{A83, C76, C98, C03, C38, B38, C62, A07, C56\}.'\}} \\
            }
        }
    }
    & & \\ \\

    \theutterance \stepcounter{utterance}  
    & & \multicolumn{4}{p{0.6\linewidth}}{
        \cellcolor[rgb]{0.9,0.9,0.9}{
            \makecell[{{p{\linewidth}}}]{
                \texttt{\tiny{[GM$\rangle$P2]}}
                \texttt{VORSCHLAG: \{'A83', 'C08', 'C98', 'C03', 'C38', 'C62', 'A07', 'C56', 'A03'\}} \\
\texttt{ARGUMENT: \{'Ich übernehme 8/9 deiner Picks 1:1 und tausche nur C76 → A03; Gesamtaufwand 3437 < 3683.', 'Deine wiederholt gesetzten Anker (A07, C62, C03) sowie C98, C56, C08 und C38 bleiben vollständig erhalten – minimaler Eingriff, sofort unterschriftsreif.', 'Falls C76 für dich unverzichtbar ist, biete ich als schlanke Alternative C08 → B38 an; Aufwand 3548; Set: \{A83, C76, C98, C03, C38, B38, C62, A07, C56\}.'\}} \\
            }
        }
    }
    & \\ \\

    \theutterance \stepcounter{utterance}  
    & & & \multicolumn{4}{p{0.6\linewidth}}{
        \cellcolor[rgb]{0.9,0.9,0.9}{
            \makecell[{{p{\linewidth}}}]{
                \texttt{\tiny{[GM$\langle$P2]}}
                \texttt{ABLEHNUNG: \{'A83', 'C08', 'C98', 'C03', 'C38', 'C62', 'A07', 'C56', 'A03'\}} \\
\texttt{VORSCHLAG: \{'A83', 'C08', 'C98', 'C03', 'C38', 'A07', 'C56', 'C10', 'C76'\}} \\
\texttt{ARGUMENT: \{'Ich übernehme 7/9 deiner Picks 1:1 und mache nur zwei zielgerichtete Swaps: C62 → C10 und A03 → C76. Gesamtaufwand: 3639 < 3683.', 'Damit bleiben deine wiederholten Anker (A07, C03, A83) vollständig erhalten; zugleich bündeln wir mehr Wirkung je Aufwand ohne das Limit anzutasten.', 'Falls C10 für dich heikel ist, kann ich die Parallelvariante sofort drehen: \{A83, C08, C98, C03, C38, A07, C56, C76, C62\} (3468).', 'Falls A07 flexibel ist, biete ich die noch schlankere, sehr fokussierte Option: \{A83, C08, C98, C03, C38, C10, C76, B38, C56\} (3524).'\}} \\
            }
        }
    }
     \\ \\

    \theutterance \stepcounter{utterance}  
    & & \multicolumn{4}{p{0.6\linewidth}}{
        \cellcolor[rgb]{0.9,0.9,0.9}{
            \makecell[{{p{\linewidth}}}]{
                \texttt{\tiny{[P1$\langle$GM]}}
                \texttt{ABLEHNUNG: \{'A83', 'C08', 'C98', 'C03', 'C38', 'C62', 'A07', 'C56', 'A03'\}} \\
\texttt{VORSCHLAG: \{'A83', 'C08', 'C98', 'C03', 'C38', 'A07', 'C56', 'C10', 'C76'\}} \\
\texttt{ARGUMENT: \{'Ich übernehme 7/9 deiner Picks 1:1 und mache nur zwei zielgerichtete Swaps: C62 → C10 und A03 → C76. Gesamtaufwand: 3639 < 3683.', 'Damit bleiben deine wiederholten Anker (A07, C03, A83) vollständig erhalten; zugleich bündeln wir mehr Wirkung je Aufwand ohne das Limit anzutasten.', 'Falls C10 für dich heikel ist, kann ich die Parallelvariante sofort drehen: \{A83, C08, C98, C03, C38, A07, C56, C76, C62\} (3468).', 'Falls A07 flexibel ist, biete ich die noch schlankere, sehr fokussierte Option: \{A83, C08, C98, C03, C38, C10, C76, B38, C56\} (3524).'\}} \\
            }
        }
    }
    & \\ \\

    \theutterance \stepcounter{utterance}  
    & \multicolumn{4}{p{0.6\linewidth}}{
        \cellcolor[rgb]{0.9,0.9,0.9}{
            \makecell[{{p{\linewidth}}}]{
                \texttt{\tiny{[P1$\rangle$GM]}}
                \texttt{ABLEHNUNG: \{'A83', 'C08', 'C98', 'C03', 'C38', 'A07', 'C56', 'C10', 'C76'\}} \\
\texttt{VORSCHLAG: \{'C56', 'A03', 'C32', 'C00', 'A07', 'C38', 'C03', 'A83', 'B38'\}} \\
\texttt{ARGUMENT: \{'Ich übernehme 5 deiner 9 Picks 1:1 (A83, C03, C38, A07, C56) und ersetze nur C08, C98, C10, C76 durch ein robustes Backbone (A03, C32, C00, B38). Gesamtaufwand: 3567 < 3683 – schlank, stabil, sofort unterschriftsreif.', 'Deine wiederholt genannten Anker bleiben gewahrt (A07, C03, A83).', 'Falls dir einzelne Hebel unverzichtbar sind, kann ich sofort drehen – alle Varianten ≤ 3683: a) Mit C98 (statt C00): Aufwand 3494 – \{C56, A03, C32, A07, C38, C03, A83, B38, C98\}. b) Mit C76 (statt C32): Aufwand 3477 – \{C56, A03, C00, A07, C38, C03, A83, B38, C76\}. c) Mit C10 (statt C00): Aufwand 3619 – \{C56, A03, C32, A07, C38, C03, A83, B38, C10\}. d) Mit C08: entweder ohne B38 (3487) – \{C56, A03, C32, C00, A07, C38, C03, A83, C08\} oder ohne C38 (3552) – \{C56, A03, C32, C00, A07, B38, C03, A83, C08\}. Gib mir deinen einen Fixpunkt, dann finalize ich sofort.'\}} \\
            }
        }
    }
    & & \\ \\

    \theutterance \stepcounter{utterance}  
    & & \multicolumn{4}{p{0.6\linewidth}}{
        \cellcolor[rgb]{0.9,0.9,0.9}{
            \makecell[{{p{\linewidth}}}]{
                \texttt{\tiny{[GM$\rangle$P2]}}
                \texttt{ABLEHNUNG: \{'A83', 'C08', 'C98', 'C03', 'C38', 'A07', 'C56', 'C10', 'C76'\}} \\
\texttt{VORSCHLAG: \{'C56', 'A03', 'C32', 'C00', 'A07', 'C38', 'C03', 'A83', 'B38'\}} \\
\texttt{ARGUMENT: \{'Ich übernehme 5 deiner 9 Picks 1:1 (A83, C03, C38, A07, C56) und ersetze nur C08, C98, C10, C76 durch ein robustes Backbone (A03, C32, C00, B38). Gesamtaufwand: 3567 < 3683 – schlank, stabil, sofort unterschriftsreif.', 'Deine wiederholt genannten Anker bleiben gewahrt (A07, C03, A83).', 'Falls dir einzelne Hebel unverzichtbar sind, kann ich sofort drehen – alle Varianten ≤ 3683: a) Mit C98 (statt C00): Aufwand 3494 – \{C56, A03, C32, A07, C38, C03, A83, B38, C98\}. b) Mit C76 (statt C32): Aufwand 3477 – \{C56, A03, C00, A07, C38, C03, A83, B38, C76\}. c) Mit C10 (statt C00): Aufwand 3619 – \{C56, A03, C32, A07, C38, C03, A83, B38, C10\}. d) Mit C08: entweder ohne B38 (3487) – \{C56, A03, C32, C00, A07, C38, C03, A83, C08\} oder ohne C38 (3552) – \{C56, A03, C32, C00, A07, B38, C03, A83, C08\}. Gib mir deinen einen Fixpunkt, dann finalize ich sofort.'\}} \\
            }
        }
    }
    & \\ \\

    \theutterance \stepcounter{utterance}  
    & & & \multicolumn{4}{p{0.6\linewidth}}{
        \cellcolor[rgb]{0.9,0.9,0.9}{
            \makecell[{{p{\linewidth}}}]{
                \texttt{\tiny{[GM$\langle$P2]}}
                \texttt{ABLEHNUNG: \{'C56', 'A03', 'C32', 'C00', 'A07', 'C38', 'C03', 'A83', 'B38'\}} \\
\texttt{VORSCHLAG: \{'A83', 'C08', 'C76', 'C98', 'C03', 'C38', 'B38', 'A07', 'C56'\}} \\
\texttt{ARGUMENT: \{'Ich übernehme 6/9 deiner Picks 1:1 (A83, C03, C38, B38, A07, C56) und tausche nur die drei schwersten Bausteine (A03, C32, C00) gegen drei schlanke, wirkungsstarke Hebel (C08, C76, C98).', 'Gesamtaufwand: 3026 << 3683 – klar im Limit, kompakt und fokussiert.', 'Das wahrt deine wiederholt gesetzten Anker (A07, C03) und bleibt sehr nah an deiner Struktur, nur effizienter je Aufwand.', 'Wenn C00 für dich unverzichtbar ist, kann ich unmittelbar drehen: ersetze C38 → C00; Set: \{A83, C08, C76, C98, C03, B38, A07, C56, C00\} (Aufwand 3652).', 'Wenn A03 ein Muss ist, drehe ich 1:1 B38 → A03; Set: \{A83, C08, C76, C98, C03, C38, A07, C56, A03\} (Aufwand 3324).'\}} \\
            }
        }
    }
     \\ \\

    \theutterance \stepcounter{utterance}  
    & & \multicolumn{4}{p{0.6\linewidth}}{
        \cellcolor[rgb]{0.9,0.9,0.9}{
            \makecell[{{p{\linewidth}}}]{
                \texttt{\tiny{[P1$\langle$GM]}}
                \texttt{ABLEHNUNG: \{'C56', 'A03', 'C32', 'C00', 'A07', 'C38', 'C03', 'A83', 'B38'\}} \\
\texttt{VORSCHLAG: \{'A83', 'C08', 'C76', 'C98', 'C03', 'C38', 'B38', 'A07', 'C56'\}} \\
\texttt{ARGUMENT: \{'Ich übernehme 6/9 deiner Picks 1:1 (A83, C03, C38, B38, A07, C56) und tausche nur die drei schwersten Bausteine (A03, C32, C00) gegen drei schlanke, wirkungsstarke Hebel (C08, C76, C98).', 'Gesamtaufwand: 3026 << 3683 – klar im Limit, kompakt und fokussiert.', 'Das wahrt deine wiederholt gesetzten Anker (A07, C03) und bleibt sehr nah an deiner Struktur, nur effizienter je Aufwand.', 'Wenn C00 für dich unverzichtbar ist, kann ich unmittelbar drehen: ersetze C38 → C00; Set: \{A83, C08, C76, C98, C03, B38, A07, C56, C00\} (Aufwand 3652).', 'Wenn A03 ein Muss ist, drehe ich 1:1 B38 → A03; Set: \{A83, C08, C76, C98, C03, C38, A07, C56, A03\} (Aufwand 3324).'\}} \\
            }
        }
    }
    & \\ \\

    \theutterance \stepcounter{utterance}  
    & \multicolumn{4}{p{0.6\linewidth}}{
        \cellcolor[rgb]{0.9,0.9,0.9}{
            \makecell[{{p{\linewidth}}}]{
                \texttt{\tiny{[P1$\rangle$GM]}}
                \texttt{ABLEHNUNG: \{'A83', 'C08', 'C76', 'C98', 'C03', 'C38', 'B38', 'A07', 'C56'\}} \\
\texttt{VORSCHLAG: \{'A83', 'C08', 'C03', 'C38', 'B38', 'A07', 'C56', 'C32', 'A03', 'C62'\}} \\
\texttt{ARGUMENT: \{'Ich übernehme 7 deiner 9 Picks 1:1 (A83, C08, C03, C38, B38, A07, C56) und ersetze nur C76 + C98 durch ein sehr solides Trio (A03, C32, C62).', 'Gesamtaufwand: 3587 <= 3683 – sauber im Limit.', 'Das wahrt deine Anker (u. a. A07, C03) und erhöht zugleich die Wirkung je Aufwand; minimaler, zielgerichteter Eingriff.', 'Falls C98 für dich unverzichtbar ist, drehe ich sofort C32 → C98: \{A83, C08, C03, C38, B38, A07, C56, A03, C62, C98\} (Aufwand 3656).', 'Falls C76 dein Fixpunkt ist, gehe ich 1:1 auf C32 → C76: \{A83, C08, C03, C38, B38, A07, C56, A03, C62, C76\} (Aufwand 3497).', 'Falls du beide (C98 und C76) brauchst, streiche ich C62 und C08 und gebe dir: \{A83, C03, C38, B38, A07, C56, A03, C98, C76\} (Aufwand 3404). Sag mir, welcher Fixpunkt dir am wichtigsten ist, dann finalize ich sofort.'\}} \\
            }
        }
    }
    & & \\ \\

    \theutterance \stepcounter{utterance}  
    & & \multicolumn{4}{p{0.6\linewidth}}{
        \cellcolor[rgb]{0.9,0.9,0.9}{
            \makecell[{{p{\linewidth}}}]{
                \texttt{\tiny{[GM$\rangle$P2]}}
                \texttt{ABLEHNUNG: \{'A83', 'C08', 'C76', 'C98', 'C03', 'C38', 'B38', 'A07', 'C56'\}} \\
\texttt{VORSCHLAG: \{'A83', 'C08', 'C03', 'C38', 'B38', 'A07', 'C56', 'C32', 'A03', 'C62'\}} \\
\texttt{ARGUMENT: \{'Ich übernehme 7 deiner 9 Picks 1:1 (A83, C08, C03, C38, B38, A07, C56) und ersetze nur C76 + C98 durch ein sehr solides Trio (A03, C32, C62).', 'Gesamtaufwand: 3587 <= 3683 – sauber im Limit.', 'Das wahrt deine Anker (u. a. A07, C03) und erhöht zugleich die Wirkung je Aufwand; minimaler, zielgerichteter Eingriff.', 'Falls C98 für dich unverzichtbar ist, drehe ich sofort C32 → C98: \{A83, C08, C03, C38, B38, A07, C56, A03, C62, C98\} (Aufwand 3656).', 'Falls C76 dein Fixpunkt ist, gehe ich 1:1 auf C32 → C76: \{A83, C08, C03, C38, B38, A07, C56, A03, C62, C76\} (Aufwand 3497).', 'Falls du beide (C98 und C76) brauchst, streiche ich C62 und C08 und gebe dir: \{A83, C03, C38, B38, A07, C56, A03, C98, C76\} (Aufwand 3404). Sag mir, welcher Fixpunkt dir am wichtigsten ist, dann finalize ich sofort.'\}} \\
            }
        }
    }
    & \\ \\

    \theutterance \stepcounter{utterance}  
    & & & \multicolumn{4}{p{0.6\linewidth}}{
        \cellcolor[rgb]{0.9,0.9,0.9}{
            \makecell[{{p{\linewidth}}}]{
                \texttt{\tiny{[GM$\langle$P2]}}
                \texttt{ABLEHNUNG: \{'A83', 'C08', 'C03', 'C38', 'B38', 'A07', 'C56', 'C32', 'A03', 'C62'\}} \\
\texttt{VORSCHLAG: \{'A83', 'C08', 'C03', 'C38', 'B38', 'A07', 'C56', 'A03', 'C62', 'C76'\}} \\
\texttt{ARGUMENT: \{'Minimaler Pivot auf deiner eigenen Option: C32 → C76. Wir behalten 9/10 deiner Picks 1:1; Aufwand 3497 < 3683 – sauber im Limit und sofort entscheidungsreif.', 'Falls dir C32 doch wichtig ist, kann ich alternativ 1:1 A03 → C76 drehen (Aufwand 3618). Bestätige kurz, welche der beiden Varianten du bevorzugst, dann finalize ich ohne weitere Schleifen.'\}} \\
            }
        }
    }
     \\ \\

    \theutterance \stepcounter{utterance}  
    & & \multicolumn{4}{p{0.6\linewidth}}{
        \cellcolor[rgb]{0.9,0.9,0.9}{
            \makecell[{{p{\linewidth}}}]{
                \texttt{\tiny{[P1$\langle$GM]}}
                \texttt{ABLEHNUNG: \{'A83', 'C08', 'C03', 'C38', 'B38', 'A07', 'C56', 'C32', 'A03', 'C62'\}} \\
\texttt{VORSCHLAG: \{'A83', 'C08', 'C03', 'C38', 'B38', 'A07', 'C56', 'A03', 'C62', 'C76'\}} \\
\texttt{ARGUMENT: \{'Minimaler Pivot auf deiner eigenen Option: C32 → C76. Wir behalten 9/10 deiner Picks 1:1; Aufwand 3497 < 3683 – sauber im Limit und sofort entscheidungsreif.', 'Falls dir C32 doch wichtig ist, kann ich alternativ 1:1 A03 → C76 drehen (Aufwand 3618). Bestätige kurz, welche der beiden Varianten du bevorzugst, dann finalize ich ohne weitere Schleifen.'\}} \\
            }
        }
    }
    & \\ \\

    \theutterance \stepcounter{utterance}  
    & \multicolumn{4}{p{0.6\linewidth}}{
        \cellcolor[rgb]{0.9,0.9,0.9}{
            \makecell[{{p{\linewidth}}}]{
                \texttt{\tiny{[P1$\rangle$GM]}}
                \texttt{ABLEHNUNG: \{'A83', 'C08', 'C03', 'C38', 'B38', 'A07', 'C56', 'A03', 'C62', 'C76'\}} \\
\texttt{VORSCHLAG: \{'C08', 'C03', 'C38', 'B38', 'A07', 'C56', 'A03', 'C62', 'C76', 'C32'\}} \\
\texttt{ARGUMENT: \{'Minimaler 1‑zu‑1‑Tausch: A83 → C32. Wir behalten 9/10 deiner Picks unverändert; Aufwand 3593 ≤ 3683 – sauber im Limit.', 'Das folgt deiner eigenen Linie (du hattest C32 bereits in Varianten drin) und stärkt das Rückgrat ohne an deinen Ankern (A07, C62, C03, C56) zu rütteln.', 'Falls A83 für dich unverzichtbar ist, biete ich die Parallelvariante C76 → C32 an; Aufwand 3587 ≤ 3683. Gib kurz durch, welche der beiden du bestätigst, dann finalize ich sofort.'\}} \\
            }
        }
    }
    & & \\ \\

    \theutterance \stepcounter{utterance}  
    & & \multicolumn{4}{p{0.6\linewidth}}{
        \cellcolor[rgb]{0.9,0.9,0.9}{
            \makecell[{{p{\linewidth}}}]{
                \texttt{\tiny{[GM$\rangle$P2]}}
                \texttt{ABLEHNUNG: \{'A83', 'C08', 'C03', 'C38', 'B38', 'A07', 'C56', 'A03', 'C62', 'C76'\}} \\
\texttt{VORSCHLAG: \{'C08', 'C03', 'C38', 'B38', 'A07', 'C56', 'A03', 'C62', 'C76', 'C32'\}} \\
\texttt{ARGUMENT: \{'Minimaler 1‑zu‑1‑Tausch: A83 → C32. Wir behalten 9/10 deiner Picks unverändert; Aufwand 3593 ≤ 3683 – sauber im Limit.', 'Das folgt deiner eigenen Linie (du hattest C32 bereits in Varianten drin) und stärkt das Rückgrat ohne an deinen Ankern (A07, C62, C03, C56) zu rütteln.', 'Falls A83 für dich unverzichtbar ist, biete ich die Parallelvariante C76 → C32 an; Aufwand 3587 ≤ 3683. Gib kurz durch, welche der beiden du bestätigst, dann finalize ich sofort.'\}} \\
            }
        }
    }
    & \\ \\

    \theutterance \stepcounter{utterance}  
    & & & \multicolumn{4}{p{0.6\linewidth}}{
        \cellcolor[rgb]{0.9,0.9,0.9}{
            \makecell[{{p{\linewidth}}}]{
                \texttt{\tiny{[GM$\langle$P2]}}
                \texttt{ABLEHNUNG: \{'C08', 'C03', 'C38', 'B38', 'A07', 'C56', 'A03', 'C62', 'C76', 'C32'\}} \\
\texttt{VORSCHLAG: \{'A83', 'C08', 'C76', 'C98', 'C03', 'C38', 'B38', 'C62', 'A07'\}} \\
\texttt{ARGUMENT: \{'Ich halte 7/10 deiner Picks 1:1 (C08, C03, C38, B38, A07, C62, C76) und tausche nur A03 + C32 → A83 + C98. Gesamtaufwand: 3627 ≤ 3683. Das ist ein minimaler, fokussierter Pivot mit sehr hohem Wirkungsbeitrag pro Aufwand – sofort entscheidungsreif.', 'Falls A03 für dich unverzichtbar ist (und C62 flexibel), biete ich diese schlanke Parallelvariante an: \{A83, C08, C76, C98, C03, C38, B38, A07, A03\} (Aufwand 3596).', 'Falls C56 dein Fixpunkt ist, drehe ich stattdessen auf: \{A83, C08, C76, C98, C03, C38, C62, A07, C56\} (Aufwand 3468). Nenne mir deinen einen Fixpunkt (A03 vs. C56), dann finalize ich sofort.'\}} \\
            }
        }
    }
     \\ \\

    \theutterance \stepcounter{utterance}  
    & & \multicolumn{4}{p{0.6\linewidth}}{
        \cellcolor[rgb]{0.9,0.9,0.9}{
            \makecell[{{p{\linewidth}}}]{
                \texttt{\tiny{[P1$\langle$GM]}}
                \texttt{ABLEHNUNG: \{'C08', 'C03', 'C38', 'B38', 'A07', 'C56', 'A03', 'C62', 'C76', 'C32'\}} \\
\texttt{VORSCHLAG: \{'A83', 'C08', 'C76', 'C98', 'C03', 'C38', 'B38', 'C62', 'A07'\}} \\
\texttt{ARGUMENT: \{'Ich halte 7/10 deiner Picks 1:1 (C08, C03, C38, B38, A07, C62, C76) und tausche nur A03 + C32 → A83 + C98. Gesamtaufwand: 3627 ≤ 3683. Das ist ein minimaler, fokussierter Pivot mit sehr hohem Wirkungsbeitrag pro Aufwand – sofort entscheidungsreif.', 'Falls A03 für dich unverzichtbar ist (und C62 flexibel), biete ich diese schlanke Parallelvariante an: \{A83, C08, C76, C98, C03, C38, B38, A07, A03\} (Aufwand 3596).', 'Falls C56 dein Fixpunkt ist, drehe ich stattdessen auf: \{A83, C08, C76, C98, C03, C38, C62, A07, C56\} (Aufwand 3468). Nenne mir deinen einen Fixpunkt (A03 vs. C56), dann finalize ich sofort.'\}} \\
            }
        }
    }
    & \\ \\

    \theutterance \stepcounter{utterance}  
    & \multicolumn{4}{p{0.6\linewidth}}{
        \cellcolor[rgb]{0.9,0.9,0.9}{
            \makecell[{{p{\linewidth}}}]{
                \texttt{\tiny{[P1$\rangle$GM]}}
                \texttt{ABLEHNUNG: \{'A83', 'C08', 'C76', 'C98', 'C03', 'C38', 'B38', 'C62', 'A07'\}} \\
\texttt{VORSCHLAG: \{'A83', 'C76', 'C98', 'C03', 'C38', 'B38', 'C62', 'A07', 'C56'\}} \\
\texttt{ARGUMENT: \{'Minimaler Pivot: Ich übernehme 8/9 deiner Picks 1:1 und tausche nur C08 → C56.', 'Aufwand 3548 ≤ 3683 – sauber im Limit.', 'Deine Anker (A07, C62, C03) und deine drei Hebel (A83, C76, C98) bleiben vollständig erhalten; ich stärke nur effizient die Basis.', 'Falls C08 für dich unverzichtbar ist, biete ich die Parallelvariante mit C08 und ohne B38 an (du selbst hattest diese Spur offen): \{A83, C08, C76, C98, C03, C38, C62, A07, C56\} – Aufwand 3468.', 'Gib mir kurz dein Go für die Haupt{-} oder Parallelvariante, dann finalize ich sofort.'\}} \\
            }
        }
    }
    & & \\ \\

    \theutterance \stepcounter{utterance}  
    & & \multicolumn{4}{p{0.6\linewidth}}{
        \cellcolor[rgb]{0.9,0.9,0.9}{
            \makecell[{{p{\linewidth}}}]{
                \texttt{\tiny{[GM$\rangle$P2]}}
                \texttt{ABLEHNUNG: \{'A83', 'C08', 'C76', 'C98', 'C03', 'C38', 'B38', 'C62', 'A07'\}} \\
\texttt{VORSCHLAG: \{'A83', 'C76', 'C98', 'C03', 'C38', 'B38', 'C62', 'A07', 'C56'\}} \\
\texttt{ARGUMENT: \{'Minimaler Pivot: Ich übernehme 8/9 deiner Picks 1:1 und tausche nur C08 → C56.', 'Aufwand 3548 ≤ 3683 – sauber im Limit.', 'Deine Anker (A07, C62, C03) und deine drei Hebel (A83, C76, C98) bleiben vollständig erhalten; ich stärke nur effizient die Basis.', 'Falls C08 für dich unverzichtbar ist, biete ich die Parallelvariante mit C08 und ohne B38 an (du selbst hattest diese Spur offen): \{A83, C08, C76, C98, C03, C38, C62, A07, C56\} – Aufwand 3468.', 'Gib mir kurz dein Go für die Haupt{-} oder Parallelvariante, dann finalize ich sofort.'\}} \\
            }
        }
    }
    & \\ \\

    \theutterance \stepcounter{utterance}  
    & & & \multicolumn{4}{p{0.6\linewidth}}{
        \cellcolor[rgb]{0.9,0.9,0.9}{
            \makecell[{{p{\linewidth}}}]{
                \texttt{\tiny{[GM$\langle$P2]}}
                \texttt{ABLEHNUNG: \{'A83', 'C76', 'C98', 'C03', 'C38', 'B38', 'C62', 'A07', 'C56'\}} \\
\texttt{VORSCHLAG: \{'A83', 'C76', 'C98', 'C03', 'C38', 'B38', 'C62', 'A07', 'C08'\}} \\
\texttt{ARGUMENT: \{'Minimaler 1‑zu‑1‑Swap: C56 → C08. Wir behalten 8/9 deiner Picks unverändert; Gesamtaufwand 3627 ≤ 3683.', 'Deine Anker (A07, C62, C03) bleiben vollständig erhalten; der Vorschlag ist fokussiert und per Aufwand stärker – sofort unterschriftsreif.', 'Falls C56 für dich unverzichtbar ist, biete ich die Parallelvariante mit C56 und ohne B38 an: \{A83, C08, C76, C98, C03, C38, C62, A07, C56\} (Aufwand 3468). Gib mir kurz dein Go für Haupt{-} oder Parallelvariante.'\}} \\
            }
        }
    }
     \\ \\

    \theutterance \stepcounter{utterance}  
    & & \multicolumn{4}{p{0.6\linewidth}}{
        \cellcolor[rgb]{0.9,0.9,0.9}{
            \makecell[{{p{\linewidth}}}]{
                \texttt{\tiny{[P1$\langle$GM]}}
                \texttt{ABLEHNUNG: \{'A83', 'C76', 'C98', 'C03', 'C38', 'B38', 'C62', 'A07', 'C56'\}} \\
\texttt{VORSCHLAG: \{'A83', 'C76', 'C98', 'C03', 'C38', 'B38', 'C62', 'A07', 'C08'\}} \\
\texttt{ARGUMENT: \{'Minimaler 1‑zu‑1‑Swap: C56 → C08. Wir behalten 8/9 deiner Picks unverändert; Gesamtaufwand 3627 ≤ 3683.', 'Deine Anker (A07, C62, C03) bleiben vollständig erhalten; der Vorschlag ist fokussiert und per Aufwand stärker – sofort unterschriftsreif.', 'Falls C56 für dich unverzichtbar ist, biete ich die Parallelvariante mit C56 und ohne B38 an: \{A83, C08, C76, C98, C03, C38, C62, A07, C56\} (Aufwand 3468). Gib mir kurz dein Go für Haupt{-} oder Parallelvariante.'\}} \\
            }
        }
    }
    & \\ \\

    \theutterance \stepcounter{utterance}  
    & \multicolumn{4}{p{0.6\linewidth}}{
        \cellcolor[rgb]{0.9,0.9,0.9}{
            \makecell[{{p{\linewidth}}}]{
                \texttt{\tiny{[P1$\rangle$GM]}}
                \texttt{ABLEHNUNG: \{'A83', 'C76', 'C98', 'C03', 'C38', 'B38', 'C62', 'A07', 'C08'\}} \\
\texttt{VORSCHLAG: \{'A83', 'C76', 'C03', 'C38', 'B38', 'C62', 'A07', 'C08', 'C56', 'A03'\}} \\
\texttt{ARGUMENT: \{'Ich übernehme 8 deiner 9 Picks 1:1 und tausche nur C98 → (C56 + A03). Gesamtaufwand 3497 < 3683.', 'Deine Anker (A07, C62, C03) bleiben vollständig erhalten; der Vorschlag ist schlanker und erhöht die Wirkung je Aufwand.', 'Falls C98 für dich unverzichtbar ist, biete ich die sofortige Parallelvariante an: \{A83, C76, C98, C03, C38, B38, C62, A07, C56\} (Aufwand 3548).'\}} \\
            }
        }
    }
    & & \\ \\

    \theutterance \stepcounter{utterance}  
    & & \multicolumn{4}{p{0.6\linewidth}}{
        \cellcolor[rgb]{0.9,0.9,0.9}{
            \makecell[{{p{\linewidth}}}]{
                \texttt{\tiny{[GM$\rangle$P2]}}
                \texttt{ABLEHNUNG: \{'A83', 'C76', 'C98', 'C03', 'C38', 'B38', 'C62', 'A07', 'C08'\}} \\
\texttt{VORSCHLAG: \{'A83', 'C76', 'C03', 'C38', 'B38', 'C62', 'A07', 'C08', 'C56', 'A03'\}} \\
\texttt{ARGUMENT: \{'Ich übernehme 8 deiner 9 Picks 1:1 und tausche nur C98 → (C56 + A03). Gesamtaufwand 3497 < 3683.', 'Deine Anker (A07, C62, C03) bleiben vollständig erhalten; der Vorschlag ist schlanker und erhöht die Wirkung je Aufwand.', 'Falls C98 für dich unverzichtbar ist, biete ich die sofortige Parallelvariante an: \{A83, C76, C98, C03, C38, B38, C62, A07, C56\} (Aufwand 3548).'\}} \\
            }
        }
    }
    & \\ \\

    \theutterance \stepcounter{utterance}  
    & & & \multicolumn{4}{p{0.6\linewidth}}{
        \cellcolor[rgb]{0.9,0.9,0.9}{
            \makecell[{{p{\linewidth}}}]{
                \texttt{\tiny{[GM$\langle$P2]}}
                \texttt{ABLEHNUNG: \{'A83', 'C76', 'C03', 'C38', 'B38', 'C62', 'A07', 'C08', 'C56', 'A03'\}} \\
\texttt{VORSCHLAG: \{'A83', 'C76', 'C98', 'C03', 'C38', 'B38', 'C62', 'A07', 'C08'\}} \\
\texttt{ARGUMENT: \{'Minimaler Pivot direkt auf deiner letzten Struktur: A03 + C56 → C98. Aufwand: 3497 {-} 517 {-} 60 + 707 = 3627 ≤ 3683. Wir behalten 8/10 deiner Picks 1:1 (A83, C76, C03, C38, B38, C62, A07, C08), deine Anker bleiben vollständig erhalten – sofort unterschriftsreif.', 'Falls C56 für dich unverzichtbar ist, behalte ich C56 und ersetze dafür B38: \{A83, C76, C98, C03, C38, C62, A07, C08, C56\} (Aufwand 3468).', 'Falls A03 ein Muss ist (und C62 etwas flexibler), drehe ich stattdessen C62 → A03: \{A83, C76, C98, C03, C38, B38, A07, C08, A03\} (Aufwand 3483). Gib mir deinen Fixpunkt (C56 vs. A03), sonst nehmen wir meinen Hauptvorschlag und schließen jetzt ab.'\}} \\
            }
        }
    }
     \\ \\

    \theutterance \stepcounter{utterance}  
    & & \multicolumn{4}{p{0.6\linewidth}}{
        \cellcolor[rgb]{0.9,0.9,0.9}{
            \makecell[{{p{\linewidth}}}]{
                \texttt{\tiny{[P1$\langle$GM]}}
                \texttt{ABLEHNUNG: \{'A83', 'C76', 'C03', 'C38', 'B38', 'C62', 'A07', 'C08', 'C56', 'A03'\}} \\
\texttt{VORSCHLAG: \{'A83', 'C76', 'C98', 'C03', 'C38', 'B38', 'C62', 'A07', 'C08'\}} \\
\texttt{ARGUMENT: \{'Minimaler Pivot direkt auf deiner letzten Struktur: A03 + C56 → C98. Aufwand: 3497 {-} 517 {-} 60 + 707 = 3627 ≤ 3683. Wir behalten 8/10 deiner Picks 1:1 (A83, C76, C03, C38, B38, C62, A07, C08), deine Anker bleiben vollständig erhalten – sofort unterschriftsreif.', 'Falls C56 für dich unverzichtbar ist, behalte ich C56 und ersetze dafür B38: \{A83, C76, C98, C03, C38, C62, A07, C08, C56\} (Aufwand 3468).', 'Falls A03 ein Muss ist (und C62 etwas flexibler), drehe ich stattdessen C62 → A03: \{A83, C76, C98, C03, C38, B38, A07, C08, A03\} (Aufwand 3483). Gib mir deinen Fixpunkt (C56 vs. A03), sonst nehmen wir meinen Hauptvorschlag und schließen jetzt ab.'\}} \\
            }
        }
    }
    & \\ \\

    \theutterance \stepcounter{utterance}  
    & \multicolumn{4}{p{0.6\linewidth}}{
        \cellcolor[rgb]{0.9,0.9,0.9}{
            \makecell[{{p{\linewidth}}}]{
                \texttt{\tiny{[P1$\rangle$GM]}}
                \texttt{VORSCHLAG: \{'A83', 'C76', 'C98', 'C03', 'C38', 'B38', 'C62', 'A07', 'C56'\}} \\
\texttt{ARGUMENT: \{'Minimaler 1{-}zu{-}1{-}Pivot auf dein letztes Paket: C08 → C56. 8/9 deiner Picks bleiben unverändert; Aufwand 3548 ≤ 3683 – sauber im Limit und per Aufwand klar stärker.', 'Deine Anker (A07, C62, C03) sowie A83/C76/C98 bleiben vollständig erhalten. Das ist dein Vorschlag mit nur einem sehr effizienten Feinschliff – sofort unterschriftsreif.', 'Falls C08 für dich unverzichtbar ist, nehme ich parallel C08 wieder rein und lasse dafür B38 raus: \{A83, C08, C76, C98, C03, C38, C62, A07, C56\} (3468). Gib mir kurz Go für Haupt{-} oder Parallelvariante.'\}} \\
            }
        }
    }
    & & \\ \\

    \theutterance \stepcounter{utterance}  
    & & \multicolumn{4}{p{0.6\linewidth}}{
        \cellcolor[rgb]{0.9,0.9,0.9}{
            \makecell[{{p{\linewidth}}}]{
                \texttt{\tiny{[GM$\rangle$P2]}}
                \texttt{VORSCHLAG: \{'A83', 'C76', 'C98', 'C03', 'C38', 'B38', 'C62', 'A07', 'C56'\}} \\
\texttt{ARGUMENT: \{'Minimaler 1{-}zu{-}1{-}Pivot auf dein letztes Paket: C08 → C56. 8/9 deiner Picks bleiben unverändert; Aufwand 3548 ≤ 3683 – sauber im Limit und per Aufwand klar stärker.', 'Deine Anker (A07, C62, C03) sowie A83/C76/C98 bleiben vollständig erhalten. Das ist dein Vorschlag mit nur einem sehr effizienten Feinschliff – sofort unterschriftsreif.', 'Falls C08 für dich unverzichtbar ist, nehme ich parallel C08 wieder rein und lasse dafür B38 raus: \{A83, C08, C76, C98, C03, C38, C62, A07, C56\} (3468). Gib mir kurz Go für Haupt{-} oder Parallelvariante.'\}} \\
            }
        }
    }
    & \\ \\

    \theutterance \stepcounter{utterance}  
    & & & \multicolumn{4}{p{0.6\linewidth}}{
        \cellcolor[rgb]{0.9,0.9,0.9}{
            \makecell[{{p{\linewidth}}}]{
                \texttt{\tiny{[GM$\langle$P2]}}
                \texttt{ABLEHNUNG: \{'A83', 'C76', 'C98', 'C03', 'C38', 'B38', 'C62', 'A07', 'C56'\}} \\
\texttt{VORSCHLAG: \{'A83', 'C76', 'C98', 'C03', 'C38', 'B38', 'C62', 'A07', 'C08'\}} \\
\texttt{ARGUMENT: \{'Minimaler 1‑zu‑1‑Tausch ggü. deinem Vorschlag: C56 → C08; Gesamtaufwand 3627 ≤ 3683.', 'Deine wiederholt gesetzten Anker (A07, C62, C03) bleiben 1:1 erhalten; das Paket ist fokussiert und sofort unterschriftsreif.', 'Falls C56 für dich nicht verhandelbar ist, halte ich C56 drin und streiche wahlweise B38 (3468) oder C38 (3533): \{A83, C76, C98, C03, C38, C62, A07, C08, C56\} ODER \{A83, C76, C98, C03, B38, C62, A07, C08, C56\}. Sag kurz, welche Variante du bevorzugst – oder nimm meinen Hauptvorschlag direkt an.'\}} \\
            }
        }
    }
     \\ \\

    \theutterance \stepcounter{utterance}  
    & & \multicolumn{4}{p{0.6\linewidth}}{
        \cellcolor[rgb]{0.9,0.9,0.9}{
            \makecell[{{p{\linewidth}}}]{
                \texttt{\tiny{[P1$\langle$GM]}}
                \texttt{ABLEHNUNG: \{'A83', 'C76', 'C98', 'C03', 'C38', 'B38', 'C62', 'A07', 'C56'\}} \\
\texttt{VORSCHLAG: \{'A83', 'C76', 'C98', 'C03', 'C38', 'B38', 'C62', 'A07', 'C08'\}} \\
\texttt{ARGUMENT: \{'Minimaler 1‑zu‑1‑Tausch ggü. deinem Vorschlag: C56 → C08; Gesamtaufwand 3627 ≤ 3683.', 'Deine wiederholt gesetzten Anker (A07, C62, C03) bleiben 1:1 erhalten; das Paket ist fokussiert und sofort unterschriftsreif.', 'Falls C56 für dich nicht verhandelbar ist, halte ich C56 drin und streiche wahlweise B38 (3468) oder C38 (3533): \{A83, C76, C98, C03, C38, C62, A07, C08, C56\} ODER \{A83, C76, C98, C03, B38, C62, A07, C08, C56\}. Sag kurz, welche Variante du bevorzugst – oder nimm meinen Hauptvorschlag direkt an.'\}} \\
            }
        }
    }
    & \\ \\

    \theutterance \stepcounter{utterance}  
    & \multicolumn{4}{p{0.6\linewidth}}{
        \cellcolor[rgb]{0.9,0.9,0.9}{
            \makecell[{{p{\linewidth}}}]{
                \texttt{\tiny{[P1$\rangle$GM]}}
                \texttt{VORSCHLAG: \{'A83', 'C76', 'C98', 'C03', 'B38', 'C62', 'A07', 'C08', 'C56'\}} \\
\texttt{ARGUMENT: \{'Das ist exakt eine deiner angebotenen Alternativen (C56 rein, C38 raus). Gesamtaufwand: 3533 ≤ 3683 – sauber im Limit.', 'Deine Anker (A07, C62, C03) sowie A83/C76/C98 bleiben 1:1 erhalten; ich nehme nur deinen eigenen Feinschliff auf.', 'Minimaler Pivot, sofort unterschriftsreif. Wenn du es bevorzugst, kann ich auch auf die Schwester‑Variante mit C38 statt B38 umstellen (3468).'\}} \\
            }
        }
    }
    & & \\ \\

    \theutterance \stepcounter{utterance}  
    & & \multicolumn{4}{p{0.6\linewidth}}{
        \cellcolor[rgb]{0.9,0.9,0.9}{
            \makecell[{{p{\linewidth}}}]{
                \texttt{\tiny{[GM$\rangle$P2]}}
                \texttt{VORSCHLAG: \{'A83', 'C76', 'C98', 'C03', 'B38', 'C62', 'A07', 'C08', 'C56'\}} \\
\texttt{ARGUMENT: \{'Das ist exakt eine deiner angebotenen Alternativen (C56 rein, C38 raus). Gesamtaufwand: 3533 ≤ 3683 – sauber im Limit.', 'Deine Anker (A07, C62, C03) sowie A83/C76/C98 bleiben 1:1 erhalten; ich nehme nur deinen eigenen Feinschliff auf.', 'Minimaler Pivot, sofort unterschriftsreif. Wenn du es bevorzugst, kann ich auch auf die Schwester‑Variante mit C38 statt B38 umstellen (3468).'\}} \\
            }
        }
    }
    & \\ \\

    \theutterance \stepcounter{utterance}  
    & & & \multicolumn{4}{p{0.6\linewidth}}{
        \cellcolor[rgb]{0.9,0.9,0.9}{
            \makecell[{{p{\linewidth}}}]{
                \texttt{\tiny{[GM$\langle$P2]}}
                \texttt{ABLEHNUNG: \{'A83', 'C76', 'C98', 'C03', 'B38', 'C62', 'A07', 'C08', 'C56'\}} \\
\texttt{VORSCHLAG: \{'A83', 'C76', 'C98', 'C03', 'B38', 'C62', 'A07', 'C08', 'C38'\}} \\
\texttt{ARGUMENT: \{'Minimaler 1‑zu‑1‑Swap ggü. deinem Paket: C56 → C38; Gesamtaufwand 3627 ≤ 3683.', 'Alle deine Anker (A07, C62, C03) bleiben unverändert; wir nutzen nur den freien Puffer für einen qualitativen Feinschliff.', 'Falls C56 für dich unverzichtbar ist, nehme ich stattdessen B38 → C38: \{A83, C76, C98, C03, C38, C62, A07, C08, C56\} (3468). Gib mir kurz Go für Haupt‑ oder Alternativvariante, dann finalize ich sofort.'\}} \\
            }
        }
    }
     \\ \\

    \theutterance \stepcounter{utterance}  
    & & \multicolumn{4}{p{0.6\linewidth}}{
        \cellcolor[rgb]{0.9,0.9,0.9}{
            \makecell[{{p{\linewidth}}}]{
                \texttt{\tiny{[P1$\langle$GM]}}
                \texttt{ABLEHNUNG: \{'A83', 'C76', 'C98', 'C03', 'B38', 'C62', 'A07', 'C08', 'C56'\}} \\
\texttt{VORSCHLAG: \{'A83', 'C76', 'C98', 'C03', 'B38', 'C62', 'A07', 'C08', 'C38'\}} \\
\texttt{ARGUMENT: \{'Minimaler 1‑zu‑1‑Swap ggü. deinem Paket: C56 → C38; Gesamtaufwand 3627 ≤ 3683.', 'Alle deine Anker (A07, C62, C03) bleiben unverändert; wir nutzen nur den freien Puffer für einen qualitativen Feinschliff.', 'Falls C56 für dich unverzichtbar ist, nehme ich stattdessen B38 → C38: \{A83, C76, C98, C03, C38, C62, A07, C08, C56\} (3468). Gib mir kurz Go für Haupt‑ oder Alternativvariante, dann finalize ich sofort.'\}} \\
            }
        }
    }
    & \\ \\

    \theutterance \stepcounter{utterance}  
    & \multicolumn{4}{p{0.6\linewidth}}{
        \cellcolor[rgb]{0.9,0.9,0.9}{
            \makecell[{{p{\linewidth}}}]{
                \texttt{\tiny{[P1$\rangle$GM]}}
                \texttt{ABLEHNUNG: \{'A83', 'C76', 'C98', 'C03', 'B38', 'C62', 'A07', 'C08', 'C38'\}} \\
\texttt{VORSCHLAG: \{'A83', 'C76', 'C98', 'C03', 'C38', 'C62', 'A07', 'C08', 'C56'\}} \\
\texttt{ARGUMENT: \{'Minimaler 1{-}zu{-}1{-}Tausch: B38 → C56. 8/9 deiner Picks bleiben unverändert; Aufwand 3468 ≤ 3683. Deine Anker (A07, C62, C03) und dein Dreikern (A83, C76, C98) plus C08/C38 bleiben vollständig erhalten. Das ist exakt eine deiner zuvor angebotenen Alternativen – schlank, effizient, sofort unterschriftsreif. Falls B38 für dich doch unverzichtbar ist, kann ich die Schwester{-}Variante mit C56 statt C08 nehmen (Aufwand 3548): \{A83, C76, C98, C03, C38, B38, C62, A07, C56\}. Gib mir kurz Go für eine der beiden, dann schließen wir ab.'\}} \\
            }
        }
    }
    & & \\ \\

    \theutterance \stepcounter{utterance}  
    & & \multicolumn{4}{p{0.6\linewidth}}{
        \cellcolor[rgb]{0.9,0.9,0.9}{
            \makecell[{{p{\linewidth}}}]{
                \texttt{\tiny{[GM$\rangle$P2]}}
                \texttt{ABLEHNUNG: \{'A83', 'C76', 'C98', 'C03', 'B38', 'C62', 'A07', 'C08', 'C38'\}} \\
\texttt{VORSCHLAG: \{'A83', 'C76', 'C98', 'C03', 'C38', 'C62', 'A07', 'C08', 'C56'\}} \\
\texttt{ARGUMENT: \{'Minimaler 1{-}zu{-}1{-}Tausch: B38 → C56. 8/9 deiner Picks bleiben unverändert; Aufwand 3468 ≤ 3683. Deine Anker (A07, C62, C03) und dein Dreikern (A83, C76, C98) plus C08/C38 bleiben vollständig erhalten. Das ist exakt eine deiner zuvor angebotenen Alternativen – schlank, effizient, sofort unterschriftsreif. Falls B38 für dich doch unverzichtbar ist, kann ich die Schwester{-}Variante mit C56 statt C08 nehmen (Aufwand 3548): \{A83, C76, C98, C03, C38, B38, C62, A07, C56\}. Gib mir kurz Go für eine der beiden, dann schließen wir ab.'\}} \\
            }
        }
    }
    & \\ \\

    \theutterance \stepcounter{utterance}  
    & & & \multicolumn{4}{p{0.6\linewidth}}{
        \cellcolor[rgb]{0.9,0.9,0.9}{
            \makecell[{{p{\linewidth}}}]{
                \texttt{\tiny{[GM$\langle$P2]}}
                \texttt{ABLEHNUNG: \{'A83', 'C76', 'C98', 'C03', 'C38', 'C62', 'A07', 'C08', 'C56'\}} \\
\texttt{VORSCHLAG: \{'A83', 'C76', 'C98', 'C03', 'C38', 'B38', 'C62', 'A07', 'C08'\}} \\
\texttt{ARGUMENT: \{'Minimaler 1‑zu‑1‑Tausch: C56 → B38. Gesamtaufwand 3627 ≤ 3683 – sauber im Limit.', 'Wir behalten 8/9 deiner Picks 1:1 (A83, C76, C98, C03, C38, C62, A07, C08); deine Anker (A07, C62, C03) bleiben vollständig erhalten.', 'Das ist die effizientere Schwester deines Vorschlags und deckt sich mit unseren zuvor diskutierten Varianten – sofort unterschriftsreif.'\}} \\
            }
        }
    }
     \\ \\

    \theutterance \stepcounter{utterance}  
    & & \multicolumn{4}{p{0.6\linewidth}}{
        \cellcolor[rgb]{0.9,0.9,0.9}{
            \makecell[{{p{\linewidth}}}]{
                \texttt{\tiny{[P1$\langle$GM]}}
                \texttt{ABLEHNUNG: \{'A83', 'C76', 'C98', 'C03', 'C38', 'C62', 'A07', 'C08', 'C56'\}} \\
\texttt{VORSCHLAG: \{'A83', 'C76', 'C98', 'C03', 'C38', 'B38', 'C62', 'A07', 'C08'\}} \\
\texttt{ARGUMENT: \{'Minimaler 1‑zu‑1‑Tausch: C56 → B38. Gesamtaufwand 3627 ≤ 3683 – sauber im Limit.', 'Wir behalten 8/9 deiner Picks 1:1 (A83, C76, C98, C03, C38, C62, A07, C08); deine Anker (A07, C62, C03) bleiben vollständig erhalten.', 'Das ist die effizientere Schwester deines Vorschlags und deckt sich mit unseren zuvor diskutierten Varianten – sofort unterschriftsreif.'\}} \\
            }
        }
    }
    & \\ \\

    \theutterance \stepcounter{utterance}  
    & \multicolumn{4}{p{0.6\linewidth}}{
        \cellcolor[rgb]{0.9,0.9,0.9}{
            \makecell[{{p{\linewidth}}}]{
                \texttt{\tiny{[P1$\rangle$GM]}}
                \texttt{VORSCHLAG: \{'A83', 'C76', 'C98', 'C03', 'B38', 'C62', 'A07', 'C08', 'C56'\}} \\
\texttt{ARGUMENT: \{'Ich greife deine eigene Alternative auf: C56 rein, C38 raus. Gesamtaufwand 3533 ≤ 3683.', 'Dein Kern bleibt 1:1 erhalten (A83, C76, C98, C03, B38, C62, A07, C08); nur ein minimaler, effizienter Feinschliff mit C56.', 'Sofort unterschriftsreif – lass uns so abschließen.'\}} \\
            }
        }
    }
    & & \\ \\

    \theutterance \stepcounter{utterance}  
    & & \multicolumn{4}{p{0.6\linewidth}}{
        \cellcolor[rgb]{0.9,0.9,0.9}{
            \makecell[{{p{\linewidth}}}]{
                \texttt{\tiny{[GM$\rangle$P2]}}
                \texttt{VORSCHLAG: \{'A83', 'C76', 'C98', 'C03', 'B38', 'C62', 'A07', 'C08', 'C56'\}} \\
\texttt{ARGUMENT: \{'Ich greife deine eigene Alternative auf: C56 rein, C38 raus. Gesamtaufwand 3533 ≤ 3683.', 'Dein Kern bleibt 1:1 erhalten (A83, C76, C98, C03, B38, C62, A07, C08); nur ein minimaler, effizienter Feinschliff mit C56.', 'Sofort unterschriftsreif – lass uns so abschließen.'\}} \\
            }
        }
    }
    & \\ \\

    \theutterance \stepcounter{utterance}  
    & & & \multicolumn{4}{p{0.6\linewidth}}{
        \cellcolor[rgb]{0.9,0.9,0.9}{
            \makecell[{{p{\linewidth}}}]{
                \texttt{\tiny{[GM$\langle$P2]}}
                \texttt{ZUSTIMMUNG: \{'A83', 'C76', 'C98', 'C03', 'B38', 'C62', 'A07', 'C08', 'C56'\}} \\
\texttt{ARGUMENT: \{'Einverstanden: Aufwand 3533 ≤ 3683, deine Anker sind gewahrt und das Paket ist effizient. Abschließen.'\}} \\
            }
        }
    }
     \\ \\

    \theutterance \stepcounter{utterance}  
    & & & \multicolumn{2}{p{0.3\linewidth}}{
        \cellcolor[rgb]{0.9,0.9,0.9}{
            \makecell[{{p{\linewidth}}}]{
                \texttt{\tiny{[GM$|$GM]}}
                \texttt{game successful} \\
            }
        }
    }
    & & \\ \\

    \theutterance \stepcounter{utterance}  
    & & & \multicolumn{2}{p{0.3\linewidth}}{
        \cellcolor[rgb]{0.9,0.9,0.9}{
            \makecell[{{p{\linewidth}}}]{
                \texttt{\tiny{[GM$|$GM]}}
                \texttt{end game} \\
            }
        }
    }
    & & \\ \\

\end{supertabular}
}

\end{document}
