
\documentclass{article}
\usepackage{colortbl}
\usepackage{makecell}
\usepackage{multirow}
\usepackage{supertabular}

\begin{document}

\newcounter{utterance}

\twocolumn

{ \footnotesize  \setcounter{utterance}{1}
\setlength{\tabcolsep}{0pt}
\begin{supertabular}{c@{$\;$}|p{.15\linewidth}@{}p{.15\linewidth}p{.15\linewidth}p{.15\linewidth}p{.15\linewidth}p{.15\linewidth}}

    \# & $\;$A & \multicolumn{4}{c}{Game Master} & $\;\:$B\\
    \hline 

    \theutterance \stepcounter{utterance}  

    & & \multicolumn{4}{p{0.6\linewidth}}{\cellcolor[rgb]{0.9,0.9,0.9}{%
	\makecell[{{p{\linewidth}}}]{% 
	  \tt {\tiny [A$\langle$GM]}  
	 Du nimmst an einem kollaborativen Verhandlungspiel Teil.\\ \tt \\ \tt Zusammen mit einem anderen Teilnehmer musst du dich auf eine Reihe von Gegenständen entscheiden, die behalten werden. Jeder von euch hat eine persönliche Verteilung über die Wichtigkeit der einzelnen Gegenstände. Jeder von euch hat eine eigene Meinung darüber, wie wichtig jeder einzelne Gegenstand ist (Gegenstandswichtigkeit). Du kennst die Wichtigkeitsverteilung des anderen Spielers nicht. Zusätzlich siehst du, wie viel Aufwand jeder Gegenstand verursacht.  \\ \tt Ihr dürft euch nur auf eine Reihe von Gegenständen einigen, wenn der Gesamtaufwand der ausgewählten Gegenstände den Maximalaufwand nicht überschreitet:\\ \tt \\ \tt Maximalaufwand = 3357\\ \tt \\ \tt Hier sind die einzelnen Aufwände der Gegenstände:\\ \tt \\ \tt Aufwand der Gegenstände = {"C76": 506, "C38": 492, "C56": 338, "A03": 122, "A07": 131, "A83": 909, "C98": 144, "C08": 716, "C62": 263, "C00": 231, "C10": 91, "C81": 651, "B38": 552, "C03": 849, "C32": 720}\\ \tt \\ \tt Hier ist deine persönliche Verteilung der Wichtigkeit der einzelnen Gegenstände:\\ \tt \\ \tt Werte der Gegenstandswichtigkeit = {"C76": 138, "C38": 583, "C56": 868, "A03": 822, "A07": 783, "A83": 65, "C98": 262, "C08": 121, "C62": 508, "C00": 780, "C10": 461, "C81": 484, "B38": 668, "C03": 389, "C32": 808}\\ \tt \\ \tt Ziel:\\ \tt \\ \tt Dein Ziel ist es, eine Reihe von Gegenständen auszuhandeln, die dir möglichst viel bringt (d. h. Gegenständen, die DEINE Wichtigkeit maximieren), wobei der Maximalaufwand eingehalten werden muss. Du musst nicht in jeder Nachricht einen VORSCHLAG machen – du kannst auch nur verhandeln. Alle Taktiken sind erlaubt!\\ \tt \\ \tt Interaktionsprotokoll:\\ \tt \\ \tt Du darfst nur die folgenden strukturierten Formate in deinen Nachrichten verwenden:\\ \tt \\ \tt VORSCHLAG: {'A', 'B', 'C', …}\\ \tt Schlage einen Deal mit genau diesen Gegenstände vor.\\ \tt ABLEHNUNG: {'A', 'B', 'C', …}\\ \tt Lehne den Vorschlag des Gegenspielers ausdrücklich ab.\\ \tt ARGUMENT: {'...'}\\ \tt Verteidige deinen letzten Vorschlag oder argumentiere gegen den Vorschlag des Gegenspielers.\\ \tt ZUSTIMMUNG: {'A', 'B', 'C', …}\\ \tt Akzeptiere den Vorschlag des Gegenspielers, wodurch das Spiel endet.\\ \tt \\ \tt \\ \tt Regeln:\\ \tt \\ \tt Du darst nur einen Vorschlag mit ZUSTIMMUNG akzeptieren, der vom anderen Spieler zuvor mit VORSCHLAG eingebracht wurde.\\ \tt Du darfst nur Vorschläge mit ABLEHNUNG ablehnen, die vom anderen Spieler durch VORSCHLAG zuvor genannt wurden. \\ \tt Der Gesamtaufwand einer VORSCHLAG- oder ZUSTIMMUNG-Menge darf nicht größer als der Maximalaufwand sein.  \\ \tt Offenbare deine versteckte Wichtigkeitsverteilung nicht.\\ \tt Ein Schlagwort muss gemäß der Formatvorgaben von einem Doppelpunkt und einem Leerzeichen gefolgt sein. Das Argument muss eine Python-Menge mit 0 oder mehr Strings sein.  \\ \tt Die Form muss also SCHLAGWORT: {...} sein.\\ \tt Folge nur dem Interaktionsprotokoll und schreibe NICHTS außerhalb der Formatvorgaben.\\ \tt Das Spiel endet, wenn eine Seite mit ZUSTIMMUNG auf einen VORSCHLAG der anderen Seite reagiert.  \\ \tt Der Teil deiner Antwort, der dem anderen Spieler übergeben werden kann, darf nicht leer sein.  \\ \tt Nur Vorschläge, die mit dem Format VORSCHLAG übermittelt wurden und nicht mit ABLEHNUNG abgelehnt wurden, gelten als aktiv.  \\ \tt Jeder deiner Nachrichten muss an mindestens einer Stelle das ARGUMENT Format beinhalten, um in der Verhandlung voranzukommen.\\ \tt  
	  } 
	   } 
	   } 
	 & \\ 
 

    \theutterance \stepcounter{utterance}  

    & \multicolumn{4}{p{0.6\linewidth}}{\cellcolor[rgb]{0.8,1,0.9}{%
	\makecell[{{p{\linewidth}}}]{% 
	  \tt {\tiny [A$\rangle$GM]}  
	 VORSCHLAG: {'A03', 'A07', 'C10', 'C00', 'C56', 'C62', 'C98', 'C32', 'B38', 'C38'}\\ \tt ARGUMENT: {'Das Paket bleibt mit Gesamtaufwand 3084 unter dem Limit 3357 und meidet die größten Aufwandstreiber (A83=909, C03=849, C08=716).', 'Ich priorisiere hier ein gutes Verhältnis von Nutzen zu Aufwand; falls dir ein bestimmter Posten fehlt, können wir über einen Tausch z. B. von B38 oder C38 sprechen, solange der Aufwand passt.'} 
	  } 
	   } 
	   } 
	 & & \\ 
 

    \theutterance \stepcounter{utterance}  

    & & \multicolumn{4}{p{0.6\linewidth}}{\cellcolor[rgb]{0.9,0.9,0.9}{%
	\makecell[{{p{\linewidth}}}]{% 
	  \tt {\tiny [GM$\rangle$B]}  
	 Du nimmst an einem kollaborativen Verhandlungspiel Teil.\\ \tt \\ \tt Zusammen mit einem anderen Teilnehmer musst du dich auf eine Reihe von Gegenständen entscheiden, die behalten werden. Jeder von euch hat eine persönliche Verteilung über die Wichtigkeit der einzelnen Gegenstände. Jeder von euch hat eine eigene Meinung darüber, wie wichtig jeder einzelne Gegenstand ist (Gegenstandswichtigkeit). Du kennst die Wichtigkeitsverteilung des anderen Spielers nicht. Zusätzlich siehst du, wie viel Aufwand jeder Gegenstand verursacht.  \\ \tt Ihr dürft euch nur auf eine Reihe von Gegenständen einigen, wenn der Gesamtaufwand der ausgewählten Gegenstände den Maximalaufwand nicht überschreitet:\\ \tt \\ \tt Maximalaufwand = 3357\\ \tt \\ \tt Hier sind die einzelnen Aufwände der Gegenstände:\\ \tt \\ \tt Aufwand der Gegenstände = {"C76": 506, "C38": 492, "C56": 338, "A03": 122, "A07": 131, "A83": 909, "C98": 144, "C08": 716, "C62": 263, "C00": 231, "C10": 91, "C81": 651, "B38": 552, "C03": 849, "C32": 720}\\ \tt \\ \tt Hier ist deine persönliche Verteilung der Wichtigkeit der einzelnen Gegenstände:\\ \tt \\ \tt Werte der Gegenstandswichtigkeit = {"C56": 65, "A03": 121, "C32": 138, "A07": 262, "C00": 389, "B38": 461, "C38": 484, "C62": 508, "C81": 583, "C10": 668, "C03": 780, "C98": 783, "C76": 808, "C08": 822, "A83": 868}\\ \tt \\ \tt Ziel:\\ \tt \\ \tt Dein Ziel ist es, eine Reihe von Gegenständen auszuhandeln, die dir möglichst viel bringt (d. h. Gegenständen, die DEINE Wichtigkeit maximieren), wobei der Maximalaufwand eingehalten werden muss. Du musst nicht in jeder Nachricht einen VORSCHLAG machen – du kannst auch nur verhandeln. Alle Taktiken sind erlaubt!\\ \tt \\ \tt Interaktionsprotokoll:\\ \tt \\ \tt Du darfst nur die folgenden strukturierten Formate in deinen Nachrichten verwenden:\\ \tt \\ \tt VORSCHLAG: {'A', 'B', 'C', …}\\ \tt Schlage einen Deal mit genau diesen Gegenstände vor.\\ \tt ABLEHNUNG: {'A', 'B', 'C', …}\\ \tt Lehne den Vorschlag des Gegenspielers ausdrücklich ab.\\ \tt ARGUMENT: {'...'}\\ \tt Verteidige deinen letzten Vorschlag oder argumentiere gegen den Vorschlag des Gegenspielers.\\ \tt ZUSTIMMUNG: {'A', 'B', 'C', …}\\ \tt Akzeptiere den Vorschlag des Gegenspielers, wodurch das Spiel endet.\\ \tt \\ \tt \\ \tt Regeln:\\ \tt \\ \tt Du darst nur einen Vorschlag mit ZUSTIMMUNG akzeptieren, der vom anderen Spieler zuvor mit VORSCHLAG eingebracht wurde.\\ \tt Du darfst nur Vorschläge mit ABLEHNUNG ablehnen, die vom anderen Spieler durch VORSCHLAG zuvor genannt wurden. \\ \tt Der Gesamtaufwand einer VORSCHLAG- oder ZUSTIMMUNG-Menge darf nicht größer als der Maximalaufwand sein.  \\ \tt Offenbare deine versteckte Wichtigkeitsverteilung nicht.\\ \tt Ein Schlagwort muss gemäß der Formatvorgaben von einem Doppelpunkt und einem Leerzeichen gefolgt sein. Das Argument muss eine Python-Menge mit 0 oder mehr Strings sein.  \\ \tt Die Form muss also SCHLAGWORT: {...} sein.\\ \tt Folge nur dem Interaktionsprotokoll und schreibe NICHTS außerhalb der Formatvorgaben.\\ \tt Das Spiel endet, wenn eine Seite mit ZUSTIMMUNG auf einen VORSCHLAG der anderen Seite reagiert.  \\ \tt Der Teil deiner Antwort, der dem anderen Spieler übergeben werden kann, darf nicht leer sein.  \\ \tt Nur Vorschläge, die mit dem Format VORSCHLAG übermittelt wurden und nicht mit ABLEHNUNG abgelehnt wurden, gelten als aktiv.  \\ \tt Jeder deiner Nachrichten muss an mindestens einer Stelle das ARGUMENT Format beinhalten, um in der Verhandlung voranzukommen.\\ \tt \\ \tt \\ \tt Du erhältst nun die erste Nachricht des anderen Spielers.\\ \tt \\ \tt VORSCHLAG: {'A03', 'A07', 'C10', 'C00', 'C56', 'C62', 'C98', 'C32', 'B38', 'C38'}\\ \tt ARGUMENT: {'Das Paket bleibt mit Gesamtaufwand 3084 unter dem Limit 3357 und meidet die größten Aufwandstreiber (A83=909, C03=849, C08=716).', 'Ich priorisiere hier ein gutes Verhältnis von Nutzen zu Aufwand; falls dir ein bestimmter Posten fehlt, können wir über einen Tausch z. B. von B38 oder C38 sprechen, solange der Aufwand passt.'} 
	  } 
	   } 
	   } 
	 & \\ 
 

    \theutterance \stepcounter{utterance}  

    & & & \multicolumn{4}{p{0.6\linewidth}}{\cellcolor[rgb]{1,0.85,0.72}{%
	\makecell[{{p{\linewidth}}}]{% 
	  \tt {\tiny [GM$\langle$B]}  
	 ABLEHNUNG: {'A03', 'A07', 'C10', 'C00', 'C56', 'C62', 'C98', 'C32', 'B38', 'C38'}\\ \tt VORSCHLAG: {'A03', 'A07', 'C10', 'C00', 'C62', 'C98', 'B38', 'C38', 'C76', 'C08'}\\ \tt ARGUMENT: {'Tausche C32 (720) und C56 (338) gegen C76 (506) und C08 (716); Gesamtaufwand = 3248 und damit ≤ 3357 (Puffer 109).', 'Wir vermeiden weiterhin die größten Aufwandstreiber A83 (909) und C03 (849); C08 ist der günstigste der drei, dafür fällt C32 (720) komplett weg.', 'So bleibt das Aufwand-Profil ausgewogen und wir müssen weder B38 noch C38 antasten.', 'Falls C08 für dich nicht geht, schlanke Alternative als Kompromiss: nur C32 → C76 (Aufwand 2870), Rest wie in deinem Paket.'} 
	  } 
	   } 
	   } 
	  \\ 
 

    \theutterance \stepcounter{utterance}  

    & & \multicolumn{4}{p{0.6\linewidth}}{\cellcolor[rgb]{0.9,0.9,0.9}{%
	\makecell[{{p{\linewidth}}}]{% 
	  \tt {\tiny [A$\langle$GM]}  
	 ABLEHNUNG: {'A03', 'A07', 'C10', 'C00', 'C56', 'C62', 'C98', 'C32', 'B38', 'C38'}\\ \tt VORSCHLAG: {'A03', 'A07', 'C10', 'C00', 'C62', 'C98', 'B38', 'C38', 'C76', 'C08'}\\ \tt ARGUMENT: {'Tausche C32 (720) und C56 (338) gegen C76 (506) und C08 (716); Gesamtaufwand = 3248 und damit ≤ 3357 (Puffer 109).', 'Wir vermeiden weiterhin die größten Aufwandstreiber A83 (909) und C03 (849); C08 ist der günstigste der drei, dafür fällt C32 (720) komplett weg.', 'So bleibt das Aufwand-Profil ausgewogen und wir müssen weder B38 noch C38 antasten.', 'Falls C08 für dich nicht geht, schlanke Alternative als Kompromiss: nur C32 → C76 (Aufwand 2870), Rest wie in deinem Paket.'} 
	  } 
	   } 
	   } 
	 & \\ 
 

    \theutterance \stepcounter{utterance}  

    & \multicolumn{4}{p{0.6\linewidth}}{\cellcolor[rgb]{0.8,1,0.9}{%
	\makecell[{{p{\linewidth}}}]{% 
	  \tt {\tiny [A$\rangle$GM]}  
	 ABLEHNUNG: {'A03', 'A07', 'C10', 'C00', 'C62', 'C98', 'B38', 'C38', 'C76', 'C08'}\\ \tt VORSCHLAG: {'A03', 'A07', 'C10', 'C00', 'C56', 'C98', 'C32', 'B38', 'C38', 'C76'}\\ \tt ARGUMENT: {'Ich komme dir bei C76 entgegen und lasse C08 draußen – so bleibt das Paket fokussiert.', 'Zur Einhaltung des Limits streiche ich statt C32/C56 den Posten C62; Gesamtaufwand = 3327 ≤ 3357 (Puffer 30).', 'B38 und C38 bleiben unangetastet; die großen Aufwandstreiber A83, C03 und C08 vermeiden wir weiterhin.', 'Falls C62 für dich essenziell ist, sag Bescheid – dann suchen wir eine Alternative über kleinere Tauschposten.'} 
	  } 
	   } 
	   } 
	 & & \\ 
 

    \theutterance \stepcounter{utterance}  

    & & \multicolumn{4}{p{0.6\linewidth}}{\cellcolor[rgb]{0.9,0.9,0.9}{%
	\makecell[{{p{\linewidth}}}]{% 
	  \tt {\tiny [GM$\rangle$B]}  
	 ABLEHNUNG: {'A03', 'A07', 'C10', 'C00', 'C62', 'C98', 'B38', 'C38', 'C76', 'C08'}\\ \tt VORSCHLAG: {'A03', 'A07', 'C10', 'C00', 'C56', 'C98', 'C32', 'B38', 'C38', 'C76'}\\ \tt ARGUMENT: {'Ich komme dir bei C76 entgegen und lasse C08 draußen – so bleibt das Paket fokussiert.', 'Zur Einhaltung des Limits streiche ich statt C32/C56 den Posten C62; Gesamtaufwand = 3327 ≤ 3357 (Puffer 30).', 'B38 und C38 bleiben unangetastet; die großen Aufwandstreiber A83, C03 und C08 vermeiden wir weiterhin.', 'Falls C62 für dich essenziell ist, sag Bescheid – dann suchen wir eine Alternative über kleinere Tauschposten.'} 
	  } 
	   } 
	   } 
	 & \\ 
 

    \theutterance \stepcounter{utterance}  

    & & & \multicolumn{4}{p{0.6\linewidth}}{\cellcolor[rgb]{1,0.85,0.72}{%
	\makecell[{{p{\linewidth}}}]{% 
	  \tt {\tiny [GM$\langle$B]}  
	 ABLEHNUNG: {'A03', 'A07', 'C10', 'C00', 'C56', 'C98', 'C32', 'B38', 'C38', 'C76'}\\ \tt VORSCHLAG: {'A03', 'A07', 'C10', 'C00', 'C98', 'C62', 'B38', 'C38', 'C76', 'C81'}\\ \tt ARGUMENT: {'Tausche C56 (338) und C32 (720) gegen C62 (263) und C81 (651): gleicher Postenumfang, aber 144 Aufwand weniger – 3327 → 3183 (Puffer 174).', 'Dein Kern bleibt erhalten: B38 und C38 bleiben drin; die großen Aufwandstreiber A83, C03 und C08 vermeiden wir weiterhin.', 'C62 hattest du anfänglich selbst im Paket; ich nehme deinen Hinweis auf und bringe ihn zurück – kombiniert mit einer moderaten C81-Stellschraube statt des schweren C32.', 'Falls C81 für dich nicht passt, schlankere Alternative ohne neues VORSCHLAG: nur C56 → C62; Aufwand 3252 (Puffer 105), Rest unverändert.'} 
	  } 
	   } 
	   } 
	  \\ 
 

    \theutterance \stepcounter{utterance}  

    & & \multicolumn{4}{p{0.6\linewidth}}{\cellcolor[rgb]{0.9,0.9,0.9}{%
	\makecell[{{p{\linewidth}}}]{% 
	  \tt {\tiny [A$\langle$GM]}  
	 ABLEHNUNG: {'A03', 'A07', 'C10', 'C00', 'C56', 'C98', 'C32', 'B38', 'C38', 'C76'}\\ \tt VORSCHLAG: {'A03', 'A07', 'C10', 'C00', 'C98', 'C62', 'B38', 'C38', 'C76', 'C81'}\\ \tt ARGUMENT: {'Tausche C56 (338) und C32 (720) gegen C62 (263) und C81 (651): gleicher Postenumfang, aber 144 Aufwand weniger – 3327 → 3183 (Puffer 174).', 'Dein Kern bleibt erhalten: B38 und C38 bleiben drin; die großen Aufwandstreiber A83, C03 und C08 vermeiden wir weiterhin.', 'C62 hattest du anfänglich selbst im Paket; ich nehme deinen Hinweis auf und bringe ihn zurück – kombiniert mit einer moderaten C81-Stellschraube statt des schweren C32.', 'Falls C81 für dich nicht passt, schlankere Alternative ohne neues VORSCHLAG: nur C56 → C62; Aufwand 3252 (Puffer 105), Rest unverändert.'} 
	  } 
	   } 
	   } 
	 & \\ 
 

    \theutterance \stepcounter{utterance}  

    & \multicolumn{4}{p{0.6\linewidth}}{\cellcolor[rgb]{0.8,1,0.9}{%
	\makecell[{{p{\linewidth}}}]{% 
	  \tt {\tiny [A$\rangle$GM]}  
	 ABLEHNUNG: {'A03', 'A07', 'C10', 'C00', 'C98', 'C62', 'B38', 'C38', 'C76', 'C81'}\\ \tt VORSCHLAG: {'A03', 'A07', 'C10', 'C00', 'C98', 'C62', 'B38', 'C38', 'C76', 'C56'}\\ \tt ARGUMENT: {'Ich tausche nur C81 (651) gegen C56 (338): −313 Aufwand bei gleicher Postenanzahl; Gesamtaufwand = 2870 (Puffer 487 ≤ 3357).', 'Deine Kernpunkte bleiben unangetastet: B38, C38 und C76 sind alle drin; die großen Aufwandstreiber A83, C03 und C08 bleiben draußen.', 'Mit dem verbleibenden Puffer passt kein weiteres Item (alle übrigen Kandidaten ≥ 651) — so bleibt das Paket schlank und effizient.', 'Falls dir C56 grundsätzlich nicht zusagt, Alternativpfad ohne neues VORSCHLAG: Behalte C81 und ersetze stattdessen C76 → C56 (Aufwand 3015), Rest unverändert.'} 
	  } 
	   } 
	   } 
	 & & \\ 
 

    \theutterance \stepcounter{utterance}  

    & & \multicolumn{4}{p{0.6\linewidth}}{\cellcolor[rgb]{0.9,0.9,0.9}{%
	\makecell[{{p{\linewidth}}}]{% 
	  \tt {\tiny [GM$\rangle$B]}  
	 ABLEHNUNG: {'A03', 'A07', 'C10', 'C00', 'C98', 'C62', 'B38', 'C38', 'C76', 'C81'}\\ \tt VORSCHLAG: {'A03', 'A07', 'C10', 'C00', 'C98', 'C62', 'B38', 'C38', 'C76', 'C56'}\\ \tt ARGUMENT: {'Ich tausche nur C81 (651) gegen C56 (338): −313 Aufwand bei gleicher Postenanzahl; Gesamtaufwand = 2870 (Puffer 487 ≤ 3357).', 'Deine Kernpunkte bleiben unangetastet: B38, C38 und C76 sind alle drin; die großen Aufwandstreiber A83, C03 und C08 bleiben draußen.', 'Mit dem verbleibenden Puffer passt kein weiteres Item (alle übrigen Kandidaten ≥ 651) — so bleibt das Paket schlank und effizient.', 'Falls dir C56 grundsätzlich nicht zusagt, Alternativpfad ohne neues VORSCHLAG: Behalte C81 und ersetze stattdessen C76 → C56 (Aufwand 3015), Rest unverändert.'} 
	  } 
	   } 
	   } 
	 & \\ 
 

    \theutterance \stepcounter{utterance}  

    & & & \multicolumn{4}{p{0.6\linewidth}}{\cellcolor[rgb]{1,0.85,0.72}{%
	\makecell[{{p{\linewidth}}}]{% 
	  \tt {\tiny [GM$\langle$B]}  
	 ABLEHNUNG: {'A03', 'A07', 'C10', 'C00', 'C98', 'C62', 'B38', 'C38', 'C76', 'C56'}\\ \tt VORSCHLAG: {'A03', 'A07', 'C10', 'C00', 'C98', 'C62', 'C38', 'C76', 'C08', 'C81'}\\ \tt ARGUMENT: {'Ersetze B38 (552) und C56 (338) durch C08 (716) und C81 (651): Netto +477 Aufwand; Gesamtaufwand = 3347 ≤ 3357 (Puffer 10).', 'Dein Kern bleibt erhalten: C38 und C76 bleiben drin; die großen Ausreißer A83 (909) und C03 (849) bleiben draußen.', 'Wir konsolidieren zwei mittelgroße Posten zu zwei stärker gewichteten – ohne das Limit zu reißen und mit sehr knapper, aber sauberer Reserve.', 'Falls B38 für dich zwingend ist, Alternativpfad ohne neues VORSCHLAG: C56 + A03 → C81 (Rest wie bei dir); Aufwand = 3061 (Puffer 296).'} 
	  } 
	   } 
	   } 
	  \\ 
 

    \theutterance \stepcounter{utterance}  

    & & \multicolumn{4}{p{0.6\linewidth}}{\cellcolor[rgb]{0.9,0.9,0.9}{%
	\makecell[{{p{\linewidth}}}]{% 
	  \tt {\tiny [A$\langle$GM]}  
	 ABLEHNUNG: {'A03', 'A07', 'C10', 'C00', 'C98', 'C62', 'B38', 'C38', 'C76', 'C56'}\\ \tt VORSCHLAG: {'A03', 'A07', 'C10', 'C00', 'C98', 'C62', 'C38', 'C76', 'C08', 'C81'}\\ \tt ARGUMENT: {'Ersetze B38 (552) und C56 (338) durch C08 (716) und C81 (651): Netto +477 Aufwand; Gesamtaufwand = 3347 ≤ 3357 (Puffer 10).', 'Dein Kern bleibt erhalten: C38 und C76 bleiben drin; die großen Ausreißer A83 (909) und C03 (849) bleiben draußen.', 'Wir konsolidieren zwei mittelgroße Posten zu zwei stärker gewichteten – ohne das Limit zu reißen und mit sehr knapper, aber sauberer Reserve.', 'Falls B38 für dich zwingend ist, Alternativpfad ohne neues VORSCHLAG: C56 + A03 → C81 (Rest wie bei dir); Aufwand = 3061 (Puffer 296).'} 
	  } 
	   } 
	   } 
	 & \\ 
 

    \theutterance \stepcounter{utterance}  

    & \multicolumn{4}{p{0.6\linewidth}}{\cellcolor[rgb]{0.8,1,0.9}{%
	\makecell[{{p{\linewidth}}}]{% 
	  \tt {\tiny [A$\rangle$GM]}  
	 ABLEHNUNG: {'A03', 'A07', 'C10', 'C00', 'C98', 'C62', 'C38', 'C76', 'C08', 'C81'}\\ \tt VORSCHLAG: {'A03', 'A07', 'C10', 'C00', 'C98', 'C62', 'C38', 'C76', 'C81', 'C56'}\\ \tt ARGUMENT: {'Einziger Tausch zu deinem Paket: C08 → C56. Dadurch sinkt der Gesamtaufwand von 3347 auf 2969 (Puffer 388 ≤ 3357).', 'Deine Kernposten bleiben drin: C38, C76, C81, C62 und C98 sind alle enthalten; die größten Aufwandstreiber A83 (909) und C03 (849) bleiben draußen.', 'Mit dem verbleibenden Puffer lässt sich kein weiterer sinnvoller Posten ergänzen (alle übrigen Kandidaten ≥ 492 Aufwand).', 'Falls C08 für dich zwingend ist, Alternativpfad ohne neues VORSCHLAG: C81 → C08 bei sonst identischem Paket; Aufwand = 3034 (Puffer 323).'} 
	  } 
	   } 
	   } 
	 & & \\ 
 

    \theutterance \stepcounter{utterance}  

    & & \multicolumn{4}{p{0.6\linewidth}}{\cellcolor[rgb]{0.9,0.9,0.9}{%
	\makecell[{{p{\linewidth}}}]{% 
	  \tt {\tiny [GM$\rangle$B]}  
	 ABLEHNUNG: {'A03', 'A07', 'C10', 'C00', 'C98', 'C62', 'C38', 'C76', 'C08', 'C81'}\\ \tt VORSCHLAG: {'A03', 'A07', 'C10', 'C00', 'C98', 'C62', 'C38', 'C76', 'C81', 'C56'}\\ \tt ARGUMENT: {'Einziger Tausch zu deinem Paket: C08 → C56. Dadurch sinkt der Gesamtaufwand von 3347 auf 2969 (Puffer 388 ≤ 3357).', 'Deine Kernposten bleiben drin: C38, C76, C81, C62 und C98 sind alle enthalten; die größten Aufwandstreiber A83 (909) und C03 (849) bleiben draußen.', 'Mit dem verbleibenden Puffer lässt sich kein weiterer sinnvoller Posten ergänzen (alle übrigen Kandidaten ≥ 492 Aufwand).', 'Falls C08 für dich zwingend ist, Alternativpfad ohne neues VORSCHLAG: C81 → C08 bei sonst identischem Paket; Aufwand = 3034 (Puffer 323).'} 
	  } 
	   } 
	   } 
	 & \\ 
 

    \theutterance \stepcounter{utterance}  

    & & & \multicolumn{4}{p{0.6\linewidth}}{\cellcolor[rgb]{1,0.85,0.72}{%
	\makecell[{{p{\linewidth}}}]{% 
	  \tt {\tiny [GM$\langle$B]}  
	 ABLEHNUNG: {'A03', 'A07', 'C10', 'C00', 'C98', 'C62', 'C38', 'C76', 'C81', 'C56'}\\ \tt VORSCHLAG: {'A03', 'A07', 'C10', 'C00', 'C98', 'C62', 'C38', 'C76', 'C81', 'B38'}\\ \tt ARGUMENT: {'Einziger Tausch: C56 (338) → B38 (552); Gesamtaufwand = 3183 ≤ 3357 (Puffer 174).', 'Ich greife deinen frühen Fokus auf B38 wieder auf; zugleich vermeiden wir weiterhin die großen Aufwandstreiber A83, C03 und C08.', 'Deine Kernposten bleiben erhalten (C38, C76, C81, C62, C98, C10, A07, A03, C00); es ist eine minimalinvasive Anpassung mit solider Reserve.', 'Falls du C56 unbedingt behalten willst: Alternativpfad ohne neues VORSCHLAG – ersetze stattdessen C00 (231) durch B38 (552); Aufwand = 3290 (Puffer 67).'} 
	  } 
	   } 
	   } 
	  \\ 
 

    \theutterance \stepcounter{utterance}  

    & & \multicolumn{4}{p{0.6\linewidth}}{\cellcolor[rgb]{0.9,0.9,0.9}{%
	\makecell[{{p{\linewidth}}}]{% 
	  \tt {\tiny [A$\langle$GM]}  
	 ABLEHNUNG: {'A03', 'A07', 'C10', 'C00', 'C98', 'C62', 'C38', 'C76', 'C81', 'C56'}\\ \tt VORSCHLAG: {'A03', 'A07', 'C10', 'C00', 'C98', 'C62', 'C38', 'C76', 'C81', 'B38'}\\ \tt ARGUMENT: {'Einziger Tausch: C56 (338) → B38 (552); Gesamtaufwand = 3183 ≤ 3357 (Puffer 174).', 'Ich greife deinen frühen Fokus auf B38 wieder auf; zugleich vermeiden wir weiterhin die großen Aufwandstreiber A83, C03 und C08.', 'Deine Kernposten bleiben erhalten (C38, C76, C81, C62, C98, C10, A07, A03, C00); es ist eine minimalinvasive Anpassung mit solider Reserve.', 'Falls du C56 unbedingt behalten willst: Alternativpfad ohne neues VORSCHLAG – ersetze stattdessen C00 (231) durch B38 (552); Aufwand = 3290 (Puffer 67).'} 
	  } 
	   } 
	   } 
	 & \\ 
 

    \theutterance \stepcounter{utterance}  

    & \multicolumn{4}{p{0.6\linewidth}}{\cellcolor[rgb]{0.8,1,0.9}{%
	\makecell[{{p{\linewidth}}}]{% 
	  \tt {\tiny [A$\rangle$GM]}  
	 ABLEHNUNG: {'A03', 'A07', 'C10', 'C00', 'C98', 'C62', 'C38', 'C76', 'C81', 'B38'}\\ \tt VORSCHLAG: {'A03', 'A07', 'C10', 'C00', 'C98', 'C62', 'C38', 'C56', 'C81', 'B38'}\\ \tt ARGUMENT: {'Einziger Tausch: C76 (506) → C56 (338). Damit sinkt der Gesamtaufwand von 3183 auf 3015 (Puffer 342 ≤ 3357).', 'Deine Kernposten bleiben komplett erhalten (C38, B38, C81, C62, C98 usw.); wir vermeiden weiterhin die großen Aufwandstreiber A83, C03 und C08.', 'Das Paket bleibt stabil und verschafft uns zusätzlich komfortable Reserve – ohne weitere Eingriffe.', 'Falls C76 für dich unverzichtbar ist, Alternativpfad ohne neues VORSCHLAG: Behalte C76 und ersetze stattdessen C81 → C56; Aufwand = 2870 (Puffer 487).'} 
	  } 
	   } 
	   } 
	 & & \\ 
 

    \theutterance \stepcounter{utterance}  

    & & \multicolumn{4}{p{0.6\linewidth}}{\cellcolor[rgb]{0.9,0.9,0.9}{%
	\makecell[{{p{\linewidth}}}]{% 
	  \tt {\tiny [GM$\rangle$B]}  
	 ABLEHNUNG: {'A03', 'A07', 'C10', 'C00', 'C98', 'C62', 'C38', 'C76', 'C81', 'B38'}\\ \tt VORSCHLAG: {'A03', 'A07', 'C10', 'C00', 'C98', 'C62', 'C38', 'C56', 'C81', 'B38'}\\ \tt ARGUMENT: {'Einziger Tausch: C76 (506) → C56 (338). Damit sinkt der Gesamtaufwand von 3183 auf 3015 (Puffer 342 ≤ 3357).', 'Deine Kernposten bleiben komplett erhalten (C38, B38, C81, C62, C98 usw.); wir vermeiden weiterhin die großen Aufwandstreiber A83, C03 und C08.', 'Das Paket bleibt stabil und verschafft uns zusätzlich komfortable Reserve – ohne weitere Eingriffe.', 'Falls C76 für dich unverzichtbar ist, Alternativpfad ohne neues VORSCHLAG: Behalte C76 und ersetze stattdessen C81 → C56; Aufwand = 2870 (Puffer 487).'} 
	  } 
	   } 
	   } 
	 & \\ 
 

    \theutterance \stepcounter{utterance}  

    & & & \multicolumn{4}{p{0.6\linewidth}}{\cellcolor[rgb]{1,0.85,0.72}{%
	\makecell[{{p{\linewidth}}}]{% 
	  \tt {\tiny [GM$\langle$B]}  
	 ABLEHNUNG: {'A03', 'A07', 'C10', 'C00', 'C98', 'C62', 'C38', 'C56', 'C81', 'B38'}\\ \tt VORSCHLAG: {'A07', 'C10', 'C00', 'C98', 'C62', 'C38', 'C81', 'B38', 'C76'}\\ \tt ARGUMENT: {'Konsolidierung: Ersetze C56 (338) und A03 (122) durch C76 (506); Netto +46 Aufwand. Gesamtaufwand = 3061 ≤ 3357 (Puffer 296).', 'Dein Kern bleibt erhalten: B38, C38, C81, C62, C98, C10, A07 und C00 sind alle drin; die großen Aufwandstreiber A83 (909), C03 (849) und C08 (716) bleiben draußen.', 'Weniger Komplexität (9 statt 10 Posten) bei nahezu gleichem Aufwand – dafür stärkerer Fokus auf einen hochwertigen Posten.', 'Falls A03 für dich wichtig ist: Alternativpfad ohne neues VORSCHLAG – behalte A03 und ersetze stattdessen A07 → C76; Aufwand = 3052 (Puffer 305).', 'Falls dir C00 weniger wichtig ist: Alternativpfad ohne neues VORSCHLAG – ersetze C56 + C00 → C76; Aufwand = 2952 (Puffer 405).'} 
	  } 
	   } 
	   } 
	  \\ 
 

    \theutterance \stepcounter{utterance}  

    & & \multicolumn{4}{p{0.6\linewidth}}{\cellcolor[rgb]{0.9,0.9,0.9}{%
	\makecell[{{p{\linewidth}}}]{% 
	  \tt {\tiny [A$\langle$GM]}  
	 ABLEHNUNG: {'A03', 'A07', 'C10', 'C00', 'C98', 'C62', 'C38', 'C56', 'C81', 'B38'}\\ \tt VORSCHLAG: {'A07', 'C10', 'C00', 'C98', 'C62', 'C38', 'C81', 'B38', 'C76'}\\ \tt ARGUMENT: {'Konsolidierung: Ersetze C56 (338) und A03 (122) durch C76 (506); Netto +46 Aufwand. Gesamtaufwand = 3061 ≤ 3357 (Puffer 296).', 'Dein Kern bleibt erhalten: B38, C38, C81, C62, C98, C10, A07 und C00 sind alle drin; die großen Aufwandstreiber A83 (909), C03 (849) und C08 (716) bleiben draußen.', 'Weniger Komplexität (9 statt 10 Posten) bei nahezu gleichem Aufwand – dafür stärkerer Fokus auf einen hochwertigen Posten.', 'Falls A03 für dich wichtig ist: Alternativpfad ohne neues VORSCHLAG – behalte A03 und ersetze stattdessen A07 → C76; Aufwand = 3052 (Puffer 305).', 'Falls dir C00 weniger wichtig ist: Alternativpfad ohne neues VORSCHLAG – ersetze C56 + C00 → C76; Aufwand = 2952 (Puffer 405).'} 
	  } 
	   } 
	   } 
	 & \\ 
 

    \theutterance \stepcounter{utterance}  

    & \multicolumn{4}{p{0.6\linewidth}}{\cellcolor[rgb]{0.8,1,0.9}{%
	\makecell[{{p{\linewidth}}}]{% 
	  \tt {\tiny [A$\rangle$GM]}  
	 ABLEHNUNG: {'A07', 'C10', 'C00', 'C98', 'C62', 'C38', 'C81', 'B38', 'C76'}\\ \tt VORSCHLAG: {'A07', 'C10', 'C00', 'C62', 'C38', 'C81', 'B38', 'C76', 'A03'}\\ \tt ARGUMENT: {'Minimaler Tausch: C98 (144) → A03 (122); Gesamtaufwand = 3039 ≤ 3357 (Puffer 318).', 'Dein Kern bleibt vollständig erhalten (B38, C38, C81, C76, C62, C10, A07, C00); wir vermeiden weiterhin A83, C03 und C08.', 'Gleiche Postenanzahl (9), spürbar mehr Wert ohne Mehraufwand.', 'Falls C98 für dich zwingend ist: Alternativpfad ohne neues VORSCHLAG – behalte C98 und ersetze stattdessen C10 (91) → C56 (338); Aufwand = 3308 (Puffer 49).'} 
	  } 
	   } 
	   } 
	 & & \\ 
 

    \theutterance \stepcounter{utterance}  

    & & \multicolumn{4}{p{0.6\linewidth}}{\cellcolor[rgb]{0.9,0.9,0.9}{%
	\makecell[{{p{\linewidth}}}]{% 
	  \tt {\tiny [GM$\rangle$B]}  
	 ABLEHNUNG: {'A07', 'C10', 'C00', 'C98', 'C62', 'C38', 'C81', 'B38', 'C76'}\\ \tt VORSCHLAG: {'A07', 'C10', 'C00', 'C62', 'C38', 'C81', 'B38', 'C76', 'A03'}\\ \tt ARGUMENT: {'Minimaler Tausch: C98 (144) → A03 (122); Gesamtaufwand = 3039 ≤ 3357 (Puffer 318).', 'Dein Kern bleibt vollständig erhalten (B38, C38, C81, C76, C62, C10, A07, C00); wir vermeiden weiterhin A83, C03 und C08.', 'Gleiche Postenanzahl (9), spürbar mehr Wert ohne Mehraufwand.', 'Falls C98 für dich zwingend ist: Alternativpfad ohne neues VORSCHLAG – behalte C98 und ersetze stattdessen C10 (91) → C56 (338); Aufwand = 3308 (Puffer 49).'} 
	  } 
	   } 
	   } 
	 & \\ 
 

    \theutterance \stepcounter{utterance}  

    & & & \multicolumn{4}{p{0.6\linewidth}}{\cellcolor[rgb]{1,0.85,0.72}{%
	\makecell[{{p{\linewidth}}}]{% 
	  \tt {\tiny [GM$\langle$B]}  
	 ABLEHNUNG: {'A07', 'C10', 'C00', 'C62', 'C38', 'C81', 'B38', 'C76', 'A03'}\\ \tt VORSCHLAG: {'A07', 'C10', 'C00', 'C62', 'C38', 'C81', 'B38', 'C76', 'A03', 'C98'}\\ \tt ARGUMENT: {'Einziger Schritt: Wir ergänzen C98 (144). Gesamtaufwand steigt nur auf 3183 ≤ 3357 (Reserve 174).', 'Dein komplettes Paket bleibt unverändert erhalten; wir nutzen lediglich sinnvoll den freien Puffer.', 'Die großen Aufwandstreiber A83 (909), C03 (849) und C08 (716) bleiben weiterhin draußen.', 'Falls du die Aufwandserhöhung vermeiden möchtest: ohne neues VORSCHLAG können wir alternativ A03 → C98 (Aufwand 3061) oder C10 → C98 (3092) denken – Rest identisch zu deinem Paket.'} 
	  } 
	   } 
	   } 
	  \\ 
 

    \theutterance \stepcounter{utterance}  

    & & \multicolumn{4}{p{0.6\linewidth}}{\cellcolor[rgb]{0.9,0.9,0.9}{%
	\makecell[{{p{\linewidth}}}]{% 
	  \tt {\tiny [A$\langle$GM]}  
	 ABLEHNUNG: {'A07', 'C10', 'C00', 'C62', 'C38', 'C81', 'B38', 'C76', 'A03'}\\ \tt VORSCHLAG: {'A07', 'C10', 'C00', 'C62', 'C38', 'C81', 'B38', 'C76', 'A03', 'C98'}\\ \tt ARGUMENT: {'Einziger Schritt: Wir ergänzen C98 (144). Gesamtaufwand steigt nur auf 3183 ≤ 3357 (Reserve 174).', 'Dein komplettes Paket bleibt unverändert erhalten; wir nutzen lediglich sinnvoll den freien Puffer.', 'Die großen Aufwandstreiber A83 (909), C03 (849) und C08 (716) bleiben weiterhin draußen.', 'Falls du die Aufwandserhöhung vermeiden möchtest: ohne neues VORSCHLAG können wir alternativ A03 → C98 (Aufwand 3061) oder C10 → C98 (3092) denken – Rest identisch zu deinem Paket.'} 
	  } 
	   } 
	   } 
	 & \\ 
 

    \theutterance \stepcounter{utterance}  

    & \multicolumn{4}{p{0.6\linewidth}}{\cellcolor[rgb]{0.8,1,0.9}{%
	\makecell[{{p{\linewidth}}}]{% 
	  \tt {\tiny [A$\rangle$GM]}  
	 ABLEHNUNG: {'A07', 'C10', 'C00', 'C62', 'C38', 'C81', 'B38', 'C76', 'A03', 'C98'}\\ \tt VORSCHLAG: {'A07', 'C00', 'C62', 'C38', 'C81', 'B38', 'C76', 'A03', 'C56'}\\ \tt ARGUMENT: {'Ich konsolidiere zwei Kleinteile (C10=91, C98=144) zu einem effizienten mittleren Posten C56=338; Gesamtaufwand = 3286 ≤ 3357 (Puffer 71).', 'Dein Kern bleibt erhalten: C76, C81, B38, C38, C62, A07, A03, C00 bleiben alle drin; die großen Aufwandstreiber A83 (909), C03 (849) und C08 (716) bleiben draußen.', 'Weniger Komplexität (9 statt 10 Posten) bei sauberer Reserve und stabilem Aufwand-Profil.', 'Falls C10 oder C98 für dich unverzichtbar ist: Ohne neues VORSCHLAG können wir alternativ genau einen von beiden behalten und stattdessen C76 → C56 tauschen; Aufwand = 2871 (behalte C10) bzw. 2924 (behalte C98).'} 
	  } 
	   } 
	   } 
	 & & \\ 
 

    \theutterance \stepcounter{utterance}  

    & & \multicolumn{4}{p{0.6\linewidth}}{\cellcolor[rgb]{0.9,0.9,0.9}{%
	\makecell[{{p{\linewidth}}}]{% 
	  \tt {\tiny [GM$\rangle$B]}  
	 ABLEHNUNG: {'A07', 'C10', 'C00', 'C62', 'C38', 'C81', 'B38', 'C76', 'A03', 'C98'}\\ \tt VORSCHLAG: {'A07', 'C00', 'C62', 'C38', 'C81', 'B38', 'C76', 'A03', 'C56'}\\ \tt ARGUMENT: {'Ich konsolidiere zwei Kleinteile (C10=91, C98=144) zu einem effizienten mittleren Posten C56=338; Gesamtaufwand = 3286 ≤ 3357 (Puffer 71).', 'Dein Kern bleibt erhalten: C76, C81, B38, C38, C62, A07, A03, C00 bleiben alle drin; die großen Aufwandstreiber A83 (909), C03 (849) und C08 (716) bleiben draußen.', 'Weniger Komplexität (9 statt 10 Posten) bei sauberer Reserve und stabilem Aufwand-Profil.', 'Falls C10 oder C98 für dich unverzichtbar ist: Ohne neues VORSCHLAG können wir alternativ genau einen von beiden behalten und stattdessen C76 → C56 tauschen; Aufwand = 2871 (behalte C10) bzw. 2924 (behalte C98).'} 
	  } 
	   } 
	   } 
	 & \\ 
 

    \theutterance \stepcounter{utterance}  

    & & & \multicolumn{4}{p{0.6\linewidth}}{\cellcolor[rgb]{1,0.85,0.72}{%
	\makecell[{{p{\linewidth}}}]{% 
	  \tt {\tiny [GM$\langle$B]}  
	 ABLEHNUNG: {'A07', 'C00', 'C62', 'C38', 'C81', 'B38', 'C76', 'A03', 'C56'}\\ \tt VORSCHLAG: {'A07', 'C10', 'C00', 'C62', 'C38', 'C81', 'B38', 'C76', 'A03', 'C98'}\\ \tt ARGUMENT: {'Tausche C56 (338) gegen die zwei schlanken Posten C10 (91) und C98 (144): −103 Aufwand bei klarer Effizienzsteigerung; Gesamtaufwand = 3183 ≤ 3357 (Reserve 174).', 'Dein Kern bleibt vollständig erhalten: B38, C38, C81, C76, C62, A07, A03 und C00 bleiben alle drin.', 'Wir vermeiden weiterhin die großen Aufwandstreiber A83 (909), C03 (849) und C08 (716).', 'Falls dir 10 Posten zu viel sind, Kompromissoption ohne neues VORSCHLAG: gleiches Paket, aber mit A03 oder A07 raus – Aufwand 3061 bzw. 3052, Rest identisch.'} 
	  } 
	   } 
	   } 
	  \\ 
 

    \theutterance \stepcounter{utterance}  

    & & \multicolumn{4}{p{0.6\linewidth}}{\cellcolor[rgb]{0.9,0.9,0.9}{%
	\makecell[{{p{\linewidth}}}]{% 
	  \tt {\tiny [A$\langle$GM]}  
	 ABLEHNUNG: {'A07', 'C00', 'C62', 'C38', 'C81', 'B38', 'C76', 'A03', 'C56'}\\ \tt VORSCHLAG: {'A07', 'C10', 'C00', 'C62', 'C38', 'C81', 'B38', 'C76', 'A03', 'C98'}\\ \tt ARGUMENT: {'Tausche C56 (338) gegen die zwei schlanken Posten C10 (91) und C98 (144): −103 Aufwand bei klarer Effizienzsteigerung; Gesamtaufwand = 3183 ≤ 3357 (Reserve 174).', 'Dein Kern bleibt vollständig erhalten: B38, C38, C81, C76, C62, A07, A03 und C00 bleiben alle drin.', 'Wir vermeiden weiterhin die großen Aufwandstreiber A83 (909), C03 (849) und C08 (716).', 'Falls dir 10 Posten zu viel sind, Kompromissoption ohne neues VORSCHLAG: gleiches Paket, aber mit A03 oder A07 raus – Aufwand 3061 bzw. 3052, Rest identisch.'} 
	  } 
	   } 
	   } 
	 & \\ 
 

    \theutterance \stepcounter{utterance}  

    & \multicolumn{4}{p{0.6\linewidth}}{\cellcolor[rgb]{0.8,1,0.9}{%
	\makecell[{{p{\linewidth}}}]{% 
	  \tt {\tiny [A$\rangle$GM]}  
	 ABLEHNUNG: {'A07', 'C10', 'C00', 'C62', 'C38', 'C81', 'B38', 'C76', 'A03', 'C98'}\\ \tt VORSCHLAG: {'A07', 'C10', 'C00', 'C62', 'C38', 'B38', 'C76', 'A03', 'C98', 'C56'}\\ \tt ARGUMENT: {'Einziger Tausch: C81 (651) → C56 (338). Ergebnis: −313 Aufwand, gleiche Postenanzahl; Gesamtaufwand = 2870 (Puffer 487 ≤ 3357).', 'Dein Kern bleibt vollständig erhalten: C76, B38, C38, C62, C98, C10, A07, A03 und C00 bleiben alle drin.', 'Wir vermeiden weiterhin die größten Aufwandstreiber A83 (909), C03 (849) und C08 (716) — bei deutlich komfortabler Reserve.', 'Falls C81 für dich unverzichtbar ist, Kompromiss ohne neues VORSCHLAG: Behalte C81 und tausche stattdessen C76 → C56; Aufwand = 3015 (Puffer 342).'} 
	  } 
	   } 
	   } 
	 & & \\ 
 

    \theutterance \stepcounter{utterance}  

    & & \multicolumn{4}{p{0.6\linewidth}}{\cellcolor[rgb]{0.9,0.9,0.9}{%
	\makecell[{{p{\linewidth}}}]{% 
	  \tt {\tiny [GM$\rangle$B]}  
	 ABLEHNUNG: {'A07', 'C10', 'C00', 'C62', 'C38', 'C81', 'B38', 'C76', 'A03', 'C98'}\\ \tt VORSCHLAG: {'A07', 'C10', 'C00', 'C62', 'C38', 'B38', 'C76', 'A03', 'C98', 'C56'}\\ \tt ARGUMENT: {'Einziger Tausch: C81 (651) → C56 (338). Ergebnis: −313 Aufwand, gleiche Postenanzahl; Gesamtaufwand = 2870 (Puffer 487 ≤ 3357).', 'Dein Kern bleibt vollständig erhalten: C76, B38, C38, C62, C98, C10, A07, A03 und C00 bleiben alle drin.', 'Wir vermeiden weiterhin die größten Aufwandstreiber A83 (909), C03 (849) und C08 (716) — bei deutlich komfortabler Reserve.', 'Falls C81 für dich unverzichtbar ist, Kompromiss ohne neues VORSCHLAG: Behalte C81 und tausche stattdessen C76 → C56; Aufwand = 3015 (Puffer 342).'} 
	  } 
	   } 
	   } 
	 & \\ 
 

    \theutterance \stepcounter{utterance}  

    & & & \multicolumn{4}{p{0.6\linewidth}}{\cellcolor[rgb]{1,0.85,0.72}{%
	\makecell[{{p{\linewidth}}}]{% 
	  \tt {\tiny [GM$\langle$B]}  
	 ABLEHNUNG: {'A07', 'C10', 'C00', 'C62', 'C38', 'B38', 'C76', 'A03', 'C98', 'C56'}\\ \tt VORSCHLAG: {'A07', 'C10', 'C00', 'C62', 'C38', 'B38', 'C76', 'A03', 'C98', 'C81'}\\ \tt ARGUMENT: {'Einziger Tausch: C56 (338) → C81 (651). Aufwand steigt moderat von 2870 auf 3183 (Puffer 174 ≤ 3357).', 'Dein Kern bleibt vollständig erhalten (B38, C38, C76, C62, C10, A07, A03, C00, C98); wir vermeiden weiterhin die großen Aufwandstreiber A83 (909), C03 (849) und C08 (716).', 'Wir nutzen den vorhandenen Puffer sinnvoll für ein hochwertigeres Item statt eines schwachen mittleren – ohne das Limit zu gefährden.', 'Falls dir 10 Posten zu viel sind oder C81 dir zu schwer erscheint: Kompromiss ohne neues VORSCHLAG – gleicher Tausch C56 → C81, zusätzlich A03 streichen; Aufwand = 3061 (komfortabler Puffer 296), restliche Posten unverändert.'} 
	  } 
	   } 
	   } 
	  \\ 
 

    \theutterance \stepcounter{utterance}  

    & & \multicolumn{4}{p{0.6\linewidth}}{\cellcolor[rgb]{0.9,0.9,0.9}{%
	\makecell[{{p{\linewidth}}}]{% 
	  \tt {\tiny [A$\langle$GM]}  
	 ABLEHNUNG: {'A07', 'C10', 'C00', 'C62', 'C38', 'B38', 'C76', 'A03', 'C98', 'C56'}\\ \tt VORSCHLAG: {'A07', 'C10', 'C00', 'C62', 'C38', 'B38', 'C76', 'A03', 'C98', 'C81'}\\ \tt ARGUMENT: {'Einziger Tausch: C56 (338) → C81 (651). Aufwand steigt moderat von 2870 auf 3183 (Puffer 174 ≤ 3357).', 'Dein Kern bleibt vollständig erhalten (B38, C38, C76, C62, C10, A07, A03, C00, C98); wir vermeiden weiterhin die großen Aufwandstreiber A83 (909), C03 (849) und C08 (716).', 'Wir nutzen den vorhandenen Puffer sinnvoll für ein hochwertigeres Item statt eines schwachen mittleren – ohne das Limit zu gefährden.', 'Falls dir 10 Posten zu viel sind oder C81 dir zu schwer erscheint: Kompromiss ohne neues VORSCHLAG – gleicher Tausch C56 → C81, zusätzlich A03 streichen; Aufwand = 3061 (komfortabler Puffer 296), restliche Posten unverändert.'} 
	  } 
	   } 
	   } 
	 & \\ 
 

    \theutterance \stepcounter{utterance}  

    & \multicolumn{4}{p{0.6\linewidth}}{\cellcolor[rgb]{0.8,1,0.9}{%
	\makecell[{{p{\linewidth}}}]{% 
	  \tt {\tiny [A$\rangle$GM]}  
	 ZUSTIMMUNG: {'A07', 'C10', 'C00', 'C62', 'C38', 'B38', 'C76', 'A03', 'C98', 'C81'}\\ \tt ARGUMENT: {'Gesamtaufwand 3183 ≤ 3357 (Reserve 174).', 'Wir vermeiden weiterhin die größten Aufwandstreiber A83 (909), C03 (849) und C08 (716).', 'Dein Kern bleibt vollständig erhalten; das ist ein fairer und balancierter Abschluss.'} 
	  } 
	   } 
	   } 
	 & & \\ 
 

    \theutterance \stepcounter{utterance}  

    & & & \multicolumn{2}{p{0.3\linewidth}}{\cellcolor[rgb]{0.95,0.95,0.95}{%
	\makecell[{{p{\linewidth}}}]{% 
	  \tt {\tiny [GM$|$GM]}  
	 game successful 
	  } 
	   } 
	   } 
	 & & \\ 
 

    \theutterance \stepcounter{utterance}  

    & & & \multicolumn{2}{p{0.3\linewidth}}{\cellcolor[rgb]{0.95,0.95,0.95}{%
	\makecell[{{p{\linewidth}}}]{% 
	  \tt {\tiny [GM$|$GM]}  
	 end game 
	  } 
	   } 
	   } 
	 & & \\ 
 

\end{supertabular}
}

\end{document}
