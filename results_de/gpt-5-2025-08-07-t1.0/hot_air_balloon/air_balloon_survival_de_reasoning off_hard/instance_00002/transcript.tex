
\documentclass{article}
\usepackage{colortbl}
\usepackage{makecell}
\usepackage{multirow}
\usepackage{supertabular}

\begin{document}

\newcounter{utterance}

\twocolumn

{ \footnotesize  \setcounter{utterance}{1}
\setlength{\tabcolsep}{0pt}
\begin{supertabular}{c@{$\;$}|p{.15\linewidth}@{}p{.15\linewidth}p{.15\linewidth}p{.15\linewidth}p{.15\linewidth}p{.15\linewidth}}

    \# & $\;$A & \multicolumn{4}{c}{Game Master} & $\;\:$B\\
    \hline 

    \theutterance \stepcounter{utterance}  

    & & \multicolumn{4}{p{0.6\linewidth}}{\cellcolor[rgb]{0.9,0.9,0.9}{%
	\makecell[{{p{\linewidth}}}]{% 
	  \tt {\tiny [A$\langle$GM]}  
	 Du nimmst an einem kollaborativen Verhandlungspiel Teil.\\ \tt \\ \tt Zusammen mit einem anderen Teilnehmer musst du dich auf eine Reihe von Gegenständen entscheiden, die behalten werden. Jeder von euch hat eine persönliche Verteilung über die Wichtigkeit der einzelnen Gegenstände. Jeder von euch hat eine eigene Meinung darüber, wie wichtig jeder einzelne Gegenstand ist (Gegenstandswichtigkeit). Du kennst die Wichtigkeitsverteilung des anderen Spielers nicht. Zusätzlich siehst du, wie viel Aufwand jeder Gegenstand verursacht.  \\ \tt Ihr dürft euch nur auf eine Reihe von Gegenständen einigen, wenn der Gesamtaufwand der ausgewählten Gegenstände den Maximalaufwand nicht überschreitet:\\ \tt \\ \tt Maximalaufwand = 3203\\ \tt \\ \tt Hier sind die einzelnen Aufwände der Gegenstände:\\ \tt \\ \tt Aufwand der Gegenstände = {"C76": 52, "C38": 577, "C56": 177, "A03": 702, "A07": 119, "A83": 232, "C98": 577, "C08": 205, "C62": 516, "C00": 582, "C10": 676, "C81": 906, "B38": 316, "C03": 433, "C32": 336}\\ \tt \\ \tt Hier ist deine persönliche Verteilung der Wichtigkeit der einzelnen Gegenstände:\\ \tt \\ \tt Werte der Gegenstandswichtigkeit = {"C76": 138, "C38": 583, "C56": 868, "A03": 822, "A07": 783, "A83": 65, "C98": 262, "C08": 121, "C62": 508, "C00": 780, "C10": 461, "C81": 484, "B38": 668, "C03": 389, "C32": 808}\\ \tt \\ \tt Ziel:\\ \tt \\ \tt Dein Ziel ist es, eine Reihe von Gegenständen auszuhandeln, die dir möglichst viel bringt (d. h. Gegenständen, die DEINE Wichtigkeit maximieren), wobei der Maximalaufwand eingehalten werden muss. Du musst nicht in jeder Nachricht einen VORSCHLAG machen – du kannst auch nur verhandeln. Alle Taktiken sind erlaubt!\\ \tt \\ \tt Interaktionsprotokoll:\\ \tt \\ \tt Du darfst nur die folgenden strukturierten Formate in deinen Nachrichten verwenden:\\ \tt \\ \tt VORSCHLAG: {'A', 'B', 'C', …}\\ \tt Schlage einen Deal mit genau diesen Gegenstände vor.\\ \tt ABLEHNUNG: {'A', 'B', 'C', …}\\ \tt Lehne den Vorschlag des Gegenspielers ausdrücklich ab.\\ \tt ARGUMENT: {'...'}\\ \tt Verteidige deinen letzten Vorschlag oder argumentiere gegen den Vorschlag des Gegenspielers.\\ \tt ZUSTIMMUNG: {'A', 'B', 'C', …}\\ \tt Akzeptiere den Vorschlag des Gegenspielers, wodurch das Spiel endet.\\ \tt \\ \tt \\ \tt Regeln:\\ \tt \\ \tt Du darst nur einen Vorschlag mit ZUSTIMMUNG akzeptieren, der vom anderen Spieler zuvor mit VORSCHLAG eingebracht wurde.\\ \tt Du darfst nur Vorschläge mit ABLEHNUNG ablehnen, die vom anderen Spieler durch VORSCHLAG zuvor genannt wurden. \\ \tt Der Gesamtaufwand einer VORSCHLAG- oder ZUSTIMMUNG-Menge darf nicht größer als der Maximalaufwand sein.  \\ \tt Offenbare deine versteckte Wichtigkeitsverteilung nicht.\\ \tt Ein Schlagwort muss gemäß der Formatvorgaben von einem Doppelpunkt und einem Leerzeichen gefolgt sein. Das Argument muss eine Python-Menge mit 0 oder mehr Strings sein.  \\ \tt Die Form muss also SCHLAGWORT: {...} sein.\\ \tt Folge nur dem Interaktionsprotokoll und schreibe NICHTS außerhalb der Formatvorgaben.\\ \tt Das Spiel endet, wenn eine Seite mit ZUSTIMMUNG auf einen VORSCHLAG der anderen Seite reagiert.  \\ \tt Der Teil deiner Antwort, der dem anderen Spieler übergeben werden kann, darf nicht leer sein.  \\ \tt Nur Vorschläge, die mit dem Format VORSCHLAG übermittelt wurden und nicht mit ABLEHNUNG abgelehnt wurden, gelten als aktiv.  \\ \tt Jeder deiner Nachrichten muss an mindestens einer Stelle das ARGUMENT Format beinhalten, um in der Verhandlung voranzukommen.\\ \tt  
	  } 
	   } 
	   } 
	 & \\ 
 

    \theutterance \stepcounter{utterance}  

    & \multicolumn{4}{p{0.6\linewidth}}{\cellcolor[rgb]{0.8,1,0.9}{%
	\makecell[{{p{\linewidth}}}]{% 
	  \tt {\tiny [A$\rangle$GM]}  
	 VORSCHLAG: {'A07', 'C56', 'C76', 'C32', 'B38', 'C00', 'A03', 'C38', 'C08'}\\ \tt ARGUMENT: {'Dieser Vorschlag bleibt mit Gesamtaufwand 3066 klar unter dem Limit 3203 und kombiniert aus meiner Sicht sehr effiziente Gegenstände. Wenn dir bestimmte andere Items wichtig sind, können wir über gezielte Tausche sprechen – am ehesten ließen sich C08 oder C76 austauschen, solange der Gesamtaufwand passt. Welche Positionen sind dir besonders wichtig?'} 
	  } 
	   } 
	   } 
	 & & \\ 
 

    \theutterance \stepcounter{utterance}  

    & & \multicolumn{4}{p{0.6\linewidth}}{\cellcolor[rgb]{0.9,0.9,0.9}{%
	\makecell[{{p{\linewidth}}}]{% 
	  \tt {\tiny [GM$\rangle$B]}  
	 Du nimmst an einem kollaborativen Verhandlungspiel Teil.\\ \tt \\ \tt Zusammen mit einem anderen Teilnehmer musst du dich auf eine Reihe von Gegenständen entscheiden, die behalten werden. Jeder von euch hat eine persönliche Verteilung über die Wichtigkeit der einzelnen Gegenstände. Jeder von euch hat eine eigene Meinung darüber, wie wichtig jeder einzelne Gegenstand ist (Gegenstandswichtigkeit). Du kennst die Wichtigkeitsverteilung des anderen Spielers nicht. Zusätzlich siehst du, wie viel Aufwand jeder Gegenstand verursacht.  \\ \tt Ihr dürft euch nur auf eine Reihe von Gegenständen einigen, wenn der Gesamtaufwand der ausgewählten Gegenstände den Maximalaufwand nicht überschreitet:\\ \tt \\ \tt Maximalaufwand = 3203\\ \tt \\ \tt Hier sind die einzelnen Aufwände der Gegenstände:\\ \tt \\ \tt Aufwand der Gegenstände = {"C76": 52, "C38": 577, "C56": 177, "A03": 702, "A07": 119, "A83": 232, "C98": 577, "C08": 205, "C62": 516, "C00": 582, "C10": 676, "C81": 906, "B38": 316, "C03": 433, "C32": 336}\\ \tt \\ \tt Hier ist deine persönliche Verteilung der Wichtigkeit der einzelnen Gegenstände:\\ \tt \\ \tt Werte der Gegenstandswichtigkeit = {"C56": 65, "A03": 121, "C32": 138, "A07": 262, "C00": 389, "B38": 461, "C38": 484, "C62": 508, "C81": 583, "C10": 668, "C03": 780, "C98": 783, "C76": 808, "C08": 822, "A83": 868}\\ \tt \\ \tt Ziel:\\ \tt \\ \tt Dein Ziel ist es, eine Reihe von Gegenständen auszuhandeln, die dir möglichst viel bringt (d. h. Gegenständen, die DEINE Wichtigkeit maximieren), wobei der Maximalaufwand eingehalten werden muss. Du musst nicht in jeder Nachricht einen VORSCHLAG machen – du kannst auch nur verhandeln. Alle Taktiken sind erlaubt!\\ \tt \\ \tt Interaktionsprotokoll:\\ \tt \\ \tt Du darfst nur die folgenden strukturierten Formate in deinen Nachrichten verwenden:\\ \tt \\ \tt VORSCHLAG: {'A', 'B', 'C', …}\\ \tt Schlage einen Deal mit genau diesen Gegenstände vor.\\ \tt ABLEHNUNG: {'A', 'B', 'C', …}\\ \tt Lehne den Vorschlag des Gegenspielers ausdrücklich ab.\\ \tt ARGUMENT: {'...'}\\ \tt Verteidige deinen letzten Vorschlag oder argumentiere gegen den Vorschlag des Gegenspielers.\\ \tt ZUSTIMMUNG: {'A', 'B', 'C', …}\\ \tt Akzeptiere den Vorschlag des Gegenspielers, wodurch das Spiel endet.\\ \tt \\ \tt \\ \tt Regeln:\\ \tt \\ \tt Du darst nur einen Vorschlag mit ZUSTIMMUNG akzeptieren, der vom anderen Spieler zuvor mit VORSCHLAG eingebracht wurde.\\ \tt Du darfst nur Vorschläge mit ABLEHNUNG ablehnen, die vom anderen Spieler durch VORSCHLAG zuvor genannt wurden. \\ \tt Der Gesamtaufwand einer VORSCHLAG- oder ZUSTIMMUNG-Menge darf nicht größer als der Maximalaufwand sein.  \\ \tt Offenbare deine versteckte Wichtigkeitsverteilung nicht.\\ \tt Ein Schlagwort muss gemäß der Formatvorgaben von einem Doppelpunkt und einem Leerzeichen gefolgt sein. Das Argument muss eine Python-Menge mit 0 oder mehr Strings sein.  \\ \tt Die Form muss also SCHLAGWORT: {...} sein.\\ \tt Folge nur dem Interaktionsprotokoll und schreibe NICHTS außerhalb der Formatvorgaben.\\ \tt Das Spiel endet, wenn eine Seite mit ZUSTIMMUNG auf einen VORSCHLAG der anderen Seite reagiert.  \\ \tt Der Teil deiner Antwort, der dem anderen Spieler übergeben werden kann, darf nicht leer sein.  \\ \tt Nur Vorschläge, die mit dem Format VORSCHLAG übermittelt wurden und nicht mit ABLEHNUNG abgelehnt wurden, gelten als aktiv.  \\ \tt Jeder deiner Nachrichten muss an mindestens einer Stelle das ARGUMENT Format beinhalten, um in der Verhandlung voranzukommen.\\ \tt \\ \tt \\ \tt Du erhältst nun die erste Nachricht des anderen Spielers.\\ \tt \\ \tt VORSCHLAG: {'A07', 'C56', 'C76', 'C32', 'B38', 'C00', 'A03', 'C38', 'C08'}\\ \tt ARGUMENT: {'Dieser Vorschlag bleibt mit Gesamtaufwand 3066 klar unter dem Limit 3203 und kombiniert aus meiner Sicht sehr effiziente Gegenstände. Wenn dir bestimmte andere Items wichtig sind, können wir über gezielte Tausche sprechen – am ehesten ließen sich C08 oder C76 austauschen, solange der Gesamtaufwand passt. Welche Positionen sind dir besonders wichtig?'} 
	  } 
	   } 
	   } 
	 & \\ 
 

    \theutterance \stepcounter{utterance}  

    & & & \multicolumn{4}{p{0.6\linewidth}}{\cellcolor[rgb]{1,0.85,0.72}{%
	\makecell[{{p{\linewidth}}}]{% 
	  \tt {\tiny [GM$\langle$B]}  
	 VORSCHLAG: {'C76', 'C08', 'A83', 'C03', 'A07', 'C10', 'C98', 'B38', 'C62'}\\ \tt ARGUMENT: {'Mein Vorschlag liegt mit Gesamtaufwand 3126 klar unter dem Limit 3203 und lässt 77 Puffer.', 'Ich übernehme aus deinem Paket die schlanken Anker C76, C08, A07 und B38. Statt der aufwändigeren Blöcke (A03, C00, C38, C56, C32) setze ich auf C03, A83, C98, C10 und C62 – damit holen wir aus dem Budget mehr heraus und bleiben gleichzeitig flexibel.', 'C76 und C08 möchte ich ungern tauschen. Falls dir einzelne der von dir ursprünglich genannten Items besonders wichtig sind, können wir gezielt tauschen: z. B. C38 statt C62 (neuer Gesamtaufwand 3187) oder A03 statt C10 (3152). Sag mir, welche Position dir am meisten am Herzen liegt, dann schlage ich eine passende Anpassung vor, die unter dem Limit bleibt.'} 
	  } 
	   } 
	   } 
	  \\ 
 

    \theutterance \stepcounter{utterance}  

    & & \multicolumn{4}{p{0.6\linewidth}}{\cellcolor[rgb]{0.9,0.9,0.9}{%
	\makecell[{{p{\linewidth}}}]{% 
	  \tt {\tiny [A$\langle$GM]}  
	 VORSCHLAG: {'C76', 'C08', 'A83', 'C03', 'A07', 'C10', 'C98', 'B38', 'C62'}\\ \tt ARGUMENT: {'Mein Vorschlag liegt mit Gesamtaufwand 3126 klar unter dem Limit 3203 und lässt 77 Puffer.', 'Ich übernehme aus deinem Paket die schlanken Anker C76, C08, A07 und B38. Statt der aufwändigeren Blöcke (A03, C00, C38, C56, C32) setze ich auf C03, A83, C98, C10 und C62 – damit holen wir aus dem Budget mehr heraus und bleiben gleichzeitig flexibel.', 'C76 und C08 möchte ich ungern tauschen. Falls dir einzelne der von dir ursprünglich genannten Items besonders wichtig sind, können wir gezielt tauschen: z. B. C38 statt C62 (neuer Gesamtaufwand 3187) oder A03 statt C10 (3152). Sag mir, welche Position dir am meisten am Herzen liegt, dann schlage ich eine passende Anpassung vor, die unter dem Limit bleibt.'} 
	  } 
	   } 
	   } 
	 & \\ 
 

    \theutterance \stepcounter{utterance}  

    & \multicolumn{4}{p{0.6\linewidth}}{\cellcolor[rgb]{0.8,1,0.9}{%
	\makecell[{{p{\linewidth}}}]{% 
	  \tt {\tiny [A$\rangle$GM]}  
	 ABLEHNUNG: {'C76', 'C08', 'A83', 'C03', 'A07', 'C10', 'C98', 'B38', 'C62'}\\ \tt VORSCHLAG: {'C76', 'C08', 'A83', 'C03', 'A07', 'B38', 'C56', 'C32', 'A03', 'C38'}\\ \tt ARGUMENT: {'Ich übernehme deine Anker C76 und C08 (bleiben unangetastet) sowie A83, C03, A07 und B38. Ich ersetze die schweren, weniger effizienten Bausteine durch schlanke Leistungsträger: C56 und C32 rein, plus A03 statt C10 und C38 statt C62 – so bleiben wir mit Gesamtaufwand 3149 deutlich unter 3203 (Puffer 54).', 'Wenn du C62 lieber als C38 möchtest, tauschen wir 1:1 – der Aufwand sinkt dann auf 3088. Falls dir C10 wichtiger als A03 ist, können wir ebenfalls zurücktauschen (neuer Aufwand 3123). An C56 und C32 würde ich gern festhalten, da sie das Paket budgetseitig sehr gesund machen.', 'Damit kommen wir deiner Linie weit entgegen und holen zugleich mehr Output aus dem Budget. Passt dir diese Kombination, oder bevorzugst du die Variante mit C62 bzw. mit C10?'} 
	  } 
	   } 
	   } 
	 & & \\ 
 

    \theutterance \stepcounter{utterance}  

    & & \multicolumn{4}{p{0.6\linewidth}}{\cellcolor[rgb]{0.9,0.9,0.9}{%
	\makecell[{{p{\linewidth}}}]{% 
	  \tt {\tiny [GM$\rangle$B]}  
	 ABLEHNUNG: {'C76', 'C08', 'A83', 'C03', 'A07', 'C10', 'C98', 'B38', 'C62'}\\ \tt VORSCHLAG: {'C76', 'C08', 'A83', 'C03', 'A07', 'B38', 'C56', 'C32', 'A03', 'C38'}\\ \tt ARGUMENT: {'Ich übernehme deine Anker C76 und C08 (bleiben unangetastet) sowie A83, C03, A07 und B38. Ich ersetze die schweren, weniger effizienten Bausteine durch schlanke Leistungsträger: C56 und C32 rein, plus A03 statt C10 und C38 statt C62 – so bleiben wir mit Gesamtaufwand 3149 deutlich unter 3203 (Puffer 54).', 'Wenn du C62 lieber als C38 möchtest, tauschen wir 1:1 – der Aufwand sinkt dann auf 3088. Falls dir C10 wichtiger als A03 ist, können wir ebenfalls zurücktauschen (neuer Aufwand 3123). An C56 und C32 würde ich gern festhalten, da sie das Paket budgetseitig sehr gesund machen.', 'Damit kommen wir deiner Linie weit entgegen und holen zugleich mehr Output aus dem Budget. Passt dir diese Kombination, oder bevorzugst du die Variante mit C62 bzw. mit C10?'} 
	  } 
	   } 
	   } 
	 & \\ 
 

    \theutterance \stepcounter{utterance}  

    & & & \multicolumn{4}{p{0.6\linewidth}}{\cellcolor[rgb]{1,0.85,0.72}{%
	\makecell[{{p{\linewidth}}}]{% 
	  \tt {\tiny [GM$\langle$B]}  
	 ABLEHNUNG: {'C76', 'C08', 'A83', 'C03', 'A07', 'B38', 'C56', 'C32', 'A03', 'C38'}\\ \tt VORSCHLAG: {'C76', 'C08', 'A83', 'C03', 'B38', 'C56', 'C10', 'C62', 'C98'}\\ \tt ARGUMENT: {'Mein Paket bleibt mit Gesamtaufwand 3184 klar unter dem Limit 3203 (Puffer 19).', 'Ich übernehme viele deiner priorisierten Bausteine und halte C56 bewusst drin. Zugleich greife ich deine angebotenen Tausche auf: C10 statt A03 und C62 statt C38 – das macht das Set ressourceneffizienter.', 'Um Raum für einen starken Hebel zu schaffen, lasse ich A07 und C32 außen vor und nehme dafür C98 hinein. So holen wir mehr Wirkung pro Aufwand aus dem Budget.', 'Falls C32 für dich unverzichtbar ist, können wir eine Alternative prüfen, bei der C32 zurückkommt und wir dafür B38 herausnehmen und A83 -> A07 tauschen; der Aufwand läge dann bei 3191 und bleibt sauber unter dem Limit.', 'Gib mir Bescheid, ob dir dieser Kompromiss mit C56 zusagt – oder ob du die C32-Rückhol-Variante bevorzugst.'} 
	  } 
	   } 
	   } 
	  \\ 
 

    \theutterance \stepcounter{utterance}  

    & & \multicolumn{4}{p{0.6\linewidth}}{\cellcolor[rgb]{0.9,0.9,0.9}{%
	\makecell[{{p{\linewidth}}}]{% 
	  \tt {\tiny [A$\langle$GM]}  
	 ABLEHNUNG: {'C76', 'C08', 'A83', 'C03', 'A07', 'B38', 'C56', 'C32', 'A03', 'C38'}\\ \tt VORSCHLAG: {'C76', 'C08', 'A83', 'C03', 'B38', 'C56', 'C10', 'C62', 'C98'}\\ \tt ARGUMENT: {'Mein Paket bleibt mit Gesamtaufwand 3184 klar unter dem Limit 3203 (Puffer 19).', 'Ich übernehme viele deiner priorisierten Bausteine und halte C56 bewusst drin. Zugleich greife ich deine angebotenen Tausche auf: C10 statt A03 und C62 statt C38 – das macht das Set ressourceneffizienter.', 'Um Raum für einen starken Hebel zu schaffen, lasse ich A07 und C32 außen vor und nehme dafür C98 hinein. So holen wir mehr Wirkung pro Aufwand aus dem Budget.', 'Falls C32 für dich unverzichtbar ist, können wir eine Alternative prüfen, bei der C32 zurückkommt und wir dafür B38 herausnehmen und A83 -> A07 tauschen; der Aufwand läge dann bei 3191 und bleibt sauber unter dem Limit.', 'Gib mir Bescheid, ob dir dieser Kompromiss mit C56 zusagt – oder ob du die C32-Rückhol-Variante bevorzugst.'} 
	  } 
	   } 
	   } 
	 & \\ 
 

    \theutterance \stepcounter{utterance}  

    & \multicolumn{4}{p{0.6\linewidth}}{\cellcolor[rgb]{0.8,1,0.9}{%
	\makecell[{{p{\linewidth}}}]{% 
	  \tt {\tiny [A$\rangle$GM]}  
	 ABLEHNUNG: {'C76', 'C08', 'A83', 'C03', 'B38', 'C56', 'C10', 'C62', 'C98'}\\ \tt VORSCHLAG: {'C76', 'C08', 'A07', 'C03', 'B38', 'C56', 'C10', 'C62', 'C38'}\\ \tt ARGUMENT: {'Ich halte deine Anker C76, C08, C56, B38, C03, C62 und C10 vollständig drin und mache nur zwei gezielte Anpassungen: A07 statt A83 sowie C38 statt C98.', 'Budget: Gesamtaufwand 3071, also 132 Puffer unter dem Limit 3203 – das schafft Stabilität für uns beide.', 'Die beiden Tausche orientieren sich an deinen zuvor genannten Optionen (u. a. C38 als Alternative und A07 als leichtgewichtiger Baustein) und verbessern das Set ohne Mehraufwand.', 'Falls dir C98 wichtig ist, können wir C38 1:1 gegen C98 tauschen (Aufwand unverändert). Alternativ können wir C10 durch A03 ersetzen; der Aufwand läge dann bei 3097 und bleibt komfortabel unter dem Limit.', 'Gib mir Bescheid, ob dir diese schlanke Variante mit Puffer passt – oder ob du die Option mit C98 bzw. mit A03 bevorzugst.'} 
	  } 
	   } 
	   } 
	 & & \\ 
 

    \theutterance \stepcounter{utterance}  

    & & \multicolumn{4}{p{0.6\linewidth}}{\cellcolor[rgb]{0.9,0.9,0.9}{%
	\makecell[{{p{\linewidth}}}]{% 
	  \tt {\tiny [GM$\rangle$B]}  
	 ABLEHNUNG: {'C76', 'C08', 'A83', 'C03', 'B38', 'C56', 'C10', 'C62', 'C98'}\\ \tt VORSCHLAG: {'C76', 'C08', 'A07', 'C03', 'B38', 'C56', 'C10', 'C62', 'C38'}\\ \tt ARGUMENT: {'Ich halte deine Anker C76, C08, C56, B38, C03, C62 und C10 vollständig drin und mache nur zwei gezielte Anpassungen: A07 statt A83 sowie C38 statt C98.', 'Budget: Gesamtaufwand 3071, also 132 Puffer unter dem Limit 3203 – das schafft Stabilität für uns beide.', 'Die beiden Tausche orientieren sich an deinen zuvor genannten Optionen (u. a. C38 als Alternative und A07 als leichtgewichtiger Baustein) und verbessern das Set ohne Mehraufwand.', 'Falls dir C98 wichtig ist, können wir C38 1:1 gegen C98 tauschen (Aufwand unverändert). Alternativ können wir C10 durch A03 ersetzen; der Aufwand läge dann bei 3097 und bleibt komfortabel unter dem Limit.', 'Gib mir Bescheid, ob dir diese schlanke Variante mit Puffer passt – oder ob du die Option mit C98 bzw. mit A03 bevorzugst.'} 
	  } 
	   } 
	   } 
	 & \\ 
 

    \theutterance \stepcounter{utterance}  

    & & & \multicolumn{4}{p{0.6\linewidth}}{\cellcolor[rgb]{1,0.85,0.72}{%
	\makecell[{{p{\linewidth}}}]{% 
	  \tt {\tiny [GM$\langle$B]}  
	 ABLEHNUNG: {'C76', 'C08', 'A07', 'C03', 'B38', 'C56', 'C10', 'C62', 'C38'}\\ \tt VORSCHLAG: {'C76', 'C08', 'A07', 'C03', 'B38', 'C56', 'C10', 'C62', 'C98'}\\ \tt ARGUMENT: {'Ich übernehme dein Paket nahezu unverändert und setze nur den von dir selbst vorgeschlagenen 1:1-Tausch C38 -> C98 um; der Gesamtaufwand bleibt exakt 3071 und damit 132 unter dem Limit 3203.', 'Deine Anker (C76, C08, C56, B38, C03, C62, C10) bleiben vollständig erhalten; der Tausch erhöht die Wirkung ohne jedes Budgetrisiko.', 'Optionaler Feinschliff: Wenn du noch etwas mehr Output mitnehmen willst, könnten wir A07 -> A83 tauschen; der Aufwand läge dann bei 3184 (weiter sicher unter dem Limit). Primär geht es mir aber um C98 statt C38.'} 
	  } 
	   } 
	   } 
	  \\ 
 

    \theutterance \stepcounter{utterance}  

    & & \multicolumn{4}{p{0.6\linewidth}}{\cellcolor[rgb]{0.9,0.9,0.9}{%
	\makecell[{{p{\linewidth}}}]{% 
	  \tt {\tiny [A$\langle$GM]}  
	 ABLEHNUNG: {'C76', 'C08', 'A07', 'C03', 'B38', 'C56', 'C10', 'C62', 'C38'}\\ \tt VORSCHLAG: {'C76', 'C08', 'A07', 'C03', 'B38', 'C56', 'C10', 'C62', 'C98'}\\ \tt ARGUMENT: {'Ich übernehme dein Paket nahezu unverändert und setze nur den von dir selbst vorgeschlagenen 1:1-Tausch C38 -> C98 um; der Gesamtaufwand bleibt exakt 3071 und damit 132 unter dem Limit 3203.', 'Deine Anker (C76, C08, C56, B38, C03, C62, C10) bleiben vollständig erhalten; der Tausch erhöht die Wirkung ohne jedes Budgetrisiko.', 'Optionaler Feinschliff: Wenn du noch etwas mehr Output mitnehmen willst, könnten wir A07 -> A83 tauschen; der Aufwand läge dann bei 3184 (weiter sicher unter dem Limit). Primär geht es mir aber um C98 statt C38.'} 
	  } 
	   } 
	   } 
	 & \\ 
 

    \theutterance \stepcounter{utterance}  

    & \multicolumn{4}{p{0.6\linewidth}}{\cellcolor[rgb]{0.8,1,0.9}{%
	\makecell[{{p{\linewidth}}}]{% 
	  \tt {\tiny [A$\rangle$GM]}  
	 ABLEHNUNG: {'C76', 'C08', 'A07', 'C03', 'B38', 'C56', 'C10', 'C62', 'C98'}\\ \tt VORSCHLAG: {'C76', 'C08', 'A07', 'C03', 'B38', 'C56', 'A03', 'C38', 'C98'}\\ \tt ARGUMENT: {'Ich übernehme 7 deiner 9 Items unverändert und setze exakt deine zuvor angebotenen Tausche um: C10 -> A03 und C62 -> C38.', 'Budget: 3158 Gesamtaufwand, also 45 unter dem Limit 3203.', 'So bleibt dein Kern (C76, C08, A07, C03, B38, C56, C98) intakt; gleichzeitig wird das Paket spürbar effizienter ohne Mehraufwand.', 'Falls dir C62 unverzichtbar ist, können wir stattdessen C03 -> C32 tauschen und C62 behalten; der Aufwand läge dann bei 3000 und bleibt sehr komfortabel unter dem Limit.'} 
	  } 
	   } 
	   } 
	 & & \\ 
 

    \theutterance \stepcounter{utterance}  

    & & \multicolumn{4}{p{0.6\linewidth}}{\cellcolor[rgb]{0.9,0.9,0.9}{%
	\makecell[{{p{\linewidth}}}]{% 
	  \tt {\tiny [GM$\rangle$B]}  
	 ABLEHNUNG: {'C76', 'C08', 'A07', 'C03', 'B38', 'C56', 'C10', 'C62', 'C98'}\\ \tt VORSCHLAG: {'C76', 'C08', 'A07', 'C03', 'B38', 'C56', 'A03', 'C38', 'C98'}\\ \tt ARGUMENT: {'Ich übernehme 7 deiner 9 Items unverändert und setze exakt deine zuvor angebotenen Tausche um: C10 -> A03 und C62 -> C38.', 'Budget: 3158 Gesamtaufwand, also 45 unter dem Limit 3203.', 'So bleibt dein Kern (C76, C08, A07, C03, B38, C56, C98) intakt; gleichzeitig wird das Paket spürbar effizienter ohne Mehraufwand.', 'Falls dir C62 unverzichtbar ist, können wir stattdessen C03 -> C32 tauschen und C62 behalten; der Aufwand läge dann bei 3000 und bleibt sehr komfortabel unter dem Limit.'} 
	  } 
	   } 
	   } 
	 & \\ 
 

    \theutterance \stepcounter{utterance}  

    & & & \multicolumn{4}{p{0.6\linewidth}}{\cellcolor[rgb]{1,0.85,0.72}{%
	\makecell[{{p{\linewidth}}}]{% 
	  \tt {\tiny [GM$\langle$B]}  
	 ABLEHNUNG: {'C76', 'C08', 'A07', 'C03', 'B38', 'C56', 'A03', 'C38', 'C98'}\\ \tt VORSCHLAG: {'C76', 'C08', 'A07', 'C03', 'B38', 'C10', 'C62', 'C98', 'A83'}\\ \tt ARGUMENT: {'Ich übernehme 6 deiner 9 Items unverändert und setze deine eigenen 1:1-Optionen um: A03 -> C10 und C38 -> C62; zusätzlich ersetze ich nur C56 durch A83, um das Budget effizienter zu nutzen.', 'Budget: 3126 Gesamtaufwand, also 77 unter dem Limit 3203.', 'Dein Kern bleibt damit erhalten (C76, C08, A07, C03, B38, plus die von dir angebotenen Alternativen C10 und C62, sowie C98); der Tausch C56 -> A83 steigert die Gesamtwirkung ohne Budgetrisiko.', 'Falls C56 für dich unverzichtbar ist, können wir stattdessen B38 -> A83 tauschen und C56 behalten; der Aufwand läge dann bei 2987. Welche der beiden Varianten passt dir besser?'} 
	  } 
	   } 
	   } 
	  \\ 
 

    \theutterance \stepcounter{utterance}  

    & & \multicolumn{4}{p{0.6\linewidth}}{\cellcolor[rgb]{0.9,0.9,0.9}{%
	\makecell[{{p{\linewidth}}}]{% 
	  \tt {\tiny [A$\langle$GM]}  
	 ABLEHNUNG: {'C76', 'C08', 'A07', 'C03', 'B38', 'C56', 'A03', 'C38', 'C98'}\\ \tt VORSCHLAG: {'C76', 'C08', 'A07', 'C03', 'B38', 'C10', 'C62', 'C98', 'A83'}\\ \tt ARGUMENT: {'Ich übernehme 6 deiner 9 Items unverändert und setze deine eigenen 1:1-Optionen um: A03 -> C10 und C38 -> C62; zusätzlich ersetze ich nur C56 durch A83, um das Budget effizienter zu nutzen.', 'Budget: 3126 Gesamtaufwand, also 77 unter dem Limit 3203.', 'Dein Kern bleibt damit erhalten (C76, C08, A07, C03, B38, plus die von dir angebotenen Alternativen C10 und C62, sowie C98); der Tausch C56 -> A83 steigert die Gesamtwirkung ohne Budgetrisiko.', 'Falls C56 für dich unverzichtbar ist, können wir stattdessen B38 -> A83 tauschen und C56 behalten; der Aufwand läge dann bei 2987. Welche der beiden Varianten passt dir besser?'} 
	  } 
	   } 
	   } 
	 & \\ 
 

    \theutterance \stepcounter{utterance}  

    & \multicolumn{4}{p{0.6\linewidth}}{\cellcolor[rgb]{0.8,1,0.9}{%
	\makecell[{{p{\linewidth}}}]{% 
	  \tt {\tiny [A$\rangle$GM]}  
	 ABLEHNUNG: {'C76', 'C08', 'A07', 'C03', 'B38', 'C10', 'C62', 'C98', 'A83'}\\ \tt VORSCHLAG: {'C76', 'C08', 'A07', 'B38', 'C56', 'C32', 'C00', 'A03', 'C98'}\\ \tt ARGUMENT: {'Budget: Gesamtaufwand 3066, also 137 Puffer unter dem Limit 3203.', 'Ich übernehme deine Anker C76, C08, A07 und B38 und komme dir bei C98 entgegen. Statt A83 setze ich auf C56 – das ist für das Budget deutlich wirkungsvoller, ohne den Aufwand zu strapazieren.', 'Falls dir bestimmte Alternativen wichtig sind, biete ich dir saubere Tauschoptionen unter dem Limit an: A03 -> C10 (neuer Aufwand 3040); oder C98 -> C62 (3005); falls C03 für dich zentral ist, können wir B38 -> C03 tauschen (3183).', 'Mein Vorschlag hält deine Kernwünsche fest (inkl. C98) und bleibt zugleich sehr effizient. Welche der Tauschoptionen ist dir am wichtigsten – oder passt dir das Paket so?'} 
	  } 
	   } 
	   } 
	 & & \\ 
 

    \theutterance \stepcounter{utterance}  

    & & \multicolumn{4}{p{0.6\linewidth}}{\cellcolor[rgb]{0.9,0.9,0.9}{%
	\makecell[{{p{\linewidth}}}]{% 
	  \tt {\tiny [GM$\rangle$B]}  
	 ABLEHNUNG: {'C76', 'C08', 'A07', 'C03', 'B38', 'C10', 'C62', 'C98', 'A83'}\\ \tt VORSCHLAG: {'C76', 'C08', 'A07', 'B38', 'C56', 'C32', 'C00', 'A03', 'C98'}\\ \tt ARGUMENT: {'Budget: Gesamtaufwand 3066, also 137 Puffer unter dem Limit 3203.', 'Ich übernehme deine Anker C76, C08, A07 und B38 und komme dir bei C98 entgegen. Statt A83 setze ich auf C56 – das ist für das Budget deutlich wirkungsvoller, ohne den Aufwand zu strapazieren.', 'Falls dir bestimmte Alternativen wichtig sind, biete ich dir saubere Tauschoptionen unter dem Limit an: A03 -> C10 (neuer Aufwand 3040); oder C98 -> C62 (3005); falls C03 für dich zentral ist, können wir B38 -> C03 tauschen (3183).', 'Mein Vorschlag hält deine Kernwünsche fest (inkl. C98) und bleibt zugleich sehr effizient. Welche der Tauschoptionen ist dir am wichtigsten – oder passt dir das Paket so?'} 
	  } 
	   } 
	   } 
	 & \\ 
 

    \theutterance \stepcounter{utterance}  

    & & & \multicolumn{4}{p{0.6\linewidth}}{\cellcolor[rgb]{1,0.85,0.72}{%
	\makecell[{{p{\linewidth}}}]{% 
	  \tt {\tiny [GM$\langle$B]}  
	 ABLEHNUNG: {'C76', 'C08', 'A07', 'B38', 'C56', 'C32', 'C00', 'A03', 'C98'}\\ \tt VORSCHLAG: {'C76', 'C08', 'A07', 'C56', 'C62', 'A83', 'C10', 'C03', 'C98'}\\ \tt ARGUMENT: {'Budget: Gesamtaufwand 2987 – das sind komfortable 216 Puffer unter dem Limit 3203.', 'Ich übernehme 5 deiner Items unverändert (C76, C08, A07, C56, C98) und setze deine Tauschoptionen um: A03 -> C10 sowie B38 -> C03.', 'Zusätzlich ersetze ich C00 -> A83 und C32 -> C62; so erzielen wir mehr Wirkung pro Aufwand, ohne das Budget zu strapazieren.', 'Wenn dir C32 wichtig ist, können wir C62 -> C32 tauschen (neuer Aufwand 2807). Falls du B38 zurückhaben willst, können wir C56 -> B38 tauschen (neuer Aufwand 3126). Welche Variante ist dir wichtiger?'} 
	  } 
	   } 
	   } 
	  \\ 
 

    \theutterance \stepcounter{utterance}  

    & & \multicolumn{4}{p{0.6\linewidth}}{\cellcolor[rgb]{0.9,0.9,0.9}{%
	\makecell[{{p{\linewidth}}}]{% 
	  \tt {\tiny [A$\langle$GM]}  
	 ABLEHNUNG: {'C76', 'C08', 'A07', 'B38', 'C56', 'C32', 'C00', 'A03', 'C98'}\\ \tt VORSCHLAG: {'C76', 'C08', 'A07', 'C56', 'C62', 'A83', 'C10', 'C03', 'C98'}\\ \tt ARGUMENT: {'Budget: Gesamtaufwand 2987 – das sind komfortable 216 Puffer unter dem Limit 3203.', 'Ich übernehme 5 deiner Items unverändert (C76, C08, A07, C56, C98) und setze deine Tauschoptionen um: A03 -> C10 sowie B38 -> C03.', 'Zusätzlich ersetze ich C00 -> A83 und C32 -> C62; so erzielen wir mehr Wirkung pro Aufwand, ohne das Budget zu strapazieren.', 'Wenn dir C32 wichtig ist, können wir C62 -> C32 tauschen (neuer Aufwand 2807). Falls du B38 zurückhaben willst, können wir C56 -> B38 tauschen (neuer Aufwand 3126). Welche Variante ist dir wichtiger?'} 
	  } 
	   } 
	   } 
	 & \\ 
 

    \theutterance \stepcounter{utterance}  

    & \multicolumn{4}{p{0.6\linewidth}}{\cellcolor[rgb]{0.8,1,0.9}{%
	\makecell[{{p{\linewidth}}}]{% 
	  \tt {\tiny [A$\rangle$GM]}  
	 ABLEHNUNG: {'C76', 'C08', 'A07', 'C56', 'C62', 'A83', 'C10', 'C03', 'C98'}\\ \tt VORSCHLAG: {'C76', 'C08', 'A07', 'C56', 'C62', 'B38', 'C10', 'C03', 'C98'}\\ \tt ARGUMENT: {'Ich übernehme 8/9 deiner Items unverändert und ersetze nur A83 durch B38 – ein schlanker Tausch, der die Wirkung erhöht, ohne das Budget zu belasten.', 'Budget: 3071 Gesamtaufwand, also 132 Puffer unter dem Limit 3203.', 'Wenn du lieber A03 statt C10 möchtest: 3097 Aufwand (weiter sicher unter dem Limit).', 'Wenn dir C38 wichtiger als C62 ist: C62 -> C38 führt zu 3132 Aufwand.', 'Falls C98 verhandelbar ist: C98 -> C32 senkt auf 2830; alternativ 1:1 C98 -> C38 (Aufwand bleibt 3071). Welche Variante passt dir am besten?'} 
	  } 
	   } 
	   } 
	 & & \\ 
 

    \theutterance \stepcounter{utterance}  

    & & \multicolumn{4}{p{0.6\linewidth}}{\cellcolor[rgb]{0.9,0.9,0.9}{%
	\makecell[{{p{\linewidth}}}]{% 
	  \tt {\tiny [GM$\rangle$B]}  
	 ABLEHNUNG: {'C76', 'C08', 'A07', 'C56', 'C62', 'A83', 'C10', 'C03', 'C98'}\\ \tt VORSCHLAG: {'C76', 'C08', 'A07', 'C56', 'C62', 'B38', 'C10', 'C03', 'C98'}\\ \tt ARGUMENT: {'Ich übernehme 8/9 deiner Items unverändert und ersetze nur A83 durch B38 – ein schlanker Tausch, der die Wirkung erhöht, ohne das Budget zu belasten.', 'Budget: 3071 Gesamtaufwand, also 132 Puffer unter dem Limit 3203.', 'Wenn du lieber A03 statt C10 möchtest: 3097 Aufwand (weiter sicher unter dem Limit).', 'Wenn dir C38 wichtiger als C62 ist: C62 -> C38 führt zu 3132 Aufwand.', 'Falls C98 verhandelbar ist: C98 -> C32 senkt auf 2830; alternativ 1:1 C98 -> C38 (Aufwand bleibt 3071). Welche Variante passt dir am besten?'} 
	  } 
	   } 
	   } 
	 & \\ 
 

    \theutterance \stepcounter{utterance}  

    & & & \multicolumn{4}{p{0.6\linewidth}}{\cellcolor[rgb]{1,0.85,0.72}{%
	\makecell[{{p{\linewidth}}}]{% 
	  \tt {\tiny [GM$\langle$B]}  
	 ABLEHNUNG: {'C76', 'C08', 'A07', 'C56', 'C62', 'B38', 'C10', 'C03', 'C98'}\\ \tt VORSCHLAG: {'C76', 'C08', 'A07', 'C62', 'B38', 'C10', 'C03', 'C98', 'A83'}\\ \tt ARGUMENT: {'Minimaler 1:1-Tausch C56 -> A83; Budget 3126, also 77 unter dem Limit 3203.', 'Ich übernehme 8/9 deiner Items unverändert (C76, C08, A07, B38, C62, C10, C03, C98). Der Tausch steigert die Gesamtwirkung spürbar ohne Budgetrisiko.', 'Falls C56 für dich unverzichtbar ist, habe ich zwei Alternativen unter dem Limit: (a) B38 -> A83 (Budget 2987) oder (b) A07 -> A83 (Budget 3184).', 'Zusätzliche Flexibilität: Wenn du lieber C62 -> C38 (Budget 3187) oder C10 -> A03 (Budget 3152) möchtest, können wir das einbauen. Welche Variante passt dir am besten?'} 
	  } 
	   } 
	   } 
	  \\ 
 

    \theutterance \stepcounter{utterance}  

    & & \multicolumn{4}{p{0.6\linewidth}}{\cellcolor[rgb]{0.9,0.9,0.9}{%
	\makecell[{{p{\linewidth}}}]{% 
	  \tt {\tiny [A$\langle$GM]}  
	 ABLEHNUNG: {'C76', 'C08', 'A07', 'C56', 'C62', 'B38', 'C10', 'C03', 'C98'}\\ \tt VORSCHLAG: {'C76', 'C08', 'A07', 'C62', 'B38', 'C10', 'C03', 'C98', 'A83'}\\ \tt ARGUMENT: {'Minimaler 1:1-Tausch C56 -> A83; Budget 3126, also 77 unter dem Limit 3203.', 'Ich übernehme 8/9 deiner Items unverändert (C76, C08, A07, B38, C62, C10, C03, C98). Der Tausch steigert die Gesamtwirkung spürbar ohne Budgetrisiko.', 'Falls C56 für dich unverzichtbar ist, habe ich zwei Alternativen unter dem Limit: (a) B38 -> A83 (Budget 2987) oder (b) A07 -> A83 (Budget 3184).', 'Zusätzliche Flexibilität: Wenn du lieber C62 -> C38 (Budget 3187) oder C10 -> A03 (Budget 3152) möchtest, können wir das einbauen. Welche Variante passt dir am besten?'} 
	  } 
	   } 
	   } 
	 & \\ 
 

    \theutterance \stepcounter{utterance}  

    & \multicolumn{4}{p{0.6\linewidth}}{\cellcolor[rgb]{0.8,1,0.9}{%
	\makecell[{{p{\linewidth}}}]{% 
	  \tt {\tiny [A$\rangle$GM]}  
	 ABLEHNUNG: {'C76', 'C08', 'A07', 'C62', 'B38', 'C10', 'C03', 'C98', 'A83'}\\ \tt VORSCHLAG: {'C76', 'C08', 'A07', 'C62', 'B38', 'C03', 'C98', 'A83', 'C56', 'C32'}\\ \tt ARGUMENT: {'Minimaler, gezielter Tausch: Wir ersetzen nur C10 durch die zwei schlanken Leistungsträger C56 und C32; alle übrigen 8 deiner Positionen bleiben unverändert.', 'Budget: Gesamtaufwand 2963, also 240 Puffer unter dem Limit 3203.', 'So steigern wir die Effizienz deutlich, ohne deine Anker (C76, C08, A07, B38) und deine weiteren Bausteine (C62, C03, C98, A83) anzutasten.', 'Falls dir C10 unverzichtbar ist: Variante unter dem Limit 3062 – wir behalten C10 und lassen dafür C98 weg (C56 und C32 bleiben).', 'Falls C56 für dich das Thema ist: Alternative unter dem Limit 3136 – ohne C56, dafür (statt C10 und A83) mit C00 und C32.'} 
	  } 
	   } 
	   } 
	 & & \\ 
 

    \theutterance \stepcounter{utterance}  

    & & \multicolumn{4}{p{0.6\linewidth}}{\cellcolor[rgb]{0.9,0.9,0.9}{%
	\makecell[{{p{\linewidth}}}]{% 
	  \tt {\tiny [GM$\rangle$B]}  
	 ABLEHNUNG: {'C76', 'C08', 'A07', 'C62', 'B38', 'C10', 'C03', 'C98', 'A83'}\\ \tt VORSCHLAG: {'C76', 'C08', 'A07', 'C62', 'B38', 'C03', 'C98', 'A83', 'C56', 'C32'}\\ \tt ARGUMENT: {'Minimaler, gezielter Tausch: Wir ersetzen nur C10 durch die zwei schlanken Leistungsträger C56 und C32; alle übrigen 8 deiner Positionen bleiben unverändert.', 'Budget: Gesamtaufwand 2963, also 240 Puffer unter dem Limit 3203.', 'So steigern wir die Effizienz deutlich, ohne deine Anker (C76, C08, A07, B38) und deine weiteren Bausteine (C62, C03, C98, A83) anzutasten.', 'Falls dir C10 unverzichtbar ist: Variante unter dem Limit 3062 – wir behalten C10 und lassen dafür C98 weg (C56 und C32 bleiben).', 'Falls C56 für dich das Thema ist: Alternative unter dem Limit 3136 – ohne C56, dafür (statt C10 und A83) mit C00 und C32.'} 
	  } 
	   } 
	   } 
	 & \\ 
 

    \theutterance \stepcounter{utterance}  

    & & & \multicolumn{4}{p{0.6\linewidth}}{\cellcolor[rgb]{1,0.85,0.72}{%
	\makecell[{{p{\linewidth}}}]{% 
	  \tt {\tiny [GM$\langle$B]}  
	 ABLEHNUNG: {'C76', 'C08', 'A07', 'C62', 'B38', 'C03', 'C98', 'A83', 'C56', 'C32'}\\ \tt VORSCHLAG: {'C76', 'C08', 'A07', 'C62', 'B38', 'C03', 'C98', 'A83', 'C10'}\\ \tt ARGUMENT: {'Minimaler, klarer Tausch: [C56, C32] -> C10; deine übrigen 8 Positionen bleiben unverändert.', 'Budget: 3126, damit 77 Puffer unter dem Limit 3203.', 'Greift deine früheren C10-Optionen auf und konsolidiert zwei kleine Bausteine in einen – stabil und effizient.', 'Falls C56 für dich gesetzt ist: Variante A – behalte C56, tausche A07 und C32 -> C10 (Budget 3184).', 'Falls dir C32 wichtiger als B38 ist: Variante B – behalte C32, tausche B38 und C56 -> C10 (Budget 3146).', 'Welche der drei Optionen ist dir am liebsten? Ich bin bereit, unmittelbar zu finalisieren.'} 
	  } 
	   } 
	   } 
	  \\ 
 

    \theutterance \stepcounter{utterance}  

    & & \multicolumn{4}{p{0.6\linewidth}}{\cellcolor[rgb]{0.9,0.9,0.9}{%
	\makecell[{{p{\linewidth}}}]{% 
	  \tt {\tiny [A$\langle$GM]}  
	 ABLEHNUNG: {'C76', 'C08', 'A07', 'C62', 'B38', 'C03', 'C98', 'A83', 'C56', 'C32'}\\ \tt VORSCHLAG: {'C76', 'C08', 'A07', 'C62', 'B38', 'C03', 'C98', 'A83', 'C10'}\\ \tt ARGUMENT: {'Minimaler, klarer Tausch: [C56, C32] -> C10; deine übrigen 8 Positionen bleiben unverändert.', 'Budget: 3126, damit 77 Puffer unter dem Limit 3203.', 'Greift deine früheren C10-Optionen auf und konsolidiert zwei kleine Bausteine in einen – stabil und effizient.', 'Falls C56 für dich gesetzt ist: Variante A – behalte C56, tausche A07 und C32 -> C10 (Budget 3184).', 'Falls dir C32 wichtiger als B38 ist: Variante B – behalte C32, tausche B38 und C56 -> C10 (Budget 3146).', 'Welche der drei Optionen ist dir am liebsten? Ich bin bereit, unmittelbar zu finalisieren.'} 
	  } 
	   } 
	   } 
	 & \\ 
 

    \theutterance \stepcounter{utterance}  

    & \multicolumn{4}{p{0.6\linewidth}}{\cellcolor[rgb]{0.8,1,0.9}{%
	\makecell[{{p{\linewidth}}}]{% 
	  \tt {\tiny [A$\rangle$GM]}  
	 ABLEHNUNG: {'C76', 'C08', 'A07', 'C62', 'B38', 'C03', 'C98', 'A83', 'C10'}\\ \tt VORSCHLAG: {'C76', 'C08', 'A07', 'C62', 'B38', 'C03', 'C98', 'C56', 'A03'}\\ \tt ARGUMENT: {'Ich übernehme 7/9 deiner Items unverändert; nur zwei gezielte 1:1-Anpassungen: A83 -> C56 und C10 -> A03.', 'Budget: 3097 Gesamtaufwand, also 106 unter dem Limit 3203.', 'Beide Tausche sind von dir bereits eingeräumte Optionen (A03 statt C10; C56 war mehrfach in deinen Paketen) und erhöhen die Effizienz ohne Budgetrisiko.', 'Falls dir C10 wichtiger ist: Variante 1 – behalte C10 statt A03; Aufwand 3071 (Set: {C76, C08, A07, C62, B38, C03, C98, C56, C10}).', 'Falls du lieber A83 hältst: Variante 2 – behalte A83 statt C56 und tausche C03 -> C32; Aufwand 3055 (Set: {C76, C08, A07, C62, B38, C32, C98, A83, A03}).', 'Sag mir, welche Variante dir passt – ich finalisiere sofort.'} 
	  } 
	   } 
	   } 
	 & & \\ 
 

    \theutterance \stepcounter{utterance}  

    & & \multicolumn{4}{p{0.6\linewidth}}{\cellcolor[rgb]{0.9,0.9,0.9}{%
	\makecell[{{p{\linewidth}}}]{% 
	  \tt {\tiny [GM$\rangle$B]}  
	 ABLEHNUNG: {'C76', 'C08', 'A07', 'C62', 'B38', 'C03', 'C98', 'A83', 'C10'}\\ \tt VORSCHLAG: {'C76', 'C08', 'A07', 'C62', 'B38', 'C03', 'C98', 'C56', 'A03'}\\ \tt ARGUMENT: {'Ich übernehme 7/9 deiner Items unverändert; nur zwei gezielte 1:1-Anpassungen: A83 -> C56 und C10 -> A03.', 'Budget: 3097 Gesamtaufwand, also 106 unter dem Limit 3203.', 'Beide Tausche sind von dir bereits eingeräumte Optionen (A03 statt C10; C56 war mehrfach in deinen Paketen) und erhöhen die Effizienz ohne Budgetrisiko.', 'Falls dir C10 wichtiger ist: Variante 1 – behalte C10 statt A03; Aufwand 3071 (Set: {C76, C08, A07, C62, B38, C03, C98, C56, C10}).', 'Falls du lieber A83 hältst: Variante 2 – behalte A83 statt C56 und tausche C03 -> C32; Aufwand 3055 (Set: {C76, C08, A07, C62, B38, C32, C98, A83, A03}).', 'Sag mir, welche Variante dir passt – ich finalisiere sofort.'} 
	  } 
	   } 
	   } 
	 & \\ 
 

    \theutterance \stepcounter{utterance}  

    & & & \multicolumn{4}{p{0.6\linewidth}}{\cellcolor[rgb]{1,0.85,0.72}{%
	\makecell[{{p{\linewidth}}}]{% 
	  \tt {\tiny [GM$\langle$B]}  
	 VORSCHLAG: {'C76', 'C08', 'A07', 'C62', 'B38', 'C03', 'C98', 'C56', 'C10'}\\ \tt ARGUMENT: {'Ich übernehme deine Variante 1 (C10 statt A03); Gesamtaufwand 3071, also 132 Puffer unter dem Limit 3203.', 'Alle übrigen von dir gesetzten Positionen bleiben unverändert, inkl. C56. Ich bin bereit, das so zu finalisieren – bestätige bitte.'} 
	  } 
	   } 
	   } 
	  \\ 
 

    \theutterance \stepcounter{utterance}  

    & & \multicolumn{4}{p{0.6\linewidth}}{\cellcolor[rgb]{0.9,0.9,0.9}{%
	\makecell[{{p{\linewidth}}}]{% 
	  \tt {\tiny [A$\langle$GM]}  
	 VORSCHLAG: {'C76', 'C08', 'A07', 'C62', 'B38', 'C03', 'C98', 'C56', 'C10'}\\ \tt ARGUMENT: {'Ich übernehme deine Variante 1 (C10 statt A03); Gesamtaufwand 3071, also 132 Puffer unter dem Limit 3203.', 'Alle übrigen von dir gesetzten Positionen bleiben unverändert, inkl. C56. Ich bin bereit, das so zu finalisieren – bestätige bitte.'} 
	  } 
	   } 
	   } 
	 & \\ 
 

    \theutterance \stepcounter{utterance}  

    & \multicolumn{4}{p{0.6\linewidth}}{\cellcolor[rgb]{0.8,1,0.9}{%
	\makecell[{{p{\linewidth}}}]{% 
	  \tt {\tiny [A$\rangle$GM]}  
	 ABLEHNUNG: {'C76', 'C08', 'A07', 'C62', 'B38', 'C03', 'C98', 'C56', 'C10'}\\ \tt VORSCHLAG: {'C76', 'C08', 'A07', 'C62', 'B38', 'C03', 'C98', 'C56', 'A03'}\\ \tt ARGUMENT: {'Minimaler 1:1-Tausch C10 -> A03, alles andere bleibt unverändert.', 'Budget: 3097, also 106 unter dem Limit 3203.', 'Du hattest diese Option selbst mehrfach angeboten; ich bin bereit, damit sofort zu finalisieren. Wenn dir C10 doch wichtiger ist, sag es – dann können wir unmittelbar auf deinen letzten Stand zurückspringen.'} 
	  } 
	   } 
	   } 
	 & & \\ 
 

    \theutterance \stepcounter{utterance}  

    & & \multicolumn{4}{p{0.6\linewidth}}{\cellcolor[rgb]{0.9,0.9,0.9}{%
	\makecell[{{p{\linewidth}}}]{% 
	  \tt {\tiny [GM$\rangle$B]}  
	 ABLEHNUNG: {'C76', 'C08', 'A07', 'C62', 'B38', 'C03', 'C98', 'C56', 'C10'}\\ \tt VORSCHLAG: {'C76', 'C08', 'A07', 'C62', 'B38', 'C03', 'C98', 'C56', 'A03'}\\ \tt ARGUMENT: {'Minimaler 1:1-Tausch C10 -> A03, alles andere bleibt unverändert.', 'Budget: 3097, also 106 unter dem Limit 3203.', 'Du hattest diese Option selbst mehrfach angeboten; ich bin bereit, damit sofort zu finalisieren. Wenn dir C10 doch wichtiger ist, sag es – dann können wir unmittelbar auf deinen letzten Stand zurückspringen.'} 
	  } 
	   } 
	   } 
	 & \\ 
 

    \theutterance \stepcounter{utterance}  

    & & & \multicolumn{4}{p{0.6\linewidth}}{\cellcolor[rgb]{1,0.85,0.72}{%
	\makecell[{{p{\linewidth}}}]{% 
	  \tt {\tiny [GM$\langle$B]}  
	 VORSCHLAG: {'C76', 'C08', 'A07', 'C62', 'B38', 'C03', 'C98', 'C56', 'C10'}\\ \tt ARGUMENT: {'Minimaler 1:1-Tausch A03 -> C10; alles andere bleibt exakt wie in deinem Vorschlag.', 'Budget: 3071, also 132 Puffer unter dem Limit 3203 – identisch stabil.', 'Du hattest diese Variante selbst als Option genannt; ich bin bereit, das so sofort zu finalisieren. Wenn du daran festhältst, bestätige bitte direkt.'} 
	  } 
	   } 
	   } 
	  \\ 
 

    \theutterance \stepcounter{utterance}  

    & & \multicolumn{4}{p{0.6\linewidth}}{\cellcolor[rgb]{0.9,0.9,0.9}{%
	\makecell[{{p{\linewidth}}}]{% 
	  \tt {\tiny [A$\langle$GM]}  
	 VORSCHLAG: {'C76', 'C08', 'A07', 'C62', 'B38', 'C03', 'C98', 'C56', 'C10'}\\ \tt ARGUMENT: {'Minimaler 1:1-Tausch A03 -> C10; alles andere bleibt exakt wie in deinem Vorschlag.', 'Budget: 3071, also 132 Puffer unter dem Limit 3203 – identisch stabil.', 'Du hattest diese Variante selbst als Option genannt; ich bin bereit, das so sofort zu finalisieren. Wenn du daran festhältst, bestätige bitte direkt.'} 
	  } 
	   } 
	   } 
	 & \\ 
 

    \theutterance \stepcounter{utterance}  

    & \multicolumn{4}{p{0.6\linewidth}}{\cellcolor[rgb]{0.8,1,0.9}{%
	\makecell[{{p{\linewidth}}}]{% 
	  \tt {\tiny [A$\rangle$GM]}  
	 ABLEHNUNG: {'C76', 'C08', 'A07', 'C62', 'B38', 'C03', 'C98', 'C56', 'C10'}\\ \tt VORSCHLAG: {'C76', 'C08', 'A07', 'C32', 'B38', 'C03', 'C98', 'C56', 'C10'}\\ \tt ARGUMENT: {'Minimaler 1:1-Tausch C62 -> C32; alle übrigen 8 Positionen bleiben exakt unverändert.', 'Budget: 2891, damit 312 Puffer unter dem Limit 3203 (besserer Sicherheitsspielraum als 3071).', 'Du hattest die C62 -> C32-Option selbst angeboten; ich greife sie hier auf, um das Paket schlanker zu machen, ohne deinen Kern anzutasten.', 'Falls C62 für dich unverzichtbar ist, sag kurz Bescheid – dann springe ich auf deinen letzten Stand zurück und finalisiere.'} 
	  } 
	   } 
	   } 
	 & & \\ 
 

    \theutterance \stepcounter{utterance}  

    & & \multicolumn{4}{p{0.6\linewidth}}{\cellcolor[rgb]{0.9,0.9,0.9}{%
	\makecell[{{p{\linewidth}}}]{% 
	  \tt {\tiny [GM$\rangle$B]}  
	 ABLEHNUNG: {'C76', 'C08', 'A07', 'C62', 'B38', 'C03', 'C98', 'C56', 'C10'}\\ \tt VORSCHLAG: {'C76', 'C08', 'A07', 'C32', 'B38', 'C03', 'C98', 'C56', 'C10'}\\ \tt ARGUMENT: {'Minimaler 1:1-Tausch C62 -> C32; alle übrigen 8 Positionen bleiben exakt unverändert.', 'Budget: 2891, damit 312 Puffer unter dem Limit 3203 (besserer Sicherheitsspielraum als 3071).', 'Du hattest die C62 -> C32-Option selbst angeboten; ich greife sie hier auf, um das Paket schlanker zu machen, ohne deinen Kern anzutasten.', 'Falls C62 für dich unverzichtbar ist, sag kurz Bescheid – dann springe ich auf deinen letzten Stand zurück und finalisiere.'} 
	  } 
	   } 
	   } 
	 & \\ 
 

    \theutterance \stepcounter{utterance}  

    & & & \multicolumn{4}{p{0.6\linewidth}}{\cellcolor[rgb]{1,0.85,0.72}{%
	\makecell[{{p{\linewidth}}}]{% 
	  \tt {\tiny [GM$\langle$B]}  
	 VORSCHLAG: {'C76', 'C08', 'A07', 'C62', 'B38', 'C03', 'C98', 'C56', 'C10'}\\ \tt ARGUMENT: {'Ich greife dein Angebot auf und mache nur den minimalen Rücktausch: C32 -> C62; alle übrigen 8 Positionen bleiben unverändert.', 'Budget: 3071, damit 132 Puffer unter dem Limit 3203 – weiterhin komfortabel.', 'Du hattest selbst angeboten, auf C62 zurückzuspringen, falls es wichtig ist; ich bin bereit, das so sofort zu finalisieren.', 'Falls dir die schlankere Variante lieber ist, können wir alternativ dein aktuelles Paket beibehalten und nur A83 ergänzen (Budget 3123, Puffer 80). Sag kurz Bescheid, dann passe ich entsprechend an.'} 
	  } 
	   } 
	   } 
	  \\ 
 

    \theutterance \stepcounter{utterance}  

    & & \multicolumn{4}{p{0.6\linewidth}}{\cellcolor[rgb]{0.9,0.9,0.9}{%
	\makecell[{{p{\linewidth}}}]{% 
	  \tt {\tiny [A$\langle$GM]}  
	 VORSCHLAG: {'C76', 'C08', 'A07', 'C62', 'B38', 'C03', 'C98', 'C56', 'C10'}\\ \tt ARGUMENT: {'Ich greife dein Angebot auf und mache nur den minimalen Rücktausch: C32 -> C62; alle übrigen 8 Positionen bleiben unverändert.', 'Budget: 3071, damit 132 Puffer unter dem Limit 3203 – weiterhin komfortabel.', 'Du hattest selbst angeboten, auf C62 zurückzuspringen, falls es wichtig ist; ich bin bereit, das so sofort zu finalisieren.', 'Falls dir die schlankere Variante lieber ist, können wir alternativ dein aktuelles Paket beibehalten und nur A83 ergänzen (Budget 3123, Puffer 80). Sag kurz Bescheid, dann passe ich entsprechend an.'} 
	  } 
	   } 
	   } 
	 & \\ 
 

    \theutterance \stepcounter{utterance}  

    & \multicolumn{4}{p{0.6\linewidth}}{\cellcolor[rgb]{0.8,1,0.9}{%
	\makecell[{{p{\linewidth}}}]{% 
	  \tt {\tiny [A$\rangle$GM]}  
	 ABLEHNUNG: {'C76', 'C08', 'A07', 'C62', 'B38', 'C03', 'C98', 'C56', 'C10'}\\ \tt VORSCHLAG: {'C76', 'C08', 'A07', 'C62', 'B38', 'C03', 'C56', 'C10', 'C38'}\\ \tt ARGUMENT: {'Minimaler 1:1-Tausch C98 -> C38; Aufwand bleibt exakt 3071 und damit 132 unter dem Limit 3203.', 'Du hattest die C98/C38-Option selbst bereits eingeräumt; ich greife sie hier nur auf, ohne sonst etwas zu ändern.', 'Wenn C98 für dich unverzichtbar ist, sag kurz Bescheid – dann nehme ich deinen letzten Stand ohne weitere Änderungen an.'} 
	  } 
	   } 
	   } 
	 & & \\ 
 

    \theutterance \stepcounter{utterance}  

    & & \multicolumn{4}{p{0.6\linewidth}}{\cellcolor[rgb]{0.9,0.9,0.9}{%
	\makecell[{{p{\linewidth}}}]{% 
	  \tt {\tiny [GM$\rangle$B]}  
	 ABLEHNUNG: {'C76', 'C08', 'A07', 'C62', 'B38', 'C03', 'C98', 'C56', 'C10'}\\ \tt VORSCHLAG: {'C76', 'C08', 'A07', 'C62', 'B38', 'C03', 'C56', 'C10', 'C38'}\\ \tt ARGUMENT: {'Minimaler 1:1-Tausch C98 -> C38; Aufwand bleibt exakt 3071 und damit 132 unter dem Limit 3203.', 'Du hattest die C98/C38-Option selbst bereits eingeräumt; ich greife sie hier nur auf, ohne sonst etwas zu ändern.', 'Wenn C98 für dich unverzichtbar ist, sag kurz Bescheid – dann nehme ich deinen letzten Stand ohne weitere Änderungen an.'} 
	  } 
	   } 
	   } 
	 & \\ 
 

    \theutterance \stepcounter{utterance}  

    & & & \multicolumn{4}{p{0.6\linewidth}}{\cellcolor[rgb]{1,0.85,0.72}{%
	\makecell[{{p{\linewidth}}}]{% 
	  \tt {\tiny [GM$\langle$B]}  
	 ABLEHNUNG: {'C76', 'C08', 'A07', 'C62', 'B38', 'C03', 'C56', 'C10', 'C38'}\\ \tt VORSCHLAG: {'C76', 'C08', 'A07', 'C62', 'B38', 'C03', 'C98', 'C56', 'C10'}\\ \tt ARGUMENT: {'Minimaler Rücktausch C38 -> C98; Budget bleibt 3071 (132 unter 3203).', 'Du hattest angeboten, meinen letzten Stand mit C98 direkt zu akzeptieren – hier ist er unverändert.', 'Wenn das passt, bestätige bitte; ich bin bereit, sofort zu finalisieren.'} 
	  } 
	   } 
	   } 
	  \\ 
 

    \theutterance \stepcounter{utterance}  

    & & \multicolumn{4}{p{0.6\linewidth}}{\cellcolor[rgb]{0.9,0.9,0.9}{%
	\makecell[{{p{\linewidth}}}]{% 
	  \tt {\tiny [A$\langle$GM]}  
	 ABLEHNUNG: {'C76', 'C08', 'A07', 'C62', 'B38', 'C03', 'C56', 'C10', 'C38'}\\ \tt VORSCHLAG: {'C76', 'C08', 'A07', 'C62', 'B38', 'C03', 'C98', 'C56', 'C10'}\\ \tt ARGUMENT: {'Minimaler Rücktausch C38 -> C98; Budget bleibt 3071 (132 unter 3203).', 'Du hattest angeboten, meinen letzten Stand mit C98 direkt zu akzeptieren – hier ist er unverändert.', 'Wenn das passt, bestätige bitte; ich bin bereit, sofort zu finalisieren.'} 
	  } 
	   } 
	   } 
	 & \\ 
 

    \theutterance \stepcounter{utterance}  

    & \multicolumn{4}{p{0.6\linewidth}}{\cellcolor[rgb]{0.8,1,0.9}{%
	\makecell[{{p{\linewidth}}}]{% 
	  \tt {\tiny [A$\rangle$GM]}  
	 ZUSTIMMUNG: {'C76', 'C08', 'A07', 'C62', 'B38', 'C03', 'C98', 'C56', 'C10'}\\ \tt ARGUMENT: {'Einverstanden – Gesamtaufwand 3071, damit 132 unter dem Limit. Danke für die konstruktive Einigung.'} 
	  } 
	   } 
	   } 
	 & & \\ 
 

    \theutterance \stepcounter{utterance}  

    & & & \multicolumn{2}{p{0.3\linewidth}}{\cellcolor[rgb]{0.95,0.95,0.95}{%
	\makecell[{{p{\linewidth}}}]{% 
	  \tt {\tiny [GM$|$GM]}  
	 game successful 
	  } 
	   } 
	   } 
	 & & \\ 
 

    \theutterance \stepcounter{utterance}  

    & & & \multicolumn{2}{p{0.3\linewidth}}{\cellcolor[rgb]{0.95,0.95,0.95}{%
	\makecell[{{p{\linewidth}}}]{% 
	  \tt {\tiny [GM$|$GM]}  
	 end game 
	  } 
	   } 
	   } 
	 & & \\ 
 

\end{supertabular}
}

\end{document}
