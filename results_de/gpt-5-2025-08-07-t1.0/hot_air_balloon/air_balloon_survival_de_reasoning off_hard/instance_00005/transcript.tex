
\documentclass{article}
\usepackage{colortbl}
\usepackage{makecell}
\usepackage{multirow}
\usepackage{supertabular}

\begin{document}

\newcounter{utterance}

\twocolumn

{ \footnotesize  \setcounter{utterance}{1}
\setlength{\tabcolsep}{0pt}
\begin{supertabular}{c@{$\;$}|p{.15\linewidth}@{}p{.15\linewidth}p{.15\linewidth}p{.15\linewidth}p{.15\linewidth}p{.15\linewidth}}

    \# & $\;$A & \multicolumn{4}{c}{Game Master} & $\;\:$B\\
    \hline 

    \theutterance \stepcounter{utterance}  

    & & \multicolumn{4}{p{0.6\linewidth}}{\cellcolor[rgb]{0.9,0.9,0.9}{%
	\makecell[{{p{\linewidth}}}]{% 
	  \tt {\tiny [A$\langle$GM]}  
	 Du nimmst an einem kollaborativen Verhandlungspiel Teil.\\ \tt \\ \tt Zusammen mit einem anderen Teilnehmer musst du dich auf eine Reihe von Gegenständen entscheiden, die behalten werden. Jeder von euch hat eine persönliche Verteilung über die Wichtigkeit der einzelnen Gegenstände. Jeder von euch hat eine eigene Meinung darüber, wie wichtig jeder einzelne Gegenstand ist (Gegenstandswichtigkeit). Du kennst die Wichtigkeitsverteilung des anderen Spielers nicht. Zusätzlich siehst du, wie viel Aufwand jeder Gegenstand verursacht.  \\ \tt Ihr dürft euch nur auf eine Reihe von Gegenständen einigen, wenn der Gesamtaufwand der ausgewählten Gegenstände den Maximalaufwand nicht überschreitet:\\ \tt \\ \tt Maximalaufwand = 2307\\ \tt \\ \tt Hier sind die einzelnen Aufwände der Gegenstände:\\ \tt \\ \tt Aufwand der Gegenstände = {"C76": 147, "C38": 698, "C56": 588, "A03": 43, "A07": 356, "A83": 80, "C98": 95, "C08": 743, "C62": 106, "C00": 308, "C10": 325, "C81": 255, "B38": 276, "C03": 543, "C32": 51}\\ \tt \\ \tt Hier ist deine persönliche Verteilung der Wichtigkeit der einzelnen Gegenstände:\\ \tt \\ \tt Werte der Gegenstandswichtigkeit = {"C76": 138, "C38": 583, "C56": 65, "A03": 262, "A07": 121, "A83": 508, "C98": 461, "C08": 484, "C62": 668, "C00": 389, "C10": 215, "C81": 97, "B38": 500, "C03": 30, "C32": 400}\\ \tt \\ \tt Ziel:\\ \tt \\ \tt Dein Ziel ist es, eine Reihe von Gegenständen auszuhandeln, die dir möglichst viel bringt (d. h. Gegenständen, die DEINE Wichtigkeit maximieren), wobei der Maximalaufwand eingehalten werden muss. Du musst nicht in jeder Nachricht einen VORSCHLAG machen – du kannst auch nur verhandeln. Alle Taktiken sind erlaubt!\\ \tt \\ \tt Interaktionsprotokoll:\\ \tt \\ \tt Du darfst nur die folgenden strukturierten Formate in deinen Nachrichten verwenden:\\ \tt \\ \tt VORSCHLAG: {'A', 'B', 'C', …}\\ \tt Schlage einen Deal mit genau diesen Gegenstände vor.\\ \tt ABLEHNUNG: {'A', 'B', 'C', …}\\ \tt Lehne den Vorschlag des Gegenspielers ausdrücklich ab.\\ \tt ARGUMENT: {'...'}\\ \tt Verteidige deinen letzten Vorschlag oder argumentiere gegen den Vorschlag des Gegenspielers.\\ \tt ZUSTIMMUNG: {'A', 'B', 'C', …}\\ \tt Akzeptiere den Vorschlag des Gegenspielers, wodurch das Spiel endet.\\ \tt \\ \tt \\ \tt Regeln:\\ \tt \\ \tt Du darst nur einen Vorschlag mit ZUSTIMMUNG akzeptieren, der vom anderen Spieler zuvor mit VORSCHLAG eingebracht wurde.\\ \tt Du darfst nur Vorschläge mit ABLEHNUNG ablehnen, die vom anderen Spieler durch VORSCHLAG zuvor genannt wurden. \\ \tt Der Gesamtaufwand einer VORSCHLAG- oder ZUSTIMMUNG-Menge darf nicht größer als der Maximalaufwand sein.  \\ \tt Offenbare deine versteckte Wichtigkeitsverteilung nicht.\\ \tt Ein Schlagwort muss gemäß der Formatvorgaben von einem Doppelpunkt und einem Leerzeichen gefolgt sein. Das Argument muss eine Python-Menge mit 0 oder mehr Strings sein.  \\ \tt Die Form muss also SCHLAGWORT: {...} sein.\\ \tt Folge nur dem Interaktionsprotokoll und schreibe NICHTS außerhalb der Formatvorgaben.\\ \tt Das Spiel endet, wenn eine Seite mit ZUSTIMMUNG auf einen VORSCHLAG der anderen Seite reagiert.  \\ \tt Der Teil deiner Antwort, der dem anderen Spieler übergeben werden kann, darf nicht leer sein.  \\ \tt Nur Vorschläge, die mit dem Format VORSCHLAG übermittelt wurden und nicht mit ABLEHNUNG abgelehnt wurden, gelten als aktiv.  \\ \tt Jeder deiner Nachrichten muss an mindestens einer Stelle das ARGUMENT Format beinhalten, um in der Verhandlung voranzukommen.\\ \tt  
	  } 
	   } 
	   } 
	 & \\ 
 

    \theutterance \stepcounter{utterance}  

    & \multicolumn{4}{p{0.6\linewidth}}{\cellcolor[rgb]{0.8,1,0.9}{%
	\makecell[{{p{\linewidth}}}]{% 
	  \tt {\tiny [A$\rangle$GM]}  
	 VORSCHLAG: {'C32', 'A83', 'C62', 'C98', 'A03', 'B38', 'C00', 'C38', 'C76', 'C10'}\\ \tt ARGUMENT: {'Dieser Zuschnitt bleibt mit Gesamtaufwand 2129 deutlich unter dem Maximalaufwand 2307 und ist damit risikominimierend.', 'Ich habe bewusst sehr aufwandsintensive und wenig attraktive Posten (z. B. C03, C56, C08) außen vor gelassen.', 'Falls dir ein bestimmter Gegenstand wichtiger ist, können wir über gezielte Ersetzungen sprechen – insbesondere bin ich bei C10 oder C76 flexibel, solange der Gesamtaufwand passt.', 'Mach gern einen konkreten Gegenvorschlag als 1-zu-1-Tausch mit ähnlichem oder geringerem Aufwand, dann prüfen wir das sofort.'} 
	  } 
	   } 
	   } 
	 & & \\ 
 

    \theutterance \stepcounter{utterance}  

    & & \multicolumn{4}{p{0.6\linewidth}}{\cellcolor[rgb]{0.9,0.9,0.9}{%
	\makecell[{{p{\linewidth}}}]{% 
	  \tt {\tiny [GM$\rangle$B]}  
	 Du nimmst an einem kollaborativen Verhandlungspiel Teil.\\ \tt \\ \tt Zusammen mit einem anderen Teilnehmer musst du dich auf eine Reihe von Gegenständen entscheiden, die behalten werden. Jeder von euch hat eine persönliche Verteilung über die Wichtigkeit der einzelnen Gegenstände. Jeder von euch hat eine eigene Meinung darüber, wie wichtig jeder einzelne Gegenstand ist (Gegenstandswichtigkeit). Du kennst die Wichtigkeitsverteilung des anderen Spielers nicht. Zusätzlich siehst du, wie viel Aufwand jeder Gegenstand verursacht.  \\ \tt Ihr dürft euch nur auf eine Reihe von Gegenständen einigen, wenn der Gesamtaufwand der ausgewählten Gegenstände den Maximalaufwand nicht überschreitet:\\ \tt \\ \tt Maximalaufwand = 2307\\ \tt \\ \tt Hier sind die einzelnen Aufwände der Gegenstände:\\ \tt \\ \tt Aufwand der Gegenstände = {"C76": 147, "C38": 698, "C56": 588, "A03": 43, "A07": 356, "A83": 80, "C98": 95, "C08": 743, "C62": 106, "C00": 308, "C10": 325, "C81": 255, "B38": 276, "C03": 543, "C32": 51}\\ \tt \\ \tt Hier ist deine persönliche Verteilung der Wichtigkeit der einzelnen Gegenstände:\\ \tt \\ \tt Werte der Gegenstandswichtigkeit = {"C62": 30, "C38": 65, "A83": 97, "B38": 121, "C08": 138, "C98": 215, "C32": 262, "C00": 389, "A03": 400, "C10": 461, "C76": 484, "A07": 500, "C81": 508, "C56": 583, "C03": 668}\\ \tt \\ \tt Ziel:\\ \tt \\ \tt Dein Ziel ist es, eine Reihe von Gegenständen auszuhandeln, die dir möglichst viel bringt (d. h. Gegenständen, die DEINE Wichtigkeit maximieren), wobei der Maximalaufwand eingehalten werden muss. Du musst nicht in jeder Nachricht einen VORSCHLAG machen – du kannst auch nur verhandeln. Alle Taktiken sind erlaubt!\\ \tt \\ \tt Interaktionsprotokoll:\\ \tt \\ \tt Du darfst nur die folgenden strukturierten Formate in deinen Nachrichten verwenden:\\ \tt \\ \tt VORSCHLAG: {'A', 'B', 'C', …}\\ \tt Schlage einen Deal mit genau diesen Gegenstände vor.\\ \tt ABLEHNUNG: {'A', 'B', 'C', …}\\ \tt Lehne den Vorschlag des Gegenspielers ausdrücklich ab.\\ \tt ARGUMENT: {'...'}\\ \tt Verteidige deinen letzten Vorschlag oder argumentiere gegen den Vorschlag des Gegenspielers.\\ \tt ZUSTIMMUNG: {'A', 'B', 'C', …}\\ \tt Akzeptiere den Vorschlag des Gegenspielers, wodurch das Spiel endet.\\ \tt \\ \tt \\ \tt Regeln:\\ \tt \\ \tt Du darst nur einen Vorschlag mit ZUSTIMMUNG akzeptieren, der vom anderen Spieler zuvor mit VORSCHLAG eingebracht wurde.\\ \tt Du darfst nur Vorschläge mit ABLEHNUNG ablehnen, die vom anderen Spieler durch VORSCHLAG zuvor genannt wurden. \\ \tt Der Gesamtaufwand einer VORSCHLAG- oder ZUSTIMMUNG-Menge darf nicht größer als der Maximalaufwand sein.  \\ \tt Offenbare deine versteckte Wichtigkeitsverteilung nicht.\\ \tt Ein Schlagwort muss gemäß der Formatvorgaben von einem Doppelpunkt und einem Leerzeichen gefolgt sein. Das Argument muss eine Python-Menge mit 0 oder mehr Strings sein.  \\ \tt Die Form muss also SCHLAGWORT: {...} sein.\\ \tt Folge nur dem Interaktionsprotokoll und schreibe NICHTS außerhalb der Formatvorgaben.\\ \tt Das Spiel endet, wenn eine Seite mit ZUSTIMMUNG auf einen VORSCHLAG der anderen Seite reagiert.  \\ \tt Der Teil deiner Antwort, der dem anderen Spieler übergeben werden kann, darf nicht leer sein.  \\ \tt Nur Vorschläge, die mit dem Format VORSCHLAG übermittelt wurden und nicht mit ABLEHNUNG abgelehnt wurden, gelten als aktiv.  \\ \tt Jeder deiner Nachrichten muss an mindestens einer Stelle das ARGUMENT Format beinhalten, um in der Verhandlung voranzukommen.\\ \tt \\ \tt \\ \tt Du erhältst nun die erste Nachricht des anderen Spielers.\\ \tt \\ \tt VORSCHLAG: {'C32', 'A83', 'C62', 'C98', 'A03', 'B38', 'C00', 'C38', 'C76', 'C10'}\\ \tt ARGUMENT: {'Dieser Zuschnitt bleibt mit Gesamtaufwand 2129 deutlich unter dem Maximalaufwand 2307 und ist damit risikominimierend.', 'Ich habe bewusst sehr aufwandsintensive und wenig attraktive Posten (z. B. C03, C56, C08) außen vor gelassen.', 'Falls dir ein bestimmter Gegenstand wichtiger ist, können wir über gezielte Ersetzungen sprechen – insbesondere bin ich bei C10 oder C76 flexibel, solange der Gesamtaufwand passt.', 'Mach gern einen konkreten Gegenvorschlag als 1-zu-1-Tausch mit ähnlichem oder geringerem Aufwand, dann prüfen wir das sofort.'} 
	  } 
	   } 
	   } 
	 & \\ 
 

    \theutterance \stepcounter{utterance}  

    & & & \multicolumn{4}{p{0.6\linewidth}}{\cellcolor[rgb]{1,0.85,0.72}{%
	\makecell[{{p{\linewidth}}}]{% 
	  \tt {\tiny [GM$\langle$B]}  
	 ABLEHNUNG: {'C32', 'A83', 'C62', 'C98', 'A03', 'B38', 'C00', 'C38', 'C76', 'C10'}\\ \tt VORSCHLAG: {'C32', 'A83', 'C62', 'C98', 'A03', 'B38', 'C00', 'C76', 'C10', 'A07', 'C81'}\\ \tt ARGUMENT: {'Ich ersetze den größten Aufwandstreiber C38 (698) durch zwei mittlere Posten A07 (356) und C81 (255); der Gesamtaufwand sinkt damit deutlich auf 2042 und bleibt mit 265 Puffer klar unter dem Maximalaufwand 2307.', 'Der Zuschnitt bleibt risikominimierend und folgt deiner Linie: keine Aufnahme der von dir kritisch genannten Schwergewichte wie C03, C56 oder C08.', 'Alle übrigen Positionen deines Vorschlags bleiben unverändert – minimal-invasive Anpassung bei besserer Verteilung der Aufwände.', 'Falls dir A07 oder C81 weniger zusagen, biete gern einen 1-zu-1-Tausch mit vergleichbarem/geringerem Aufwand an; ich prüfe das sofort und bin bei Bedarf flexibel – bevorzugt ohne C76/C10 anzutasten.'} 
	  } 
	   } 
	   } 
	  \\ 
 

    \theutterance \stepcounter{utterance}  

    & & \multicolumn{4}{p{0.6\linewidth}}{\cellcolor[rgb]{0.9,0.9,0.9}{%
	\makecell[{{p{\linewidth}}}]{% 
	  \tt {\tiny [A$\langle$GM]}  
	 ABLEHNUNG: {'C32', 'A83', 'C62', 'C98', 'A03', 'B38', 'C00', 'C38', 'C76', 'C10'}\\ \tt VORSCHLAG: {'C32', 'A83', 'C62', 'C98', 'A03', 'B38', 'C00', 'C76', 'C10', 'A07', 'C81'}\\ \tt ARGUMENT: {'Ich ersetze den größten Aufwandstreiber C38 (698) durch zwei mittlere Posten A07 (356) und C81 (255); der Gesamtaufwand sinkt damit deutlich auf 2042 und bleibt mit 265 Puffer klar unter dem Maximalaufwand 2307.', 'Der Zuschnitt bleibt risikominimierend und folgt deiner Linie: keine Aufnahme der von dir kritisch genannten Schwergewichte wie C03, C56 oder C08.', 'Alle übrigen Positionen deines Vorschlags bleiben unverändert – minimal-invasive Anpassung bei besserer Verteilung der Aufwände.', 'Falls dir A07 oder C81 weniger zusagen, biete gern einen 1-zu-1-Tausch mit vergleichbarem/geringerem Aufwand an; ich prüfe das sofort und bin bei Bedarf flexibel – bevorzugt ohne C76/C10 anzutasten.'} 
	  } 
	   } 
	   } 
	 & \\ 
 

    \theutterance \stepcounter{utterance}  

    & \multicolumn{4}{p{0.6\linewidth}}{\cellcolor[rgb]{0.8,1,0.9}{%
	\makecell[{{p{\linewidth}}}]{% 
	  \tt {\tiny [A$\rangle$GM]}  
	 ABLEHNUNG: {'C32', 'A83', 'C62', 'C98', 'A03', 'B38', 'C00', 'C76', 'C10', 'A07', 'C81'}\\ \tt VORSCHLAG: {'C32', 'A83', 'C62', 'C98', 'A03', 'B38', 'C00', 'C76', 'A07', 'C38'}\\ \tt ARGUMENT: {'Ich konsolidiere zwei mittlere Posten (C10: 325, C81: 255) in einen starken Einzelposten (C38: 698); der Gesamtaufwand bleibt mit 2160 klar unter 2307 – 147 Puffer.', '9 deiner 11 Positionen bleiben unverändert; A07 bleibt explizit drin und auch C76 wird nicht angetastet, wie von dir bevorzugt.', 'Die Linie bleibt risikominimierend: weiterhin keine Aufnahme von C03, C56 oder C08.', 'Falls C10 für dich unverzichtbar ist, Alternative ohne Vorschlagswechsel: Wir können statt C10 lieber C00 herausnehmen (Kosten dann 2177). Oder – wenn dir A07 entbehrlicher ist – C10 behalten und A07 streichen (Kosten dann 2059). Gib mir deine Präferenz, dann passe ich sofort an.'} 
	  } 
	   } 
	   } 
	 & & \\ 
 

    \theutterance \stepcounter{utterance}  

    & & \multicolumn{4}{p{0.6\linewidth}}{\cellcolor[rgb]{0.9,0.9,0.9}{%
	\makecell[{{p{\linewidth}}}]{% 
	  \tt {\tiny [GM$\rangle$B]}  
	 ABLEHNUNG: {'C32', 'A83', 'C62', 'C98', 'A03', 'B38', 'C00', 'C76', 'C10', 'A07', 'C81'}\\ \tt VORSCHLAG: {'C32', 'A83', 'C62', 'C98', 'A03', 'B38', 'C00', 'C76', 'A07', 'C38'}\\ \tt ARGUMENT: {'Ich konsolidiere zwei mittlere Posten (C10: 325, C81: 255) in einen starken Einzelposten (C38: 698); der Gesamtaufwand bleibt mit 2160 klar unter 2307 – 147 Puffer.', '9 deiner 11 Positionen bleiben unverändert; A07 bleibt explizit drin und auch C76 wird nicht angetastet, wie von dir bevorzugt.', 'Die Linie bleibt risikominimierend: weiterhin keine Aufnahme von C03, C56 oder C08.', 'Falls C10 für dich unverzichtbar ist, Alternative ohne Vorschlagswechsel: Wir können statt C10 lieber C00 herausnehmen (Kosten dann 2177). Oder – wenn dir A07 entbehrlicher ist – C10 behalten und A07 streichen (Kosten dann 2059). Gib mir deine Präferenz, dann passe ich sofort an.'} 
	  } 
	   } 
	   } 
	 & \\ 
 

    \theutterance \stepcounter{utterance}  

    & & & \multicolumn{4}{p{0.6\linewidth}}{\cellcolor[rgb]{1,0.85,0.72}{%
	\makecell[{{p{\linewidth}}}]{% 
	  \tt {\tiny [GM$\langle$B]}  
	 ABLEHNUNG: {'C32', 'A83', 'C62', 'C98', 'A03', 'B38', 'C00', 'C76', 'A07', 'C38'}\\ \tt VORSCHLAG: {'C03', 'A03', 'C32', 'C76', 'C81', 'C10', 'C00', 'A07', 'A83', 'C98'}\\ \tt ARGUMENT: {'Zwei gezielte Pakettauschs ab deinem Vorschlag: (1) C38 (698) -> C10 (325) + C81 (255): -118 Aufwand, feinere Verteilung, weniger Klumpenrisiko. (2) B38 (276) + C62 (106) -> C03 (543): +161 Aufwand. Netto +43; Gesamtaufwand 2203, also 104 Puffer unter 2307.', '7 deiner 10 Positionen bleiben unverändert (C32, A83, C98, A03, C00, C76, A07). Die Linie bleibt risikominimierend: kein C56, kein C08.', 'Der Zuschnitt erhöht die Gesamtqualität der Auswahl ohne den Budgetrahmen auszureizen und vermeidet einen einzelnen sehr teuren Blockposten.', 'Falls C03 für dich nicht machbar ist: Alternative ohne Vorschlagswechsel – wir lassen alles wie in deinem Vorschlag, tauschen nur C38 gegen C10 + C81 (Aufwand dann 2042). Gib kurz OK und ich bestätige entsprechend.'} 
	  } 
	   } 
	   } 
	  \\ 
 

    \theutterance \stepcounter{utterance}  

    & & \multicolumn{4}{p{0.6\linewidth}}{\cellcolor[rgb]{0.9,0.9,0.9}{%
	\makecell[{{p{\linewidth}}}]{% 
	  \tt {\tiny [A$\langle$GM]}  
	 ABLEHNUNG: {'C32', 'A83', 'C62', 'C98', 'A03', 'B38', 'C00', 'C76', 'A07', 'C38'}\\ \tt VORSCHLAG: {'C03', 'A03', 'C32', 'C76', 'C81', 'C10', 'C00', 'A07', 'A83', 'C98'}\\ \tt ARGUMENT: {'Zwei gezielte Pakettauschs ab deinem Vorschlag: (1) C38 (698) -> C10 (325) + C81 (255): -118 Aufwand, feinere Verteilung, weniger Klumpenrisiko. (2) B38 (276) + C62 (106) -> C03 (543): +161 Aufwand. Netto +43; Gesamtaufwand 2203, also 104 Puffer unter 2307.', '7 deiner 10 Positionen bleiben unverändert (C32, A83, C98, A03, C00, C76, A07). Die Linie bleibt risikominimierend: kein C56, kein C08.', 'Der Zuschnitt erhöht die Gesamtqualität der Auswahl ohne den Budgetrahmen auszureizen und vermeidet einen einzelnen sehr teuren Blockposten.', 'Falls C03 für dich nicht machbar ist: Alternative ohne Vorschlagswechsel – wir lassen alles wie in deinem Vorschlag, tauschen nur C38 gegen C10 + C81 (Aufwand dann 2042). Gib kurz OK und ich bestätige entsprechend.'} 
	  } 
	   } 
	   } 
	 & \\ 
 

    \theutterance \stepcounter{utterance}  

    & \multicolumn{4}{p{0.6\linewidth}}{\cellcolor[rgb]{0.8,1,0.9}{%
	\makecell[{{p{\linewidth}}}]{% 
	  \tt {\tiny [A$\rangle$GM]}  
	 ABLEHNUNG: {'C03', 'A03', 'C32', 'C76', 'C81', 'C10', 'C00', 'A07', 'A83', 'C98'}\\ \tt VORSCHLAG: {'C32', 'A83', 'C62', 'C98', 'A03', 'B38', 'C00', 'C38', 'C76', 'C10'}\\ \tt ARGUMENT: {'3-zu-3-Tausch: raus C03(543), A07(356), C81(255); rein C38(698), B38(276), C62(106) – Gesamtaufwand sinkt von 2203 auf 2129 (178 Puffer unter 2307).', '7 deiner 10 Positionen bleiben unverändert (A03, C32, C76, C10, C00, A83, C98) – minimal-invasive Anpassung.', 'Ausgewogene Mischung: viele sehr niedrige Aufwände (43, 51, 80, 95, 106, 147) balancieren den einen 698er-Posten; keine Aufnahme der echten Schwergewichte C56 oder C08.', 'Falls C38 für dich ein No-Go ist: Alternative in gleicher Logik – ersetze C38 durch A07 + C81; Aufwand dann 2042. Gib kurz Bescheid, dann stelle ich das sofort um.', 'Wenn dir ein anderer 1-zu-1-Tausch mit ähnlichem/geringerem Aufwand wichtig ist, sag an; ich prüfe ihn direkt, solange der Gesamtaufwand <= 2307 bleibt.'} 
	  } 
	   } 
	   } 
	 & & \\ 
 

    \theutterance \stepcounter{utterance}  

    & & \multicolumn{4}{p{0.6\linewidth}}{\cellcolor[rgb]{0.9,0.9,0.9}{%
	\makecell[{{p{\linewidth}}}]{% 
	  \tt {\tiny [GM$\rangle$B]}  
	 ABLEHNUNG: {'C03', 'A03', 'C32', 'C76', 'C81', 'C10', 'C00', 'A07', 'A83', 'C98'}\\ \tt VORSCHLAG: {'C32', 'A83', 'C62', 'C98', 'A03', 'B38', 'C00', 'C38', 'C76', 'C10'}\\ \tt ARGUMENT: {'3-zu-3-Tausch: raus C03(543), A07(356), C81(255); rein C38(698), B38(276), C62(106) – Gesamtaufwand sinkt von 2203 auf 2129 (178 Puffer unter 2307).', '7 deiner 10 Positionen bleiben unverändert (A03, C32, C76, C10, C00, A83, C98) – minimal-invasive Anpassung.', 'Ausgewogene Mischung: viele sehr niedrige Aufwände (43, 51, 80, 95, 106, 147) balancieren den einen 698er-Posten; keine Aufnahme der echten Schwergewichte C56 oder C08.', 'Falls C38 für dich ein No-Go ist: Alternative in gleicher Logik – ersetze C38 durch A07 + C81; Aufwand dann 2042. Gib kurz Bescheid, dann stelle ich das sofort um.', 'Wenn dir ein anderer 1-zu-1-Tausch mit ähnlichem/geringerem Aufwand wichtig ist, sag an; ich prüfe ihn direkt, solange der Gesamtaufwand <= 2307 bleibt.'} 
	  } 
	   } 
	   } 
	 & \\ 
 

    \theutterance \stepcounter{utterance}  

    & & & \multicolumn{4}{p{0.6\linewidth}}{\cellcolor[rgb]{1,0.85,0.72}{%
	\makecell[{{p{\linewidth}}}]{% 
	  \tt {\tiny [GM$\langle$B]}  
	 ABLEHNUNG: {'C32', 'A83', 'C62', 'C98', 'A03', 'B38', 'C00', 'C38', 'C76', 'C10'}\\ \tt VORSCHLAG: {'C32', 'A83', 'C62', 'C98', 'A03', 'B38', 'C00', 'C76', 'C10', 'A07', 'C81'}\\ \tt ARGUMENT: {'Wie von dir selbst als Alternative angeboten: Wir ersetzen C38 (698) durch A07 (356) und C81 (255). Gesamtaufwand 2042, also 265 Puffer unter 2307.', '9 deiner 10 Positionen bleiben unverändert; der Zuschnitt reduziert Klumpenrisiko und bleibt bei deiner Linie (kein C03, kein C56, kein C08).', 'Wenn du strikt 10 Positionen bevorzugst: gleiche Logik ohne B38 – {C32, A83, C62, C98, A03, A07, C00, C81, C76, C10} mit Aufwand 1766. Bestätige kurz, welche Variante dir lieber ist.', 'Falls du C38 unbedingt halten willst, brauche ich im Gegenzug A07; streichen wir dafür C62 und A83, landen wir bei 2299. Gib mir das Go, dann stelle ich entsprechend um.'} 
	  } 
	   } 
	   } 
	  \\ 
 

    \theutterance \stepcounter{utterance}  

    & & \multicolumn{4}{p{0.6\linewidth}}{\cellcolor[rgb]{0.9,0.9,0.9}{%
	\makecell[{{p{\linewidth}}}]{% 
	  \tt {\tiny [A$\langle$GM]}  
	 ABLEHNUNG: {'C32', 'A83', 'C62', 'C98', 'A03', 'B38', 'C00', 'C38', 'C76', 'C10'}\\ \tt VORSCHLAG: {'C32', 'A83', 'C62', 'C98', 'A03', 'B38', 'C00', 'C76', 'C10', 'A07', 'C81'}\\ \tt ARGUMENT: {'Wie von dir selbst als Alternative angeboten: Wir ersetzen C38 (698) durch A07 (356) und C81 (255). Gesamtaufwand 2042, also 265 Puffer unter 2307.', '9 deiner 10 Positionen bleiben unverändert; der Zuschnitt reduziert Klumpenrisiko und bleibt bei deiner Linie (kein C03, kein C56, kein C08).', 'Wenn du strikt 10 Positionen bevorzugst: gleiche Logik ohne B38 – {C32, A83, C62, C98, A03, A07, C00, C81, C76, C10} mit Aufwand 1766. Bestätige kurz, welche Variante dir lieber ist.', 'Falls du C38 unbedingt halten willst, brauche ich im Gegenzug A07; streichen wir dafür C62 und A83, landen wir bei 2299. Gib mir das Go, dann stelle ich entsprechend um.'} 
	  } 
	   } 
	   } 
	 & \\ 
 

    \theutterance \stepcounter{utterance}  

    & \multicolumn{4}{p{0.6\linewidth}}{\cellcolor[rgb]{0.8,1,0.9}{%
	\makecell[{{p{\linewidth}}}]{% 
	  \tt {\tiny [A$\rangle$GM]}  
	 ZUSTIMMUNG: {'C32', 'A83', 'C62', 'C98', 'A03', 'B38', 'C00', 'C76', 'C10', 'A07', 'C81'}\\ \tt ARGUMENT: {'Einverstanden – der Gesamtaufwand von 2042 bleibt klar unter 2307; solide, ausgewogene Auswahl mit gutem Puffer.'} 
	  } 
	   } 
	   } 
	 & & \\ 
 

    \theutterance \stepcounter{utterance}  

    & & & \multicolumn{2}{p{0.3\linewidth}}{\cellcolor[rgb]{0.95,0.95,0.95}{%
	\makecell[{{p{\linewidth}}}]{% 
	  \tt {\tiny [GM$|$GM]}  
	 game successful 
	  } 
	   } 
	   } 
	 & & \\ 
 

    \theutterance \stepcounter{utterance}  

    & & & \multicolumn{2}{p{0.3\linewidth}}{\cellcolor[rgb]{0.95,0.95,0.95}{%
	\makecell[{{p{\linewidth}}}]{% 
	  \tt {\tiny [GM$|$GM]}  
	 end game 
	  } 
	   } 
	   } 
	 & & \\ 
 

\end{supertabular}
}

\end{document}
