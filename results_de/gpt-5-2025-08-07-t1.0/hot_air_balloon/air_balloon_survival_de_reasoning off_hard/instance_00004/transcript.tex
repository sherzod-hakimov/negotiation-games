
\documentclass{article}
\usepackage{colortbl}
\usepackage{makecell}
\usepackage{multirow}
\usepackage{supertabular}

\begin{document}

\newcounter{utterance}

\twocolumn

{ \footnotesize  \setcounter{utterance}{1}
\setlength{\tabcolsep}{0pt}
\begin{supertabular}{c@{$\;$}|p{.15\linewidth}@{}p{.15\linewidth}p{.15\linewidth}p{.15\linewidth}p{.15\linewidth}p{.15\linewidth}}

    \# & $\;$A & \multicolumn{4}{c}{Game Master} & $\;\:$B\\
    \hline 

    \theutterance \stepcounter{utterance}  

    & & \multicolumn{4}{p{0.6\linewidth}}{\cellcolor[rgb]{0.9,0.9,0.9}{%
	\makecell[{{p{\linewidth}}}]{% 
	  \tt {\tiny [A$\langle$GM]}  
	 Du nimmst an einem kollaborativen Verhandlungspiel Teil.\\ \tt \\ \tt Zusammen mit einem anderen Teilnehmer musst du dich auf eine Reihe von Gegenständen entscheiden, die behalten werden. Jeder von euch hat eine persönliche Verteilung über die Wichtigkeit der einzelnen Gegenstände. Jeder von euch hat eine eigene Meinung darüber, wie wichtig jeder einzelne Gegenstand ist (Gegenstandswichtigkeit). Du kennst die Wichtigkeitsverteilung des anderen Spielers nicht. Zusätzlich siehst du, wie viel Aufwand jeder Gegenstand verursacht.  \\ \tt Ihr dürft euch nur auf eine Reihe von Gegenständen einigen, wenn der Gesamtaufwand der ausgewählten Gegenstände den Maximalaufwand nicht überschreitet:\\ \tt \\ \tt Maximalaufwand = 2984\\ \tt \\ \tt Hier sind die einzelnen Aufwände der Gegenstände:\\ \tt \\ \tt Aufwand der Gegenstände = {"C76": 350, "C38": 276, "C56": 616, "A03": 737, "A07": 531, "A83": 389, "C98": 24, "C08": 125, "C62": 338, "C00": 356, "C10": 143, "C81": 117, "B38": 257, "C03": 921, "C32": 789}\\ \tt \\ \tt Hier ist deine persönliche Verteilung der Wichtigkeit der einzelnen Gegenstände:\\ \tt \\ \tt Werte der Gegenstandswichtigkeit = {"C76": 138, "C38": 583, "C56": 868, "A03": 822, "A07": 783, "A83": 65, "C98": 262, "C08": 121, "C62": 508, "C00": 780, "C10": 461, "C81": 484, "B38": 668, "C03": 389, "C32": 808}\\ \tt \\ \tt Ziel:\\ \tt \\ \tt Dein Ziel ist es, eine Reihe von Gegenständen auszuhandeln, die dir möglichst viel bringt (d. h. Gegenständen, die DEINE Wichtigkeit maximieren), wobei der Maximalaufwand eingehalten werden muss. Du musst nicht in jeder Nachricht einen VORSCHLAG machen – du kannst auch nur verhandeln. Alle Taktiken sind erlaubt!\\ \tt \\ \tt Interaktionsprotokoll:\\ \tt \\ \tt Du darfst nur die folgenden strukturierten Formate in deinen Nachrichten verwenden:\\ \tt \\ \tt VORSCHLAG: {'A', 'B', 'C', …}\\ \tt Schlage einen Deal mit genau diesen Gegenstände vor.\\ \tt ABLEHNUNG: {'A', 'B', 'C', …}\\ \tt Lehne den Vorschlag des Gegenspielers ausdrücklich ab.\\ \tt ARGUMENT: {'...'}\\ \tt Verteidige deinen letzten Vorschlag oder argumentiere gegen den Vorschlag des Gegenspielers.\\ \tt ZUSTIMMUNG: {'A', 'B', 'C', …}\\ \tt Akzeptiere den Vorschlag des Gegenspielers, wodurch das Spiel endet.\\ \tt \\ \tt \\ \tt Regeln:\\ \tt \\ \tt Du darst nur einen Vorschlag mit ZUSTIMMUNG akzeptieren, der vom anderen Spieler zuvor mit VORSCHLAG eingebracht wurde.\\ \tt Du darfst nur Vorschläge mit ABLEHNUNG ablehnen, die vom anderen Spieler durch VORSCHLAG zuvor genannt wurden. \\ \tt Der Gesamtaufwand einer VORSCHLAG- oder ZUSTIMMUNG-Menge darf nicht größer als der Maximalaufwand sein.  \\ \tt Offenbare deine versteckte Wichtigkeitsverteilung nicht.\\ \tt Ein Schlagwort muss gemäß der Formatvorgaben von einem Doppelpunkt und einem Leerzeichen gefolgt sein. Das Argument muss eine Python-Menge mit 0 oder mehr Strings sein.  \\ \tt Die Form muss also SCHLAGWORT: {...} sein.\\ \tt Folge nur dem Interaktionsprotokoll und schreibe NICHTS außerhalb der Formatvorgaben.\\ \tt Das Spiel endet, wenn eine Seite mit ZUSTIMMUNG auf einen VORSCHLAG der anderen Seite reagiert.  \\ \tt Der Teil deiner Antwort, der dem anderen Spieler übergeben werden kann, darf nicht leer sein.  \\ \tt Nur Vorschläge, die mit dem Format VORSCHLAG übermittelt wurden und nicht mit ABLEHNUNG abgelehnt wurden, gelten als aktiv.  \\ \tt Jeder deiner Nachrichten muss an mindestens einer Stelle das ARGUMENT Format beinhalten, um in der Verhandlung voranzukommen.\\ \tt  
	  } 
	   } 
	   } 
	 & \\ 
 

    \theutterance \stepcounter{utterance}  

    & \multicolumn{4}{p{0.6\linewidth}}{\cellcolor[rgb]{0.8,1,0.9}{%
	\makecell[{{p{\linewidth}}}]{% 
	  \tt {\tiny [A$\rangle$GM]}  
	 VORSCHLAG: {'C98', 'C81', 'C10', 'B38', 'C00', 'C38', 'C62', 'A07', 'C56', 'C08'}\\ \tt ARGUMENT: {'Wir bleiben mit Gesamtaufwand 2783 unter dem Limit 2984 und behalten Puffer fuer Anpassungen', 'Die Auswahl fokussiert auf sehr gutes Nutzen zu Aufwand Verhaeltnis mit besonders effizienten Posten wie C98, C81 und C10 sowie starken Bausteinen wie B38 und C00', 'Schluesselposten mit hoher Wirkung wie C56, A07 und C38 sind enthalten', 'Ich bin offen bis zu rund 200 Aufwand umzuschichten falls dir bestimmte Alternativen wichtiger sind', 'Wenn dir andere Posten zentral sind schlage bitte konkrete Tausche vor zum Beispiel einen mittleren Posten gegen deinen Favoriten'} 
	  } 
	   } 
	   } 
	 & & \\ 
 

    \theutterance \stepcounter{utterance}  

    & & \multicolumn{4}{p{0.6\linewidth}}{\cellcolor[rgb]{0.9,0.9,0.9}{%
	\makecell[{{p{\linewidth}}}]{% 
	  \tt {\tiny [GM$\rangle$B]}  
	 Du nimmst an einem kollaborativen Verhandlungspiel Teil.\\ \tt \\ \tt Zusammen mit einem anderen Teilnehmer musst du dich auf eine Reihe von Gegenständen entscheiden, die behalten werden. Jeder von euch hat eine persönliche Verteilung über die Wichtigkeit der einzelnen Gegenstände. Jeder von euch hat eine eigene Meinung darüber, wie wichtig jeder einzelne Gegenstand ist (Gegenstandswichtigkeit). Du kennst die Wichtigkeitsverteilung des anderen Spielers nicht. Zusätzlich siehst du, wie viel Aufwand jeder Gegenstand verursacht.  \\ \tt Ihr dürft euch nur auf eine Reihe von Gegenständen einigen, wenn der Gesamtaufwand der ausgewählten Gegenstände den Maximalaufwand nicht überschreitet:\\ \tt \\ \tt Maximalaufwand = 2984\\ \tt \\ \tt Hier sind die einzelnen Aufwände der Gegenstände:\\ \tt \\ \tt Aufwand der Gegenstände = {"C76": 350, "C38": 276, "C56": 616, "A03": 737, "A07": 531, "A83": 389, "C98": 24, "C08": 125, "C62": 338, "C00": 356, "C10": 143, "C81": 117, "B38": 257, "C03": 921, "C32": 789}\\ \tt \\ \tt Hier ist deine persönliche Verteilung der Wichtigkeit der einzelnen Gegenstände:\\ \tt \\ \tt Werte der Gegenstandswichtigkeit = {"C56": 65, "A03": 121, "C32": 138, "A07": 262, "C00": 389, "B38": 461, "C38": 484, "C62": 508, "C81": 583, "C10": 668, "C03": 780, "C98": 783, "C76": 808, "C08": 822, "A83": 868}\\ \tt \\ \tt Ziel:\\ \tt \\ \tt Dein Ziel ist es, eine Reihe von Gegenständen auszuhandeln, die dir möglichst viel bringt (d. h. Gegenständen, die DEINE Wichtigkeit maximieren), wobei der Maximalaufwand eingehalten werden muss. Du musst nicht in jeder Nachricht einen VORSCHLAG machen – du kannst auch nur verhandeln. Alle Taktiken sind erlaubt!\\ \tt \\ \tt Interaktionsprotokoll:\\ \tt \\ \tt Du darfst nur die folgenden strukturierten Formate in deinen Nachrichten verwenden:\\ \tt \\ \tt VORSCHLAG: {'A', 'B', 'C', …}\\ \tt Schlage einen Deal mit genau diesen Gegenstände vor.\\ \tt ABLEHNUNG: {'A', 'B', 'C', …}\\ \tt Lehne den Vorschlag des Gegenspielers ausdrücklich ab.\\ \tt ARGUMENT: {'...'}\\ \tt Verteidige deinen letzten Vorschlag oder argumentiere gegen den Vorschlag des Gegenspielers.\\ \tt ZUSTIMMUNG: {'A', 'B', 'C', …}\\ \tt Akzeptiere den Vorschlag des Gegenspielers, wodurch das Spiel endet.\\ \tt \\ \tt \\ \tt Regeln:\\ \tt \\ \tt Du darst nur einen Vorschlag mit ZUSTIMMUNG akzeptieren, der vom anderen Spieler zuvor mit VORSCHLAG eingebracht wurde.\\ \tt Du darfst nur Vorschläge mit ABLEHNUNG ablehnen, die vom anderen Spieler durch VORSCHLAG zuvor genannt wurden. \\ \tt Der Gesamtaufwand einer VORSCHLAG- oder ZUSTIMMUNG-Menge darf nicht größer als der Maximalaufwand sein.  \\ \tt Offenbare deine versteckte Wichtigkeitsverteilung nicht.\\ \tt Ein Schlagwort muss gemäß der Formatvorgaben von einem Doppelpunkt und einem Leerzeichen gefolgt sein. Das Argument muss eine Python-Menge mit 0 oder mehr Strings sein.  \\ \tt Die Form muss also SCHLAGWORT: {...} sein.\\ \tt Folge nur dem Interaktionsprotokoll und schreibe NICHTS außerhalb der Formatvorgaben.\\ \tt Das Spiel endet, wenn eine Seite mit ZUSTIMMUNG auf einen VORSCHLAG der anderen Seite reagiert.  \\ \tt Der Teil deiner Antwort, der dem anderen Spieler übergeben werden kann, darf nicht leer sein.  \\ \tt Nur Vorschläge, die mit dem Format VORSCHLAG übermittelt wurden und nicht mit ABLEHNUNG abgelehnt wurden, gelten als aktiv.  \\ \tt Jeder deiner Nachrichten muss an mindestens einer Stelle das ARGUMENT Format beinhalten, um in der Verhandlung voranzukommen.\\ \tt \\ \tt \\ \tt Du erhältst nun die erste Nachricht des anderen Spielers.\\ \tt \\ \tt VORSCHLAG: {'C98', 'C81', 'C10', 'B38', 'C00', 'C38', 'C62', 'A07', 'C56', 'C08'}\\ \tt ARGUMENT: {'Wir bleiben mit Gesamtaufwand 2783 unter dem Limit 2984 und behalten Puffer fuer Anpassungen', 'Die Auswahl fokussiert auf sehr gutes Nutzen zu Aufwand Verhaeltnis mit besonders effizienten Posten wie C98, C81 und C10 sowie starken Bausteinen wie B38 und C00', 'Schluesselposten mit hoher Wirkung wie C56, A07 und C38 sind enthalten', 'Ich bin offen bis zu rund 200 Aufwand umzuschichten falls dir bestimmte Alternativen wichtiger sind', 'Wenn dir andere Posten zentral sind schlage bitte konkrete Tausche vor zum Beispiel einen mittleren Posten gegen deinen Favoriten'} 
	  } 
	   } 
	   } 
	 & \\ 
 

    \theutterance \stepcounter{utterance}  

    & & & \multicolumn{4}{p{0.6\linewidth}}{\cellcolor[rgb]{1,0.85,0.72}{%
	\makecell[{{p{\linewidth}}}]{% 
	  \tt {\tiny [GM$\langle$B]}  
	 ABLEHNUNG: {'C98', 'C81', 'C10', 'B38', 'C00', 'C38', 'C62', 'A07', 'C56', 'C08'}\\ \tt VORSCHLAG: {'C98', 'C81', 'C10', 'B38', 'C00', 'C38', 'C62', 'A07', 'C08', 'A83', 'C76'}\\ \tt ARGUMENT: {'Ich uebernehme dein effizientes Grundgeruest (C98, C81, C10, B38, C00, C38, C62, C08, A07) unveraendert und ersetze nur C56 durch zwei starke, gut skalierende Posten A83 und C76', 'Gesamtaufwand 2906 bleibt unter dem Limit 2984; wir haben 78 Puffer fuer kurzfristige Anpassungen', 'Der Tausch reduziert das Risiko eines einzelnen schweren Postens und verbessert das Gesamtverhaeltnis von Wirkung zu Aufwand deutlich', 'Die Anpassung liegt in deinem erwaehnten Spielraum (< 200 Aufwand) und wahrt nahezu alle von dir gesetzten Schluesselposten', 'Falls C56 fuer dich unverzichtbar ist, koennen wir alternativ C56 behalten und stattdessen A07 durch A83 ersetzen (Aufwand wuerde auf 2641 sinken); sag Bescheid, dann passe ich den Vorschlag entsprechend an'} 
	  } 
	   } 
	   } 
	  \\ 
 

    \theutterance \stepcounter{utterance}  

    & & \multicolumn{4}{p{0.6\linewidth}}{\cellcolor[rgb]{0.9,0.9,0.9}{%
	\makecell[{{p{\linewidth}}}]{% 
	  \tt {\tiny [A$\langle$GM]}  
	 ABLEHNUNG: {'C98', 'C81', 'C10', 'B38', 'C00', 'C38', 'C62', 'A07', 'C56', 'C08'}\\ \tt VORSCHLAG: {'C98', 'C81', 'C10', 'B38', 'C00', 'C38', 'C62', 'A07', 'C08', 'A83', 'C76'}\\ \tt ARGUMENT: {'Ich uebernehme dein effizientes Grundgeruest (C98, C81, C10, B38, C00, C38, C62, C08, A07) unveraendert und ersetze nur C56 durch zwei starke, gut skalierende Posten A83 und C76', 'Gesamtaufwand 2906 bleibt unter dem Limit 2984; wir haben 78 Puffer fuer kurzfristige Anpassungen', 'Der Tausch reduziert das Risiko eines einzelnen schweren Postens und verbessert das Gesamtverhaeltnis von Wirkung zu Aufwand deutlich', 'Die Anpassung liegt in deinem erwaehnten Spielraum (< 200 Aufwand) und wahrt nahezu alle von dir gesetzten Schluesselposten', 'Falls C56 fuer dich unverzichtbar ist, koennen wir alternativ C56 behalten und stattdessen A07 durch A83 ersetzen (Aufwand wuerde auf 2641 sinken); sag Bescheid, dann passe ich den Vorschlag entsprechend an'} 
	  } 
	   } 
	   } 
	 & \\ 
 

    \theutterance \stepcounter{utterance}  

    & \multicolumn{4}{p{0.6\linewidth}}{\cellcolor[rgb]{0.8,1,0.9}{%
	\makecell[{{p{\linewidth}}}]{% 
	  \tt {\tiny [A$\rangle$GM]}  
	 ABLEHNUNG: {'C98', 'C81', 'C10', 'B38', 'C00', 'C38', 'C62', 'A07', 'C08', 'A83', 'C76'}\\ \tt VORSCHLAG: {'C98', 'C81', 'C10', 'B38', 'C00', 'C38', 'A07', 'C56', 'C08', 'C76'}\\ \tt ARGUMENT: {'Ich lehne den Ersatz von C56 durch A83+C76 ab; C56 liefert in unserem Set ein sehr solides Wirkung-zu-Aufwand-Profil', 'Kompromiss: Wir nehmen C76 wie von dir gewuenscht auf und geben dafuer C62 ab; Gesamtaufwand 2795 < 2984, mit reichlich Puffer', 'Dein effizientes Grundgeruest bleibt erhalten (C98, C81, C10, B38, C00, C38, C08, A07) und die Last ist gut diversifiziert (kein Einzelposten > ~21% des Budgets)', 'Wenn dir A83 wichtiger als C76 ist, kann ich alternativ C62 gegen A83 tauschen (Aufwand ca. 2834); sag mir, welcher der beiden dir wichtiger ist', 'Nenne mir gern 1–2 absolute Favoriten von dir, dann finalisiere ich den Korb ohne das Limit zu verletzen'} 
	  } 
	   } 
	   } 
	 & & \\ 
 

    \theutterance \stepcounter{utterance}  

    & & \multicolumn{4}{p{0.6\linewidth}}{\cellcolor[rgb]{0.9,0.9,0.9}{%
	\makecell[{{p{\linewidth}}}]{% 
	  \tt {\tiny [GM$\rangle$B]}  
	 ABLEHNUNG: {'C98', 'C81', 'C10', 'B38', 'C00', 'C38', 'C62', 'A07', 'C08', 'A83', 'C76'}\\ \tt VORSCHLAG: {'C98', 'C81', 'C10', 'B38', 'C00', 'C38', 'A07', 'C56', 'C08', 'C76'}\\ \tt ARGUMENT: {'Ich lehne den Ersatz von C56 durch A83+C76 ab; C56 liefert in unserem Set ein sehr solides Wirkung-zu-Aufwand-Profil', 'Kompromiss: Wir nehmen C76 wie von dir gewuenscht auf und geben dafuer C62 ab; Gesamtaufwand 2795 < 2984, mit reichlich Puffer', 'Dein effizientes Grundgeruest bleibt erhalten (C98, C81, C10, B38, C00, C38, C08, A07) und die Last ist gut diversifiziert (kein Einzelposten > ~21% des Budgets)', 'Wenn dir A83 wichtiger als C76 ist, kann ich alternativ C62 gegen A83 tauschen (Aufwand ca. 2834); sag mir, welcher der beiden dir wichtiger ist', 'Nenne mir gern 1–2 absolute Favoriten von dir, dann finalisiere ich den Korb ohne das Limit zu verletzen'} 
	  } 
	   } 
	   } 
	 & \\ 
 

    \theutterance \stepcounter{utterance}  

    & & & \multicolumn{4}{p{0.6\linewidth}}{\cellcolor[rgb]{1,0.85,0.72}{%
	\makecell[{{p{\linewidth}}}]{% 
	  \tt {\tiny [GM$\langle$B]}  
	 ABLEHNUNG: {'C98', 'C81', 'C10', 'B38', 'C00', 'C38', 'A07', 'C56', 'C08', 'C76'}\\ \tt VORSCHLAG: {'C98', 'C81', 'C10', 'B38', 'C00', 'C38', 'A07', 'C08', 'C76', 'A83', 'C62'}\\ \tt ARGUMENT: {'Ich uebernehme deinen aktuellen Korb nahezu unveraendert, streiche nur C56 und ersetze ihn durch A83 und nehme C62 wieder dazu – das staerkt die Effizienz mit zwei mittleren Posten statt eines schweren Einzelblocks', 'Gesamtaufwand 2906 bleibt klar unter dem Limit 2984; wir behalten 78 Puffer und bleiben in deinem angepeilten Anpassungskorridor (Delta zum jetzigen Set: +111 Aufwand)', 'Deine betonten Schluesselposten A07 und C38 bleiben voll erhalten; C76 ist ebenfalls drin wie von dir vorgeschlagen', 'Das Risiko ist besser verteilt (keine Abhaengigkeit von einem einzelnen teuren Posten) bei sehr gutem Nutzen-zu-Aufwand-Profil durch die schlanken Elemente', 'Falls C56 fuer dich absolut gesetzt ist, kann ich als Minimalvariante auch nur C56 -> A83 tauschen und alles andere unveraendert lassen (Aufwand wuerde auf 2568 fallen); gib mir dann kurz Gruenlicht'} 
	  } 
	   } 
	   } 
	  \\ 
 

    \theutterance \stepcounter{utterance}  

    & & \multicolumn{4}{p{0.6\linewidth}}{\cellcolor[rgb]{0.9,0.9,0.9}{%
	\makecell[{{p{\linewidth}}}]{% 
	  \tt {\tiny [A$\langle$GM]}  
	 ABLEHNUNG: {'C98', 'C81', 'C10', 'B38', 'C00', 'C38', 'A07', 'C56', 'C08', 'C76'}\\ \tt VORSCHLAG: {'C98', 'C81', 'C10', 'B38', 'C00', 'C38', 'A07', 'C08', 'C76', 'A83', 'C62'}\\ \tt ARGUMENT: {'Ich uebernehme deinen aktuellen Korb nahezu unveraendert, streiche nur C56 und ersetze ihn durch A83 und nehme C62 wieder dazu – das staerkt die Effizienz mit zwei mittleren Posten statt eines schweren Einzelblocks', 'Gesamtaufwand 2906 bleibt klar unter dem Limit 2984; wir behalten 78 Puffer und bleiben in deinem angepeilten Anpassungskorridor (Delta zum jetzigen Set: +111 Aufwand)', 'Deine betonten Schluesselposten A07 und C38 bleiben voll erhalten; C76 ist ebenfalls drin wie von dir vorgeschlagen', 'Das Risiko ist besser verteilt (keine Abhaengigkeit von einem einzelnen teuren Posten) bei sehr gutem Nutzen-zu-Aufwand-Profil durch die schlanken Elemente', 'Falls C56 fuer dich absolut gesetzt ist, kann ich als Minimalvariante auch nur C56 -> A83 tauschen und alles andere unveraendert lassen (Aufwand wuerde auf 2568 fallen); gib mir dann kurz Gruenlicht'} 
	  } 
	   } 
	   } 
	 & \\ 
 

    \theutterance \stepcounter{utterance}  

    & \multicolumn{4}{p{0.6\linewidth}}{\cellcolor[rgb]{0.8,1,0.9}{%
	\makecell[{{p{\linewidth}}}]{% 
	  \tt {\tiny [A$\rangle$GM]}  
	 ABLEHNUNG: {'C98', 'C81', 'C10', 'B38', 'C00', 'C38', 'A07', 'C08', 'C76', 'A83', 'C62'}\\ \tt VORSCHLAG: {'C98', 'C81', 'C10', 'B38', 'C00', 'C38', 'A07', 'C56', 'C08', 'A83'}\\ \tt ARGUMENT: {'Ich uebernehme dein Grundgeruest und lasse A83 drin, fuehre aber C56 als starken Hebel wieder ein und gebe dafuer nur C62 ab', 'Gesamtaufwand 2834 < 2984; damit bleiben 150 Puffer fuer dich', 'So vermeiden wir zwei zusaetzliche Mid-Posten parallel zu C56 und halten das Set fokussiert ohne das Budget zu strecken', 'Falls dir C76 wichtiger als A83 ist, tausche ich A83 -> C76 bei gleicher Struktur (Aufwand 2795) – C56 bleibt in jedem Fall gesetzt', 'Damit bleiben alle hocheffizienten Elemente bestehen (C98, C81, C10, B38, C00, C38, A07), und wir erhalten Robustheit durch C56 ohne Uebergewichtung einzelner Teile'} 
	  } 
	   } 
	   } 
	 & & \\ 
 

    \theutterance \stepcounter{utterance}  

    & & \multicolumn{4}{p{0.6\linewidth}}{\cellcolor[rgb]{0.9,0.9,0.9}{%
	\makecell[{{p{\linewidth}}}]{% 
	  \tt {\tiny [GM$\rangle$B]}  
	 ABLEHNUNG: {'C98', 'C81', 'C10', 'B38', 'C00', 'C38', 'A07', 'C08', 'C76', 'A83', 'C62'}\\ \tt VORSCHLAG: {'C98', 'C81', 'C10', 'B38', 'C00', 'C38', 'A07', 'C56', 'C08', 'A83'}\\ \tt ARGUMENT: {'Ich uebernehme dein Grundgeruest und lasse A83 drin, fuehre aber C56 als starken Hebel wieder ein und gebe dafuer nur C62 ab', 'Gesamtaufwand 2834 < 2984; damit bleiben 150 Puffer fuer dich', 'So vermeiden wir zwei zusaetzliche Mid-Posten parallel zu C56 und halten das Set fokussiert ohne das Budget zu strecken', 'Falls dir C76 wichtiger als A83 ist, tausche ich A83 -> C76 bei gleicher Struktur (Aufwand 2795) – C56 bleibt in jedem Fall gesetzt', 'Damit bleiben alle hocheffizienten Elemente bestehen (C98, C81, C10, B38, C00, C38, A07), und wir erhalten Robustheit durch C56 ohne Uebergewichtung einzelner Teile'} 
	  } 
	   } 
	   } 
	 & \\ 
 

    \theutterance \stepcounter{utterance}  

    & & & \multicolumn{4}{p{0.6\linewidth}}{\cellcolor[rgb]{1,0.85,0.72}{%
	\makecell[{{p{\linewidth}}}]{% 
	  \tt {\tiny [GM$\langle$B]}  
	 ABLEHNUNG: {'C98', 'C81', 'C10', 'B38', 'C00', 'C38', 'A07', 'C56', 'C08', 'A83'}\\ \tt VORSCHLAG: {'C98', 'C81', 'C10', 'C00', 'C38', 'A07', 'C56', 'C08', 'A83', 'C76'}\\ \tt ARGUMENT: {'Ich uebernehme deine drei Schluesselposten C56, A07 und C38 unveraendert sowie das effiziente Fundament (C98, C81, C10, C08) und A83', 'Kleiner Tausch: B38 -> C76 erhoeht den Gesamtaufwand nur um 93 auf 2927 (< 2984), wir behalten 57 Puffer', 'Der Tausch liegt in deinem angekuendigten Umschichtungsrahmen (< 200 Aufwand) und verbessert die Robustheit ohne das Set aufzublasen', 'Wenn B38 fuer dich unverzichtbar ist, biete ich als gleichwertige Alternative C00 -> C76 an (Gesamtaufwand 2828), alle anderen Posten bleiben identisch', 'Sag mir, welche der beiden Varianten dir lieber ist; beide respektieren deine Kernpunkte und bleiben klar unter dem Limit'} 
	  } 
	   } 
	   } 
	  \\ 
 

    \theutterance \stepcounter{utterance}  

    & & \multicolumn{4}{p{0.6\linewidth}}{\cellcolor[rgb]{0.9,0.9,0.9}{%
	\makecell[{{p{\linewidth}}}]{% 
	  \tt {\tiny [A$\langle$GM]}  
	 ABLEHNUNG: {'C98', 'C81', 'C10', 'B38', 'C00', 'C38', 'A07', 'C56', 'C08', 'A83'}\\ \tt VORSCHLAG: {'C98', 'C81', 'C10', 'C00', 'C38', 'A07', 'C56', 'C08', 'A83', 'C76'}\\ \tt ARGUMENT: {'Ich uebernehme deine drei Schluesselposten C56, A07 und C38 unveraendert sowie das effiziente Fundament (C98, C81, C10, C08) und A83', 'Kleiner Tausch: B38 -> C76 erhoeht den Gesamtaufwand nur um 93 auf 2927 (< 2984), wir behalten 57 Puffer', 'Der Tausch liegt in deinem angekuendigten Umschichtungsrahmen (< 200 Aufwand) und verbessert die Robustheit ohne das Set aufzublasen', 'Wenn B38 fuer dich unverzichtbar ist, biete ich als gleichwertige Alternative C00 -> C76 an (Gesamtaufwand 2828), alle anderen Posten bleiben identisch', 'Sag mir, welche der beiden Varianten dir lieber ist; beide respektieren deine Kernpunkte und bleiben klar unter dem Limit'} 
	  } 
	   } 
	   } 
	 & \\ 
 

    \theutterance \stepcounter{utterance}  

    & \multicolumn{4}{p{0.6\linewidth}}{\cellcolor[rgb]{0.8,1,0.9}{%
	\makecell[{{p{\linewidth}}}]{% 
	  \tt {\tiny [A$\rangle$GM]}  
	 ABLEHNUNG: {'C98', 'C81', 'C10', 'C00', 'C38', 'A07', 'C56', 'C08', 'A83', 'C76'}\\ \tt VORSCHLAG: {'C98', 'C81', 'C10', 'C00', 'C38', 'A07', 'C56', 'C08', 'C76', 'B38'}\\ \tt ARGUMENT: {'Ein schlanker 1-zu-1 Tausch: A83 -> B38; Gesamtaufwand sinkt von 2927 auf 2795 (< 2984) und laesst 189 Puffer', 'Dein Kern bleibt unangetastet (C56, A07, C38) plus das effiziente Fundament (C98, C81, C10, C08, C00); zusaetzlich halte ich C76 fuer dich drin', 'Der Tausch verbessert das Nutzen-zu-Aufwand-Profil und reduziert Abhaengigkeit von schwach skalierenden Posten', 'Wenn A83 fuer dich unverzichtbar ist, kann ich alternativ C76 -> A83 tauschen bei gleicher Struktur; Aufwand waere dann 2834', 'Ich kann hier direkt finalisieren, wenn dir diese minimalinvasive Anpassung passt'} 
	  } 
	   } 
	   } 
	 & & \\ 
 

    \theutterance \stepcounter{utterance}  

    & & \multicolumn{4}{p{0.6\linewidth}}{\cellcolor[rgb]{0.9,0.9,0.9}{%
	\makecell[{{p{\linewidth}}}]{% 
	  \tt {\tiny [GM$\rangle$B]}  
	 ABLEHNUNG: {'C98', 'C81', 'C10', 'C00', 'C38', 'A07', 'C56', 'C08', 'A83', 'C76'}\\ \tt VORSCHLAG: {'C98', 'C81', 'C10', 'C00', 'C38', 'A07', 'C56', 'C08', 'C76', 'B38'}\\ \tt ARGUMENT: {'Ein schlanker 1-zu-1 Tausch: A83 -> B38; Gesamtaufwand sinkt von 2927 auf 2795 (< 2984) und laesst 189 Puffer', 'Dein Kern bleibt unangetastet (C56, A07, C38) plus das effiziente Fundament (C98, C81, C10, C08, C00); zusaetzlich halte ich C76 fuer dich drin', 'Der Tausch verbessert das Nutzen-zu-Aufwand-Profil und reduziert Abhaengigkeit von schwach skalierenden Posten', 'Wenn A83 fuer dich unverzichtbar ist, kann ich alternativ C76 -> A83 tauschen bei gleicher Struktur; Aufwand waere dann 2834', 'Ich kann hier direkt finalisieren, wenn dir diese minimalinvasive Anpassung passt'} 
	  } 
	   } 
	   } 
	 & \\ 
 

    \theutterance \stepcounter{utterance}  

    & & & \multicolumn{4}{p{0.6\linewidth}}{\cellcolor[rgb]{1,0.85,0.72}{%
	\makecell[{{p{\linewidth}}}]{% 
	  \tt {\tiny [GM$\langle$B]}  
	 ABLEHNUNG: {'C98', 'C81', 'C10', 'C00', 'C38', 'A07', 'C56', 'C08', 'C76', 'B38'}\\ \tt VORSCHLAG: {'C98', 'C81', 'C10', 'C38', 'A07', 'C56', 'C08', 'C76', 'B38', 'A83'}\\ \tt ARGUMENT: {'Minimaler, zielgerichteter Tausch: C00 -> A83; Gesamtaufwand 2828 < 2984 (156 Puffer). Alles andere aus deinem Korb bleibt unveraendert inklusive deiner Kernpunkte C56, A07, C38 sowie B38 und C76', 'Du hattest C00 bereits als tauschbar eingerauemt; ich bleibe in deinem Umschichtungsrahmen und wahre deine gesetzten Prioritaeten', 'Der Wechsel hebt das erwartete Nutzen-zu-Aufwand-Verhaeltnis spuerbar an, ohne neue Risiken oder Budgetdruck zu erzeugen', 'Falls C00 fuer dich unverzichtbar ist, kann ich als Alternativvariante C76 -> A83 anbieten (Aufwand 2834) bei sonst identischem Set'} 
	  } 
	   } 
	   } 
	  \\ 
 

    \theutterance \stepcounter{utterance}  

    & & \multicolumn{4}{p{0.6\linewidth}}{\cellcolor[rgb]{0.9,0.9,0.9}{%
	\makecell[{{p{\linewidth}}}]{% 
	  \tt {\tiny [A$\langle$GM]}  
	 ABLEHNUNG: {'C98', 'C81', 'C10', 'C00', 'C38', 'A07', 'C56', 'C08', 'C76', 'B38'}\\ \tt VORSCHLAG: {'C98', 'C81', 'C10', 'C38', 'A07', 'C56', 'C08', 'C76', 'B38', 'A83'}\\ \tt ARGUMENT: {'Minimaler, zielgerichteter Tausch: C00 -> A83; Gesamtaufwand 2828 < 2984 (156 Puffer). Alles andere aus deinem Korb bleibt unveraendert inklusive deiner Kernpunkte C56, A07, C38 sowie B38 und C76', 'Du hattest C00 bereits als tauschbar eingerauemt; ich bleibe in deinem Umschichtungsrahmen und wahre deine gesetzten Prioritaeten', 'Der Wechsel hebt das erwartete Nutzen-zu-Aufwand-Verhaeltnis spuerbar an, ohne neue Risiken oder Budgetdruck zu erzeugen', 'Falls C00 fuer dich unverzichtbar ist, kann ich als Alternativvariante C76 -> A83 anbieten (Aufwand 2834) bei sonst identischem Set'} 
	  } 
	   } 
	   } 
	 & \\ 
 

    \theutterance \stepcounter{utterance}  

    & \multicolumn{4}{p{0.6\linewidth}}{\cellcolor[rgb]{0.8,1,0.9}{%
	\makecell[{{p{\linewidth}}}]{% 
	  \tt {\tiny [A$\rangle$GM]}  
	 ABLEHNUNG: {'C98', 'C81', 'C10', 'C38', 'A07', 'C56', 'C08', 'C76', 'B38', 'A83'}\\ \tt VORSCHLAG: {'C98', 'C81', 'C10', 'C38', 'A07', 'C56', 'C08', 'B38', 'A83', 'C00'}\\ \tt ARGUMENT: {'Minimaler 1-zu-1 Tausch: C76 -> C00; Gesamtaufwand 2834 < 2984 (150 Puffer)', 'Dein Kern bleibt unveraendert (C56, A07, C38, B38) sowie das effiziente Fundament (C98, C81, C10, C08); ich lasse A83 ausdruecklich drin', 'C00 bietet in diesem Setup deutlich bessere Stabilitaet und Effizienz als C76 ohne das Budget zu belasten', 'Falls C76 fuer dich gesetzt ist, Alternativpfad: A83 -> C62 bei sonst identischem Set (Aufwand 2777; 207 Puffer). Sag mir kurz, welche Variante du bevorzugst'} 
	  } 
	   } 
	   } 
	 & & \\ 
 

    \theutterance \stepcounter{utterance}  

    & & \multicolumn{4}{p{0.6\linewidth}}{\cellcolor[rgb]{0.9,0.9,0.9}{%
	\makecell[{{p{\linewidth}}}]{% 
	  \tt {\tiny [GM$\rangle$B]}  
	 ABLEHNUNG: {'C98', 'C81', 'C10', 'C38', 'A07', 'C56', 'C08', 'C76', 'B38', 'A83'}\\ \tt VORSCHLAG: {'C98', 'C81', 'C10', 'C38', 'A07', 'C56', 'C08', 'B38', 'A83', 'C00'}\\ \tt ARGUMENT: {'Minimaler 1-zu-1 Tausch: C76 -> C00; Gesamtaufwand 2834 < 2984 (150 Puffer)', 'Dein Kern bleibt unveraendert (C56, A07, C38, B38) sowie das effiziente Fundament (C98, C81, C10, C08); ich lasse A83 ausdruecklich drin', 'C00 bietet in diesem Setup deutlich bessere Stabilitaet und Effizienz als C76 ohne das Budget zu belasten', 'Falls C76 fuer dich gesetzt ist, Alternativpfad: A83 -> C62 bei sonst identischem Set (Aufwand 2777; 207 Puffer). Sag mir kurz, welche Variante du bevorzugst'} 
	  } 
	   } 
	   } 
	 & \\ 
 

    \theutterance \stepcounter{utterance}  

    & & & \multicolumn{4}{p{0.6\linewidth}}{\cellcolor[rgb]{1,0.85,0.72}{%
	\makecell[{{p{\linewidth}}}]{% 
	  \tt {\tiny [GM$\langle$B]}  
	 ZUSTIMMUNG: {'C98', 'C81', 'C10', 'C38', 'A07', 'C56', 'C08', 'B38', 'A83', 'C00'}\\ \tt ARGUMENT: {'Gesamtaufwand 2834 liegt klar unter dem Limit 2984 und laesst ausreichend Puffer', 'Das Paket ist ausgewogen und enthaelt mehrere sehr effiziente Posten bei guter Diversifikation', 'Deine zentralen Punkte bleiben gewahrt; damit ist der Vorschlag fuer mich tragfaehig', 'Ich finalisiere hiermit'} 
	  } 
	   } 
	   } 
	  \\ 
 

    \theutterance \stepcounter{utterance}  

    & & & \multicolumn{2}{p{0.3\linewidth}}{\cellcolor[rgb]{0.95,0.95,0.95}{%
	\makecell[{{p{\linewidth}}}]{% 
	  \tt {\tiny [GM$|$GM]}  
	 game successful 
	  } 
	   } 
	   } 
	 & & \\ 
 

    \theutterance \stepcounter{utterance}  

    & & & \multicolumn{2}{p{0.3\linewidth}}{\cellcolor[rgb]{0.95,0.95,0.95}{%
	\makecell[{{p{\linewidth}}}]{% 
	  \tt {\tiny [GM$|$GM]}  
	 end game 
	  } 
	   } 
	   } 
	 & & \\ 
 

\end{supertabular}
}

\end{document}
