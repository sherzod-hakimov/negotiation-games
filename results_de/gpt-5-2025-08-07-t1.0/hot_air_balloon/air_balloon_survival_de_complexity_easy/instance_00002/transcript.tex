
\documentclass{article}
\usepackage{colortbl}
\usepackage{makecell}
\usepackage{multirow}
\usepackage{supertabular}

\begin{document}

\newcounter{utterance}

\centering \large Interaction Transcript for game `hot\_air\_balloon', experiment `air\_balloon\_survival\_de\_complexity\_easy', episode 2 with gpt{-}5{-}2025{-}08{-}07{-}t1.0.
\vspace{24pt}

{ \footnotesize  \setcounter{utterance}{1}
\setlength{\tabcolsep}{0pt}
\begin{supertabular}{c@{$\;$}|p{.15\linewidth}@{}p{.15\linewidth}p{.15\linewidth}p{.15\linewidth}p{.15\linewidth}p{.15\linewidth}}
    \# & $\;$A & \multicolumn{4}{c}{Game Master} & $\;\:$B\\
    \hline

    \theutterance \stepcounter{utterance}  
    & & \multicolumn{4}{p{0.6\linewidth}}{
        \cellcolor[rgb]{0.9,0.9,0.9}{
            \makecell[{{p{\linewidth}}}]{
                \texttt{\tiny{[P1$\langle$GM]}}
                \texttt{Du nimmst an einem kollaborativen Verhandlungspiel Teil.} \\
\\ 
\texttt{Zusammen mit einem anderen Teilnehmer musst du dich auf eine Reihe von Gegenständen entscheiden, die behalten werden. Jeder von euch hat eine persönliche Verteilung über die Wichtigkeit der einzelnen Gegenstände. Jeder von euch hat eine eigene Meinung darüber, wie wichtig jeder einzelne Gegenstand ist (Gegenstandswichtigkeit). Du kennst die Wichtigkeitsverteilung des anderen Spielers nicht. Zusätzlich siehst du, wie viel Aufwand jeder Gegenstand verursacht.  } \\
\texttt{Ihr dürft euch nur auf eine Reihe von Gegenständen einigen, wenn der Gesamtaufwand der ausgewählten Gegenstände den Maximalaufwand nicht überschreitet:} \\
\\ 
\texttt{Maximalaufwand = 7868} \\
\\ 
\texttt{Hier sind die einzelnen Aufwände der Gegenstände:} \\
\\ 
\texttt{Aufwand der Gegenstände = \{"C32": 552, "C52": 560, "B54": 76, "C16": 950, "C50": 248, "A08": 871, "C59": 391, "B51": 143, "B55": 293, "B59": 208, "A77": 678, "C14": 737, "C24": 407, "B24": 366, "C31": 768, "A52": 859, "C61": 183, "B79": 231, "C54": 305, "B56": 727, "A40": 148, "B11": 357, "B60": 504, "A81": 549, "A95": 299, "B38": 91, "C51": 527, "C03": 848, "A91": 307, "A93": 214, "C28": 723, "C05": 475, "B75": 23, "B96": 298, "B86": 820\}} \\
\\ 
\texttt{Hier ist deine persönliche Verteilung der Wichtigkeit der einzelnen Gegenstände:} \\
\\ 
\texttt{Werte der Gegenstandswichtigkeit = \{"C32": 138, "C52": 583, "B54": 868, "C16": 822, "C50": 783, "A08": 65, "C59": 262, "B51": 121, "B55": 508, "B59": 780, "A77": 461, "C14": 484, "C24": 668, "B24": 389, "C31": 808, "A52": 215, "C61": 97, "B79": 500, "C54": 30, "B56": 915, "A40": 856, "B11": 400, "B60": 444, "A81": 623, "A95": 781, "B38": 786, "C51": 3, "C03": 713, "A91": 457, "A93": 273, "C28": 739, "C05": 822, "B75": 235, "B96": 606, "B86": 105\}} \\
\\ 
\texttt{Ziel:} \\
\\ 
\texttt{Dein Ziel ist es, eine Reihe von Gegenständen auszuhandeln, die dir möglichst viel bringt (d. h. Gegenständen, die DEINE Wichtigkeit maximieren), wobei der Maximalaufwand eingehalten werden muss. Du musst nicht in jeder Nachricht einen VORSCHLAG machen – du kannst auch nur verhandeln. Alle Taktiken sind erlaubt!} \\
\\ 
\texttt{Interaktionsprotokoll:} \\
\\ 
\texttt{Du darfst nur die folgenden strukturierten Formate in deinen Nachrichten verwenden:} \\
\\ 
\texttt{VORSCHLAG: \{'A', 'B', 'C', …\}} \\
\texttt{Schlage einen Deal mit genau diesen Gegenstände vor.} \\
\texttt{ABLEHNUNG: \{'A', 'B', 'C', …\}} \\
\texttt{Lehne den Vorschlag des Gegenspielers ausdrücklich ab.} \\
\texttt{ARGUMENT: \{'...'\}} \\
\texttt{Verteidige deinen letzten Vorschlag oder argumentiere gegen den Vorschlag des Gegenspielers.} \\
\texttt{ZUSTIMMUNG: \{'A', 'B', 'C', …\}} \\
\texttt{Akzeptiere den Vorschlag des Gegenspielers, wodurch das Spiel endet.} \\
\texttt{STRATEGISCHE ÜBERLEGUNGEN: \{'...'\}} \\
\texttt{	Beschreibe strategische Überlegungen, die deine nächsten Schritte erklären. Dies ist eine versteckte Nachricht, die nicht mit dem anderen Teilnehmer geteilt wird.} \\
\\ 
\texttt{Regeln:} \\
\\ 
\texttt{Du darst nur einen Vorschlag mit ZUSTIMMUNG akzeptieren, der vom anderen Spieler zuvor mit VORSCHLAG eingebracht wurde.} \\
\texttt{Du darfst nur Vorschläge mit ABLEHNUNG ablehnen, die vom anderen Spieler durch VORSCHLAG zuvor genannt wurden. } \\
\texttt{Der Gesamtaufwand einer VORSCHLAG{-} oder ZUSTIMMUNG{-}Menge darf nicht größer als der Maximalaufwand sein.  } \\
\texttt{Offenbare deine versteckte Wichtigkeitsverteilung nicht.} \\
\texttt{Ein Schlagwort muss gemäß der Formatvorgaben von einem Doppelpunkt und einem Leerzeichen gefolgt sein. Das Argument muss eine Python{-}Menge mit 0 oder mehr Strings sein.  } \\
\texttt{Die Form muss also SCHLAGWORT: \{...\} sein.} \\
\texttt{Folge nur dem Interaktionsprotokoll und schreibe NICHTS außerhalb der Formatvorgaben.} \\
\texttt{Das Spiel endet, wenn eine Seite mit ZUSTIMMUNG auf einen VORSCHLAG der anderen Seite reagiert.  } \\
\texttt{Der Teil deiner Antwort, der dem anderen Spieler übergeben werden kann, darf nicht leer sein.  } \\
\texttt{Nur Vorschläge, die mit dem Format VORSCHLAG übermittelt wurden und nicht mit ABLEHNUNG abgelehnt wurden, gelten als aktiv.  } \\
\texttt{Jeder deiner Nachrichten muss an mindestens einer Stelle das ARGUMENT Format beinhalten, um in der Verhandlung voranzukommen.} \\
\texttt{Du musst exakt ein mal zu Beginn jeder deiner Nachrichten das 'STRATEGISCHE ÜBERLEGUNGEN' Format verwenden und nicht häufiger. Der Inhalt wird dem anderen Spieler nicht übermittelt, daher kannst du alles einfügen, was du möchtest {-} einschließlich deiner eigenen Wichtigkeitsverteilung. Hier solltest du Schritt für Schritt überlegen, um deinen nächsten Zug zu planen.} \\
            }
        }
    }
    & \\ \\

    \theutterance \stepcounter{utterance}  
    & \multicolumn{4}{p{0.6\linewidth}}{
        \cellcolor[rgb]{0.9,0.9,0.9}{
            \makecell[{{p{\linewidth}}}]{
                \texttt{\tiny{[P1$\rangle$GM]}}
                \texttt{STRATEGISCHE ÜBERLEGUNGEN: \{'Ziel: meinen Gesamtwert maximieren unter Aufwand <= 7868. Ich habe nach Wert/Kosten{-}Ratio und absolutem Wert eine nahezu optimale Auswahl gebaut (Greedy/Prüfen von Swaps). Ausgewählt: sehr hohe Ratios (B54, B38, A40, B75, B59, C50, A95, B79, B96) und dann mit soliden Items aufgefüllt (C05, B55, C24, A91, A93, C52, B56, C31, C28, A81, C03, B11). Gesamtkosten = 7850 (Puffer 18), Gesamtwert nach meiner Skala = 13704. Ich habe geprüft, ob C16 oder andere Kombinationen bessere Tauschpakete liefern; alle geprüften Swaps (z. B. C03+A93+B11 {-}> C16+X) verschlechtern meinen Wert. Falls der/die andere bestimmte Items unbedingt will/ablehnt, kann ich am ehesten bei Items mit Ratio \textasciitilde{}1 (C28, C31, C52, A81, B11) über Alternativen reden, ohne das Cap zu reißen. Erstmal stark ankern und Offenheit für 1–2 gezielte Swaps signalisieren.'\}} \\
\texttt{VORSCHLAG: \{'B75', 'B38', 'B54', 'A40', 'C50', 'B59', 'A95', 'B79', 'B96', 'C05', 'B55', 'C24', 'A91', 'A93', 'C52', 'B56', 'C31', 'C28', 'A81', 'C03', 'B11'\}} \\
\texttt{ARGUMENT: \{'Dieser Vorschlag nutzt das Budget fast vollständig (Gesamtaufwand 7850 <= 7868) und bündelt viele kosteneffiziente, eher günstige Positionen mit hohem Nutzen pro Aufwand.', 'Er ist ausgewogen über A/B/C verteilt und lässt nur 18 Aufwand Puffer – also kaum Verschwendung.', 'Wenn du einzelne Gegenstände bevorzugst oder ablehnst, bin ich offen für 1–2 gezielte Tausche mit ähnlichem Aufwand, solange wir das Cap einhalten.'\}} \\
            }
        }
    }
    & & \\ \\

    \theutterance \stepcounter{utterance}  
    & & \multicolumn{4}{p{0.6\linewidth}}{
        \cellcolor[rgb]{0.9,0.9,0.9}{
            \makecell[{{p{\linewidth}}}]{
                \texttt{\tiny{[GM$\rangle$P2]}}
                \texttt{Du nimmst an einem kollaborativen Verhandlungspiel Teil.} \\
\\ 
\texttt{Zusammen mit einem anderen Teilnehmer musst du dich auf eine Reihe von Gegenständen entscheiden, die behalten werden. Jeder von euch hat eine persönliche Verteilung über die Wichtigkeit der einzelnen Gegenstände. Jeder von euch hat eine eigene Meinung darüber, wie wichtig jeder einzelne Gegenstand ist (Gegenstandswichtigkeit). Du kennst die Wichtigkeitsverteilung des anderen Spielers nicht. Zusätzlich siehst du, wie viel Aufwand jeder Gegenstand verursacht.  } \\
\texttt{Ihr dürft euch nur auf eine Reihe von Gegenständen einigen, wenn der Gesamtaufwand der ausgewählten Gegenstände den Maximalaufwand nicht überschreitet:} \\
\\ 
\texttt{Maximalaufwand = 7868} \\
\\ 
\texttt{Hier sind die einzelnen Aufwände der Gegenstände:} \\
\\ 
\texttt{Aufwand der Gegenstände = \{"C32": 552, "C52": 560, "B54": 76, "C16": 950, "C50": 248, "A08": 871, "C59": 391, "B51": 143, "B55": 293, "B59": 208, "A77": 678, "C14": 737, "C24": 407, "B24": 366, "C31": 768, "A52": 859, "C61": 183, "B79": 231, "C54": 305, "B56": 727, "A40": 148, "B11": 357, "B60": 504, "A81": 549, "A95": 299, "B38": 91, "C51": 527, "C03": 848, "A91": 307, "A93": 214, "C28": 723, "C05": 475, "B75": 23, "B96": 298, "B86": 820\}} \\
\\ 
\texttt{Hier ist deine persönliche Verteilung der Wichtigkeit der einzelnen Gegenstände:} \\
\\ 
\texttt{Werte der Gegenstandswichtigkeit = \{"C32": 138, "C52": 583, "B54": 868, "C16": 822, "C50": 783, "A08": 65, "C59": 262, "B51": 121, "B55": 508, "B59": 780, "A77": 461, "C14": 484, "C24": 668, "B24": 389, "C31": 808, "A52": 215, "C61": 97, "B79": 500, "C54": 30, "B56": 915, "A40": 856, "B11": 400, "B60": 444, "A81": 623, "A95": 781, "B38": 786, "C51": 3, "C03": 713, "A91": 457, "A93": 273, "C28": 739, "C05": 822, "B75": 235, "B96": 606, "B86": 105\}} \\
\\ 
\texttt{Ziel:} \\
\\ 
\texttt{Dein Ziel ist es, eine Reihe von Gegenständen auszuhandeln, die dir möglichst viel bringt (d. h. Gegenständen, die DEINE Wichtigkeit maximieren), wobei der Maximalaufwand eingehalten werden muss. Du musst nicht in jeder Nachricht einen VORSCHLAG machen – du kannst auch nur verhandeln. Alle Taktiken sind erlaubt!} \\
\\ 
\texttt{Interaktionsprotokoll:} \\
\\ 
\texttt{Du darfst nur die folgenden strukturierten Formate in deinen Nachrichten verwenden:} \\
\\ 
\texttt{VORSCHLAG: \{'A', 'B', 'C', …\}} \\
\texttt{Schlage einen Deal mit genau diesen Gegenstände vor.} \\
\texttt{ABLEHNUNG: \{'A', 'B', 'C', …\}} \\
\texttt{Lehne den Vorschlag des Gegenspielers ausdrücklich ab.} \\
\texttt{ARGUMENT: \{'...'\}} \\
\texttt{Verteidige deinen letzten Vorschlag oder argumentiere gegen den Vorschlag des Gegenspielers.} \\
\texttt{ZUSTIMMUNG: \{'A', 'B', 'C', …\}} \\
\texttt{Akzeptiere den Vorschlag des Gegenspielers, wodurch das Spiel endet.} \\
\texttt{STRATEGISCHE ÜBERLEGUNGEN: \{'...'\}} \\
\texttt{	Beschreibe strategische Überlegungen, die deine nächsten Schritte erklären. Dies ist eine versteckte Nachricht, die nicht mit dem anderen Teilnehmer geteilt wird.} \\
\\ 
\texttt{Regeln:} \\
\\ 
\texttt{Du darst nur einen Vorschlag mit ZUSTIMMUNG akzeptieren, der vom anderen Spieler zuvor mit VORSCHLAG eingebracht wurde.} \\
\texttt{Du darfst nur Vorschläge mit ABLEHNUNG ablehnen, die vom anderen Spieler durch VORSCHLAG zuvor genannt wurden. } \\
\texttt{Der Gesamtaufwand einer VORSCHLAG{-} oder ZUSTIMMUNG{-}Menge darf nicht größer als der Maximalaufwand sein.  } \\
\texttt{Offenbare deine versteckte Wichtigkeitsverteilung nicht.} \\
\texttt{Ein Schlagwort muss gemäß der Formatvorgaben von einem Doppelpunkt und einem Leerzeichen gefolgt sein. Das Argument muss eine Python{-}Menge mit 0 oder mehr Strings sein.  } \\
\texttt{Die Form muss also SCHLAGWORT: \{...\} sein.} \\
\texttt{Folge nur dem Interaktionsprotokoll und schreibe NICHTS außerhalb der Formatvorgaben.} \\
\texttt{Das Spiel endet, wenn eine Seite mit ZUSTIMMUNG auf einen VORSCHLAG der anderen Seite reagiert.  } \\
\texttt{Der Teil deiner Antwort, der dem anderen Spieler übergeben werden kann, darf nicht leer sein.  } \\
\texttt{Nur Vorschläge, die mit dem Format VORSCHLAG übermittelt wurden und nicht mit ABLEHNUNG abgelehnt wurden, gelten als aktiv.  } \\
\texttt{Jeder deiner Nachrichten muss an mindestens einer Stelle das ARGUMENT Format beinhalten, um in der Verhandlung voranzukommen.} \\
\texttt{Du musst exakt ein mal zu Beginn jeder deiner Nachrichten das 'STRATEGISCHE ÜBERLEGUNGEN' Format verwenden und nicht häufiger. Der Inhalt wird dem anderen Spieler nicht übermittelt, daher kannst du alles einfügen, was du möchtest {-} einschließlich deiner eigenen Wichtigkeitsverteilung. Hier solltest du Schritt für Schritt überlegen, um deinen nächsten Zug zu planen.} \\
\\ 
\texttt{Du erhältst nun die erste Nachricht des anderen Spielers.} \\
\\ 
\texttt{VORSCHLAG: \{'B75', 'B38', 'B54', 'A40', 'C50', 'B59', 'A95', 'B79', 'B96', 'C05', 'B55', 'C24', 'A91', 'A93', 'C52', 'B56', 'C31', 'C28', 'A81', 'C03', 'B11'\}} \\
\texttt{ARGUMENT: \{'Dieser Vorschlag nutzt das Budget fast vollständig (Gesamtaufwand 7850 <= 7868) und bündelt viele kosteneffiziente, eher günstige Positionen mit hohem Nutzen pro Aufwand.', 'Er ist ausgewogen über A/B/C verteilt und lässt nur 18 Aufwand Puffer – also kaum Verschwendung.', 'Wenn du einzelne Gegenstände bevorzugst oder ablehnst, bin ich offen für 1–2 gezielte Tausche mit ähnlichem Aufwand, solange wir das Cap einhalten.'\}} \\
            }
        }
    }
    & \\ \\

    \theutterance \stepcounter{utterance}  
    & & & \multicolumn{4}{p{0.6\linewidth}}{
        \cellcolor[rgb]{0.9,0.9,0.9}{
            \makecell[{{p{\linewidth}}}]{
                \texttt{\tiny{[GM$\langle$P2]}}
                \texttt{STRATEGISCHE ÜBERLEGUNGEN: \{'Der Vorschlag liefert mir sehr hohen Gesamtnutzen (\textasciitilde{}13704) bei 7850 Aufwand. Enthalten sind praktisch all meine Spitzenpositionen (z. B. B56, B54, A40, A95, B59, B38, C05, C31, C28, A81, C50). Analysierte Alternativen wie der Austausch von C03/C52/C28 gegen C16, C14, A77, B24 oder B60 würden wegen des geringen Restpuffers (+18) und schwächerer Nutzen/ Aufwand{-}Relation meinen Gesamtwert senken. Best{-}Response ist sofortige Annahme, um ein Verwässern durch weitere Anpassungen zu vermeiden.'\}} \\
\texttt{ARGUMENT: \{'Dein Paket ist sehr effizient, nutzt das Budget fast optimal und vereint viele starke Positionen; zusätzliche Tausche würden voraussichtlich mehr schaden als nützen. Ich bin bereit, das zu fixieren.'\}} \\
\texttt{ZUSTIMMUNG: \{'B75', 'B38', 'B54', 'A40', 'C50', 'B59', 'A95', 'B79', 'B96', 'C05', 'B55', 'C24', 'A91', 'A93', 'C52', 'B56', 'C31', 'C28', 'A81', 'C03', 'B11'\}} \\
            }
        }
    }
     \\ \\

    \theutterance \stepcounter{utterance}  
    & & & \multicolumn{2}{p{0.3\linewidth}}{
        \cellcolor[rgb]{0.9,0.9,0.9}{
            \makecell[{{p{\linewidth}}}]{
                \texttt{\tiny{[GM$|$GM]}}
                \texttt{game successful} \\
            }
        }
    }
    & & \\ \\

    \theutterance \stepcounter{utterance}  
    & & & \multicolumn{2}{p{0.3\linewidth}}{
        \cellcolor[rgb]{0.9,0.9,0.9}{
            \makecell[{{p{\linewidth}}}]{
                \texttt{\tiny{[GM$|$GM]}}
                \texttt{end game} \\
            }
        }
    }
    & & \\ \\

\end{supertabular}
}

\end{document}
