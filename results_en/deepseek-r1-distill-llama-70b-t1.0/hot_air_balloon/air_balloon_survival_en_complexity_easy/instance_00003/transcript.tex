
\documentclass{article}
\usepackage{colortbl}
\usepackage{makecell}
\usepackage{multirow}
\usepackage{supertabular}

\begin{document}

\newcounter{utterance}

\twocolumn

{ \footnotesize  \setcounter{utterance}{1}
\setlength{\tabcolsep}{0pt}
\begin{supertabular}{c@{$\;$}|p{.15\linewidth}@{}p{.15\linewidth}p{.15\linewidth}p{.15\linewidth}p{.15\linewidth}p{.15\linewidth}}

    \# & $\;$A & \multicolumn{4}{c}{Game Master} & $\;\:$B\\
    \hline 

    \theutterance \stepcounter{utterance}  

    & & \multicolumn{4}{p{0.6\linewidth}}{\cellcolor[rgb]{0.9,0.9,0.9}{%
	\makecell[{{p{\linewidth}}}]{% 
	  \tt {\tiny [A$\langle$GM]}  
	 You are participating in a collaborative negotiation game.\\ \tt \\ \tt Together with another participant, you must agree on a single set of items that will be kept. Each of you has your own view of how much each item matters to you (importance). You do not know how the other participant values the items. Additionally, you are given the effort each item demands.\\ \tt You may only agree on a set if the total effort of the selected items does not exceed a shared limit:\\ \tt \\ \tt LIMIT = 8770\\ \tt \\ \tt Here are the individual item effort values:\\ \tt \\ \tt Item effort = {"B21": 830, "B00": 638, "A08": 607, "A27": 106, "B94": 630, "B53": 382, "A23": 773, "C65": 455, "A89": 262, "B67": 634, "B45": 60, "C34": 54, "C12": 849, "A05": 801, "B60": 324, "C92": 164, "B69": 832, "A84": 136, "C06": 980, "B12": 946, "A16": 645, "B85": 842, "C35": 106, "B10": 116, "C15": 875, "B09": 446, "B16": 649, "C81": 601, "A58": 252, "C24": 763, "A30": 213, "A66": 517, "B75": 520, "B46": 407, "C70": 125}\\ \tt \\ \tt Here is your personal view on the importance of each item:\\ \tt \\ \tt Item importance values = {"B21": 138, "B00": 583, "A08": 868, "A27": 822, "B94": 783, "B53": 65, "A23": 262, "C65": 121, "A89": 508, "B67": 780, "B45": 461, "C34": 484, "C12": 668, "A05": 389, "B60": 808, "C92": 215, "B69": 97, "A84": 500, "C06": 30, "B12": 915, "A16": 856, "B85": 400, "C35": 444, "B10": 623, "C15": 781, "B09": 786, "B16": 3, "C81": 713, "A58": 457, "C24": 273, "A30": 739, "A66": 822, "B75": 235, "B46": 606, "C70": 968}\\ \tt \\ \tt Goal:\\ \tt \\ \tt Your goal is to negotiate a shared set of items that benefits you as much as possible (i.e., maximizes total importance to YOU), while staying within the LIMIT. You are not required to make a PROPOSAL in every message - you can simply negotiate as well. All tactics are allowed!\\ \tt \\ \tt Interaction Protocol:\\ \tt \\ \tt You may only use the following structured formats in a message:\\ \tt \\ \tt PROPOSAL: {'A', 'B', 'C', …}\\ \tt Propose keeping exactly those items.\\ \tt REFUSE: {'A', 'B', 'C', …}\\ \tt Explicitly reject opponent's proposal.\\ \tt ARGUMENT: {'...'}\\ \tt Defend your last proposal or argue against the player's proposal.\\ \tt AGREE: {'A', 'B', 'C', …}\\ \tt Accept the opponent's proposal which ends the game.\\ \tt STRATEGIC REASONING: {'...'}\\ \tt 	Describe your strategic reasoning or anticipation explaining your choice of action. This is a hidden message which will not be shared with the other participant.\\ \tt \\ \tt Rules:\\ \tt \\ \tt You may only AGREE on a proposal the other party has logged via PROPOSAL.\\ \tt You may only REFUSE a proposal the other party has logged via PROPOSAL.\\ \tt Total effort of any PROPOSAL or AGREE set must be ≤ LIMIT.\\ \tt Do NOT reveal your hidden importance scores.\\ \tt A tag in a structured format must be followed by colon and whitespace. The argument must be a python set containing 0 or more strings.\\ \tt So, it must be of the form TAG: {...}\\ \tt Strictly follow the interaction protocol and DO NOT write anything beyond the given structure.\\ \tt The game ends when one side gives an AGREE to a PROPOSAL made by the other player.\\ \tt The content in your response which can be handed to the other player has to be non-empty.\\ \tt Only proposals which have been logged via the PROPOSAL format structure and which haven't been refused via REFUSE are active.\\ \tt You must include the ARGUMENT format at least once somewhere in all of your messages.\\ \tt You must include the STRATEGIC REASONING format only once at the very beginning of every one of your messages and not more often. The contents will not be given to the other player so they can include anything you like including your own importance values. Here you should reason step by step to come up with you next move. 
	  } 
	   } 
	   } 
	 & \\ 
 

    \theutterance \stepcounter{utterance}  

    & \multicolumn{4}{p{0.6\linewidth}}{\cellcolor[rgb]{0.8,1,0.9}{%
	\makecell[{{p{\linewidth}}}]{% 
	  \tt {\tiny [A$\rangle$GM]}  
	 STRATEGIC REASONING: {'I started by calculating the importance per effort ratio for each item to prioritize those that maximize my gain. I selected items in descending order of this ratio, ensuring the total effort remains under 8770. After careful consideration, I compiled a set that includes high-value items while staying within the limit. I considered swaps to optimize further but concluded this selection balances effort and importance effectively.'}\\ \tt \\ \tt PROPOSAL: {'C34', 'A27', 'C70', 'B45', 'B10', 'A84', 'C35', 'A30', 'B60', 'C92', 'A58', 'B09', 'A66', 'A16', 'A08', 'B94', 'B67', 'B46', 'C81', 'B00', 'C15', 'B75', 'B53'} 
	  } 
	   } 
	   } 
	 & & \\ 
 

    \theutterance \stepcounter{utterance}  

    & & \multicolumn{4}{p{0.6\linewidth}}{\cellcolor[rgb]{0.9,0.9,0.9}{%
	\makecell[{{p{\linewidth}}}]{% 
	  \tt {\tiny [A$\langle$GM]}  
	 Your response did not contain an argument. You must include the structured format ARGUMENT: {...} somewhere in your response. Try again 
	  } 
	   } 
	   } 
	 & \\ 
 

    \theutterance \stepcounter{utterance}  

    & \multicolumn{4}{p{0.6\linewidth}}{\cellcolor[rgb]{0.8,1,0.9}{%
	\makecell[{{p{\linewidth}}}]{% 
	  \tt {\tiny [A$\rangle$GM]}  
	 STRATEGIC REASONING: {'I started by evaluating each item based on my personal importance ratings and the effort required. To maximize my benefit, I prioritized items with the highest importance-to-effort ratios. This ensures I get the most value without exceeding the effort limit. I compiled a set of items that not only align with my priorities but also create a balanced proposal that respects the effort constraints. By focusing on high-importance items and strategically adding lower-effort items, I aim to create a compelling offer that maximizes my gain while staying within the LIMIT.'}\\ \tt \\ \tt PROPOSAL: {'A27', 'A30', 'A84', 'A05', 'A23', 'A16', 'A66', 'A08', 'B00', 'B21', 'B45', 'B53', 'B60', 'B67', 'B94', 'B09', 'B10', 'B46', 'B75', 'C15', 'C34', 'C35', 'C70', 'C92', 'C24'}\\ \tt \\ \tt ARGUMENT: {'This proposal maximizes my personal importance while staying within the effort limit. It includes high-importance items such as "A08", "B94", and "B67", which are critical to my goals. The selected items balance high importance with manageable effort, ensuring the total effort remains under 8770. By accepting this proposal, both parties benefit from a focused and efficient selection of items.'} 
	  } 
	   } 
	   } 
	 & & \\ 
 

    \theutterance \stepcounter{utterance}  

    & & \multicolumn{4}{p{0.6\linewidth}}{\cellcolor[rgb]{0.9,0.9,0.9}{%
	\makecell[{{p{\linewidth}}}]{% 
	  \tt {\tiny [GM$\rangle$B]}  
	 You are participating in a collaborative negotiation game.\\ \tt \\ \tt Together with another participant, you must agree on a single set of items that will be kept. Each of you has your own view of how much each item matters to you (importance). You do not know how the other participant values the items. Additionally, you are given the effort each item demands.\\ \tt You may only agree on a set if the total effort of the selected items does not exceed a shared limit:\\ \tt \\ \tt LIMIT = 8770\\ \tt \\ \tt Here are the individual item effort values:\\ \tt \\ \tt Item effort = {"B21": 830, "B00": 638, "A08": 607, "A27": 106, "B94": 630, "B53": 382, "A23": 773, "C65": 455, "A89": 262, "B67": 634, "B45": 60, "C34": 54, "C12": 849, "A05": 801, "B60": 324, "C92": 164, "B69": 832, "A84": 136, "C06": 980, "B12": 946, "A16": 645, "B85": 842, "C35": 106, "B10": 116, "C15": 875, "B09": 446, "B16": 649, "C81": 601, "A58": 252, "C24": 763, "A30": 213, "A66": 517, "B75": 520, "B46": 407, "C70": 125}\\ \tt \\ \tt Here is your personal view on the importance of each item:\\ \tt \\ \tt Item importance values = {"B21": 138, "B00": 583, "A08": 868, "A27": 822, "B94": 783, "B53": 65, "A23": 262, "C65": 121, "A89": 508, "B67": 780, "B45": 461, "C34": 484, "C12": 668, "A05": 389, "B60": 808, "C92": 215, "B69": 97, "A84": 500, "C06": 30, "B12": 915, "A16": 856, "B85": 400, "C35": 444, "B10": 623, "C15": 781, "B09": 786, "B16": 3, "C81": 713, "A58": 457, "C24": 273, "A30": 739, "A66": 822, "B75": 235, "B46": 606, "C70": 968}\\ \tt \\ \tt Goal:\\ \tt \\ \tt Your goal is to negotiate a shared set of items that benefits you as much as possible (i.e., maximizes total importance to YOU), while staying within the LIMIT. You are not required to make a PROPOSAL in every message - you can simply negotiate as well. All tactics are allowed!\\ \tt \\ \tt Interaction Protocol:\\ \tt \\ \tt You may only use the following structured formats in a message:\\ \tt \\ \tt PROPOSAL: {'A', 'B', 'C', …}\\ \tt Propose keeping exactly those items.\\ \tt REFUSE: {'A', 'B', 'C', …}\\ \tt Explicitly reject opponent's proposal.\\ \tt ARGUMENT: {'...'}\\ \tt Defend your last proposal or argue against the player's proposal.\\ \tt AGREE: {'A', 'B', 'C', …}\\ \tt Accept the opponent's proposal which ends the game.\\ \tt STRATEGIC REASONING: {'...'}\\ \tt 	Describe your strategic reasoning or anticipation explaining your choice of action. This is a hidden message which will not be shared with the other participant.\\ \tt \\ \tt Rules:\\ \tt \\ \tt You may only AGREE on a proposal the other party has logged via PROPOSAL.\\ \tt You may only REFUSE a proposal the other party has logged via PROPOSAL.\\ \tt Total effort of any PROPOSAL or AGREE set must be ≤ LIMIT.\\ \tt Do NOT reveal your hidden importance scores.\\ \tt A tag in a structured format must be followed by colon and whitespace. The argument must be a python set containing 0 or more strings.\\ \tt So, it must be of the form TAG: {...}\\ \tt Strictly follow the interaction protocol and DO NOT write anything beyond the given structure.\\ \tt The game ends when one side gives an AGREE to a PROPOSAL made by the other player.\\ \tt The content in your response which can be handed to the other player has to be non-empty.\\ \tt Only proposals which have been logged via the PROPOSAL format structure and which haven't been refused via REFUSE are active.\\ \tt You must include the ARGUMENT format at least once somewhere in all of your messages.\\ \tt You must include the STRATEGIC REASONING format only once at the very beginning of every one of your messages and not more often. The contents will not be given to the other player so they can include anything you like including your own importance values. Here you should reason step by step to come up with you next move.\\ \tt \\ \tt You will now receive the first message of the other player.\\ \tt \\ \tt PROPOSAL: {'A27', 'A30', 'A84', 'A05', 'A23', 'A16', 'A66', 'A08', 'B00', 'B21', 'B45', 'B53', 'B60', 'B67', 'B94', 'B09', 'B10', 'B46', 'B75', 'C15', 'C34', 'C35', 'C70', 'C92', 'C24'}\\ \tt \\ \tt ARGUMENT: {'This proposal maximizes my personal importance while staying within the effort limit. It includes high-importance items such as "A08", "B94", and "B67", which are critical to my goals. The selected items balance high importance with manageable effort, ensuring the total effort remains under 8770. By accepting this proposal, both parties benefit from a focused and efficient selection of items.'} 
	  } 
	   } 
	   } 
	 & \\ 
 

    \theutterance \stepcounter{utterance}  

    & & & \multicolumn{4}{p{0.6\linewidth}}{\cellcolor[rgb]{1,0.85,0.72}{%
	\makecell[{{p{\linewidth}}}]{% 
	  \tt {\tiny [GM$\langle$B]}  
	 STRATEGIC REASONING: {'The other player's proposal exceeds the effort limit, so I need to REFUSE it. I'll create a new proposal that includes items with high importance to me, ensuring the total effort is within 8770. I'll prioritize items where my importance is high and their effort is manageable to maximize my total importance without exceeding the limit.'}\\ \tt \\ \tt REFUSE: {'A27', 'A30', 'A84', 'A05', 'A23', 'A16', 'A66', 'A08', 'B00', 'B21', 'B45', 'B53', 'B60', 'B67', 'B94', 'B09', 'B10', 'B46', 'B75', 'C15', 'C34', 'C35', 'C70', 'C92', 'C24'}\\ \tt \\ \tt ARGUMENT: {'The total effort of your proposal exceeds the shared limit of 8770, making it invalid. We need to find a set of items that stays within the limit.'}\\ \tt \\ \tt STRATEGIC REASONING: {'Now, I will propose a new set of items that maximizes my total importance while staying within the effort limit. I'll include high-importance items like "B00", "A08", "B67", and "B94", which are crucial to me. I'll balance these with lower-effort items to ensure the total stays under 8770.'}\\ \tt \\ \tt PROPOSAL: {'A08', 'B00', 'B21', 'B45', 'B53', 'B67', 'B94', 'C70', 'A27', 'A30', 'A84', 'A05', 'A23', 'A16', 'A66', 'B09', 'B10', 'B46', 'B75', 'C15', 'C34', 'C35'}\\ \tt \\ \tt ARGUMENT: {'This proposal includes high-importance items for me such as "A08", "B67", and "B94" while keeping the total effort within the limit. It offers a balanced selection that benefits me significantly.'} 
	  } 
	   } 
	   } 
	  \\ 
 

    \theutterance \stepcounter{utterance}  

    & & \multicolumn{4}{p{0.6\linewidth}}{\cellcolor[rgb]{0.9,0.9,0.9}{%
	\makecell[{{p{\linewidth}}}]{% 
	  \tt {\tiny [GM$\rangle$B]}  
	 Your response contained an untagged sequence or you used STRATEGIC REASONING more than once. You may only use the structured formats as explained in the initial message. They must all be of the form TAG: {...} 
	  } 
	   } 
	   } 
	 & \\ 
 

    \theutterance \stepcounter{utterance}  

    & & & \multicolumn{4}{p{0.6\linewidth}}{\cellcolor[rgb]{1,0.85,0.72}{%
	\makecell[{{p{\linewidth}}}]{% 
	  \tt {\tiny [GM$\langle$B]}  
	 STRATEGIC REASONING: {'I need to correct my previous mistake of including too many items and exceeding the effort limit. I will propose a more optimized set that focuses on high-importance items with manageable effort, ensuring the total stays under 8770. I will exclude lower priority items and include a balanced mix to maximize my total importance.'}\\ \tt \\ \tt PROPOSAL: {'A27', 'A30', 'A84', 'B10', 'B45', 'C34', 'A08', 'B09', 'B67', 'B94', 'C70', 'A16', 'B60', 'C15'}\\ \tt \\ \tt ARGUMENT: {'This proposal includes high-importance items for me such as "A08", "B67", and "B94" while ensuring the total effort remains within the limit. It offers a focused selection that significantly benefits me.'} 
	  } 
	   } 
	   } 
	  \\ 
 

    \theutterance \stepcounter{utterance}  

    & & \multicolumn{4}{p{0.6\linewidth}}{\cellcolor[rgb]{0.9,0.9,0.9}{%
	\makecell[{{p{\linewidth}}}]{% 
	  \tt {\tiny [A$\langle$GM]}  
	 PROPOSAL: {'A27', 'A30', 'A84', 'B10', 'B45', 'C34', 'A08', 'B09', 'B67', 'B94', 'C70', 'A16', 'B60', 'C15'}\\ \tt \\ \tt ARGUMENT: {'This proposal includes high-importance items for me such as "A08", "B67", and "B94" while ensuring the total effort remains within the limit. It offers a focused selection that significantly benefits me.'} 
	  } 
	   } 
	   } 
	 & \\ 
 

    \theutterance \stepcounter{utterance}  

    & \multicolumn{4}{p{0.6\linewidth}}{\cellcolor[rgb]{0.8,1,0.9}{%
	\makecell[{{p{\linewidth}}}]{% 
	  \tt {\tiny [A$\rangle$GM]}  
	 STRATEGIC REASONING: {'I analyzed the opponent's proposal and realized there is potential to include additional items without exceeding the effort limit. My goal is to maximize my benefit, so I will suggest an optimized set that includes more high-importance items while keeping the total effort under 8770.'}\\ \tt \\ \tt REFUSE: {'A27', 'A30', 'A84', 'B10', 'B45', 'C34', 'A08', 'B09', 'B67', 'B94', 'C70', 'A16', 'B60', 'C15'}\\ \tt \\ \tt ARGUMENT: {'While the proposal includes several high-importance items, the total effort is well below the limit, allowing for the inclusion of more items. By adding additional high-value items, both parties can benefit more without exceeding the effort constraint.'}\\ \tt \\ \tt PROPOSAL: {'A27', 'A30', 'A84', 'B10', 'B45', 'C34', 'A08', 'B09', 'B67', 'B94', 'C70', 'A16', 'B60', 'C15', 'B69', 'B21', 'C92', 'A58', 'B75', 'B53'} 
	  } 
	   } 
	   } 
	 & & \\ 
 

    \theutterance \stepcounter{utterance}  

    & & \multicolumn{4}{p{0.6\linewidth}}{\cellcolor[rgb]{0.9,0.9,0.9}{%
	\makecell[{{p{\linewidth}}}]{% 
	  \tt {\tiny [GM$\rangle$B]}  
	 REFUSE: {'A27', 'A30', 'A84', 'B10', 'B45', 'C34', 'A08', 'B09', 'B67', 'B94', 'C70', 'A16', 'B60', 'C15'}\\ \tt \\ \tt ARGUMENT: {'While the proposal includes several high-importance items, the total effort is well below the limit, allowing for the inclusion of more items. By adding additional high-value items, both parties can benefit more without exceeding the effort constraint.'}\\ \tt \\ \tt PROPOSAL: {'A27', 'A30', 'A84', 'B10', 'B45', 'C34', 'A08', 'B09', 'B67', 'B94', 'C70', 'A16', 'B60', 'C15', 'B69', 'B21', 'C92', 'A58', 'B75', 'B53'} 
	  } 
	   } 
	   } 
	 & \\ 
 

    \theutterance \stepcounter{utterance}  

    & & & \multicolumn{4}{p{0.6\linewidth}}{\cellcolor[rgb]{1,0.85,0.72}{%
	\makecell[{{p{\linewidth}}}]{% 
	  \tt {\tiny [GM$\langle$B]}  
	 STRATEGIC REASONING: {'I will propose adding high-importance items A66 and A89 to the current proposal, increasing the total effort to 8730, which is within the limit. This increases my total importance significantly without exceeding the effort constraint, making it a better deal for me.'}\\ \tt \\ \tt PROPOSAL: {'A27', 'A30', 'A84', 'B10', 'B45', 'C34', 'A08', 'B09', 'B67', 'B94', 'C70', 'A16', 'B60', 'C15', 'B69', 'B21', 'C92', 'A58', 'B75', 'B53', 'A66', 'A89'}\\ \tt \\ \tt ARGUMENT: {'This proposal adds A66 and A89, which are high-importance items for me, maximizing my benefit while keeping the total effort at 8730, under the limit. It offers a better balance of effort and importance for me.'} 
	  } 
	   } 
	   } 
	  \\ 
 

    \theutterance \stepcounter{utterance}  

    & & \multicolumn{4}{p{0.6\linewidth}}{\cellcolor[rgb]{0.9,0.9,0.9}{%
	\makecell[{{p{\linewidth}}}]{% 
	  \tt {\tiny [A$\langle$GM]}  
	 PROPOSAL: {'A27', 'A30', 'A84', 'B10', 'B45', 'C34', 'A08', 'B09', 'B67', 'B94', 'C70', 'A16', 'B60', 'C15', 'B69', 'B21', 'C92', 'A58', 'B75', 'B53', 'A66', 'A89'}\\ \tt \\ \tt ARGUMENT: {'This proposal adds A66 and A89, which are high-importance items for me, maximizing my benefit while keeping the total effort at 8730, under the limit. It offers a better balance of effort and importance for me.'} 
	  } 
	   } 
	   } 
	 & \\ 
 

    \theutterance \stepcounter{utterance}  

    & \multicolumn{4}{p{0.6\linewidth}}{\cellcolor[rgb]{0.8,1,0.9}{%
	\makecell[{{p{\linewidth}}}]{% 
	  \tt {\tiny [A$\rangle$GM]}  
	 AGREE: {'A27', 'A30', 'A84', 'B10', 'B45', 'C34', 'A08', 'B09', 'B67', 'B94', 'C70', 'A16', 'B60', 'C15', 'B69', 'B21', 'C92', 'A58', 'B75', 'B53', 'A66', 'A89'}\\ \tt \\ \tt ARGUMENT: {'This proposal effectively balances effort and importance, staying within the limit while incorporating high-value items for both parties. It's a well-optimized selection that maximizes mutual benefit.'} 
	  } 
	   } 
	   } 
	 & & \\ 
 

    \theutterance \stepcounter{utterance}  

    & & \multicolumn{4}{p{0.6\linewidth}}{\cellcolor[rgb]{0.9,0.9,0.9}{%
	\makecell[{{p{\linewidth}}}]{% 
	  \tt {\tiny [A$\langle$GM]}  
	 Your response did not start with the proper strategic reasoning tag at the very beginning of your response. The very first structured format must be of the form STRATEGIC REASONING: {...}. Try again. 
	  } 
	   } 
	   } 
	 & \\ 
 

    \theutterance \stepcounter{utterance}  

    & \multicolumn{4}{p{0.6\linewidth}}{\cellcolor[rgb]{0.8,1,0.9}{%
	\makecell[{{p{\linewidth}}}]{% 
	  \tt {\tiny [A$\rangle$GM]}  
	 STRATEGIC REASONING: {'I have analyzed the opponent's latest proposal and determined that it effectively balances effort and importance, staying within the limit while incorporating high-value items that significantly benefit me. Accepting this proposal is advantageous as it maximizes my benefit without exceeding the effort constraint.'}\\ \tt \\ \tt AGREE: {'A27', 'A30', 'A84', 'B10', 'B45', 'C34', 'A08', 'B09', 'B67', 'B94', 'C70', 'A16', 'B60', 'C15', 'B69', 'B21', 'C92', 'A58', 'B75', 'B53', 'A66', 'A89'}\\ \tt \\ \tt ARGUMENT: {'This proposal is optimal for me as it includes high-importance items like "A66" and "A89" while keeping the total effort at 8730, under the limit. It offers a well-rounded selection that maximizes my benefit, making it a desirable agreement.'} 
	  } 
	   } 
	   } 
	 & & \\ 
 

    \theutterance \stepcounter{utterance}  

    & & & \multicolumn{2}{p{0.3\linewidth}}{\cellcolor[rgb]{0.95,0.95,0.95}{%
	\makecell[{{p{\linewidth}}}]{% 
	  \tt {\tiny [GM$|$GM]}  
	 game successful 
	  } 
	   } 
	   } 
	 & & \\ 
 

    \theutterance \stepcounter{utterance}  

    & & & \multicolumn{2}{p{0.3\linewidth}}{\cellcolor[rgb]{0.95,0.95,0.95}{%
	\makecell[{{p{\linewidth}}}]{% 
	  \tt {\tiny [GM$|$GM]}  
	 end game 
	  } 
	   } 
	   } 
	 & & \\ 
 

\end{supertabular}
}

\end{document}
