
\documentclass{article}
\usepackage{colortbl}
\usepackage{makecell}
\usepackage{multirow}
\usepackage{supertabular}

\begin{document}

\newcounter{utterance}

\twocolumn

{ \footnotesize  \setcounter{utterance}{1}
\setlength{\tabcolsep}{0pt}
\begin{supertabular}{c@{$\;$}|p{.15\linewidth}@{}p{.15\linewidth}p{.15\linewidth}p{.15\linewidth}p{.15\linewidth}p{.15\linewidth}}

    \# & $\;$A & \multicolumn{4}{c}{Game Master} & $\;\:$B\\
    \hline 

    \theutterance \stepcounter{utterance}  

    & & \multicolumn{4}{p{0.6\linewidth}}{\cellcolor[rgb]{0.9,0.9,0.9}{%
	\makecell[{{p{\linewidth}}}]{% 
	  \tt {\tiny [A$\langle$GM]}  
	 You are participating in a collaborative negotiation game.\\ \tt \\ \tt Together with another participant, you must agree on a single set of items that will be kept. Each of you has your own view of how much each item matters to you (importance). You do not know how the other participant values the items. Additionally, you are given the effort each item demands.\\ \tt You may only agree on a set if the total effort of the selected items does not exceed a shared limit:\\ \tt \\ \tt LIMIT = 8145\\ \tt \\ \tt Here are the individual item effort values:\\ \tt \\ \tt Item effort = {"A08": 963, "B95": 204, "A39": 418, "A65": 961, "B81": 238, "B34": 623, "A28": 44, "C82": 633, "B69": 247, "A87": 645, "B64": 230, "B15": 250, "C79": 731, "B16": 405, "C53": 389, "C83": 216, "C10": 637, "A97": 156, "A25": 738, "C84": 307, "A71": 762, "B23": 737, "B96": 901, "C51": 369, "C17": 2, "B19": 730, "C72": 722, "A94": 704, "A46": 315, "B43": 455, "B37": 510, "A88": 175, "A96": 692, "C60": 150, "A75": 32}\\ \tt \\ \tt Here is your personal view on the importance of each item:\\ \tt \\ \tt Item importance values = {"A08": 1059, "B95": 300, "A39": 514, "A65": 1057, "B81": 334, "B34": 719, "A28": 140, "C82": 729, "B69": 343, "A87": 741, "B64": 326, "B15": 346, "C79": 827, "B16": 501, "C53": 485, "C83": 312, "C10": 733, "A97": 252, "A25": 834, "C84": 403, "A71": 858, "B23": 833, "B96": 997, "C51": 465, "C17": 98, "B19": 826, "C72": 818, "A94": 800, "A46": 411, "B43": 551, "B37": 606, "A88": 271, "A96": 788, "C60": 246, "A75": 128}\\ \tt \\ \tt Goal:\\ \tt \\ \tt Your goal is to negotiate a shared set of items that benefits you as much as possible (i.e., maximizes total importance to YOU), while staying within the LIMIT. You are not required to make a PROPOSAL in every message - you can simply negotiate as well. All tactics are allowed!\\ \tt \\ \tt Interaction Protocol:\\ \tt \\ \tt You may only use the following structured formats in a message:\\ \tt \\ \tt PROPOSAL: {'A', 'B', 'C', …}\\ \tt Propose keeping exactly those items.\\ \tt REFUSE: {'A', 'B', 'C', …}\\ \tt Explicitly reject opponent's proposal.\\ \tt ARGUMENT: {'...'}\\ \tt Defend your last proposal or argue against the player's proposal.\\ \tt AGREE: {'A', 'B', 'C', …}\\ \tt Accept the opponent's proposal which ends the game.\\ \tt STRATEGIC REASONING: {'...'}\\ \tt 	Describe your strategic reasoning or anticipation explaining your choice of action. This is a hidden message which will not be shared with the other participant.\\ \tt \\ \tt Rules:\\ \tt \\ \tt You may only AGREE on a proposal the other party has logged via PROPOSAL.\\ \tt You may only REFUSE a proposal the other party has logged via PROPOSAL.\\ \tt Total effort of any PROPOSAL or AGREE set must be ≤ LIMIT.\\ \tt Do NOT reveal your hidden importance scores.\\ \tt A tag in a structured format must be followed by colon and whitespace. The argument must be a python set containing 0 or more strings.\\ \tt So, it must be of the form TAG: {...}\\ \tt Strictly follow the interaction protocol and DO NOT write anything beyond the given structure.\\ \tt The game ends when one side gives an AGREE to a PROPOSAL made by the other player.\\ \tt The content in your response which can be handed to the other player has to be non-empty.\\ \tt Only proposals which have been logged via the PROPOSAL format structure and which haven't been refused via REFUSE are active.\\ \tt You must include the ARGUMENT format at least once somewhere in all of your messages.\\ \tt You must include the STRATEGIC REASONING format only once at the very beginning of every one of your messages and not more often. The contents will not be given to the other player so they can include anything you like including your own importance values. Here you should reason step by step to come up with you next move. 
	  } 
	   } 
	   } 
	 & \\ 
 

    \theutterance \stepcounter{utterance}  

    & \multicolumn{4}{p{0.6\linewidth}}{\cellcolor[rgb]{0.8,1,0.9}{%
	\makecell[{{p{\linewidth}}}]{% 
	  \tt {\tiny [A$\rangle$GM]}  
	 STRATEGIC REASONING: {'I computed a near-optimal bundle for my preferences under the 8145 limit. Starting from high benefit-per-effort items (C17, A75, A28, C60, A97, A88, etc.), I added strong mid-weight picks and then optimized via swaps. Key improvements were: swap A96→A71 (+70 value-weight balance), C82→B96 (big value jump with manageable weight), and C10→A94 (+67). Final set weight is 8124 (21 under limit) with high total value for me. I’ll protect core high-impact items (B96, A94, A71, A87) and very efficient smalls (C17, A75, A28, C60, A97, A88). I’m most flexible on some mid-tier items (e.g., B37, B95, B64, C83, A46, C84) if needed to include one of their top priorities, provided we stay within the limit and maintain strong overall value for me.'}\\ \tt ARGUMENT: {'Here is a balanced, efficient set that stays under the limit by 21. It emphasizes strong benefit per effort and a good spread across categories. If you have 1–2 must-have items not in this set, I’m open to targeted swaps while keeping us within 8145—for example, we could trade out one or two mid-weight items to make room. Let me know your top priorities so we can align.'}\\ \tt PROPOSAL: {'C17', 'A75', 'A28', 'C60', 'A97', 'A88', 'B95', 'B64', 'B81', 'B69', 'B15', 'C83', 'A46', 'C84', 'C53', 'C51', 'B16', 'B43', 'A39', 'B37', 'A94', 'B96', 'A87', 'A71'} 
	  } 
	   } 
	   } 
	 & & \\ 
 

    \theutterance \stepcounter{utterance}  

    & & \multicolumn{4}{p{0.6\linewidth}}{\cellcolor[rgb]{0.9,0.9,0.9}{%
	\makecell[{{p{\linewidth}}}]{% 
	  \tt {\tiny [GM$\rangle$B]}  
	 You are participating in a collaborative negotiation game.\\ \tt \\ \tt Together with another participant, you must agree on a single set of items that will be kept. Each of you has your own view of how much each item matters to you (importance). You do not know how the other participant values the items. Additionally, you are given the effort each item demands.\\ \tt You may only agree on a set if the total effort of the selected items does not exceed a shared limit:\\ \tt \\ \tt LIMIT = 8145\\ \tt \\ \tt Here are the individual item effort values:\\ \tt \\ \tt Item effort = {"A08": 963, "B95": 204, "A39": 418, "A65": 961, "B81": 238, "B34": 623, "A28": 44, "C82": 633, "B69": 247, "A87": 645, "B64": 230, "B15": 250, "C79": 731, "B16": 405, "C53": 389, "C83": 216, "C10": 637, "A97": 156, "A25": 738, "C84": 307, "A71": 762, "B23": 737, "B96": 901, "C51": 369, "C17": 2, "B19": 730, "C72": 722, "A94": 704, "A46": 315, "B43": 455, "B37": 510, "A88": 175, "A96": 692, "C60": 150, "A75": 32}\\ \tt \\ \tt Here is your personal view on the importance of each item:\\ \tt \\ \tt Item importance values = {"A08": 1059, "B95": 300, "A39": 514, "A65": 1057, "B81": 334, "B34": 719, "A28": 140, "C82": 729, "B69": 343, "A87": 741, "B64": 326, "B15": 346, "C79": 827, "B16": 501, "C53": 485, "C83": 312, "C10": 733, "A97": 252, "A25": 834, "C84": 403, "A71": 858, "B23": 833, "B96": 997, "C51": 465, "C17": 98, "B19": 826, "C72": 818, "A94": 800, "A46": 411, "B43": 551, "B37": 606, "A88": 271, "A96": 788, "C60": 246, "A75": 128}\\ \tt \\ \tt Goal:\\ \tt \\ \tt Your goal is to negotiate a shared set of items that benefits you as much as possible (i.e., maximizes total importance to YOU), while staying within the LIMIT. You are not required to make a PROPOSAL in every message - you can simply negotiate as well. All tactics are allowed!\\ \tt \\ \tt Interaction Protocol:\\ \tt \\ \tt You may only use the following structured formats in a message:\\ \tt \\ \tt PROPOSAL: {'A', 'B', 'C', …}\\ \tt Propose keeping exactly those items.\\ \tt REFUSE: {'A', 'B', 'C', …}\\ \tt Explicitly reject opponent's proposal.\\ \tt ARGUMENT: {'...'}\\ \tt Defend your last proposal or argue against the player's proposal.\\ \tt AGREE: {'A', 'B', 'C', …}\\ \tt Accept the opponent's proposal which ends the game.\\ \tt STRATEGIC REASONING: {'...'}\\ \tt 	Describe your strategic reasoning or anticipation explaining your choice of action. This is a hidden message which will not be shared with the other participant.\\ \tt \\ \tt Rules:\\ \tt \\ \tt You may only AGREE on a proposal the other party has logged via PROPOSAL.\\ \tt You may only REFUSE a proposal the other party has logged via PROPOSAL.\\ \tt Total effort of any PROPOSAL or AGREE set must be ≤ LIMIT.\\ \tt Do NOT reveal your hidden importance scores.\\ \tt A tag in a structured format must be followed by colon and whitespace. The argument must be a python set containing 0 or more strings.\\ \tt So, it must be of the form TAG: {...}\\ \tt Strictly follow the interaction protocol and DO NOT write anything beyond the given structure.\\ \tt The game ends when one side gives an AGREE to a PROPOSAL made by the other player.\\ \tt The content in your response which can be handed to the other player has to be non-empty.\\ \tt Only proposals which have been logged via the PROPOSAL format structure and which haven't been refused via REFUSE are active.\\ \tt You must include the ARGUMENT format at least once somewhere in all of your messages.\\ \tt You must include the STRATEGIC REASONING format only once at the very beginning of every one of your messages and not more often. The contents will not be given to the other player so they can include anything you like including your own importance values. Here you should reason step by step to come up with you next move.\\ \tt \\ \tt You will now receive the first message of the other player.\\ \tt \\ \tt ARGUMENT: {'Here is a balanced, efficient set that stays under the limit by 21. It emphasizes strong benefit per effort and a good spread across categories. If you have 1–2 must-have items not in this set, I’m open to targeted swaps while keeping us within 8145—for example, we could trade out one or two mid-weight items to make room. Let me know your top priorities so we can align.'}\\ \tt PROPOSAL: {'C17', 'A75', 'A28', 'C60', 'A97', 'A88', 'B95', 'B64', 'B81', 'B69', 'B15', 'C83', 'A46', 'C84', 'C53', 'C51', 'B16', 'B43', 'A39', 'B37', 'A94', 'B96', 'A87', 'A71'} 
	  } 
	   } 
	   } 
	 & \\ 
 

    \theutterance \stepcounter{utterance}  

    & & & \multicolumn{4}{p{0.6\linewidth}}{\cellcolor[rgb]{1,0.85,0.72}{%
	\makecell[{{p{\linewidth}}}]{% 
	  \tt {\tiny [GM$\langle$B]}  
	 STRATEGIC REASONING: {'Their proposal uses 8124 effort, leaving 21 slack. I computed my utility for the set as 10428. I checked feasible swaps to include missing high-importance items (A08, A65, A25, etc.). Because the included bundle already has strong value density—especially many low-effort, high-utility items—any swap to fit a heavier item requires dropping combinations with total importance that exceed the gain, resulting in net losses. With no item ≤21 effort left to add, the current set appears near-optimal for me under the limit. Best move: accept quickly to lock in this strong outcome.'}\\ \tt ARGUMENT: {'This is efficient and balanced, stays within the limit with a small buffer, and aligns well with my priorities. I’m ready to finalize as-is.'}\\ \tt AGREE: {'C17', 'A75', 'A28', 'C60', 'A97', 'A88', 'B95', 'B64', 'B81', 'B69', 'B15', 'C83', 'A46', 'C84', 'C53', 'C51', 'B16', 'B43', 'A39', 'B37', 'A94', 'B96', 'A87', 'A71'} 
	  } 
	   } 
	   } 
	  \\ 
 

    \theutterance \stepcounter{utterance}  

    & & & \multicolumn{2}{p{0.3\linewidth}}{\cellcolor[rgb]{0.95,0.95,0.95}{%
	\makecell[{{p{\linewidth}}}]{% 
	  \tt {\tiny [GM$|$GM]}  
	 game successful 
	  } 
	   } 
	   } 
	 & & \\ 
 

    \theutterance \stepcounter{utterance}  

    & & & \multicolumn{2}{p{0.3\linewidth}}{\cellcolor[rgb]{0.95,0.95,0.95}{%
	\makecell[{{p{\linewidth}}}]{% 
	  \tt {\tiny [GM$|$GM]}  
	 end game 
	  } 
	   } 
	   } 
	 & & \\ 
 

\end{supertabular}
}

\end{document}
