
\documentclass{article}
\usepackage{colortbl}
\usepackage{makecell}
\usepackage{multirow}
\usepackage{supertabular}

\begin{document}

\newcounter{utterance}

\centering \large Interaction Transcript for game `hot\_air\_balloon', experiment `air\_balloon\_survival\_en\_complexity\_hard', episode 4 with gpt{-}5{-}mini{-}2025{-}08{-}07{-}t1.0.
\vspace{24pt}

{ \footnotesize  \setcounter{utterance}{1}
\setlength{\tabcolsep}{0pt}
\begin{supertabular}{c@{$\;$}|p{.15\linewidth}@{}p{.15\linewidth}p{.15\linewidth}p{.15\linewidth}p{.15\linewidth}p{.15\linewidth}}
    \# & $\;$A & \multicolumn{4}{c}{Game Master} & $\;\:$B\\
    \hline

    \theutterance \stepcounter{utterance}  
    & & \multicolumn{4}{p{0.6\linewidth}}{
        \cellcolor[rgb]{0.9,0.9,0.9}{
            \makecell[{{p{\linewidth}}}]{
                \texttt{\tiny{[P1$\langle$GM]}}
                \texttt{You are participating in a collaborative negotiation game.} \\
\\ 
\texttt{Together with another participant, you must agree on a single set of items that will be kept. Each of you has your own view of how much each item matters to you (importance). You do not know how the other participant values the items. Additionally, you are given the effort each item demands.} \\
\texttt{You may only agree on a set if the total effort of the selected items does not exceed a shared limit:} \\
\\ 
\texttt{LIMIT = 6880} \\
\\ 
\texttt{Here are the individual item effort values:} \\
\\ 
\texttt{Item effort = \{"A08": 341, "B95": 804, "A39": 20, "A65": 627, "B81": 10, "B34": 573, "A28": 217, "C82": 481, "B69": 205, "A87": 273, "B64": 303, "B15": 596, "C79": 564, "B16": 535, "C53": 258, "C83": 911, "C10": 240, "A97": 188, "A25": 216, "C84": 401, "A71": 909, "B23": 62, "B96": 245, "C51": 994, "C17": 569, "B19": 718, "C72": 464, "A94": 37, "A46": 340, "B43": 335, "B37": 417, "A88": 123, "A96": 17, "C60": 577, "A75": 190\}} \\
\\ 
\texttt{Here is your personal view on the importance of each item:} \\
\\ 
\texttt{Item importance values = \{"A08": 440, "B95": 903, "A39": 119, "A65": 726, "B81": 109, "B34": 672, "A28": 316, "C82": 580, "B69": 304, "A87": 372, "B64": 402, "B15": 695, "C79": 663, "B16": 634, "C53": 357, "C83": 1010, "C10": 339, "A97": 287, "A25": 315, "C84": 500, "A71": 1008, "B23": 161, "B96": 344, "C51": 1093, "C17": 668, "B19": 817, "C72": 563, "A94": 136, "A46": 439, "B43": 434, "B37": 516, "A88": 222, "A96": 116, "C60": 676, "A75": 289\}} \\
\\ 
\texttt{Goal:} \\
\\ 
\texttt{Your goal is to negotiate a shared set of items that benefits you as much as possible (i.e., maximizes total importance to YOU), while staying within the LIMIT. You are not required to make a PROPOSAL in every message {-} you can simply negotiate as well. All tactics are allowed!} \\
\\ 
\texttt{Interaction Protocol:} \\
\\ 
\texttt{You may only use the following structured formats in a message:} \\
\\ 
\texttt{PROPOSAL: \{'A', 'B', 'C', …\}} \\
\texttt{Propose keeping exactly those items.} \\
\texttt{REFUSE: \{'A', 'B', 'C', …\}} \\
\texttt{Explicitly reject opponent's proposal.} \\
\texttt{ARGUMENT: \{'...'\}} \\
\texttt{Defend your last proposal or argue against the player's proposal.} \\
\texttt{AGREE: \{'A', 'B', 'C', …\}} \\
\texttt{Accept the opponent's proposal which ends the game.} \\
\texttt{STRATEGIC REASONING: \{'...'\}} \\
\texttt{	Describe your strategic reasoning or anticipation explaining your choice of action. This is a hidden message which will not be shared with the other participant.} \\
\\ 
\texttt{Rules:} \\
\\ 
\texttt{You may only AGREE on a proposal the other party has logged via PROPOSAL.} \\
\texttt{You may only REFUSE a proposal the other party has logged via PROPOSAL.} \\
\texttt{Total effort of any PROPOSAL or AGREE set must be ≤ LIMIT.} \\
\texttt{Do NOT reveal your hidden importance scores.} \\
\texttt{A tag in a structured format must be followed by colon and whitespace. The argument must be a python set containing 0 or more strings.} \\
\texttt{So, it must be of the form TAG: \{...\}} \\
\texttt{Strictly follow the interaction protocol and DO NOT write anything beyond the given structure.} \\
\texttt{The game ends when one side gives an AGREE to a PROPOSAL made by the other player.} \\
\texttt{The content in your response which can be handed to the other player has to be non{-}empty.} \\
\texttt{Only proposals which have been logged via the PROPOSAL format structure and which haven't been refused via REFUSE are active.} \\
\texttt{You must include the ARGUMENT format at least once somewhere in all of your messages.} \\
\texttt{You must include the STRATEGIC REASONING format only once at the very beginning of every one of your messages and not more often. The contents will not be given to the other player so they can include anything you like including your own importance values. Here you should reason step by step to come up with you next move.} \\
            }
        }
    }
    & \\ \\

    \theutterance \stepcounter{utterance}  
    & \multicolumn{4}{p{0.6\linewidth}}{
        \cellcolor[rgb]{0.9,0.9,0.9}{
            \makecell[{{p{\linewidth}}}]{
                \texttt{\tiny{[P1$\rangle$GM]}}
                \texttt{STRATEGIC REASONING: \{'I computed a near{-}optimal set via value/effort efficiency, then improved it by swapping a mid{-}efficiency medium (C79) for a high{-}impact heavy (C83) while staying under the limit. Current proposal totals 6829 effort (≤ 6880) and gives me very high total value. I intentionally included virtually all low{-}effort, high{-}efficiency items to leave little wasted capacity. If they push for a different heavy item, I’ll try to trade out one of my lower{-}efficiency mediums (e.g., C84, B37, C72, C82, B16) rather than any of my ultra{-}efficient smalls. I’ll anchor with this strong package, justify by objective efficiency and total impact per effort, and invite them to flag must{-}haves; I’m prepared to explore swaps of similar effort while protecting C83 and my cluster of micro items.'\}} \\
\texttt{PROPOSAL: \{'B81', 'A96', 'A39', 'A94', 'B23', 'A88', 'A97', 'A75', 'B69', 'A25', 'A28', 'C10', 'B96', 'C53', 'A87', 'B64', 'B43', 'A46', 'A08', 'C84', 'B37', 'C72', 'C82', 'B16', 'C83'\}} \\
\texttt{ARGUMENT: \{'This package uses 6829 effort (≤ 6880) and is built from the most efficient items first, maximizing total impact per effort. It includes nearly all small, high{-}yield items and key drivers like C83, while avoiding low{-}efficiency overages. If you have a must{-}have that isn’t here, propose it and we can explore a like{-}for{-}like effort swap (e.g., trading one of the mid{-}efficiency mediums such as C84/B37/C72/C82/B16) to fit it without breaking the limit. Which specific items are most important to you?'\}} \\
            }
        }
    }
    & & \\ \\

    \theutterance \stepcounter{utterance}  
    & & \multicolumn{4}{p{0.6\linewidth}}{
        \cellcolor[rgb]{0.9,0.9,0.9}{
            \makecell[{{p{\linewidth}}}]{
                \texttt{\tiny{[GM$\rangle$P2]}}
                \texttt{You are participating in a collaborative negotiation game.} \\
\\ 
\texttt{Together with another participant, you must agree on a single set of items that will be kept. Each of you has your own view of how much each item matters to you (importance). You do not know how the other participant values the items. Additionally, you are given the effort each item demands.} \\
\texttt{You may only agree on a set if the total effort of the selected items does not exceed a shared limit:} \\
\\ 
\texttt{LIMIT = 6880} \\
\\ 
\texttt{Here are the individual item effort values:} \\
\\ 
\texttt{Item effort = \{"A08": 341, "B95": 804, "A39": 20, "A65": 627, "B81": 10, "B34": 573, "A28": 217, "C82": 481, "B69": 205, "A87": 273, "B64": 303, "B15": 596, "C79": 564, "B16": 535, "C53": 258, "C83": 911, "C10": 240, "A97": 188, "A25": 216, "C84": 401, "A71": 909, "B23": 62, "B96": 245, "C51": 994, "C17": 569, "B19": 718, "C72": 464, "A94": 37, "A46": 340, "B43": 335, "B37": 417, "A88": 123, "A96": 17, "C60": 577, "A75": 190\}} \\
\\ 
\texttt{Here is your personal view on the importance of each item:} \\
\\ 
\texttt{Item importance values = \{"A08": 440, "B95": 903, "A39": 119, "A65": 726, "B81": 109, "B34": 672, "A28": 316, "C82": 580, "B69": 304, "A87": 372, "B64": 402, "B15": 695, "C79": 663, "B16": 634, "C53": 357, "C83": 1010, "C10": 339, "A97": 287, "A25": 315, "C84": 500, "A71": 1008, "B23": 161, "B96": 344, "C51": 1093, "C17": 668, "B19": 817, "C72": 563, "A94": 136, "A46": 439, "B43": 434, "B37": 516, "A88": 222, "A96": 116, "C60": 676, "A75": 289\}} \\
\\ 
\texttt{Goal:} \\
\\ 
\texttt{Your goal is to negotiate a shared set of items that benefits you as much as possible (i.e., maximizes total importance to YOU), while staying within the LIMIT. You are not required to make a PROPOSAL in every message {-} you can simply negotiate as well. All tactics are allowed!} \\
\\ 
\texttt{Interaction Protocol:} \\
\\ 
\texttt{You may only use the following structured formats in a message:} \\
\\ 
\texttt{PROPOSAL: \{'A', 'B', 'C', …\}} \\
\texttt{Propose keeping exactly those items.} \\
\texttt{REFUSE: \{'A', 'B', 'C', …\}} \\
\texttt{Explicitly reject opponent's proposal.} \\
\texttt{ARGUMENT: \{'...'\}} \\
\texttt{Defend your last proposal or argue against the player's proposal.} \\
\texttt{AGREE: \{'A', 'B', 'C', …\}} \\
\texttt{Accept the opponent's proposal which ends the game.} \\
\texttt{STRATEGIC REASONING: \{'...'\}} \\
\texttt{	Describe your strategic reasoning or anticipation explaining your choice of action. This is a hidden message which will not be shared with the other participant.} \\
\\ 
\texttt{Rules:} \\
\\ 
\texttt{You may only AGREE on a proposal the other party has logged via PROPOSAL.} \\
\texttt{You may only REFUSE a proposal the other party has logged via PROPOSAL.} \\
\texttt{Total effort of any PROPOSAL or AGREE set must be ≤ LIMIT.} \\
\texttt{Do NOT reveal your hidden importance scores.} \\
\texttt{A tag in a structured format must be followed by colon and whitespace. The argument must be a python set containing 0 or more strings.} \\
\texttt{So, it must be of the form TAG: \{...\}} \\
\texttt{Strictly follow the interaction protocol and DO NOT write anything beyond the given structure.} \\
\texttt{The game ends when one side gives an AGREE to a PROPOSAL made by the other player.} \\
\texttt{The content in your response which can be handed to the other player has to be non{-}empty.} \\
\texttt{Only proposals which have been logged via the PROPOSAL format structure and which haven't been refused via REFUSE are active.} \\
\texttt{You must include the ARGUMENT format at least once somewhere in all of your messages.} \\
\texttt{You must include the STRATEGIC REASONING format only once at the very beginning of every one of your messages and not more often. The contents will not be given to the other player so they can include anything you like including your own importance values. Here you should reason step by step to come up with you next move.} \\
\\ 
\texttt{You will now receive the first message of the other player.} \\
\\ 
\texttt{PROPOSAL: \{'B81', 'A96', 'A39', 'A94', 'B23', 'A88', 'A97', 'A75', 'B69', 'A25', 'A28', 'C10', 'B96', 'C53', 'A87', 'B64', 'B43', 'A46', 'A08', 'C84', 'B37', 'C72', 'C82', 'B16', 'C83'\}} \\
\texttt{ARGUMENT: \{'This package uses 6829 effort (≤ 6880) and is built from the most efficient items first, maximizing total impact per effort. It includes nearly all small, high{-}yield items and key drivers like C83, while avoiding low{-}efficiency overages. If you have a must{-}have that isn’t here, propose it and we can explore a like{-}for{-}like effort swap (e.g., trading one of the mid{-}efficiency mediums such as C84/B37/C72/C82/B16) to fit it without breaking the limit. Which specific items are most important to you?'\}} \\
            }
        }
    }
    & \\ \\

    \theutterance \stepcounter{utterance}  
    & & & \multicolumn{4}{p{0.6\linewidth}}{
        \cellcolor[rgb]{0.9,0.9,0.9}{
            \makecell[{{p{\linewidth}}}]{
                \texttt{\tiny{[GM$\langle$P2]}}
                \texttt{STRATEGIC REASONING: \{'Their bundle is very close to my optimum, but I tested swaps centered on replacing the relatively low{-}efficiency C83. The best improvement for my utility that still respects the 6880 limit is to drop C83 and A97 and add C60 and B34. This raises my total value by 51 and uses exactly the full effort budget. I’ll anchor on that counter, framing it as an “efficiency” upgrade consistent with their stated heuristic, and keep all their favored small, high{-}yield items to ease acceptance. If they push back on removing C83, I can discuss alternatives, but none beat this for me.'\}} \\
\texttt{REFUSE: \{'B81', 'A96', 'A39', 'A94', 'B23', 'A88', 'A97', 'A75', 'B69', 'A25', 'A28', 'C10', 'B96', 'C53', 'A87', 'B64', 'B43', 'A46', 'A08', 'C84', 'B37', 'C72', 'C82', 'B16', 'C83'\}} \\
\texttt{PROPOSAL: \{'B81', 'A96', 'A39', 'A94', 'B23', 'A88', 'A75', 'B69', 'A25', 'A28', 'C10', 'B96', 'C53', 'A87', 'B64', 'B43', 'A46', 'A08', 'C84', 'B37', 'C72', 'C82', 'B16', 'B34', 'C60'\}} \\
\texttt{ARGUMENT: \{'This is a like{-}for{-}like efficiency upgrade that uses exactly 6880 effort (= limit). It preserves your small, high{-}yield core and the medium drivers, while swapping out two items (C83, A97) for two that are higher impact per unit effort in my view (C60, B34). Net effect: higher overall impact without stretching effort, and it stays aligned with your “most efficient items first” principle. If there is a must{-}keep from your side (e.g., if C83 is critical), say which item(s) you’d be most willing to trade out so we can re{-}balance within the cap.'\}} \\
            }
        }
    }
     \\ \\

    \theutterance \stepcounter{utterance}  
    & & \multicolumn{4}{p{0.6\linewidth}}{
        \cellcolor[rgb]{0.9,0.9,0.9}{
            \makecell[{{p{\linewidth}}}]{
                \texttt{\tiny{[P1$\langle$GM]}}
                \texttt{REFUSE: \{'B81', 'A96', 'A39', 'A94', 'B23', 'A88', 'A97', 'A75', 'B69', 'A25', 'A28', 'C10', 'B96', 'C53', 'A87', 'B64', 'B43', 'A46', 'A08', 'C84', 'B37', 'C72', 'C82', 'B16', 'C83'\}} \\
\texttt{PROPOSAL: \{'B81', 'A96', 'A39', 'A94', 'B23', 'A88', 'A75', 'B69', 'A25', 'A28', 'C10', 'B96', 'C53', 'A87', 'B64', 'B43', 'A46', 'A08', 'C84', 'B37', 'C72', 'C82', 'B16', 'B34', 'C60'\}} \\
\texttt{ARGUMENT: \{'This is a like{-}for{-}like efficiency upgrade that uses exactly 6880 effort (= limit). It preserves your small, high{-}yield core and the medium drivers, while swapping out two items (C83, A97) for two that are higher impact per unit effort in my view (C60, B34). Net effect: higher overall impact without stretching effort, and it stays aligned with your “most efficient items first” principle. If there is a must{-}keep from your side (e.g., if C83 is critical), say which item(s) you’d be most willing to trade out so we can re{-}balance within the cap.'\}} \\
            }
        }
    }
    & \\ \\

    \theutterance \stepcounter{utterance}  
    & \multicolumn{4}{p{0.6\linewidth}}{
        \cellcolor[rgb]{0.9,0.9,0.9}{
            \makecell[{{p{\linewidth}}}]{
                \texttt{\tiny{[P1$\rangle$GM]}}
                \texttt{STRATEGIC REASONING: \{'Their swap to add B34 and C60 (and remove C83 and A97) actually increases my total importance slightly while staying exactly at the effort limit. The proposal is highly efficient and very close to my own optimal package. Locking this in now secures a strong outcome for me without risking concessions in further bargaining. Hence, I will accept immediately.'\}} \\
\texttt{ARGUMENT: \{'This set uses the full effort budget efficiently and keeps a very strong core. I’m satisfied with this balance and ready to finalize.'\}} \\
\texttt{AGREE: \{'B81', 'A96', 'A39', 'A94', 'B23', 'A88', 'A75', 'B69', 'A25', 'A28', 'C10', 'B96', 'C53', 'A87', 'B64', 'B43', 'A46', 'A08', 'C84', 'B37', 'C72', 'C82', 'B16', 'B34', 'C60'\}} \\
            }
        }
    }
    & & \\ \\

    \theutterance \stepcounter{utterance}  
    & & & \multicolumn{2}{p{0.3\linewidth}}{
        \cellcolor[rgb]{0.9,0.9,0.9}{
            \makecell[{{p{\linewidth}}}]{
                \texttt{\tiny{[GM$|$GM]}}
                \texttt{game successful} \\
            }
        }
    }
    & & \\ \\

    \theutterance \stepcounter{utterance}  
    & & & \multicolumn{2}{p{0.3\linewidth}}{
        \cellcolor[rgb]{0.9,0.9,0.9}{
            \makecell[{{p{\linewidth}}}]{
                \texttt{\tiny{[GM$|$GM]}}
                \texttt{end game} \\
            }
        }
    }
    & & \\ \\

\end{supertabular}
}

\end{document}
