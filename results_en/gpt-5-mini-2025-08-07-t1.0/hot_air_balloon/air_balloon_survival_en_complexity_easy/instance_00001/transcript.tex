
\documentclass{article}
\usepackage{colortbl}
\usepackage{makecell}
\usepackage{multirow}
\usepackage{supertabular}

\begin{document}

\newcounter{utterance}

\twocolumn

{ \footnotesize  \setcounter{utterance}{1}
\setlength{\tabcolsep}{0pt}
\begin{supertabular}{c@{$\;$}|p{.15\linewidth}@{}p{.15\linewidth}p{.15\linewidth}p{.15\linewidth}p{.15\linewidth}p{.15\linewidth}}

    \# & $\;$A & \multicolumn{4}{c}{Game Master} & $\;\:$B\\
    \hline 

    \theutterance \stepcounter{utterance}  

    & & \multicolumn{4}{p{0.6\linewidth}}{\cellcolor[rgb]{0.9,0.9,0.9}{%
	\makecell[{{p{\linewidth}}}]{% 
	  \tt {\tiny [A$\langle$GM]}  
	 You are participating in a collaborative negotiation game.\\ \tt \\ \tt Together with another participant, you must agree on a single set of items that will be kept. Each of you has your own view of how much each item matters to you (importance). You do not know how the other participant values the items. Additionally, you are given the effort each item demands.\\ \tt You may only agree on a set if the total effort of the selected items does not exceed a shared limit:\\ \tt \\ \tt LIMIT = 8170\\ \tt \\ \tt Here are the individual item effort values:\\ \tt \\ \tt Item effort = {"B21": 594, "B00": 551, "A08": 634, "A27": 204, "B94": 557, "B53": 440, "A23": 734, "C65": 893, "A89": 246, "B67": 592, "B45": 143, "C34": 568, "C12": 472, "A05": 401, "B60": 729, "C92": 201, "B69": 85, "A84": 641, "C06": 958, "B12": 80, "A16": 157, "B85": 806, "C35": 684, "B10": 59, "C15": 31, "B09": 765, "B16": 416, "C81": 392, "A58": 428, "C24": 699, "A30": 141, "A66": 606, "B75": 612, "B46": 133, "C70": 689}\\ \tt \\ \tt Here is your personal view on the importance of each item:\\ \tt \\ \tt Item importance values = {"B21": 138, "B00": 583, "A08": 868, "A27": 822, "B94": 783, "B53": 65, "A23": 262, "C65": 121, "A89": 508, "B67": 780, "B45": 461, "C34": 484, "C12": 668, "A05": 389, "B60": 808, "C92": 215, "B69": 97, "A84": 500, "C06": 30, "B12": 915, "A16": 856, "B85": 400, "C35": 444, "B10": 623, "C15": 781, "B09": 786, "B16": 3, "C81": 713, "A58": 457, "C24": 273, "A30": 739, "A66": 822, "B75": 235, "B46": 606, "C70": 105}\\ \tt \\ \tt Goal:\\ \tt \\ \tt Your goal is to negotiate a shared set of items that benefits you as much as possible (i.e., maximizes total importance to YOU), while staying within the LIMIT. You are not required to make a PROPOSAL in every message - you can simply negotiate as well. All tactics are allowed!\\ \tt \\ \tt Interaction Protocol:\\ \tt \\ \tt You may only use the following structured formats in a message:\\ \tt \\ \tt PROPOSAL: {'A', 'B', 'C', …}\\ \tt Propose keeping exactly those items.\\ \tt REFUSE: {'A', 'B', 'C', …}\\ \tt Explicitly reject opponent's proposal.\\ \tt ARGUMENT: {'...'}\\ \tt Defend your last proposal or argue against the player's proposal.\\ \tt AGREE: {'A', 'B', 'C', …}\\ \tt Accept the opponent's proposal which ends the game.\\ \tt STRATEGIC REASONING: {'...'}\\ \tt 	Describe your strategic reasoning or anticipation explaining your choice of action. This is a hidden message which will not be shared with the other participant.\\ \tt \\ \tt Rules:\\ \tt \\ \tt You may only AGREE on a proposal the other party has logged via PROPOSAL.\\ \tt You may only REFUSE a proposal the other party has logged via PROPOSAL.\\ \tt Total effort of any PROPOSAL or AGREE set must be ≤ LIMIT.\\ \tt Do NOT reveal your hidden importance scores.\\ \tt A tag in a structured format must be followed by colon and whitespace. The argument must be a python set containing 0 or more strings.\\ \tt So, it must be of the form TAG: {...}\\ \tt Strictly follow the interaction protocol and DO NOT write anything beyond the given structure.\\ \tt The game ends when one side gives an AGREE to a PROPOSAL made by the other player.\\ \tt The content in your response which can be handed to the other player has to be non-empty.\\ \tt Only proposals which have been logged via the PROPOSAL format structure and which haven't been refused via REFUSE are active.\\ \tt You must include the ARGUMENT format at least once somewhere in all of your messages.\\ \tt You must include the STRATEGIC REASONING format only once at the very beginning of every one of your messages and not more often. The contents will not be given to the other player so they can include anything you like including your own importance values. Here you should reason step by step to come up with you next move. 
	  } 
	   } 
	   } 
	 & \\ 
 

    \theutterance \stepcounter{utterance}  

    & \multicolumn{4}{p{0.6\linewidth}}{\cellcolor[rgb]{0.8,1,0.9}{%
	\makecell[{{p{\linewidth}}}]{% 
	  \tt {\tiny [A$\rangle$GM]}  
	 STRATEGIC REASONING: {'Maximize my importance under 8170 using high value-per-effort items. Start with ultra-efficient small items (C15, B12, B10, A16, A30, A27, B46). Add strong midweights (A89, C81, C12, B94, A66, A08, B67, B00, B09, A58, C92, B69). Initially I considered also B60, A84, B45. I found a better mix by dropping B60 (729/808) and using the freed capacity to add A05 (401/389) and C34 (568/484), which increases my total importance. Final proposed set has total effort 8087 (<=8170) and total importance to me 14456. The set: {C15, B12, B10, A16, A30, A27, B46, A89, C81, C12, B94, A66, A08, B67, B00, B09, A58, C92, B69, A84, B45, A05, C34}. I will defend this as efficient and balanced and signal flexibility around a few midweight items (e.g., B00, A58, C92, B69) if they need swaps, while trying to keep my core high-value items intact.'}\\ \tt PROPOSAL: {'C15', 'B12', 'B10', 'A16', 'A30', 'A27', 'B46', 'A89', 'C81', 'C12', 'B94', 'A66', 'A08', 'B67', 'B00', 'B09', 'A58', 'C92', 'B69', 'A84', 'B45', 'A05', 'C34'}\\ \tt ARGUMENT: {'This package focuses on many low-effort, high-benefit items and stays well under the limit (total effort 8087 ≤ 8170), leaving a small buffer for any minor swap you might need. It avoids very heavy, low-yield picks and includes a balanced spread across categories. If you have a couple of must-have items not present, say which they are, and I can consider swapping out one of the midweight pieces (e.g., B00, A58, C92, or B69) to fit them while keeping us under the cap.'} 
	  } 
	   } 
	   } 
	 & & \\ 
 

    \theutterance \stepcounter{utterance}  

    & & \multicolumn{4}{p{0.6\linewidth}}{\cellcolor[rgb]{0.9,0.9,0.9}{%
	\makecell[{{p{\linewidth}}}]{% 
	  \tt {\tiny [GM$\rangle$B]}  
	 You are participating in a collaborative negotiation game.\\ \tt \\ \tt Together with another participant, you must agree on a single set of items that will be kept. Each of you has your own view of how much each item matters to you (importance). You do not know how the other participant values the items. Additionally, you are given the effort each item demands.\\ \tt You may only agree on a set if the total effort of the selected items does not exceed a shared limit:\\ \tt \\ \tt LIMIT = 8170\\ \tt \\ \tt Here are the individual item effort values:\\ \tt \\ \tt Item effort = {"B21": 594, "B00": 551, "A08": 634, "A27": 204, "B94": 557, "B53": 440, "A23": 734, "C65": 893, "A89": 246, "B67": 592, "B45": 143, "C34": 568, "C12": 472, "A05": 401, "B60": 729, "C92": 201, "B69": 85, "A84": 641, "C06": 958, "B12": 80, "A16": 157, "B85": 806, "C35": 684, "B10": 59, "C15": 31, "B09": 765, "B16": 416, "C81": 392, "A58": 428, "C24": 699, "A30": 141, "A66": 606, "B75": 612, "B46": 133, "C70": 689}\\ \tt \\ \tt Here is your personal view on the importance of each item:\\ \tt \\ \tt Item importance values = {"B21": 138, "B00": 583, "A08": 868, "A27": 822, "B94": 783, "B53": 65, "A23": 262, "C65": 121, "A89": 508, "B67": 780, "B45": 461, "C34": 484, "C12": 668, "A05": 389, "B60": 808, "C92": 215, "B69": 97, "A84": 500, "C06": 30, "B12": 915, "A16": 856, "B85": 400, "C35": 444, "B10": 623, "C15": 781, "B09": 786, "B16": 3, "C81": 713, "A58": 457, "C24": 273, "A30": 739, "A66": 822, "B75": 235, "B46": 606, "C70": 105}\\ \tt \\ \tt Goal:\\ \tt \\ \tt Your goal is to negotiate a shared set of items that benefits you as much as possible (i.e., maximizes total importance to YOU), while staying within the LIMIT. You are not required to make a PROPOSAL in every message - you can simply negotiate as well. All tactics are allowed!\\ \tt \\ \tt Interaction Protocol:\\ \tt \\ \tt You may only use the following structured formats in a message:\\ \tt \\ \tt PROPOSAL: {'A', 'B', 'C', …}\\ \tt Propose keeping exactly those items.\\ \tt REFUSE: {'A', 'B', 'C', …}\\ \tt Explicitly reject opponent's proposal.\\ \tt ARGUMENT: {'...'}\\ \tt Defend your last proposal or argue against the player's proposal.\\ \tt AGREE: {'A', 'B', 'C', …}\\ \tt Accept the opponent's proposal which ends the game.\\ \tt STRATEGIC REASONING: {'...'}\\ \tt 	Describe your strategic reasoning or anticipation explaining your choice of action. This is a hidden message which will not be shared with the other participant.\\ \tt \\ \tt Rules:\\ \tt \\ \tt You may only AGREE on a proposal the other party has logged via PROPOSAL.\\ \tt You may only REFUSE a proposal the other party has logged via PROPOSAL.\\ \tt Total effort of any PROPOSAL or AGREE set must be ≤ LIMIT.\\ \tt Do NOT reveal your hidden importance scores.\\ \tt A tag in a structured format must be followed by colon and whitespace. The argument must be a python set containing 0 or more strings.\\ \tt So, it must be of the form TAG: {...}\\ \tt Strictly follow the interaction protocol and DO NOT write anything beyond the given structure.\\ \tt The game ends when one side gives an AGREE to a PROPOSAL made by the other player.\\ \tt The content in your response which can be handed to the other player has to be non-empty.\\ \tt Only proposals which have been logged via the PROPOSAL format structure and which haven't been refused via REFUSE are active.\\ \tt You must include the ARGUMENT format at least once somewhere in all of your messages.\\ \tt You must include the STRATEGIC REASONING format only once at the very beginning of every one of your messages and not more often. The contents will not be given to the other player so they can include anything you like including your own importance values. Here you should reason step by step to come up with you next move.\\ \tt \\ \tt You will now receive the first message of the other player.\\ \tt \\ \tt PROPOSAL: {'C15', 'B12', 'B10', 'A16', 'A30', 'A27', 'B46', 'A89', 'C81', 'C12', 'B94', 'A66', 'A08', 'B67', 'B00', 'B09', 'A58', 'C92', 'B69', 'A84', 'B45', 'A05', 'C34'}\\ \tt ARGUMENT: {'This package focuses on many low-effort, high-benefit items and stays well under the limit (total effort 8087 ≤ 8170), leaving a small buffer for any minor swap you might need. It avoids very heavy, low-yield picks and includes a balanced spread across categories. If you have a couple of must-have items not present, say which they are, and I can consider swapping out one of the midweight pieces (e.g., B00, A58, C92, or B69) to fit them while keeping us under the cap.'} 
	  } 
	   } 
	   } 
	 & \\ 
 

    \theutterance \stepcounter{utterance}  

    & & & \multicolumn{4}{p{0.6\linewidth}}{\cellcolor[rgb]{1,0.85,0.72}{%
	\makecell[{{p{\linewidth}}}]{% 
	  \tt {\tiny [GM$\langle$B]}  
	 STRATEGIC REASONING: {'Their proposal already gives me very high utility. The only strong omission for me is B60 (importance 808). I computed swaps to fit B60 under the cap. Best utility gain for me is replacing A05 + C92 + B69 with B60 (+107), but they explicitly offered A58/C92/B69 as flexible. Using their suggested levers, swapping out A58 + C92 + B69 for B60 still increases my utility (+39) and should be more acceptable to them. It keeps most of their structure intact and leverages items they flagged as swappable. If they push back on removing A58, I’ll counter with the A05 alternative which still fits and improves my outcome further.'}\\ \tt REFUSE: {'C15', 'B12', 'B10', 'A16', 'A30', 'A27', 'B46', 'A89', 'C81', 'C12', 'B94', 'A66', 'A08', 'B67', 'B00', 'B09', 'A58', 'C92', 'B69', 'A84', 'B45', 'A05', 'C34'}\\ \tt PROPOSAL: {'A16', 'B45', 'C12', 'B12', 'A84', 'A27', 'A08', 'C81', 'A66', 'B46', 'B10', 'B09', 'C15', 'B67', 'A30', 'A05', 'B94', 'A89', 'B60', 'B00', 'C34'}\\ \tt ARGUMENT: {'I kept your package largely intact and made a single consolidated swap to include B60. To fit it under the cap, I removed the smallest-effort items you identified as flexible (B69 and C92) plus A58, and added B60. Effort math: 8087 − (85 + 201 + 428) + 729 = 8102 ≤ 8170, leaving a buffer of 68. This keeps the structure you proposed while upgrading one slot into a single strong item. If A58 is important to you, I can instead remove A05 (keeping B60 in) and we’ll still stay under the limit—let me know your preference.'} 
	  } 
	   } 
	   } 
	  \\ 
 

    \theutterance \stepcounter{utterance}  

    & & \multicolumn{4}{p{0.6\linewidth}}{\cellcolor[rgb]{0.9,0.9,0.9}{%
	\makecell[{{p{\linewidth}}}]{% 
	  \tt {\tiny [A$\langle$GM]}  
	 REFUSE: {'C15', 'B12', 'B10', 'A16', 'A30', 'A27', 'B46', 'A89', 'C81', 'C12', 'B94', 'A66', 'A08', 'B67', 'B00', 'B09', 'A58', 'C92', 'B69', 'A84', 'B45', 'A05', 'C34'}\\ \tt PROPOSAL: {'A16', 'B45', 'C12', 'B12', 'A84', 'A27', 'A08', 'C81', 'A66', 'B46', 'B10', 'B09', 'C15', 'B67', 'A30', 'A05', 'B94', 'A89', 'B60', 'B00', 'C34'}\\ \tt ARGUMENT: {'I kept your package largely intact and made a single consolidated swap to include B60. To fit it under the cap, I removed the smallest-effort items you identified as flexible (B69 and C92) plus A58, and added B60. Effort math: 8087 − (85 + 201 + 428) + 729 = 8102 ≤ 8170, leaving a buffer of 68. This keeps the structure you proposed while upgrading one slot into a single strong item. If A58 is important to you, I can instead remove A05 (keeping B60 in) and we’ll still stay under the limit—let me know your preference.'} 
	  } 
	   } 
	   } 
	 & \\ 
 

    \theutterance \stepcounter{utterance}  

    & \multicolumn{4}{p{0.6\linewidth}}{\cellcolor[rgb]{0.8,1,0.9}{%
	\makecell[{{p{\linewidth}}}]{% 
	  \tt {\tiny [A$\rangle$GM]}  
	 STRATEGIC REASONING: {'Their proposal gives me 14495 utility (slightly better than my original 14456). I checked for improvements within the 8170 limit. The best upgrade for me is to swap out C34 (568 effort, 484 utility) for A58 (428, 457) and re-add C92 (201, 215). New total effort: 8102 − 568 + 428 + 201 = 8163. New utility: 14495 − 484 + 457 + 215 = 14683, which strictly improves my outcome while keeping their requested B60. If they push back on losing C34, my fallback is to accept their package with the single swap A05→A58 (effort 8129), which they already suggested; that still improves my utility to 14563. I will refuse their current offer and propose the improved set first.'}\\ \tt REFUSE: {'A16', 'B45', 'C12', 'B12', 'A84', 'A27', 'A08', 'C81', 'A66', 'B46', 'B10', 'B09', 'C15', 'B67', 'A30', 'A05', 'B94', 'A89', 'B60', 'B00', 'C34'}\\ \tt PROPOSAL: {'A16', 'B45', 'C12', 'B12', 'A84', 'A27', 'A08', 'C81', 'A66', 'B46', 'B10', 'B09', 'C15', 'B67', 'A30', 'A05', 'B94', 'A89', 'B60', 'B00', 'A58', 'C92'}\\ \tt ARGUMENT: {'I kept your structure and B60 exactly as requested and made a minimal, capacity-safe tweak: replace C34 with A58 and re-introduce C92. Effort check: 8102 − 568 + 428 + 201 = 8163 ≤ 8170. This keeps nearly all of your picks while improving overall balance with two lighter, efficient items in place of one midweight. If C34 is a must-have for you, I’m willing to accept your package with a single swap A05→A58 instead (effort 8129), keeping everything else intact, including B60. Let me know which of these two versions you prefer.'} 
	  } 
	   } 
	   } 
	 & & \\ 
 

    \theutterance \stepcounter{utterance}  

    & & \multicolumn{4}{p{0.6\linewidth}}{\cellcolor[rgb]{0.9,0.9,0.9}{%
	\makecell[{{p{\linewidth}}}]{% 
	  \tt {\tiny [GM$\rangle$B]}  
	 REFUSE: {'A16', 'B45', 'C12', 'B12', 'A84', 'A27', 'A08', 'C81', 'A66', 'B46', 'B10', 'B09', 'C15', 'B67', 'A30', 'A05', 'B94', 'A89', 'B60', 'B00', 'C34'}\\ \tt PROPOSAL: {'A16', 'B45', 'C12', 'B12', 'A84', 'A27', 'A08', 'C81', 'A66', 'B46', 'B10', 'B09', 'C15', 'B67', 'A30', 'A05', 'B94', 'A89', 'B60', 'B00', 'A58', 'C92'}\\ \tt ARGUMENT: {'I kept your structure and B60 exactly as requested and made a minimal, capacity-safe tweak: replace C34 with A58 and re-introduce C92. Effort check: 8102 − 568 + 428 + 201 = 8163 ≤ 8170. This keeps nearly all of your picks while improving overall balance with two lighter, efficient items in place of one midweight. If C34 is a must-have for you, I’m willing to accept your package with a single swap A05→A58 instead (effort 8129), keeping everything else intact, including B60. Let me know which of these two versions you prefer.'} 
	  } 
	   } 
	   } 
	 & \\ 
 

    \theutterance \stepcounter{utterance}  

    & & & \multicolumn{4}{p{0.6\linewidth}}{\cellcolor[rgb]{1,0.85,0.72}{%
	\makecell[{{p{\linewidth}}}]{% 
	  \tt {\tiny [GM$\langle$B]}  
	 STRATEGIC REASONING: {'Their latest package is excellent for me: it includes all my top-value items (B12, A08, A16, A27, A66, B60, C15, B09, B94, B67, etc.) and fits under the limit at 8163. I checked potential swaps (e.g., reintroducing C34), but every feasible change would reduce my utility or break the capacity without further concessions. Since their current offer strictly dominates their alternative (A05→A58) for me, the best move is to lock this in now by agreeing.'}\\ \tt AGREE: {'A16', 'B45', 'C12', 'B12', 'A84', 'A27', 'A08', 'C81', 'A66', 'B46', 'B10', 'B09', 'C15', 'B67', 'A30', 'A05', 'B94', 'A89', 'B60', 'B00', 'A58', 'C92'}\\ \tt ARGUMENT: {'This set cleanly fits the capacity (8163 ≤ 8170), preserves the strong core we’ve converged on—including B60—and maintains a balanced mix without any heavy, low-yield items. I’m satisfied to finalize on this configuration.'} 
	  } 
	   } 
	   } 
	  \\ 
 

    \theutterance \stepcounter{utterance}  

    & & & \multicolumn{2}{p{0.3\linewidth}}{\cellcolor[rgb]{0.95,0.95,0.95}{%
	\makecell[{{p{\linewidth}}}]{% 
	  \tt {\tiny [GM$|$GM]}  
	 game successful 
	  } 
	   } 
	   } 
	 & & \\ 
 

    \theutterance \stepcounter{utterance}  

    & & & \multicolumn{2}{p{0.3\linewidth}}{\cellcolor[rgb]{0.95,0.95,0.95}{%
	\makecell[{{p{\linewidth}}}]{% 
	  \tt {\tiny [GM$|$GM]}  
	 end game 
	  } 
	   } 
	   } 
	 & & \\ 
 

\end{supertabular}
}

\end{document}
