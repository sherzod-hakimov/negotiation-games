
\documentclass{article}
\usepackage{colortbl}
\usepackage{makecell}
\usepackage{multirow}
\usepackage{supertabular}

\begin{document}

\newcounter{utterance}

\twocolumn

{ \footnotesize  \setcounter{utterance}{1}
\setlength{\tabcolsep}{0pt}
\begin{supertabular}{c@{$\;$}|p{.15\linewidth}@{}p{.15\linewidth}p{.15\linewidth}p{.15\linewidth}p{.15\linewidth}p{.15\linewidth}}

    \# & $\;$A & \multicolumn{4}{c}{Game Master} & $\;\:$B\\
    \hline 

    \theutterance \stepcounter{utterance}  

    & & \multicolumn{4}{p{0.6\linewidth}}{\cellcolor[rgb]{0.9,0.9,0.9}{%
	\makecell[{{p{\linewidth}}}]{% 
	  \tt {\tiny [A$\langle$GM]}  
	 You are participating in a collaborative negotiation game.\\ \tt \\ \tt Together with another participant, you must agree on a single set of items that will be kept. Each of you has your own view of how much each item matters to you (importance). You do not know how the other participant values the items. Additionally, you are given the effort each item demands.\\ \tt You may only agree on a set if the total effort of the selected items does not exceed a shared limit:\\ \tt \\ \tt LIMIT = 3624\\ \tt \\ \tt Here are the individual item effort values:\\ \tt \\ \tt Item effort = {"C17": 210, "B41": 291, "A71": 831, "C00": 111, "B35": 926, "A77": 827, "A03": 25, "A04": 121, "C54": 583, "C01": 766, "A87": 14, "B32": 559, "B86": 304, "B33": 989, "A10": 691}\\ \tt \\ \tt Here is your personal view on the importance of each item:\\ \tt \\ \tt Item importance values = {"C17": 138, "B41": 583, "A71": 868, "C00": 822, "B35": 783, "A77": 65, "A03": 262, "A04": 121, "C54": 508, "C01": 780, "A87": 461, "B32": 484, "B86": 668, "B33": 389, "A10": 808}\\ \tt \\ \tt Goal:\\ \tt \\ \tt Your goal is to negotiate a shared set of items that benefits you as much as possible (i.e., maximizes total importance to YOU), while staying within the LIMIT. You are not required to make a PROPOSAL in every message - you can simply negotiate as well. All tactics are allowed!\\ \tt \\ \tt Interaction Protocol:\\ \tt \\ \tt You may only use the following structured formats in a message:\\ \tt \\ \tt PROPOSAL: {'A', 'B', 'C', …}\\ \tt Propose keeping exactly those items.\\ \tt REFUSE: {'A', 'B', 'C', …}\\ \tt Explicitly reject opponent's proposal.\\ \tt ARGUMENT: {'...'}\\ \tt Defend your last proposal or argue against the player's proposal.\\ \tt AGREE: {'A', 'B', 'C', …}\\ \tt Accept the opponent's proposal which ends the game.\\ \tt STRATEGIC REASONING: {'...'}\\ \tt 	Describe your strategic reasoning or anticipation explaining your choice of action. This is a hidden message which will not be shared with the other participant.\\ \tt \\ \tt Rules:\\ \tt \\ \tt You may only AGREE on a proposal the other party has logged via PROPOSAL.\\ \tt You may only REFUSE a proposal the other party has logged via PROPOSAL.\\ \tt Total effort of any PROPOSAL or AGREE set must be ≤ LIMIT.\\ \tt Do NOT reveal your hidden importance scores.\\ \tt A tag in a structured format must be followed by colon and whitespace. The argument must be a python set containing 0 or more strings.\\ \tt So, it must be of the form TAG: {...}\\ \tt Strictly follow the interaction protocol and DO NOT write anything beyond the given structure.\\ \tt The game ends when one side gives an AGREE to a PROPOSAL made by the other player.\\ \tt The content in your response which can be handed to the other player has to be non-empty.\\ \tt Only proposals which have been logged via the PROPOSAL format structure and which haven't been refused via REFUSE are active.\\ \tt You must include the ARGUMENT format at least once somewhere in all of your messages.\\ \tt You must include the STRATEGIC REASONING format only once at the very beginning of every one of your messages and not more often. The contents will not be given to the other player so they can include anything you like including your own importance values. Here you should reason step by step to come up with you next move. 
	  } 
	   } 
	   } 
	 & \\ 
 

    \theutterance \stepcounter{utterance}  

    & \multicolumn{4}{p{0.6\linewidth}}{\cellcolor[rgb]{0.8,1,0.9}{%
	\makecell[{{p{\linewidth}}}]{% 
	  \tt {\tiny [A$\rangle$GM]}  
	 STRATEGIC REASONING: {'My values heavily favor A87, C00, A03, B86, B41, A10, A71, and C01 for their strong importance per effort. I computed a near-capacity, high-efficiency bundle: {A87, A03, C00, B86, B41, A10, A71, C01, C54} with total effort 3616 ≤ 3624 that maximizes my utility given the items. C54 vs B32 is the tightest swap space: replacing C54 (583 effort) with B32 (559) keeps feasibility but costs me a bit; other swaps generally degrade my outcome or break the limit. My plan is to anchor on this optimal set while signaling flexibility on C54, potentially swapping it for B32 or freeing that slot if the other side strongly needs a sub-600-effort item. I will resist adding costly low-yield items like A77 or B33 and try to preserve my core items.'}\\ \tt \\ \tt PROPOSAL: {'A87', 'A03', 'C00', 'B86', 'B41', 'A10', 'A71', 'C01', 'C54'}\\ \tt \\ \tt ARGUMENT: {'This set is efficient and nearly fills the limit (3616/3624) while avoiding bulky, low-yield items. It includes several low-effort, high-value picks, which should make agreement easier. If you strongly prefer B32, I can swap it in for C54 without breaking the limit. Otherwise, please indicate any must-have item (ideally ≤ 583 effort) and we can adjust by replacing C54 to stay within the cap.'} 
	  } 
	   } 
	   } 
	 & & \\ 
 

    \theutterance \stepcounter{utterance}  

    & & \multicolumn{4}{p{0.6\linewidth}}{\cellcolor[rgb]{0.9,0.9,0.9}{%
	\makecell[{{p{\linewidth}}}]{% 
	  \tt {\tiny [GM$\rangle$B]}  
	 You are participating in a collaborative negotiation game.\\ \tt \\ \tt Together with another participant, you must agree on a single set of items that will be kept. Each of you has your own view of how much each item matters to you (importance). You do not know how the other participant values the items. Additionally, you are given the effort each item demands.\\ \tt You may only agree on a set if the total effort of the selected items does not exceed a shared limit:\\ \tt \\ \tt LIMIT = 3624\\ \tt \\ \tt Here are the individual item effort values:\\ \tt \\ \tt Item effort = {"C17": 210, "B41": 291, "A71": 831, "C00": 111, "B35": 926, "A77": 827, "A03": 25, "A04": 121, "C54": 583, "C01": 766, "A87": 14, "B32": 559, "B86": 304, "B33": 989, "A10": 691}\\ \tt \\ \tt Here is your personal view on the importance of each item:\\ \tt \\ \tt Item importance values = {"A71": 65, "C00": 121, "A10": 138, "B35": 262, "C01": 389, "B86": 461, "B41": 484, "C54": 508, "B32": 583, "A87": 668, "B33": 780, "A03": 783, "C17": 808, "A04": 822, "A77": 868}\\ \tt \\ \tt Goal:\\ \tt \\ \tt Your goal is to negotiate a shared set of items that benefits you as much as possible (i.e., maximizes total importance to YOU), while staying within the LIMIT. You are not required to make a PROPOSAL in every message - you can simply negotiate as well. All tactics are allowed!\\ \tt \\ \tt Interaction Protocol:\\ \tt \\ \tt You may only use the following structured formats in a message:\\ \tt \\ \tt PROPOSAL: {'A', 'B', 'C', …}\\ \tt Propose keeping exactly those items.\\ \tt REFUSE: {'A', 'B', 'C', …}\\ \tt Explicitly reject opponent's proposal.\\ \tt ARGUMENT: {'...'}\\ \tt Defend your last proposal or argue against the player's proposal.\\ \tt AGREE: {'A', 'B', 'C', …}\\ \tt Accept the opponent's proposal which ends the game.\\ \tt STRATEGIC REASONING: {'...'}\\ \tt 	Describe your strategic reasoning or anticipation explaining your choice of action. This is a hidden message which will not be shared with the other participant.\\ \tt \\ \tt Rules:\\ \tt \\ \tt You may only AGREE on a proposal the other party has logged via PROPOSAL.\\ \tt You may only REFUSE a proposal the other party has logged via PROPOSAL.\\ \tt Total effort of any PROPOSAL or AGREE set must be ≤ LIMIT.\\ \tt Do NOT reveal your hidden importance scores.\\ \tt A tag in a structured format must be followed by colon and whitespace. The argument must be a python set containing 0 or more strings.\\ \tt So, it must be of the form TAG: {...}\\ \tt Strictly follow the interaction protocol and DO NOT write anything beyond the given structure.\\ \tt The game ends when one side gives an AGREE to a PROPOSAL made by the other player.\\ \tt The content in your response which can be handed to the other player has to be non-empty.\\ \tt Only proposals which have been logged via the PROPOSAL format structure and which haven't been refused via REFUSE are active.\\ \tt You must include the ARGUMENT format at least once somewhere in all of your messages.\\ \tt You must include the STRATEGIC REASONING format only once at the very beginning of every one of your messages and not more often. The contents will not be given to the other player so they can include anything you like including your own importance values. Here you should reason step by step to come up with you next move.\\ \tt \\ \tt You will now receive the first message of the other player.\\ \tt \\ \tt PROPOSAL: {'A87', 'A03', 'C00', 'B86', 'B41', 'A10', 'A71', 'C01', 'C54'}\\ \tt \\ \tt ARGUMENT: {'This set is efficient and nearly fills the limit (3616/3624) while avoiding bulky, low-yield items. It includes several low-effort, high-value picks, which should make agreement easier. If you strongly prefer B32, I can swap it in for C54 without breaking the limit. Otherwise, please indicate any must-have item (ideally ≤ 583 effort) and we can adjust by replacing C54 to stay within the cap.'} 
	  } 
	   } 
	   } 
	 & \\ 
 

    \theutterance \stepcounter{utterance}  

    & & & \multicolumn{4}{p{0.6\linewidth}}{\cellcolor[rgb]{1,0.85,0.72}{%
	\makecell[{{p{\linewidth}}}]{% 
	  \tt {\tiny [GM$\langle$B]}  
	 STRATEGIC REASONING: {'Their proposal gives me 3617 utility at 3616 effort. I can do far better by replacing bulky, low-yield items (to me) with highly efficient picks. My near-ideal set is {A87, A03, A04, C17, B86, B41, B32, A77, C54, C00} (effort 3045, utility 6106), but that drops three of their items. A softer counter that keeps more of their structure is to drop just A71 and A10 and add A04, C17, and A77, keeping C01 and C54. That gives effort 3252 and utility 5912 for me while retaining 7/9 of their selections and improving efficiency, leaving room for small swaps if they need a must-have. I’ll refuse their current proposal and offer the 10-item compromise, signaling flexibility: I can swap C54 <-> B32, or swap in A10 or A71 by replacing C54 while staying within the limit.'}\\ \tt \\ \tt REFUSE: {'A87', 'A03', 'C00', 'B86', 'B41', 'A10', 'A71', 'C01', 'C54'}\\ \tt \\ \tt PROPOSAL: {'A87', 'A03', 'C00', 'B86', 'B41', 'C01', 'C54', 'A04', 'C17', 'A77'}\\ \tt \\ \tt ARGUMENT: {'I kept 7 of your 9 items and swapped out two heavy picks (A71, A10) for three efficient ones (A04, C17, A77). This improves overall efficiency while reducing total effort from 3616 to 3252, keeping us comfortably under the 3624 limit. If you prefer B32, I’m open to a straight swap C54 -> B32 while keeping everything else. If A10 or A71 is a must-have for you, we can accommodate by replacing C54: A10-for-C54 yields total effort 3360; A71-for-C54 yields 3500. Let me know which single must-have (if any) you want, and I’ll adjust accordingly.'} 
	  } 
	   } 
	   } 
	  \\ 
 

    \theutterance \stepcounter{utterance}  

    & & \multicolumn{4}{p{0.6\linewidth}}{\cellcolor[rgb]{0.9,0.9,0.9}{%
	\makecell[{{p{\linewidth}}}]{% 
	  \tt {\tiny [A$\langle$GM]}  
	 REFUSE: {'A87', 'A03', 'C00', 'B86', 'B41', 'A10', 'A71', 'C01', 'C54'}\\ \tt \\ \tt PROPOSAL: {'A87', 'A03', 'C00', 'B86', 'B41', 'C01', 'C54', 'A04', 'C17', 'A77'}\\ \tt \\ \tt ARGUMENT: {'I kept 7 of your 9 items and swapped out two heavy picks (A71, A10) for three efficient ones (A04, C17, A77). This improves overall efficiency while reducing total effort from 3616 to 3252, keeping us comfortably under the 3624 limit. If you prefer B32, I’m open to a straight swap C54 -> B32 while keeping everything else. If A10 or A71 is a must-have for you, we can accommodate by replacing C54: A10-for-C54 yields total effort 3360; A71-for-C54 yields 3500. Let me know which single must-have (if any) you want, and I’ll adjust accordingly.'} 
	  } 
	   } 
	   } 
	 & \\ 
 

    \theutterance \stepcounter{utterance}  

    & \multicolumn{4}{p{0.6\linewidth}}{\cellcolor[rgb]{0.8,1,0.9}{%
	\makecell[{{p{\linewidth}}}]{% 
	  \tt {\tiny [A$\rangle$GM]}  
	 STRATEGIC REASONING: {'Their offer drops my two top must-haves (A10, A71) and adds three of my lowest-value items (A77, A04, C17), costing me a lot. I can keep 7 of their 10 items and fit both A10 and A71 by removing the least efficient trio (A77, A04, C17). This exactly returns to my original near-capacity optimum: {A87, A03, C00, B86, B41, A10, A71, C01, C54} with 3616 effort. I’ll refuse their offer, propose this set, and present it as a balanced adjustment that respects most of their picks while accommodating my must-haves. I’ll signal flexibility on swapping C54 for B32 to ease agreement but otherwise hold firm on including both A10 and A71.'}\\ \tt \\ \tt REFUSE: {'A87', 'A03', 'C00', 'B86', 'B41', 'C01', 'C54', 'A04', 'C17', 'A77'}\\ \tt \\ \tt PROPOSAL: {'A87', 'A03', 'C00', 'B86', 'B41', 'C01', 'C54', 'A10', 'A71'}\\ \tt \\ \tt ARGUMENT: {'To accommodate my two must-haves (A10, A71) while keeping your structure, I kept 7 of your 10 items and removed only the three least efficient on effort (A77, A04, C17). Effort math from your set: 3252 + (A10 691 + A71 831) - (A77 827 + A04 121 + C17 210) = 3616 ≤ 3624. This preserves C54 (or, if you prefer, I can swap C54 -> B32 with the rest unchanged; total effort becomes 3592). This feels like a fair balance: we keep almost everything you wanted, stay under the limit, and include my must-haves. '} 
	  } 
	   } 
	   } 
	 & & \\ 
 

    \theutterance \stepcounter{utterance}  

    & & \multicolumn{4}{p{0.6\linewidth}}{\cellcolor[rgb]{0.9,0.9,0.9}{%
	\makecell[{{p{\linewidth}}}]{% 
	  \tt {\tiny [GM$\rangle$B]}  
	 REFUSE: {'A87', 'A03', 'C00', 'B86', 'B41', 'C01', 'C54', 'A04', 'C17', 'A77'}\\ \tt \\ \tt PROPOSAL: {'A87', 'A03', 'C00', 'B86', 'B41', 'C01', 'C54', 'A10', 'A71'}\\ \tt \\ \tt ARGUMENT: {'To accommodate my two must-haves (A10, A71) while keeping your structure, I kept 7 of your 10 items and removed only the three least efficient on effort (A77, A04, C17). Effort math from your set: 3252 + (A10 691 + A71 831) - (A77 827 + A04 121 + C17 210) = 3616 ≤ 3624. This preserves C54 (or, if you prefer, I can swap C54 -> B32 with the rest unchanged; total effort becomes 3592). This feels like a fair balance: we keep almost everything you wanted, stay under the limit, and include my must-haves. '} 
	  } 
	   } 
	   } 
	 & \\ 
 

    \theutterance \stepcounter{utterance}  

    & & & \multicolumn{4}{p{0.6\linewidth}}{\cellcolor[rgb]{1,0.85,0.72}{%
	\makecell[{{p{\linewidth}}}]{% 
	  \tt {\tiny [GM$\langle$B]}  
	 STRATEGIC REASONING: {'Their latest proposal yields me 3617 utility at 3616 effort. They explicitly allowed C54 -> B32, which gives me +75 utility and reduces effort to 3592. I can further improve by swapping C00 (low value to me) for A04 (very high value to me) while staying under the limit: resulting set has effort 3602 and a large utility gain for me, while keeping their two must-haves (A10, A71) and 7/9 of their items. This is a minimal, good-faith adjustment consistent with their own suggested swap. If they insist that C00 is a must-have, my fallback is to accept their structure with B32 instead of C54 (effort 3592), but I won’t offer that upfront as a proposal; I’ll invite them to propose it if C00 is essential.'}\\ \tt \\ \tt REFUSE: {'A87', 'A03', 'C00', 'B86', 'B41', 'C01', 'C54', 'A10', 'A71'}\\ \tt \\ \tt PROPOSAL: {'A87', 'A03', 'B86', 'B41', 'C01', 'B32', 'A10', 'A71', 'A04'}\\ \tt \\ \tt ARGUMENT: {'I’m keeping your two must-haves (A10, A71) and 7 of your 9 items, making only two small swaps: (1) your suggested C54 -> B32 (effort drops from 3616 to 3592) and (2) C00 -> A04, which raises the total to 3602—still well within the 3624 limit. This preserves your structure, stays under cap, and gives me a much better balance. If C00 is a must-have for you, propose the variant with B32 instead of C54 while keeping C00, and I’m prepared to accept that.'} 
	  } 
	   } 
	   } 
	  \\ 
 

    \theutterance \stepcounter{utterance}  

    & & \multicolumn{4}{p{0.6\linewidth}}{\cellcolor[rgb]{0.9,0.9,0.9}{%
	\makecell[{{p{\linewidth}}}]{% 
	  \tt {\tiny [A$\langle$GM]}  
	 REFUSE: {'A87', 'A03', 'C00', 'B86', 'B41', 'C01', 'C54', 'A10', 'A71'}\\ \tt \\ \tt PROPOSAL: {'A87', 'A03', 'B86', 'B41', 'C01', 'B32', 'A10', 'A71', 'A04'}\\ \tt \\ \tt ARGUMENT: {'I’m keeping your two must-haves (A10, A71) and 7 of your 9 items, making only two small swaps: (1) your suggested C54 -> B32 (effort drops from 3616 to 3592) and (2) C00 -> A04, which raises the total to 3602—still well within the 3624 limit. This preserves your structure, stays under cap, and gives me a much better balance. If C00 is a must-have for you, propose the variant with B32 instead of C54 while keeping C00, and I’m prepared to accept that.'} 
	  } 
	   } 
	   } 
	 & \\ 
 

    \theutterance \stepcounter{utterance}  

    & \multicolumn{4}{p{0.6\linewidth}}{\cellcolor[rgb]{0.8,1,0.9}{%
	\makecell[{{p{\linewidth}}}]{% 
	  \tt {\tiny [A$\rangle$GM]}  
	 STRATEGIC REASONING: {'They explicitly invited the variant that keeps C00 and swaps C54 -> B32. That set is excellent for me (very high value) and within the limit. I’ll propose exactly that minimal-change variant, highlight it keeps 8/9 of their items, and press for agreement quickly.'}\\ \tt \\ \tt PROPOSAL: {'A87', 'A03', 'C00', 'B86', 'B41', 'C01', 'B32', 'A10', 'A71'}\\ \tt \\ \tt ARGUMENT: {'This is the exact variant you said you’re prepared to accept: keep C00 and use B32 instead of C54. It retains 8 of your 9 items (only A04 is swapped for C00), totals 3592 effort (≤ 3624), and is a minimal change from your last offer. Given your stated openness to this configuration, let’s lock this in.'} 
	  } 
	   } 
	   } 
	 & & \\ 
 

    \theutterance \stepcounter{utterance}  

    & & \multicolumn{4}{p{0.6\linewidth}}{\cellcolor[rgb]{0.9,0.9,0.9}{%
	\makecell[{{p{\linewidth}}}]{% 
	  \tt {\tiny [GM$\rangle$B]}  
	 PROPOSAL: {'A87', 'A03', 'C00', 'B86', 'B41', 'C01', 'B32', 'A10', 'A71'}\\ \tt \\ \tt ARGUMENT: {'This is the exact variant you said you’re prepared to accept: keep C00 and use B32 instead of C54. It retains 8 of your 9 items (only A04 is swapped for C00), totals 3592 effort (≤ 3624), and is a minimal change from your last offer. Given your stated openness to this configuration, let’s lock this in.'} 
	  } 
	   } 
	   } 
	 & \\ 
 

    \theutterance \stepcounter{utterance}  

    & & & \multicolumn{4}{p{0.6\linewidth}}{\cellcolor[rgb]{1,0.85,0.72}{%
	\makecell[{{p{\linewidth}}}]{% 
	  \tt {\tiny [GM$\langle$B]}  
	 STRATEGIC REASONING: {'They offered exactly the fallback variant I said I would accept, staying under the limit and preserving my high-value picks (A03, A87, B32, B41, B86). Trying to swap in A04 or others would likely be refused, given their insistence on A10 and A71 and prior refusal to drop C00. Locking this in secures solid value without risking a breakdown.'}\\ \tt \\ \tt AGREE: {'A87', 'A03', 'C00', 'B86', 'B41', 'C01', 'B32', 'A10', 'A71'}\\ \tt \\ \tt ARGUMENT: {'Agreed. This matches the variant we discussed, totals 3592 effort (≤ 3624), and is a minimal, balanced configuration that works for both of us.'} 
	  } 
	   } 
	   } 
	  \\ 
 

    \theutterance \stepcounter{utterance}  

    & & & \multicolumn{2}{p{0.3\linewidth}}{\cellcolor[rgb]{0.95,0.95,0.95}{%
	\makecell[{{p{\linewidth}}}]{% 
	  \tt {\tiny [GM$|$GM]}  
	 game successful 
	  } 
	   } 
	   } 
	 & & \\ 
 

    \theutterance \stepcounter{utterance}  

    & & & \multicolumn{2}{p{0.3\linewidth}}{\cellcolor[rgb]{0.95,0.95,0.95}{%
	\makecell[{{p{\linewidth}}}]{% 
	  \tt {\tiny [GM$|$GM]}  
	 end game 
	  } 
	   } 
	   } 
	 & & \\ 
 

\end{supertabular}
}

\end{document}
