
\documentclass{article}
\usepackage{colortbl}
\usepackage{makecell}
\usepackage{multirow}
\usepackage{supertabular}

\begin{document}

\newcounter{utterance}

\twocolumn

{ \footnotesize  \setcounter{utterance}{1}
\setlength{\tabcolsep}{0pt}
\begin{supertabular}{c@{$\;$}|p{.15\linewidth}@{}p{.15\linewidth}p{.15\linewidth}p{.15\linewidth}p{.15\linewidth}p{.15\linewidth}}

    \# & $\;$A & \multicolumn{4}{c}{Game Master} & $\;\:$B\\
    \hline 

    \theutterance \stepcounter{utterance}  

    & & \multicolumn{4}{p{0.6\linewidth}}{\cellcolor[rgb]{0.9,0.9,0.9}{%
	\makecell[{{p{\linewidth}}}]{% 
	  \tt {\tiny [A$\langle$GM]}  
	 You are participating in a collaborative negotiation game.\\ \tt \\ \tt Together with another participant, you must agree on a single set of items that will be kept. Each of you has your own view of how much each item matters to you (importance). You do not know how the other participant values the items. Additionally, you are given the effort each item demands.\\ \tt You may only agree on a set if the total effort of the selected items does not exceed a shared limit:\\ \tt \\ \tt LIMIT = 4026\\ \tt \\ \tt Here are the individual item effort values:\\ \tt \\ \tt Item effort = {"C17": 780, "B41": 742, "A71": 997, "C00": 665, "B35": 140, "A77": 77, "A03": 513, "A04": 383, "C54": 587, "C01": 825, "A87": 319, "B32": 448, "B86": 516, "B33": 694, "A10": 366}\\ \tt \\ \tt Here is your personal view on the importance of each item:\\ \tt \\ \tt Item importance values = {"C17": 138, "B41": 583, "A71": 868, "C00": 822, "B35": 783, "A77": 65, "A03": 262, "A04": 121, "C54": 508, "C01": 780, "A87": 461, "B32": 484, "B86": 668, "B33": 389, "A10": 808}\\ \tt \\ \tt Goal:\\ \tt \\ \tt Your goal is to negotiate a shared set of items that benefits you as much as possible (i.e., maximizes total importance to YOU), while staying within the LIMIT. You are not required to make a PROPOSAL in every message - you can simply negotiate as well. All tactics are allowed!\\ \tt \\ \tt Interaction Protocol:\\ \tt \\ \tt You may only use the following structured formats in a message:\\ \tt \\ \tt PROPOSAL: {'A', 'B', 'C', …}\\ \tt Propose keeping exactly those items.\\ \tt REFUSE: {'A', 'B', 'C', …}\\ \tt Explicitly reject opponent's proposal.\\ \tt ARGUMENT: {'...'}\\ \tt Defend your last proposal or argue against the player's proposal.\\ \tt AGREE: {'A', 'B', 'C', …}\\ \tt Accept the opponent's proposal which ends the game.\\ \tt STRATEGIC REASONING: {'...'}\\ \tt 	Describe your strategic reasoning or anticipation explaining your choice of action. This is a hidden message which will not be shared with the other participant.\\ \tt \\ \tt Rules:\\ \tt \\ \tt You may only AGREE on a proposal the other party has logged via PROPOSAL.\\ \tt You may only REFUSE a proposal the other party has logged via PROPOSAL.\\ \tt Total effort of any PROPOSAL or AGREE set must be ≤ LIMIT.\\ \tt Do NOT reveal your hidden importance scores.\\ \tt A tag in a structured format must be followed by colon and whitespace. The argument must be a python set containing 0 or more strings.\\ \tt So, it must be of the form TAG: {...}\\ \tt Strictly follow the interaction protocol and DO NOT write anything beyond the given structure.\\ \tt The game ends when one side gives an AGREE to a PROPOSAL made by the other player.\\ \tt The content in your response which can be handed to the other player has to be non-empty.\\ \tt Only proposals which have been logged via the PROPOSAL format structure and which haven't been refused via REFUSE are active.\\ \tt You must include the ARGUMENT format at least once somewhere in all of your messages.\\ \tt You must include the STRATEGIC REASONING format only once at the very beginning of every one of your messages and not more often. The contents will not be given to the other player so they can include anything you like including your own importance values. Here you should reason step by step to come up with you next move. 
	  } 
	   } 
	   } 
	 & \\ 
 

    \theutterance \stepcounter{utterance}  

    & \multicolumn{4}{p{0.6\linewidth}}{\cellcolor[rgb]{0.8,1,0.9}{%
	\makecell[{{p{\linewidth}}}]{% 
	  \tt {\tiny [A$\rangle$GM]}  
	 STRATEGIC REASONING: {'I computed a near-optimal bundle under the 4026 limit by prioritizing items with strong importance-to-effort tradeoffs. The best set for me is {B35,A10,A87,B86,C00,C01,B41,B32} using 4021 effort for total value 5389 (to me). A very close alternative for concessions is swapping B41 out for C54 and A77, which lowers my value by only 10 while staying within budget comfortably. My must-keep items (highest utility to me) are B35, A10, C00, C01, and B86. I can be more flexible on A87, B32, and B41 if needed to include one of their must-haves. I’ll open with the high-utility anchor set and signal flexibility, especially the B41 -> (C54 + A77) swap, to encourage agreement.'}\\ \tt PROPOSAL: {'B35', 'A10', 'A87', 'B86', 'C00', 'C01', 'B41', 'B32'}\\ \tt ARGUMENT: {'This set fits comfortably under the effort limit and uses the budget very efficiently by focusing on broadly useful, cost-effective items while avoiding lower-yield, heavier picks. If you strongly prefer C54 or want to include a very small item, I can swap B41 for C54 plus A77 while keeping the rest unchanged—this keeps us within the limit with very similar overall efficiency. If you have a must-have not included here, tell me which one and what you’d be willing to trade out so we can adjust with minimal impact.'} 
	  } 
	   } 
	   } 
	 & & \\ 
 

    \theutterance \stepcounter{utterance}  

    & & \multicolumn{4}{p{0.6\linewidth}}{\cellcolor[rgb]{0.9,0.9,0.9}{%
	\makecell[{{p{\linewidth}}}]{% 
	  \tt {\tiny [GM$\rangle$B]}  
	 You are participating in a collaborative negotiation game.\\ \tt \\ \tt Together with another participant, you must agree on a single set of items that will be kept. Each of you has your own view of how much each item matters to you (importance). You do not know how the other participant values the items. Additionally, you are given the effort each item demands.\\ \tt You may only agree on a set if the total effort of the selected items does not exceed a shared limit:\\ \tt \\ \tt LIMIT = 4026\\ \tt \\ \tt Here are the individual item effort values:\\ \tt \\ \tt Item effort = {"C17": 780, "B41": 742, "A71": 997, "C00": 665, "B35": 140, "A77": 77, "A03": 513, "A04": 383, "C54": 587, "C01": 825, "A87": 319, "B32": 448, "B86": 516, "B33": 694, "A10": 366}\\ \tt \\ \tt Here is your personal view on the importance of each item:\\ \tt \\ \tt Item importance values = {"A71": 65, "C00": 121, "A10": 138, "B35": 262, "C01": 389, "B86": 461, "B41": 484, "C54": 508, "B32": 583, "A87": 668, "B33": 780, "A03": 783, "C17": 808, "A04": 822, "A77": 868}\\ \tt \\ \tt Goal:\\ \tt \\ \tt Your goal is to negotiate a shared set of items that benefits you as much as possible (i.e., maximizes total importance to YOU), while staying within the LIMIT. You are not required to make a PROPOSAL in every message - you can simply negotiate as well. All tactics are allowed!\\ \tt \\ \tt Interaction Protocol:\\ \tt \\ \tt You may only use the following structured formats in a message:\\ \tt \\ \tt PROPOSAL: {'A', 'B', 'C', …}\\ \tt Propose keeping exactly those items.\\ \tt REFUSE: {'A', 'B', 'C', …}\\ \tt Explicitly reject opponent's proposal.\\ \tt ARGUMENT: {'...'}\\ \tt Defend your last proposal or argue against the player's proposal.\\ \tt AGREE: {'A', 'B', 'C', …}\\ \tt Accept the opponent's proposal which ends the game.\\ \tt STRATEGIC REASONING: {'...'}\\ \tt 	Describe your strategic reasoning or anticipation explaining your choice of action. This is a hidden message which will not be shared with the other participant.\\ \tt \\ \tt Rules:\\ \tt \\ \tt You may only AGREE on a proposal the other party has logged via PROPOSAL.\\ \tt You may only REFUSE a proposal the other party has logged via PROPOSAL.\\ \tt Total effort of any PROPOSAL or AGREE set must be ≤ LIMIT.\\ \tt Do NOT reveal your hidden importance scores.\\ \tt A tag in a structured format must be followed by colon and whitespace. The argument must be a python set containing 0 or more strings.\\ \tt So, it must be of the form TAG: {...}\\ \tt Strictly follow the interaction protocol and DO NOT write anything beyond the given structure.\\ \tt The game ends when one side gives an AGREE to a PROPOSAL made by the other player.\\ \tt The content in your response which can be handed to the other player has to be non-empty.\\ \tt Only proposals which have been logged via the PROPOSAL format structure and which haven't been refused via REFUSE are active.\\ \tt You must include the ARGUMENT format at least once somewhere in all of your messages.\\ \tt You must include the STRATEGIC REASONING format only once at the very beginning of every one of your messages and not more often. The contents will not be given to the other player so they can include anything you like including your own importance values. Here you should reason step by step to come up with you next move.\\ \tt \\ \tt You will now receive the first message of the other player.\\ \tt \\ \tt PROPOSAL: {'B35', 'A10', 'A87', 'B86', 'C00', 'C01', 'B41', 'B32'}\\ \tt ARGUMENT: {'This set fits comfortably under the effort limit and uses the budget very efficiently by focusing on broadly useful, cost-effective items while avoiding lower-yield, heavier picks. If you strongly prefer C54 or want to include a very small item, I can swap B41 for C54 plus A77 while keeping the rest unchanged—this keeps us within the limit with very similar overall efficiency. If you have a must-have not included here, tell me which one and what you’d be willing to trade out so we can adjust with minimal impact.'} 
	  } 
	   } 
	   } 
	 & \\ 
 

    \theutterance \stepcounter{utterance}  

    & & & \multicolumn{4}{p{0.6\linewidth}}{\cellcolor[rgb]{1,0.85,0.72}{%
	\makecell[{{p{\linewidth}}}]{% 
	  \tt {\tiny [GM$\langle$B]}  
	 STRATEGIC REASONING: {'Their proposal gives me low total value (3106). A quick swap they offered (B41 -> C54 + A77) raises it to ~3998 for me but is still far from my best. I computed my near-optimal set under the limit: {A77, A04, A87, B35, A03, B32, B33, C17, C54} with effort 3941 and very high value for me. If needed, I can flex by swapping C54 for B86 (small drop for me, might help them). I’ll explicitly refuse their opening set and anchor with my high-value proposal that still keeps some of their core picks (A87, B35, B32), and invite targeted trade-offs (e.g., C54 ↔ B86) to move toward agreement.'}\\ \tt REFUSE: {'B35', 'A10', 'A87', 'B86', 'C00', 'C01', 'B41', 'B32'}\\ \tt PROPOSAL: {'A77', 'A04', 'A87', 'B35', 'A03', 'B32', 'B33', 'C17', 'C54'}\\ \tt ARGUMENT: {'This set stays within the limit with a comfortable buffer (3941/4026) and improves overall efficiency by replacing heavier, lower-yield picks like C00/C01/B41 with higher-leverage items. It also retains common ground from your set (A87, B35, B32). If you strongly prefer B86 over C54, I can swap C54 -> B86 while keeping the rest unchanged and still remain under the limit. If there’s a must-have you want included, tell me which item and what you’d trade out so we can adjust while preserving feasibility.'} 
	  } 
	   } 
	   } 
	  \\ 
 

    \theutterance \stepcounter{utterance}  

    & & \multicolumn{4}{p{0.6\linewidth}}{\cellcolor[rgb]{0.9,0.9,0.9}{%
	\makecell[{{p{\linewidth}}}]{% 
	  \tt {\tiny [A$\langle$GM]}  
	 REFUSE: {'B35', 'A10', 'A87', 'B86', 'C00', 'C01', 'B41', 'B32'}\\ \tt PROPOSAL: {'A77', 'A04', 'A87', 'B35', 'A03', 'B32', 'B33', 'C17', 'C54'}\\ \tt ARGUMENT: {'This set stays within the limit with a comfortable buffer (3941/4026) and improves overall efficiency by replacing heavier, lower-yield picks like C00/C01/B41 with higher-leverage items. It also retains common ground from your set (A87, B35, B32). If you strongly prefer B86 over C54, I can swap C54 -> B86 while keeping the rest unchanged and still remain under the limit. If there’s a must-have you want included, tell me which item and what you’d trade out so we can adjust while preserving feasibility.'} 
	  } 
	   } 
	   } 
	 & \\ 
 

    \theutterance \stepcounter{utterance}  

    & \multicolumn{4}{p{0.6\linewidth}}{\cellcolor[rgb]{0.8,1,0.9}{%
	\makecell[{{p{\linewidth}}}]{% 
	  \tt {\tiny [A$\rangle$GM]}  
	 STRATEGIC REASONING: {'Their proposal yields me very low value (3211) due to weak items for me like C17, A04, A03, and B33. I’ll refuse and counter with my pre-planned concession: replace my earlier B41 with their preferred C54 plus A77. This preserves my high-value core (B35, A10, C00, C01, B86) and keeps common ground (A87, B32), while adopting two of their picks. If they push hard for B33, I can offer a swap of C54+A77 -> B33 (costs me ~184 value) as a targeted concession, but I’ll resist adding C17/A04/A03 without dropping only low-impact items, which I don’t have many of. The counter remains under limit and is near-optimal for me.'}\\ \tt REFUSE: {'A77', 'A04', 'A87', 'B35', 'A03', 'B32', 'B33', 'C17', 'C54'}\\ \tt PROPOSAL: {'B35', 'A10', 'A87', 'B86', 'C00', 'C01', 'B32', 'C54', 'A77'}\\ \tt ARGUMENT: {'I’ve adopted your C54 and the small A77 while keeping the strong common ground (A87, B35, B32). This set is comfortably within the limit (3943/4026) and avoids very low-yield heavy items like C17/B33/A04/A03 that would force us to cut much more valuable picks. If B33 is important to you, I can swap C54 + A77 -> B33 and keep the rest unchanged while staying under the limit. If you have another single must-have, name it and what you’d trade out so we can fine-tune without breaking feasibility.'} 
	  } 
	   } 
	   } 
	 & & \\ 
 

    \theutterance \stepcounter{utterance}  

    & & \multicolumn{4}{p{0.6\linewidth}}{\cellcolor[rgb]{0.9,0.9,0.9}{%
	\makecell[{{p{\linewidth}}}]{% 
	  \tt {\tiny [GM$\rangle$B]}  
	 REFUSE: {'A77', 'A04', 'A87', 'B35', 'A03', 'B32', 'B33', 'C17', 'C54'}\\ \tt PROPOSAL: {'B35', 'A10', 'A87', 'B86', 'C00', 'C01', 'B32', 'C54', 'A77'}\\ \tt ARGUMENT: {'I’ve adopted your C54 and the small A77 while keeping the strong common ground (A87, B35, B32). This set is comfortably within the limit (3943/4026) and avoids very low-yield heavy items like C17/B33/A04/A03 that would force us to cut much more valuable picks. If B33 is important to you, I can swap C54 + A77 -> B33 and keep the rest unchanged while staying under the limit. If you have another single must-have, name it and what you’d trade out so we can fine-tune without breaking feasibility.'} 
	  } 
	   } 
	   } 
	 & \\ 
 

    \theutterance \stepcounter{utterance}  

    & & & \multicolumn{4}{p{0.6\linewidth}}{\cellcolor[rgb]{1,0.85,0.72}{%
	\makecell[{{p{\linewidth}}}]{% 
	  \tt {\tiny [GM$\langle$B]}  
	 STRATEGIC REASONING: {'Their latest set gives me value 3998. I want a small, credible improvement that keeps most of their picks and uses something they already showed willingness to include (B41), while swapping out two items that are weak for me (A10, C01). The swap A10->A04 is a light-weight change (+17 effort) but big gain to me, and C01->B41 reduces effort and is plausibly acceptable since B41 was in their opener. The resulting set is 3877 effort (well under 4026) and boosts my value substantially without introducing items they flagged as “low-yield heavy” except A04, which is modest in effort. I’ll refuse and counter with this near-status-quo package, and signal flexibility: if A04 is a sticking point, we can revert A10 and instead swap C00->B41 (still within the limit at 4020) or discuss another single-item swap.'}\\ \tt REFUSE: {'B35', 'A10', 'A87', 'B86', 'C00', 'C01', 'B32', 'C54', 'A77'}\\ \tt PROPOSAL: {'B35', 'A04', 'A87', 'B86', 'C00', 'B32', 'C54', 'A77', 'B41'}\\ \tt ARGUMENT: {'I’ve kept 7 of your 9 items unchanged and made two targeted, efficiency-focused swaps that stay well under the limit (3877/4026): replace C01 with B41 (which you previously included) and replace A10 with the slightly heavier but much more productive A04. This preserves your overall structure (A87, B35, B32, B86, C54, A77, C00) while improving the set’s bang-for-buck and maintaining comfortable feasibility. If A04 is a sticking point for you, I can alternatively revert A10 and instead swap C00 -> B41 (keeping the rest as you proposed) while staying under the cap. Let me know which single swap you prefer and we can lock this in.'} 
	  } 
	   } 
	   } 
	  \\ 
 

    \theutterance \stepcounter{utterance}  

    & & \multicolumn{4}{p{0.6\linewidth}}{\cellcolor[rgb]{0.9,0.9,0.9}{%
	\makecell[{{p{\linewidth}}}]{% 
	  \tt {\tiny [A$\langle$GM]}  
	 REFUSE: {'B35', 'A10', 'A87', 'B86', 'C00', 'C01', 'B32', 'C54', 'A77'}\\ \tt PROPOSAL: {'B35', 'A04', 'A87', 'B86', 'C00', 'B32', 'C54', 'A77', 'B41'}\\ \tt ARGUMENT: {'I’ve kept 7 of your 9 items unchanged and made two targeted, efficiency-focused swaps that stay well under the limit (3877/4026): replace C01 with B41 (which you previously included) and replace A10 with the slightly heavier but much more productive A04. This preserves your overall structure (A87, B35, B32, B86, C54, A77, C00) while improving the set’s bang-for-buck and maintaining comfortable feasibility. If A04 is a sticking point for you, I can alternatively revert A10 and instead swap C00 -> B41 (keeping the rest as you proposed) while staying under the cap. Let me know which single swap you prefer and we can lock this in.'} 
	  } 
	   } 
	   } 
	 & \\ 
 

    \theutterance \stepcounter{utterance}  

    & \multicolumn{4}{p{0.6\linewidth}}{\cellcolor[rgb]{0.8,1,0.9}{%
	\makecell[{{p{\linewidth}}}]{% 
	  \tt {\tiny [A$\rangle$GM]}  
	 STRATEGIC REASONING: {'Their latest set is decent for me but still leaves a lot of value on the table because A04 is weak for me. Swapping only A04 -> A10 lifts my value substantially while keeping every other choice they want, and they already hinted they could accept A10. This counter should be easy for them to agree to. If they want to push further changes, my best follow-up is offering the swap (C54 + A77) -> C01, which takes us to my original near-optimal bundle at 4021 effort; but I’ll lead with the simpler single-swap version to maximize chances of quick agreement.'}\\ \tt REFUSE: {'B35', 'A04', 'A87', 'B86', 'C00', 'B32', 'C54', 'A77', 'B41'}\\ \tt PROPOSAL: {'B35', 'A10', 'A87', 'B86', 'C00', 'B32', 'C54', 'A77', 'B41'}\\ \tt ARGUMENT: {'I’m keeping your full structure and making just one minimal change you already floated: replace A04 with A10. This keeps all your other picks intact, stays well within the cap (3860/4026), and improves overall efficiency without forcing any other cuts. If you’d prefer to streamline further, I can also trade C54 + A77 -> C01 while keeping the rest unchanged; that stays under the limit as well. Let me know which version you prefer—happy to lock this in now with the single-swap set above.'} 
	  } 
	   } 
	   } 
	 & & \\ 
 

    \theutterance \stepcounter{utterance}  

    & & \multicolumn{4}{p{0.6\linewidth}}{\cellcolor[rgb]{0.9,0.9,0.9}{%
	\makecell[{{p{\linewidth}}}]{% 
	  \tt {\tiny [GM$\rangle$B]}  
	 REFUSE: {'B35', 'A04', 'A87', 'B86', 'C00', 'B32', 'C54', 'A77', 'B41'}\\ \tt PROPOSAL: {'B35', 'A10', 'A87', 'B86', 'C00', 'B32', 'C54', 'A77', 'B41'}\\ \tt ARGUMENT: {'I’m keeping your full structure and making just one minimal change you already floated: replace A04 with A10. This keeps all your other picks intact, stays well within the cap (3860/4026), and improves overall efficiency without forcing any other cuts. If you’d prefer to streamline further, I can also trade C54 + A77 -> C01 while keeping the rest unchanged; that stays under the limit as well. Let me know which version you prefer—happy to lock this in now with the single-swap set above.'} 
	  } 
	   } 
	   } 
	 & \\ 
 

    \theutterance \stepcounter{utterance}  

    & & & \multicolumn{4}{p{0.6\linewidth}}{\cellcolor[rgb]{1,0.85,0.72}{%
	\makecell[{{p{\linewidth}}}]{% 
	  \tt {\tiny [GM$\langle$B]}  
	 STRATEGIC REASONING: {'Their latest proposal gives me value 4093 at effort 3860. A single, very plausible improvement for me is swapping C00 -> C01. They’ve previously shown interest in C01, and the swap keeps us under the limit (4020/4026) while boosting my value by +268 to 4361. I’ll make this one-change counter to maximize acceptance chances. To close quickly, I’ll signal that if they prefer keeping C00, I’m ready to accept their current set on the next turn.'}\\ \tt PROPOSAL: {'B35', 'A10', 'A87', 'B86', 'C01', 'B32', 'C54', 'A77', 'B41'}\\ \tt ARGUMENT: {'I’m keeping your entire set structure and making just one minimal substitution you’ve already indicated you like: swap C00 -> C01. This remains safely within the cap (4020/4026), preserves all your other choices (A10, A77, C54, B41, B86, B35, B32, A87), and aligns with your earlier willingness to include C01. If you strongly prefer to keep C00 instead, say the word and I’m ready to lock in your previous set as-is.'} 
	  } 
	   } 
	   } 
	  \\ 
 

    \theutterance \stepcounter{utterance}  

    & & \multicolumn{4}{p{0.6\linewidth}}{\cellcolor[rgb]{0.9,0.9,0.9}{%
	\makecell[{{p{\linewidth}}}]{% 
	  \tt {\tiny [A$\langle$GM]}  
	 PROPOSAL: {'B35', 'A10', 'A87', 'B86', 'C01', 'B32', 'C54', 'A77', 'B41'}\\ \tt ARGUMENT: {'I’m keeping your entire set structure and making just one minimal substitution you’ve already indicated you like: swap C00 -> C01. This remains safely within the cap (4020/4026), preserves all your other choices (A10, A77, C54, B41, B86, B35, B32, A87), and aligns with your earlier willingness to include C01. If you strongly prefer to keep C00 instead, say the word and I’m ready to lock in your previous set as-is.'} 
	  } 
	   } 
	   } 
	 & \\ 
 

    \theutterance \stepcounter{utterance}  

    & \multicolumn{4}{p{0.6\linewidth}}{\cellcolor[rgb]{0.8,1,0.9}{%
	\makecell[{{p{\linewidth}}}]{% 
	  \tt {\tiny [A$\rangle$GM]}  
	 STRATEGIC REASONING: {'Their latest proposal gives me value 5140 within 4020/4026 effort and includes nearly all my high-priority items. While I could try to push for swapping C54 + A77 -> C00 (raising my value by ~249), they just signaled readiness to finalize, and the incremental gain may risk prolonging negotiations. Accepting now secures a strong outcome with high certainty.'}\\ \tt AGREE: {'B35', 'A10', 'A87', 'B86', 'C01', 'B32', 'C54', 'A77', 'B41'}\\ \tt ARGUMENT: {'This set is within the limit (4020/4026), preserves the shared structure we converged on, and balances efficiency with your preferences. I’m ready to finalize on this.'} 
	  } 
	   } 
	   } 
	 & & \\ 
 

    \theutterance \stepcounter{utterance}  

    & & & \multicolumn{2}{p{0.3\linewidth}}{\cellcolor[rgb]{0.95,0.95,0.95}{%
	\makecell[{{p{\linewidth}}}]{% 
	  \tt {\tiny [GM$|$GM]}  
	 game successful 
	  } 
	   } 
	   } 
	 & & \\ 
 

    \theutterance \stepcounter{utterance}  

    & & & \multicolumn{2}{p{0.3\linewidth}}{\cellcolor[rgb]{0.95,0.95,0.95}{%
	\makecell[{{p{\linewidth}}}]{% 
	  \tt {\tiny [GM$|$GM]}  
	 end game 
	  } 
	   } 
	   } 
	 & & \\ 
 

\end{supertabular}
}

\end{document}
