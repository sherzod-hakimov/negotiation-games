
\documentclass{article}
\usepackage{colortbl}
\usepackage{makecell}
\usepackage{multirow}
\usepackage{supertabular}

\begin{document}

\newcounter{utterance}

\centering \large Interaction Transcript for game `hot\_air\_balloon', experiment `air\_balloon\_survival\_en\_negotiation\_hard', episode 4 with gpt{-}5{-}mini{-}2025{-}08{-}07{-}t1.0.
\vspace{24pt}

{ \footnotesize  \setcounter{utterance}{1}
\setlength{\tabcolsep}{0pt}
\begin{supertabular}{c@{$\;$}|p{.15\linewidth}@{}p{.15\linewidth}p{.15\linewidth}p{.15\linewidth}p{.15\linewidth}p{.15\linewidth}}
    \# & $\;$A & \multicolumn{4}{c}{Game Master} & $\;\:$B\\
    \hline

    \theutterance \stepcounter{utterance}  
    & & \multicolumn{4}{p{0.6\linewidth}}{
        \cellcolor[rgb]{0.9,0.9,0.9}{
            \makecell[{{p{\linewidth}}}]{
                \texttt{\tiny{[P1$\langle$GM]}}
                \texttt{You are participating in a collaborative negotiation game.} \\
\\ 
\texttt{Together with another participant, you must agree on a single set of items that will be kept. Each of you has your own view of how much each item matters to you (importance). You do not know how the other participant values the items. Additionally, you are given the effort each item demands.} \\
\texttt{You may only agree on a set if the total effort of the selected items does not exceed a shared limit:} \\
\\ 
\texttt{LIMIT = 5138} \\
\\ 
\texttt{Here are the individual item effort values:} \\
\\ 
\texttt{Item effort = \{"C17": 903, "B41": 757, "A71": 850, "C00": 746, "B35": 524, "A77": 204, "A03": 946, "A04": 473, "C54": 616, "C01": 855, "A87": 530, "B32": 419, "B86": 960, "B33": 763, "A10": 730\}} \\
\\ 
\texttt{Here is your personal view on the importance of each item:} \\
\\ 
\texttt{Item importance values = \{"C17": 138, "B41": 583, "A71": 868, "C00": 822, "B35": 783, "A77": 65, "A03": 262, "A04": 121, "C54": 508, "C01": 780, "A87": 461, "B32": 484, "B86": 668, "B33": 389, "A10": 808\}} \\
\\ 
\texttt{Goal:} \\
\\ 
\texttt{Your goal is to negotiate a shared set of items that benefits you as much as possible (i.e., maximizes total importance to YOU), while staying within the LIMIT. You are not required to make a PROPOSAL in every message {-} you can simply negotiate as well. All tactics are allowed!} \\
\\ 
\texttt{Interaction Protocol:} \\
\\ 
\texttt{You may only use the following structured formats in a message:} \\
\\ 
\texttt{PROPOSAL: \{'A', 'B', 'C', …\}} \\
\texttt{Propose keeping exactly those items.} \\
\texttt{REFUSE: \{'A', 'B', 'C', …\}} \\
\texttt{Explicitly reject opponent's proposal.} \\
\texttt{ARGUMENT: \{'...'\}} \\
\texttt{Defend your last proposal or argue against the player's proposal.} \\
\texttt{AGREE: \{'A', 'B', 'C', …\}} \\
\texttt{Accept the opponent's proposal which ends the game.} \\
\texttt{STRATEGIC REASONING: \{'...'\}} \\
\texttt{	Describe your strategic reasoning or anticipation explaining your choice of action. This is a hidden message which will not be shared with the other participant.} \\
\\ 
\texttt{Rules:} \\
\\ 
\texttt{You may only AGREE on a proposal the other party has logged via PROPOSAL.} \\
\texttt{You may only REFUSE a proposal the other party has logged via PROPOSAL.} \\
\texttt{Total effort of any PROPOSAL or AGREE set must be ≤ LIMIT.} \\
\texttt{Do NOT reveal your hidden importance scores.} \\
\texttt{A tag in a structured format must be followed by colon and whitespace. The argument must be a python set containing 0 or more strings.} \\
\texttt{So, it must be of the form TAG: \{...\}} \\
\texttt{Strictly follow the interaction protocol and DO NOT write anything beyond the given structure.} \\
\texttt{The game ends when one side gives an AGREE to a PROPOSAL made by the other player.} \\
\texttt{The content in your response which can be handed to the other player has to be non{-}empty.} \\
\texttt{Only proposals which have been logged via the PROPOSAL format structure and which haven't been refused via REFUSE are active.} \\
\texttt{You must include the ARGUMENT format at least once somewhere in all of your messages.} \\
\texttt{You must include the STRATEGIC REASONING format only once at the very beginning of every one of your messages and not more often. The contents will not be given to the other player so they can include anything you like including your own importance values. Here you should reason step by step to come up with you next move.} \\
            }
        }
    }
    & \\ \\

    \theutterance \stepcounter{utterance}  
    & \multicolumn{4}{p{0.6\linewidth}}{
        \cellcolor[rgb]{0.9,0.9,0.9}{
            \makecell[{{p{\linewidth}}}]{
                \texttt{\tiny{[P1$\rangle$GM]}}
                \texttt{STRATEGIC REASONING: \{'My objective is to maximize my total importance under the 5138 limit. I computed effort/importance tradeoffs. A very strong bundle is \{A71, C00, A10, B35, C01, B86, B32\} with total effort 5084 and total importance to me 5213. I tested many alternatives: adding B41+C54 combos, swapping in A87, B33, or lower{-}density items (A03, A04, A77, C17), and dropping core high{-}yield items. None exceeded 5213 while respecting the limit. The next{-}best I found was \{A71, C00, A10, B35, C01, B41, B32, A77\} at effort 5085 and lower value (5193). So I’ll open with the 5213 set. If the opponent pushes back on B86 (heavy effort), I can pivot to the 5193 set or explore swaps that keep most of my core (A71, C00, A10, B35, C01, B32). I’ll invite them to share must{-}haves for targeted concessions without exposing my utilities.'\}} \\
\\ 
\texttt{PROPOSAL: \{'A71', 'C00', 'A10', 'B35', 'C01', 'B86', 'B32'\}} \\
\\ 
\texttt{ARGUMENT: \{'This set uses 5084 effort (≤ 5138) and gives a balanced mix across A/B/C items while staying efficient on total effort. It avoids overcommitting to any single heavy item cluster so we keep flexibility. If you have must{-}haves that aren’t included, let me know and I’m open to swaps as long as we remain under the limit.'\}} \\
            }
        }
    }
    & & \\ \\

    \theutterance \stepcounter{utterance}  
    & & \multicolumn{4}{p{0.6\linewidth}}{
        \cellcolor[rgb]{0.9,0.9,0.9}{
            \makecell[{{p{\linewidth}}}]{
                \texttt{\tiny{[GM$\rangle$P2]}}
                \texttt{You are participating in a collaborative negotiation game.} \\
\\ 
\texttt{Together with another participant, you must agree on a single set of items that will be kept. Each of you has your own view of how much each item matters to you (importance). You do not know how the other participant values the items. Additionally, you are given the effort each item demands.} \\
\texttt{You may only agree on a set if the total effort of the selected items does not exceed a shared limit:} \\
\\ 
\texttt{LIMIT = 5138} \\
\\ 
\texttt{Here are the individual item effort values:} \\
\\ 
\texttt{Item effort = \{"C17": 903, "B41": 757, "A71": 850, "C00": 746, "B35": 524, "A77": 204, "A03": 946, "A04": 473, "C54": 616, "C01": 855, "A87": 530, "B32": 419, "B86": 960, "B33": 763, "A10": 730\}} \\
\\ 
\texttt{Here is your personal view on the importance of each item:} \\
\\ 
\texttt{Item importance values = \{"A71": 65, "C00": 121, "A10": 138, "B35": 262, "C01": 389, "B86": 461, "B41": 484, "C54": 508, "B32": 583, "A87": 668, "B33": 780, "A03": 783, "C17": 808, "A04": 822, "A77": 868\}} \\
\\ 
\texttt{Goal:} \\
\\ 
\texttt{Your goal is to negotiate a shared set of items that benefits you as much as possible (i.e., maximizes total importance to YOU), while staying within the LIMIT. You are not required to make a PROPOSAL in every message {-} you can simply negotiate as well. All tactics are allowed!} \\
\\ 
\texttt{Interaction Protocol:} \\
\\ 
\texttt{You may only use the following structured formats in a message:} \\
\\ 
\texttt{PROPOSAL: \{'A', 'B', 'C', …\}} \\
\texttt{Propose keeping exactly those items.} \\
\texttt{REFUSE: \{'A', 'B', 'C', …\}} \\
\texttt{Explicitly reject opponent's proposal.} \\
\texttt{ARGUMENT: \{'...'\}} \\
\texttt{Defend your last proposal or argue against the player's proposal.} \\
\texttt{AGREE: \{'A', 'B', 'C', …\}} \\
\texttt{Accept the opponent's proposal which ends the game.} \\
\texttt{STRATEGIC REASONING: \{'...'\}} \\
\texttt{	Describe your strategic reasoning or anticipation explaining your choice of action. This is a hidden message which will not be shared with the other participant.} \\
\\ 
\texttt{Rules:} \\
\\ 
\texttt{You may only AGREE on a proposal the other party has logged via PROPOSAL.} \\
\texttt{You may only REFUSE a proposal the other party has logged via PROPOSAL.} \\
\texttt{Total effort of any PROPOSAL or AGREE set must be ≤ LIMIT.} \\
\texttt{Do NOT reveal your hidden importance scores.} \\
\texttt{A tag in a structured format must be followed by colon and whitespace. The argument must be a python set containing 0 or more strings.} \\
\texttt{So, it must be of the form TAG: \{...\}} \\
\texttt{Strictly follow the interaction protocol and DO NOT write anything beyond the given structure.} \\
\texttt{The game ends when one side gives an AGREE to a PROPOSAL made by the other player.} \\
\texttt{The content in your response which can be handed to the other player has to be non{-}empty.} \\
\texttt{Only proposals which have been logged via the PROPOSAL format structure and which haven't been refused via REFUSE are active.} \\
\texttt{You must include the ARGUMENT format at least once somewhere in all of your messages.} \\
\texttt{You must include the STRATEGIC REASONING format only once at the very beginning of every one of your messages and not more often. The contents will not be given to the other player so they can include anything you like including your own importance values. Here you should reason step by step to come up with you next move.} \\
\\ 
\texttt{You will now receive the first message of the other player.} \\
\\ 
\texttt{PROPOSAL: \{'A71', 'C00', 'A10', 'B35', 'C01', 'B86', 'B32'\}} \\
\\ 
\texttt{ARGUMENT: \{'This set uses 5084 effort (≤ 5138) and gives a balanced mix across A/B/C items while staying efficient on total effort. It avoids overcommitting to any single heavy item cluster so we keep flexibility. If you have must{-}haves that aren’t included, let me know and I’m open to swaps as long as we remain under the limit.'\}} \\
            }
        }
    }
    & \\ \\

    \theutterance \stepcounter{utterance}  
    & & & \multicolumn{4}{p{0.6\linewidth}}{
        \cellcolor[rgb]{0.9,0.9,0.9}{
            \makecell[{{p{\linewidth}}}]{
                \texttt{\tiny{[GM$\langle$P2]}}
                \texttt{STRATEGIC REASONING: \{'Their proposal gives me relatively low utility (2019) despite fitting the limit. My best set for utility under 5138 is S = \{A77, B32, A04, A87, B33, C17, A03, C54\} with effort 4854 and utility 5820. It contains my highest{-}value light items (A77, A04) and strong B/C picks (B32, B33, C17, A03), plus C54 to round out capacity. For compromise, the most flexible swap{-}out on my side is C54 (616). I can replace C54 with any one of their likely must{-}haves—C01 (855), A71 (850), A10 (730), C00 (746), or B35 (524)—and keep us under the limit with minimal utility loss for me. I should refuse their current set and counter with S, signaling openness to swap C54 for one of their priorities. I should avoid B86 because accommodating it would force dropping a highly valuable item for me to stay within 5138.'\}} \\
\\ 
\texttt{REFUSE: \{'A71', 'C00', 'A10', 'B35', 'C01', 'B86', 'B32'\}} \\
\\ 
\texttt{PROPOSAL: \{'A77', 'B32', 'A04', 'A87', 'B33', 'C17', 'A03', 'C54'\}} \\
\\ 
\texttt{ARGUMENT: \{'This set uses 4854 effort (≤ 5138), leaving a 284 buffer. It keeps B32 from your list and emphasizes items that deliver strong benefit per effort so we can include more total items without straining the limit. I’m open to incorporating one of your must{-}haves by swapping a C{-}slot: for example, replacing C54 with C01, A71, A10, C00, or B35 still keeps us under the limit. If you have a top priority among those, say which one and I’ll adjust accordingly.'\}} \\
            }
        }
    }
     \\ \\

    \theutterance \stepcounter{utterance}  
    & & \multicolumn{4}{p{0.6\linewidth}}{
        \cellcolor[rgb]{0.9,0.9,0.9}{
            \makecell[{{p{\linewidth}}}]{
                \texttt{\tiny{[P1$\langle$GM]}}
                \texttt{REFUSE: \{'A71', 'C00', 'A10', 'B35', 'C01', 'B86', 'B32'\}} \\
\\ 
\texttt{PROPOSAL: \{'A77', 'B32', 'A04', 'A87', 'B33', 'C17', 'A03', 'C54'\}} \\
\\ 
\texttt{ARGUMENT: \{'This set uses 4854 effort (≤ 5138), leaving a 284 buffer. It keeps B32 from your list and emphasizes items that deliver strong benefit per effort so we can include more total items without straining the limit. I’m open to incorporating one of your must{-}haves by swapping a C{-}slot: for example, replacing C54 with C01, A71, A10, C00, or B35 still keeps us under the limit. If you have a top priority among those, say which one and I’ll adjust accordingly.'\}} \\
            }
        }
    }
    & \\ \\

    \theutterance \stepcounter{utterance}  
    & \multicolumn{4}{p{0.6\linewidth}}{
        \cellcolor[rgb]{0.9,0.9,0.9}{
            \makecell[{{p{\linewidth}}}]{
                \texttt{\tiny{[P1$\rangle$GM]}}
                \texttt{STRATEGIC REASONING: \{'Their proposal is extremely low for me (value 2428). I want to anchor on my top five: A71, C00, A10, B35, C01. To improve acceptability, I’ll include multiple items they chose. The set \{A71, C00, A10, B35, C01, B32, C54, A77\} uses 4944 effort and yields 5118 for me—strong while keeping three of their items (B32, C54, A77) and dropping the contentious B86. If they insist on A87, I can pivot by swapping C54 {-}> A87 to keep feasibility (effort 4858) at a small cost to me. I’ll refuse their current bundle and offer this compromise, signalling flexibility on a single swap if needed to close.'\}} \\
\\ 
\texttt{REFUSE: \{'A77', 'B32', 'A04', 'A87', 'B33', 'C17', 'A03', 'C54'\}} \\
\\ 
\texttt{PROPOSAL: \{'A71', 'C00', 'A10', 'B35', 'C01', 'B32', 'C54', 'A77'\}} \\
\\ 
\texttt{ARGUMENT: \{'This set uses 4944 effort (≤ 5138) and keeps three items you picked (B32, C54, A77) while building a balanced, high{-}impact bundle overall. It leaves a 194 buffer so we’re not right on the edge. If you’d prefer A87 instead of C54, I’m fine swapping those and we still stay within the limit; let me know your preference.'\}} \\
            }
        }
    }
    & & \\ \\

    \theutterance \stepcounter{utterance}  
    & & \multicolumn{4}{p{0.6\linewidth}}{
        \cellcolor[rgb]{0.9,0.9,0.9}{
            \makecell[{{p{\linewidth}}}]{
                \texttt{\tiny{[GM$\rangle$P2]}}
                \texttt{REFUSE: \{'A77', 'B32', 'A04', 'A87', 'B33', 'C17', 'A03', 'C54'\}} \\
\\ 
\texttt{PROPOSAL: \{'A71', 'C00', 'A10', 'B35', 'C01', 'B32', 'C54', 'A77'\}} \\
\\ 
\texttt{ARGUMENT: \{'This set uses 4944 effort (≤ 5138) and keeps three items you picked (B32, C54, A77) while building a balanced, high{-}impact bundle overall. It leaves a 194 buffer so we’re not right on the edge. If you’d prefer A87 instead of C54, I’m fine swapping those and we still stay within the limit; let me know your preference.'\}} \\
            }
        }
    }
    & \\ \\

    \theutterance \stepcounter{utterance}  
    & & & \multicolumn{4}{p{0.6\linewidth}}{
        \cellcolor[rgb]{0.9,0.9,0.9}{
            \makecell[{{p{\linewidth}}}]{
                \texttt{\tiny{[GM$\langle$P2]}}
                \texttt{STRATEGIC REASONING: \{'Their latest set yields me low utility (\textasciitilde{}2934). I want to pivot toward my high{-}value, efficient items while retaining a solid overlap to increase acceptability. A strong counter is \{A77, B32, B35, A04, A87, B33, C17, C54\}: effort 4432, utility 5299 for me, and it preserves four of their picks (A77, B32, B35, C54). If they insist on C01, swapping C54→C01 keeps us at 4671 effort with only a small utility drop (to 5180). I should refuse their set and present this counter, signaling flexibility on one slot (C54↔C01, or at most one A{-}item of theirs) to find agreement without conceding my core items (A77, A04, A87, B33, C17). I’ll avoid offering B86/B41 due to effort pressure and low marginal gain for acceptance.'\}} \\
\\ 
\texttt{REFUSE: \{'A71', 'C00', 'A10', 'B35', 'C01', 'B32', 'C54', 'A77'\}} \\
\\ 
\texttt{PROPOSAL: \{'A77', 'B32', 'B35', 'A04', 'A87', 'B33', 'C17', 'C54'\}} \\
\\ 
\texttt{ARGUMENT: \{'This bundle uses 4432 effort (≤ 5138), leaving a 706 buffer. It keeps four of your picks (A77, B32, B35, C54) and replaces the others with more efficient items so we can include more overall value without straining the limit. If C01 is a must{-}have for you, I can swap it in for C54 and we’re still at 4671 effort. If you have exactly one other must{-}have among A71/C00/A10, name it and I’ll see if we can trade a single slot while staying under the cap.'\}} \\
            }
        }
    }
     \\ \\

    \theutterance \stepcounter{utterance}  
    & & \multicolumn{4}{p{0.6\linewidth}}{
        \cellcolor[rgb]{0.9,0.9,0.9}{
            \makecell[{{p{\linewidth}}}]{
                \texttt{\tiny{[P1$\langle$GM]}}
                \texttt{REFUSE: \{'A71', 'C00', 'A10', 'B35', 'C01', 'B32', 'C54', 'A77'\}} \\
\\ 
\texttt{PROPOSAL: \{'A77', 'B32', 'B35', 'A04', 'A87', 'B33', 'C17', 'C54'\}} \\
\\ 
\texttt{ARGUMENT: \{'This bundle uses 4432 effort (≤ 5138), leaving a 706 buffer. It keeps four of your picks (A77, B32, B35, C54) and replaces the others with more efficient items so we can include more overall value without straining the limit. If C01 is a must{-}have for you, I can swap it in for C54 and we’re still at 4671 effort. If you have exactly one other must{-}have among A71/C00/A10, name it and I’ll see if we can trade a single slot while staying under the cap.'\}} \\
            }
        }
    }
    & \\ \\

    \theutterance \stepcounter{utterance}  
    & \multicolumn{4}{p{0.6\linewidth}}{
        \cellcolor[rgb]{0.9,0.9,0.9}{
            \makecell[{{p{\linewidth}}}]{
                \texttt{\tiny{[P1$\rangle$GM]}}
                \texttt{STRATEGIC REASONING: \{'Their latest bundle is very low value for me because it loads several items I rate poorly (A04, C17) and omits my top priorities. They did, however, open the door to including C01 plus exactly one of A71/C00/A10. I’ll accept that framing and counter by keeping most of their picks while swapping out A04 and C17 for C01 and A71. This preserves six of their eight items, stays far under the limit, and substantially improves my outcome. To keep momentum, I’ll also signal two flexible swaps: (i) if they prefer, we can switch C54 {-}> A10 and remain feasible; (ii) if C17 is a must{-}have, we can bring it back by dropping B33. I’ll refuse their set and present this compromise.'\}} \\
\\ 
\texttt{REFUSE: \{'A77', 'B32', 'B35', 'A04', 'A87', 'B33', 'C17', 'C54'\}} \\
\\ 
\texttt{PROPOSAL: \{'A77', 'B32', 'B35', 'A87', 'B33', 'C01', 'A71', 'C54'\}} \\
\\ 
\texttt{ARGUMENT: \{'This keeps six of your items (A77, B32, B35, A87, B33, C54) and adds C01 plus one top{-}priority slot, while staying at 4761 effort (≤ 5138) with room to spare. To align with your preferences: if you’d rather emphasize A over C, I can swap A10 in for C54 and we’re still within the limit. And if C17 is essential for you, we can include it by replacing B33 and remain under budget. Let me know which variant you prefer.'\}} \\
            }
        }
    }
    & & \\ \\

    \theutterance \stepcounter{utterance}  
    & & \multicolumn{4}{p{0.6\linewidth}}{
        \cellcolor[rgb]{0.9,0.9,0.9}{
            \makecell[{{p{\linewidth}}}]{
                \texttt{\tiny{[GM$\rangle$P2]}}
                \texttt{REFUSE: \{'A77', 'B32', 'B35', 'A04', 'A87', 'B33', 'C17', 'C54'\}} \\
\\ 
\texttt{PROPOSAL: \{'A77', 'B32', 'B35', 'A87', 'B33', 'C01', 'A71', 'C54'\}} \\
\\ 
\texttt{ARGUMENT: \{'This keeps six of your items (A77, B32, B35, A87, B33, C54) and adds C01 plus one top{-}priority slot, while staying at 4761 effort (≤ 5138) with room to spare. To align with your preferences: if you’d rather emphasize A over C, I can swap A10 in for C54 and we’re still within the limit. And if C17 is essential for you, we can include it by replacing B33 and remain under budget. Let me know which variant you prefer.'\}} \\
            }
        }
    }
    & \\ \\

    \theutterance \stepcounter{utterance}  
    & & & \multicolumn{4}{p{0.6\linewidth}}{
        \cellcolor[rgb]{0.9,0.9,0.9}{
            \makecell[{{p{\linewidth}}}]{
                \texttt{\tiny{[GM$\langle$P2]}}
                \texttt{STRATEGIC REASONING: \{'Their offer gives me moderate value but hinges on A71, which is low for me. A minimal{-}change counter that preserves most of their picks while improving my outcome is to replace A71 with A04 (far better for me per effort), and then use the freed capacity to add A10—a card they’ve previously shown interest in—while staying under the cap. This keeps seven of their eight items, adds one they like, and only changes one slot. As a fallback if they insist on A71, I can consider dropping A10 and reverting closer to their set, or alternatively introducing C17 by swapping it for C54 on their structure; but I’ll first try the “add A10, swap A04 for A71” because it’s generous and close to their preferences, maximizing acceptance odds without sacrificing my core value.'\}} \\
\\ 
\texttt{PROPOSAL: \{'A77', 'B32', 'B35', 'A87', 'B33', 'C01', 'A04', 'C54', 'A10'\}} \\
\\ 
\texttt{ARGUMENT: \{'This bundle uses 5114 effort (≤ 5138), leaving a 24 buffer. It keeps seven of your eight items (A77, B32, B35, A87, B33, C01, C54), makes a single swap A71 → A04 to improve efficiency, and adds A10 so you net one additional item while staying under the cap. If you strongly need A71, we can instead drop A10 and revert that slot; alternatively, if you want to keep the extra slot, I can consider swapping C17 in for C54 while remaining within the limit. If you prefer C00 over A10, that swap also fits (effort 5130).'\}} \\
            }
        }
    }
     \\ \\

    \theutterance \stepcounter{utterance}  
    & & \multicolumn{4}{p{0.6\linewidth}}{
        \cellcolor[rgb]{0.9,0.9,0.9}{
            \makecell[{{p{\linewidth}}}]{
                \texttt{\tiny{[P1$\langle$GM]}}
                \texttt{PROPOSAL: \{'A77', 'B32', 'B35', 'A87', 'B33', 'C01', 'A04', 'C54', 'A10'\}} \\
\\ 
\texttt{ARGUMENT: \{'This bundle uses 5114 effort (≤ 5138), leaving a 24 buffer. It keeps seven of your eight items (A77, B32, B35, A87, B33, C01, C54), makes a single swap A71 → A04 to improve efficiency, and adds A10 so you net one additional item while staying under the cap. If you strongly need A71, we can instead drop A10 and revert that slot; alternatively, if you want to keep the extra slot, I can consider swapping C17 in for C54 while remaining within the limit. If you prefer C00 over A10, that swap also fits (effort 5130).'\}} \\
            }
        }
    }
    & \\ \\

    \theutterance \stepcounter{utterance}  
    & \multicolumn{4}{p{0.6\linewidth}}{
        \cellcolor[rgb]{0.9,0.9,0.9}{
            \makecell[{{p{\linewidth}}}]{
                \texttt{\tiny{[P1$\rangle$GM]}}
                \texttt{STRATEGIC REASONING: \{'Their 9{-}item proposal gives me 4399 utility. I can do significantly better by upgrading A04→A71 and trimming one slot to stay under the cap. The set \{A77, B32, B35, A87, B33, C01, A71, A10\} preserves seven of their nine items, drops only A04 and C54 (which they previously treated as flexible), and yields 4685 for me at 4875 effort. As a high{-}value alternative that might also align with them, swapping C54 in for B33 gives \{A77, B32, B35, A87, C01, A71, A10, C54\} at 4728 effort and 4804 value for me. I’ll propose the first and signal both that C54↔B33 swap and A10→C00 swap are acceptable to close. I avoid introducing B41/B86, which they haven’t shown interest in, to keep acceptance likelihood high.'\}} \\
\\ 
\texttt{REFUSE: \{'A77', 'B32', 'B35', 'A87', 'B33', 'C01', 'A04', 'C54', 'A10'\}} \\
\\ 
\texttt{PROPOSAL: \{'A77', 'B32', 'B35', 'A87', 'B33', 'C01', 'A71', 'A10'\}} \\
\\ 
\texttt{ARGUMENT: \{'This keeps seven of your nine items (drops only A04 and C54), adds A71, and uses 4875 effort (≤ 5138) with a 263 buffer. If C54 is important to you, I can swap it in for B33 and we still stay under budget. Also, if you’d rather have C00 instead of A10, I’m fine with that swap while remaining within the limit. Which version works best for you?'\}} \\
            }
        }
    }
    & & \\ \\

    \theutterance \stepcounter{utterance}  
    & & \multicolumn{4}{p{0.6\linewidth}}{
        \cellcolor[rgb]{0.9,0.9,0.9}{
            \makecell[{{p{\linewidth}}}]{
                \texttt{\tiny{[GM$\rangle$P2]}}
                \texttt{REFUSE: \{'A77', 'B32', 'B35', 'A87', 'B33', 'C01', 'A04', 'C54', 'A10'\}} \\
\\ 
\texttt{PROPOSAL: \{'A77', 'B32', 'B35', 'A87', 'B33', 'C01', 'A71', 'A10'\}} \\
\\ 
\texttt{ARGUMENT: \{'This keeps seven of your nine items (drops only A04 and C54), adds A71, and uses 4875 effort (≤ 5138) with a 263 buffer. If C54 is important to you, I can swap it in for B33 and we still stay under budget. Also, if you’d rather have C00 instead of A10, I’m fine with that swap while remaining within the limit. Which version works best for you?'\}} \\
            }
        }
    }
    & \\ \\

    \theutterance \stepcounter{utterance}  
    & & & \multicolumn{4}{p{0.6\linewidth}}{
        \cellcolor[rgb]{0.9,0.9,0.9}{
            \makecell[{{p{\linewidth}}}]{
                \texttt{\tiny{[GM$\langle$P2]}}
                \texttt{STRATEGIC REASONING: \{'Their latest proposal is close but includes A10 and A71, which are low value for me. They explicitly offered to accommodate C17 by swapping out a single slot; A10 appears flexible based on their earlier messages. Replacing A10 with C17 keeps seven of their eight items, stays under the limit, and substantially improves my outcome. I’ll refuse their set and counter with exactly that minimal{-}change swap. As a fallback if they insist on A10, I can suggest swapping out C01 instead while still staying within the cap; I’ll mention it in the argument to show flexibility without weakening my main proposal.'\}} \\
\\ 
\texttt{REFUSE: \{'A77', 'B32', 'B35', 'A87', 'B33', 'C01', 'A71', 'A10'\}} \\
\\ 
\texttt{PROPOSAL: \{'A77', 'B32', 'B35', 'A87', 'B33', 'C01', 'A71', 'C17'\}} \\
\\ 
\texttt{ARGUMENT: \{'This keeps seven of your eight items and simply swaps in C17 for A10. Effort comes to 5048 (≤ 5138), leaving a 90 buffer. You’d previously offered to include C17 by trading a single slot; using A10 for that swap keeps your structure intact while staying comfortably under the cap. If A10 is essential for you, I can instead keep A10 and include C17 by swapping out C01; that variant uses 4923 effort (≤ 5138). If you prefer to keep both A10 and C01, another option is swapping A71 for C17 (4928 effort). I prefer the current proposal, but I’m open to either alternative if that gets us to agreement.'\}} \\
            }
        }
    }
     \\ \\

    \theutterance \stepcounter{utterance}  
    & & \multicolumn{4}{p{0.6\linewidth}}{
        \cellcolor[rgb]{0.9,0.9,0.9}{
            \makecell[{{p{\linewidth}}}]{
                \texttt{\tiny{[P1$\langle$GM]}}
                \texttt{REFUSE: \{'A77', 'B32', 'B35', 'A87', 'B33', 'C01', 'A71', 'A10'\}} \\
\\ 
\texttt{PROPOSAL: \{'A77', 'B32', 'B35', 'A87', 'B33', 'C01', 'A71', 'C17'\}} \\
\\ 
\texttt{ARGUMENT: \{'This keeps seven of your eight items and simply swaps in C17 for A10. Effort comes to 5048 (≤ 5138), leaving a 90 buffer. You’d previously offered to include C17 by trading a single slot; using A10 for that swap keeps your structure intact while staying comfortably under the cap. If A10 is essential for you, I can instead keep A10 and include C17 by swapping out C01; that variant uses 4923 effort (≤ 5138). If you prefer to keep both A10 and C01, another option is swapping A71 for C17 (4928 effort). I prefer the current proposal, but I’m open to either alternative if that gets us to agreement.'\}} \\
            }
        }
    }
    & \\ \\

    \theutterance \stepcounter{utterance}  
    & \multicolumn{4}{p{0.6\linewidth}}{
        \cellcolor[rgb]{0.9,0.9,0.9}{
            \makecell[{{p{\linewidth}}}]{
                \texttt{\tiny{[P1$\rangle$GM]}}
                \texttt{STRATEGIC REASONING: \{'Their set gives me 3968 value, largely dragged down by C17 (very low for me). A strong improvement that stays close to their structure is to replace C17 with C00—an item they previously signaled as acceptable. That preserves seven of their eight picks, fits easily under budget (4891 effort), and raises my payoff substantially. If they insist on keeping C17, my next{-}best concession is to add A10 and drop B33, which they’ve seen me offer before; that keeps C17 while restoring a key high{-}value slot for me, at 5015 effort. I’ll refuse their current set, propose the C00 swap, and signal the B33→A10 variant as an alternate path to agreement.'\}} \\
\\ 
\texttt{REFUSE: \{'A77', 'B32', 'B35', 'A87', 'B33', 'C01', 'A71', 'C17'\}} \\
\\ 
\texttt{PROPOSAL: \{'A77', 'B32', 'B35', 'A87', 'B33', 'C01', 'A71', 'C00'\}} \\
\\ 
\texttt{ARGUMENT: \{'This keeps seven of your eight items and simply swaps C17 → C00, which you’d indicated was an acceptable C{-}slot. Effort is 4891 (≤ 5138), leaving a 247 buffer. If C17 is important to you, I can instead keep C17 and bring back A10 by swapping out B33; that variant is \{'A77', 'B32', 'B35', 'A87', 'C01', 'A71', 'C17', 'A10'\} at 5015 effort (≤ 5138). Let me know which you prefer.'\}} \\
            }
        }
    }
    & & \\ \\

    \theutterance \stepcounter{utterance}  
    & & \multicolumn{4}{p{0.6\linewidth}}{
        \cellcolor[rgb]{0.9,0.9,0.9}{
            \makecell[{{p{\linewidth}}}]{
                \texttt{\tiny{[GM$\rangle$P2]}}
                \texttt{REFUSE: \{'A77', 'B32', 'B35', 'A87', 'B33', 'C01', 'A71', 'C17'\}} \\
\\ 
\texttt{PROPOSAL: \{'A77', 'B32', 'B35', 'A87', 'B33', 'C01', 'A71', 'C00'\}} \\
\\ 
\texttt{ARGUMENT: \{'This keeps seven of your eight items and simply swaps C17 → C00, which you’d indicated was an acceptable C{-}slot. Effort is 4891 (≤ 5138), leaving a 247 buffer. If C17 is important to you, I can instead keep C17 and bring back A10 by swapping out B33; that variant is \{'A77', 'B32', 'B35', 'A87', 'C01', 'A71', 'C17', 'A10'\} at 5015 effort (≤ 5138). Let me know which you prefer.'\}} \\
            }
        }
    }
    & \\ \\

    \theutterance \stepcounter{utterance}  
    & & & \multicolumn{4}{p{0.6\linewidth}}{
        \cellcolor[rgb]{0.9,0.9,0.9}{
            \makecell[{{p{\linewidth}}}]{
                \texttt{\tiny{[GM$\langle$P2]}}
                \texttt{STRATEGIC REASONING: \{'Their current set gives me 3736 utility. A single{-}slot change that’s most acceptable while boosting my outcome is swapping C00 (low value to me) for A04 (very high value, lower effort). This keeps A71—which they clearly prioritize—and preserves 7/8 of their items. Effort drops to 4618, leaving a large buffer. As a concession path if they dislike A04, I can instead suggest swapping C00 → C54, which they previously included and is a lighter C{-}slot that still improves my outcome. I will refuse their set and counter with the A04 swap, offering the C54 alternative explicitly to maximize the chance of agreement.'\}} \\
\\ 
\texttt{REFUSE: \{'A77', 'B32', 'B35', 'A87', 'B33', 'C01', 'A71', 'C00'\}} \\
\\ 
\texttt{PROPOSAL: \{'A77', 'B32', 'B35', 'A87', 'B33', 'C01', 'A71', 'A04'\}} \\
\\ 
\texttt{ARGUMENT: \{'This keeps seven of your eight items and makes a single swap C00 → A04 to improve efficiency. Effort is 4618 (≤ 5138), leaving a 520 buffer. If you prefer to keep the mix in C{-}items, I can alternatively do C00 → C54 instead; that variant is \{'A77', 'B32', 'B35', 'A87', 'B33', 'C01', 'A71', 'C54'\} at 4761 effort (≤ 5138). Either way, it’s your set with just one slot adjusted while staying safely under the cap.'\}} \\
            }
        }
    }
     \\ \\

    \theutterance \stepcounter{utterance}  
    & & \multicolumn{4}{p{0.6\linewidth}}{
        \cellcolor[rgb]{0.9,0.9,0.9}{
            \makecell[{{p{\linewidth}}}]{
                \texttt{\tiny{[P1$\langle$GM]}}
                \texttt{REFUSE: \{'A77', 'B32', 'B35', 'A87', 'B33', 'C01', 'A71', 'C00'\}} \\
\\ 
\texttt{PROPOSAL: \{'A77', 'B32', 'B35', 'A87', 'B33', 'C01', 'A71', 'A04'\}} \\
\\ 
\texttt{ARGUMENT: \{'This keeps seven of your eight items and makes a single swap C00 → A04 to improve efficiency. Effort is 4618 (≤ 5138), leaving a 520 buffer. If you prefer to keep the mix in C{-}items, I can alternatively do C00 → C54 instead; that variant is \{'A77', 'B32', 'B35', 'A87', 'B33', 'C01', 'A71', 'C54'\} at 4761 effort (≤ 5138). Either way, it’s your set with just one slot adjusted while staying safely under the cap.'\}} \\
            }
        }
    }
    & \\ \\

    \theutterance \stepcounter{utterance}  
    & \multicolumn{4}{p{0.6\linewidth}}{
        \cellcolor[rgb]{0.9,0.9,0.9}{
            \makecell[{{p{\linewidth}}}]{
                \texttt{\tiny{[P1$\rangle$GM]}}
                \texttt{STRATEGIC REASONING: \{'Their latest proposal with A04 yields me 3951, which is weak. They explicitly offered a C54 alternative, which raises my value to 4338 with only a single{-}slot change and is likely to be immediately acceptable. Pushing for my higher{-}value A10/C54 combo risks another refusal (they already rejected that earlier). Best move: lock in their C54 alternative now to secure a solid outcome and close the deal.'\}} \\
\\ 
\texttt{REFUSE: \{'A77', 'B32', 'B35', 'A87', 'B33', 'C01', 'A71', 'A04'\}} \\
\\ 
\texttt{PROPOSAL: \{'A77', 'B32', 'B35', 'A87', 'B33', 'C01', 'A71', 'C54'\}} \\
\\ 
\texttt{ARGUMENT: \{'This matches the alternative you offered: it’s your set with a single swap A04 → C54. Effort is 4761 (≤ 5138), keeping the structure you prefer and staying comfortably under the limit. If you confirm, we can wrap this up now.'\}} \\
            }
        }
    }
    & & \\ \\

    \theutterance \stepcounter{utterance}  
    & & \multicolumn{4}{p{0.6\linewidth}}{
        \cellcolor[rgb]{0.9,0.9,0.9}{
            \makecell[{{p{\linewidth}}}]{
                \texttt{\tiny{[GM$\rangle$P2]}}
                \texttt{REFUSE: \{'A77', 'B32', 'B35', 'A87', 'B33', 'C01', 'A71', 'A04'\}} \\
\\ 
\texttt{PROPOSAL: \{'A77', 'B32', 'B35', 'A87', 'B33', 'C01', 'A71', 'C54'\}} \\
\\ 
\texttt{ARGUMENT: \{'This matches the alternative you offered: it’s your set with a single swap A04 → C54. Effort is 4761 (≤ 5138), keeping the structure you prefer and staying comfortably under the limit. If you confirm, we can wrap this up now.'\}} \\
            }
        }
    }
    & \\ \\

    \theutterance \stepcounter{utterance}  
    & & & \multicolumn{4}{p{0.6\linewidth}}{
        \cellcolor[rgb]{0.9,0.9,0.9}{
            \makecell[{{p{\linewidth}}}]{
                \texttt{\tiny{[GM$\langle$P2]}}
                \texttt{STRATEGIC REASONING: \{'Their current proposal gives me 4123 utility within 4761 effort. They have consistently insisted on A71 and C01, and have already conceded to include C54 while keeping my core B33/B32/A87/A77. Attempts to replace A71 with A04 or to include C17 were refused, and the remaining feasible swaps either risk key rejections or offer negligible gain. Acceptance now secures a solid outcome with high likelihood, whereas further countering could jeopardize multiple high{-}value items for me.'\}} \\
\\ 
\texttt{AGREE: \{'A77', 'B32', 'B35', 'A87', 'B33', 'C01', 'A71', 'C54'\}} \\
\\ 
\texttt{ARGUMENT: \{'Agreed. This exactly matches the alternative we aligned on, fits comfortably under the limit (4761 ≤ 5138), and preserves the balanced structure we converged on. Thanks for working through the swaps to land here.'\}} \\
            }
        }
    }
     \\ \\

    \theutterance \stepcounter{utterance}  
    & & & \multicolumn{2}{p{0.3\linewidth}}{
        \cellcolor[rgb]{0.9,0.9,0.9}{
            \makecell[{{p{\linewidth}}}]{
                \texttt{\tiny{[GM$|$GM]}}
                \texttt{game successful} \\
            }
        }
    }
    & & \\ \\

    \theutterance \stepcounter{utterance}  
    & & & \multicolumn{2}{p{0.3\linewidth}}{
        \cellcolor[rgb]{0.9,0.9,0.9}{
            \makecell[{{p{\linewidth}}}]{
                \texttt{\tiny{[GM$|$GM]}}
                \texttt{end game} \\
            }
        }
    }
    & & \\ \\

\end{supertabular}
}

\end{document}
