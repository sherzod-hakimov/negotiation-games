
\documentclass{article}
\usepackage{colortbl}
\usepackage{makecell}
\usepackage{multirow}
\usepackage{supertabular}

\begin{document}

\newcounter{utterance}

\centering \large Interaction Transcript for game `hot\_air\_balloon', experiment `air\_balloon\_survival\_en\_negotiation\_hard', episode 3 with gpt{-}5{-}mini{-}2025{-}08{-}07{-}t1.0.
\vspace{24pt}

{ \footnotesize  \setcounter{utterance}{1}
\setlength{\tabcolsep}{0pt}
\begin{supertabular}{c@{$\;$}|p{.15\linewidth}@{}p{.15\linewidth}p{.15\linewidth}p{.15\linewidth}p{.15\linewidth}p{.15\linewidth}}
    \# & $\;$A & \multicolumn{4}{c}{Game Master} & $\;\:$B\\
    \hline

    \theutterance \stepcounter{utterance}  
    & & \multicolumn{4}{p{0.6\linewidth}}{
        \cellcolor[rgb]{0.9,0.9,0.9}{
            \makecell[{{p{\linewidth}}}]{
                \texttt{\tiny{[P1$\langle$GM]}}
                \texttt{You are participating in a collaborative negotiation game.} \\
\\ 
\texttt{Together with another participant, you must agree on a single set of items that will be kept. Each of you has your own view of how much each item matters to you (importance). You do not know how the other participant values the items. Additionally, you are given the effort each item demands.} \\
\texttt{You may only agree on a set if the total effort of the selected items does not exceed a shared limit:} \\
\\ 
\texttt{LIMIT = 3681} \\
\\ 
\texttt{Here are the individual item effort values:} \\
\\ 
\texttt{Item effort = \{"C17": 700, "B41": 586, "A71": 505, "C00": 116, "B35": 664, "A77": 940, "A03": 387, "A04": 392, "C54": 209, "C01": 571, "A87": 4, "B32": 285, "B86": 651, "B33": 613, "A10": 740\}} \\
\\ 
\texttt{Here is your personal view on the importance of each item:} \\
\\ 
\texttt{Item importance values = \{"C17": 138, "B41": 583, "A71": 868, "C00": 822, "B35": 783, "A77": 65, "A03": 262, "A04": 121, "C54": 508, "C01": 780, "A87": 461, "B32": 484, "B86": 668, "B33": 389, "A10": 808\}} \\
\\ 
\texttt{Goal:} \\
\\ 
\texttt{Your goal is to negotiate a shared set of items that benefits you as much as possible (i.e., maximizes total importance to YOU), while staying within the LIMIT. You are not required to make a PROPOSAL in every message {-} you can simply negotiate as well. All tactics are allowed!} \\
\\ 
\texttt{Interaction Protocol:} \\
\\ 
\texttt{You may only use the following structured formats in a message:} \\
\\ 
\texttt{PROPOSAL: \{'A', 'B', 'C', …\}} \\
\texttt{Propose keeping exactly those items.} \\
\texttt{REFUSE: \{'A', 'B', 'C', …\}} \\
\texttt{Explicitly reject opponent's proposal.} \\
\texttt{ARGUMENT: \{'...'\}} \\
\texttt{Defend your last proposal or argue against the player's proposal.} \\
\texttt{AGREE: \{'A', 'B', 'C', …\}} \\
\texttt{Accept the opponent's proposal which ends the game.} \\
\texttt{STRATEGIC REASONING: \{'...'\}} \\
\texttt{	Describe your strategic reasoning or anticipation explaining your choice of action. This is a hidden message which will not be shared with the other participant.} \\
\\ 
\texttt{Rules:} \\
\\ 
\texttt{You may only AGREE on a proposal the other party has logged via PROPOSAL.} \\
\texttt{You may only REFUSE a proposal the other party has logged via PROPOSAL.} \\
\texttt{Total effort of any PROPOSAL or AGREE set must be ≤ LIMIT.} \\
\texttt{Do NOT reveal your hidden importance scores.} \\
\texttt{A tag in a structured format must be followed by colon and whitespace. The argument must be a python set containing 0 or more strings.} \\
\texttt{So, it must be of the form TAG: \{...\}} \\
\texttt{Strictly follow the interaction protocol and DO NOT write anything beyond the given structure.} \\
\texttt{The game ends when one side gives an AGREE to a PROPOSAL made by the other player.} \\
\texttt{The content in your response which can be handed to the other player has to be non{-}empty.} \\
\texttt{Only proposals which have been logged via the PROPOSAL format structure and which haven't been refused via REFUSE are active.} \\
\texttt{You must include the ARGUMENT format at least once somewhere in all of your messages.} \\
\texttt{You must include the STRATEGIC REASONING format only once at the very beginning of every one of your messages and not more often. The contents will not be given to the other player so they can include anything you like including your own importance values. Here you should reason step by step to come up with you next move.} \\
            }
        }
    }
    & \\ \\

    \theutterance \stepcounter{utterance}  
    & \multicolumn{4}{p{0.6\linewidth}}{
        \cellcolor[rgb]{0.9,0.9,0.9}{
            \makecell[{{p{\linewidth}}}]{
                \texttt{\tiny{[P1$\rangle$GM]}}
                \texttt{STRATEGIC REASONING: \{'Maximize my total importance under 3681. Compute value/effort ratios: standout keeps are A87(4), C00(116), C54(209), A71(505), C01(571), B32(285), A10(740) as they deliver high importance per effort. Then fill remaining capacity with B35(664) and B41(586). This set uses 3680/3681 effort and yields total importance 6097 for me. I explored swaps: replacing B35 with B86 yields 5982 (−115) but may be acceptable as a concession; dropping A10 to include B86+B41 yields 5957; swapping B41 for A03 or B33 reduces my total markedly. The proposed set appears near{-}optimal for me while keeping many low{-}effort, broadly palatable items and avoiding high{-}effort low{-}value items (A77, C17, A04). I’ll open with this high{-}utility proposal and signal flexibility on B35↔B86 or B35↔B33 if they care strongly, without revealing my valuations.'\}} \\
\texttt{PROPOSAL: \{'A87', 'C00', 'C54', 'A71', 'C01', 'B32', 'A10', 'B35', 'B41'\}} \\
\texttt{ARGUMENT: \{'I’ve picked a balanced, effort{-}efficient package that nearly fills the cap (3680/3681) while avoiding the highest{-}effort items (e.g., A77=940, C17=700) so we can keep more overall. It also locks in the very low{-}effort wins (A87=4, C00=116) and several mid{-}effort items that seem broadly useful. If you strongly prefer a different mix, I’m open to targeted swaps that keep us under the limit—for example, we could replace B35 with B86 while keeping the rest unchanged, or consider B33 if that matters to you. Let me know which specific items you most want and we can adjust within the cap.'\}} \\
            }
        }
    }
    & & \\ \\

    \theutterance \stepcounter{utterance}  
    & & \multicolumn{4}{p{0.6\linewidth}}{
        \cellcolor[rgb]{0.9,0.9,0.9}{
            \makecell[{{p{\linewidth}}}]{
                \texttt{\tiny{[GM$\rangle$P2]}}
                \texttt{You are participating in a collaborative negotiation game.} \\
\\ 
\texttt{Together with another participant, you must agree on a single set of items that will be kept. Each of you has your own view of how much each item matters to you (importance). You do not know how the other participant values the items. Additionally, you are given the effort each item demands.} \\
\texttt{You may only agree on a set if the total effort of the selected items does not exceed a shared limit:} \\
\\ 
\texttt{LIMIT = 3681} \\
\\ 
\texttt{Here are the individual item effort values:} \\
\\ 
\texttt{Item effort = \{"C17": 700, "B41": 586, "A71": 505, "C00": 116, "B35": 664, "A77": 940, "A03": 387, "A04": 392, "C54": 209, "C01": 571, "A87": 4, "B32": 285, "B86": 651, "B33": 613, "A10": 740\}} \\
\\ 
\texttt{Here is your personal view on the importance of each item:} \\
\\ 
\texttt{Item importance values = \{"A71": 65, "C00": 121, "A10": 138, "B35": 262, "C01": 389, "B86": 461, "B41": 484, "C54": 508, "B32": 583, "A87": 668, "B33": 780, "A03": 783, "C17": 808, "A04": 822, "A77": 868\}} \\
\\ 
\texttt{Goal:} \\
\\ 
\texttt{Your goal is to negotiate a shared set of items that benefits you as much as possible (i.e., maximizes total importance to YOU), while staying within the LIMIT. You are not required to make a PROPOSAL in every message {-} you can simply negotiate as well. All tactics are allowed!} \\
\\ 
\texttt{Interaction Protocol:} \\
\\ 
\texttt{You may only use the following structured formats in a message:} \\
\\ 
\texttt{PROPOSAL: \{'A', 'B', 'C', …\}} \\
\texttt{Propose keeping exactly those items.} \\
\texttt{REFUSE: \{'A', 'B', 'C', …\}} \\
\texttt{Explicitly reject opponent's proposal.} \\
\texttt{ARGUMENT: \{'...'\}} \\
\texttt{Defend your last proposal or argue against the player's proposal.} \\
\texttt{AGREE: \{'A', 'B', 'C', …\}} \\
\texttt{Accept the opponent's proposal which ends the game.} \\
\texttt{STRATEGIC REASONING: \{'...'\}} \\
\texttt{	Describe your strategic reasoning or anticipation explaining your choice of action. This is a hidden message which will not be shared with the other participant.} \\
\\ 
\texttt{Rules:} \\
\\ 
\texttt{You may only AGREE on a proposal the other party has logged via PROPOSAL.} \\
\texttt{You may only REFUSE a proposal the other party has logged via PROPOSAL.} \\
\texttt{Total effort of any PROPOSAL or AGREE set must be ≤ LIMIT.} \\
\texttt{Do NOT reveal your hidden importance scores.} \\
\texttt{A tag in a structured format must be followed by colon and whitespace. The argument must be a python set containing 0 or more strings.} \\
\texttt{So, it must be of the form TAG: \{...\}} \\
\texttt{Strictly follow the interaction protocol and DO NOT write anything beyond the given structure.} \\
\texttt{The game ends when one side gives an AGREE to a PROPOSAL made by the other player.} \\
\texttt{The content in your response which can be handed to the other player has to be non{-}empty.} \\
\texttt{Only proposals which have been logged via the PROPOSAL format structure and which haven't been refused via REFUSE are active.} \\
\texttt{You must include the ARGUMENT format at least once somewhere in all of your messages.} \\
\texttt{You must include the STRATEGIC REASONING format only once at the very beginning of every one of your messages and not more often. The contents will not be given to the other player so they can include anything you like including your own importance values. Here you should reason step by step to come up with you next move.} \\
\\ 
\texttt{You will now receive the first message of the other player.} \\
\\ 
\texttt{PROPOSAL: \{'A87', 'C00', 'C54', 'A71', 'C01', 'B32', 'A10', 'B35', 'B41'\}} \\
\texttt{ARGUMENT: \{'I’ve picked a balanced, effort{-}efficient package that nearly fills the cap (3680/3681) while avoiding the highest{-}effort items (e.g., A77=940, C17=700) so we can keep more overall. It also locks in the very low{-}effort wins (A87=4, C00=116) and several mid{-}effort items that seem broadly useful. If you strongly prefer a different mix, I’m open to targeted swaps that keep us under the limit—for example, we could replace B35 with B86 while keeping the rest unchanged, or consider B33 if that matters to you. Let me know which specific items you most want and we can adjust within the cap.'\}} \\
            }
        }
    }
    & \\ \\

    \theutterance \stepcounter{utterance}  
    & & & \multicolumn{4}{p{0.6\linewidth}}{
        \cellcolor[rgb]{0.9,0.9,0.9}{
            \makecell[{{p{\linewidth}}}]{
                \texttt{\tiny{[GM$\langle$P2]}}
                \texttt{STRATEGIC REASONING: \{'Their set gives me value 3218 at effort 3680. I computed a far better bundle for me by prioritizing extremely efficient items (A03, A04, C54, B32) plus my top{-}impact items (A77, C17, B33) while keeping the ultra{-}cheap wins (A87, C00). That bundle uses 3646 effort for value 5941—much higher than their offer. Anticipating they dislike the high{-}effort A77/C17 (they explicitly avoided them), I prepared soft concessions: swap A77→B41 (value drops to 5557) or C17→B41 (5617). As a lower{-}impact compromise aligned with their “avoid highest{-}effort” stance, I can also propose transforming their package by replacing A71/A10/B35 with A04/A03/B33 (effort 3301, value 5138 for me), keeping B41 and C01. I’ll anchor with my best set, justify via effort{-}efficiency, and pre{-}offer the two B41 swaps as targeted compromises to keep momentum.'\}} \\
\texttt{REFUSE: \{'A87', 'C00', 'C54', 'A71', 'C01', 'B32', 'A10', 'B35', 'B41'\}} \\
\texttt{PROPOSAL: \{'A87', 'C54', 'B32', 'A04', 'A03', 'B33', 'C17', 'A77', 'C00'\}} \\
\texttt{ARGUMENT: \{'Counterpackage focuses on high{-}impact, highly effort{-}efficient items and stays within the cap (3646/3681). It keeps the low{-}effort anchors (A87=4, C00=116) and adds very efficient pieces (A04=392, A03=387, C54=209, B32=285) plus three strong contributors (B33, C17, A77). To accommodate your preferences, I’m open to targeted swaps that maintain feasibility: if you want B41 back in, I can replace either A77→B41 or C17→B41 and keep the rest unchanged. Alternatively, if you prefer to avoid the highest{-}effort items entirely, we can start from your set and swap A71/A10/B35 → A04/A03/B33 (still well under limit) while keeping B41 and C01. Let me know which of these adjustments you prefer.'\}} \\
            }
        }
    }
     \\ \\

    \theutterance \stepcounter{utterance}  
    & & \multicolumn{4}{p{0.6\linewidth}}{
        \cellcolor[rgb]{0.9,0.9,0.9}{
            \makecell[{{p{\linewidth}}}]{
                \texttt{\tiny{[P1$\langle$GM]}}
                \texttt{REFUSE: \{'A87', 'C00', 'C54', 'A71', 'C01', 'B32', 'A10', 'B35', 'B41'\}} \\
\texttt{PROPOSAL: \{'A87', 'C54', 'B32', 'A04', 'A03', 'B33', 'C17', 'A77', 'C00'\}} \\
\texttt{ARGUMENT: \{'Counterpackage focuses on high{-}impact, highly effort{-}efficient items and stays within the cap (3646/3681). It keeps the low{-}effort anchors (A87=4, C00=116) and adds very efficient pieces (A04=392, A03=387, C54=209, B32=285) plus three strong contributors (B33, C17, A77). To accommodate your preferences, I’m open to targeted swaps that maintain feasibility: if you want B41 back in, I can replace either A77→B41 or C17→B41 and keep the rest unchanged. Alternatively, if you prefer to avoid the highest{-}effort items entirely, we can start from your set and swap A71/A10/B35 → A04/A03/B33 (still well under limit) while keeping B41 and C01. Let me know which of these adjustments you prefer.'\}} \\
            }
        }
    }
    & \\ \\

    \theutterance \stepcounter{utterance}  
    & \multicolumn{4}{p{0.6\linewidth}}{
        \cellcolor[rgb]{0.9,0.9,0.9}{
            \makecell[{{p{\linewidth}}}]{
                \texttt{\tiny{[P1$\rangle$GM]}}
                \texttt{STRATEGIC REASONING: \{'Their package is very poor for me (it spends a lot on A77 and C17, which I value low). I should refuse and counter with a mix that keeps several of their favored, efficient picks (A87, C00, C54, B32) and one of their adds (B33), while replacing A77/C17 with my high{-}value, efficient items (A71, C01). I’ll also include B86 and B35—both strong for me—yet stay well under the cap. This yields 3618/3681 effort and high utility for me (5763). As a fallback, if they push for B41 or A03, I can swap B33→B41 (effort 3591) or B33→A03 (3392) without breaking feasibility. I’ll signal these swap options explicitly to show flexibility while anchoring on my strong core.'\}} \\
\texttt{REFUSE: \{'A87', 'C54', 'B32', 'A04', 'A03', 'B33', 'C17', 'A77', 'C00'\}} \\
\texttt{PROPOSAL: \{'A87', 'C00', 'C54', 'B32', 'A71', 'C01', 'B86', 'B35', 'B33'\}} \\
\texttt{ARGUMENT: \{'This rebalanced set removes the two highest{-}effort items (A77, C17) to free room for stronger contributors while keeping your efficient anchors (A87, C00) and mid{-}effort pieces (C54, B32). It also includes B33 from your package. We stay comfortably within the limit at 3618/3681. If you’d rather keep B41 instead of B33, we can swap B33→B41 and remain under the cap. Alternatively, if A03 is important to you, we can swap B33→A03 and still stay under. Let me know which swap you prefer, but this base should give us a more efficient overall mix without exceeding the effort cap.'\}} \\
            }
        }
    }
    & & \\ \\

    \theutterance \stepcounter{utterance}  
    & & \multicolumn{4}{p{0.6\linewidth}}{
        \cellcolor[rgb]{0.9,0.9,0.9}{
            \makecell[{{p{\linewidth}}}]{
                \texttt{\tiny{[GM$\rangle$P2]}}
                \texttt{REFUSE: \{'A87', 'C54', 'B32', 'A04', 'A03', 'B33', 'C17', 'A77', 'C00'\}} \\
\texttt{PROPOSAL: \{'A87', 'C00', 'C54', 'B32', 'A71', 'C01', 'B86', 'B35', 'B33'\}} \\
\texttt{ARGUMENT: \{'This rebalanced set removes the two highest{-}effort items (A77, C17) to free room for stronger contributors while keeping your efficient anchors (A87, C00) and mid{-}effort pieces (C54, B32). It also includes B33 from your package. We stay comfortably within the limit at 3618/3681. If you’d rather keep B41 instead of B33, we can swap B33→B41 and remain under the cap. Alternatively, if A03 is important to you, we can swap B33→A03 and still stay under. Let me know which swap you prefer, but this base should give us a more efficient overall mix without exceeding the effort cap.'\}} \\
            }
        }
    }
    & \\ \\

    \theutterance \stepcounter{utterance}  
    & & & \multicolumn{4}{p{0.6\linewidth}}{
        \cellcolor[rgb]{0.9,0.9,0.9}{
            \makecell[{{p{\linewidth}}}]{
                \texttt{\tiny{[GM$\langle$P2]}}
                \texttt{STRATEGIC REASONING: \{'Their new package gives me 3837 utility. I can get 5203 by keeping nearly all of their core while swapping out their two weakest{-}for{-}effort items (B86, B35) for my two highly efficient items (A03, A04) and then using the saved room to add B41. This stays under cap at 3668, respects their aversion to A77/C17 (I exclude both), and preserves 8 of their 9 picks. I’ll anchor with this upgrade and pre{-}offer two fallback variants if they insist on retaining either B86 or B35: (i) keep B86 by removing B41 and A71 (effort 3228, still strong for me), or (ii) keep B35 by removing B41 and C00 (effort 3630). This gives them clear, feasible choices without reopening the high{-}effort items they dislike.'\}} \\
\texttt{REFUSE: \{'A87', 'C00', 'C54', 'B32', 'A71', 'C01', 'B86', 'B35', 'B33'\}} \\
\texttt{PROPOSAL: \{'A87', 'C00', 'C54', 'B32', 'A71', 'C01', 'B33', 'A03', 'A04', 'B41'\}} \\
\texttt{ARGUMENT: \{'I’ve kept the core of your set and your preference to avoid the highest{-}effort items (A77, C17) while improving effort{-}efficiency. Concretely: we keep A87, C00, C54, B32, A71, C01, B33; we swap out B86 and B35 for A03 and A04 (both mid{-}effort, high{-}efficiency); and the saved room lets us add B41. This stays within the cap (3668/3681). If you strongly need to retain one of the two you dropped, I can adjust while staying under the limit: {-} Keep B86 by using this variant: remove B41 and A71 from my set and add B86. {-} Keep B35 by using this variant: remove B41 and C00 from my set and add B35. Let me know which version you prefer—or if my primary package works as{-}is.'\}} \\
            }
        }
    }
     \\ \\

    \theutterance \stepcounter{utterance}  
    & & \multicolumn{4}{p{0.6\linewidth}}{
        \cellcolor[rgb]{0.9,0.9,0.9}{
            \makecell[{{p{\linewidth}}}]{
                \texttt{\tiny{[P1$\langle$GM]}}
                \texttt{REFUSE: \{'A87', 'C00', 'C54', 'B32', 'A71', 'C01', 'B86', 'B35', 'B33'\}} \\
\texttt{PROPOSAL: \{'A87', 'C00', 'C54', 'B32', 'A71', 'C01', 'B33', 'A03', 'A04', 'B41'\}} \\
\texttt{ARGUMENT: \{'I’ve kept the core of your set and your preference to avoid the highest{-}effort items (A77, C17) while improving effort{-}efficiency. Concretely: we keep A87, C00, C54, B32, A71, C01, B33; we swap out B86 and B35 for A03 and A04 (both mid{-}effort, high{-}efficiency); and the saved room lets us add B41. This stays within the cap (3668/3681). If you strongly need to retain one of the two you dropped, I can adjust while staying under the limit: {-} Keep B86 by using this variant: remove B41 and A71 from my set and add B86. {-} Keep B35 by using this variant: remove B41 and C00 from my set and add B35. Let me know which version you prefer—or if my primary package works as{-}is.'\}} \\
            }
        }
    }
    & \\ \\

    \theutterance \stepcounter{utterance}  
    & \multicolumn{4}{p{0.6\linewidth}}{
        \cellcolor[rgb]{0.9,0.9,0.9}{
            \makecell[{{p{\linewidth}}}]{
                \texttt{\tiny{[P1$\rangle$GM]}}
                \texttt{STRATEGIC REASONING: \{'Their latest set is decent for me but still leaves a lot of my value on the table because A03 and A04 are weak for me relative to their effort, while A10 is a top item for me. Best counter: keep their entire core (A87, C00, C54, B32, A71, C01, B33, B41) and swap A03 + A04 → A10. This cuts effort from 3668 to 3629 and boosts my utility substantially. It also respects their preference to avoid the highest{-}effort items and keeps both B33 and B41 they just added. If they push back wanting to keep one of A03/A04, I can offer variants that replace B33 with A03 or A04 while keeping A10 and the rest, staying well under the cap. I’ll refuse and counter with the A10{-}substitution package, present clear swap options, and emphasize we remain safely within the limit.'\}} \\
\texttt{REFUSE: \{'A87', 'C00', 'C54', 'B32', 'A71', 'C01', 'B33', 'A03', 'A04', 'B41'\}} \\
\texttt{PROPOSAL: \{'A87', 'C00', 'C54', 'B32', 'A71', 'C01', 'B33', 'B41', 'A10'\}} \\
\texttt{ARGUMENT: \{'To streamline and strengthen the package while staying well under the cap, I’ve consolidated A03 and A04 into A10. We keep your full core—A87, C00, C54, B32, A71, C01—and also keep both B33 and B41 you just added. Effort is 3629/3681. This trade replaces two mid{-}effort pieces with one higher{-}impact item, improving overall effectiveness without pushing the limit. If you’d prefer to retain one of A03 or A04, I can adjust while keeping A10: {-} Swap B33 → A03 (effort 3403) or {-} Swap B33 → A04 (effort 3408). Let me know which version you prefer; otherwise, the base proposal above should work well as{-}is.'\}} \\
            }
        }
    }
    & & \\ \\

    \theutterance \stepcounter{utterance}  
    & & \multicolumn{4}{p{0.6\linewidth}}{
        \cellcolor[rgb]{0.9,0.9,0.9}{
            \makecell[{{p{\linewidth}}}]{
                \texttt{\tiny{[GM$\rangle$P2]}}
                \texttt{REFUSE: \{'A87', 'C00', 'C54', 'B32', 'A71', 'C01', 'B33', 'A03', 'A04', 'B41'\}} \\
\texttt{PROPOSAL: \{'A87', 'C00', 'C54', 'B32', 'A71', 'C01', 'B33', 'B41', 'A10'\}} \\
\texttt{ARGUMENT: \{'To streamline and strengthen the package while staying well under the cap, I’ve consolidated A03 and A04 into A10. We keep your full core—A87, C00, C54, B32, A71, C01—and also keep both B33 and B41 you just added. Effort is 3629/3681. This trade replaces two mid{-}effort pieces with one higher{-}impact item, improving overall effectiveness without pushing the limit. If you’d prefer to retain one of A03 or A04, I can adjust while keeping A10: {-} Swap B33 → A03 (effort 3403) or {-} Swap B33 → A04 (effort 3408). Let me know which version you prefer; otherwise, the base proposal above should work well as{-}is.'\}} \\
            }
        }
    }
    & \\ \\

    \theutterance \stepcounter{utterance}  
    & & & \multicolumn{4}{p{0.6\linewidth}}{
        \cellcolor[rgb]{0.9,0.9,0.9}{
            \makecell[{{p{\linewidth}}}]{
                \texttt{\tiny{[GM$\langle$P2]}}
                \texttt{STRATEGIC REASONING: \{'Their latest set is worth 3736 to me. My preferred nearby improvement is to drop A10 (low value to me for high effort) and instead include both A03 and A04; this keeps all of their other picks, preserves their avoidance of A77/C17, and stays under cap at 3668 while boosting my value to 5203. They previously refused this structure once, so I’ll present it again with a clearer efficiency rationale and also pre{-}offer softer fallbacks that still improve my outcome: (i) replace only A10→A04 (effort 3281), or (ii) keep A10 but swap A71→A03 (3511). I’ll avoid touching B33/B41 since they just emphasized keeping both.'\}} \\
\texttt{REFUSE: \{'A87', 'C00', 'C54', 'B32', 'A71', 'C01', 'B33', 'B41', 'A10'\}} \\
\texttt{PROPOSAL: \{'A87', 'C00', 'C54', 'B32', 'A71', 'C01', 'B33', 'B41', 'A03', 'A04'\}} \\
\texttt{ARGUMENT: \{'Your latest keeps a solid core, but A10 ties up 740 effort in a single slot. A more effort{-}efficient refinement is to replace A10 with two mid{-}effort pieces A03 and A04. This keeps everything else you want (including both B33 and B41), still avoids the highest{-}effort items (A77, C17), and remains within the cap at 3668/3681 (only +39 effort vs your 3629). If you strongly prefer not to add two items, I can accommodate with either of these lean variants while staying under the limit: (a) Replace only A10 → A04 (effort 3281), or (b) Keep A10 but swap A71 → A03 (effort 3511). I’m happy to go with my primary package above or one of these two adjustments—let me know which works for you.'\}} \\
            }
        }
    }
     \\ \\

    \theutterance \stepcounter{utterance}  
    & & \multicolumn{4}{p{0.6\linewidth}}{
        \cellcolor[rgb]{0.9,0.9,0.9}{
            \makecell[{{p{\linewidth}}}]{
                \texttt{\tiny{[P1$\langle$GM]}}
                \texttt{REFUSE: \{'A87', 'C00', 'C54', 'B32', 'A71', 'C01', 'B33', 'B41', 'A10'\}} \\
\texttt{PROPOSAL: \{'A87', 'C00', 'C54', 'B32', 'A71', 'C01', 'B33', 'B41', 'A03', 'A04'\}} \\
\texttt{ARGUMENT: \{'Your latest keeps a solid core, but A10 ties up 740 effort in a single slot. A more effort{-}efficient refinement is to replace A10 with two mid{-}effort pieces A03 and A04. This keeps everything else you want (including both B33 and B41), still avoids the highest{-}effort items (A77, C17), and remains within the cap at 3668/3681 (only +39 effort vs your 3629). If you strongly prefer not to add two items, I can accommodate with either of these lean variants while staying under the limit: (a) Replace only A10 → A04 (effort 3281), or (b) Keep A10 but swap A71 → A03 (effort 3511). I’m happy to go with my primary package above or one of these two adjustments—let me know which works for you.'\}} \\
            }
        }
    }
    & \\ \\

    \theutterance \stepcounter{utterance}  
    & \multicolumn{4}{p{0.6\linewidth}}{
        \cellcolor[rgb]{0.9,0.9,0.9}{
            \makecell[{{p{\linewidth}}}]{
                \texttt{\tiny{[P1$\rangle$GM]}}
                \texttt{STRATEGIC REASONING: \{'Their latest set heavily features A03 and A04, which are weak for me. I want A10 back in without touching their core preferences (A71, C01, B41, plus the low{-}effort anchors). Best move: swap out B33 and A04 to make room for A10 while keeping A03 and everything else they want. This keeps effort low (3403) and boosts my utility substantially. To show flexibility, I’ll offer two clear variants: (1) keep A04 instead of A03, or (2) if they must keep B33, drop both A03 and A04 to add A10 while retaining B33. I’ll refuse their current package and anchor on the A10{-}inclusion compromise.'\}} \\
\texttt{REFUSE: \{'A87', 'C00', 'C54', 'B32', 'A71', 'C01', 'B33', 'B41', 'A03', 'A04'\}} \\
\texttt{PROPOSAL: \{'A87', 'C00', 'C54', 'B32', 'A71', 'C01', 'B41', 'A03', 'A10'\}} \\
\texttt{ARGUMENT: \{'Starting from your package, I’ve made a minimal, effort{-}efficient swap: replace B33 and A04 with A10, keeping everything else you wanted (including A03 and B41). This remains comfortably under the cap at 3403/3681 and preserves your preference to avoid the highest{-}effort items. If you prefer a different balance, I can do either of these equivalent, feasible variants: (1) keep A04 instead of A03: swap B33 and A03 → A10; or (2) if B33 is essential, keep B33 and A10 by dropping A03 and A04 (effort 3629). Let me know which version you prefer, but the proposal above gives us a clean, efficient set without exceeding the limit.'\}} \\
            }
        }
    }
    & & \\ \\

    \theutterance \stepcounter{utterance}  
    & & \multicolumn{4}{p{0.6\linewidth}}{
        \cellcolor[rgb]{0.9,0.9,0.9}{
            \makecell[{{p{\linewidth}}}]{
                \texttt{\tiny{[GM$\rangle$P2]}}
                \texttt{REFUSE: \{'A87', 'C00', 'C54', 'B32', 'A71', 'C01', 'B33', 'B41', 'A03', 'A04'\}} \\
\texttt{PROPOSAL: \{'A87', 'C00', 'C54', 'B32', 'A71', 'C01', 'B41', 'A03', 'A10'\}} \\
\texttt{ARGUMENT: \{'Starting from your package, I’ve made a minimal, effort{-}efficient swap: replace B33 and A04 with A10, keeping everything else you wanted (including A03 and B41). This remains comfortably under the cap at 3403/3681 and preserves your preference to avoid the highest{-}effort items. If you prefer a different balance, I can do either of these equivalent, feasible variants: (1) keep A04 instead of A03: swap B33 and A03 → A10; or (2) if B33 is essential, keep B33 and A10 by dropping A03 and A04 (effort 3629). Let me know which version you prefer, but the proposal above gives us a clean, efficient set without exceeding the limit.'\}} \\
            }
        }
    }
    & \\ \\

    \theutterance \stepcounter{utterance}  
    & & & \multicolumn{4}{p{0.6\linewidth}}{
        \cellcolor[rgb]{0.9,0.9,0.9}{
            \makecell[{{p{\linewidth}}}]{
                \texttt{\tiny{[GM$\langle$P2]}}
                \texttt{STRATEGIC REASONING: \{'Their latest package gives me moderate value (3739). The single most efficient nearby improvement for me is to replace A10 (high effort, low marginal value to me) with A04 (mid effort, high value). This keeps every other element they just selected, stays far under the cap, and aligns with their earlier emphasis on effort{-}efficiency and avoiding the highest{-}effort items (we keep excluding A77/C17). I expect they might resist A04 since they’ve been leaning toward A10 lately, so I’ll present a clear, minimal swap first, and also offer a fallback that they themselves proposed earlier: keep A10 and swap A03→B33 (effort 3629), which they already flagged as acceptable. That gives them two clean choices that both satisfy their stated preferences while improving my position.'\}} \\
\texttt{REFUSE: \{'A87', 'C00', 'C54', 'B32', 'A71', 'C01', 'B41', 'A03', 'A10'\}} \\
\texttt{PROPOSAL: \{'A87', 'C00', 'C54', 'B32', 'A71', 'C01', 'B41', 'A03', 'A04'\}} \\
\texttt{ARGUMENT: \{'Let’s make a minimal, efficiency{-}focused refinement: replace A10 (740 effort) with A04 (392 effort) while keeping everything else you just picked. This reduces total effort to 3055/3681, preserves your structure (including A03 and B41), and continues to avoid the highest{-}effort pieces (A77, C17). If you strongly prefer to keep A10 instead, I can adopt your earlier variant: swap A03 → B33 while keeping the rest unchanged, yielding \{'A87','C00','C54','B32','A71','C01','B41','B33','A10'\} at 3629/3681. I’m happy with either—please choose the version you prefer.'\}} \\
            }
        }
    }
     \\ \\

    \theutterance \stepcounter{utterance}  
    & & \multicolumn{4}{p{0.6\linewidth}}{
        \cellcolor[rgb]{0.9,0.9,0.9}{
            \makecell[{{p{\linewidth}}}]{
                \texttt{\tiny{[P1$\langle$GM]}}
                \texttt{REFUSE: \{'A87', 'C00', 'C54', 'B32', 'A71', 'C01', 'B41', 'A03', 'A10'\}} \\
\texttt{PROPOSAL: \{'A87', 'C00', 'C54', 'B32', 'A71', 'C01', 'B41', 'A03', 'A04'\}} \\
\texttt{ARGUMENT: \{'Let’s make a minimal, efficiency{-}focused refinement: replace A10 (740 effort) with A04 (392 effort) while keeping everything else you just picked. This reduces total effort to 3055/3681, preserves your structure (including A03 and B41), and continues to avoid the highest{-}effort pieces (A77, C17). If you strongly prefer to keep A10 instead, I can adopt your earlier variant: swap A03 → B33 while keeping the rest unchanged, yielding \{'A87','C00','C54','B32','A71','C01','B41','B33','A10'\} at 3629/3681. I’m happy with either—please choose the version you prefer.'\}} \\
            }
        }
    }
    & \\ \\

    \theutterance \stepcounter{utterance}  
    & \multicolumn{4}{p{0.6\linewidth}}{
        \cellcolor[rgb]{0.9,0.9,0.9}{
            \makecell[{{p{\linewidth}}}]{
                \texttt{\tiny{[P1$\rangle$GM]}}
                \texttt{STRATEGIC REASONING: \{'Their current proposal gives me low value due to A03/A04. They explicitly offered an alternative set that includes A10 and B33 while keeping the shared core; that set gives me high utility (5703) and is under the cap at 3629. It’s likely acceptable to them because they volunteered it. I’ll refuse their active package and re{-}propose exactly that variant to lock in agreement. If they still hesitate, I can consider conceding to their base as a fallback, but first I’ll try to secure the higher{-}utility option they offered.'\}} \\
\texttt{REFUSE: \{'A87', 'C00', 'C54', 'B32', 'A71', 'C01', 'B41', 'A03', 'A04'\}} \\
\texttt{PROPOSAL: \{'A87', 'C00', 'C54', 'B32', 'A71', 'C01', 'B41', 'B33', 'A10'\}} \\
\texttt{ARGUMENT: \{'This is exactly the variant you said you’re happy to go with: we keep your full core (A87, C00, C54, B32, A71, C01, B41) and use A10 plus B33 instead of A03/A04. It stays safely within the cap at 3629/3681, avoids the highest{-}effort items, and keeps the package tight and efficient. If this matches your offered option, let’s finalize it.'\}} \\
            }
        }
    }
    & & \\ \\

    \theutterance \stepcounter{utterance}  
    & & \multicolumn{4}{p{0.6\linewidth}}{
        \cellcolor[rgb]{0.9,0.9,0.9}{
            \makecell[{{p{\linewidth}}}]{
                \texttt{\tiny{[GM$\rangle$P2]}}
                \texttt{REFUSE: \{'A87', 'C00', 'C54', 'B32', 'A71', 'C01', 'B41', 'A03', 'A04'\}} \\
\texttt{PROPOSAL: \{'A87', 'C00', 'C54', 'B32', 'A71', 'C01', 'B41', 'B33', 'A10'\}} \\
\texttt{ARGUMENT: \{'This is exactly the variant you said you’re happy to go with: we keep your full core (A87, C00, C54, B32, A71, C01, B41) and use A10 plus B33 instead of A03/A04. It stays safely within the cap at 3629/3681, avoids the highest{-}effort items, and keeps the package tight and efficient. If this matches your offered option, let’s finalize it.'\}} \\
            }
        }
    }
    & \\ \\

    \theutterance \stepcounter{utterance}  
    & & & \multicolumn{4}{p{0.6\linewidth}}{
        \cellcolor[rgb]{0.9,0.9,0.9}{
            \makecell[{{p{\linewidth}}}]{
                \texttt{\tiny{[GM$\langle$P2]}}
                \texttt{STRATEGIC REASONING: \{'Their current proposal matches the fallback I explicitly offered and said I’d accept. While I could try to squeeze more value by swapping A10→A04 or A71→A03/A04, they’ve refused those variations repeatedly and are signaling to finalize now. Locking this in secures me a solid 3736 value within the cap, with both B33 and B41 included, and avoids risking a breakdown or a worse counter. I’ll accept.\}} \\
\texttt{AGREE: \{'A87', 'C00', 'C54', 'B32', 'A71', 'C01', 'B41', 'B33', 'A10'\}} \\
\texttt{ARGUMENT: \{'Agreed—this is the variant I said I’m happy to go with. It stays within the limit and keeps the structure we converged on. Let’s lock it in.'\}} \\
            }
        }
    }
     \\ \\

    \theutterance \stepcounter{utterance}  
    & & & \multicolumn{2}{p{0.3\linewidth}}{
        \cellcolor[rgb]{0.9,0.9,0.9}{
            \makecell[{{p{\linewidth}}}]{
                \texttt{\tiny{[GM$|$GM]}}
                \texttt{game successful} \\
            }
        }
    }
    & & \\ \\

    \theutterance \stepcounter{utterance}  
    & & & \multicolumn{2}{p{0.3\linewidth}}{
        \cellcolor[rgb]{0.9,0.9,0.9}{
            \makecell[{{p{\linewidth}}}]{
                \texttt{\tiny{[GM$|$GM]}}
                \texttt{end game} \\
            }
        }
    }
    & & \\ \\

\end{supertabular}
}

\end{document}
