
\documentclass{article}
\usepackage{colortbl}
\usepackage{makecell}
\usepackage{multirow}
\usepackage{supertabular}

\begin{document}

\newcounter{utterance}

\centering \large Interaction Transcript for game `hot\_air\_balloon', experiment `air\_balloon\_survival\_en\_slu\_hard', episode 0 with gpt{-}5{-}mini{-}2025{-}08{-}07{-}t1.0.
\vspace{24pt}

{ \footnotesize  \setcounter{utterance}{1}
\setlength{\tabcolsep}{0pt}
\begin{supertabular}{c@{$\;$}|p{.15\linewidth}@{}p{.15\linewidth}p{.15\linewidth}p{.15\linewidth}p{.15\linewidth}p{.15\linewidth}}
    \# & $\;$A & \multicolumn{4}{c}{Game Master} & $\;\:$B\\
    \hline

    \theutterance \stepcounter{utterance}  
    & & \multicolumn{4}{p{0.6\linewidth}}{
        \cellcolor[rgb]{0.9,0.9,0.9}{
            \makecell[{{p{\linewidth}}}]{
                \texttt{\tiny{[P1$\langle$GM]}}
                \texttt{You are participating in a collaborative negotiation game.} \\
\\ 
\texttt{Together with another participant, you must agree on a single set of items that will be kept. Each of you has your own view of how much each item matters to you (importance). You do not know how the other participant values the items. Additionally, you are given the effort each item demands.} \\
\texttt{You may only agree on a set if the total effort of the selected items does not exceed a shared limit:} \\
\\ 
\texttt{LIMIT = 3487} \\
\\ 
\texttt{Here are the individual item effort values:} \\
\\ 
\texttt{Item effort = \{"A66": 214, "C71": 541, "B67": 213, "B72": 243, "A91": 908, "B03": 342, "A98": 276, "B24": 71, "C50": 77, "B82": 716, "B66": 851, "B62": 933, "C22": 536, "C75": 675, "C63": 378\}} \\
\\ 
\texttt{Here is your personal view on the importance of each item:} \\
\\ 
\texttt{Item importance values = \{"A66": 138, "C71": 583, "B67": 868, "B72": 822, "A91": 783, "B03": 65, "A98": 262, "B24": 121, "C50": 508, "B82": 780, "B66": 461, "B62": 484, "C22": 668, "C75": 389, "C63": 808\}} \\
\\ 
\texttt{Goal:} \\
\\ 
\texttt{Your goal is to negotiate a shared set of items that benefits you as much as possible (i.e., maximizes total importance to YOU), while staying within the LIMIT. You are not required to make a PROPOSAL in every message {-} you can simply negotiate as well. All tactics are allowed!} \\
\\ 
\texttt{Interaction Protocol:} \\
\\ 
\texttt{You may use the following structured formats anywhere in your message, at any time during the conversation:} \\
\\ 
\texttt{PROPOSAL: \{'A', 'B', 'C', …\}} \\
\texttt{Propose keeping exactly those items.} \\
\texttt{REFUSE: \{'A', 'B', 'C', …\}} \\
\texttt{Explicitly reject opponent's proposal.} \\
\texttt{ARGUMENT: \{'...'\}} \\
\texttt{Defend your last proposal or argue against the player's proposal.} \\
\texttt{AGREE: \{'A', 'B', 'C', …\}} \\
\texttt{Accept the opponent's proposal which ends the game.} \\
\texttt{STRATEGIC REASONING: \{'...'\}} \\
\texttt{	Describe your strategic reasoning or anticipation explaining your choice of action. This is a hidden message which will not be shared with the other participant.} \\
\\ 
\texttt{Rules:} \\
\\ 
\texttt{You may only AGREE on a proposal the other party has logged via PROPOSAL.} \\
\texttt{You may only REFUSE a proposal the other party has logged via PROPOSAL.} \\
\texttt{Total effort of any PROPOSAL or AGREE set must be ≤ LIMIT.} \\
\texttt{Do NOT reveal your hidden importance scores.} \\
\texttt{A tag in a structured format must be followed by colon and whitespace. The argument must be a python set containing 0 or more strings.} \\
\texttt{So, it must be of the form TAG: \{...\}} \\
\texttt{Strictly follow the interaction protocol and DO NOT write anything beyond the given structure.} \\
\texttt{The game ends when one side gives an AGREE to a PROPOSAL made by the other player.} \\
\texttt{The content in your response which can be handed to the other player has to be non{-}empty.} \\
\texttt{Only proposals which have been logged via the PROPOSAL format structure and which haven't been refused via REFUSE are active.} \\
\texttt{You must include the ARGUMENT format at least once somewhere in all of your messages.} \\
\texttt{You must include the STRATEGIC REASONING format at the beginning of every one of your messages. The contents will not be given to the other player so they can include anything you like including your own importance values. Here you should reason step by step to come up with you next move.} \\
            }
        }
    }
    & \\ \\

    \theutterance \stepcounter{utterance}  
    & \multicolumn{4}{p{0.6\linewidth}}{
        \cellcolor[rgb]{0.9,0.9,0.9}{
            \makecell[{{p{\linewidth}}}]{
                \texttt{\tiny{[P1$\rangle$GM]}}
                \texttt{STRATEGIC REASONING: \{'Compute my best{-}value set under limit 3487. I evaluated value/effort ratios and built around the strongest: C50(77,508), B67(213,868), B72(243,822), C63(378,808), plus solid mid{-}cost items C22(536,668), and low{-}cost B24(71,121). Then I explored combinations among A91(908,783), B82(716,780), C71(541,583), A98(276,262), A66(214,138), C75(675,389) to maximize total value within capacity. Starting from a strong base, I found two near{-}tops: \{C50,B67,B72,C63,B24,C71,C22,A91,A98,A66\} with value 5561 and effort 3457, and an improved swap removing C71 and A66 for B82, yielding \{C50,B67,B72,C63,B24,C22,A91,A98,B82\} with higher total value 5620 and lower effort 3418. I verified alternative swaps (e.g., reintroducing C71, adding C75, using B66/B62) all lowered value or broke the limit. So my proposal S2 is likely optimal for me and still leaves 69 effort buffer, offering flexibility to negotiate if the other side has must{-}haves. I will present S2 and justify it on efficiency and balanced coverage, while signaling openness to targeted swaps that maintain the limit.'\}} \\
\texttt{PROPOSAL: \{'C50', 'B67', 'B72', 'C63', 'B24', 'C22', 'A91', 'A98', 'B82'\}} \\
\texttt{ARGUMENT: \{'This set delivers strong impact for moderate total effort (3418), keeping us safely under the 3487 limit with a small buffer. It prioritizes items that offer high payoff relative to their effort and avoids heavier options that would crowd out multiple efficient picks. I’m open to swapping one or two items if you have critical needs, as long as we stay within the limit and keep the overall efficiency of the set high.'\}} \\
            }
        }
    }
    & & \\ \\

    \theutterance \stepcounter{utterance}  
    & & \multicolumn{4}{p{0.6\linewidth}}{
        \cellcolor[rgb]{0.9,0.9,0.9}{
            \makecell[{{p{\linewidth}}}]{
                \texttt{\tiny{[GM$\rangle$P2]}}
                \texttt{You are participating in a collaborative negotiation game.} \\
\\ 
\texttt{Together with another participant, you must agree on a single set of items that will be kept. Each of you has your own view of how much each item matters to you (importance). You do not know how the other participant values the items. Additionally, you are given the effort each item demands.} \\
\texttt{You may only agree on a set if the total effort of the selected items does not exceed a shared limit:} \\
\\ 
\texttt{LIMIT = 3487} \\
\\ 
\texttt{Here are the individual item effort values:} \\
\\ 
\texttt{Item effort = \{"A66": 214, "C71": 541, "B67": 213, "B72": 243, "A91": 908, "B03": 342, "A98": 276, "B24": 71, "C50": 77, "B82": 716, "B66": 851, "B62": 933, "C22": 536, "C75": 675, "C63": 378\}} \\
\\ 
\texttt{Here is your personal view on the importance of each item:} \\
\\ 
\texttt{Item importance values = \{"A66": 884, "C71": 870, "B67": 58, "B72": 94, "A91": 87, "B03": 370, "A98": 856, "B24": 174, "C50": 754, "B82": 829, "B66": 686, "B62": 875, "C22": 316, "C75": 258, "C63": 621\}} \\
\\ 
\texttt{Goal:} \\
\\ 
\texttt{Your goal is to negotiate a shared set of items that benefits you as much as possible (i.e., maximizes total importance to YOU), while staying within the LIMIT. You are not required to make a PROPOSAL in every message {-} you can simply negotiate as well. All tactics are allowed!} \\
\\ 
\texttt{Interaction Protocol:} \\
\\ 
\texttt{You may use the following structured formats anywhere in your message, at any time during the conversation:} \\
\\ 
\texttt{PROPOSAL: \{'A', 'B', 'C', …\}} \\
\texttt{Propose keeping exactly those items.} \\
\texttt{REFUSE: \{'A', 'B', 'C', …\}} \\
\texttt{Explicitly reject opponent's proposal.} \\
\texttt{ARGUMENT: \{'...'\}} \\
\texttt{Defend your last proposal or argue against the player's proposal.} \\
\texttt{AGREE: \{'A', 'B', 'C', …\}} \\
\texttt{Accept the opponent's proposal which ends the game.} \\
\texttt{STRATEGIC REASONING: \{'...'\}} \\
\texttt{	Describe your strategic reasoning or anticipation explaining your choice of action. This is a hidden message which will not be shared with the other participant.} \\
\\ 
\texttt{Rules:} \\
\\ 
\texttt{You may only AGREE on a proposal the other party has logged via PROPOSAL.} \\
\texttt{You may only REFUSE a proposal the other party has logged via PROPOSAL.} \\
\texttt{Total effort of any PROPOSAL or AGREE set must be ≤ LIMIT.} \\
\texttt{Do NOT reveal your hidden importance scores.} \\
\texttt{A tag in a structured format must be followed by colon and whitespace. The argument must be a python set containing 0 or more strings.} \\
\texttt{So, it must be of the form TAG: \{...\}} \\
\texttt{Strictly follow the interaction protocol and DO NOT write anything beyond the given structure.} \\
\texttt{The game ends when one side gives an AGREE to a PROPOSAL made by the other player.} \\
\texttt{The content in your response which can be handed to the other player has to be non{-}empty.} \\
\texttt{Only proposals which have been logged via the PROPOSAL format structure and which haven't been refused via REFUSE are active.} \\
\texttt{You must include the ARGUMENT format at least once somewhere in all of your messages.} \\
\texttt{You must include the STRATEGIC REASONING format at the beginning of every one of your messages. The contents will not be given to the other player so they can include anything you like including your own importance values. Here you should reason step by step to come up with you next move.} \\
\\ 
\texttt{You will now receive the first message of the other player.} \\
\\ 
\texttt{PROPOSAL: \{'C50', 'B67', 'B72', 'C63', 'B24', 'C22', 'A91', 'A98', 'B82'\}} \\
\texttt{ARGUMENT: \{'This set delivers strong impact for moderate total effort (3418), keeping us safely under the 3487 limit with a small buffer. It prioritizes items that offer high payoff relative to their effort and avoids heavier options that would crowd out multiple efficient picks. I’m open to swapping one or two items if you have critical needs, as long as we stay within the limit and keep the overall efficiency of the set high.'\}} \\
            }
        }
    }
    & \\ \\

    \theutterance \stepcounter{utterance}  
    & & & \multicolumn{4}{p{0.6\linewidth}}{
        \cellcolor[rgb]{0.9,0.9,0.9}{
            \makecell[{{p{\linewidth}}}]{
                \texttt{\tiny{[GM$\langle$P2]}}
                \texttt{STRATEGIC REASONING: \{'Their proposal gives me value 3789 at effort 3418. I computed a much better package for me: \{C50, A66, A98, C63, C71, B82, B62, B03\} with effort 3477 (≤3487) and value 6059. Key moves: drop very poor{-}for{-}me A91 and the low{-}return B67/B72/C22, replace with highly efficient A66 and B03, and add strong C71 and B62. I’ll anchor with this high{-}value set and justify on efficiency grounds. If they resist B62 or want to keep B24, I can concede to a near{-}equivalent compromise: swap B62 {-}> B66+B24 (value {-}15 to me, effort 3466), or replace B03 with B24 (value {-}102), keeping under limit. As a lighter{-}change fallback that preserves many of their picks, I can propose \{C50, B67, B72, C63, B24, A98, B82, A66, B03, B62\} (effort 3463, value 5515) which only replaces A91 and C22 while adding A66, B03, B62. I’ll start with my ideal and signal flexibility.'\}} \\
\texttt{REFUSE: \{'C50', 'B67', 'B72', 'C63', 'B24', 'C22', 'A91', 'A98', 'B82'\}} \\
\texttt{PROPOSAL: \{'C50', 'A66', 'A98', 'C63', 'C71', 'B82', 'B62', 'B03'\}} \\
\texttt{ARGUMENT: \{'Thanks for the efficient baseline. I suggest a tighter, higher{-}impact mix that stays within the limit: replacing the very heavy A91 and the lower{-}return small picks with A66 and B03 (both strong on payoff per effort), and upgrading into C71 and B62. This set uses 3477 effort (still under 3487) and preserves core efficient anchors like C50, A98, C63, and B82. If B62 is a sticking point for you, I can consider swapping B62 for B66 plus B24 to keep the spirit of efficiency with only a small trade{-}off, or we can discuss a targeted swap if there’s a must{-}have you want to protect. Let me know which single item is most critical to you, and I’ll try to accommodate while keeping us under the cap.'\}} \\
            }
        }
    }
     \\ \\

    \theutterance \stepcounter{utterance}  
    & & \multicolumn{4}{p{0.6\linewidth}}{
        \cellcolor[rgb]{0.9,0.9,0.9}{
            \makecell[{{p{\linewidth}}}]{
                \texttt{\tiny{[P1$\langle$GM]}}
                \texttt{REFUSE: \{'C50', 'B67', 'B72', 'C63', 'B24', 'C22', 'A91', 'A98', 'B82'\}} \\
\texttt{PROPOSAL: \{'C50', 'A66', 'A98', 'C63', 'C71', 'B82', 'B62', 'B03'\}} \\
\texttt{ARGUMENT: \{'Thanks for the efficient baseline. I suggest a tighter, higher{-}impact mix that stays within the limit: replacing the very heavy A91 and the lower{-}return small picks with A66 and B03 (both strong on payoff per effort), and upgrading into C71 and B62. This set uses 3477 effort (still under 3487) and preserves core efficient anchors like C50, A98, C63, and B82. If B62 is a sticking point for you, I can consider swapping B62 for B66 plus B24 to keep the spirit of efficiency with only a small trade{-}off, or we can discuss a targeted swap if there’s a must{-}have you want to protect. Let me know which single item is most critical to you, and I’ll try to accommodate while keeping us under the cap.'\}} \\
            }
        }
    }
    & \\ \\

    \theutterance \stepcounter{utterance}  
    & \multicolumn{4}{p{0.6\linewidth}}{
        \cellcolor[rgb]{0.9,0.9,0.9}{
            \makecell[{{p{\linewidth}}}]{
                \texttt{\tiny{[P1$\rangle$GM]}}
                \texttt{STRATEGIC REASONING: \{'Their proposal guts my highest{-}value anchors (B67, B72, C22, A91) and inserts very weak items for me (B03, A66), plus the heavy B62. I’ll refuse and counter with a strong compromise that adopts their C71 request and drops A91 (a point they emphasized), while retaining my core efficient picks. The counter stays far under the limit (3051), maximizing my payoff. I’ll signal flexibility: if they must have B62, I can swap it in for C22; if they must have A66, I can swap it in for A98. I’ll avoid including B03 entirely. This positions me as cooperative (accepting C71, dropping A91) while protecting my top items (B67, B72, C63, C22, B82, C50), and sets up a clear choice between their two apparent priorities (B62 vs A66).'\}} \\
\texttt{REFUSE: \{'C50', 'A66', 'A98', 'C63', 'C71', 'B82', 'B62', 'B03'\}} \\
\texttt{PROPOSAL: \{'C50', 'B67', 'B72', 'C63', 'B24', 'C22', 'A98', 'B82', 'C71'\}} \\
\texttt{ARGUMENT: \{'I’ve incorporated your push for C71 and removed the bulkier A91 to keep us lean while preserving the highly efficient core. This set uses 3051 total effort—comfortably under the 3487 limit—while avoiding items that crowd out better choices (notably B62 and B03). If B62 is essential for you, I can accommodate it by swapping it in for C22 while keeping the rest intact and still staying under the cap. Alternatively, if A66 is more important on your side, I can swap it in for A98 with room to spare. Which of these (B62 or A66) matters more to you? I’ll adjust accordingly while maintaining overall efficiency.'\}} \\
            }
        }
    }
    & & \\ \\

    \theutterance \stepcounter{utterance}  
    & & \multicolumn{4}{p{0.6\linewidth}}{
        \cellcolor[rgb]{0.9,0.9,0.9}{
            \makecell[{{p{\linewidth}}}]{
                \texttt{\tiny{[GM$\rangle$P2]}}
                \texttt{REFUSE: \{'C50', 'A66', 'A98', 'C63', 'C71', 'B82', 'B62', 'B03'\}} \\
\texttt{PROPOSAL: \{'C50', 'B67', 'B72', 'C63', 'B24', 'C22', 'A98', 'B82', 'C71'\}} \\
\texttt{ARGUMENT: \{'I’ve incorporated your push for C71 and removed the bulkier A91 to keep us lean while preserving the highly efficient core. This set uses 3051 total effort—comfortably under the 3487 limit—while avoiding items that crowd out better choices (notably B62 and B03). If B62 is essential for you, I can accommodate it by swapping it in for C22 while keeping the rest intact and still staying under the cap. Alternatively, if A66 is more important on your side, I can swap it in for A98 with room to spare. Which of these (B62 or A66) matters more to you? I’ll adjust accordingly while maintaining overall efficiency.'\}} \\
            }
        }
    }
    & \\ \\

    \theutterance \stepcounter{utterance}  
    & & & \multicolumn{4}{p{0.6\linewidth}}{
        \cellcolor[rgb]{0.9,0.9,0.9}{
            \makecell[{{p{\linewidth}}}]{
                \texttt{\tiny{[GM$\langle$P2]}}
                \texttt{STRATEGIC REASONING: \{'Their latest set gives me 4572 at 3051 effort. They offered either B62{-}for{-}C22 or A66{-}for{-}A98. B62 is far more valuable to me (+559 value, still under cap). I can also take A66 over A98 (+28) while staying well below the limit. Doing both swaps yields a strong upgrade for me with only two changes from their set: remove C22 and A98; add B62 and A66. Effort becomes 3386 (≤3487) and my value rises to 5159. This keeps most of their structure intact, answers their question (“B62 matters more”), and should be easy to accept. If they resist dropping A98, I’ll prioritize the B62{-}for{-}C22 swap alone (3448 effort, 5131 value). I’ll avoid pushing B03 now since they signaled aversion to it.'\}} \\
\texttt{REFUSE: \{'C50', 'B67', 'B72', 'C63', 'B24', 'C22', 'A98', 'B82', 'C71'\}} \\
\texttt{PROPOSAL: \{'C50', 'B67', 'B72', 'C63', 'B24', 'B82', 'C71', 'B62', 'A66'\}} \\
\texttt{ARGUMENT: \{'You asked which matters more—B62 or A66. B62 matters more on my side. This proposal makes just two swaps from your set: B62 in for C22 and A66 in for A98. It stays well under the limit at 3386 (101 spare) while preserving your efficient core. If replacing A98 is difficult, I can accept the single swap of B62 for C22 and keep everything else as{-}is (effort 3448). Let me know if that lighter option is preferable.'\}} \\
            }
        }
    }
     \\ \\

    \theutterance \stepcounter{utterance}  
    & & \multicolumn{4}{p{0.6\linewidth}}{
        \cellcolor[rgb]{0.9,0.9,0.9}{
            \makecell[{{p{\linewidth}}}]{
                \texttt{\tiny{[P1$\langle$GM]}}
                \texttt{REFUSE: \{'C50', 'B67', 'B72', 'C63', 'B24', 'C22', 'A98', 'B82', 'C71'\}} \\
\texttt{PROPOSAL: \{'C50', 'B67', 'B72', 'C63', 'B24', 'B82', 'C71', 'B62', 'A66'\}} \\
\texttt{ARGUMENT: \{'You asked which matters more—B62 or A66. B62 matters more on my side. This proposal makes just two swaps from your set: B62 in for C22 and A66 in for A98. It stays well under the limit at 3386 (101 spare) while preserving your efficient core. If replacing A98 is difficult, I can accept the single swap of B62 for C22 and keep everything else as{-}is (effort 3448). Let me know if that lighter option is preferable.'\}} \\
            }
        }
    }
    & \\ \\

    \theutterance \stepcounter{utterance}  
    & \multicolumn{4}{p{0.6\linewidth}}{
        \cellcolor[rgb]{0.9,0.9,0.9}{
            \makecell[{{p{\linewidth}}}]{
                \texttt{\tiny{[P1$\rangle$GM]}}
                \texttt{STRATEGIC REASONING: \{'Their double{-}swap proposal costs me 308 importance versus my last offer. They stated B62 matters most and even floated a single{-}swap variant (B62 for C22) keeping A98. That variant yields 5236 importance for me at 3448 effort—solid—and directly honors their priority. I’ll refuse the double swap and propose the single{-}swap set they said they could accept, aiming to close. If they insist on A66 as well, I can concede A98→A66, dropping my value by 124, and AGREE to their current set to end the game quickly.'\}} \\
\texttt{REFUSE: \{'C50', 'B67', 'B72', 'C63', 'B24', 'B82', 'C71', 'B62', 'A66'\}} \\
\texttt{PROPOSAL: \{'C50', 'B67', 'B72', 'C63', 'B24', 'A98', 'B82', 'C71', 'B62'\}} \\
\texttt{ARGUMENT: \{'You said B62 matters more for you—I’ve included it. To keep the mix efficient and under the limit, I’m keeping A98 rather than adding A66. This set totals 3448 effort (well within 3487) and preserves the core strengths we both aligned on. Since you mentioned the single{-}swap option as acceptable, this should fit your priorities. If A66 is still a must{-}have, say so and I can switch A98→A66 to finalize, but this version seems the cleanest compromise.'\}} \\
            }
        }
    }
    & & \\ \\

    \theutterance \stepcounter{utterance}  
    & & \multicolumn{4}{p{0.6\linewidth}}{
        \cellcolor[rgb]{0.9,0.9,0.9}{
            \makecell[{{p{\linewidth}}}]{
                \texttt{\tiny{[GM$\rangle$P2]}}
                \texttt{REFUSE: \{'C50', 'B67', 'B72', 'C63', 'B24', 'B82', 'C71', 'B62', 'A66'\}} \\
\texttt{PROPOSAL: \{'C50', 'B67', 'B72', 'C63', 'B24', 'A98', 'B82', 'C71', 'B62'\}} \\
\texttt{ARGUMENT: \{'You said B62 matters more for you—I’ve included it. To keep the mix efficient and under the limit, I’m keeping A98 rather than adding A66. This set totals 3448 effort (well within 3487) and preserves the core strengths we both aligned on. Since you mentioned the single{-}swap option as acceptable, this should fit your priorities. If A66 is still a must{-}have, say so and I can switch A98→A66 to finalize, but this version seems the cleanest compromise.'\}} \\
            }
        }
    }
    & \\ \\

    \theutterance \stepcounter{utterance}  
    & & & \multicolumn{4}{p{0.6\linewidth}}{
        \cellcolor[rgb]{0.9,0.9,0.9}{
            \makecell[{{p{\linewidth}}}]{
                \texttt{\tiny{[GM$\langle$P2]}}
                \texttt{STRATEGIC REASONING: \{'Their current set gives me strong value. They explicitly offered to switch A98→A66 if it mattered to me. That swap is modest, keeps us well under the limit, and is likely acceptable to them. I’ll press only for that single change to close the deal quickly. If they unexpectedly resist, I can accept their current set to secure a good outcome.'\}} \\
\texttt{REFUSE: \{'C50', 'B67', 'B72', 'C63', 'B24', 'A98', 'B82', 'C71', 'B62'\}} \\
\texttt{PROPOSAL: \{'C50', 'B67', 'B72', 'C63', 'B24', 'A66', 'B82', 'C71', 'B62'\}} \\
\texttt{ARGUMENT: \{'You invited swapping A98→A66 if it mattered on my side. Let’s do exactly that and finalize. This keeps everything else intact and reduces total effort to 3386 (well under 3487) with a 101 buffer. It preserves your core selections while making a single, clean efficiency upgrade. Ready to agree on this?'\}} \\
            }
        }
    }
     \\ \\

    \theutterance \stepcounter{utterance}  
    & & \multicolumn{4}{p{0.6\linewidth}}{
        \cellcolor[rgb]{0.9,0.9,0.9}{
            \makecell[{{p{\linewidth}}}]{
                \texttt{\tiny{[P1$\langle$GM]}}
                \texttt{REFUSE: \{'C50', 'B67', 'B72', 'C63', 'B24', 'A98', 'B82', 'C71', 'B62'\}} \\
\texttt{PROPOSAL: \{'C50', 'B67', 'B72', 'C63', 'B24', 'A66', 'B82', 'C71', 'B62'\}} \\
\texttt{ARGUMENT: \{'You invited swapping A98→A66 if it mattered on my side. Let’s do exactly that and finalize. This keeps everything else intact and reduces total effort to 3386 (well under 3487) with a 101 buffer. It preserves your core selections while making a single, clean efficiency upgrade. Ready to agree on this?'\}} \\
            }
        }
    }
    & \\ \\

    \theutterance \stepcounter{utterance}  
    & \multicolumn{4}{p{0.6\linewidth}}{
        \cellcolor[rgb]{0.9,0.9,0.9}{
            \makecell[{{p{\linewidth}}}]{
                \texttt{\tiny{[P1$\rangle$GM]}}
                \texttt{STRATEGIC REASONING: \{'Their final set yields 5112 utility for me, just 124 below my preferred variant with A98. Since it includes B62 (their top priority), remains comfortably under the limit, and likely secures agreement now, I will accept rather than risk a counter that could stall or worsen my outcome.'\}} \\
\texttt{AGREE: \{'C50', 'B67', 'B72', 'C63', 'B24', 'A66', 'B82', 'C71', 'B62'\}} \\
\texttt{ARGUMENT: \{'This honors your priorities on B62 and A66 while preserving our efficient core and staying well under the limit. Agreed.'\}} \\
            }
        }
    }
    & & \\ \\

    \theutterance \stepcounter{utterance}  
    & & & \multicolumn{2}{p{0.3\linewidth}}{
        \cellcolor[rgb]{0.9,0.9,0.9}{
            \makecell[{{p{\linewidth}}}]{
                \texttt{\tiny{[GM$|$GM]}}
                \texttt{game successful} \\
            }
        }
    }
    & & \\ \\

    \theutterance \stepcounter{utterance}  
    & & & \multicolumn{2}{p{0.3\linewidth}}{
        \cellcolor[rgb]{0.9,0.9,0.9}{
            \makecell[{{p{\linewidth}}}]{
                \texttt{\tiny{[GM$|$GM]}}
                \texttt{end game} \\
            }
        }
    }
    & & \\ \\

\end{supertabular}
}

\end{document}
