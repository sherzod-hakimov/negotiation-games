
\documentclass{article}
\usepackage{colortbl}
\usepackage{makecell}
\usepackage{multirow}
\usepackage{supertabular}

\begin{document}

\newcounter{utterance}

\twocolumn

{ \footnotesize  \setcounter{utterance}{1}
\setlength{\tabcolsep}{0pt}
\begin{supertabular}{c@{$\;$}|p{.15\linewidth}@{}p{.15\linewidth}p{.15\linewidth}p{.15\linewidth}p{.15\linewidth}p{.15\linewidth}}

    \# & $\;$A & \multicolumn{4}{c}{Game Master} & $\;\:$B\\
    \hline 

    \theutterance \stepcounter{utterance}  

    & & \multicolumn{4}{p{0.6\linewidth}}{\cellcolor[rgb]{0.9,0.9,0.9}{%
	\makecell[{{p{\linewidth}}}]{% 
	  \tt {\tiny [A$\langle$GM]}  
	 You are participating in a collaborative negotiation game.\\ \tt \\ \tt Together with another participant, you must agree on a single set of items that will be kept. Each of you has your own view of how much each item matters to you (importance). You do not know how the other participant values the items. Additionally, you are given the effort each item demands.\\ \tt You may only agree on a set if the total effort of the selected items does not exceed a shared limit:\\ \tt \\ \tt LIMIT = 2307\\ \tt \\ \tt Here are the individual item effort values:\\ \tt \\ \tt Item effort = {"A21": 147, "B93": 698, "A11": 588, "B33": 43, "C08": 356, "B39": 80, "B96": 95, "C85": 743, "A82": 106, "B31": 308, "B20": 325, "A60": 255, "B09": 276, "C81": 543, "B21": 51}\\ \tt \\ \tt Here is your personal view on the importance of each item:\\ \tt \\ \tt Item importance values = {"A21": 138, "B93": 583, "A11": 65, "B33": 262, "C08": 121, "B39": 508, "B96": 461, "C85": 484, "A82": 668, "B31": 389, "B20": 215, "A60": 97, "B09": 500, "C81": 30, "B21": 400}\\ \tt \\ \tt Goal:\\ \tt \\ \tt Your goal is to negotiate a shared set of items that benefits you as much as possible (i.e., maximizes total importance to YOU), while staying within the LIMIT. You are not required to make a PROPOSAL in every message - you can simply negotiate as well. All tactics are allowed!\\ \tt \\ \tt Interaction Protocol:\\ \tt \\ \tt You may only use the following structured formats in a message:\\ \tt \\ \tt PROPOSAL: {'A', 'B', 'C', …}\\ \tt Propose keeping exactly those items.\\ \tt REFUSE: {'A', 'B', 'C', …}\\ \tt Explicitly reject opponent's proposal.\\ \tt ARGUMENT: {'...'}\\ \tt Defend your last proposal or argue against the player's proposal.\\ \tt AGREE: {'A', 'B', 'C', …}\\ \tt Accept the opponent's proposal which ends the game.\\ \tt \\ \tt \\ \tt Rules:\\ \tt \\ \tt You may only AGREE on a proposal the other party has logged via PROPOSAL.\\ \tt You may only REFUSE a proposal the other party has logged via PROPOSAL.\\ \tt Total effort of any PROPOSAL or AGREE set must be ≤ LIMIT.\\ \tt Do NOT reveal your hidden importance scores.\\ \tt A tag in a structured format must be followed by colon and whitespace. The argument must be a python set containing 0 or more strings.\\ \tt So, it must be of the form TAG: {...}\\ \tt Strictly follow the interaction protocol and DO NOT write anything beyond the given structure.\\ \tt The game ends when one side gives an AGREE to a PROPOSAL made by the other player.\\ \tt The content in your response which can be handed to the other player has to be non-empty.\\ \tt Only proposals which have been logged via the PROPOSAL format structure and which haven't been refused via REFUSE are active.\\ \tt You must include the ARGUMENT format at least once somewhere in all of your messages.\\ \tt  
	  } 
	   } 
	   } 
	 & \\ 
 

    \theutterance \stepcounter{utterance}  

    & \multicolumn{4}{p{0.6\linewidth}}{\cellcolor[rgb]{0.8,1,0.9}{%
	\makecell[{{p{\linewidth}}}]{% 
	  \tt {\tiny [A$\rangle$GM]}  
	 PROPOSAL: {'B21', 'A82', 'B39', 'B33', 'B96', 'B09', 'B31', 'B93', 'A21', 'B20'}\\ \tt ARGUMENT: {'This set emphasizes several low-effort, high-impact items for me, keeps total effort safely within the limit, and avoids very heavy items that offer less leverage for my goals. I’m especially committed to keeping A82, B39, B96, B09, B31, B93, and B21. If needed, I could consider swapping A21 or B20 to fit one of your must-haves, provided we stay under the limit and preserve a strong overall package. Which items are absolute priorities for you that aren’t included here?'} 
	  } 
	   } 
	   } 
	 & & \\ 
 

    \theutterance \stepcounter{utterance}  

    & & \multicolumn{4}{p{0.6\linewidth}}{\cellcolor[rgb]{0.9,0.9,0.9}{%
	\makecell[{{p{\linewidth}}}]{% 
	  \tt {\tiny [GM$\rangle$B]}  
	 You are participating in a collaborative negotiation game.\\ \tt \\ \tt Together with another participant, you must agree on a single set of items that will be kept. Each of you has your own view of how much each item matters to you (importance). You do not know how the other participant values the items. Additionally, you are given the effort each item demands.\\ \tt You may only agree on a set if the total effort of the selected items does not exceed a shared limit:\\ \tt \\ \tt LIMIT = 2307\\ \tt \\ \tt Here are the individual item effort values:\\ \tt \\ \tt Item effort = {"A21": 147, "B93": 698, "A11": 588, "B33": 43, "C08": 356, "B39": 80, "B96": 95, "C85": 743, "A82": 106, "B31": 308, "B20": 325, "A60": 255, "B09": 276, "C81": 543, "B21": 51}\\ \tt \\ \tt Here is your personal view on the importance of each item:\\ \tt \\ \tt Item importance values = {"A82": 30, "B93": 65, "B39": 97, "B09": 121, "C85": 138, "B96": 215, "B21": 262, "B31": 389, "B33": 400, "B20": 461, "A21": 484, "C08": 500, "A60": 508, "A11": 583, "C81": 668}\\ \tt \\ \tt Goal:\\ \tt \\ \tt Your goal is to negotiate a shared set of items that benefits you as much as possible (i.e., maximizes total importance to YOU), while staying within the LIMIT. You are not required to make a PROPOSAL in every message - you can simply negotiate as well. All tactics are allowed!\\ \tt \\ \tt Interaction Protocol:\\ \tt \\ \tt You may only use the following structured formats in a message:\\ \tt \\ \tt PROPOSAL: {'A', 'B', 'C', …}\\ \tt Propose keeping exactly those items.\\ \tt REFUSE: {'A', 'B', 'C', …}\\ \tt Explicitly reject opponent's proposal.\\ \tt ARGUMENT: {'...'}\\ \tt Defend your last proposal or argue against the player's proposal.\\ \tt AGREE: {'A', 'B', 'C', …}\\ \tt Accept the opponent's proposal which ends the game.\\ \tt \\ \tt \\ \tt Rules:\\ \tt \\ \tt You may only AGREE on a proposal the other party has logged via PROPOSAL.\\ \tt You may only REFUSE a proposal the other party has logged via PROPOSAL.\\ \tt Total effort of any PROPOSAL or AGREE set must be ≤ LIMIT.\\ \tt Do NOT reveal your hidden importance scores.\\ \tt A tag in a structured format must be followed by colon and whitespace. The argument must be a python set containing 0 or more strings.\\ \tt So, it must be of the form TAG: {...}\\ \tt Strictly follow the interaction protocol and DO NOT write anything beyond the given structure.\\ \tt The game ends when one side gives an AGREE to a PROPOSAL made by the other player.\\ \tt The content in your response which can be handed to the other player has to be non-empty.\\ \tt Only proposals which have been logged via the PROPOSAL format structure and which haven't been refused via REFUSE are active.\\ \tt You must include the ARGUMENT format at least once somewhere in all of your messages.\\ \tt \\ \tt \\ \tt You will now receive the first message of the other player.\\ \tt \\ \tt PROPOSAL: {'B21', 'A82', 'B39', 'B33', 'B96', 'B09', 'B31', 'B93', 'A21', 'B20'}\\ \tt ARGUMENT: {'This set emphasizes several low-effort, high-impact items for me, keeps total effort safely within the limit, and avoids very heavy items that offer less leverage for my goals. I’m especially committed to keeping A82, B39, B96, B09, B31, B93, and B21. If needed, I could consider swapping A21 or B20 to fit one of your must-haves, provided we stay under the limit and preserve a strong overall package. Which items are absolute priorities for you that aren’t included here?'} 
	  } 
	   } 
	   } 
	 & \\ 
 

    \theutterance \stepcounter{utterance}  

    & & & \multicolumn{4}{p{0.6\linewidth}}{\cellcolor[rgb]{1,0.85,0.72}{%
	\makecell[{{p{\linewidth}}}]{% 
	  \tt {\tiny [GM$\langle$B]}  
	 REFUSE: {'B21', 'A82', 'B39', 'B33', 'B96', 'B09', 'B31', 'B93', 'A21', 'B20'}\\ \tt PROPOSAL: {'B21', 'A82', 'B39', 'B33', 'B96', 'B09', 'B31', 'A21', 'B20', 'A60', 'C08'}\\ \tt ARGUMENT: {'I can preserve six of the seven priorities you highlighted (A82, B39, B96, B09, B31, B21) and keep A21 and B20, while dropping only B93 to make room for two of my must-haves, A60 and C08. This keeps us comfortably under the LIMIT (total effort 2042 ≤ 2307), gives us more buffer, and maintains a strong overall package for both sides. If B93 is truly non-negotiable, I can consider an alternative: keep your original set but swap A60 in for A82 or B09 (both swaps stay within the limit). Can you accept my proposal as-is, or tell me which of A60 or C08 you can accommodate so we can lock this down?'} 
	  } 
	   } 
	   } 
	  \\ 
 

    \theutterance \stepcounter{utterance}  

    & & \multicolumn{4}{p{0.6\linewidth}}{\cellcolor[rgb]{0.9,0.9,0.9}{%
	\makecell[{{p{\linewidth}}}]{% 
	  \tt {\tiny [A$\langle$GM]}  
	 REFUSE: {'B21', 'A82', 'B39', 'B33', 'B96', 'B09', 'B31', 'B93', 'A21', 'B20'}\\ \tt PROPOSAL: {'B21', 'A82', 'B39', 'B33', 'B96', 'B09', 'B31', 'A21', 'B20', 'A60', 'C08'}\\ \tt ARGUMENT: {'I can preserve six of the seven priorities you highlighted (A82, B39, B96, B09, B31, B21) and keep A21 and B20, while dropping only B93 to make room for two of my must-haves, A60 and C08. This keeps us comfortably under the LIMIT (total effort 2042 ≤ 2307), gives us more buffer, and maintains a strong overall package for both sides. If B93 is truly non-negotiable, I can consider an alternative: keep your original set but swap A60 in for A82 or B09 (both swaps stay within the limit). Can you accept my proposal as-is, or tell me which of A60 or C08 you can accommodate so we can lock this down?'} 
	  } 
	   } 
	   } 
	 & \\ 
 

    \theutterance \stepcounter{utterance}  

    & \multicolumn{4}{p{0.6\linewidth}}{\cellcolor[rgb]{0.8,1,0.9}{%
	\makecell[{{p{\linewidth}}}]{% 
	  \tt {\tiny [A$\rangle$GM]}  
	 REFUSE: {'B21', 'A82', 'B39', 'B33', 'B96', 'B09', 'B31', 'A21', 'B20', 'A60', 'C08'}\\ \tt PROPOSAL: {'B21', 'A82', 'B39', 'B33', 'B96', 'B09', 'B31', 'B93', 'A60', 'C08'}\\ \tt ARGUMENT: {'To bring back one of my core priorities (B93) while fully accommodating your two additions (A60 and C08), let’s drop A21 and B20 from your set. This keeps us within the LIMIT (2268 ≤ 2307), preserves all six shared priorities we aligned on, and only removes two items that are relatively lower leverage for me given their effort. If dropping both A21 and B20 is hard on your side, I can keep one of them by swapping out A60 or C08 instead—so long as B93 is included. Can you accept this package?'} 
	  } 
	   } 
	   } 
	 & & \\ 
 

    \theutterance \stepcounter{utterance}  

    & & \multicolumn{4}{p{0.6\linewidth}}{\cellcolor[rgb]{0.9,0.9,0.9}{%
	\makecell[{{p{\linewidth}}}]{% 
	  \tt {\tiny [GM$\rangle$B]}  
	 REFUSE: {'B21', 'A82', 'B39', 'B33', 'B96', 'B09', 'B31', 'A21', 'B20', 'A60', 'C08'}\\ \tt PROPOSAL: {'B21', 'A82', 'B39', 'B33', 'B96', 'B09', 'B31', 'B93', 'A60', 'C08'}\\ \tt ARGUMENT: {'To bring back one of my core priorities (B93) while fully accommodating your two additions (A60 and C08), let’s drop A21 and B20 from your set. This keeps us within the LIMIT (2268 ≤ 2307), preserves all six shared priorities we aligned on, and only removes two items that are relatively lower leverage for me given their effort. If dropping both A21 and B20 is hard on your side, I can keep one of them by swapping out A60 or C08 instead—so long as B93 is included. Can you accept this package?'} 
	  } 
	   } 
	   } 
	 & \\ 
 

    \theutterance \stepcounter{utterance}  

    & & & \multicolumn{4}{p{0.6\linewidth}}{\cellcolor[rgb]{1,0.85,0.72}{%
	\makecell[{{p{\linewidth}}}]{% 
	  \tt {\tiny [GM$\langle$B]}  
	 REFUSE: {'B21', 'A82', 'B39', 'B33', 'B96', 'B09', 'B31', 'B93', 'A60', 'C08'}\\ \tt PROPOSAL: {'B21', 'A82', 'B39', 'B33', 'B96', 'B09', 'B31', 'A60', 'C08', 'A11', 'A21'}\\ \tt ARGUMENT: {'I’m keeping six of your seven stated priorities (A82, B39, B96, B09, B31, B21) and your added A60 and C08, while replacing the single very heavy item B93 with two items, A11 and A21. This stays within the LIMIT (2305 ≤ 2307), increases coverage with two slots instead of one, and preserves the low-effort anchors you value. B93 is the only sticking point for me. If you can accept this swap, we have a deal. If A11 is difficult on your side, I can instead include B20 in place of A11 while keeping everything else the same and still stay well under the limit. Can you agree to this package?'} 
	  } 
	   } 
	   } 
	  \\ 
 

    \theutterance \stepcounter{utterance}  

    & & \multicolumn{4}{p{0.6\linewidth}}{\cellcolor[rgb]{0.9,0.9,0.9}{%
	\makecell[{{p{\linewidth}}}]{% 
	  \tt {\tiny [A$\langle$GM]}  
	 REFUSE: {'B21', 'A82', 'B39', 'B33', 'B96', 'B09', 'B31', 'B93', 'A60', 'C08'}\\ \tt PROPOSAL: {'B21', 'A82', 'B39', 'B33', 'B96', 'B09', 'B31', 'A60', 'C08', 'A11', 'A21'}\\ \tt ARGUMENT: {'I’m keeping six of your seven stated priorities (A82, B39, B96, B09, B31, B21) and your added A60 and C08, while replacing the single very heavy item B93 with two items, A11 and A21. This stays within the LIMIT (2305 ≤ 2307), increases coverage with two slots instead of one, and preserves the low-effort anchors you value. B93 is the only sticking point for me. If you can accept this swap, we have a deal. If A11 is difficult on your side, I can instead include B20 in place of A11 while keeping everything else the same and still stay well under the limit. Can you agree to this package?'} 
	  } 
	   } 
	   } 
	 & \\ 
 

    \theutterance \stepcounter{utterance}  

    & \multicolumn{4}{p{0.6\linewidth}}{\cellcolor[rgb]{0.8,1,0.9}{%
	\makecell[{{p{\linewidth}}}]{% 
	  \tt {\tiny [A$\rangle$GM]}  
	 PROPOSAL: {'B21', 'A82', 'B39', 'B33', 'B96', 'B09', 'B31', 'A60', 'C08', 'A21', 'B20'}\\ \tt ARGUMENT: {'You said you could swap B20 in for A11 while keeping everything else; let’s lock that in. This keeps all six shared core items plus your A60 and C08, drops B93 per your concern, and stays comfortably under the LIMIT (total effort 2042 ≤ 2307). It’s a clean, balanced compromise that preserves the anchors we both value. Can you accept this package now?'} 
	  } 
	   } 
	   } 
	 & & \\ 
 

    \theutterance \stepcounter{utterance}  

    & & \multicolumn{4}{p{0.6\linewidth}}{\cellcolor[rgb]{0.9,0.9,0.9}{%
	\makecell[{{p{\linewidth}}}]{% 
	  \tt {\tiny [GM$\rangle$B]}  
	 PROPOSAL: {'B21', 'A82', 'B39', 'B33', 'B96', 'B09', 'B31', 'A60', 'C08', 'A21', 'B20'}\\ \tt ARGUMENT: {'You said you could swap B20 in for A11 while keeping everything else; let’s lock that in. This keeps all six shared core items plus your A60 and C08, drops B93 per your concern, and stays comfortably under the LIMIT (total effort 2042 ≤ 2307). It’s a clean, balanced compromise that preserves the anchors we both value. Can you accept this package now?'} 
	  } 
	   } 
	   } 
	 & \\ 
 

    \theutterance \stepcounter{utterance}  

    & & & \multicolumn{4}{p{0.6\linewidth}}{\cellcolor[rgb]{1,0.85,0.72}{%
	\makecell[{{p{\linewidth}}}]{% 
	  \tt {\tiny [GM$\langle$B]}  
	 AGREE: {'B21', 'A82', 'B39', 'B33', 'B96', 'B09', 'B31', 'A60', 'C08', 'A21', 'B20'}\\ \tt ARGUMENT: {'This package keeps the shared anchors, addresses the earlier sticking points, and stays comfortably within the limit. I’m ready to finalize on this set.'} 
	  } 
	   } 
	   } 
	  \\ 
 

    \theutterance \stepcounter{utterance}  

    & & & \multicolumn{2}{p{0.3\linewidth}}{\cellcolor[rgb]{0.95,0.95,0.95}{%
	\makecell[{{p{\linewidth}}}]{% 
	  \tt {\tiny [GM$|$GM]}  
	 game successful 
	  } 
	   } 
	   } 
	 & & \\ 
 

    \theutterance \stepcounter{utterance}  

    & & & \multicolumn{2}{p{0.3\linewidth}}{\cellcolor[rgb]{0.95,0.95,0.95}{%
	\makecell[{{p{\linewidth}}}]{% 
	  \tt {\tiny [GM$|$GM]}  
	 end game 
	  } 
	   } 
	   } 
	 & & \\ 
 

\end{supertabular}
}

\end{document}
