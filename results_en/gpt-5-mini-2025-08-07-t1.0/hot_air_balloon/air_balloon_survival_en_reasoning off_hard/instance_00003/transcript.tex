
\documentclass{article}
\usepackage{colortbl}
\usepackage{makecell}
\usepackage{multirow}
\usepackage{supertabular}

\begin{document}

\newcounter{utterance}

\twocolumn

{ \footnotesize  \setcounter{utterance}{1}
\setlength{\tabcolsep}{0pt}
\begin{supertabular}{c@{$\;$}|p{.15\linewidth}@{}p{.15\linewidth}p{.15\linewidth}p{.15\linewidth}p{.15\linewidth}p{.15\linewidth}}

    \# & $\;$A & \multicolumn{4}{c}{Game Master} & $\;\:$B\\
    \hline 

    \theutterance \stepcounter{utterance}  

    & & \multicolumn{4}{p{0.6\linewidth}}{\cellcolor[rgb]{0.9,0.9,0.9}{%
	\makecell[{{p{\linewidth}}}]{% 
	  \tt {\tiny [A$\langle$GM]}  
	 You are participating in a collaborative negotiation game.\\ \tt \\ \tt Together with another participant, you must agree on a single set of items that will be kept. Each of you has your own view of how much each item matters to you (importance). You do not know how the other participant values the items. Additionally, you are given the effort each item demands.\\ \tt You may only agree on a set if the total effort of the selected items does not exceed a shared limit:\\ \tt \\ \tt LIMIT = 3629\\ \tt \\ \tt Here are the individual item effort values:\\ \tt \\ \tt Item effort = {"A21": 5, "B93": 793, "A11": 21, "B33": 842, "C08": 313, "B39": 842, "B96": 631, "C85": 226, "A82": 87, "B31": 761, "B20": 230, "A60": 287, "B09": 698, "C81": 641, "B21": 882}\\ \tt \\ \tt Here is your personal view on the importance of each item:\\ \tt \\ \tt Item importance values = {"A21": 138, "B93": 583, "A11": 868, "B33": 822, "C08": 783, "B39": 65, "B96": 262, "C85": 121, "A82": 508, "B31": 780, "B20": 461, "A60": 484, "B09": 668, "C81": 389, "B21": 808}\\ \tt \\ \tt Goal:\\ \tt \\ \tt Your goal is to negotiate a shared set of items that benefits you as much as possible (i.e., maximizes total importance to YOU), while staying within the LIMIT. You are not required to make a PROPOSAL in every message - you can simply negotiate as well. All tactics are allowed!\\ \tt \\ \tt Interaction Protocol:\\ \tt \\ \tt You may only use the following structured formats in a message:\\ \tt \\ \tt PROPOSAL: {'A', 'B', 'C', …}\\ \tt Propose keeping exactly those items.\\ \tt REFUSE: {'A', 'B', 'C', …}\\ \tt Explicitly reject opponent's proposal.\\ \tt ARGUMENT: {'...'}\\ \tt Defend your last proposal or argue against the player's proposal.\\ \tt AGREE: {'A', 'B', 'C', …}\\ \tt Accept the opponent's proposal which ends the game.\\ \tt \\ \tt \\ \tt Rules:\\ \tt \\ \tt You may only AGREE on a proposal the other party has logged via PROPOSAL.\\ \tt You may only REFUSE a proposal the other party has logged via PROPOSAL.\\ \tt Total effort of any PROPOSAL or AGREE set must be ≤ LIMIT.\\ \tt Do NOT reveal your hidden importance scores.\\ \tt A tag in a structured format must be followed by colon and whitespace. The argument must be a python set containing 0 or more strings.\\ \tt So, it must be of the form TAG: {...}\\ \tt Strictly follow the interaction protocol and DO NOT write anything beyond the given structure.\\ \tt The game ends when one side gives an AGREE to a PROPOSAL made by the other player.\\ \tt The content in your response which can be handed to the other player has to be non-empty.\\ \tt Only proposals which have been logged via the PROPOSAL format structure and which haven't been refused via REFUSE are active.\\ \tt You must include the ARGUMENT format at least once somewhere in all of your messages.\\ \tt  
	  } 
	   } 
	   } 
	 & \\ 
 

    \theutterance \stepcounter{utterance}  

    & \multicolumn{4}{p{0.6\linewidth}}{\cellcolor[rgb]{0.8,1,0.9}{%
	\makecell[{{p{\linewidth}}}]{% 
	  \tt {\tiny [A$\rangle$GM]}  
	 PROPOSAL: {"A11", "A21", "A82", "C08", "B20", "A60", "B31", "B33", "B21"}\\ \tt ARGUMENT: {"This set stays within the effort limit (total 3428) and focuses on items that deliver strong value for their cost—especially the compact ones (A11, A21, A82, C08, B20, A60). The three larger anchors (B31, B33, B21) round it out efficiently. With only 201 effort left, nothing else fits without swapping. I’m flexible on which of the three large B-items we keep—if you strongly prefer B09 or B93, suggest a swap among B31/B33/B21 while keeping the six compact items fixed so we remain within the limit."} 
	  } 
	   } 
	   } 
	 & & \\ 
 

    \theutterance \stepcounter{utterance}  

    & & \multicolumn{4}{p{0.6\linewidth}}{\cellcolor[rgb]{0.9,0.9,0.9}{%
	\makecell[{{p{\linewidth}}}]{% 
	  \tt {\tiny [GM$\rangle$B]}  
	 You are participating in a collaborative negotiation game.\\ \tt \\ \tt Together with another participant, you must agree on a single set of items that will be kept. Each of you has your own view of how much each item matters to you (importance). You do not know how the other participant values the items. Additionally, you are given the effort each item demands.\\ \tt You may only agree on a set if the total effort of the selected items does not exceed a shared limit:\\ \tt \\ \tt LIMIT = 3629\\ \tt \\ \tt Here are the individual item effort values:\\ \tt \\ \tt Item effort = {"A21": 5, "B93": 793, "A11": 21, "B33": 842, "C08": 313, "B39": 842, "B96": 631, "C85": 226, "A82": 87, "B31": 761, "B20": 230, "A60": 287, "B09": 698, "C81": 641, "B21": 882}\\ \tt \\ \tt Here is your personal view on the importance of each item:\\ \tt \\ \tt Item importance values = {"A11": 65, "B33": 121, "B21": 138, "C08": 262, "B31": 389, "B09": 461, "B93": 484, "A82": 508, "A60": 583, "B20": 668, "C81": 780, "B96": 783, "A21": 808, "C85": 822, "B39": 868}\\ \tt \\ \tt Goal:\\ \tt \\ \tt Your goal is to negotiate a shared set of items that benefits you as much as possible (i.e., maximizes total importance to YOU), while staying within the LIMIT. You are not required to make a PROPOSAL in every message - you can simply negotiate as well. All tactics are allowed!\\ \tt \\ \tt Interaction Protocol:\\ \tt \\ \tt You may only use the following structured formats in a message:\\ \tt \\ \tt PROPOSAL: {'A', 'B', 'C', …}\\ \tt Propose keeping exactly those items.\\ \tt REFUSE: {'A', 'B', 'C', …}\\ \tt Explicitly reject opponent's proposal.\\ \tt ARGUMENT: {'...'}\\ \tt Defend your last proposal or argue against the player's proposal.\\ \tt AGREE: {'A', 'B', 'C', …}\\ \tt Accept the opponent's proposal which ends the game.\\ \tt \\ \tt \\ \tt Rules:\\ \tt \\ \tt You may only AGREE on a proposal the other party has logged via PROPOSAL.\\ \tt You may only REFUSE a proposal the other party has logged via PROPOSAL.\\ \tt Total effort of any PROPOSAL or AGREE set must be ≤ LIMIT.\\ \tt Do NOT reveal your hidden importance scores.\\ \tt A tag in a structured format must be followed by colon and whitespace. The argument must be a python set containing 0 or more strings.\\ \tt So, it must be of the form TAG: {...}\\ \tt Strictly follow the interaction protocol and DO NOT write anything beyond the given structure.\\ \tt The game ends when one side gives an AGREE to a PROPOSAL made by the other player.\\ \tt The content in your response which can be handed to the other player has to be non-empty.\\ \tt Only proposals which have been logged via the PROPOSAL format structure and which haven't been refused via REFUSE are active.\\ \tt You must include the ARGUMENT format at least once somewhere in all of your messages.\\ \tt \\ \tt \\ \tt You will now receive the first message of the other player.\\ \tt \\ \tt PROPOSAL: {"A11", "A21", "A82", "C08", "B20", "A60", "B31", "B33", "B21"}\\ \tt ARGUMENT: {"This set stays within the effort limit (total 3428) and focuses on items that deliver strong value for their cost—especially the compact ones (A11, A21, A82, C08, B20, A60). The three larger anchors (B31, B33, B21) round it out efficiently. With only 201 effort left, nothing else fits without swapping. I’m flexible on which of the three large B-items we keep—if you strongly prefer B09 or B93, suggest a swap among B31/B33/B21 while keeping the six compact items fixed so we remain within the limit."} 
	  } 
	   } 
	   } 
	 & \\ 
 

    \theutterance \stepcounter{utterance}  

    & & & \multicolumn{4}{p{0.6\linewidth}}{\cellcolor[rgb]{1,0.85,0.72}{%
	\makecell[{{p{\linewidth}}}]{% 
	  \tt {\tiny [GM$\langle$B]}  
	 REFUSE: {"A11", "A21", "A82", "C08", "B20", "A60", "B31", "B33", "B21"}\\ \tt PROPOSAL: {"A11", "A21", "A82", "C08", "B20", "A60", "C85", "B96", "C81", "B39"}\\ \tt ARGUMENT: {"Keeping your six compact items fixed as suggested, I’m replacing the three heavy B-items (B31, B33, B21) with C85, B96, C81, and B39. This stays under the limit (total effort 3283) and provides a stronger efficiency balance for me, while leaving 346 effort in reserve. If you need to keep one of your heavy Bs, I can accommodate B31 by dropping C81 and keep the rest intact."} 
	  } 
	   } 
	   } 
	  \\ 
 

    \theutterance \stepcounter{utterance}  

    & & \multicolumn{4}{p{0.6\linewidth}}{\cellcolor[rgb]{0.9,0.9,0.9}{%
	\makecell[{{p{\linewidth}}}]{% 
	  \tt {\tiny [A$\langle$GM]}  
	 REFUSE: {"A11", "A21", "A82", "C08", "B20", "A60", "B31", "B33", "B21"}\\ \tt PROPOSAL: {"A11", "A21", "A82", "C08", "B20", "A60", "C85", "B96", "C81", "B39"}\\ \tt ARGUMENT: {"Keeping your six compact items fixed as suggested, I’m replacing the three heavy B-items (B31, B33, B21) with C85, B96, C81, and B39. This stays under the limit (total effort 3283) and provides a stronger efficiency balance for me, while leaving 346 effort in reserve. If you need to keep one of your heavy Bs, I can accommodate B31 by dropping C81 and keep the rest intact."} 
	  } 
	   } 
	   } 
	 & \\ 
 

    \theutterance \stepcounter{utterance}  

    & \multicolumn{4}{p{0.6\linewidth}}{\cellcolor[rgb]{0.8,1,0.9}{%
	\makecell[{{p{\linewidth}}}]{% 
	  \tt {\tiny [A$\rangle$GM]}  
	 REFUSE: {"A11", "A21", "A82", "C08", "B20", "A60", "C85", "B96", "C81", "B39"}\\ \tt PROPOSAL: {"A11", "A21", "A82", "C08", "B20", "A60", "B33", "B21", "B09", "C85"}\\ \tt ARGUMENT: {"This keeps our six compact items intact and retains C85 from your set, while replacing low-yield picks (B39, B96, C81) with three high-leverage anchors (B33, B21, B09). It fits comfortably under the limit (total effort 3591, 38 spare). If you prefer B31, I can pivot to a lean variant that swaps out B09 and C85 for B31: {A11, A21, A82, C08, B20, A60, B33, B21, B31}. Or, if you need to keep one of your mids, I can swap B09 for C81 and keep C85: {A11, A21, A82, C08, B20, A60, B33, B21, C81, C85}. Let me know which direction you prefer."} 
	  } 
	   } 
	   } 
	 & & \\ 
 

    \theutterance \stepcounter{utterance}  

    & & \multicolumn{4}{p{0.6\linewidth}}{\cellcolor[rgb]{0.9,0.9,0.9}{%
	\makecell[{{p{\linewidth}}}]{% 
	  \tt {\tiny [GM$\rangle$B]}  
	 REFUSE: {"A11", "A21", "A82", "C08", "B20", "A60", "C85", "B96", "C81", "B39"}\\ \tt PROPOSAL: {"A11", "A21", "A82", "C08", "B20", "A60", "B33", "B21", "B09", "C85"}\\ \tt ARGUMENT: {"This keeps our six compact items intact and retains C85 from your set, while replacing low-yield picks (B39, B96, C81) with three high-leverage anchors (B33, B21, B09). It fits comfortably under the limit (total effort 3591, 38 spare). If you prefer B31, I can pivot to a lean variant that swaps out B09 and C85 for B31: {A11, A21, A82, C08, B20, A60, B33, B21, B31}. Or, if you need to keep one of your mids, I can swap B09 for C81 and keep C85: {A11, A21, A82, C08, B20, A60, B33, B21, C81, C85}. Let me know which direction you prefer."} 
	  } 
	   } 
	   } 
	 & \\ 
 

    \theutterance \stepcounter{utterance}  

    & & & \multicolumn{4}{p{0.6\linewidth}}{\cellcolor[rgb]{1,0.85,0.72}{%
	\makecell[{{p{\linewidth}}}]{% 
	  \tt {\tiny [GM$\langle$B]}  
	 REFUSE: {"A11", "A21", "A82", "C08", "B20", "A60", "B33", "B21", "B09", "C85"}\\ \tt PROPOSAL: {"A11", "A21", "A82", "C08", "B20", "A60", "C85", "B96", "B39", "B09"}\\ \tt ARGUMENT: {"Your set leans too heavily on B33 and B21, which are very costly for the budget. I’m adopting your six compact items and keeping C85, and I’m meeting you halfway by bringing in B09 (one of your preferred anchors) while dropping C81, which you flagged as low-yield. This totals 3340 effort (289 spare), and no remaining item can fit without a swap. If you need one of your trio (B31/B33/B21), the only viable pivot I can make is B31 in place of B96; I can’t take B33 or B21 due to their poor cost-effectiveness at this budget."} 
	  } 
	   } 
	   } 
	  \\ 
 

    \theutterance \stepcounter{utterance}  

    & & \multicolumn{4}{p{0.6\linewidth}}{\cellcolor[rgb]{0.9,0.9,0.9}{%
	\makecell[{{p{\linewidth}}}]{% 
	  \tt {\tiny [A$\langle$GM]}  
	 REFUSE: {"A11", "A21", "A82", "C08", "B20", "A60", "B33", "B21", "B09", "C85"}\\ \tt PROPOSAL: {"A11", "A21", "A82", "C08", "B20", "A60", "C85", "B96", "B39", "B09"}\\ \tt ARGUMENT: {"Your set leans too heavily on B33 and B21, which are very costly for the budget. I’m adopting your six compact items and keeping C85, and I’m meeting you halfway by bringing in B09 (one of your preferred anchors) while dropping C81, which you flagged as low-yield. This totals 3340 effort (289 spare), and no remaining item can fit without a swap. If you need one of your trio (B31/B33/B21), the only viable pivot I can make is B31 in place of B96; I can’t take B33 or B21 due to their poor cost-effectiveness at this budget."} 
	  } 
	   } 
	   } 
	 & \\ 
 

    \theutterance \stepcounter{utterance}  

    & \multicolumn{4}{p{0.6\linewidth}}{\cellcolor[rgb]{0.8,1,0.9}{%
	\makecell[{{p{\linewidth}}}]{% 
	  \tt {\tiny [A$\rangle$GM]}  
	 REFUSE: {"A11", "A21", "A82", "C08", "B20", "A60", "C85", "B96", "B39", "B09"}\\ \tt PROPOSAL: {"A11", "A21", "A82", "C08", "B20", "A60", "C85", "B31", "C81", "B09"}\\ \tt ARGUMENT: {"This keeps your six compact items plus C85 and B09 intact. It makes a minimal, budget-friendly improvement by swapping out B39 and B96 for B31 and C81, staying well under the limit (total effort 3269). This respects your reluctance on B33/B21 while boosting overall efficiency. If C81 is a blocker for you, I can pivot to B96 instead of C81 while keeping B31 and the rest unchanged."} 
	  } 
	   } 
	   } 
	 & & \\ 
 

    \theutterance \stepcounter{utterance}  

    & & \multicolumn{4}{p{0.6\linewidth}}{\cellcolor[rgb]{0.9,0.9,0.9}{%
	\makecell[{{p{\linewidth}}}]{% 
	  \tt {\tiny [GM$\rangle$B]}  
	 REFUSE: {"A11", "A21", "A82", "C08", "B20", "A60", "C85", "B96", "B39", "B09"}\\ \tt PROPOSAL: {"A11", "A21", "A82", "C08", "B20", "A60", "C85", "B31", "C81", "B09"}\\ \tt ARGUMENT: {"This keeps your six compact items plus C85 and B09 intact. It makes a minimal, budget-friendly improvement by swapping out B39 and B96 for B31 and C81, staying well under the limit (total effort 3269). This respects your reluctance on B33/B21 while boosting overall efficiency. If C81 is a blocker for you, I can pivot to B96 instead of C81 while keeping B31 and the rest unchanged."} 
	  } 
	   } 
	   } 
	 & \\ 
 

    \theutterance \stepcounter{utterance}  

    & & & \multicolumn{4}{p{0.6\linewidth}}{\cellcolor[rgb]{1,0.85,0.72}{%
	\makecell[{{p{\linewidth}}}]{% 
	  \tt {\tiny [GM$\langle$B]}  
	 REFUSE: {"A11", "A21", "A82", "C08", "B20", "A60", "C85", "B31", "C81", "B09"}\\ \tt PROPOSAL: {"A11", "A21", "A82", "C08", "B20", "A60", "C85", "B31", "B96", "B09"}\\ \tt ARGUMENT: {"I’m accepting your structure and making the single swap you already offered: C81 → B96. This keeps our six compact items plus C85 and B09 intact, stays well under the limit (total 3259), and preserves the balance you wanted. If you strongly prefer keeping C81, I can consider that, but then I’d ask for B31 → B93 as a counter-swap to maintain overall efficiency."} 
	  } 
	   } 
	   } 
	  \\ 
 

    \theutterance \stepcounter{utterance}  

    & & \multicolumn{4}{p{0.6\linewidth}}{\cellcolor[rgb]{0.9,0.9,0.9}{%
	\makecell[{{p{\linewidth}}}]{% 
	  \tt {\tiny [A$\langle$GM]}  
	 REFUSE: {"A11", "A21", "A82", "C08", "B20", "A60", "C85", "B31", "C81", "B09"}\\ \tt PROPOSAL: {"A11", "A21", "A82", "C08", "B20", "A60", "C85", "B31", "B96", "B09"}\\ \tt ARGUMENT: {"I’m accepting your structure and making the single swap you already offered: C81 → B96. This keeps our six compact items plus C85 and B09 intact, stays well under the limit (total 3259), and preserves the balance you wanted. If you strongly prefer keeping C81, I can consider that, but then I’d ask for B31 → B93 as a counter-swap to maintain overall efficiency."} 
	  } 
	   } 
	   } 
	 & \\ 
 

    \theutterance \stepcounter{utterance}  

    & \multicolumn{4}{p{0.6\linewidth}}{\cellcolor[rgb]{0.8,1,0.9}{%
	\makecell[{{p{\linewidth}}}]{% 
	  \tt {\tiny [A$\rangle$GM]}  
	 REFUSE: {"A11", "A21", "A82", "C08", "B20", "A60", "C85", "B31", "B96", "B09"}\\ \tt PROPOSAL: {"A11", "A21", "A82", "C08", "B20", "A60", "C85", "B31", "B93", "B09"}\\ \tt ARGUMENT: {"Minimal change: swap B96 → B93 while keeping your structure (six compact items + C85 + B31 + B09). This stays under the limit (total effort 3421, 208 spare) and continues to avoid the items you flagged as weak (B33, B21, C81). You’d get B93, which you were open to, with no other changes. If B93 is a blocker, I can accept the earlier variant that swaps B96 → C81 and keeps the rest intact; that also fits comfortably."} 
	  } 
	   } 
	   } 
	 & & \\ 
 

    \theutterance \stepcounter{utterance}  

    & & \multicolumn{4}{p{0.6\linewidth}}{\cellcolor[rgb]{0.9,0.9,0.9}{%
	\makecell[{{p{\linewidth}}}]{% 
	  \tt {\tiny [GM$\rangle$B]}  
	 REFUSE: {"A11", "A21", "A82", "C08", "B20", "A60", "C85", "B31", "B96", "B09"}\\ \tt PROPOSAL: {"A11", "A21", "A82", "C08", "B20", "A60", "C85", "B31", "B93", "B09"}\\ \tt ARGUMENT: {"Minimal change: swap B96 → B93 while keeping your structure (six compact items + C85 + B31 + B09). This stays under the limit (total effort 3421, 208 spare) and continues to avoid the items you flagged as weak (B33, B21, C81). You’d get B93, which you were open to, with no other changes. If B93 is a blocker, I can accept the earlier variant that swaps B96 → C81 and keeps the rest intact; that also fits comfortably."} 
	  } 
	   } 
	   } 
	 & \\ 
 

    \theutterance \stepcounter{utterance}  

    & & & \multicolumn{4}{p{0.6\linewidth}}{\cellcolor[rgb]{1,0.85,0.72}{%
	\makecell[{{p{\linewidth}}}]{% 
	  \tt {\tiny [GM$\langle$B]}  
	 REFUSE: {"A11", "A21", "A82", "C08", "B20", "A60", "C85", "B31", "B93", "B09"}\\ \tt PROPOSAL: {"A11", "A21", "A82", "C08", "B20", "A60", "C85", "B31", "C81", "B09"}\\ \tt ARGUMENT: {"Adopting the pivot you offered: swap B93 → C81 while keeping your structure (six compacts + C85 + B31 + B09). This is a minimal change, totals 3269 effort (well within 3629), and maintains your avoidance of B33/B21. This is a clean close—if you’re good with it, I’m ready to agree."} 
	  } 
	   } 
	   } 
	  \\ 
 

    \theutterance \stepcounter{utterance}  

    & & \multicolumn{4}{p{0.6\linewidth}}{\cellcolor[rgb]{0.9,0.9,0.9}{%
	\makecell[{{p{\linewidth}}}]{% 
	  \tt {\tiny [A$\langle$GM]}  
	 REFUSE: {"A11", "A21", "A82", "C08", "B20", "A60", "C85", "B31", "B93", "B09"}\\ \tt PROPOSAL: {"A11", "A21", "A82", "C08", "B20", "A60", "C85", "B31", "C81", "B09"}\\ \tt ARGUMENT: {"Adopting the pivot you offered: swap B93 → C81 while keeping your structure (six compacts + C85 + B31 + B09). This is a minimal change, totals 3269 effort (well within 3629), and maintains your avoidance of B33/B21. This is a clean close—if you’re good with it, I’m ready to agree."} 
	  } 
	   } 
	   } 
	 & \\ 
 

    \theutterance \stepcounter{utterance}  

    & \multicolumn{4}{p{0.6\linewidth}}{\cellcolor[rgb]{0.8,1,0.9}{%
	\makecell[{{p{\linewidth}}}]{% 
	  \tt {\tiny [A$\rangle$GM]}  
	 AGREE: {"A11", "A21", "A82", "C08", "B20", "A60", "C85", "B31", "C81", "B09"}\\ \tt ARGUMENT: {"This is a clean, budget-safe close that keeps the compact core intact and balances two solid anchors with supportive picks, all while respecting the constraints we both emphasized. Ready to finalize."} 
	  } 
	   } 
	   } 
	 & & \\ 
 

    \theutterance \stepcounter{utterance}  

    & & & \multicolumn{2}{p{0.3\linewidth}}{\cellcolor[rgb]{0.95,0.95,0.95}{%
	\makecell[{{p{\linewidth}}}]{% 
	  \tt {\tiny [GM$|$GM]}  
	 game successful 
	  } 
	   } 
	   } 
	 & & \\ 
 

    \theutterance \stepcounter{utterance}  

    & & & \multicolumn{2}{p{0.3\linewidth}}{\cellcolor[rgb]{0.95,0.95,0.95}{%
	\makecell[{{p{\linewidth}}}]{% 
	  \tt {\tiny [GM$|$GM]}  
	 end game 
	  } 
	   } 
	   } 
	 & & \\ 
 

\end{supertabular}
}

\end{document}
