
\documentclass{article}
\usepackage{colortbl}
\usepackage{makecell}
\usepackage{multirow}
\usepackage{supertabular}

\begin{document}

\newcounter{utterance}

\centering \large Interaction Transcript for game `dond', experiment `semi\_en', episode 0 with claude{-}sonnet{-}4{-}20250514{-}t0.0.
\vspace{24pt}

{ \footnotesize  \setcounter{utterance}{1}
\setlength{\tabcolsep}{0pt}
\begin{supertabular}{c@{$\;$}|p{.15\linewidth}@{}p{.15\linewidth}p{.15\linewidth}p{.15\linewidth}p{.15\linewidth}p{.15\linewidth}}
    \# & $\;$A & \multicolumn{4}{c}{Game Master} & $\;\:$B\\
    \hline

    \theutterance \stepcounter{utterance}  
    & & \multicolumn{4}{p{0.6\linewidth}}{
        \cellcolor[rgb]{0.9,0.9,0.9}{
            \makecell[{{p{\linewidth}}}]{
                \texttt{\tiny{[P1$\langle$GM]}}
                \texttt{You are playing a negotiation game in which you have to agree on how to divide a set of items among you and another player.} \\
\\ 
\texttt{Rules:} \\
\texttt{(a) You and the other player are given a set of items. Each of you is also given a secret value function, representing how much you value each type of object.} \\
\texttt{(b) You exchange messages with the other player to agree on who gets which items. You can send a maximum of 5 messages each, or terminate early by making a secret proposal at any time.} \\
\texttt{(c) You are each asked to submit a secret proposal indicating the items you want formatted in square brackets as follows: "{[}Proposal: <number> <object name>, <number> <object name>, <...>{]}"} \\
\texttt{(d) If your proposals are complementary, i.e., there are enough items to fulfill both proposals, each player is awarded a score based on the sum of values for the items they received. Otherwise, both of you get zero points.} \\
\texttt{(e) Your goal is to maximize the score you receive.} \\
\\ 
\texttt{Let us start.} \\
\\ 
\texttt{The set of available items is:} \\
\\ 
\texttt{3 crayons, 1 chalk, 2 bolts} \\
\\ 
\texttt{Your secret value function is:} \\
\\ 
\texttt{crayon: 0, chalk: 0, bolt: 5} \\
\\ 
\texttt{IMPORTANT: Your messages, unless it is the secret proposal, are directly transmitted to the other player, so do not include any response to the rules or text announcing your message. To make a secret proposal, use the indicated format. Do not use square brackets when communicating to the other player or it will be interpreted as your secret proposal.} \\
            }
        }
    }
    & \\ \\

    \theutterance \stepcounter{utterance}  
    & \multicolumn{4}{p{0.6\linewidth}}{
        \cellcolor[rgb]{0.9,0.9,0.9}{
            \makecell[{{p{\linewidth}}}]{
                \texttt{\tiny{[P1$\rangle$GM]}}
                \texttt{Hello! I see we have 3 crayons, 1 chalk, and 2 bolts to divide between us. I'm curious about what items you're most interested in. I have some preferences but I'd like to hear your thoughts first on what would work well for you.} \\
            }
        }
    }
    & & \\ \\

    \theutterance \stepcounter{utterance}  
    & & & \multicolumn{2}{p{0.3\linewidth}}{
        \cellcolor[rgb]{0.9,0.9,0.9}{
            \makecell[{{p{\linewidth}}}]{
                \texttt{\tiny{[GM$|$GM]}}
                \texttt{continue} \\
            }
        }
    }
    & & \\ \\

    \theutterance \stepcounter{utterance}  
    & & & \multicolumn{2}{p{0.3\linewidth}}{
        \cellcolor[rgb]{0.9,0.9,0.9}{
            \makecell[{{p{\linewidth}}}]{
                \texttt{\tiny{[GM$|$GM]}}
                \texttt{Hello! I see we have 3 crayons, 1 chalk, and 2 bolts to divide between us. I'm curious about what items you're most interested in. I have some preferences but I'd like to hear your thoughts first on what would work well for you.} \\
            }
        }
    }
    & & \\ \\

    \theutterance \stepcounter{utterance}  
    & & \multicolumn{4}{p{0.6\linewidth}}{
        \cellcolor[rgb]{0.9,0.9,0.9}{
            \makecell[{{p{\linewidth}}}]{
                \texttt{\tiny{[GM$\rangle$P2]}}
                \texttt{You are playing a negotiation game in which you have to agree on how to divide a set of items among you and another player.} \\
\\ 
\texttt{Rules:} \\
\texttt{(a) You and the other player are given a set of items. Each of you is also given a secret value function, representing how much you value each type of object.} \\
\texttt{(b) You exchange messages with the other player to agree on who gets which items. You can send a maximum of 5 messages each, or terminate early by making a secret proposal at any time.} \\
\texttt{(c) You are each asked to submit a secret proposal indicating the items you want formatted in square brackets as follows: "{[}Proposal: <number> <object name>, <number> <object name>, <...>{]}"} \\
\texttt{(d) If your proposals are complementary, i.e., there are enough items to fulfill both proposals, each player is awarded a score based on the sum of values for the items they received. Otherwise, both of you get zero points.} \\
\texttt{(e) Your goal is to maximize the score you receive.} \\
\\ 
\texttt{Let us start.} \\
\\ 
\texttt{The set of available items is:} \\
\\ 
\texttt{3 crayons, 1 chalk, 2 bolts} \\
\\ 
\texttt{Your secret value function is:} \\
\\ 
\texttt{crayon: 1, chalk: 3, bolt: 2} \\
\\ 
\texttt{IMPORTANT: Your messages, unless it is the secret proposal, are directly transmitted to the other player, so do not include any response to the rules or text announcing your message. To make a secret proposal, use the indicated format. Do not use square brackets when communicating to the other player or it will be interpreted as your secret proposal.} \\
\\ 
\\ 
\texttt{Hello! I see we have 3 crayons, 1 chalk, and 2 bolts to divide between us. I'm curious about what items you're most interested in. I have some preferences but I'd like to hear your thoughts first on what would work well for you.} \\
            }
        }
    }
    & \\ \\

    \theutterance \stepcounter{utterance}  
    & & & \multicolumn{4}{p{0.6\linewidth}}{
        \cellcolor[rgb]{0.9,0.9,0.9}{
            \makecell[{{p{\linewidth}}}]{
                \texttt{\tiny{[GM$\langle$P2]}}
                \texttt{I'm interested in making sure we both get items that are valuable to us! I'm particularly drawn to the chalk and would really like to have that if possible. I'm also quite interested in the bolts. How do you feel about the crayons? Are those something you'd be interested in, or do you have other preferences among the items?} \\
            }
        }
    }
     \\ \\

    \theutterance \stepcounter{utterance}  
    & & & \multicolumn{2}{p{0.3\linewidth}}{
        \cellcolor[rgb]{0.9,0.9,0.9}{
            \makecell[{{p{\linewidth}}}]{
                \texttt{\tiny{[GM$|$GM]}}
                \texttt{continue} \\
            }
        }
    }
    & & \\ \\

    \theutterance \stepcounter{utterance}  
    & & & \multicolumn{2}{p{0.3\linewidth}}{
        \cellcolor[rgb]{0.9,0.9,0.9}{
            \makecell[{{p{\linewidth}}}]{
                \texttt{\tiny{[GM$|$GM]}}
                \texttt{I'm interested in making sure we both get items that are valuable to us! I'm particularly drawn to the chalk and would really like to have that if possible. I'm also quite interested in the bolts. How do you feel about the crayons? Are those something you'd be interested in, or do you have other preferences among the items?} \\
            }
        }
    }
    & & \\ \\

    \theutterance \stepcounter{utterance}  
    & & \multicolumn{4}{p{0.6\linewidth}}{
        \cellcolor[rgb]{0.9,0.9,0.9}{
            \makecell[{{p{\linewidth}}}]{
                \texttt{\tiny{[P1$\langle$GM]}}
                \texttt{I'm interested in making sure we both get items that are valuable to us! I'm particularly drawn to the chalk and would really like to have that if possible. I'm also quite interested in the bolts. How do you feel about the crayons? Are those something you'd be interested in, or do you have other preferences among the items?} \\
            }
        }
    }
    & \\ \\

    \theutterance \stepcounter{utterance}  
    & \multicolumn{4}{p{0.6\linewidth}}{
        \cellcolor[rgb]{0.9,0.9,0.9}{
            \makecell[{{p{\linewidth}}}]{
                \texttt{\tiny{[P1$\rangle$GM]}}
                \texttt{That works out well! I'm actually not very interested in the crayons at all, so you're welcome to those. I do have a strong preference for the bolts though {-} they would be quite valuable to me. Since you mentioned being drawn to the chalk, what if you took the chalk and all the crayons? That would leave me with the 2 bolts. Would that division work for you?} \\
            }
        }
    }
    & & \\ \\

    \theutterance \stepcounter{utterance}  
    & & & \multicolumn{2}{p{0.3\linewidth}}{
        \cellcolor[rgb]{0.9,0.9,0.9}{
            \makecell[{{p{\linewidth}}}]{
                \texttt{\tiny{[GM$|$GM]}}
                \texttt{continue} \\
            }
        }
    }
    & & \\ \\

    \theutterance \stepcounter{utterance}  
    & & & \multicolumn{2}{p{0.3\linewidth}}{
        \cellcolor[rgb]{0.9,0.9,0.9}{
            \makecell[{{p{\linewidth}}}]{
                \texttt{\tiny{[GM$|$GM]}}
                \texttt{That works out well! I'm actually not very interested in the crayons at all, so you're welcome to those. I do have a strong preference for the bolts though {-} they would be quite valuable to me. Since you mentioned being drawn to the chalk, what if you took the chalk and all the crayons? That would leave me with the 2 bolts. Would that division work for you?} \\
            }
        }
    }
    & & \\ \\

    \theutterance \stepcounter{utterance}  
    & & \multicolumn{4}{p{0.6\linewidth}}{
        \cellcolor[rgb]{0.9,0.9,0.9}{
            \makecell[{{p{\linewidth}}}]{
                \texttt{\tiny{[GM$\rangle$P2]}}
                \texttt{That works out well! I'm actually not very interested in the crayons at all, so you're welcome to those. I do have a strong preference for the bolts though {-} they would be quite valuable to me. Since you mentioned being drawn to the chalk, what if you took the chalk and all the crayons? That would leave me with the 2 bolts. Would that division work for you?} \\
            }
        }
    }
    & \\ \\

    \theutterance \stepcounter{utterance}  
    & & & \multicolumn{4}{p{0.6\linewidth}}{
        \cellcolor[rgb]{0.9,0.9,0.9}{
            \makecell[{{p{\linewidth}}}]{
                \texttt{\tiny{[GM$\langle$P2]}}
                \texttt{That sounds like a great arrangement! You get the bolts which you said are valuable to you, and I get the chalk which I really wanted plus all the crayons. It seems like we both get items we prefer, so this feels like a fair division to me.} \\
\\ 
\texttt{{[}Proposal: 3 crayons, 1 chalk{]}} \\
            }
        }
    }
     \\ \\

    \theutterance \stepcounter{utterance}  
    & & & \multicolumn{2}{p{0.3\linewidth}}{
        \cellcolor[rgb]{0.9,0.9,0.9}{
            \makecell[{{p{\linewidth}}}]{
                \texttt{\tiny{[GM$|$GM]}}
                \texttt{proposal} \\
            }
        }
    }
    & & \\ \\

    \theutterance \stepcounter{utterance}  
    & & & \multicolumn{2}{p{0.3\linewidth}}{
        \cellcolor[rgb]{0.9,0.9,0.9}{
            \makecell[{{p{\linewidth}}}]{
                \texttt{\tiny{[GM$|$GM]}}
                [3, 1, 0]
            }
        }
    }
    & & \\ \\

    \theutterance \stepcounter{utterance}  
    & & \multicolumn{4}{p{0.6\linewidth}}{
        \cellcolor[rgb]{0.9,0.9,0.9}{
            \makecell[{{p{\linewidth}}}]{
                \texttt{\tiny{[P1$\langle$GM]}}
                \texttt{The other player has submitted a secret proposal. It is now your turn to submit a proposal of your own.} \\
            }
        }
    }
    & \\ \\

    \theutterance \stepcounter{utterance}  
    & \multicolumn{4}{p{0.6\linewidth}}{
        \cellcolor[rgb]{0.9,0.9,0.9}{
            \makecell[{{p{\linewidth}}}]{
                \texttt{\tiny{[P1$\rangle$GM]}}
                \texttt{{[}Proposal: 2 bolt{]}} \\
            }
        }
    }
    & & \\ \\

    \theutterance \stepcounter{utterance}  
    & & & \multicolumn{2}{p{0.3\linewidth}}{
        \cellcolor[rgb]{0.9,0.9,0.9}{
            \makecell[{{p{\linewidth}}}]{
                \texttt{\tiny{[GM$|$GM]}}
                \texttt{proposal} \\
            }
        }
    }
    & & \\ \\

    \theutterance \stepcounter{utterance}  
    & & & \multicolumn{2}{p{0.3\linewidth}}{
        \cellcolor[rgb]{0.9,0.9,0.9}{
            \makecell[{{p{\linewidth}}}]{
                \texttt{\tiny{[GM$|$GM]}}
                [0, 0, 2]
            }
        }
    }
    & & \\ \\

    \theutterance \stepcounter{utterance}  
    & & & \multicolumn{2}{p{0.3\linewidth}}{
        \cellcolor[rgb]{0.9,0.9,0.9}{
            \makecell[{{p{\linewidth}}}]{
                \texttt{\tiny{[GM$|$GM]}}
                [[0, 0, 2], [3, 1, 0]]
            }
        }
    }
    & & \\ \\

\end{supertabular}
}

\end{document}
