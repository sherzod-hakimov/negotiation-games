
\documentclass{article}
\usepackage{colortbl}
\usepackage{makecell}
\usepackage{multirow}
\usepackage{supertabular}

\begin{document}

\newcounter{utterance}

\centering \large Interaction Transcript for game `hot\_air\_balloon', experiment `air\_balloon\_survival\_en\_reasoning off\_hard', episode 1 with claude{-}sonnet{-}4{-}20250514{-}t0.0.
\vspace{24pt}

{ \footnotesize  \setcounter{utterance}{1}
\setlength{\tabcolsep}{0pt}
\begin{supertabular}{c@{$\;$}|p{.15\linewidth}@{}p{.15\linewidth}p{.15\linewidth}p{.15\linewidth}p{.15\linewidth}p{.15\linewidth}}
    \# & $\;$A & \multicolumn{4}{c}{Game Master} & $\;\:$B\\
    \hline

    \theutterance \stepcounter{utterance}  
    & & \multicolumn{4}{p{0.6\linewidth}}{
        \cellcolor[rgb]{0.9,0.9,0.9}{
            \makecell[{{p{\linewidth}}}]{
                \texttt{\tiny{[P1$\langle$GM]}}
                \texttt{You are participating in a collaborative negotiation game.} \\
\\ 
\texttt{Together with another participant, you must agree on a single set of items that will be kept. Each of you has your own view of how much each item matters to you (importance). You do not know how the other participant values the items. Additionally, you are given the effort each item demands.} \\
\texttt{You may only agree on a set if the total effort of the selected items does not exceed a shared limit:} \\
\\ 
\texttt{LIMIT = 3357} \\
\\ 
\texttt{Here are the individual item effort values:} \\
\\ 
\texttt{Item effort = \{"A21": 506, "B93": 492, "A11": 338, "B33": 122, "C08": 131, "B39": 909, "B96": 144, "C85": 716, "A82": 263, "B31": 231, "B20": 91, "A60": 651, "B09": 552, "C81": 849, "B21": 720\}} \\
\\ 
\texttt{Here is your personal view on the importance of each item:} \\
\\ 
\texttt{Item importance values = \{"A21": 138, "B93": 583, "A11": 868, "B33": 822, "C08": 783, "B39": 65, "B96": 262, "C85": 121, "A82": 508, "B31": 780, "B20": 461, "A60": 484, "B09": 668, "C81": 389, "B21": 808\}} \\
\\ 
\texttt{Goal:} \\
\\ 
\texttt{Your goal is to negotiate a shared set of items that benefits you as much as possible (i.e., maximizes total importance to YOU), while staying within the LIMIT. You are not required to make a PROPOSAL in every message {-} you can simply negotiate as well. All tactics are allowed!} \\
\\ 
\texttt{Interaction Protocol:} \\
\\ 
\texttt{You may only use the following structured formats in a message:} \\
\\ 
\texttt{PROPOSAL: \{'A', 'B', 'C', …\}} \\
\texttt{Propose keeping exactly those items.} \\
\texttt{REFUSE: \{'A', 'B', 'C', …\}} \\
\texttt{Explicitly reject opponent's proposal.} \\
\texttt{ARGUMENT: \{'...'\}} \\
\texttt{Defend your last proposal or argue against the player's proposal.} \\
\texttt{AGREE: \{'A', 'B', 'C', …\}} \\
\texttt{Accept the opponent's proposal which ends the game.} \\
\\ 
\\ 
\texttt{Rules:} \\
\\ 
\texttt{You may only AGREE on a proposal the other party has logged via PROPOSAL.} \\
\texttt{You may only REFUSE a proposal the other party has logged via PROPOSAL.} \\
\texttt{Total effort of any PROPOSAL or AGREE set must be ≤ LIMIT.} \\
\texttt{Do NOT reveal your hidden importance scores.} \\
\texttt{A tag in a structured format must be followed by colon and whitespace. The argument must be a python set containing 0 or more strings.} \\
\texttt{So, it must be of the form TAG: \{...\}} \\
\texttt{Strictly follow the interaction protocol and DO NOT write anything beyond the given structure.} \\
\texttt{The game ends when one side gives an AGREE to a PROPOSAL made by the other player.} \\
\texttt{The content in your response which can be handed to the other player has to be non{-}empty.} \\
\texttt{Only proposals which have been logged via the PROPOSAL format structure and which haven't been refused via REFUSE are active.} \\
\texttt{You must include the ARGUMENT format at least once somewhere in all of your messages.} \\
            }
        }
    }
    & \\ \\

    \theutterance \stepcounter{utterance}  
    & \multicolumn{4}{p{0.6\linewidth}}{
        \cellcolor[rgb]{0.9,0.9,0.9}{
            \makecell[{{p{\linewidth}}}]{
                \texttt{\tiny{[P1$\rangle$GM]}}
                \texttt{PROPOSAL: \{'B33', 'C08', 'B20', 'B31', 'A11', 'A82', 'B96', 'B09', 'B93', 'B21'\}} \\
\\ 
\texttt{ARGUMENT: \{'This selection prioritizes items that offer strong value relative to} \\
            }
        }
    }
    & & \\ \\

    \theutterance \stepcounter{utterance}  
    & & \multicolumn{4}{p{0.6\linewidth}}{
        \cellcolor[rgb]{0.9,0.9,0.9}{
            \makecell[{{p{\linewidth}}}]{
                \texttt{\tiny{[P1$\langle$GM]}}
                \texttt{Your response did not contain an argument. You must include the structured format ARGUMENT: \{...\} somewhere in your response. Try again} \\
            }
        }
    }
    & \\ \\

    \theutterance \stepcounter{utterance}  
    & \multicolumn{4}{p{0.6\linewidth}}{
        \cellcolor[rgb]{0.9,0.9,0.9}{
            \makecell[{{p{\linewidth}}}]{
                \texttt{\tiny{[P1$\rangle$GM]}}
                \texttt{PROPOSAL: \{'B33', 'C08', 'B20', 'B31', 'A11', 'A82', 'B96', 'B09', 'B93', 'B21'\}} \\
\\ 
\texttt{ARGUMENT: \{'This selection includes high{-}value items while staying well within our effort budget, leaving room for potential adjustments if needed'\}} \\
            }
        }
    }
    & & \\ \\

    \theutterance \stepcounter{utterance}  
    & & \multicolumn{4}{p{0.6\linewidth}}{
        \cellcolor[rgb]{0.9,0.9,0.9}{
            \makecell[{{p{\linewidth}}}]{
                \texttt{\tiny{[GM$\rangle$P2]}}
                \texttt{You are participating in a collaborative negotiation game.} \\
\\ 
\texttt{Together with another participant, you must agree on a single set of items that will be kept. Each of you has your own view of how much each item matters to you (importance). You do not know how the other participant values the items. Additionally, you are given the effort each item demands.} \\
\texttt{You may only agree on a set if the total effort of the selected items does not exceed a shared limit:} \\
\\ 
\texttt{LIMIT = 3357} \\
\\ 
\texttt{Here are the individual item effort values:} \\
\\ 
\texttt{Item effort = \{"A21": 506, "B93": 492, "A11": 338, "B33": 122, "C08": 131, "B39": 909, "B96": 144, "C85": 716, "A82": 263, "B31": 231, "B20": 91, "A60": 651, "B09": 552, "C81": 849, "B21": 720\}} \\
\\ 
\texttt{Here is your personal view on the importance of each item:} \\
\\ 
\texttt{Item importance values = \{"A11": 65, "B33": 121, "B21": 138, "C08": 262, "B31": 389, "B09": 461, "B93": 484, "A82": 508, "A60": 583, "B20": 668, "C81": 780, "B96": 783, "A21": 808, "C85": 822, "B39": 868\}} \\
\\ 
\texttt{Goal:} \\
\\ 
\texttt{Your goal is to negotiate a shared set of items that benefits you as much as possible (i.e., maximizes total importance to YOU), while staying within the LIMIT. You are not required to make a PROPOSAL in every message {-} you can simply negotiate as well. All tactics are allowed!} \\
\\ 
\texttt{Interaction Protocol:} \\
\\ 
\texttt{You may only use the following structured formats in a message:} \\
\\ 
\texttt{PROPOSAL: \{'A', 'B', 'C', …\}} \\
\texttt{Propose keeping exactly those items.} \\
\texttt{REFUSE: \{'A', 'B', 'C', …\}} \\
\texttt{Explicitly reject opponent's proposal.} \\
\texttt{ARGUMENT: \{'...'\}} \\
\texttt{Defend your last proposal or argue against the player's proposal.} \\
\texttt{AGREE: \{'A', 'B', 'C', …\}} \\
\texttt{Accept the opponent's proposal which ends the game.} \\
\\ 
\\ 
\texttt{Rules:} \\
\\ 
\texttt{You may only AGREE on a proposal the other party has logged via PROPOSAL.} \\
\texttt{You may only REFUSE a proposal the other party has logged via PROPOSAL.} \\
\texttt{Total effort of any PROPOSAL or AGREE set must be ≤ LIMIT.} \\
\texttt{Do NOT reveal your hidden importance scores.} \\
\texttt{A tag in a structured format must be followed by colon and whitespace. The argument must be a python set containing 0 or more strings.} \\
\texttt{So, it must be of the form TAG: \{...\}} \\
\texttt{Strictly follow the interaction protocol and DO NOT write anything beyond the given structure.} \\
\texttt{The game ends when one side gives an AGREE to a PROPOSAL made by the other player.} \\
\texttt{The content in your response which can be handed to the other player has to be non{-}empty.} \\
\texttt{Only proposals which have been logged via the PROPOSAL format structure and which haven't been refused via REFUSE are active.} \\
\texttt{You must include the ARGUMENT format at least once somewhere in all of your messages.} \\
\\ 
\\ 
\texttt{You will now receive the first message of the other player.} \\
\\ 
\texttt{PROPOSAL: \{'B33', 'C08', 'B20', 'B31', 'A11', 'A82', 'B96', 'B09', 'B93', 'B21'\}} \\
\\ 
\texttt{ARGUMENT: \{'This selection includes high{-}value items while staying well within our effort budget, leaving room for potential adjustments if needed'\}} \\
            }
        }
    }
    & \\ \\

    \theutterance \stepcounter{utterance}  
    & & & \multicolumn{4}{p{0.6\linewidth}}{
        \cellcolor[rgb]{0.9,0.9,0.9}{
            \makecell[{{p{\linewidth}}}]{
                \texttt{\tiny{[GM$\langle$P2]}}
                \texttt{REFUSE: \{'B33', 'C08', 'B20', 'B31', 'A11', 'A82', 'B96', 'B09', 'B93', 'B21'\}} \\
\\ 
\texttt{PROPOSAL: \{'B39', 'C85', 'A21', 'B96', 'B20', 'A82', 'A60'\}} \\
\\ 
\texttt{ARGUMENT: \{'My proposal includes several high{-}impact items that provide excellent value while maintaining efficient use of our effort budget. This combination achieves better overall optimization than your initial selection.'\}} \\
            }
        }
    }
     \\ \\

    \theutterance \stepcounter{utterance}  
    & & \multicolumn{4}{p{0.6\linewidth}}{
        \cellcolor[rgb]{0.9,0.9,0.9}{
            \makecell[{{p{\linewidth}}}]{
                \texttt{\tiny{[P1$\langle$GM]}}
                \texttt{REFUSE: \{'B33', 'C08', 'B20', 'B31', 'A11', 'A82', 'B96', 'B09', 'B93', 'B21'\}} \\
\\ 
\texttt{PROPOSAL: \{'B39', 'C85', 'A21', 'B96', 'B20', 'A82', 'A60'\}} \\
\\ 
\texttt{ARGUMENT: \{'My proposal includes several high{-}impact items that provide excellent value while maintaining efficient use of our effort budget. This combination achieves better overall optimization than your initial selection.'\}} \\
            }
        }
    }
    & \\ \\

    \theutterance \stepcounter{utterance}  
    & \multicolumn{4}{p{0.6\linewidth}}{
        \cellcolor[rgb]{0.9,0.9,0.9}{
            \makecell[{{p{\linewidth}}}]{
                \texttt{\tiny{[P1$\rangle$GM]}}
                \texttt{REFUSE: \{'B39', 'C85', 'A21', 'B96', 'B20', 'A82', 'A60'\}} \\
\\ 
\texttt{PROPOSAL: \{'B33', 'C08', 'B20', 'B31', 'A11', 'A82', 'B96', 'B93'\}} \\
\\ 
\texttt{ARGUMENT: \{'Your proposal includes several items that seem inefficient given their high effort costs relative to potential benefits. My revised proposal removes some lower{-}priority items to create a more balanced selection that maximizes our collective advantage while using our effort budget more strategically'\}} \\
            }
        }
    }
    & & \\ \\

    \theutterance \stepcounter{utterance}  
    & & \multicolumn{4}{p{0.6\linewidth}}{
        \cellcolor[rgb]{0.9,0.9,0.9}{
            \makecell[{{p{\linewidth}}}]{
                \texttt{\tiny{[GM$\rangle$P2]}}
                \texttt{REFUSE: \{'B39', 'C85', 'A21', 'B96', 'B20', 'A82', 'A60'\}} \\
\\ 
\texttt{PROPOSAL: \{'B33', 'C08', 'B20', 'B31', 'A11', 'A82', 'B96', 'B93'\}} \\
\\ 
\texttt{ARGUMENT: \{'Your proposal includes several items that seem inefficient given their high effort costs relative to potential benefits. My revised proposal removes some lower{-}priority items to create a more balanced selection that maximizes our collective advantage while using our effort budget more strategically'\}} \\
            }
        }
    }
    & \\ \\

    \theutterance \stepcounter{utterance}  
    & & & \multicolumn{4}{p{0.6\linewidth}}{
        \cellcolor[rgb]{0.9,0.9,0.9}{
            \makecell[{{p{\linewidth}}}]{
                \texttt{\tiny{[GM$\langle$P2]}}
                \texttt{REFUSE: \{'B33', 'C08', 'B20', 'B31', 'A11', 'A82', 'B96', 'B93'\}} \\
\\ 
\texttt{PROPOSAL: \{'C85', 'A21', 'B96', 'B20', 'A82', 'B31', 'C08'\}} \\
\\ 
\texttt{ARGUMENT: \{'Your approach removes too many high{-}value opportunities. My counter{-}proposal keeps some of your preferred items while incorporating elements that provide superior returns on our effort investment. This balanced approach should satisfy both our priorities more effectively.'\}} \\
            }
        }
    }
     \\ \\

    \theutterance \stepcounter{utterance}  
    & & \multicolumn{4}{p{0.6\linewidth}}{
        \cellcolor[rgb]{0.9,0.9,0.9}{
            \makecell[{{p{\linewidth}}}]{
                \texttt{\tiny{[P1$\langle$GM]}}
                \texttt{REFUSE: \{'B33', 'C08', 'B20', 'B31', 'A11', 'A82', 'B96', 'B93'\}} \\
\\ 
\texttt{PROPOSAL: \{'C85', 'A21', 'B96', 'B20', 'A82', 'B31', 'C08'\}} \\
\\ 
\texttt{ARGUMENT: \{'Your approach removes too many high{-}value opportunities. My counter{-}proposal keeps some of your preferred items while incorporating elements that provide superior returns on our effort investment. This balanced approach should satisfy both our priorities more effectively.'\}} \\
            }
        }
    }
    & \\ \\

    \theutterance \stepcounter{utterance}  
    & \multicolumn{4}{p{0.6\linewidth}}{
        \cellcolor[rgb]{0.9,0.9,0.9}{
            \makecell[{{p{\linewidth}}}]{
                \texttt{\tiny{[P1$\rangle$GM]}}
                \texttt{REFUSE: \{'C85', 'A21', 'B96', 'B20', 'A82', 'B31', 'C08'\}} \\
\\ 
\texttt{PROPOSAL: \{'B33', 'C08', 'B31', 'A11', 'B93', 'B21', 'B20'\}} \\
\\ 
\texttt{ARGUMENT: \{'Your proposal excludes some critical high{-}performing items that we should prioritize. My counter{-}proposal focuses on the most valuable opportunities while still maintaining reasonable effort expenditure. This selection represents a more strategic approach to maximizing our combined benefits within the constraints'\}} \\
            }
        }
    }
    & & \\ \\

    \theutterance \stepcounter{utterance}  
    & & \multicolumn{4}{p{0.6\linewidth}}{
        \cellcolor[rgb]{0.9,0.9,0.9}{
            \makecell[{{p{\linewidth}}}]{
                \texttt{\tiny{[GM$\rangle$P2]}}
                \texttt{REFUSE: \{'C85', 'A21', 'B96', 'B20', 'A82', 'B31', 'C08'\}} \\
\\ 
\texttt{PROPOSAL: \{'B33', 'C08', 'B31', 'A11', 'B93', 'B21', 'B20'\}} \\
\\ 
\texttt{ARGUMENT: \{'Your proposal excludes some critical high{-}performing items that we should prioritize. My counter{-}proposal focuses on the most valuable opportunities while still maintaining reasonable effort expenditure. This selection represents a more strategic approach to maximizing our combined benefits within the constraints'\}} \\
            }
        }
    }
    & \\ \\

    \theutterance \stepcounter{utterance}  
    & & & \multicolumn{4}{p{0.6\linewidth}}{
        \cellcolor[rgb]{0.9,0.9,0.9}{
            \makecell[{{p{\linewidth}}}]{
                \texttt{\tiny{[GM$\langle$P2]}}
                \texttt{REFUSE: \{'B33', 'C08', 'B31', 'A11', 'B93', 'B21', 'B20'\}} \\
\\ 
\texttt{PROPOSAL: \{'C85', 'A21', 'B96', 'B20', 'B31', 'C08', 'A82'\}} \\
\\ 
\texttt{ARGUMENT: \{'Your latest proposal significantly underutilizes our effort capacity and misses key high{-}value items. My refined proposal maintains the core valuable elements while ensuring we make full strategic use of our available effort budget. This approach delivers substantially better results for both parties.'\}} \\
            }
        }
    }
     \\ \\

    \theutterance \stepcounter{utterance}  
    & & \multicolumn{4}{p{0.6\linewidth}}{
        \cellcolor[rgb]{0.9,0.9,0.9}{
            \makecell[{{p{\linewidth}}}]{
                \texttt{\tiny{[P1$\langle$GM]}}
                \texttt{REFUSE: \{'B33', 'C08', 'B31', 'A11', 'B93', 'B21', 'B20'\}} \\
\\ 
\texttt{PROPOSAL: \{'C85', 'A21', 'B96', 'B20', 'B31', 'C08', 'A82'\}} \\
\\ 
\texttt{ARGUMENT: \{'Your latest proposal significantly underutilizes our effort capacity and misses key high{-}value items. My refined proposal maintains the core valuable elements while ensuring we make full strategic use of our available effort budget. This approach delivers substantially better results for both parties.'\}} \\
            }
        }
    }
    & \\ \\

    \theutterance \stepcounter{utterance}  
    & \multicolumn{4}{p{0.6\linewidth}}{
        \cellcolor[rgb]{0.9,0.9,0.9}{
            \makecell[{{p{\linewidth}}}]{
                \texttt{\tiny{[P1$\rangle$GM]}}
                \texttt{REFUSE: \{'C85', 'A21', 'B96', 'B20', 'B31', 'C08', 'A82'\}} \\
\\ 
\texttt{PROPOSAL: \{'B33', 'C08', 'B31', 'A11', 'B21', 'B09', 'B20'\}} \\
\\ 
\texttt{ARGUMENT: \{'Your proposal includes several items with poor effort{-}to{-}value ratios that waste our limited budget. My counter{-}proposal eliminates these inefficiencies and focuses on a core set of genuinely valuable items that provide much better returns. This selection uses our effort more wisely while delivering superior overall results'\}} \\
            }
        }
    }
    & & \\ \\

    \theutterance \stepcounter{utterance}  
    & & \multicolumn{4}{p{0.6\linewidth}}{
        \cellcolor[rgb]{0.9,0.9,0.9}{
            \makecell[{{p{\linewidth}}}]{
                \texttt{\tiny{[GM$\rangle$P2]}}
                \texttt{REFUSE: \{'C85', 'A21', 'B96', 'B20', 'B31', 'C08', 'A82'\}} \\
\\ 
\texttt{PROPOSAL: \{'B33', 'C08', 'B31', 'A11', 'B21', 'B09', 'B20'\}} \\
\\ 
\texttt{ARGUMENT: \{'Your proposal includes several items with poor effort{-}to{-}value ratios that waste our limited budget. My counter{-}proposal eliminates these inefficiencies and focuses on a core set of genuinely valuable items that provide much better returns. This selection uses our effort more wisely while delivering superior overall results'\}} \\
            }
        }
    }
    & \\ \\

    \theutterance \stepcounter{utterance}  
    & & & \multicolumn{4}{p{0.6\linewidth}}{
        \cellcolor[rgb]{0.9,0.9,0.9}{
            \makecell[{{p{\linewidth}}}]{
                \texttt{\tiny{[GM$\langle$P2]}}
                \texttt{REFUSE: \{'B33', 'C08', 'B31', 'A11', 'B21', 'B09', 'B20'\}} \\
\\ 
\texttt{PROPOSAL: \{'C85', 'A21', 'C81', 'B96', 'B20', 'C08'\}} \\
\\ 
\texttt{ARGUMENT: \{'Your proposal continues to exclude the most valuable items in favor of lower{-}priority selections. My counter{-}proposal strategically combines the highest{-}value opportunities while staying within our effort constraints. This approach maximizes our collective benefit and represents a much more efficient allocation of our limited resources.'\}} \\
            }
        }
    }
     \\ \\

    \theutterance \stepcounter{utterance}  
    & & \multicolumn{4}{p{0.6\linewidth}}{
        \cellcolor[rgb]{0.9,0.9,0.9}{
            \makecell[{{p{\linewidth}}}]{
                \texttt{\tiny{[P1$\langle$GM]}}
                \texttt{REFUSE: \{'B33', 'C08', 'B31', 'A11', 'B21', 'B09', 'B20'\}} \\
\\ 
\texttt{PROPOSAL: \{'C85', 'A21', 'C81', 'B96', 'B20', 'C08'\}} \\
\\ 
\texttt{ARGUMENT: \{'Your proposal continues to exclude the most valuable items in favor of lower{-}priority selections. My counter{-}proposal strategically combines the highest{-}value opportunities while staying within our effort constraints. This approach maximizes our collective benefit and represents a much more efficient allocation of our limited resources.'\}} \\
            }
        }
    }
    & \\ \\

    \theutterance \stepcounter{utterance}  
    & \multicolumn{4}{p{0.6\linewidth}}{
        \cellcolor[rgb]{0.9,0.9,0.9}{
            \makecell[{{p{\linewidth}}}]{
                \texttt{\tiny{[P1$\rangle$GM]}}
                \texttt{REFUSE: \{'C85', 'A21', 'C81', 'B96', 'B20', 'C08'\}} \\
\\ 
\texttt{PROPOSAL: \{'B33', 'C08', 'B31', 'A11', 'B21', 'B20', 'A82'\}} \\
\\ 
\texttt{ARGUMENT: \{'Your proposal focuses too heavily on items that may not provide the best overall value for our collaboration. My counter{-}proposal includes a more diverse mix of high{-}performing items that should better serve both our interests while maintaining efficient effort utilization and staying comfortably within our budget constraints'\}} \\
            }
        }
    }
    & & \\ \\

    \theutterance \stepcounter{utterance}  
    & & \multicolumn{4}{p{0.6\linewidth}}{
        \cellcolor[rgb]{0.9,0.9,0.9}{
            \makecell[{{p{\linewidth}}}]{
                \texttt{\tiny{[GM$\rangle$P2]}}
                \texttt{REFUSE: \{'C85', 'A21', 'C81', 'B96', 'B20', 'C08'\}} \\
\\ 
\texttt{PROPOSAL: \{'B33', 'C08', 'B31', 'A11', 'B21', 'B20', 'A82'\}} \\
\\ 
\texttt{ARGUMENT: \{'Your proposal focuses too heavily on items that may not provide the best overall value for our collaboration. My counter{-}proposal includes a more diverse mix of high{-}performing items that should better serve both our interests while maintaining efficient effort utilization and staying comfortably within our budget constraints'\}} \\
            }
        }
    }
    & \\ \\

    \theutterance \stepcounter{utterance}  
    & & & \multicolumn{4}{p{0.6\linewidth}}{
        \cellcolor[rgb]{0.9,0.9,0.9}{
            \makecell[{{p{\linewidth}}}]{
                \texttt{\tiny{[GM$\langle$P2]}}
                \texttt{REFUSE: \{'B33', 'C08', 'B31', 'A11', 'B21', 'B20', 'A82'\}} \\
\\ 
\texttt{PROPOSAL: \{'C85', 'A21', 'B96', 'B20', 'C08', 'B31'\}} \\
\\ 
\texttt{ARGUMENT: \{'Your proposal consistently avoids the most valuable items that could significantly benefit our collaboration. My counter{-}proposal incorporates key high{-}value elements while keeping some items you seem to prefer. This represents a fair compromise that maximizes our collective success within the effort constraints.'\}} \\
            }
        }
    }
     \\ \\

    \theutterance \stepcounter{utterance}  
    & & \multicolumn{4}{p{0.6\linewidth}}{
        \cellcolor[rgb]{0.9,0.9,0.9}{
            \makecell[{{p{\linewidth}}}]{
                \texttt{\tiny{[P1$\langle$GM]}}
                \texttt{REFUSE: \{'B33', 'C08', 'B31', 'A11', 'B21', 'B20', 'A82'\}} \\
\\ 
\texttt{PROPOSAL: \{'C85', 'A21', 'B96', 'B20', 'C08', 'B31'\}} \\
\\ 
\texttt{ARGUMENT: \{'Your proposal consistently avoids the most valuable items that could significantly benefit our collaboration. My counter{-}proposal incorporates key high{-}value elements while keeping some items you seem to prefer. This represents a fair compromise that maximizes our collective success within the effort constraints.'\}} \\
            }
        }
    }
    & \\ \\

    \theutterance \stepcounter{utterance}  
    & \multicolumn{4}{p{0.6\linewidth}}{
        \cellcolor[rgb]{0.9,0.9,0.9}{
            \makecell[{{p{\linewidth}}}]{
                \texttt{\tiny{[P1$\rangle$GM]}}
                \texttt{AGREE: \{'C85', 'A21', 'B96', 'B20', 'C08', 'B31'\}} \\
\\ 
\texttt{ARGUMENT: \{'After careful consideration, this proposal achieves a good balance between our different priorities while making efficient use of our effort budget. The combination provides solid value and represents a fair compromise that both parties can benefit from'\}} \\
            }
        }
    }
    & & \\ \\

    \theutterance \stepcounter{utterance}  
    & & & \multicolumn{2}{p{0.3\linewidth}}{
        \cellcolor[rgb]{0.9,0.9,0.9}{
            \makecell[{{p{\linewidth}}}]{
                \texttt{\tiny{[GM$|$GM]}}
                \texttt{game successful} \\
            }
        }
    }
    & & \\ \\

    \theutterance \stepcounter{utterance}  
    & & & \multicolumn{2}{p{0.3\linewidth}}{
        \cellcolor[rgb]{0.9,0.9,0.9}{
            \makecell[{{p{\linewidth}}}]{
                \texttt{\tiny{[GM$|$GM]}}
                \texttt{end game} \\
            }
        }
    }
    & & \\ \\

\end{supertabular}
}

\end{document}
