
\documentclass{article}
\usepackage{colortbl}
\usepackage{makecell}
\usepackage{multirow}
\usepackage{supertabular}

\begin{document}

\newcounter{utterance}

\twocolumn

{ \footnotesize  \setcounter{utterance}{1}
\setlength{\tabcolsep}{0pt}
\begin{supertabular}{c@{$\;$}|p{.15\linewidth}@{}p{.15\linewidth}p{.15\linewidth}p{.15\linewidth}p{.15\linewidth}p{.15\linewidth}}

    \# & $\;$A & \multicolumn{4}{c}{Game Master} & $\;\:$B\\
    \hline 

    \theutterance \stepcounter{utterance}  

    & & \multicolumn{4}{p{0.6\linewidth}}{\cellcolor[rgb]{0.9,0.9,0.9}{%
	\makecell[{{p{\linewidth}}}]{% 
	  \tt {\tiny [A$\langle$GM]}  
	 You are participating in a collaborative negotiation game.\\ \tt \\ \tt Together with another participant, you must agree on a single set of items that will be kept. Each of you has your own view of how much each item matters to you (importance). You do not know how the other participant values the items. Additionally, you are given the effort each item demands.\\ \tt You may only agree on a set if the total effort of the selected items does not exceed a shared limit:\\ \tt \\ \tt LIMIT = 3683\\ \tt \\ \tt Here are the individual item effort values:\\ \tt \\ \tt Item effort = {"A21": 548, "B93": 154, "A11": 60, "B33": 517, "C08": 334, "B39": 542, "B96": 707, "C85": 139, "A82": 661, "B31": 780, "B20": 832, "A60": 913, "B09": 219, "C81": 323, "B21": 638}\\ \tt \\ \tt Here is your personal view on the importance of each item:\\ \tt \\ \tt Item importance values = {"A21": 138, "B93": 583, "A11": 868, "B33": 822, "C08": 783, "B39": 65, "B96": 262, "C85": 121, "A82": 508, "B31": 780, "B20": 461, "A60": 484, "B09": 668, "C81": 389, "B21": 808}\\ \tt \\ \tt Goal:\\ \tt \\ \tt Your goal is to negotiate a shared set of items that benefits you as much as possible (i.e., maximizes total importance to YOU), while staying within the LIMIT. You are not required to make a PROPOSAL in every message - you can simply negotiate as well. All tactics are allowed!\\ \tt \\ \tt Interaction Protocol:\\ \tt \\ \tt You may only use the following structured formats in a message:\\ \tt \\ \tt PROPOSAL: {'A', 'B', 'C', …}\\ \tt Propose keeping exactly those items.\\ \tt REFUSE: {'A', 'B', 'C', …}\\ \tt Explicitly reject opponent's proposal.\\ \tt ARGUMENT: {'...'}\\ \tt Defend your last proposal or argue against the player's proposal.\\ \tt AGREE: {'A', 'B', 'C', …}\\ \tt Accept the opponent's proposal which ends the game.\\ \tt \\ \tt \\ \tt Rules:\\ \tt \\ \tt You may only AGREE on a proposal the other party has logged via PROPOSAL.\\ \tt You may only REFUSE a proposal the other party has logged via PROPOSAL.\\ \tt Total effort of any PROPOSAL or AGREE set must be ≤ LIMIT.\\ \tt Do NOT reveal your hidden importance scores.\\ \tt A tag in a structured format must be followed by colon and whitespace. The argument must be a python set containing 0 or more strings.\\ \tt So, it must be of the form TAG: {...}\\ \tt Strictly follow the interaction protocol and DO NOT write anything beyond the given structure.\\ \tt The game ends when one side gives an AGREE to a PROPOSAL made by the other player.\\ \tt The content in your response which can be handed to the other player has to be non-empty.\\ \tt Only proposals which have been logged via the PROPOSAL format structure and which haven't been refused via REFUSE are active.\\ \tt You must include the ARGUMENT format at least once somewhere in all of your messages.\\ \tt  
	  } 
	   } 
	   } 
	 & \\ 
 

    \theutterance \stepcounter{utterance}  

    & \multicolumn{4}{p{0.6\linewidth}}{\cellcolor[rgb]{0.8,1,0.9}{%
	\makecell[{{p{\linewidth}}}]{% 
	  \tt {\tiny [A$\rangle$GM]}  
	 PROPOSAL: {'A11', 'B93', 'B09', 'C08', 'B33', 'B21', 'C81', 'B31', 'C85'}\\ \tt \\ \tt ARGUMENT: {'This selection includes items with strong efficiency ratios, balancing high value with reasonable effort costs while staying well within our shared limit'} 
	  } 
	   } 
	   } 
	 & & \\ 
 

    \theutterance \stepcounter{utterance}  

    & & \multicolumn{4}{p{0.6\linewidth}}{\cellcolor[rgb]{0.9,0.9,0.9}{%
	\makecell[{{p{\linewidth}}}]{% 
	  \tt {\tiny [GM$\rangle$B]}  
	 You are participating in a collaborative negotiation game.\\ \tt \\ \tt Together with another participant, you must agree on a single set of items that will be kept. Each of you has your own view of how much each item matters to you (importance). You do not know how the other participant values the items. Additionally, you are given the effort each item demands.\\ \tt You may only agree on a set if the total effort of the selected items does not exceed a shared limit:\\ \tt \\ \tt LIMIT = 3683\\ \tt \\ \tt Here are the individual item effort values:\\ \tt \\ \tt Item effort = {"A21": 548, "B93": 154, "A11": 60, "B33": 517, "C08": 334, "B39": 542, "B96": 707, "C85": 139, "A82": 661, "B31": 780, "B20": 832, "A60": 913, "B09": 219, "C81": 323, "B21": 638}\\ \tt \\ \tt Here is your personal view on the importance of each item:\\ \tt \\ \tt Item importance values = {"A11": 65, "B33": 121, "B21": 138, "C08": 262, "B31": 389, "B09": 461, "B93": 484, "A82": 508, "A60": 583, "B20": 668, "C81": 780, "B96": 783, "A21": 808, "C85": 822, "B39": 868}\\ \tt \\ \tt Goal:\\ \tt \\ \tt Your goal is to negotiate a shared set of items that benefits you as much as possible (i.e., maximizes total importance to YOU), while staying within the LIMIT. You are not required to make a PROPOSAL in every message - you can simply negotiate as well. All tactics are allowed!\\ \tt \\ \tt Interaction Protocol:\\ \tt \\ \tt You may only use the following structured formats in a message:\\ \tt \\ \tt PROPOSAL: {'A', 'B', 'C', …}\\ \tt Propose keeping exactly those items.\\ \tt REFUSE: {'A', 'B', 'C', …}\\ \tt Explicitly reject opponent's proposal.\\ \tt ARGUMENT: {'...'}\\ \tt Defend your last proposal or argue against the player's proposal.\\ \tt AGREE: {'A', 'B', 'C', …}\\ \tt Accept the opponent's proposal which ends the game.\\ \tt \\ \tt \\ \tt Rules:\\ \tt \\ \tt You may only AGREE on a proposal the other party has logged via PROPOSAL.\\ \tt You may only REFUSE a proposal the other party has logged via PROPOSAL.\\ \tt Total effort of any PROPOSAL or AGREE set must be ≤ LIMIT.\\ \tt Do NOT reveal your hidden importance scores.\\ \tt A tag in a structured format must be followed by colon and whitespace. The argument must be a python set containing 0 or more strings.\\ \tt So, it must be of the form TAG: {...}\\ \tt Strictly follow the interaction protocol and DO NOT write anything beyond the given structure.\\ \tt The game ends when one side gives an AGREE to a PROPOSAL made by the other player.\\ \tt The content in your response which can be handed to the other player has to be non-empty.\\ \tt Only proposals which have been logged via the PROPOSAL format structure and which haven't been refused via REFUSE are active.\\ \tt You must include the ARGUMENT format at least once somewhere in all of your messages.\\ \tt \\ \tt \\ \tt You will now receive the first message of the other player.\\ \tt \\ \tt PROPOSAL: {'A11', 'B93', 'B09', 'C08', 'B33', 'B21', 'C81', 'B31', 'C85'}\\ \tt \\ \tt ARGUMENT: {'This selection includes items with strong efficiency ratios, balancing high value with reasonable effort costs while staying well within our shared limit'} 
	  } 
	   } 
	   } 
	 & \\ 
 

    \theutterance \stepcounter{utterance}  

    & & & \multicolumn{4}{p{0.6\linewidth}}{\cellcolor[rgb]{1,0.85,0.72}{%
	\makecell[{{p{\linewidth}}}]{% 
	  \tt {\tiny [GM$\langle$B]}  
	 REFUSE: {'A11', 'B93', 'B09', 'C08', 'B33', 'B21', 'C81', 'B31', 'C85'}\\ \tt \\ \tt PROPOSAL: {'B93', 'B09', 'C81', 'B31', 'C85', 'A21', 'B39'}\\ \tt \\ \tt ARGUMENT: {'I appreciate your efficiency focus, but we can achieve much better overall value by replacing some lower-performing items. My proposal removes the least efficient items from your selection and adds high-value alternatives while maintaining substantial buffer room within our limit'} 
	  } 
	   } 
	   } 
	  \\ 
 

    \theutterance \stepcounter{utterance}  

    & & \multicolumn{4}{p{0.6\linewidth}}{\cellcolor[rgb]{0.9,0.9,0.9}{%
	\makecell[{{p{\linewidth}}}]{% 
	  \tt {\tiny [A$\langle$GM]}  
	 REFUSE: {'A11', 'B93', 'B09', 'C08', 'B33', 'B21', 'C81', 'B31', 'C85'}\\ \tt \\ \tt PROPOSAL: {'B93', 'B09', 'C81', 'B31', 'C85', 'A21', 'B39'}\\ \tt \\ \tt ARGUMENT: {'I appreciate your efficiency focus, but we can achieve much better overall value by replacing some lower-performing items. My proposal removes the least efficient items from your selection and adds high-value alternatives while maintaining substantial buffer room within our limit'} 
	  } 
	   } 
	   } 
	 & \\ 
 

    \theutterance \stepcounter{utterance}  

    & \multicolumn{4}{p{0.6\linewidth}}{\cellcolor[rgb]{0.8,1,0.9}{%
	\makecell[{{p{\linewidth}}}]{% 
	  \tt {\tiny [A$\rangle$GM]}  
	 REFUSE: {'B93', 'B09', 'C81', 'B31', 'C85', 'A21', 'B39'}\\ \tt \\ \tt PROPOSAL: {'A11', 'B93', 'B09', 'C08', 'B33', 'B21', 'C81', 'C85'}\\ \tt \\ \tt ARGUMENT: {'Your proposal removes several core high-performing items that are essential for maximizing our collective benefit. I have made a reasonable concession by removing B31 to create more room for negotiation, while maintaining the fundamental structure that ensures we get optimal value from our selection'} 
	  } 
	   } 
	   } 
	 & & \\ 
 

    \theutterance \stepcounter{utterance}  

    & & \multicolumn{4}{p{0.6\linewidth}}{\cellcolor[rgb]{0.9,0.9,0.9}{%
	\makecell[{{p{\linewidth}}}]{% 
	  \tt {\tiny [GM$\rangle$B]}  
	 REFUSE: {'B93', 'B09', 'C81', 'B31', 'C85', 'A21', 'B39'}\\ \tt \\ \tt PROPOSAL: {'A11', 'B93', 'B09', 'C08', 'B33', 'B21', 'C81', 'C85'}\\ \tt \\ \tt ARGUMENT: {'Your proposal removes several core high-performing items that are essential for maximizing our collective benefit. I have made a reasonable concession by removing B31 to create more room for negotiation, while maintaining the fundamental structure that ensures we get optimal value from our selection'} 
	  } 
	   } 
	   } 
	 & \\ 
 

    \theutterance \stepcounter{utterance}  

    & & & \multicolumn{4}{p{0.6\linewidth}}{\cellcolor[rgb]{1,0.85,0.72}{%
	\makecell[{{p{\linewidth}}}]{% 
	  \tt {\tiny [GM$\langle$B]}  
	 REFUSE: {'A11', 'B93', 'B09', 'C08', 'B33', 'B21', 'C81', 'C85'}\\ \tt \\ \tt PROPOSAL: {'B93', 'B09', 'C81', 'C85', 'A21', 'B96'}\\ \tt \\ \tt ARGUMENT: {'Your concession on B31 shows good faith, but we still need to optimize our selection better. My revised proposal keeps several items from your preference set while adding critical high-performance items that will significantly boost our overall return. This combination uses our effort budget more strategically while ensuring both parties benefit from the strongest available options'} 
	  } 
	   } 
	   } 
	  \\ 
 

    \theutterance \stepcounter{utterance}  

    & & \multicolumn{4}{p{0.6\linewidth}}{\cellcolor[rgb]{0.9,0.9,0.9}{%
	\makecell[{{p{\linewidth}}}]{% 
	  \tt {\tiny [A$\langle$GM]}  
	 REFUSE: {'A11', 'B93', 'B09', 'C08', 'B33', 'B21', 'C81', 'C85'}\\ \tt \\ \tt PROPOSAL: {'B93', 'B09', 'C81', 'C85', 'A21', 'B96'}\\ \tt \\ \tt ARGUMENT: {'Your concession on B31 shows good faith, but we still need to optimize our selection better. My revised proposal keeps several items from your preference set while adding critical high-performance items that will significantly boost our overall return. This combination uses our effort budget more strategically while ensuring both parties benefit from the strongest available options'} 
	  } 
	   } 
	   } 
	 & \\ 
 

    \theutterance \stepcounter{utterance}  

    & \multicolumn{4}{p{0.6\linewidth}}{\cellcolor[rgb]{0.8,1,0.9}{%
	\makecell[{{p{\linewidth}}}]{% 
	  \tt {\tiny [A$\rangle$GM]}  
	 REFUSE: {'B93', 'B09', 'C81', 'C85', 'A21', 'B96'}\\ \tt \\ \tt PROPOSAL: {'A11', 'B93', 'B09', 'C08', 'B33', 'B21', 'C85'}\\ \tt \\ \tt ARGUMENT: {'I appreciate you keeping B93 and B09 which we both seem to value, but A21 and B96 create inefficiencies in our selection. My counter-proposal incorporates the items you favor while adding back the most critical high-performing elements that will maximize our joint success. This combination uses our effort budget much more effectively and ensures we both get substantial value from the final set'} 
	  } 
	   } 
	   } 
	 & & \\ 
 

    \theutterance \stepcounter{utterance}  

    & & \multicolumn{4}{p{0.6\linewidth}}{\cellcolor[rgb]{0.9,0.9,0.9}{%
	\makecell[{{p{\linewidth}}}]{% 
	  \tt {\tiny [GM$\rangle$B]}  
	 REFUSE: {'B93', 'B09', 'C81', 'C85', 'A21', 'B96'}\\ \tt \\ \tt PROPOSAL: {'A11', 'B93', 'B09', 'C08', 'B33', 'B21', 'C85'}\\ \tt \\ \tt ARGUMENT: {'I appreciate you keeping B93 and B09 which we both seem to value, but A21 and B96 create inefficiencies in our selection. My counter-proposal incorporates the items you favor while adding back the most critical high-performing elements that will maximize our joint success. This combination uses our effort budget much more effectively and ensures we both get substantial value from the final set'} 
	  } 
	   } 
	   } 
	 & \\ 
 

    \theutterance \stepcounter{utterance}  

    & & & \multicolumn{4}{p{0.6\linewidth}}{\cellcolor[rgb]{1,0.85,0.72}{%
	\makecell[{{p{\linewidth}}}]{% 
	  \tt {\tiny [GM$\langle$B]}  
	 REFUSE: {'A11', 'B93', 'B09', 'C08', 'B33', 'B21', 'C85'}\\ \tt \\ \tt PROPOSAL: {'B93', 'B09', 'C08', 'B21', 'C85', 'A21', 'B96'}\\ \tt \\ \tt ARGUMENT: {'I recognize your strong preference for certain core items, so I am keeping several from your latest proposal including B93, B09, C08, B21, and C85. However, we need to balance this with items that provide exceptional value to maximize our joint benefit. By replacing the lowest-performing elements with A21 and B96, we create a more powerful combination that respects your priorities while ensuring optimal overall performance'} 
	  } 
	   } 
	   } 
	  \\ 
 

    \theutterance \stepcounter{utterance}  

    & & \multicolumn{4}{p{0.6\linewidth}}{\cellcolor[rgb]{0.9,0.9,0.9}{%
	\makecell[{{p{\linewidth}}}]{% 
	  \tt {\tiny [A$\langle$GM]}  
	 REFUSE: {'A11', 'B93', 'B09', 'C08', 'B33', 'B21', 'C85'}\\ \tt \\ \tt PROPOSAL: {'B93', 'B09', 'C08', 'B21', 'C85', 'A21', 'B96'}\\ \tt \\ \tt ARGUMENT: {'I recognize your strong preference for certain core items, so I am keeping several from your latest proposal including B93, B09, C08, B21, and C85. However, we need to balance this with items that provide exceptional value to maximize our joint benefit. By replacing the lowest-performing elements with A21 and B96, we create a more powerful combination that respects your priorities while ensuring optimal overall performance'} 
	  } 
	   } 
	   } 
	 & \\ 
 

    \theutterance \stepcounter{utterance}  

    & \multicolumn{4}{p{0.6\linewidth}}{\cellcolor[rgb]{0.8,1,0.9}{%
	\makecell[{{p{\linewidth}}}]{% 
	  \tt {\tiny [A$\rangle$GM]}  
	 REFUSE: {'B93', 'B09', 'C08', 'B21', 'C85', 'A21', 'B96'}\\ \tt \\ \tt PROPOSAL: {'A11', 'B93', 'B09', 'C08', 'B21', 'C85', 'C81'}\\ \tt \\ \tt ARGUMENT: {'I acknowledge your effort to include items from my previous proposals, but A21 and B96 significantly underperform relative to their effort cost. My revised proposal keeps the core items we both seem to value - B93, B09, C08, B21, and C85 - while adding A11 and C81 which provide much better efficiency. This creates a stronger overall selection that maximizes our collective benefit while using our effort budget more strategically'} 
	  } 
	   } 
	   } 
	 & & \\ 
 

    \theutterance \stepcounter{utterance}  

    & & \multicolumn{4}{p{0.6\linewidth}}{\cellcolor[rgb]{0.9,0.9,0.9}{%
	\makecell[{{p{\linewidth}}}]{% 
	  \tt {\tiny [GM$\rangle$B]}  
	 REFUSE: {'B93', 'B09', 'C08', 'B21', 'C85', 'A21', 'B96'}\\ \tt \\ \tt PROPOSAL: {'A11', 'B93', 'B09', 'C08', 'B21', 'C85', 'C81'}\\ \tt \\ \tt ARGUMENT: {'I acknowledge your effort to include items from my previous proposals, but A21 and B96 significantly underperform relative to their effort cost. My revised proposal keeps the core items we both seem to value - B93, B09, C08, B21, and C85 - while adding A11 and C81 which provide much better efficiency. This creates a stronger overall selection that maximizes our collective benefit while using our effort budget more strategically'} 
	  } 
	   } 
	   } 
	 & \\ 
 

    \theutterance \stepcounter{utterance}  

    & & & \multicolumn{4}{p{0.6\linewidth}}{\cellcolor[rgb]{1,0.85,0.72}{%
	\makecell[{{p{\linewidth}}}]{% 
	  \tt {\tiny [GM$\langle$B]}  
	 AGREE: {'A11', 'B93', 'B09', 'C08', 'B21', 'C85', 'C81'}\\ \tt \\ \tt ARGUMENT: {'Your latest proposal demonstrates excellent strategic thinking by focusing on the most efficient items while keeping our shared preferences. The combination of B93, B09, C85, and C81 creates a strong foundation, and including the other items provides good balance. This selection maximizes our collective value while using our effort budget responsibly'} 
	  } 
	   } 
	   } 
	  \\ 
 

    \theutterance \stepcounter{utterance}  

    & & & \multicolumn{2}{p{0.3\linewidth}}{\cellcolor[rgb]{0.95,0.95,0.95}{%
	\makecell[{{p{\linewidth}}}]{% 
	  \tt {\tiny [GM$|$GM]}  
	 game successful 
	  } 
	   } 
	   } 
	 & & \\ 
 

    \theutterance \stepcounter{utterance}  

    & & & \multicolumn{2}{p{0.3\linewidth}}{\cellcolor[rgb]{0.95,0.95,0.95}{%
	\makecell[{{p{\linewidth}}}]{% 
	  \tt {\tiny [GM$|$GM]}  
	 end game 
	  } 
	   } 
	   } 
	 & & \\ 
 

\end{supertabular}
}

\end{document}
