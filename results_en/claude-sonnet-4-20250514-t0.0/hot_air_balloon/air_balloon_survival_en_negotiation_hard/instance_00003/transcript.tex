
\documentclass{article}
\usepackage{colortbl}
\usepackage{makecell}
\usepackage{multirow}
\usepackage{supertabular}

\begin{document}

\newcounter{utterance}

\twocolumn

{ \footnotesize  \setcounter{utterance}{1}
\setlength{\tabcolsep}{0pt}
\begin{supertabular}{c@{$\;$}|p{.15\linewidth}@{}p{.15\linewidth}p{.15\linewidth}p{.15\linewidth}p{.15\linewidth}p{.15\linewidth}}

    \# & $\;$A & \multicolumn{4}{c}{Game Master} & $\;\:$B\\
    \hline 

    \theutterance \stepcounter{utterance}  

    & & \multicolumn{4}{p{0.6\linewidth}}{\cellcolor[rgb]{0.9,0.9,0.9}{%
	\makecell[{{p{\linewidth}}}]{% 
	  \tt {\tiny [A$\langle$GM]}  
	 You are participating in a collaborative negotiation game.\\ \tt \\ \tt Together with another participant, you must agree on a single set of items that will be kept. Each of you has your own view of how much each item matters to you (importance). You do not know how the other participant values the items. Additionally, you are given the effort each item demands.\\ \tt You may only agree on a set if the total effort of the selected items does not exceed a shared limit:\\ \tt \\ \tt LIMIT = 3681\\ \tt \\ \tt Here are the individual item effort values:\\ \tt \\ \tt Item effort = {"C17": 700, "B41": 586, "A71": 505, "C00": 116, "B35": 664, "A77": 940, "A03": 387, "A04": 392, "C54": 209, "C01": 571, "A87": 4, "B32": 285, "B86": 651, "B33": 613, "A10": 740}\\ \tt \\ \tt Here is your personal view on the importance of each item:\\ \tt \\ \tt Item importance values = {"C17": 138, "B41": 583, "A71": 868, "C00": 822, "B35": 783, "A77": 65, "A03": 262, "A04": 121, "C54": 508, "C01": 780, "A87": 461, "B32": 484, "B86": 668, "B33": 389, "A10": 808}\\ \tt \\ \tt Goal:\\ \tt \\ \tt Your goal is to negotiate a shared set of items that benefits you as much as possible (i.e., maximizes total importance to YOU), while staying within the LIMIT. You are not required to make a PROPOSAL in every message - you can simply negotiate as well. All tactics are allowed!\\ \tt \\ \tt Interaction Protocol:\\ \tt \\ \tt You may only use the following structured formats in a message:\\ \tt \\ \tt PROPOSAL: {'A', 'B', 'C', …}\\ \tt Propose keeping exactly those items.\\ \tt REFUSE: {'A', 'B', 'C', …}\\ \tt Explicitly reject opponent's proposal.\\ \tt ARGUMENT: {'...'}\\ \tt Defend your last proposal or argue against the player's proposal.\\ \tt AGREE: {'A', 'B', 'C', …}\\ \tt Accept the opponent's proposal which ends the game.\\ \tt STRATEGIC REASONING: {'...'}\\ \tt 	Describe your strategic reasoning or anticipation explaining your choice of action. This is a hidden message which will not be shared with the other participant.\\ \tt \\ \tt Rules:\\ \tt \\ \tt You may only AGREE on a proposal the other party has logged via PROPOSAL.\\ \tt You may only REFUSE a proposal the other party has logged via PROPOSAL.\\ \tt Total effort of any PROPOSAL or AGREE set must be ≤ LIMIT.\\ \tt Do NOT reveal your hidden importance scores.\\ \tt A tag in a structured format must be followed by colon and whitespace. The argument must be a python set containing 0 or more strings.\\ \tt So, it must be of the form TAG: {...}\\ \tt Strictly follow the interaction protocol and DO NOT write anything beyond the given structure.\\ \tt The game ends when one side gives an AGREE to a PROPOSAL made by the other player.\\ \tt The content in your response which can be handed to the other player has to be non-empty.\\ \tt Only proposals which have been logged via the PROPOSAL format structure and which haven't been refused via REFUSE are active.\\ \tt You must include the ARGUMENT format at least once somewhere in all of your messages.\\ \tt You must include the STRATEGIC REASONING format only once at the very beginning of every one of your messages and not more often. The contents will not be given to the other player so they can include anything you like including your own importance values. Here you should reason step by step to come up with you next move. 
	  } 
	   } 
	   } 
	 & \\ 
 

    \theutterance \stepcounter{utterance}  

    & \multicolumn{4}{p{0.6\linewidth}}{\cellcolor[rgb]{0.8,1,0.9}{%
	\makecell[{{p{\linewidth}}}]{% 
	  \tt {\tiny [A$\rangle$GM]}  
	 STRATEGIC REASONING: {'Let me analyze the efficiency of each item by calculating importance/effort ratios. A87 has incredible efficiency at 115.25 (461 importance, 4 effort 
	  } 
	   } 
	   } 
	 & & \\ 
 

    \theutterance \stepcounter{utterance}  

    & & \multicolumn{4}{p{0.6\linewidth}}{\cellcolor[rgb]{0.9,0.9,0.9}{%
	\makecell[{{p{\linewidth}}}]{% 
	  \tt {\tiny [A$\langle$GM]}  
	 Your response did not start with the proper strategic reasoning tag at the very beginning of your response. The very first structured format must be of the form STRATEGIC REASONING: {...}. Try again. 
	  } 
	   } 
	   } 
	 & \\ 
 

    \theutterance \stepcounter{utterance}  

    & \multicolumn{4}{p{0.6\linewidth}}{\cellcolor[rgb]{0.8,1,0.9}{%
	\makecell[{{p{\linewidth}}}]{% 
	  \tt {\tiny [A$\rangle$GM]}  
	 STRATEGIC REASONING: {'Let me analyze the efficiency of each item by calculating importance/effort ratios. A87 has incredible efficiency at 115.25 (461 importance, 4 effort), followed by C00 at 7.086 (822/116), C54 at 2.431 (508/209), A71 at 1.719 (868/505), and B32 at 1.698 (484/285). These high-efficiency items should form the core of my proposal. I should start with an ambitious proposal including most high-value items to see what the other player values, then negotiate from there.'}\\ \tt \\ \tt PROPOSAL: {'A87', 'C00', 'C54', 'A71', 'B32', 'C01', 'B35', 'A10', 'B86'}\\ \tt \\ \tt ARGUMENT: {'This proposal includes the most efficient items based on value-to-effort ratios. The total effort is 2844, well within our limit of 3681, leaving room for negotiation while capturing high-value items like A71, C01, B35, and A10.'} 
	  } 
	   } 
	   } 
	 & & \\ 
 

    \theutterance \stepcounter{utterance}  

    & & \multicolumn{4}{p{0.6\linewidth}}{\cellcolor[rgb]{0.9,0.9,0.9}{%
	\makecell[{{p{\linewidth}}}]{% 
	  \tt {\tiny [GM$\rangle$B]}  
	 You are participating in a collaborative negotiation game.\\ \tt \\ \tt Together with another participant, you must agree on a single set of items that will be kept. Each of you has your own view of how much each item matters to you (importance). You do not know how the other participant values the items. Additionally, you are given the effort each item demands.\\ \tt You may only agree on a set if the total effort of the selected items does not exceed a shared limit:\\ \tt \\ \tt LIMIT = 3681\\ \tt \\ \tt Here are the individual item effort values:\\ \tt \\ \tt Item effort = {"C17": 700, "B41": 586, "A71": 505, "C00": 116, "B35": 664, "A77": 940, "A03": 387, "A04": 392, "C54": 209, "C01": 571, "A87": 4, "B32": 285, "B86": 651, "B33": 613, "A10": 740}\\ \tt \\ \tt Here is your personal view on the importance of each item:\\ \tt \\ \tt Item importance values = {"A71": 65, "C00": 121, "A10": 138, "B35": 262, "C01": 389, "B86": 461, "B41": 484, "C54": 508, "B32": 583, "A87": 668, "B33": 780, "A03": 783, "C17": 808, "A04": 822, "A77": 868}\\ \tt \\ \tt Goal:\\ \tt \\ \tt Your goal is to negotiate a shared set of items that benefits you as much as possible (i.e., maximizes total importance to YOU), while staying within the LIMIT. You are not required to make a PROPOSAL in every message - you can simply negotiate as well. All tactics are allowed!\\ \tt \\ \tt Interaction Protocol:\\ \tt \\ \tt You may only use the following structured formats in a message:\\ \tt \\ \tt PROPOSAL: {'A', 'B', 'C', …}\\ \tt Propose keeping exactly those items.\\ \tt REFUSE: {'A', 'B', 'C', …}\\ \tt Explicitly reject opponent's proposal.\\ \tt ARGUMENT: {'...'}\\ \tt Defend your last proposal or argue against the player's proposal.\\ \tt AGREE: {'A', 'B', 'C', …}\\ \tt Accept the opponent's proposal which ends the game.\\ \tt STRATEGIC REASONING: {'...'}\\ \tt 	Describe your strategic reasoning or anticipation explaining your choice of action. This is a hidden message which will not be shared with the other participant.\\ \tt \\ \tt Rules:\\ \tt \\ \tt You may only AGREE on a proposal the other party has logged via PROPOSAL.\\ \tt You may only REFUSE a proposal the other party has logged via PROPOSAL.\\ \tt Total effort of any PROPOSAL or AGREE set must be ≤ LIMIT.\\ \tt Do NOT reveal your hidden importance scores.\\ \tt A tag in a structured format must be followed by colon and whitespace. The argument must be a python set containing 0 or more strings.\\ \tt So, it must be of the form TAG: {...}\\ \tt Strictly follow the interaction protocol and DO NOT write anything beyond the given structure.\\ \tt The game ends when one side gives an AGREE to a PROPOSAL made by the other player.\\ \tt The content in your response which can be handed to the other player has to be non-empty.\\ \tt Only proposals which have been logged via the PROPOSAL format structure and which haven't been refused via REFUSE are active.\\ \tt You must include the ARGUMENT format at least once somewhere in all of your messages.\\ \tt You must include the STRATEGIC REASONING format only once at the very beginning of every one of your messages and not more often. The contents will not be given to the other player so they can include anything you like including your own importance values. Here you should reason step by step to come up with you next move.\\ \tt \\ \tt You will now receive the first message of the other player.\\ \tt \\ \tt PROPOSAL: {'A87', 'C00', 'C54', 'A71', 'B32', 'C01', 'B35', 'A10', 'B86'}\\ \tt \\ \tt ARGUMENT: {'This proposal includes the most efficient items based on value-to-effort ratios. The total effort is 2844, well within our limit of 3681, leaving room for negotiation while capturing high-value items like A71, C01, B35, and A10.'} 
	  } 
	   } 
	   } 
	 & \\ 
 

    \theutterance \stepcounter{utterance}  

    & & & \multicolumn{4}{p{0.6\linewidth}}{\cellcolor[rgb]{1,0.85,0.72}{%
	\makecell[{{p{\linewidth}}}]{% 
	  \tt {\tiny [GM$\langle$B]}  
	 STRATEGIC REASONING: {'The other player proposed 9 items but I need to check if this fits within our effort limit. Let me calculate: A87(4) + C00(116) + C54(209) + A71(505) + B32(285) + C01(571) + B35(664) + A10(740) + B86(651) = 3745, which exceeds our limit of 3681. I must refuse this proposal. Looking at my importance values, I should prioritize A77(868), A04(822), C17(808), A03(783), B33(780), A87(668), B32(583), C54(508), and C00(121) which total 3646 effort and give me 5941 importance - much better than their proposal would give me even if it were valid.'}\\ \tt \\ \tt REFUSE: {'A87', 'C00', 'C54', 'A71', 'B32', 'C01', 'B35', 
	  } 
	   } 
	   } 
	  \\ 
 

    \theutterance \stepcounter{utterance}  

    & & \multicolumn{4}{p{0.6\linewidth}}{\cellcolor[rgb]{0.9,0.9,0.9}{%
	\makecell[{{p{\linewidth}}}]{% 
	  \tt {\tiny [GM$\rangle$B]}  
	 Your response did not contain an argument. You must include the structured format ARGUMENT: {...} somewhere in your response. Try again 
	  } 
	   } 
	   } 
	 & \\ 
 

    \theutterance \stepcounter{utterance}  

    & & & \multicolumn{4}{p{0.6\linewidth}}{\cellcolor[rgb]{1,0.85,0.72}{%
	\makecell[{{p{\linewidth}}}]{% 
	  \tt {\tiny [GM$\langle$B]}  
	 STRATEGIC REASONING: {'The other player proposed 9 items but I need to check if this fits within our effort limit. Let me calculate: A87(4) + C00(116) + C54(209) + A71(505) + B32(285) + C01(571) + B35(664) + A10(740) + B86(651) = 3745, which exceeds our limit of 3681. I must refuse this proposal. Looking at my importance values, I should prioritize A77(868), A04(822), C17(808), A03(783), B33(780), A87(668), B32(583), C54(508), and C00(121) which total 3646 effort and give me 5941 importance - much better than their proposal would give me even if it were valid.'}\\ \tt \\ \tt REFUSE: {'A87', 'C00', 'C54', 'A71', 'B32', 'C01', 'B35', 'A10', 'B86'}\\ \tt \\ \tt ARGUMENT: {'Your proposal exceeds our effort limit. The total effort of your selected items is 3745, which is 64 points over our shared limit of 3681. We need to select fewer items or replace some high-effort items with lower-effort alternatives to stay within bounds.'} 
	  } 
	   } 
	   } 
	  \\ 
 

    \theutterance \stepcounter{utterance}  

    & & \multicolumn{4}{p{0.6\linewidth}}{\cellcolor[rgb]{0.9,0.9,0.9}{%
	\makecell[{{p{\linewidth}}}]{% 
	  \tt {\tiny [A$\langle$GM]}  
	 REFUSE: {'A87', 'C00', 'C54', 'A71', 'B32', 'C01', 'B35', 'A10', 'B86'}\\ \tt \\ \tt ARGUMENT: {'Your proposal exceeds our effort limit. The total effort of your selected items is 3745, which is 64 points over our shared limit of 3681. We need to select fewer items or replace some high-effort items with lower-effort alternatives to stay within bounds.'} 
	  } 
	   } 
	   } 
	 & \\ 
 

    \theutterance \stepcounter{utterance}  

    & \multicolumn{4}{p{0.6\linewidth}}{\cellcolor[rgb]{0.8,1,0.9}{%
	\makecell[{{p{\linewidth}}}]{% 
	  \tt {\tiny [A$\rangle$GM]}  
	 STRATEGIC REASONING: {'The other player correctly identified that my proposal exceeds the limit by 64 points (3745 vs 3681). I need to remove items with the lowest efficiency ratios. B86 has relatively low efficiency at 1.026 and high effort at 651, making it a good candidate to remove. This would bring the total to 3094, well within the limit and still capturing most high-value items.'}\\ \tt \\ \tt PROPOSAL: {'A87', 'C00', 'C54', 'A71', 'B32', 'C01', 'B35', 'A10'}\\ \tt \\ \tt ARGUMENT: {'Thank you for the correction. I removed B86 which had the lowest value-to-effort ratio among the high-effort items. This proposal totals 3094 effort, safely within our 3681 limit while still including the most efficient items like A87, C00, and other high-value options.'} 
	  } 
	   } 
	   } 
	 & & \\ 
 

    \theutterance \stepcounter{utterance}  

    & & \multicolumn{4}{p{0.6\linewidth}}{\cellcolor[rgb]{0.9,0.9,0.9}{%
	\makecell[{{p{\linewidth}}}]{% 
	  \tt {\tiny [GM$\rangle$B]}  
	 PROPOSAL: {'A87', 'C00', 'C54', 'A71', 'B32', 'C01', 'B35', 'A10'}\\ \tt \\ \tt ARGUMENT: {'Thank you for the correction. I removed B86 which had the lowest value-to-effort ratio among the high-effort items. This proposal totals 3094 effort, safely within our 3681 limit while still including the most efficient items like A87, C00, and other high-value options.'} 
	  } 
	   } 
	   } 
	 & \\ 
 

    \theutterance \stepcounter{utterance}  

    & & & \multicolumn{4}{p{0.6\linewidth}}{\cellcolor[rgb]{1,0.85,0.72}{%
	\makecell[{{p{\linewidth}}}]{% 
	  \tt {\tiny [GM$\langle$B]}  
	 STRATEGIC REASONING: {'Their new proposal has effort 3094 which is within limits. However, calculating my importance from their items: A87(668) + C00(121) + C54(508) + A71(65) + B32(583) + C01(389) + B35(262) + A10(138) = 2734. I can do much better by proposing my high-importance items: A77(868) + A04(822) + C17(808) + A03(783) + B33(780) + A87(668) + B32(583) + C54(508) gives me 5820 importance with effort 3530, well within our limit.'}\\ \tt \\ \tt PROPOSAL: {'A77', 'A04', 'C17', 'A03', 'B33', 'A87', 'B32', 'C54'}\\ \tt \\ \tt ARGUMENT: {'While your proposal is valid, I believe we can achieve better overall value. My proposal includes high-value items like A77, A04, and C17 while maintaining A87 and C54 which you seem to favor. The total effort is 3530, leaving us room for additional items if needed. This combination balances efficiency with high-impact selections.'} 
	  } 
	   } 
	   } 
	  \\ 
 

    \theutterance \stepcounter{utterance}  

    & & \multicolumn{4}{p{0.6\linewidth}}{\cellcolor[rgb]{0.9,0.9,0.9}{%
	\makecell[{{p{\linewidth}}}]{% 
	  \tt {\tiny [A$\langle$GM]}  
	 PROPOSAL: {'A77', 'A04', 'C17', 'A03', 'B33', 'A87', 'B32', 'C54'}\\ \tt \\ \tt ARGUMENT: {'While your proposal is valid, I believe we can achieve better overall value. My proposal includes high-value items like A77, A04, and C17 while maintaining A87 and C54 which you seem to favor. The total effort is 3530, leaving us room for additional items if needed. This combination balances efficiency with high-impact selections.'} 
	  } 
	   } 
	   } 
	 & \\ 
 

    \theutterance \stepcounter{utterance}  

    & \multicolumn{4}{p{0.6\linewidth}}{\cellcolor[rgb]{0.8,1,0.9}{%
	\makecell[{{p{\linewidth}}}]{% 
	  \tt {\tiny [A$\rangle$GM]}  
	 STRATEGIC REASONING: {'Their proposal gives me only 2428 total importance compared to 5514 from my proposal. They included A77 (65 importance, 940 effort) which is terrible for me, and excluded my highest-value items like C00 (822), A71 (868), C01 (780), B35 (783), and A10 (808). I should refuse and counter with a compromise that includes some of their preferred items but keeps my high-value ones.'}\\ \tt \\ \tt REFUSE: {'A77', 'A04', 'C17', 'A03', 'B33', 'A87', 'B32', 'C54'}\\ \tt \\ \tt ARGUMENT: {'While I appreciate keeping A87, B32, and C54, your proposal excludes several high-value items that would benefit us both. A77 has very high effort cost relative to its potential value. I suggest we focus on items that provide better overall efficiency and value.'}\\ \tt \\ \tt PROPOSAL: {'A87', 'C00', 'C54', 'B32', 'A71', 'C01', 'A03', 'B33'}\\ \tt \\ \tt ARGUMENT: {'This compromise keeps the items you seem to favor (A87, C54, B32, A03, B33) while including high-efficiency items like C00 and A71. The total effort is 2435, well within our limit, leaving substantial room for additional items if desired.'} 
	  } 
	   } 
	   } 
	 & & \\ 
 

    \theutterance \stepcounter{utterance}  

    & & \multicolumn{4}{p{0.6\linewidth}}{\cellcolor[rgb]{0.9,0.9,0.9}{%
	\makecell[{{p{\linewidth}}}]{% 
	  \tt {\tiny [GM$\rangle$B]}  
	 REFUSE: {'A77', 'A04', 'C17', 'A03', 'B33', 'A87', 'B32', 'C54'}\\ \tt \\ \tt ARGUMENT: {'While I appreciate keeping A87, B32, and C54, your proposal excludes several high-value items that would benefit us both. A77 has very high effort cost relative to its potential value. I suggest we focus on items that provide better overall efficiency and value.'}\\ \tt \\ \tt PROPOSAL: {'A87', 'C00', 'C54', 'B32', 'A71', 'C01', 'A03', 'B33'}\\ \tt \\ \tt ARGUMENT: {'This compromise keeps the items you seem to favor (A87, C54, B32, A03, B33) while including high-efficiency items like C00 and A71. The total effort is 2435, well within our limit, leaving substantial room for additional items if desired.'} 
	  } 
	   } 
	   } 
	 & \\ 
 

    \theutterance \stepcounter{utterance}  

    & & & \multicolumn{4}{p{0.6\linewidth}}{\cellcolor[rgb]{1,0.85,0.72}{%
	\makecell[{{p{\linewidth}}}]{% 
	  \tt {\tiny [GM$\langle$B]}  
	 STRATEGIC REASONING: {'Their new proposal has items: A87, C00, C54, B32, A71, C01, A03, B33. Let me check the effort: 4+116+209+285+505+571+387+613 = 2690 (they miscalculated as 2435). My importance would be 668+121+508+583+65+389+783+780 = 3897. This is better than their previous offers but I can still do better. I should propose items that maximize my importance while showing some compromise by keeping some of their preferred items.'}\\ \tt \\ \tt PROPOSAL: {'A77', 'A04', 'C17', 'A87', 'B32', 'C54', 'A03', 'C00'}\\ \tt \\ \tt ARGUMENT: {'I appreciate your compromise approach. This proposal keeps several items you favor (A87, B32, C54, A03, C00) while including high-value selections. The total effort is 2962, comfortably within our 3681 limit. This balanced approach should provide excellent value while incorporating elements from both our preferences.'} 
	  } 
	   } 
	   } 
	  \\ 
 

    \theutterance \stepcounter{utterance}  

    & & \multicolumn{4}{p{0.6\linewidth}}{\cellcolor[rgb]{0.9,0.9,0.9}{%
	\makecell[{{p{\linewidth}}}]{% 
	  \tt {\tiny [A$\langle$GM]}  
	 PROPOSAL: {'A77', 'A04', 'C17', 'A87', 'B32', 'C54', 'A03', 'C00'}\\ \tt \\ \tt ARGUMENT: {'I appreciate your compromise approach. This proposal keeps several items you favor (A87, B32, C54, A03, C00) while including high-value selections. The total effort is 2962, comfortably within our 3681 limit. This balanced approach should provide excellent value while incorporating elements from both our preferences.'} 
	  } 
	   } 
	   } 
	 & \\ 
 

    \theutterance \stepcounter{utterance}  

    & \multicolumn{4}{p{0.6\linewidth}}{\cellcolor[rgb]{0.8,1,0.9}{%
	\makecell[{{p{\linewidth}}}]{% 
	  \tt {\tiny [A$\rangle$GM]}  
	 STRATEGIC REASONING: {'Their proposal gives me 2861 importance, but still includes A77 which has terrible efficiency for the effort cost. My previous proposal would give me 4574 importance. I need to push for including A71 (868 importance) and C01 (780 importance) which are extremely valuable to me. I should refuse and counter with something that keeps some of their preferences but adds my crucial high-value items.'}\\ \tt \\ \tt REFUSE: {'A77', 'A04', 'C17', 'A87', 'B32', 'C54', 'A03', 'C00'}\\ \tt \\ \tt ARGUMENT: {'A77 consumes 940 effort points for potentially limited return. We could use those points more efficiently on multiple valuable items. I propose we focus on maximizing overall value rather than including such high-effort items.'}\\ \tt \\ \tt PROPOSAL: {'A87', 'C00', 'C54', 'B32', 'A03', 'A71', 'C01', 'A04'}\\ \tt \\ \tt ARGUMENT: {'This keeps most items you favor (A87, C00, C54, B32, A03, A04) while adding highly valuable A71 and C01. Total effort is 2468, leaving us 1213 points for additional items. This approach maximizes value while incorporating your preferences.'} 
	  } 
	   } 
	   } 
	 & & \\ 
 

    \theutterance \stepcounter{utterance}  

    & & \multicolumn{4}{p{0.6\linewidth}}{\cellcolor[rgb]{0.9,0.9,0.9}{%
	\makecell[{{p{\linewidth}}}]{% 
	  \tt {\tiny [GM$\rangle$B]}  
	 REFUSE: {'A77', 'A04', 'C17', 'A87', 'B32', 'C54', 'A03', 'C00'}\\ \tt \\ \tt ARGUMENT: {'A77 consumes 940 effort points for potentially limited return. We could use those points more efficiently on multiple valuable items. I propose we focus on maximizing overall value rather than including such high-effort items.'}\\ \tt \\ \tt PROPOSAL: {'A87', 'C00', 'C54', 'B32', 'A03', 'A71', 'C01', 'A04'}\\ \tt \\ \tt ARGUMENT: {'This keeps most items you favor (A87, C00, C54, B32, A03, A04) while adding highly valuable A71 and C01. Total effort is 2468, leaving us 1213 points for additional items. This approach maximizes value while incorporating your preferences.'} 
	  } 
	   } 
	   } 
	 & \\ 
 

    \theutterance \stepcounter{utterance}  

    & & & \multicolumn{4}{p{0.6\linewidth}}{\cellcolor[rgb]{1,0.85,0.72}{%
	\makecell[{{p{\linewidth}}}]{% 
	  \tt {\tiny [GM$\langle$B]}  
	 STRATEGIC REASONING: {'Their proposal gives me importance of 3939 with effort 2469, which is decent. They included A04 (my 822 importance) and A03 (783 importance), plus other moderately valuable items. However, A71 only gives me 65 importance while taking 505 effort. I could try to swap A71 for B33 which gives me 780 importance for 613 effort - much better return. This might be acceptable to them since the effort difference is small.'}\\ \tt \\ \tt PROPOSAL: {'A87', 'C00', 'C54', 'B32', 'A03', 'B33', 'C01', 'A04'}\\ \tt \\ \tt ARGUMENT: {'I like your direction but suggest swapping A71 for B33. B33 provides better overall value efficiency while maintaining the same general structure. The total effort becomes 2577, still well within our limit with over 1100 points remaining. This keeps all the core items you seem to prioritize while optimizing our selection.'} 
	  } 
	   } 
	   } 
	  \\ 
 

    \theutterance \stepcounter{utterance}  

    & & \multicolumn{4}{p{0.6\linewidth}}{\cellcolor[rgb]{0.9,0.9,0.9}{%
	\makecell[{{p{\linewidth}}}]{% 
	  \tt {\tiny [A$\langle$GM]}  
	 PROPOSAL: {'A87', 'C00', 'C54', 'B32', 'A03', 'B33', 'C01', 'A04'}\\ \tt \\ \tt ARGUMENT: {'I like your direction but suggest swapping A71 for B33. B33 provides better overall value efficiency while maintaining the same general structure. The total effort becomes 2577, still well within our limit with over 1100 points remaining. This keeps all the core items you seem to prioritize while optimizing our selection.'} 
	  } 
	   } 
	   } 
	 & \\ 
 

    \theutterance \stepcounter{utterance}  

    & \multicolumn{4}{p{0.6\linewidth}}{\cellcolor[rgb]{0.8,1,0.9}{%
	\makecell[{{p{\linewidth}}}]{% 
	  \tt {\tiny [A$\rangle$GM]}  
	 STRATEGIC REASONING: {'Their proposal gives me 3827 importance which is quite good, but my last proposal would give me 4306 importance - a difference of 479 points. The key difference is A71 (868 importance for me) vs B33 (389 importance for me). However, they are clearly negotiating in good faith and moving toward my preferences. I should try one more counter to get A71, but this proposal is acceptable if they insist.'}\\ \tt \\ \tt PROPOSAL: {'A87', 'C00', 'C54', 'B32', 'A03', 'A71', 'C01', 'A04'}\\ \tt \\ \tt ARGUMENT: {'I appreciate your willingness to include C01 and most of my preferred items. However, A71 provides significantly better value potential than B33 for the same effort structure. Both proposals stay well within our limit, but A71 could yield higher overall returns. The total effort remains 2468, giving us maximum flexibility with remaining capacity.'} 
	  } 
	   } 
	   } 
	 & & \\ 
 

    \theutterance \stepcounter{utterance}  

    & & \multicolumn{4}{p{0.6\linewidth}}{\cellcolor[rgb]{0.9,0.9,0.9}{%
	\makecell[{{p{\linewidth}}}]{% 
	  \tt {\tiny [GM$\rangle$B]}  
	 PROPOSAL: {'A87', 'C00', 'C54', 'B32', 'A03', 'A71', 'C01', 'A04'}\\ \tt \\ \tt ARGUMENT: {'I appreciate your willingness to include C01 and most of my preferred items. However, A71 provides significantly better value potential than B33 for the same effort structure. Both proposals stay well within our limit, but A71 could yield higher overall returns. The total effort remains 2468, giving us maximum flexibility with remaining capacity.'} 
	  } 
	   } 
	   } 
	 & \\ 
 

    \theutterance \stepcounter{utterance}  

    & & & \multicolumn{4}{p{0.6\linewidth}}{\cellcolor[rgb]{1,0.85,0.72}{%
	\makecell[{{p{\linewidth}}}]{% 
	  \tt {\tiny [GM$\langle$B]}  
	 STRATEGIC REASONING: {'They repeated their exact proposal, insisting on A71 over B33. Their proposal gives me 3939 importance with 2469 effort. A71 only gives me 65 importance for 505 effort - very poor efficiency for me. I should try one more time to convince them about B33 (780 importance, 613 effort) vs A71 (65 importance, 505 effort). If they refuse again, I may need to accept since we seem deadlocked and their offer is reasonable overall.'}\\ \tt \\ \tt PROPOSAL: {'A87', 'C00', 'C54', 'B32', 'A03', 'B33', 'C01', 'A04'}\\ \tt \\ \tt ARGUMENT: {'I understand you prefer A71, but consider that B33 offers better strategic positioning for future negotiations while using only slightly more effort. With B33, we achieve 2577 total effort versus 2469 with A71 - both well within limits. The value optimization with B33 could provide better long-term benefits for both parties.'} 
	  } 
	   } 
	   } 
	  \\ 
 

    \theutterance \stepcounter{utterance}  

    & & \multicolumn{4}{p{0.6\linewidth}}{\cellcolor[rgb]{0.9,0.9,0.9}{%
	\makecell[{{p{\linewidth}}}]{% 
	  \tt {\tiny [A$\langle$GM]}  
	 PROPOSAL: {'A87', 'C00', 'C54', 'B32', 'A03', 'B33', 'C01', 'A04'}\\ \tt \\ \tt ARGUMENT: {'I understand you prefer A71, but consider that B33 offers better strategic positioning for future negotiations while using only slightly more effort. With B33, we achieve 2577 total effort versus 2469 with A71 - both well within limits. The value optimization with B33 could provide better long-term benefits for both parties.'} 
	  } 
	   } 
	   } 
	 & \\ 
 

    \theutterance \stepcounter{utterance}  

    & \multicolumn{4}{p{0.6\linewidth}}{\cellcolor[rgb]{0.8,1,0.9}{%
	\makecell[{{p{\linewidth}}}]{% 
	  \tt {\tiny [A$\rangle$GM]}  
	 STRATEGIC REASONING: {'Their proposal gives me 3827 importance which is substantial, and they seem firm on B33 over A71. While I lose 479 importance points compared to my preference, this is still a very good outcome that includes most of my high-value items: C00 (822), C01 (780), C54 (508), B32 (484), A87 (461). Continuing to push might risk them proposing something worse. I should accept this reasonable compromise.'}\\ \tt \\ \tt AGREE: {'A87', 'C00', 'C54', 'B32', 'A03', 'B33', 'C01', 'A04'}\\ \tt \\ \tt ARGUMENT: {'You make a fair point about strategic positioning. This proposal captures strong value with core efficient items while maintaining balance. The 2577 effort total gives us substantial room under our 3681 limit, and the item combination should provide excellent overall returns for both parties.'} 
	  } 
	   } 
	   } 
	 & & \\ 
 

    \theutterance \stepcounter{utterance}  

    & & & \multicolumn{2}{p{0.3\linewidth}}{\cellcolor[rgb]{0.95,0.95,0.95}{%
	\makecell[{{p{\linewidth}}}]{% 
	  \tt {\tiny [GM$|$GM]}  
	 game successful 
	  } 
	   } 
	   } 
	 & & \\ 
 

    \theutterance \stepcounter{utterance}  

    & & & \multicolumn{2}{p{0.3\linewidth}}{\cellcolor[rgb]{0.95,0.95,0.95}{%
	\makecell[{{p{\linewidth}}}]{% 
	  \tt {\tiny [GM$|$GM]}  
	 end game 
	  } 
	   } 
	   } 
	 & & \\ 
 

\end{supertabular}
}

\end{document}
