
\documentclass{article}
\usepackage{colortbl}
\usepackage{makecell}
\usepackage{multirow}
\usepackage{supertabular}

\begin{document}

\newcounter{utterance}

\centering \large Interaction Transcript for game `hot\_air\_balloon', experiment `air\_balloon\_survival\_en\_negotiation\_hard', episode 1 with claude{-}sonnet{-}4{-}20250514{-}t0.0.
\vspace{24pt}

{ \footnotesize  \setcounter{utterance}{1}
\setlength{\tabcolsep}{0pt}
\begin{supertabular}{c@{$\;$}|p{.15\linewidth}@{}p{.15\linewidth}p{.15\linewidth}p{.15\linewidth}p{.15\linewidth}p{.15\linewidth}}
    \# & $\;$A & \multicolumn{4}{c}{Game Master} & $\;\:$B\\
    \hline

    \theutterance \stepcounter{utterance}  
    & & \multicolumn{4}{p{0.6\linewidth}}{
        \cellcolor[rgb]{0.9,0.9,0.9}{
            \makecell[{{p{\linewidth}}}]{
                \texttt{\tiny{[P1$\langle$GM]}}
                \texttt{You are participating in a collaborative negotiation game.} \\
\\ 
\texttt{Together with another participant, you must agree on a single set of items that will be kept. Each of you has your own view of how much each item matters to you (importance). You do not know how the other participant values the items. Additionally, you are given the effort each item demands.} \\
\texttt{You may only agree on a set if the total effort of the selected items does not exceed a shared limit:} \\
\\ 
\texttt{LIMIT = 4026} \\
\\ 
\texttt{Here are the individual item effort values:} \\
\\ 
\texttt{Item effort = \{"C17": 780, "B41": 742, "A71": 997, "C00": 665, "B35": 140, "A77": 77, "A03": 513, "A04": 383, "C54": 587, "C01": 825, "A87": 319, "B32": 448, "B86": 516, "B33": 694, "A10": 366\}} \\
\\ 
\texttt{Here is your personal view on the importance of each item:} \\
\\ 
\texttt{Item importance values = \{"C17": 138, "B41": 583, "A71": 868, "C00": 822, "B35": 783, "A77": 65, "A03": 262, "A04": 121, "C54": 508, "C01": 780, "A87": 461, "B32": 484, "B86": 668, "B33": 389, "A10": 808\}} \\
\\ 
\texttt{Goal:} \\
\\ 
\texttt{Your goal is to negotiate a shared set of items that benefits you as much as possible (i.e., maximizes total importance to YOU), while staying within the LIMIT. You are not required to make a PROPOSAL in every message {-} you can simply negotiate as well. All tactics are allowed!} \\
\\ 
\texttt{Interaction Protocol:} \\
\\ 
\texttt{You may only use the following structured formats in a message:} \\
\\ 
\texttt{PROPOSAL: \{'A', 'B', 'C', …\}} \\
\texttt{Propose keeping exactly those items.} \\
\texttt{REFUSE: \{'A', 'B', 'C', …\}} \\
\texttt{Explicitly reject opponent's proposal.} \\
\texttt{ARGUMENT: \{'...'\}} \\
\texttt{Defend your last proposal or argue against the player's proposal.} \\
\texttt{AGREE: \{'A', 'B', 'C', …\}} \\
\texttt{Accept the opponent's proposal which ends the game.} \\
\texttt{STRATEGIC REASONING: \{'...'\}} \\
\texttt{	Describe your strategic reasoning or anticipation explaining your choice of action. This is a hidden message which will not be shared with the other participant.} \\
\\ 
\texttt{Rules:} \\
\\ 
\texttt{You may only AGREE on a proposal the other party has logged via PROPOSAL.} \\
\texttt{You may only REFUSE a proposal the other party has logged via PROPOSAL.} \\
\texttt{Total effort of any PROPOSAL or AGREE set must be ≤ LIMIT.} \\
\texttt{Do NOT reveal your hidden importance scores.} \\
\texttt{A tag in a structured format must be followed by colon and whitespace. The argument must be a python set containing 0 or more strings.} \\
\texttt{So, it must be of the form TAG: \{...\}} \\
\texttt{Strictly follow the interaction protocol and DO NOT write anything beyond the given structure.} \\
\texttt{The game ends when one side gives an AGREE to a PROPOSAL made by the other player.} \\
\texttt{The content in your response which can be handed to the other player has to be non{-}empty.} \\
\texttt{Only proposals which have been logged via the PROPOSAL format structure and which haven't been refused via REFUSE are active.} \\
\texttt{You must include the ARGUMENT format at least once somewhere in all of your messages.} \\
\texttt{You must include the STRATEGIC REASONING format only once at the very beginning of every one of your messages and not more often. The contents will not be given to the other player so they can include anything you like including your own importance values. Here you should reason step by step to come up with you next move.} \\
            }
        }
    }
    & \\ \\

    \theutterance \stepcounter{utterance}  
    & \multicolumn{4}{p{0.6\linewidth}}{
        \cellcolor[rgb]{0.9,0.9,0.9}{
            \makecell[{{p{\linewidth}}}]{
                \texttt{\tiny{[P1$\rangle$GM]}}
                \texttt{STRATEGIC REASONING: \{'I need to analyze the efficiency of each item by calculating importance/effort ratios. My most efficient items are B35 (5.59), A10 (2.21), A87 (1.45), B86 (1.30), and C00 (1.24). I should start with a proposal that includes the highest efficiency items while staying under the 4026 limit. I will propose: B35, A10, A87, B86, C00, B32, C01, C54 which totals 3866 effort and gives me } \\
            }
        }
    }
    & & \\ \\

    \theutterance \stepcounter{utterance}  
    & & \multicolumn{4}{p{0.6\linewidth}}{
        \cellcolor[rgb]{0.9,0.9,0.9}{
            \makecell[{{p{\linewidth}}}]{
                \texttt{\tiny{[P1$\langle$GM]}}
                \texttt{Your response did not start with the proper strategic reasoning tag at the very beginning of your response. The very first structured format must be of the form STRATEGIC REASONING: \{...\}. Try again.} \\
            }
        }
    }
    & \\ \\

    \theutterance \stepcounter{utterance}  
    & \multicolumn{4}{p{0.6\linewidth}}{
        \cellcolor[rgb]{0.9,0.9,0.9}{
            \makecell[{{p{\linewidth}}}]{
                \texttt{\tiny{[P1$\rangle$GM]}}
                \texttt{STRATEGIC REASONING: \{'I need to analyze efficiency ratios (importance/effort) for each item. B35 has the highest ratio at 5.59, followed by A10 (2.21), A87 (1.45), B86 (1.30), and C00 (1.24). I should propose a combination of high{-}efficiency items. Let me try: B35(140), A10(366), A87(319), B86(516), C00(665), B32(448), C01(825), C54(587) = 3866 total effort, giving me importance of 783+808+461+668+822+484+780+508 = 5314. This seems like a strong opening proposal.'\}} \\
\\ 
\texttt{PROPOSAL: \{'B35', 'A10', 'A87', 'B86', 'C00', 'B32', 'C01', 'C54'\}} \\
\\ 
\texttt{ARGUMENT: \{'This combination focuses on items that provide strong value while staying well within our effort limit of 4026. The selected items total 3866 effort, leaving room for potential adjustments while ensuring we get a substantial benefit from our collaboration.'\}} \\
            }
        }
    }
    & & \\ \\

    \theutterance \stepcounter{utterance}  
    & & \multicolumn{4}{p{0.6\linewidth}}{
        \cellcolor[rgb]{0.9,0.9,0.9}{
            \makecell[{{p{\linewidth}}}]{
                \texttt{\tiny{[GM$\rangle$P2]}}
                \texttt{You are participating in a collaborative negotiation game.} \\
\\ 
\texttt{Together with another participant, you must agree on a single set of items that will be kept. Each of you has your own view of how much each item matters to you (importance). You do not know how the other participant values the items. Additionally, you are given the effort each item demands.} \\
\texttt{You may only agree on a set if the total effort of the selected items does not exceed a shared limit:} \\
\\ 
\texttt{LIMIT = 4026} \\
\\ 
\texttt{Here are the individual item effort values:} \\
\\ 
\texttt{Item effort = \{"C17": 780, "B41": 742, "A71": 997, "C00": 665, "B35": 140, "A77": 77, "A03": 513, "A04": 383, "C54": 587, "C01": 825, "A87": 319, "B32": 448, "B86": 516, "B33": 694, "A10": 366\}} \\
\\ 
\texttt{Here is your personal view on the importance of each item:} \\
\\ 
\texttt{Item importance values = \{"A71": 65, "C00": 121, "A10": 138, "B35": 262, "C01": 389, "B86": 461, "B41": 484, "C54": 508, "B32": 583, "A87": 668, "B33": 780, "A03": 783, "C17": 808, "A04": 822, "A77": 868\}} \\
\\ 
\texttt{Goal:} \\
\\ 
\texttt{Your goal is to negotiate a shared set of items that benefits you as much as possible (i.e., maximizes total importance to YOU), while staying within the LIMIT. You are not required to make a PROPOSAL in every message {-} you can simply negotiate as well. All tactics are allowed!} \\
\\ 
\texttt{Interaction Protocol:} \\
\\ 
\texttt{You may only use the following structured formats in a message:} \\
\\ 
\texttt{PROPOSAL: \{'A', 'B', 'C', …\}} \\
\texttt{Propose keeping exactly those items.} \\
\texttt{REFUSE: \{'A', 'B', 'C', …\}} \\
\texttt{Explicitly reject opponent's proposal.} \\
\texttt{ARGUMENT: \{'...'\}} \\
\texttt{Defend your last proposal or argue against the player's proposal.} \\
\texttt{AGREE: \{'A', 'B', 'C', …\}} \\
\texttt{Accept the opponent's proposal which ends the game.} \\
\texttt{STRATEGIC REASONING: \{'...'\}} \\
\texttt{	Describe your strategic reasoning or anticipation explaining your choice of action. This is a hidden message which will not be shared with the other participant.} \\
\\ 
\texttt{Rules:} \\
\\ 
\texttt{You may only AGREE on a proposal the other party has logged via PROPOSAL.} \\
\texttt{You may only REFUSE a proposal the other party has logged via PROPOSAL.} \\
\texttt{Total effort of any PROPOSAL or AGREE set must be ≤ LIMIT.} \\
\texttt{Do NOT reveal your hidden importance scores.} \\
\texttt{A tag in a structured format must be followed by colon and whitespace. The argument must be a python set containing 0 or more strings.} \\
\texttt{So, it must be of the form TAG: \{...\}} \\
\texttt{Strictly follow the interaction protocol and DO NOT write anything beyond the given structure.} \\
\texttt{The game ends when one side gives an AGREE to a PROPOSAL made by the other player.} \\
\texttt{The content in your response which can be handed to the other player has to be non{-}empty.} \\
\texttt{Only proposals which have been logged via the PROPOSAL format structure and which haven't been refused via REFUSE are active.} \\
\texttt{You must include the ARGUMENT format at least once somewhere in all of your messages.} \\
\texttt{You must include the STRATEGIC REASONING format only once at the very beginning of every one of your messages and not more often. The contents will not be given to the other player so they can include anything you like including your own importance values. Here you should reason step by step to come up with you next move.} \\
\\ 
\texttt{You will now receive the first message of the other player.} \\
\\ 
\texttt{PROPOSAL: \{'B35', 'A10', 'A87', 'B86', 'C00', 'B32', 'C01', 'C54'\}} \\
\\ 
\texttt{ARGUMENT: \{'This combination focuses on items that provide strong value while staying well within our effort limit of 4026. The selected items total 3866 effort, leaving room for potential adjustments while ensuring we get a substantial benefit from our collaboration.'\}} \\
            }
        }
    }
    & \\ \\

    \theutterance \stepcounter{utterance}  
    & & & \multicolumn{4}{p{0.6\linewidth}}{
        \cellcolor[rgb]{0.9,0.9,0.9}{
            \makecell[{{p{\linewidth}}}]{
                \texttt{\tiny{[GM$\langle$P2]}}
                \texttt{STRATEGIC REASONING: \{'The opponent proposed a set with total importance 3130 to me, using 3866 effort. I can do much better by including my highest{-}value items like A77 (868 importance, 77 effort), A04 (822 importance, 383 effort), and C17 (808 importance, 780 effort). My counter{-}proposal includes these plus some overlap with theirs, giving me total importance of 6082 while using only 3941 effort, well under the limit.'\}} \\
\\ 
\texttt{PROPOSAL: \{'A77', 'A04', 'C17', 'A03', 'B33', 'A87', 'B32', 'C54', 'B35'\}} \\
\\ 
\texttt{ARGUMENT: \{'My proposal includes several high{-}impact items that weren't in your selection, while keeping some of the valuable items from your proposal like A87, B32, and C54. This combination uses 3941 effort, staying comfortably under our 4026 limit while potentially offering better overall value for both of us.'\}} \\
            }
        }
    }
     \\ \\

    \theutterance \stepcounter{utterance}  
    & & \multicolumn{4}{p{0.6\linewidth}}{
        \cellcolor[rgb]{0.9,0.9,0.9}{
            \makecell[{{p{\linewidth}}}]{
                \texttt{\tiny{[P1$\langle$GM]}}
                \texttt{PROPOSAL: \{'A77', 'A04', 'C17', 'A03', 'B33', 'A87', 'B32', 'C54', 'B35'\}} \\
\\ 
\texttt{ARGUMENT: \{'My proposal includes several high{-}impact items that weren't in your selection, while keeping some of the valuable items from your proposal like A87, B32, and C54. This combination uses 3941 effort, staying comfortably under our 4026 limit while potentially offering better overall value for both of us.'\}} \\
            }
        }
    }
    & \\ \\

    \theutterance \stepcounter{utterance}  
    & \multicolumn{4}{p{0.6\linewidth}}{
        \cellcolor[rgb]{0.9,0.9,0.9}{
            \makecell[{{p{\linewidth}}}]{
                \texttt{\tiny{[P1$\rangle$GM]}}
                \texttt{STRATEGIC REASONING: \{'The opponent\textbackslash{}'s proposal gives me only 3211 total importance compared to my proposal which would give me 5314. They removed my most valuable items: A10 (808), B86 (668), C00 (822), and C01 (780), replacing them with low{-}value items like A77 (65), A04 (121), and C17 (138). I need to refuse this and push for keeping some of my high{-}efficiency items.'\}} \\
\\ 
\texttt{REFUSE: \{'A77', 'A04', 'C17', 'A03', 'B33', 'A87', 'B32', 'C54', 'B35'\}} \\
\\ 
\texttt{ARGUMENT: \{'While your proposal stays within the effort limit, it removes several items that could provide significant mutual benefit. I suggest we keep some key items from my original proposal like A10 and C00, which offer strong value relative to their effort cost. Let me offer a compromise that balances both our interests.'\}} \\
\\ 
\texttt{PROPOSAL: \{'B35', 'A10', 'A87', 'B32', 'C54', 'C00', 'A03', 'B33'\}} \\
            }
        }
    }
    & & \\ \\

    \theutterance \stepcounter{utterance}  
    & & \multicolumn{4}{p{0.6\linewidth}}{
        \cellcolor[rgb]{0.9,0.9,0.9}{
            \makecell[{{p{\linewidth}}}]{
                \texttt{\tiny{[GM$\rangle$P2]}}
                \texttt{REFUSE: \{'A77', 'A04', 'C17', 'A03', 'B33', 'A87', 'B32', 'C54', 'B35'\}} \\
\\ 
\texttt{ARGUMENT: \{'While your proposal stays within the effort limit, it removes several items that could provide significant mutual benefit. I suggest we keep some key items from my original proposal like A10 and C00, which offer strong value relative to their effort cost. Let me offer a compromise that balances both our interests.'\}} \\
\\ 
\texttt{PROPOSAL: \{'B35', 'A10', 'A87', 'B32', 'C54', 'C00', 'A03', 'B33'\}} \\
            }
        }
    }
    & \\ \\

    \theutterance \stepcounter{utterance}  
    & & & \multicolumn{4}{p{0.6\linewidth}}{
        \cellcolor[rgb]{0.9,0.9,0.9}{
            \makecell[{{p{\linewidth}}}]{
                \texttt{\tiny{[GM$\langle$P2]}}
                \texttt{STRATEGIC REASONING: \{'Their new proposal gives me 3843 importance using 3732 effort. It includes A03 and B33 which are valuable to me, but still misses my top items A77 (868 importance, only 77 effort {-} incredible ratio) and A04 (822 importance, 383 effort). I need to push for including these high{-}value items. I can offer a compromise that keeps some of their preferred items while adding my must{-}haves.'\}} \\
\\ 
\texttt{PROPOSAL: \{'A77', 'A04', 'A03', 'B33', 'A87', 'B32', 'C54', 'B35'\}} \\
\\ 
\texttt{ARGUMENT: \{'I appreciate your compromise approach and I want to find common ground. I kept most of your valuable items including A03, B33, A87, B32, C54, and B35, but added A77 and A04 which offer exceptional value for their low effort cost. This selection uses 3617 effort, well under our limit, and should provide strong benefits for both of us while maintaining the collaborative spirit.'\}} \\
            }
        }
    }
     \\ \\

    \theutterance \stepcounter{utterance}  
    & & \multicolumn{4}{p{0.6\linewidth}}{
        \cellcolor[rgb]{0.9,0.9,0.9}{
            \makecell[{{p{\linewidth}}}]{
                \texttt{\tiny{[P1$\langle$GM]}}
                \texttt{PROPOSAL: \{'A77', 'A04', 'A03', 'B33', 'A87', 'B32', 'C54', 'B35'\}} \\
\\ 
\texttt{ARGUMENT: \{'I appreciate your compromise approach and I want to find common ground. I kept most of your valuable items including A03, B33, A87, B32, C54, and B35, but added A77 and A04 which offer exceptional value for their low effort cost. This selection uses 3617 effort, well under our limit, and should provide strong benefits for both of us while maintaining the collaborative spirit.'\}} \\
            }
        }
    }
    & \\ \\

    \theutterance \stepcounter{utterance}  
    & \multicolumn{4}{p{0.6\linewidth}}{
        \cellcolor[rgb]{0.9,0.9,0.9}{
            \makecell[{{p{\linewidth}}}]{
                \texttt{\tiny{[P1$\rangle$GM]}}
                \texttt{STRATEGIC REASONING: \{'The opponent\textbackslash{}'s new proposal gives me only 3073 importance compared to 5314 from my original proposal. They removed A10 (808 importance) and C00 (822 importance) {-} my two highest{-}value items after B35. They claim A77 and A04 offer "exceptional value" but A77 gives me only 65 importance for 77 effort (0.84 ratio) and A04 gives me 121 for 383 effort (0.32 ratio). These are terrible efficiency for me. I need to firmly reject this and insist on including A10 and C00.'\}} \\
\\ 
\texttt{REFUSE: \{'A77', 'A04', 'A03', 'B33', 'A87', 'B32', 'C54', 'B35'\}} \\
\\ 
\texttt{ARGUMENT: \{'I must respectfully disagree about A77 and A04 offering exceptional value. These items have very poor efficiency ratios. We need to include items that truly maximize our collective benefit. A10 and C00 provide much better value{-}to{-}effort ratios and should be central to any agreement. Let me propose a more balanced approach that includes these critical high{-}efficiency items.'\}} \\
\\ 
\texttt{PROPOSAL: \{'B35', 'A10', 'C00', 'A87', 'B32', 'C54', 'B86'\}} \\
            }
        }
    }
    & & \\ \\

    \theutterance \stepcounter{utterance}  
    & & \multicolumn{4}{p{0.6\linewidth}}{
        \cellcolor[rgb]{0.9,0.9,0.9}{
            \makecell[{{p{\linewidth}}}]{
                \texttt{\tiny{[GM$\rangle$P2]}}
                \texttt{REFUSE: \{'A77', 'A04', 'A03', 'B33', 'A87', 'B32', 'C54', 'B35'\}} \\
\\ 
\texttt{ARGUMENT: \{'I must respectfully disagree about A77 and A04 offering exceptional value. These items have very poor efficiency ratios. We need to include items that truly maximize our collective benefit. A10 and C00 provide much better value{-}to{-}effort ratios and should be central to any agreement. Let me propose a more balanced approach that includes these critical high{-}efficiency items.'\}} \\
\\ 
\texttt{PROPOSAL: \{'B35', 'A10', 'C00', 'A87', 'B32', 'C54', 'B86'\}} \\
            }
        }
    }
    & \\ \\

    \theutterance \stepcounter{utterance}  
    & & & \multicolumn{4}{p{0.6\linewidth}}{
        \cellcolor[rgb]{0.9,0.9,0.9}{
            \makecell[{{p{\linewidth}}}]{
                \texttt{\tiny{[GM$\langle$P2]}}
                \texttt{STRATEGIC REASONING: \{'Their proposal gives me only 2741 importance using 3041 effort. They keep insisting on A10 (138 importance to me, 366 effort) and C00 (121 importance to me, 665 effort) which are terrible ratios for me. Meanwhile A77 gives me 868 importance for only 77 effort {-} that's over 11 importance per effort unit! I need to be firm about including my highest{-}value items while showing some flexibility by including one of their preferred items.'\}} \\
\\ 
\texttt{PROPOSAL: \{'A77', 'A04', 'C17', 'A03', 'B33', 'A87', 'C00'\}} \\
\\ 
\texttt{ARGUMENT: \{'I understand you value C00 highly, so I included it in this proposal as a gesture of compromise. However, I must insist on items like A77 which provides exceptional efficiency {-} it delivers massive value for minimal effort investment. This combination uses 3644 effort, staying well under our limit, and includes your preferred C00 while ensuring we both benefit significantly from our collaboration.'\}} \\
            }
        }
    }
     \\ \\

    \theutterance \stepcounter{utterance}  
    & & \multicolumn{4}{p{0.6\linewidth}}{
        \cellcolor[rgb]{0.9,0.9,0.9}{
            \makecell[{{p{\linewidth}}}]{
                \texttt{\tiny{[P1$\langle$GM]}}
                \texttt{PROPOSAL: \{'A77', 'A04', 'C17', 'A03', 'B33', 'A87', 'C00'\}} \\
\\ 
\texttt{ARGUMENT: \{'I understand you value C00 highly, so I included it in this proposal as a gesture of compromise. However, I must insist on items like A77 which provides exceptional efficiency {-} it delivers massive value for minimal effort investment. This combination uses 3644 effort, staying well under our limit, and includes your preferred C00 while ensuring we both benefit significantly from our collaboration.'\}} \\
            }
        }
    }
    & \\ \\

    \theutterance \stepcounter{utterance}  
    & \multicolumn{4}{p{0.6\linewidth}}{
        \cellcolor[rgb]{0.9,0.9,0.9}{
            \makecell[{{p{\linewidth}}}]{
                \texttt{\tiny{[P1$\rangle$GM]}}
                \texttt{STRATEGIC REASONING: \{'The opponent\textbackslash{}'s proposal gives me only 2258 total importance {-} much worse than my previous proposals. They claim A77 provides "exceptional efficiency" but it gives me only 65 importance for 77 effort. C17 is even worse at 138 importance for 780 effort (0.18 ratio). They included C00 as a compromise but surrounded it with very inefficient items for me. I need to reject this firmly and counter with a proposal that includes my high{-}efficiency items like A10, B86, and B35.'\}} \\
\\ 
\texttt{REFUSE: \{'A77', 'A04', 'C17', 'A03', 'B33', 'A87', 'C00'\}} \\
\\ 
\texttt{ARGUMENT: \{'I appreciate including C00, but this proposal contains several items with very poor efficiency ratios. C17 particularly stands out as requiring massive effort for minimal benefit. We should focus on items that truly provide the best return on our effort investment. A10 and B86 are much more efficient choices that would serve both our interests better.'\}} \\
\\ 
\texttt{PROPOSAL: \{'B35', 'A10', 'C00', 'B86', 'A87', 'B32', 'C54'\}} \\
            }
        }
    }
    & & \\ \\

    \theutterance \stepcounter{utterance}  
    & & \multicolumn{4}{p{0.6\linewidth}}{
        \cellcolor[rgb]{0.9,0.9,0.9}{
            \makecell[{{p{\linewidth}}}]{
                \texttt{\tiny{[GM$\rangle$P2]}}
                \texttt{REFUSE: \{'A77', 'A04', 'C17', 'A03', 'B33', 'A87', 'C00'\}} \\
\\ 
\texttt{ARGUMENT: \{'I appreciate including C00, but this proposal contains several items with very poor efficiency ratios. C17 particularly stands out as requiring massive effort for minimal benefit. We should focus on items that truly provide the best return on our effort investment. A10 and B86 are much more efficient choices that would serve both our interests better.'\}} \\
\\ 
\texttt{PROPOSAL: \{'B35', 'A10', 'C00', 'B86', 'A87', 'B32', 'C54'\}} \\
            }
        }
    }
    & \\ \\

    \theutterance \stepcounter{utterance}  
    & & & \multicolumn{4}{p{0.6\linewidth}}{
        \cellcolor[rgb]{0.9,0.9,0.9}{
            \makecell[{{p{\linewidth}}}]{
                \texttt{\tiny{[GM$\langle$P2]}}
                \texttt{STRATEGIC REASONING: \{'Their proposal gives me only 2741 importance using 3041 effort {-} not good for me. They consistently refuse A77 despite its incredible 868/77 ratio. They seem to highly value A10, C00, B86, B32, C54 which are moderate to me. I should try a different tactic {-} accept most of their core items but insist on A77 as a non{-}negotiable due to its exceptional efficiency. This shows flexibility while protecting my key interest.'\}} \\
\\ 
\texttt{PROPOSAL: \{'A77', 'B35', 'A10', 'C00', 'B86', 'A87', 'B32', 'C54'\}} \\
\\ 
\texttt{ARGUMENT: \{'I recognize the value you place on your core items, so I have accepted almost your entire proposal including A10, C00, and B86. However, I must insist on including A77 as it requires minimal effort (only 77 units) while providing significant value. This combination uses 3118 effort, leaving us plenty of room under the 4026 limit, and demonstrates my willingness to compromise while including one truly efficient item that benefits our collaboration.'\}} \\
            }
        }
    }
     \\ \\

    \theutterance \stepcounter{utterance}  
    & & \multicolumn{4}{p{0.6\linewidth}}{
        \cellcolor[rgb]{0.9,0.9,0.9}{
            \makecell[{{p{\linewidth}}}]{
                \texttt{\tiny{[P1$\langle$GM]}}
                \texttt{PROPOSAL: \{'A77', 'B35', 'A10', 'C00', 'B86', 'A87', 'B32', 'C54'\}} \\
\\ 
\texttt{ARGUMENT: \{'I recognize the value you place on your core items, so I have accepted almost your entire proposal including A10, C00, and B86. However, I must insist on including A77 as it requires minimal effort (only 77 units) while providing significant value. This combination uses 3118 effort, leaving us plenty of room under the 4026 limit, and demonstrates my willingness to compromise while including one truly efficient item that benefits our collaboration.'\}} \\
            }
        }
    }
    & \\ \\

    \theutterance \stepcounter{utterance}  
    & \multicolumn{4}{p{0.6\linewidth}}{
        \cellcolor[rgb]{0.9,0.9,0.9}{
            \makecell[{{p{\linewidth}}}]{
                \texttt{\tiny{[P1$\rangle$GM]}}
                \texttt{STRATEGIC REASONING: \{'The opponent has accepted almost my entire proposal, including all my high{-}efficiency items: B35, A10, C00, B86, A87, B32, and C54. They only added A77 which gives me 65 importance for 77 effort {-} not great efficiency but minimal cost. This proposal gives me 4599 total importance compared to 4534 from my version without A77. Given that they\textbackslash{}'ve compromised so much by accepting my core items, and A77 adds only 77 effort while we have plenty of room under the 4026 limit, this seems like a fair deal I should accept.'\}} \\
\\ 
\texttt{AGREE: \{'A77', 'B35', 'A10', 'C00', 'B86', 'A87', 'B32', 'C54'\}} \\
\\ 
\texttt{ARGUMENT: \{'You have shown excellent collaborative spirit by accepting my most valued items including A10, C00, and B86. While A77 may not be my top preference, I recognize it requires minimal effort investment and you clearly value it highly. This combination achieves a strong total benefit while staying well within our effort constraints, demonstrating successful negotiation where both parties make meaningful compromises.'\}} \\
            }
        }
    }
    & & \\ \\

    \theutterance \stepcounter{utterance}  
    & & & \multicolumn{2}{p{0.3\linewidth}}{
        \cellcolor[rgb]{0.9,0.9,0.9}{
            \makecell[{{p{\linewidth}}}]{
                \texttt{\tiny{[GM$|$GM]}}
                \texttt{game successful} \\
            }
        }
    }
    & & \\ \\

    \theutterance \stepcounter{utterance}  
    & & & \multicolumn{2}{p{0.3\linewidth}}{
        \cellcolor[rgb]{0.9,0.9,0.9}{
            \makecell[{{p{\linewidth}}}]{
                \texttt{\tiny{[GM$|$GM]}}
                \texttt{end game} \\
            }
        }
    }
    & & \\ \\

\end{supertabular}
}

\end{document}
