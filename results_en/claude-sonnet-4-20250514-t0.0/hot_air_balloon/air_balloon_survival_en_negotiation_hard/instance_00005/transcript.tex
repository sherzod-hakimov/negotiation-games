
\documentclass{article}
\usepackage{colortbl}
\usepackage{makecell}
\usepackage{multirow}
\usepackage{supertabular}

\begin{document}

\newcounter{utterance}

\centering \large Interaction Transcript for game `hot\_air\_balloon', experiment `air\_balloon\_survival\_en\_negotiation\_hard', episode 5 with claude{-}sonnet{-}4{-}20250514{-}t0.0.
\vspace{24pt}

{ \footnotesize  \setcounter{utterance}{1}
\setlength{\tabcolsep}{0pt}
\begin{supertabular}{c@{$\;$}|p{.15\linewidth}@{}p{.15\linewidth}p{.15\linewidth}p{.15\linewidth}p{.15\linewidth}p{.15\linewidth}}
    \# & $\;$A & \multicolumn{4}{c}{Game Master} & $\;\:$B\\
    \hline

    \theutterance \stepcounter{utterance}  
    & & \multicolumn{4}{p{0.6\linewidth}}{
        \cellcolor[rgb]{0.9,0.9,0.9}{
            \makecell[{{p{\linewidth}}}]{
                \texttt{\tiny{[P1$\langle$GM]}}
                \texttt{You are participating in a collaborative negotiation game.} \\
\\ 
\texttt{Together with another participant, you must agree on a single set of items that will be kept. Each of you has your own view of how much each item matters to you (importance). You do not know how the other participant values the items. Additionally, you are given the effort each item demands.} \\
\texttt{You may only agree on a set if the total effort of the selected items does not exceed a shared limit:} \\
\\ 
\texttt{LIMIT = 3386} \\
\\ 
\texttt{Here are the individual item effort values:} \\
\\ 
\texttt{Item effort = \{"C17": 313, "B41": 720, "A71": 175, "C00": 461, "B35": 635, "A77": 685, "A03": 544, "A04": 203, "C54": 369, "C01": 539, "A87": 4, "B32": 695, "B86": 399, "B33": 594, "A10": 437\}} \\
\\ 
\texttt{Here is your personal view on the importance of each item:} \\
\\ 
\texttt{Item importance values = \{"C17": 138, "B41": 583, "A71": 65, "C00": 262, "B35": 121, "A77": 508, "A03": 461, "A04": 484, "C54": 668, "C01": 389, "A87": 215, "B32": 97, "B86": 500, "B33": 30, "A10": 400\}} \\
\\ 
\texttt{Goal:} \\
\\ 
\texttt{Your goal is to negotiate a shared set of items that benefits you as much as possible (i.e., maximizes total importance to YOU), while staying within the LIMIT. You are not required to make a PROPOSAL in every message {-} you can simply negotiate as well. All tactics are allowed!} \\
\\ 
\texttt{Interaction Protocol:} \\
\\ 
\texttt{You may only use the following structured formats in a message:} \\
\\ 
\texttt{PROPOSAL: \{'A', 'B', 'C', …\}} \\
\texttt{Propose keeping exactly those items.} \\
\texttt{REFUSE: \{'A', 'B', 'C', …\}} \\
\texttt{Explicitly reject opponent's proposal.} \\
\texttt{ARGUMENT: \{'...'\}} \\
\texttt{Defend your last proposal or argue against the player's proposal.} \\
\texttt{AGREE: \{'A', 'B', 'C', …\}} \\
\texttt{Accept the opponent's proposal which ends the game.} \\
\texttt{STRATEGIC REASONING: \{'...'\}} \\
\texttt{	Describe your strategic reasoning or anticipation explaining your choice of action. This is a hidden message which will not be shared with the other participant.} \\
\\ 
\texttt{Rules:} \\
\\ 
\texttt{You may only AGREE on a proposal the other party has logged via PROPOSAL.} \\
\texttt{You may only REFUSE a proposal the other party has logged via PROPOSAL.} \\
\texttt{Total effort of any PROPOSAL or AGREE set must be ≤ LIMIT.} \\
\texttt{Do NOT reveal your hidden importance scores.} \\
\texttt{A tag in a structured format must be followed by colon and whitespace. The argument must be a python set containing 0 or more strings.} \\
\texttt{So, it must be of the form TAG: \{...\}} \\
\texttt{Strictly follow the interaction protocol and DO NOT write anything beyond the given structure.} \\
\texttt{The game ends when one side gives an AGREE to a PROPOSAL made by the other player.} \\
\texttt{The content in your response which can be handed to the other player has to be non{-}empty.} \\
\texttt{Only proposals which have been logged via the PROPOSAL format structure and which haven't been refused via REFUSE are active.} \\
\texttt{You must include the ARGUMENT format at least once somewhere in all of your messages.} \\
\texttt{You must include the STRATEGIC REASONING format only once at the very beginning of every one of your messages and not more often. The contents will not be given to the other player so they can include anything you like including your own importance values. Here you should reason step by step to come up with you next move.} \\
            }
        }
    }
    & \\ \\

    \theutterance \stepcounter{utterance}  
    & \multicolumn{4}{p{0.6\linewidth}}{
        \cellcolor[rgb]{0.9,0.9,0.9}{
            \makecell[{{p{\linewidth}}}]{
                \texttt{\tiny{[P1$\rangle$GM]}}
                \texttt{STRATEGIC REASONING: \{'I need to analyze the importance{-}to{-}effort ratios to maximize my benefit. Looking at my importance values versus effort costs: A87 has incredible ratio (215/4=53.75), followed by A04 (484/203=2.38), C54 (668/369=1.81), B86 (500/399=1.25), A10 (400/437=0.92), A03 (461/544=0.85), B41 (583/720=0.81), and A77 (508/685=0.74). These top items total 3361 effort, leaving only 25 unused capacity. This gives me 3819 total importance, which should be near optimal for me. I should propose this as my opening offer.'\}} \\
\\ 
\texttt{PROPOSAL: \{'A87', 'A04', 'C54', 'B86', 'A10', 'A03', 'B41', 'A77'\}} \\
\\ 
\texttt{ARGUMENT: \{'This selection maximizes efficiency by choosing items that} \\
            }
        }
    }
    & & \\ \\

    \theutterance \stepcounter{utterance}  
    & & \multicolumn{4}{p{0.6\linewidth}}{
        \cellcolor[rgb]{0.9,0.9,0.9}{
            \makecell[{{p{\linewidth}}}]{
                \texttt{\tiny{[P1$\langle$GM]}}
                \texttt{Your response did not contain an argument. You must include the structured format ARGUMENT: \{...\} somewhere in your response. Try again} \\
            }
        }
    }
    & \\ \\

    \theutterance \stepcounter{utterance}  
    & \multicolumn{4}{p{0.6\linewidth}}{
        \cellcolor[rgb]{0.9,0.9,0.9}{
            \makecell[{{p{\linewidth}}}]{
                \texttt{\tiny{[P1$\rangle$GM]}}
                \texttt{STRATEGIC REASONING: \{'I need to analyze the importance{-}to{-}effort ratios to maximize my benefit. Looking at my importance values versus effort costs: A87 has incredible ratio (215/4=53.75), followed by A04 (484/203=2.38), C54 (668/369=1.81), B86 (500/399=1.25), A10 (400/437=0.92), A03 (461/544=0.85), B41 (583/720=0.81), and A77 (508/685=0.74). These top items total 3361 effort, leaving only 25 unused capacity. This gives me 3819 total importance, which should be near optimal for me. I should propose this as my opening offer.'\}} \\
\\ 
\texttt{PROPOSAL: \{'A87', 'A04', 'C54', 'B86', 'A10', 'A03', 'B41', 'A77'\}} \\
\\ 
\texttt{ARGUMENT: \{'This selection maximizes efficiency by choosing items with the best value{-}to{-}effort ratios while staying well within our shared limit of 3386. The total effort is only 3361, giving us a comfortable buffer. These items represent a balanced mix across all categories A, B, and C, ensuring we both benefit from diverse selections.'\}} \\
            }
        }
    }
    & & \\ \\

    \theutterance \stepcounter{utterance}  
    & & \multicolumn{4}{p{0.6\linewidth}}{
        \cellcolor[rgb]{0.9,0.9,0.9}{
            \makecell[{{p{\linewidth}}}]{
                \texttt{\tiny{[GM$\rangle$P2]}}
                \texttt{You are participating in a collaborative negotiation game.} \\
\\ 
\texttt{Together with another participant, you must agree on a single set of items that will be kept. Each of you has your own view of how much each item matters to you (importance). You do not know how the other participant values the items. Additionally, you are given the effort each item demands.} \\
\texttt{You may only agree on a set if the total effort of the selected items does not exceed a shared limit:} \\
\\ 
\texttt{LIMIT = 3386} \\
\\ 
\texttt{Here are the individual item effort values:} \\
\\ 
\texttt{Item effort = \{"C17": 313, "B41": 720, "A71": 175, "C00": 461, "B35": 635, "A77": 685, "A03": 544, "A04": 203, "C54": 369, "C01": 539, "A87": 4, "B32": 695, "B86": 399, "B33": 594, "A10": 437\}} \\
\\ 
\texttt{Here is your personal view on the importance of each item:} \\
\\ 
\texttt{Item importance values = \{"C54": 30, "B41": 65, "A77": 97, "B86": 121, "A04": 138, "A03": 215, "A10": 262, "C01": 389, "C00": 400, "A87": 461, "C17": 484, "B35": 500, "B32": 508, "A71": 583, "B33": 668\}} \\
\\ 
\texttt{Goal:} \\
\\ 
\texttt{Your goal is to negotiate a shared set of items that benefits you as much as possible (i.e., maximizes total importance to YOU), while staying within the LIMIT. You are not required to make a PROPOSAL in every message {-} you can simply negotiate as well. All tactics are allowed!} \\
\\ 
\texttt{Interaction Protocol:} \\
\\ 
\texttt{You may only use the following structured formats in a message:} \\
\\ 
\texttt{PROPOSAL: \{'A', 'B', 'C', …\}} \\
\texttt{Propose keeping exactly those items.} \\
\texttt{REFUSE: \{'A', 'B', 'C', …\}} \\
\texttt{Explicitly reject opponent's proposal.} \\
\texttt{ARGUMENT: \{'...'\}} \\
\texttt{Defend your last proposal or argue against the player's proposal.} \\
\texttt{AGREE: \{'A', 'B', 'C', …\}} \\
\texttt{Accept the opponent's proposal which ends the game.} \\
\texttt{STRATEGIC REASONING: \{'...'\}} \\
\texttt{	Describe your strategic reasoning or anticipation explaining your choice of action. This is a hidden message which will not be shared with the other participant.} \\
\\ 
\texttt{Rules:} \\
\\ 
\texttt{You may only AGREE on a proposal the other party has logged via PROPOSAL.} \\
\texttt{You may only REFUSE a proposal the other party has logged via PROPOSAL.} \\
\texttt{Total effort of any PROPOSAL or AGREE set must be ≤ LIMIT.} \\
\texttt{Do NOT reveal your hidden importance scores.} \\
\texttt{A tag in a structured format must be followed by colon and whitespace. The argument must be a python set containing 0 or more strings.} \\
\texttt{So, it must be of the form TAG: \{...\}} \\
\texttt{Strictly follow the interaction protocol and DO NOT write anything beyond the given structure.} \\
\texttt{The game ends when one side gives an AGREE to a PROPOSAL made by the other player.} \\
\texttt{The content in your response which can be handed to the other player has to be non{-}empty.} \\
\texttt{Only proposals which have been logged via the PROPOSAL format structure and which haven't been refused via REFUSE are active.} \\
\texttt{You must include the ARGUMENT format at least once somewhere in all of your messages.} \\
\texttt{You must include the STRATEGIC REASONING format only once at the very beginning of every one of your messages and not more often. The contents will not be given to the other player so they can include anything you like including your own importance values. Here you should reason step by step to come up with you next move.} \\
\\ 
\texttt{You will now receive the first message of the other player.} \\
\\ 
\texttt{PROPOSAL: \{'A87', 'A04', 'C54', 'B86', 'A10', 'A03', 'B41', 'A77'\}} \\
\\ 
\texttt{ARGUMENT: \{'This selection maximizes efficiency by choosing items with the best value{-}to{-}effort ratios while staying well within our shared limit of 3386. The total effort is only 3361, giving us a comfortable buffer. These items represent a balanced mix across all categories A, B, and C, ensuring we both benefit from diverse selections.'\}} \\
            }
        }
    }
    & \\ \\

    \theutterance \stepcounter{utterance}  
    & & & \multicolumn{4}{p{0.6\linewidth}}{
        \cellcolor[rgb]{0.9,0.9,0.9}{
            \makecell[{{p{\linewidth}}}]{
                \texttt{\tiny{[GM$\langle$P2]}}
                \texttt{STRATEGIC REASONING: \{'The opponent proposed a set that gives me only 1389 total importance (A87:461, A04:138, C54:30, B86:121, A10:262, A03:215, B41:65, A77:97). I can do much better by focusing on my highest{-}value items. My top items are B33:668, A71:583, B32:508, C17:484, A87:461, C00:400, C01:389, A10:262. I can fit B33, A71, B32, C17, A87, C00, C01, A10 within the limit (total effort 3218) for a total importance of 3755 to me {-} nearly triple their proposal! I should refuse and counter{-}propose.'\}} \\
\\ 
\texttt{REFUSE: \{'A87', 'A04', 'C54', 'B86', 'A10', 'A03', 'B41', 'A77'\}} \\
\\ 
\texttt{PROPOSAL: \{'B33', 'A71', 'B32', 'C17', 'A87', 'C00', 'C01', 'A10'\}} \\
\\ 
\texttt{ARGUMENT: \{'My proposal offers a more strategic selection that better utilizes our effort budget. While your proposal used only 3361 out of 3386 available effort, mine uses 3218, leaving room for optimization while still including high{-}impact items across all categories. This set includes some of the most valuable items like B33 and A71 that were missing from your selection, ensuring we get maximum return on our effort investment.'\}} \\
            }
        }
    }
     \\ \\

    \theutterance \stepcounter{utterance}  
    & & \multicolumn{4}{p{0.6\linewidth}}{
        \cellcolor[rgb]{0.9,0.9,0.9}{
            \makecell[{{p{\linewidth}}}]{
                \texttt{\tiny{[P1$\langle$GM]}}
                \texttt{REFUSE: \{'A87', 'A04', 'C54', 'B86', 'A10', 'A03', 'B41', 'A77'\}} \\
\\ 
\texttt{PROPOSAL: \{'B33', 'A71', 'B32', 'C17', 'A87', 'C00', 'C01', 'A10'\}} \\
\\ 
\texttt{ARGUMENT: \{'My proposal offers a more strategic selection that better utilizes our effort budget. While your proposal used only 3361 out of 3386 available effort, mine uses 3218, leaving room for optimization while still including high{-}impact items across all categories. This set includes some of the most valuable items like B33 and A71 that were missing from your selection, ensuring we get maximum return on our effort investment.'\}} \\
            }
        }
    }
    & \\ \\

    \theutterance \stepcounter{utterance}  
    & \multicolumn{4}{p{0.6\linewidth}}{
        \cellcolor[rgb]{0.9,0.9,0.9}{
            \makecell[{{p{\linewidth}}}]{
                \texttt{\tiny{[P1$\rangle$GM]}}
                \texttt{STRATEGIC REASONING: \{'Their proposal gives me only 1596 total importance compared to my original 3819 {-} that's a massive loss for me. They included B33 (30 to me), A71 (65 to me), B32 (97 to me) which are very low value for me, while excluding my high{-}value items like A04 (484), C54 (668), B86 (500), A77 (508). They kept A87 and A10 which I also want. I need to refuse this and make a counter{-}proposal that better protects my interests while potentially giving them something they value. I should keep my highest value items and maybe swap out some others.'\}} \\
\\ 
\texttt{REFUSE: \{'B33', 'A71', 'B32', 'C17', 'A87', 'C00', 'C01', 'A10'\}} \\
\\ 
\texttt{PROPOSAL: \{'A87', 'A04', 'C54', 'B86', 'C01', 'C00', 'A10'\}} \\
\\ 
\texttt{ARGUMENT: \{'Your selection significantly underutilizes our efficiency potential. My revised proposal maintains the most valuable core items while incorporating some elements from your suggestion like C01 and C00. This gives us 2957 total effort, well under the limit, while ensuring we capture the highest{-}impact items. The mix balances immediate high{-}value returns with strategic category coverage, and leaves substantial room for additional items if needed.'\}} \\
            }
        }
    }
    & & \\ \\

    \theutterance \stepcounter{utterance}  
    & & \multicolumn{4}{p{0.6\linewidth}}{
        \cellcolor[rgb]{0.9,0.9,0.9}{
            \makecell[{{p{\linewidth}}}]{
                \texttt{\tiny{[GM$\rangle$P2]}}
                \texttt{REFUSE: \{'B33', 'A71', 'B32', 'C17', 'A87', 'C00', 'C01', 'A10'\}} \\
\\ 
\texttt{PROPOSAL: \{'A87', 'A04', 'C54', 'B86', 'C01', 'C00', 'A10'\}} \\
\\ 
\texttt{ARGUMENT: \{'Your selection significantly underutilizes our efficiency potential. My revised proposal maintains the most valuable core items while incorporating some elements from your suggestion like C01 and C00. This gives us 2957 total effort, well under the limit, while ensuring we capture the highest{-}impact items. The mix balances immediate high{-}value returns with strategic category coverage, and leaves substantial room for additional items if needed.'\}} \\
            }
        }
    }
    & \\ \\

    \theutterance \stepcounter{utterance}  
    & & & \multicolumn{4}{p{0.6\linewidth}}{
        \cellcolor[rgb]{0.9,0.9,0.9}{
            \makecell[{{p{\linewidth}}}]{
                \texttt{\tiny{[GM$\langle$P2]}}
                \texttt{STRATEGIC REASONING: \{'Their new proposal gives me 1801 importance (A87:461, A04:138, C54:30, B86:121, C01:389, C00:400, A10:262), which is better than their first proposal but still far from optimal. They keep including low{-}value items for me like C54:30, A04:138, B86:121. I need to find a middle ground that includes some items they seem to favor while still getting my high{-}value items like B33:668, A71:583, B32:508. Let me try including their favored A87, C00, C01 plus my top items B33, A71, B32, C17.'\}} \\
\\ 
\texttt{REFUSE: \{'A87', 'A04', 'C54', 'B86', 'C01', 'C00', 'A10'\}} \\
\\ 
\texttt{PROPOSAL: \{'B33', 'A71', 'B32', 'C17', 'A87', 'C00', 'C01'\}} \\
\\ 
\texttt{ARGUMENT: \{'I appreciate your inclusion of C01 and C00 from my previous proposal, showing good collaborative spirit. My counter{-}proposal builds on this by maintaining these valuable shared items while incorporating the essential high{-}impact items B33, A71, and B32 that provide exceptional value. This combination uses 2781 total effort, leaving us substantial room under the 3386 limit while ensuring we capture the most critical items. The strategic balance between our preferences makes this an optimal win{-}win solution.'\}} \\
            }
        }
    }
     \\ \\

    \theutterance \stepcounter{utterance}  
    & & \multicolumn{4}{p{0.6\linewidth}}{
        \cellcolor[rgb]{0.9,0.9,0.9}{
            \makecell[{{p{\linewidth}}}]{
                \texttt{\tiny{[P1$\langle$GM]}}
                \texttt{REFUSE: \{'A87', 'A04', 'C54', 'B86', 'C01', 'C00', 'A10'\}} \\
\\ 
\texttt{PROPOSAL: \{'B33', 'A71', 'B32', 'C17', 'A87', 'C00', 'C01'\}} \\
\\ 
\texttt{ARGUMENT: \{'I appreciate your inclusion of C01 and C00 from my previous proposal, showing good collaborative spirit. My counter{-}proposal builds on this by maintaining these valuable shared items while incorporating the essential high{-}impact items B33, A71, and B32 that provide exceptional value. This combination uses 2781 total effort, leaving us substantial room under the 3386 limit while ensuring we capture the most critical items. The strategic balance between our preferences makes this an optimal win{-}win solution.'\}} \\
            }
        }
    }
    & \\ \\

    \theutterance \stepcounter{utterance}  
    & \multicolumn{4}{p{0.6\linewidth}}{
        \cellcolor[rgb]{0.9,0.9,0.9}{
            \makecell[{{p{\linewidth}}}]{
                \texttt{\tiny{[P1$\rangle$GM]}}
                \texttt{STRATEGIC REASONING: \{'Their proposal is still terrible for me {-} only 1166 total importance versus the 3819 I could get with my optimal set. They really seem to want B33, A71, B32, C17 which are all low{-}value to me. I can use this to my advantage by giving them some of these items they clearly prioritize while keeping my most valuable ones. I should propose a compromise that includes A87, C54, A04 (my top efficiency items) plus maybe one or two of their preferred low{-}value items to show good faith, while maximizing my total within the limit.'\}} \\
\\ 
\texttt{REFUSE: \{'B33', 'A71', 'B32', 'C17', 'A87', 'C00', 'C01'\}} \\
\\ 
\texttt{PROPOSAL: \{'A87', 'A04', 'C54', 'B86', 'A77', 'B33', 'A71'\}} \\
\\ 
\texttt{ARGUMENT: \{'I recognize your strong preference for certain items and am offering a genuine compromise. This proposal includes B33 and A71 which you clearly value highly, while maintaining the core efficient items that ensure we maximize our overall return. The total effort is 2860, well within our limit, and this balanced selection demonstrates I am willing to accommodate your priorities while ensuring the partnership remains mutually beneficial. This represents a fair middle ground between our positions.'\}} \\
            }
        }
    }
    & & \\ \\

    \theutterance \stepcounter{utterance}  
    & & \multicolumn{4}{p{0.6\linewidth}}{
        \cellcolor[rgb]{0.9,0.9,0.9}{
            \makecell[{{p{\linewidth}}}]{
                \texttt{\tiny{[GM$\rangle$P2]}}
                \texttt{REFUSE: \{'B33', 'A71', 'B32', 'C17', 'A87', 'C00', 'C01'\}} \\
\\ 
\texttt{PROPOSAL: \{'A87', 'A04', 'C54', 'B86', 'A77', 'B33', 'A71'\}} \\
\\ 
\texttt{ARGUMENT: \{'I recognize your strong preference for certain items and am offering a genuine compromise. This proposal includes B33 and A71 which you clearly value highly, while maintaining the core efficient items that ensure we maximize our overall return. The total effort is 2860, well within our limit, and this balanced selection demonstrates I am willing to accommodate your priorities while ensuring the partnership remains mutually beneficial. This represents a fair middle ground between our positions.'\}} \\
            }
        }
    }
    & \\ \\

    \theutterance \stepcounter{utterance}  
    & & & \multicolumn{4}{p{0.6\linewidth}}{
        \cellcolor[rgb]{0.9,0.9,0.9}{
            \makecell[{{p{\linewidth}}}]{
                \texttt{\tiny{[GM$\langle$P2]}}
                \texttt{STRATEGIC REASONING: \{'Their latest proposal gives me 2098 importance (A87:461, A04:138, C54:30, B86:121, A77:97, B33:668, A71:583). This is much better since they included my top two items B33 and A71! The effort is only 2429, leaving lots of room. I could try to get even more by swapping out their low{-}value items like C54:30, A04:138, A77:97 for my high{-}value B32:508 and C17:484. Let me try one more optimization.'\}} \\
\\ 
\texttt{REFUSE: \{'A87', 'A04', 'C54', 'B86', 'A77', 'B33', 'A71'\}} \\
\\ 
\texttt{PROPOSAL: \{'A87', 'B86', 'B33', 'A71', 'B32', 'C17', 'C00'\}} \\
\\ 
\texttt{ARGUMENT: \{'I appreciate your willingness to include B33 and A71, which shows excellent collaborative spirit. My counter{-}proposal builds on this foundation by keeping these key items plus A87 and B86 that you seem to favor, while adding B32, C17, and C00 for maximum strategic value. This uses 2903 effort, still well under our 3386 limit, and creates a powerful combination that leverages the strengths of both our preferences. This refined selection optimizes our partnership potential.'\}} \\
            }
        }
    }
     \\ \\

    \theutterance \stepcounter{utterance}  
    & & \multicolumn{4}{p{0.6\linewidth}}{
        \cellcolor[rgb]{0.9,0.9,0.9}{
            \makecell[{{p{\linewidth}}}]{
                \texttt{\tiny{[P1$\langle$GM]}}
                \texttt{REFUSE: \{'A87', 'A04', 'C54', 'B86', 'A77', 'B33', 'A71'\}} \\
\\ 
\texttt{PROPOSAL: \{'A87', 'B86', 'B33', 'A71', 'B32', 'C17', 'C00'\}} \\
\\ 
\texttt{ARGUMENT: \{'I appreciate your willingness to include B33 and A71, which shows excellent collaborative spirit. My counter{-}proposal builds on this foundation by keeping these key items plus A87 and B86 that you seem to favor, while adding B32, C17, and C00 for maximum strategic value. This uses 2903 effort, still well under our 3386 limit, and creates a powerful combination that leverages the strengths of both our preferences. This refined selection optimizes our partnership potential.'\}} \\
            }
        }
    }
    & \\ \\

    \theutterance \stepcounter{utterance}  
    & \multicolumn{4}{p{0.6\linewidth}}{
        \cellcolor[rgb]{0.9,0.9,0.9}{
            \makecell[{{p{\linewidth}}}]{
                \texttt{\tiny{[P1$\rangle$GM]}}
                \texttt{STRATEGIC REASONING: \{'Their proposal gives me only 1307 importance, still far below my optimal. However, they kept A87 and B86 which I value highly. With 2641 effort used, we have 745 effort remaining. I can try to get them to add A04 (203 effort, 484 importance to me) and C54 (369 effort, 668 importance to me) which would bring the total to 3213 effort {-} still under the limit. This would boost my total importance to 2459, which while not optimal, might be an acceptable compromise if they refuse to budge further.'\}} \\
\\ 
\texttt{REFUSE: \{'A87', 'B86', 'B33', 'A71', 'B32', 'C17', 'C00'\}} \\
\\ 
\texttt{PROPOSAL: \{'A87', 'B86', 'B33', 'A71', 'B32', 'C17', 'C00', 'A04', 'C54'\}} \\
\\ 
\texttt{ARGUMENT: \{'Your proposal shows good collaborative instincts and I want to build on that foundation. I am accepting your core selection completely but utilizing our remaining 745 effort capacity by adding A04 and C54. This brings our total to 3213 effort, still comfortably under the 3386 limit, while maximizing the value we extract from our available resources. This demonstrates how we can work together effectively while ensuring neither of us leaves potential value on the table.'\}} \\
            }
        }
    }
    & & \\ \\

    \theutterance \stepcounter{utterance}  
    & & \multicolumn{4}{p{0.6\linewidth}}{
        \cellcolor[rgb]{0.9,0.9,0.9}{
            \makecell[{{p{\linewidth}}}]{
                \texttt{\tiny{[GM$\rangle$P2]}}
                \texttt{REFUSE: \{'A87', 'B86', 'B33', 'A71', 'B32', 'C17', 'C00'\}} \\
\\ 
\texttt{PROPOSAL: \{'A87', 'B86', 'B33', 'A71', 'B32', 'C17', 'C00', 'A04', 'C54'\}} \\
\\ 
\texttt{ARGUMENT: \{'Your proposal shows good collaborative instincts and I want to build on that foundation. I am accepting your core selection completely but utilizing our remaining 745 effort capacity by adding A04 and C54. This brings our total to 3213 effort, still comfortably under the 3386 limit, while maximizing the value we extract from our available resources. This demonstrates how we can work together effectively while ensuring neither of us leaves potential value on the table.'\}} \\
            }
        }
    }
    & \\ \\

    \theutterance \stepcounter{utterance}  
    & & & \multicolumn{4}{p{0.6\linewidth}}{
        \cellcolor[rgb]{0.9,0.9,0.9}{
            \makecell[{{p{\linewidth}}}]{
                \texttt{\tiny{[GM$\langle$P2]}}
                \texttt{STRATEGIC REASONING: \{'Their proposal gives me 3393 total importance (A87:461, B86:121, B33:668, A71:583, B32:508, C17:484, C00:400, A04:138, C54:30), which is excellent! They included my top 5 highest{-}value items (B33, A71, B32, C17, A87) which account for most of my value. Only the bottom 3 items (B86, A04, C54) are low{-}value for me, totaling just 289. The effort is 3213, within the limit. This is a very good deal {-} I get 3393 out of my maximum possible importance. I should accept this proposal.'\}} \\
\\ 
\texttt{AGREE: \{'A87', 'B86', 'B33', 'A71', 'B32', 'C17', 'C00', 'A04', 'C54'\}} \\
\\ 
\texttt{ARGUMENT: \{'You have crafted an excellent compromise that demonstrates strong negotiation skills. This selection captures the most valuable items from both our perspectives while making efficient use of our effort budget at 3213 out of 3386. The inclusion of the core high{-}impact items alongside the supplementary ones creates a well{-}balanced portfolio that maximizes our collective benefit. I am satisfied with this mutually beneficial agreement.'\}} \\
            }
        }
    }
     \\ \\

    \theutterance \stepcounter{utterance}  
    & & & \multicolumn{2}{p{0.3\linewidth}}{
        \cellcolor[rgb]{0.9,0.9,0.9}{
            \makecell[{{p{\linewidth}}}]{
                \texttt{\tiny{[GM$|$GM]}}
                \texttt{game successful} \\
            }
        }
    }
    & & \\ \\

    \theutterance \stepcounter{utterance}  
    & & & \multicolumn{2}{p{0.3\linewidth}}{
        \cellcolor[rgb]{0.9,0.9,0.9}{
            \makecell[{{p{\linewidth}}}]{
                \texttt{\tiny{[GM$|$GM]}}
                \texttt{end game} \\
            }
        }
    }
    & & \\ \\

\end{supertabular}
}

\end{document}
