
\documentclass{article}
\usepackage{colortbl}
\usepackage{makecell}
\usepackage{multirow}
\usepackage{supertabular}

\begin{document}

\newcounter{utterance}

\twocolumn

{ \footnotesize  \setcounter{utterance}{1}
\setlength{\tabcolsep}{0pt}
\begin{supertabular}{c@{$\;$}|p{.15\linewidth}@{}p{.15\linewidth}p{.15\linewidth}p{.15\linewidth}p{.15\linewidth}p{.15\linewidth}}

    \# & $\;$A & \multicolumn{4}{c}{Game Master} & $\;\:$B\\
    \hline 

    \theutterance \stepcounter{utterance}  

    & & \multicolumn{4}{p{0.6\linewidth}}{\cellcolor[rgb]{0.9,0.9,0.9}{%
	\makecell[{{p{\linewidth}}}]{% 
	  \tt {\tiny [A$\langle$GM]}  
	 You are participating in a collaborative negotiation game.\\ \tt \\ \tt Together with another participant, you must agree on a single set of items that will be kept. Each of you has your own view of how much each item matters to you (importance). You do not know how the other participant values the items. Additionally, you are given the effort each item demands.\\ \tt You may only agree on a set if the total effort of the selected items does not exceed a shared limit:\\ \tt \\ \tt LIMIT = 5138\\ \tt \\ \tt Here are the individual item effort values:\\ \tt \\ \tt Item effort = {"C17": 903, "B41": 757, "A71": 850, "C00": 746, "B35": 524, "A77": 204, "A03": 946, "A04": 473, "C54": 616, "C01": 855, "A87": 530, "B32": 419, "B86": 960, "B33": 763, "A10": 730}\\ \tt \\ \tt Here is your personal view on the importance of each item:\\ \tt \\ \tt Item importance values = {"C17": 138, "B41": 583, "A71": 868, "C00": 822, "B35": 783, "A77": 65, "A03": 262, "A04": 121, "C54": 508, "C01": 780, "A87": 461, "B32": 484, "B86": 668, "B33": 389, "A10": 808}\\ \tt \\ \tt Goal:\\ \tt \\ \tt Your goal is to negotiate a shared set of items that benefits you as much as possible (i.e., maximizes total importance to YOU), while staying within the LIMIT. You are not required to make a PROPOSAL in every message - you can simply negotiate as well. All tactics are allowed!\\ \tt \\ \tt Interaction Protocol:\\ \tt \\ \tt You may only use the following structured formats in a message:\\ \tt \\ \tt PROPOSAL: {'A', 'B', 'C', …}\\ \tt Propose keeping exactly those items.\\ \tt REFUSE: {'A', 'B', 'C', …}\\ \tt Explicitly reject opponent's proposal.\\ \tt ARGUMENT: {'...'}\\ \tt Defend your last proposal or argue against the player's proposal.\\ \tt AGREE: {'A', 'B', 'C', …}\\ \tt Accept the opponent's proposal which ends the game.\\ \tt STRATEGIC REASONING: {'...'}\\ \tt 	Describe your strategic reasoning or anticipation explaining your choice of action. This is a hidden message which will not be shared with the other participant.\\ \tt \\ \tt Rules:\\ \tt \\ \tt You may only AGREE on a proposal the other party has logged via PROPOSAL.\\ \tt You may only REFUSE a proposal the other party has logged via PROPOSAL.\\ \tt Total effort of any PROPOSAL or AGREE set must be ≤ LIMIT.\\ \tt Do NOT reveal your hidden importance scores.\\ \tt A tag in a structured format must be followed by colon and whitespace. The argument must be a python set containing 0 or more strings.\\ \tt So, it must be of the form TAG: {...}\\ \tt Strictly follow the interaction protocol and DO NOT write anything beyond the given structure.\\ \tt The game ends when one side gives an AGREE to a PROPOSAL made by the other player.\\ \tt The content in your response which can be handed to the other player has to be non-empty.\\ \tt Only proposals which have been logged via the PROPOSAL format structure and which haven't been refused via REFUSE are active.\\ \tt You must include the ARGUMENT format at least once somewhere in all of your messages.\\ \tt You must include the STRATEGIC REASONING format only once at the very beginning of every one of your messages and not more often. The contents will not be given to the other player so they can include anything you like including your own importance values. Here you should reason step by step to come up with you next move. 
	  } 
	   } 
	   } 
	 & \\ 
 

    \theutterance \stepcounter{utterance}  

    & \multicolumn{4}{p{0.6\linewidth}}{\cellcolor[rgb]{0.8,1,0.9}{%
	\makecell[{{p{\linewidth}}}]{% 
	  \tt {\tiny [A$\rangle$GM]}  
	 STRATEGIC REASONING: {'I need to maximize my total importance while staying within the effort limit of 5138. Let me calculate importance-to-effort ratios: B35 (1.494), B32 (1.155), A10 (1.107), C00 (1.102), A71 (1.021) are my top ratios. Starting with these high-efficiency items: B35(524)+B32(419)+A10(730)+C00(746)+A71(850)=3269 effort, giving me 3765 importance. I have 1869 capacity left. Adding C01(855) gives 4124 total effort and 4545 importance. Adding A87(530) gives 4654 effort and 5006 importance. Finally adding A04(473) gives 5127 total effort and 5127 importance, using almost all capacity efficiently.'}\\ \tt \\ \tt PROPOSAL: {'B35', 'B32', 'A10', 'C00', 'A71', 'C01', 'A87', 'A04'}\\ \tt \\ \tt ARGUMENT: {'This selection efficiently uses 5127 out of our 5138 effort limit, including several high-value items. The combination balances different item categories and maximizes our shared benefit within the constraint.'} 
	  } 
	   } 
	   } 
	 & & \\ 
 

    \theutterance \stepcounter{utterance}  

    & & \multicolumn{4}{p{0.6\linewidth}}{\cellcolor[rgb]{0.9,0.9,0.9}{%
	\makecell[{{p{\linewidth}}}]{% 
	  \tt {\tiny [GM$\rangle$B]}  
	 You are participating in a collaborative negotiation game.\\ \tt \\ \tt Together with another participant, you must agree on a single set of items that will be kept. Each of you has your own view of how much each item matters to you (importance). You do not know how the other participant values the items. Additionally, you are given the effort each item demands.\\ \tt You may only agree on a set if the total effort of the selected items does not exceed a shared limit:\\ \tt \\ \tt LIMIT = 5138\\ \tt \\ \tt Here are the individual item effort values:\\ \tt \\ \tt Item effort = {"C17": 903, "B41": 757, "A71": 850, "C00": 746, "B35": 524, "A77": 204, "A03": 946, "A04": 473, "C54": 616, "C01": 855, "A87": 530, "B32": 419, "B86": 960, "B33": 763, "A10": 730}\\ \tt \\ \tt Here is your personal view on the importance of each item:\\ \tt \\ \tt Item importance values = {"A71": 65, "C00": 121, "A10": 138, "B35": 262, "C01": 389, "B86": 461, "B41": 484, "C54": 508, "B32": 583, "A87": 668, "B33": 780, "A03": 783, "C17": 808, "A04": 822, "A77": 868}\\ \tt \\ \tt Goal:\\ \tt \\ \tt Your goal is to negotiate a shared set of items that benefits you as much as possible (i.e., maximizes total importance to YOU), while staying within the LIMIT. You are not required to make a PROPOSAL in every message - you can simply negotiate as well. All tactics are allowed!\\ \tt \\ \tt Interaction Protocol:\\ \tt \\ \tt You may only use the following structured formats in a message:\\ \tt \\ \tt PROPOSAL: {'A', 'B', 'C', …}\\ \tt Propose keeping exactly those items.\\ \tt REFUSE: {'A', 'B', 'C', …}\\ \tt Explicitly reject opponent's proposal.\\ \tt ARGUMENT: {'...'}\\ \tt Defend your last proposal or argue against the player's proposal.\\ \tt AGREE: {'A', 'B', 'C', …}\\ \tt Accept the opponent's proposal which ends the game.\\ \tt STRATEGIC REASONING: {'...'}\\ \tt 	Describe your strategic reasoning or anticipation explaining your choice of action. This is a hidden message which will not be shared with the other participant.\\ \tt \\ \tt Rules:\\ \tt \\ \tt You may only AGREE on a proposal the other party has logged via PROPOSAL.\\ \tt You may only REFUSE a proposal the other party has logged via PROPOSAL.\\ \tt Total effort of any PROPOSAL or AGREE set must be ≤ LIMIT.\\ \tt Do NOT reveal your hidden importance scores.\\ \tt A tag in a structured format must be followed by colon and whitespace. The argument must be a python set containing 0 or more strings.\\ \tt So, it must be of the form TAG: {...}\\ \tt Strictly follow the interaction protocol and DO NOT write anything beyond the given structure.\\ \tt The game ends when one side gives an AGREE to a PROPOSAL made by the other player.\\ \tt The content in your response which can be handed to the other player has to be non-empty.\\ \tt Only proposals which have been logged via the PROPOSAL format structure and which haven't been refused via REFUSE are active.\\ \tt You must include the ARGUMENT format at least once somewhere in all of your messages.\\ \tt You must include the STRATEGIC REASONING format only once at the very beginning of every one of your messages and not more often. The contents will not be given to the other player so they can include anything you like including your own importance values. Here you should reason step by step to come up with you next move.\\ \tt \\ \tt You will now receive the first message of the other player.\\ \tt \\ \tt PROPOSAL: {'B35', 'B32', 'A10', 'C00', 'A71', 'C01', 'A87', 'A04'}\\ \tt \\ \tt ARGUMENT: {'This selection efficiently uses 5127 out of our 5138 effort limit, including several high-value items. The combination balances different item categories and maximizes our shared benefit within the constraint.'} 
	  } 
	   } 
	   } 
	 & \\ 
 

    \theutterance \stepcounter{utterance}  

    & & & \multicolumn{4}{p{0.6\linewidth}}{\cellcolor[rgb]{1,0.85,0.72}{%
	\makecell[{{p{\linewidth}}}]{% 
	  \tt {\tiny [GM$\langle$B]}  
	 STRATEGIC REASONING: {'I need to analyze the opponent\'s proposal against my own importance values. Their proposal gives me B35(262), B32(583), A10(138), C00(121), A71(65), C01(389), A87(668), A04(822) for total importance of 3048. However, I can do much better by including my highest-value items like A77(868 importance, only 204 effort), C17(808 importance), A03(783 importance), and B33(780 importance). I should counter with a proposal that maximizes my benefit while staying under the effort limit.'}\\ \tt \\ \tt REFUSE: {'B35', 'B32', 'A10', 'C00', 'A71', 'C01', 'A87', 'A04'}\\ \tt \\ \tt PROPOSAL: {'A77', 'A04', 'C17', 'B33', 'A87', 'B 
	  } 
	   } 
	   } 
	  \\ 
 

    \theutterance \stepcounter{utterance}  

    & & \multicolumn{4}{p{0.6\linewidth}}{\cellcolor[rgb]{0.9,0.9,0.9}{%
	\makecell[{{p{\linewidth}}}]{% 
	  \tt {\tiny [GM$\rangle$B]}  
	 Your response did not contain an argument. You must include the structured format ARGUMENT: {...} somewhere in your response. Try again 
	  } 
	   } 
	   } 
	 & \\ 
 

    \theutterance \stepcounter{utterance}  

    & & & \multicolumn{4}{p{0.6\linewidth}}{\cellcolor[rgb]{1,0.85,0.72}{%
	\makecell[{{p{\linewidth}}}]{% 
	  \tt {\tiny [GM$\langle$B]}  
	 STRATEGIC REASONING: {'I need to analyze the opponent\'s proposal against my own importance values. Their proposal gives me B35(262), B32(583), A10(138), C00(121), A71(65), C01(389), A87(668), A04(822) for total importance of 3048. However, I can do much better by including my highest-value items like A77(868 importance, only 204 effort), C17(808 importance), A03(783 importance), and B33(780 importance). I should counter with a proposal that maximizes my benefit while staying under the effort limit.'}\\ \tt \\ \tt REFUSE: {'B35', 'B32', 'A10', 'C00', 'A71', 'C01', 'A87', 'A04'}\\ \tt \\ \tt PROPOSAL: {'A77', 'A04', 'C17', 'B33', 'A87'}\\ \tt \\ \tt ARGUMENT: {'My proposal uses only 2873 effort out of our 5138 limit, leaving substantial room for additional valuable items. This combination prioritizes efficiency by including A77 which has very low effort cost relative to value, while still incorporating several substantial items that can benefit both parties.'} 
	  } 
	   } 
	   } 
	  \\ 
 

    \theutterance \stepcounter{utterance}  

    & & \multicolumn{4}{p{0.6\linewidth}}{\cellcolor[rgb]{0.9,0.9,0.9}{%
	\makecell[{{p{\linewidth}}}]{% 
	  \tt {\tiny [A$\langle$GM]}  
	 REFUSE: {'B35', 'B32', 'A10', 'C00', 'A71', 'C01', 'A87', 'A04'}\\ \tt \\ \tt PROPOSAL: {'A77', 'A04', 'C17', 'B33', 'A87'}\\ \tt \\ \tt ARGUMENT: {'My proposal uses only 2873 effort out of our 5138 limit, leaving substantial room for additional valuable items. This combination prioritizes efficiency by including A77 which has very low effort cost relative to value, while still incorporating several substantial items that can benefit both parties.'} 
	  } 
	   } 
	   } 
	 & \\ 
 

    \theutterance \stepcounter{utterance}  

    & \multicolumn{4}{p{0.6\linewidth}}{\cellcolor[rgb]{0.8,1,0.9}{%
	\makecell[{{p{\linewidth}}}]{% 
	  \tt {\tiny [A$\rangle$GM]}  
	 STRATEGIC REASONING: {'Their proposal gives me only 1174 importance using just 2873 effort (56% of capacity) - this is very inefficient for me. They included A77 (very low value to me), A04, C17 (low value), B33, and A87. Since they kept A87 and A04 from my original proposal, these might be important to them. I should refuse and counter with a compromise that includes some items they seem to want (A87, A04) while adding my high-efficiency items (B35, A10, C00, A71) to better utilize our capacity and increase my value significantly.'}\\ \tt \\ \tt REFUSE: {'A77', 'A04', 'C17', 'B33', 'A87'}\\ \tt \\ \tt PROPOSAL: {'A87', 'A04', 'B35', 'A10', 'C00', 'A71'}\\ \tt \\ \tt ARGUMENT: {'I kept two items from your proposal (A87 and A04) as a compromise, but added high-value items to better utilize our 5138 effort limit. This proposal uses 3853 effort (75% capacity) compared to your 2873 (56%), giving us much more value for the available resources while still including items you seem to prioritize.'} 
	  } 
	   } 
	   } 
	 & & \\ 
 

    \theutterance \stepcounter{utterance}  

    & & \multicolumn{4}{p{0.6\linewidth}}{\cellcolor[rgb]{0.9,0.9,0.9}{%
	\makecell[{{p{\linewidth}}}]{% 
	  \tt {\tiny [GM$\rangle$B]}  
	 REFUSE: {'A77', 'A04', 'C17', 'B33', 'A87'}\\ \tt \\ \tt PROPOSAL: {'A87', 'A04', 'B35', 'A10', 'C00', 'A71'}\\ \tt \\ \tt ARGUMENT: {'I kept two items from your proposal (A87 and A04) as a compromise, but added high-value items to better utilize our 5138 effort limit. This proposal uses 3853 effort (75% capacity) compared to your 2873 (56%), giving us much more value for the available resources while still including items you seem to prioritize.'} 
	  } 
	   } 
	   } 
	 & \\ 
 

    \theutterance \stepcounter{utterance}  

    & & & \multicolumn{4}{p{0.6\linewidth}}{\cellcolor[rgb]{1,0.85,0.72}{%
	\makecell[{{p{\linewidth}}}]{% 
	  \tt {\tiny [GM$\langle$B]}  
	 STRATEGIC REASONING: {'The opponent\'s proposal gives me only 2076 total importance (A87:668, A04:822, B35:262, A10:138, C00:121, A71:65) compared to my previous proposal which gave me 3946 importance. They\'re including low-value items for me like A71, A10, and C00. I should refuse this and make a counter-proposal that uses more of the effort budget while prioritizing my high-importance items. I can add B32 to my previous proposal to increase the total effort usage and make it more appealing.'}\\ \tt \\ \tt REFUSE: {'A87', 'A04', 'B35', 'A10', 'C00', 'A71'}\\ \tt \\ \tt PROPOSAL: {'A77', 'A04', 'C17', 'B33', 'A87', 'B32'}\\ \tt \\ \tt ARGUMENT: {'This proposal uses 3292 out of 5138 effort (64% capacity) and includes six valuable items across all categories. I\'ve added B32 to my previous proposal to increase our resource utilization while maintaining a balanced selection that should provide good value for both parties.'} 
	  } 
	   } 
	   } 
	  \\ 
 

    \theutterance \stepcounter{utterance}  

    & & \multicolumn{4}{p{0.6\linewidth}}{\cellcolor[rgb]{0.9,0.9,0.9}{%
	\makecell[{{p{\linewidth}}}]{% 
	  \tt {\tiny [A$\langle$GM]}  
	 REFUSE: {'A87', 'A04', 'B35', 'A10', 'C00', 'A71'}\\ \tt \\ \tt PROPOSAL: {'A77', 'A04', 'C17', 'B33', 'A87', 'B32'}\\ \tt \\ \tt ARGUMENT: {'This proposal uses 3292 out of 5138 effort (64% capacity) and includes six valuable items across all categories. I\'ve added B32 to my previous proposal to increase our resource utilization while maintaining a balanced selection that should provide good value for both parties.'} 
	  } 
	   } 
	   } 
	 & \\ 
 

    \theutterance \stepcounter{utterance}  

    & \multicolumn{4}{p{0.6\linewidth}}{\cellcolor[rgb]{0.8,1,0.9}{%
	\makecell[{{p{\linewidth}}}]{% 
	  \tt {\tiny [A$\rangle$GM]}  
	 STRATEGIC REASONING: {'Their proposal gives me only 1658 importance using 3292 effort - still quite inefficient. They consistently include A77 (65 importance, 204 effort) and C17 (138 importance, 903 effort) which are very poor value for me. Since they seem attached to A04, B33, A87, and now B32, I should try a compromise keeping these four items they prefer but replacing A77 and C17 with my high-efficiency items B35 and C00. This would give me: 121+389+461+484+783+822=3060 importance vs their 1658, while using similar effort capacity.'}\\ \tt \\ \tt REFUSE: {'A77', 'A04', 'C17', 'B33', 'A87', 'B32'}\\ \tt \\ \tt PROPOSAL: {'A04', 'B33', 'A87', 'B32', 'B35', 'C00'}\\ \tt \\ \tt ARGUMENT: {'I kept four items from your proposal that you seem to value (A04, B33, A87, B32) but replaced A77 and C17 with B35 and C00 to better utilize our effort capacity. This uses 3939 effort (77% of our limit) compared to your 3292 (64%), providing significantly more value while respecting items you consistently include in your proposals.'} 
	  } 
	   } 
	   } 
	 & & \\ 
 

    \theutterance \stepcounter{utterance}  

    & & \multicolumn{4}{p{0.6\linewidth}}{\cellcolor[rgb]{0.9,0.9,0.9}{%
	\makecell[{{p{\linewidth}}}]{% 
	  \tt {\tiny [GM$\rangle$B]}  
	 REFUSE: {'A77', 'A04', 'C17', 'B33', 'A87', 'B32'}\\ \tt \\ \tt PROPOSAL: {'A04', 'B33', 'A87', 'B32', 'B35', 'C00'}\\ \tt \\ \tt ARGUMENT: {'I kept four items from your proposal that you seem to value (A04, B33, A87, B32) but replaced A77 and C17 with B35 and C00 to better utilize our effort capacity. This uses 3939 effort (77% of our limit) compared to your 3292 (64%), providing significantly more value while respecting items you consistently include in your proposals.'} 
	  } 
	   } 
	   } 
	 & \\ 
 

    \theutterance \stepcounter{utterance}  

    & & & \multicolumn{4}{p{0.6\linewidth}}{\cellcolor[rgb]{1,0.85,0.72}{%
	\makecell[{{p{\linewidth}}}]{% 
	  \tt {\tiny [GM$\langle$B]}  
	 STRATEGIC REASONING: {'The opponent\'s proposal gives me 3236 importance using 3455 effort. They keep rejecting my highest-value items A77 and C17. However, I notice they haven\'t seen A03 yet, which has 783 importance for me. I should try to include A03 along with other high-value items like A04, B33, and A87. I need to find a combination that uses more of our effort budget while maximizing my benefit.'}\\ \tt \\ \tt REFUSE: {'A04', 'B33', 'A87', 'B32', 'B35', 'C00'}\\ \tt \\ \tt PROPOSAL: {'A04', 'A03', 'B33', 'A87', 'B32', 'A77'}\\ \tt \\ \tt ARGUMENT: {'This proposal uses 4335 effort (84% of our capacity) compared to your 3455 (67%), significantly better utilizing our shared resources. I\'ve kept the core items you seem to value like A04, B33, A87, and B32, but replaced the lower-efficiency items with A03 and A77 to maximize our combined benefit within the effort constraints.'} 
	  } 
	   } 
	   } 
	  \\ 
 

    \theutterance \stepcounter{utterance}  

    & & \multicolumn{4}{p{0.6\linewidth}}{\cellcolor[rgb]{0.9,0.9,0.9}{%
	\makecell[{{p{\linewidth}}}]{% 
	  \tt {\tiny [A$\langle$GM]}  
	 REFUSE: {'A04', 'B33', 'A87', 'B32', 'B35', 'C00'}\\ \tt \\ \tt PROPOSAL: {'A04', 'A03', 'B33', 'A87', 'B32', 'A77'}\\ \tt \\ \tt ARGUMENT: {'This proposal uses 4335 effort (84% of our capacity) compared to your 3455 (67%), significantly better utilizing our shared resources. I\'ve kept the core items you seem to value like A04, B33, A87, and B32, but replaced the lower-efficiency items with A03 and A77 to maximize our combined benefit within the effort constraints.'} 
	  } 
	   } 
	   } 
	 & \\ 
 

    \theutterance \stepcounter{utterance}  

    & \multicolumn{4}{p{0.6\linewidth}}{\cellcolor[rgb]{0.8,1,0.9}{%
	\makecell[{{p{\linewidth}}}]{% 
	  \tt {\tiny [A$\rangle$GM]}  
	 STRATEGIC REASONING: {'Their proposal gives me only 1782 importance using 3335 effort (they miscalculated as 4335). They keep insisting on A77 (terrible 65 importance for 204 effort) and now A03 (poor 262 importance for 946 effort). Since they seem committed to A04, A87, B32, B33, let me try keeping these four core items they want (2185 effort, 1455 importance for me) and add my four most efficient remaining items: B35, A10, C00, A71 (2850 effort, 3856 importance). Total: 5035 effort (within limit), 5311 importance for me. This is a strong compromise.'}\\ \tt \\ \tt REFUSE: {'A04', 'A03', 'B33', 'A87', 'B32', 'A77'}\\ \tt \\ \tt PROPOSAL: {'A04', 'B33', 'A87', 'B32', 'B35', 'A10', 'C00', 'A71'}\\ \tt \\ \tt ARGUMENT: {'I kept the four core items you consistently propose (A04, B33, A87, B32) showing good faith, and added four high-efficiency items to maximize our shared value. This uses 5035 effort (98% of our 5138 capacity) compared to your 3335 (65%), much better resource utilization. This represents a strong compromise between both our preferences while maximizing total benefit.'} 
	  } 
	   } 
	   } 
	 & & \\ 
 

    \theutterance \stepcounter{utterance}  

    & & \multicolumn{4}{p{0.6\linewidth}}{\cellcolor[rgb]{0.9,0.9,0.9}{%
	\makecell[{{p{\linewidth}}}]{% 
	  \tt {\tiny [GM$\rangle$B]}  
	 REFUSE: {'A04', 'A03', 'B33', 'A87', 'B32', 'A77'}\\ \tt \\ \tt PROPOSAL: {'A04', 'B33', 'A87', 'B32', 'B35', 'A10', 'C00', 'A71'}\\ \tt \\ \tt ARGUMENT: {'I kept the four core items you consistently propose (A04, B33, A87, B32) showing good faith, and added four high-efficiency items to maximize our shared value. This uses 5035 effort (98% of our 5138 capacity) compared to your 3335 (65%), much better resource utilization. This represents a strong compromise between both our preferences while maximizing total benefit.'} 
	  } 
	   } 
	   } 
	 & \\ 
 

    \theutterance \stepcounter{utterance}  

    & & & \multicolumn{4}{p{0.6\linewidth}}{\cellcolor[rgb]{1,0.85,0.72}{%
	\makecell[{{p{\linewidth}}}]{% 
	  \tt {\tiny [GM$\langle$B]}  
	 STRATEGIC REASONING: {'The opponent\'s proposal gives me 3439 importance using 5035 effort (98% capacity). While this uses our resources efficiently, I can still do better by replacing some lower-value items with higher-value ones for me. Since they keep rejecting A77 and C17, maybe I can try including C54 and C01 instead, which I haven\'t pushed for yet. Let me try a counter-proposal that maintains some of their preferred items but optimizes for my benefit.'}\\ \tt \\ \tt REFUSE: {'A04', 'B33', 'A87', 'B32', 'B35', 'A10', 'C00', 'A71'}\\ \tt \\ \tt PROPOSAL: {'A04', 'B33', 'A87', 'B32', 'C54', 'C01', 'B41'}\\ \tt \\ \tt ARGUMENT: {'I kept your four consistently proposed core items (A04, B33, A87, B32) to show compromise, but replaced the four lower-efficiency items with C54, C01, and B41. This uses 4820 effort (94% of our capacity) while providing better overall value distribution. The selection maintains balance across item categories and represents a fair middle ground between our different preferences.'} 
	  } 
	   } 
	   } 
	  \\ 
 

    \theutterance \stepcounter{utterance}  

    & & \multicolumn{4}{p{0.6\linewidth}}{\cellcolor[rgb]{0.9,0.9,0.9}{%
	\makecell[{{p{\linewidth}}}]{% 
	  \tt {\tiny [A$\langle$GM]}  
	 REFUSE: {'A04', 'B33', 'A87', 'B32', 'B35', 'A10', 'C00', 'A71'}\\ \tt \\ \tt PROPOSAL: {'A04', 'B33', 'A87', 'B32', 'C54', 'C01', 'B41'}\\ \tt \\ \tt ARGUMENT: {'I kept your four consistently proposed core items (A04, B33, A87, B32) to show compromise, but replaced the four lower-efficiency items with C54, C01, and B41. This uses 4820 effort (94% of our capacity) while providing better overall value distribution. The selection maintains balance across item categories and represents a fair middle ground between our different preferences.'} 
	  } 
	   } 
	   } 
	 & \\ 
 

    \theutterance \stepcounter{utterance}  

    & \multicolumn{4}{p{0.6\linewidth}}{\cellcolor[rgb]{0.8,1,0.9}{%
	\makecell[{{p{\linewidth}}}]{% 
	  \tt {\tiny [A$\rangle$GM]}  
	 STRATEGIC REASONING: {'Their proposal gives me 3326 importance using 4413 effort (they miscalculated as 4820). They keep the same 4 core items they always include (A04, B33, A87, B32) which gives me 1455 importance for 2185 effort. With remaining 2953 capacity, I could add B35+A10+C00 (2000 effort, 2413 importance) for total 4185 effort and 3868 importance - much better than their 3326. This keeps their preferred core items while optimizing the remaining selections for maximum value.'}\\ \tt \\ \tt REFUSE: {'A04', 'B33', 'A87', 'B32', 'C54', 'C01', 'B41'}\\ \tt \\ \tt PROPOSAL: {'A04', 'B33', 'A87', 'B32', 'B35', 'A10', 'C00'}\\ \tt \\ \tt ARGUMENT: {'I maintained your four consistently requested core items (A04, B33, A87, B32) as a significant compromise, then optimized the remaining capacity by adding three high-efficiency items (B35, A10, C00). This uses 4185 effort (81% of our limit) and provides excellent value-to-effort ratio while respecting your clear preferences for those four specific items.'} 
	  } 
	   } 
	   } 
	 & & \\ 
 

    \theutterance \stepcounter{utterance}  

    & & \multicolumn{4}{p{0.6\linewidth}}{\cellcolor[rgb]{0.9,0.9,0.9}{%
	\makecell[{{p{\linewidth}}}]{% 
	  \tt {\tiny [GM$\rangle$B]}  
	 REFUSE: {'A04', 'B33', 'A87', 'B32', 'C54', 'C01', 'B41'}\\ \tt \\ \tt PROPOSAL: {'A04', 'B33', 'A87', 'B32', 'B35', 'A10', 'C00'}\\ \tt \\ \tt ARGUMENT: {'I maintained your four consistently requested core items (A04, B33, A87, B32) as a significant compromise, then optimized the remaining capacity by adding three high-efficiency items (B35, A10, C00). This uses 4185 effort (81% of our limit) and provides excellent value-to-effort ratio while respecting your clear preferences for those four specific items.'} 
	  } 
	   } 
	   } 
	 & \\ 
 

    \theutterance \stepcounter{utterance}  

    & & & \multicolumn{4}{p{0.6\linewidth}}{\cellcolor[rgb]{1,0.85,0.72}{%
	\makecell[{{p{\linewidth}}}]{% 
	  \tt {\tiny [GM$\langle$B]}  
	 STRATEGIC REASONING: {'The opponent\'s proposal gives me 3374 importance using 4185 effort (81% capacity). They seem very committed to those four core items and keep rejecting my highest-value items. With 953 effort remaining, I could add A77 (204 effort) to make better use of our resources. This might be my last chance to get A77 included before I need to consider accepting their offer.'}\\ \tt \\ \tt REFUSE: {'A04', 'B33', 'A87', 'B32', 'B35', 'A10', 'C00'}\\ \tt \\ \tt PROPOSAL: {'A04', 'B33', 'A87', 'B32', 'B35', 'A10', 'C00', 'A77'}\\ \tt \\ \tt ARGUMENT: {'I accepted all seven items from your proposal and simply added A77 to better utilize our remaining effort capacity. This uses 4389 effort (85% of our limit) compared to your 4185 (81%), providing additional value while keeping every single item you specifically requested. This represents maximum compromise on my part while optimizing our shared resources.'} 
	  } 
	   } 
	   } 
	  \\ 
 

    \theutterance \stepcounter{utterance}  

    & & \multicolumn{4}{p{0.6\linewidth}}{\cellcolor[rgb]{0.9,0.9,0.9}{%
	\makecell[{{p{\linewidth}}}]{% 
	  \tt {\tiny [A$\langle$GM]}  
	 REFUSE: {'A04', 'B33', 'A87', 'B32', 'B35', 'A10', 'C00'}\\ \tt \\ \tt PROPOSAL: {'A04', 'B33', 'A87', 'B32', 'B35', 'A10', 'C00', 'A77'}\\ \tt \\ \tt ARGUMENT: {'I accepted all seven items from your proposal and simply added A77 to better utilize our remaining effort capacity. This uses 4389 effort (85% of our limit) compared to your 4185 (81%), providing additional value while keeping every single item you specifically requested. This represents maximum compromise on my part while optimizing our shared resources.'} 
	  } 
	   } 
	   } 
	 & \\ 
 

    \theutterance \stepcounter{utterance}  

    & \multicolumn{4}{p{0.6\linewidth}}{\cellcolor[rgb]{0.8,1,0.9}{%
	\makecell[{{p{\linewidth}}}]{% 
	  \tt {\tiny [A$\rangle$GM]}  
	 STRATEGIC REASONING: {'Their proposal includes all 7 items I requested plus A77. This gives me 3933 importance using 4389 effort (85% capacity). While A77 is low-efficiency for me (65 importance, 204 effort), I am getting all my high-value items: B35(783), A10(808), C00(822) plus the compromise items. This represents 3933/5138 = 76% of maximum possible importance if I used all effort optimally. Given their consistent resistance to my proposals and their willingness to include all my preferred items, this may be the best deal I can achieve. Accepting this would end negotiations favorably.'}\\ \tt \\ \tt AGREE: {'A04', 'B33', 'A87', 'B32', 'B35', 'A10', 'C00', 'A77'}\\ \tt \\ \tt ARGUMENT: {'You have shown excellent compromise by including all seven items I specifically requested while adding A77 to optimize our resource utilization. This achieves 85% capacity usage and represents a fair balance between both our preferences. I accept this proposal as it maximizes our combined benefit within the effort constraints.'} 
	  } 
	   } 
	   } 
	 & & \\ 
 

    \theutterance \stepcounter{utterance}  

    & & & \multicolumn{2}{p{0.3\linewidth}}{\cellcolor[rgb]{0.95,0.95,0.95}{%
	\makecell[{{p{\linewidth}}}]{% 
	  \tt {\tiny [GM$|$GM]}  
	 game successful 
	  } 
	   } 
	   } 
	 & & \\ 
 

    \theutterance \stepcounter{utterance}  

    & & & \multicolumn{2}{p{0.3\linewidth}}{\cellcolor[rgb]{0.95,0.95,0.95}{%
	\makecell[{{p{\linewidth}}}]{% 
	  \tt {\tiny [GM$|$GM]}  
	 end game 
	  } 
	   } 
	   } 
	 & & \\ 
 

\end{supertabular}
}

\end{document}
