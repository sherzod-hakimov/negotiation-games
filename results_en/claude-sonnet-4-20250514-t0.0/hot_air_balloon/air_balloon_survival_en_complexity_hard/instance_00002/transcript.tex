
\documentclass{article}
\usepackage{colortbl}
\usepackage{makecell}
\usepackage{multirow}
\usepackage{supertabular}

\begin{document}

\newcounter{utterance}

\twocolumn

{ \footnotesize  \setcounter{utterance}{1}
\setlength{\tabcolsep}{0pt}
\begin{supertabular}{c@{$\;$}|p{.15\linewidth}@{}p{.15\linewidth}p{.15\linewidth}p{.15\linewidth}p{.15\linewidth}p{.15\linewidth}}

    \# & $\;$A & \multicolumn{4}{c}{Game Master} & $\;\:$B\\
    \hline 

    \theutterance \stepcounter{utterance}  

    & & \multicolumn{4}{p{0.6\linewidth}}{\cellcolor[rgb]{0.9,0.9,0.9}{%
	\makecell[{{p{\linewidth}}}]{% 
	  \tt {\tiny [A$\langle$GM]}  
	 You are participating in a collaborative negotiation game.\\ \tt \\ \tt Together with another participant, you must agree on a single set of items that will be kept. Each of you has your own view of how much each item matters to you (importance). You do not know how the other participant values the items. Additionally, you are given the effort each item demands.\\ \tt You may only agree on a set if the total effort of the selected items does not exceed a shared limit:\\ \tt \\ \tt LIMIT = 9322\\ \tt \\ \tt Here are the individual item effort values:\\ \tt \\ \tt Item effort = {"A08": 380, "B95": 448, "A39": 568, "A65": 351, "B81": 896, "B34": 821, "A28": 526, "C82": 502, "B69": 326, "A87": 962, "B64": 619, "B15": 115, "C79": 598, "B16": 662, "C53": 300, "C83": 830, "C10": 562, "A97": 679, "A25": 283, "C84": 440, "A71": 12, "B23": 863, "B96": 319, "C51": 771, "C17": 89, "B19": 655, "C72": 504, "A94": 118, "A46": 514, "B43": 226, "B37": 892, "A88": 621, "A96": 766, "C60": 659, "A75": 768}\\ \tt \\ \tt Here is your personal view on the importance of each item:\\ \tt \\ \tt Item importance values = {"A08": 476, "B95": 544, "A39": 664, "A65": 447, "B81": 992, "B34": 917, "A28": 622, "C82": 598, "B69": 422, "A87": 1058, "B64": 715, "B15": 211, "C79": 694, "B16": 758, "C53": 396, "C83": 926, "C10": 658, "A97": 775, "A25": 379, "C84": 536, "A71": 108, "B23": 959, "B96": 415, "C51": 867, "C17": 185, "B19": 751, "C72": 600, "A94": 214, "A46": 610, "B43": 322, "B37": 988, "A88": 717, "A96": 862, "C60": 755, "A75": 864}\\ \tt \\ \tt Goal:\\ \tt \\ \tt Your goal is to negotiate a shared set of items that benefits you as much as possible (i.e., maximizes total importance to YOU), while staying within the LIMIT. You are not required to make a PROPOSAL in every message - you can simply negotiate as well. All tactics are allowed!\\ \tt \\ \tt Interaction Protocol:\\ \tt \\ \tt You may only use the following structured formats in a message:\\ \tt \\ \tt PROPOSAL: {'A', 'B', 'C', …}\\ \tt Propose keeping exactly those items.\\ \tt REFUSE: {'A', 'B', 'C', …}\\ \tt Explicitly reject opponent's proposal.\\ \tt ARGUMENT: {'...'}\\ \tt Defend your last proposal or argue against the player's proposal.\\ \tt AGREE: {'A', 'B', 'C', …}\\ \tt Accept the opponent's proposal which ends the game.\\ \tt STRATEGIC REASONING: {'...'}\\ \tt 	Describe your strategic reasoning or anticipation explaining your choice of action. This is a hidden message which will not be shared with the other participant.\\ \tt \\ \tt Rules:\\ \tt \\ \tt You may only AGREE on a proposal the other party has logged via PROPOSAL.\\ \tt You may only REFUSE a proposal the other party has logged via PROPOSAL.\\ \tt Total effort of any PROPOSAL or AGREE set must be ≤ LIMIT.\\ \tt Do NOT reveal your hidden importance scores.\\ \tt A tag in a structured format must be followed by colon and whitespace. The argument must be a python set containing 0 or more strings.\\ \tt So, it must be of the form TAG: {...}\\ \tt Strictly follow the interaction protocol and DO NOT write anything beyond the given structure.\\ \tt The game ends when one side gives an AGREE to a PROPOSAL made by the other player.\\ \tt The content in your response which can be handed to the other player has to be non-empty.\\ \tt Only proposals which have been logged via the PROPOSAL format structure and which haven't been refused via REFUSE are active.\\ \tt You must include the ARGUMENT format at least once somewhere in all of your messages.\\ \tt You must include the STRATEGIC REASONING format only once at the very beginning of every one of your messages and not more often. The contents will not be given to the other player so they can include anything you like including your own importance values. Here you should reason step by step to come up with you next move. 
	  } 
	   } 
	   } 
	 & \\ 
 

    \theutterance \stepcounter{utterance}  

    & \multicolumn{4}{p{0.6\linewidth}}{\cellcolor[rgb]{0.8,1,0.9}{%
	\makecell[{{p{\linewidth}}}]{% 
	  \tt {\tiny [A$\rangle$GM]}  
	 STRATEGIC REASONING: {'I need to analyze the efficiency of each item by calculating importance/effort ratios to identify the best value items. The highest efficiency items are A71 (9.00), C17 (2.08), B15 (1.83), A94 (1.81), and B43 (1.42). I should start with these and then add high importance items like A87 (1058), B37 (988), B81 (992), B23 (959), C83 (926), and B34 (917). Let me create an initial proposal that maximizes my value while staying under the 9322 limit.'}\\ \tt \\ \tt PROPOSAL: {'A71', 'C17', 'B15', 'A94', 'B43', 'A87', 'B37', 'B81', 'B23', 'C83', 'B34', 'A39', 'A28', 'B64', 'C79', 'B16', 'A97', 'C51', 'B19', 'C60', 'A75', 'A88', 'A96', 'A46', 'C72', 'C82', 'B69', 'A65', 'A08', 'C84', 'B96', 'C53', 'A25'}\\ \tt \\ \tt ARGUMENT: {'This proposal includes the highest efficiency items like A71, C17, B15, and A94 which provide excellent value, combined with several high-importance items. The total effort is 8980, well within our limit of 9322, leaving room for potential adjustments while maximizing overall value.'} 
	  } 
	   } 
	   } 
	 & & \\ 
 

    \theutterance \stepcounter{utterance}  

    & & \multicolumn{4}{p{0.6\linewidth}}{\cellcolor[rgb]{0.9,0.9,0.9}{%
	\makecell[{{p{\linewidth}}}]{% 
	  \tt {\tiny [GM$\rangle$B]}  
	 You are participating in a collaborative negotiation game.\\ \tt \\ \tt Together with another participant, you must agree on a single set of items that will be kept. Each of you has your own view of how much each item matters to you (importance). You do not know how the other participant values the items. Additionally, you are given the effort each item demands.\\ \tt You may only agree on a set if the total effort of the selected items does not exceed a shared limit:\\ \tt \\ \tt LIMIT = 9322\\ \tt \\ \tt Here are the individual item effort values:\\ \tt \\ \tt Item effort = {"A08": 380, "B95": 448, "A39": 568, "A65": 351, "B81": 896, "B34": 821, "A28": 526, "C82": 502, "B69": 326, "A87": 962, "B64": 619, "B15": 115, "C79": 598, "B16": 662, "C53": 300, "C83": 830, "C10": 562, "A97": 679, "A25": 283, "C84": 440, "A71": 12, "B23": 863, "B96": 319, "C51": 771, "C17": 89, "B19": 655, "C72": 504, "A94": 118, "A46": 514, "B43": 226, "B37": 892, "A88": 621, "A96": 766, "C60": 659, "A75": 768}\\ \tt \\ \tt Here is your personal view on the importance of each item:\\ \tt \\ \tt Item importance values = {"A08": 476, "B95": 544, "A39": 664, "A65": 447, "B81": 992, "B34": 917, "A28": 622, "C82": 598, "B69": 422, "A87": 1058, "B64": 715, "B15": 211, "C79": 694, "B16": 758, "C53": 396, "C83": 926, "C10": 658, "A97": 775, "A25": 379, "C84": 536, "A71": 108, "B23": 959, "B96": 415, "C51": 867, "C17": 185, "B19": 751, "C72": 600, "A94": 214, "A46": 610, "B43": 322, "B37": 988, "A88": 717, "A96": 862, "C60": 755, "A75": 864}\\ \tt \\ \tt Goal:\\ \tt \\ \tt Your goal is to negotiate a shared set of items that benefits you as much as possible (i.e., maximizes total importance to YOU), while staying within the LIMIT. You are not required to make a PROPOSAL in every message - you can simply negotiate as well. All tactics are allowed!\\ \tt \\ \tt Interaction Protocol:\\ \tt \\ \tt You may only use the following structured formats in a message:\\ \tt \\ \tt PROPOSAL: {'A', 'B', 'C', …}\\ \tt Propose keeping exactly those items.\\ \tt REFUSE: {'A', 'B', 'C', …}\\ \tt Explicitly reject opponent's proposal.\\ \tt ARGUMENT: {'...'}\\ \tt Defend your last proposal or argue against the player's proposal.\\ \tt AGREE: {'A', 'B', 'C', …}\\ \tt Accept the opponent's proposal which ends the game.\\ \tt STRATEGIC REASONING: {'...'}\\ \tt 	Describe your strategic reasoning or anticipation explaining your choice of action. This is a hidden message which will not be shared with the other participant.\\ \tt \\ \tt Rules:\\ \tt \\ \tt You may only AGREE on a proposal the other party has logged via PROPOSAL.\\ \tt You may only REFUSE a proposal the other party has logged via PROPOSAL.\\ \tt Total effort of any PROPOSAL or AGREE set must be ≤ LIMIT.\\ \tt Do NOT reveal your hidden importance scores.\\ \tt A tag in a structured format must be followed by colon and whitespace. The argument must be a python set containing 0 or more strings.\\ \tt So, it must be of the form TAG: {...}\\ \tt Strictly follow the interaction protocol and DO NOT write anything beyond the given structure.\\ \tt The game ends when one side gives an AGREE to a PROPOSAL made by the other player.\\ \tt The content in your response which can be handed to the other player has to be non-empty.\\ \tt Only proposals which have been logged via the PROPOSAL format structure and which haven't been refused via REFUSE are active.\\ \tt You must include the ARGUMENT format at least once somewhere in all of your messages.\\ \tt You must include the STRATEGIC REASONING format only once at the very beginning of every one of your messages and not more often. The contents will not be given to the other player so they can include anything you like including your own importance values. Here you should reason step by step to come up with you next move.\\ \tt \\ \tt You will now receive the first message of the other player.\\ \tt \\ \tt PROPOSAL: {'A71', 'C17', 'B15', 'A94', 'B43', 'A87', 'B37', 'B81', 'B23', 'C83', 'B34', 'A39', 'A28', 'B64', 'C79', 'B16', 'A97', 'C51', 'B19', 'C60', 'A75', 'A88', 'A96', 'A46', 'C72', 'C82', 'B69', 'A65', 'A08', 'C84', 'B96', 'C53', 'A25'}\\ \tt \\ \tt ARGUMENT: {'This proposal includes the highest efficiency items like A71, C17, B15, and A94 which provide excellent value, combined with several high-importance items. The total effort is 8980, well within our limit of 9322, leaving room for potential adjustments while maximizing overall value.'} 
	  } 
	   } 
	   } 
	 & \\ 
 

    \theutterance \stepcounter{utterance}  

    & & & \multicolumn{4}{p{0.6\linewidth}}{\cellcolor[rgb]{1,0.85,0.72}{%
	\makecell[{{p{\linewidth}}}]{% 
	  \tt {\tiny [GM$\langle$B]}  
	 STRATEGIC REASONING: {'The opponent proposed keeping almost all items (33 out of 35), excluding only B95 and C10. However, I need to verify the effort calculation - with my calculations, this proposal would require 17, 
	  } 
	   } 
	   } 
	  \\ 
 

    \theutterance \stepcounter{utterance}  

    & & \multicolumn{4}{p{0.6\linewidth}}{\cellcolor[rgb]{0.9,0.9,0.9}{%
	\makecell[{{p{\linewidth}}}]{% 
	  \tt {\tiny [GM$\rangle$B]}  
	 Your response did not start with the proper strategic reasoning tag at the very beginning of your response. The very first structured format must be of the form STRATEGIC REASONING: {...}. Try again. 
	  } 
	   } 
	   } 
	 & \\ 
 

    \theutterance \stepcounter{utterance}  

    & & & \multicolumn{4}{p{0.6\linewidth}}{\cellcolor[rgb]{1,0.85,0.72}{%
	\makecell[{{p{\linewidth}}}]{% 
	  \tt {\tiny [GM$\langle$B]}  
	 STRATEGIC REASONING: {'The opponent proposed keeping 33 out of 35 items, but this violates the effort constraint. Let me calculate: the total effort would be approximately 15,570, which far exceeds our limit of 9322. This is either an error or a test. I should refuse this proposal and counter with a more realistic selection focusing on items with high importance-to-effort ratios for me. Key high-value items for me include A87 (1058 importance, 962 effort), B37 (988 importance, 892 effort), B81 (992 importance, 896 effort), but I need to be more selective to stay within limits.'}\\ \tt \\ \tt REFUSE: {'A71', 'C17', 'B15', 'A94', 'B43', 'A87', 'B37', 'B81', 'B23', 'C83', 'B34', 'A39', 'A28', 'B64', 'C79', 'B16', 'A97', 'C51', 'B19', 'C60', 'A75', 'A88', 'A96', 'A46', 'C72', 'C82', 'B69', 'A65', 'A08', 'C84', 'B96', 'C53', 'A25'}\\ \tt \\ \tt ARGUMENT: {'Your proposal totals approximately 15,570 effort, which significantly exceeds our limit of 9322. We need to be more selective and focus on the most valuable items while staying within constraints.'}\\ \tt \\ \tt PROPOSAL: {'A87', 'B37', 'B81', 'B23', 'C83', 'B34', 'A39', 'B64', 'C79', 'B16', 'A97', 'C51', 'A75', 'A96'} 
	  } 
	   } 
	   } 
	  \\ 
 

    \theutterance \stepcounter{utterance}  

    & & \multicolumn{4}{p{0.6\linewidth}}{\cellcolor[rgb]{0.9,0.9,0.9}{%
	\makecell[{{p{\linewidth}}}]{% 
	  \tt {\tiny [A$\langle$GM]}  
	 REFUSE: {'A71', 'C17', 'B15', 'A94', 'B43', 'A87', 'B37', 'B81', 'B23', 'C83', 'B34', 'A39', 'A28', 'B64', 'C79', 'B16', 'A97', 'C51', 'B19', 'C60', 'A75', 'A88', 'A96', 'A46', 'C72', 'C82', 'B69', 'A65', 'A08', 'C84', 'B96', 'C53', 'A25'}\\ \tt \\ \tt ARGUMENT: {'Your proposal totals approximately 15,570 effort, which significantly exceeds our limit of 9322. We need to be more selective and focus on the most valuable items while staying within constraints.'}\\ \tt \\ \tt PROPOSAL: {'A87', 'B37', 'B81', 'B23', 'C83', 'B34', 'A39', 'B64', 'C79', 'B16', 'A97', 'C51', 'A75', 'A96'} 
	  } 
	   } 
	   } 
	 & \\ 
 

    \theutterance \stepcounter{utterance}  

    & \multicolumn{4}{p{0.6\linewidth}}{\cellcolor[rgb]{0.8,1,0.9}{%
	\makecell[{{p{\linewidth}}}]{% 
	  \tt {\tiny [A$\rangle$GM]}  
	 STRATEGIC REASONING: {'The other player correctly identified my calculation error - my proposal was way over the limit. However, their counter-proposal also exceeds the limit at 10,695 effort. I need to refuse this and propose a more realistic set. I should focus on the highest importance items that fit within the constraint, prioritizing items like A87 (1058 importance), B37 (988), B81 (992), B23 (959), C83 (926), and B34 (917) while being more selective.'}\\ \tt \\ \tt REFUSE: {'A87', 'B37', 'B81', 'B23', 'C83', 'B34', 'A39', 'B64', 'C79', 'B16', 'A97', 'C51', 'A75', 'A96'}\\ \tt \\ \tt ARGUMENT: {'Your proposal totals 10,695 effort, which also exceeds our limit of 9322. We need to be more selective. Let me propose a more feasible set that stays within our constraints.'}\\ \tt \\ \tt PROPOSAL: {'A87', 'B37', 'B81', 'B23', 'C83', 'B34', 'A39', 'B64', 'C79', 'A71', 'C17', 'B15'} 
	  } 
	   } 
	   } 
	 & & \\ 
 

    \theutterance \stepcounter{utterance}  

    & & \multicolumn{4}{p{0.6\linewidth}}{\cellcolor[rgb]{0.9,0.9,0.9}{%
	\makecell[{{p{\linewidth}}}]{% 
	  \tt {\tiny [GM$\rangle$B]}  
	 REFUSE: {'A87', 'B37', 'B81', 'B23', 'C83', 'B34', 'A39', 'B64', 'C79', 'B16', 'A97', 'C51', 'A75', 'A96'}\\ \tt \\ \tt ARGUMENT: {'Your proposal totals 10,695 effort, which also exceeds our limit of 9322. We need to be more selective. Let me propose a more feasible set that stays within our constraints.'}\\ \tt \\ \tt PROPOSAL: {'A87', 'B37', 'B81', 'B23', 'C83', 'B34', 'A39', 'B64', 'C79', 'A71', 'C17', 'B15'} 
	  } 
	   } 
	   } 
	 & \\ 
 

    \theutterance \stepcounter{utterance}  

    & & & \multicolumn{4}{p{0.6\linewidth}}{\cellcolor[rgb]{1,0.85,0.72}{%
	\makecell[{{p{\linewidth}}}]{% 
	  \tt {\tiny [GM$\langle$B]}  
	 STRATEGIC REASONING: {'The opponent has proposed 12 items with total effort of 6,265, well within our limit. This gives us 3,057 effort remaining. The proposal includes many of my highest-value items like A87, B37, B81, B23, C83, and B34. However, I could potentially add more valuable items with the remaining budget. Items like B16 (758 importance, 662 effort), A97 (775 importance, 679 effort), or C51 (867 importance, 771 effort) could fit within the remaining space. I should counter-propose to include some additional high-value items.'}\\ \tt \\ \tt REFUSE: {'A87', 'B37', 'B81', 'B23', 'C83', 'B34', 'A39', 'B64', 'C79', 'A71', 'C17', 'B15'}\\ \tt \\ \tt ARGUMENT: {'Your proposal is within limits and includes good items, but we have 3,057 effort remaining that we could use for additional valuable items. Let me suggest adding a few more high-value options.'}\\ \tt \\ \tt PROPOSAL: {'A87', 'B37', 'B81', 'B23', 'C83', 'B34', 'A39', 'B64', 'C79', 'B16', 'A97', 'C51', 'A71', 'C17', 'B15'} 
	  } 
	   } 
	   } 
	  \\ 
 

    \theutterance \stepcounter{utterance}  

    & & \multicolumn{4}{p{0.6\linewidth}}{\cellcolor[rgb]{0.9,0.9,0.9}{%
	\makecell[{{p{\linewidth}}}]{% 
	  \tt {\tiny [A$\langle$GM]}  
	 REFUSE: {'A87', 'B37', 'B81', 'B23', 'C83', 'B34', 'A39', 'B64', 'C79', 'A71', 'C17', 'B15'}\\ \tt \\ \tt ARGUMENT: {'Your proposal is within limits and includes good items, but we have 3,057 effort remaining that we could use for additional valuable items. Let me suggest adding a few more high-value options.'}\\ \tt \\ \tt PROPOSAL: {'A87', 'B37', 'B81', 'B23', 'C83', 'B34', 'A39', 'B64', 'C79', 'B16', 'A97', 'C51', 'A71', 'C17', 'B15'} 
	  } 
	   } 
	   } 
	 & \\ 
 

    \theutterance \stepcounter{utterance}  

    & \multicolumn{4}{p{0.6\linewidth}}{\cellcolor[rgb]{0.8,1,0.9}{%
	\makecell[{{p{\linewidth}}}]{% 
	  \tt {\tiny [A$\rangle$GM]}  
	 STRATEGIC REASONING: {'The opponent's proposal totals 9,877 effort, which exceeds our limit of 9,322 by 555. I need to refuse this and suggest a more conservative approach. I should focus on keeping the absolute highest importance items to me while staying safely under the limit. The core high-value items are A87, B37, B81, B23, C83, and B34, which together total 5,264 effort, leaving 4,058 for additional items.'}\\ \tt \\ \tt REFUSE: {'A87', 'B37', 'B81', 'B23', 'C83', 'B34', 'A39', 'B64', 'C79', 'B16', 'A97', 'C51', 'A71', 'C17', 'B15'}\\ \tt \\ \tt ARGUMENT: {'Your proposal totals 9,877 effort, which exceeds our limit of 9,322. We need to remove some items to stay within constraints while maximizing value.'}\\ \tt \\ \tt PROPOSAL: {'A87', 'B37', 'B81', 'B23', 'C83', 'B34', 'A39', 'B64', 'C79', 'A97', 'C51', 'A71', 'C17', 'B15', 'A94'} 
	  } 
	   } 
	   } 
	 & & \\ 
 

    \theutterance \stepcounter{utterance}  

    & & \multicolumn{4}{p{0.6\linewidth}}{\cellcolor[rgb]{0.9,0.9,0.9}{%
	\makecell[{{p{\linewidth}}}]{% 
	  \tt {\tiny [GM$\rangle$B]}  
	 REFUSE: {'A87', 'B37', 'B81', 'B23', 'C83', 'B34', 'A39', 'B64', 'C79', 'B16', 'A97', 'C51', 'A71', 'C17', 'B15'}\\ \tt \\ \tt ARGUMENT: {'Your proposal totals 9,877 effort, which exceeds our limit of 9,322. We need to remove some items to stay within constraints while maximizing value.'}\\ \tt \\ \tt PROPOSAL: {'A87', 'B37', 'B81', 'B23', 'C83', 'B34', 'A39', 'B64', 'C79', 'A97', 'C51', 'A71', 'C17', 'B15', 'A94'} 
	  } 
	   } 
	   } 
	 & \\ 
 

    \theutterance \stepcounter{utterance}  

    & & & \multicolumn{4}{p{0.6\linewidth}}{\cellcolor[rgb]{1,0.85,0.72}{%
	\makecell[{{p{\linewidth}}}]{% 
	  \tt {\tiny [GM$\langle$B]}  
	 STRATEGIC REASONING: {'The opponent\'s proposal totals 8,833 effort, leaving us 489 effort remaining within our 9,322 limit. This proposal includes most of my highest-value items like A87 (1058), B37 (988), B81 (992), B23 (959), C83 (926), and B34 (917). The total importance to me would be 10,373. With 489 effort remaining, I could potentially add one more valuable item like A65 (447 importance, 351 effort) or C84 (536 importance, 440 effort). Let me try to add A65 to maximize value while staying within limits.'}\\ \tt \\ \tt REFUSE: {'A87', 'B37', 'B81', 'B23', 'C83', 'B34', 'A39', 'B64', 'C79', 'A97', 'C51', 'A71', 'C17', 'B15', 'A94'}\\ \tt \\ \tt ARGUMENT: {'Your proposal is good and within limits with 489 effort remaining. We can squeeze in one more valuable item to maximize our return on the available capacity.'}\\ \tt \\ \tt PROPOSAL: {'A87', 'B37', 'B81', 'B23', 'C83', 'B34', 'A39', 'B64', 'C79', 'A97', 'C51', 'A71', 'C17', 'B15', 'A94', 'A65'} 
	  } 
	   } 
	   } 
	  \\ 
 

    \theutterance \stepcounter{utterance}  

    & & \multicolumn{4}{p{0.6\linewidth}}{\cellcolor[rgb]{0.9,0.9,0.9}{%
	\makecell[{{p{\linewidth}}}]{% 
	  \tt {\tiny [A$\langle$GM]}  
	 REFUSE: {'A87', 'B37', 'B81', 'B23', 'C83', 'B34', 'A39', 'B64', 'C79', 'A97', 'C51', 'A71', 'C17', 'B15', 'A94'}\\ \tt \\ \tt ARGUMENT: {'Your proposal is good and within limits with 489 effort remaining. We can squeeze in one more valuable item to maximize our return on the available capacity.'}\\ \tt \\ \tt PROPOSAL: {'A87', 'B37', 'B81', 'B23', 'C83', 'B34', 'A39', 'B64', 'C79', 'A97', 'C51', 'A71', 'C17', 'B15', 'A94', 'A65'} 
	  } 
	   } 
	   } 
	 & \\ 
 

    \theutterance \stepcounter{utterance}  

    & \multicolumn{4}{p{0.6\linewidth}}{\cellcolor[rgb]{0.8,1,0.9}{%
	\makecell[{{p{\linewidth}}}]{% 
	  \tt {\tiny [A$\rangle$GM]}  
	 STRATEGIC REASONING: {'The opponent\'s proposal totals 9,184 effort, which is well within our 9,322 limit with only 138 remaining. Looking at my importance values, this set gives me a total importance of 10,720, which includes all my highest priority items: A87 (1058), B37 (988), B81 (992), B23 (959), C83 (926), and B34 (917). This is an excellent outcome for me and uses our capacity very efficiently. I should accept this proposal.'}\\ \tt \\ \tt AGREE: {'A87', 'B37', 'B81', 'B23', 'C83', 'B34', 'A39', 'B64', 'C79', 'A97', 'C51', 'A71', 'C17', 'B15', 'A94', 'A65'}\\ \tt \\ \tt ARGUMENT: {'This proposal efficiently uses our capacity at 9,184 effort out of 9,322 limit while including many high-value items. It represents a good balance that maximizes our collective benefit within the constraints.'} 
	  } 
	   } 
	   } 
	 & & \\ 
 

    \theutterance \stepcounter{utterance}  

    & & & \multicolumn{2}{p{0.3\linewidth}}{\cellcolor[rgb]{0.95,0.95,0.95}{%
	\makecell[{{p{\linewidth}}}]{% 
	  \tt {\tiny [GM$|$GM]}  
	 game successful 
	  } 
	   } 
	   } 
	 & & \\ 
 

    \theutterance \stepcounter{utterance}  

    & & & \multicolumn{2}{p{0.3\linewidth}}{\cellcolor[rgb]{0.95,0.95,0.95}{%
	\makecell[{{p{\linewidth}}}]{% 
	  \tt {\tiny [GM$|$GM]}  
	 end game 
	  } 
	   } 
	   } 
	 & & \\ 
 

\end{supertabular}
}

\end{document}
