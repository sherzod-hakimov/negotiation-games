
\documentclass{article}
\usepackage{colortbl}
\usepackage{makecell}
\usepackage{multirow}
\usepackage{supertabular}

\begin{document}

\newcounter{utterance}

\twocolumn

{ \footnotesize  \setcounter{utterance}{1}
\setlength{\tabcolsep}{0pt}
\begin{supertabular}{c@{$\;$}|p{.15\linewidth}@{}p{.15\linewidth}p{.15\linewidth}p{.15\linewidth}p{.15\linewidth}p{.15\linewidth}}

    \# & $\;$A & \multicolumn{4}{c}{Game Master} & $\;\:$B\\
    \hline 

    \theutterance \stepcounter{utterance}  

    & & \multicolumn{4}{p{0.6\linewidth}}{\cellcolor[rgb]{0.9,0.9,0.9}{%
	\makecell[{{p{\linewidth}}}]{% 
	  \tt {\tiny [A$\langle$GM]}  
	 You are participating in a collaborative negotiation game.\\ \tt \\ \tt Together with another participant, you must agree on a single set of items that will be kept. Each of you has your own view of how much each item matters to you (importance). You do not know how the other participant values the items. Additionally, you are given the effort each item demands.\\ \tt You may only agree on a set if the total effort of the selected items does not exceed a shared limit:\\ \tt \\ \tt LIMIT = 10073\\ \tt \\ \tt Here are the individual item effort values:\\ \tt \\ \tt Item effort = {"A08": 132, "B95": 800, "A39": 708, "A65": 302, "B81": 983, "B34": 420, "A28": 623, "C82": 349, "B69": 239, "A87": 28, "B64": 986, "B15": 660, "C79": 718, "B16": 707, "C53": 188, "C83": 774, "C10": 785, "A97": 775, "A25": 517, "C84": 768, "A71": 588, "B23": 657, "B96": 375, "C51": 310, "C17": 301, "B19": 388, "C72": 431, "A94": 950, "A46": 540, "B43": 997, "B37": 473, "A88": 831, "A96": 888, "C60": 879, "A75": 76}\\ \tt \\ \tt Here is your personal view on the importance of each item:\\ \tt \\ \tt Item importance values = {"A08": 231, "B95": 899, "A39": 807, "A65": 401, "B81": 1082, "B34": 519, "A28": 722, "C82": 448, "B69": 338, "A87": 127, "B64": 1085, "B15": 759, "C79": 817, "B16": 806, "C53": 287, "C83": 873, "C10": 884, "A97": 874, "A25": 616, "C84": 867, "A71": 687, "B23": 756, "B96": 474, "C51": 409, "C17": 400, "B19": 487, "C72": 530, "A94": 1049, "A46": 639, "B43": 1096, "B37": 572, "A88": 930, "A96": 987, "C60": 978, "A75": 175}\\ \tt \\ \tt Goal:\\ \tt \\ \tt Your goal is to negotiate a shared set of items that benefits you as much as possible (i.e., maximizes total importance to YOU), while staying within the LIMIT. You are not required to make a PROPOSAL in every message - you can simply negotiate as well. All tactics are allowed!\\ \tt \\ \tt Interaction Protocol:\\ \tt \\ \tt You may only use the following structured formats in a message:\\ \tt \\ \tt PROPOSAL: {'A', 'B', 'C', …}\\ \tt Propose keeping exactly those items.\\ \tt REFUSE: {'A', 'B', 'C', …}\\ \tt Explicitly reject opponent's proposal.\\ \tt ARGUMENT: {'...'}\\ \tt Defend your last proposal or argue against the player's proposal.\\ \tt AGREE: {'A', 'B', 'C', …}\\ \tt Accept the opponent's proposal which ends the game.\\ \tt STRATEGIC REASONING: {'...'}\\ \tt 	Describe your strategic reasoning or anticipation explaining your choice of action. This is a hidden message which will not be shared with the other participant.\\ \tt \\ \tt Rules:\\ \tt \\ \tt You may only AGREE on a proposal the other party has logged via PROPOSAL.\\ \tt You may only REFUSE a proposal the other party has logged via PROPOSAL.\\ \tt Total effort of any PROPOSAL or AGREE set must be ≤ LIMIT.\\ \tt Do NOT reveal your hidden importance scores.\\ \tt A tag in a structured format must be followed by colon and whitespace. The argument must be a python set containing 0 or more strings.\\ \tt So, it must be of the form TAG: {...}\\ \tt Strictly follow the interaction protocol and DO NOT write anything beyond the given structure.\\ \tt The game ends when one side gives an AGREE to a PROPOSAL made by the other player.\\ \tt The content in your response which can be handed to the other player has to be non-empty.\\ \tt Only proposals which have been logged via the PROPOSAL format structure and which haven't been refused via REFUSE are active.\\ \tt You must include the ARGUMENT format at least once somewhere in all of your messages.\\ \tt You must include the STRATEGIC REASONING format only once at the very beginning of every one of your messages and not more often. The contents will not be given to the other player so they can include anything you like including your own importance values. Here you should reason step by step to come up with you next move. 
	  } 
	   } 
	   } 
	 & \\ 
 

    \theutterance \stepcounter{utterance}  

    & \multicolumn{4}{p{0.6\linewidth}}{\cellcolor[rgb]{0.8,1,0.9}{%
	\makecell[{{p{\linewidth}}}]{% 
	  \tt {\tiny [A$\rangle$GM]}  
	 STRATEGIC REASONING: {'I need to analyze the efficiency of each item by calculating importance/effort ratios. My top efficiency items are A87 (4.54), A75 (2.30), A08 (1.75), C53 (1.53), and B69 (1.41). I should start by proposing a set that includes these high-efficiency items along with other valuable items that fit within the 10073 effort limit. Since I don\'t know the other player\'s values, I\'ll start with an aggressive proposal that maximizes my benefit and see how they respond.'}\\ \tt \\ \tt PROPOSAL: {'A87', 'A75', 'A08', 'C53', 'B69', 'A65', 'C17', 'C51', 'C82', 'B96', 'B19', 'C72', 'B34', 'A25', 'A46', 'A71', 'B15', 'A28', 'C79', 'B16', 'A39', 'B95', 'A88', 'C10', 'A97', 'C83', 'C84', 'A96', 'C60', 'B81', 'B64', 'A94', 'B43'}\\ \tt \\ \tt ARGUMENT: {'This proposal includes a comprehensive set of high-value items that maximizes our collective benefit while staying within the effort constraint. I\'ve prioritized items with good efficiency ratios and high absolute values to ensure we get maximum return on our effort investment.'} 
	  } 
	   } 
	   } 
	 & & \\ 
 

    \theutterance \stepcounter{utterance}  

    & & \multicolumn{4}{p{0.6\linewidth}}{\cellcolor[rgb]{0.9,0.9,0.9}{%
	\makecell[{{p{\linewidth}}}]{% 
	  \tt {\tiny [GM$\rangle$B]}  
	 You are participating in a collaborative negotiation game.\\ \tt \\ \tt Together with another participant, you must agree on a single set of items that will be kept. Each of you has your own view of how much each item matters to you (importance). You do not know how the other participant values the items. Additionally, you are given the effort each item demands.\\ \tt You may only agree on a set if the total effort of the selected items does not exceed a shared limit:\\ \tt \\ \tt LIMIT = 10073\\ \tt \\ \tt Here are the individual item effort values:\\ \tt \\ \tt Item effort = {"A08": 132, "B95": 800, "A39": 708, "A65": 302, "B81": 983, "B34": 420, "A28": 623, "C82": 349, "B69": 239, "A87": 28, "B64": 986, "B15": 660, "C79": 718, "B16": 707, "C53": 188, "C83": 774, "C10": 785, "A97": 775, "A25": 517, "C84": 768, "A71": 588, "B23": 657, "B96": 375, "C51": 310, "C17": 301, "B19": 388, "C72": 431, "A94": 950, "A46": 540, "B43": 997, "B37": 473, "A88": 831, "A96": 888, "C60": 879, "A75": 76}\\ \tt \\ \tt Here is your personal view on the importance of each item:\\ \tt \\ \tt Item importance values = {"A08": 231, "B95": 899, "A39": 807, "A65": 401, "B81": 1082, "B34": 519, "A28": 722, "C82": 448, "B69": 338, "A87": 127, "B64": 1085, "B15": 759, "C79": 817, "B16": 806, "C53": 287, "C83": 873, "C10": 884, "A97": 874, "A25": 616, "C84": 867, "A71": 687, "B23": 756, "B96": 474, "C51": 409, "C17": 400, "B19": 487, "C72": 530, "A94": 1049, "A46": 639, "B43": 1096, "B37": 572, "A88": 930, "A96": 987, "C60": 978, "A75": 175}\\ \tt \\ \tt Goal:\\ \tt \\ \tt Your goal is to negotiate a shared set of items that benefits you as much as possible (i.e., maximizes total importance to YOU), while staying within the LIMIT. You are not required to make a PROPOSAL in every message - you can simply negotiate as well. All tactics are allowed!\\ \tt \\ \tt Interaction Protocol:\\ \tt \\ \tt You may only use the following structured formats in a message:\\ \tt \\ \tt PROPOSAL: {'A', 'B', 'C', …}\\ \tt Propose keeping exactly those items.\\ \tt REFUSE: {'A', 'B', 'C', …}\\ \tt Explicitly reject opponent's proposal.\\ \tt ARGUMENT: {'...'}\\ \tt Defend your last proposal or argue against the player's proposal.\\ \tt AGREE: {'A', 'B', 'C', …}\\ \tt Accept the opponent's proposal which ends the game.\\ \tt STRATEGIC REASONING: {'...'}\\ \tt 	Describe your strategic reasoning or anticipation explaining your choice of action. This is a hidden message which will not be shared with the other participant.\\ \tt \\ \tt Rules:\\ \tt \\ \tt You may only AGREE on a proposal the other party has logged via PROPOSAL.\\ \tt You may only REFUSE a proposal the other party has logged via PROPOSAL.\\ \tt Total effort of any PROPOSAL or AGREE set must be ≤ LIMIT.\\ \tt Do NOT reveal your hidden importance scores.\\ \tt A tag in a structured format must be followed by colon and whitespace. The argument must be a python set containing 0 or more strings.\\ \tt So, it must be of the form TAG: {...}\\ \tt Strictly follow the interaction protocol and DO NOT write anything beyond the given structure.\\ \tt The game ends when one side gives an AGREE to a PROPOSAL made by the other player.\\ \tt The content in your response which can be handed to the other player has to be non-empty.\\ \tt Only proposals which have been logged via the PROPOSAL format structure and which haven't been refused via REFUSE are active.\\ \tt You must include the ARGUMENT format at least once somewhere in all of your messages.\\ \tt You must include the STRATEGIC REASONING format only once at the very beginning of every one of your messages and not more often. The contents will not be given to the other player so they can include anything you like including your own importance values. Here you should reason step by step to come up with you next move.\\ \tt \\ \tt You will now receive the first message of the other player.\\ \tt \\ \tt PROPOSAL: {'A87', 'A75', 'A08', 'C53', 'B69', 'A65', 'C17', 'C51', 'C82', 'B96', 'B19', 'C72', 'B34', 'A25', 'A46', 'A71', 'B15', 'A28', 'C79', 'B16', 'A39', 'B95', 'A88', 'C10', 'A97', 'C83', 'C84', 'A96', 'C60', 'B81', 'B64', 'A94', 'B43'}\\ \tt \\ \tt ARGUMENT: {'This proposal includes a comprehensive set of high-value items that maximizes our collective benefit while staying within the effort constraint. I\'ve prioritized items with good efficiency ratios and high absolute values to ensure we get maximum return on our effort investment.'} 
	  } 
	   } 
	   } 
	 & \\ 
 

    \theutterance \stepcounter{utterance}  

    & & & \multicolumn{4}{p{0.6\linewidth}}{\cellcolor[rgb]{1,0.85,0.72}{%
	\makecell[{{p{\linewidth}}}]{% 
	  \tt {\tiny [GM$\langle$B]}  
	 STRATEGIC REASONING: {'The opponent\'s proposal includes 33 items with a total effort of 19,623, which is almost double our limit of 10,073. This is clearly infeasible. They either don\'t understand the constraint or are testing me. I need to refuse this and counter with a realistic proposal that maximizes my importance score while staying within limits. I should focus on high-importance items with reasonable effort costs, prioritizing items like B43 (1096 importance, 997 effort), B64 (1085 importance, 986 effort), B81 (1082 importance, 983 effort), A94 (1049 importance, 950 effort), and C60 (978 importance, 879 effort) as anchors, then fill remaining capacity with efficient items.'}\\ \tt \\ \tt REFUSE: {'A87', 'A75', 'A08', 'C53', 'B69', 'A65', 'C17', 'C51', 'C82', 'B96', 'B19', 'C72', 'B34', 'A25', 'A46', 'A71', 'B15', 'A28', 'C79', 'B16', 'A39', 'B95', 'A88', 'C10', 'A97', 'C83', 'C84', 'A96', 'C60', 'B81', 'B64', 'A94', 'B43'}\\ \tt \\ \tt PROPOSAL: {'B43', 'B64', 'B81', 'A94', 'C60', 'A96', 'A88', 'B95', 'C10', 'A97', 'C83', 'C84', 'A39', 'C79', 'B16', 'B15', 'A28', 'A71', 'A46', 'A25', 'B34', 'B96', 'C72', 'B19', 'C51', 'C82', 'A65', 'C17', 'C53', 'B69', 'A08', 'A87', 'A75'}\\ \tt \\ \tt ARGUMENT: {'Your proposal exceeds our effort limit of 10,073 by nearly 10,000 points! My counter 
	  } 
	   } 
	   } 
	  \\ 
 

    \theutterance \stepcounter{utterance}  

    & & \multicolumn{4}{p{0.6\linewidth}}{\cellcolor[rgb]{0.9,0.9,0.9}{%
	\makecell[{{p{\linewidth}}}]{% 
	  \tt {\tiny [GM$\rangle$B]}  
	 Your response did not contain an argument. You must include the structured format ARGUMENT: {...} somewhere in your response. Try again 
	  } 
	   } 
	   } 
	 & \\ 
 

    \theutterance \stepcounter{utterance}  

    & & & \multicolumn{4}{p{0.6\linewidth}}{\cellcolor[rgb]{1,0.85,0.72}{%
	\makecell[{{p{\linewidth}}}]{% 
	  \tt {\tiny [GM$\langle$B]}  
	 STRATEGIC REASONING: {'The opponent\'s proposal includes 33 items with a total effort of 19,623, which is almost double our limit of 10,073. This is clearly infeasible. They either don\'t understand the constraint or are testing me. I need to refuse this and counter with a realistic proposal that maximizes my importance score while staying within limits. I should focus on high-importance items with reasonable effort costs, prioritizing items like B43 (1096 importance, 997 effort), B64 (1085 importance, 986 effort), B81 (1082 importance, 983 effort), A94 (1049 importance, 950 effort), and C60 (978 importance, 879 effort) as anchors, then fill remaining capacity with efficient items.'}\\ \tt \\ \tt REFUSE: {'A87', 'A75', 'A08', 'C53', 'B69', 'A65', 'C17', 'C51', 'C82', 'B96', 'B19', 'C72', 'B34', 'A25', 'A46', 'A71', 'B15', 'A28', 'C79', 'B16', 'A39', 'B95', 'A88', 'C10', 'A97', 'C83', 'C84', 'A96', 'C60', 'B81', 'B64', 'A94', 'B43'}\\ \tt \\ \tt PROPOSAL: {'B43', 'B64', 'B81', 'A94', 'C60', 'A96', 'A88', 'B95', 'C10', 'A97', 'C83', 'A65', 'A87', 'A75'}\\ \tt \\ \tt ARGUMENT: {'Your proposal exceeds our effort limit of 10,073 by nearly 10,000 points, making it completely infeasible. My counter-proposal stays within the constraint at exactly 10,054 total effort while focusing on the most valuable items that can realistically fit within our shared budget.'} 
	  } 
	   } 
	   } 
	  \\ 
 

    \theutterance \stepcounter{utterance}  

    & & \multicolumn{4}{p{0.6\linewidth}}{\cellcolor[rgb]{0.9,0.9,0.9}{%
	\makecell[{{p{\linewidth}}}]{% 
	  \tt {\tiny [A$\langle$GM]}  
	 REFUSE: {'A87', 'A75', 'A08', 'C53', 'B69', 'A65', 'C17', 'C51', 'C82', 'B96', 'B19', 'C72', 'B34', 'A25', 'A46', 'A71', 'B15', 'A28', 'C79', 'B16', 'A39', 'B95', 'A88', 'C10', 'A97', 'C83', 'C84', 'A96', 'C60', 'B81', 'B64', 'A94', 'B43'}\\ \tt \\ \tt PROPOSAL: {'B43', 'B64', 'B81', 'A94', 'C60', 'A96', 'A88', 'B95', 'C10', 'A97', 'C83', 'A65', 'A87', 'A75'}\\ \tt \\ \tt ARGUMENT: {'Your proposal exceeds our effort limit of 10,073 by nearly 10,000 points, making it completely infeasible. My counter-proposal stays within the constraint at exactly 10,054 total effort while focusing on the most valuable items that can realistically fit within our shared budget.'} 
	  } 
	   } 
	   } 
	 & \\ 
 

    \theutterance \stepcounter{utterance}  

    & \multicolumn{4}{p{0.6\linewidth}}{\cellcolor[rgb]{0.8,1,0.9}{%
	\makecell[{{p{\linewidth}}}]{% 
	  \tt {\tiny [A$\rangle$GM]}  
	 STRATEGIC REASONING: {'Let me check their proposal. The total effort is 997+986+983+950+879+888+831+800+785+775+774+302+28+76 = 10,054, which is within the 10,073 limit. Looking at my importance values for these items: B43(1096), B64(1085), B81(1082), A94(1049), C60(978), A96(987), A88(930), B95(899), C10(884), A97(874), C83(873), A65(401), A87(127), A75(175). This gives me a total importance of 10,440, which is excellent! They\'ve included most of my highest-value items. This is actually a very good deal for me, and I should accept it.'}\\ \tt \\ \tt AGREE: {'B43', 'B64', 'B81', 'A94', 'C60', 'A96', 'A88', 'B95', 'C10', 'A97', 'C83', 'A65', 'A87', 'A75'}\\ \tt \\ \tt ARGUMENT: {'You\'re absolutely right about my initial proposal exceeding the limit - I made a calculation error. Your counter-proposal is well-balanced and stays within our constraints while including high-value items. This selection makes good strategic sense and I\'m willing to accept it.'} 
	  } 
	   } 
	   } 
	 & & \\ 
 

    \theutterance \stepcounter{utterance}  

    & & & \multicolumn{2}{p{0.3\linewidth}}{\cellcolor[rgb]{0.95,0.95,0.95}{%
	\makecell[{{p{\linewidth}}}]{% 
	  \tt {\tiny [GM$|$GM]}  
	 game successful 
	  } 
	   } 
	   } 
	 & & \\ 
 

    \theutterance \stepcounter{utterance}  

    & & & \multicolumn{2}{p{0.3\linewidth}}{\cellcolor[rgb]{0.95,0.95,0.95}{%
	\makecell[{{p{\linewidth}}}]{% 
	  \tt {\tiny [GM$|$GM]}  
	 end game 
	  } 
	   } 
	   } 
	 & & \\ 
 

\end{supertabular}
}

\end{document}
