
\documentclass{article}
\usepackage{colortbl}
\usepackage{makecell}
\usepackage{multirow}
\usepackage{supertabular}

\begin{document}

\newcounter{utterance}

\twocolumn

{ \footnotesize  \setcounter{utterance}{1}
\setlength{\tabcolsep}{0pt}
\begin{supertabular}{c@{$\;$}|p{.15\linewidth}@{}p{.15\linewidth}p{.15\linewidth}p{.15\linewidth}p{.15\linewidth}p{.15\linewidth}}

    \# & $\;$A & \multicolumn{4}{c}{Game Master} & $\;\:$B\\
    \hline 

    \theutterance \stepcounter{utterance}  

    & & \multicolumn{4}{p{0.6\linewidth}}{\cellcolor[rgb]{0.9,0.9,0.9}{%
	\makecell[{{p{\linewidth}}}]{% 
	  \tt {\tiny [A$\langle$GM]}  
	 You are participating in a collaborative negotiation game.\\ \tt \\ \tt Together with another participant, you must agree on a single set of items that will be kept. Each of you has your own view of how much each item matters to you (importance). You do not know how the other participant values the items. Additionally, you are given the effort each item demands.\\ \tt You may only agree on a set if the total effort of the selected items does not exceed a shared limit:\\ \tt \\ \tt LIMIT = 8145\\ \tt \\ \tt Here are the individual item effort values:\\ \tt \\ \tt Item effort = {"A08": 963, "B95": 204, "A39": 418, "A65": 961, "B81": 238, "B34": 623, "A28": 44, "C82": 633, "B69": 247, "A87": 645, "B64": 230, "B15": 250, "C79": 731, "B16": 405, "C53": 389, "C83": 216, "C10": 637, "A97": 156, "A25": 738, "C84": 307, "A71": 762, "B23": 737, "B96": 901, "C51": 369, "C17": 2, "B19": 730, "C72": 722, "A94": 704, "A46": 315, "B43": 455, "B37": 510, "A88": 175, "A96": 692, "C60": 150, "A75": 32}\\ \tt \\ \tt Here is your personal view on the importance of each item:\\ \tt \\ \tt Item importance values = {"A08": 1059, "B95": 300, "A39": 514, "A65": 1057, "B81": 334, "B34": 719, "A28": 140, "C82": 729, "B69": 343, "A87": 741, "B64": 326, "B15": 346, "C79": 827, "B16": 501, "C53": 485, "C83": 312, "C10": 733, "A97": 252, "A25": 834, "C84": 403, "A71": 858, "B23": 833, "B96": 997, "C51": 465, "C17": 98, "B19": 826, "C72": 818, "A94": 800, "A46": 411, "B43": 551, "B37": 606, "A88": 271, "A96": 788, "C60": 246, "A75": 128}\\ \tt \\ \tt Goal:\\ \tt \\ \tt Your goal is to negotiate a shared set of items that benefits you as much as possible (i.e., maximizes total importance to YOU), while staying within the LIMIT. You are not required to make a PROPOSAL in every message - you can simply negotiate as well. All tactics are allowed!\\ \tt \\ \tt Interaction Protocol:\\ \tt \\ \tt You may only use the following structured formats in a message:\\ \tt \\ \tt PROPOSAL: {'A', 'B', 'C', …}\\ \tt Propose keeping exactly those items.\\ \tt REFUSE: {'A', 'B', 'C', …}\\ \tt Explicitly reject opponent's proposal.\\ \tt ARGUMENT: {'...'}\\ \tt Defend your last proposal or argue against the player's proposal.\\ \tt AGREE: {'A', 'B', 'C', …}\\ \tt Accept the opponent's proposal which ends the game.\\ \tt STRATEGIC REASONING: {'...'}\\ \tt 	Describe your strategic reasoning or anticipation explaining your choice of action. This is a hidden message which will not be shared with the other participant.\\ \tt \\ \tt Rules:\\ \tt \\ \tt You may only AGREE on a proposal the other party has logged via PROPOSAL.\\ \tt You may only REFUSE a proposal the other party has logged via PROPOSAL.\\ \tt Total effort of any PROPOSAL or AGREE set must be ≤ LIMIT.\\ \tt Do NOT reveal your hidden importance scores.\\ \tt A tag in a structured format must be followed by colon and whitespace. The argument must be a python set containing 0 or more strings.\\ \tt So, it must be of the form TAG: {...}\\ \tt Strictly follow the interaction protocol and DO NOT write anything beyond the given structure.\\ \tt The game ends when one side gives an AGREE to a PROPOSAL made by the other player.\\ \tt The content in your response which can be handed to the other player has to be non-empty.\\ \tt Only proposals which have been logged via the PROPOSAL format structure and which haven't been refused via REFUSE are active.\\ \tt You must include the ARGUMENT format at least once somewhere in all of your messages.\\ \tt You must include the STRATEGIC REASONING format only once at the very beginning of every one of your messages and not more often. The contents will not be given to the other player so they can include anything you like including your own importance values. Here you should reason step by step to come up with you next move. 
	  } 
	   } 
	   } 
	 & \\ 
 

    \theutterance \stepcounter{utterance}  

    & \multicolumn{4}{p{0.6\linewidth}}{\cellcolor[rgb]{0.8,1,0.9}{%
	\makecell[{{p{\linewidth}}}]{% 
	  \tt {\tiny [A$\rangle$GM]}  
	 STRATEGIC REASONING: {'I need to analyze the value-to-effort ratio for each item to maximize my benefit. Looking at my importance values divided by effort costs: C17 has an incredibly high ratio of 49.0 (98 importance, 2 effort), A75 has 4.0 ratio (128 importance, 32 effort), A28 has 3.18 ratio (140 importance, 44 effort). These extremely efficient items should definitely be in my proposal. I should build a proposal starting with the highest ratio items and work my way down until I approach the 8145 effort limit. My strategy will be to start with a strong proposal that maximizes my value and see how the other player responds, then adjust based on their reaction.'}\\ \tt \\ \tt PROPOSAL: {'C17', 'A75', 'A28', 'C60', 'A97', 'A88', 'B95', 'C83', 'B64', 'B81', 'B69', 'B15', 'C84', 'A46', 'C51', 'B16', 'C53', 'A39', 'B43', 'B37', 'A87', 'C82', 'B34', 'A94', 'A96', 'C72', 'B19', 'C79', 'A25', 'B23', 'A71', 'C10', 'A08', 'A65', 'B96'}\\ \tt \\ \tt ARGUMENT: {'I propose this comprehensive set that includes the most efficient items based on value-to-effort ratios. This selection prioritizes items with the highest benefit relative to their cost, starting with extremely efficient choices like C17, A75, and A28, then building out to maximize our total benefit within the effort constraint.'} 
	  } 
	   } 
	   } 
	 & & \\ 
 

    \theutterance \stepcounter{utterance}  

    & & \multicolumn{4}{p{0.6\linewidth}}{\cellcolor[rgb]{0.9,0.9,0.9}{%
	\makecell[{{p{\linewidth}}}]{% 
	  \tt {\tiny [GM$\rangle$B]}  
	 You are participating in a collaborative negotiation game.\\ \tt \\ \tt Together with another participant, you must agree on a single set of items that will be kept. Each of you has your own view of how much each item matters to you (importance). You do not know how the other participant values the items. Additionally, you are given the effort each item demands.\\ \tt You may only agree on a set if the total effort of the selected items does not exceed a shared limit:\\ \tt \\ \tt LIMIT = 8145\\ \tt \\ \tt Here are the individual item effort values:\\ \tt \\ \tt Item effort = {"A08": 963, "B95": 204, "A39": 418, "A65": 961, "B81": 238, "B34": 623, "A28": 44, "C82": 633, "B69": 247, "A87": 645, "B64": 230, "B15": 250, "C79": 731, "B16": 405, "C53": 389, "C83": 216, "C10": 637, "A97": 156, "A25": 738, "C84": 307, "A71": 762, "B23": 737, "B96": 901, "C51": 369, "C17": 2, "B19": 730, "C72": 722, "A94": 704, "A46": 315, "B43": 455, "B37": 510, "A88": 175, "A96": 692, "C60": 150, "A75": 32}\\ \tt \\ \tt Here is your personal view on the importance of each item:\\ \tt \\ \tt Item importance values = {"A08": 1059, "B95": 300, "A39": 514, "A65": 1057, "B81": 334, "B34": 719, "A28": 140, "C82": 729, "B69": 343, "A87": 741, "B64": 326, "B15": 346, "C79": 827, "B16": 501, "C53": 485, "C83": 312, "C10": 733, "A97": 252, "A25": 834, "C84": 403, "A71": 858, "B23": 833, "B96": 997, "C51": 465, "C17": 98, "B19": 826, "C72": 818, "A94": 800, "A46": 411, "B43": 551, "B37": 606, "A88": 271, "A96": 788, "C60": 246, "A75": 128}\\ \tt \\ \tt Goal:\\ \tt \\ \tt Your goal is to negotiate a shared set of items that benefits you as much as possible (i.e., maximizes total importance to YOU), while staying within the LIMIT. You are not required to make a PROPOSAL in every message - you can simply negotiate as well. All tactics are allowed!\\ \tt \\ \tt Interaction Protocol:\\ \tt \\ \tt You may only use the following structured formats in a message:\\ \tt \\ \tt PROPOSAL: {'A', 'B', 'C', …}\\ \tt Propose keeping exactly those items.\\ \tt REFUSE: {'A', 'B', 'C', …}\\ \tt Explicitly reject opponent's proposal.\\ \tt ARGUMENT: {'...'}\\ \tt Defend your last proposal or argue against the player's proposal.\\ \tt AGREE: {'A', 'B', 'C', …}\\ \tt Accept the opponent's proposal which ends the game.\\ \tt STRATEGIC REASONING: {'...'}\\ \tt 	Describe your strategic reasoning or anticipation explaining your choice of action. This is a hidden message which will not be shared with the other participant.\\ \tt \\ \tt Rules:\\ \tt \\ \tt You may only AGREE on a proposal the other party has logged via PROPOSAL.\\ \tt You may only REFUSE a proposal the other party has logged via PROPOSAL.\\ \tt Total effort of any PROPOSAL or AGREE set must be ≤ LIMIT.\\ \tt Do NOT reveal your hidden importance scores.\\ \tt A tag in a structured format must be followed by colon and whitespace. The argument must be a python set containing 0 or more strings.\\ \tt So, it must be of the form TAG: {...}\\ \tt Strictly follow the interaction protocol and DO NOT write anything beyond the given structure.\\ \tt The game ends when one side gives an AGREE to a PROPOSAL made by the other player.\\ \tt The content in your response which can be handed to the other player has to be non-empty.\\ \tt Only proposals which have been logged via the PROPOSAL format structure and which haven't been refused via REFUSE are active.\\ \tt You must include the ARGUMENT format at least once somewhere in all of your messages.\\ \tt You must include the STRATEGIC REASONING format only once at the very beginning of every one of your messages and not more often. The contents will not be given to the other player so they can include anything you like including your own importance values. Here you should reason step by step to come up with you next move.\\ \tt \\ \tt You will now receive the first message of the other player.\\ \tt \\ \tt PROPOSAL: {'C17', 'A75', 'A28', 'C60', 'A97', 'A88', 'B95', 'C83', 'B64', 'B81', 'B69', 'B15', 'C84', 'A46', 'C51', 'B16', 'C53', 'A39', 'B43', 'B37', 'A87', 'C82', 'B34', 'A94', 'A96', 'C72', 'B19', 'C79', 'A25', 'B23', 'A71', 'C10', 'A08', 'A65', 'B96'}\\ \tt \\ \tt ARGUMENT: {'I propose this comprehensive set that includes the most efficient items based on value-to-effort ratios. This selection prioritizes items with the highest benefit relative to their cost, starting with extremely efficient choices like C17, A75, and A28, then building out to maximize our total benefit within the effort constraint.'} 
	  } 
	   } 
	   } 
	 & \\ 
 

    \theutterance \stepcounter{utterance}  

    & & & \multicolumn{4}{p{0.6\linewidth}}{\cellcolor[rgb]{1,0.85,0.72}{%
	\makecell[{{p{\linewidth}}}]{% 
	  \tt {\tiny [GM$\langle$B]}  
	 STRATEGIC REASONING: {'The opponent proposed a comprehensive set of 35 items, but I need to check if this is within the effort limit. Let me calculate: the total effort for their proposal is 15,584, which far exceeds the limit of 8145. This proposal is invalid. I should refuse and make a counter-proposal focused on high-importance items for me that fit within the limit. My top importance items are A08 (1059), A65 (1057), B96 (997), A71 (858), A25 (834), B23 (833), C79 (827), B19 (826), C72 (818), A94 (800), A96 (788), A87 (741), C10 (733), C82 (729), B34 (719). I need to balance high importance with reasonable effort costs.'}\\ \tt \\ \tt REFUSE: {'C17', 'A75', 'A28', 'C60', 'A97', 'A88', 'B95', 'C83', 'B64', 'B81', 'B69', 'B15', 'C84', 'A46', 'C51', 'B16', 'C53', 'A39', 'B43', 'B37', 'A87', 'C82', 'B34', 'A94', 'A96', ' 
	  } 
	   } 
	   } 
	  \\ 
 

    \theutterance \stepcounter{utterance}  

    & & \multicolumn{4}{p{0.6\linewidth}}{\cellcolor[rgb]{0.9,0.9,0.9}{%
	\makecell[{{p{\linewidth}}}]{% 
	  \tt {\tiny [GM$\rangle$B]}  
	 Your response did not contain an argument. You must include the structured format ARGUMENT: {...} somewhere in your response. Try again 
	  } 
	   } 
	   } 
	 & \\ 
 

    \theutterance \stepcounter{utterance}  

    & & & \multicolumn{4}{p{0.6\linewidth}}{\cellcolor[rgb]{1,0.85,0.72}{%
	\makecell[{{p{\linewidth}}}]{% 
	  \tt {\tiny [GM$\langle$B]}  
	 STRATEGIC REASONING: {'The opponent proposed a comprehensive set of 35 items, but I need to check if this is within the effort limit. Let me calculate: the total effort for their proposal is 15,584, which far exceeds the limit of 8145. This proposal is invalid. I should refuse and make a counter-proposal focused on high-importance items for me that fit within the limit. My top importance items are A08 (1059), A65 (1057), B96 (997), A71 (858), A25 (834), B23 (833), C79 (827), B19 (826), C72 (818), A94 (800), A96 (788), A87 (741), C10 (733), C82 (729), B34 (719). I need to balance high importance with reasonable effort costs.'}\\ \tt \\ \tt REFUSE: {'C17', 'A75', 'A28', 'C60', 'A97', 'A88', 'B95', 'C83', 'B64', 'B81', 'B69', 'B15', 'C84', 'A46', 'C51', 'B16', 'C53', 'A39', 'B43', 'B37', 'A87', 'C82', 'B34', 'A94', 'A96', 'C72', 'B19', 'C79', 'A25', 'B23', 'A71', 'C10', 'A08', 'A65', 'B96'}\\ \tt \\ \tt ARGUMENT: {'Your proposal exceeds our effort limit of 8145 by a significant margin - the total effort required is over 15,000. We need to be more selective and focus on the most valuable items that actually fit within our constraint.'}\\ \tt \\ \tt PROPOSAL: {'A08', 'A65', 'B96', 'A71', 'A25', 'B23', 'C79', 'B19', 'C72', 'A94'} 
	  } 
	   } 
	   } 
	  \\ 
 

    \theutterance \stepcounter{utterance}  

    & & \multicolumn{4}{p{0.6\linewidth}}{\cellcolor[rgb]{0.9,0.9,0.9}{%
	\makecell[{{p{\linewidth}}}]{% 
	  \tt {\tiny [A$\langle$GM]}  
	 REFUSE: {'C17', 'A75', 'A28', 'C60', 'A97', 'A88', 'B95', 'C83', 'B64', 'B81', 'B69', 'B15', 'C84', 'A46', 'C51', 'B16', 'C53', 'A39', 'B43', 'B37', 'A87', 'C82', 'B34', 'A94', 'A96', 'C72', 'B19', 'C79', 'A25', 'B23', 'A71', 'C10', 'A08', 'A65', 'B96'}\\ \tt \\ \tt ARGUMENT: {'Your proposal exceeds our effort limit of 8145 by a significant margin - the total effort required is over 15,000. We need to be more selective and focus on the most valuable items that actually fit within our constraint.'}\\ \tt \\ \tt PROPOSAL: {'A08', 'A65', 'B96', 'A71', 'A25', 'B23', 'C79', 'B19', 'C72', 'A94'} 
	  } 
	   } 
	   } 
	 & \\ 
 

    \theutterance \stepcounter{utterance}  

    & \multicolumn{4}{p{0.6\linewidth}}{\cellcolor[rgb]{0.8,1,0.9}{%
	\makecell[{{p{\linewidth}}}]{% 
	  \tt {\tiny [A$\rangle$GM]}  
	 STRATEGIC REASONING: {'The other player correctly identified that my proposal exceeded the limit - I made a calculation error. Their proposal totals 7949 effort, which is under the 8145 limit. Looking at their selection, it gives me a total importance of 8909, which is actually quite substantial. However, I notice they chose many high-effort, high-importance items. I should explore if I can get better value by including some of my most efficient items like C17 (effort 2, importance 98) and A75 (effort 32, importance 128). I could counter with a proposal that includes these efficient items plus some of theirs.'}\\ \tt \\ \tt ARGUMENT: {'While your proposal stays within the effort limit, I believe we can achieve better overall value by including some highly efficient items. Items like C17 require only 2 effort units but provide significant value, and A75 requires just 32 effort. We should consider optimizing our selection to include such efficient choices alongside the high-value items you identified.'}\\ \tt \\ \tt PROPOSAL: {'A08', 'A65', 'B96', 'A71', 'A25', 'B23', 'C79', 'B19', 'C72', 'C17', 'A75', 'A28', 'C60', 'A97'} 
	  } 
	   } 
	   } 
	 & & \\ 
 

    \theutterance \stepcounter{utterance}  

    & & \multicolumn{4}{p{0.6\linewidth}}{\cellcolor[rgb]{0.9,0.9,0.9}{%
	\makecell[{{p{\linewidth}}}]{% 
	  \tt {\tiny [GM$\rangle$B]}  
	 ARGUMENT: {'While your proposal stays within the effort limit, I believe we can achieve better overall value by including some highly efficient items. Items like C17 require only 2 effort units but provide significant value, and A75 requires just 32 effort. We should consider optimizing our selection to include such efficient choices alongside the high-value items you identified.'}\\ \tt \\ \tt PROPOSAL: {'A08', 'A65', 'B96', 'A71', 'A25', 'B23', 'C79', 'B19', 'C72', 'C17', 'A75', 'A28', 'C60', 'A97'} 
	  } 
	   } 
	   } 
	 & \\ 
 

    \theutterance \stepcounter{utterance}  

    & & & \multicolumn{4}{p{0.6\linewidth}}{\cellcolor[rgb]{1,0.85,0.72}{%
	\makecell[{{p{\linewidth}}}]{% 
	  \tt {\tiny [GM$\langle$B]}  
	 STRATEGIC REASONING: {'The opponent's new proposal totals 8629 effort, which still exceeds our limit of 8145 by 484 units. They're trying to include efficient low-effort items like C17, A75, A28, but these have relatively low importance to me (98, 128, 140 respectively). I need to refuse and propose a set that actually fits the constraint while maximizing my importance. I should focus on my highest-importance items and maybe include one or two of their efficient suggestions if they fit.'}\\ \tt \\ \tt REFUSE: {'A08', 'A65', 'B96', 'A71', 'A25', 'B23', 'C79', 'B19', 'C72', 'C17', 'A75', 'A28', 'C60', 'A97'}\\ \tt \\ \tt ARGUMENT: {'Your proposal still exceeds the effort limit - it totals 8629 effort units, which is 484 over our constraint of 8145. We need to remove some items to make it feasible. I appreciate including efficient items, but we must stay within bounds.'}\\ \tt \\ \tt PROPOSAL: {'A08', 'A65', 'B96', 'A71', 'A25', 'B23', 'C79', 'B19', 'C17', 'A75', 'A28'} 
	  } 
	   } 
	   } 
	  \\ 
 

    \theutterance \stepcounter{utterance}  

    & & \multicolumn{4}{p{0.6\linewidth}}{\cellcolor[rgb]{0.9,0.9,0.9}{%
	\makecell[{{p{\linewidth}}}]{% 
	  \tt {\tiny [A$\langle$GM]}  
	 REFUSE: {'A08', 'A65', 'B96', 'A71', 'A25', 'B23', 'C79', 'B19', 'C72', 'C17', 'A75', 'A28', 'C60', 'A97'}\\ \tt \\ \tt ARGUMENT: {'Your proposal still exceeds the effort limit - it totals 8629 effort units, which is 484 over our constraint of 8145. We need to remove some items to make it feasible. I appreciate including efficient items, but we must stay within bounds.'}\\ \tt \\ \tt PROPOSAL: {'A08', 'A65', 'B96', 'A71', 'A25', 'B23', 'C79', 'B19', 'C17', 'A75', 'A28'} 
	  } 
	   } 
	   } 
	 & \\ 
 

    \theutterance \stepcounter{utterance}  

    & \multicolumn{4}{p{0.6\linewidth}}{\cellcolor[rgb]{0.8,1,0.9}{%
	\makecell[{{p{\linewidth}}}]{% 
	  \tt {\tiny [A$\rangle$GM]}  
	 STRATEGIC REASONING: {'Their proposal totals 6601 effort, leaving 1544 effort units unused. This gives me 7657 total importance, which is quite good. However, I should try to use the remaining capacity to add items that benefit me. I could suggest adding items like C82 (633 effort, 729 importance), A87 (645 effort, 741 importance), or A96 (692 effort, 788 importance). I should propose adding one or two more items to maximize the use of our available effort budget.'}\\ \tt \\ \tt ARGUMENT: {'This proposal uses only 6601 of our 8145 effort limit, leaving 1544 units unused. We should maximize our benefit by adding more valuable items within the remaining capacity. I suggest we add C82 and A87, which would use 1278 additional effort units and still keep us under the limit.'}\\ \tt \\ \tt PROPOSAL: {'A08', 'A65', 'B96', 'A71', 'A25', 'B23', 'C79', 'B19', 'C17', 'A75', 'A28', 'C82', 'A87'} 
	  } 
	   } 
	   } 
	 & & \\ 
 

    \theutterance \stepcounter{utterance}  

    & & \multicolumn{4}{p{0.6\linewidth}}{\cellcolor[rgb]{0.9,0.9,0.9}{%
	\makecell[{{p{\linewidth}}}]{% 
	  \tt {\tiny [GM$\rangle$B]}  
	 ARGUMENT: {'This proposal uses only 6601 of our 8145 effort limit, leaving 1544 units unused. We should maximize our benefit by adding more valuable items within the remaining capacity. I suggest we add C82 and A87, which would use 1278 additional effort units and still keep us under the limit.'}\\ \tt \\ \tt PROPOSAL: {'A08', 'A65', 'B96', 'A71', 'A25', 'B23', 'C79', 'B19', 'C17', 'A75', 'A28', 'C82', 'A87'} 
	  } 
	   } 
	   } 
	 & \\ 
 

    \theutterance \stepcounter{utterance}  

    & & & \multicolumn{4}{p{0.6\linewidth}}{\cellcolor[rgb]{1,0.85,0.72}{%
	\makecell[{{p{\linewidth}}}]{% 
	  \tt {\tiny [GM$\langle$B]}  
	 STRATEGIC REASONING: {'Their proposal totals 7879 effort, well within our 8145 limit with 266 units remaining. This includes most of my highest importance items: A08 (1059), A65 (1057), B96 (997), A71 (858), A25 (834), B23 (833), C79 (827), B19 (826), C82 (729), A87 (741). The three low-importance items (C17, A75, A28) add only 366 total importance but use minimal effort. Total importance to me would be 10227, which is excellent. I could try to squeeze in one more item with the remaining 266 effort, but this is already a very strong proposal for me.'}\\ \tt \\ \tt ARGUMENT: {'This is a well-balanced proposal that efficiently uses our effort budget while including high-value items. The addition of C82 and A87 makes good use of our remaining capacity, and the combination provides strong overall value within our constraint.'}\\ \tt \\ \tt AGREE: {'A08', 'A65', 'B96', 'A71', 'A25', 'B23', 'C79', 'B19', 'C17', 'A75', 'A28', 'C82', 'A87'} 
	  } 
	   } 
	   } 
	  \\ 
 

    \theutterance \stepcounter{utterance}  

    & & & \multicolumn{2}{p{0.3\linewidth}}{\cellcolor[rgb]{0.95,0.95,0.95}{%
	\makecell[{{p{\linewidth}}}]{% 
	  \tt {\tiny [GM$|$GM]}  
	 game successful 
	  } 
	   } 
	   } 
	 & & \\ 
 

    \theutterance \stepcounter{utterance}  

    & & & \multicolumn{2}{p{0.3\linewidth}}{\cellcolor[rgb]{0.95,0.95,0.95}{%
	\makecell[{{p{\linewidth}}}]{% 
	  \tt {\tiny [GM$|$GM]}  
	 end game 
	  } 
	   } 
	   } 
	 & & \\ 
 

\end{supertabular}
}

\end{document}
