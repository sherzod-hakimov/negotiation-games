
\documentclass{article}
\usepackage{colortbl}
\usepackage{makecell}
\usepackage{multirow}
\usepackage{supertabular}

\begin{document}

\newcounter{utterance}

\twocolumn

{ \footnotesize  \setcounter{utterance}{1}
\setlength{\tabcolsep}{0pt}
\begin{supertabular}{c@{$\;$}|p{.15\linewidth}@{}p{.15\linewidth}p{.15\linewidth}p{.15\linewidth}p{.15\linewidth}p{.15\linewidth}}

    \# & $\;$A & \multicolumn{4}{c}{Game Master} & $\;\:$B\\
    \hline 

    \theutterance \stepcounter{utterance}  

    & & \multicolumn{4}{p{0.6\linewidth}}{\cellcolor[rgb]{0.9,0.9,0.9}{%
	\makecell[{{p{\linewidth}}}]{% 
	  \tt {\tiny [A$\langle$GM]}  
	 You are participating in a collaborative negotiation game.\\ \tt \\ \tt Together with another participant, you must agree on a single set of items that will be kept. Each of you has your own view of how much each item matters to you (importance). You do not know how the other participant values the items. Additionally, you are given the effort each item demands.\\ \tt You may only agree on a set if the total effort of the selected items does not exceed a shared limit:\\ \tt \\ \tt LIMIT = 9612\\ \tt \\ \tt Here are the individual item effort values:\\ \tt \\ \tt Item effort = {"B21": 935, "B00": 928, "A08": 725, "A27": 218, "B94": 839, "B53": 970, "A23": 394, "C65": 677, "A89": 946, "B67": 530, "B45": 138, "C34": 834, "C12": 734, "A05": 593, "B60": 261, "C92": 743, "B69": 4, "A84": 735, "C06": 124, "B12": 829, "A16": 207, "B85": 782, "C35": 577, "B10": 388, "C15": 679, "B09": 494, "B16": 559, "C81": 629, "A58": 237, "C24": 275, "A30": 39, "A66": 654, "B75": 172, "B46": 688, "C70": 687}\\ \tt \\ \tt Here is your personal view on the importance of each item:\\ \tt \\ \tt Item importance values = {"B21": 138, "B00": 583, "A08": 868, "A27": 822, "B94": 783, "B53": 65, "A23": 262, "C65": 121, "A89": 508, "B67": 780, "B45": 461, "C34": 484, "C12": 668, "A05": 389, "B60": 808, "C92": 215, "B69": 97, "A84": 500, "C06": 30, "B12": 915, "A16": 856, "B85": 400, "C35": 444, "B10": 623, "C15": 781, "B09": 786, "B16": 3, "C81": 713, "A58": 457, "C24": 273, "A30": 739, "A66": 822, "B75": 235, "B46": 606, "C70": 968}\\ \tt \\ \tt Goal:\\ \tt \\ \tt Your goal is to negotiate a shared set of items that benefits you as much as possible (i.e., maximizes total importance to YOU), while staying within the LIMIT. You are not required to make a PROPOSAL in every message - you can simply negotiate as well. All tactics are allowed!\\ \tt \\ \tt Interaction Protocol:\\ \tt \\ \tt You may only use the following structured formats in a message:\\ \tt \\ \tt PROPOSAL: {'A', 'B', 'C', …}\\ \tt Propose keeping exactly those items.\\ \tt REFUSE: {'A', 'B', 'C', …}\\ \tt Explicitly reject opponent's proposal.\\ \tt ARGUMENT: {'...'}\\ \tt Defend your last proposal or argue against the player's proposal.\\ \tt AGREE: {'A', 'B', 'C', …}\\ \tt Accept the opponent's proposal which ends the game.\\ \tt STRATEGIC REASONING: {'...'}\\ \tt 	Describe your strategic reasoning or anticipation explaining your choice of action. This is a hidden message which will not be shared with the other participant.\\ \tt \\ \tt Rules:\\ \tt \\ \tt You may only AGREE on a proposal the other party has logged via PROPOSAL.\\ \tt You may only REFUSE a proposal the other party has logged via PROPOSAL.\\ \tt Total effort of any PROPOSAL or AGREE set must be ≤ LIMIT.\\ \tt Do NOT reveal your hidden importance scores.\\ \tt A tag in a structured format must be followed by colon and whitespace. The argument must be a python set containing 0 or more strings.\\ \tt So, it must be of the form TAG: {...}\\ \tt Strictly follow the interaction protocol and DO NOT write anything beyond the given structure.\\ \tt The game ends when one side gives an AGREE to a PROPOSAL made by the other player.\\ \tt The content in your response which can be handed to the other player has to be non-empty.\\ \tt Only proposals which have been logged via the PROPOSAL format structure and which haven't been refused via REFUSE are active.\\ \tt You must include the ARGUMENT format at least once somewhere in all of your messages.\\ \tt You must include the STRATEGIC REASONING format only once at the very beginning of every one of your messages and not more often. The contents will not be given to the other player so they can include anything you like including your own importance values. Here you should reason step by step to come up with you next move. 
	  } 
	   } 
	   } 
	 & \\ 
 

    \theutterance \stepcounter{utterance}  

    & \multicolumn{4}{p{0.6\linewidth}}{\cellcolor[rgb]{0.8,1,0.9}{%
	\makecell[{{p{\linewidth}}}]{% 
	  \tt {\tiny [A$\rangle$GM]}  
	 STRATEGIC REASONING: {'I need to analyze the efficiency of each item by calculating importance/effort ratios. The highest efficiency items are B69 (24.25), A30 (18.949), A16 (4.135), A27 (3.770), B45 (3.341), and B60 (3.096). These give me the most value per effort unit. I should start with a proposal including these high-efficiency items and fill the remaining effort budget with other valuable items. My strategy is to begin with an ambitious proposal that maximizes my utility and then negotiate from there.'}\\ \tt \\ \tt PROPOSAL: {'B69', 'A30', 'A16', 'A27', 'B45', 'B60', 'A08', 'B67', 'B10', 'B09', 'A58', 'C70', 'B75', 'A66', 'C15', 'C81', 'B12', 'A84', 'C12', 'B94'}\\ \tt \\ \tt ARGUMENT: {'This proposal includes the most efficient items that provide high value relative to their effort cost. The total effort is 8,847 which is well within our limit of 9,612, leaving room for adjustments. These items represent a balanced selection that should provide good overall value while maintaining flexibility for negotiation.'} 
	  } 
	   } 
	   } 
	 & & \\ 
 

    \theutterance \stepcounter{utterance}  

    & & \multicolumn{4}{p{0.6\linewidth}}{\cellcolor[rgb]{0.9,0.9,0.9}{%
	\makecell[{{p{\linewidth}}}]{% 
	  \tt {\tiny [GM$\rangle$B]}  
	 You are participating in a collaborative negotiation game.\\ \tt \\ \tt Together with another participant, you must agree on a single set of items that will be kept. Each of you has your own view of how much each item matters to you (importance). You do not know how the other participant values the items. Additionally, you are given the effort each item demands.\\ \tt You may only agree on a set if the total effort of the selected items does not exceed a shared limit:\\ \tt \\ \tt LIMIT = 9612\\ \tt \\ \tt Here are the individual item effort values:\\ \tt \\ \tt Item effort = {"B21": 935, "B00": 928, "A08": 725, "A27": 218, "B94": 839, "B53": 970, "A23": 394, "C65": 677, "A89": 946, "B67": 530, "B45": 138, "C34": 834, "C12": 734, "A05": 593, "B60": 261, "C92": 743, "B69": 4, "A84": 735, "C06": 124, "B12": 829, "A16": 207, "B85": 782, "C35": 577, "B10": 388, "C15": 679, "B09": 494, "B16": 559, "C81": 629, "A58": 237, "C24": 275, "A30": 39, "A66": 654, "B75": 172, "B46": 688, "C70": 687}\\ \tt \\ \tt Here is your personal view on the importance of each item:\\ \tt \\ \tt Item importance values = {"B21": 138, "B00": 583, "A08": 868, "A27": 822, "B94": 783, "B53": 65, "A23": 262, "C65": 121, "A89": 508, "B67": 780, "B45": 461, "C34": 484, "C12": 668, "A05": 389, "B60": 808, "C92": 215, "B69": 97, "A84": 500, "C06": 30, "B12": 915, "A16": 856, "B85": 400, "C35": 444, "B10": 623, "C15": 781, "B09": 786, "B16": 3, "C81": 713, "A58": 457, "C24": 273, "A30": 739, "A66": 822, "B75": 235, "B46": 606, "C70": 968}\\ \tt \\ \tt Goal:\\ \tt \\ \tt Your goal is to negotiate a shared set of items that benefits you as much as possible (i.e., maximizes total importance to YOU), while staying within the LIMIT. You are not required to make a PROPOSAL in every message - you can simply negotiate as well. All tactics are allowed!\\ \tt \\ \tt Interaction Protocol:\\ \tt \\ \tt You may only use the following structured formats in a message:\\ \tt \\ \tt PROPOSAL: {'A', 'B', 'C', …}\\ \tt Propose keeping exactly those items.\\ \tt REFUSE: {'A', 'B', 'C', …}\\ \tt Explicitly reject opponent's proposal.\\ \tt ARGUMENT: {'...'}\\ \tt Defend your last proposal or argue against the player's proposal.\\ \tt AGREE: {'A', 'B', 'C', …}\\ \tt Accept the opponent's proposal which ends the game.\\ \tt STRATEGIC REASONING: {'...'}\\ \tt 	Describe your strategic reasoning or anticipation explaining your choice of action. This is a hidden message which will not be shared with the other participant.\\ \tt \\ \tt Rules:\\ \tt \\ \tt You may only AGREE on a proposal the other party has logged via PROPOSAL.\\ \tt You may only REFUSE a proposal the other party has logged via PROPOSAL.\\ \tt Total effort of any PROPOSAL or AGREE set must be ≤ LIMIT.\\ \tt Do NOT reveal your hidden importance scores.\\ \tt A tag in a structured format must be followed by colon and whitespace. The argument must be a python set containing 0 or more strings.\\ \tt So, it must be of the form TAG: {...}\\ \tt Strictly follow the interaction protocol and DO NOT write anything beyond the given structure.\\ \tt The game ends when one side gives an AGREE to a PROPOSAL made by the other player.\\ \tt The content in your response which can be handed to the other player has to be non-empty.\\ \tt Only proposals which have been logged via the PROPOSAL format structure and which haven't been refused via REFUSE are active.\\ \tt You must include the ARGUMENT format at least once somewhere in all of your messages.\\ \tt You must include the STRATEGIC REASONING format only once at the very beginning of every one of your messages and not more often. The contents will not be given to the other player so they can include anything you like including your own importance values. Here you should reason step by step to come up with you next move.\\ \tt \\ \tt You will now receive the first message of the other player.\\ \tt \\ \tt PROPOSAL: {'B69', 'A30', 'A16', 'A27', 'B45', 'B60', 'A08', 'B67', 'B10', 'B09', 'A58', 'C70', 'B75', 'A66', 'C15', 'C81', 'B12', 'A84', 'C12', 'B94'}\\ \tt \\ \tt ARGUMENT: {'This proposal includes the most efficient items that provide high value relative to their effort cost. The total effort is 8,847 which is well within our limit of 9,612, leaving room for adjustments. These items represent a balanced selection that should provide good overall value while maintaining flexibility for negotiation.'} 
	  } 
	   } 
	   } 
	 & \\ 
 

    \theutterance \stepcounter{utterance}  

    & & & \multicolumn{4}{p{0.6\linewidth}}{\cellcolor[rgb]{1,0.85,0.72}{%
	\makecell[{{p{\linewidth}}}]{% 
	  \tt {\tiny [GM$\langle$B]}  
	 STRATEGIC REASONING: {'The opponent proposed 20 items with total effort 8,459, leaving 1,153 unused capacity. This proposal gives me total importance of 13,126, which is quite good. However, I should check if I can improve it by swapping some items. Key high-importance items for me in their proposal: C70 (968), B12 (915), A08 (868), A16 (856), A66 (822), A27 (822), B60 (808), B09 (786), C15 (781), B67 (780), B94 (783). I notice they excluded some potentially valuable items like A05, C35, B46, C24, A23, C06, B85, A89, B00, B21, B53, C34, C92. Let me see if I can make strategic substitutions.'}\\ \tt \\ \tt ARGUMENT: {'I appreciate this comprehensive proposal that includes many valuable items while staying well within our effort limit. The selection appears thoughtful, particularly including efficient items like B69 with minimal effort cost. However, I believe we can optimize this further given the remaining capacity of over 1,100 effort points. Some adjustments might help us both achieve better outcomes.'}\\ \tt \\ \tt PROPOSAL: {'B69', 'A30', 'A16', 'A27', 'B45', 'B60', 'A08', 'B67', 'B10', 'B09', 'A58', 'C70', 'A66', 'C15', 'C81', 'B12', 'A84', 'C12', 'B94', 'A05', 'C35'} 
	  } 
	   } 
	   } 
	  \\ 
 

    \theutterance \stepcounter{utterance}  

    & & \multicolumn{4}{p{0.6\linewidth}}{\cellcolor[rgb]{0.9,0.9,0.9}{%
	\makecell[{{p{\linewidth}}}]{% 
	  \tt {\tiny [A$\langle$GM]}  
	 ARGUMENT: {'I appreciate this comprehensive proposal that includes many valuable items while staying well within our effort limit. The selection appears thoughtful, particularly including efficient items like B69 with minimal effort cost. However, I believe we can optimize this further given the remaining capacity of over 1,100 effort points. Some adjustments might help us both achieve better outcomes.'}\\ \tt \\ \tt PROPOSAL: {'B69', 'A30', 'A16', 'A27', 'B45', 'B60', 'A08', 'B67', 'B10', 'B09', 'A58', 'C70', 'A66', 'C15', 'C81', 'B12', 'A84', 'C12', 'B94', 'A05', 'C35'} 
	  } 
	   } 
	   } 
	 & \\ 
 

    \theutterance \stepcounter{utterance}  

    & \multicolumn{4}{p{0.6\linewidth}}{\cellcolor[rgb]{0.8,1,0.9}{%
	\makecell[{{p{\linewidth}}}]{% 
	  \tt {\tiny [A$\rangle$GM]}  
	 STRATEGIC REASONING: {'The opponent has modified my proposal by removing B75 (effort 172, importance 235 to me) and adding A05 (effort 593, importance 389 to me) and C35 (effort 577, importance 444 to me). Let me calculate: this gives me a net gain of +598 importance points (389+444-235) while the total effort is 9,438, still within our limit. This is actually better for me than my original proposal, so I should accept this immediately.'}\\ \tt \\ \tt AGREE: {'B69', 'A30', 'A16', 'A27', 'B45', 'B60', 'A08', 'B67', 'B10', 'B09', 'A58', 'C70', 'A66', 'C15', 'C81', 'B12', 'A84', 'C12', 'B94', 'A05', 'C35'}\\ \tt \\ \tt ARGUMENT: {'This revised proposal is excellent - it efficiently uses our available effort capacity and includes high-value items. The addition of A05 and C35 while removing B75 creates a more optimal selection. I agree to this set as it represents a strong collaborative outcome within our constraints.'} 
	  } 
	   } 
	   } 
	 & & \\ 
 

    \theutterance \stepcounter{utterance}  

    & & & \multicolumn{2}{p{0.3\linewidth}}{\cellcolor[rgb]{0.95,0.95,0.95}{%
	\makecell[{{p{\linewidth}}}]{% 
	  \tt {\tiny [GM$|$GM]}  
	 game successful 
	  } 
	   } 
	   } 
	 & & \\ 
 

    \theutterance \stepcounter{utterance}  

    & & & \multicolumn{2}{p{0.3\linewidth}}{\cellcolor[rgb]{0.95,0.95,0.95}{%
	\makecell[{{p{\linewidth}}}]{% 
	  \tt {\tiny [GM$|$GM]}  
	 end game 
	  } 
	   } 
	   } 
	 & & \\ 
 

\end{supertabular}
}

\end{document}
