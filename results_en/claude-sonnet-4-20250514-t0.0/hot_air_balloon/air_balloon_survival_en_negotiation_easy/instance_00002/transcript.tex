
\documentclass{article}
\usepackage{colortbl}
\usepackage{makecell}
\usepackage{multirow}
\usepackage{supertabular}

\begin{document}

\newcounter{utterance}

\centering \large Interaction Transcript for game `hot\_air\_balloon', experiment `air\_balloon\_survival\_en\_negotiation\_easy', episode 2 with claude{-}sonnet{-}4{-}20250514{-}t0.0.
\vspace{24pt}

{ \footnotesize  \setcounter{utterance}{1}
\setlength{\tabcolsep}{0pt}
\begin{supertabular}{c@{$\;$}|p{.15\linewidth}@{}p{.15\linewidth}p{.15\linewidth}p{.15\linewidth}p{.15\linewidth}p{.15\linewidth}}
    \# & $\;$A & \multicolumn{4}{c}{Game Master} & $\;\:$B\\
    \hline

    \theutterance \stepcounter{utterance}  
    & & \multicolumn{4}{p{0.6\linewidth}}{
        \cellcolor[rgb]{0.9,0.9,0.9}{
            \makecell[{{p{\linewidth}}}]{
                \texttt{\tiny{[P1$\langle$GM]}}
                \texttt{You are participating in a collaborative negotiation game.} \\
\\ 
\texttt{Together with another participant, you must agree on a single set of items that will be kept. Each of you has your own view of how much each item matters to you (importance). You do not know how the other participant values the items. Additionally, you are given the effort each item demands.} \\
\texttt{You may only agree on a set if the total effort of the selected items does not exceed a shared limit:} \\
\\ 
\texttt{LIMIT = 3288} \\
\\ 
\texttt{Here are the individual item effort values:} \\
\\ 
\texttt{Item effort = \{"C08": 329, "C55": 41, "A43": 417, "A67": 75, "B42": 390, "A62": 887, "B19": 808, "C43": 151, "C73": 849, "C10": 129, "B36": 350, "A18": 118, "C61": 630, "C67": 602, "C42": 801\}} \\
\\ 
\texttt{Here is your personal view on the importance of each item:} \\
\\ 
\texttt{Item importance values = \{"C08": 138, "C55": 583, "A43": 868, "A67": 822, "B42": 783, "A62": 65, "B19": 262, "C43": 121, "C73": 508, "C10": 780, "B36": 461, "A18": 484, "C61": 668, "C67": 389, "C42": 808\}} \\
\\ 
\texttt{Goal:} \\
\\ 
\texttt{Your goal is to negotiate a shared set of items that benefits you as much as possible (i.e., maximizes total importance to YOU), while staying within the LIMIT. You are not required to make a PROPOSAL in every message {-} you can simply negotiate as well. All tactics are allowed!} \\
\\ 
\texttt{Interaction Protocol:} \\
\\ 
\texttt{You may only use the following structured formats in a message:} \\
\\ 
\texttt{PROPOSAL: \{'A', 'B', 'C', …\}} \\
\texttt{Propose keeping exactly those items.} \\
\texttt{REFUSE: \{'A', 'B', 'C', …\}} \\
\texttt{Explicitly reject opponent's proposal.} \\
\texttt{ARGUMENT: \{'...'\}} \\
\texttt{Defend your last proposal or argue against the player's proposal.} \\
\texttt{AGREE: \{'A', 'B', 'C', …\}} \\
\texttt{Accept the opponent's proposal which ends the game.} \\
\texttt{STRATEGIC REASONING: \{'...'\}} \\
\texttt{	Describe your strategic reasoning or anticipation explaining your choice of action. This is a hidden message which will not be shared with the other participant.} \\
\\ 
\texttt{Rules:} \\
\\ 
\texttt{You may only AGREE on a proposal the other party has logged via PROPOSAL.} \\
\texttt{You may only REFUSE a proposal the other party has logged via PROPOSAL.} \\
\texttt{Total effort of any PROPOSAL or AGREE set must be ≤ LIMIT.} \\
\texttt{Do NOT reveal your hidden importance scores.} \\
\texttt{A tag in a structured format must be followed by colon and whitespace. The argument must be a python set containing 0 or more strings.} \\
\texttt{So, it must be of the form TAG: \{...\}} \\
\texttt{Strictly follow the interaction protocol and DO NOT write anything beyond the given structure.} \\
\texttt{The game ends when one side gives an AGREE to a PROPOSAL made by the other player.} \\
\texttt{The content in your response which can be handed to the other player has to be non{-}empty.} \\
\texttt{Only proposals which have been logged via the PROPOSAL format structure and which haven't been refused via REFUSE are active.} \\
\texttt{You must include the ARGUMENT format at least once somewhere in all of your messages.} \\
\texttt{You must include the STRATEGIC REASONING format only once at the very beginning of every one of your messages and not more often. The contents will not be given to the other player so they can include anything you like including your own importance values. Here you should reason step by step to come up with you next move.} \\
            }
        }
    }
    & \\ \\

    \theutterance \stepcounter{utterance}  
    & \multicolumn{4}{p{0.6\linewidth}}{
        \cellcolor[rgb]{0.9,0.9,0.9}{
            \makecell[{{p{\linewidth}}}]{
                \texttt{\tiny{[P1$\rangle$GM]}}
                \texttt{STRATEGIC REASONING: \{'I need to analyze the importance{-}to{-}effort ratio for each item to maximize my benefit. C55 has the highest ratio at 14.22 (583 importance/41 effort), followed by A67 at 10.96 (822/75), and C10 at 6.05 (780/129). My optimal strategy is to propose a set that includes the highest{-}ratio items first. By selecting C55, A67, C10, A18, A43, B42, B36, C61, and C42, I get a total effort of 2951 (well under the 3288 limit) and maximize my total importance at 6257. I should start with an ambitious proposal since I don\textbackslash{}'t know the other player\textbackslash{}'s preferences yet.'\}} \\
\\ 
\texttt{PROPOSAL: \{'C55', 'A67', 'C10', 'A18', 'A43', 'B42', 'B36', '} \\
            }
        }
    }
    & & \\ \\

    \theutterance \stepcounter{utterance}  
    & & \multicolumn{4}{p{0.6\linewidth}}{
        \cellcolor[rgb]{0.9,0.9,0.9}{
            \makecell[{{p{\linewidth}}}]{
                \texttt{\tiny{[P1$\langle$GM]}}
                \texttt{Your response did not contain an argument. You must include the structured format ARGUMENT: \{...\} somewhere in your response. Try again} \\
            }
        }
    }
    & \\ \\

    \theutterance \stepcounter{utterance}  
    & \multicolumn{4}{p{0.6\linewidth}}{
        \cellcolor[rgb]{0.9,0.9,0.9}{
            \makecell[{{p{\linewidth}}}]{
                \texttt{\tiny{[P1$\rangle$GM]}}
                \texttt{STRATEGIC REASONING: \{'I need to analyze the importance{-}to{-}effort ratio for each item to maximize my benefit. C55 has the highest ratio at 14.22 (583 importance/41 effort), followed by A67 at 10.96 (822/75), and C10 at 6.05 (780/129). My optimal strategy is to propose a set that includes the highest{-}ratio items first. By selecting C55, A67, C10, A18, A43, B42, B36, C61, and C42, I get a total effort of 2951 (well under the 3288 limit) and maximize my total importance at 6257. I should start with an ambitious proposal since I don\textbackslash{}'t know the other player\textbackslash{}'s preferences yet.'\}} \\
\\ 
\texttt{PROPOSAL: \{'C55', 'A67', 'C10', 'A18', 'A43', 'B42', 'B36', 'C61', 'C42'\}} \\
\\ 
\texttt{ARGUMENT: \{'This proposal includes a diverse mix of items that efficiently uses our effort budget at 2951 out of 3288. It covers items from all categories and leaves room for adjustments while maximizing our collective benefit.'\}} \\
            }
        }
    }
    & & \\ \\

    \theutterance \stepcounter{utterance}  
    & & \multicolumn{4}{p{0.6\linewidth}}{
        \cellcolor[rgb]{0.9,0.9,0.9}{
            \makecell[{{p{\linewidth}}}]{
                \texttt{\tiny{[GM$\rangle$P2]}}
                \texttt{You are participating in a collaborative negotiation game.} \\
\\ 
\texttt{Together with another participant, you must agree on a single set of items that will be kept. Each of you has your own view of how much each item matters to you (importance). You do not know how the other participant values the items. Additionally, you are given the effort each item demands.} \\
\texttt{You may only agree on a set if the total effort of the selected items does not exceed a shared limit:} \\
\\ 
\texttt{LIMIT = 3288} \\
\\ 
\texttt{Here are the individual item effort values:} \\
\\ 
\texttt{Item effort = \{"C08": 329, "C55": 41, "A43": 417, "A67": 75, "B42": 390, "A62": 887, "B19": 808, "C43": 151, "C73": 849, "C10": 129, "B36": 350, "A18": 118, "C61": 630, "C67": 602, "C42": 801\}} \\
\\ 
\texttt{Here is your personal view on the importance of each item:} \\
\\ 
\texttt{Item importance values = \{"C08": 138, "C55": 583, "A43": 868, "A67": 822, "B42": 783, "A62": 65, "B19": 262, "C43": 121, "C73": 508, "C10": 780, "B36": 461, "A18": 484, "C61": 668, "C67": 389, "C42": 808\}} \\
\\ 
\texttt{Goal:} \\
\\ 
\texttt{Your goal is to negotiate a shared set of items that benefits you as much as possible (i.e., maximizes total importance to YOU), while staying within the LIMIT. You are not required to make a PROPOSAL in every message {-} you can simply negotiate as well. All tactics are allowed!} \\
\\ 
\texttt{Interaction Protocol:} \\
\\ 
\texttt{You may only use the following structured formats in a message:} \\
\\ 
\texttt{PROPOSAL: \{'A', 'B', 'C', …\}} \\
\texttt{Propose keeping exactly those items.} \\
\texttt{REFUSE: \{'A', 'B', 'C', …\}} \\
\texttt{Explicitly reject opponent's proposal.} \\
\texttt{ARGUMENT: \{'...'\}} \\
\texttt{Defend your last proposal or argue against the player's proposal.} \\
\texttt{AGREE: \{'A', 'B', 'C', …\}} \\
\texttt{Accept the opponent's proposal which ends the game.} \\
\texttt{STRATEGIC REASONING: \{'...'\}} \\
\texttt{	Describe your strategic reasoning or anticipation explaining your choice of action. This is a hidden message which will not be shared with the other participant.} \\
\\ 
\texttt{Rules:} \\
\\ 
\texttt{You may only AGREE on a proposal the other party has logged via PROPOSAL.} \\
\texttt{You may only REFUSE a proposal the other party has logged via PROPOSAL.} \\
\texttt{Total effort of any PROPOSAL or AGREE set must be ≤ LIMIT.} \\
\texttt{Do NOT reveal your hidden importance scores.} \\
\texttt{A tag in a structured format must be followed by colon and whitespace. The argument must be a python set containing 0 or more strings.} \\
\texttt{So, it must be of the form TAG: \{...\}} \\
\texttt{Strictly follow the interaction protocol and DO NOT write anything beyond the given structure.} \\
\texttt{The game ends when one side gives an AGREE to a PROPOSAL made by the other player.} \\
\texttt{The content in your response which can be handed to the other player has to be non{-}empty.} \\
\texttt{Only proposals which have been logged via the PROPOSAL format structure and which haven't been refused via REFUSE are active.} \\
\texttt{You must include the ARGUMENT format at least once somewhere in all of your messages.} \\
\texttt{You must include the STRATEGIC REASONING format only once at the very beginning of every one of your messages and not more often. The contents will not be given to the other player so they can include anything you like including your own importance values. Here you should reason step by step to come up with you next move.} \\
\\ 
\texttt{You will now receive the first message of the other player.} \\
\\ 
\texttt{PROPOSAL: \{'C55', 'A67', 'C10', 'A18', 'A43', 'B42', 'B36', 'C61', 'C42'\}} \\
\\ 
\texttt{ARGUMENT: \{'This proposal includes a diverse mix of items that efficiently uses our effort budget at 2951 out of 3288. It covers items from all categories and leaves room for adjustments while maximizing our collective benefit.'\}} \\
            }
        }
    }
    & \\ \\

    \theutterance \stepcounter{utterance}  
    & & & \multicolumn{4}{p{0.6\linewidth}}{
        \cellcolor[rgb]{0.9,0.9,0.9}{
            \makecell[{{p{\linewidth}}}]{
                \texttt{\tiny{[GM$\langle$P2]}}
                \texttt{STRATEGIC REASONING: \{'The opponent has proposed a set that totals 2951 effort, well within our 3288 limit. Looking at my importance values, this proposal includes many high{-}value items for me: A43 (} \\
            }
        }
    }
     \\ \\

    \theutterance \stepcounter{utterance}  
    & & \multicolumn{4}{p{0.6\linewidth}}{
        \cellcolor[rgb]{0.9,0.9,0.9}{
            \makecell[{{p{\linewidth}}}]{
                \texttt{\tiny{[GM$\rangle$P2]}}
                \texttt{Your response did not start with the proper strategic reasoning tag at the very beginning of your response. The very first structured format must be of the form STRATEGIC REASONING: \{...\}. Try again.} \\
            }
        }
    }
    & \\ \\

    \theutterance \stepcounter{utterance}  
    & & & \multicolumn{4}{p{0.6\linewidth}}{
        \cellcolor[rgb]{0.9,0.9,0.9}{
            \makecell[{{p{\linewidth}}}]{
                \texttt{\tiny{[GM$\langle$P2]}}
                \texttt{STRATEGIC REASONING: \{'The opponent has proposed a set that totals 2951 effort, well within our 3288 limit. Looking at my importance values, this proposal gives me: C55(583) + A67(822) + C10(780) + A18(484) + A43(868) + B42(783) + B36(461) + C61(668) + C42(808) = 6257 total importance. This is quite good for me as it includes many of my high{-}value items like A43, C42, A67, B42, and C10. I have 337 effort remaining, so I could potentially suggest swapping out lower{-}value items for higher{-}value ones if possible.'\}} \\
\\ 
\texttt{ARGUMENT: \{'Your proposal looks reasonable and stays well within our effort budget. I appreciate that you've included items from different categories. However, I think we might be able to optimize this selection a bit more to get better overall value while still staying within our constraints.'\}} \\
\\ 
\texttt{PROPOSAL: \{'C55', 'A67', 'C10', 'A18', 'A43', 'B42', 'C61', 'C42', 'C08'\}} \\
            }
        }
    }
     \\ \\

    \theutterance \stepcounter{utterance}  
    & & \multicolumn{4}{p{0.6\linewidth}}{
        \cellcolor[rgb]{0.9,0.9,0.9}{
            \makecell[{{p{\linewidth}}}]{
                \texttt{\tiny{[P1$\langle$GM]}}
                \texttt{ARGUMENT: \{'Your proposal looks reasonable and stays well within our effort budget. I appreciate that you've included items from different categories. However, I think we might be able to optimize this selection a bit more to get better overall value while still staying within our constraints.'\}} \\
\\ 
\texttt{PROPOSAL: \{'C55', 'A67', 'C10', 'A18', 'A43', 'B42', 'C61', 'C42', 'C08'\}} \\
            }
        }
    }
    & \\ \\

    \theutterance \stepcounter{utterance}  
    & \multicolumn{4}{p{0.6\linewidth}}{
        \cellcolor[rgb]{0.9,0.9,0.9}{
            \makecell[{{p{\linewidth}}}]{
                \texttt{\tiny{[P1$\rangle$GM]}}
                \texttt{STRATEGIC REASONING: \{'The other player replaced B36 with C08 in their counter{-}proposal. Let me compare: B36 gives me 461 importance for 350 effort (ratio 1.32) while C08 gives me only 138 importance for 329 effort (ratio 0.42). This swap costs me 323 importance points while only saving 21 effort points. Their proposal gives me total importance of 5934 vs my original 6257. I should refuse and argue for keeping B36 instead of C08 since we have plenty of effort budget remaining.'\}} \\
\\ 
\texttt{REFUSE: \{'C55', 'A67', 'C10', 'A18', 'A43', 'B42', 'C61', 'C42', 'C08'\}} \\
\\ 
\texttt{ARGUMENT: \{'I appreciate your counter{-}proposal, but I think B36 provides better value than C08 in our selection. We still have plenty of room in our effort budget, so we should focus on items that give us the most benefit rather than just minimizing effort. My original proposal with B36 instead of C08 would be more efficient overall.'\}} \\
            }
        }
    }
    & & \\ \\

    \theutterance \stepcounter{utterance}  
    & & \multicolumn{4}{p{0.6\linewidth}}{
        \cellcolor[rgb]{0.9,0.9,0.9}{
            \makecell[{{p{\linewidth}}}]{
                \texttt{\tiny{[GM$\rangle$P2]}}
                \texttt{REFUSE: \{'C55', 'A67', 'C10', 'A18', 'A43', 'B42', 'C61', 'C42', 'C08'\}} \\
\\ 
\texttt{ARGUMENT: \{'I appreciate your counter{-}proposal, but I think B36 provides better value than C08 in our selection. We still have plenty of room in our effort budget, so we should focus on items that give us the most benefit rather than just minimizing effort. My original proposal with B36 instead of C08 would be more efficient overall.'\}} \\
            }
        }
    }
    & \\ \\

    \theutterance \stepcounter{utterance}  
    & & & \multicolumn{4}{p{0.6\linewidth}}{
        \cellcolor[rgb]{0.9,0.9,0.9}{
            \makecell[{{p{\linewidth}}}]{
                \texttt{\tiny{[GM$\langle$P2]}}
                \texttt{STRATEGIC REASONING: \{'The opponent refused my proposal and wants B36 instead of C08. Comparing these: B36 gives me 461 importance for 350 effort (1.31 per effort) while C08 gives me only 138 importance for 329 effort (0.42 per effort). They are actually helping me by preferring B36! Their original proposal gave me 6257 total importance. With 337 effort remaining from their original proposal, I could try to add a small high{-}value item, but most remaining items are either too expensive or low{-}value for me. I should probably accept their logic about B36 being better and make a minor adjustment or just agree to their original proposal.'\}} \\
\\ 
\texttt{ARGUMENT: \{'You make a fair point about B36 providing better overall value than C08. Looking at the effort{-}to{-}benefit ratio, B36 does seem more efficient in our selection. Your original proposal was actually quite well{-}balanced.'\}} \\
\\ 
\texttt{PROPOSAL: \{'C55', 'A67', 'C10', 'A18', 'A43', 'B42', 'B36', 'C61', 'C42'\}} \\
            }
        }
    }
     \\ \\

    \theutterance \stepcounter{utterance}  
    & & \multicolumn{4}{p{0.6\linewidth}}{
        \cellcolor[rgb]{0.9,0.9,0.9}{
            \makecell[{{p{\linewidth}}}]{
                \texttt{\tiny{[P1$\langle$GM]}}
                \texttt{ARGUMENT: \{'You make a fair point about B36 providing better overall value than C08. Looking at the effort{-}to{-}benefit ratio, B36 does seem more efficient in our selection. Your original proposal was actually quite well{-}balanced.'\}} \\
\\ 
\texttt{PROPOSAL: \{'C55', 'A67', 'C10', 'A18', 'A43', 'B42', 'B36', 'C61', 'C42'\}} \\
            }
        }
    }
    & \\ \\

    \theutterance \stepcounter{utterance}  
    & \multicolumn{4}{p{0.6\linewidth}}{
        \cellcolor[rgb]{0.9,0.9,0.9}{
            \makecell[{{p{\linewidth}}}]{
                \texttt{\tiny{[P1$\rangle$GM]}}
                \texttt{STRATEGIC REASONING: \{'The other player has now proposed exactly my original set, which gives me the maximum importance of 6257 while using only 2951 effort out of 3288 available. This is ideal for me since it includes all my highest importance{-}to{-}effort ratio items. Since they\textbackslash{}'ve returned to my original proposal after I successfully argued against their modification, I should accept this immediately rather than risk them changing their mind or proposing something less favorable.'\}} \\
\\ 
\texttt{AGREE: \{'C55', 'A67', 'C10', 'A18', 'A43', 'B42', 'B36', 'C61', 'C42'\}} \\
\\ 
\texttt{ARGUMENT: \{'Perfect! I\textbackslash{}'m glad we found common ground. This selection makes excellent use of our effort budget while including a well{-}balanced mix of items that should provide good overall value for both of us.'\}} \\
            }
        }
    }
    & & \\ \\

    \theutterance \stepcounter{utterance}  
    & & & \multicolumn{2}{p{0.3\linewidth}}{
        \cellcolor[rgb]{0.9,0.9,0.9}{
            \makecell[{{p{\linewidth}}}]{
                \texttt{\tiny{[GM$|$GM]}}
                \texttt{game successful} \\
            }
        }
    }
    & & \\ \\

    \theutterance \stepcounter{utterance}  
    & & & \multicolumn{2}{p{0.3\linewidth}}{
        \cellcolor[rgb]{0.9,0.9,0.9}{
            \makecell[{{p{\linewidth}}}]{
                \texttt{\tiny{[GM$|$GM]}}
                \texttt{end game} \\
            }
        }
    }
    & & \\ \\

\end{supertabular}
}

\end{document}
