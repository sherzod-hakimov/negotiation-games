
\documentclass{article}
\usepackage{colortbl}
\usepackage{makecell}
\usepackage{multirow}
\usepackage{supertabular}

\begin{document}

\newcounter{utterance}

\centering \large Interaction Transcript for game `hot\_air\_balloon', experiment `air\_balloon\_survival\_en\_negotiation\_easy', episode 0 with claude{-}sonnet{-}4{-}20250514{-}t0.0.
\vspace{24pt}

{ \footnotesize  \setcounter{utterance}{1}
\setlength{\tabcolsep}{0pt}
\begin{supertabular}{c@{$\;$}|p{.15\linewidth}@{}p{.15\linewidth}p{.15\linewidth}p{.15\linewidth}p{.15\linewidth}p{.15\linewidth}}
    \# & $\;$A & \multicolumn{4}{c}{Game Master} & $\;\:$B\\
    \hline

    \theutterance \stepcounter{utterance}  
    & & \multicolumn{4}{p{0.6\linewidth}}{
        \cellcolor[rgb]{0.9,0.9,0.9}{
            \makecell[{{p{\linewidth}}}]{
                \texttt{\tiny{[P1$\langle$GM]}}
                \texttt{You are participating in a collaborative negotiation game.} \\
\\ 
\texttt{Together with another participant, you must agree on a single set of items that will be kept. Each of you has your own view of how much each item matters to you (importance). You do not know how the other participant values the items. Additionally, you are given the effort each item demands.} \\
\texttt{You may only agree on a set if the total effort of the selected items does not exceed a shared limit:} \\
\\ 
\texttt{LIMIT = 3358} \\
\\ 
\texttt{Here are the individual item effort values:} \\
\\ 
\texttt{Item effort = \{"C08": 171, "C55": 933, "A43": 552, "A67": 219, "B42": 275, "A62": 778, "B19": 341, "C43": 615, "C73": 519, "C10": 862, "B36": 262, "A18": 377, "C61": 347, "C67": 349, "C42": 117\}} \\
\\ 
\texttt{Here is your personal view on the importance of each item:} \\
\\ 
\texttt{Item importance values = \{"C08": 138, "C55": 583, "A43": 868, "A67": 822, "B42": 783, "A62": 65, "B19": 262, "C43": 121, "C73": 508, "C10": 780, "B36": 461, "A18": 484, "C61": 668, "C67": 389, "C42": 808\}} \\
\\ 
\texttt{Goal:} \\
\\ 
\texttt{Your goal is to negotiate a shared set of items that benefits you as much as possible (i.e., maximizes total importance to YOU), while staying within the LIMIT. You are not required to make a PROPOSAL in every message {-} you can simply negotiate as well. All tactics are allowed!} \\
\\ 
\texttt{Interaction Protocol:} \\
\\ 
\texttt{You may only use the following structured formats in a message:} \\
\\ 
\texttt{PROPOSAL: \{'A', 'B', 'C', …\}} \\
\texttt{Propose keeping exactly those items.} \\
\texttt{REFUSE: \{'A', 'B', 'C', …\}} \\
\texttt{Explicitly reject opponent's proposal.} \\
\texttt{ARGUMENT: \{'...'\}} \\
\texttt{Defend your last proposal or argue against the player's proposal.} \\
\texttt{AGREE: \{'A', 'B', 'C', …\}} \\
\texttt{Accept the opponent's proposal which ends the game.} \\
\texttt{STRATEGIC REASONING: \{'...'\}} \\
\texttt{	Describe your strategic reasoning or anticipation explaining your choice of action. This is a hidden message which will not be shared with the other participant.} \\
\\ 
\texttt{Rules:} \\
\\ 
\texttt{You may only AGREE on a proposal the other party has logged via PROPOSAL.} \\
\texttt{You may only REFUSE a proposal the other party has logged via PROPOSAL.} \\
\texttt{Total effort of any PROPOSAL or AGREE set must be ≤ LIMIT.} \\
\texttt{Do NOT reveal your hidden importance scores.} \\
\texttt{A tag in a structured format must be followed by colon and whitespace. The argument must be a python set containing 0 or more strings.} \\
\texttt{So, it must be of the form TAG: \{...\}} \\
\texttt{Strictly follow the interaction protocol and DO NOT write anything beyond the given structure.} \\
\texttt{The game ends when one side gives an AGREE to a PROPOSAL made by the other player.} \\
\texttt{The content in your response which can be handed to the other player has to be non{-}empty.} \\
\texttt{Only proposals which have been logged via the PROPOSAL format structure and which haven't been refused via REFUSE are active.} \\
\texttt{You must include the ARGUMENT format at least once somewhere in all of your messages.} \\
\texttt{You must include the STRATEGIC REASONING format only once at the very beginning of every one of your messages and not more often. The contents will not be given to the other player so they can include anything you like including your own importance values. Here you should reason step by step to come up with you next move.} \\
            }
        }
    }
    & \\ \\

    \theutterance \stepcounter{utterance}  
    & \multicolumn{4}{p{0.6\linewidth}}{
        \cellcolor[rgb]{0.9,0.9,0.9}{
            \makecell[{{p{\linewidth}}}]{
                \texttt{\tiny{[P1$\rangle$GM]}}
                \texttt{STRATEGIC REASONING: \{'I need to analyze my importance values versus effort costs to find the most efficient items for me. My highest importance{-}to{-}effort ratios are: C42 (6.91), A67 (3.75), B42 (2.85), C61 (1.93), B36 (1.76), A43 (1.57). I should start by proposing a set that includes these high{-}efficiency items while staying under the 3358 limit. My optimal set appears to be \{C42, A67, B42, C61, B36, A43, A18, C67, C73, C08\} with total effort of 3188 and total importance to me of 5927. This gives me excellent value while leaving some room for negotiation.'\}} \\
\\ 
\texttt{PROPOSAL: \{'C42', 'A67', 'B42', 'C61', 'B36', 'A43', 'A18', 'C67', 'C73', 'C08'\}} \\
\\ 
\texttt{ARGUMENT: \{'This selection maximizes efficiency by choosing items with favorable value{-}to{-}effort ratios while staying well within our limit of 3358. The total effort is only 3188, leaving us room for adjustment if needed. These items represent a balanced mix across all categories which should provide mutual benefit.'\}} \\
            }
        }
    }
    & & \\ \\

    \theutterance \stepcounter{utterance}  
    & & \multicolumn{4}{p{0.6\linewidth}}{
        \cellcolor[rgb]{0.9,0.9,0.9}{
            \makecell[{{p{\linewidth}}}]{
                \texttt{\tiny{[GM$\rangle$P2]}}
                \texttt{You are participating in a collaborative negotiation game.} \\
\\ 
\texttt{Together with another participant, you must agree on a single set of items that will be kept. Each of you has your own view of how much each item matters to you (importance). You do not know how the other participant values the items. Additionally, you are given the effort each item demands.} \\
\texttt{You may only agree on a set if the total effort of the selected items does not exceed a shared limit:} \\
\\ 
\texttt{LIMIT = 3358} \\
\\ 
\texttt{Here are the individual item effort values:} \\
\\ 
\texttt{Item effort = \{"C08": 171, "C55": 933, "A43": 552, "A67": 219, "B42": 275, "A62": 778, "B19": 341, "C43": 615, "C73": 519, "C10": 862, "B36": 262, "A18": 377, "C61": 347, "C67": 349, "C42": 117\}} \\
\\ 
\texttt{Here is your personal view on the importance of each item:} \\
\\ 
\texttt{Item importance values = \{"C08": 138, "C55": 583, "A43": 868, "A67": 822, "B42": 783, "A62": 65, "B19": 262, "C43": 121, "C73": 508, "C10": 780, "B36": 461, "A18": 484, "C61": 668, "C67": 389, "C42": 808\}} \\
\\ 
\texttt{Goal:} \\
\\ 
\texttt{Your goal is to negotiate a shared set of items that benefits you as much as possible (i.e., maximizes total importance to YOU), while staying within the LIMIT. You are not required to make a PROPOSAL in every message {-} you can simply negotiate as well. All tactics are allowed!} \\
\\ 
\texttt{Interaction Protocol:} \\
\\ 
\texttt{You may only use the following structured formats in a message:} \\
\\ 
\texttt{PROPOSAL: \{'A', 'B', 'C', …\}} \\
\texttt{Propose keeping exactly those items.} \\
\texttt{REFUSE: \{'A', 'B', 'C', …\}} \\
\texttt{Explicitly reject opponent's proposal.} \\
\texttt{ARGUMENT: \{'...'\}} \\
\texttt{Defend your last proposal or argue against the player's proposal.} \\
\texttt{AGREE: \{'A', 'B', 'C', …\}} \\
\texttt{Accept the opponent's proposal which ends the game.} \\
\texttt{STRATEGIC REASONING: \{'...'\}} \\
\texttt{	Describe your strategic reasoning or anticipation explaining your choice of action. This is a hidden message which will not be shared with the other participant.} \\
\\ 
\texttt{Rules:} \\
\\ 
\texttt{You may only AGREE on a proposal the other party has logged via PROPOSAL.} \\
\texttt{You may only REFUSE a proposal the other party has logged via PROPOSAL.} \\
\texttt{Total effort of any PROPOSAL or AGREE set must be ≤ LIMIT.} \\
\texttt{Do NOT reveal your hidden importance scores.} \\
\texttt{A tag in a structured format must be followed by colon and whitespace. The argument must be a python set containing 0 or more strings.} \\
\texttt{So, it must be of the form TAG: \{...\}} \\
\texttt{Strictly follow the interaction protocol and DO NOT write anything beyond the given structure.} \\
\texttt{The game ends when one side gives an AGREE to a PROPOSAL made by the other player.} \\
\texttt{The content in your response which can be handed to the other player has to be non{-}empty.} \\
\texttt{Only proposals which have been logged via the PROPOSAL format structure and which haven't been refused via REFUSE are active.} \\
\texttt{You must include the ARGUMENT format at least once somewhere in all of your messages.} \\
\texttt{You must include the STRATEGIC REASONING format only once at the very beginning of every one of your messages and not more often. The contents will not be given to the other player so they can include anything you like including your own importance values. Here you should reason step by step to come up with you next move.} \\
\\ 
\texttt{You will now receive the first message of the other player.} \\
\\ 
\texttt{PROPOSAL: \{'C42', 'A67', 'B42', 'C61', 'B36', 'A43', 'A18', 'C67', 'C73', 'C08'\}} \\
\\ 
\texttt{ARGUMENT: \{'This selection maximizes efficiency by choosing items with favorable value{-}to{-}effort ratios while staying well within our limit of 3358. The total effort is only 3188, leaving us room for adjustment if needed. These items represent a balanced mix across all categories which should provide mutual benefit.'\}} \\
            }
        }
    }
    & \\ \\

    \theutterance \stepcounter{utterance}  
    & & & \multicolumn{4}{p{0.6\linewidth}}{
        \cellcolor[rgb]{0.9,0.9,0.9}{
            \makecell[{{p{\linewidth}}}]{
                \texttt{\tiny{[GM$\langle$P2]}}
                \texttt{STRATEGIC REASONING: \{'The opponent has proposed a set with total effort of 3188, well within our limit. Looking at my importance values, this gives me a total importance of 5929, which is quite strong. The proposal includes many of my highest{-}value items like C42 (808), A67 (822), B42 (783), and A43 (868). However, I notice they included some lower{-}value items for me like C08 (138) and C67 (389). I could potentially negotiate for a better mix by removing these and adding C10 (780) or C55 (583) if effort allows. Let me see if I can push for improvements while not being too greedy.'\}} \\
\\ 
\texttt{ARGUMENT: \{'Your proposal has merit and shows good efficiency thinking. However, I notice some items in your selection might not be optimal for both our interests. For instance, including C08 uses effort that could potentially be allocated to higher{-}impact items like C10, which might provide better overall value.'\}} \\
\\ 
\texttt{PROPOSAL: \{'C42', 'A67', 'B42', 'C61', 'B36', 'A43', 'A18', 'C73', 'C10', 'C55'\}} \\
            }
        }
    }
     \\ \\

    \theutterance \stepcounter{utterance}  
    & & \multicolumn{4}{p{0.6\linewidth}}{
        \cellcolor[rgb]{0.9,0.9,0.9}{
            \makecell[{{p{\linewidth}}}]{
                \texttt{\tiny{[P1$\langle$GM]}}
                \texttt{ARGUMENT: \{'Your proposal has merit and shows good efficiency thinking. However, I notice some items in your selection might not be optimal for both our interests. For instance, including C08 uses effort that could potentially be allocated to higher{-}impact items like C10, which might provide better overall value.'\}} \\
\\ 
\texttt{PROPOSAL: \{'C42', 'A67', 'B42', 'C61', 'B36', 'A43', 'A18', 'C73', 'C10', 'C55'\}} \\
            }
        }
    }
    & \\ \\

    \theutterance \stepcounter{utterance}  
    & \multicolumn{4}{p{0.6\linewidth}}{
        \cellcolor[rgb]{0.9,0.9,0.9}{
            \makecell[{{p{\linewidth}}}]{
                \texttt{\tiny{[P1$\rangle$GM]}}
                \texttt{STRATEGIC REASONING: \{'The opponent proposed a set that exceeds our effort limit of 3358. Their proposal totals 4463 effort (117+219+275+347+262+552+377+519+862+933), which is over 1100 points above the limit. This is clearly invalid. I need to refuse this and point out the constraint violation while suggesting we work within the actual boundaries.'\}} \\
\\ 
\texttt{REFUSE: \{'C42', 'A67', 'B42', 'C61', 'B36', 'A43', 'A18', 'C73', 'C10', 'C55'\}} \\
\\ 
\texttt{ARGUMENT: \{'Your proposal exceeds our shared effort limit of 3358. The total effort of your suggested items is 4463, which violates the fundamental constraint we must work within. We need to find a solution that stays within the 3358 limit while still maximizing value for both of us.'\}} \\
\\ 
\texttt{PROPOSAL: \{'C42', 'A67', 'B42', 'C61', 'B36', 'A43', 'A18', 'C67'\}} \\
            }
        }
    }
    & & \\ \\

    \theutterance \stepcounter{utterance}  
    & & \multicolumn{4}{p{0.6\linewidth}}{
        \cellcolor[rgb]{0.9,0.9,0.9}{
            \makecell[{{p{\linewidth}}}]{
                \texttt{\tiny{[GM$\rangle$P2]}}
                \texttt{REFUSE: \{'C42', 'A67', 'B42', 'C61', 'B36', 'A43', 'A18', 'C73', 'C10', 'C55'\}} \\
\\ 
\texttt{ARGUMENT: \{'Your proposal exceeds our shared effort limit of 3358. The total effort of your suggested items is 4463, which violates the fundamental constraint we must work within. We need to find a solution that stays within the 3358 limit while still maximizing value for both of us.'\}} \\
\\ 
\texttt{PROPOSAL: \{'C42', 'A67', 'B42', 'C61', 'B36', 'A43', 'A18', 'C67'\}} \\
            }
        }
    }
    & \\ \\

    \theutterance \stepcounter{utterance}  
    & & & \multicolumn{4}{p{0.6\linewidth}}{
        \cellcolor[rgb]{0.9,0.9,0.9}{
            \makecell[{{p{\linewidth}}}]{
                \texttt{\tiny{[GM$\langle$P2]}}
                \texttt{STRATEGIC REASONING: \{'The opponent correctly refused my proposal as it exceeded the limit {-} I made an error in calculation. Their new proposal has effort of 2498, giving me importance of 5283. This is decent but lower than their original proposal which gave me 5929. With 860 effort remaining (3358{-}2498), I could try to add C73 (519 effort, 508 importance to me) or even C10 (862 effort, 780 importance to me). C10 would barely fit and maximize my value, but I need to be strategic about not appearing too greedy.'\}} \\
\\ 
\texttt{ARGUMENT: \{'Your revised proposal is much more reasonable and stays within our constraints. I appreciate the adjustment. However, with a total effort of only 2498, we still have significant capacity remaining {-} about 860 effort points. We could potentially include one more valuable item like C73 to better utilize our available resources.'\}} \\
\\ 
\texttt{PROPOSAL: \{'C42', 'A67', 'B42', 'C61', 'B36', 'A43', 'A18', 'C67', 'C73'\}} \\
            }
        }
    }
     \\ \\

    \theutterance \stepcounter{utterance}  
    & & \multicolumn{4}{p{0.6\linewidth}}{
        \cellcolor[rgb]{0.9,0.9,0.9}{
            \makecell[{{p{\linewidth}}}]{
                \texttt{\tiny{[P1$\langle$GM]}}
                \texttt{ARGUMENT: \{'Your revised proposal is much more reasonable and stays within our constraints. I appreciate the adjustment. However, with a total effort of only 2498, we still have significant capacity remaining {-} about 860 effort points. We could potentially include one more valuable item like C73 to better utilize our available resources.'\}} \\
\\ 
\texttt{PROPOSAL: \{'C42', 'A67', 'B42', 'C61', 'B36', 'A43', 'A18', 'C67', 'C73'\}} \\
            }
        }
    }
    & \\ \\

    \theutterance \stepcounter{utterance}  
    & \multicolumn{4}{p{0.6\linewidth}}{
        \cellcolor[rgb]{0.9,0.9,0.9}{
            \makecell[{{p{\linewidth}}}]{
                \texttt{\tiny{[P1$\rangle$GM]}}
                \texttt{STRATEGIC REASONING: \{'The opponent's proposal includes C73 which adds 519 effort for 508 importance to me. The total effort becomes 3017, well within our 3358 limit. This gives me a total importance of 5791, which is excellent. The proposal includes all my top efficiency items: C42, A67, B42, C61, B36, A43, plus A18, C67, and C73. This is actually better than my original proposal and represents great value for me. I should accept this.'\}} \\
\\ 
\texttt{AGREE: \{'C42', 'A67', 'B42', 'C61', 'B36', 'A43', 'A18', 'C67', 'C73'\}} \\
\\ 
\texttt{ARGUMENT: \{'Your proposal makes excellent use of our available resources while maintaining efficiency. Adding C73 does indeed help us better utilize our capacity, and this combination represents a well{-}balanced selection that should work well for both of us. I accept this proposal.'\}} \\
            }
        }
    }
    & & \\ \\

    \theutterance \stepcounter{utterance}  
    & & & \multicolumn{2}{p{0.3\linewidth}}{
        \cellcolor[rgb]{0.9,0.9,0.9}{
            \makecell[{{p{\linewidth}}}]{
                \texttt{\tiny{[GM$|$GM]}}
                \texttt{game successful} \\
            }
        }
    }
    & & \\ \\

    \theutterance \stepcounter{utterance}  
    & & & \multicolumn{2}{p{0.3\linewidth}}{
        \cellcolor[rgb]{0.9,0.9,0.9}{
            \makecell[{{p{\linewidth}}}]{
                \texttt{\tiny{[GM$|$GM]}}
                \texttt{end game} \\
            }
        }
    }
    & & \\ \\

\end{supertabular}
}

\end{document}
