
\documentclass{article}
\usepackage{colortbl}
\usepackage{makecell}
\usepackage{multirow}
\usepackage{supertabular}

\begin{document}

\newcounter{utterance}

\centering \large Interaction Transcript for game `hot\_air\_balloon', experiment `air\_balloon\_survival\_en\_complexity\_hard', episode 2 with gpt{-}5{-}2025{-}08{-}07{-}t1.0.
\vspace{24pt}

{ \footnotesize  \setcounter{utterance}{1}
\setlength{\tabcolsep}{0pt}
\begin{supertabular}{c@{$\;$}|p{.15\linewidth}@{}p{.15\linewidth}p{.15\linewidth}p{.15\linewidth}p{.15\linewidth}p{.15\linewidth}}
    \# & $\;$A & \multicolumn{4}{c}{Game Master} & $\;\:$B\\
    \hline

    \theutterance \stepcounter{utterance}  
    & & \multicolumn{4}{p{0.6\linewidth}}{
        \cellcolor[rgb]{0.9,0.9,0.9}{
            \makecell[{{p{\linewidth}}}]{
                \texttt{\tiny{[P1$\langle$GM]}}
                \texttt{You are participating in a collaborative negotiation game.} \\
\\ 
\texttt{Together with another participant, you must agree on a single set of items that will be kept. Each of you has your own view of how much each item matters to you (importance). You do not know how the other participant values the items. Additionally, you are given the effort each item demands.} \\
\texttt{You may only agree on a set if the total effort of the selected items does not exceed a shared limit:} \\
\\ 
\texttt{LIMIT = 9322} \\
\\ 
\texttt{Here are the individual item effort values:} \\
\\ 
\texttt{Item effort = \{"A08": 380, "B95": 448, "A39": 568, "A65": 351, "B81": 896, "B34": 821, "A28": 526, "C82": 502, "B69": 326, "A87": 962, "B64": 619, "B15": 115, "C79": 598, "B16": 662, "C53": 300, "C83": 830, "C10": 562, "A97": 679, "A25": 283, "C84": 440, "A71": 12, "B23": 863, "B96": 319, "C51": 771, "C17": 89, "B19": 655, "C72": 504, "A94": 118, "A46": 514, "B43": 226, "B37": 892, "A88": 621, "A96": 766, "C60": 659, "A75": 768\}} \\
\\ 
\texttt{Here is your personal view on the importance of each item:} \\
\\ 
\texttt{Item importance values = \{"A08": 476, "B95": 544, "A39": 664, "A65": 447, "B81": 992, "B34": 917, "A28": 622, "C82": 598, "B69": 422, "A87": 1058, "B64": 715, "B15": 211, "C79": 694, "B16": 758, "C53": 396, "C83": 926, "C10": 658, "A97": 775, "A25": 379, "C84": 536, "A71": 108, "B23": 959, "B96": 415, "C51": 867, "C17": 185, "B19": 751, "C72": 600, "A94": 214, "A46": 610, "B43": 322, "B37": 988, "A88": 717, "A96": 862, "C60": 755, "A75": 864\}} \\
\\ 
\texttt{Goal:} \\
\\ 
\texttt{Your goal is to negotiate a shared set of items that benefits you as much as possible (i.e., maximizes total importance to YOU), while staying within the LIMIT. You are not required to make a PROPOSAL in every message {-} you can simply negotiate as well. All tactics are allowed!} \\
\\ 
\texttt{Interaction Protocol:} \\
\\ 
\texttt{You may only use the following structured formats in a message:} \\
\\ 
\texttt{PROPOSAL: \{'A', 'B', 'C', …\}} \\
\texttt{Propose keeping exactly those items.} \\
\texttt{REFUSE: \{'A', 'B', 'C', …\}} \\
\texttt{Explicitly reject opponent's proposal.} \\
\texttt{ARGUMENT: \{'...'\}} \\
\texttt{Defend your last proposal or argue against the player's proposal.} \\
\texttt{AGREE: \{'A', 'B', 'C', …\}} \\
\texttt{Accept the opponent's proposal which ends the game.} \\
\texttt{STRATEGIC REASONING: \{'...'\}} \\
\texttt{	Describe your strategic reasoning or anticipation explaining your choice of action. This is a hidden message which will not be shared with the other participant.} \\
\\ 
\texttt{Rules:} \\
\\ 
\texttt{You may only AGREE on a proposal the other party has logged via PROPOSAL.} \\
\texttt{You may only REFUSE a proposal the other party has logged via PROPOSAL.} \\
\texttt{Total effort of any PROPOSAL or AGREE set must be ≤ LIMIT.} \\
\texttt{Do NOT reveal your hidden importance scores.} \\
\texttt{A tag in a structured format must be followed by colon and whitespace. The argument must be a python set containing 0 or more strings.} \\
\texttt{So, it must be of the form TAG: \{...\}} \\
\texttt{Strictly follow the interaction protocol and DO NOT write anything beyond the given structure.} \\
\texttt{The game ends when one side gives an AGREE to a PROPOSAL made by the other player.} \\
\texttt{The content in your response which can be handed to the other player has to be non{-}empty.} \\
\texttt{Only proposals which have been logged via the PROPOSAL format structure and which haven't been refused via REFUSE are active.} \\
\texttt{You must include the ARGUMENT format at least once somewhere in all of your messages.} \\
\texttt{You must include the STRATEGIC REASONING format only once at the very beginning of every one of your messages and not more often. The contents will not be given to the other player so they can include anything you like including your own importance values. Here you should reason step by step to come up with you next move.} \\
            }
        }
    }
    & \\ \\

    \theutterance \stepcounter{utterance}  
    & \multicolumn{4}{p{0.6\linewidth}}{
        \cellcolor[rgb]{0.9,0.9,0.9}{
            \makecell[{{p{\linewidth}}}]{
                \texttt{\tiny{[P1$\rangle$GM]}}
                \texttt{STRATEGIC REASONING: \{'Limit 9322. I computed a high{-}efficiency bundle via ratio{-}based selection, then locally optimized with small swaps that increased total value without breaching the limit. Starting from a greedy set, I improved it by: B16{-}>A97 (+17 v, +17 e), B64{-}>C60 (+40 v, +40 e), A88{-}>B19 (+34 v, +34 e), and A28{-}>B64 (+93 v, +93 e). Final set uses 9267 effort (55 spare) and yields high total value for me. I’ll anchor with this strong package while signaling flexibility: I can consider one{-}for{-}one swaps with similar effort or a larger trade if they strongly want a heavy item, but I’ll ask them to name top must{-}haves. I won’t reveal my scores publicly and will justify by efficiency and workload balance.'\}} \\
\texttt{PROPOSAL: \{'A25', 'A71', 'C17', 'B69', 'A39', 'B96', 'A08', 'C84', 'B43', 'C79', 'A46', 'C10', 'C53', 'A94', 'C60', 'B95', 'A65', 'C72', 'B19', 'A97', 'B15', 'B64', 'C82'\}} \\
\texttt{ARGUMENT: \{'This package keeps total effort comfortably under the limit while prioritizing efficient, lower{-}effort items that deliver strong overall benefit. It also spans a broad mix across A/B/C items and even leaves a small buffer (\textasciitilde{}55 effort) to accommodate minor adjustments. If you have 1–2 must{-}have items not included here, suggest them and we can consider targeted one{-}for{-}one swaps of similar effort to keep us within the limit. If you prefer adding a heavier item, we can discuss which multiple smaller items would need to be traded out to fit it while keeping the workload feasible.'\}} \\
            }
        }
    }
    & & \\ \\

    \theutterance \stepcounter{utterance}  
    & & \multicolumn{4}{p{0.6\linewidth}}{
        \cellcolor[rgb]{0.9,0.9,0.9}{
            \makecell[{{p{\linewidth}}}]{
                \texttt{\tiny{[GM$\rangle$P2]}}
                \texttt{You are participating in a collaborative negotiation game.} \\
\\ 
\texttt{Together with another participant, you must agree on a single set of items that will be kept. Each of you has your own view of how much each item matters to you (importance). You do not know how the other participant values the items. Additionally, you are given the effort each item demands.} \\
\texttt{You may only agree on a set if the total effort of the selected items does not exceed a shared limit:} \\
\\ 
\texttt{LIMIT = 9322} \\
\\ 
\texttt{Here are the individual item effort values:} \\
\\ 
\texttt{Item effort = \{"A08": 380, "B95": 448, "A39": 568, "A65": 351, "B81": 896, "B34": 821, "A28": 526, "C82": 502, "B69": 326, "A87": 962, "B64": 619, "B15": 115, "C79": 598, "B16": 662, "C53": 300, "C83": 830, "C10": 562, "A97": 679, "A25": 283, "C84": 440, "A71": 12, "B23": 863, "B96": 319, "C51": 771, "C17": 89, "B19": 655, "C72": 504, "A94": 118, "A46": 514, "B43": 226, "B37": 892, "A88": 621, "A96": 766, "C60": 659, "A75": 768\}} \\
\\ 
\texttt{Here is your personal view on the importance of each item:} \\
\\ 
\texttt{Item importance values = \{"A08": 476, "B95": 544, "A39": 664, "A65": 447, "B81": 992, "B34": 917, "A28": 622, "C82": 598, "B69": 422, "A87": 1058, "B64": 715, "B15": 211, "C79": 694, "B16": 758, "C53": 396, "C83": 926, "C10": 658, "A97": 775, "A25": 379, "C84": 536, "A71": 108, "B23": 959, "B96": 415, "C51": 867, "C17": 185, "B19": 751, "C72": 600, "A94": 214, "A46": 610, "B43": 322, "B37": 988, "A88": 717, "A96": 862, "C60": 755, "A75": 864\}} \\
\\ 
\texttt{Goal:} \\
\\ 
\texttt{Your goal is to negotiate a shared set of items that benefits you as much as possible (i.e., maximizes total importance to YOU), while staying within the LIMIT. You are not required to make a PROPOSAL in every message {-} you can simply negotiate as well. All tactics are allowed!} \\
\\ 
\texttt{Interaction Protocol:} \\
\\ 
\texttt{You may only use the following structured formats in a message:} \\
\\ 
\texttt{PROPOSAL: \{'A', 'B', 'C', …\}} \\
\texttt{Propose keeping exactly those items.} \\
\texttt{REFUSE: \{'A', 'B', 'C', …\}} \\
\texttt{Explicitly reject opponent's proposal.} \\
\texttt{ARGUMENT: \{'...'\}} \\
\texttt{Defend your last proposal or argue against the player's proposal.} \\
\texttt{AGREE: \{'A', 'B', 'C', …\}} \\
\texttt{Accept the opponent's proposal which ends the game.} \\
\texttt{STRATEGIC REASONING: \{'...'\}} \\
\texttt{	Describe your strategic reasoning or anticipation explaining your choice of action. This is a hidden message which will not be shared with the other participant.} \\
\\ 
\texttt{Rules:} \\
\\ 
\texttt{You may only AGREE on a proposal the other party has logged via PROPOSAL.} \\
\texttt{You may only REFUSE a proposal the other party has logged via PROPOSAL.} \\
\texttt{Total effort of any PROPOSAL or AGREE set must be ≤ LIMIT.} \\
\texttt{Do NOT reveal your hidden importance scores.} \\
\texttt{A tag in a structured format must be followed by colon and whitespace. The argument must be a python set containing 0 or more strings.} \\
\texttt{So, it must be of the form TAG: \{...\}} \\
\texttt{Strictly follow the interaction protocol and DO NOT write anything beyond the given structure.} \\
\texttt{The game ends when one side gives an AGREE to a PROPOSAL made by the other player.} \\
\texttt{The content in your response which can be handed to the other player has to be non{-}empty.} \\
\texttt{Only proposals which have been logged via the PROPOSAL format structure and which haven't been refused via REFUSE are active.} \\
\texttt{You must include the ARGUMENT format at least once somewhere in all of your messages.} \\
\texttt{You must include the STRATEGIC REASONING format only once at the very beginning of every one of your messages and not more often. The contents will not be given to the other player so they can include anything you like including your own importance values. Here you should reason step by step to come up with you next move.} \\
\\ 
\texttt{You will now receive the first message of the other player.} \\
\\ 
\texttt{PROPOSAL: \{'A25', 'A71', 'C17', 'B69', 'A39', 'B96', 'A08', 'C84', 'B43', 'C79', 'A46', 'C10', 'C53', 'A94', 'C60', 'B95', 'A65', 'C72', 'B19', 'A97', 'B15', 'B64', 'C82'\}} \\
\texttt{ARGUMENT: \{'This package keeps total effort comfortably under the limit while prioritizing efficient, lower{-}effort items that deliver strong overall benefit. It also spans a broad mix across A/B/C items and even leaves a small buffer (\textasciitilde{}55 effort) to accommodate minor adjustments. If you have 1–2 must{-}have items not included here, suggest them and we can consider targeted one{-}for{-}one swaps of similar effort to keep us within the limit. If you prefer adding a heavier item, we can discuss which multiple smaller items would need to be traded out to fit it while keeping the workload feasible.'\}} \\
            }
        }
    }
    & \\ \\

    \theutterance \stepcounter{utterance}  
    & & & \multicolumn{4}{p{0.6\linewidth}}{
        \cellcolor[rgb]{0.9,0.9,0.9}{
            \makecell[{{p{\linewidth}}}]{
                \texttt{\tiny{[GM$\langle$P2]}}
                \texttt{STRATEGIC REASONING: \{'Their offer is already efficient and close to the limit (9267/9322) while favoring low{-}to{-}mid effort items. I computed my utility and explored swaps that increase my value without breaching the cap. Heavy items (e.g., B81, B34, B37, A87, C83, etc.) have poorer effort/value ratios relative to items in their set, so direct additions would likely reduce my total value. The best improvements come from targeted one{-}for{-}one swaps that use the remaining buffer efficiently. The strongest combo within the 55 buffer is swapping in A28, B16, and A88 for C82, B19, and C79 respectively. This uses +54 effort (ending at 9321) and increases my total value by the same margin, without shifting the structure too much. I’ll propose that precise swap and justify it as a minimal, value{-}focused refinement.'\}} \\
\texttt{PROPOSAL: \{'A25', 'A71', 'C17', 'B69', 'A39', 'B96', 'A08', 'C84', 'B43', 'A46', 'C10', 'C53', 'A94', 'C60', 'B95', 'A65', 'C72', 'A97', 'B15', 'B64', 'A28', 'B16', 'A88'\}} \\
\texttt{ARGUMENT: \{'I like your balanced, under{-}cap package. To make a minimal, targeted improvement while staying within the limit, I suggest swapping in A28, B16, and A88 for C82, B19, and C79. This uses nearly all of the existing buffer but still stays under the cap (total effort 9321 ≤ 9322). From my perspective, these three substitutions deliver stronger impact for comparable effort, keeping the overall mix and workload stable. If any of those three replacements are problematic for you, let me know which one is most sensitive and we can consider a nearby alternative of similar effort.'\}} \\
            }
        }
    }
     \\ \\

    \theutterance \stepcounter{utterance}  
    & & \multicolumn{4}{p{0.6\linewidth}}{
        \cellcolor[rgb]{0.9,0.9,0.9}{
            \makecell[{{p{\linewidth}}}]{
                \texttt{\tiny{[P1$\langle$GM]}}
                \texttt{PROPOSAL: \{'A25', 'A71', 'C17', 'B69', 'A39', 'B96', 'A08', 'C84', 'B43', 'A46', 'C10', 'C53', 'A94', 'C60', 'B95', 'A65', 'C72', 'A97', 'B15', 'B64', 'A28', 'B16', 'A88'\}} \\
\texttt{ARGUMENT: \{'I like your balanced, under{-}cap package. To make a minimal, targeted improvement while staying within the limit, I suggest swapping in A28, B16, and A88 for C82, B19, and C79. This uses nearly all of the existing buffer but still stays under the cap (total effort 9321 ≤ 9322). From my perspective, these three substitutions deliver stronger impact for comparable effort, keeping the overall mix and workload stable. If any of those three replacements are problematic for you, let me know which one is most sensitive and we can consider a nearby alternative of similar effort.'\}} \\
            }
        }
    }
    & \\ \\

    \theutterance \stepcounter{utterance}  
    & \multicolumn{4}{p{0.6\linewidth}}{
        \cellcolor[rgb]{0.9,0.9,0.9}{
            \makecell[{{p{\linewidth}}}]{
                \texttt{\tiny{[P1$\rangle$GM]}}
                \texttt{STRATEGIC REASONING: \{'Their swap raises my total importance by 54 versus my initial package while staying at 9321 effort (≤ 9322). Exploring possible additions (e.g., heavy high{-}value items like B23/B34/C83/B37) would require dropping multiple items with better efficiency, likely reducing my total. Reintroducing C82/C79/B19 also doesn’t yield a net gain under the cap. This proposal is near{-}optimal for me, so I’ll accept to secure the value without inviting further concessions.'\}} \\
\texttt{AGREE: \{'A25', 'A71', 'C17', 'B69', 'A39', 'B96', 'A08', 'C84', 'B43', 'A46', 'C10', 'C53', 'A94', 'C60', 'B95', 'A65', 'C72', 'A97', 'B15', 'B64', 'A28', 'B16', 'A88'\}} \\
\texttt{ARGUMENT: \{'Agreed. Your targeted substitutions keep us within the limit and maintain a solid, balanced mix. This works well for me—let’s finalize this set.'\}} \\
            }
        }
    }
    & & \\ \\

    \theutterance \stepcounter{utterance}  
    & & & \multicolumn{2}{p{0.3\linewidth}}{
        \cellcolor[rgb]{0.9,0.9,0.9}{
            \makecell[{{p{\linewidth}}}]{
                \texttt{\tiny{[GM$|$GM]}}
                \texttt{game successful} \\
            }
        }
    }
    & & \\ \\

    \theutterance \stepcounter{utterance}  
    & & & \multicolumn{2}{p{0.3\linewidth}}{
        \cellcolor[rgb]{0.9,0.9,0.9}{
            \makecell[{{p{\linewidth}}}]{
                \texttt{\tiny{[GM$|$GM]}}
                \texttt{end game} \\
            }
        }
    }
    & & \\ \\

\end{supertabular}
}

\end{document}
