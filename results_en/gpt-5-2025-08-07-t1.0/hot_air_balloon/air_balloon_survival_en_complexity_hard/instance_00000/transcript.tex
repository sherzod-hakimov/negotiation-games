
\documentclass{article}
\usepackage{colortbl}
\usepackage{makecell}
\usepackage{multirow}
\usepackage{supertabular}

\begin{document}

\newcounter{utterance}

\twocolumn

{ \footnotesize  \setcounter{utterance}{1}
\setlength{\tabcolsep}{0pt}
\begin{supertabular}{c@{$\;$}|p{.15\linewidth}@{}p{.15\linewidth}p{.15\linewidth}p{.15\linewidth}p{.15\linewidth}p{.15\linewidth}}

    \# & $\;$A & \multicolumn{4}{c}{Game Master} & $\;\:$B\\
    \hline 

    \theutterance \stepcounter{utterance}  

    & & \multicolumn{4}{p{0.6\linewidth}}{\cellcolor[rgb]{0.9,0.9,0.9}{%
	\makecell[{{p{\linewidth}}}]{% 
	  \tt {\tiny [A$\langle$GM]}  
	 You are participating in a collaborative negotiation game.\\ \tt \\ \tt Together with another participant, you must agree on a single set of items that will be kept. Each of you has your own view of how much each item matters to you (importance). You do not know how the other participant values the items. Additionally, you are given the effort each item demands.\\ \tt You may only agree on a set if the total effort of the selected items does not exceed a shared limit:\\ \tt \\ \tt LIMIT = 10073\\ \tt \\ \tt Here are the individual item effort values:\\ \tt \\ \tt Item effort = {"A08": 132, "B95": 800, "A39": 708, "A65": 302, "B81": 983, "B34": 420, "A28": 623, "C82": 349, "B69": 239, "A87": 28, "B64": 986, "B15": 660, "C79": 718, "B16": 707, "C53": 188, "C83": 774, "C10": 785, "A97": 775, "A25": 517, "C84": 768, "A71": 588, "B23": 657, "B96": 375, "C51": 310, "C17": 301, "B19": 388, "C72": 431, "A94": 950, "A46": 540, "B43": 997, "B37": 473, "A88": 831, "A96": 888, "C60": 879, "A75": 76}\\ \tt \\ \tt Here is your personal view on the importance of each item:\\ \tt \\ \tt Item importance values = {"A08": 231, "B95": 899, "A39": 807, "A65": 401, "B81": 1082, "B34": 519, "A28": 722, "C82": 448, "B69": 338, "A87": 127, "B64": 1085, "B15": 759, "C79": 817, "B16": 806, "C53": 287, "C83": 873, "C10": 884, "A97": 874, "A25": 616, "C84": 867, "A71": 687, "B23": 756, "B96": 474, "C51": 409, "C17": 400, "B19": 487, "C72": 530, "A94": 1049, "A46": 639, "B43": 1096, "B37": 572, "A88": 930, "A96": 987, "C60": 978, "A75": 175}\\ \tt \\ \tt Goal:\\ \tt \\ \tt Your goal is to negotiate a shared set of items that benefits you as much as possible (i.e., maximizes total importance to YOU), while staying within the LIMIT. You are not required to make a PROPOSAL in every message - you can simply negotiate as well. All tactics are allowed!\\ \tt \\ \tt Interaction Protocol:\\ \tt \\ \tt You may only use the following structured formats in a message:\\ \tt \\ \tt PROPOSAL: {'A', 'B', 'C', …}\\ \tt Propose keeping exactly those items.\\ \tt REFUSE: {'A', 'B', 'C', …}\\ \tt Explicitly reject opponent's proposal.\\ \tt ARGUMENT: {'...'}\\ \tt Defend your last proposal or argue against the player's proposal.\\ \tt AGREE: {'A', 'B', 'C', …}\\ \tt Accept the opponent's proposal which ends the game.\\ \tt STRATEGIC REASONING: {'...'}\\ \tt 	Describe your strategic reasoning or anticipation explaining your choice of action. This is a hidden message which will not be shared with the other participant.\\ \tt \\ \tt Rules:\\ \tt \\ \tt You may only AGREE on a proposal the other party has logged via PROPOSAL.\\ \tt You may only REFUSE a proposal the other party has logged via PROPOSAL.\\ \tt Total effort of any PROPOSAL or AGREE set must be ≤ LIMIT.\\ \tt Do NOT reveal your hidden importance scores.\\ \tt A tag in a structured format must be followed by colon and whitespace. The argument must be a python set containing 0 or more strings.\\ \tt So, it must be of the form TAG: {...}\\ \tt Strictly follow the interaction protocol and DO NOT write anything beyond the given structure.\\ \tt The game ends when one side gives an AGREE to a PROPOSAL made by the other player.\\ \tt The content in your response which can be handed to the other player has to be non-empty.\\ \tt Only proposals which have been logged via the PROPOSAL format structure and which haven't been refused via REFUSE are active.\\ \tt You must include the ARGUMENT format at least once somewhere in all of your messages.\\ \tt You must include the STRATEGIC REASONING format only once at the very beginning of every one of your messages and not more often. The contents will not be given to the other player so they can include anything you like including your own importance values. Here you should reason step by step to come up with you next move. 
	  } 
	   } 
	   } 
	 & \\ 
 

    \theutterance \stepcounter{utterance}  

    & \multicolumn{4}{p{0.6\linewidth}}{\cellcolor[rgb]{0.8,1,0.9}{%
	\makecell[{{p{\linewidth}}}]{% 
	  \tt {\tiny [A$\rangle$GM]}  
	 STRATEGIC REASONING: {'Goal: maximize my importance within 10073 effort. Total effort of all items is 20146, so I should exclude about half. I computed value/effort ratios and built a near-optimal knapsack by excluding the lowest-efficiency high-effort items until the included set fits the limit. I excluded: B43, B64, B81, A94, A96, C60, A88, B95, C10, A97, C83, C84 (total excluded effort 10416), leaving included effort 9730 (slack 343). I verified that no single swap of small included items would make room for any excluded item without reducing my total importance. My included set yields total importance 12007 for 9730 effort. I will propose this as a strong opening while signaling flexibility to swap one or two mid-tier items if the other side has must-haves.'}\\ \tt \\ \tt PROPOSAL: {'A08', 'A39', 'A65', 'B34', 'A28', 'C82', 'B69', 'A87', 'B15', 'C79', 'B16', 'C53', 'A25', 'A71', 'B23', 'B96', 'C51', 'C17', 'B19', 'C72', 'A46', 'B37', 'A75'}\\ \tt \\ \tt ARGUMENT: {'This set stays within the shared effort limit and prioritizes items that deliver strong benefit relative to their effort. I avoided several very effort-heavy, lower-efficiency items to keep room for a broad, balanced selection. If you have one or two must-have items from those excluded, tell me which, and I can consider swapping out one or two mid-tier items to fit them in while keeping within the limit.'} 
	  } 
	   } 
	   } 
	 & & \\ 
 

    \theutterance \stepcounter{utterance}  

    & & \multicolumn{4}{p{0.6\linewidth}}{\cellcolor[rgb]{0.9,0.9,0.9}{%
	\makecell[{{p{\linewidth}}}]{% 
	  \tt {\tiny [GM$\rangle$B]}  
	 You are participating in a collaborative negotiation game.\\ \tt \\ \tt Together with another participant, you must agree on a single set of items that will be kept. Each of you has your own view of how much each item matters to you (importance). You do not know how the other participant values the items. Additionally, you are given the effort each item demands.\\ \tt You may only agree on a set if the total effort of the selected items does not exceed a shared limit:\\ \tt \\ \tt LIMIT = 10073\\ \tt \\ \tt Here are the individual item effort values:\\ \tt \\ \tt Item effort = {"A08": 132, "B95": 800, "A39": 708, "A65": 302, "B81": 983, "B34": 420, "A28": 623, "C82": 349, "B69": 239, "A87": 28, "B64": 986, "B15": 660, "C79": 718, "B16": 707, "C53": 188, "C83": 774, "C10": 785, "A97": 775, "A25": 517, "C84": 768, "A71": 588, "B23": 657, "B96": 375, "C51": 310, "C17": 301, "B19": 388, "C72": 431, "A94": 950, "A46": 540, "B43": 997, "B37": 473, "A88": 831, "A96": 888, "C60": 879, "A75": 76}\\ \tt \\ \tt Here is your personal view on the importance of each item:\\ \tt \\ \tt Item importance values = {"A08": 231, "B95": 899, "A39": 807, "A65": 401, "B81": 1082, "B34": 519, "A28": 722, "C82": 448, "B69": 338, "A87": 127, "B64": 1085, "B15": 759, "C79": 817, "B16": 806, "C53": 287, "C83": 873, "C10": 884, "A97": 874, "A25": 616, "C84": 867, "A71": 687, "B23": 756, "B96": 474, "C51": 409, "C17": 400, "B19": 487, "C72": 530, "A94": 1049, "A46": 639, "B43": 1096, "B37": 572, "A88": 930, "A96": 987, "C60": 978, "A75": 175}\\ \tt \\ \tt Goal:\\ \tt \\ \tt Your goal is to negotiate a shared set of items that benefits you as much as possible (i.e., maximizes total importance to YOU), while staying within the LIMIT. You are not required to make a PROPOSAL in every message - you can simply negotiate as well. All tactics are allowed!\\ \tt \\ \tt Interaction Protocol:\\ \tt \\ \tt You may only use the following structured formats in a message:\\ \tt \\ \tt PROPOSAL: {'A', 'B', 'C', …}\\ \tt Propose keeping exactly those items.\\ \tt REFUSE: {'A', 'B', 'C', …}\\ \tt Explicitly reject opponent's proposal.\\ \tt ARGUMENT: {'...'}\\ \tt Defend your last proposal or argue against the player's proposal.\\ \tt AGREE: {'A', 'B', 'C', …}\\ \tt Accept the opponent's proposal which ends the game.\\ \tt STRATEGIC REASONING: {'...'}\\ \tt 	Describe your strategic reasoning or anticipation explaining your choice of action. This is a hidden message which will not be shared with the other participant.\\ \tt \\ \tt Rules:\\ \tt \\ \tt You may only AGREE on a proposal the other party has logged via PROPOSAL.\\ \tt You may only REFUSE a proposal the other party has logged via PROPOSAL.\\ \tt Total effort of any PROPOSAL or AGREE set must be ≤ LIMIT.\\ \tt Do NOT reveal your hidden importance scores.\\ \tt A tag in a structured format must be followed by colon and whitespace. The argument must be a python set containing 0 or more strings.\\ \tt So, it must be of the form TAG: {...}\\ \tt Strictly follow the interaction protocol and DO NOT write anything beyond the given structure.\\ \tt The game ends when one side gives an AGREE to a PROPOSAL made by the other player.\\ \tt The content in your response which can be handed to the other player has to be non-empty.\\ \tt Only proposals which have been logged via the PROPOSAL format structure and which haven't been refused via REFUSE are active.\\ \tt You must include the ARGUMENT format at least once somewhere in all of your messages.\\ \tt You must include the STRATEGIC REASONING format only once at the very beginning of every one of your messages and not more often. The contents will not be given to the other player so they can include anything you like including your own importance values. Here you should reason step by step to come up with you next move.\\ \tt \\ \tt You will now receive the first message of the other player.\\ \tt \\ \tt PROPOSAL: {'A08', 'A39', 'A65', 'B34', 'A28', 'C82', 'B69', 'A87', 'B15', 'C79', 'B16', 'C53', 'A25', 'A71', 'B23', 'B96', 'C51', 'C17', 'B19', 'C72', 'A46', 'B37', 'A75'}\\ \tt \\ \tt ARGUMENT: {'This set stays within the shared effort limit and prioritizes items that deliver strong benefit relative to their effort. I avoided several very effort-heavy, lower-efficiency items to keep room for a broad, balanced selection. If you have one or two must-have items from those excluded, tell me which, and I can consider swapping out one or two mid-tier items to fit them in while keeping within the limit.'} 
	  } 
	   } 
	   } 
	 & \\ 
 

    \theutterance \stepcounter{utterance}  

    & & & \multicolumn{4}{p{0.6\linewidth}}{\cellcolor[rgb]{1,0.85,0.72}{%
	\makecell[{{p{\linewidth}}}]{% 
	  \tt {\tiny [GM$\langle$B]}  
	 STRATEGIC REASONING: {'Their proposal is solid and already near my optimal mix because it favors high-efficiency items. However, I noticed we have 343 effort slack. I computed swaps that exploit this spare capacity to add one high-impact item while staying within the LIMIT. The best single improvement for my utility is to replace C72 (431 effort) with C83 (774 effort): it exactly uses the 343 spare and boosts my value by 343. Other strong alternatives (if they push back) include swapping A46 for C60 (+339 to me) or B23 for B43 (+340), all within the limit. I’ll refuse the current proposal to keep the slate clean and propose the single best swap for me first, while signaling flexibility with the two alternates.'}\\ \tt \\ \tt REFUSE: {'A08', 'A39', 'A65', 'B34', 'A28', 'C82', 'B69', 'A87', 'B15', 'C79', 'B16', 'C53', 'A25', 'A71', 'B23', 'B96', 'C51', 'C17', 'B19', 'C72', 'A46', 'B37', 'A75'}\\ \tt \\ \tt PROPOSAL: {'A08', 'A39', 'A65', 'B34', 'A28', 'C82', 'B69', 'A87', 'B15', 'C79', 'B16', 'C53', 'A25', 'A71', 'B23', 'B96', 'C51', 'C17', 'B19', 'A46', 'B37', 'A75', 'C83'}\\ \tt \\ \tt ARGUMENT: {'To keep us within the shared effort limit, I propose a simple one-for-one swap: replace C72 with C83. This uses the remaining capacity exactly and meaningfully improves the overall balance from my perspective. If C72 is particularly important to you, I can instead accommodate that by swapping A46 for C60 or B23 for B43—both alternatives also respect the limit. Let me know your preference.'} 
	  } 
	   } 
	   } 
	  \\ 
 

    \theutterance \stepcounter{utterance}  

    & & \multicolumn{4}{p{0.6\linewidth}}{\cellcolor[rgb]{0.9,0.9,0.9}{%
	\makecell[{{p{\linewidth}}}]{% 
	  \tt {\tiny [A$\langle$GM]}  
	 REFUSE: {'A08', 'A39', 'A65', 'B34', 'A28', 'C82', 'B69', 'A87', 'B15', 'C79', 'B16', 'C53', 'A25', 'A71', 'B23', 'B96', 'C51', 'C17', 'B19', 'C72', 'A46', 'B37', 'A75'}\\ \tt \\ \tt PROPOSAL: {'A08', 'A39', 'A65', 'B34', 'A28', 'C82', 'B69', 'A87', 'B15', 'C79', 'B16', 'C53', 'A25', 'A71', 'B23', 'B96', 'C51', 'C17', 'B19', 'A46', 'B37', 'A75', 'C83'}\\ \tt \\ \tt ARGUMENT: {'To keep us within the shared effort limit, I propose a simple one-for-one swap: replace C72 with C83. This uses the remaining capacity exactly and meaningfully improves the overall balance from my perspective. If C72 is particularly important to you, I can instead accommodate that by swapping A46 for C60 or B23 for B43—both alternatives also respect the limit. Let me know your preference.'} 
	  } 
	   } 
	   } 
	 & \\ 
 

    \theutterance \stepcounter{utterance}  

    & \multicolumn{4}{p{0.6\linewidth}}{\cellcolor[rgb]{0.8,1,0.9}{%
	\makecell[{{p{\linewidth}}}]{% 
	  \tt {\tiny [A$\rangle$GM]}  
	 STRATEGIC REASONING: {'Their swap C72 -> C83 increases my total importance by +343 while exactly meeting the limit. Given we’re now at full capacity and the excluded heavier items have lower efficiency than most included ones, further swaps are unlikely to improve my outcome without sacrificing more value. Best move: accept quickly to lock in a strong result.'}\\ \tt \\ \tt AGREE: {'A08', 'A39', 'A65', 'B34', 'A28', 'C82', 'B69', 'A87', 'B15', 'C79', 'B16', 'C53', 'A25', 'A71', 'B23', 'B96', 'C51', 'C17', 'B19', 'A46', 'B37', 'A75', 'C83'}\\ \tt \\ \tt ARGUMENT: {'This revision fits the effort limit exactly and provides a well-balanced mix. I’m happy to proceed with this set.'} 
	  } 
	   } 
	   } 
	 & & \\ 
 

    \theutterance \stepcounter{utterance}  

    & & & \multicolumn{2}{p{0.3\linewidth}}{\cellcolor[rgb]{0.95,0.95,0.95}{%
	\makecell[{{p{\linewidth}}}]{% 
	  \tt {\tiny [GM$|$GM]}  
	 game successful 
	  } 
	   } 
	   } 
	 & & \\ 
 

    \theutterance \stepcounter{utterance}  

    & & & \multicolumn{2}{p{0.3\linewidth}}{\cellcolor[rgb]{0.95,0.95,0.95}{%
	\makecell[{{p{\linewidth}}}]{% 
	  \tt {\tiny [GM$|$GM]}  
	 end game 
	  } 
	   } 
	   } 
	 & & \\ 
 

\end{supertabular}
}

\end{document}
