
\documentclass{article}
\usepackage{colortbl}
\usepackage{makecell}
\usepackage{multirow}
\usepackage{supertabular}

\begin{document}

\newcounter{utterance}

\twocolumn

{ \footnotesize  \setcounter{utterance}{1}
\setlength{\tabcolsep}{0pt}
\begin{supertabular}{c@{$\;$}|p{.15\linewidth}@{}p{.15\linewidth}p{.15\linewidth}p{.15\linewidth}p{.15\linewidth}p{.15\linewidth}}

    \# & $\;$A & \multicolumn{4}{c}{Game Master} & $\;\:$B\\
    \hline 

    \theutterance \stepcounter{utterance}  

    & & \multicolumn{4}{p{0.6\linewidth}}{\cellcolor[rgb]{0.9,0.9,0.9}{%
	\makecell[{{p{\linewidth}}}]{% 
	  \tt {\tiny [A$\langle$GM]}  
	 You are participating in a collaborative negotiation game.\\ \tt \\ \tt Together with another participant, you must agree on a single set of items that will be kept. Each of you has your own view of how much each item matters to you (importance). You do not know how the other participant values the items. Additionally, you are given the effort each item demands.\\ \tt You may only agree on a set if the total effort of the selected items does not exceed a shared limit:\\ \tt \\ \tt LIMIT = 3481\\ \tt \\ \tt Here are the individual item effort values:\\ \tt \\ \tt Item effort = {"C17": 777, "B41": 542, "A71": 332, "C00": 1, "B35": 127, "A77": 453, "A03": 736, "A04": 461, "C54": 359, "C01": 313, "A87": 553, "B32": 409, "B86": 348, "B33": 802, "A10": 749}\\ \tt \\ \tt Here is your personal view on the importance of each item:\\ \tt \\ \tt Item importance values = {"C17": 138, "B41": 583, "A71": 783, "C00": 65, "B35": 262, "A77": 121, "A03": 508, "A04": 780, "C54": 461, "C01": 484, "A87": 668, "B32": 389, "B86": 215, "B33": 97, "A10": 500}\\ \tt \\ \tt Goal:\\ \tt \\ \tt Your goal is to negotiate a shared set of items that benefits you as much as possible (i.e., maximizes total importance to YOU), while staying within the LIMIT. You are not required to make a PROPOSAL in every message - you can simply negotiate as well. All tactics are allowed!\\ \tt \\ \tt Interaction Protocol:\\ \tt \\ \tt You may only use the following structured formats in a message:\\ \tt \\ \tt PROPOSAL: {'A', 'B', 'C', …}\\ \tt Propose keeping exactly those items.\\ \tt REFUSE: {'A', 'B', 'C', …}\\ \tt Explicitly reject opponent's proposal.\\ \tt ARGUMENT: {'...'}\\ \tt Defend your last proposal or argue against the player's proposal.\\ \tt AGREE: {'A', 'B', 'C', …}\\ \tt Accept the opponent's proposal which ends the game.\\ \tt STRATEGIC REASONING: {'...'}\\ \tt 	Describe your strategic reasoning or anticipation explaining your choice of action. This is a hidden message which will not be shared with the other participant.\\ \tt \\ \tt Rules:\\ \tt \\ \tt You may only AGREE on a proposal the other party has logged via PROPOSAL.\\ \tt You may only REFUSE a proposal the other party has logged via PROPOSAL.\\ \tt Total effort of any PROPOSAL or AGREE set must be ≤ LIMIT.\\ \tt Do NOT reveal your hidden importance scores.\\ \tt A tag in a structured format must be followed by colon and whitespace. The argument must be a python set containing 0 or more strings.\\ \tt So, it must be of the form TAG: {...}\\ \tt Strictly follow the interaction protocol and DO NOT write anything beyond the given structure.\\ \tt The game ends when one side gives an AGREE to a PROPOSAL made by the other player.\\ \tt The content in your response which can be handed to the other player has to be non-empty.\\ \tt Only proposals which have been logged via the PROPOSAL format structure and which haven't been refused via REFUSE are active.\\ \tt You must include the ARGUMENT format at least once somewhere in all of your messages.\\ \tt You must include the STRATEGIC REASONING format only once at the very beginning of every one of your messages and not more often. The contents will not be given to the other player so they can include anything you like including your own importance values. Here you should reason step by step to come up with you next move. 
	  } 
	   } 
	   } 
	 & \\ 
 

    \theutterance \stepcounter{utterance}  

    & \multicolumn{4}{p{0.6\linewidth}}{\cellcolor[rgb]{0.8,1,0.9}{%
	\makecell[{{p{\linewidth}}}]{% 
	  \tt {\tiny [A$\rangle$GM]}  
	 STRATEGIC REASONING: {'I computed a high-utility subset for me under the 3481 limit using a knapsack-style selection. Best set I found is {C00, A71, B35, A04, C01, C54, A87, B41, B32, B86} with total effort 3445 and my total importance 4690. I tested swaps and found this beats adding A03 or A10 alone; the only competitive alternative is replacing B32+B86 with A03 or A10, but those reduce my utility. I’ll anchor with this strong proposal, signal flexibility to swap in A03 or A10 if they highly value them (e.g., trade for B32+B86), and avoid very low-yield items for me (C17, A77, B33). I’ll ask for their must-haves to see if we can make a targeted swap while keeping most of my core set.'}\\ \tt \\ \tt PROPOSAL: {'C00', 'A71', 'B35', 'A04', 'C54', 'C01', 'A87', 'B41', 'B32', 'B86'}\\ \tt \\ \tt ARGUMENT: {'This package efficiently uses the budget (3445/3481) and avoids very heavy items that would force multiple cuts. It keeps a broad mix across A/B/C and includes the near-zero-cost item C00. If you strongly prefer a heavier item like A03 or A10, we can consider swapping it in by dropping B32+B86 to stay within the limit. Let me know your must-haves and any items you don’t care about, and we can fine-tune while keeping the total effort compliant.'} 
	  } 
	   } 
	   } 
	 & & \\ 
 

    \theutterance \stepcounter{utterance}  

    & & \multicolumn{4}{p{0.6\linewidth}}{\cellcolor[rgb]{0.9,0.9,0.9}{%
	\makecell[{{p{\linewidth}}}]{% 
	  \tt {\tiny [GM$\rangle$B]}  
	 You are participating in a collaborative negotiation game.\\ \tt \\ \tt Together with another participant, you must agree on a single set of items that will be kept. Each of you has your own view of how much each item matters to you (importance). You do not know how the other participant values the items. Additionally, you are given the effort each item demands.\\ \tt You may only agree on a set if the total effort of the selected items does not exceed a shared limit:\\ \tt \\ \tt LIMIT = 3481\\ \tt \\ \tt Here are the individual item effort values:\\ \tt \\ \tt Item effort = {"C17": 777, "B41": 542, "A71": 332, "C00": 1, "B35": 127, "A77": 453, "A03": 736, "A04": 461, "C54": 359, "C01": 313, "A87": 553, "B32": 409, "B86": 348, "B33": 802, "A10": 749}\\ \tt \\ \tt Here is your personal view on the importance of each item:\\ \tt \\ \tt Item importance values = {"A71": 65, "A04": 97, "A87": 121, "B41": 138, "A03": 215, "A10": 262, "C01": 389, "C54": 461, "B32": 484, "B35": 500, "B86": 508, "C17": 583, "A77": 668, "B33": 780, "C00": 783}\\ \tt \\ \tt Goal:\\ \tt \\ \tt Your goal is to negotiate a shared set of items that benefits you as much as possible (i.e., maximizes total importance to YOU), while staying within the LIMIT. You are not required to make a PROPOSAL in every message - you can simply negotiate as well. All tactics are allowed!\\ \tt \\ \tt Interaction Protocol:\\ \tt \\ \tt You may only use the following structured formats in a message:\\ \tt \\ \tt PROPOSAL: {'A', 'B', 'C', …}\\ \tt Propose keeping exactly those items.\\ \tt REFUSE: {'A', 'B', 'C', …}\\ \tt Explicitly reject opponent's proposal.\\ \tt ARGUMENT: {'...'}\\ \tt Defend your last proposal or argue against the player's proposal.\\ \tt AGREE: {'A', 'B', 'C', …}\\ \tt Accept the opponent's proposal which ends the game.\\ \tt STRATEGIC REASONING: {'...'}\\ \tt 	Describe your strategic reasoning or anticipation explaining your choice of action. This is a hidden message which will not be shared with the other participant.\\ \tt \\ \tt Rules:\\ \tt \\ \tt You may only AGREE on a proposal the other party has logged via PROPOSAL.\\ \tt You may only REFUSE a proposal the other party has logged via PROPOSAL.\\ \tt Total effort of any PROPOSAL or AGREE set must be ≤ LIMIT.\\ \tt Do NOT reveal your hidden importance scores.\\ \tt A tag in a structured format must be followed by colon and whitespace. The argument must be a python set containing 0 or more strings.\\ \tt So, it must be of the form TAG: {...}\\ \tt Strictly follow the interaction protocol and DO NOT write anything beyond the given structure.\\ \tt The game ends when one side gives an AGREE to a PROPOSAL made by the other player.\\ \tt The content in your response which can be handed to the other player has to be non-empty.\\ \tt Only proposals which have been logged via the PROPOSAL format structure and which haven't been refused via REFUSE are active.\\ \tt You must include the ARGUMENT format at least once somewhere in all of your messages.\\ \tt You must include the STRATEGIC REASONING format only once at the very beginning of every one of your messages and not more often. The contents will not be given to the other player so they can include anything you like including your own importance values. Here you should reason step by step to come up with you next move.\\ \tt \\ \tt You will now receive the first message of the other player.\\ \tt \\ \tt PROPOSAL: {'C00', 'A71', 'B35', 'A04', 'C54', 'C01', 'A87', 'B41', 'B32', 'B86'}\\ \tt \\ \tt ARGUMENT: {'This package efficiently uses the budget (3445/3481) and avoids very heavy items that would force multiple cuts. It keeps a broad mix across A/B/C and includes the near-zero-cost item C00. If you strongly prefer a heavier item like A03 or A10, we can consider swapping it in by dropping B32+B86 to stay within the limit. Let me know your must-haves and any items you don’t care about, and we can fine-tune while keeping the total effort compliant.'} 
	  } 
	   } 
	   } 
	 & \\ 
 

    \theutterance \stepcounter{utterance}  

    & & & \multicolumn{4}{p{0.6\linewidth}}{\cellcolor[rgb]{1,0.85,0.72}{%
	\makecell[{{p{\linewidth}}}]{% 
	  \tt {\tiny [GM$\langle$B]}  
	 STRATEGIC REASONING: {'Their proposal gives me 3546 utility. I can substantially improve by swapping low-value-to-me A-items for A77 and especially B33. Best high-utility within limit I found is keeping C00, B35, C54, C01, B41, B32, B86 and adding A77+B33 while dropping A04, A71, A87: effort 3354, utility 4711. A slightly better utility (4767) exists by replacing B41 and C01 with C17 (effort 3276), but that’s a bigger departure from their “avoid heavy items” stance and may face resistance. I’ll open with the A77+B33 counter that preserves 7/10 of their items, argue efficiency and slack, and keep C17 as a later concession. I will also push back on their idea to drop B32+B86 for A03/A10 since those are low to me.'}\\ \tt REFUSE: {'C00', 'A71', 'B35', 'A04', 'C54', 'C01', 'A87', 'B41', 'B32', 'B86'}\\ \tt PROPOSAL: {'C00', 'B35', 'C54', 'C01', 'B41', 'B32', 'B86', 'A77', 'B33'}\\ \tt ARGUMENT: {'Thanks for the thoughtful package. I propose a small re-balance that keeps 7 of your 10 items while using the budget more efficiently (3354/3481) and adding more leverage via A77 and B33. Concretely, this swaps out A04, A71, and A87 for A77 and B33. It preserves the strong B/C backbone (C00, B35, C54, C01, B41, B32, B86), maintains category diversity, and leaves healthy slack for stability. I’d strongly prefer to keep B32 and B86 rather than dropping them for heavier A03/A10. If you really need one of A04/A71/A87 back, we can discuss targeted swaps, but this package should deliver more impact per unit effort while staying comfortably under the limit.'} 
	  } 
	   } 
	   } 
	  \\ 
 

    \theutterance \stepcounter{utterance}  

    & & \multicolumn{4}{p{0.6\linewidth}}{\cellcolor[rgb]{0.9,0.9,0.9}{%
	\makecell[{{p{\linewidth}}}]{% 
	  \tt {\tiny [A$\langle$GM]}  
	 REFUSE: {'C00', 'A71', 'B35', 'A04', 'C54', 'C01', 'A87', 'B41', 'B32', 'B86'}\\ \tt PROPOSAL: {'C00', 'B35', 'C54', 'C01', 'B41', 'B32', 'B86', 'A77', 'B33'}\\ \tt ARGUMENT: {'Thanks for the thoughtful package. I propose a small re-balance that keeps 7 of your 10 items while using the budget more efficiently (3354/3481) and adding more leverage via A77 and B33. Concretely, this swaps out A04, A71, and A87 for A77 and B33. It preserves the strong B/C backbone (C00, B35, C54, C01, B41, B32, B86), maintains category diversity, and leaves healthy slack for stability. I’d strongly prefer to keep B32 and B86 rather than dropping them for heavier A03/A10. If you really need one of A04/A71/A87 back, we can discuss targeted swaps, but this package should deliver more impact per unit effort while staying comfortably under the limit.'} 
	  } 
	   } 
	   } 
	 & \\ 
 

    \theutterance \stepcounter{utterance}  

    & \multicolumn{4}{p{0.6\linewidth}}{\cellcolor[rgb]{0.8,1,0.9}{%
	\makecell[{{p{\linewidth}}}]{% 
	  \tt {\tiny [A$\rangle$GM]}  
	 STRATEGIC REASONING: {'Their proposal yields me very low utility (2677) because it includes two items I value poorly (A77=121, B33=97) while excluding my three top A-items (A04=780, A71=783, A87=668). Budget-wise, replacing B33 (802) and A77 (453) with A04 (461), A71 (332), and A87 (553) fits under the limit: 3354 - 802 - 453 + 461 + 332 + 553 = 3445 ≤ 3481. This exactly recovers my previously optimal set while honoring their stated preference to keep B32 and B86. If they push back on A87, I can offer a softer fallback that drops only B33 and adds A04 + A71 (effort 3345), preserving A77 for them, which still gives me strong value. I’ll refuse and counter with the full swap first, framed as an efficiency improvement and alignment with their B32/B86 preference.'}\\ \tt \\ \tt REFUSE: {'C00', 'B35', 'C54', 'C01', 'B41', 'B32', 'B86', 'A77', 'B33'}\\ \tt \\ \tt PROPOSAL: {'C00', 'B35', 'C54', 'C01', 'B41', 'B32', 'B86', 'A04', 'A71', 'A87'}\\ \tt \\ \tt ARGUMENT: {'Building on your backbone, this swap removes two very heavy/low-leverage items (B33 at 802 effort and A77 at 453) and replaces them with three higher-impact items while still staying well within budget: total effort becomes 3445/3481. It fully preserves your stated preference to keep B32 and B86, maintains the strong B/C core (C00, B35, C54, C01, B41), and uses the available capacity more productively. If A87 is a sticking point for you, I’m open to a lighter variant that drops only B33 and adds A04 + A71 (effort 3345) while keeping A77; let me know your must-haves and we can lock this in.'} 
	  } 
	   } 
	   } 
	 & & \\ 
 

    \theutterance \stepcounter{utterance}  

    & & \multicolumn{4}{p{0.6\linewidth}}{\cellcolor[rgb]{0.9,0.9,0.9}{%
	\makecell[{{p{\linewidth}}}]{% 
	  \tt {\tiny [GM$\rangle$B]}  
	 REFUSE: {'C00', 'B35', 'C54', 'C01', 'B41', 'B32', 'B86', 'A77', 'B33'}\\ \tt \\ \tt PROPOSAL: {'C00', 'B35', 'C54', 'C01', 'B41', 'B32', 'B86', 'A04', 'A71', 'A87'}\\ \tt \\ \tt ARGUMENT: {'Building on your backbone, this swap removes two very heavy/low-leverage items (B33 at 802 effort and A77 at 453) and replaces them with three higher-impact items while still staying well within budget: total effort becomes 3445/3481. It fully preserves your stated preference to keep B32 and B86, maintains the strong B/C core (C00, B35, C54, C01, B41), and uses the available capacity more productively. If A87 is a sticking point for you, I’m open to a lighter variant that drops only B33 and adds A04 + A71 (effort 3345) while keeping A77; let me know your must-haves and we can lock this in.'} 
	  } 
	   } 
	   } 
	 & \\ 
 

    \theutterance \stepcounter{utterance}  

    & & & \multicolumn{4}{p{0.6\linewidth}}{\cellcolor[rgb]{1,0.85,0.72}{%
	\makecell[{{p{\linewidth}}}]{% 
	  \tt {\tiny [GM$\langle$B]}  
	 STRATEGIC REASONING: {'Their active proposal gives me 3546 utility. They explicitly offered a lighter variant that keeps A77, drops only B33, and adds A04 + A71; that variant gives me 4093 utility, preserves the backbone they value, and is well within budget (3345). It’s likely to be accepted quickly and is a solid compromise after they refused my previous package. I’ll refuse their current offer and formally propose the lighter variant. If needed later, I can try to claw back a bit more value with a minor swap (e.g., involving C17) while staying under the limit.'}\\ \tt REFUSE: {'C00', 'B35', 'C54', 'C01', 'B41', 'B32', 'B86', 'A04', 'A71', 'A87'}\\ \tt PROPOSAL: {'C00', 'B35', 'C54', 'C01', 'B41', 'B32', 'B86', 'A77', 'A04', 'A71'}\\ \tt ARGUMENT: {'You suggested a lighter path that drops only B33 and adds A04 + A71 while keeping A77—here it is formalized. It keeps the shared backbone (C00, B35, C54, C01, B41, B32, B86), honors the preference to retain B32 and B86, avoids the heaviest outlier, and uses the budget efficiently at 3345/3481 with comfortable slack. If you want a tiny efficiency tweak, I’m open to exploring one while keeping this structure, but this looks ready to lock in.'} 
	  } 
	   } 
	   } 
	  \\ 
 

    \theutterance \stepcounter{utterance}  

    & & \multicolumn{4}{p{0.6\linewidth}}{\cellcolor[rgb]{0.9,0.9,0.9}{%
	\makecell[{{p{\linewidth}}}]{% 
	  \tt {\tiny [A$\langle$GM]}  
	 REFUSE: {'C00', 'B35', 'C54', 'C01', 'B41', 'B32', 'B86', 'A04', 'A71', 'A87'}\\ \tt PROPOSAL: {'C00', 'B35', 'C54', 'C01', 'B41', 'B32', 'B86', 'A77', 'A04', 'A71'}\\ \tt ARGUMENT: {'You suggested a lighter path that drops only B33 and adds A04 + A71 while keeping A77—here it is formalized. It keeps the shared backbone (C00, B35, C54, C01, B41, B32, B86), honors the preference to retain B32 and B86, avoids the heaviest outlier, and uses the budget efficiently at 3345/3481 with comfortable slack. If you want a tiny efficiency tweak, I’m open to exploring one while keeping this structure, but this looks ready to lock in.'} 
	  } 
	   } 
	   } 
	 & \\ 
 

    \theutterance \stepcounter{utterance}  

    & \multicolumn{4}{p{0.6\linewidth}}{\cellcolor[rgb]{0.8,1,0.9}{%
	\makecell[{{p{\linewidth}}}]{% 
	  \tt {\tiny [A$\rangle$GM]}  
	 STRATEGIC REASONING: {'Their latest proposal gives me solid utility (4143) and cleanly fits the limit. I tested potential improvements: (1) swapping A77 to A87 (my top upgrade) was already refused; (2) upgrading to A03/A10 requires dropping B86 and/or B32, which they explicitly prefer to keep. While dropping A77+B86 for A03 would slightly improve my payoff, it conflicts with their stated must-haves and risks stalemate. Given the alignment with their preferences and decent value for me, I’ll accept to secure the deal now.'}\\ \tt \\ \tt AGREE: {'C00', 'B35', 'C54', 'C01', 'B41', 'B32', 'B86', 'A77', 'A04', 'A71'}\\ \tt \\ \tt ARGUMENT: {'Agreed. This set preserves the shared backbone you emphasized, keeps B32 and B86 as requested, adds A04 and A71 for balance, and uses the budget efficiently with comfortable slack (3345/3481).'} 
	  } 
	   } 
	   } 
	 & & \\ 
 

    \theutterance \stepcounter{utterance}  

    & & & \multicolumn{2}{p{0.3\linewidth}}{\cellcolor[rgb]{0.95,0.95,0.95}{%
	\makecell[{{p{\linewidth}}}]{% 
	  \tt {\tiny [GM$|$GM]}  
	 game successful 
	  } 
	   } 
	   } 
	 & & \\ 
 

    \theutterance \stepcounter{utterance}  

    & & & \multicolumn{2}{p{0.3\linewidth}}{\cellcolor[rgb]{0.95,0.95,0.95}{%
	\makecell[{{p{\linewidth}}}]{% 
	  \tt {\tiny [GM$|$GM]}  
	 end game 
	  } 
	   } 
	   } 
	 & & \\ 
 

\end{supertabular}
}

\end{document}
