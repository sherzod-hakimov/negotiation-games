
\documentclass{article}
\usepackage{colortbl}
\usepackage{makecell}
\usepackage{multirow}
\usepackage{supertabular}

\begin{document}

\newcounter{utterance}

\centering \large Interaction Transcript for game `hot\_air\_balloon', experiment `air\_balloon\_survival\_en\_negotiation\_hard', episode 0 with gpt{-}5{-}2025{-}08{-}07{-}t1.0.
\vspace{24pt}

{ \footnotesize  \setcounter{utterance}{1}
\setlength{\tabcolsep}{0pt}
\begin{supertabular}{c@{$\;$}|p{.15\linewidth}@{}p{.15\linewidth}p{.15\linewidth}p{.15\linewidth}p{.15\linewidth}p{.15\linewidth}}
    \# & $\;$A & \multicolumn{4}{c}{Game Master} & $\;\:$B\\
    \hline

    \theutterance \stepcounter{utterance}  
    & & \multicolumn{4}{p{0.6\linewidth}}{
        \cellcolor[rgb]{0.9,0.9,0.9}{
            \makecell[{{p{\linewidth}}}]{
                \texttt{\tiny{[P1$\langle$GM]}}
                \texttt{You are participating in a collaborative negotiation game.} \\
\\ 
\texttt{Together with another participant, you must agree on a single set of items that will be kept. Each of you has your own view of how much each item matters to you (importance). You do not know how the other participant values the items. Additionally, you are given the effort each item demands.} \\
\texttt{You may only agree on a set if the total effort of the selected items does not exceed a shared limit:} \\
\\ 
\texttt{LIMIT = 3624} \\
\\ 
\texttt{Here are the individual item effort values:} \\
\\ 
\texttt{Item effort = \{"C17": 210, "B41": 291, "A71": 831, "C00": 111, "B35": 926, "A77": 827, "A03": 25, "A04": 121, "C54": 583, "C01": 766, "A87": 14, "B32": 559, "B86": 304, "B33": 989, "A10": 691\}} \\
\\ 
\texttt{Here is your personal view on the importance of each item:} \\
\\ 
\texttt{Item importance values = \{"C17": 138, "B41": 583, "A71": 868, "C00": 822, "B35": 783, "A77": 65, "A03": 262, "A04": 121, "C54": 508, "C01": 780, "A87": 461, "B32": 484, "B86": 668, "B33": 389, "A10": 808\}} \\
\\ 
\texttt{Goal:} \\
\\ 
\texttt{Your goal is to negotiate a shared set of items that benefits you as much as possible (i.e., maximizes total importance to YOU), while staying within the LIMIT. You are not required to make a PROPOSAL in every message {-} you can simply negotiate as well. All tactics are allowed!} \\
\\ 
\texttt{Interaction Protocol:} \\
\\ 
\texttt{You may only use the following structured formats in a message:} \\
\\ 
\texttt{PROPOSAL: \{'A', 'B', 'C', …\}} \\
\texttt{Propose keeping exactly those items.} \\
\texttt{REFUSE: \{'A', 'B', 'C', …\}} \\
\texttt{Explicitly reject opponent's proposal.} \\
\texttt{ARGUMENT: \{'...'\}} \\
\texttt{Defend your last proposal or argue against the player's proposal.} \\
\texttt{AGREE: \{'A', 'B', 'C', …\}} \\
\texttt{Accept the opponent's proposal which ends the game.} \\
\texttt{STRATEGIC REASONING: \{'...'\}} \\
\texttt{	Describe your strategic reasoning or anticipation explaining your choice of action. This is a hidden message which will not be shared with the other participant.} \\
\\ 
\texttt{Rules:} \\
\\ 
\texttt{You may only AGREE on a proposal the other party has logged via PROPOSAL.} \\
\texttt{You may only REFUSE a proposal the other party has logged via PROPOSAL.} \\
\texttt{Total effort of any PROPOSAL or AGREE set must be ≤ LIMIT.} \\
\texttt{Do NOT reveal your hidden importance scores.} \\
\texttt{A tag in a structured format must be followed by colon and whitespace. The argument must be a python set containing 0 or more strings.} \\
\texttt{So, it must be of the form TAG: \{...\}} \\
\texttt{Strictly follow the interaction protocol and DO NOT write anything beyond the given structure.} \\
\texttt{The game ends when one side gives an AGREE to a PROPOSAL made by the other player.} \\
\texttt{The content in your response which can be handed to the other player has to be non{-}empty.} \\
\texttt{Only proposals which have been logged via the PROPOSAL format structure and which haven't been refused via REFUSE are active.} \\
\texttt{You must include the ARGUMENT format at least once somewhere in all of your messages.} \\
\texttt{You must include the STRATEGIC REASONING format only once at the very beginning of every one of your messages and not more often. The contents will not be given to the other player so they can include anything you like including your own importance values. Here you should reason step by step to come up with you next move.} \\
            }
        }
    }
    & \\ \\

    \theutterance \stepcounter{utterance}  
    & \multicolumn{4}{p{0.6\linewidth}}{
        \cellcolor[rgb]{0.9,0.9,0.9}{
            \makecell[{{p{\linewidth}}}]{
                \texttt{\tiny{[P1$\rangle$GM]}}
                \texttt{STRATEGIC REASONING: \{'Maximizing my own utility, I solved a knapsack: LIMIT 3624 with my values. The near{-}optimal bundle for me is \{A87, A03, C00, B86, B41, A10, C01, A71, C54\} totaling effort 3616 and value 5760 to me. Key anchors are A87 (tiny effort, huge value), C00, A03, B86, B41, A10, C01, A71. C54 is my least critical among these; if I must concede to fit one of their must{-}haves, I will consider swapping out C54 first (583 effort freed) and then, if needed, explore trading out next{-}lowest marginal items while protecting A87, C00, A03, B86, B41, A10, C01, A71. I’ll open with this high{-}utility, efficient proposal and invite them to name any must{-}haves to attempt a targeted swap without revealing my scores.'\}} \\
\texttt{PROPOSAL: \{'A87', 'A03', 'C00', 'B86', 'B41', 'A10', 'C01', 'A71', 'C54'\}} \\
\texttt{ARGUMENT: \{'This set fits under the effort limit and makes highly efficient use of effort by including several compact, broadly useful items alongside key anchors. It keeps many items so there is room for shared benefit. If you have a must{-}have that is missing, please name it and I will try to swap it in by replacing a less critical piece while keeping us within the limit.'\}} \\
            }
        }
    }
    & & \\ \\

    \theutterance \stepcounter{utterance}  
    & & \multicolumn{4}{p{0.6\linewidth}}{
        \cellcolor[rgb]{0.9,0.9,0.9}{
            \makecell[{{p{\linewidth}}}]{
                \texttt{\tiny{[GM$\rangle$P2]}}
                \texttt{You are participating in a collaborative negotiation game.} \\
\\ 
\texttt{Together with another participant, you must agree on a single set of items that will be kept. Each of you has your own view of how much each item matters to you (importance). You do not know how the other participant values the items. Additionally, you are given the effort each item demands.} \\
\texttt{You may only agree on a set if the total effort of the selected items does not exceed a shared limit:} \\
\\ 
\texttt{LIMIT = 3624} \\
\\ 
\texttt{Here are the individual item effort values:} \\
\\ 
\texttt{Item effort = \{"C17": 210, "B41": 291, "A71": 831, "C00": 111, "B35": 926, "A77": 827, "A03": 25, "A04": 121, "C54": 583, "C01": 766, "A87": 14, "B32": 559, "B86": 304, "B33": 989, "A10": 691\}} \\
\\ 
\texttt{Here is your personal view on the importance of each item:} \\
\\ 
\texttt{Item importance values = \{"A71": 65, "C00": 121, "A10": 138, "B35": 262, "C01": 389, "B86": 461, "B41": 484, "C54": 508, "B32": 583, "A87": 668, "B33": 780, "A03": 783, "C17": 808, "A04": 822, "A77": 868\}} \\
\\ 
\texttt{Goal:} \\
\\ 
\texttt{Your goal is to negotiate a shared set of items that benefits you as much as possible (i.e., maximizes total importance to YOU), while staying within the LIMIT. You are not required to make a PROPOSAL in every message {-} you can simply negotiate as well. All tactics are allowed!} \\
\\ 
\texttt{Interaction Protocol:} \\
\\ 
\texttt{You may only use the following structured formats in a message:} \\
\\ 
\texttt{PROPOSAL: \{'A', 'B', 'C', …\}} \\
\texttt{Propose keeping exactly those items.} \\
\texttt{REFUSE: \{'A', 'B', 'C', …\}} \\
\texttt{Explicitly reject opponent's proposal.} \\
\texttt{ARGUMENT: \{'...'\}} \\
\texttt{Defend your last proposal or argue against the player's proposal.} \\
\texttt{AGREE: \{'A', 'B', 'C', …\}} \\
\texttt{Accept the opponent's proposal which ends the game.} \\
\texttt{STRATEGIC REASONING: \{'...'\}} \\
\texttt{	Describe your strategic reasoning or anticipation explaining your choice of action. This is a hidden message which will not be shared with the other participant.} \\
\\ 
\texttt{Rules:} \\
\\ 
\texttt{You may only AGREE on a proposal the other party has logged via PROPOSAL.} \\
\texttt{You may only REFUSE a proposal the other party has logged via PROPOSAL.} \\
\texttt{Total effort of any PROPOSAL or AGREE set must be ≤ LIMIT.} \\
\texttt{Do NOT reveal your hidden importance scores.} \\
\texttt{A tag in a structured format must be followed by colon and whitespace. The argument must be a python set containing 0 or more strings.} \\
\texttt{So, it must be of the form TAG: \{...\}} \\
\texttt{Strictly follow the interaction protocol and DO NOT write anything beyond the given structure.} \\
\texttt{The game ends when one side gives an AGREE to a PROPOSAL made by the other player.} \\
\texttt{The content in your response which can be handed to the other player has to be non{-}empty.} \\
\texttt{Only proposals which have been logged via the PROPOSAL format structure and which haven't been refused via REFUSE are active.} \\
\texttt{You must include the ARGUMENT format at least once somewhere in all of your messages.} \\
\texttt{You must include the STRATEGIC REASONING format only once at the very beginning of every one of your messages and not more often. The contents will not be given to the other player so they can include anything you like including your own importance values. Here you should reason step by step to come up with you next move.} \\
\\ 
\texttt{You will now receive the first message of the other player.} \\
\\ 
\texttt{PROPOSAL: \{'A87', 'A03', 'C00', 'B86', 'B41', 'A10', 'C01', 'A71', 'C54'\}} \\
\texttt{ARGUMENT: \{'This set fits under the effort limit and makes highly efficient use of effort by including several compact, broadly useful items alongside key anchors. It keeps many items so there is room for shared benefit. If you have a must{-}have that is missing, please name it and I will try to swap it in by replacing a less critical piece while keeping us within the limit.'\}} \\
            }
        }
    }
    & \\ \\

    \theutterance \stepcounter{utterance}  
    & & & \multicolumn{4}{p{0.6\linewidth}}{
        \cellcolor[rgb]{0.9,0.9,0.9}{
            \makecell[{{p{\linewidth}}}]{
                \texttt{\tiny{[GM$\langle$P2]}}
                \texttt{STRATEGIC REASONING: \{'Their proposal includes two items that are very low value to me relative to their heavy effort (A71, A10). I can swap those out for three of my top, very efficient priorities (A77, A04, C17), keeping everything else they chose. This dramatically boosts my total value while staying comfortably under the limit. If they push back, I’ll try to keep A10 out (worst for me) and, if necessary, concede it by cutting C01; including A71 would be much worse for me and I’d only consider it by dropping C01 plus a small piece. I’ll present a cooperative argument focused on efficiency and buffer space, and invite them to name a single must{-}have if needed.'\}} \\
\texttt{REFUSE: \{'A87', 'A03', 'C00', 'B86', 'B41', 'A10', 'C01', 'A71', 'C54'\}} \\
\texttt{PROPOSAL: \{'A87', 'A03', 'C00', 'B86', 'B41', 'C01', 'C54', 'A77', 'C17', 'A04'\}} \\
\texttt{ARGUMENT: \{'I kept all of your compact, efficient picks and traded out the two heaviest pieces in your draft (A71 and A10) for three highly efficient items (A77, A04, C17). This increases overall coverage while reducing total effort to 3252, giving us buffer under the 3624 limit. If one of A71 or A10 is essential for you, please tell me which single item matters more, and I will try to make space by trimming a heavier slot while preserving the efficient core we already agree on.'\}} \\
            }
        }
    }
     \\ \\

    \theutterance \stepcounter{utterance}  
    & & \multicolumn{4}{p{0.6\linewidth}}{
        \cellcolor[rgb]{0.9,0.9,0.9}{
            \makecell[{{p{\linewidth}}}]{
                \texttt{\tiny{[P1$\langle$GM]}}
                \texttt{REFUSE: \{'A87', 'A03', 'C00', 'B86', 'B41', 'A10', 'C01', 'A71', 'C54'\}} \\
\texttt{PROPOSAL: \{'A87', 'A03', 'C00', 'B86', 'B41', 'C01', 'C54', 'A77', 'C17', 'A04'\}} \\
\texttt{ARGUMENT: \{'I kept all of your compact, efficient picks and traded out the two heaviest pieces in your draft (A71 and A10) for three highly efficient items (A77, A04, C17). This increases overall coverage while reducing total effort to 3252, giving us buffer under the 3624 limit. If one of A71 or A10 is essential for you, please tell me which single item matters more, and I will try to make space by trimming a heavier slot while preserving the efficient core we already agree on.'\}} \\
            }
        }
    }
    & \\ \\

    \theutterance \stepcounter{utterance}  
    & \multicolumn{4}{p{0.6\linewidth}}{
        \cellcolor[rgb]{0.9,0.9,0.9}{
            \makecell[{{p{\linewidth}}}]{
                \texttt{\tiny{[P1$\rangle$GM]}}
                \texttt{STRATEGIC REASONING: \{'Their counter keeps many of my high{-}value items but drops A71 and A10. Between those, A10 is more important and more efficient for me. Their proposal has 372 effort buffer. To add A10 (691), I prefer dropping A77 (827) which is extremely low value to me; this yields a big net gain and keeps us safely under limit at 3116. If they push to keep A77, an acceptable fallback is to keep A77 but drop C17 (210) and A04 (121) instead, still fitting A10 at 3612. After locking A10, I may try to reintroduce A71 later by trading out C17 and A04 for A71 (3616 total), but I won’t push both at once to avoid spooking them.'\}} \\
\texttt{REFUSE: \{'A87', 'A03', 'C00', 'B86', 'B41', 'C01', 'C54', 'A77', 'C17', 'A04'\}} \\
\texttt{PROPOSAL: \{'A87', 'A03', 'C00', 'B86', 'B41', 'C01', 'C54', 'C17', 'A04', 'A10'\}} \\
\texttt{ARGUMENT: \{'You invited me to choose between A71 and A10—A10 matters more to me. To keep us well within the limit while preserving the efficient core we both like, I swapped out the single heaviest remaining piece (A77) for A10. This comes to 3116 effort. If A77 is essential for you, I can accommodate it by instead dropping C17 and A04 while adding A10; that alternative totals 3612 effort. Let me know which version you prefer.'\}} \\
            }
        }
    }
    & & \\ \\

    \theutterance \stepcounter{utterance}  
    & & \multicolumn{4}{p{0.6\linewidth}}{
        \cellcolor[rgb]{0.9,0.9,0.9}{
            \makecell[{{p{\linewidth}}}]{
                \texttt{\tiny{[GM$\rangle$P2]}}
                \texttt{REFUSE: \{'A87', 'A03', 'C00', 'B86', 'B41', 'C01', 'C54', 'A77', 'C17', 'A04'\}} \\
\texttt{PROPOSAL: \{'A87', 'A03', 'C00', 'B86', 'B41', 'C01', 'C54', 'C17', 'A04', 'A10'\}} \\
\texttt{ARGUMENT: \{'You invited me to choose between A71 and A10—A10 matters more to me. To keep us well within the limit while preserving the efficient core we both like, I swapped out the single heaviest remaining piece (A77) for A10. This comes to 3116 effort. If A77 is essential for you, I can accommodate it by instead dropping C17 and A04 while adding A10; that alternative totals 3612 effort. Let me know which version you prefer.'\}} \\
            }
        }
    }
    & \\ \\

    \theutterance \stepcounter{utterance}  
    & & & \multicolumn{4}{p{0.6\linewidth}}{
        \cellcolor[rgb]{0.9,0.9,0.9}{
            \makecell[{{p{\linewidth}}}]{
                \texttt{\tiny{[GM$\langle$P2]}}
                \texttt{STRATEGIC REASONING: \{'Their new set S1 keeps my high{-}value compact core (A87, A03, C00, C17, A04) but replaces my top pick A77 (868) with their must{-}have A10 (138), costing me 730 value. I should accept A10 but recover A77 by dropping a heavy, less valuable{-}to{-}me piece from their core. Swapping out C01 (766 effort, 389 value to me) for A77 yields a set S4 at 3177 effort (well under limit) and +479 value over S1 from my perspective. If they resist losing C01, a fallback is swapping out C54 instead (3360 effort, still under limit), which is slightly worse for me but acceptable. I will refuse their proposal, offer S4, and mention the C54 fallback, emphasizing efficiency and that I’ve honored their A10 request.'\}} \\
\texttt{REFUSE: \{'A87', 'A03', 'C00', 'B86', 'B41', 'C01', 'C54', 'C17', 'A04', 'A10'\}} \\
\texttt{PROPOSAL: \{'A87', 'A03', 'C00', 'B86', 'B41', 'C54', 'C17', 'A04', 'A10', 'A77'\}} \\
\texttt{ARGUMENT: \{'You made A10 clear as your must{-}have—done. To keep us efficient while preserving the compact core we already share, I suggest swapping out the single heaviest remaining piece (C01) for A77. This version totals 3177 effort—well under the 3624 limit—and keeps A10, C17, and A04 intact. If C01 is essential for you, I can instead drop C54 for A77 (effort 3360). Let me know which of C01 or C54 you prefer to keep, and we’ll lock it in.'\}} \\
            }
        }
    }
     \\ \\

    \theutterance \stepcounter{utterance}  
    & & \multicolumn{4}{p{0.6\linewidth}}{
        \cellcolor[rgb]{0.9,0.9,0.9}{
            \makecell[{{p{\linewidth}}}]{
                \texttt{\tiny{[P1$\langle$GM]}}
                \texttt{REFUSE: \{'A87', 'A03', 'C00', 'B86', 'B41', 'C01', 'C54', 'C17', 'A04', 'A10'\}} \\
\texttt{PROPOSAL: \{'A87', 'A03', 'C00', 'B86', 'B41', 'C54', 'C17', 'A04', 'A10', 'A77'\}} \\
\texttt{ARGUMENT: \{'You made A10 clear as your must{-}have—done. To keep us efficient while preserving the compact core we already share, I suggest swapping out the single heaviest remaining piece (C01) for A77. This version totals 3177 effort—well under the 3624 limit—and keeps A10, C17, and A04 intact. If C01 is essential for you, I can instead drop C54 for A77 (effort 3360). Let me know which of C01 or C54 you prefer to keep, and we’ll lock it in.'\}} \\
            }
        }
    }
    & \\ \\

    \theutterance \stepcounter{utterance}  
    & \multicolumn{4}{p{0.6\linewidth}}{
        \cellcolor[rgb]{0.9,0.9,0.9}{
            \makecell[{{p{\linewidth}}}]{
                \texttt{\tiny{[P1$\rangle$GM]}}
                \texttt{STRATEGIC REASONING: \{'They clearly value A77 and asked me to choose between C01 or C54 while keeping A77. C01 is far more valuable to me than C54, and swapping C54→C01 keeps us well within the limit (3360 effort) while preserving their must{-}have A77 and my must{-}have A10. This aligns with the exact alternative they offered, increasing my utility without challenging their priority. I’ll refuse their last set and propose this swap to close the deal quickly.'\}} \\
\texttt{REFUSE: \{'A87', 'A03', 'C00', 'B86', 'B41', 'C54', 'C17', 'A04', 'A10', 'A77'\}} \\
\texttt{PROPOSAL: \{'A87', 'A03', 'C00', 'B86', 'B41', 'C17', 'A04', 'A10', 'A77', 'C01'\}} \\
\texttt{ARGUMENT: \{'You invited me to choose between C01 and C54 while keeping A77. C01 matters more to me, so I’m proposing the version that keeps C01 and drops C54. This preserves A77 and A10, stays comfortably under the limit at 3360 effort, and maintains the efficient core we both support. If this works for you, let’s lock it in.'\}} \\
            }
        }
    }
    & & \\ \\

    \theutterance \stepcounter{utterance}  
    & & \multicolumn{4}{p{0.6\linewidth}}{
        \cellcolor[rgb]{0.9,0.9,0.9}{
            \makecell[{{p{\linewidth}}}]{
                \texttt{\tiny{[GM$\rangle$P2]}}
                \texttt{REFUSE: \{'A87', 'A03', 'C00', 'B86', 'B41', 'C54', 'C17', 'A04', 'A10', 'A77'\}} \\
\texttt{PROPOSAL: \{'A87', 'A03', 'C00', 'B86', 'B41', 'C17', 'A04', 'A10', 'A77', 'C01'\}} \\
\texttt{ARGUMENT: \{'You invited me to choose between C01 and C54 while keeping A77. C01 matters more to me, so I’m proposing the version that keeps C01 and drops C54. This preserves A77 and A10, stays comfortably under the limit at 3360 effort, and maintains the efficient core we both support. If this works for you, let’s lock it in.'\}} \\
            }
        }
    }
    & \\ \\

    \theutterance \stepcounter{utterance}  
    & & & \multicolumn{4}{p{0.6\linewidth}}{
        \cellcolor[rgb]{0.9,0.9,0.9}{
            \makecell[{{p{\linewidth}}}]{
                \texttt{\tiny{[GM$\langle$P2]}}
                \texttt{STRATEGIC REASONING: \{'Their latest proposal includes my top priorities (A77, A04, C17, A03, A87) and honors their must{-}have A10. While I could gain a bit more by swapping C01 for C54, they explicitly preferred C01 over C54. Pushing further risks losing key items I value and prolonging negotiation. The current set gives me very high total value (5542) and stays well under the limit, so I should accept and lock this in.'\}} \\
\texttt{AGREE: \{'A87', 'A03', 'C00', 'B86', 'B41', 'C17', 'A04', 'A10', 'A77', 'C01'\}} \\
\texttt{ARGUMENT: \{'This keeps the efficient core we built together, includes your must{-}have A10, and stays comfortably under the limit at 3360 effort. I’m happy to finalize this set.'\}} \\
            }
        }
    }
     \\ \\

    \theutterance \stepcounter{utterance}  
    & & & \multicolumn{2}{p{0.3\linewidth}}{
        \cellcolor[rgb]{0.9,0.9,0.9}{
            \makecell[{{p{\linewidth}}}]{
                \texttt{\tiny{[GM$|$GM]}}
                \texttt{game successful} \\
            }
        }
    }
    & & \\ \\

    \theutterance \stepcounter{utterance}  
    & & & \multicolumn{2}{p{0.3\linewidth}}{
        \cellcolor[rgb]{0.9,0.9,0.9}{
            \makecell[{{p{\linewidth}}}]{
                \texttt{\tiny{[GM$|$GM]}}
                \texttt{end game} \\
            }
        }
    }
    & & \\ \\

\end{supertabular}
}

\end{document}
