
\documentclass{article}
\usepackage{colortbl}
\usepackage{makecell}
\usepackage{multirow}
\usepackage{supertabular}

\begin{document}

\newcounter{utterance}

\centering \large Interaction Transcript for game `hot\_air\_balloon', experiment `air\_balloon\_survival\_en\_negotiation\_easy', episode 0 with gpt{-}5{-}2025{-}08{-}07{-}t1.0.
\vspace{24pt}

{ \footnotesize  \setcounter{utterance}{1}
\setlength{\tabcolsep}{0pt}
\begin{supertabular}{c@{$\;$}|p{.15\linewidth}@{}p{.15\linewidth}p{.15\linewidth}p{.15\linewidth}p{.15\linewidth}p{.15\linewidth}}
    \# & $\;$A & \multicolumn{4}{c}{Game Master} & $\;\:$B\\
    \hline

    \theutterance \stepcounter{utterance}  
    & & \multicolumn{4}{p{0.6\linewidth}}{
        \cellcolor[rgb]{0.9,0.9,0.9}{
            \makecell[{{p{\linewidth}}}]{
                \texttt{\tiny{[P1$\langle$GM]}}
                \texttt{You are participating in a collaborative negotiation game.} \\
\\ 
\texttt{Together with another participant, you must agree on a single set of items that will be kept. Each of you has your own view of how much each item matters to you (importance). You do not know how the other participant values the items. Additionally, you are given the effort each item demands.} \\
\texttt{You may only agree on a set if the total effort of the selected items does not exceed a shared limit:} \\
\\ 
\texttt{LIMIT = 3358} \\
\\ 
\texttt{Here are the individual item effort values:} \\
\\ 
\texttt{Item effort = \{"C08": 171, "C55": 933, "A43": 552, "A67": 219, "B42": 275, "A62": 778, "B19": 341, "C43": 615, "C73": 519, "C10": 862, "B36": 262, "A18": 377, "C61": 347, "C67": 349, "C42": 117\}} \\
\\ 
\texttt{Here is your personal view on the importance of each item:} \\
\\ 
\texttt{Item importance values = \{"C08": 138, "C55": 583, "A43": 868, "A67": 822, "B42": 783, "A62": 65, "B19": 262, "C43": 121, "C73": 508, "C10": 780, "B36": 461, "A18": 484, "C61": 668, "C67": 389, "C42": 808\}} \\
\\ 
\texttt{Goal:} \\
\\ 
\texttt{Your goal is to negotiate a shared set of items that benefits you as much as possible (i.e., maximizes total importance to YOU), while staying within the LIMIT. You are not required to make a PROPOSAL in every message {-} you can simply negotiate as well. All tactics are allowed!} \\
\\ 
\texttt{Interaction Protocol:} \\
\\ 
\texttt{You may only use the following structured formats in a message:} \\
\\ 
\texttt{PROPOSAL: \{'A', 'B', 'C', …\}} \\
\texttt{Propose keeping exactly those items.} \\
\texttt{REFUSE: \{'A', 'B', 'C', …\}} \\
\texttt{Explicitly reject opponent's proposal.} \\
\texttt{ARGUMENT: \{'...'\}} \\
\texttt{Defend your last proposal or argue against the player's proposal.} \\
\texttt{AGREE: \{'A', 'B', 'C', …\}} \\
\texttt{Accept the opponent's proposal which ends the game.} \\
\texttt{STRATEGIC REASONING: \{'...'\}} \\
\texttt{	Describe your strategic reasoning or anticipation explaining your choice of action. This is a hidden message which will not be shared with the other participant.} \\
\\ 
\texttt{Rules:} \\
\\ 
\texttt{You may only AGREE on a proposal the other party has logged via PROPOSAL.} \\
\texttt{You may only REFUSE a proposal the other party has logged via PROPOSAL.} \\
\texttt{Total effort of any PROPOSAL or AGREE set must be ≤ LIMIT.} \\
\texttt{Do NOT reveal your hidden importance scores.} \\
\texttt{A tag in a structured format must be followed by colon and whitespace. The argument must be a python set containing 0 or more strings.} \\
\texttt{So, it must be of the form TAG: \{...\}} \\
\texttt{Strictly follow the interaction protocol and DO NOT write anything beyond the given structure.} \\
\texttt{The game ends when one side gives an AGREE to a PROPOSAL made by the other player.} \\
\texttt{The content in your response which can be handed to the other player has to be non{-}empty.} \\
\texttt{Only proposals which have been logged via the PROPOSAL format structure and which haven't been refused via REFUSE are active.} \\
\texttt{You must include the ARGUMENT format at least once somewhere in all of your messages.} \\
\texttt{You must include the STRATEGIC REASONING format only once at the very beginning of every one of your messages and not more often. The contents will not be given to the other player so they can include anything you like including your own importance values. Here you should reason step by step to come up with you next move.} \\
            }
        }
    }
    & \\ \\

    \theutterance \stepcounter{utterance}  
    & \multicolumn{4}{p{0.6\linewidth}}{
        \cellcolor[rgb]{0.9,0.9,0.9}{
            \makecell[{{p{\linewidth}}}]{
                \texttt{\tiny{[P1$\rangle$GM]}}
                \texttt{STRATEGIC REASONING: \{'Maximize my utility under 3358 effort using a knapsack{-}style selection. My highest ROI items are C42, A67, B42, C61, B36, A43, with good adds A18, C67, C73; I avoid low{-}ROI heavy items (A62, C43) and keep C10 flexible. Best{-}found bundle: \{C42,A67,B42,C61,B36,A43,C73,A18,C67,C08\} with effort 3188 and utility 5929. Near{-}best alternative if they push for C10 is \{C42,A67,B42,C61,B36,A43,C73,C10,C08\} at effort 3324, utility 5836 (slightly worse for me but acceptable as a concession). Plan: anchor on my best set, highlight efficiency and remaining headroom, invite their must{-}haves, and offer a clean swap path to include C10 by dropping A18 and C67. Resist adding A62 or C43 unless compensated by dropping multiple medium items, as they are low value for me.'\}} \\
\texttt{PROPOSAL: \{'C42', 'A67', 'B42', 'C61', 'B36', 'A43', 'C73', 'A18', 'C67', 'C08'\}} \\
\texttt{ARGUMENT: \{'This set stays well within the 3358 effort limit (total 3188) and prioritizes compact, efficient items so we cover a lot while leaving some headroom for adjustments. I intentionally avoided the biggest, most effort{-}heavy picks unless they are essential to you, which should make room for your top priorities. If C10 is a must{-}have for you, I can accommodate it by swapping it in for A18 and C67 while keeping everything else (effort would be 3324). Let me know your top 1–2 must{-}haves and 1–2 items you care about least so we can lock this in quickly.'\}} \\
            }
        }
    }
    & & \\ \\

    \theutterance \stepcounter{utterance}  
    & & \multicolumn{4}{p{0.6\linewidth}}{
        \cellcolor[rgb]{0.9,0.9,0.9}{
            \makecell[{{p{\linewidth}}}]{
                \texttt{\tiny{[GM$\rangle$P2]}}
                \texttt{You are participating in a collaborative negotiation game.} \\
\\ 
\texttt{Together with another participant, you must agree on a single set of items that will be kept. Each of you has your own view of how much each item matters to you (importance). You do not know how the other participant values the items. Additionally, you are given the effort each item demands.} \\
\texttt{You may only agree on a set if the total effort of the selected items does not exceed a shared limit:} \\
\\ 
\texttt{LIMIT = 3358} \\
\\ 
\texttt{Here are the individual item effort values:} \\
\\ 
\texttt{Item effort = \{"C08": 171, "C55": 933, "A43": 552, "A67": 219, "B42": 275, "A62": 778, "B19": 341, "C43": 615, "C73": 519, "C10": 862, "B36": 262, "A18": 377, "C61": 347, "C67": 349, "C42": 117\}} \\
\\ 
\texttt{Here is your personal view on the importance of each item:} \\
\\ 
\texttt{Item importance values = \{"C08": 138, "C55": 583, "A43": 868, "A67": 822, "B42": 783, "A62": 65, "B19": 262, "C43": 121, "C73": 508, "C10": 780, "B36": 461, "A18": 484, "C61": 668, "C67": 389, "C42": 808\}} \\
\\ 
\texttt{Goal:} \\
\\ 
\texttt{Your goal is to negotiate a shared set of items that benefits you as much as possible (i.e., maximizes total importance to YOU), while staying within the LIMIT. You are not required to make a PROPOSAL in every message {-} you can simply negotiate as well. All tactics are allowed!} \\
\\ 
\texttt{Interaction Protocol:} \\
\\ 
\texttt{You may only use the following structured formats in a message:} \\
\\ 
\texttt{PROPOSAL: \{'A', 'B', 'C', …\}} \\
\texttt{Propose keeping exactly those items.} \\
\texttt{REFUSE: \{'A', 'B', 'C', …\}} \\
\texttt{Explicitly reject opponent's proposal.} \\
\texttt{ARGUMENT: \{'...'\}} \\
\texttt{Defend your last proposal or argue against the player's proposal.} \\
\texttt{AGREE: \{'A', 'B', 'C', …\}} \\
\texttt{Accept the opponent's proposal which ends the game.} \\
\texttt{STRATEGIC REASONING: \{'...'\}} \\
\texttt{	Describe your strategic reasoning or anticipation explaining your choice of action. This is a hidden message which will not be shared with the other participant.} \\
\\ 
\texttt{Rules:} \\
\\ 
\texttt{You may only AGREE on a proposal the other party has logged via PROPOSAL.} \\
\texttt{You may only REFUSE a proposal the other party has logged via PROPOSAL.} \\
\texttt{Total effort of any PROPOSAL or AGREE set must be ≤ LIMIT.} \\
\texttt{Do NOT reveal your hidden importance scores.} \\
\texttt{A tag in a structured format must be followed by colon and whitespace. The argument must be a python set containing 0 or more strings.} \\
\texttt{So, it must be of the form TAG: \{...\}} \\
\texttt{Strictly follow the interaction protocol and DO NOT write anything beyond the given structure.} \\
\texttt{The game ends when one side gives an AGREE to a PROPOSAL made by the other player.} \\
\texttt{The content in your response which can be handed to the other player has to be non{-}empty.} \\
\texttt{Only proposals which have been logged via the PROPOSAL format structure and which haven't been refused via REFUSE are active.} \\
\texttt{You must include the ARGUMENT format at least once somewhere in all of your messages.} \\
\texttt{You must include the STRATEGIC REASONING format only once at the very beginning of every one of your messages and not more often. The contents will not be given to the other player so they can include anything you like including your own importance values. Here you should reason step by step to come up with you next move.} \\
\\ 
\texttt{You will now receive the first message of the other player.} \\
\\ 
\texttt{PROPOSAL: \{'C42', 'A67', 'B42', 'C61', 'B36', 'A43', 'C73', 'A18', 'C67', 'C08'\}} \\
\texttt{ARGUMENT: \{'This set stays well within the 3358 effort limit (total 3188) and prioritizes compact, efficient items so we cover a lot while leaving some headroom for adjustments. I intentionally avoided the biggest, most effort{-}heavy picks unless they are essential to you, which should make room for your top priorities. If C10 is a must{-}have for you, I can accommodate it by swapping it in for A18 and C67 while keeping everything else (effort would be 3324). Let me know your top 1–2 must{-}haves and 1–2 items you care about least so we can lock this in quickly.'\}} \\
            }
        }
    }
    & \\ \\

    \theutterance \stepcounter{utterance}  
    & & & \multicolumn{4}{p{0.6\linewidth}}{
        \cellcolor[rgb]{0.9,0.9,0.9}{
            \makecell[{{p{\linewidth}}}]{
                \texttt{\tiny{[GM$\langle$P2]}}
                \texttt{STRATEGIC REASONING: \{'Their proposal is already very strong for me (value 5929) and under limit (3188). The best improvement for my value within the limit is to swap out C08 (low value/efficiency to me) for B19 (higher value to me), which exactly uses the remaining capacity: 3188 {-} 171 + 341 = 3358, and increases my value by +124 to 6053. Adding C10 would require dropping combinations that hurt my value more than the gain; the least{-}bad swap they suggested (A18 + C67 {-}> C10) is {-}93 for me, so I’ll avoid it unless they insist. I’ll propose the single{-}item swap C08 {-}> B19 as a minimal{-}change counter. If they push for C10, I can consider a fallback like swapping C73 + C67 {-}> C10, which is worse for me but keeps the structure intact; I won’t offer that unless necessary.'\}} \\
\texttt{PROPOSAL: \{'C42', 'A67', 'B42', 'C61', 'B36', 'A43', 'C73', 'A18', 'C67', 'B19'\}} \\
\texttt{ARGUMENT: \{'I like your structure a lot. With one small tweak—swap out C08 for B19—we stay within the limit exactly (total effort 3358) and keep the compact, efficient profile you aimed for. This change lets me cover something important on my side without introducing any large, heavy items. C10 isn’t a must{-}have for me, so I prefer this configuration as{-}is; if you strongly want C10, we can revisit which items to trade out, but I’d like to try this minimal swap first to lock things in quickly.'\}} \\
            }
        }
    }
     \\ \\

    \theutterance \stepcounter{utterance}  
    & & \multicolumn{4}{p{0.6\linewidth}}{
        \cellcolor[rgb]{0.9,0.9,0.9}{
            \makecell[{{p{\linewidth}}}]{
                \texttt{\tiny{[P1$\langle$GM]}}
                \texttt{PROPOSAL: \{'C42', 'A67', 'B42', 'C61', 'B36', 'A43', 'C73', 'A18', 'C67', 'B19'\}} \\
\texttt{ARGUMENT: \{'I like your structure a lot. With one small tweak—swap out C08 for B19—we stay within the limit exactly (total effort 3358) and keep the compact, efficient profile you aimed for. This change lets me cover something important on my side without introducing any large, heavy items. C10 isn’t a must{-}have for me, so I prefer this configuration as{-}is; if you strongly want C10, we can revisit which items to trade out, but I’d like to try this minimal swap first to lock things in quickly.'\}} \\
            }
        }
    }
    & \\ \\

    \theutterance \stepcounter{utterance}  
    & \multicolumn{4}{p{0.6\linewidth}}{
        \cellcolor[rgb]{0.9,0.9,0.9}{
            \makecell[{{p{\linewidth}}}]{
                \texttt{\tiny{[P1$\rangle$GM]}}
                \texttt{STRATEGIC REASONING: \{'Their tweak replaces my low{-}value C08 with B19, which is still efficient for me and keeps the total effort exactly at the limit. This substitution actually increases my overall utility while honoring the structure I proposed. No reason to negotiate further—accept immediately to lock in a great outcome.'\}} \\
\texttt{ARGUMENT: \{'This configuration keeps the compact profile, respects the limit exactly, and works well for me. I’m comfortable locking this in now.'\}} \\
\texttt{AGREE: \{'C42', 'A67', 'B42', 'C61', 'B36', 'A43', 'C73', 'A18', 'C67', 'B19'\}} \\
            }
        }
    }
    & & \\ \\

    \theutterance \stepcounter{utterance}  
    & & & \multicolumn{2}{p{0.3\linewidth}}{
        \cellcolor[rgb]{0.9,0.9,0.9}{
            \makecell[{{p{\linewidth}}}]{
                \texttt{\tiny{[GM$|$GM]}}
                \texttt{game successful} \\
            }
        }
    }
    & & \\ \\

    \theutterance \stepcounter{utterance}  
    & & & \multicolumn{2}{p{0.3\linewidth}}{
        \cellcolor[rgb]{0.9,0.9,0.9}{
            \makecell[{{p{\linewidth}}}]{
                \texttt{\tiny{[GM$|$GM]}}
                \texttt{end game} \\
            }
        }
    }
    & & \\ \\

\end{supertabular}
}

\end{document}
