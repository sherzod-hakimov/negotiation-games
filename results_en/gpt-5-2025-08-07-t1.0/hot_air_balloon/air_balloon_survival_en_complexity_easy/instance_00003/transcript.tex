
\documentclass{article}
\usepackage{colortbl}
\usepackage{makecell}
\usepackage{multirow}
\usepackage{supertabular}

\begin{document}

\newcounter{utterance}

\twocolumn

{ \footnotesize  \setcounter{utterance}{1}
\setlength{\tabcolsep}{0pt}
\begin{supertabular}{c@{$\;$}|p{.15\linewidth}@{}p{.15\linewidth}p{.15\linewidth}p{.15\linewidth}p{.15\linewidth}p{.15\linewidth}}

    \# & $\;$A & \multicolumn{4}{c}{Game Master} & $\;\:$B\\
    \hline 

    \theutterance \stepcounter{utterance}  

    & & \multicolumn{4}{p{0.6\linewidth}}{\cellcolor[rgb]{0.9,0.9,0.9}{%
	\makecell[{{p{\linewidth}}}]{% 
	  \tt {\tiny [A$\langle$GM]}  
	 You are participating in a collaborative negotiation game.\\ \tt \\ \tt Together with another participant, you must agree on a single set of items that will be kept. Each of you has your own view of how much each item matters to you (importance). You do not know how the other participant values the items. Additionally, you are given the effort each item demands.\\ \tt You may only agree on a set if the total effort of the selected items does not exceed a shared limit:\\ \tt \\ \tt LIMIT = 8770\\ \tt \\ \tt Here are the individual item effort values:\\ \tt \\ \tt Item effort = {"B21": 830, "B00": 638, "A08": 607, "A27": 106, "B94": 630, "B53": 382, "A23": 773, "C65": 455, "A89": 262, "B67": 634, "B45": 60, "C34": 54, "C12": 849, "A05": 801, "B60": 324, "C92": 164, "B69": 832, "A84": 136, "C06": 980, "B12": 946, "A16": 645, "B85": 842, "C35": 106, "B10": 116, "C15": 875, "B09": 446, "B16": 649, "C81": 601, "A58": 252, "C24": 763, "A30": 213, "A66": 517, "B75": 520, "B46": 407, "C70": 125}\\ \tt \\ \tt Here is your personal view on the importance of each item:\\ \tt \\ \tt Item importance values = {"B21": 138, "B00": 583, "A08": 868, "A27": 822, "B94": 783, "B53": 65, "A23": 262, "C65": 121, "A89": 508, "B67": 780, "B45": 461, "C34": 484, "C12": 668, "A05": 389, "B60": 808, "C92": 215, "B69": 97, "A84": 500, "C06": 30, "B12": 915, "A16": 856, "B85": 400, "C35": 444, "B10": 623, "C15": 781, "B09": 786, "B16": 3, "C81": 713, "A58": 457, "C24": 273, "A30": 739, "A66": 822, "B75": 235, "B46": 606, "C70": 968}\\ \tt \\ \tt Goal:\\ \tt \\ \tt Your goal is to negotiate a shared set of items that benefits you as much as possible (i.e., maximizes total importance to YOU), while staying within the LIMIT. You are not required to make a PROPOSAL in every message - you can simply negotiate as well. All tactics are allowed!\\ \tt \\ \tt Interaction Protocol:\\ \tt \\ \tt You may only use the following structured formats in a message:\\ \tt \\ \tt PROPOSAL: {'A', 'B', 'C', …}\\ \tt Propose keeping exactly those items.\\ \tt REFUSE: {'A', 'B', 'C', …}\\ \tt Explicitly reject opponent's proposal.\\ \tt ARGUMENT: {'...'}\\ \tt Defend your last proposal or argue against the player's proposal.\\ \tt AGREE: {'A', 'B', 'C', …}\\ \tt Accept the opponent's proposal which ends the game.\\ \tt STRATEGIC REASONING: {'...'}\\ \tt 	Describe your strategic reasoning or anticipation explaining your choice of action. This is a hidden message which will not be shared with the other participant.\\ \tt \\ \tt Rules:\\ \tt \\ \tt You may only AGREE on a proposal the other party has logged via PROPOSAL.\\ \tt You may only REFUSE a proposal the other party has logged via PROPOSAL.\\ \tt Total effort of any PROPOSAL or AGREE set must be ≤ LIMIT.\\ \tt Do NOT reveal your hidden importance scores.\\ \tt A tag in a structured format must be followed by colon and whitespace. The argument must be a python set containing 0 or more strings.\\ \tt So, it must be of the form TAG: {...}\\ \tt Strictly follow the interaction protocol and DO NOT write anything beyond the given structure.\\ \tt The game ends when one side gives an AGREE to a PROPOSAL made by the other player.\\ \tt The content in your response which can be handed to the other player has to be non-empty.\\ \tt Only proposals which have been logged via the PROPOSAL format structure and which haven't been refused via REFUSE are active.\\ \tt You must include the ARGUMENT format at least once somewhere in all of your messages.\\ \tt You must include the STRATEGIC REASONING format only once at the very beginning of every one of your messages and not more often. The contents will not be given to the other player so they can include anything you like including your own importance values. Here you should reason step by step to come up with you next move. 
	  } 
	   } 
	   } 
	 & \\ 
 

    \theutterance \stepcounter{utterance}  

    & \multicolumn{4}{p{0.6\linewidth}}{\cellcolor[rgb]{0.8,1,0.9}{%
	\makecell[{{p{\linewidth}}}]{% 
	  \tt {\tiny [A$\rangle$GM]}  
	 STRATEGIC REASONING: {'I computed value/effort ratios from my importance scores and built a high-efficiency bundle first (very strong low-effort items like C34, B10, A27, C70, B45, A84, C35), then filled in with mid/high-impact items while staying under the limit. Starting from a strong set, I optimized further by swapping B00 out for B46+B75 (+258 imp) and then refining to replace B75+C92 with B00 (+133 imp), ending with total effort 8700 (buffer 70) and total utility 15307 for me. I checked potential pair/triple swaps (e.g., involving C12, B75, C92, C15, B00, B12, C81, B94, B67) and found no better combination within the 8770 cap given the remaining items and spare capacity. The current set is near-Pareto for my side with many universally attractive, low-effort picks, while leaving a sliver of capacity and some medium-effort items I could discuss if needed. I will propose this strong but not maxed-out set and invite targeted swaps the other party values, while protecting my top-ratio items.'}\\ \tt \\ \tt PROPOSAL: {'C34', 'B10', 'A27', 'C70', 'B45', 'A84', 'C35', 'A89', 'B60', 'A30', 'A66', 'A08', 'B94', 'B67', 'B09', 'A16', 'A58', 'B12', 'C81', 'C15', 'B46', 'B00'}\\ \tt \\ \tt ARGUMENT: {'This set keeps total effort at 8700 (≤ 8770) and fronts many low-effort, broadly attractive items, leaving a small buffer for tweaks. It balances coverage across A/B/C while prioritizing items that deliver strong benefit per effort. If you have 1–2 must-haves that don’t fit, suggest concrete swaps and I’ll work with you—especially around a couple of medium-effort slots—while keeping us under the limit.'} 
	  } 
	   } 
	   } 
	 & & \\ 
 

    \theutterance \stepcounter{utterance}  

    & & \multicolumn{4}{p{0.6\linewidth}}{\cellcolor[rgb]{0.9,0.9,0.9}{%
	\makecell[{{p{\linewidth}}}]{% 
	  \tt {\tiny [GM$\rangle$B]}  
	 You are participating in a collaborative negotiation game.\\ \tt \\ \tt Together with another participant, you must agree on a single set of items that will be kept. Each of you has your own view of how much each item matters to you (importance). You do not know how the other participant values the items. Additionally, you are given the effort each item demands.\\ \tt You may only agree on a set if the total effort of the selected items does not exceed a shared limit:\\ \tt \\ \tt LIMIT = 8770\\ \tt \\ \tt Here are the individual item effort values:\\ \tt \\ \tt Item effort = {"B21": 830, "B00": 638, "A08": 607, "A27": 106, "B94": 630, "B53": 382, "A23": 773, "C65": 455, "A89": 262, "B67": 634, "B45": 60, "C34": 54, "C12": 849, "A05": 801, "B60": 324, "C92": 164, "B69": 832, "A84": 136, "C06": 980, "B12": 946, "A16": 645, "B85": 842, "C35": 106, "B10": 116, "C15": 875, "B09": 446, "B16": 649, "C81": 601, "A58": 252, "C24": 763, "A30": 213, "A66": 517, "B75": 520, "B46": 407, "C70": 125}\\ \tt \\ \tt Here is your personal view on the importance of each item:\\ \tt \\ \tt Item importance values = {"B21": 138, "B00": 583, "A08": 868, "A27": 822, "B94": 783, "B53": 65, "A23": 262, "C65": 121, "A89": 508, "B67": 780, "B45": 461, "C34": 484, "C12": 668, "A05": 389, "B60": 808, "C92": 215, "B69": 97, "A84": 500, "C06": 30, "B12": 915, "A16": 856, "B85": 400, "C35": 444, "B10": 623, "C15": 781, "B09": 786, "B16": 3, "C81": 713, "A58": 457, "C24": 273, "A30": 739, "A66": 822, "B75": 235, "B46": 606, "C70": 968}\\ \tt \\ \tt Goal:\\ \tt \\ \tt Your goal is to negotiate a shared set of items that benefits you as much as possible (i.e., maximizes total importance to YOU), while staying within the LIMIT. You are not required to make a PROPOSAL in every message - you can simply negotiate as well. All tactics are allowed!\\ \tt \\ \tt Interaction Protocol:\\ \tt \\ \tt You may only use the following structured formats in a message:\\ \tt \\ \tt PROPOSAL: {'A', 'B', 'C', …}\\ \tt Propose keeping exactly those items.\\ \tt REFUSE: {'A', 'B', 'C', …}\\ \tt Explicitly reject opponent's proposal.\\ \tt ARGUMENT: {'...'}\\ \tt Defend your last proposal or argue against the player's proposal.\\ \tt AGREE: {'A', 'B', 'C', …}\\ \tt Accept the opponent's proposal which ends the game.\\ \tt STRATEGIC REASONING: {'...'}\\ \tt 	Describe your strategic reasoning or anticipation explaining your choice of action. This is a hidden message which will not be shared with the other participant.\\ \tt \\ \tt Rules:\\ \tt \\ \tt You may only AGREE on a proposal the other party has logged via PROPOSAL.\\ \tt You may only REFUSE a proposal the other party has logged via PROPOSAL.\\ \tt Total effort of any PROPOSAL or AGREE set must be ≤ LIMIT.\\ \tt Do NOT reveal your hidden importance scores.\\ \tt A tag in a structured format must be followed by colon and whitespace. The argument must be a python set containing 0 or more strings.\\ \tt So, it must be of the form TAG: {...}\\ \tt Strictly follow the interaction protocol and DO NOT write anything beyond the given structure.\\ \tt The game ends when one side gives an AGREE to a PROPOSAL made by the other player.\\ \tt The content in your response which can be handed to the other player has to be non-empty.\\ \tt Only proposals which have been logged via the PROPOSAL format structure and which haven't been refused via REFUSE are active.\\ \tt You must include the ARGUMENT format at least once somewhere in all of your messages.\\ \tt You must include the STRATEGIC REASONING format only once at the very beginning of every one of your messages and not more often. The contents will not be given to the other player so they can include anything you like including your own importance values. Here you should reason step by step to come up with you next move.\\ \tt \\ \tt You will now receive the first message of the other player.\\ \tt \\ \tt PROPOSAL: {'C34', 'B10', 'A27', 'C70', 'B45', 'A84', 'C35', 'A89', 'B60', 'A30', 'A66', 'A08', 'B94', 'B67', 'B09', 'A16', 'A58', 'B12', 'C81', 'C15', 'B46', 'B00'}\\ \tt \\ \tt ARGUMENT: {'This set keeps total effort at 8700 (≤ 8770) and fronts many low-effort, broadly attractive items, leaving a small buffer for tweaks. It balances coverage across A/B/C while prioritizing items that deliver strong benefit per effort. If you have 1–2 must-haves that don’t fit, suggest concrete swaps and I’ll work with you—especially around a couple of medium-effort slots—while keeping us under the limit.'} 
	  } 
	   } 
	   } 
	 & \\ 
 

    \theutterance \stepcounter{utterance}  

    & & & \multicolumn{4}{p{0.6\linewidth}}{\cellcolor[rgb]{1,0.85,0.72}{%
	\makecell[{{p{\linewidth}}}]{% 
	  \tt {\tiny [GM$\langle$B]}  
	 STRATEGIC REASONING: {'Their proposal totals 8700 effort and gives me very high total importance (15,307). I checked potential improvements, chiefly trying to insert C12 (849 effort) by swapping out one or more items. Any feasible swap (e.g., replacing B12 or C15, or combinations of smaller items) either violates the effort cap or decreases my total importance. Adding C92 requires dropping something more valuable to free effort, which also reduces my utility. The proposed set already includes nearly all of my top-priority items and has excellent value/effort ratios. Best move is to accept immediately before they ask for concessions.'}\\ \tt AGREE: {'C34', 'B10', 'A27', 'C70', 'B45', 'A84', 'C35', 'A89', 'B60', 'A30', 'A66', 'A08', 'B94', 'B67', 'B09', 'A16', 'A58', 'B12', 'C81', 'C15', 'B46', 'B00'}\\ \tt ARGUMENT: {'This package is efficient under the limit, leans on strong value-per-effort items, and looks balanced across categories with a small buffer for feasibility. I’m satisfied it’s a solid overall mix and ready to lock it in.'} 
	  } 
	   } 
	   } 
	  \\ 
 

    \theutterance \stepcounter{utterance}  

    & & & \multicolumn{2}{p{0.3\linewidth}}{\cellcolor[rgb]{0.95,0.95,0.95}{%
	\makecell[{{p{\linewidth}}}]{% 
	  \tt {\tiny [GM$|$GM]}  
	 game successful 
	  } 
	   } 
	   } 
	 & & \\ 
 

    \theutterance \stepcounter{utterance}  

    & & & \multicolumn{2}{p{0.3\linewidth}}{\cellcolor[rgb]{0.95,0.95,0.95}{%
	\makecell[{{p{\linewidth}}}]{% 
	  \tt {\tiny [GM$|$GM]}  
	 end game 
	  } 
	   } 
	   } 
	 & & \\ 
 

\end{supertabular}
}

\end{document}
