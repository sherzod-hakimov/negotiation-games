
\documentclass{article}
\usepackage{colortbl}
\usepackage{makecell}
\usepackage{multirow}
\usepackage{supertabular}

\begin{document}

\newcounter{utterance}

\twocolumn

{ \footnotesize  \setcounter{utterance}{1}
\setlength{\tabcolsep}{0pt}
\begin{supertabular}{c@{$\;$}|p{.15\linewidth}@{}p{.15\linewidth}p{.15\linewidth}p{.15\linewidth}p{.15\linewidth}p{.15\linewidth}}

    \# & $\;$A & \multicolumn{4}{c}{Game Master} & $\;\:$B\\
    \hline 

    \theutterance \stepcounter{utterance}  

    & & \multicolumn{4}{p{0.6\linewidth}}{\cellcolor[rgb]{0.9,0.9,0.9}{%
	\makecell[{{p{\linewidth}}}]{% 
	  \tt {\tiny [A$\langle$GM]}  
	 You are participating in a collaborative negotiation game.\\ \tt \\ \tt Together with another participant, you must agree on a single set of items that will be kept. Each of you has your own view of how much each item matters to you (importance). You do not know how the other participant values the items. Additionally, you are given the effort each item demands.\\ \tt You may only agree on a set if the total effort of the selected items does not exceed a shared limit:\\ \tt \\ \tt LIMIT = 7868\\ \tt \\ \tt Here are the individual item effort values:\\ \tt \\ \tt Item effort = {"B21": 552, "B00": 560, "A08": 76, "A27": 950, "B94": 248, "B53": 871, "A23": 391, "C65": 143, "A89": 293, "B67": 208, "B45": 678, "C34": 737, "C12": 407, "A05": 366, "B60": 768, "C92": 859, "B69": 183, "A84": 231, "C06": 305, "B12": 727, "A16": 148, "B85": 357, "C35": 504, "B10": 549, "C15": 299, "B09": 91, "B16": 527, "C81": 848, "A58": 307, "C24": 214, "A30": 723, "A66": 475, "B75": 23, "B46": 298, "C70": 820}\\ \tt \\ \tt Here is your personal view on the importance of each item:\\ \tt \\ \tt Item importance values = {"B21": 138, "B00": 583, "A08": 868, "A27": 822, "B94": 783, "B53": 65, "A23": 262, "C65": 121, "A89": 508, "B67": 780, "B45": 461, "C34": 484, "C12": 668, "A05": 389, "B60": 808, "C92": 215, "B69": 97, "A84": 500, "C06": 30, "B12": 915, "A16": 856, "B85": 400, "C35": 444, "B10": 623, "C15": 781, "B09": 786, "B16": 3, "C81": 713, "A58": 457, "C24": 273, "A30": 739, "A66": 822, "B75": 235, "B46": 606, "C70": 105}\\ \tt \\ \tt Goal:\\ \tt \\ \tt Your goal is to negotiate a shared set of items that benefits you as much as possible (i.e., maximizes total importance to YOU), while staying within the LIMIT. You are not required to make a PROPOSAL in every message - you can simply negotiate as well. All tactics are allowed!\\ \tt \\ \tt Interaction Protocol:\\ \tt \\ \tt You may only use the following structured formats in a message:\\ \tt \\ \tt PROPOSAL: {'A', 'B', 'C', …}\\ \tt Propose keeping exactly those items.\\ \tt REFUSE: {'A', 'B', 'C', …}\\ \tt Explicitly reject opponent's proposal.\\ \tt ARGUMENT: {'...'}\\ \tt Defend your last proposal or argue against the player's proposal.\\ \tt AGREE: {'A', 'B', 'C', …}\\ \tt Accept the opponent's proposal which ends the game.\\ \tt STRATEGIC REASONING: {'...'}\\ \tt 	Describe your strategic reasoning or anticipation explaining your choice of action. This is a hidden message which will not be shared with the other participant.\\ \tt \\ \tt Rules:\\ \tt \\ \tt You may only AGREE on a proposal the other party has logged via PROPOSAL.\\ \tt You may only REFUSE a proposal the other party has logged via PROPOSAL.\\ \tt Total effort of any PROPOSAL or AGREE set must be ≤ LIMIT.\\ \tt Do NOT reveal your hidden importance scores.\\ \tt A tag in a structured format must be followed by colon and whitespace. The argument must be a python set containing 0 or more strings.\\ \tt So, it must be of the form TAG: {...}\\ \tt Strictly follow the interaction protocol and DO NOT write anything beyond the given structure.\\ \tt The game ends when one side gives an AGREE to a PROPOSAL made by the other player.\\ \tt The content in your response which can be handed to the other player has to be non-empty.\\ \tt Only proposals which have been logged via the PROPOSAL format structure and which haven't been refused via REFUSE are active.\\ \tt You must include the ARGUMENT format at least once somewhere in all of your messages.\\ \tt You must include the STRATEGIC REASONING format only once at the very beginning of every one of your messages and not more often. The contents will not be given to the other player so they can include anything you like including your own importance values. Here you should reason step by step to come up with you next move. 
	  } 
	   } 
	   } 
	 & \\ 
 

    \theutterance \stepcounter{utterance}  

    & \multicolumn{4}{p{0.6\linewidth}}{\cellcolor[rgb]{0.8,1,0.9}{%
	\makecell[{{p{\linewidth}}}]{% 
	  \tt {\tiny [A$\rangle$GM]}  
	 STRATEGIC REASONING: {'I computed a near-optimal bundle for my utility under the effort limit. Starting from a high value/effort core (A08, B09, A16, B67, B94, C15, etc.), I noticed two heavier, lower-ratio items (A27, C81) in the initial greedy fill. Replacing them with a combo of medium-effort, high-yield items (B00, B85, A05, C35) increased my total value while keeping total effort well under 7868. The proposed set uses 7801 effort, leaving a small buffer. If the other side pushes for changes, my first preference for flexibility would be to consider swaps around C65 or A58 (as effort levers) provided I don’t give up too much value; I’ll avoid conceding high-yield items like A08, B09, A16, B67, B94, C15, A66, B12. I’ll present the strong set and invite targeted swaps if they have must-haves, without revealing my importance scores.'}\\ \tt PROPOSAL: {'A08', 'B75', 'B09', 'A16', 'B67', 'B94', 'C15', 'A84', 'B46', 'A66', 'A89', 'C12', 'B12', 'A30', 'B60', 'B10', 'A58', 'C65', 'B00', 'B85', 'A05', 'C35'}\\ \tt ARGUMENT: {'This set stays within the LIMIT (total effort = 7801 ≤ 7868) and provides a balanced mix across A/B/C while prioritizing items that give strong value for their effort. It avoids very high-effort, low-return items, which helps keep us under budget with a small buffer. If you have a must-have that we missed, propose a targeted swap under the limit—e.g., we can consider replacing one or two mid-effort items to accommodate your priority, as long as the total stays within 7868. Let me know which specific items matter most to you and I can suggest concrete swaps that fit.'} 
	  } 
	   } 
	   } 
	 & & \\ 
 

    \theutterance \stepcounter{utterance}  

    & & \multicolumn{4}{p{0.6\linewidth}}{\cellcolor[rgb]{0.9,0.9,0.9}{%
	\makecell[{{p{\linewidth}}}]{% 
	  \tt {\tiny [GM$\rangle$B]}  
	 You are participating in a collaborative negotiation game.\\ \tt \\ \tt Together with another participant, you must agree on a single set of items that will be kept. Each of you has your own view of how much each item matters to you (importance). You do not know how the other participant values the items. Additionally, you are given the effort each item demands.\\ \tt You may only agree on a set if the total effort of the selected items does not exceed a shared limit:\\ \tt \\ \tt LIMIT = 7868\\ \tt \\ \tt Here are the individual item effort values:\\ \tt \\ \tt Item effort = {"B21": 552, "B00": 560, "A08": 76, "A27": 950, "B94": 248, "B53": 871, "A23": 391, "C65": 143, "A89": 293, "B67": 208, "B45": 678, "C34": 737, "C12": 407, "A05": 366, "B60": 768, "C92": 859, "B69": 183, "A84": 231, "C06": 305, "B12": 727, "A16": 148, "B85": 357, "C35": 504, "B10": 549, "C15": 299, "B09": 91, "B16": 527, "C81": 848, "A58": 307, "C24": 214, "A30": 723, "A66": 475, "B75": 23, "B46": 298, "C70": 820}\\ \tt \\ \tt Here is your personal view on the importance of each item:\\ \tt \\ \tt Item importance values = {"B21": 138, "B00": 583, "A08": 868, "A27": 822, "B94": 783, "B53": 65, "A23": 262, "C65": 121, "A89": 508, "B67": 780, "B45": 461, "C34": 484, "C12": 668, "A05": 389, "B60": 808, "C92": 215, "B69": 97, "A84": 500, "C06": 30, "B12": 915, "A16": 856, "B85": 400, "C35": 444, "B10": 623, "C15": 781, "B09": 786, "B16": 3, "C81": 713, "A58": 457, "C24": 273, "A30": 739, "A66": 822, "B75": 235, "B46": 606, "C70": 105}\\ \tt \\ \tt Goal:\\ \tt \\ \tt Your goal is to negotiate a shared set of items that benefits you as much as possible (i.e., maximizes total importance to YOU), while staying within the LIMIT. You are not required to make a PROPOSAL in every message - you can simply negotiate as well. All tactics are allowed!\\ \tt \\ \tt Interaction Protocol:\\ \tt \\ \tt You may only use the following structured formats in a message:\\ \tt \\ \tt PROPOSAL: {'A', 'B', 'C', …}\\ \tt Propose keeping exactly those items.\\ \tt REFUSE: {'A', 'B', 'C', …}\\ \tt Explicitly reject opponent's proposal.\\ \tt ARGUMENT: {'...'}\\ \tt Defend your last proposal or argue against the player's proposal.\\ \tt AGREE: {'A', 'B', 'C', …}\\ \tt Accept the opponent's proposal which ends the game.\\ \tt STRATEGIC REASONING: {'...'}\\ \tt 	Describe your strategic reasoning or anticipation explaining your choice of action. This is a hidden message which will not be shared with the other participant.\\ \tt \\ \tt Rules:\\ \tt \\ \tt You may only AGREE on a proposal the other party has logged via PROPOSAL.\\ \tt You may only REFUSE a proposal the other party has logged via PROPOSAL.\\ \tt Total effort of any PROPOSAL or AGREE set must be ≤ LIMIT.\\ \tt Do NOT reveal your hidden importance scores.\\ \tt A tag in a structured format must be followed by colon and whitespace. The argument must be a python set containing 0 or more strings.\\ \tt So, it must be of the form TAG: {...}\\ \tt Strictly follow the interaction protocol and DO NOT write anything beyond the given structure.\\ \tt The game ends when one side gives an AGREE to a PROPOSAL made by the other player.\\ \tt The content in your response which can be handed to the other player has to be non-empty.\\ \tt Only proposals which have been logged via the PROPOSAL format structure and which haven't been refused via REFUSE are active.\\ \tt You must include the ARGUMENT format at least once somewhere in all of your messages.\\ \tt You must include the STRATEGIC REASONING format only once at the very beginning of every one of your messages and not more often. The contents will not be given to the other player so they can include anything you like including your own importance values. Here you should reason step by step to come up with you next move.\\ \tt \\ \tt You will now receive the first message of the other player.\\ \tt \\ \tt PROPOSAL: {'A08', 'B75', 'B09', 'A16', 'B67', 'B94', 'C15', 'A84', 'B46', 'A66', 'A89', 'C12', 'B12', 'A30', 'B60', 'B10', 'A58', 'C65', 'B00', 'B85', 'A05', 'C35'}\\ \tt ARGUMENT: {'This set stays within the LIMIT (total effort = 7801 ≤ 7868) and provides a balanced mix across A/B/C while prioritizing items that give strong value for their effort. It avoids very high-effort, low-return items, which helps keep us under budget with a small buffer. If you have a must-have that we missed, propose a targeted swap under the limit—e.g., we can consider replacing one or two mid-effort items to accommodate your priority, as long as the total stays within 7868. Let me know which specific items matter most to you and I can suggest concrete swaps that fit.'} 
	  } 
	   } 
	   } 
	 & \\ 
 

    \theutterance \stepcounter{utterance}  

    & & & \multicolumn{4}{p{0.6\linewidth}}{\cellcolor[rgb]{1,0.85,0.72}{%
	\makecell[{{p{\linewidth}}}]{% 
	  \tt {\tiny [GM$\langle$B]}  
	 STRATEGIC REASONING: {'Your proposed set already captures nearly all of my highest-value, low-effort items and uses the budget efficiently with only 67 effort slack. I evaluated potential swaps (e.g., replacing lower-density items like C35 or C65 with options not included such as C24, A23, B45, C34, C81, A27) and each feasible combination either exceeds the limit or reduces my total value. There are no remaining items with effort ≤67 to add, and replacing any included item to fit heavier high-value items results in a net loss for me. Therefore, accepting your proposal maximizes my outcome under the constraint.'}\\ \tt ARGUMENT: {'This selection fits under the LIMIT with minimal slack and includes a strong set of priorities for me. I don’t see any swaps that both respect the limit and improve the overall mix, so I’m ready to finalize.'}\\ \tt AGREE: {'A08', 'B75', 'B09', 'A16', 'B67', 'B94', 'C15', 'A84', 'B46', 'A66', 'A89', 'C12', 'B12', 'A30', 'B60', 'B10', 'A58', 'C65', 'B00', 'B85', 'A05', 'C35'} 
	  } 
	   } 
	   } 
	  \\ 
 

    \theutterance \stepcounter{utterance}  

    & & & \multicolumn{2}{p{0.3\linewidth}}{\cellcolor[rgb]{0.95,0.95,0.95}{%
	\makecell[{{p{\linewidth}}}]{% 
	  \tt {\tiny [GM$|$GM]}  
	 game successful 
	  } 
	   } 
	   } 
	 & & \\ 
 

    \theutterance \stepcounter{utterance}  

    & & & \multicolumn{2}{p{0.3\linewidth}}{\cellcolor[rgb]{0.95,0.95,0.95}{%
	\makecell[{{p{\linewidth}}}]{% 
	  \tt {\tiny [GM$|$GM]}  
	 end game 
	  } 
	   } 
	   } 
	 & & \\ 
 

\end{supertabular}
}

\end{document}
