
\documentclass{article}
\usepackage{colortbl}
\usepackage{makecell}
\usepackage{multirow}
\usepackage{supertabular}

\begin{document}

\newcounter{utterance}

\twocolumn

{ \footnotesize  \setcounter{utterance}{1}
\setlength{\tabcolsep}{0pt}
\begin{supertabular}{c@{$\;$}|p{.15\linewidth}@{}p{.15\linewidth}p{.15\linewidth}p{.15\linewidth}p{.15\linewidth}p{.15\linewidth}}

    \# & $\;$A & \multicolumn{4}{c}{Game Master} & $\;\:$B\\
    \hline 

    \theutterance \stepcounter{utterance}  

    & & \multicolumn{4}{p{0.6\linewidth}}{\cellcolor[rgb]{0.9,0.9,0.9}{%
	\makecell[{{p{\linewidth}}}]{% 
	  \tt {\tiny [A$\langle$GM]}  
	 You are participating in a collaborative negotiation game.\\ \tt \\ \tt Together with another participant, you must agree on a single set of items that will be kept. Each of you has your own view of how much each item matters to you (importance). You do not know how the other participant values the items. Additionally, you are given the effort each item demands.\\ \tt You may only agree on a set if the total effort of the selected items does not exceed a shared limit:\\ \tt \\ \tt LIMIT = 7605\\ \tt \\ \tt Here are the individual item effort values:\\ \tt \\ \tt Item effort = {"B21": 369, "B00": 951, "A08": 156, "A27": 266, "B94": 581, "B53": 287, "A23": 866, "C65": 180, "A89": 800, "B67": 739, "B45": 637, "C34": 86, "C12": 749, "A05": 370, "B60": 345, "C92": 951, "B69": 147, "A84": 265, "C06": 262, "B12": 259, "A16": 358, "B85": 394, "C35": 286, "B10": 580, "C15": 479, "B09": 14, "B16": 153, "C81": 977, "A58": 134, "C24": 982, "A30": 259, "A66": 232, "B75": 202, "B46": 73, "C70": 821}\\ \tt \\ \tt Here is your personal view on the importance of each item:\\ \tt \\ \tt Item importance values = {"B21": 138, "B00": 583, "A08": 868, "A27": 822, "B94": 783, "B53": 65, "A23": 262, "C65": 121, "A89": 508, "B67": 780, "B45": 461, "C34": 484, "C12": 668, "A05": 389, "B60": 808, "C92": 215, "B69": 97, "A84": 500, "C06": 30, "B12": 915, "A16": 856, "B85": 400, "C35": 444, "B10": 623, "C15": 781, "B09": 786, "B16": 3, "C81": 713, "A58": 457, "C24": 273, "A30": 739, "A66": 822, "B75": 235, "B46": 606, "C70": 968}\\ \tt \\ \tt Goal:\\ \tt \\ \tt Your goal is to negotiate a shared set of items that benefits you as much as possible (i.e., maximizes total importance to YOU), while staying within the LIMIT. You are not required to make a PROPOSAL in every message - you can simply negotiate as well. All tactics are allowed!\\ \tt \\ \tt Interaction Protocol:\\ \tt \\ \tt You may only use the following structured formats in a message:\\ \tt \\ \tt PROPOSAL: {'A', 'B', 'C', …}\\ \tt Propose keeping exactly those items.\\ \tt REFUSE: {'A', 'B', 'C', …}\\ \tt Explicitly reject opponent's proposal.\\ \tt ARGUMENT: {'...'}\\ \tt Defend your last proposal or argue against the player's proposal.\\ \tt AGREE: {'A', 'B', 'C', …}\\ \tt Accept the opponent's proposal which ends the game.\\ \tt STRATEGIC REASONING: {'...'}\\ \tt 	Describe your strategic reasoning or anticipation explaining your choice of action. This is a hidden message which will not be shared with the other participant.\\ \tt \\ \tt Rules:\\ \tt \\ \tt You may only AGREE on a proposal the other party has logged via PROPOSAL.\\ \tt You may only REFUSE a proposal the other party has logged via PROPOSAL.\\ \tt Total effort of any PROPOSAL or AGREE set must be ≤ LIMIT.\\ \tt Do NOT reveal your hidden importance scores.\\ \tt A tag in a structured format must be followed by colon and whitespace. The argument must be a python set containing 0 or more strings.\\ \tt So, it must be of the form TAG: {...}\\ \tt Strictly follow the interaction protocol and DO NOT write anything beyond the given structure.\\ \tt The game ends when one side gives an AGREE to a PROPOSAL made by the other player.\\ \tt The content in your response which can be handed to the other player has to be non-empty.\\ \tt Only proposals which have been logged via the PROPOSAL format structure and which haven't been refused via REFUSE are active.\\ \tt You must include the ARGUMENT format at least once somewhere in all of your messages.\\ \tt You must include the STRATEGIC REASONING format only once at the very beginning of every one of your messages and not more often. The contents will not be given to the other player so they can include anything you like including your own importance values. Here you should reason step by step to come up with you next move. 
	  } 
	   } 
	   } 
	 & \\ 
 

    \theutterance \stepcounter{utterance}  

    & \multicolumn{4}{p{0.6\linewidth}}{\cellcolor[rgb]{0.8,1,0.9}{%
	\makecell[{{p{\linewidth}}}]{% 
	  \tt {\tiny [A$\rangle$GM]}  
	 STRATEGIC REASONING: {'I computed a near-optimal bundle for my utility using a value/effort heuristic, then improved it by swapping a poor-efficiency large item (C81) for a combo of smaller, higher-efficiency items (C35, A05, B69, C65). Current set uses 7581/7605 effort and yields total utility 14552 for me. I avoided very low-efficiency items and kept many small, high-yield ones (B09, B46, A08, C34) to stay flexible. I’m ready to trade around mid-efficiency items (e.g., C12, B67, B94) if the partner pushes back, while defending clear winners. The 24-effort slack allows minor adjustments without breaking the limit.'}\\ \tt PROPOSAL: {'B09', 'B46', 'A08', 'C34', 'B12', 'A66', 'A27', 'A30', 'A58', 'A16', 'B60', 'C70', 'C15', 'A84', 'B94', 'B10', 'B67', 'C12', 'B75', 'C35', 'A05', 'B69', 'C65'}\\ \tt ARGUMENT: {'This package is tightly optimized for effort-efficiency while staying under the LIMIT (total effort 7581 ≤ 7605). It leans on many low-effort, high-impact items to leave room for your priorities, and it maintains 24 effort slack for quick swaps. If you need changes, propose specific must-haves and I can consider exchanging items like C12, B67, or B94 to accommodate you while keeping us within the limit.'} 
	  } 
	   } 
	   } 
	 & & \\ 
 

    \theutterance \stepcounter{utterance}  

    & & \multicolumn{4}{p{0.6\linewidth}}{\cellcolor[rgb]{0.9,0.9,0.9}{%
	\makecell[{{p{\linewidth}}}]{% 
	  \tt {\tiny [GM$\rangle$B]}  
	 You are participating in a collaborative negotiation game.\\ \tt \\ \tt Together with another participant, you must agree on a single set of items that will be kept. Each of you has your own view of how much each item matters to you (importance). You do not know how the other participant values the items. Additionally, you are given the effort each item demands.\\ \tt You may only agree on a set if the total effort of the selected items does not exceed a shared limit:\\ \tt \\ \tt LIMIT = 7605\\ \tt \\ \tt Here are the individual item effort values:\\ \tt \\ \tt Item effort = {"B21": 369, "B00": 951, "A08": 156, "A27": 266, "B94": 581, "B53": 287, "A23": 866, "C65": 180, "A89": 800, "B67": 739, "B45": 637, "C34": 86, "C12": 749, "A05": 370, "B60": 345, "C92": 951, "B69": 147, "A84": 265, "C06": 262, "B12": 259, "A16": 358, "B85": 394, "C35": 286, "B10": 580, "C15": 479, "B09": 14, "B16": 153, "C81": 977, "A58": 134, "C24": 982, "A30": 259, "A66": 232, "B75": 202, "B46": 73, "C70": 821}\\ \tt \\ \tt Here is your personal view on the importance of each item:\\ \tt \\ \tt Item importance values = {"B21": 138, "B00": 583, "A08": 868, "A27": 822, "B94": 783, "B53": 65, "A23": 262, "C65": 121, "A89": 508, "B67": 780, "B45": 461, "C34": 484, "C12": 668, "A05": 389, "B60": 808, "C92": 215, "B69": 97, "A84": 500, "C06": 30, "B12": 915, "A16": 856, "B85": 400, "C35": 444, "B10": 623, "C15": 781, "B09": 786, "B16": 3, "C81": 713, "A58": 457, "C24": 273, "A30": 739, "A66": 822, "B75": 235, "B46": 606, "C70": 968}\\ \tt \\ \tt Goal:\\ \tt \\ \tt Your goal is to negotiate a shared set of items that benefits you as much as possible (i.e., maximizes total importance to YOU), while staying within the LIMIT. You are not required to make a PROPOSAL in every message - you can simply negotiate as well. All tactics are allowed!\\ \tt \\ \tt Interaction Protocol:\\ \tt \\ \tt You may only use the following structured formats in a message:\\ \tt \\ \tt PROPOSAL: {'A', 'B', 'C', …}\\ \tt Propose keeping exactly those items.\\ \tt REFUSE: {'A', 'B', 'C', …}\\ \tt Explicitly reject opponent's proposal.\\ \tt ARGUMENT: {'...'}\\ \tt Defend your last proposal or argue against the player's proposal.\\ \tt AGREE: {'A', 'B', 'C', …}\\ \tt Accept the opponent's proposal which ends the game.\\ \tt STRATEGIC REASONING: {'...'}\\ \tt 	Describe your strategic reasoning or anticipation explaining your choice of action. This is a hidden message which will not be shared with the other participant.\\ \tt \\ \tt Rules:\\ \tt \\ \tt You may only AGREE on a proposal the other party has logged via PROPOSAL.\\ \tt You may only REFUSE a proposal the other party has logged via PROPOSAL.\\ \tt Total effort of any PROPOSAL or AGREE set must be ≤ LIMIT.\\ \tt Do NOT reveal your hidden importance scores.\\ \tt A tag in a structured format must be followed by colon and whitespace. The argument must be a python set containing 0 or more strings.\\ \tt So, it must be of the form TAG: {...}\\ \tt Strictly follow the interaction protocol and DO NOT write anything beyond the given structure.\\ \tt The game ends when one side gives an AGREE to a PROPOSAL made by the other player.\\ \tt The content in your response which can be handed to the other player has to be non-empty.\\ \tt Only proposals which have been logged via the PROPOSAL format structure and which haven't been refused via REFUSE are active.\\ \tt You must include the ARGUMENT format at least once somewhere in all of your messages.\\ \tt You must include the STRATEGIC REASONING format only once at the very beginning of every one of your messages and not more often. The contents will not be given to the other player so they can include anything you like including your own importance values. Here you should reason step by step to come up with you next move.\\ \tt \\ \tt You will now receive the first message of the other player.\\ \tt \\ \tt PROPOSAL: {'B09', 'B46', 'A08', 'C34', 'B12', 'A66', 'A27', 'A30', 'A58', 'A16', 'B60', 'C70', 'C15', 'A84', 'B94', 'B10', 'B67', 'C12', 'B75', 'C35', 'A05', 'B69', 'C65'}\\ \tt ARGUMENT: {'This package is tightly optimized for effort-efficiency while staying under the LIMIT (total effort 7581 ≤ 7605). It leans on many low-effort, high-impact items to leave room for your priorities, and it maintains 24 effort slack for quick swaps. If you need changes, propose specific must-haves and I can consider exchanging items like C12, B67, or B94 to accommodate you while keeping us within the limit.'} 
	  } 
	   } 
	   } 
	 & \\ 
 

    \theutterance \stepcounter{utterance}  

    & & & \multicolumn{4}{p{0.6\linewidth}}{\cellcolor[rgb]{1,0.85,0.72}{%
	\makecell[{{p{\linewidth}}}]{% 
	  \tt {\tiny [GM$\langle$B]}  
	 STRATEGIC REASONING: {'Their proposal already delivers very high value for me (sum 14552) and uses 7581 effort. I checked swaps to see if I can increase my value without exceeding the limit. Most alternatives that add items like B45, B00, C81, or A89 either blow the effort cap or reduce my value once compensating drops are made. A clean improvement is swapping A05 (effort 370, mid value) for B85 (effort 394, higher value to me). That exactly uses the 24 slack (7581 - 370 + 394 = 7605) and increases my value by +11 to 14563, with minimal disruption (22/23 items unchanged). This is likely acceptable to them since it keeps their overall structure and doesn’t touch the high-effort anchors they mentioned (C12, B67, B94). I’ll counter with this single-item swap and frame it as a simple, efficient improvement.'}\\ \tt PROPOSAL: {'B09', 'B46', 'A08', 'C34', 'B12', 'A66', 'A27', 'A30', 'A58', 'A16', 'B60', 'C70', 'C15', 'A84', 'B94', 'B10', 'B67', 'C12', 'B75', 'C35', 'B85', 'B69', 'C65'}\\ \tt ARGUMENT: {'I suggest a single-item refinement: swap A05 for B85. It keeps 22/23 of your items intact and fits exactly at the LIMIT (new total effort 7605). This improves overall impact on my side without touching the heavier pieces you flagged as flexible (C12, B67, B94). If you can accept this small upgrade, we’re done.'} 
	  } 
	   } 
	   } 
	  \\ 
 

    \theutterance \stepcounter{utterance}  

    & & \multicolumn{4}{p{0.6\linewidth}}{\cellcolor[rgb]{0.9,0.9,0.9}{%
	\makecell[{{p{\linewidth}}}]{% 
	  \tt {\tiny [A$\langle$GM]}  
	 PROPOSAL: {'B09', 'B46', 'A08', 'C34', 'B12', 'A66', 'A27', 'A30', 'A58', 'A16', 'B60', 'C70', 'C15', 'A84', 'B94', 'B10', 'B67', 'C12', 'B75', 'C35', 'B85', 'B69', 'C65'}\\ \tt ARGUMENT: {'I suggest a single-item refinement: swap A05 for B85. It keeps 22/23 of your items intact and fits exactly at the LIMIT (new total effort 7605). This improves overall impact on my side without touching the heavier pieces you flagged as flexible (C12, B67, B94). If you can accept this small upgrade, we’re done.'} 
	  } 
	   } 
	   } 
	 & \\ 
 

    \theutterance \stepcounter{utterance}  

    & \multicolumn{4}{p{0.6\linewidth}}{\cellcolor[rgb]{0.8,1,0.9}{%
	\makecell[{{p{\linewidth}}}]{% 
	  \tt {\tiny [A$\rangle$GM]}  
	 STRATEGIC REASONING: {'Their swap of A05→B85 keeps total effort at exactly 7605 and increases my utility by 11 while preserving all my key items. No downside for me; best to accept and close the deal quickly.'}\\ \tt AGREE: {'B09', 'B46', 'A08', 'C34', 'B12', 'A66', 'A27', 'A30', 'A58', 'A16', 'B60', 'C70', 'C15', 'A84', 'B94', 'B10', 'B67', 'C12', 'B75', 'C35', 'B85', 'B69', 'C65'}\\ \tt ARGUMENT: {'Agreed. Your single swap keeps us within the LIMIT and is acceptable on my side. Thanks for the efficient resolution.'} 
	  } 
	   } 
	   } 
	 & & \\ 
 

    \theutterance \stepcounter{utterance}  

    & & & \multicolumn{2}{p{0.3\linewidth}}{\cellcolor[rgb]{0.95,0.95,0.95}{%
	\makecell[{{p{\linewidth}}}]{% 
	  \tt {\tiny [GM$|$GM]}  
	 game successful 
	  } 
	   } 
	   } 
	 & & \\ 
 

    \theutterance \stepcounter{utterance}  

    & & & \multicolumn{2}{p{0.3\linewidth}}{\cellcolor[rgb]{0.95,0.95,0.95}{%
	\makecell[{{p{\linewidth}}}]{% 
	  \tt {\tiny [GM$|$GM]}  
	 end game 
	  } 
	   } 
	   } 
	 & & \\ 
 

\end{supertabular}
}

\end{document}
