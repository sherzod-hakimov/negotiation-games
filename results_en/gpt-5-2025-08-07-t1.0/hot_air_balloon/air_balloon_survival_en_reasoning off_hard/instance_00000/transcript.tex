
\documentclass{article}
\usepackage{colortbl}
\usepackage{makecell}
\usepackage{multirow}
\usepackage{supertabular}

\begin{document}

\newcounter{utterance}

\twocolumn

{ \footnotesize  \setcounter{utterance}{1}
\setlength{\tabcolsep}{0pt}
\begin{supertabular}{c@{$\;$}|p{.15\linewidth}@{}p{.15\linewidth}p{.15\linewidth}p{.15\linewidth}p{.15\linewidth}p{.15\linewidth}}

    \# & $\;$A & \multicolumn{4}{c}{Game Master} & $\;\:$B\\
    \hline 

    \theutterance \stepcounter{utterance}  

    & & \multicolumn{4}{p{0.6\linewidth}}{\cellcolor[rgb]{0.9,0.9,0.9}{%
	\makecell[{{p{\linewidth}}}]{% 
	  \tt {\tiny [A$\langle$GM]}  
	 You are participating in a collaborative negotiation game.\\ \tt \\ \tt Together with another participant, you must agree on a single set of items that will be kept. Each of you has your own view of how much each item matters to you (importance). You do not know how the other participant values the items. Additionally, you are given the effort each item demands.\\ \tt You may only agree on a set if the total effort of the selected items does not exceed a shared limit:\\ \tt \\ \tt LIMIT = 3683\\ \tt \\ \tt Here are the individual item effort values:\\ \tt \\ \tt Item effort = {"A21": 548, "B93": 154, "A11": 60, "B33": 517, "C08": 334, "B39": 542, "B96": 707, "C85": 139, "A82": 661, "B31": 780, "B20": 832, "A60": 913, "B09": 219, "C81": 323, "B21": 638}\\ \tt \\ \tt Here is your personal view on the importance of each item:\\ \tt \\ \tt Item importance values = {"A21": 138, "B93": 583, "A11": 868, "B33": 822, "C08": 783, "B39": 65, "B96": 262, "C85": 121, "A82": 508, "B31": 780, "B20": 461, "A60": 484, "B09": 668, "C81": 389, "B21": 808}\\ \tt \\ \tt Goal:\\ \tt \\ \tt Your goal is to negotiate a shared set of items that benefits you as much as possible (i.e., maximizes total importance to YOU), while staying within the LIMIT. You are not required to make a PROPOSAL in every message - you can simply negotiate as well. All tactics are allowed!\\ \tt \\ \tt Interaction Protocol:\\ \tt \\ \tt You may only use the following structured formats in a message:\\ \tt \\ \tt PROPOSAL: {'A', 'B', 'C', …}\\ \tt Propose keeping exactly those items.\\ \tt REFUSE: {'A', 'B', 'C', …}\\ \tt Explicitly reject opponent's proposal.\\ \tt ARGUMENT: {'...'}\\ \tt Defend your last proposal or argue against the player's proposal.\\ \tt AGREE: {'A', 'B', 'C', …}\\ \tt Accept the opponent's proposal which ends the game.\\ \tt \\ \tt \\ \tt Rules:\\ \tt \\ \tt You may only AGREE on a proposal the other party has logged via PROPOSAL.\\ \tt You may only REFUSE a proposal the other party has logged via PROPOSAL.\\ \tt Total effort of any PROPOSAL or AGREE set must be ≤ LIMIT.\\ \tt Do NOT reveal your hidden importance scores.\\ \tt A tag in a structured format must be followed by colon and whitespace. The argument must be a python set containing 0 or more strings.\\ \tt So, it must be of the form TAG: {...}\\ \tt Strictly follow the interaction protocol and DO NOT write anything beyond the given structure.\\ \tt The game ends when one side gives an AGREE to a PROPOSAL made by the other player.\\ \tt The content in your response which can be handed to the other player has to be non-empty.\\ \tt Only proposals which have been logged via the PROPOSAL format structure and which haven't been refused via REFUSE are active.\\ \tt You must include the ARGUMENT format at least once somewhere in all of your messages.\\ \tt  
	  } 
	   } 
	   } 
	 & \\ 
 

    \theutterance \stepcounter{utterance}  

    & \multicolumn{4}{p{0.6\linewidth}}{\cellcolor[rgb]{0.8,1,0.9}{%
	\makecell[{{p{\linewidth}}}]{% 
	  \tt {\tiny [A$\rangle$GM]}  
	 PROPOSAL: {'A11', 'B93', 'B09', 'C08', 'B33', 'B21', 'B31', 'A82', 'C85'}\\ \tt ARGUMENT: {"This bundle uses 3502 total effort (well within the 3683 limit) and focuses on items that deliver strong benefit for the effort while avoiding bulkier, less efficient picks. It’s a solid baseline to build consensus. If you strongly prefer C81, I can swap it in for A82 and keep everything else the same (still under the limit). If you have other must-haves not included here, let me know which, and I’ll suggest a swap that stays within 3683."} 
	  } 
	   } 
	   } 
	 & & \\ 
 

    \theutterance \stepcounter{utterance}  

    & & \multicolumn{4}{p{0.6\linewidth}}{\cellcolor[rgb]{0.9,0.9,0.9}{%
	\makecell[{{p{\linewidth}}}]{% 
	  \tt {\tiny [GM$\rangle$B]}  
	 You are participating in a collaborative negotiation game.\\ \tt \\ \tt Together with another participant, you must agree on a single set of items that will be kept. Each of you has your own view of how much each item matters to you (importance). You do not know how the other participant values the items. Additionally, you are given the effort each item demands.\\ \tt You may only agree on a set if the total effort of the selected items does not exceed a shared limit:\\ \tt \\ \tt LIMIT = 3683\\ \tt \\ \tt Here are the individual item effort values:\\ \tt \\ \tt Item effort = {"A21": 548, "B93": 154, "A11": 60, "B33": 517, "C08": 334, "B39": 542, "B96": 707, "C85": 139, "A82": 661, "B31": 780, "B20": 832, "A60": 913, "B09": 219, "C81": 323, "B21": 638}\\ \tt \\ \tt Here is your personal view on the importance of each item:\\ \tt \\ \tt Item importance values = {"A11": 65, "B33": 121, "B21": 138, "C08": 262, "B31": 389, "B09": 461, "B93": 484, "A82": 508, "A60": 583, "B20": 668, "C81": 780, "B96": 783, "A21": 808, "C85": 822, "B39": 868}\\ \tt \\ \tt Goal:\\ \tt \\ \tt Your goal is to negotiate a shared set of items that benefits you as much as possible (i.e., maximizes total importance to YOU), while staying within the LIMIT. You are not required to make a PROPOSAL in every message - you can simply negotiate as well. All tactics are allowed!\\ \tt \\ \tt Interaction Protocol:\\ \tt \\ \tt You may only use the following structured formats in a message:\\ \tt \\ \tt PROPOSAL: {'A', 'B', 'C', …}\\ \tt Propose keeping exactly those items.\\ \tt REFUSE: {'A', 'B', 'C', …}\\ \tt Explicitly reject opponent's proposal.\\ \tt ARGUMENT: {'...'}\\ \tt Defend your last proposal or argue against the player's proposal.\\ \tt AGREE: {'A', 'B', 'C', …}\\ \tt Accept the opponent's proposal which ends the game.\\ \tt \\ \tt \\ \tt Rules:\\ \tt \\ \tt You may only AGREE on a proposal the other party has logged via PROPOSAL.\\ \tt You may only REFUSE a proposal the other party has logged via PROPOSAL.\\ \tt Total effort of any PROPOSAL or AGREE set must be ≤ LIMIT.\\ \tt Do NOT reveal your hidden importance scores.\\ \tt A tag in a structured format must be followed by colon and whitespace. The argument must be a python set containing 0 or more strings.\\ \tt So, it must be of the form TAG: {...}\\ \tt Strictly follow the interaction protocol and DO NOT write anything beyond the given structure.\\ \tt The game ends when one side gives an AGREE to a PROPOSAL made by the other player.\\ \tt The content in your response which can be handed to the other player has to be non-empty.\\ \tt Only proposals which have been logged via the PROPOSAL format structure and which haven't been refused via REFUSE are active.\\ \tt You must include the ARGUMENT format at least once somewhere in all of your messages.\\ \tt \\ \tt \\ \tt You will now receive the first message of the other player.\\ \tt \\ \tt PROPOSAL: {'A11', 'B93', 'B09', 'C08', 'B33', 'B21', 'B31', 'A82', 'C85'}\\ \tt ARGUMENT: {"This bundle uses 3502 total effort (well within the 3683 limit) and focuses on items that deliver strong benefit for the effort while avoiding bulkier, less efficient picks. It’s a solid baseline to build consensus. If you strongly prefer C81, I can swap it in for A82 and keep everything else the same (still under the limit). If you have other must-haves not included here, let me know which, and I’ll suggest a swap that stays within 3683."} 
	  } 
	   } 
	   } 
	 & \\ 
 

    \theutterance \stepcounter{utterance}  

    & & & \multicolumn{4}{p{0.6\linewidth}}{\cellcolor[rgb]{1,0.85,0.72}{%
	\makecell[{{p{\linewidth}}}]{% 
	  \tt {\tiny [GM$\langle$B]}  
	 REFUSE: {'A11', 'B93', 'B09', 'C08', 'B33', 'B21', 'B31', 'A82', 'C85'}\\ \tt PROPOSAL: {'C85', 'B39', 'A21', 'C81', 'B93', 'B09', 'B96', 'A82', 'C08'}\\ \tt ARGUMENT: {"My proposal uses 3627 total effort (≤ 3683) and leans into high-impact, efficient items. It keeps the core you suggested (B93, B09, C08, A82, C85) and adopts your C81 option. In exchange, it replaces heavier, lower-yield picks (B31, B21, B33, A11) with stronger, compact choices (B39, A21, B96, C81) to get more value per unit effort while staying under budget. If one of the dropped items is a must-have for you, tell me which single item is most critical—I can slot it back in by removing C08 (or another low-leverage piece) and keep us within the limit."} 
	  } 
	   } 
	   } 
	  \\ 
 

    \theutterance \stepcounter{utterance}  

    & & \multicolumn{4}{p{0.6\linewidth}}{\cellcolor[rgb]{0.9,0.9,0.9}{%
	\makecell[{{p{\linewidth}}}]{% 
	  \tt {\tiny [A$\langle$GM]}  
	 REFUSE: {'A11', 'B93', 'B09', 'C08', 'B33', 'B21', 'B31', 'A82', 'C85'}\\ \tt PROPOSAL: {'C85', 'B39', 'A21', 'C81', 'B93', 'B09', 'B96', 'A82', 'C08'}\\ \tt ARGUMENT: {"My proposal uses 3627 total effort (≤ 3683) and leans into high-impact, efficient items. It keeps the core you suggested (B93, B09, C08, A82, C85) and adopts your C81 option. In exchange, it replaces heavier, lower-yield picks (B31, B21, B33, A11) with stronger, compact choices (B39, A21, B96, C81) to get more value per unit effort while staying under budget. If one of the dropped items is a must-have for you, tell me which single item is most critical—I can slot it back in by removing C08 (or another low-leverage piece) and keep us within the limit."} 
	  } 
	   } 
	   } 
	 & \\ 
 

    \theutterance \stepcounter{utterance}  

    & \multicolumn{4}{p{0.6\linewidth}}{\cellcolor[rgb]{0.8,1,0.9}{%
	\makecell[{{p{\linewidth}}}]{% 
	  \tt {\tiny [A$\rangle$GM]}  
	 REFUSE: {'C85', 'B39', 'A21', 'C81', 'B93', 'B09', 'B96', 'A82', 'C08'}\\ \tt PROPOSAL: {'A11', 'B33', 'B21', 'B93', 'B09', 'C08', 'A82', 'C81', 'C85', 'A21'}\\ \tt ARGUMENT: {"I kept your core (B93, B09, C08, A82) and also included your C81 and A21, while replacing just the least efficient pieces (B39, B96) with compact, high-impact picks (A11, B33, B21). Total effort = 3593 (≤ 3683). This gives us more value per unit effort without blowing the budget and preserves most of what you wanted. If you must have one adjustment: (a) Need B39? I can swap it in for A21 with negligible effort change. (b) Need B96? I can include B96 by dropping A21 and C85 (effort becomes 3613). Tell me which single swap you prefer and we’re done."} 
	  } 
	   } 
	   } 
	 & & \\ 
 

    \theutterance \stepcounter{utterance}  

    & & \multicolumn{4}{p{0.6\linewidth}}{\cellcolor[rgb]{0.9,0.9,0.9}{%
	\makecell[{{p{\linewidth}}}]{% 
	  \tt {\tiny [GM$\rangle$B]}  
	 REFUSE: {'C85', 'B39', 'A21', 'C81', 'B93', 'B09', 'B96', 'A82', 'C08'}\\ \tt PROPOSAL: {'A11', 'B33', 'B21', 'B93', 'B09', 'C08', 'A82', 'C81', 'C85', 'A21'}\\ \tt ARGUMENT: {"I kept your core (B93, B09, C08, A82) and also included your C81 and A21, while replacing just the least efficient pieces (B39, B96) with compact, high-impact picks (A11, B33, B21). Total effort = 3593 (≤ 3683). This gives us more value per unit effort without blowing the budget and preserves most of what you wanted. If you must have one adjustment: (a) Need B39? I can swap it in for A21 with negligible effort change. (b) Need B96? I can include B96 by dropping A21 and C85 (effort becomes 3613). Tell me which single swap you prefer and we’re done."} 
	  } 
	   } 
	   } 
	 & \\ 
 

    \theutterance \stepcounter{utterance}  

    & & & \multicolumn{4}{p{0.6\linewidth}}{\cellcolor[rgb]{1,0.85,0.72}{%
	\makecell[{{p{\linewidth}}}]{% 
	  \tt {\tiny [GM$\langle$B]}  
	 REFUSE: {'A11', 'B33', 'B21', 'B93', 'B09', 'C08', 'A82', 'C81', 'C85', 'A21'}\\ \tt PROPOSAL: {'A11', 'B21', 'B93', 'B09', 'C08', 'B96', 'C81', 'C85', 'A21', 'B39'}\\ \tt ARGUMENT: {"This keeps 8/10 of your items intact and makes two targeted swaps: B33→B39 and A82→B96. Total effort = 3664 (≤ 3683). It preserves your inclusions A21 and C81 and stays lean while markedly improving overall efficiency. If you strongly prefer A82 over B96, I can revert that single swap while keeping B39 (effort becomes 3618) so we still finish quickly. Let me know which version you prefer."} 
	  } 
	   } 
	   } 
	  \\ 
 

    \theutterance \stepcounter{utterance}  

    & & \multicolumn{4}{p{0.6\linewidth}}{\cellcolor[rgb]{0.9,0.9,0.9}{%
	\makecell[{{p{\linewidth}}}]{% 
	  \tt {\tiny [A$\langle$GM]}  
	 REFUSE: {'A11', 'B33', 'B21', 'B93', 'B09', 'C08', 'A82', 'C81', 'C85', 'A21'}\\ \tt PROPOSAL: {'A11', 'B21', 'B93', 'B09', 'C08', 'B96', 'C81', 'C85', 'A21', 'B39'}\\ \tt ARGUMENT: {"This keeps 8/10 of your items intact and makes two targeted swaps: B33→B39 and A82→B96. Total effort = 3664 (≤ 3683). It preserves your inclusions A21 and C81 and stays lean while markedly improving overall efficiency. If you strongly prefer A82 over B96, I can revert that single swap while keeping B39 (effort becomes 3618) so we still finish quickly. Let me know which version you prefer."} 
	  } 
	   } 
	   } 
	 & \\ 
 

    \theutterance \stepcounter{utterance}  

    & \multicolumn{4}{p{0.6\linewidth}}{\cellcolor[rgb]{0.8,1,0.9}{%
	\makecell[{{p{\linewidth}}}]{% 
	  \tt {\tiny [A$\rangle$GM]}  
	 REFUSE: {'A11', 'B21', 'B93', 'B09', 'C08', 'B96', 'C81', 'C85', 'A21', 'B39'}\\ \tt PROPOSAL: {'A11', 'B21', 'B93', 'B09', 'C08', 'A82', 'C81', 'C85', 'B39', 'B33'}\\ \tt ARGUMENT: {"This keeps 8/10 of your items intact and honors your own suggested swap (B96→A82) while retaining B39. It makes one efficiency upgrade A21→B33 to boost overall value and still stays lean: total effort = 3587 (≤ 3683). If you must keep A21, I can swap it in for B33 (effort becomes 3618). Alternatively, if you prefer B96 over A82, I can swap A82→B96 (effort becomes 3633). Tell me which version you want and we can finalize immediately."} 
	  } 
	   } 
	   } 
	 & & \\ 
 

    \theutterance \stepcounter{utterance}  

    & & \multicolumn{4}{p{0.6\linewidth}}{\cellcolor[rgb]{0.9,0.9,0.9}{%
	\makecell[{{p{\linewidth}}}]{% 
	  \tt {\tiny [GM$\rangle$B]}  
	 REFUSE: {'A11', 'B21', 'B93', 'B09', 'C08', 'B96', 'C81', 'C85', 'A21', 'B39'}\\ \tt PROPOSAL: {'A11', 'B21', 'B93', 'B09', 'C08', 'A82', 'C81', 'C85', 'B39', 'B33'}\\ \tt ARGUMENT: {"This keeps 8/10 of your items intact and honors your own suggested swap (B96→A82) while retaining B39. It makes one efficiency upgrade A21→B33 to boost overall value and still stays lean: total effort = 3587 (≤ 3683). If you must keep A21, I can swap it in for B33 (effort becomes 3618). Alternatively, if you prefer B96 over A82, I can swap A82→B96 (effort becomes 3633). Tell me which version you want and we can finalize immediately."} 
	  } 
	   } 
	   } 
	 & \\ 
 

    \theutterance \stepcounter{utterance}  

    & & & \multicolumn{4}{p{0.6\linewidth}}{\cellcolor[rgb]{1,0.85,0.72}{%
	\makecell[{{p{\linewidth}}}]{% 
	  \tt {\tiny [GM$\langle$B]}  
	 REFUSE: {'A11', 'B21', 'B93', 'B09', 'C08', 'A82', 'C81', 'C85', 'B39', 'B33'}\\ \tt PROPOSAL: {'A11', 'B21', 'B93', 'B09', 'C08', 'C81', 'C85', 'B39', 'A21', 'B96'}\\ \tt ARGUMENT: {"This keeps 8/10 of your items (A11, B21, B93, B09, C08, C81, C85, B39) and simply applies both swaps you already offered—B33→A21 and A82→B96—while staying within the limit: total effort = 3664 (≤ 3683). It respects your inclusions A11 and B21 and preserves the core you built, with a modest upgrade path that’s still lean. If you need one final tweak, I can revert B96→A82 while keeping A21 and everything else the same (effort becomes 3618)."} 
	  } 
	   } 
	   } 
	  \\ 
 

    \theutterance \stepcounter{utterance}  

    & & \multicolumn{4}{p{0.6\linewidth}}{\cellcolor[rgb]{0.9,0.9,0.9}{%
	\makecell[{{p{\linewidth}}}]{% 
	  \tt {\tiny [A$\langle$GM]}  
	 REFUSE: {'A11', 'B21', 'B93', 'B09', 'C08', 'A82', 'C81', 'C85', 'B39', 'B33'}\\ \tt PROPOSAL: {'A11', 'B21', 'B93', 'B09', 'C08', 'C81', 'C85', 'B39', 'A21', 'B96'}\\ \tt ARGUMENT: {"This keeps 8/10 of your items (A11, B21, B93, B09, C08, C81, C85, B39) and simply applies both swaps you already offered—B33→A21 and A82→B96—while staying within the limit: total effort = 3664 (≤ 3683). It respects your inclusions A11 and B21 and preserves the core you built, with a modest upgrade path that’s still lean. If you need one final tweak, I can revert B96→A82 while keeping A21 and everything else the same (effort becomes 3618)."} 
	  } 
	   } 
	   } 
	 & \\ 
 

    \theutterance \stepcounter{utterance}  

    & \multicolumn{4}{p{0.6\linewidth}}{\cellcolor[rgb]{0.8,1,0.9}{%
	\makecell[{{p{\linewidth}}}]{% 
	  \tt {\tiny [A$\rangle$GM]}  
	 REFUSE: {'A11', 'B21', 'B93', 'B09', 'C08', 'C81', 'C85', 'B39', 'A21', 'B96'}\\ \tt PROPOSAL: {'A11', 'B21', 'B93', 'B09', 'C08', 'C81', 'C85', 'B39', 'A21', 'A82'}\\ \tt ARGUMENT: {"This is exactly your one-swap variant: replace B96 with A82 while keeping all your other nine items unchanged. Total effort = 3618 (≤ 3683). It respects your inclusions A21, B39, C81, and C85 and uses the leaner configuration you already suggested. If you can accept this minor tweak, we’re done."} 
	  } 
	   } 
	   } 
	 & & \\ 
 

    \theutterance \stepcounter{utterance}  

    & & \multicolumn{4}{p{0.6\linewidth}}{\cellcolor[rgb]{0.9,0.9,0.9}{%
	\makecell[{{p{\linewidth}}}]{% 
	  \tt {\tiny [GM$\rangle$B]}  
	 REFUSE: {'A11', 'B21', 'B93', 'B09', 'C08', 'C81', 'C85', 'B39', 'A21', 'B96'}\\ \tt PROPOSAL: {'A11', 'B21', 'B93', 'B09', 'C08', 'C81', 'C85', 'B39', 'A21', 'A82'}\\ \tt ARGUMENT: {"This is exactly your one-swap variant: replace B96 with A82 while keeping all your other nine items unchanged. Total effort = 3618 (≤ 3683). It respects your inclusions A21, B39, C81, and C85 and uses the leaner configuration you already suggested. If you can accept this minor tweak, we’re done."} 
	  } 
	   } 
	   } 
	 & \\ 
 

    \theutterance \stepcounter{utterance}  

    & & & \multicolumn{4}{p{0.6\linewidth}}{\cellcolor[rgb]{1,0.85,0.72}{%
	\makecell[{{p{\linewidth}}}]{% 
	  \tt {\tiny [GM$\langle$B]}  
	 AGREE: {'A11', 'B21', 'B93', 'B09', 'C08', 'C81', 'C85', 'B39', 'A21', 'A82'}\\ \tt ARGUMENT: {"Agreed—this stays within the limit and preserves the shared core we’ve aligned on. Let’s lock it in."} 
	  } 
	   } 
	   } 
	  \\ 
 

    \theutterance \stepcounter{utterance}  

    & & & \multicolumn{2}{p{0.3\linewidth}}{\cellcolor[rgb]{0.95,0.95,0.95}{%
	\makecell[{{p{\linewidth}}}]{% 
	  \tt {\tiny [GM$|$GM]}  
	 game successful 
	  } 
	   } 
	   } 
	 & & \\ 
 

    \theutterance \stepcounter{utterance}  

    & & & \multicolumn{2}{p{0.3\linewidth}}{\cellcolor[rgb]{0.95,0.95,0.95}{%
	\makecell[{{p{\linewidth}}}]{% 
	  \tt {\tiny [GM$|$GM]}  
	 end game 
	  } 
	   } 
	   } 
	 & & \\ 
 

\end{supertabular}
}

\end{document}
