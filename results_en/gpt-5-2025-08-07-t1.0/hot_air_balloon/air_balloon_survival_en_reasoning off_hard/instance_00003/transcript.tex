
\documentclass{article}
\usepackage{colortbl}
\usepackage{makecell}
\usepackage{multirow}
\usepackage{supertabular}

\begin{document}

\newcounter{utterance}

\twocolumn

{ \footnotesize  \setcounter{utterance}{1}
\setlength{\tabcolsep}{0pt}
\begin{supertabular}{c@{$\;$}|p{.15\linewidth}@{}p{.15\linewidth}p{.15\linewidth}p{.15\linewidth}p{.15\linewidth}p{.15\linewidth}}

    \# & $\;$A & \multicolumn{4}{c}{Game Master} & $\;\:$B\\
    \hline 

    \theutterance \stepcounter{utterance}  

    & & \multicolumn{4}{p{0.6\linewidth}}{\cellcolor[rgb]{0.9,0.9,0.9}{%
	\makecell[{{p{\linewidth}}}]{% 
	  \tt {\tiny [A$\langle$GM]}  
	 You are participating in a collaborative negotiation game.\\ \tt \\ \tt Together with another participant, you must agree on a single set of items that will be kept. Each of you has your own view of how much each item matters to you (importance). You do not know how the other participant values the items. Additionally, you are given the effort each item demands.\\ \tt You may only agree on a set if the total effort of the selected items does not exceed a shared limit:\\ \tt \\ \tt LIMIT = 3629\\ \tt \\ \tt Here are the individual item effort values:\\ \tt \\ \tt Item effort = {"A21": 5, "B93": 793, "A11": 21, "B33": 842, "C08": 313, "B39": 842, "B96": 631, "C85": 226, "A82": 87, "B31": 761, "B20": 230, "A60": 287, "B09": 698, "C81": 641, "B21": 882}\\ \tt \\ \tt Here is your personal view on the importance of each item:\\ \tt \\ \tt Item importance values = {"A21": 138, "B93": 583, "A11": 868, "B33": 822, "C08": 783, "B39": 65, "B96": 262, "C85": 121, "A82": 508, "B31": 780, "B20": 461, "A60": 484, "B09": 668, "C81": 389, "B21": 808}\\ \tt \\ \tt Goal:\\ \tt \\ \tt Your goal is to negotiate a shared set of items that benefits you as much as possible (i.e., maximizes total importance to YOU), while staying within the LIMIT. You are not required to make a PROPOSAL in every message - you can simply negotiate as well. All tactics are allowed!\\ \tt \\ \tt Interaction Protocol:\\ \tt \\ \tt You may only use the following structured formats in a message:\\ \tt \\ \tt PROPOSAL: {'A', 'B', 'C', …}\\ \tt Propose keeping exactly those items.\\ \tt REFUSE: {'A', 'B', 'C', …}\\ \tt Explicitly reject opponent's proposal.\\ \tt ARGUMENT: {'...'}\\ \tt Defend your last proposal or argue against the player's proposal.\\ \tt AGREE: {'A', 'B', 'C', …}\\ \tt Accept the opponent's proposal which ends the game.\\ \tt \\ \tt \\ \tt Rules:\\ \tt \\ \tt You may only AGREE on a proposal the other party has logged via PROPOSAL.\\ \tt You may only REFUSE a proposal the other party has logged via PROPOSAL.\\ \tt Total effort of any PROPOSAL or AGREE set must be ≤ LIMIT.\\ \tt Do NOT reveal your hidden importance scores.\\ \tt A tag in a structured format must be followed by colon and whitespace. The argument must be a python set containing 0 or more strings.\\ \tt So, it must be of the form TAG: {...}\\ \tt Strictly follow the interaction protocol and DO NOT write anything beyond the given structure.\\ \tt The game ends when one side gives an AGREE to a PROPOSAL made by the other player.\\ \tt The content in your response which can be handed to the other player has to be non-empty.\\ \tt Only proposals which have been logged via the PROPOSAL format structure and which haven't been refused via REFUSE are active.\\ \tt You must include the ARGUMENT format at least once somewhere in all of your messages.\\ \tt  
	  } 
	   } 
	   } 
	 & \\ 
 

    \theutterance \stepcounter{utterance}  

    & \multicolumn{4}{p{0.6\linewidth}}{\cellcolor[rgb]{0.8,1,0.9}{%
	\makecell[{{p{\linewidth}}}]{% 
	  \tt {\tiny [A$\rangle$GM]}  
	 PROPOSAL: {'A11', 'A21', 'A82', 'C08', 'B20', 'A60', 'B33', 'B21', 'B09', 'C85'}\\ \tt ARGUMENT: {"This selection prioritizes high-impact, effort-efficient items across categories, stays within the limit with a bit of room, and avoids high-effort/low-yield picks. If you strongly prefer B31 over B09, I'm open to swapping B09 out and dropping C85 to accommodate B31 while keeping the rest intact."} 
	  } 
	   } 
	   } 
	 & & \\ 
 

    \theutterance \stepcounter{utterance}  

    & & \multicolumn{4}{p{0.6\linewidth}}{\cellcolor[rgb]{0.9,0.9,0.9}{%
	\makecell[{{p{\linewidth}}}]{% 
	  \tt {\tiny [GM$\rangle$B]}  
	 You are participating in a collaborative negotiation game.\\ \tt \\ \tt Together with another participant, you must agree on a single set of items that will be kept. Each of you has your own view of how much each item matters to you (importance). You do not know how the other participant values the items. Additionally, you are given the effort each item demands.\\ \tt You may only agree on a set if the total effort of the selected items does not exceed a shared limit:\\ \tt \\ \tt LIMIT = 3629\\ \tt \\ \tt Here are the individual item effort values:\\ \tt \\ \tt Item effort = {"A21": 5, "B93": 793, "A11": 21, "B33": 842, "C08": 313, "B39": 842, "B96": 631, "C85": 226, "A82": 87, "B31": 761, "B20": 230, "A60": 287, "B09": 698, "C81": 641, "B21": 882}\\ \tt \\ \tt Here is your personal view on the importance of each item:\\ \tt \\ \tt Item importance values = {"A11": 65, "B33": 121, "B21": 138, "C08": 262, "B31": 389, "B09": 461, "B93": 484, "A82": 508, "A60": 583, "B20": 668, "C81": 780, "B96": 783, "A21": 808, "C85": 822, "B39": 868}\\ \tt \\ \tt Goal:\\ \tt \\ \tt Your goal is to negotiate a shared set of items that benefits you as much as possible (i.e., maximizes total importance to YOU), while staying within the LIMIT. You are not required to make a PROPOSAL in every message - you can simply negotiate as well. All tactics are allowed!\\ \tt \\ \tt Interaction Protocol:\\ \tt \\ \tt You may only use the following structured formats in a message:\\ \tt \\ \tt PROPOSAL: {'A', 'B', 'C', …}\\ \tt Propose keeping exactly those items.\\ \tt REFUSE: {'A', 'B', 'C', …}\\ \tt Explicitly reject opponent's proposal.\\ \tt ARGUMENT: {'...'}\\ \tt Defend your last proposal or argue against the player's proposal.\\ \tt AGREE: {'A', 'B', 'C', …}\\ \tt Accept the opponent's proposal which ends the game.\\ \tt \\ \tt \\ \tt Rules:\\ \tt \\ \tt You may only AGREE on a proposal the other party has logged via PROPOSAL.\\ \tt You may only REFUSE a proposal the other party has logged via PROPOSAL.\\ \tt Total effort of any PROPOSAL or AGREE set must be ≤ LIMIT.\\ \tt Do NOT reveal your hidden importance scores.\\ \tt A tag in a structured format must be followed by colon and whitespace. The argument must be a python set containing 0 or more strings.\\ \tt So, it must be of the form TAG: {...}\\ \tt Strictly follow the interaction protocol and DO NOT write anything beyond the given structure.\\ \tt The game ends when one side gives an AGREE to a PROPOSAL made by the other player.\\ \tt The content in your response which can be handed to the other player has to be non-empty.\\ \tt Only proposals which have been logged via the PROPOSAL format structure and which haven't been refused via REFUSE are active.\\ \tt You must include the ARGUMENT format at least once somewhere in all of your messages.\\ \tt \\ \tt \\ \tt You will now receive the first message of the other player.\\ \tt \\ \tt PROPOSAL: {'A11', 'A21', 'A82', 'C08', 'B20', 'A60', 'B33', 'B21', 'B09', 'C85'}\\ \tt ARGUMENT: {"This selection prioritizes high-impact, effort-efficient items across categories, stays within the limit with a bit of room, and avoids high-effort/low-yield picks. If you strongly prefer B31 over B09, I'm open to swapping B09 out and dropping C85 to accommodate B31 while keeping the rest intact."} 
	  } 
	   } 
	   } 
	 & \\ 
 

    \theutterance \stepcounter{utterance}  

    & & & \multicolumn{4}{p{0.6\linewidth}}{\cellcolor[rgb]{1,0.85,0.72}{%
	\makecell[{{p{\linewidth}}}]{% 
	  \tt {\tiny [GM$\langle$B]}  
	 ARGUMENT: {"I like most of your structure. To tighten effort and improve balance, I suggest swapping out the two heaviest, less efficient B-picks (B33 and B21) for B39 and B96. This keeps 8 of your 10 items, reduces total effort from 3591 to 3340 (well under the 3629 limit), and preserves strong coverage across categories. Also, I prefer to keep C85; with the extra room, if B31 is important to you, we can later swap B09 -> B31 and still remain under the limit."}\\ \tt PROPOSAL: {'A11', 'A21', 'A82', 'C08', 'B20', 'A60', 'B39', 'B96', 'B09', 'C85'} 
	  } 
	   } 
	   } 
	  \\ 
 

    \theutterance \stepcounter{utterance}  

    & & \multicolumn{4}{p{0.6\linewidth}}{\cellcolor[rgb]{0.9,0.9,0.9}{%
	\makecell[{{p{\linewidth}}}]{% 
	  \tt {\tiny [A$\langle$GM]}  
	 ARGUMENT: {"I like most of your structure. To tighten effort and improve balance, I suggest swapping out the two heaviest, less efficient B-picks (B33 and B21) for B39 and B96. This keeps 8 of your 10 items, reduces total effort from 3591 to 3340 (well under the 3629 limit), and preserves strong coverage across categories. Also, I prefer to keep C85; with the extra room, if B31 is important to you, we can later swap B09 -> B31 and still remain under the limit."}\\ \tt PROPOSAL: {'A11', 'A21', 'A82', 'C08', 'B20', 'A60', 'B39', 'B96', 'B09', 'C85'} 
	  } 
	   } 
	   } 
	 & \\ 
 

    \theutterance \stepcounter{utterance}  

    & \multicolumn{4}{p{0.6\linewidth}}{\cellcolor[rgb]{0.8,1,0.9}{%
	\makecell[{{p{\linewidth}}}]{% 
	  \tt {\tiny [A$\rangle$GM]}  
	 PROPOSAL: {'A11', 'A21', 'A82', 'C08', 'B20', 'A60', 'B33', 'B96', 'B31', 'C85'}\\ \tt ARGUMENT: {"Appreciate the leaner mix. To meet you halfway, I’ll keep C85 and B96 and take you up on the B09→B31 swap. In exchange, let's swap B39→B33—same effort, so total effort is 3403, well under 3629. This preserves your footprint and coverage while better balancing the B-slot for me. If you strongly prefer keeping B39, an alternative is to keep B09 and instead do B96→B21 (effort 3591)."} 
	  } 
	   } 
	   } 
	 & & \\ 
 

    \theutterance \stepcounter{utterance}  

    & & \multicolumn{4}{p{0.6\linewidth}}{\cellcolor[rgb]{0.9,0.9,0.9}{%
	\makecell[{{p{\linewidth}}}]{% 
	  \tt {\tiny [GM$\rangle$B]}  
	 PROPOSAL: {'A11', 'A21', 'A82', 'C08', 'B20', 'A60', 'B33', 'B96', 'B31', 'C85'}\\ \tt ARGUMENT: {"Appreciate the leaner mix. To meet you halfway, I’ll keep C85 and B96 and take you up on the B09→B31 swap. In exchange, let's swap B39→B33—same effort, so total effort is 3403, well under 3629. This preserves your footprint and coverage while better balancing the B-slot for me. If you strongly prefer keeping B39, an alternative is to keep B09 and instead do B96→B21 (effort 3591)."} 
	  } 
	   } 
	   } 
	 & \\ 
 

    \theutterance \stepcounter{utterance}  

    & & & \multicolumn{4}{p{0.6\linewidth}}{\cellcolor[rgb]{1,0.85,0.72}{%
	\makecell[{{p{\linewidth}}}]{% 
	  \tt {\tiny [GM$\langle$B]}  
	 REFUSE: {'A11', 'A21', 'A82', 'C08', 'B20', 'A60', 'B33', 'B96', 'B31', 'C85'}\\ \tt ARGUMENT: {"Thanks for meeting me halfway. I can’t accept B33—at identical effort to B39, it’s a clear downgrade, so keeping B39 is a clean, no-effort-cost upgrade. Also, B31 is very effort-heavy; swapping it for C81 gives a better return per unit effort and still preserves 8 of your 10 picks (A11, A21, A82, C08, B20, A60, C85, B96) while staying well under the limit (3283 total effort). If B31 is critical for you, I’m willing to do the same package but with B31 instead of C81 while keeping B39; that version is 3403 effort—say the word and I’ll confirm."}\\ \tt PROPOSAL: {'A11', 'A21', 'A82', 'C08', 'B20', 'A60', 'B39', 'B96', 'C85', 'C81'} 
	  } 
	   } 
	   } 
	  \\ 
 

    \theutterance \stepcounter{utterance}  

    & & \multicolumn{4}{p{0.6\linewidth}}{\cellcolor[rgb]{0.9,0.9,0.9}{%
	\makecell[{{p{\linewidth}}}]{% 
	  \tt {\tiny [A$\langle$GM]}  
	 REFUSE: {'A11', 'A21', 'A82', 'C08', 'B20', 'A60', 'B33', 'B96', 'B31', 'C85'}\\ \tt ARGUMENT: {"Thanks for meeting me halfway. I can’t accept B33—at identical effort to B39, it’s a clear downgrade, so keeping B39 is a clean, no-effort-cost upgrade. Also, B31 is very effort-heavy; swapping it for C81 gives a better return per unit effort and still preserves 8 of your 10 picks (A11, A21, A82, C08, B20, A60, C85, B96) while staying well under the limit (3283 total effort). If B31 is critical for you, I’m willing to do the same package but with B31 instead of C81 while keeping B39; that version is 3403 effort—say the word and I’ll confirm."}\\ \tt PROPOSAL: {'A11', 'A21', 'A82', 'C08', 'B20', 'A60', 'B39', 'B96', 'C85', 'C81'} 
	  } 
	   } 
	   } 
	 & \\ 
 

    \theutterance \stepcounter{utterance}  

    & \multicolumn{4}{p{0.6\linewidth}}{\cellcolor[rgb]{0.8,1,0.9}{%
	\makecell[{{p{\linewidth}}}]{% 
	  \tt {\tiny [A$\rangle$GM]}  
	 REFUSE: {'A11', 'A21', 'A82', 'C08', 'B20', 'A60', 'B39', 'B96', 'C85', 'C81'}\\ \tt PROPOSAL: {'A11', 'A21', 'A82', 'C08', 'B20', 'A60', 'B96', 'B31', 'B09', 'C85'}\\ \tt ARGUMENT: {"I’m keeping 8 of your 10 items (dropping only B39 and C81) and slotting in B31 and B09. This preserves your structure with C85 and B96, stays comfortably under the limit at 3259 effort, and provides a stronger overall balance for me without adding new category gaps. If B39 is absolutely essential for you, I can swap B96→B39 while keeping the rest unchanged (effort 3470), but I’d prefer to avoid B39 given its low return on my side."} 
	  } 
	   } 
	   } 
	 & & \\ 
 

    \theutterance \stepcounter{utterance}  

    & & \multicolumn{4}{p{0.6\linewidth}}{\cellcolor[rgb]{0.9,0.9,0.9}{%
	\makecell[{{p{\linewidth}}}]{% 
	  \tt {\tiny [GM$\rangle$B]}  
	 REFUSE: {'A11', 'A21', 'A82', 'C08', 'B20', 'A60', 'B39', 'B96', 'C85', 'C81'}\\ \tt PROPOSAL: {'A11', 'A21', 'A82', 'C08', 'B20', 'A60', 'B96', 'B31', 'B09', 'C85'}\\ \tt ARGUMENT: {"I’m keeping 8 of your 10 items (dropping only B39 and C81) and slotting in B31 and B09. This preserves your structure with C85 and B96, stays comfortably under the limit at 3259 effort, and provides a stronger overall balance for me without adding new category gaps. If B39 is absolutely essential for you, I can swap B96→B39 while keeping the rest unchanged (effort 3470), but I’d prefer to avoid B39 given its low return on my side."} 
	  } 
	   } 
	   } 
	 & \\ 
 

    \theutterance \stepcounter{utterance}  

    & & & \multicolumn{4}{p{0.6\linewidth}}{\cellcolor[rgb]{1,0.85,0.72}{%
	\makecell[{{p{\linewidth}}}]{% 
	  \tt {\tiny [GM$\langle$B]}  
	 PROPOSAL: {'A11', 'A21', 'A82', 'C08', 'B20', 'A60', 'B39', 'B31', 'B09', 'C85'}\\ \tt ARGUMENT: {"Adopting your own alternative: this keeps 9 of your 10 items intact and only swaps B96→B39. Effort is 3470 ≤ 3629. It preserves your B31 and B09, keeps C85, and is a minimal-change path that works well on my side. Ready to finalize on this."} 
	  } 
	   } 
	   } 
	  \\ 
 

    \theutterance \stepcounter{utterance}  

    & & \multicolumn{4}{p{0.6\linewidth}}{\cellcolor[rgb]{0.9,0.9,0.9}{%
	\makecell[{{p{\linewidth}}}]{% 
	  \tt {\tiny [A$\langle$GM]}  
	 PROPOSAL: {'A11', 'A21', 'A82', 'C08', 'B20', 'A60', 'B39', 'B31', 'B09', 'C85'}\\ \tt ARGUMENT: {"Adopting your own alternative: this keeps 9 of your 10 items intact and only swaps B96→B39. Effort is 3470 ≤ 3629. It preserves your B31 and B09, keeps C85, and is a minimal-change path that works well on my side. Ready to finalize on this."} 
	  } 
	   } 
	   } 
	 & \\ 
 

    \theutterance \stepcounter{utterance}  

    & \multicolumn{4}{p{0.6\linewidth}}{\cellcolor[rgb]{0.8,1,0.9}{%
	\makecell[{{p{\linewidth}}}]{% 
	  \tt {\tiny [A$\rangle$GM]}  
	 AGREE: {'A11', 'A21', 'A82', 'C08', 'B20', 'A60', 'B39', 'B31', 'B09', 'C85'}\\ \tt ARGUMENT: {"Agreed. This is a lean, balanced set under the limit that preserves the core we both prioritized. Locking it in."} 
	  } 
	   } 
	   } 
	 & & \\ 
 

    \theutterance \stepcounter{utterance}  

    & & & \multicolumn{2}{p{0.3\linewidth}}{\cellcolor[rgb]{0.95,0.95,0.95}{%
	\makecell[{{p{\linewidth}}}]{% 
	  \tt {\tiny [GM$|$GM]}  
	 game successful 
	  } 
	   } 
	   } 
	 & & \\ 
 

    \theutterance \stepcounter{utterance}  

    & & & \multicolumn{2}{p{0.3\linewidth}}{\cellcolor[rgb]{0.95,0.95,0.95}{%
	\makecell[{{p{\linewidth}}}]{% 
	  \tt {\tiny [GM$|$GM]}  
	 end game 
	  } 
	   } 
	   } 
	 & & \\ 
 

\end{supertabular}
}

\end{document}
