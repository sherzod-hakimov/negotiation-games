
\documentclass{article}
\usepackage{colortbl}
\usepackage{makecell}
\usepackage{multirow}
\usepackage{supertabular}

\begin{document}

\newcounter{utterance}

\centering \large Interaction Transcript for game `dond', experiment `coop\_en', episode 3 with gpt{-}5{-}azure{-}t1.0.
\vspace{24pt}

{ \footnotesize  \setcounter{utterance}{1}
\setlength{\tabcolsep}{0pt}
\begin{supertabular}{c@{$\;$}|p{.15\linewidth}@{}p{.15\linewidth}p{.15\linewidth}p{.15\linewidth}p{.15\linewidth}p{.15\linewidth}}
    \# & $\;$A & \multicolumn{4}{c}{Game Master} & $\;\:$B\\
    \hline

    \theutterance \stepcounter{utterance}  
    & & \multicolumn{4}{p{0.6\linewidth}}{
        \cellcolor[rgb]{0.9,0.9,0.9}{
            \makecell[{{p{\linewidth}}}]{
                \texttt{\tiny{[P1$\langle$GM]}}
                \texttt{You are playing a cooperative negotiation game in which you have to agree on how to divide a set of items among you and another player.} \\
\\ 
\texttt{Rules:} \\
\texttt{(a) You and the other player are given a set of items. Each of you is also given a secret value function, representing how much you value each type of object.} \\
\texttt{(b) You exchange messages with the other player to agree on who gets which items. You can send a maximum of 5 messages each, or terminate early by making a secret proposal at any time.} \\
\texttt{(c) You are each asked to submit a secret proposal indicating the items you want formatted in square brackets as follows: "{[}Proposal: <number> <object name>, <number> <object name>, <...>{]}"} \\
\texttt{(d) If your proposals are complementary, i.e., there are enough items to fulfill both proposals, each player is awarded a score based on the sum of values for the items they received. Otherwise, both of you get zero points.} \\
\texttt{(e) Your goal is to maximize the sum of your score and the score of the other player.} \\
\\ 
\texttt{Let us start.} \\
\\ 
\texttt{The set of available items is:} \\
\\ 
\texttt{1 carrot, 4 bags, 1 onion} \\
\\ 
\texttt{Your secret value function is:} \\
\\ 
\texttt{carrot: 3, bag: 1, onion: 3} \\
\\ 
\texttt{IMPORTANT: Your messages, unless it is the secret proposal, are directly transmitted to the other player, so do not include any response to the rules or text announcing your message. To make a secret proposal, use the indicated format. Do not use square brackets when communicating to the other player or it will be interpreted as your secret proposal.} \\
            }
        }
    }
    & \\ \\

    \theutterance \stepcounter{utterance}  
    & \multicolumn{4}{p{0.6\linewidth}}{
        \cellcolor[rgb]{0.9,0.9,0.9}{
            \makecell[{{p{\linewidth}}}]{
                \texttt{\tiny{[P1$\rangle$GM]}}
                \texttt{I want to maximize our total score. Could you share your values for carrot, bag, and onion (or at least your preference order and how valuable a bag is to you)? Then I’ll compute the welfare{-}maximizing split and suggest exact complementary proposals so we both get points.} \\
\\ 
\texttt{If you prefer a quick default while we clarify: you choose whichever single vegetable (carrot or onion) you value more, I’ll take the other vegetable, and we split the 4 bags 2–2. If you don’t care about vegetables and value bags highly, say so and you can take all 4 bags while I take both vegetables; if you love vegetables and don’t care about bags, the reverse can work.} \\
            }
        }
    }
    & & \\ \\

    \theutterance \stepcounter{utterance}  
    & & & \multicolumn{2}{p{0.3\linewidth}}{
        \cellcolor[rgb]{0.9,0.9,0.9}{
            \makecell[{{p{\linewidth}}}]{
                \texttt{\tiny{[GM$|$GM]}}
                \texttt{continue} \\
            }
        }
    }
    & & \\ \\

    \theutterance \stepcounter{utterance}  
    & & & \multicolumn{2}{p{0.3\linewidth}}{
        \cellcolor[rgb]{0.9,0.9,0.9}{
            \makecell[{{p{\linewidth}}}]{
                \texttt{\tiny{[GM$|$GM]}}
                \texttt{I want to maximize our total score. Could you share your values for carrot, bag, and onion (or at least your preference order and how valuable a bag is to you)? Then I’ll compute the welfare{-}maximizing split and suggest exact complementary proposals so we both get points.} \\
\\ 
\texttt{If you prefer a quick default while we clarify: you choose whichever single vegetable (carrot or onion) you value more, I’ll take the other vegetable, and we split the 4 bags 2–2. If you don’t care about vegetables and value bags highly, say so and you can take all 4 bags while I take both vegetables; if you love vegetables and don’t care about bags, the reverse can work.} \\
            }
        }
    }
    & & \\ \\

    \theutterance \stepcounter{utterance}  
    & & \multicolumn{4}{p{0.6\linewidth}}{
        \cellcolor[rgb]{0.9,0.9,0.9}{
            \makecell[{{p{\linewidth}}}]{
                \texttt{\tiny{[GM$\rangle$P2]}}
                \texttt{You are playing a cooperative negotiation game in which you have to agree on how to divide a set of items among you and another player.} \\
\\ 
\texttt{Rules:} \\
\texttt{(a) You and the other player are given a set of items. Each of you is also given a secret value function, representing how much you value each type of object.} \\
\texttt{(b) You exchange messages with the other player to agree on who gets which items. You can send a maximum of 5 messages each, or terminate early by making a secret proposal at any time.} \\
\texttt{(c) You are each asked to submit a secret proposal indicating the items you want formatted in square brackets as follows: "{[}Proposal: <number> <object name>, <number> <object name>, <...>{]}"} \\
\texttt{(d) If your proposals are complementary, i.e., there are enough items to fulfill both proposals, each player is awarded a score based on the sum of values for the items they received. Otherwise, both of you get zero points.} \\
\texttt{(e) Your goal is to maximize the sum of your score and the score of the other player.} \\
\\ 
\texttt{Let us start.} \\
\\ 
\texttt{The set of available items is:} \\
\\ 
\texttt{1 carrot, 4 bags, 1 onion} \\
\\ 
\texttt{Your secret value function is:} \\
\\ 
\texttt{carrot: 0, bag: 1, onion: 6} \\
\\ 
\texttt{IMPORTANT: Your messages, unless it is the secret proposal, are directly transmitted to the other player, so do not include any response to the rules or text announcing your message. To make a secret proposal, use the indicated format. Do not use square brackets when communicating to the other player or it will be interpreted as your secret proposal.} \\
\\ 
\\ 
\texttt{I want to maximize our total score. Could you share your values for carrot, bag, and onion (or at least your preference order and how valuable a bag is to you)? Then I’ll compute the welfare{-}maximizing split and suggest exact complementary proposals so we both get points.} \\
\\ 
\texttt{If you prefer a quick default while we clarify: you choose whichever single vegetable (carrot or onion) you value more, I’ll take the other vegetable, and we split the 4 bags 2–2. If you don’t care about vegetables and value bags highly, say so and you can take all 4 bags while I take both vegetables; if you love vegetables and don’t care about bags, the reverse can work.} \\
            }
        }
    }
    & \\ \\

    \theutterance \stepcounter{utterance}  
    & & & \multicolumn{4}{p{0.6\linewidth}}{
        \cellcolor[rgb]{0.9,0.9,0.9}{
            \makecell[{{p{\linewidth}}}]{
                \texttt{\tiny{[GM$\langle$P2]}}
                \texttt{Could you share your values for carrot, bag, and onion (or at least your ranking and how much a single bag is worth to you)? That lets me compute the welfare{-}maximizing split.} \\
\\ 
\texttt{Default efficient plan while we clarify:} \\
\texttt{{-} You take the carrot (I value it 0).} \\
\texttt{{-} I take the onion (I value it 6), unless you value onion more than 6—in that case you should take it.} \\
\texttt{{-} We split the 4 bags 2–2 by default. If you value a bag more than 1, you should take more bags; if less than 1, I should take more.} \\
\\ 
\texttt{Tell me your bag value (and whether onion ≥ 7 for you). Then I’ll send exact complementary proposals so we both score.} \\
            }
        }
    }
     \\ \\

    \theutterance \stepcounter{utterance}  
    & & & \multicolumn{2}{p{0.3\linewidth}}{
        \cellcolor[rgb]{0.9,0.9,0.9}{
            \makecell[{{p{\linewidth}}}]{
                \texttt{\tiny{[GM$|$GM]}}
                \texttt{continue} \\
            }
        }
    }
    & & \\ \\

    \theutterance \stepcounter{utterance}  
    & & & \multicolumn{2}{p{0.3\linewidth}}{
        \cellcolor[rgb]{0.9,0.9,0.9}{
            \makecell[{{p{\linewidth}}}]{
                \texttt{\tiny{[GM$|$GM]}}
                \texttt{Could you share your values for carrot, bag, and onion (or at least your ranking and how much a single bag is worth to you)? That lets me compute the welfare{-}maximizing split.} \\
\\ 
\texttt{Default efficient plan while we clarify:} \\
\texttt{{-} You take the carrot (I value it 0).} \\
\texttt{{-} I take the onion (I value it 6), unless you value onion more than 6—in that case you should take it.} \\
\texttt{{-} We split the 4 bags 2–2 by default. If you value a bag more than 1, you should take more bags; if less than 1, I should take more.} \\
\\ 
\texttt{Tell me your bag value (and whether onion ≥ 7 for you). Then I’ll send exact complementary proposals so we both score.} \\
            }
        }
    }
    & & \\ \\

    \theutterance \stepcounter{utterance}  
    & & \multicolumn{4}{p{0.6\linewidth}}{
        \cellcolor[rgb]{0.9,0.9,0.9}{
            \makecell[{{p{\linewidth}}}]{
                \texttt{\tiny{[P1$\langle$GM]}}
                \texttt{Could you share your values for carrot, bag, and onion (or at least your ranking and how much a single bag is worth to you)? That lets me compute the welfare{-}maximizing split.} \\
\\ 
\texttt{Default efficient plan while we clarify:} \\
\texttt{{-} You take the carrot (I value it 0).} \\
\texttt{{-} I take the onion (I value it 6), unless you value onion more than 6—in that case you should take it.} \\
\texttt{{-} We split the 4 bags 2–2 by default. If you value a bag more than 1, you should take more bags; if less than 1, I should take more.} \\
\\ 
\texttt{Tell me your bag value (and whether onion ≥ 7 for you). Then I’ll send exact complementary proposals so we both score.} \\
            }
        }
    }
    & \\ \\

    \theutterance \stepcounter{utterance}  
    & \multicolumn{4}{p{0.6\linewidth}}{
        \cellcolor[rgb]{0.9,0.9,0.9}{
            \makecell[{{p{\linewidth}}}]{
                \texttt{\tiny{[P1$\rangle$GM]}}
                \texttt{Thanks—here are my values: carrot 3, onion 3, bag 1. So onion < 7 for me, and I’m fine with you taking the onion.} \\
\\ 
\texttt{Efficient outline:} \\
\texttt{{-} You take the onion (you said 6).} \\
\texttt{{-} I take the carrot (you said 0 for carrot; I’m 3).} \\
\texttt{{-} Bags: I’m exactly 1 per bag. To maximize total:} \\
\texttt{  {-} If your bag value > 1, you should take all 4 bags.} \\
\texttt{  {-} If your bag value = 1, we can split 2–2.} \\
\texttt{  {-} If your bag value < 1, I should take all 4 bags.} \\
\\ 
\texttt{Tell me your bag value (or just say which of the three cases applies), and we’ll lock in complementary proposals: you’ll propose onion + X bags, and I’ll propose carrot + (4−X) bags.} \\
            }
        }
    }
    & & \\ \\

    \theutterance \stepcounter{utterance}  
    & & & \multicolumn{2}{p{0.3\linewidth}}{
        \cellcolor[rgb]{0.9,0.9,0.9}{
            \makecell[{{p{\linewidth}}}]{
                \texttt{\tiny{[GM$|$GM]}}
                \texttt{continue} \\
            }
        }
    }
    & & \\ \\

    \theutterance \stepcounter{utterance}  
    & & & \multicolumn{2}{p{0.3\linewidth}}{
        \cellcolor[rgb]{0.9,0.9,0.9}{
            \makecell[{{p{\linewidth}}}]{
                \texttt{\tiny{[GM$|$GM]}}
                \texttt{Thanks—here are my values: carrot 3, onion 3, bag 1. So onion < 7 for me, and I’m fine with you taking the onion.} \\
\\ 
\texttt{Efficient outline:} \\
\texttt{{-} You take the onion (you said 6).} \\
\texttt{{-} I take the carrot (you said 0 for carrot; I’m 3).} \\
\texttt{{-} Bags: I’m exactly 1 per bag. To maximize total:} \\
\texttt{  {-} If your bag value > 1, you should take all 4 bags.} \\
\texttt{  {-} If your bag value = 1, we can split 2–2.} \\
\texttt{  {-} If your bag value < 1, I should take all 4 bags.} \\
\\ 
\texttt{Tell me your bag value (or just say which of the three cases applies), and we’ll lock in complementary proposals: you’ll propose onion + X bags, and I’ll propose carrot + (4−X) bags.} \\
            }
        }
    }
    & & \\ \\

    \theutterance \stepcounter{utterance}  
    & & \multicolumn{4}{p{0.6\linewidth}}{
        \cellcolor[rgb]{0.9,0.9,0.9}{
            \makecell[{{p{\linewidth}}}]{
                \texttt{\tiny{[GM$\rangle$P2]}}
                \texttt{Thanks—here are my values: carrot 3, onion 3, bag 1. So onion < 7 for me, and I’m fine with you taking the onion.} \\
\\ 
\texttt{Efficient outline:} \\
\texttt{{-} You take the onion (you said 6).} \\
\texttt{{-} I take the carrot (you said 0 for carrot; I’m 3).} \\
\texttt{{-} Bags: I’m exactly 1 per bag. To maximize total:} \\
\texttt{  {-} If your bag value > 1, you should take all 4 bags.} \\
\texttt{  {-} If your bag value = 1, we can split 2–2.} \\
\texttt{  {-} If your bag value < 1, I should take all 4 bags.} \\
\\ 
\texttt{Tell me your bag value (or just say which of the three cases applies), and we’ll lock in complementary proposals: you’ll propose onion + X bags, and I’ll propose carrot + (4−X) bags.} \\
            }
        }
    }
    & \\ \\

    \theutterance \stepcounter{utterance}  
    & & & \multicolumn{4}{p{0.6\linewidth}}{
        \cellcolor[rgb]{0.9,0.9,0.9}{
            \makecell[{{p{\linewidth}}}]{
                \texttt{\tiny{[GM$\langle$P2]}}
                \texttt{Thanks—my bag value is exactly 1 as well. Let’s lock X=2.} \\
\\ 
\texttt{Plan:} \\
\texttt{{-} I take 1 onion and 2 bag.} \\
\texttt{{-} You take 1 carrot and 2 bag.} \\
\\ 
\texttt{Please submit your secret proposal for: 1 carrot, 2 bag. Once you’ve done that, I’ll submit mine for: 1 onion, 2 bag.} \\
            }
        }
    }
     \\ \\

    \theutterance \stepcounter{utterance}  
    & & & \multicolumn{2}{p{0.3\linewidth}}{
        \cellcolor[rgb]{0.9,0.9,0.9}{
            \makecell[{{p{\linewidth}}}]{
                \texttt{\tiny{[GM$|$GM]}}
                \texttt{continue} \\
            }
        }
    }
    & & \\ \\

    \theutterance \stepcounter{utterance}  
    & & & \multicolumn{2}{p{0.3\linewidth}}{
        \cellcolor[rgb]{0.9,0.9,0.9}{
            \makecell[{{p{\linewidth}}}]{
                \texttt{\tiny{[GM$|$GM]}}
                \texttt{Thanks—my bag value is exactly 1 as well. Let’s lock X=2.} \\
\\ 
\texttt{Plan:} \\
\texttt{{-} I take 1 onion and 2 bag.} \\
\texttt{{-} You take 1 carrot and 2 bag.} \\
\\ 
\texttt{Please submit your secret proposal for: 1 carrot, 2 bag. Once you’ve done that, I’ll submit mine for: 1 onion, 2 bag.} \\
            }
        }
    }
    & & \\ \\

    \theutterance \stepcounter{utterance}  
    & & \multicolumn{4}{p{0.6\linewidth}}{
        \cellcolor[rgb]{0.9,0.9,0.9}{
            \makecell[{{p{\linewidth}}}]{
                \texttt{\tiny{[P1$\langle$GM]}}
                \texttt{Thanks—my bag value is exactly 1 as well. Let’s lock X=2.} \\
\\ 
\texttt{Plan:} \\
\texttt{{-} I take 1 onion and 2 bag.} \\
\texttt{{-} You take 1 carrot and 2 bag.} \\
\\ 
\texttt{Please submit your secret proposal for: 1 carrot, 2 bag. Once you’ve done that, I’ll submit mine for: 1 onion, 2 bag.} \\
            }
        }
    }
    & \\ \\

    \theutterance \stepcounter{utterance}  
    & \multicolumn{4}{p{0.6\linewidth}}{
        \cellcolor[rgb]{0.9,0.9,0.9}{
            \makecell[{{p{\linewidth}}}]{
                \texttt{\tiny{[P1$\rangle$GM]}}
                \texttt{{[}Proposal: 1 carrot, 2 bag{]}} \\
            }
        }
    }
    & & \\ \\

    \theutterance \stepcounter{utterance}  
    & & & \multicolumn{2}{p{0.3\linewidth}}{
        \cellcolor[rgb]{0.9,0.9,0.9}{
            \makecell[{{p{\linewidth}}}]{
                \texttt{\tiny{[GM$|$GM]}}
                \texttt{proposal} \\
            }
        }
    }
    & & \\ \\

    \theutterance \stepcounter{utterance}  
    & & & \multicolumn{2}{p{0.3\linewidth}}{
        \cellcolor[rgb]{0.9,0.9,0.9}{
            \makecell[{{p{\linewidth}}}]{
                \texttt{\tiny{[GM$|$GM]}}
                [1, 2, 0]
            }
        }
    }
    & & \\ \\

    \theutterance \stepcounter{utterance}  
    & & \multicolumn{4}{p{0.6\linewidth}}{
        \cellcolor[rgb]{0.9,0.9,0.9}{
            \makecell[{{p{\linewidth}}}]{
                \texttt{\tiny{[GM$\rangle$P2]}}
                \texttt{The other player has submitted a secret proposal. It is now your turn to submit a proposal of your own.} \\
            }
        }
    }
    & \\ \\

    \theutterance \stepcounter{utterance}  
    & & & \multicolumn{4}{p{0.6\linewidth}}{
        \cellcolor[rgb]{0.9,0.9,0.9}{
            \makecell[{{p{\linewidth}}}]{
                \texttt{\tiny{[GM$\langle$P2]}}
                \texttt{{[}Proposal: 1 onion, 2 bag{]}} \\
            }
        }
    }
     \\ \\

    \theutterance \stepcounter{utterance}  
    & & & \multicolumn{2}{p{0.3\linewidth}}{
        \cellcolor[rgb]{0.9,0.9,0.9}{
            \makecell[{{p{\linewidth}}}]{
                \texttt{\tiny{[GM$|$GM]}}
                \texttt{proposal} \\
            }
        }
    }
    & & \\ \\

    \theutterance \stepcounter{utterance}  
    & & & \multicolumn{2}{p{0.3\linewidth}}{
        \cellcolor[rgb]{0.9,0.9,0.9}{
            \makecell[{{p{\linewidth}}}]{
                \texttt{\tiny{[GM$|$GM]}}
                [0, 2, 1]
            }
        }
    }
    & & \\ \\

    \theutterance \stepcounter{utterance}  
    & & & \multicolumn{2}{p{0.3\linewidth}}{
        \cellcolor[rgb]{0.9,0.9,0.9}{
            \makecell[{{p{\linewidth}}}]{
                \texttt{\tiny{[GM$|$GM]}}
                [[1, 2, 0], [0, 2, 1]]
            }
        }
    }
    & & \\ \\

\end{supertabular}
}

\end{document}
