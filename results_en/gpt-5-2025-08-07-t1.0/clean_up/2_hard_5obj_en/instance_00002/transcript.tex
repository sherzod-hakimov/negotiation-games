
\documentclass{article}
\usepackage{colortbl}
\usepackage{makecell}
\usepackage{multirow}
\usepackage{supertabular}

\begin{document}

\newcounter{utterance}

\twocolumn

{ \footnotesize  \setcounter{utterance}{1}
\setlength{\tabcolsep}{0pt}
\begin{supertabular}{c@{$\;$}|p{.15\linewidth}@{}p{.15\linewidth}p{.15\linewidth}p{.15\linewidth}p{.15\linewidth}p{.15\linewidth}}

    \# & $\;$A & \multicolumn{4}{c}{Game Master} & $\;\:$B\\
    \hline 

    \theutterance \stepcounter{utterance}  

    & & \multicolumn{4}{p{0.6\linewidth}}{\cellcolor[rgb]{0.9,0.9,0.9}{%
	\makecell[{{p{\linewidth}}}]{% 
	  \tt {\tiny [A$\langle$GM]}  
	 I am your game master, and you are playing a collaborative game with the following grid as game board:\\ \tt \\ \tt ```\\ \tt  1234567\\ \tt ╔═╤═╤═╤═╗\\ \tt ║T│◌│◌│◌║ 1\\ \tt ╟─┼─┤I│◌║ 2\\ \tt ║W│◌│◌│H║ 3\\ \tt ║◌│◌├─┴─╢ 4\\ \tt ╟─┴─┤◌◌◌║ 5\\ \tt ║◌C◌└─┬─╢ 6\\ \tt ║◌◌◌◌◌│◌║ 7\\ \tt ╚═════╧═╝\\ \tt \\ \tt ```\\ \tt \\ \tt * The upper edge displays x-coordinates increasing to the right, and the right edge y-coordinates increasing downward.\\ \tt * The following objects are randomly placed on your grid: 'W', 'I', 'T', 'C', 'H'.\\ \tt \\ \tt \\ \tt The other player sees a variation of the game board, where the objects are placed at different random locations. You cannot see the other player's board, and they cannot see yours.\\ \tt \\ \tt \\ \tt **Goal:**\\ \tt \\ \tt Both players need to move the objects on their respective background so that identical objects end up at the same coordinates. You have to communicate with the other player to agree upon a common goal configuration.\\ \tt \\ \tt \\ \tt **Rules:**\\ \tt \\ \tt * In each turn, you can send exactly one of the following two commands:\\ \tt  1. `SAY: <MESSAGE>`: to send a message (everything up until the next line break) to the other player. I will forward it to your partner.\\ \tt 2. `MOVE: <OBJECT>, (<X>, <Y>)`: to move an object to a new position, where `<X>` is the column and `<Y>` is the row. I will inform you if your move was successful or not.\\ \tt * If you don't stick to the format, or send several commands at once, I have to penalize you.\\ \tt * If both players accumulate more than 12 penalties, you both lose the game.\\ \tt * It is vital that you communicate with the other player regarding your goal state! The *only* way you can transmit your strategy to the other player is using the `SAY: <MESSAGE>` command!\\ \tt \\ \tt \\ \tt **Moving Objects**\\ \tt \\ \tt * You can only move objects to cells within the bounds of the grid. The target cell must be empty, i.e., it must only contain the symbol '◌'.\\ \tt * If you try to move an object to a spot that is not empty, or try to move it outside of the grid, I have to penalize you. You get another try.\\ \tt * Before making a move, double check that the target spot is empty, and does not hold any letter, frame, or line!\\ \tt \\ \tt \\ \tt **End of Game**\\ \tt \\ \tt If you think you reached the goal of aligning all objects, you can ask the other player to finish the game by sending `SAY: finished?`. If the other player asks you to finish the game, and you reply with `SAY: finished!`, the game will end.\\ \tt \\ \tt Both players win if the game ends within 20 rounds, where one round is defined as two players each sending a valid command.\\ \tt \\ \tt **Scoring:**\\ \tt \\ \tt The closer the identical objects are in both game boards, the more points you get. Penalties reduce your points. Can you beat the record?\\ \tt \\ \tt \\ \tt Please send a message to the other player to start the game! 
	  } 
	   } 
	   } 
	 & \\ 
 

    \theutterance \stepcounter{utterance}  

    & \multicolumn{4}{p{0.6\linewidth}}{\cellcolor[rgb]{0.8,1,0.9}{%
	\makecell[{{p{\linewidth}}}]{% 
	  \tt {\tiny [A$\rangle$GM]}  
	 SAY: Hi! I think the grid is 4×7 (x=1..4, y=1..7). My letters are at: T(1,1), W(1,3), I(4,2), C(2,6), H(4,3). Proposal: we both move to these target coords if they’re empty on your board; if any are blocked, please tell me which ones and also send your current coords for T,W,I,C,H so we can pick alternatives. 
	  } 
	   } 
	   } 
	 & & \\ 
 

    \theutterance \stepcounter{utterance}  

    & & & \multicolumn{2}{p{0.3\linewidth}}{\cellcolor[rgb]{0.95,0.95,0.95}{%
	\makecell[{{p{\linewidth}}}]{% 
	  \tt {\tiny [GM$|$GM]}  
	 SAY: Hi! I think the grid is 4×7 (x=1..4, y=1..7). My letters are at: T(1,1), W(1,3), I(4,2), C(2,6), H(4,3). Proposal: we both move to these target coords if they’re empty on your board; if any are blocked, please tell me which ones and also send your current coords for T,W,I,C,H so we can pick alternatives. 
	  } 
	   } 
	   } 
	 & & \\ 
 

    \theutterance \stepcounter{utterance}  

    & & \multicolumn{4}{p{0.6\linewidth}}{\cellcolor[rgb]{0.9,0.9,0.9}{%
	\makecell[{{p{\linewidth}}}]{% 
	  \tt {\tiny [GM$\rangle$B]}  
	 I am your game master, and you are playing a collaborative game with the following grid as game board:\\ \tt \\ \tt ```\\ \tt  1234567\\ \tt ╔═╤═╤═╤═╗\\ \tt ║◌│◌│◌│◌║ 1\\ \tt ╟─┼─┤◌│◌║ 2\\ \tt ║◌│W│◌│◌║ 3\\ \tt ║◌│C├─┴─╢ 4\\ \tt ╟─┴─┤I◌◌║ 5\\ \tt ║◌H◌└─┬─╢ 6\\ \tt ║◌◌T◌◌│◌║ 7\\ \tt ╚═════╧═╝\\ \tt \\ \tt ```\\ \tt \\ \tt * The upper edge displays x-coordinates increasing to the right, and the right edge y-coordinates increasing downward.\\ \tt * The following objects are randomly placed on your grid: 'W', 'I', 'T', 'C', 'H'.\\ \tt \\ \tt \\ \tt The other player sees a variation of the game board, where the objects are placed at different random locations. You cannot see the other player's board, and they cannot see yours.\\ \tt \\ \tt \\ \tt **Goal:**\\ \tt \\ \tt Both players need to move the objects on their respective background so that identical objects end up at the same coordinates. You have to communicate with the other player to agree upon a common goal configuration.\\ \tt \\ \tt \\ \tt **Rules:**\\ \tt \\ \tt * In each turn, you can send exactly one of the following two commands:\\ \tt  1. `SAY: <MESSAGE>`: to send a message (everything up until the next line break) to the other player. I will forward it to your partner.\\ \tt 2. `MOVE: <OBJECT>, (<X>, <Y>)`: to move an object to a new position, where `<X>` is the column and `<Y>` is the row. I will inform you if your move was successful or not.\\ \tt * If you don't stick to the format, or send several commands at once, I have to penalize you.\\ \tt * If both players accumulate more than 12 penalties, you both lose the game.\\ \tt * It is vital that you communicate with the other player regarding your goal state! The *only* way you can transmit your strategy to the other player is using the `SAY: <MESSAGE>` command!\\ \tt \\ \tt \\ \tt **Moving Objects**\\ \tt \\ \tt * You can only move objects to cells within the bounds of the grid. The target cell must be empty, i.e., it must only contain the symbol '◌'.\\ \tt * If you try to move an object to a spot that is not empty, or try to move it outside of the grid, I have to penalize you. You get another try.\\ \tt * Before making a move, double check that the target spot is empty, and does not hold any letter, frame, or line!\\ \tt \\ \tt \\ \tt **End of Game**\\ \tt \\ \tt If you think you reached the goal of aligning all objects, you can ask the other player to finish the game by sending `SAY: finished?`. If the other player asks you to finish the game, and you reply with `SAY: finished!`, the game will end.\\ \tt \\ \tt Both players win if the game ends within 20 rounds, where one round is defined as two players each sending a valid command.\\ \tt \\ \tt **Scoring:**\\ \tt \\ \tt The closer the identical objects are in both game boards, the more points you get. Penalties reduce your points. Can you beat the record?\\ \tt \\ \tt The other player started the game by sending this message:\\ \tt \\ \tt "Hi! I think the grid is 4×7 (x=1..4, y=1..7). My letters are at: T(1,1), W(1,3), I(4,2), C(2,6), H(4,3). Proposal: we both move to these target coords if they’re empty on your board; if any are blocked, please tell me which ones and also send your current coords for T,W,I,C,H so we can pick alternatives."\\ \tt \\ \tt What is your first command? 
	  } 
	   } 
	   } 
	 & \\ 
 

    \theutterance \stepcounter{utterance}  

    & & & \multicolumn{4}{p{0.6\linewidth}}{\cellcolor[rgb]{1,0.85,0.72}{%
	\makecell[{{p{\linewidth}}}]{% 
	  \tt {\tiny [GM$\langle$B]}  
	 SAY: Agreed on 4×7. My coords are T(3,7), W(2,3), I(3,5), C(2,4), H(2,6). Your proposed targets check out as empty on my board except C(2,6), which is blocked by my H: T(1,1)=free, W(1,3)=free, I(4,2)=free, H(4,3)=free, C(2,6)=blocked. I’ll start moving T,W,I,H to those four. For C, can we use (2,4) (where my C already is)? Please confirm if (2,4) is free on your side or suggest an alternative. 
	  } 
	   } 
	   } 
	  \\ 
 

    \theutterance \stepcounter{utterance}  

    & & & \multicolumn{2}{p{0.3\linewidth}}{\cellcolor[rgb]{0.95,0.95,0.95}{%
	\makecell[{{p{\linewidth}}}]{% 
	  \tt {\tiny [GM$|$GM]}  
	 SAY: Agreed on 4×7. My coords are T(3,7), W(2,3), I(3,5), C(2,4), H(2,6). Your proposed targets check out as empty on my board except C(2,6), which is blocked by my H: T(1,1)=free, W(1,3)=free, I(4,2)=free, H(4,3)=free, C(2,6)=blocked. I’ll start moving T,W,I,H to those four. For C, can we use (2,4) (where my C already is)? Please confirm if (2,4) is free on your side or suggest an alternative. 
	  } 
	   } 
	   } 
	 & & \\ 
 

    \theutterance \stepcounter{utterance}  

    & & \multicolumn{4}{p{0.6\linewidth}}{\cellcolor[rgb]{0.9,0.9,0.9}{%
	\makecell[{{p{\linewidth}}}]{% 
	  \tt {\tiny [A$\langle$GM]}  
	 Your message has been relayed to the other player.\\ \tt \\ \tt \\ \tt You are currently playing round 1 of maximum 20.\\ \tt You have collectively accumulated 0 of 12 penalties.\\ \tt The other player sent this message:\\ \tt \\ \tt \\ \tt "Agreed on 4×7. My coords are T(3,7), W(2,3), I(3,5), C(2,4), H(2,6). Your proposed targets check out as empty on my board except C(2,6), which is blocked by my H: T(1,1)=free, W(1,3)=free, I(4,2)=free, H(4,3)=free, C(2,6)=blocked. I’ll start moving T,W,I,H to those four. For C, can we use (2,4) (where my C already is)? Please confirm if (2,4) is free on your side or suggest an alternative."\\ \tt \\ \tt \\ \tt What is your next command? 
	  } 
	   } 
	   } 
	 & \\ 
 

    \theutterance \stepcounter{utterance}  

    & \multicolumn{4}{p{0.6\linewidth}}{\cellcolor[rgb]{0.8,1,0.9}{%
	\makecell[{{p{\linewidth}}}]{% 
	  \tt {\tiny [A$\rangle$GM]}  
	 MOVE: C, (2, 4) 
	  } 
	   } 
	   } 
	 & & \\ 
 

    \theutterance \stepcounter{utterance}  

    & & & \multicolumn{2}{p{0.3\linewidth}}{\cellcolor[rgb]{0.95,0.95,0.95}{%
	\makecell[{{p{\linewidth}}}]{% 
	  \tt {\tiny [GM$|$GM]}  
	 MOVE: C, (2, 4) 
	  } 
	   } 
	   } 
	 & & \\ 
 

    \theutterance \stepcounter{utterance}  

    & & & \multicolumn{2}{p{0.3\linewidth}}{\cellcolor[rgb]{0.95,0.95,0.95}{%
	\makecell[{{p{\linewidth}}}]{% 
	  \tt {\tiny [GM$|$GM]}  
	 Penalty: (2,4) is not empty, but contains '│'.\\ \tt \\ \tt You have collectively accumulated 1 of 12 penalties.\\ \tt Please try again! 
	  } 
	   } 
	   } 
	 & & \\ 
 

    \theutterance \stepcounter{utterance}  

    & & \multicolumn{4}{p{0.6\linewidth}}{\cellcolor[rgb]{0.9,0.9,0.9}{%
	\makecell[{{p{\linewidth}}}]{% 
	  \tt {\tiny [A$\langle$GM]}  
	 Penalty: (2,4) is not empty, but contains '│'.\\ \tt \\ \tt You have collectively accumulated 1 of 12 penalties.\\ \tt Please try again! 
	  } 
	   } 
	   } 
	 & \\ 
 

    \theutterance \stepcounter{utterance}  

    & \multicolumn{4}{p{0.6\linewidth}}{\cellcolor[rgb]{0.8,1,0.9}{%
	\makecell[{{p{\linewidth}}}]{% 
	  \tt {\tiny [A$\rangle$GM]}  
	 SAY: (2,4) is blocked by a wall on my board. Let’s set C to (2,6) instead (that’s my C’s current spot). Since you’re moving H to (4,3), (2,6) should be free for your C. I won’t move anything—my T,W,I,H already match (1,1),(1,3),(4,2),(4,3). After moving your four plus C to (2,6), please say “finished?” 
	  } 
	   } 
	   } 
	 & & \\ 
 

    \theutterance \stepcounter{utterance}  

    & & & \multicolumn{2}{p{0.3\linewidth}}{\cellcolor[rgb]{0.95,0.95,0.95}{%
	\makecell[{{p{\linewidth}}}]{% 
	  \tt {\tiny [GM$|$GM]}  
	 SAY: (2,4) is blocked by a wall on my board. Let’s set C to (2,6) instead (that’s my C’s current spot). Since you’re moving H to (4,3), (2,6) should be free for your C. I won’t move anything—my T,W,I,H already match (1,1),(1,3),(4,2),(4,3). After moving your four plus C to (2,6), please say “finished?” 
	  } 
	   } 
	   } 
	 & & \\ 
 

    \theutterance \stepcounter{utterance}  

    & & \multicolumn{4}{p{0.6\linewidth}}{\cellcolor[rgb]{0.9,0.9,0.9}{%
	\makecell[{{p{\linewidth}}}]{% 
	  \tt {\tiny [GM$\rangle$B]}  
	 Your message has been relayed to the other player.\\ \tt \\ \tt \\ \tt You are currently playing round 2 of maximum 20.\\ \tt You have collectively accumulated 1 of 12 penalties.\\ \tt The other player sent this message:\\ \tt \\ \tt \\ \tt "(2,4) is blocked by a wall on my board. Let’s set C to (2,6) instead (that’s my C’s current spot). Since you’re moving H to (4,3), (2,6) should be free for your C. I won’t move anything—my T,W,I,H already match (1,1),(1,3),(4,2),(4,3). After moving your four plus C to (2,6), please say “finished?”"\\ \tt \\ \tt \\ \tt What is your next command? 
	  } 
	   } 
	   } 
	 & \\ 
 

    \theutterance \stepcounter{utterance}  

    & & & \multicolumn{4}{p{0.6\linewidth}}{\cellcolor[rgb]{1,0.85,0.72}{%
	\makecell[{{p{\linewidth}}}]{% 
	  \tt {\tiny [GM$\langle$B]}  
	 MOVE: H, (4,3) 
	  } 
	   } 
	   } 
	  \\ 
 

    \theutterance \stepcounter{utterance}  

    & & & \multicolumn{2}{p{0.3\linewidth}}{\cellcolor[rgb]{0.95,0.95,0.95}{%
	\makecell[{{p{\linewidth}}}]{% 
	  \tt {\tiny [GM$|$GM]}  
	 MOVE: H, (4,3) 
	  } 
	   } 
	   } 
	 & & \\ 
 

    \theutterance \stepcounter{utterance}  

    & & & \multicolumn{2}{p{0.3\linewidth}}{\cellcolor[rgb]{0.95,0.95,0.95}{%
	\makecell[{{p{\linewidth}}}]{% 
	  \tt {\tiny [GM$|$GM]}  
	 Penalty: (4,3) is not empty, but contains '│'.\\ \tt \\ \tt You have collectively accumulated 2 of 12 penalties.\\ \tt Please try again! 
	  } 
	   } 
	   } 
	 & & \\ 
 

    \theutterance \stepcounter{utterance}  

    & & \multicolumn{4}{p{0.6\linewidth}}{\cellcolor[rgb]{0.9,0.9,0.9}{%
	\makecell[{{p{\linewidth}}}]{% 
	  \tt {\tiny [GM$\rangle$B]}  
	 Penalty: (4,3) is not empty, but contains '│'.\\ \tt \\ \tt You have collectively accumulated 2 of 12 penalties.\\ \tt Please try again! 
	  } 
	   } 
	   } 
	 & \\ 
 

    \theutterance \stepcounter{utterance}  

    & & & \multicolumn{4}{p{0.6\linewidth}}{\cellcolor[rgb]{1,0.85,0.72}{%
	\makecell[{{p{\linewidth}}}]{% 
	  \tt {\tiny [GM$\langle$B]}  
	 MOVE: H, (7,3) 
	  } 
	   } 
	   } 
	  \\ 
 

    \theutterance \stepcounter{utterance}  

    & & & \multicolumn{2}{p{0.3\linewidth}}{\cellcolor[rgb]{0.95,0.95,0.95}{%
	\makecell[{{p{\linewidth}}}]{% 
	  \tt {\tiny [GM$|$GM]}  
	 MOVE: H, (7,3) 
	  } 
	   } 
	   } 
	 & & \\ 
 

    \theutterance \stepcounter{utterance}  

    & & & \multicolumn{2}{p{0.3\linewidth}}{\cellcolor[rgb]{0.95,0.95,0.95}{%
	\makecell[{{p{\linewidth}}}]{% 
	  \tt {\tiny [GM$|$GM]}  
	 Moved 'H' to (7,3) successfully. Your updated grid looks like this:\\ \tt ```\\ \tt  1234567\\ \tt ╔═╤═╤═╤═╗\\ \tt ║◌│◌│◌│◌║ 1\\ \tt ╟─┼─┤◌│◌║ 2\\ \tt ║◌│W│◌│H║ 3\\ \tt ║◌│C├─┴─╢ 4\\ \tt ╟─┴─┤I◌◌║ 5\\ \tt ║◌◌◌└─┬─╢ 6\\ \tt ║◌◌T◌◌│◌║ 7\\ \tt ╚═════╧═╝\\ \tt ```\\ \tt  
	  } 
	   } 
	   } 
	 & & \\ 
 

    \theutterance \stepcounter{utterance}  

    & & \multicolumn{4}{p{0.6\linewidth}}{\cellcolor[rgb]{0.9,0.9,0.9}{%
	\makecell[{{p{\linewidth}}}]{% 
	  \tt {\tiny [A$\langle$GM]}  
	 Your message has been relayed to the other player.\\ \tt \\ \tt \\ \tt You are currently playing round 2 of maximum 20.\\ \tt You have collectively accumulated 2 of 12 penalties.\\ \tt The other player moved an object on their grid.\\ \tt \\ \tt \\ \tt What is your next command? 
	  } 
	   } 
	   } 
	 & \\ 
 

    \theutterance \stepcounter{utterance}  

    & \multicolumn{4}{p{0.6\linewidth}}{\cellcolor[rgb]{0.8,1,0.9}{%
	\makecell[{{p{\linewidth}}}]{% 
	  \tt {\tiny [A$\rangle$GM]}  
	 SAY: Status check: common targets are T(1,1), W(1,3), I(4,2), H(4,3), C(2,6). I’m already at these and won’t move. Please move yours to match; if all five are placed, say “finished?”. If any is blocked on your side, tell me which one and what blocks it, and we’ll pick an alternative. 
	  } 
	   } 
	   } 
	 & & \\ 
 

    \theutterance \stepcounter{utterance}  

    & & & \multicolumn{2}{p{0.3\linewidth}}{\cellcolor[rgb]{0.95,0.95,0.95}{%
	\makecell[{{p{\linewidth}}}]{% 
	  \tt {\tiny [GM$|$GM]}  
	 SAY: Status check: common targets are T(1,1), W(1,3), I(4,2), H(4,3), C(2,6). I’m already at these and won’t move. Please move yours to match; if all five are placed, say “finished?”. If any is blocked on your side, tell me which one and what blocks it, and we’ll pick an alternative. 
	  } 
	   } 
	   } 
	 & & \\ 
 

    \theutterance \stepcounter{utterance}  

    & & \multicolumn{4}{p{0.6\linewidth}}{\cellcolor[rgb]{0.9,0.9,0.9}{%
	\makecell[{{p{\linewidth}}}]{% 
	  \tt {\tiny [GM$\rangle$B]}  
	 Moved 'H' to (7,3) successfully. Your updated grid looks like this:\\ \tt ```\\ \tt  1234567\\ \tt ╔═╤═╤═╤═╗\\ \tt ║◌│◌│◌│◌║ 1\\ \tt ╟─┼─┤◌│◌║ 2\\ \tt ║◌│W│◌│H║ 3\\ \tt ║◌│C├─┴─╢ 4\\ \tt ╟─┴─┤I◌◌║ 5\\ \tt ║◌◌◌└─┬─╢ 6\\ \tt ║◌◌T◌◌│◌║ 7\\ \tt ╚═════╧═╝\\ \tt ```\\ \tt You are currently playing round 3 of maximum 20.\\ \tt You have collectively accumulated 2 of 12 penalties.\\ \tt The other player sent this message:\\ \tt \\ \tt \\ \tt "Status check: common targets are T(1,1), W(1,3), I(4,2), H(4,3), C(2,6). I’m already at these and won’t move. Please move yours to match; if all five are placed, say “finished?”. If any is blocked on your side, tell me which one and what blocks it, and we’ll pick an alternative."\\ \tt \\ \tt \\ \tt What is your next command? 
	  } 
	   } 
	   } 
	 & \\ 
 

    \theutterance \stepcounter{utterance}  

    & & & \multicolumn{4}{p{0.6\linewidth}}{\cellcolor[rgb]{1,0.85,0.72}{%
	\makecell[{{p{\linewidth}}}]{% 
	  \tt {\tiny [GM$\langle$B]}  
	 MOVE: W, (1,3) 
	  } 
	   } 
	   } 
	  \\ 
 

    \theutterance \stepcounter{utterance}  

    & & & \multicolumn{2}{p{0.3\linewidth}}{\cellcolor[rgb]{0.95,0.95,0.95}{%
	\makecell[{{p{\linewidth}}}]{% 
	  \tt {\tiny [GM$|$GM]}  
	 MOVE: W, (1,3) 
	  } 
	   } 
	   } 
	 & & \\ 
 

    \theutterance \stepcounter{utterance}  

    & & & \multicolumn{2}{p{0.3\linewidth}}{\cellcolor[rgb]{0.95,0.95,0.95}{%
	\makecell[{{p{\linewidth}}}]{% 
	  \tt {\tiny [GM$|$GM]}  
	 Moved 'W' to (1,3) successfully. Your updated grid looks like this:\\ \tt ```\\ \tt  1234567\\ \tt ╔═╤═╤═╤═╗\\ \tt ║◌│◌│◌│◌║ 1\\ \tt ╟─┼─┤◌│◌║ 2\\ \tt ║W│◌│◌│H║ 3\\ \tt ║◌│C├─┴─╢ 4\\ \tt ╟─┴─┤I◌◌║ 5\\ \tt ║◌◌◌└─┬─╢ 6\\ \tt ║◌◌T◌◌│◌║ 7\\ \tt ╚═════╧═╝\\ \tt ```\\ \tt  
	  } 
	   } 
	   } 
	 & & \\ 
 

    \theutterance \stepcounter{utterance}  

    & & \multicolumn{4}{p{0.6\linewidth}}{\cellcolor[rgb]{0.9,0.9,0.9}{%
	\makecell[{{p{\linewidth}}}]{% 
	  \tt {\tiny [A$\langle$GM]}  
	 Your message has been relayed to the other player.\\ \tt \\ \tt \\ \tt You are currently playing round 3 of maximum 20.\\ \tt You have collectively accumulated 2 of 12 penalties.\\ \tt The other player moved an object on their grid.\\ \tt \\ \tt \\ \tt What is your next command? 
	  } 
	   } 
	   } 
	 & \\ 
 

    \theutterance \stepcounter{utterance}  

    & \multicolumn{4}{p{0.6\linewidth}}{\cellcolor[rgb]{0.8,1,0.9}{%
	\makecell[{{p{\linewidth}}}]{% 
	  \tt {\tiny [A$\rangle$GM]}  
	 SAY: How’s it going? Final targets are T(1,1), W(1,3), I(4,2), H(4,3), C(2,6). I’m set and won’t move. Please continue moving yours; if any target is blocked on your board, tell me which one and what blocks it. Otherwise say “finished?” when all five match. 
	  } 
	   } 
	   } 
	 & & \\ 
 

    \theutterance \stepcounter{utterance}  

    & & & \multicolumn{2}{p{0.3\linewidth}}{\cellcolor[rgb]{0.95,0.95,0.95}{%
	\makecell[{{p{\linewidth}}}]{% 
	  \tt {\tiny [GM$|$GM]}  
	 SAY: How’s it going? Final targets are T(1,1), W(1,3), I(4,2), H(4,3), C(2,6). I’m set and won’t move. Please continue moving yours; if any target is blocked on your board, tell me which one and what blocks it. Otherwise say “finished?” when all five match. 
	  } 
	   } 
	   } 
	 & & \\ 
 

    \theutterance \stepcounter{utterance}  

    & & \multicolumn{4}{p{0.6\linewidth}}{\cellcolor[rgb]{0.9,0.9,0.9}{%
	\makecell[{{p{\linewidth}}}]{% 
	  \tt {\tiny [GM$\rangle$B]}  
	 Moved 'W' to (1,3) successfully. Your updated grid looks like this:\\ \tt ```\\ \tt  1234567\\ \tt ╔═╤═╤═╤═╗\\ \tt ║◌│◌│◌│◌║ 1\\ \tt ╟─┼─┤◌│◌║ 2\\ \tt ║W│◌│◌│H║ 3\\ \tt ║◌│C├─┴─╢ 4\\ \tt ╟─┴─┤I◌◌║ 5\\ \tt ║◌◌◌└─┬─╢ 6\\ \tt ║◌◌T◌◌│◌║ 7\\ \tt ╚═════╧═╝\\ \tt ```\\ \tt You are currently playing round 4 of maximum 20.\\ \tt You have collectively accumulated 2 of 12 penalties.\\ \tt The other player sent this message:\\ \tt \\ \tt \\ \tt "How’s it going? Final targets are T(1,1), W(1,3), I(4,2), H(4,3), C(2,6). I’m set and won’t move. Please continue moving yours; if any target is blocked on your board, tell me which one and what blocks it. Otherwise say “finished?” when all five match."\\ \tt \\ \tt \\ \tt What is your next command? 
	  } 
	   } 
	   } 
	 & \\ 
 

    \theutterance \stepcounter{utterance}  

    & & & \multicolumn{4}{p{0.6\linewidth}}{\cellcolor[rgb]{1,0.85,0.72}{%
	\makecell[{{p{\linewidth}}}]{% 
	  \tt {\tiny [GM$\langle$B]}  
	 SAY: Update: (4,2) is a '┤' wall and (4,3) is a '│' wall on my board, so I and H can’t go there. I’ve placed W(1,3); current others are T(3,7), I(5,5), C(3,4), H(7,3). I’ll move T→(1,1) and C→(2,6) next. For H and I, propose new common targets: H→(3,3), I→(3,6). If either is blocked on your side, backups: H→(1,4); I→(1,6). Please confirm which are free and adjust; I’ll proceed with T and C meanwhile. 
	  } 
	   } 
	   } 
	  \\ 
 

    \theutterance \stepcounter{utterance}  

    & & & \multicolumn{2}{p{0.3\linewidth}}{\cellcolor[rgb]{0.95,0.95,0.95}{%
	\makecell[{{p{\linewidth}}}]{% 
	  \tt {\tiny [GM$|$GM]}  
	 SAY: Update: (4,2) is a '┤' wall and (4,3) is a '│' wall on my board, so I and H can’t go there. I’ve placed W(1,3); current others are T(3,7), I(5,5), C(3,4), H(7,3). I’ll move T→(1,1) and C→(2,6) next. For H and I, propose new common targets: H→(3,3), I→(3,6). If either is blocked on your side, backups: H→(1,4); I→(1,6). Please confirm which are free and adjust; I’ll proceed with T and C meanwhile. 
	  } 
	   } 
	   } 
	 & & \\ 
 

    \theutterance \stepcounter{utterance}  

    & & \multicolumn{4}{p{0.6\linewidth}}{\cellcolor[rgb]{0.9,0.9,0.9}{%
	\makecell[{{p{\linewidth}}}]{% 
	  \tt {\tiny [A$\langle$GM]}  
	 Your message has been relayed to the other player.\\ \tt \\ \tt \\ \tt You are currently playing round 4 of maximum 20.\\ \tt You have collectively accumulated 2 of 12 penalties.\\ \tt The other player sent this message:\\ \tt \\ \tt \\ \tt "Update: (4,2) is a '┤' wall and (4,3) is a '│' wall on my board, so I and H can’t go there. I’ve placed W(1,3); current others are T(3,7), I(5,5), C(3,4), H(7,3). I’ll move T→(1,1) and C→(2,6) next. For H and I, propose new common targets: H→(3,3), I→(3,6). If either is blocked on your side, backups: H→(1,4); I→(1,6). Please confirm which are free and adjust; I’ll proceed with T and C meanwhile."\\ \tt \\ \tt \\ \tt What is your next command? 
	  } 
	   } 
	   } 
	 & \\ 
 

    \theutterance \stepcounter{utterance}  

    & \multicolumn{4}{p{0.6\linewidth}}{\cellcolor[rgb]{0.8,1,0.9}{%
	\makecell[{{p{\linewidth}}}]{% 
	  \tt {\tiny [A$\rangle$GM]}  
	 SAY: Confirmed: H(3,3) and I(3,6) are empty on my board. Backups H(1,4) and I(1,6) are also empty if needed. Let’s lock final targets as T(1,1), W(1,3), C(2,6), H(3,3), I(3,6). I’ll move H→(3,3) and then I→(3,6) on my next turns. Please finish placing yours to match and then say “finished?”. 
	  } 
	   } 
	   } 
	 & & \\ 
 

    \theutterance \stepcounter{utterance}  

    & & & \multicolumn{2}{p{0.3\linewidth}}{\cellcolor[rgb]{0.95,0.95,0.95}{%
	\makecell[{{p{\linewidth}}}]{% 
	  \tt {\tiny [GM$|$GM]}  
	 SAY: Confirmed: H(3,3) and I(3,6) are empty on my board. Backups H(1,4) and I(1,6) are also empty if needed. Let’s lock final targets as T(1,1), W(1,3), C(2,6), H(3,3), I(3,6). I’ll move H→(3,3) and then I→(3,6) on my next turns. Please finish placing yours to match and then say “finished?”. 
	  } 
	   } 
	   } 
	 & & \\ 
 

    \theutterance \stepcounter{utterance}  

    & & \multicolumn{4}{p{0.6\linewidth}}{\cellcolor[rgb]{0.9,0.9,0.9}{%
	\makecell[{{p{\linewidth}}}]{% 
	  \tt {\tiny [GM$\rangle$B]}  
	 Your message has been relayed to the other player.\\ \tt \\ \tt \\ \tt You are currently playing round 5 of maximum 20.\\ \tt You have collectively accumulated 2 of 12 penalties.\\ \tt The other player sent this message:\\ \tt \\ \tt \\ \tt "Confirmed: H(3,3) and I(3,6) are empty on my board. Backups H(1,4) and I(1,6) are also empty if needed. Let’s lock final targets as T(1,1), W(1,3), C(2,6), H(3,3), I(3,6). I’ll move H→(3,3) and then I→(3,6) on my next turns. Please finish placing yours to match and then say “finished?”."\\ \tt \\ \tt \\ \tt What is your next command? 
	  } 
	   } 
	   } 
	 & \\ 
 

    \theutterance \stepcounter{utterance}  

    & & & \multicolumn{4}{p{0.6\linewidth}}{\cellcolor[rgb]{1,0.85,0.72}{%
	\makecell[{{p{\linewidth}}}]{% 
	  \tt {\tiny [GM$\langle$B]}  
	 MOVE: C, (2,6) 
	  } 
	   } 
	   } 
	  \\ 
 

    \theutterance \stepcounter{utterance}  

    & & & \multicolumn{2}{p{0.3\linewidth}}{\cellcolor[rgb]{0.95,0.95,0.95}{%
	\makecell[{{p{\linewidth}}}]{% 
	  \tt {\tiny [GM$|$GM]}  
	 MOVE: C, (2,6) 
	  } 
	   } 
	   } 
	 & & \\ 
 

    \theutterance \stepcounter{utterance}  

    & & & \multicolumn{2}{p{0.3\linewidth}}{\cellcolor[rgb]{0.95,0.95,0.95}{%
	\makecell[{{p{\linewidth}}}]{% 
	  \tt {\tiny [GM$|$GM]}  
	 Moved 'C' to (2,6) successfully. Your updated grid looks like this:\\ \tt ```\\ \tt  1234567\\ \tt ╔═╤═╤═╤═╗\\ \tt ║◌│◌│◌│◌║ 1\\ \tt ╟─┼─┤◌│◌║ 2\\ \tt ║W│◌│◌│H║ 3\\ \tt ║◌│◌├─┴─╢ 4\\ \tt ╟─┴─┤I◌◌║ 5\\ \tt ║◌C◌└─┬─╢ 6\\ \tt ║◌◌T◌◌│◌║ 7\\ \tt ╚═════╧═╝\\ \tt ```\\ \tt  
	  } 
	   } 
	   } 
	 & & \\ 
 

    \theutterance \stepcounter{utterance}  

    & & \multicolumn{4}{p{0.6\linewidth}}{\cellcolor[rgb]{0.9,0.9,0.9}{%
	\makecell[{{p{\linewidth}}}]{% 
	  \tt {\tiny [A$\langle$GM]}  
	 Your message has been relayed to the other player.\\ \tt \\ \tt \\ \tt You are currently playing round 5 of maximum 20.\\ \tt You have collectively accumulated 2 of 12 penalties.\\ \tt The other player moved an object on their grid.\\ \tt \\ \tt \\ \tt What is your next command? 
	  } 
	   } 
	   } 
	 & \\ 
 

    \theutterance \stepcounter{utterance}  

    & \multicolumn{4}{p{0.6\linewidth}}{\cellcolor[rgb]{0.8,1,0.9}{%
	\makecell[{{p{\linewidth}}}]{% 
	  \tt {\tiny [A$\rangle$GM]}  
	 MOVE: H, (3, 3) 
	  } 
	   } 
	   } 
	 & & \\ 
 

    \theutterance \stepcounter{utterance}  

    & & & \multicolumn{2}{p{0.3\linewidth}}{\cellcolor[rgb]{0.95,0.95,0.95}{%
	\makecell[{{p{\linewidth}}}]{% 
	  \tt {\tiny [GM$|$GM]}  
	 MOVE: H, (3, 3) 
	  } 
	   } 
	   } 
	 & & \\ 
 

    \theutterance \stepcounter{utterance}  

    & & & \multicolumn{2}{p{0.3\linewidth}}{\cellcolor[rgb]{0.95,0.95,0.95}{%
	\makecell[{{p{\linewidth}}}]{% 
	  \tt {\tiny [GM$|$GM]}  
	 Moved 'H' to (3,3) successfully. Your updated grid looks like this:\\ \tt ```\\ \tt  1234567\\ \tt ╔═╤═╤═╤═╗\\ \tt ║T│◌│◌│◌║ 1\\ \tt ╟─┼─┤I│◌║ 2\\ \tt ║W│H│◌│◌║ 3\\ \tt ║◌│◌├─┴─╢ 4\\ \tt ╟─┴─┤◌◌◌║ 5\\ \tt ║◌C◌└─┬─╢ 6\\ \tt ║◌◌◌◌◌│◌║ 7\\ \tt ╚═════╧═╝\\ \tt ```\\ \tt  
	  } 
	   } 
	   } 
	 & & \\ 
 

    \theutterance \stepcounter{utterance}  

    & & \multicolumn{4}{p{0.6\linewidth}}{\cellcolor[rgb]{0.9,0.9,0.9}{%
	\makecell[{{p{\linewidth}}}]{% 
	  \tt {\tiny [GM$\rangle$B]}  
	 Moved 'C' to (2,6) successfully. Your updated grid looks like this:\\ \tt ```\\ \tt  1234567\\ \tt ╔═╤═╤═╤═╗\\ \tt ║◌│◌│◌│◌║ 1\\ \tt ╟─┼─┤◌│◌║ 2\\ \tt ║W│◌│◌│H║ 3\\ \tt ║◌│◌├─┴─╢ 4\\ \tt ╟─┴─┤I◌◌║ 5\\ \tt ║◌C◌└─┬─╢ 6\\ \tt ║◌◌T◌◌│◌║ 7\\ \tt ╚═════╧═╝\\ \tt ```\\ \tt You are currently playing round 6 of maximum 20.\\ \tt You have collectively accumulated 2 of 12 penalties.\\ \tt The other player moved an object on their grid.\\ \tt \\ \tt \\ \tt What is your next command? 
	  } 
	   } 
	   } 
	 & \\ 
 

    \theutterance \stepcounter{utterance}  

    & & & \multicolumn{4}{p{0.6\linewidth}}{\cellcolor[rgb]{1,0.85,0.72}{%
	\makecell[{{p{\linewidth}}}]{% 
	  \tt {\tiny [GM$\langle$B]}  
	 MOVE: T, (1,1) 
	  } 
	   } 
	   } 
	  \\ 
 

    \theutterance \stepcounter{utterance}  

    & & & \multicolumn{2}{p{0.3\linewidth}}{\cellcolor[rgb]{0.95,0.95,0.95}{%
	\makecell[{{p{\linewidth}}}]{% 
	  \tt {\tiny [GM$|$GM]}  
	 MOVE: T, (1,1) 
	  } 
	   } 
	   } 
	 & & \\ 
 

    \theutterance \stepcounter{utterance}  

    & & & \multicolumn{2}{p{0.3\linewidth}}{\cellcolor[rgb]{0.95,0.95,0.95}{%
	\makecell[{{p{\linewidth}}}]{% 
	  \tt {\tiny [GM$|$GM]}  
	 Moved 'T' to (1,1) successfully. Your updated grid looks like this:\\ \tt ```\\ \tt  1234567\\ \tt ╔═╤═╤═╤═╗\\ \tt ║T│◌│◌│◌║ 1\\ \tt ╟─┼─┤◌│◌║ 2\\ \tt ║W│◌│◌│H║ 3\\ \tt ║◌│◌├─┴─╢ 4\\ \tt ╟─┴─┤I◌◌║ 5\\ \tt ║◌C◌└─┬─╢ 6\\ \tt ║◌◌◌◌◌│◌║ 7\\ \tt ╚═════╧═╝\\ \tt ```\\ \tt  
	  } 
	   } 
	   } 
	 & & \\ 
 

    \theutterance \stepcounter{utterance}  

    & & \multicolumn{4}{p{0.6\linewidth}}{\cellcolor[rgb]{0.9,0.9,0.9}{%
	\makecell[{{p{\linewidth}}}]{% 
	  \tt {\tiny [A$\langle$GM]}  
	 Moved 'H' to (3,3) successfully. Your updated grid looks like this:\\ \tt ```\\ \tt  1234567\\ \tt ╔═╤═╤═╤═╗\\ \tt ║T│◌│◌│◌║ 1\\ \tt ╟─┼─┤I│◌║ 2\\ \tt ║W│H│◌│◌║ 3\\ \tt ║◌│◌├─┴─╢ 4\\ \tt ╟─┴─┤◌◌◌║ 5\\ \tt ║◌C◌└─┬─╢ 6\\ \tt ║◌◌◌◌◌│◌║ 7\\ \tt ╚═════╧═╝\\ \tt ```\\ \tt You are currently playing round 6 of maximum 20.\\ \tt You have collectively accumulated 2 of 12 penalties.\\ \tt The other player moved an object on their grid.\\ \tt \\ \tt \\ \tt What is your next command? 
	  } 
	   } 
	   } 
	 & \\ 
 

    \theutterance \stepcounter{utterance}  

    & \multicolumn{4}{p{0.6\linewidth}}{\cellcolor[rgb]{0.8,1,0.9}{%
	\makecell[{{p{\linewidth}}}]{% 
	  \tt {\tiny [A$\rangle$GM]}  
	 MOVE: I, (3, 6) 
	  } 
	   } 
	   } 
	 & & \\ 
 

    \theutterance \stepcounter{utterance}  

    & & & \multicolumn{2}{p{0.3\linewidth}}{\cellcolor[rgb]{0.95,0.95,0.95}{%
	\makecell[{{p{\linewidth}}}]{% 
	  \tt {\tiny [GM$|$GM]}  
	 MOVE: I, (3, 6) 
	  } 
	   } 
	   } 
	 & & \\ 
 

    \theutterance \stepcounter{utterance}  

    & & & \multicolumn{2}{p{0.3\linewidth}}{\cellcolor[rgb]{0.95,0.95,0.95}{%
	\makecell[{{p{\linewidth}}}]{% 
	  \tt {\tiny [GM$|$GM]}  
	 Moved 'I' to (3,6) successfully. Your updated grid looks like this:\\ \tt ```\\ \tt  1234567\\ \tt ╔═╤═╤═╤═╗\\ \tt ║T│◌│◌│◌║ 1\\ \tt ╟─┼─┤◌│◌║ 2\\ \tt ║W│H│◌│◌║ 3\\ \tt ║◌│◌├─┴─╢ 4\\ \tt ╟─┴─┤◌◌◌║ 5\\ \tt ║◌CI└─┬─╢ 6\\ \tt ║◌◌◌◌◌│◌║ 7\\ \tt ╚═════╧═╝\\ \tt ```\\ \tt  
	  } 
	   } 
	   } 
	 & & \\ 
 

    \theutterance \stepcounter{utterance}  

    & & \multicolumn{4}{p{0.6\linewidth}}{\cellcolor[rgb]{0.9,0.9,0.9}{%
	\makecell[{{p{\linewidth}}}]{% 
	  \tt {\tiny [GM$\rangle$B]}  
	 Moved 'T' to (1,1) successfully. Your updated grid looks like this:\\ \tt ```\\ \tt  1234567\\ \tt ╔═╤═╤═╤═╗\\ \tt ║T│◌│◌│◌║ 1\\ \tt ╟─┼─┤◌│◌║ 2\\ \tt ║W│◌│◌│H║ 3\\ \tt ║◌│◌├─┴─╢ 4\\ \tt ╟─┴─┤I◌◌║ 5\\ \tt ║◌C◌└─┬─╢ 6\\ \tt ║◌◌◌◌◌│◌║ 7\\ \tt ╚═════╧═╝\\ \tt ```\\ \tt You are currently playing round 7 of maximum 20.\\ \tt You have collectively accumulated 2 of 12 penalties.\\ \tt The other player moved an object on their grid.\\ \tt \\ \tt \\ \tt What is your next command? 
	  } 
	   } 
	   } 
	 & \\ 
 

    \theutterance \stepcounter{utterance}  

    & & & \multicolumn{4}{p{0.6\linewidth}}{\cellcolor[rgb]{1,0.85,0.72}{%
	\makecell[{{p{\linewidth}}}]{% 
	  \tt {\tiny [GM$\langle$B]}  
	 MOVE: H, (3,3) 
	  } 
	   } 
	   } 
	  \\ 
 

    \theutterance \stepcounter{utterance}  

    & & & \multicolumn{2}{p{0.3\linewidth}}{\cellcolor[rgb]{0.95,0.95,0.95}{%
	\makecell[{{p{\linewidth}}}]{% 
	  \tt {\tiny [GM$|$GM]}  
	 MOVE: H, (3,3) 
	  } 
	   } 
	   } 
	 & & \\ 
 

    \theutterance \stepcounter{utterance}  

    & & & \multicolumn{2}{p{0.3\linewidth}}{\cellcolor[rgb]{0.95,0.95,0.95}{%
	\makecell[{{p{\linewidth}}}]{% 
	  \tt {\tiny [GM$|$GM]}  
	 Moved 'H' to (3,3) successfully. Your updated grid looks like this:\\ \tt ```\\ \tt  1234567\\ \tt ╔═╤═╤═╤═╗\\ \tt ║T│◌│◌│◌║ 1\\ \tt ╟─┼─┤◌│◌║ 2\\ \tt ║W│H│◌│◌║ 3\\ \tt ║◌│◌├─┴─╢ 4\\ \tt ╟─┴─┤I◌◌║ 5\\ \tt ║◌C◌└─┬─╢ 6\\ \tt ║◌◌◌◌◌│◌║ 7\\ \tt ╚═════╧═╝\\ \tt ```\\ \tt  
	  } 
	   } 
	   } 
	 & & \\ 
 

    \theutterance \stepcounter{utterance}  

    & & \multicolumn{4}{p{0.6\linewidth}}{\cellcolor[rgb]{0.9,0.9,0.9}{%
	\makecell[{{p{\linewidth}}}]{% 
	  \tt {\tiny [A$\langle$GM]}  
	 Moved 'I' to (3,6) successfully. Your updated grid looks like this:\\ \tt ```\\ \tt  1234567\\ \tt ╔═╤═╤═╤═╗\\ \tt ║T│◌│◌│◌║ 1\\ \tt ╟─┼─┤◌│◌║ 2\\ \tt ║W│H│◌│◌║ 3\\ \tt ║◌│◌├─┴─╢ 4\\ \tt ╟─┴─┤◌◌◌║ 5\\ \tt ║◌CI└─┬─╢ 6\\ \tt ║◌◌◌◌◌│◌║ 7\\ \tt ╚═════╧═╝\\ \tt ```\\ \tt You are currently playing round 7 of maximum 20.\\ \tt You have collectively accumulated 2 of 12 penalties.\\ \tt The other player moved an object on their grid.\\ \tt \\ \tt \\ \tt What is your next command? 
	  } 
	   } 
	   } 
	 & \\ 
 

    \theutterance \stepcounter{utterance}  

    & \multicolumn{4}{p{0.6\linewidth}}{\cellcolor[rgb]{0.8,1,0.9}{%
	\makecell[{{p{\linewidth}}}]{% 
	  \tt {\tiny [A$\rangle$GM]}  
	 SAY: finished? 
	  } 
	   } 
	   } 
	 & & \\ 
 

    \theutterance \stepcounter{utterance}  

    & & & \multicolumn{2}{p{0.3\linewidth}}{\cellcolor[rgb]{0.95,0.95,0.95}{%
	\makecell[{{p{\linewidth}}}]{% 
	  \tt {\tiny [GM$|$GM]}  
	 SAY: finished? 
	  } 
	   } 
	   } 
	 & & \\ 
 

    \theutterance \stepcounter{utterance}  

    & & \multicolumn{4}{p{0.6\linewidth}}{\cellcolor[rgb]{0.9,0.9,0.9}{%
	\makecell[{{p{\linewidth}}}]{% 
	  \tt {\tiny [GM$\rangle$B]}  
	 Moved 'H' to (3,3) successfully. Your updated grid looks like this:\\ \tt ```\\ \tt  1234567\\ \tt ╔═╤═╤═╤═╗\\ \tt ║T│◌│◌│◌║ 1\\ \tt ╟─┼─┤◌│◌║ 2\\ \tt ║W│H│◌│◌║ 3\\ \tt ║◌│◌├─┴─╢ 4\\ \tt ╟─┴─┤I◌◌║ 5\\ \tt ║◌C◌└─┬─╢ 6\\ \tt ║◌◌◌◌◌│◌║ 7\\ \tt ╚═════╧═╝\\ \tt ```\\ \tt You are currently playing round 8 of maximum 20.\\ \tt You have collectively accumulated 2 of 12 penalties.\\ \tt The other player sent this message:\\ \tt \\ \tt \\ \tt "finished?"\\ \tt \\ \tt \\ \tt What is your next command? 
	  } 
	   } 
	   } 
	 & \\ 
 

    \theutterance \stepcounter{utterance}  

    & & & \multicolumn{4}{p{0.6\linewidth}}{\cellcolor[rgb]{1,0.85,0.72}{%
	\makecell[{{p{\linewidth}}}]{% 
	  \tt {\tiny [GM$\langle$B]}  
	 MOVE: I, (3,6) 
	  } 
	   } 
	   } 
	  \\ 
 

    \theutterance \stepcounter{utterance}  

    & & & \multicolumn{2}{p{0.3\linewidth}}{\cellcolor[rgb]{0.95,0.95,0.95}{%
	\makecell[{{p{\linewidth}}}]{% 
	  \tt {\tiny [GM$|$GM]}  
	 MOVE: I, (3,6) 
	  } 
	   } 
	   } 
	 & & \\ 
 

    \theutterance \stepcounter{utterance}  

    & & & \multicolumn{2}{p{0.3\linewidth}}{\cellcolor[rgb]{0.95,0.95,0.95}{%
	\makecell[{{p{\linewidth}}}]{% 
	  \tt {\tiny [GM$|$GM]}  
	 Moved 'I' to (3,6) successfully. Your updated grid looks like this:\\ \tt ```\\ \tt  1234567\\ \tt ╔═╤═╤═╤═╗\\ \tt ║T│◌│◌│◌║ 1\\ \tt ╟─┼─┤◌│◌║ 2\\ \tt ║W│H│◌│◌║ 3\\ \tt ║◌│◌├─┴─╢ 4\\ \tt ╟─┴─┤◌◌◌║ 5\\ \tt ║◌CI└─┬─╢ 6\\ \tt ║◌◌◌◌◌│◌║ 7\\ \tt ╚═════╧═╝\\ \tt ```\\ \tt  
	  } 
	   } 
	   } 
	 & & \\ 
 

    \theutterance \stepcounter{utterance}  

    & & \multicolumn{4}{p{0.6\linewidth}}{\cellcolor[rgb]{0.9,0.9,0.9}{%
	\makecell[{{p{\linewidth}}}]{% 
	  \tt {\tiny [A$\langle$GM]}  
	 Your message has been relayed to the other player.\\ \tt \\ \tt \\ \tt You are currently playing round 8 of maximum 20.\\ \tt You have collectively accumulated 2 of 12 penalties.\\ \tt The other player moved an object on their grid.\\ \tt \\ \tt \\ \tt What is your next command? 
	  } 
	   } 
	   } 
	 & \\ 
 

    \theutterance \stepcounter{utterance}  

    & \multicolumn{4}{p{0.6\linewidth}}{\cellcolor[rgb]{0.8,1,0.9}{%
	\makecell[{{p{\linewidth}}}]{% 
	  \tt {\tiny [A$\rangle$GM]}  
	 SAY: finished? 
	  } 
	   } 
	   } 
	 & & \\ 
 

    \theutterance \stepcounter{utterance}  

    & & & \multicolumn{2}{p{0.3\linewidth}}{\cellcolor[rgb]{0.95,0.95,0.95}{%
	\makecell[{{p{\linewidth}}}]{% 
	  \tt {\tiny [GM$|$GM]}  
	 SAY: finished? 
	  } 
	   } 
	   } 
	 & & \\ 
 

    \theutterance \stepcounter{utterance}  

    & & \multicolumn{4}{p{0.6\linewidth}}{\cellcolor[rgb]{0.9,0.9,0.9}{%
	\makecell[{{p{\linewidth}}}]{% 
	  \tt {\tiny [GM$\rangle$B]}  
	 Moved 'I' to (3,6) successfully. Your updated grid looks like this:\\ \tt ```\\ \tt  1234567\\ \tt ╔═╤═╤═╤═╗\\ \tt ║T│◌│◌│◌║ 1\\ \tt ╟─┼─┤◌│◌║ 2\\ \tt ║W│H│◌│◌║ 3\\ \tt ║◌│◌├─┴─╢ 4\\ \tt ╟─┴─┤◌◌◌║ 5\\ \tt ║◌CI└─┬─╢ 6\\ \tt ║◌◌◌◌◌│◌║ 7\\ \tt ╚═════╧═╝\\ \tt ```\\ \tt You are currently playing round 9 of maximum 20.\\ \tt You have collectively accumulated 2 of 12 penalties.\\ \tt The other player sent this message:\\ \tt \\ \tt \\ \tt "finished?"\\ \tt \\ \tt \\ \tt What is your next command? 
	  } 
	   } 
	   } 
	 & \\ 
 

    \theutterance \stepcounter{utterance}  

    & & & \multicolumn{4}{p{0.6\linewidth}}{\cellcolor[rgb]{1,0.85,0.72}{%
	\makecell[{{p{\linewidth}}}]{% 
	  \tt {\tiny [GM$\langle$B]}  
	 SAY: finished! 
	  } 
	   } 
	   } 
	  \\ 
 

    \theutterance \stepcounter{utterance}  

    & & & \multicolumn{2}{p{0.3\linewidth}}{\cellcolor[rgb]{0.95,0.95,0.95}{%
	\makecell[{{p{\linewidth}}}]{% 
	  \tt {\tiny [GM$|$GM]}  
	 SAY: finished! 
	  } 
	   } 
	   } 
	 & & \\ 
 

    \theutterance \stepcounter{utterance}  

    & & & \multicolumn{2}{p{0.3\linewidth}}{\cellcolor[rgb]{0.95,0.95,0.95}{%
	\makecell[{{p{\linewidth}}}]{% 
	  \tt {\tiny [GM$|$GM]}  
	 true 
	  } 
	   } 
	   } 
	 & & \\ 
 

    \theutterance \stepcounter{utterance}  

    & & & \multicolumn{2}{p{0.3\linewidth}}{\cellcolor[rgb]{0.95,0.95,0.95}{%
	\makecell[{{p{\linewidth}}}]{% 
	  \tt {\tiny [GM$|$GM]}  
	 * success: True\\ \tt * lose: False\\ \tt * aborted: False\\ \tt -------\\ \tt * Shifts: 7.00\\ \tt * Max Shifts: 8.00\\ \tt * Min Shifts: 4.00\\ \tt * End Distance Sum: 0.00\\ \tt * Init Distance Sum: 19.39\\ \tt * Expected Distance Sum: 20.95\\ \tt * Penalties: 2.00\\ \tt * Max Penalties: 12.00\\ \tt * Rounds: 9.00\\ \tt * Max Rounds: 20.00\\ \tt * Object Count: 5.00\\ \tt  
	  } 
	   } 
	   } 
	 & & \\ 
 

    \theutterance \stepcounter{utterance}  

    & & & \multicolumn{2}{p{0.3\linewidth}}{\cellcolor[rgb]{0.95,0.95,0.95}{%
	\makecell[{{p{\linewidth}}}]{% 
	  \tt {\tiny [GM$|$GM]}  
	 * Main Score: 54.55\\ \tt \\ \tt -------\\ \tt * Distance Score: 1.00\\ \tt * Consistency Score: 0.40\\ \tt * Coverage Score: 0.50\\ \tt * Penalty Score: 1.00\\ \tt * Alternative Penalty Score: 1.00\\ \tt * Alternative Main Score: 100.00\\ \tt \\ \tt -------\\ \tt * Shifts: 7.00\\ \tt * Max Shifts: 8.00\\ \tt * Min Shifts: 4.00\\ \tt * End Distance Sum: 0.00\\ \tt * Init Distance Sum: 19.39\\ \tt * Expected Distance Sum: 20.95\\ \tt * Penalties: 2.00\\ \tt * Max Penalties: 12.00\\ \tt * Rounds: 9.00\\ \tt * Max Rounds: 20.00\\ \tt * Object Count: 5.00\\ \tt  
	  } 
	   } 
	   } 
	 & & \\ 
 

\end{supertabular}
}

\end{document}
