
\documentclass{article}
\usepackage{colortbl}
\usepackage{makecell}
\usepackage{multirow}
\usepackage{supertabular}

\begin{document}

\newcounter{utterance}

\twocolumn

{ \footnotesize  \setcounter{utterance}{1}
\setlength{\tabcolsep}{0pt}
\begin{supertabular}{c@{$\;$}|p{.15\linewidth}@{}p{.15\linewidth}p{.15\linewidth}p{.15\linewidth}p{.15\linewidth}p{.15\linewidth}}

    \# & $\;$A & \multicolumn{4}{c}{Game Master} & $\;\:$B\\
    \hline 

    \theutterance \stepcounter{utterance}  

    & & \multicolumn{4}{p{0.6\linewidth}}{\cellcolor[rgb]{0.9,0.9,0.9}{%
	\makecell[{{p{\linewidth}}}]{% 
	  \tt {\tiny [A$\langle$GM]}  
	 I am your game master, and you are playing a collaborative game with the following grid as game board:\\ \tt \\ \tt ```\\ \tt  1234567\\ \tt ╔══╤════╗\\ \tt ║◌T│D◌◌◌║ 1\\ \tt ╟──┘◌◌┌─╢ 2\\ \tt ║SA◌◌◌│◌║ 3\\ \tt ╟─┐◌┌─┼─╢ 4\\ \tt ║◌├─┤◌│O║ 5\\ \tt ╟─┤◌├─┤◌║ 6\\ \tt ║◌│P│M│◌║ 7\\ \tt ╚═╧═╧═╧═╝\\ \tt \\ \tt ```\\ \tt \\ \tt * The upper edge displays x-coordinates increasing to the right, and the right edge y-coordinates increasing downward.\\ \tt * The following objects are randomly placed on your grid: 'P', 'O', 'T', 'S', 'D', 'A', 'M'.\\ \tt \\ \tt \\ \tt The other player sees a variation of the game board, where the objects are placed at different random locations. You cannot see the other player's board, and they cannot see yours.\\ \tt \\ \tt \\ \tt **Goal:**\\ \tt \\ \tt Both players need to move the objects on their respective background so that identical objects end up at the same coordinates. You have to communicate with the other player to agree upon a common goal configuration.\\ \tt \\ \tt \\ \tt **Rules:**\\ \tt \\ \tt * In each turn, you can send exactly one of the following two commands:\\ \tt  1. `SAY: <MESSAGE>`: to send a message (everything up until the next line break) to the other player. I will forward it to your partner.\\ \tt 2. `MOVE: <OBJECT>, (<X>, <Y>)`: to move an object to a new position, where `<X>` is the column and `<Y>` is the row. I will inform you if your move was successful or not.\\ \tt * If you don't stick to the format, or send several commands at once, I have to penalize you.\\ \tt * If both players accumulate more than 16 penalties, you both lose the game.\\ \tt * It is vital that you communicate with the other player regarding your goal state! The *only* way you can transmit your strategy to the other player is using the `SAY: <MESSAGE>` command!\\ \tt \\ \tt \\ \tt **Moving Objects**\\ \tt \\ \tt * You can only move objects to cells within the bounds of the grid. The target cell must be empty, i.e., it must only contain the symbol '◌'.\\ \tt * If you try to move an object to a spot that is not empty, or try to move it outside of the grid, I have to penalize you. You get another try.\\ \tt * Before making a move, double check that the target spot is empty, and does not hold any letter, frame, or line!\\ \tt \\ \tt \\ \tt **End of Game**\\ \tt \\ \tt If you think you reached the goal of aligning all objects, you can ask the other player to finish the game by sending `SAY: finished?`. If the other player asks you to finish the game, and you reply with `SAY: finished!`, the game will end.\\ \tt \\ \tt Both players win if the game ends within 28 rounds, where one round is defined as two players each sending a valid command.\\ \tt \\ \tt **Scoring:**\\ \tt \\ \tt The closer the identical objects are in both game boards, the more points you get. Penalties reduce your points. Can you beat the record?\\ \tt \\ \tt \\ \tt Please send a message to the other player to start the game! 
	  } 
	   } 
	   } 
	 & \\ 
 

    \theutterance \stepcounter{utterance}  

    & \multicolumn{4}{p{0.6\linewidth}}{\cellcolor[rgb]{0.8,1,0.9}{%
	\makecell[{{p{\linewidth}}}]{% 
	  \tt {\tiny [A$\rangle$GM]}  
	 SAY: Hello, let's start by describing our initial boards. I'll describe mine, and then you can describe yours. My board has the following objects: 'P' at (,7), 'O' at (5,5), 'T' at (,1), 'S' at (1,3), 'D' at (2,1), 'A' at (3,3), 'M' at (,7). Please respond with the positions of these objects on your board. 
	  } 
	   } 
	   } 
	 & & \\ 
 

    \theutterance \stepcounter{utterance}  

    & & & \multicolumn{2}{p{0.3\linewidth}}{\cellcolor[rgb]{0.95,0.95,0.95}{%
	\makecell[{{p{\linewidth}}}]{% 
	  \tt {\tiny [GM$|$GM]}  
	 SAY: Hello, let's start by describing our initial boards. I'll describe mine, and then you can describe yours. My board has the following objects: 'P' at (,7), 'O' at (5,5), 'T' at (,1), 'S' at (1,3), 'D' at (2,1), 'A' at (3,3), 'M' at (,7). Please respond with the positions of these objects on your board. 
	  } 
	   } 
	   } 
	 & & \\ 
 

    \theutterance \stepcounter{utterance}  

    & & \multicolumn{4}{p{0.6\linewidth}}{\cellcolor[rgb]{0.9,0.9,0.9}{%
	\makecell[{{p{\linewidth}}}]{% 
	  \tt {\tiny [GM$\rangle$B]}  
	 I am your game master, and you are playing a collaborative game with the following grid as game board:\\ \tt \\ \tt ```\\ \tt  1234567\\ \tt ╔══╤════╗\\ \tt ║◌S│◌◌◌◌║ 1\\ \tt ╟──┘◌◌┌─╢ 2\\ \tt ║A◌O◌◌│◌║ 3\\ \tt ╟─┐◌┌─┼─╢ 4\\ \tt ║◌├─┤T│D║ 5\\ \tt ╟─┤◌├─┤◌║ 6\\ \tt ║P│◌│M│◌║ 7\\ \tt ╚═╧═╧═╧═╝\\ \tt \\ \tt ```\\ \tt \\ \tt * The upper edge displays x-coordinates increasing to the right, and the right edge y-coordinates increasing downward.\\ \tt * The following objects are randomly placed on your grid: 'P', 'O', 'T', 'S', 'D', 'A', 'M'.\\ \tt \\ \tt \\ \tt The other player sees a variation of the game board, where the objects are placed at different random locations. You cannot see the other player's board, and they cannot see yours.\\ \tt \\ \tt \\ \tt **Goal:**\\ \tt \\ \tt Both players need to move the objects on their respective background so that identical objects end up at the same coordinates. You have to communicate with the other player to agree upon a common goal configuration.\\ \tt \\ \tt \\ \tt **Rules:**\\ \tt \\ \tt * In each turn, you can send exactly one of the following two commands:\\ \tt  1. `SAY: <MESSAGE>`: to send a message (everything up until the next line break) to the other player. I will forward it to your partner.\\ \tt 2. `MOVE: <OBJECT>, (<X>, <Y>)`: to move an object to a new position, where `<X>` is the column and `<Y>` is the row. I will inform you if your move was successful or not.\\ \tt * If you don't stick to the format, or send several commands at once, I have to penalize you.\\ \tt * If both players accumulate more than 16 penalties, you both lose the game.\\ \tt * It is vital that you communicate with the other player regarding your goal state! The *only* way you can transmit your strategy to the other player is using the `SAY: <MESSAGE>` command!\\ \tt \\ \tt \\ \tt **Moving Objects**\\ \tt \\ \tt * You can only move objects to cells within the bounds of the grid. The target cell must be empty, i.e., it must only contain the symbol '◌'.\\ \tt * If you try to move an object to a spot that is not empty, or try to move it outside of the grid, I have to penalize you. You get another try.\\ \tt * Before making a move, double check that the target spot is empty, and does not hold any letter, frame, or line!\\ \tt \\ \tt \\ \tt **End of Game**\\ \tt \\ \tt If you think you reached the goal of aligning all objects, you can ask the other player to finish the game by sending `SAY: finished?`. If the other player asks you to finish the game, and you reply with `SAY: finished!`, the game will end.\\ \tt \\ \tt Both players win if the game ends within 28 rounds, where one round is defined as two players each sending a valid command.\\ \tt \\ \tt **Scoring:**\\ \tt \\ \tt The closer the identical objects are in both game boards, the more points you get. Penalties reduce your points. Can you beat the record?\\ \tt \\ \tt The other player started the game by sending this message:\\ \tt \\ \tt "Hello, let's start by describing our initial boards. I'll describe mine, and then you can describe yours. My board has the following objects: 'P' at (,7), 'O' at (5,5), 'T' at (,1), 'S' at (1,3), 'D' at (2,1), 'A' at (3,3), 'M' at (,7). Please respond with the positions of these objects on your board."\\ \tt \\ \tt What is your first command? 
	  } 
	   } 
	   } 
	 & \\ 
 

    \theutterance \stepcounter{utterance}  

    & & & \multicolumn{4}{p{0.6\linewidth}}{\cellcolor[rgb]{1,0.85,0.72}{%
	\makecell[{{p{\linewidth}}}]{% 
	  \tt {\tiny [GM$\langle$B]}  
	 My first command is:\\ \tt \\ \tt `SAY: Hello! On my board, 'P' is at (1,7), 'O' is at (2,3), 'T' is at (4,5), 'S' is at (1,1), 'D' is at (4,5), 'A' is at (1,3), 'M' is at (3,7). Let's discuss how to align them.` 
	  } 
	   } 
	   } 
	  \\ 
 

    \theutterance \stepcounter{utterance}  

    & & & \multicolumn{2}{p{0.3\linewidth}}{\cellcolor[rgb]{0.95,0.95,0.95}{%
	\makecell[{{p{\linewidth}}}]{% 
	  \tt {\tiny [GM$|$GM]}  
	 My first command is:\\ \tt \\ \tt `SAY: Hello! On my board, 'P' is at (1,7), 'O' is at (2,3), 'T' is at (4,5), 'S' is at (1,1), 'D' is at (4,5), 'A' is at (1,3), 'M' is at (3,7). Let's discuss how to align them.` 
	  } 
	   } 
	   } 
	 & & \\ 
 

    \theutterance \stepcounter{utterance}  

    & & & \multicolumn{2}{p{0.3\linewidth}}{\cellcolor[rgb]{0.95,0.95,0.95}{%
	\makecell[{{p{\linewidth}}}]{% 
	  \tt {\tiny [GM$|$GM]}  
	 Invalid format: head is not empty: 'My first command is:\\ \tt \\ \tt ' 
	  } 
	   } 
	   } 
	 & & \\ 
 

    \theutterance \stepcounter{utterance}  

    & & \multicolumn{4}{p{0.6\linewidth}}{\cellcolor[rgb]{0.9,0.9,0.9}{%
	\makecell[{{p{\linewidth}}}]{% 
	  \tt {\tiny [GM$\rangle$B]}  
	 Penalty: Your message must not contain anything before the command!\\ \tt \\ \tt Make sure that your response only contains either `SAY: <MESSAGE>` or `MOVE: <OBJECT>, (<X>, <Y>)`, and nothing else!\\ \tt \\ \tt You have collectively accumulated 1 of 16 penalties.\\ \tt Please try again! 
	  } 
	   } 
	   } 
	 & \\ 
 

    \theutterance \stepcounter{utterance}  

    & & & \multicolumn{4}{p{0.6\linewidth}}{\cellcolor[rgb]{1,0.85,0.72}{%
	\makecell[{{p{\linewidth}}}]{% 
	  \tt {\tiny [GM$\langle$B]}  
	 SAY: On my board, 'P' is at (1,7), 'O' is at (2,3), 'T' is at (4,5), 'S' is at (1,1), 'D' is at (4,5), 'A' is at (1,3), 'M' is at (3,7). Let's discuss how to align them. 
	  } 
	   } 
	   } 
	  \\ 
 

    \theutterance \stepcounter{utterance}  

    & & & \multicolumn{2}{p{0.3\linewidth}}{\cellcolor[rgb]{0.95,0.95,0.95}{%
	\makecell[{{p{\linewidth}}}]{% 
	  \tt {\tiny [GM$|$GM]}  
	 SAY: On my board, 'P' is at (1,7), 'O' is at (2,3), 'T' is at (4,5), 'S' is at (1,1), 'D' is at (4,5), 'A' is at (1,3), 'M' is at (3,7). Let's discuss how to align them. 
	  } 
	   } 
	   } 
	 & & \\ 
 

    \theutterance \stepcounter{utterance}  

    & & \multicolumn{4}{p{0.6\linewidth}}{\cellcolor[rgb]{0.9,0.9,0.9}{%
	\makecell[{{p{\linewidth}}}]{% 
	  \tt {\tiny [A$\langle$GM]}  
	 Your message has been relayed to the other player.\\ \tt \\ \tt \\ \tt You are currently playing round 1 of maximum 28.\\ \tt You have collectively accumulated 1 of 16 penalties.\\ \tt The other player sent this message:\\ \tt \\ \tt \\ \tt "On my board, 'P' is at (1,7), 'O' is at (2,3), 'T' is at (4,5), 'S' is at (1,1), 'D' is at (4,5), 'A' is at (1,3), 'M' is at (3,7). Let's discuss how to align them."\\ \tt \\ \tt \\ \tt What is your next command? 
	  } 
	   } 
	   } 
	 & \\ 
 

    \theutterance \stepcounter{utterance}  

    & \multicolumn{4}{p{0.6\linewidth}}{\cellcolor[rgb]{0.8,1,0.9}{%
	\makecell[{{p{\linewidth}}}]{% 
	  \tt {\tiny [A$\rangle$GM]}  
	 SAY: Thank you for sharing the positions of the objects on your board. It seems like we have some differences. Let's start by trying to align one object at a time. How about we start with 'P'? On my board, 'P' is at (3,7), and on your board, it's at (1,7). Would you be willing to move your 'P' to (3,7) to match mine, or should I move mine to (1,7) to match yours? 
	  } 
	   } 
	   } 
	 & & \\ 
 

    \theutterance \stepcounter{utterance}  

    & & & \multicolumn{2}{p{0.3\linewidth}}{\cellcolor[rgb]{0.95,0.95,0.95}{%
	\makecell[{{p{\linewidth}}}]{% 
	  \tt {\tiny [GM$|$GM]}  
	 SAY: Thank you for sharing the positions of the objects on your board. It seems like we have some differences. Let's start by trying to align one object at a time. How about we start with 'P'? On my board, 'P' is at (3,7), and on your board, it's at (1,7). Would you be willing to move your 'P' to (3,7) to match mine, or should I move mine to (1,7) to match yours? 
	  } 
	   } 
	   } 
	 & & \\ 
 

    \theutterance \stepcounter{utterance}  

    & & \multicolumn{4}{p{0.6\linewidth}}{\cellcolor[rgb]{0.9,0.9,0.9}{%
	\makecell[{{p{\linewidth}}}]{% 
	  \tt {\tiny [GM$\rangle$B]}  
	 Your message has been relayed to the other player.\\ \tt \\ \tt \\ \tt You are currently playing round 2 of maximum 28.\\ \tt You have collectively accumulated 1 of 16 penalties.\\ \tt The other player sent this message:\\ \tt \\ \tt \\ \tt "Thank you for sharing the positions of the objects on your board. It seems like we have some differences. Let's start by trying to align one object at a time. How about we start with 'P'? On my board, 'P' is at (3,7), and on your board, it's at (1,7). Would you be willing to move your 'P' to (3,7) to match mine, or should I move mine to (1,7) to match yours?"\\ \tt \\ \tt \\ \tt What is your next command? 
	  } 
	   } 
	   } 
	 & \\ 
 

    \theutterance \stepcounter{utterance}  

    & & & \multicolumn{4}{p{0.6\linewidth}}{\cellcolor[rgb]{1,0.85,0.72}{%
	\makecell[{{p{\linewidth}}}]{% 
	  \tt {\tiny [GM$\langle$B]}  
	 SAY: I'm willing to move my 'P' to (3,7) to match yours, so we can align the 'P' objects. Please confirm if the spot (3,7) is empty on your board after I make the move. 
	  } 
	   } 
	   } 
	  \\ 
 

    \theutterance \stepcounter{utterance}  

    & & & \multicolumn{2}{p{0.3\linewidth}}{\cellcolor[rgb]{0.95,0.95,0.95}{%
	\makecell[{{p{\linewidth}}}]{% 
	  \tt {\tiny [GM$|$GM]}  
	 SAY: I'm willing to move my 'P' to (3,7) to match yours, so we can align the 'P' objects. Please confirm if the spot (3,7) is empty on your board after I make the move. 
	  } 
	   } 
	   } 
	 & & \\ 
 

    \theutterance \stepcounter{utterance}  

    & & \multicolumn{4}{p{0.6\linewidth}}{\cellcolor[rgb]{0.9,0.9,0.9}{%
	\makecell[{{p{\linewidth}}}]{% 
	  \tt {\tiny [A$\langle$GM]}  
	 Your message has been relayed to the other player.\\ \tt \\ \tt \\ \tt You are currently playing round 2 of maximum 28.\\ \tt You have collectively accumulated 1 of 16 penalties.\\ \tt The other player sent this message:\\ \tt \\ \tt \\ \tt "I'm willing to move my 'P' to (3,7) to match yours, so we can align the 'P' objects. Please confirm if the spot (3,7) is empty on your board after I make the move."\\ \tt \\ \tt \\ \tt What is your next command? 
	  } 
	   } 
	   } 
	 & \\ 
 

    \theutterance \stepcounter{utterance}  

    & \multicolumn{4}{p{0.6\linewidth}}{\cellcolor[rgb]{0.8,1,0.9}{%
	\makecell[{{p{\linewidth}}}]{% 
	  \tt {\tiny [A$\rangle$GM]}  
	 SAY: Yes, the spot (3,7) is currently occupied by 'P' on my board, but that's the object we're trying to align. Once you move your 'P' to (3,7) on your board, we can consider the 'P' objects aligned. However, I notice that on my board, 'M' is also at (4,7), very close to 'P'. How about we try to align 'M' as well? On your board, 'M' is at (3,7), which is the same column where we're trying to put 'P'. Should we try to move 'M' to a different position to make space for 'P', or move 'P' to a different column to avoid conflicting with 'M'? 
	  } 
	   } 
	   } 
	 & & \\ 
 

    \theutterance \stepcounter{utterance}  

    & & & \multicolumn{2}{p{0.3\linewidth}}{\cellcolor[rgb]{0.95,0.95,0.95}{%
	\makecell[{{p{\linewidth}}}]{% 
	  \tt {\tiny [GM$|$GM]}  
	 SAY: Yes, the spot (3,7) is currently occupied by 'P' on my board, but that's the object we're trying to align. Once you move your 'P' to (3,7) on your board, we can consider the 'P' objects aligned. However, I notice that on my board, 'M' is also at (4,7), very close to 'P'. How about we try to align 'M' as well? On your board, 'M' is at (3,7), which is the same column where we're trying to put 'P'. Should we try to move 'M' to a different position to make space for 'P', or move 'P' to a different column to avoid conflicting with 'M'? 
	  } 
	   } 
	   } 
	 & & \\ 
 

    \theutterance \stepcounter{utterance}  

    & & \multicolumn{4}{p{0.6\linewidth}}{\cellcolor[rgb]{0.9,0.9,0.9}{%
	\makecell[{{p{\linewidth}}}]{% 
	  \tt {\tiny [GM$\rangle$B]}  
	 Your message has been relayed to the other player.\\ \tt \\ \tt \\ \tt You are currently playing round 3 of maximum 28.\\ \tt You have collectively accumulated 1 of 16 penalties.\\ \tt The other player sent this message:\\ \tt \\ \tt \\ \tt "Yes, the spot (3,7) is currently occupied by 'P' on my board, but that's the object we're trying to align. Once you move your 'P' to (3,7) on your board, we can consider the 'P' objects aligned. However, I notice that on my board, 'M' is also at (4,7), very close to 'P'. How about we try to align 'M' as well? On your board, 'M' is at (3,7), which is the same column where we're trying to put 'P'. Should we try to move 'M' to a different position to make space for 'P', or move 'P' to a different column to avoid conflicting with 'M'?"\\ \tt \\ \tt \\ \tt What is your next command? 
	  } 
	   } 
	   } 
	 & \\ 
 

    \theutterance \stepcounter{utterance}  

    & & & \multicolumn{4}{p{0.6\linewidth}}{\cellcolor[rgb]{1,0.85,0.72}{%
	\makecell[{{p{\linewidth}}}]{% 
	  \tt {\tiny [GM$\langle$B]}  
	 SAY: Let's move my 'M' to a different position to make space for 'P'. How about we move 'M' to (4,7) or another empty spot that won't conflict with other objects? We can discuss the best position for 'M' after 'P' is aligned. 
	  } 
	   } 
	   } 
	  \\ 
 

    \theutterance \stepcounter{utterance}  

    & & & \multicolumn{2}{p{0.3\linewidth}}{\cellcolor[rgb]{0.95,0.95,0.95}{%
	\makecell[{{p{\linewidth}}}]{% 
	  \tt {\tiny [GM$|$GM]}  
	 SAY: Let's move my 'M' to a different position to make space for 'P'. How about we move 'M' to (4,7) or another empty spot that won't conflict with other objects? We can discuss the best position for 'M' after 'P' is aligned. 
	  } 
	   } 
	   } 
	 & & \\ 
 

    \theutterance \stepcounter{utterance}  

    & & \multicolumn{4}{p{0.6\linewidth}}{\cellcolor[rgb]{0.9,0.9,0.9}{%
	\makecell[{{p{\linewidth}}}]{% 
	  \tt {\tiny [A$\langle$GM]}  
	 Your message has been relayed to the other player.\\ \tt \\ \tt \\ \tt You are currently playing round 3 of maximum 28.\\ \tt You have collectively accumulated 1 of 16 penalties.\\ \tt The other player sent this message:\\ \tt \\ \tt \\ \tt "Let's move my 'M' to a different position to make space for 'P'. How about we move 'M' to (4,7) or another empty spot that won't conflict with other objects? We can discuss the best position for 'M' after 'P' is aligned."\\ \tt \\ \tt \\ \tt What is your next command? 
	  } 
	   } 
	   } 
	 & \\ 
 

    \theutterance \stepcounter{utterance}  

    & \multicolumn{4}{p{0.6\linewidth}}{\cellcolor[rgb]{0.8,1,0.9}{%
	\makecell[{{p{\linewidth}}}]{% 
	  \tt {\tiny [A$\rangle$GM]}  
	 SAY: That sounds like a good plan. On my board, (4,7) is empty, so if you move your 'M' to (4,7), it won't conflict with any other objects. However, before we make any moves, let's confirm that we're on the same page. To align 'P', you will move your 'P' from (1,7) to (3,7), and to make space for it, you will move your 'M' from (3,7) to (4,7). Is that correct? If so, I'll wait for your move to be confirmed before we proceed with the next steps. 
	  } 
	   } 
	   } 
	 & & \\ 
 

    \theutterance \stepcounter{utterance}  

    & & & \multicolumn{2}{p{0.3\linewidth}}{\cellcolor[rgb]{0.95,0.95,0.95}{%
	\makecell[{{p{\linewidth}}}]{% 
	  \tt {\tiny [GM$|$GM]}  
	 SAY: That sounds like a good plan. On my board, (4,7) is empty, so if you move your 'M' to (4,7), it won't conflict with any other objects. However, before we make any moves, let's confirm that we're on the same page. To align 'P', you will move your 'P' from (1,7) to (3,7), and to make space for it, you will move your 'M' from (3,7) to (4,7). Is that correct? If so, I'll wait for your move to be confirmed before we proceed with the next steps. 
	  } 
	   } 
	   } 
	 & & \\ 
 

    \theutterance \stepcounter{utterance}  

    & & \multicolumn{4}{p{0.6\linewidth}}{\cellcolor[rgb]{0.9,0.9,0.9}{%
	\makecell[{{p{\linewidth}}}]{% 
	  \tt {\tiny [GM$\rangle$B]}  
	 Your message has been relayed to the other player.\\ \tt \\ \tt \\ \tt You are currently playing round 4 of maximum 28.\\ \tt You have collectively accumulated 1 of 16 penalties.\\ \tt The other player sent this message:\\ \tt \\ \tt \\ \tt "That sounds like a good plan. On my board, (4,7) is empty, so if you move your 'M' to (4,7), it won't conflict with any other objects. However, before we make any moves, let's confirm that we're on the same page. To align 'P', you will move your 'P' from (1,7) to (3,7), and to make space for it, you will move your 'M' from (3,7) to (4,7). Is that correct? If so, I'll wait for your move to be confirmed before we proceed with the next steps."\\ \tt \\ \tt \\ \tt What is your next command? 
	  } 
	   } 
	   } 
	 & \\ 
 

    \theutterance \stepcounter{utterance}  

    & & & \multicolumn{4}{p{0.6\linewidth}}{\cellcolor[rgb]{1,0.85,0.72}{%
	\makecell[{{p{\linewidth}}}]{% 
	  \tt {\tiny [GM$\langle$B]}  
	 SAY: Yes, that's correct. I will move my 'M' from (3,7) to (4,7) and then move my 'P' from (1,7) to (3,7) to align it with yours. I'll make the moves now. 
	  } 
	   } 
	   } 
	  \\ 
 

    \theutterance \stepcounter{utterance}  

    & & & \multicolumn{2}{p{0.3\linewidth}}{\cellcolor[rgb]{0.95,0.95,0.95}{%
	\makecell[{{p{\linewidth}}}]{% 
	  \tt {\tiny [GM$|$GM]}  
	 SAY: Yes, that's correct. I will move my 'M' from (3,7) to (4,7) and then move my 'P' from (1,7) to (3,7) to align it with yours. I'll make the moves now. 
	  } 
	   } 
	   } 
	 & & \\ 
 

    \theutterance \stepcounter{utterance}  

    & & \multicolumn{4}{p{0.6\linewidth}}{\cellcolor[rgb]{0.9,0.9,0.9}{%
	\makecell[{{p{\linewidth}}}]{% 
	  \tt {\tiny [A$\langle$GM]}  
	 Your message has been relayed to the other player.\\ \tt \\ \tt \\ \tt You are currently playing round 4 of maximum 28.\\ \tt You have collectively accumulated 1 of 16 penalties.\\ \tt The other player sent this message:\\ \tt \\ \tt \\ \tt "Yes, that's correct. I will move my 'M' from (3,7) to (4,7) and then move my 'P' from (1,7) to (3,7) to align it with yours. I'll make the moves now."\\ \tt \\ \tt \\ \tt What is your next command? 
	  } 
	   } 
	   } 
	 & \\ 
 

    \theutterance \stepcounter{utterance}  

    & \multicolumn{4}{p{0.6\linewidth}}{\cellcolor[rgb]{0.8,1,0.9}{%
	\makecell[{{p{\linewidth}}}]{% 
	  \tt {\tiny [A$\rangle$GM]}  
	 SAY: Wait, don't make the moves yet! We need to make sure that the moves are done in a way that can be replicated on both boards. Since we can't see each other's boards, we need to make one move at a time and confirm the outcome before making the next move. Please make the first move, which is to move your 'M' from (3,7) to (4,7), and then I'll confirm the outcome on my board before you proceed with moving your 'P'. 
	  } 
	   } 
	   } 
	 & & \\ 
 

    \theutterance \stepcounter{utterance}  

    & & & \multicolumn{2}{p{0.3\linewidth}}{\cellcolor[rgb]{0.95,0.95,0.95}{%
	\makecell[{{p{\linewidth}}}]{% 
	  \tt {\tiny [GM$|$GM]}  
	 SAY: Wait, don't make the moves yet! We need to make sure that the moves are done in a way that can be replicated on both boards. Since we can't see each other's boards, we need to make one move at a time and confirm the outcome before making the next move. Please make the first move, which is to move your 'M' from (3,7) to (4,7), and then I'll confirm the outcome on my board before you proceed with moving your 'P'. 
	  } 
	   } 
	   } 
	 & & \\ 
 

    \theutterance \stepcounter{utterance}  

    & & \multicolumn{4}{p{0.6\linewidth}}{\cellcolor[rgb]{0.9,0.9,0.9}{%
	\makecell[{{p{\linewidth}}}]{% 
	  \tt {\tiny [GM$\rangle$B]}  
	 Your message has been relayed to the other player.\\ \tt \\ \tt \\ \tt You are currently playing round 5 of maximum 28.\\ \tt You have collectively accumulated 1 of 16 penalties.\\ \tt The other player sent this message:\\ \tt \\ \tt \\ \tt "Wait, don't make the moves yet! We need to make sure that the moves are done in a way that can be replicated on both boards. Since we can't see each other's boards, we need to make one move at a time and confirm the outcome before making the next move. Please make the first move, which is to move your 'M' from (3,7) to (4,7), and then I'll confirm the outcome on my board before you proceed with moving your 'P'."\\ \tt \\ \tt \\ \tt What is your next command? 
	  } 
	   } 
	   } 
	 & \\ 
 

    \theutterance \stepcounter{utterance}  

    & & & \multicolumn{4}{p{0.6\linewidth}}{\cellcolor[rgb]{1,0.85,0.72}{%
	\makecell[{{p{\linewidth}}}]{% 
	  \tt {\tiny [GM$\langle$B]}  
	 MOVE: M, (4,7) 
	  } 
	   } 
	   } 
	  \\ 
 

    \theutterance \stepcounter{utterance}  

    & & & \multicolumn{2}{p{0.3\linewidth}}{\cellcolor[rgb]{0.95,0.95,0.95}{%
	\makecell[{{p{\linewidth}}}]{% 
	  \tt {\tiny [GM$|$GM]}  
	 MOVE: M, (4,7) 
	  } 
	   } 
	   } 
	 & & \\ 
 

    \theutterance \stepcounter{utterance}  

    & & & \multicolumn{2}{p{0.3\linewidth}}{\cellcolor[rgb]{0.95,0.95,0.95}{%
	\makecell[{{p{\linewidth}}}]{% 
	  \tt {\tiny [GM$|$GM]}  
	 Penalty: (4,7) is not empty, but contains '│'.\\ \tt \\ \tt You have collectively accumulated 2 of 16 penalties.\\ \tt Please try again! 
	  } 
	   } 
	   } 
	 & & \\ 
 

    \theutterance \stepcounter{utterance}  

    & & \multicolumn{4}{p{0.6\linewidth}}{\cellcolor[rgb]{0.9,0.9,0.9}{%
	\makecell[{{p{\linewidth}}}]{% 
	  \tt {\tiny [GM$\rangle$B]}  
	 Penalty: (4,7) is not empty, but contains '│'.\\ \tt \\ \tt You have collectively accumulated 2 of 16 penalties.\\ \tt Please try again! 
	  } 
	   } 
	   } 
	 & \\ 
 

    \theutterance \stepcounter{utterance}  

    & & & \multicolumn{4}{p{0.6\linewidth}}{\cellcolor[rgb]{1,0.85,0.72}{%
	\makecell[{{p{\linewidth}}}]{% 
	  \tt {\tiny [GM$\langle$B]}  
	 MOVE: M, (5,7) 
	  } 
	   } 
	   } 
	  \\ 
 

    \theutterance \stepcounter{utterance}  

    & & & \multicolumn{2}{p{0.3\linewidth}}{\cellcolor[rgb]{0.95,0.95,0.95}{%
	\makecell[{{p{\linewidth}}}]{% 
	  \tt {\tiny [GM$|$GM]}  
	 MOVE: M, (5,7) 
	  } 
	   } 
	   } 
	 & & \\ 
 

    \theutterance \stepcounter{utterance}  

    & & & \multicolumn{2}{p{0.3\linewidth}}{\cellcolor[rgb]{0.95,0.95,0.95}{%
	\makecell[{{p{\linewidth}}}]{% 
	  \tt {\tiny [GM$|$GM]}  
	 Penalty: (5,7) is not empty, but contains 'M'.\\ \tt \\ \tt You have collectively accumulated 3 of 16 penalties.\\ \tt Please try again! 
	  } 
	   } 
	   } 
	 & & \\ 
 

    \theutterance \stepcounter{utterance}  

    & & \multicolumn{4}{p{0.6\linewidth}}{\cellcolor[rgb]{0.9,0.9,0.9}{%
	\makecell[{{p{\linewidth}}}]{% 
	  \tt {\tiny [GM$\rangle$B]}  
	 Penalty: (5,7) is not empty, but contains 'M'.\\ \tt \\ \tt You have collectively accumulated 3 of 16 penalties.\\ \tt Please try again! 
	  } 
	   } 
	   } 
	 & \\ 
 

    \theutterance \stepcounter{utterance}  

    & & & \multicolumn{4}{p{0.6\linewidth}}{\cellcolor[rgb]{1,0.85,0.72}{%
	\makecell[{{p{\linewidth}}}]{% 
	  \tt {\tiny [GM$\langle$B]}  
	 MOVE: M, (2,7) 
	  } 
	   } 
	   } 
	  \\ 
 

    \theutterance \stepcounter{utterance}  

    & & & \multicolumn{2}{p{0.3\linewidth}}{\cellcolor[rgb]{0.95,0.95,0.95}{%
	\makecell[{{p{\linewidth}}}]{% 
	  \tt {\tiny [GM$|$GM]}  
	 MOVE: M, (2,7) 
	  } 
	   } 
	   } 
	 & & \\ 
 

    \theutterance \stepcounter{utterance}  

    & & & \multicolumn{2}{p{0.3\linewidth}}{\cellcolor[rgb]{0.95,0.95,0.95}{%
	\makecell[{{p{\linewidth}}}]{% 
	  \tt {\tiny [GM$|$GM]}  
	 Penalty: (2,7) is not empty, but contains '│'.\\ \tt \\ \tt You have collectively accumulated 4 of 16 penalties.\\ \tt Please try again! 
	  } 
	   } 
	   } 
	 & & \\ 
 

    \theutterance \stepcounter{utterance}  

    & & \multicolumn{4}{p{0.6\linewidth}}{\cellcolor[rgb]{0.9,0.9,0.9}{%
	\makecell[{{p{\linewidth}}}]{% 
	  \tt {\tiny [GM$\rangle$B]}  
	 Penalty: (2,7) is not empty, but contains '│'.\\ \tt \\ \tt You have collectively accumulated 4 of 16 penalties.\\ \tt Please try again! 
	  } 
	   } 
	   } 
	 & \\ 
 

    \theutterance \stepcounter{utterance}  

    & & & \multicolumn{4}{p{0.6\linewidth}}{\cellcolor[rgb]{1,0.85,0.72}{%
	\makecell[{{p{\linewidth}}}]{% 
	  \tt {\tiny [GM$\langle$B]}  
	 MOVE: M, (6,7) 
	  } 
	   } 
	   } 
	  \\ 
 

    \theutterance \stepcounter{utterance}  

    & & & \multicolumn{2}{p{0.3\linewidth}}{\cellcolor[rgb]{0.95,0.95,0.95}{%
	\makecell[{{p{\linewidth}}}]{% 
	  \tt {\tiny [GM$|$GM]}  
	 MOVE: M, (6,7) 
	  } 
	   } 
	   } 
	 & & \\ 
 

    \theutterance \stepcounter{utterance}  

    & & & \multicolumn{2}{p{0.3\linewidth}}{\cellcolor[rgb]{0.95,0.95,0.95}{%
	\makecell[{{p{\linewidth}}}]{% 
	  \tt {\tiny [GM$|$GM]}  
	 Penalty: (6,7) is not empty, but contains '│'.\\ \tt \\ \tt You have collectively accumulated 5 of 16 penalties.\\ \tt Please try again! 
	  } 
	   } 
	   } 
	 & & \\ 
 

    \theutterance \stepcounter{utterance}  

    & & \multicolumn{4}{p{0.6\linewidth}}{\cellcolor[rgb]{0.9,0.9,0.9}{%
	\makecell[{{p{\linewidth}}}]{% 
	  \tt {\tiny [GM$\rangle$B]}  
	 Penalty: (6,7) is not empty, but contains '│'.\\ \tt \\ \tt You have collectively accumulated 5 of 16 penalties.\\ \tt Please try again! 
	  } 
	   } 
	   } 
	 & \\ 
 

    \theutterance \stepcounter{utterance}  

    & & & \multicolumn{4}{p{0.6\linewidth}}{\cellcolor[rgb]{1,0.85,0.72}{%
	\makecell[{{p{\linewidth}}}]{% 
	  \tt {\tiny [GM$\langle$B]}  
	 MOVE: M, (5,6) 
	  } 
	   } 
	   } 
	  \\ 
 

    \theutterance \stepcounter{utterance}  

    & & & \multicolumn{2}{p{0.3\linewidth}}{\cellcolor[rgb]{0.95,0.95,0.95}{%
	\makecell[{{p{\linewidth}}}]{% 
	  \tt {\tiny [GM$|$GM]}  
	 MOVE: M, (5,6) 
	  } 
	   } 
	   } 
	 & & \\ 
 

    \theutterance \stepcounter{utterance}  

    & & & \multicolumn{2}{p{0.3\linewidth}}{\cellcolor[rgb]{0.95,0.95,0.95}{%
	\makecell[{{p{\linewidth}}}]{% 
	  \tt {\tiny [GM$|$GM]}  
	 Penalty: (5,6) is not empty, but contains '─'.\\ \tt \\ \tt You have collectively accumulated 6 of 16 penalties.\\ \tt Please try again! 
	  } 
	   } 
	   } 
	 & & \\ 
 

    \theutterance \stepcounter{utterance}  

    & & \multicolumn{4}{p{0.6\linewidth}}{\cellcolor[rgb]{0.9,0.9,0.9}{%
	\makecell[{{p{\linewidth}}}]{% 
	  \tt {\tiny [GM$\rangle$B]}  
	 Penalty: (5,6) is not empty, but contains '─'.\\ \tt \\ \tt You have collectively accumulated 6 of 16 penalties.\\ \tt Please try again! 
	  } 
	   } 
	   } 
	 & \\ 
 

    \theutterance \stepcounter{utterance}  

    & & & \multicolumn{4}{p{0.6\linewidth}}{\cellcolor[rgb]{1,0.85,0.72}{%
	\makecell[{{p{\linewidth}}}]{% 
	  \tt {\tiny [GM$\langle$B]}  
	 MOVE: M, (6,6) 
	  } 
	   } 
	   } 
	  \\ 
 

    \theutterance \stepcounter{utterance}  

    & & & \multicolumn{2}{p{0.3\linewidth}}{\cellcolor[rgb]{0.95,0.95,0.95}{%
	\makecell[{{p{\linewidth}}}]{% 
	  \tt {\tiny [GM$|$GM]}  
	 MOVE: M, (6,6) 
	  } 
	   } 
	   } 
	 & & \\ 
 

    \theutterance \stepcounter{utterance}  

    & & & \multicolumn{2}{p{0.3\linewidth}}{\cellcolor[rgb]{0.95,0.95,0.95}{%
	\makecell[{{p{\linewidth}}}]{% 
	  \tt {\tiny [GM$|$GM]}  
	 Penalty: (6,6) is not empty, but contains '┤'.\\ \tt \\ \tt You have collectively accumulated 7 of 16 penalties.\\ \tt Please try again! 
	  } 
	   } 
	   } 
	 & & \\ 
 

    \theutterance \stepcounter{utterance}  

    & & \multicolumn{4}{p{0.6\linewidth}}{\cellcolor[rgb]{0.9,0.9,0.9}{%
	\makecell[{{p{\linewidth}}}]{% 
	  \tt {\tiny [GM$\rangle$B]}  
	 Penalty: (6,6) is not empty, but contains '┤'.\\ \tt \\ \tt You have collectively accumulated 7 of 16 penalties.\\ \tt Please try again! 
	  } 
	   } 
	   } 
	 & \\ 
 

    \theutterance \stepcounter{utterance}  

    & & & \multicolumn{4}{p{0.6\linewidth}}{\cellcolor[rgb]{1,0.85,0.72}{%
	\makecell[{{p{\linewidth}}}]{% 
	  \tt {\tiny [GM$\langle$B]}  
	 MOVE: M, (7,7) 
	  } 
	   } 
	   } 
	  \\ 
 

    \theutterance \stepcounter{utterance}  

    & & & \multicolumn{2}{p{0.3\linewidth}}{\cellcolor[rgb]{0.95,0.95,0.95}{%
	\makecell[{{p{\linewidth}}}]{% 
	  \tt {\tiny [GM$|$GM]}  
	 MOVE: M, (7,7) 
	  } 
	   } 
	   } 
	 & & \\ 
 

    \theutterance \stepcounter{utterance}  

    & & & \multicolumn{2}{p{0.3\linewidth}}{\cellcolor[rgb]{0.95,0.95,0.95}{%
	\makecell[{{p{\linewidth}}}]{% 
	  \tt {\tiny [GM$|$GM]}  
	 Moved 'M' to (7,7) successfully. Your updated grid looks like this:\\ \tt ```\\ \tt  1234567\\ \tt ╔══╤════╗\\ \tt ║◌S│◌◌◌◌║ 1\\ \tt ╟──┘◌◌┌─╢ 2\\ \tt ║A◌O◌◌│◌║ 3\\ \tt ╟─┐◌┌─┼─╢ 4\\ \tt ║◌├─┤T│D║ 5\\ \tt ╟─┤◌├─┤◌║ 6\\ \tt ║P│◌│◌│M║ 7\\ \tt ╚═╧═╧═╧═╝\\ \tt ```\\ \tt  
	  } 
	   } 
	   } 
	 & & \\ 
 

    \theutterance \stepcounter{utterance}  

    & & \multicolumn{4}{p{0.6\linewidth}}{\cellcolor[rgb]{0.9,0.9,0.9}{%
	\makecell[{{p{\linewidth}}}]{% 
	  \tt {\tiny [A$\langle$GM]}  
	 Your message has been relayed to the other player.\\ \tt \\ \tt \\ \tt You are currently playing round 5 of maximum 28.\\ \tt You have collectively accumulated 7 of 16 penalties.\\ \tt The other player moved an object on their grid.\\ \tt \\ \tt \\ \tt What is your next command? 
	  } 
	   } 
	   } 
	 & \\ 
 

    \theutterance \stepcounter{utterance}  

    & \multicolumn{4}{p{0.6\linewidth}}{\cellcolor[rgb]{0.8,1,0.9}{%
	\makecell[{{p{\linewidth}}}]{% 
	  \tt {\tiny [A$\rangle$GM]}  
	 SAY: It seems like you've made a move. Can you please confirm what move you made? Did you move your 'M' from (3,7) to (4,7) as planned? Also, what's the current state of your board? Have any other objects been moved or are they still in their original positions? 
	  } 
	   } 
	   } 
	 & & \\ 
 

    \theutterance \stepcounter{utterance}  

    & & & \multicolumn{2}{p{0.3\linewidth}}{\cellcolor[rgb]{0.95,0.95,0.95}{%
	\makecell[{{p{\linewidth}}}]{% 
	  \tt {\tiny [GM$|$GM]}  
	 SAY: It seems like you've made a move. Can you please confirm what move you made? Did you move your 'M' from (3,7) to (4,7) as planned? Also, what's the current state of your board? Have any other objects been moved or are they still in their original positions? 
	  } 
	   } 
	   } 
	 & & \\ 
 

    \theutterance \stepcounter{utterance}  

    & & \multicolumn{4}{p{0.6\linewidth}}{\cellcolor[rgb]{0.9,0.9,0.9}{%
	\makecell[{{p{\linewidth}}}]{% 
	  \tt {\tiny [GM$\rangle$B]}  
	 Moved 'M' to (7,7) successfully. Your updated grid looks like this:\\ \tt ```\\ \tt  1234567\\ \tt ╔══╤════╗\\ \tt ║◌S│◌◌◌◌║ 1\\ \tt ╟──┘◌◌┌─╢ 2\\ \tt ║A◌O◌◌│◌║ 3\\ \tt ╟─┐◌┌─┼─╢ 4\\ \tt ║◌├─┤T│D║ 5\\ \tt ╟─┤◌├─┤◌║ 6\\ \tt ║P│◌│◌│M║ 7\\ \tt ╚═╧═╧═╧═╝\\ \tt ```\\ \tt You are currently playing round 6 of maximum 28.\\ \tt You have collectively accumulated 7 of 16 penalties.\\ \tt The other player sent this message:\\ \tt \\ \tt \\ \tt "It seems like you've made a move. Can you please confirm what move you made? Did you move your 'M' from (3,7) to (4,7) as planned? Also, what's the current state of your board? Have any other objects been moved or are they still in their original positions?"\\ \tt \\ \tt \\ \tt What is your next command? 
	  } 
	   } 
	   } 
	 & \\ 
 

    \theutterance \stepcounter{utterance}  

    & & & \multicolumn{4}{p{0.6\linewidth}}{\cellcolor[rgb]{1,0.85,0.72}{%
	\makecell[{{p{\linewidth}}}]{% 
	  \tt {\tiny [GM$\langle$B]}  
	 SAY: I moved my 'M' from (3,7) to (7,7), not (4,7) as originally planned, since (4,7) was not a valid move. My current board state is as shown above, with 'M' now at (7,7) and all other objects in their original positions except for the move I just made. Next, I plan to move 'P' from (1,7) to (3,7) to align it with yours. 
	  } 
	   } 
	   } 
	  \\ 
 

    \theutterance \stepcounter{utterance}  

    & & & \multicolumn{2}{p{0.3\linewidth}}{\cellcolor[rgb]{0.95,0.95,0.95}{%
	\makecell[{{p{\linewidth}}}]{% 
	  \tt {\tiny [GM$|$GM]}  
	 SAY: I moved my 'M' from (3,7) to (7,7), not (4,7) as originally planned, since (4,7) was not a valid move. My current board state is as shown above, with 'M' now at (7,7) and all other objects in their original positions except for the move I just made. Next, I plan to move 'P' from (1,7) to (3,7) to align it with yours. 
	  } 
	   } 
	   } 
	 & & \\ 
 

    \theutterance \stepcounter{utterance}  

    & & \multicolumn{4}{p{0.6\linewidth}}{\cellcolor[rgb]{0.9,0.9,0.9}{%
	\makecell[{{p{\linewidth}}}]{% 
	  \tt {\tiny [A$\langle$GM]}  
	 Your message has been relayed to the other player.\\ \tt \\ \tt \\ \tt You are currently playing round 6 of maximum 28.\\ \tt You have collectively accumulated 7 of 16 penalties.\\ \tt The other player sent this message:\\ \tt \\ \tt \\ \tt "I moved my 'M' from (3,7) to (7,7), not (4,7) as originally planned, since (4,7) was not a valid move. My current board state is as shown above, with 'M' now at (7,7) and all other objects in their original positions except for the move I just made. Next, I plan to move 'P' from (1,7) to (3,7) to align it with yours."\\ \tt \\ \tt \\ \tt What is your next command? 
	  } 
	   } 
	   } 
	 & \\ 
 

    \theutterance \stepcounter{utterance}  

    & \multicolumn{4}{p{0.6\linewidth}}{\cellcolor[rgb]{0.8,1,0.9}{%
	\makecell[{{p{\linewidth}}}]{% 
	  \tt {\tiny [A$\rangle$GM]}  
	 SAY: Thank you for clarifying the move. I apologize for not considering the possibility that (4,7) might not be a valid move on your board. Moving your 'M' to (7,7) is a good alternative. Now, regarding your plan to move 'P' from (1,7) to (3,7), I agree that it's a good next step. Before you make the move, I just want to confirm that the spot (3,7) is empty on your board after moving 'M' to (7,7). If it is, please go ahead and move 'P' to (3,7), and then we can assess the next steps to align the other objects. 
	  } 
	   } 
	   } 
	 & & \\ 
 

    \theutterance \stepcounter{utterance}  

    & & & \multicolumn{2}{p{0.3\linewidth}}{\cellcolor[rgb]{0.95,0.95,0.95}{%
	\makecell[{{p{\linewidth}}}]{% 
	  \tt {\tiny [GM$|$GM]}  
	 SAY: Thank you for clarifying the move. I apologize for not considering the possibility that (4,7) might not be a valid move on your board. Moving your 'M' to (7,7) is a good alternative. Now, regarding your plan to move 'P' from (1,7) to (3,7), I agree that it's a good next step. Before you make the move, I just want to confirm that the spot (3,7) is empty on your board after moving 'M' to (7,7). If it is, please go ahead and move 'P' to (3,7), and then we can assess the next steps to align the other objects. 
	  } 
	   } 
	   } 
	 & & \\ 
 

    \theutterance \stepcounter{utterance}  

    & & \multicolumn{4}{p{0.6\linewidth}}{\cellcolor[rgb]{0.9,0.9,0.9}{%
	\makecell[{{p{\linewidth}}}]{% 
	  \tt {\tiny [GM$\rangle$B]}  
	 Your message has been relayed to the other player.\\ \tt \\ \tt \\ \tt You are currently playing round 7 of maximum 28.\\ \tt You have collectively accumulated 7 of 16 penalties.\\ \tt The other player sent this message:\\ \tt \\ \tt \\ \tt "Thank you for clarifying the move. I apologize for not considering the possibility that (4,7) might not be a valid move on your board. Moving your 'M' to (7,7) is a good alternative. Now, regarding your plan to move 'P' from (1,7) to (3,7), I agree that it's a good next step. Before you make the move, I just want to confirm that the spot (3,7) is empty on your board after moving 'M' to (7,7). If it is, please go ahead and move 'P' to (3,7), and then we can assess the next steps to align the other objects."\\ \tt \\ \tt \\ \tt What is your next command? 
	  } 
	   } 
	   } 
	 & \\ 
 

    \theutterance \stepcounter{utterance}  

    & & & \multicolumn{4}{p{0.6\linewidth}}{\cellcolor[rgb]{1,0.85,0.72}{%
	\makecell[{{p{\linewidth}}}]{% 
	  \tt {\tiny [GM$\langle$B]}  
	 MOVE: P, (3,7) 
	  } 
	   } 
	   } 
	  \\ 
 

    \theutterance \stepcounter{utterance}  

    & & & \multicolumn{2}{p{0.3\linewidth}}{\cellcolor[rgb]{0.95,0.95,0.95}{%
	\makecell[{{p{\linewidth}}}]{% 
	  \tt {\tiny [GM$|$GM]}  
	 MOVE: P, (3,7) 
	  } 
	   } 
	   } 
	 & & \\ 
 

    \theutterance \stepcounter{utterance}  

    & & & \multicolumn{2}{p{0.3\linewidth}}{\cellcolor[rgb]{0.95,0.95,0.95}{%
	\makecell[{{p{\linewidth}}}]{% 
	  \tt {\tiny [GM$|$GM]}  
	 Moved 'P' to (3,7) successfully. Your updated grid looks like this:\\ \tt ```\\ \tt  1234567\\ \tt ╔══╤════╗\\ \tt ║◌S│◌◌◌◌║ 1\\ \tt ╟──┘◌◌┌─╢ 2\\ \tt ║A◌O◌◌│◌║ 3\\ \tt ╟─┐◌┌─┼─╢ 4\\ \tt ║◌├─┤T│D║ 5\\ \tt ╟─┤◌├─┤◌║ 6\\ \tt ║◌│P│◌│M║ 7\\ \tt ╚═╧═╧═╧═╝\\ \tt ```\\ \tt  
	  } 
	   } 
	   } 
	 & & \\ 
 

    \theutterance \stepcounter{utterance}  

    & & \multicolumn{4}{p{0.6\linewidth}}{\cellcolor[rgb]{0.9,0.9,0.9}{%
	\makecell[{{p{\linewidth}}}]{% 
	  \tt {\tiny [A$\langle$GM]}  
	 Your message has been relayed to the other player.\\ \tt \\ \tt \\ \tt You are currently playing round 7 of maximum 28.\\ \tt You have collectively accumulated 7 of 16 penalties.\\ \tt The other player moved an object on their grid.\\ \tt \\ \tt \\ \tt What is your next command? 
	  } 
	   } 
	   } 
	 & \\ 
 

    \theutterance \stepcounter{utterance}  

    & \multicolumn{4}{p{0.6\linewidth}}{\cellcolor[rgb]{0.8,1,0.9}{%
	\makecell[{{p{\linewidth}}}]{% 
	  \tt {\tiny [A$\rangle$GM]}  
	 SAY: It seems like you've made the move to align 'P'. Can you please confirm that you've moved your 'P' from (1,7) to (3,7)? If so, our 'P' objects should now be aligned. Next, I'd like to discuss aligning the 'M' objects. On my board, 'M' is at (4,7), and on your board, 'M' is now at (7,7). Should we try to move one of the 'M' objects to a common position, or leave them as is for now and focus on aligning other objects? 
	  } 
	   } 
	   } 
	 & & \\ 
 

    \theutterance \stepcounter{utterance}  

    & & & \multicolumn{2}{p{0.3\linewidth}}{\cellcolor[rgb]{0.95,0.95,0.95}{%
	\makecell[{{p{\linewidth}}}]{% 
	  \tt {\tiny [GM$|$GM]}  
	 SAY: It seems like you've made the move to align 'P'. Can you please confirm that you've moved your 'P' from (1,7) to (3,7)? If so, our 'P' objects should now be aligned. Next, I'd like to discuss aligning the 'M' objects. On my board, 'M' is at (4,7), and on your board, 'M' is now at (7,7). Should we try to move one of the 'M' objects to a common position, or leave them as is for now and focus on aligning other objects? 
	  } 
	   } 
	   } 
	 & & \\ 
 

    \theutterance \stepcounter{utterance}  

    & & \multicolumn{4}{p{0.6\linewidth}}{\cellcolor[rgb]{0.9,0.9,0.9}{%
	\makecell[{{p{\linewidth}}}]{% 
	  \tt {\tiny [GM$\rangle$B]}  
	 Moved 'P' to (3,7) successfully. Your updated grid looks like this:\\ \tt ```\\ \tt  1234567\\ \tt ╔══╤════╗\\ \tt ║◌S│◌◌◌◌║ 1\\ \tt ╟──┘◌◌┌─╢ 2\\ \tt ║A◌O◌◌│◌║ 3\\ \tt ╟─┐◌┌─┼─╢ 4\\ \tt ║◌├─┤T│D║ 5\\ \tt ╟─┤◌├─┤◌║ 6\\ \tt ║◌│P│◌│M║ 7\\ \tt ╚═╧═╧═╧═╝\\ \tt ```\\ \tt You are currently playing round 8 of maximum 28.\\ \tt You have collectively accumulated 7 of 16 penalties.\\ \tt The other player sent this message:\\ \tt \\ \tt \\ \tt "It seems like you've made the move to align 'P'. Can you please confirm that you've moved your 'P' from (1,7) to (3,7)? If so, our 'P' objects should now be aligned. Next, I'd like to discuss aligning the 'M' objects. On my board, 'M' is at (4,7), and on your board, 'M' is now at (7,7). Should we try to move one of the 'M' objects to a common position, or leave them as is for now and focus on aligning other objects?"\\ \tt \\ \tt \\ \tt What is your next command? 
	  } 
	   } 
	   } 
	 & \\ 
 

    \theutterance \stepcounter{utterance}  

    & & & \multicolumn{4}{p{0.6\linewidth}}{\cellcolor[rgb]{1,0.85,0.72}{%
	\makecell[{{p{\linewidth}}}]{% 
	  \tt {\tiny [GM$\langle$B]}  
	 SAY: Yes, I've moved my 'P' from (1,7) to (3,7) and our 'P' objects should now be aligned. Regarding the 'M' objects, I think it's a good idea to try to move one of them to a common position. How about we aim to move my 'M' from (7,7) to (4,7) to match the position of your 'M'? Would that be a good next step? 
	  } 
	   } 
	   } 
	  \\ 
 

    \theutterance \stepcounter{utterance}  

    & & & \multicolumn{2}{p{0.3\linewidth}}{\cellcolor[rgb]{0.95,0.95,0.95}{%
	\makecell[{{p{\linewidth}}}]{% 
	  \tt {\tiny [GM$|$GM]}  
	 SAY: Yes, I've moved my 'P' from (1,7) to (3,7) and our 'P' objects should now be aligned. Regarding the 'M' objects, I think it's a good idea to try to move one of them to a common position. How about we aim to move my 'M' from (7,7) to (4,7) to match the position of your 'M'? Would that be a good next step? 
	  } 
	   } 
	   } 
	 & & \\ 
 

    \theutterance \stepcounter{utterance}  

    & & \multicolumn{4}{p{0.6\linewidth}}{\cellcolor[rgb]{0.9,0.9,0.9}{%
	\makecell[{{p{\linewidth}}}]{% 
	  \tt {\tiny [A$\langle$GM]}  
	 Your message has been relayed to the other player.\\ \tt \\ \tt \\ \tt You are currently playing round 8 of maximum 28.\\ \tt You have collectively accumulated 7 of 16 penalties.\\ \tt The other player sent this message:\\ \tt \\ \tt \\ \tt "Yes, I've moved my 'P' from (1,7) to (3,7) and our 'P' objects should now be aligned. Regarding the 'M' objects, I think it's a good idea to try to move one of them to a common position. How about we aim to move my 'M' from (7,7) to (4,7) to match the position of your 'M'? Would that be a good next step?"\\ \tt \\ \tt \\ \tt What is your next command? 
	  } 
	   } 
	   } 
	 & \\ 
 

    \theutterance \stepcounter{utterance}  

    & \multicolumn{4}{p{0.6\linewidth}}{\cellcolor[rgb]{0.8,1,0.9}{%
	\makecell[{{p{\linewidth}}}]{% 
	  \tt {\tiny [A$\rangle$GM]}  
	 SAY: That sounds like a great plan. Moving your 'M' from (7,7) to (4,7) would indeed align the 'M' objects. Before you make the move, I just want to confirm that the spot (4,7) is empty on your board. If it is, please go ahead and move your 'M' to (4,7). After that, we can take a step back and assess the overall state of our boards to decide which object to align next. Perhaps we can focus on the 'T' objects or the 'O' objects next? 
	  } 
	   } 
	   } 
	 & & \\ 
 

    \theutterance \stepcounter{utterance}  

    & & & \multicolumn{2}{p{0.3\linewidth}}{\cellcolor[rgb]{0.95,0.95,0.95}{%
	\makecell[{{p{\linewidth}}}]{% 
	  \tt {\tiny [GM$|$GM]}  
	 SAY: That sounds like a great plan. Moving your 'M' from (7,7) to (4,7) would indeed align the 'M' objects. Before you make the move, I just want to confirm that the spot (4,7) is empty on your board. If it is, please go ahead and move your 'M' to (4,7). After that, we can take a step back and assess the overall state of our boards to decide which object to align next. Perhaps we can focus on the 'T' objects or the 'O' objects next? 
	  } 
	   } 
	   } 
	 & & \\ 
 

    \theutterance \stepcounter{utterance}  

    & & \multicolumn{4}{p{0.6\linewidth}}{\cellcolor[rgb]{0.9,0.9,0.9}{%
	\makecell[{{p{\linewidth}}}]{% 
	  \tt {\tiny [GM$\rangle$B]}  
	 Your message has been relayed to the other player.\\ \tt \\ \tt \\ \tt You are currently playing round 9 of maximum 28.\\ \tt You have collectively accumulated 7 of 16 penalties.\\ \tt The other player sent this message:\\ \tt \\ \tt \\ \tt "That sounds like a great plan. Moving your 'M' from (7,7) to (4,7) would indeed align the 'M' objects. Before you make the move, I just want to confirm that the spot (4,7) is empty on your board. If it is, please go ahead and move your 'M' to (4,7). After that, we can take a step back and assess the overall state of our boards to decide which object to align next. Perhaps we can focus on the 'T' objects or the 'O' objects next?"\\ \tt \\ \tt \\ \tt What is your next command? 
	  } 
	   } 
	   } 
	 & \\ 
 

    \theutterance \stepcounter{utterance}  

    & & & \multicolumn{4}{p{0.6\linewidth}}{\cellcolor[rgb]{1,0.85,0.72}{%
	\makecell[{{p{\linewidth}}}]{% 
	  \tt {\tiny [GM$\langle$B]}  
	 MOVE: M, (4,7) 
	  } 
	   } 
	   } 
	  \\ 
 

    \theutterance \stepcounter{utterance}  

    & & & \multicolumn{2}{p{0.3\linewidth}}{\cellcolor[rgb]{0.95,0.95,0.95}{%
	\makecell[{{p{\linewidth}}}]{% 
	  \tt {\tiny [GM$|$GM]}  
	 MOVE: M, (4,7) 
	  } 
	   } 
	   } 
	 & & \\ 
 

    \theutterance \stepcounter{utterance}  

    & & & \multicolumn{2}{p{0.3\linewidth}}{\cellcolor[rgb]{0.95,0.95,0.95}{%
	\makecell[{{p{\linewidth}}}]{% 
	  \tt {\tiny [GM$|$GM]}  
	 Penalty: (4,7) is not empty, but contains '│'.\\ \tt \\ \tt You have collectively accumulated 8 of 16 penalties.\\ \tt Please try again! 
	  } 
	   } 
	   } 
	 & & \\ 
 

    \theutterance \stepcounter{utterance}  

    & & \multicolumn{4}{p{0.6\linewidth}}{\cellcolor[rgb]{0.9,0.9,0.9}{%
	\makecell[{{p{\linewidth}}}]{% 
	  \tt {\tiny [GM$\rangle$B]}  
	 Penalty: (4,7) is not empty, but contains '│'.\\ \tt \\ \tt You have collectively accumulated 8 of 16 penalties.\\ \tt Please try again! 
	  } 
	   } 
	   } 
	 & \\ 
 

    \theutterance \stepcounter{utterance}  

    & & & \multicolumn{4}{p{0.6\linewidth}}{\cellcolor[rgb]{1,0.85,0.72}{%
	\makecell[{{p{\linewidth}}}]{% 
	  \tt {\tiny [GM$\langle$B]}  
	 SAY: It seems that moving my 'M' to (4,7) is not possible since it's not empty. Let me try to find an alternative position for 'M' that matches your 'M' position. Can we discuss other possible positions for 'M'? 
	  } 
	   } 
	   } 
	  \\ 
 

    \theutterance \stepcounter{utterance}  

    & & & \multicolumn{2}{p{0.3\linewidth}}{\cellcolor[rgb]{0.95,0.95,0.95}{%
	\makecell[{{p{\linewidth}}}]{% 
	  \tt {\tiny [GM$|$GM]}  
	 SAY: It seems that moving my 'M' to (4,7) is not possible since it's not empty. Let me try to find an alternative position for 'M' that matches your 'M' position. Can we discuss other possible positions for 'M'? 
	  } 
	   } 
	   } 
	 & & \\ 
 

    \theutterance \stepcounter{utterance}  

    & & \multicolumn{4}{p{0.6\linewidth}}{\cellcolor[rgb]{0.9,0.9,0.9}{%
	\makecell[{{p{\linewidth}}}]{% 
	  \tt {\tiny [A$\langle$GM]}  
	 Your message has been relayed to the other player.\\ \tt \\ \tt \\ \tt You are currently playing round 9 of maximum 28.\\ \tt You have collectively accumulated 8 of 16 penalties.\\ \tt The other player sent this message:\\ \tt \\ \tt \\ \tt "It seems that moving my 'M' to (4,7) is not possible since it's not empty. Let me try to find an alternative position for 'M' that matches your 'M' position. Can we discuss other possible positions for 'M'?"\\ \tt \\ \tt \\ \tt What is your next command? 
	  } 
	   } 
	   } 
	 & \\ 
 

    \theutterance \stepcounter{utterance}  

    & \multicolumn{4}{p{0.6\linewidth}}{\cellcolor[rgb]{0.8,1,0.9}{%
	\makecell[{{p{\linewidth}}}]{% 
	  \tt {\tiny [A$\rangle$GM]}  
	 SAY: Okay, no problem. If (4,7) is not available on your board, let's explore other options. Since my 'M' is at (4,7), I'd like to suggest alternative positions that are close to (4,7). How about (3,7), (5,7), or (4,6)? Are any of these positions available on your board for your 'M' object? We can discuss the pros and cons of each option and choose the best one to align our 'M' objects. 
	  } 
	   } 
	   } 
	 & & \\ 
 

    \theutterance \stepcounter{utterance}  

    & & & \multicolumn{2}{p{0.3\linewidth}}{\cellcolor[rgb]{0.95,0.95,0.95}{%
	\makecell[{{p{\linewidth}}}]{% 
	  \tt {\tiny [GM$|$GM]}  
	 SAY: Okay, no problem. If (4,7) is not available on your board, let's explore other options. Since my 'M' is at (4,7), I'd like to suggest alternative positions that are close to (4,7). How about (3,7), (5,7), or (4,6)? Are any of these positions available on your board for your 'M' object? We can discuss the pros and cons of each option and choose the best one to align our 'M' objects. 
	  } 
	   } 
	   } 
	 & & \\ 
 

    \theutterance \stepcounter{utterance}  

    & & \multicolumn{4}{p{0.6\linewidth}}{\cellcolor[rgb]{0.9,0.9,0.9}{%
	\makecell[{{p{\linewidth}}}]{% 
	  \tt {\tiny [GM$\rangle$B]}  
	 Your message has been relayed to the other player.\\ \tt \\ \tt \\ \tt You are currently playing round 10 of maximum 28.\\ \tt You have collectively accumulated 8 of 16 penalties.\\ \tt The other player sent this message:\\ \tt \\ \tt \\ \tt "Okay, no problem. If (4,7) is not available on your board, let's explore other options. Since my 'M' is at (4,7), I'd like to suggest alternative positions that are close to (4,7). How about (3,7), (5,7), or (4,6)? Are any of these positions available on your board for your 'M' object? We can discuss the pros and cons of each option and choose the best one to align our 'M' objects."\\ \tt \\ \tt \\ \tt What is your next command? 
	  } 
	   } 
	   } 
	 & \\ 
 

    \theutterance \stepcounter{utterance}  

    & & & \multicolumn{4}{p{0.6\linewidth}}{\cellcolor[rgb]{1,0.85,0.72}{%
	\makecell[{{p{\linewidth}}}]{% 
	  \tt {\tiny [GM$\langle$B]}  
	 SAY: Let me check the availability of those positions on my board. (3,7) is currently occupied by 'P', which we just aligned. (5,7) contains '║', so it's not available. (4,6) contains '┤', so it's not available either. However, I can try moving my 'M' to (5,6) or other nearby empty cells. Would you like me to try one of those positions? 
	  } 
	   } 
	   } 
	  \\ 
 

    \theutterance \stepcounter{utterance}  

    & & & \multicolumn{2}{p{0.3\linewidth}}{\cellcolor[rgb]{0.95,0.95,0.95}{%
	\makecell[{{p{\linewidth}}}]{% 
	  \tt {\tiny [GM$|$GM]}  
	 SAY: Let me check the availability of those positions on my board. (3,7) is currently occupied by 'P', which we just aligned. (5,7) contains '║', so it's not available. (4,6) contains '┤', so it's not available either. However, I can try moving my 'M' to (5,6) or other nearby empty cells. Would you like me to try one of those positions? 
	  } 
	   } 
	   } 
	 & & \\ 
 

    \theutterance \stepcounter{utterance}  

    & & \multicolumn{4}{p{0.6\linewidth}}{\cellcolor[rgb]{0.9,0.9,0.9}{%
	\makecell[{{p{\linewidth}}}]{% 
	  \tt {\tiny [A$\langle$GM]}  
	 Your message has been relayed to the other player.\\ \tt \\ \tt \\ \tt You are currently playing round 10 of maximum 28.\\ \tt You have collectively accumulated 8 of 16 penalties.\\ \tt The other player sent this message:\\ \tt \\ \tt \\ \tt "Let me check the availability of those positions on my board. (3,7) is currently occupied by 'P', which we just aligned. (5,7) contains '║', so it's not available. (4,6) contains '┤', so it's not available either. However, I can try moving my 'M' to (5,6) or other nearby empty cells. Would you like me to try one of those positions?"\\ \tt \\ \tt \\ \tt What is your next command? 
	  } 
	   } 
	   } 
	 & \\ 
 

    \theutterance \stepcounter{utterance}  

    & \multicolumn{4}{p{0.6\linewidth}}{\cellcolor[rgb]{0.8,1,0.9}{%
	\makecell[{{p{\linewidth}}}]{% 
	  \tt {\tiny [A$\rangle$GM]}  
	 SAY: Thank you for checking the availability of those positions. I understand that (3,7) is occupied by 'P', and (5,7) and (4,6) are not available due to the grid structure. Moving your 'M' to (5,6) could be a good alternative. Before you make the move, I'd like to consider the implications of having 'M' at (5,6) on our overall alignment strategy. How would moving 'M' to (5,6) affect the potential alignment of other objects, such as 'T', 'O', 'D', 'S', and 'A'? Should we prioritize aligning one of these objects next, or focus on finding a better position for 'M'? 
	  } 
	   } 
	   } 
	 & & \\ 
 

    \theutterance \stepcounter{utterance}  

    & & & \multicolumn{2}{p{0.3\linewidth}}{\cellcolor[rgb]{0.95,0.95,0.95}{%
	\makecell[{{p{\linewidth}}}]{% 
	  \tt {\tiny [GM$|$GM]}  
	 SAY: Thank you for checking the availability of those positions. I understand that (3,7) is occupied by 'P', and (5,7) and (4,6) are not available due to the grid structure. Moving your 'M' to (5,6) could be a good alternative. Before you make the move, I'd like to consider the implications of having 'M' at (5,6) on our overall alignment strategy. How would moving 'M' to (5,6) affect the potential alignment of other objects, such as 'T', 'O', 'D', 'S', and 'A'? Should we prioritize aligning one of these objects next, or focus on finding a better position for 'M'? 
	  } 
	   } 
	   } 
	 & & \\ 
 

    \theutterance \stepcounter{utterance}  

    & & \multicolumn{4}{p{0.6\linewidth}}{\cellcolor[rgb]{0.9,0.9,0.9}{%
	\makecell[{{p{\linewidth}}}]{% 
	  \tt {\tiny [GM$\rangle$B]}  
	 Your message has been relayed to the other player.\\ \tt \\ \tt \\ \tt You are currently playing round 11 of maximum 28.\\ \tt You have collectively accumulated 8 of 16 penalties.\\ \tt The other player sent this message:\\ \tt \\ \tt \\ \tt "Thank you for checking the availability of those positions. I understand that (3,7) is occupied by 'P', and (5,7) and (4,6) are not available due to the grid structure. Moving your 'M' to (5,6) could be a good alternative. Before you make the move, I'd like to consider the implications of having 'M' at (5,6) on our overall alignment strategy. How would moving 'M' to (5,6) affect the potential alignment of other objects, such as 'T', 'O', 'D', 'S', and 'A'? Should we prioritize aligning one of these objects next, or focus on finding a better position for 'M'?"\\ \tt \\ \tt \\ \tt What is your next command? 
	  } 
	   } 
	   } 
	 & \\ 
 

    \theutterance \stepcounter{utterance}  

    & & & \multicolumn{4}{p{0.6\linewidth}}{\cellcolor[rgb]{1,0.85,0.72}{%
	\makecell[{{p{\linewidth}}}]{% 
	  \tt {\tiny [GM$\langle$B]}  
	 SAY: Moving 'M' to (5,6) might limit our options for aligning 'T' and 'D' since they are currently close to that area. However, it could also create opportunities for aligning 'O' and 'A' if we can find a way to move them to matching positions. Considering our goal is to align all objects, I think it's worth trying to move 'M' to (5,6) and then reassess our strategy. We can always adjust our plan if needed. What are your thoughts on prioritizing 'O' or 'A' next, or do you have a different suggestion? 
	  } 
	   } 
	   } 
	  \\ 
 

    \theutterance \stepcounter{utterance}  

    & & & \multicolumn{2}{p{0.3\linewidth}}{\cellcolor[rgb]{0.95,0.95,0.95}{%
	\makecell[{{p{\linewidth}}}]{% 
	  \tt {\tiny [GM$|$GM]}  
	 SAY: Moving 'M' to (5,6) might limit our options for aligning 'T' and 'D' since they are currently close to that area. However, it could also create opportunities for aligning 'O' and 'A' if we can find a way to move them to matching positions. Considering our goal is to align all objects, I think it's worth trying to move 'M' to (5,6) and then reassess our strategy. We can always adjust our plan if needed. What are your thoughts on prioritizing 'O' or 'A' next, or do you have a different suggestion? 
	  } 
	   } 
	   } 
	 & & \\ 
 

    \theutterance \stepcounter{utterance}  

    & & \multicolumn{4}{p{0.6\linewidth}}{\cellcolor[rgb]{0.9,0.9,0.9}{%
	\makecell[{{p{\linewidth}}}]{% 
	  \tt {\tiny [A$\langle$GM]}  
	 Your message has been relayed to the other player.\\ \tt \\ \tt \\ \tt You are currently playing round 11 of maximum 28.\\ \tt You have collectively accumulated 8 of 16 penalties.\\ \tt The other player sent this message:\\ \tt \\ \tt \\ \tt "Moving 'M' to (5,6) might limit our options for aligning 'T' and 'D' since they are currently close to that area. However, it could also create opportunities for aligning 'O' and 'A' if we can find a way to move them to matching positions. Considering our goal is to align all objects, I think it's worth trying to move 'M' to (5,6) and then reassess our strategy. We can always adjust our plan if needed. What are your thoughts on prioritizing 'O' or 'A' next, or do you have a different suggestion?"\\ \tt \\ \tt \\ \tt What is your next command? 
	  } 
	   } 
	   } 
	 & \\ 
 

    \theutterance \stepcounter{utterance}  

    & \multicolumn{4}{p{0.6\linewidth}}{\cellcolor[rgb]{0.8,1,0.9}{%
	\makecell[{{p{\linewidth}}}]{% 
	  \tt {\tiny [A$\rangle$GM]}  
	 SAY: I agree that moving 'M' to (5,6) could have both positive and negative effects on our alignment strategy. It's a good point that it might limit our options for 'T' and 'D', but it could also create opportunities for 'O' and 'A'. Considering our current progress, I think it's a good idea to move 'M' to (5,6) and reassess our strategy. Regarding prioritizing 'O' or 'A' next, I think 'O' might be a good choice. On my board, 'O' is at (5,5), and on your board, 'O' is at (2,3). If we can find a way to move them to matching positions, it could help us make progress. What do you think about trying to align 'O' next, and do you have any suggestions for how to do it? 
	  } 
	   } 
	   } 
	 & & \\ 
 

    \theutterance \stepcounter{utterance}  

    & & & \multicolumn{2}{p{0.3\linewidth}}{\cellcolor[rgb]{0.95,0.95,0.95}{%
	\makecell[{{p{\linewidth}}}]{% 
	  \tt {\tiny [GM$|$GM]}  
	 SAY: I agree that moving 'M' to (5,6) could have both positive and negative effects on our alignment strategy. It's a good point that it might limit our options for 'T' and 'D', but it could also create opportunities for 'O' and 'A'. Considering our current progress, I think it's a good idea to move 'M' to (5,6) and reassess our strategy. Regarding prioritizing 'O' or 'A' next, I think 'O' might be a good choice. On my board, 'O' is at (5,5), and on your board, 'O' is at (2,3). If we can find a way to move them to matching positions, it could help us make progress. What do you think about trying to align 'O' next, and do you have any suggestions for how to do it? 
	  } 
	   } 
	   } 
	 & & \\ 
 

    \theutterance \stepcounter{utterance}  

    & & \multicolumn{4}{p{0.6\linewidth}}{\cellcolor[rgb]{0.9,0.9,0.9}{%
	\makecell[{{p{\linewidth}}}]{% 
	  \tt {\tiny [GM$\rangle$B]}  
	 Your message has been relayed to the other player.\\ \tt \\ \tt \\ \tt You are currently playing round 12 of maximum 28.\\ \tt You have collectively accumulated 8 of 16 penalties.\\ \tt The other player sent this message:\\ \tt \\ \tt \\ \tt "I agree that moving 'M' to (5,6) could have both positive and negative effects on our alignment strategy. It's a good point that it might limit our options for 'T' and 'D', but it could also create opportunities for 'O' and 'A'. Considering our current progress, I think it's a good idea to move 'M' to (5,6) and reassess our strategy. Regarding prioritizing 'O' or 'A' next, I think 'O' might be a good choice. On my board, 'O' is at (5,5), and on your board, 'O' is at (2,3). If we can find a way to move them to matching positions, it could help us make progress. What do you think about trying to align 'O' next, and do you have any suggestions for how to do it?"\\ \tt \\ \tt \\ \tt What is your next command? 
	  } 
	   } 
	   } 
	 & \\ 
 

    \theutterance \stepcounter{utterance}  

    & & & \multicolumn{4}{p{0.6\linewidth}}{\cellcolor[rgb]{1,0.85,0.72}{%
	\makecell[{{p{\linewidth}}}]{% 
	  \tt {\tiny [GM$\langle$B]}  
	 SAY: Aligning 'O' sounds like a good next step. Since 'O' is at (2,3) on my board and (5,5) on your board, we could try to find a middle ground or a position that works for both boards. How about we aim to move both 'O' objects to a position like (4,4) or another spot that could be feasible for both boards? We can discuss the specific target position and come up with a plan to move 'O' to that spot. 
	  } 
	   } 
	   } 
	  \\ 
 

    \theutterance \stepcounter{utterance}  

    & & & \multicolumn{2}{p{0.3\linewidth}}{\cellcolor[rgb]{0.95,0.95,0.95}{%
	\makecell[{{p{\linewidth}}}]{% 
	  \tt {\tiny [GM$|$GM]}  
	 SAY: Aligning 'O' sounds like a good next step. Since 'O' is at (2,3) on my board and (5,5) on your board, we could try to find a middle ground or a position that works for both boards. How about we aim to move both 'O' objects to a position like (4,4) or another spot that could be feasible for both boards? We can discuss the specific target position and come up with a plan to move 'O' to that spot. 
	  } 
	   } 
	   } 
	 & & \\ 
 

    \theutterance \stepcounter{utterance}  

    & & \multicolumn{4}{p{0.6\linewidth}}{\cellcolor[rgb]{0.9,0.9,0.9}{%
	\makecell[{{p{\linewidth}}}]{% 
	  \tt {\tiny [A$\langle$GM]}  
	 Your message has been relayed to the other player.\\ \tt \\ \tt \\ \tt You are currently playing round 12 of maximum 28.\\ \tt You have collectively accumulated 8 of 16 penalties.\\ \tt The other player sent this message:\\ \tt \\ \tt \\ \tt "Aligning 'O' sounds like a good next step. Since 'O' is at (2,3) on my board and (5,5) on your board, we could try to find a middle ground or a position that works for both boards. How about we aim to move both 'O' objects to a position like (4,4) or another spot that could be feasible for both boards? We can discuss the specific target position and come up with a plan to move 'O' to that spot."\\ \tt \\ \tt \\ \tt What is your next command? 
	  } 
	   } 
	   } 
	 & \\ 
 

    \theutterance \stepcounter{utterance}  

    & \multicolumn{4}{p{0.6\linewidth}}{\cellcolor[rgb]{0.8,1,0.9}{%
	\makecell[{{p{\linewidth}}}]{% 
	  \tt {\tiny [A$\rangle$GM]}  
	 SAY: Finding a middle ground for 'O' is a great idea. I like the suggestion of aiming for a position like (4,4) or another spot that could work for both boards. Before we decide on a specific target position, let me check the availability of (4,4) on my board. If it's empty, it could be a good candidate. If not, we can discuss other options. Can you please also check the availability of (4,4) on your board and let me know if it's empty or occupied? This will help us determine if (4,4) is a viable target position for 'O'. 
	  } 
	   } 
	   } 
	 & & \\ 
 

    \theutterance \stepcounter{utterance}  

    & & & \multicolumn{2}{p{0.3\linewidth}}{\cellcolor[rgb]{0.95,0.95,0.95}{%
	\makecell[{{p{\linewidth}}}]{% 
	  \tt {\tiny [GM$|$GM]}  
	 SAY: Finding a middle ground for 'O' is a great idea. I like the suggestion of aiming for a position like (4,4) or another spot that could work for both boards. Before we decide on a specific target position, let me check the availability of (4,4) on my board. If it's empty, it could be a good candidate. If not, we can discuss other options. Can you please also check the availability of (4,4) on your board and let me know if it's empty or occupied? This will help us determine if (4,4) is a viable target position for 'O'. 
	  } 
	   } 
	   } 
	 & & \\ 
 

    \theutterance \stepcounter{utterance}  

    & & \multicolumn{4}{p{0.6\linewidth}}{\cellcolor[rgb]{0.9,0.9,0.9}{%
	\makecell[{{p{\linewidth}}}]{% 
	  \tt {\tiny [GM$\rangle$B]}  
	 Your message has been relayed to the other player.\\ \tt \\ \tt \\ \tt You are currently playing round 13 of maximum 28.\\ \tt You have collectively accumulated 8 of 16 penalties.\\ \tt The other player sent this message:\\ \tt \\ \tt \\ \tt "Finding a middle ground for 'O' is a great idea. I like the suggestion of aiming for a position like (4,4) or another spot that could work for both boards. Before we decide on a specific target position, let me check the availability of (4,4) on my board. If it's empty, it could be a good candidate. If not, we can discuss other options. Can you please also check the availability of (4,4) on your board and let me know if it's empty or occupied? This will help us determine if (4,4) is a viable target position for 'O'."\\ \tt \\ \tt \\ \tt What is your next command? 
	  } 
	   } 
	   } 
	 & \\ 
 

    \theutterance \stepcounter{utterance}  

    & & & \multicolumn{4}{p{0.6\linewidth}}{\cellcolor[rgb]{1,0.85,0.72}{%
	\makecell[{{p{\linewidth}}}]{% 
	  \tt {\tiny [GM$\langle$B]}  
	 SAY: I've checked the availability of (4,4) on my board, and it's empty, containing only '◌'. It looks like (4,4) could be a viable target position for 'O' on my end. Please let me know if it's available on your board as well, and we can proceed with planning the move for 'O'. Also, I still need to move my 'M' to (5,6) as previously discussed, should I proceed with that move before or after aligning 'O'? 
	  } 
	   } 
	   } 
	  \\ 
 

    \theutterance \stepcounter{utterance}  

    & & & \multicolumn{2}{p{0.3\linewidth}}{\cellcolor[rgb]{0.95,0.95,0.95}{%
	\makecell[{{p{\linewidth}}}]{% 
	  \tt {\tiny [GM$|$GM]}  
	 SAY: I've checked the availability of (4,4) on my board, and it's empty, containing only '◌'. It looks like (4,4) could be a viable target position for 'O' on my end. Please let me know if it's available on your board as well, and we can proceed with planning the move for 'O'. Also, I still need to move my 'M' to (5,6) as previously discussed, should I proceed with that move before or after aligning 'O'? 
	  } 
	   } 
	   } 
	 & & \\ 
 

    \theutterance \stepcounter{utterance}  

    & & \multicolumn{4}{p{0.6\linewidth}}{\cellcolor[rgb]{0.9,0.9,0.9}{%
	\makecell[{{p{\linewidth}}}]{% 
	  \tt {\tiny [A$\langle$GM]}  
	 Your message has been relayed to the other player.\\ \tt \\ \tt \\ \tt You are currently playing round 13 of maximum 28.\\ \tt You have collectively accumulated 8 of 16 penalties.\\ \tt The other player sent this message:\\ \tt \\ \tt \\ \tt "I've checked the availability of (4,4) on my board, and it's empty, containing only '◌'. It looks like (4,4) could be a viable target position for 'O' on my end. Please let me know if it's available on your board as well, and we can proceed with planning the move for 'O'. Also, I still need to move my 'M' to (5,6) as previously discussed, should I proceed with that move before or after aligning 'O'?"\\ \tt \\ \tt \\ \tt What is your next command? 
	  } 
	   } 
	   } 
	 & \\ 
 

    \theutterance \stepcounter{utterance}  

    & \multicolumn{4}{p{0.6\linewidth}}{\cellcolor[rgb]{0.8,1,0.9}{%
	\makecell[{{p{\linewidth}}}]{% 
	  \tt {\tiny [A$\rangle$GM]}  
	 SAY: I've checked the availability of (4,4) on my board, and it's also empty, containing only '◌'. So, (4,4) is a viable target position for 'O' on both boards. Regarding the move of your 'M' to (5,6), I think it would be best to proceed with that move before aligning 'O'. This is because moving 'M' to (5,6) might affect the availability of certain paths or positions for 'O', so it's better to get that move out of the way first. Please go ahead and move your 'M' to (5,6), and then we can focus on planning the move for 'O' to (4,4). 
	  } 
	   } 
	   } 
	 & & \\ 
 

    \theutterance \stepcounter{utterance}  

    & & & \multicolumn{2}{p{0.3\linewidth}}{\cellcolor[rgb]{0.95,0.95,0.95}{%
	\makecell[{{p{\linewidth}}}]{% 
	  \tt {\tiny [GM$|$GM]}  
	 SAY: I've checked the availability of (4,4) on my board, and it's also empty, containing only '◌'. So, (4,4) is a viable target position for 'O' on both boards. Regarding the move of your 'M' to (5,6), I think it would be best to proceed with that move before aligning 'O'. This is because moving 'M' to (5,6) might affect the availability of certain paths or positions for 'O', so it's better to get that move out of the way first. Please go ahead and move your 'M' to (5,6), and then we can focus on planning the move for 'O' to (4,4). 
	  } 
	   } 
	   } 
	 & & \\ 
 

    \theutterance \stepcounter{utterance}  

    & & \multicolumn{4}{p{0.6\linewidth}}{\cellcolor[rgb]{0.9,0.9,0.9}{%
	\makecell[{{p{\linewidth}}}]{% 
	  \tt {\tiny [GM$\rangle$B]}  
	 Your message has been relayed to the other player.\\ \tt \\ \tt \\ \tt You are currently playing round 14 of maximum 28.\\ \tt You have collectively accumulated 8 of 16 penalties.\\ \tt The other player sent this message:\\ \tt \\ \tt \\ \tt "I've checked the availability of (4,4) on my board, and it's also empty, containing only '◌'. So, (4,4) is a viable target position for 'O' on both boards. Regarding the move of your 'M' to (5,6), I think it would be best to proceed with that move before aligning 'O'. This is because moving 'M' to (5,6) might affect the availability of certain paths or positions for 'O', so it's better to get that move out of the way first. Please go ahead and move your 'M' to (5,6), and then we can focus on planning the move for 'O' to (4,4)."\\ \tt \\ \tt \\ \tt What is your next command? 
	  } 
	   } 
	   } 
	 & \\ 
 

    \theutterance \stepcounter{utterance}  

    & & & \multicolumn{4}{p{0.6\linewidth}}{\cellcolor[rgb]{1,0.85,0.72}{%
	\makecell[{{p{\linewidth}}}]{% 
	  \tt {\tiny [GM$\langle$B]}  
	 MOVE: M, (5,6) 
	  } 
	   } 
	   } 
	  \\ 
 

    \theutterance \stepcounter{utterance}  

    & & & \multicolumn{2}{p{0.3\linewidth}}{\cellcolor[rgb]{0.95,0.95,0.95}{%
	\makecell[{{p{\linewidth}}}]{% 
	  \tt {\tiny [GM$|$GM]}  
	 MOVE: M, (5,6) 
	  } 
	   } 
	   } 
	 & & \\ 
 

    \theutterance \stepcounter{utterance}  

    & & & \multicolumn{2}{p{0.3\linewidth}}{\cellcolor[rgb]{0.95,0.95,0.95}{%
	\makecell[{{p{\linewidth}}}]{% 
	  \tt {\tiny [GM$|$GM]}  
	 Penalty: (5,6) is not empty, but contains '─'.\\ \tt \\ \tt You have collectively accumulated 9 of 16 penalties.\\ \tt Please try again! 
	  } 
	   } 
	   } 
	 & & \\ 
 

    \theutterance \stepcounter{utterance}  

    & & \multicolumn{4}{p{0.6\linewidth}}{\cellcolor[rgb]{0.9,0.9,0.9}{%
	\makecell[{{p{\linewidth}}}]{% 
	  \tt {\tiny [GM$\rangle$B]}  
	 Penalty: (5,6) is not empty, but contains '─'.\\ \tt \\ \tt You have collectively accumulated 9 of 16 penalties.\\ \tt Please try again! 
	  } 
	   } 
	   } 
	 & \\ 
 

    \theutterance \stepcounter{utterance}  

    & & & \multicolumn{4}{p{0.6\linewidth}}{\cellcolor[rgb]{1,0.85,0.72}{%
	\makecell[{{p{\linewidth}}}]{% 
	  \tt {\tiny [GM$\langle$B]}  
	 MOVE: M, (6,5) 
	  } 
	   } 
	   } 
	  \\ 
 

    \theutterance \stepcounter{utterance}  

    & & & \multicolumn{2}{p{0.3\linewidth}}{\cellcolor[rgb]{0.95,0.95,0.95}{%
	\makecell[{{p{\linewidth}}}]{% 
	  \tt {\tiny [GM$|$GM]}  
	 MOVE: M, (6,5) 
	  } 
	   } 
	   } 
	 & & \\ 
 

    \theutterance \stepcounter{utterance}  

    & & & \multicolumn{2}{p{0.3\linewidth}}{\cellcolor[rgb]{0.95,0.95,0.95}{%
	\makecell[{{p{\linewidth}}}]{% 
	  \tt {\tiny [GM$|$GM]}  
	 Penalty: (6,5) is not empty, but contains '│'.\\ \tt \\ \tt You have collectively accumulated 10 of 16 penalties.\\ \tt Please try again! 
	  } 
	   } 
	   } 
	 & & \\ 
 

    \theutterance \stepcounter{utterance}  

    & & \multicolumn{4}{p{0.6\linewidth}}{\cellcolor[rgb]{0.9,0.9,0.9}{%
	\makecell[{{p{\linewidth}}}]{% 
	  \tt {\tiny [GM$\rangle$B]}  
	 Penalty: (6,5) is not empty, but contains '│'.\\ \tt \\ \tt You have collectively accumulated 10 of 16 penalties.\\ \tt Please try again! 
	  } 
	   } 
	   } 
	 & \\ 
 

    \theutterance \stepcounter{utterance}  

    & & & \multicolumn{4}{p{0.6\linewidth}}{\cellcolor[rgb]{1,0.85,0.72}{%
	\makecell[{{p{\linewidth}}}]{% 
	  \tt {\tiny [GM$\langle$B]}  
	 SAY: It seems that moving 'M' to (5,6) or nearby positions is not possible due to the grid structure. Let me try to find an alternative position for 'M' that is empty and can be moved to. Can we discuss other options for moving 'M' before proceeding with aligning 'O'? 
	  } 
	   } 
	   } 
	  \\ 
 

    \theutterance \stepcounter{utterance}  

    & & & \multicolumn{2}{p{0.3\linewidth}}{\cellcolor[rgb]{0.95,0.95,0.95}{%
	\makecell[{{p{\linewidth}}}]{% 
	  \tt {\tiny [GM$|$GM]}  
	 SAY: It seems that moving 'M' to (5,6) or nearby positions is not possible due to the grid structure. Let me try to find an alternative position for 'M' that is empty and can be moved to. Can we discuss other options for moving 'M' before proceeding with aligning 'O'? 
	  } 
	   } 
	   } 
	 & & \\ 
 

    \theutterance \stepcounter{utterance}  

    & & \multicolumn{4}{p{0.6\linewidth}}{\cellcolor[rgb]{0.9,0.9,0.9}{%
	\makecell[{{p{\linewidth}}}]{% 
	  \tt {\tiny [A$\langle$GM]}  
	 Your message has been relayed to the other player.\\ \tt \\ \tt \\ \tt You are currently playing round 14 of maximum 28.\\ \tt You have collectively accumulated 10 of 16 penalties.\\ \tt The other player sent this message:\\ \tt \\ \tt \\ \tt "It seems that moving 'M' to (5,6) or nearby positions is not possible due to the grid structure. Let me try to find an alternative position for 'M' that is empty and can be moved to. Can we discuss other options for moving 'M' before proceeding with aligning 'O'?"\\ \tt \\ \tt \\ \tt What is your next command? 
	  } 
	   } 
	   } 
	 & \\ 
 

    \theutterance \stepcounter{utterance}  

    & \multicolumn{4}{p{0.6\linewidth}}{\cellcolor[rgb]{0.8,1,0.9}{%
	\makecell[{{p{\linewidth}}}]{% 
	  \tt {\tiny [A$\rangle$GM]}  
	 SAY: It seems like moving 'M' to (5,6) is not feasible due to the grid structure. Let's explore other options for moving 'M'. Since we previously discussed moving 'M' to a position that matches mine, which is at (4,7), but that spot is not available on your board, we could consider other positions that might be closer to (4,7). How about we try to find an empty spot on your board that is relatively close to (4,7), such as (3,7), (5,7), or even (4,5) or (4,6) if they are available? We can discuss the pros and cons of each option and choose the best alternative for 'M'. Once we've moved 'M' to a suitable position, we can refocus on aligning 'O' to (4,4). 
	  } 
	   } 
	   } 
	 & & \\ 
 

    \theutterance \stepcounter{utterance}  

    & & & \multicolumn{2}{p{0.3\linewidth}}{\cellcolor[rgb]{0.95,0.95,0.95}{%
	\makecell[{{p{\linewidth}}}]{% 
	  \tt {\tiny [GM$|$GM]}  
	 SAY: It seems like moving 'M' to (5,6) is not feasible due to the grid structure. Let's explore other options for moving 'M'. Since we previously discussed moving 'M' to a position that matches mine, which is at (4,7), but that spot is not available on your board, we could consider other positions that might be closer to (4,7). How about we try to find an empty spot on your board that is relatively close to (4,7), such as (3,7), (5,7), or even (4,5) or (4,6) if they are available? We can discuss the pros and cons of each option and choose the best alternative for 'M'. Once we've moved 'M' to a suitable position, we can refocus on aligning 'O' to (4,4). 
	  } 
	   } 
	   } 
	 & & \\ 
 

    \theutterance \stepcounter{utterance}  

    & & \multicolumn{4}{p{0.6\linewidth}}{\cellcolor[rgb]{0.9,0.9,0.9}{%
	\makecell[{{p{\linewidth}}}]{% 
	  \tt {\tiny [GM$\rangle$B]}  
	 Your message has been relayed to the other player.\\ \tt \\ \tt \\ \tt You are currently playing round 15 of maximum 28.\\ \tt You have collectively accumulated 10 of 16 penalties.\\ \tt The other player sent this message:\\ \tt \\ \tt \\ \tt "It seems like moving 'M' to (5,6) is not feasible due to the grid structure. Let's explore other options for moving 'M'. Since we previously discussed moving 'M' to a position that matches mine, which is at (4,7), but that spot is not available on your board, we could consider other positions that might be closer to (4,7). How about we try to find an empty spot on your board that is relatively close to (4,7), such as (3,7), (5,7), or even (4,5) or (4,6) if they are available? We can discuss the pros and cons of each option and choose the best alternative for 'M'. Once we've moved 'M' to a suitable position, we can refocus on aligning 'O' to (4,4)."\\ \tt \\ \tt \\ \tt What is your next command? 
	  } 
	   } 
	   } 
	 & \\ 
 

    \theutterance \stepcounter{utterance}  

    & & & \multicolumn{4}{p{0.6\linewidth}}{\cellcolor[rgb]{1,0.85,0.72}{%
	\makecell[{{p{\linewidth}}}]{% 
	  \tt {\tiny [GM$\langle$B]}  
	 SAY: I can try moving 'M' to one of the positions you suggested, but I need to check their availability on my board. (3,7) is currently occupied by 'P', which we aligned earlier. (5,7) contains '║', so it's not available. (4,5) contains 'T', and (4,6) contains '┤', so neither of those positions is available either. However, I can try moving 'M' to a different position that is empty. How about I move 'M' to a position like (2,6) or another empty cell that could work as an alternative? Should I try one of those positions or explore other options? 
	  } 
	   } 
	   } 
	  \\ 
 

    \theutterance \stepcounter{utterance}  

    & & & \multicolumn{2}{p{0.3\linewidth}}{\cellcolor[rgb]{0.95,0.95,0.95}{%
	\makecell[{{p{\linewidth}}}]{% 
	  \tt {\tiny [GM$|$GM]}  
	 SAY: I can try moving 'M' to one of the positions you suggested, but I need to check their availability on my board. (3,7) is currently occupied by 'P', which we aligned earlier. (5,7) contains '║', so it's not available. (4,5) contains 'T', and (4,6) contains '┤', so neither of those positions is available either. However, I can try moving 'M' to a different position that is empty. How about I move 'M' to a position like (2,6) or another empty cell that could work as an alternative? Should I try one of those positions or explore other options? 
	  } 
	   } 
	   } 
	 & & \\ 
 

    \theutterance \stepcounter{utterance}  

    & & \multicolumn{4}{p{0.6\linewidth}}{\cellcolor[rgb]{0.9,0.9,0.9}{%
	\makecell[{{p{\linewidth}}}]{% 
	  \tt {\tiny [A$\langle$GM]}  
	 Your message has been relayed to the other player.\\ \tt \\ \tt \\ \tt You are currently playing round 15 of maximum 28.\\ \tt You have collectively accumulated 10 of 16 penalties.\\ \tt The other player sent this message:\\ \tt \\ \tt \\ \tt "I can try moving 'M' to one of the positions you suggested, but I need to check their availability on my board. (3,7) is currently occupied by 'P', which we aligned earlier. (5,7) contains '║', so it's not available. (4,5) contains 'T', and (4,6) contains '┤', so neither of those positions is available either. However, I can try moving 'M' to a different position that is empty. How about I move 'M' to a position like (2,6) or another empty cell that could work as an alternative? Should I try one of those positions or explore other options?"\\ \tt \\ \tt \\ \tt What is your next command? 
	  } 
	   } 
	   } 
	 & \\ 
 

    \theutterance \stepcounter{utterance}  

    & \multicolumn{4}{p{0.6\linewidth}}{\cellcolor[rgb]{0.8,1,0.9}{%
	\makecell[{{p{\linewidth}}}]{% 
	  \tt {\tiny [A$\rangle$GM]}  
	 SAY: Thank you for checking the availability of those positions. It seems like most of the positions I suggested are not available on your board. Moving 'M' to (2,6) could be a viable alternative. Before we decide on that, let me consider the implications of having 'M' at (2,6) on our overall alignment strategy. Since 'M' is currently at (7,7) on your board, moving it to (2,6) would be a significant change. How would this move affect the potential alignment of other objects, such as 'T', 'O', 'D', 'S', and 'A'? Are there any potential conflicts or opportunities that we should consider before making this move? Additionally, are you sure that (2,6) is empty on your board and not occupied by any other object or grid structure element? 
	  } 
	   } 
	   } 
	 & & \\ 
 

    \theutterance \stepcounter{utterance}  

    & & & \multicolumn{2}{p{0.3\linewidth}}{\cellcolor[rgb]{0.95,0.95,0.95}{%
	\makecell[{{p{\linewidth}}}]{% 
	  \tt {\tiny [GM$|$GM]}  
	 SAY: Thank you for checking the availability of those positions. It seems like most of the positions I suggested are not available on your board. Moving 'M' to (2,6) could be a viable alternative. Before we decide on that, let me consider the implications of having 'M' at (2,6) on our overall alignment strategy. Since 'M' is currently at (7,7) on your board, moving it to (2,6) would be a significant change. How would this move affect the potential alignment of other objects, such as 'T', 'O', 'D', 'S', and 'A'? Are there any potential conflicts or opportunities that we should consider before making this move? Additionally, are you sure that (2,6) is empty on your board and not occupied by any other object or grid structure element? 
	  } 
	   } 
	   } 
	 & & \\ 
 

    \theutterance \stepcounter{utterance}  

    & & \multicolumn{4}{p{0.6\linewidth}}{\cellcolor[rgb]{0.9,0.9,0.9}{%
	\makecell[{{p{\linewidth}}}]{% 
	  \tt {\tiny [GM$\rangle$B]}  
	 Your message has been relayed to the other player.\\ \tt \\ \tt \\ \tt You are currently playing round 16 of maximum 28.\\ \tt You have collectively accumulated 10 of 16 penalties.\\ \tt The other player sent this message:\\ \tt \\ \tt \\ \tt "Thank you for checking the availability of those positions. It seems like most of the positions I suggested are not available on your board. Moving 'M' to (2,6) could be a viable alternative. Before we decide on that, let me consider the implications of having 'M' at (2,6) on our overall alignment strategy. Since 'M' is currently at (7,7) on your board, moving it to (2,6) would be a significant change. How would this move affect the potential alignment of other objects, such as 'T', 'O', 'D', 'S', and 'A'? Are there any potential conflicts or opportunities that we should consider before making this move? Additionally, are you sure that (2,6) is empty on your board and not occupied by any other object or grid structure element?"\\ \tt \\ \tt \\ \tt What is your next command? 
	  } 
	   } 
	   } 
	 & \\ 
 

    \theutterance \stepcounter{utterance}  

    & & & \multicolumn{4}{p{0.6\linewidth}}{\cellcolor[rgb]{1,0.85,0.72}{%
	\makecell[{{p{\linewidth}}}]{% 
	  \tt {\tiny [GM$\langle$B]}  
	 SAY: I've double-checked my board, and (2,6) is indeed empty, containing only '◌'. Moving 'M' to (2,6) could potentially create opportunities for aligning other objects, such as 'T' or 'D', since it would be moving 'M' to a more central position on the board. However, it could also create conflicts or limitations for aligning 'O' or 'A', depending on their potential movement paths. Considering our goal is to align all objects, I think it's worth exploring this move and adjusting our strategy as needed. Before making the move, I'd like to confirm: are you ready to proceed with aligning 'O' to (4,4) after I move 'M' to (2,6), or do you want to reassess our strategy after the move? 
	  } 
	   } 
	   } 
	  \\ 
 

    \theutterance \stepcounter{utterance}  

    & & & \multicolumn{2}{p{0.3\linewidth}}{\cellcolor[rgb]{0.95,0.95,0.95}{%
	\makecell[{{p{\linewidth}}}]{% 
	  \tt {\tiny [GM$|$GM]}  
	 SAY: I've double-checked my board, and (2,6) is indeed empty, containing only '◌'. Moving 'M' to (2,6) could potentially create opportunities for aligning other objects, such as 'T' or 'D', since it would be moving 'M' to a more central position on the board. However, it could also create conflicts or limitations for aligning 'O' or 'A', depending on their potential movement paths. Considering our goal is to align all objects, I think it's worth exploring this move and adjusting our strategy as needed. Before making the move, I'd like to confirm: are you ready to proceed with aligning 'O' to (4,4) after I move 'M' to (2,6), or do you want to reassess our strategy after the move? 
	  } 
	   } 
	   } 
	 & & \\ 
 

    \theutterance \stepcounter{utterance}  

    & & \multicolumn{4}{p{0.6\linewidth}}{\cellcolor[rgb]{0.9,0.9,0.9}{%
	\makecell[{{p{\linewidth}}}]{% 
	  \tt {\tiny [A$\langle$GM]}  
	 Your message has been relayed to the other player.\\ \tt \\ \tt \\ \tt You are currently playing round 16 of maximum 28.\\ \tt You have collectively accumulated 10 of 16 penalties.\\ \tt The other player sent this message:\\ \tt \\ \tt \\ \tt "I've double-checked my board, and (2,6) is indeed empty, containing only '◌'. Moving 'M' to (2,6) could potentially create opportunities for aligning other objects, such as 'T' or 'D', since it would be moving 'M' to a more central position on the board. However, it could also create conflicts or limitations for aligning 'O' or 'A', depending on their potential movement paths. Considering our goal is to align all objects, I think it's worth exploring this move and adjusting our strategy as needed. Before making the move, I'd like to confirm: are you ready to proceed with aligning 'O' to (4,4) after I move 'M' to (2,6), or do you want to reassess our strategy after the move?"\\ \tt \\ \tt \\ \tt What is your next command? 
	  } 
	   } 
	   } 
	 & \\ 
 

    \theutterance \stepcounter{utterance}  

    & \multicolumn{4}{p{0.6\linewidth}}{\cellcolor[rgb]{0.8,1,0.9}{%
	\makecell[{{p{\linewidth}}}]{% 
	  \tt {\tiny [A$\rangle$GM]}  
	 SAY: I'm glad we've confirmed that (2,6) is empty on your board. Considering the potential opportunities and conflicts that moving 'M' to (2,6) could create, I think it's a good idea to proceed with the move and reassess our strategy afterwards. Considerable progress has been made in our conversation so far, but the number of rounds is also increasing. Please go ahead and move your 'M' to (2,6). After the move, let's reassess our strategy and decide on the next steps, including potentially aligning 'O' to (4,4) or focusing on other objects. We can adjust our plan as needed to ensure we're making progress towards our goal of aligning all objects within the remaining rounds. 
	  } 
	   } 
	   } 
	 & & \\ 
 

    \theutterance \stepcounter{utterance}  

    & & & \multicolumn{2}{p{0.3\linewidth}}{\cellcolor[rgb]{0.95,0.95,0.95}{%
	\makecell[{{p{\linewidth}}}]{% 
	  \tt {\tiny [GM$|$GM]}  
	 SAY: I'm glad we've confirmed that (2,6) is empty on your board. Considering the potential opportunities and conflicts that moving 'M' to (2,6) could create, I think it's a good idea to proceed with the move and reassess our strategy afterwards. Considerable progress has been made in our conversation so far, but the number of rounds is also increasing. Please go ahead and move your 'M' to (2,6). After the move, let's reassess our strategy and decide on the next steps, including potentially aligning 'O' to (4,4) or focusing on other objects. We can adjust our plan as needed to ensure we're making progress towards our goal of aligning all objects within the remaining rounds. 
	  } 
	   } 
	   } 
	 & & \\ 
 

    \theutterance \stepcounter{utterance}  

    & & \multicolumn{4}{p{0.6\linewidth}}{\cellcolor[rgb]{0.9,0.9,0.9}{%
	\makecell[{{p{\linewidth}}}]{% 
	  \tt {\tiny [GM$\rangle$B]}  
	 Your message has been relayed to the other player.\\ \tt \\ \tt \\ \tt You are currently playing round 17 of maximum 28.\\ \tt You have collectively accumulated 10 of 16 penalties.\\ \tt The other player sent this message:\\ \tt \\ \tt \\ \tt "I'm glad we've confirmed that (2,6) is empty on your board. Considering the potential opportunities and conflicts that moving 'M' to (2,6) could create, I think it's a good idea to proceed with the move and reassess our strategy afterwards. Considerable progress has been made in our conversation so far, but the number of rounds is also increasing. Please go ahead and move your 'M' to (2,6). After the move, let's reassess our strategy and decide on the next steps, including potentially aligning 'O' to (4,4) or focusing on other objects. We can adjust our plan as needed to ensure we're making progress towards our goal of aligning all objects within the remaining rounds."\\ \tt \\ \tt \\ \tt What is your next command? 
	  } 
	   } 
	   } 
	 & \\ 
 

    \theutterance \stepcounter{utterance}  

    & & & \multicolumn{4}{p{0.6\linewidth}}{\cellcolor[rgb]{1,0.85,0.72}{%
	\makecell[{{p{\linewidth}}}]{% 
	  \tt {\tiny [GM$\langle$B]}  
	 MOVE: M, (2,6) 
	  } 
	   } 
	   } 
	  \\ 
 

    \theutterance \stepcounter{utterance}  

    & & & \multicolumn{2}{p{0.3\linewidth}}{\cellcolor[rgb]{0.95,0.95,0.95}{%
	\makecell[{{p{\linewidth}}}]{% 
	  \tt {\tiny [GM$|$GM]}  
	 MOVE: M, (2,6) 
	  } 
	   } 
	   } 
	 & & \\ 
 

    \theutterance \stepcounter{utterance}  

    & & & \multicolumn{2}{p{0.3\linewidth}}{\cellcolor[rgb]{0.95,0.95,0.95}{%
	\makecell[{{p{\linewidth}}}]{% 
	  \tt {\tiny [GM$|$GM]}  
	 Penalty: (2,6) is not empty, but contains '┤'.\\ \tt \\ \tt You have collectively accumulated 11 of 16 penalties.\\ \tt Please try again! 
	  } 
	   } 
	   } 
	 & & \\ 
 

    \theutterance \stepcounter{utterance}  

    & & \multicolumn{4}{p{0.6\linewidth}}{\cellcolor[rgb]{0.9,0.9,0.9}{%
	\makecell[{{p{\linewidth}}}]{% 
	  \tt {\tiny [GM$\rangle$B]}  
	 Penalty: (2,6) is not empty, but contains '┤'.\\ \tt \\ \tt You have collectively accumulated 11 of 16 penalties.\\ \tt Please try again! 
	  } 
	   } 
	   } 
	 & \\ 
 

    \theutterance \stepcounter{utterance}  

    & & & \multicolumn{4}{p{0.6\linewidth}}{\cellcolor[rgb]{1,0.85,0.72}{%
	\makecell[{{p{\linewidth}}}]{% 
	  \tt {\tiny [GM$\langle$B]}  
	 MOVE: M, (2,5) 
	  } 
	   } 
	   } 
	  \\ 
 

    \theutterance \stepcounter{utterance}  

    & & & \multicolumn{2}{p{0.3\linewidth}}{\cellcolor[rgb]{0.95,0.95,0.95}{%
	\makecell[{{p{\linewidth}}}]{% 
	  \tt {\tiny [GM$|$GM]}  
	 MOVE: M, (2,5) 
	  } 
	   } 
	   } 
	 & & \\ 
 

    \theutterance \stepcounter{utterance}  

    & & & \multicolumn{2}{p{0.3\linewidth}}{\cellcolor[rgb]{0.95,0.95,0.95}{%
	\makecell[{{p{\linewidth}}}]{% 
	  \tt {\tiny [GM$|$GM]}  
	 Penalty: (2,5) is not empty, but contains '├'.\\ \tt \\ \tt You have collectively accumulated 12 of 16 penalties.\\ \tt Please try again! 
	  } 
	   } 
	   } 
	 & & \\ 
 

    \theutterance \stepcounter{utterance}  

    & & \multicolumn{4}{p{0.6\linewidth}}{\cellcolor[rgb]{0.9,0.9,0.9}{%
	\makecell[{{p{\linewidth}}}]{% 
	  \tt {\tiny [GM$\rangle$B]}  
	 Penalty: (2,5) is not empty, but contains '├'.\\ \tt \\ \tt You have collectively accumulated 12 of 16 penalties.\\ \tt Please try again! 
	  } 
	   } 
	   } 
	 & \\ 
 

    \theutterance \stepcounter{utterance}  

    & & & \multicolumn{4}{p{0.6\linewidth}}{\cellcolor[rgb]{1,0.85,0.72}{%
	\makecell[{{p{\linewidth}}}]{% 
	  \tt {\tiny [GM$\langle$B]}  
	 SAY: It seems that moving 'M' to (2,6) or (2,5) is not possible due to the grid structure. Let me try to find another empty position for 'M'. Can we discuss other options for moving 'M' before proceeding with aligning 'O'? 
	  } 
	   } 
	   } 
	  \\ 
 

    \theutterance \stepcounter{utterance}  

    & & & \multicolumn{2}{p{0.3\linewidth}}{\cellcolor[rgb]{0.95,0.95,0.95}{%
	\makecell[{{p{\linewidth}}}]{% 
	  \tt {\tiny [GM$|$GM]}  
	 SAY: It seems that moving 'M' to (2,6) or (2,5) is not possible due to the grid structure. Let me try to find another empty position for 'M'. Can we discuss other options for moving 'M' before proceeding with aligning 'O'? 
	  } 
	   } 
	   } 
	 & & \\ 
 

    \theutterance \stepcounter{utterance}  

    & & \multicolumn{4}{p{0.6\linewidth}}{\cellcolor[rgb]{0.9,0.9,0.9}{%
	\makecell[{{p{\linewidth}}}]{% 
	  \tt {\tiny [A$\langle$GM]}  
	 Your message has been relayed to the other player.\\ \tt \\ \tt \\ \tt You are currently playing round 17 of maximum 28.\\ \tt You have collectively accumulated 12 of 16 penalties.\\ \tt The other player sent this message:\\ \tt \\ \tt \\ \tt "It seems that moving 'M' to (2,6) or (2,5) is not possible due to the grid structure. Let me try to find another empty position for 'M'. Can we discuss other options for moving 'M' before proceeding with aligning 'O'?"\\ \tt \\ \tt \\ \tt What is your next command? 
	  } 
	   } 
	   } 
	 & \\ 
 

    \theutterance \stepcounter{utterance}  

    & \multicolumn{4}{p{0.6\linewidth}}{\cellcolor[rgb]{0.8,1,0.9}{%
	\makecell[{{p{\linewidth}}}]{% 
	  \tt {\tiny [A$\rangle$GM]}  
	 SAY: It seems like we're running into another obstacle with moving 'M'. Don't worry, let's explore other options. At this point, I think it's essential to consider the overall state of our boards and the objects that are still not aligned. Instead of focusing solely on moving 'M', let's take a step back and look at the bigger picture. Which objects do you think would be the easiest to align next, considering their current positions on both boards? Perhaps we can make progress with 'O', 'T', 'D', 'S', or 'A' while we continue to figure out a suitable position for 'M'. Can you give me an update on the current positions of these objects on your board, and we can discuss potential next steps? 
	  } 
	   } 
	   } 
	 & & \\ 
 

    \theutterance \stepcounter{utterance}  

    & & & \multicolumn{2}{p{0.3\linewidth}}{\cellcolor[rgb]{0.95,0.95,0.95}{%
	\makecell[{{p{\linewidth}}}]{% 
	  \tt {\tiny [GM$|$GM]}  
	 SAY: It seems like we're running into another obstacle with moving 'M'. Don't worry, let's explore other options. At this point, I think it's essential to consider the overall state of our boards and the objects that are still not aligned. Instead of focusing solely on moving 'M', let's take a step back and look at the bigger picture. Which objects do you think would be the easiest to align next, considering their current positions on both boards? Perhaps we can make progress with 'O', 'T', 'D', 'S', or 'A' while we continue to figure out a suitable position for 'M'. Can you give me an update on the current positions of these objects on your board, and we can discuss potential next steps? 
	  } 
	   } 
	   } 
	 & & \\ 
 

    \theutterance \stepcounter{utterance}  

    & & \multicolumn{4}{p{0.6\linewidth}}{\cellcolor[rgb]{0.9,0.9,0.9}{%
	\makecell[{{p{\linewidth}}}]{% 
	  \tt {\tiny [GM$\rangle$B]}  
	 Your message has been relayed to the other player.\\ \tt \\ \tt \\ \tt You are currently playing round 18 of maximum 28.\\ \tt You have collectively accumulated 12 of 16 penalties.\\ \tt The other player sent this message:\\ \tt \\ \tt \\ \tt "It seems like we're running into another obstacle with moving 'M'. Don't worry, let's explore other options. At this point, I think it's essential to consider the overall state of our boards and the objects that are still not aligned. Instead of focusing solely on moving 'M', let's take a step back and look at the bigger picture. Which objects do you think would be the easiest to align next, considering their current positions on both boards? Perhaps we can make progress with 'O', 'T', 'D', 'S', or 'A' while we continue to figure out a suitable position for 'M'. Can you give me an update on the current positions of these objects on your board, and we can discuss potential next steps?"\\ \tt \\ \tt \\ \tt What is your next command? 
	  } 
	   } 
	   } 
	 & \\ 
 

    \theutterance \stepcounter{utterance}  

    & & & \multicolumn{4}{p{0.6\linewidth}}{\cellcolor[rgb]{1,0.85,0.72}{%
	\makecell[{{p{\linewidth}}}]{% 
	  \tt {\tiny [GM$\langle$B]}  
	 SAY: Let me give you an update on the current positions of the objects on my board. 'P' is at (3,7), 'O' is at (2,3), 'T' is at (4,5), 'D' is at (4,5), 'S' is at (1,1), 'A' is at (1,3), and 'M' is at (7,7). Considering these positions, I think 'O' might still be a good candidate to align next, as we previously discussed moving it to (4,4). Alternatively, we could also consider 'S' or 'A', as they are relatively close to each other on my board and might be easier to align. What are your thoughts on these options, and do you have any suggestions for how to proceed? 
	  } 
	   } 
	   } 
	  \\ 
 

    \theutterance \stepcounter{utterance}  

    & & & \multicolumn{2}{p{0.3\linewidth}}{\cellcolor[rgb]{0.95,0.95,0.95}{%
	\makecell[{{p{\linewidth}}}]{% 
	  \tt {\tiny [GM$|$GM]}  
	 SAY: Let me give you an update on the current positions of the objects on my board. 'P' is at (3,7), 'O' is at (2,3), 'T' is at (4,5), 'D' is at (4,5), 'S' is at (1,1), 'A' is at (1,3), and 'M' is at (7,7). Considering these positions, I think 'O' might still be a good candidate to align next, as we previously discussed moving it to (4,4). Alternatively, we could also consider 'S' or 'A', as they are relatively close to each other on my board and might be easier to align. What are your thoughts on these options, and do you have any suggestions for how to proceed? 
	  } 
	   } 
	   } 
	 & & \\ 
 

    \theutterance \stepcounter{utterance}  

    & & \multicolumn{4}{p{0.6\linewidth}}{\cellcolor[rgb]{0.9,0.9,0.9}{%
	\makecell[{{p{\linewidth}}}]{% 
	  \tt {\tiny [A$\langle$GM]}  
	 Your message has been relayed to the other player.\\ \tt \\ \tt \\ \tt You are currently playing round 18 of maximum 28.\\ \tt You have collectively accumulated 12 of 16 penalties.\\ \tt The other player sent this message:\\ \tt \\ \tt \\ \tt "Let me give you an update on the current positions of the objects on my board. 'P' is at (3,7), 'O' is at (2,3), 'T' is at (4,5), 'D' is at (4,5), 'S' is at (1,1), 'A' is at (1,3), and 'M' is at (7,7). Considering these positions, I think 'O' might still be a good candidate to align next, as we previously discussed moving it to (4,4). Alternatively, we could also consider 'S' or 'A', as they are relatively close to each other on my board and might be easier to align. What are your thoughts on these options, and do you have any suggestions for how to proceed?"\\ \tt \\ \tt \\ \tt What is your next command? 
	  } 
	   } 
	   } 
	 & \\ 
 

    \theutterance \stepcounter{utterance}  

    & \multicolumn{4}{p{0.6\linewidth}}{\cellcolor[rgb]{0.8,1,0.9}{%
	\makecell[{{p{\linewidth}}}]{% 
	  \tt {\tiny [A$\rangle$GM]}  
	 SAY: Thank you for the update on the object positions on your board. I agree that 'O' is still a good candidate to align next, and moving it to (4,4) could be a good goal. However, considering the penalties we've accumulated and the number of rounds left, I think we should also explore other options that might be more feasible in the short term. Aligning 'S' or 'A' could be a good alternative, as you mentioned they are relatively close to each other on your board. But before we decide on a specific object to align, I'd like to ask: have you noticed any patterns or commonalities between our boards that could help us make progress? For example, are there any objects that are closer to being aligned than we think, or any grid structures that could be used to our advantage? Let's discuss any potential insights you might have, and then we can decide on the best course of action. 
	  } 
	   } 
	   } 
	 & & \\ 
 

    \theutterance \stepcounter{utterance}  

    & & & \multicolumn{2}{p{0.3\linewidth}}{\cellcolor[rgb]{0.95,0.95,0.95}{%
	\makecell[{{p{\linewidth}}}]{% 
	  \tt {\tiny [GM$|$GM]}  
	 SAY: Thank you for the update on the object positions on your board. I agree that 'O' is still a good candidate to align next, and moving it to (4,4) could be a good goal. However, considering the penalties we've accumulated and the number of rounds left, I think we should also explore other options that might be more feasible in the short term. Aligning 'S' or 'A' could be a good alternative, as you mentioned they are relatively close to each other on your board. But before we decide on a specific object to align, I'd like to ask: have you noticed any patterns or commonalities between our boards that could help us make progress? For example, are there any objects that are closer to being aligned than we think, or any grid structures that could be used to our advantage? Let's discuss any potential insights you might have, and then we can decide on the best course of action. 
	  } 
	   } 
	   } 
	 & & \\ 
 

    \theutterance \stepcounter{utterance}  

    & & \multicolumn{4}{p{0.6\linewidth}}{\cellcolor[rgb]{0.9,0.9,0.9}{%
	\makecell[{{p{\linewidth}}}]{% 
	  \tt {\tiny [GM$\rangle$B]}  
	 Your message has been relayed to the other player.\\ \tt \\ \tt \\ \tt You are currently playing round 19 of maximum 28.\\ \tt You have collectively accumulated 12 of 16 penalties.\\ \tt The other player sent this message:\\ \tt \\ \tt \\ \tt "Thank you for the update on the object positions on your board. I agree that 'O' is still a good candidate to align next, and moving it to (4,4) could be a good goal. However, considering the penalties we've accumulated and the number of rounds left, I think we should also explore other options that might be more feasible in the short term. Aligning 'S' or 'A' could be a good alternative, as you mentioned they are relatively close to each other on your board. But before we decide on a specific object to align, I'd like to ask: have you noticed any patterns or commonalities between our boards that could help us make progress? For example, are there any objects that are closer to being aligned than we think, or any grid structures that could be used to our advantage? Let's discuss any potential insights you might have, and then we can decide on the best course of action."\\ \tt \\ \tt \\ \tt What is your next command? 
	  } 
	   } 
	   } 
	 & \\ 
 

    \theutterance \stepcounter{utterance}  

    & & & \multicolumn{4}{p{0.6\linewidth}}{\cellcolor[rgb]{1,0.85,0.72}{%
	\makecell[{{p{\linewidth}}}]{% 
	  \tt {\tiny [GM$\langle$B]}  
	 SAY: Upon reviewing our conversation and the object positions, I notice that 'S' and 'A' are not only close to each other on my board, but they are also in the same column (column 1) as 'P', which we've already aligned. This could be a useful pattern to exploit. Additionally, I observe that the grid structure, with its various walls and corners, might be limiting our movement options, but it could also be used to our advantage if we can find ways to move objects around these obstacles. I also wonder if there might be other objects that are closer to being aligned than we think, perhaps by using the grid structure to "slide" objects into place. What are your thoughts on these observations, and do you have any similar insights or ideas for how to make progress? 
	  } 
	   } 
	   } 
	  \\ 
 

    \theutterance \stepcounter{utterance}  

    & & & \multicolumn{2}{p{0.3\linewidth}}{\cellcolor[rgb]{0.95,0.95,0.95}{%
	\makecell[{{p{\linewidth}}}]{% 
	  \tt {\tiny [GM$|$GM]}  
	 SAY: Upon reviewing our conversation and the object positions, I notice that 'S' and 'A' are not only close to each other on my board, but they are also in the same column (column 1) as 'P', which we've already aligned. This could be a useful pattern to exploit. Additionally, I observe that the grid structure, with its various walls and corners, might be limiting our movement options, but it could also be used to our advantage if we can find ways to move objects around these obstacles. I also wonder if there might be other objects that are closer to being aligned than we think, perhaps by using the grid structure to "slide" objects into place. What are your thoughts on these observations, and do you have any similar insights or ideas for how to make progress? 
	  } 
	   } 
	   } 
	 & & \\ 
 

    \theutterance \stepcounter{utterance}  

    & & \multicolumn{4}{p{0.6\linewidth}}{\cellcolor[rgb]{0.9,0.9,0.9}{%
	\makecell[{{p{\linewidth}}}]{% 
	  \tt {\tiny [A$\langle$GM]}  
	 Your message has been relayed to the other player.\\ \tt \\ \tt \\ \tt You are currently playing round 19 of maximum 28.\\ \tt You have collectively accumulated 12 of 16 penalties.\\ \tt The other player sent this message:\\ \tt \\ \tt \\ \tt "Upon reviewing our conversation and the object positions, I notice that 'S' and 'A' are not only close to each other on my board, but they are also in the same column (column 1) as 'P', which we've already aligned. This could be a useful pattern to exploit. Additionally, I observe that the grid structure, with its various walls and corners, might be limiting our movement options, but it could also be used to our advantage if we can find ways to move objects around these obstacles. I also wonder if there might be other objects that are closer to being aligned than we think, perhaps by using the grid structure to "slide" objects into place. What are your thoughts on these observations, and do you have any similar insights or ideas for how to make progress?"\\ \tt \\ \tt \\ \tt What is your next command? 
	  } 
	   } 
	   } 
	 & \\ 
 

    \theutterance \stepcounter{utterance}  

    & \multicolumn{4}{p{0.6\linewidth}}{\cellcolor[rgb]{0.8,1,0.9}{%
	\makecell[{{p{\linewidth}}}]{% 
	  \tt {\tiny [A$\rangle$GM]}  
	 SAY: I think you've made some excellent observations about the object positions and the grid structure. The fact that 'S' and 'A' are in the same column as 'P' could be very useful, as it might allow us to create a "chain" of aligned objects. I also like your idea of using the grid structure to "slide" objects into place, as this could help us navigate around obstacles and find new paths for alignment. Considering these insights, I think our next step should be to try to align 'S' or 'A' with their corresponding objects on my board. Since 'S' and 'A' are in column 1 on your board, let's focus on moving one of them to a position that could match the position of the same object on my board. Can you tell me the positions of 'S' and 'A' on my board, so we can discuss potential target positions for alignment? 
	  } 
	   } 
	   } 
	 & & \\ 
 

    \theutterance \stepcounter{utterance}  

    & & & \multicolumn{2}{p{0.3\linewidth}}{\cellcolor[rgb]{0.95,0.95,0.95}{%
	\makecell[{{p{\linewidth}}}]{% 
	  \tt {\tiny [GM$|$GM]}  
	 SAY: I think you've made some excellent observations about the object positions and the grid structure. The fact that 'S' and 'A' are in the same column as 'P' could be very useful, as it might allow us to create a "chain" of aligned objects. I also like your idea of using the grid structure to "slide" objects into place, as this could help us navigate around obstacles and find new paths for alignment. Considering these insights, I think our next step should be to try to align 'S' or 'A' with their corresponding objects on my board. Since 'S' and 'A' are in column 1 on your board, let's focus on moving one of them to a position that could match the position of the same object on my board. Can you tell me the positions of 'S' and 'A' on my board, so we can discuss potential target positions for alignment? 
	  } 
	   } 
	   } 
	 & & \\ 
 

    \theutterance \stepcounter{utterance}  

    & & \multicolumn{4}{p{0.6\linewidth}}{\cellcolor[rgb]{0.9,0.9,0.9}{%
	\makecell[{{p{\linewidth}}}]{% 
	  \tt {\tiny [GM$\rangle$B]}  
	 Your message has been relayed to the other player.\\ \tt \\ \tt \\ \tt You are currently playing round 20 of maximum 28.\\ \tt You have collectively accumulated 12 of 16 penalties.\\ \tt The other player sent this message:\\ \tt \\ \tt \\ \tt "I think you've made some excellent observations about the object positions and the grid structure. The fact that 'S' and 'A' are in the same column as 'P' could be very useful, as it might allow us to create a "chain" of aligned objects. I also like your idea of using the grid structure to "slide" objects into place, as this could help us navigate around obstacles and find new paths for alignment. Considering these insights, I think our next step should be to try to align 'S' or 'A' with their corresponding objects on my board. Since 'S' and 'A' are in column 1 on your board, let's focus on moving one of them to a position that could match the position of the same object on my board. Can you tell me the positions of 'S' and 'A' on my board, so we can discuss potential target positions for alignment?"\\ \tt \\ \tt \\ \tt What is your next command? 
	  } 
	   } 
	   } 
	 & \\ 
 

    \theutterance \stepcounter{utterance}  

    & & & \multicolumn{4}{p{0.6\linewidth}}{\cellcolor[rgb]{1,0.85,0.72}{%
	\makecell[{{p{\linewidth}}}]{% 
	  \tt {\tiny [GM$\langle$B]}  
	 SAY: According to our earlier conversation, on your board, 'S' is at (1,3) and 'A' is at (3,3). Since 'S' is already at (1,1) on my board, moving it to (1,3) could be a good option to align it with your 'S'. Alternatively, we could try to move 'A' to a position that matches your 'A' at (3,3), but that would require moving my 'A' from (1,3) to (3,3). Which of these options do you think would be more feasible, or do you have a different suggestion for aligning 'S' or 'A'? 
	  } 
	   } 
	   } 
	  \\ 
 

    \theutterance \stepcounter{utterance}  

    & & & \multicolumn{2}{p{0.3\linewidth}}{\cellcolor[rgb]{0.95,0.95,0.95}{%
	\makecell[{{p{\linewidth}}}]{% 
	  \tt {\tiny [GM$|$GM]}  
	 SAY: According to our earlier conversation, on your board, 'S' is at (1,3) and 'A' is at (3,3). Since 'S' is already at (1,1) on my board, moving it to (1,3) could be a good option to align it with your 'S'. Alternatively, we could try to move 'A' to a position that matches your 'A' at (3,3), but that would require moving my 'A' from (1,3) to (3,3). Which of these options do you think would be more feasible, or do you have a different suggestion for aligning 'S' or 'A'? 
	  } 
	   } 
	   } 
	 & & \\ 
 

    \theutterance \stepcounter{utterance}  

    & & \multicolumn{4}{p{0.6\linewidth}}{\cellcolor[rgb]{0.9,0.9,0.9}{%
	\makecell[{{p{\linewidth}}}]{% 
	  \tt {\tiny [A$\langle$GM]}  
	 Your message has been relayed to the other player.\\ \tt \\ \tt \\ \tt You are currently playing round 20 of maximum 28.\\ \tt You have collectively accumulated 12 of 16 penalties.\\ \tt The other player sent this message:\\ \tt \\ \tt \\ \tt "According to our earlier conversation, on your board, 'S' is at (1,3) and 'A' is at (3,3). Since 'S' is already at (1,1) on my board, moving it to (1,3) could be a good option to align it with your 'S'. Alternatively, we could try to move 'A' to a position that matches your 'A' at (3,3), but that would require moving my 'A' from (1,3) to (3,3). Which of these options do you think would be more feasible, or do you have a different suggestion for aligning 'S' or 'A'?"\\ \tt \\ \tt \\ \tt What is your next command? 
	  } 
	   } 
	   } 
	 & \\ 
 

    \theutterance \stepcounter{utterance}  

    & \multicolumn{4}{p{0.6\linewidth}}{\cellcolor[rgb]{0.8,1,0.9}{%
	\makecell[{{p{\linewidth}}}]{% 
	  \tt {\tiny [A$\rangle$GM]}  
	 SAY: Thank you for reminding me of the positions of 'S' and 'A' on my board. I think moving your 'S' from (1,1) to (1,3) could be a good option to align it with my 'S'. This move seems relatively straightforward, and it would be a good test of our ability to align objects. Alternatively, moving your 'A' from (1,3) to (3,3) could also be a good option, but it might require more moves to achieve. Before we decide, I'd like to ask: are there any obstacles or other objects on your board that could prevent or hinder the movement of 'S' from (1,1) to (1,3)? Please check your board and let me know if this move is feasible. If it is, we can proceed with aligning 'S'. If not, we can explore other options for 'A' or other objects. 
	  } 
	   } 
	   } 
	 & & \\ 
 

    \theutterance \stepcounter{utterance}  

    & & & \multicolumn{2}{p{0.3\linewidth}}{\cellcolor[rgb]{0.95,0.95,0.95}{%
	\makecell[{{p{\linewidth}}}]{% 
	  \tt {\tiny [GM$|$GM]}  
	 SAY: Thank you for reminding me of the positions of 'S' and 'A' on my board. I think moving your 'S' from (1,1) to (1,3) could be a good option to align it with my 'S'. This move seems relatively straightforward, and it would be a good test of our ability to align objects. Alternatively, moving your 'A' from (1,3) to (3,3) could also be a good option, but it might require more moves to achieve. Before we decide, I'd like to ask: are there any obstacles or other objects on your board that could prevent or hinder the movement of 'S' from (1,1) to (1,3)? Please check your board and let me know if this move is feasible. If it is, we can proceed with aligning 'S'. If not, we can explore other options for 'A' or other objects. 
	  } 
	   } 
	   } 
	 & & \\ 
 

    \theutterance \stepcounter{utterance}  

    & & \multicolumn{4}{p{0.6\linewidth}}{\cellcolor[rgb]{0.9,0.9,0.9}{%
	\makecell[{{p{\linewidth}}}]{% 
	  \tt {\tiny [GM$\rangle$B]}  
	 Your message has been relayed to the other player.\\ \tt \\ \tt \\ \tt You are currently playing round 21 of maximum 28.\\ \tt You have collectively accumulated 12 of 16 penalties.\\ \tt The other player sent this message:\\ \tt \\ \tt \\ \tt "Thank you for reminding me of the positions of 'S' and 'A' on my board. I think moving your 'S' from (1,1) to (1,3) could be a good option to align it with my 'S'. This move seems relatively straightforward, and it would be a good test of our ability to align objects. Alternatively, moving your 'A' from (1,3) to (3,3) could also be a good option, but it might require more moves to achieve. Before we decide, I'd like to ask: are there any obstacles or other objects on your board that could prevent or hinder the movement of 'S' from (1,1) to (1,3)? Please check your board and let me know if this move is feasible. If it is, we can proceed with aligning 'S'. If not, we can explore other options for 'A' or other objects."\\ \tt \\ \tt \\ \tt What is your next command? 
	  } 
	   } 
	   } 
	 & \\ 
 

    \theutterance \stepcounter{utterance}  

    & & & \multicolumn{4}{p{0.6\linewidth}}{\cellcolor[rgb]{1,0.85,0.72}{%
	\makecell[{{p{\linewidth}}}]{% 
	  \tt {\tiny [GM$\langle$B]}  
	 SAY: I've checked my board, and there are no obstacles or objects blocking the path from (1,1) to (1,3) that would prevent 'S' from moving to (1,3). The cell at (1,2) is empty, containing only '◌', and the cell at (1,3) is currently occupied by 'A'. So, the move is feasible, but it would require moving 'A' out of the way first. We could potentially move 'A' to a different position to make room for 'S', and then proceed with aligning 'S'. What are your thoughts on this plan, and should we proceed with moving 'A' to clear the path for 'S'? 
	  } 
	   } 
	   } 
	  \\ 
 

    \theutterance \stepcounter{utterance}  

    & & & \multicolumn{2}{p{0.3\linewidth}}{\cellcolor[rgb]{0.95,0.95,0.95}{%
	\makecell[{{p{\linewidth}}}]{% 
	  \tt {\tiny [GM$|$GM]}  
	 SAY: I've checked my board, and there are no obstacles or objects blocking the path from (1,1) to (1,3) that would prevent 'S' from moving to (1,3). The cell at (1,2) is empty, containing only '◌', and the cell at (1,3) is currently occupied by 'A'. So, the move is feasible, but it would require moving 'A' out of the way first. We could potentially move 'A' to a different position to make room for 'S', and then proceed with aligning 'S'. What are your thoughts on this plan, and should we proceed with moving 'A' to clear the path for 'S'? 
	  } 
	   } 
	   } 
	 & & \\ 
 

    \theutterance \stepcounter{utterance}  

    & & \multicolumn{4}{p{0.6\linewidth}}{\cellcolor[rgb]{0.9,0.9,0.9}{%
	\makecell[{{p{\linewidth}}}]{% 
	  \tt {\tiny [A$\langle$GM]}  
	 Your message has been relayed to the other player.\\ \tt \\ \tt \\ \tt You are currently playing round 21 of maximum 28.\\ \tt You have collectively accumulated 12 of 16 penalties.\\ \tt The other player sent this message:\\ \tt \\ \tt \\ \tt "I've checked my board, and there are no obstacles or objects blocking the path from (1,1) to (1,3) that would prevent 'S' from moving to (1,3). The cell at (1,2) is empty, containing only '◌', and the cell at (1,3) is currently occupied by 'A'. So, the move is feasible, but it would require moving 'A' out of the way first. We could potentially move 'A' to a different position to make room for 'S', and then proceed with aligning 'S'. What are your thoughts on this plan, and should we proceed with moving 'A' to clear the path for 'S'?"\\ \tt \\ \tt \\ \tt What is your next command? 
	  } 
	   } 
	   } 
	 & \\ 
 

    \theutterance \stepcounter{utterance}  

    & \multicolumn{4}{p{0.6\linewidth}}{\cellcolor[rgb]{0.8,1,0.9}{%
	\makecell[{{p{\linewidth}}}]{% 
	  \tt {\tiny [A$\rangle$GM]}  
	 SAY: I'm glad to hear that the path from (1,1) to (1,3) is relatively clear. However, I understand that 'A' is currently occupying the cell at (1,3), which means we need to move it out of the way before we can align 'S'. Moving 'A' to a different position could be a good solution, but we need to make sure that we're not creating a new problem by blocking another object's path or limiting our future options. Before we proceed, let's think about where we could move 'A' to. Since 'A' is currently at (1,3) on your board, and 'A' is at (3,3) on my board, could we potentially move your 'A' to a position that's closer to (3,3), such as (2,3) or (3,2)? This might help us create a better opportunity for aligning 'A' in the future, while also clearing the path for 'S'. What do you think about this idea? 
	  } 
	   } 
	   } 
	 & & \\ 
 

    \theutterance \stepcounter{utterance}  

    & & & \multicolumn{2}{p{0.3\linewidth}}{\cellcolor[rgb]{0.95,0.95,0.95}{%
	\makecell[{{p{\linewidth}}}]{% 
	  \tt {\tiny [GM$|$GM]}  
	 SAY: I'm glad to hear that the path from (1,1) to (1,3) is relatively clear. However, I understand that 'A' is currently occupying the cell at (1,3), which means we need to move it out of the way before we can align 'S'. Moving 'A' to a different position could be a good solution, but we need to make sure that we're not creating a new problem by blocking another object's path or limiting our future options. Before we proceed, let's think about where we could move 'A' to. Since 'A' is currently at (1,3) on your board, and 'A' is at (3,3) on my board, could we potentially move your 'A' to a position that's closer to (3,3), such as (2,3) or (3,2)? This might help us create a better opportunity for aligning 'A' in the future, while also clearing the path for 'S'. What do you think about this idea? 
	  } 
	   } 
	   } 
	 & & \\ 
 

    \theutterance \stepcounter{utterance}  

    & & \multicolumn{4}{p{0.6\linewidth}}{\cellcolor[rgb]{0.9,0.9,0.9}{%
	\makecell[{{p{\linewidth}}}]{% 
	  \tt {\tiny [GM$\rangle$B]}  
	 Your message has been relayed to the other player.\\ \tt \\ \tt \\ \tt You are currently playing round 22 of maximum 28.\\ \tt You have collectively accumulated 12 of 16 penalties.\\ \tt The other player sent this message:\\ \tt \\ \tt \\ \tt "I'm glad to hear that the path from (1,1) to (1,3) is relatively clear. However, I understand that 'A' is currently occupying the cell at (1,3), which means we need to move it out of the way before we can align 'S'. Moving 'A' to a different position could be a good solution, but we need to make sure that we're not creating a new problem by blocking another object's path or limiting our future options. Before we proceed, let's think about where we could move 'A' to. Since 'A' is currently at (1,3) on your board, and 'A' is at (3,3) on my board, could we potentially move your 'A' to a position that's closer to (3,3), such as (2,3) or (3,2)? This might help us create a better opportunity for aligning 'A' in the future, while also clearing the path for 'S'. What do you think about this idea?"\\ \tt \\ \tt \\ \tt What is your next command? 
	  } 
	   } 
	   } 
	 & \\ 
 

    \theutterance \stepcounter{utterance}  

    & & & \multicolumn{4}{p{0.6\linewidth}}{\cellcolor[rgb]{1,0.85,0.72}{%
	\makecell[{{p{\linewidth}}}]{% 
	  \tt {\tiny [GM$\langle$B]}  
	 SAY: Moving 'A' to (2,3) or (3,2) could be a good idea, as it would not only clear the path for 'S' but also potentially set up 'A' for future alignment. On my board, (2,3) is currently occupied by 'O', so that's not a viable option. However, (3,2) is empty, containing only '◌'. Moving 'A' to (3,2) could work, but we should consider how this move might affect the overall alignment strategy and whether it could create any new conflicts or limitations. What are your thoughts on moving 'A' to (3,2), and should we proceed with this move to clear the path for 'S'? 
	  } 
	   } 
	   } 
	  \\ 
 

    \theutterance \stepcounter{utterance}  

    & & & \multicolumn{2}{p{0.3\linewidth}}{\cellcolor[rgb]{0.95,0.95,0.95}{%
	\makecell[{{p{\linewidth}}}]{% 
	  \tt {\tiny [GM$|$GM]}  
	 SAY: Moving 'A' to (2,3) or (3,2) could be a good idea, as it would not only clear the path for 'S' but also potentially set up 'A' for future alignment. On my board, (2,3) is currently occupied by 'O', so that's not a viable option. However, (3,2) is empty, containing only '◌'. Moving 'A' to (3,2) could work, but we should consider how this move might affect the overall alignment strategy and whether it could create any new conflicts or limitations. What are your thoughts on moving 'A' to (3,2), and should we proceed with this move to clear the path for 'S'? 
	  } 
	   } 
	   } 
	 & & \\ 
 

    \theutterance \stepcounter{utterance}  

    & & \multicolumn{4}{p{0.6\linewidth}}{\cellcolor[rgb]{0.9,0.9,0.9}{%
	\makecell[{{p{\linewidth}}}]{% 
	  \tt {\tiny [A$\langle$GM]}  
	 Your message has been relayed to the other player.\\ \tt \\ \tt \\ \tt You are currently playing round 22 of maximum 28.\\ \tt You have collectively accumulated 12 of 16 penalties.\\ \tt The other player sent this message:\\ \tt \\ \tt \\ \tt "Moving 'A' to (2,3) or (3,2) could be a good idea, as it would not only clear the path for 'S' but also potentially set up 'A' for future alignment. On my board, (2,3) is currently occupied by 'O', so that's not a viable option. However, (3,2) is empty, containing only '◌'. Moving 'A' to (3,2) could work, but we should consider how this move might affect the overall alignment strategy and whether it could create any new conflicts or limitations. What are your thoughts on moving 'A' to (3,2), and should we proceed with this move to clear the path for 'S'?"\\ \tt \\ \tt \\ \tt What is your next command? 
	  } 
	   } 
	   } 
	 & \\ 
 

    \theutterance \stepcounter{utterance}  

    & \multicolumn{4}{p{0.6\linewidth}}{\cellcolor[rgb]{0.8,1,0.9}{%
	\makecell[{{p{\linewidth}}}]{% 
	  \tt {\tiny [A$\rangle$GM]}  
	 SAY: I'm glad we found a potential new home for 'A' at (3,2). Considering the empty cell at (3,2) and the need to clear the path for 'S', I think moving 'A' to (3,2) is a good idea. This move seems to have a relatively low risk of creating new conflicts or limitations, and it could potentially set up 'A' for future alignment. Before we proceed, let's confirm that moving 'A' to (3,2) won't block any other objects or create a deadlock. If everything looks good, we can move 'A' to (3,2) and then proceed with moving 'S' to (1,3) to align it with my 'S'. After that, we can reassess our strategy and decide on the next steps. Please confirm that you're ready to move 'A' to (3,2) and that it won't cause any issues on your board. 
	  } 
	   } 
	   } 
	 & & \\ 
 

    \theutterance \stepcounter{utterance}  

    & & & \multicolumn{2}{p{0.3\linewidth}}{\cellcolor[rgb]{0.95,0.95,0.95}{%
	\makecell[{{p{\linewidth}}}]{% 
	  \tt {\tiny [GM$|$GM]}  
	 SAY: I'm glad we found a potential new home for 'A' at (3,2). Considering the empty cell at (3,2) and the need to clear the path for 'S', I think moving 'A' to (3,2) is a good idea. This move seems to have a relatively low risk of creating new conflicts or limitations, and it could potentially set up 'A' for future alignment. Before we proceed, let's confirm that moving 'A' to (3,2) won't block any other objects or create a deadlock. If everything looks good, we can move 'A' to (3,2) and then proceed with moving 'S' to (1,3) to align it with my 'S'. After that, we can reassess our strategy and decide on the next steps. Please confirm that you're ready to move 'A' to (3,2) and that it won't cause any issues on your board. 
	  } 
	   } 
	   } 
	 & & \\ 
 

    \theutterance \stepcounter{utterance}  

    & & \multicolumn{4}{p{0.6\linewidth}}{\cellcolor[rgb]{0.9,0.9,0.9}{%
	\makecell[{{p{\linewidth}}}]{% 
	  \tt {\tiny [GM$\rangle$B]}  
	 Your message has been relayed to the other player.\\ \tt \\ \tt \\ \tt You are currently playing round 23 of maximum 28.\\ \tt You have collectively accumulated 12 of 16 penalties.\\ \tt The other player sent this message:\\ \tt \\ \tt \\ \tt "I'm glad we found a potential new home for 'A' at (3,2). Considering the empty cell at (3,2) and the need to clear the path for 'S', I think moving 'A' to (3,2) is a good idea. This move seems to have a relatively low risk of creating new conflicts or limitations, and it could potentially set up 'A' for future alignment. Before we proceed, let's confirm that moving 'A' to (3,2) won't block any other objects or create a deadlock. If everything looks good, we can move 'A' to (3,2) and then proceed with moving 'S' to (1,3) to align it with my 'S'. After that, we can reassess our strategy and decide on the next steps. Please confirm that you're ready to move 'A' to (3,2) and that it won't cause any issues on your board."\\ \tt \\ \tt \\ \tt What is your next command? 
	  } 
	   } 
	   } 
	 & \\ 
 

    \theutterance \stepcounter{utterance}  

    & & & \multicolumn{4}{p{0.6\linewidth}}{\cellcolor[rgb]{1,0.85,0.72}{%
	\makecell[{{p{\linewidth}}}]{% 
	  \tt {\tiny [GM$\langle$B]}  
	 MOVE: A, (3,2) 
	  } 
	   } 
	   } 
	  \\ 
 

    \theutterance \stepcounter{utterance}  

    & & & \multicolumn{2}{p{0.3\linewidth}}{\cellcolor[rgb]{0.95,0.95,0.95}{%
	\makecell[{{p{\linewidth}}}]{% 
	  \tt {\tiny [GM$|$GM]}  
	 MOVE: A, (3,2) 
	  } 
	   } 
	   } 
	 & & \\ 
 

    \theutterance \stepcounter{utterance}  

    & & & \multicolumn{2}{p{0.3\linewidth}}{\cellcolor[rgb]{0.95,0.95,0.95}{%
	\makecell[{{p{\linewidth}}}]{% 
	  \tt {\tiny [GM$|$GM]}  
	 Penalty: (3,2) is not empty, but contains '┘'.\\ \tt \\ \tt You have collectively accumulated 13 of 16 penalties.\\ \tt Please try again! 
	  } 
	   } 
	   } 
	 & & \\ 
 

    \theutterance \stepcounter{utterance}  

    & & \multicolumn{4}{p{0.6\linewidth}}{\cellcolor[rgb]{0.9,0.9,0.9}{%
	\makecell[{{p{\linewidth}}}]{% 
	  \tt {\tiny [GM$\rangle$B]}  
	 Penalty: (3,2) is not empty, but contains '┘'.\\ \tt \\ \tt You have collectively accumulated 13 of 16 penalties.\\ \tt Please try again! 
	  } 
	   } 
	   } 
	 & \\ 
 

    \theutterance \stepcounter{utterance}  

    & & & \multicolumn{4}{p{0.6\linewidth}}{\cellcolor[rgb]{1,0.85,0.72}{%
	\makecell[{{p{\linewidth}}}]{% 
	  \tt {\tiny [GM$\langle$B]}  
	 SAY: It seems that moving 'A' to (3,2) is not possible due to the grid structure. Let me try to find an alternative position for 'A' that is empty and can be moved to. Can we discuss other options for moving 'A' before proceeding with aligning 'S'? 
	  } 
	   } 
	   } 
	  \\ 
 

    \theutterance \stepcounter{utterance}  

    & & & \multicolumn{2}{p{0.3\linewidth}}{\cellcolor[rgb]{0.95,0.95,0.95}{%
	\makecell[{{p{\linewidth}}}]{% 
	  \tt {\tiny [GM$|$GM]}  
	 SAY: It seems that moving 'A' to (3,2) is not possible due to the grid structure. Let me try to find an alternative position for 'A' that is empty and can be moved to. Can we discuss other options for moving 'A' before proceeding with aligning 'S'? 
	  } 
	   } 
	   } 
	 & & \\ 
 

    \theutterance \stepcounter{utterance}  

    & & \multicolumn{4}{p{0.6\linewidth}}{\cellcolor[rgb]{0.9,0.9,0.9}{%
	\makecell[{{p{\linewidth}}}]{% 
	  \tt {\tiny [A$\langle$GM]}  
	 Your message has been relayed to the other player.\\ \tt \\ \tt \\ \tt You are currently playing round 23 of maximum 28.\\ \tt You have collectively accumulated 13 of 16 penalties.\\ \tt The other player sent this message:\\ \tt \\ \tt \\ \tt "It seems that moving 'A' to (3,2) is not possible due to the grid structure. Let me try to find an alternative position for 'A' that is empty and can be moved to. Can we discuss other options for moving 'A' before proceeding with aligning 'S'?"\\ \tt \\ \tt \\ \tt What is your next command? 
	  } 
	   } 
	   } 
	 & \\ 
 

    \theutterance \stepcounter{utterance}  

    & \multicolumn{4}{p{0.6\linewidth}}{\cellcolor[rgb]{0.8,1,0.9}{%
	\makecell[{{p{\linewidth}}}]{% 
	  \tt {\tiny [A$\rangle$GM]}  
	 SAY: It seems like we've hit another obstacle with the grid structure. Don't worry, let's explore other options for moving 'A'. At this point, I think it's essential to consider the overall progress we've made and the number of rounds we have left. We need to make the most of our remaining moves to align as many objects as possible. Instead of focusing solely on moving 'A' to a specific position, let's think about the bigger picture. Which objects are closest to being aligned, and what moves can we make to take advantage of that? Perhaps we can make progress with 'O', 'T', or 'D' while we figure out a new plan for 'A' and 'S'. Can you give me an update on the current positions of these objects on your board, and we can discuss potential next steps? We should also start thinking about when to try to finish the game, as we're running out of rounds. 
	  } 
	   } 
	   } 
	 & & \\ 
 

    \theutterance \stepcounter{utterance}  

    & & & \multicolumn{2}{p{0.3\linewidth}}{\cellcolor[rgb]{0.95,0.95,0.95}{%
	\makecell[{{p{\linewidth}}}]{% 
	  \tt {\tiny [GM$|$GM]}  
	 SAY: It seems like we've hit another obstacle with the grid structure. Don't worry, let's explore other options for moving 'A'. At this point, I think it's essential to consider the overall progress we've made and the number of rounds we have left. We need to make the most of our remaining moves to align as many objects as possible. Instead of focusing solely on moving 'A' to a specific position, let's think about the bigger picture. Which objects are closest to being aligned, and what moves can we make to take advantage of that? Perhaps we can make progress with 'O', 'T', or 'D' while we figure out a new plan for 'A' and 'S'. Can you give me an update on the current positions of these objects on your board, and we can discuss potential next steps? We should also start thinking about when to try to finish the game, as we're running out of rounds. 
	  } 
	   } 
	   } 
	 & & \\ 
 

    \theutterance \stepcounter{utterance}  

    & & \multicolumn{4}{p{0.6\linewidth}}{\cellcolor[rgb]{0.9,0.9,0.9}{%
	\makecell[{{p{\linewidth}}}]{% 
	  \tt {\tiny [GM$\rangle$B]}  
	 Your message has been relayed to the other player.\\ \tt \\ \tt \\ \tt You are currently playing round 24 of maximum 28.\\ \tt You have collectively accumulated 13 of 16 penalties.\\ \tt The other player sent this message:\\ \tt \\ \tt \\ \tt "It seems like we've hit another obstacle with the grid structure. Don't worry, let's explore other options for moving 'A'. At this point, I think it's essential to consider the overall progress we've made and the number of rounds we have left. We need to make the most of our remaining moves to align as many objects as possible. Instead of focusing solely on moving 'A' to a specific position, let's think about the bigger picture. Which objects are closest to being aligned, and what moves can we make to take advantage of that? Perhaps we can make progress with 'O', 'T', or 'D' while we figure out a new plan for 'A' and 'S'. Can you give me an update on the current positions of these objects on your board, and we can discuss potential next steps? We should also start thinking about when to try to finish the game, as we're running out of rounds."\\ \tt \\ \tt \\ \tt What is your next command? 
	  } 
	   } 
	   } 
	 & \\ 
 

    \theutterance \stepcounter{utterance}  

    & & & \multicolumn{4}{p{0.6\linewidth}}{\cellcolor[rgb]{1,0.85,0.72}{%
	\makecell[{{p{\linewidth}}}]{% 
	  \tt {\tiny [GM$\langle$B]}  
	 SAY: Let me give you an update on the current positions of the objects on my board. 'P' is at (3,7), 'O' is at (2,3), 'T' is at (4,5), 'D' is at (4,5), 'S' is at (1,1), 'A' is at (1,3), and 'M' is at (7,7). Considering these positions and the fact that we're running out of rounds, I think we should prioritize aligning the objects that are closest to being aligned. Considering our previous discussions, I think 'S' is still a good candidate to align next, as it only needs to move to (1,3) to match your 'S'. However, since 'A' is currently occupying (1,3), we need to find a new position for 'A' first. Let's try to find a simple move for 'A' that can clear the path for 'S'. If we can align 'S', we can then reassess our strategy and decide on the next steps. We should also start thinking about when to try to finish the game, as you mentioned. Perhaps we can aim to finish the game in the next few rounds, depending on how our next moves go. What are your thoughts on this plan, and should we try to align 'S' as soon as possible? 
	  } 
	   } 
	   } 
	  \\ 
 

    \theutterance \stepcounter{utterance}  

    & & & \multicolumn{2}{p{0.3\linewidth}}{\cellcolor[rgb]{0.95,0.95,0.95}{%
	\makecell[{{p{\linewidth}}}]{% 
	  \tt {\tiny [GM$|$GM]}  
	 SAY: Let me give you an update on the current positions of the objects on my board. 'P' is at (3,7), 'O' is at (2,3), 'T' is at (4,5), 'D' is at (4,5), 'S' is at (1,1), 'A' is at (1,3), and 'M' is at (7,7). Considering these positions and the fact that we're running out of rounds, I think we should prioritize aligning the objects that are closest to being aligned. Considering our previous discussions, I think 'S' is still a good candidate to align next, as it only needs to move to (1,3) to match your 'S'. However, since 'A' is currently occupying (1,3), we need to find a new position for 'A' first. Let's try to find a simple move for 'A' that can clear the path for 'S'. If we can align 'S', we can then reassess our strategy and decide on the next steps. We should also start thinking about when to try to finish the game, as you mentioned. Perhaps we can aim to finish the game in the next few rounds, depending on how our next moves go. What are your thoughts on this plan, and should we try to align 'S' as soon as possible? 
	  } 
	   } 
	   } 
	 & & \\ 
 

    \theutterance \stepcounter{utterance}  

    & & \multicolumn{4}{p{0.6\linewidth}}{\cellcolor[rgb]{0.9,0.9,0.9}{%
	\makecell[{{p{\linewidth}}}]{% 
	  \tt {\tiny [A$\langle$GM]}  
	 Your message has been relayed to the other player.\\ \tt \\ \tt \\ \tt You are currently playing round 24 of maximum 28.\\ \tt You have collectively accumulated 13 of 16 penalties.\\ \tt The other player sent this message:\\ \tt \\ \tt \\ \tt "Let me give you an update on the current positions of the objects on my board. 'P' is at (3,7), 'O' is at (2,3), 'T' is at (4,5), 'D' is at (4,5), 'S' is at (1,1), 'A' is at (1,3), and 'M' is at (7,7). Considering these positions and the fact that we're running out of rounds, I think we should prioritize aligning the objects that are closest to being aligned. Considering our previous discussions, I think 'S' is still a good candidate to align next, as it only needs to move to (1,3) to match your 'S'. However, since 'A' is currently occupying (1,3), we need to find a new position for 'A' first. Let's try to find a simple move for 'A' that can clear the path for 'S'. If we can align 'S', we can then reassess our strategy and decide on the next steps. We should also start thinking about when to try to finish the game, as you mentioned. Perhaps we can aim to finish the game in the next few rounds, depending on how our next moves go. What are your thoughts on this plan, and should we try to align 'S' as soon as possible?"\\ \tt \\ \tt \\ \tt What is your next command? 
	  } 
	   } 
	   } 
	 & \\ 
 

    \theutterance \stepcounter{utterance}  

    & \multicolumn{4}{p{0.6\linewidth}}{\cellcolor[rgb]{0.8,1,0.9}{%
	\makecell[{{p{\linewidth}}}]{% 
	  \tt {\tiny [A$\rangle$GM]}  
	 SAY: I think your plan to prioritize aligning 'S' is a good one, considering its proximity to the target position. Given the current positions of the objects on your board and the limited number of rounds left, it's essential to focus on making progress with the objects that are closest to being aligned. Aligning 'S' would be a significant achievement, and it could help us build momentum for the remaining rounds. To clear the path for 'S', let's try to find a simple move for 'A' that doesn't require too much planning or risk. Since 'A' is currently at (1,3), could we try moving it to a nearby empty cell, such as (1,2) or (1,4), if either of those cells is available? If not, we can explore other options. Once we've moved 'A' and cleared the path, we can proceed with moving 'S' to (1,3) to align it with my 'S'. After that, we can reassess our strategy and decide on the next steps, keeping in mind that we should aim to finish the game soon. What do you think about trying to move 'A' to a nearby empty cell to clear the path for 'S'? 
	  } 
	   } 
	   } 
	 & & \\ 
 

    \theutterance \stepcounter{utterance}  

    & & & \multicolumn{2}{p{0.3\linewidth}}{\cellcolor[rgb]{0.95,0.95,0.95}{%
	\makecell[{{p{\linewidth}}}]{% 
	  \tt {\tiny [GM$|$GM]}  
	 SAY: I think your plan to prioritize aligning 'S' is a good one, considering its proximity to the target position. Given the current positions of the objects on your board and the limited number of rounds left, it's essential to focus on making progress with the objects that are closest to being aligned. Aligning 'S' would be a significant achievement, and it could help us build momentum for the remaining rounds. To clear the path for 'S', let's try to find a simple move for 'A' that doesn't require too much planning or risk. Since 'A' is currently at (1,3), could we try moving it to a nearby empty cell, such as (1,2) or (1,4), if either of those cells is available? If not, we can explore other options. Once we've moved 'A' and cleared the path, we can proceed with moving 'S' to (1,3) to align it with my 'S'. After that, we can reassess our strategy and decide on the next steps, keeping in mind that we should aim to finish the game soon. What do you think about trying to move 'A' to a nearby empty cell to clear the path for 'S'? 
	  } 
	   } 
	   } 
	 & & \\ 
 

    \theutterance \stepcounter{utterance}  

    & & \multicolumn{4}{p{0.6\linewidth}}{\cellcolor[rgb]{0.9,0.9,0.9}{%
	\makecell[{{p{\linewidth}}}]{% 
	  \tt {\tiny [GM$\rangle$B]}  
	 Your message has been relayed to the other player.\\ \tt \\ \tt \\ \tt You are currently playing round 25 of maximum 28.\\ \tt You have collectively accumulated 13 of 16 penalties.\\ \tt The other player sent this message:\\ \tt \\ \tt \\ \tt "I think your plan to prioritize aligning 'S' is a good one, considering its proximity to the target position. Given the current positions of the objects on your board and the limited number of rounds left, it's essential to focus on making progress with the objects that are closest to being aligned. Aligning 'S' would be a significant achievement, and it could help us build momentum for the remaining rounds. To clear the path for 'S', let's try to find a simple move for 'A' that doesn't require too much planning or risk. Since 'A' is currently at (1,3), could we try moving it to a nearby empty cell, such as (1,2) or (1,4), if either of those cells is available? If not, we can explore other options. Once we've moved 'A' and cleared the path, we can proceed with moving 'S' to (1,3) to align it with my 'S'. After that, we can reassess our strategy and decide on the next steps, keeping in mind that we should aim to finish the game soon. What do you think about trying to move 'A' to a nearby empty cell to clear the path for 'S'?"\\ \tt \\ \tt \\ \tt What is your next command? 
	  } 
	   } 
	   } 
	 & \\ 
 

    \theutterance \stepcounter{utterance}  

    & & & \multicolumn{4}{p{0.6\linewidth}}{\cellcolor[rgb]{1,0.85,0.72}{%
	\makecell[{{p{\linewidth}}}]{% 
	  \tt {\tiny [GM$\langle$B]}  
	 SAY: I've checked the nearby cells, and (1,2) is empty, containing only '◌'. Moving 'A' to (1,2) could be a good option to clear the path for 'S'. This move seems relatively simple and low-risk, and it would allow us to proceed with aligning 'S' to (1,3). I'm ready to move 'A' to (1,2) and then proceed with moving 'S' to (1,3). After that, we can reassess our strategy and decide on the next steps, with the goal of finishing the game soon. What are your thoughts on this plan, and should we proceed with moving 'A' to (1,2)? 
	  } 
	   } 
	   } 
	  \\ 
 

    \theutterance \stepcounter{utterance}  

    & & & \multicolumn{2}{p{0.3\linewidth}}{\cellcolor[rgb]{0.95,0.95,0.95}{%
	\makecell[{{p{\linewidth}}}]{% 
	  \tt {\tiny [GM$|$GM]}  
	 SAY: I've checked the nearby cells, and (1,2) is empty, containing only '◌'. Moving 'A' to (1,2) could be a good option to clear the path for 'S'. This move seems relatively simple and low-risk, and it would allow us to proceed with aligning 'S' to (1,3). I'm ready to move 'A' to (1,2) and then proceed with moving 'S' to (1,3). After that, we can reassess our strategy and decide on the next steps, with the goal of finishing the game soon. What are your thoughts on this plan, and should we proceed with moving 'A' to (1,2)? 
	  } 
	   } 
	   } 
	 & & \\ 
 

    \theutterance \stepcounter{utterance}  

    & & \multicolumn{4}{p{0.6\linewidth}}{\cellcolor[rgb]{0.9,0.9,0.9}{%
	\makecell[{{p{\linewidth}}}]{% 
	  \tt {\tiny [A$\langle$GM]}  
	 Your message has been relayed to the other player.\\ \tt \\ \tt \\ \tt You are currently playing round 25 of maximum 28.\\ \tt You have collectively accumulated 13 of 16 penalties.\\ \tt The other player sent this message:\\ \tt \\ \tt \\ \tt "I've checked the nearby cells, and (1,2) is empty, containing only '◌'. Moving 'A' to (1,2) could be a good option to clear the path for 'S'. This move seems relatively simple and low-risk, and it would allow us to proceed with aligning 'S' to (1,3). I'm ready to move 'A' to (1,2) and then proceed with moving 'S' to (1,3). After that, we can reassess our strategy and decide on the next steps, with the goal of finishing the game soon. What are your thoughts on this plan, and should we proceed with moving 'A' to (1,2)?"\\ \tt \\ \tt \\ \tt What is your next command? 
	  } 
	   } 
	   } 
	 & \\ 
 

    \theutterance \stepcounter{utterance}  

    & \multicolumn{4}{p{0.6\linewidth}}{\cellcolor[rgb]{0.8,1,0.9}{%
	\makecell[{{p{\linewidth}}}]{% 
	  \tt {\tiny [A$\rangle$GM]}  
	 SAY: I think moving 'A' to (1,2) is a great idea. It's a simple and low-risk move that will clear the path for 'S' and allow us to make progress. I agree that we should proceed with moving 'A' to (1,2) and then moving 'S' to (1,3) to align it with my 'S'. After that, we can reassess our strategy and decide on the next steps. Considering the number of rounds left, I think we should aim to finish the game within the next 2-3 rounds. Let's make the move and see how it goes. Please go ahead and move 'A' to (1,2), and then we can proceed with moving 'S' to (1,3). After that, we can discuss our next steps and plan our approach to finishing the game. 
	  } 
	   } 
	   } 
	 & & \\ 
 

    \theutterance \stepcounter{utterance}  

    & & & \multicolumn{2}{p{0.3\linewidth}}{\cellcolor[rgb]{0.95,0.95,0.95}{%
	\makecell[{{p{\linewidth}}}]{% 
	  \tt {\tiny [GM$|$GM]}  
	 SAY: I think moving 'A' to (1,2) is a great idea. It's a simple and low-risk move that will clear the path for 'S' and allow us to make progress. I agree that we should proceed with moving 'A' to (1,2) and then moving 'S' to (1,3) to align it with my 'S'. After that, we can reassess our strategy and decide on the next steps. Considering the number of rounds left, I think we should aim to finish the game within the next 2-3 rounds. Let's make the move and see how it goes. Please go ahead and move 'A' to (1,2), and then we can proceed with moving 'S' to (1,3). After that, we can discuss our next steps and plan our approach to finishing the game. 
	  } 
	   } 
	   } 
	 & & \\ 
 

    \theutterance \stepcounter{utterance}  

    & & \multicolumn{4}{p{0.6\linewidth}}{\cellcolor[rgb]{0.9,0.9,0.9}{%
	\makecell[{{p{\linewidth}}}]{% 
	  \tt {\tiny [GM$\rangle$B]}  
	 Your message has been relayed to the other player.\\ \tt \\ \tt \\ \tt You are currently playing round 26 of maximum 28.\\ \tt You have collectively accumulated 13 of 16 penalties.\\ \tt The other player sent this message:\\ \tt \\ \tt \\ \tt "I think moving 'A' to (1,2) is a great idea. It's a simple and low-risk move that will clear the path for 'S' and allow us to make progress. I agree that we should proceed with moving 'A' to (1,2) and then moving 'S' to (1,3) to align it with my 'S'. After that, we can reassess our strategy and decide on the next steps. Considering the number of rounds left, I think we should aim to finish the game within the next 2-3 rounds. Let's make the move and see how it goes. Please go ahead and move 'A' to (1,2), and then we can proceed with moving 'S' to (1,3). After that, we can discuss our next steps and plan our approach to finishing the game."\\ \tt \\ \tt \\ \tt What is your next command? 
	  } 
	   } 
	   } 
	 & \\ 
 

    \theutterance \stepcounter{utterance}  

    & & & \multicolumn{4}{p{0.6\linewidth}}{\cellcolor[rgb]{1,0.85,0.72}{%
	\makecell[{{p{\linewidth}}}]{% 
	  \tt {\tiny [GM$\langle$B]}  
	 MOVE: A, (1,2) 
	  } 
	   } 
	   } 
	  \\ 
 

    \theutterance \stepcounter{utterance}  

    & & & \multicolumn{2}{p{0.3\linewidth}}{\cellcolor[rgb]{0.95,0.95,0.95}{%
	\makecell[{{p{\linewidth}}}]{% 
	  \tt {\tiny [GM$|$GM]}  
	 MOVE: A, (1,2) 
	  } 
	   } 
	   } 
	 & & \\ 
 

    \theutterance \stepcounter{utterance}  

    & & & \multicolumn{2}{p{0.3\linewidth}}{\cellcolor[rgb]{0.95,0.95,0.95}{%
	\makecell[{{p{\linewidth}}}]{% 
	  \tt {\tiny [GM$|$GM]}  
	 Penalty: (1,2) is not empty, but contains '─'.\\ \tt \\ \tt You have collectively accumulated 14 of 16 penalties.\\ \tt Please try again! 
	  } 
	   } 
	   } 
	 & & \\ 
 

    \theutterance \stepcounter{utterance}  

    & & \multicolumn{4}{p{0.6\linewidth}}{\cellcolor[rgb]{0.9,0.9,0.9}{%
	\makecell[{{p{\linewidth}}}]{% 
	  \tt {\tiny [GM$\rangle$B]}  
	 Penalty: (1,2) is not empty, but contains '─'.\\ \tt \\ \tt You have collectively accumulated 14 of 16 penalties.\\ \tt Please try again! 
	  } 
	   } 
	   } 
	 & \\ 
 

    \theutterance \stepcounter{utterance}  

    & & & \multicolumn{4}{p{0.6\linewidth}}{\cellcolor[rgb]{1,0.85,0.72}{%
	\makecell[{{p{\linewidth}}}]{% 
	  \tt {\tiny [GM$\langle$B]}  
	 SAY: It seems that moving 'A' to (1,2) is not possible due to the grid structure. Let me try to find an alternative position for 'A' that is empty and can be moved to. Can we discuss other options for moving 'A' before proceeding with aligning 'S'? Perhaps we can try moving 'A' to a different nearby empty cell, such as (1,4) or another available position. 
	  } 
	   } 
	   } 
	  \\ 
 

    \theutterance \stepcounter{utterance}  

    & & & \multicolumn{2}{p{0.3\linewidth}}{\cellcolor[rgb]{0.95,0.95,0.95}{%
	\makecell[{{p{\linewidth}}}]{% 
	  \tt {\tiny [GM$|$GM]}  
	 SAY: It seems that moving 'A' to (1,2) is not possible due to the grid structure. Let me try to find an alternative position for 'A' that is empty and can be moved to. Can we discuss other options for moving 'A' before proceeding with aligning 'S'? Perhaps we can try moving 'A' to a different nearby empty cell, such as (1,4) or another available position. 
	  } 
	   } 
	   } 
	 & & \\ 
 

    \theutterance \stepcounter{utterance}  

    & & \multicolumn{4}{p{0.6\linewidth}}{\cellcolor[rgb]{0.9,0.9,0.9}{%
	\makecell[{{p{\linewidth}}}]{% 
	  \tt {\tiny [A$\langle$GM]}  
	 Your message has been relayed to the other player.\\ \tt \\ \tt \\ \tt You are currently playing round 26 of maximum 28.\\ \tt You have collectively accumulated 14 of 16 penalties.\\ \tt The other player sent this message:\\ \tt \\ \tt \\ \tt "It seems that moving 'A' to (1,2) is not possible due to the grid structure. Let me try to find an alternative position for 'A' that is empty and can be moved to. Can we discuss other options for moving 'A' before proceeding with aligning 'S'? Perhaps we can try moving 'A' to a different nearby empty cell, such as (1,4) or another available position."\\ \tt \\ \tt \\ \tt What is your next command? 
	  } 
	   } 
	   } 
	 & \\ 
 

    \theutterance \stepcounter{utterance}  

    & \multicolumn{4}{p{0.6\linewidth}}{\cellcolor[rgb]{0.8,1,0.9}{%
	\makecell[{{p{\linewidth}}}]{% 
	  \tt {\tiny [A$\rangle$GM]}  
	 SAY: It seems like we've hit another obstacle with the grid structure. Let's think carefully about our next move, considering we only have 2 rounds left. Since moving 'A' to (1,2) is not possible, let's explore other nearby options. You mentioned (1,4) as a potential alternative. Could you please check if (1,4) is empty and available for 'A' to move to? If it is, we can try moving 'A' to (1,4) and then proceed with aligning 'S' to (1,3). If not, we'll need to think of another plan. We're running out of time, so let's make our next move count. Please check the availability of (1,4) and let me know if it's a viable option. 
	  } 
	   } 
	   } 
	 & & \\ 
 

    \theutterance \stepcounter{utterance}  

    & & & \multicolumn{2}{p{0.3\linewidth}}{\cellcolor[rgb]{0.95,0.95,0.95}{%
	\makecell[{{p{\linewidth}}}]{% 
	  \tt {\tiny [GM$|$GM]}  
	 SAY: It seems like we've hit another obstacle with the grid structure. Let's think carefully about our next move, considering we only have 2 rounds left. Since moving 'A' to (1,2) is not possible, let's explore other nearby options. You mentioned (1,4) as a potential alternative. Could you please check if (1,4) is empty and available for 'A' to move to? If it is, we can try moving 'A' to (1,4) and then proceed with aligning 'S' to (1,3). If not, we'll need to think of another plan. We're running out of time, so let's make our next move count. Please check the availability of (1,4) and let me know if it's a viable option. 
	  } 
	   } 
	   } 
	 & & \\ 
 

    \theutterance \stepcounter{utterance}  

    & & \multicolumn{4}{p{0.6\linewidth}}{\cellcolor[rgb]{0.9,0.9,0.9}{%
	\makecell[{{p{\linewidth}}}]{% 
	  \tt {\tiny [GM$\rangle$B]}  
	 Your message has been relayed to the other player.\\ \tt \\ \tt \\ \tt You are currently playing round 27 of maximum 28.\\ \tt You have collectively accumulated 14 of 16 penalties.\\ \tt The other player sent this message:\\ \tt \\ \tt \\ \tt "It seems like we've hit another obstacle with the grid structure. Let's think carefully about our next move, considering we only have 2 rounds left. Since moving 'A' to (1,2) is not possible, let's explore other nearby options. You mentioned (1,4) as a potential alternative. Could you please check if (1,4) is empty and available for 'A' to move to? If it is, we can try moving 'A' to (1,4) and then proceed with aligning 'S' to (1,3). If not, we'll need to think of another plan. We're running out of time, so let's make our next move count. Please check the availability of (1,4) and let me know if it's a viable option."\\ \tt \\ \tt \\ \tt What is your next command? 
	  } 
	   } 
	   } 
	 & \\ 
 

    \theutterance \stepcounter{utterance}  

    & & & \multicolumn{4}{p{0.6\linewidth}}{\cellcolor[rgb]{1,0.85,0.72}{%
	\makecell[{{p{\linewidth}}}]{% 
	  \tt {\tiny [GM$\langle$B]}  
	 SAY: I've checked the availability of (1,4), and it's empty, containing only '◌'. Moving 'A' to (1,4) is a viable option. I'll proceed with moving 'A' to (1,4) to clear the path for 'S'. After that, I can move 'S' to (1,3) to align it with your 'S'. Please confirm that you're ready for me to make these moves, and we can plan our final steps to finish the game. 
	  } 
	   } 
	   } 
	  \\ 
 

    \theutterance \stepcounter{utterance}  

    & & & \multicolumn{2}{p{0.3\linewidth}}{\cellcolor[rgb]{0.95,0.95,0.95}{%
	\makecell[{{p{\linewidth}}}]{% 
	  \tt {\tiny [GM$|$GM]}  
	 SAY: I've checked the availability of (1,4), and it's empty, containing only '◌'. Moving 'A' to (1,4) is a viable option. I'll proceed with moving 'A' to (1,4) to clear the path for 'S'. After that, I can move 'S' to (1,3) to align it with your 'S'. Please confirm that you're ready for me to make these moves, and we can plan our final steps to finish the game. 
	  } 
	   } 
	   } 
	 & & \\ 
 

    \theutterance \stepcounter{utterance}  

    & & \multicolumn{4}{p{0.6\linewidth}}{\cellcolor[rgb]{0.9,0.9,0.9}{%
	\makecell[{{p{\linewidth}}}]{% 
	  \tt {\tiny [A$\langle$GM]}  
	 Your message has been relayed to the other player.\\ \tt \\ \tt \\ \tt You are currently playing round 27 of maximum 28.\\ \tt You have collectively accumulated 14 of 16 penalties.\\ \tt The other player sent this message:\\ \tt \\ \tt \\ \tt "I've checked the availability of (1,4), and it's empty, containing only '◌'. Moving 'A' to (1,4) is a viable option. I'll proceed with moving 'A' to (1,4) to clear the path for 'S'. After that, I can move 'S' to (1,3) to align it with your 'S'. Please confirm that you're ready for me to make these moves, and we can plan our final steps to finish the game."\\ \tt \\ \tt \\ \tt What is your next command? 
	  } 
	   } 
	   } 
	 & \\ 
 

    \theutterance \stepcounter{utterance}  

    & \multicolumn{4}{p{0.6\linewidth}}{\cellcolor[rgb]{0.8,1,0.9}{%
	\makecell[{{p{\linewidth}}}]{% 
	  \tt {\tiny [A$\rangle$GM]}  
	 SAY: I'm glad we found a viable option for moving 'A'. Yes, please proceed with moving 'A' to (1,4) to clear the path for 'S'. After that, you can move 'S' to (1,3) to align it with my 'S'. I'm ready for you to make these moves. Once you've done so, we can assess the final state of our boards and plan our last move to finish the game. Since this is our second-to-last round, let's make sure we're setting ourselves up for a successful conclusion. Please go ahead and make the moves, and then we can discuss our final strategy. After you've moved 'S' to (1,3), I'll check my board to see if we've made sufficient progress to finish the game. If so, I'll let you know, and we can proceed with ending the game. 
	  } 
	   } 
	   } 
	 & & \\ 
 

    \theutterance \stepcounter{utterance}  

    & & & \multicolumn{2}{p{0.3\linewidth}}{\cellcolor[rgb]{0.95,0.95,0.95}{%
	\makecell[{{p{\linewidth}}}]{% 
	  \tt {\tiny [GM$|$GM]}  
	 SAY: I'm glad we found a viable option for moving 'A'. Yes, please proceed with moving 'A' to (1,4) to clear the path for 'S'. After that, you can move 'S' to (1,3) to align it with my 'S'. I'm ready for you to make these moves. Once you've done so, we can assess the final state of our boards and plan our last move to finish the game. Since this is our second-to-last round, let's make sure we're setting ourselves up for a successful conclusion. Please go ahead and make the moves, and then we can discuss our final strategy. After you've moved 'S' to (1,3), I'll check my board to see if we've made sufficient progress to finish the game. If so, I'll let you know, and we can proceed with ending the game. 
	  } 
	   } 
	   } 
	 & & \\ 
 

    \theutterance \stepcounter{utterance}  

    & & \multicolumn{4}{p{0.6\linewidth}}{\cellcolor[rgb]{0.9,0.9,0.9}{%
	\makecell[{{p{\linewidth}}}]{% 
	  \tt {\tiny [GM$\rangle$B]}  
	 Your message has been relayed to the other player.\\ \tt \\ \tt \\ \tt You are currently playing round 28 of maximum 28.\\ \tt You have collectively accumulated 14 of 16 penalties.\\ \tt The other player sent this message:\\ \tt \\ \tt \\ \tt "I'm glad we found a viable option for moving 'A'. Yes, please proceed with moving 'A' to (1,4) to clear the path for 'S'. After that, you can move 'S' to (1,3) to align it with my 'S'. I'm ready for you to make these moves. Once you've done so, we can assess the final state of our boards and plan our last move to finish the game. Since this is our second-to-last round, let's make sure we're setting ourselves up for a successful conclusion. Please go ahead and make the moves, and then we can discuss our final strategy. After you've moved 'S' to (1,3), I'll check my board to see if we've made sufficient progress to finish the game. If so, I'll let you know, and we can proceed with ending the game."\\ \tt \\ \tt \\ \tt What is your next command? 
	  } 
	   } 
	   } 
	 & \\ 
 

    \theutterance \stepcounter{utterance}  

    & & & \multicolumn{4}{p{0.6\linewidth}}{\cellcolor[rgb]{1,0.85,0.72}{%
	\makecell[{{p{\linewidth}}}]{% 
	  \tt {\tiny [GM$\langle$B]}  
	 MOVE: A, (1,4) 
	  } 
	   } 
	   } 
	  \\ 
 

    \theutterance \stepcounter{utterance}  

    & & & \multicolumn{2}{p{0.3\linewidth}}{\cellcolor[rgb]{0.95,0.95,0.95}{%
	\makecell[{{p{\linewidth}}}]{% 
	  \tt {\tiny [GM$|$GM]}  
	 MOVE: A, (1,4) 
	  } 
	   } 
	   } 
	 & & \\ 
 

    \theutterance \stepcounter{utterance}  

    & & & \multicolumn{2}{p{0.3\linewidth}}{\cellcolor[rgb]{0.95,0.95,0.95}{%
	\makecell[{{p{\linewidth}}}]{% 
	  \tt {\tiny [GM$|$GM]}  
	 Penalty: (1,4) is not empty, but contains '─'.\\ \tt \\ \tt You have collectively accumulated 15 of 16 penalties.\\ \tt Please try again! 
	  } 
	   } 
	   } 
	 & & \\ 
 

    \theutterance \stepcounter{utterance}  

    & & \multicolumn{4}{p{0.6\linewidth}}{\cellcolor[rgb]{0.9,0.9,0.9}{%
	\makecell[{{p{\linewidth}}}]{% 
	  \tt {\tiny [GM$\rangle$B]}  
	 Penalty: (1,4) is not empty, but contains '─'.\\ \tt \\ \tt You have collectively accumulated 15 of 16 penalties.\\ \tt Please try again! 
	  } 
	   } 
	   } 
	 & \\ 
 

    \theutterance \stepcounter{utterance}  

    & & & \multicolumn{4}{p{0.6\linewidth}}{\cellcolor[rgb]{1,0.85,0.72}{%
	\makecell[{{p{\linewidth}}}]{% 
	  \tt {\tiny [GM$\langle$B]}  
	 SAY: It seems that moving 'A' to (1,4) is not possible due to the grid structure. With only one round left, I think we should try to make the most of our last move. Considering the current state of our boards and the objects that are closest to being aligned, I suggest we focus on making a move that will bring us closest to achieving our goal. However, since we've accumulated 15 penalties, our next move needs to be extremely strategic. Let's discuss our options and plan our final move carefully to maximize our chances of success. What are your thoughts on our last move, and should we try to align a different object or make a different strategic move? 
	  } 
	   } 
	   } 
	  \\ 
 

    \theutterance \stepcounter{utterance}  

    & & & \multicolumn{2}{p{0.3\linewidth}}{\cellcolor[rgb]{0.95,0.95,0.95}{%
	\makecell[{{p{\linewidth}}}]{% 
	  \tt {\tiny [GM$|$GM]}  
	 SAY: It seems that moving 'A' to (1,4) is not possible due to the grid structure. With only one round left, I think we should try to make the most of our last move. Considering the current state of our boards and the objects that are closest to being aligned, I suggest we focus on making a move that will bring us closest to achieving our goal. However, since we've accumulated 15 penalties, our next move needs to be extremely strategic. Let's discuss our options and plan our final move carefully to maximize our chances of success. What are your thoughts on our last move, and should we try to align a different object or make a different strategic move? 
	  } 
	   } 
	   } 
	 & & \\ 
 

    \theutterance \stepcounter{utterance}  

    & & & \multicolumn{2}{p{0.3\linewidth}}{\cellcolor[rgb]{0.95,0.95,0.95}{%
	\makecell[{{p{\linewidth}}}]{% 
	  \tt {\tiny [GM$|$GM]}  
	 Maximum number of rounds reached 
	  } 
	   } 
	   } 
	 & & \\ 
 

    \theutterance \stepcounter{utterance}  

    & & & \multicolumn{2}{p{0.3\linewidth}}{\cellcolor[rgb]{0.95,0.95,0.95}{%
	\makecell[{{p{\linewidth}}}]{% 
	  \tt {\tiny [GM$|$GM]}  
	 * success: True\\ \tt * lose: False\\ \tt * aborted: False\\ \tt -------\\ \tt * Shifts: 1.00\\ \tt * Max Shifts: 12.00\\ \tt * Min Shifts: 6.00\\ \tt * End Distance Sum: 19.71\\ \tt * Init Distance Sum: 19.71\\ \tt * Expected Distance Sum: 29.33\\ \tt * Penalties: 15.00\\ \tt * Max Penalties: 16.00\\ \tt * Rounds: 28.00\\ \tt * Max Rounds: 28.00\\ \tt * Object Count: 7.00\\ \tt  
	  } 
	   } 
	   } 
	 & & \\ 
 

    \theutterance \stepcounter{utterance}  

    & & & \multicolumn{2}{p{0.3\linewidth}}{\cellcolor[rgb]{0.95,0.95,0.95}{%
	\makecell[{{p{\linewidth}}}]{% 
	  \tt {\tiny [GM$|$GM]}  
	 * Main Score: 0.97\\ \tt \\ \tt -------\\ \tt * Distance Score: 0.16\\ \tt * Coverage Score: 0.05\\ \tt * Penalty Score: 0.13\\ \tt * Alternative Penalty Score: 0.24\\ \tt * Alternative Main Score: 3.86\\ \tt \\ \tt -------\\ \tt * Shifts: 1.00\\ \tt * Max Shifts: 12.00\\ \tt * Min Shifts: 6.00\\ \tt * End Distance Sum: 19.71\\ \tt * Init Distance Sum: 19.71\\ \tt * Expected Distance Sum: 29.33\\ \tt * Penalties: 15.00\\ \tt * Max Penalties: 16.00\\ \tt * Rounds: 28.00\\ \tt * Max Rounds: 28.00\\ \tt * Object Count: 7.00\\ \tt  
	  } 
	   } 
	   } 
	 & & \\ 
 

\end{supertabular}
}

\end{document}
