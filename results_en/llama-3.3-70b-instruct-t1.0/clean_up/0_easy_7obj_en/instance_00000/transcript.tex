
\documentclass{article}
\usepackage{colortbl}
\usepackage{makecell}
\usepackage{multirow}
\usepackage{supertabular}

\begin{document}

\newcounter{utterance}

\centering \large Interaction Transcript for game `clean\_up', experiment `0\_easy\_7obj\_en', episode 0 with llama{-}3.3{-}70b{-}instruct{-}t1.0.
\vspace{24pt}

{ \footnotesize  \setcounter{utterance}{1}
\setlength{\tabcolsep}{0pt}
\begin{supertabular}{c@{$\;$}|p{.15\linewidth}@{}p{.15\linewidth}p{.15\linewidth}p{.15\linewidth}p{.15\linewidth}p{.15\linewidth}}
    \# & $\;$A & \multicolumn{4}{c}{Game Master} & $\;\:$B\\
    \hline

    \theutterance \stepcounter{utterance}  
    & & \multicolumn{4}{p{0.6\linewidth}}{
        \cellcolor[rgb]{0.9,0.9,0.9}{
            \makecell[{{p{\linewidth}}}]{
                \texttt{\tiny{[P1$\langle$GM]}}
                \texttt{I am your game master, and you are playing a collaborative game with the following grid as game board:} \\
\\ 
\texttt{```} \\
\texttt{ 1234567} \\
\texttt{╔═╤═══╤═╗} \\
\texttt{║◌│◌◌◌│◌║ 1} \\
\texttt{╟─┼──┬┘◌║ 2} \\
\texttt{║◌│◌◌│S◌║ 3} \\
\texttt{╟─┴──┘DT║ 4} \\
\texttt{║◌O◌◌◌◌◌║ 5} \\
\texttt{║◌◌◌◌◌◌◌║ 6} \\
\texttt{║◌◌APM◌◌║ 7} \\
\texttt{╚═══════╝} \\
\\ 
\texttt{```} \\
\\ 
\texttt{* The upper edge displays x{-}coordinates increasing to the right, and the right edge y{-}coordinates increasing downward.} \\
\texttt{* The following objects are randomly placed on your grid: 'P', 'O', 'T', 'S', 'D', 'A', 'M'.} \\
\\ 
\\ 
\texttt{The other player sees a variation of the game board, where the objects are placed at different random locations. You cannot see the other player's board, and they cannot see yours.} \\
\\ 
\\ 
\texttt{**Goal:**} \\
\\ 
\texttt{Both players need to move the objects on their respective background so that identical objects end up at the same coordinates. You have to communicate with the other player to agree upon a common goal configuration.} \\
\\ 
\\ 
\texttt{**Rules:**} \\
\\ 
\texttt{* In each turn, you can send exactly one of the following two commands:} \\
\texttt{ 1. `SAY: <MESSAGE>`: to send a message (everything up until the next line break) to the other player. I will forward it to your partner.} \\
\texttt{2. `MOVE: <OBJECT>, (<X>, <Y>)`: to move an object to a new position, where `<X>` is the column and `<Y>` is the row. I will inform you if your move was successful or not.} \\
\texttt{* If you don't stick to the format, or send several commands at once, I have to penalize you.} \\
\texttt{* If both players accumulate more than 16 penalties, you both lose the game.} \\
\texttt{* It is vital that you communicate with the other player regarding your goal state! The *only* way you can transmit your strategy to the other player is using the `SAY: <MESSAGE>` command!} \\
\\ 
\\ 
\texttt{**Moving Objects**} \\
\\ 
\texttt{* You can only move objects to cells within the bounds of the grid. The target cell must be empty, i.e., it must only contain the symbol '◌'.} \\
\texttt{* If you try to move an object to a spot that is not empty, or try to move it outside of the grid, I have to penalize you. You get another try.} \\
\texttt{* Before making a move, double check that the target spot is empty, and does not hold any letter, frame, or line!} \\
\\ 
\\ 
\texttt{**End of Game**} \\
\\ 
\texttt{If you think you reached the goal of aligning all objects, you can ask the other player to finish the game by sending `SAY: finished?`. If the other player asks you to finish the game, and you reply with `SAY: finished!`, the game will end.} \\
\\ 
\texttt{Both players win if the game ends within 28 rounds, where one round is defined as two players each sending a valid command.} \\
\\ 
\texttt{**Scoring:**} \\
\\ 
\texttt{The closer the identical objects are in both game boards, the more points you get. Penalties reduce your points. Can you beat the record?} \\
\\ 
\\ 
\texttt{Please send a message to the other player to start the game!} \\
            }
        }
    }
    & \\ \\

    \theutterance \stepcounter{utterance}  
    & \multicolumn{4}{p{0.6\linewidth}}{
        \cellcolor[rgb]{0.9,0.9,0.9}{
            \makecell[{{p{\linewidth}}}]{
                \texttt{\tiny{[P1$\rangle$GM]}}
                \texttt{SAY: Hello, let's start by identifying the objects on our boards. I have the objects 'P', 'O', 'T', 'S', 'D', 'A', and 'M'. Can you please confirm the objects you have on your board and let's discuss a strategy to align them?} \\
            }
        }
    }
    & & \\ \\

    \theutterance \stepcounter{utterance}  
    & & & \multicolumn{2}{p{0.3\linewidth}}{
        \cellcolor[rgb]{0.9,0.9,0.9}{
            \makecell[{{p{\linewidth}}}]{
                \texttt{\tiny{[GM$|$GM]}}
                \texttt{SAY: Hello, let's start by identifying the objects on our boards. I have the objects 'P', 'O', 'T', 'S', 'D', 'A', and 'M'. Can you please confirm the objects you have on your board and let's discuss a strategy to align them?} \\
            }
        }
    }
    & & \\ \\

    \theutterance \stepcounter{utterance}  
    & & \multicolumn{4}{p{0.6\linewidth}}{
        \cellcolor[rgb]{0.9,0.9,0.9}{
            \makecell[{{p{\linewidth}}}]{
                \texttt{\tiny{[GM$\rangle$P2]}}
                \texttt{I am your game master, and you are playing a collaborative game with the following grid as game board:} \\
\\ 
\texttt{```} \\
\texttt{ 1234567} \\
\texttt{╔═╤═══╤═╗} \\
\texttt{║◌│◌◌◌│◌║ 1} \\
\texttt{╟─┼──┬┘◌║ 2} \\
\texttt{║◌│◌M│◌T║ 3} \\
\texttt{╟─┴──┘P◌║ 4} \\
\texttt{║D◌◌◌◌A◌║ 5} \\
\texttt{║◌◌◌◌O◌◌║ 6} \\
\texttt{║◌◌S◌◌◌◌║ 7} \\
\texttt{╚═══════╝} \\
\\ 
\texttt{```} \\
\\ 
\texttt{* The upper edge displays x{-}coordinates increasing to the right, and the right edge y{-}coordinates increasing downward.} \\
\texttt{* The following objects are randomly placed on your grid: 'P', 'O', 'T', 'S', 'D', 'A', 'M'.} \\
\\ 
\\ 
\texttt{The other player sees a variation of the game board, where the objects are placed at different random locations. You cannot see the other player's board, and they cannot see yours.} \\
\\ 
\\ 
\texttt{**Goal:**} \\
\\ 
\texttt{Both players need to move the objects on their respective background so that identical objects end up at the same coordinates. You have to communicate with the other player to agree upon a common goal configuration.} \\
\\ 
\\ 
\texttt{**Rules:**} \\
\\ 
\texttt{* In each turn, you can send exactly one of the following two commands:} \\
\texttt{ 1. `SAY: <MESSAGE>`: to send a message (everything up until the next line break) to the other player. I will forward it to your partner.} \\
\texttt{2. `MOVE: <OBJECT>, (<X>, <Y>)`: to move an object to a new position, where `<X>` is the column and `<Y>` is the row. I will inform you if your move was successful or not.} \\
\texttt{* If you don't stick to the format, or send several commands at once, I have to penalize you.} \\
\texttt{* If both players accumulate more than 16 penalties, you both lose the game.} \\
\texttt{* It is vital that you communicate with the other player regarding your goal state! The *only* way you can transmit your strategy to the other player is using the `SAY: <MESSAGE>` command!} \\
\\ 
\\ 
\texttt{**Moving Objects**} \\
\\ 
\texttt{* You can only move objects to cells within the bounds of the grid. The target cell must be empty, i.e., it must only contain the symbol '◌'.} \\
\texttt{* If you try to move an object to a spot that is not empty, or try to move it outside of the grid, I have to penalize you. You get another try.} \\
\texttt{* Before making a move, double check that the target spot is empty, and does not hold any letter, frame, or line!} \\
\\ 
\\ 
\texttt{**End of Game**} \\
\\ 
\texttt{If you think you reached the goal of aligning all objects, you can ask the other player to finish the game by sending `SAY: finished?`. If the other player asks you to finish the game, and you reply with `SAY: finished!`, the game will end.} \\
\\ 
\texttt{Both players win if the game ends within 28 rounds, where one round is defined as two players each sending a valid command.} \\
\\ 
\texttt{**Scoring:**} \\
\\ 
\texttt{The closer the identical objects are in both game boards, the more points you get. Penalties reduce your points. Can you beat the record?} \\
\\ 
\texttt{The other player started the game by sending this message:} \\
\\ 
\texttt{"Hello, let's start by identifying the objects on our boards. I have the objects 'P', 'O', 'T', 'S', 'D', 'A', and 'M'. Can you please confirm the objects you have on your board and let's discuss a strategy to align them?"} \\
\\ 
\texttt{What is your first command?} \\
            }
        }
    }
    & \\ \\

    \theutterance \stepcounter{utterance}  
    & & & \multicolumn{4}{p{0.6\linewidth}}{
        \cellcolor[rgb]{0.9,0.9,0.9}{
            \makecell[{{p{\linewidth}}}]{
                \texttt{\tiny{[GM$\langle$P2]}}
                \texttt{Since the other player has initiated the conversation, my first command will be to respond to their message and confirm the objects on my board.} \\
\\ 
\texttt{`SAY: Hello! Yes, I also have the objects 'P', 'O', 'T', 'S', 'D', 'A', and 'M' on my board. Let's discuss a strategy to align them. How about we try to place each object in a specific location, such as the top{-}left or center of the board? What are your thoughts?`} \\
            }
        }
    }
     \\ \\

    \theutterance \stepcounter{utterance}  
    & & & \multicolumn{2}{p{0.3\linewidth}}{
        \cellcolor[rgb]{0.9,0.9,0.9}{
            \makecell[{{p{\linewidth}}}]{
                \texttt{\tiny{[GM$|$GM]}}
                \texttt{Since the other player has initiated the conversation, my first command will be to respond to their message and confirm the objects on my board.} \\
\\ 
\texttt{`SAY: Hello! Yes, I also have the objects 'P', 'O', 'T', 'S', 'D', 'A', and 'M' on my board. Let's discuss a strategy to align them. How about we try to place each object in a specific location, such as the top{-}left or center of the board? What are your thoughts?`} \\
            }
        }
    }
    & & \\ \\

    \theutterance \stepcounter{utterance}  
    & & & \multicolumn{2}{p{0.3\linewidth}}{
        \cellcolor[rgb]{0.9,0.9,0.9}{
            \makecell[{{p{\linewidth}}}]{
                \texttt{\tiny{[GM$|$GM]}}
                \texttt{Invalid format: head is not empty: 'Since the other player has initiated the conversation, my first command will be to respond to their message and confirm the objects on my board.} \\
\\ 
\texttt{'} \\
            }
        }
    }
    & & \\ \\

    \theutterance \stepcounter{utterance}  
    & & \multicolumn{4}{p{0.6\linewidth}}{
        \cellcolor[rgb]{0.9,0.9,0.9}{
            \makecell[{{p{\linewidth}}}]{
                \texttt{\tiny{[GM$\rangle$P2]}}
                \texttt{Penalty: Your message must not contain anything before the command!} \\
\\ 
\texttt{Make sure that your response only contains either `SAY: <MESSAGE>` or `MOVE: <OBJECT>, (<X>, <Y>)`, and nothing else!} \\
\\ 
\texttt{You have collectively accumulated 1 of 16 penalties.} \\
\texttt{Please try again!} \\
            }
        }
    }
    & \\ \\

    \theutterance \stepcounter{utterance}  
    & & & \multicolumn{4}{p{0.6\linewidth}}{
        \cellcolor[rgb]{0.9,0.9,0.9}{
            \makecell[{{p{\linewidth}}}]{
                \texttt{\tiny{[GM$\langle$P2]}}
                \texttt{SAY: Hello! Yes, I also have the objects 'P', 'O', 'T', 'S', 'D', 'A', and 'M' on my board. Let's discuss a strategy to align them. How about we try to place each object in a specific location, such as the top{-}left or center of the board? What are your thoughts?} \\
            }
        }
    }
     \\ \\

    \theutterance \stepcounter{utterance}  
    & & & \multicolumn{2}{p{0.3\linewidth}}{
        \cellcolor[rgb]{0.9,0.9,0.9}{
            \makecell[{{p{\linewidth}}}]{
                \texttt{\tiny{[GM$|$GM]}}
                \texttt{SAY: Hello! Yes, I also have the objects 'P', 'O', 'T', 'S', 'D', 'A', and 'M' on my board. Let's discuss a strategy to align them. How about we try to place each object in a specific location, such as the top{-}left or center of the board? What are your thoughts?} \\
            }
        }
    }
    & & \\ \\

    \theutterance \stepcounter{utterance}  
    & & \multicolumn{4}{p{0.6\linewidth}}{
        \cellcolor[rgb]{0.9,0.9,0.9}{
            \makecell[{{p{\linewidth}}}]{
                \texttt{\tiny{[P1$\langle$GM]}}
                \texttt{Your message has been relayed to the other player.} \\
\\ 
\\ 
\texttt{You are currently playing round 1 of maximum 28.} \\
\texttt{You have collectively accumulated 1 of 16 penalties.} \\
\texttt{The other player sent this message:} \\
\\ 
\\ 
\texttt{"Hello! Yes, I also have the objects 'P', 'O', 'T', 'S', 'D', 'A', and 'M' on my board. Let's discuss a strategy to align them. How about we try to place each object in a specific location, such as the top{-}left or center of the board? What are your thoughts?"} \\
\\ 
\\ 
\texttt{What is your next command?} \\
            }
        }
    }
    & \\ \\

    \theutterance \stepcounter{utterance}  
    & \multicolumn{4}{p{0.6\linewidth}}{
        \cellcolor[rgb]{0.9,0.9,0.9}{
            \makecell[{{p{\linewidth}}}]{
                \texttt{\tiny{[P1$\rangle$GM]}}
                \texttt{SAY: That sounds like a great idea! Let's designate a specific location for each object. How about we use the first row for 'P', 'O', and 'T', the second row for 'S' and 'D', and the last row for 'A' and 'M'? We can discuss the exact column positions for each object. What do you think about this general layout?} \\
            }
        }
    }
    & & \\ \\

    \theutterance \stepcounter{utterance}  
    & & & \multicolumn{2}{p{0.3\linewidth}}{
        \cellcolor[rgb]{0.9,0.9,0.9}{
            \makecell[{{p{\linewidth}}}]{
                \texttt{\tiny{[GM$|$GM]}}
                \texttt{SAY: That sounds like a great idea! Let's designate a specific location for each object. How about we use the first row for 'P', 'O', and 'T', the second row for 'S' and 'D', and the last row for 'A' and 'M'? We can discuss the exact column positions for each object. What do you think about this general layout?} \\
            }
        }
    }
    & & \\ \\

    \theutterance \stepcounter{utterance}  
    & & \multicolumn{4}{p{0.6\linewidth}}{
        \cellcolor[rgb]{0.9,0.9,0.9}{
            \makecell[{{p{\linewidth}}}]{
                \texttt{\tiny{[GM$\rangle$P2]}}
                \texttt{Your message has been relayed to the other player.} \\
\\ 
\\ 
\texttt{You are currently playing round 2 of maximum 28.} \\
\texttt{You have collectively accumulated 1 of 16 penalties.} \\
\texttt{The other player sent this message:} \\
\\ 
\\ 
\texttt{"That sounds like a great idea! Let's designate a specific location for each object. How about we use the first row for 'P', 'O', and 'T', the second row for 'S' and 'D', and the last row for 'A' and 'M'? We can discuss the exact column positions for each object. What do you think about this general layout?"} \\
\\ 
\\ 
\texttt{What is your next command?} \\
            }
        }
    }
    & \\ \\

    \theutterance \stepcounter{utterance}  
    & & & \multicolumn{4}{p{0.6\linewidth}}{
        \cellcolor[rgb]{0.9,0.9,0.9}{
            \makecell[{{p{\linewidth}}}]{
                \texttt{\tiny{[GM$\langle$P2]}}
                \texttt{SAY: That's a good starting point! Using the first row for 'P', 'O', and 'T', the second row for 'S' and 'D', and the last row for 'A' and 'M' sounds like a logical layout. How about we assign specific columns to each object, such as 'P' at (1,1), 'O' at (3,1), 'T' at (5,1), 'S' at (2,2), 'D' at (4,2), 'A' at (3,7), and 'M' at (5,7)? What are your thoughts on these specific positions?} \\
            }
        }
    }
     \\ \\

    \theutterance \stepcounter{utterance}  
    & & & \multicolumn{2}{p{0.3\linewidth}}{
        \cellcolor[rgb]{0.9,0.9,0.9}{
            \makecell[{{p{\linewidth}}}]{
                \texttt{\tiny{[GM$|$GM]}}
                \texttt{SAY: That's a good starting point! Using the first row for 'P', 'O', and 'T', the second row for 'S' and 'D', and the last row for 'A' and 'M' sounds like a logical layout. How about we assign specific columns to each object, such as 'P' at (1,1), 'O' at (3,1), 'T' at (5,1), 'S' at (2,2), 'D' at (4,2), 'A' at (3,7), and 'M' at (5,7)? What are your thoughts on these specific positions?} \\
            }
        }
    }
    & & \\ \\

    \theutterance \stepcounter{utterance}  
    & & \multicolumn{4}{p{0.6\linewidth}}{
        \cellcolor[rgb]{0.9,0.9,0.9}{
            \makecell[{{p{\linewidth}}}]{
                \texttt{\tiny{[P1$\langle$GM]}}
                \texttt{Your message has been relayed to the other player.} \\
\\ 
\\ 
\texttt{You are currently playing round 2 of maximum 28.} \\
\texttt{You have collectively accumulated 1 of 16 penalties.} \\
\texttt{The other player sent this message:} \\
\\ 
\\ 
\texttt{"That's a good starting point! Using the first row for 'P', 'O', and 'T', the second row for 'S' and 'D', and the last row for 'A' and 'M' sounds like a logical layout. How about we assign specific columns to each object, such as 'P' at (1,1), 'O' at (3,1), 'T' at (5,1), 'S' at (2,2), 'D' at (4,2), 'A' at (3,7), and 'M' at (5,7)? What are your thoughts on these specific positions?"} \\
\\ 
\\ 
\texttt{What is your next command?} \\
            }
        }
    }
    & \\ \\

    \theutterance \stepcounter{utterance}  
    & \multicolumn{4}{p{0.6\linewidth}}{
        \cellcolor[rgb]{0.9,0.9,0.9}{
            \makecell[{{p{\linewidth}}}]{
                \texttt{\tiny{[P1$\rangle$GM]}}
                \texttt{SAY: I like the assignments you've proposed. The positions 'P' at (1,1), 'O' at (3,1), 'T' at (5,1), 'S' at (2,2), 'D' at (4,2), 'A' at (3,7), and 'M' at (5,7) seem logical and organized. Before we proceed, I'd like to confirm that we're on the same page. Can you confirm that your board currently does not have any objects at these assigned positions, and that you're ready to start moving objects to these positions?} \\
            }
        }
    }
    & & \\ \\

    \theutterance \stepcounter{utterance}  
    & & & \multicolumn{2}{p{0.3\linewidth}}{
        \cellcolor[rgb]{0.9,0.9,0.9}{
            \makecell[{{p{\linewidth}}}]{
                \texttt{\tiny{[GM$|$GM]}}
                \texttt{SAY: I like the assignments you've proposed. The positions 'P' at (1,1), 'O' at (3,1), 'T' at (5,1), 'S' at (2,2), 'D' at (4,2), 'A' at (3,7), and 'M' at (5,7) seem logical and organized. Before we proceed, I'd like to confirm that we're on the same page. Can you confirm that your board currently does not have any objects at these assigned positions, and that you're ready to start moving objects to these positions?} \\
            }
        }
    }
    & & \\ \\

    \theutterance \stepcounter{utterance}  
    & & \multicolumn{4}{p{0.6\linewidth}}{
        \cellcolor[rgb]{0.9,0.9,0.9}{
            \makecell[{{p{\linewidth}}}]{
                \texttt{\tiny{[GM$\rangle$P2]}}
                \texttt{Your message has been relayed to the other player.} \\
\\ 
\\ 
\texttt{You are currently playing round 3 of maximum 28.} \\
\texttt{You have collectively accumulated 1 of 16 penalties.} \\
\texttt{The other player sent this message:} \\
\\ 
\\ 
\texttt{"I like the assignments you've proposed. The positions 'P' at (1,1), 'O' at (3,1), 'T' at (5,1), 'S' at (2,2), 'D' at (4,2), 'A' at (3,7), and 'M' at (5,7) seem logical and organized. Before we proceed, I'd like to confirm that we're on the same page. Can you confirm that your board currently does not have any objects at these assigned positions, and that you're ready to start moving objects to these positions?"} \\
\\ 
\\ 
\texttt{What is your next command?} \\
            }
        }
    }
    & \\ \\

    \theutterance \stepcounter{utterance}  
    & & & \multicolumn{4}{p{0.6\linewidth}}{
        \cellcolor[rgb]{0.9,0.9,0.9}{
            \makecell[{{p{\linewidth}}}]{
                \texttt{\tiny{[GM$\langle$P2]}}
                \texttt{SAY: Yes, I confirm that my board currently does not have any objects at the assigned positions, except for 'M' which is already at (3,3) and 'P' which is at (4,4) and other objects are in different positions. I'm ready to start moving objects to the assigned positions. I'll start by moving the objects that are not in their assigned positions. Let's synchronize our moves to avoid conflicts. Shall we start moving the objects one by one, or move all the objects that are not in their correct positions at once?} \\
            }
        }
    }
     \\ \\

    \theutterance \stepcounter{utterance}  
    & & & \multicolumn{2}{p{0.3\linewidth}}{
        \cellcolor[rgb]{0.9,0.9,0.9}{
            \makecell[{{p{\linewidth}}}]{
                \texttt{\tiny{[GM$|$GM]}}
                \texttt{SAY: Yes, I confirm that my board currently does not have any objects at the assigned positions, except for 'M' which is already at (3,3) and 'P' which is at (4,4) and other objects are in different positions. I'm ready to start moving objects to the assigned positions. I'll start by moving the objects that are not in their assigned positions. Let's synchronize our moves to avoid conflicts. Shall we start moving the objects one by one, or move all the objects that are not in their correct positions at once?} \\
            }
        }
    }
    & & \\ \\

    \theutterance \stepcounter{utterance}  
    & & \multicolumn{4}{p{0.6\linewidth}}{
        \cellcolor[rgb]{0.9,0.9,0.9}{
            \makecell[{{p{\linewidth}}}]{
                \texttt{\tiny{[P1$\langle$GM]}}
                \texttt{Your message has been relayed to the other player.} \\
\\ 
\\ 
\texttt{You are currently playing round 3 of maximum 28.} \\
\texttt{You have collectively accumulated 1 of 16 penalties.} \\
\texttt{The other player sent this message:} \\
\\ 
\\ 
\texttt{"Yes, I confirm that my board currently does not have any objects at the assigned positions, except for 'M' which is already at (3,3) and 'P' which is at (4,4) and other objects are in different positions. I'm ready to start moving objects to the assigned positions. I'll start by moving the objects that are not in their assigned positions. Let's synchronize our moves to avoid conflicts. Shall we start moving the objects one by one, or move all the objects that are not in their correct positions at once?"} \\
\\ 
\\ 
\texttt{What is your next command?} \\
            }
        }
    }
    & \\ \\

    \theutterance \stepcounter{utterance}  
    & \multicolumn{4}{p{0.6\linewidth}}{
        \cellcolor[rgb]{0.9,0.9,0.9}{
            \makecell[{{p{\linewidth}}}]{
                \texttt{\tiny{[P1$\rangle$GM]}}
                \texttt{SAY: It's good that we're confirming the positions. I also have objects that are not in their assigned positions. To avoid conflicts, let's move the objects one by one. I suggest we start with the objects that are furthest from their assigned positions. Can you please tell me which object on your board is furthest from its assigned position, and I'll do the same? We can then decide who moves first and what the first move should be.} \\
            }
        }
    }
    & & \\ \\

    \theutterance \stepcounter{utterance}  
    & & & \multicolumn{2}{p{0.3\linewidth}}{
        \cellcolor[rgb]{0.9,0.9,0.9}{
            \makecell[{{p{\linewidth}}}]{
                \texttt{\tiny{[GM$|$GM]}}
                \texttt{SAY: It's good that we're confirming the positions. I also have objects that are not in their assigned positions. To avoid conflicts, let's move the objects one by one. I suggest we start with the objects that are furthest from their assigned positions. Can you please tell me which object on your board is furthest from its assigned position, and I'll do the same? We can then decide who moves first and what the first move should be.} \\
            }
        }
    }
    & & \\ \\

    \theutterance \stepcounter{utterance}  
    & & \multicolumn{4}{p{0.6\linewidth}}{
        \cellcolor[rgb]{0.9,0.9,0.9}{
            \makecell[{{p{\linewidth}}}]{
                \texttt{\tiny{[GM$\rangle$P2]}}
                \texttt{Your message has been relayed to the other player.} \\
\\ 
\\ 
\texttt{You are currently playing round 4 of maximum 28.} \\
\texttt{You have collectively accumulated 1 of 16 penalties.} \\
\texttt{The other player sent this message:} \\
\\ 
\\ 
\texttt{"It's good that we're confirming the positions. I also have objects that are not in their assigned positions. To avoid conflicts, let's move the objects one by one. I suggest we start with the objects that are furthest from their assigned positions. Can you please tell me which object on your board is furthest from its assigned position, and I'll do the same? We can then decide who moves first and what the first move should be."} \\
\\ 
\\ 
\texttt{What is your next command?} \\
            }
        }
    }
    & \\ \\

    \theutterance \stepcounter{utterance}  
    & & & \multicolumn{4}{p{0.6\linewidth}}{
        \cellcolor[rgb]{0.9,0.9,0.9}{
            \makecell[{{p{\linewidth}}}]{
                \texttt{\tiny{[GM$\langle$P2]}}
                \texttt{SAY: After reviewing my board, I see that 'D' is at (1,5) which is quite far from its assigned position (4,2). I also notice that 'A' is at (5,5) and 'S' is at (3,7) but 'D' is the furthest. Let me know which object is furthest from its assigned position on your board, and we can decide who moves first.} \\
            }
        }
    }
     \\ \\

    \theutterance \stepcounter{utterance}  
    & & & \multicolumn{2}{p{0.3\linewidth}}{
        \cellcolor[rgb]{0.9,0.9,0.9}{
            \makecell[{{p{\linewidth}}}]{
                \texttt{\tiny{[GM$|$GM]}}
                \texttt{SAY: After reviewing my board, I see that 'D' is at (1,5) which is quite far from its assigned position (4,2). I also notice that 'A' is at (5,5) and 'S' is at (3,7) but 'D' is the furthest. Let me know which object is furthest from its assigned position on your board, and we can decide who moves first.} \\
            }
        }
    }
    & & \\ \\

    \theutterance \stepcounter{utterance}  
    & & \multicolumn{4}{p{0.6\linewidth}}{
        \cellcolor[rgb]{0.9,0.9,0.9}{
            \makecell[{{p{\linewidth}}}]{
                \texttt{\tiny{[P1$\langle$GM]}}
                \texttt{Your message has been relayed to the other player.} \\
\\ 
\\ 
\texttt{You are currently playing round 4 of maximum 28.} \\
\texttt{You have collectively accumulated 1 of 16 penalties.} \\
\texttt{The other player sent this message:} \\
\\ 
\\ 
\texttt{"After reviewing my board, I see that 'D' is at (1,5) which is quite far from its assigned position (4,2). I also notice that 'A' is at (5,5) and 'S' is at (3,7) but 'D' is the furthest. Let me know which object is furthest from its assigned position on your board, and we can decide who moves first."} \\
\\ 
\\ 
\texttt{What is your next command?} \\
            }
        }
    }
    & \\ \\

    \theutterance \stepcounter{utterance}  
    & \multicolumn{4}{p{0.6\linewidth}}{
        \cellcolor[rgb]{0.9,0.9,0.9}{
            \makecell[{{p{\linewidth}}}]{
                \texttt{\tiny{[P1$\rangle$GM]}}
                \texttt{SAY: On my board, 'O' is currently at (2,5) which is furthest from its assigned position (3,1). Since 'D' on your board is at (1,5) and needs to move to (4,2), and 'O' on my board is at (2,5) and needs to move to (3,1), it seems like we both have objects that need to move a significant distance. How about we both move our respective objects ('D' on your board and 'O' on my board) to an intermediate position that is closer to their final destination, and then reassess from there?} \\
            }
        }
    }
    & & \\ \\

    \theutterance \stepcounter{utterance}  
    & & & \multicolumn{2}{p{0.3\linewidth}}{
        \cellcolor[rgb]{0.9,0.9,0.9}{
            \makecell[{{p{\linewidth}}}]{
                \texttt{\tiny{[GM$|$GM]}}
                \texttt{SAY: On my board, 'O' is currently at (2,5) which is furthest from its assigned position (3,1). Since 'D' on your board is at (1,5) and needs to move to (4,2), and 'O' on my board is at (2,5) and needs to move to (3,1), it seems like we both have objects that need to move a significant distance. How about we both move our respective objects ('D' on your board and 'O' on my board) to an intermediate position that is closer to their final destination, and then reassess from there?} \\
            }
        }
    }
    & & \\ \\

    \theutterance \stepcounter{utterance}  
    & & \multicolumn{4}{p{0.6\linewidth}}{
        \cellcolor[rgb]{0.9,0.9,0.9}{
            \makecell[{{p{\linewidth}}}]{
                \texttt{\tiny{[GM$\rangle$P2]}}
                \texttt{Your message has been relayed to the other player.} \\
\\ 
\\ 
\texttt{You are currently playing round 5 of maximum 28.} \\
\texttt{You have collectively accumulated 1 of 16 penalties.} \\
\texttt{The other player sent this message:} \\
\\ 
\\ 
\texttt{"On my board, 'O' is currently at (2,5) which is furthest from its assigned position (3,1). Since 'D' on your board is at (1,5) and needs to move to (4,2), and 'O' on my board is at (2,5) and needs to move to (3,1), it seems like we both have objects that need to move a significant distance. How about we both move our respective objects ('D' on your board and 'O' on my board) to an intermediate position that is closer to their final destination, and then reassess from there?"} \\
\\ 
\\ 
\texttt{What is your next command?} \\
            }
        }
    }
    & \\ \\

    \theutterance \stepcounter{utterance}  
    & & & \multicolumn{4}{p{0.6\linewidth}}{
        \cellcolor[rgb]{0.9,0.9,0.9}{
            \makecell[{{p{\linewidth}}}]{
                \texttt{\tiny{[GM$\langle$P2]}}
                \texttt{SAY: Moving 'D' and 'O' to an intermediate position sounds like a good plan. For 'D' on my board, I can move it from (1,5) to (1,3) as an intermediate step, which would bring it closer to its final destination (4,2). What intermediate position were you thinking of moving 'O' to on your board, and shall we make our first moves to these intermediate positions simultaneously?} \\
            }
        }
    }
     \\ \\

    \theutterance \stepcounter{utterance}  
    & & & \multicolumn{2}{p{0.3\linewidth}}{
        \cellcolor[rgb]{0.9,0.9,0.9}{
            \makecell[{{p{\linewidth}}}]{
                \texttt{\tiny{[GM$|$GM]}}
                \texttt{SAY: Moving 'D' and 'O' to an intermediate position sounds like a good plan. For 'D' on my board, I can move it from (1,5) to (1,3) as an intermediate step, which would bring it closer to its final destination (4,2). What intermediate position were you thinking of moving 'O' to on your board, and shall we make our first moves to these intermediate positions simultaneously?} \\
            }
        }
    }
    & & \\ \\

    \theutterance \stepcounter{utterance}  
    & & \multicolumn{4}{p{0.6\linewidth}}{
        \cellcolor[rgb]{0.9,0.9,0.9}{
            \makecell[{{p{\linewidth}}}]{
                \texttt{\tiny{[P1$\langle$GM]}}
                \texttt{Your message has been relayed to the other player.} \\
\\ 
\\ 
\texttt{You are currently playing round 5 of maximum 28.} \\
\texttt{You have collectively accumulated 1 of 16 penalties.} \\
\texttt{The other player sent this message:} \\
\\ 
\\ 
\texttt{"Moving 'D' and 'O' to an intermediate position sounds like a good plan. For 'D' on my board, I can move it from (1,5) to (1,3) as an intermediate step, which would bring it closer to its final destination (4,2). What intermediate position were you thinking of moving 'O' to on your board, and shall we make our first moves to these intermediate positions simultaneously?"} \\
\\ 
\\ 
\texttt{What is your next command?} \\
            }
        }
    }
    & \\ \\

    \theutterance \stepcounter{utterance}  
    & \multicolumn{4}{p{0.6\linewidth}}{
        \cellcolor[rgb]{0.9,0.9,0.9}{
            \makecell[{{p{\linewidth}}}]{
                \texttt{\tiny{[P1$\rangle$GM]}}
                \texttt{SAY: For 'O' on my board, I was thinking of moving it from (2,5) to (2,3) as an intermediate step, which would also bring it closer to its final destination (3,1). Moving 'D' to (1,3) and 'O' to (2,3) simultaneously sounds like a good first step. To confirm, I will move 'O' to (2,3) and you will move 'D' to (1,3). Let's make these moves and then reassess the board before deciding on the next steps. I'll make the move on my board. MOVE: O, (2, 3)} \\
            }
        }
    }
    & & \\ \\

    \theutterance \stepcounter{utterance}  
    & & & \multicolumn{2}{p{0.3\linewidth}}{
        \cellcolor[rgb]{0.9,0.9,0.9}{
            \makecell[{{p{\linewidth}}}]{
                \texttt{\tiny{[GM$|$GM]}}
                \texttt{SAY: For 'O' on my board, I was thinking of moving it from (2,5) to (2,3) as an intermediate step, which would also bring it closer to its final destination (3,1). Moving 'D' to (1,3) and 'O' to (2,3) simultaneously sounds like a good first step. To confirm, I will move 'O' to (2,3) and you will move 'D' to (1,3). Let's make these moves and then reassess the board before deciding on the next steps. I'll make the move on my board. MOVE: O, (2, 3)} \\
            }
        }
    }
    & & \\ \\

    \theutterance \stepcounter{utterance}  
    & & & \multicolumn{2}{p{0.3\linewidth}}{
        \cellcolor[rgb]{0.9,0.9,0.9}{
            \makecell[{{p{\linewidth}}}]{
                \texttt{\tiny{[GM$|$GM]}}
                \texttt{Invalid response format: SAY: For 'O' on my board, I was thinking of moving it from (2,5) to (2,3) as an intermediate step, which would also bring it closer to its final destination (3,1). Moving 'D' to (1,3) and 'O' to (2,3) simultaneously sounds like a good first step. To confirm, I will move 'O' to (2,3) and you will move 'D' to (1,3). Let's make these moves and then reassess the board before deciding on the next steps. I'll make the move on my board. MOVE: O, (2, 3)} \\
            }
        }
    }
    & & \\ \\

    \theutterance \stepcounter{utterance}  
    & & \multicolumn{4}{p{0.6\linewidth}}{
        \cellcolor[rgb]{0.9,0.9,0.9}{
            \makecell[{{p{\linewidth}}}]{
                \texttt{\tiny{[P1$\langle$GM]}}
                \texttt{Penalty: Your message contains more than one command!} \\
\\ 
\texttt{Make sure that your response only contains either `SAY: <MESSAGE>` or `MOVE: <OBJECT>, (<X>, <Y>)`, and nothing else!} \\
\\ 
\texttt{You have collectively accumulated 2 of 16 penalties.} \\
\texttt{Please try again!} \\
            }
        }
    }
    & \\ \\

    \theutterance \stepcounter{utterance}  
    & \multicolumn{4}{p{0.6\linewidth}}{
        \cellcolor[rgb]{0.9,0.9,0.9}{
            \makecell[{{p{\linewidth}}}]{
                \texttt{\tiny{[P1$\rangle$GM]}}
                \texttt{SAY: For 'O' on my board, I was thinking of moving it from (2,5) to (2,3) as an intermediate step, which would also bring it closer to its final destination (3,1). Moving 'D' to (1,3) and 'O' to (2,3) simultaneously sounds like a good first step. To confirm, I will move 'O' to (2,3) and you will move 'D' to (1,3). Let's make these moves and then reassess the board before deciding on the next steps.} \\
            }
        }
    }
    & & \\ \\

    \theutterance \stepcounter{utterance}  
    & & & \multicolumn{2}{p{0.3\linewidth}}{
        \cellcolor[rgb]{0.9,0.9,0.9}{
            \makecell[{{p{\linewidth}}}]{
                \texttt{\tiny{[GM$|$GM]}}
                \texttt{SAY: For 'O' on my board, I was thinking of moving it from (2,5) to (2,3) as an intermediate step, which would also bring it closer to its final destination (3,1). Moving 'D' to (1,3) and 'O' to (2,3) simultaneously sounds like a good first step. To confirm, I will move 'O' to (2,3) and you will move 'D' to (1,3). Let's make these moves and then reassess the board before deciding on the next steps.} \\
            }
        }
    }
    & & \\ \\

    \theutterance \stepcounter{utterance}  
    & & \multicolumn{4}{p{0.6\linewidth}}{
        \cellcolor[rgb]{0.9,0.9,0.9}{
            \makecell[{{p{\linewidth}}}]{
                \texttt{\tiny{[GM$\rangle$P2]}}
                \texttt{Your message has been relayed to the other player.} \\
\\ 
\\ 
\texttt{You are currently playing round 6 of maximum 28.} \\
\texttt{You have collectively accumulated 2 of 16 penalties.} \\
\texttt{The other player sent this message:} \\
\\ 
\\ 
\texttt{"For 'O' on my board, I was thinking of moving it from (2,5) to (2,3) as an intermediate step, which would also bring it closer to its final destination (3,1). Moving 'D' to (1,3) and 'O' to (2,3) simultaneously sounds like a good first step. To confirm, I will move 'O' to (2,3) and you will move 'D' to (1,3). Let's make these moves and then reassess the board before deciding on the next steps."} \\
\\ 
\\ 
\texttt{What is your next command?} \\
            }
        }
    }
    & \\ \\

    \theutterance \stepcounter{utterance}  
    & & & \multicolumn{4}{p{0.6\linewidth}}{
        \cellcolor[rgb]{0.9,0.9,0.9}{
            \makecell[{{p{\linewidth}}}]{
                \texttt{\tiny{[GM$\langle$P2]}}
                \texttt{SAY: Yes, that's correct. You will move 'O' to (2,3) and I will move 'D' to (1,3). Let's make these moves. I'm ready to execute my part of the plan. Please confirm once you've made your move, and then I'll proceed with moving 'D' to (1,3).} \\
            }
        }
    }
     \\ \\

    \theutterance \stepcounter{utterance}  
    & & & \multicolumn{2}{p{0.3\linewidth}}{
        \cellcolor[rgb]{0.9,0.9,0.9}{
            \makecell[{{p{\linewidth}}}]{
                \texttt{\tiny{[GM$|$GM]}}
                \texttt{SAY: Yes, that's correct. You will move 'O' to (2,3) and I will move 'D' to (1,3). Let's make these moves. I'm ready to execute my part of the plan. Please confirm once you've made your move, and then I'll proceed with moving 'D' to (1,3).} \\
            }
        }
    }
    & & \\ \\

    \theutterance \stepcounter{utterance}  
    & & \multicolumn{4}{p{0.6\linewidth}}{
        \cellcolor[rgb]{0.9,0.9,0.9}{
            \makecell[{{p{\linewidth}}}]{
                \texttt{\tiny{[P1$\langle$GM]}}
                \texttt{Your message has been relayed to the other player.} \\
\\ 
\\ 
\texttt{You are currently playing round 6 of maximum 28.} \\
\texttt{You have collectively accumulated 2 of 16 penalties.} \\
\texttt{The other player sent this message:} \\
\\ 
\\ 
\texttt{"Yes, that's correct. You will move 'O' to (2,3) and I will move 'D' to (1,3). Let's make these moves. I'm ready to execute my part of the plan. Please confirm once you've made your move, and then I'll proceed with moving 'D' to (1,3)."} \\
\\ 
\\ 
\texttt{What is your next command?} \\
            }
        }
    }
    & \\ \\

    \theutterance \stepcounter{utterance}  
    & \multicolumn{4}{p{0.6\linewidth}}{
        \cellcolor[rgb]{0.9,0.9,0.9}{
            \makecell[{{p{\linewidth}}}]{
                \texttt{\tiny{[P1$\rangle$GM]}}
                \texttt{MOVE: O, (2, 3)} \\
            }
        }
    }
    & & \\ \\

    \theutterance \stepcounter{utterance}  
    & & & \multicolumn{2}{p{0.3\linewidth}}{
        \cellcolor[rgb]{0.9,0.9,0.9}{
            \makecell[{{p{\linewidth}}}]{
                \texttt{\tiny{[GM$|$GM]}}
                \texttt{MOVE: O, (2, 3)} \\
            }
        }
    }
    & & \\ \\

    \theutterance \stepcounter{utterance}  
    & & & \multicolumn{2}{p{0.3\linewidth}}{
        \cellcolor[rgb]{0.9,0.9,0.9}{
            \makecell[{{p{\linewidth}}}]{
                \texttt{\tiny{[GM$|$GM]}}
                \texttt{Penalty: (2,3) is not empty, but contains '│'.} \\
\\ 
\texttt{You have collectively accumulated 3 of 16 penalties.} \\
\texttt{Please try again!} \\
            }
        }
    }
    & & \\ \\

    \theutterance \stepcounter{utterance}  
    & & \multicolumn{4}{p{0.6\linewidth}}{
        \cellcolor[rgb]{0.9,0.9,0.9}{
            \makecell[{{p{\linewidth}}}]{
                \texttt{\tiny{[P1$\langle$GM]}}
                \texttt{Penalty: (2,3) is not empty, but contains '│'.} \\
\\ 
\texttt{You have collectively accumulated 3 of 16 penalties.} \\
\texttt{Please try again!} \\
            }
        }
    }
    & \\ \\

    \theutterance \stepcounter{utterance}  
    & \multicolumn{4}{p{0.6\linewidth}}{
        \cellcolor[rgb]{0.9,0.9,0.9}{
            \makecell[{{p{\linewidth}}}]{
                \texttt{\tiny{[P1$\rangle$GM]}}
                \texttt{MOVE: O, (3, 3)} \\
            }
        }
    }
    & & \\ \\

    \theutterance \stepcounter{utterance}  
    & & & \multicolumn{2}{p{0.3\linewidth}}{
        \cellcolor[rgb]{0.9,0.9,0.9}{
            \makecell[{{p{\linewidth}}}]{
                \texttt{\tiny{[GM$|$GM]}}
                \texttt{MOVE: O, (3, 3)} \\
            }
        }
    }
    & & \\ \\

    \theutterance \stepcounter{utterance}  
    & & & \multicolumn{2}{p{0.3\linewidth}}{
        \cellcolor[rgb]{0.9,0.9,0.9}{
            \makecell[{{p{\linewidth}}}]{
                \texttt{\tiny{[GM$|$GM]}}
                \texttt{Moved 'O' to (3,3) successfully. Your updated grid looks like this:} \\
\texttt{```} \\
\texttt{ 1234567} \\
\texttt{╔═╤═══╤═╗} \\
\texttt{║◌│◌◌◌│◌║ 1} \\
\texttt{╟─┼──┬┘◌║ 2} \\
\texttt{║◌│O◌│S◌║ 3} \\
\texttt{╟─┴──┘DT║ 4} \\
\texttt{║◌◌◌◌◌◌◌║ 5} \\
\texttt{║◌◌◌◌◌◌◌║ 6} \\
\texttt{║◌◌APM◌◌║ 7} \\
\texttt{╚═══════╝} \\
\texttt{```} \\
            }
        }
    }
    & & \\ \\

    \theutterance \stepcounter{utterance}  
    & & \multicolumn{4}{p{0.6\linewidth}}{
        \cellcolor[rgb]{0.9,0.9,0.9}{
            \makecell[{{p{\linewidth}}}]{
                \texttt{\tiny{[GM$\rangle$P2]}}
                \texttt{Your message has been relayed to the other player.} \\
\\ 
\\ 
\texttt{You are currently playing round 7 of maximum 28.} \\
\texttt{You have collectively accumulated 3 of 16 penalties.} \\
\texttt{The other player moved an object on their grid.} \\
\\ 
\\ 
\texttt{What is your next command?} \\
            }
        }
    }
    & \\ \\

    \theutterance \stepcounter{utterance}  
    & & & \multicolumn{4}{p{0.6\linewidth}}{
        \cellcolor[rgb]{0.9,0.9,0.9}{
            \makecell[{{p{\linewidth}}}]{
                \texttt{\tiny{[GM$\langle$P2]}}
                \texttt{MOVE: D, (1,3)} \\
            }
        }
    }
     \\ \\

    \theutterance \stepcounter{utterance}  
    & & & \multicolumn{2}{p{0.3\linewidth}}{
        \cellcolor[rgb]{0.9,0.9,0.9}{
            \makecell[{{p{\linewidth}}}]{
                \texttt{\tiny{[GM$|$GM]}}
                \texttt{MOVE: D, (1,3)} \\
            }
        }
    }
    & & \\ \\

    \theutterance \stepcounter{utterance}  
    & & & \multicolumn{2}{p{0.3\linewidth}}{
        \cellcolor[rgb]{0.9,0.9,0.9}{
            \makecell[{{p{\linewidth}}}]{
                \texttt{\tiny{[GM$|$GM]}}
                \texttt{Moved 'D' to (1,3) successfully. Your updated grid looks like this:} \\
\texttt{```} \\
\texttt{ 1234567} \\
\texttt{╔═╤═══╤═╗} \\
\texttt{║◌│◌◌◌│◌║ 1} \\
\texttt{╟─┼──┬┘◌║ 2} \\
\texttt{║D│◌M│◌T║ 3} \\
\texttt{╟─┴──┘P◌║ 4} \\
\texttt{║◌◌◌◌◌A◌║ 5} \\
\texttt{║◌◌◌◌O◌◌║ 6} \\
\texttt{║◌◌S◌◌◌◌║ 7} \\
\texttt{╚═══════╝} \\
\texttt{```} \\
            }
        }
    }
    & & \\ \\

    \theutterance \stepcounter{utterance}  
    & & \multicolumn{4}{p{0.6\linewidth}}{
        \cellcolor[rgb]{0.9,0.9,0.9}{
            \makecell[{{p{\linewidth}}}]{
                \texttt{\tiny{[P1$\langle$GM]}}
                \texttt{Moved 'O' to (3,3) successfully. Your updated grid looks like this:} \\
\texttt{```} \\
\texttt{ 1234567} \\
\texttt{╔═╤═══╤═╗} \\
\texttt{║◌│◌◌◌│◌║ 1} \\
\texttt{╟─┼──┬┘◌║ 2} \\
\texttt{║◌│O◌│S◌║ 3} \\
\texttt{╟─┴──┘DT║ 4} \\
\texttt{║◌◌◌◌◌◌◌║ 5} \\
\texttt{║◌◌◌◌◌◌◌║ 6} \\
\texttt{║◌◌APM◌◌║ 7} \\
\texttt{╚═══════╝} \\
\texttt{```} \\
\texttt{You are currently playing round 7 of maximum 28.} \\
\texttt{You have collectively accumulated 3 of 16 penalties.} \\
\texttt{The other player moved an object on their grid.} \\
\\ 
\\ 
\texttt{What is your next command?} \\
            }
        }
    }
    & \\ \\

    \theutterance \stepcounter{utterance}  
    & \multicolumn{4}{p{0.6\linewidth}}{
        \cellcolor[rgb]{0.9,0.9,0.9}{
            \makecell[{{p{\linewidth}}}]{
                \texttt{\tiny{[P1$\rangle$GM]}}
                \texttt{SAY: I've moved 'O' to (3,3) on my board. Now it's your turn. Please confirm that you've moved 'D' to (1,3) on your board as planned. What's the current state of your board?} \\
            }
        }
    }
    & & \\ \\

    \theutterance \stepcounter{utterance}  
    & & & \multicolumn{2}{p{0.3\linewidth}}{
        \cellcolor[rgb]{0.9,0.9,0.9}{
            \makecell[{{p{\linewidth}}}]{
                \texttt{\tiny{[GM$|$GM]}}
                \texttt{SAY: I've moved 'O' to (3,3) on my board. Now it's your turn. Please confirm that you've moved 'D' to (1,3) on your board as planned. What's the current state of your board?} \\
            }
        }
    }
    & & \\ \\

    \theutterance \stepcounter{utterance}  
    & & \multicolumn{4}{p{0.6\linewidth}}{
        \cellcolor[rgb]{0.9,0.9,0.9}{
            \makecell[{{p{\linewidth}}}]{
                \texttt{\tiny{[GM$\rangle$P2]}}
                \texttt{Moved 'D' to (1,3) successfully. Your updated grid looks like this:} \\
\texttt{```} \\
\texttt{ 1234567} \\
\texttt{╔═╤═══╤═╗} \\
\texttt{║◌│◌◌◌│◌║ 1} \\
\texttt{╟─┼──┬┘◌║ 2} \\
\texttt{║D│◌M│◌T║ 3} \\
\texttt{╟─┴──┘P◌║ 4} \\
\texttt{║◌◌◌◌◌A◌║ 5} \\
\texttt{║◌◌◌◌O◌◌║ 6} \\
\texttt{║◌◌S◌◌◌◌║ 7} \\
\texttt{╚═══════╝} \\
\texttt{```} \\
\texttt{You are currently playing round 8 of maximum 28.} \\
\texttt{You have collectively accumulated 3 of 16 penalties.} \\
\texttt{The other player sent this message:} \\
\\ 
\\ 
\texttt{"I've moved 'O' to (3,3) on my board. Now it's your turn. Please confirm that you've moved 'D' to (1,3) on your board as planned. What's the current state of your board?"} \\
\\ 
\\ 
\texttt{What is your next command?} \\
            }
        }
    }
    & \\ \\

    \theutterance \stepcounter{utterance}  
    & & & \multicolumn{4}{p{0.6\linewidth}}{
        \cellcolor[rgb]{0.9,0.9,0.9}{
            \makecell[{{p{\linewidth}}}]{
                \texttt{\tiny{[GM$\langle$P2]}}
                \texttt{SAY: Yes, I've moved 'D' to (1,3) as planned, but it looks like there was a conflict with 'M' already being at (3,3). On my board, 'D' is now at (1,3) and 'M' is still at (3,3). The current state of my board is as shown above. It seems we need to reassess and adjust our plan to avoid conflicts like this. What's your suggestion on how to proceed?} \\
            }
        }
    }
     \\ \\

    \theutterance \stepcounter{utterance}  
    & & & \multicolumn{2}{p{0.3\linewidth}}{
        \cellcolor[rgb]{0.9,0.9,0.9}{
            \makecell[{{p{\linewidth}}}]{
                \texttt{\tiny{[GM$|$GM]}}
                \texttt{SAY: Yes, I've moved 'D' to (1,3) as planned, but it looks like there was a conflict with 'M' already being at (3,3). On my board, 'D' is now at (1,3) and 'M' is still at (3,3). The current state of my board is as shown above. It seems we need to reassess and adjust our plan to avoid conflicts like this. What's your suggestion on how to proceed?} \\
            }
        }
    }
    & & \\ \\

    \theutterance \stepcounter{utterance}  
    & & \multicolumn{4}{p{0.6\linewidth}}{
        \cellcolor[rgb]{0.9,0.9,0.9}{
            \makecell[{{p{\linewidth}}}]{
                \texttt{\tiny{[P1$\langle$GM]}}
                \texttt{Your message has been relayed to the other player.} \\
\\ 
\\ 
\texttt{You are currently playing round 8 of maximum 28.} \\
\texttt{You have collectively accumulated 3 of 16 penalties.} \\
\texttt{The other player sent this message:} \\
\\ 
\\ 
\texttt{"Yes, I've moved 'D' to (1,3) as planned, but it looks like there was a conflict with 'M' already being at (3,3). On my board, 'D' is now at (1,3) and 'M' is still at (3,3). The current state of my board is as shown above. It seems we need to reassess and adjust our plan to avoid conflicts like this. What's your suggestion on how to proceed?"} \\
\\ 
\\ 
\texttt{What is your next command?} \\
            }
        }
    }
    & \\ \\

    \theutterance \stepcounter{utterance}  
    & \multicolumn{4}{p{0.6\linewidth}}{
        \cellcolor[rgb]{0.9,0.9,0.9}{
            \makecell[{{p{\linewidth}}}]{
                \texttt{\tiny{[P1$\rangle$GM]}}
                \texttt{SAY: It seems like we've encountered an unexpected conflict. Let's reassess the situation. Since 'O' is now at (3,3) on my board and 'M' is at (3,3) on your board, we need to find an alternative position for either 'O' or 'M'. How about we try to move 'M' to its final assigned position (5,7) on both boards, and then re{-}evaluate the positions of 'O' and other objects?} \\
            }
        }
    }
    & & \\ \\

    \theutterance \stepcounter{utterance}  
    & & & \multicolumn{2}{p{0.3\linewidth}}{
        \cellcolor[rgb]{0.9,0.9,0.9}{
            \makecell[{{p{\linewidth}}}]{
                \texttt{\tiny{[GM$|$GM]}}
                \texttt{SAY: It seems like we've encountered an unexpected conflict. Let's reassess the situation. Since 'O' is now at (3,3) on my board and 'M' is at (3,3) on your board, we need to find an alternative position for either 'O' or 'M'. How about we try to move 'M' to its final assigned position (5,7) on both boards, and then re{-}evaluate the positions of 'O' and other objects?} \\
            }
        }
    }
    & & \\ \\

    \theutterance \stepcounter{utterance}  
    & & \multicolumn{4}{p{0.6\linewidth}}{
        \cellcolor[rgb]{0.9,0.9,0.9}{
            \makecell[{{p{\linewidth}}}]{
                \texttt{\tiny{[GM$\rangle$P2]}}
                \texttt{Your message has been relayed to the other player.} \\
\\ 
\\ 
\texttt{You are currently playing round 9 of maximum 28.} \\
\texttt{You have collectively accumulated 3 of 16 penalties.} \\
\texttt{The other player sent this message:} \\
\\ 
\\ 
\texttt{"It seems like we've encountered an unexpected conflict. Let's reassess the situation. Since 'O' is now at (3,3) on my board and 'M' is at (3,3) on your board, we need to find an alternative position for either 'O' or 'M'. How about we try to move 'M' to its final assigned position (5,7) on both boards, and then re{-}evaluate the positions of 'O' and other objects?"} \\
\\ 
\\ 
\texttt{What is your next command?} \\
            }
        }
    }
    & \\ \\

    \theutterance \stepcounter{utterance}  
    & & & \multicolumn{4}{p{0.6\linewidth}}{
        \cellcolor[rgb]{0.9,0.9,0.9}{
            \makecell[{{p{\linewidth}}}]{
                \texttt{\tiny{[GM$\langle$P2]}}
                \texttt{SAY: Moving 'M' to its final assigned position (5,7) is a good idea. I can move 'M' from (3,3) to (5,7) on my board. Please confirm that you'll do the same on your board, moving 'M' to (5,7), so we can ensure both boards have 'M' in the same position. After that, we can re{-}evaluate the positions of 'O' and other objects.} \\
            }
        }
    }
     \\ \\

    \theutterance \stepcounter{utterance}  
    & & & \multicolumn{2}{p{0.3\linewidth}}{
        \cellcolor[rgb]{0.9,0.9,0.9}{
            \makecell[{{p{\linewidth}}}]{
                \texttt{\tiny{[GM$|$GM]}}
                \texttt{SAY: Moving 'M' to its final assigned position (5,7) is a good idea. I can move 'M' from (3,3) to (5,7) on my board. Please confirm that you'll do the same on your board, moving 'M' to (5,7), so we can ensure both boards have 'M' in the same position. After that, we can re{-}evaluate the positions of 'O' and other objects.} \\
            }
        }
    }
    & & \\ \\

    \theutterance \stepcounter{utterance}  
    & & \multicolumn{4}{p{0.6\linewidth}}{
        \cellcolor[rgb]{0.9,0.9,0.9}{
            \makecell[{{p{\linewidth}}}]{
                \texttt{\tiny{[P1$\langle$GM]}}
                \texttt{Your message has been relayed to the other player.} \\
\\ 
\\ 
\texttt{You are currently playing round 9 of maximum 28.} \\
\texttt{You have collectively accumulated 3 of 16 penalties.} \\
\texttt{The other player sent this message:} \\
\\ 
\\ 
\texttt{"Moving 'M' to its final assigned position (5,7) is a good idea. I can move 'M' from (3,3) to (5,7) on my board. Please confirm that you'll do the same on your board, moving 'M' to (5,7), so we can ensure both boards have 'M' in the same position. After that, we can re{-}evaluate the positions of 'O' and other objects."} \\
\\ 
\\ 
\texttt{What is your next command?} \\
            }
        }
    }
    & \\ \\

    \theutterance \stepcounter{utterance}  
    & \multicolumn{4}{p{0.6\linewidth}}{
        \cellcolor[rgb]{0.9,0.9,0.9}{
            \makecell[{{p{\linewidth}}}]{
                \texttt{\tiny{[P1$\rangle$GM]}}
                \texttt{MOVE: M, (5, 7)} \\
            }
        }
    }
    & & \\ \\

    \theutterance \stepcounter{utterance}  
    & & & \multicolumn{2}{p{0.3\linewidth}}{
        \cellcolor[rgb]{0.9,0.9,0.9}{
            \makecell[{{p{\linewidth}}}]{
                \texttt{\tiny{[GM$|$GM]}}
                \texttt{MOVE: M, (5, 7)} \\
            }
        }
    }
    & & \\ \\

    \theutterance \stepcounter{utterance}  
    & & & \multicolumn{2}{p{0.3\linewidth}}{
        \cellcolor[rgb]{0.9,0.9,0.9}{
            \makecell[{{p{\linewidth}}}]{
                \texttt{\tiny{[GM$|$GM]}}
                \texttt{Penalty: (5,7) is not empty, but contains 'M'.} \\
\\ 
\texttt{You have collectively accumulated 4 of 16 penalties.} \\
\texttt{Please try again!} \\
            }
        }
    }
    & & \\ \\

    \theutterance \stepcounter{utterance}  
    & & \multicolumn{4}{p{0.6\linewidth}}{
        \cellcolor[rgb]{0.9,0.9,0.9}{
            \makecell[{{p{\linewidth}}}]{
                \texttt{\tiny{[P1$\langle$GM]}}
                \texttt{Penalty: (5,7) is not empty, but contains 'M'.} \\
\\ 
\texttt{You have collectively accumulated 4 of 16 penalties.} \\
\texttt{Please try again!} \\
            }
        }
    }
    & \\ \\

    \theutterance \stepcounter{utterance}  
    & \multicolumn{4}{p{0.6\linewidth}}{
        \cellcolor[rgb]{0.9,0.9,0.9}{
            \makecell[{{p{\linewidth}}}]{
                \texttt{\tiny{[P1$\rangle$GM]}}
                \texttt{SAY: It seems that 'M' is already at (5,7) on my board, so I don't need to move it. Please proceed with moving 'M' to (5,7) on your board, and then we can re{-}evaluate the positions of 'O' and other objects.} \\
            }
        }
    }
    & & \\ \\

    \theutterance \stepcounter{utterance}  
    & & & \multicolumn{2}{p{0.3\linewidth}}{
        \cellcolor[rgb]{0.9,0.9,0.9}{
            \makecell[{{p{\linewidth}}}]{
                \texttt{\tiny{[GM$|$GM]}}
                \texttt{SAY: It seems that 'M' is already at (5,7) on my board, so I don't need to move it. Please proceed with moving 'M' to (5,7) on your board, and then we can re{-}evaluate the positions of 'O' and other objects.} \\
            }
        }
    }
    & & \\ \\

    \theutterance \stepcounter{utterance}  
    & & \multicolumn{4}{p{0.6\linewidth}}{
        \cellcolor[rgb]{0.9,0.9,0.9}{
            \makecell[{{p{\linewidth}}}]{
                \texttt{\tiny{[GM$\rangle$P2]}}
                \texttt{Your message has been relayed to the other player.} \\
\\ 
\\ 
\texttt{You are currently playing round 10 of maximum 28.} \\
\texttt{You have collectively accumulated 4 of 16 penalties.} \\
\texttt{The other player sent this message:} \\
\\ 
\\ 
\texttt{"It seems that 'M' is already at (5,7) on my board, so I don't need to move it. Please proceed with moving 'M' to (5,7) on your board, and then we can re{-}evaluate the positions of 'O' and other objects."} \\
\\ 
\\ 
\texttt{What is your next command?} \\
            }
        }
    }
    & \\ \\

    \theutterance \stepcounter{utterance}  
    & & & \multicolumn{4}{p{0.6\linewidth}}{
        \cellcolor[rgb]{0.9,0.9,0.9}{
            \makecell[{{p{\linewidth}}}]{
                \texttt{\tiny{[GM$\langle$P2]}}
                \texttt{MOVE: M, (5,7)} \\
            }
        }
    }
     \\ \\

    \theutterance \stepcounter{utterance}  
    & & & \multicolumn{2}{p{0.3\linewidth}}{
        \cellcolor[rgb]{0.9,0.9,0.9}{
            \makecell[{{p{\linewidth}}}]{
                \texttt{\tiny{[GM$|$GM]}}
                \texttt{MOVE: M, (5,7)} \\
            }
        }
    }
    & & \\ \\

    \theutterance \stepcounter{utterance}  
    & & & \multicolumn{2}{p{0.3\linewidth}}{
        \cellcolor[rgb]{0.9,0.9,0.9}{
            \makecell[{{p{\linewidth}}}]{
                \texttt{\tiny{[GM$|$GM]}}
                \texttt{Moved 'M' to (5,7) successfully. Your updated grid looks like this:} \\
\texttt{```} \\
\texttt{ 1234567} \\
\texttt{╔═╤═══╤═╗} \\
\texttt{║◌│◌◌◌│◌║ 1} \\
\texttt{╟─┼──┬┘◌║ 2} \\
\texttt{║D│◌◌│◌T║ 3} \\
\texttt{╟─┴──┘P◌║ 4} \\
\texttt{║◌◌◌◌◌A◌║ 5} \\
\texttt{║◌◌◌◌O◌◌║ 6} \\
\texttt{║◌◌S◌M◌◌║ 7} \\
\texttt{╚═══════╝} \\
\texttt{```} \\
            }
        }
    }
    & & \\ \\

    \theutterance \stepcounter{utterance}  
    & & \multicolumn{4}{p{0.6\linewidth}}{
        \cellcolor[rgb]{0.9,0.9,0.9}{
            \makecell[{{p{\linewidth}}}]{
                \texttt{\tiny{[P1$\langle$GM]}}
                \texttt{Your message has been relayed to the other player.} \\
\\ 
\\ 
\texttt{You are currently playing round 10 of maximum 28.} \\
\texttt{You have collectively accumulated 4 of 16 penalties.} \\
\texttt{The other player moved an object on their grid.} \\
\\ 
\\ 
\texttt{What is your next command?} \\
            }
        }
    }
    & \\ \\

    \theutterance \stepcounter{utterance}  
    & \multicolumn{4}{p{0.6\linewidth}}{
        \cellcolor[rgb]{0.9,0.9,0.9}{
            \makecell[{{p{\linewidth}}}]{
                \texttt{\tiny{[P1$\rangle$GM]}}
                \texttt{SAY: Now that 'M' is at (5,7) on both boards, let's focus on 'O'. I had previously moved 'O' to (3,3) on my board. What's the current position of 'O' on your board, and what's the next step to get 'O' to its assigned position (3,1)?} \\
            }
        }
    }
    & & \\ \\

    \theutterance \stepcounter{utterance}  
    & & & \multicolumn{2}{p{0.3\linewidth}}{
        \cellcolor[rgb]{0.9,0.9,0.9}{
            \makecell[{{p{\linewidth}}}]{
                \texttt{\tiny{[GM$|$GM]}}
                \texttt{SAY: Now that 'M' is at (5,7) on both boards, let's focus on 'O'. I had previously moved 'O' to (3,3) on my board. What's the current position of 'O' on your board, and what's the next step to get 'O' to its assigned position (3,1)?} \\
            }
        }
    }
    & & \\ \\

    \theutterance \stepcounter{utterance}  
    & & \multicolumn{4}{p{0.6\linewidth}}{
        \cellcolor[rgb]{0.9,0.9,0.9}{
            \makecell[{{p{\linewidth}}}]{
                \texttt{\tiny{[GM$\rangle$P2]}}
                \texttt{Moved 'M' to (5,7) successfully. Your updated grid looks like this:} \\
\texttt{```} \\
\texttt{ 1234567} \\
\texttt{╔═╤═══╤═╗} \\
\texttt{║◌│◌◌◌│◌║ 1} \\
\texttt{╟─┼──┬┘◌║ 2} \\
\texttt{║D│◌◌│◌T║ 3} \\
\texttt{╟─┴──┘P◌║ 4} \\
\texttt{║◌◌◌◌◌A◌║ 5} \\
\texttt{║◌◌◌◌O◌◌║ 6} \\
\texttt{║◌◌S◌M◌◌║ 7} \\
\texttt{╚═══════╝} \\
\texttt{```} \\
\texttt{You are currently playing round 11 of maximum 28.} \\
\texttt{You have collectively accumulated 4 of 16 penalties.} \\
\texttt{The other player sent this message:} \\
\\ 
\\ 
\texttt{"Now that 'M' is at (5,7) on both boards, let's focus on 'O'. I had previously moved 'O' to (3,3) on my board. What's the current position of 'O' on your board, and what's the next step to get 'O' to its assigned position (3,1)?"} \\
\\ 
\\ 
\texttt{What is your next command?} \\
            }
        }
    }
    & \\ \\

    \theutterance \stepcounter{utterance}  
    & & & \multicolumn{4}{p{0.6\linewidth}}{
        \cellcolor[rgb]{0.9,0.9,0.9}{
            \makecell[{{p{\linewidth}}}]{
                \texttt{\tiny{[GM$\langle$P2]}}
                \texttt{SAY: On my board, 'O' is currently at (6,6), but since we are trying to get 'O' to its assigned position (3,1), I think there might be some confusion. I don't have 'O' at (3,3) like you do on your board. To get 'O' to (3,1), I can start moving it from its current position (6,6) to an intermediate position, and then eventually to (3,1). What's your plan for moving 'O' from (3,3) to (3,1) on your board?} \\
            }
        }
    }
     \\ \\

    \theutterance \stepcounter{utterance}  
    & & & \multicolumn{2}{p{0.3\linewidth}}{
        \cellcolor[rgb]{0.9,0.9,0.9}{
            \makecell[{{p{\linewidth}}}]{
                \texttt{\tiny{[GM$|$GM]}}
                \texttt{SAY: On my board, 'O' is currently at (6,6), but since we are trying to get 'O' to its assigned position (3,1), I think there might be some confusion. I don't have 'O' at (3,3) like you do on your board. To get 'O' to (3,1), I can start moving it from its current position (6,6) to an intermediate position, and then eventually to (3,1). What's your plan for moving 'O' from (3,3) to (3,1) on your board?} \\
            }
        }
    }
    & & \\ \\

    \theutterance \stepcounter{utterance}  
    & & \multicolumn{4}{p{0.6\linewidth}}{
        \cellcolor[rgb]{0.9,0.9,0.9}{
            \makecell[{{p{\linewidth}}}]{
                \texttt{\tiny{[P1$\langle$GM]}}
                \texttt{Your message has been relayed to the other player.} \\
\\ 
\\ 
\texttt{You are currently playing round 11 of maximum 28.} \\
\texttt{You have collectively accumulated 4 of 16 penalties.} \\
\texttt{The other player sent this message:} \\
\\ 
\\ 
\texttt{"On my board, 'O' is currently at (6,6), but since we are trying to get 'O' to its assigned position (3,1), I think there might be some confusion. I don't have 'O' at (3,3) like you do on your board. To get 'O' to (3,1), I can start moving it from its current position (6,6) to an intermediate position, and then eventually to (3,1). What's your plan for moving 'O' from (3,3) to (3,1) on your board?"} \\
\\ 
\\ 
\texttt{What is your next command?} \\
            }
        }
    }
    & \\ \\

    \theutterance \stepcounter{utterance}  
    & \multicolumn{4}{p{0.6\linewidth}}{
        \cellcolor[rgb]{0.9,0.9,0.9}{
            \makecell[{{p{\linewidth}}}]{
                \texttt{\tiny{[P1$\rangle$GM]}}
                \texttt{SAY: I understand the situation now. To move 'O' from (3,3) to (3,1) on my board, I can move it up two rows. I'll move 'O' to (3,2) as an intermediate step, and then eventually to (3,1). How about you move 'O' from (6,6) to an intermediate position, such as (6,5) or (5,6), to start bringing it closer to (3,1)?} \\
            }
        }
    }
    & & \\ \\

    \theutterance \stepcounter{utterance}  
    & & & \multicolumn{2}{p{0.3\linewidth}}{
        \cellcolor[rgb]{0.9,0.9,0.9}{
            \makecell[{{p{\linewidth}}}]{
                \texttt{\tiny{[GM$|$GM]}}
                \texttt{SAY: I understand the situation now. To move 'O' from (3,3) to (3,1) on my board, I can move it up two rows. I'll move 'O' to (3,2) as an intermediate step, and then eventually to (3,1). How about you move 'O' from (6,6) to an intermediate position, such as (6,5) or (5,6), to start bringing it closer to (3,1)?} \\
            }
        }
    }
    & & \\ \\

    \theutterance \stepcounter{utterance}  
    & & \multicolumn{4}{p{0.6\linewidth}}{
        \cellcolor[rgb]{0.9,0.9,0.9}{
            \makecell[{{p{\linewidth}}}]{
                \texttt{\tiny{[GM$\rangle$P2]}}
                \texttt{Your message has been relayed to the other player.} \\
\\ 
\\ 
\texttt{You are currently playing round 12 of maximum 28.} \\
\texttt{You have collectively accumulated 4 of 16 penalties.} \\
\texttt{The other player sent this message:} \\
\\ 
\\ 
\texttt{"I understand the situation now. To move 'O' from (3,3) to (3,1) on my board, I can move it up two rows. I'll move 'O' to (3,2) as an intermediate step, and then eventually to (3,1). How about you move 'O' from (6,6) to an intermediate position, such as (6,5) or (5,6), to start bringing it closer to (3,1)?"} \\
\\ 
\\ 
\texttt{What is your next command?} \\
            }
        }
    }
    & \\ \\

    \theutterance \stepcounter{utterance}  
    & & & \multicolumn{4}{p{0.6\linewidth}}{
        \cellcolor[rgb]{0.9,0.9,0.9}{
            \makecell[{{p{\linewidth}}}]{
                \texttt{\tiny{[GM$\langle$P2]}}
                \texttt{MOVE: O, (6,5)} \\
            }
        }
    }
     \\ \\

    \theutterance \stepcounter{utterance}  
    & & & \multicolumn{2}{p{0.3\linewidth}}{
        \cellcolor[rgb]{0.9,0.9,0.9}{
            \makecell[{{p{\linewidth}}}]{
                \texttt{\tiny{[GM$|$GM]}}
                \texttt{MOVE: O, (6,5)} \\
            }
        }
    }
    & & \\ \\

    \theutterance \stepcounter{utterance}  
    & & & \multicolumn{2}{p{0.3\linewidth}}{
        \cellcolor[rgb]{0.9,0.9,0.9}{
            \makecell[{{p{\linewidth}}}]{
                \texttt{\tiny{[GM$|$GM]}}
                \texttt{Penalty: (6,5) is not empty, but contains 'A'.} \\
\\ 
\texttt{You have collectively accumulated 5 of 16 penalties.} \\
\texttt{Please try again!} \\
            }
        }
    }
    & & \\ \\

    \theutterance \stepcounter{utterance}  
    & & \multicolumn{4}{p{0.6\linewidth}}{
        \cellcolor[rgb]{0.9,0.9,0.9}{
            \makecell[{{p{\linewidth}}}]{
                \texttt{\tiny{[GM$\rangle$P2]}}
                \texttt{Penalty: (6,5) is not empty, but contains 'A'.} \\
\\ 
\texttt{You have collectively accumulated 5 of 16 penalties.} \\
\texttt{Please try again!} \\
            }
        }
    }
    & \\ \\

    \theutterance \stepcounter{utterance}  
    & & & \multicolumn{4}{p{0.6\linewidth}}{
        \cellcolor[rgb]{0.9,0.9,0.9}{
            \makecell[{{p{\linewidth}}}]{
                \texttt{\tiny{[GM$\langle$P2]}}
                \texttt{MOVE: O, (5,6)} \\
            }
        }
    }
     \\ \\

    \theutterance \stepcounter{utterance}  
    & & & \multicolumn{2}{p{0.3\linewidth}}{
        \cellcolor[rgb]{0.9,0.9,0.9}{
            \makecell[{{p{\linewidth}}}]{
                \texttt{\tiny{[GM$|$GM]}}
                \texttt{MOVE: O, (5,6)} \\
            }
        }
    }
    & & \\ \\

    \theutterance \stepcounter{utterance}  
    & & & \multicolumn{2}{p{0.3\linewidth}}{
        \cellcolor[rgb]{0.9,0.9,0.9}{
            \makecell[{{p{\linewidth}}}]{
                \texttt{\tiny{[GM$|$GM]}}
                \texttt{Penalty: (5,6) is not empty, but contains 'O'.} \\
\\ 
\texttt{You have collectively accumulated 6 of 16 penalties.} \\
\texttt{Please try again!} \\
            }
        }
    }
    & & \\ \\

    \theutterance \stepcounter{utterance}  
    & & \multicolumn{4}{p{0.6\linewidth}}{
        \cellcolor[rgb]{0.9,0.9,0.9}{
            \makecell[{{p{\linewidth}}}]{
                \texttt{\tiny{[GM$\rangle$P2]}}
                \texttt{Penalty: (5,6) is not empty, but contains 'O'.} \\
\\ 
\texttt{You have collectively accumulated 6 of 16 penalties.} \\
\texttt{Please try again!} \\
            }
        }
    }
    & \\ \\

    \theutterance \stepcounter{utterance}  
    & & & \multicolumn{4}{p{0.6\linewidth}}{
        \cellcolor[rgb]{0.9,0.9,0.9}{
            \makecell[{{p{\linewidth}}}]{
                \texttt{\tiny{[GM$\langle$P2]}}
                \texttt{MOVE: O, (5,5)} \\
            }
        }
    }
     \\ \\

    \theutterance \stepcounter{utterance}  
    & & & \multicolumn{2}{p{0.3\linewidth}}{
        \cellcolor[rgb]{0.9,0.9,0.9}{
            \makecell[{{p{\linewidth}}}]{
                \texttt{\tiny{[GM$|$GM]}}
                \texttt{MOVE: O, (5,5)} \\
            }
        }
    }
    & & \\ \\

    \theutterance \stepcounter{utterance}  
    & & & \multicolumn{2}{p{0.3\linewidth}}{
        \cellcolor[rgb]{0.9,0.9,0.9}{
            \makecell[{{p{\linewidth}}}]{
                \texttt{\tiny{[GM$|$GM]}}
                \texttt{Moved 'O' to (5,5) successfully. Your updated grid looks like this:} \\
\texttt{```} \\
\texttt{ 1234567} \\
\texttt{╔═╤═══╤═╗} \\
\texttt{║◌│◌◌◌│◌║ 1} \\
\texttt{╟─┼──┬┘◌║ 2} \\
\texttt{║D│◌◌│◌T║ 3} \\
\texttt{╟─┴──┘P◌║ 4} \\
\texttt{║◌◌◌◌OA◌║ 5} \\
\texttt{║◌◌◌◌◌◌◌║ 6} \\
\texttt{║◌◌S◌M◌◌║ 7} \\
\texttt{╚═══════╝} \\
\texttt{```} \\
            }
        }
    }
    & & \\ \\

    \theutterance \stepcounter{utterance}  
    & & \multicolumn{4}{p{0.6\linewidth}}{
        \cellcolor[rgb]{0.9,0.9,0.9}{
            \makecell[{{p{\linewidth}}}]{
                \texttt{\tiny{[P1$\langle$GM]}}
                \texttt{Your message has been relayed to the other player.} \\
\\ 
\\ 
\texttt{You are currently playing round 12 of maximum 28.} \\
\texttt{You have collectively accumulated 6 of 16 penalties.} \\
\texttt{The other player moved an object on their grid.} \\
\\ 
\\ 
\texttt{What is your next command?} \\
            }
        }
    }
    & \\ \\

    \theutterance \stepcounter{utterance}  
    & \multicolumn{4}{p{0.6\linewidth}}{
        \cellcolor[rgb]{0.9,0.9,0.9}{
            \makecell[{{p{\linewidth}}}]{
                \texttt{\tiny{[P1$\rangle$GM]}}
                \texttt{MOVE: O, (3, 2)} \\
            }
        }
    }
    & & \\ \\

    \theutterance \stepcounter{utterance}  
    & & & \multicolumn{2}{p{0.3\linewidth}}{
        \cellcolor[rgb]{0.9,0.9,0.9}{
            \makecell[{{p{\linewidth}}}]{
                \texttt{\tiny{[GM$|$GM]}}
                \texttt{MOVE: O, (3, 2)} \\
            }
        }
    }
    & & \\ \\

    \theutterance \stepcounter{utterance}  
    & & & \multicolumn{2}{p{0.3\linewidth}}{
        \cellcolor[rgb]{0.9,0.9,0.9}{
            \makecell[{{p{\linewidth}}}]{
                \texttt{\tiny{[GM$|$GM]}}
                \texttt{Penalty: (3,2) is not empty, but contains '─'.} \\
\\ 
\texttt{You have collectively accumulated 7 of 16 penalties.} \\
\texttt{Please try again!} \\
            }
        }
    }
    & & \\ \\

    \theutterance \stepcounter{utterance}  
    & & \multicolumn{4}{p{0.6\linewidth}}{
        \cellcolor[rgb]{0.9,0.9,0.9}{
            \makecell[{{p{\linewidth}}}]{
                \texttt{\tiny{[P1$\langle$GM]}}
                \texttt{Penalty: (3,2) is not empty, but contains '─'.} \\
\\ 
\texttt{You have collectively accumulated 7 of 16 penalties.} \\
\texttt{Please try again!} \\
            }
        }
    }
    & \\ \\

    \theutterance \stepcounter{utterance}  
    & \multicolumn{4}{p{0.6\linewidth}}{
        \cellcolor[rgb]{0.9,0.9,0.9}{
            \makecell[{{p{\linewidth}}}]{
                \texttt{\tiny{[P1$\rangle$GM]}}
                \texttt{SAY: It seems that I need to find an alternative path for 'O' to move from (3,3) to (3,1). Since (3,2) is occupied by '─', I'll need to consider a different intermediate position. Can we discuss possible alternative routes for 'O' to reach (3,1) on both boards?} \\
            }
        }
    }
    & & \\ \\

    \theutterance \stepcounter{utterance}  
    & & & \multicolumn{2}{p{0.3\linewidth}}{
        \cellcolor[rgb]{0.9,0.9,0.9}{
            \makecell[{{p{\linewidth}}}]{
                \texttt{\tiny{[GM$|$GM]}}
                \texttt{SAY: It seems that I need to find an alternative path for 'O' to move from (3,3) to (3,1). Since (3,2) is occupied by '─', I'll need to consider a different intermediate position. Can we discuss possible alternative routes for 'O' to reach (3,1) on both boards?} \\
            }
        }
    }
    & & \\ \\

    \theutterance \stepcounter{utterance}  
    & & \multicolumn{4}{p{0.6\linewidth}}{
        \cellcolor[rgb]{0.9,0.9,0.9}{
            \makecell[{{p{\linewidth}}}]{
                \texttt{\tiny{[GM$\rangle$P2]}}
                \texttt{Moved 'O' to (5,5) successfully. Your updated grid looks like this:} \\
\texttt{```} \\
\texttt{ 1234567} \\
\texttt{╔═╤═══╤═╗} \\
\texttt{║◌│◌◌◌│◌║ 1} \\
\texttt{╟─┼──┬┘◌║ 2} \\
\texttt{║D│◌◌│◌T║ 3} \\
\texttt{╟─┴──┘P◌║ 4} \\
\texttt{║◌◌◌◌OA◌║ 5} \\
\texttt{║◌◌◌◌◌◌◌║ 6} \\
\texttt{║◌◌S◌M◌◌║ 7} \\
\texttt{╚═══════╝} \\
\texttt{```} \\
\texttt{You are currently playing round 13 of maximum 28.} \\
\texttt{You have collectively accumulated 7 of 16 penalties.} \\
\texttt{The other player sent this message:} \\
\\ 
\\ 
\texttt{"It seems that I need to find an alternative path for 'O' to move from (3,3) to (3,1). Since (3,2) is occupied by '─', I'll need to consider a different intermediate position. Can we discuss possible alternative routes for 'O' to reach (3,1) on both boards?"} \\
\\ 
\\ 
\texttt{What is your next command?} \\
            }
        }
    }
    & \\ \\

    \theutterance \stepcounter{utterance}  
    & & & \multicolumn{4}{p{0.6\linewidth}}{
        \cellcolor[rgb]{0.9,0.9,0.9}{
            \makecell[{{p{\linewidth}}}]{
                \texttt{\tiny{[GM$\langle$P2]}}
                \texttt{SAY: On my board, I've moved 'O' to (5,5) as an intermediate step. For 'O' to reach (3,1) from (5,5), it could potentially move to (5,1) and then to (3,1), or to (3,5) and then up to (3,1). Considering the occupied spaces on your board, such as (3,2) being occupied by '─', we should find a route that avoids these obstacles. What are your thoughts on possible alternative routes for 'O' to reach (3,1) on your board?} \\
            }
        }
    }
     \\ \\

    \theutterance \stepcounter{utterance}  
    & & & \multicolumn{2}{p{0.3\linewidth}}{
        \cellcolor[rgb]{0.9,0.9,0.9}{
            \makecell[{{p{\linewidth}}}]{
                \texttt{\tiny{[GM$|$GM]}}
                \texttt{SAY: On my board, I've moved 'O' to (5,5) as an intermediate step. For 'O' to reach (3,1) from (5,5), it could potentially move to (5,1) and then to (3,1), or to (3,5) and then up to (3,1). Considering the occupied spaces on your board, such as (3,2) being occupied by '─', we should find a route that avoids these obstacles. What are your thoughts on possible alternative routes for 'O' to reach (3,1) on your board?} \\
            }
        }
    }
    & & \\ \\

    \theutterance \stepcounter{utterance}  
    & & \multicolumn{4}{p{0.6\linewidth}}{
        \cellcolor[rgb]{0.9,0.9,0.9}{
            \makecell[{{p{\linewidth}}}]{
                \texttt{\tiny{[P1$\langle$GM]}}
                \texttt{Your message has been relayed to the other player.} \\
\\ 
\\ 
\texttt{You are currently playing round 13 of maximum 28.} \\
\texttt{You have collectively accumulated 7 of 16 penalties.} \\
\texttt{The other player sent this message:} \\
\\ 
\\ 
\texttt{"On my board, I've moved 'O' to (5,5) as an intermediate step. For 'O' to reach (3,1) from (5,5), it could potentially move to (5,1) and then to (3,1), or to (3,5) and then up to (3,1). Considering the occupied spaces on your board, such as (3,2) being occupied by '─', we should find a route that avoids these obstacles. What are your thoughts on possible alternative routes for 'O' to reach (3,1) on your board?"} \\
\\ 
\\ 
\texttt{What is your next command?} \\
            }
        }
    }
    & \\ \\

    \theutterance \stepcounter{utterance}  
    & \multicolumn{4}{p{0.6\linewidth}}{
        \cellcolor[rgb]{0.9,0.9,0.9}{
            \makecell[{{p{\linewidth}}}]{
                \texttt{\tiny{[P1$\rangle$GM]}}
                \texttt{SAY: Given the obstacles on my board, I think a possible route for 'O' to reach (3,1) could be to move it from (3,3) to (4,3), then to (4,1), and finally to (3,1). This route avoids the occupied spaces, including (3,2) which has '─'. What do you think about this proposed route, and can you suggest a similar route for 'O' to reach (3,1) on your board, considering its current position at (5,5)?} \\
            }
        }
    }
    & & \\ \\

    \theutterance \stepcounter{utterance}  
    & & & \multicolumn{2}{p{0.3\linewidth}}{
        \cellcolor[rgb]{0.9,0.9,0.9}{
            \makecell[{{p{\linewidth}}}]{
                \texttt{\tiny{[GM$|$GM]}}
                \texttt{SAY: Given the obstacles on my board, I think a possible route for 'O' to reach (3,1) could be to move it from (3,3) to (4,3), then to (4,1), and finally to (3,1). This route avoids the occupied spaces, including (3,2) which has '─'. What do you think about this proposed route, and can you suggest a similar route for 'O' to reach (3,1) on your board, considering its current position at (5,5)?} \\
            }
        }
    }
    & & \\ \\

    \theutterance \stepcounter{utterance}  
    & & \multicolumn{4}{p{0.6\linewidth}}{
        \cellcolor[rgb]{0.9,0.9,0.9}{
            \makecell[{{p{\linewidth}}}]{
                \texttt{\tiny{[GM$\rangle$P2]}}
                \texttt{Your message has been relayed to the other player.} \\
\\ 
\\ 
\texttt{You are currently playing round 14 of maximum 28.} \\
\texttt{You have collectively accumulated 7 of 16 penalties.} \\
\texttt{The other player sent this message:} \\
\\ 
\\ 
\texttt{"Given the obstacles on my board, I think a possible route for 'O' to reach (3,1) could be to move it from (3,3) to (4,3), then to (4,1), and finally to (3,1). This route avoids the occupied spaces, including (3,2) which has '─'. What do you think about this proposed route, and can you suggest a similar route for 'O' to reach (3,1) on your board, considering its current position at (5,5)?"} \\
\\ 
\\ 
\texttt{What is your next command?} \\
            }
        }
    }
    & \\ \\

    \theutterance \stepcounter{utterance}  
    & & & \multicolumn{4}{p{0.6\linewidth}}{
        \cellcolor[rgb]{0.9,0.9,0.9}{
            \makecell[{{p{\linewidth}}}]{
                \texttt{\tiny{[GM$\langle$P2]}}
                \texttt{SAY: The proposed route for your board seems reasonable, avoiding the occupied spaces. For my board, considering 'O' is currently at (5,5), a possible route could be to move 'O' from (5,5) to (5,1), and then to (3,1), which would bring it to the desired position. Alternatively, another route could be to move 'O' from (5,5) to (3,5), and then up to (3,1), similar to the route you proposed for your board but adapted to my board's current state. Which of these routes do you think would be more feasible, or do you have any suggestions for improvement?} \\
            }
        }
    }
     \\ \\

    \theutterance \stepcounter{utterance}  
    & & & \multicolumn{2}{p{0.3\linewidth}}{
        \cellcolor[rgb]{0.9,0.9,0.9}{
            \makecell[{{p{\linewidth}}}]{
                \texttt{\tiny{[GM$|$GM]}}
                \texttt{SAY: The proposed route for your board seems reasonable, avoiding the occupied spaces. For my board, considering 'O' is currently at (5,5), a possible route could be to move 'O' from (5,5) to (5,1), and then to (3,1), which would bring it to the desired position. Alternatively, another route could be to move 'O' from (5,5) to (3,5), and then up to (3,1), similar to the route you proposed for your board but adapted to my board's current state. Which of these routes do you think would be more feasible, or do you have any suggestions for improvement?} \\
            }
        }
    }
    & & \\ \\

    \theutterance \stepcounter{utterance}  
    & & \multicolumn{4}{p{0.6\linewidth}}{
        \cellcolor[rgb]{0.9,0.9,0.9}{
            \makecell[{{p{\linewidth}}}]{
                \texttt{\tiny{[P1$\langle$GM]}}
                \texttt{Your message has been relayed to the other player.} \\
\\ 
\\ 
\texttt{You are currently playing round 14 of maximum 28.} \\
\texttt{You have collectively accumulated 7 of 16 penalties.} \\
\texttt{The other player sent this message:} \\
\\ 
\\ 
\texttt{"The proposed route for your board seems reasonable, avoiding the occupied spaces. For my board, considering 'O' is currently at (5,5), a possible route could be to move 'O' from (5,5) to (5,1), and then to (3,1), which would bring it to the desired position. Alternatively, another route could be to move 'O' from (5,5) to (3,5), and then up to (3,1), similar to the route you proposed for your board but adapted to my board's current state. Which of these routes do you think would be more feasible, or do you have any suggestions for improvement?"} \\
\\ 
\\ 
\texttt{What is your next command?} \\
            }
        }
    }
    & \\ \\

    \theutterance \stepcounter{utterance}  
    & \multicolumn{4}{p{0.6\linewidth}}{
        \cellcolor[rgb]{0.9,0.9,0.9}{
            \makecell[{{p{\linewidth}}}]{
                \texttt{\tiny{[P1$\rangle$GM]}}
                \texttt{SAY: Both routes you proposed for your board seem feasible. However, considering the goal of aligning the objects on both boards, I think it would be beneficial to synchronize our movements as much as possible. How about we both move 'O' to the (4,1) position or equivalent as an intermediate step, and then from there to (3,1)? This could help us ensure that 'O' is aligned on both boards. What are your thoughts on this synchronized approach?} \\
            }
        }
    }
    & & \\ \\

    \theutterance \stepcounter{utterance}  
    & & & \multicolumn{2}{p{0.3\linewidth}}{
        \cellcolor[rgb]{0.9,0.9,0.9}{
            \makecell[{{p{\linewidth}}}]{
                \texttt{\tiny{[GM$|$GM]}}
                \texttt{SAY: Both routes you proposed for your board seem feasible. However, considering the goal of aligning the objects on both boards, I think it would be beneficial to synchronize our movements as much as possible. How about we both move 'O' to the (4,1) position or equivalent as an intermediate step, and then from there to (3,1)? This could help us ensure that 'O' is aligned on both boards. What are your thoughts on this synchronized approach?} \\
            }
        }
    }
    & & \\ \\

    \theutterance \stepcounter{utterance}  
    & & \multicolumn{4}{p{0.6\linewidth}}{
        \cellcolor[rgb]{0.9,0.9,0.9}{
            \makecell[{{p{\linewidth}}}]{
                \texttt{\tiny{[GM$\rangle$P2]}}
                \texttt{Your message has been relayed to the other player.} \\
\\ 
\\ 
\texttt{You are currently playing round 15 of maximum 28.} \\
\texttt{You have collectively accumulated 7 of 16 penalties.} \\
\texttt{The other player sent this message:} \\
\\ 
\\ 
\texttt{"Both routes you proposed for your board seem feasible. However, considering the goal of aligning the objects on both boards, I think it would be beneficial to synchronize our movements as much as possible. How about we both move 'O' to the (4,1) position or equivalent as an intermediate step, and then from there to (3,1)? This could help us ensure that 'O' is aligned on both boards. What are your thoughts on this synchronized approach?"} \\
\\ 
\\ 
\texttt{What is your next command?} \\
            }
        }
    }
    & \\ \\

    \theutterance \stepcounter{utterance}  
    & & & \multicolumn{4}{p{0.6\linewidth}}{
        \cellcolor[rgb]{0.9,0.9,0.9}{
            \makecell[{{p{\linewidth}}}]{
                \texttt{\tiny{[GM$\langle$P2]}}
                \texttt{SAY: I agree that synchronizing our movements could help align the objects on both boards. Moving 'O' to (4,1) or an equivalent position as an intermediate step before reaching (3,1) is a good plan. On my board, I can move 'O' from (5,5) to (5,1) and then to (4,1) to match the intermediate position you suggested, and finally to (3,1). Let's proceed with this synchronized approach, making sure to communicate each step to ensure we're aligned. Shall we start by moving 'O' to the intermediate position, and then confirm with each other before making the final move to (3,1)?} \\
            }
        }
    }
     \\ \\

    \theutterance \stepcounter{utterance}  
    & & & \multicolumn{2}{p{0.3\linewidth}}{
        \cellcolor[rgb]{0.9,0.9,0.9}{
            \makecell[{{p{\linewidth}}}]{
                \texttt{\tiny{[GM$|$GM]}}
                \texttt{SAY: I agree that synchronizing our movements could help align the objects on both boards. Moving 'O' to (4,1) or an equivalent position as an intermediate step before reaching (3,1) is a good plan. On my board, I can move 'O' from (5,5) to (5,1) and then to (4,1) to match the intermediate position you suggested, and finally to (3,1). Let's proceed with this synchronized approach, making sure to communicate each step to ensure we're aligned. Shall we start by moving 'O' to the intermediate position, and then confirm with each other before making the final move to (3,1)?} \\
            }
        }
    }
    & & \\ \\

    \theutterance \stepcounter{utterance}  
    & & \multicolumn{4}{p{0.6\linewidth}}{
        \cellcolor[rgb]{0.9,0.9,0.9}{
            \makecell[{{p{\linewidth}}}]{
                \texttt{\tiny{[P1$\langle$GM]}}
                \texttt{Your message has been relayed to the other player.} \\
\\ 
\\ 
\texttt{You are currently playing round 15 of maximum 28.} \\
\texttt{You have collectively accumulated 7 of 16 penalties.} \\
\texttt{The other player sent this message:} \\
\\ 
\\ 
\texttt{"I agree that synchronizing our movements could help align the objects on both boards. Moving 'O' to (4,1) or an equivalent position as an intermediate step before reaching (3,1) is a good plan. On my board, I can move 'O' from (5,5) to (5,1) and then to (4,1) to match the intermediate position you suggested, and finally to (3,1). Let's proceed with this synchronized approach, making sure to communicate each step to ensure we're aligned. Shall we start by moving 'O' to the intermediate position, and then confirm with each other before making the final move to (3,1)?"} \\
\\ 
\\ 
\texttt{What is your next command?} \\
            }
        }
    }
    & \\ \\

    \theutterance \stepcounter{utterance}  
    & \multicolumn{4}{p{0.6\linewidth}}{
        \cellcolor[rgb]{0.9,0.9,0.9}{
            \makecell[{{p{\linewidth}}}]{
                \texttt{\tiny{[P1$\rangle$GM]}}
                \texttt{MOVE: O, (4, 3)} \\
            }
        }
    }
    & & \\ \\

    \theutterance \stepcounter{utterance}  
    & & & \multicolumn{2}{p{0.3\linewidth}}{
        \cellcolor[rgb]{0.9,0.9,0.9}{
            \makecell[{{p{\linewidth}}}]{
                \texttt{\tiny{[GM$|$GM]}}
                \texttt{MOVE: O, (4, 3)} \\
            }
        }
    }
    & & \\ \\

    \theutterance \stepcounter{utterance}  
    & & & \multicolumn{2}{p{0.3\linewidth}}{
        \cellcolor[rgb]{0.9,0.9,0.9}{
            \makecell[{{p{\linewidth}}}]{
                \texttt{\tiny{[GM$|$GM]}}
                \texttt{Moved 'O' to (4,3) successfully. Your updated grid looks like this:} \\
\texttt{```} \\
\texttt{ 1234567} \\
\texttt{╔═╤═══╤═╗} \\
\texttt{║◌│◌◌◌│◌║ 1} \\
\texttt{╟─┼──┬┘◌║ 2} \\
\texttt{║◌│◌O│S◌║ 3} \\
\texttt{╟─┴──┘DT║ 4} \\
\texttt{║◌◌◌◌◌◌◌║ 5} \\
\texttt{║◌◌◌◌◌◌◌║ 6} \\
\texttt{║◌◌APM◌◌║ 7} \\
\texttt{╚═══════╝} \\
\texttt{```} \\
            }
        }
    }
    & & \\ \\

    \theutterance \stepcounter{utterance}  
    & & \multicolumn{4}{p{0.6\linewidth}}{
        \cellcolor[rgb]{0.9,0.9,0.9}{
            \makecell[{{p{\linewidth}}}]{
                \texttt{\tiny{[GM$\rangle$P2]}}
                \texttt{Your message has been relayed to the other player.} \\
\\ 
\\ 
\texttt{You are currently playing round 16 of maximum 28.} \\
\texttt{You have collectively accumulated 7 of 16 penalties.} \\
\texttt{The other player moved an object on their grid.} \\
\\ 
\\ 
\texttt{What is your next command?} \\
            }
        }
    }
    & \\ \\

    \theutterance \stepcounter{utterance}  
    & & & \multicolumn{4}{p{0.6\linewidth}}{
        \cellcolor[rgb]{0.9,0.9,0.9}{
            \makecell[{{p{\linewidth}}}]{
                \texttt{\tiny{[GM$\langle$P2]}}
                \texttt{MOVE: O, (5,1)} \\
            }
        }
    }
     \\ \\

    \theutterance \stepcounter{utterance}  
    & & & \multicolumn{2}{p{0.3\linewidth}}{
        \cellcolor[rgb]{0.9,0.9,0.9}{
            \makecell[{{p{\linewidth}}}]{
                \texttt{\tiny{[GM$|$GM]}}
                \texttt{MOVE: O, (5,1)} \\
            }
        }
    }
    & & \\ \\

    \theutterance \stepcounter{utterance}  
    & & & \multicolumn{2}{p{0.3\linewidth}}{
        \cellcolor[rgb]{0.9,0.9,0.9}{
            \makecell[{{p{\linewidth}}}]{
                \texttt{\tiny{[GM$|$GM]}}
                \texttt{Moved 'O' to (5,1) successfully. Your updated grid looks like this:} \\
\texttt{```} \\
\texttt{ 1234567} \\
\texttt{╔═╤═══╤═╗} \\
\texttt{║◌│◌◌O│◌║ 1} \\
\texttt{╟─┼──┬┘◌║ 2} \\
\texttt{║D│◌◌│◌T║ 3} \\
\texttt{╟─┴──┘P◌║ 4} \\
\texttt{║◌◌◌◌◌A◌║ 5} \\
\texttt{║◌◌◌◌◌◌◌║ 6} \\
\texttt{║◌◌S◌M◌◌║ 7} \\
\texttt{╚═══════╝} \\
\texttt{```} \\
            }
        }
    }
    & & \\ \\

    \theutterance \stepcounter{utterance}  
    & & \multicolumn{4}{p{0.6\linewidth}}{
        \cellcolor[rgb]{0.9,0.9,0.9}{
            \makecell[{{p{\linewidth}}}]{
                \texttt{\tiny{[P1$\langle$GM]}}
                \texttt{Moved 'O' to (4,3) successfully. Your updated grid looks like this:} \\
\texttt{```} \\
\texttt{ 1234567} \\
\texttt{╔═╤═══╤═╗} \\
\texttt{║◌│◌◌◌│◌║ 1} \\
\texttt{╟─┼──┬┘◌║ 2} \\
\texttt{║◌│◌O│S◌║ 3} \\
\texttt{╟─┴──┘DT║ 4} \\
\texttt{║◌◌◌◌◌◌◌║ 5} \\
\texttt{║◌◌◌◌◌◌◌║ 6} \\
\texttt{║◌◌APM◌◌║ 7} \\
\texttt{╚═══════╝} \\
\texttt{```} \\
\texttt{You are currently playing round 16 of maximum 28.} \\
\texttt{You have collectively accumulated 7 of 16 penalties.} \\
\texttt{The other player moved an object on their grid.} \\
\\ 
\\ 
\texttt{What is your next command?} \\
            }
        }
    }
    & \\ \\

    \theutterance \stepcounter{utterance}  
    & \multicolumn{4}{p{0.6\linewidth}}{
        \cellcolor[rgb]{0.9,0.9,0.9}{
            \makecell[{{p{\linewidth}}}]{
                \texttt{\tiny{[P1$\rangle$GM]}}
                \texttt{SAY: I've moved 'O' to (4,3) on my board. Now, I'll plan the next step to get 'O' to (3,1). I can move 'O' from (4,3) to (4,1) and then to (3,1). Please confirm the current position of 'O' on your board and let's synchronize our next moves to ensure 'O' reaches (3,1) on both boards.} \\
            }
        }
    }
    & & \\ \\

    \theutterance \stepcounter{utterance}  
    & & & \multicolumn{2}{p{0.3\linewidth}}{
        \cellcolor[rgb]{0.9,0.9,0.9}{
            \makecell[{{p{\linewidth}}}]{
                \texttt{\tiny{[GM$|$GM]}}
                \texttt{SAY: I've moved 'O' to (4,3) on my board. Now, I'll plan the next step to get 'O' to (3,1). I can move 'O' from (4,3) to (4,1) and then to (3,1). Please confirm the current position of 'O' on your board and let's synchronize our next moves to ensure 'O' reaches (3,1) on both boards.} \\
            }
        }
    }
    & & \\ \\

    \theutterance \stepcounter{utterance}  
    & & \multicolumn{4}{p{0.6\linewidth}}{
        \cellcolor[rgb]{0.9,0.9,0.9}{
            \makecell[{{p{\linewidth}}}]{
                \texttt{\tiny{[GM$\rangle$P2]}}
                \texttt{Moved 'O' to (5,1) successfully. Your updated grid looks like this:} \\
\texttt{```} \\
\texttt{ 1234567} \\
\texttt{╔═╤═══╤═╗} \\
\texttt{║◌│◌◌O│◌║ 1} \\
\texttt{╟─┼──┬┘◌║ 2} \\
\texttt{║D│◌◌│◌T║ 3} \\
\texttt{╟─┴──┘P◌║ 4} \\
\texttt{║◌◌◌◌◌A◌║ 5} \\
\texttt{║◌◌◌◌◌◌◌║ 6} \\
\texttt{║◌◌S◌M◌◌║ 7} \\
\texttt{╚═══════╝} \\
\texttt{```} \\
\texttt{You are currently playing round 17 of maximum 28.} \\
\texttt{You have collectively accumulated 7 of 16 penalties.} \\
\texttt{The other player sent this message:} \\
\\ 
\\ 
\texttt{"I've moved 'O' to (4,3) on my board. Now, I'll plan the next step to get 'O' to (3,1). I can move 'O' from (4,3) to (4,1) and then to (3,1). Please confirm the current position of 'O' on your board and let's synchronize our next moves to ensure 'O' reaches (3,1) on both boards."} \\
\\ 
\\ 
\texttt{What is your next command?} \\
            }
        }
    }
    & \\ \\

    \theutterance \stepcounter{utterance}  
    & & & \multicolumn{4}{p{0.6\linewidth}}{
        \cellcolor[rgb]{0.9,0.9,0.9}{
            \makecell[{{p{\linewidth}}}]{
                \texttt{\tiny{[GM$\langle$P2]}}
                \texttt{SAY: I've moved 'O' to (5,1) on my board. Now that we have 'O' at (5,1) on my board and you have 'O' at (4,3) on your board, we can plan the next step to get 'O' to (3,1) on both boards. Since you plan to move 'O' from (4,3) to (4,1) and then to (3,1), I can move 'O' from (5,1) to (4,1) and then to (3,1) to match your moves. Let's proceed with this plan and confirm with each other after each move to ensure we're synchronized.} \\
            }
        }
    }
     \\ \\

    \theutterance \stepcounter{utterance}  
    & & & \multicolumn{2}{p{0.3\linewidth}}{
        \cellcolor[rgb]{0.9,0.9,0.9}{
            \makecell[{{p{\linewidth}}}]{
                \texttt{\tiny{[GM$|$GM]}}
                \texttt{SAY: I've moved 'O' to (5,1) on my board. Now that we have 'O' at (5,1) on my board and you have 'O' at (4,3) on your board, we can plan the next step to get 'O' to (3,1) on both boards. Since you plan to move 'O' from (4,3) to (4,1) and then to (3,1), I can move 'O' from (5,1) to (4,1) and then to (3,1) to match your moves. Let's proceed with this plan and confirm with each other after each move to ensure we're synchronized.} \\
            }
        }
    }
    & & \\ \\

    \theutterance \stepcounter{utterance}  
    & & \multicolumn{4}{p{0.6\linewidth}}{
        \cellcolor[rgb]{0.9,0.9,0.9}{
            \makecell[{{p{\linewidth}}}]{
                \texttt{\tiny{[P1$\langle$GM]}}
                \texttt{Your message has been relayed to the other player.} \\
\\ 
\\ 
\texttt{You are currently playing round 17 of maximum 28.} \\
\texttt{You have collectively accumulated 7 of 16 penalties.} \\
\texttt{The other player sent this message:} \\
\\ 
\\ 
\texttt{"I've moved 'O' to (5,1) on my board. Now that we have 'O' at (5,1) on my board and you have 'O' at (4,3) on your board, we can plan the next step to get 'O' to (3,1) on both boards. Since you plan to move 'O' from (4,3) to (4,1) and then to (3,1), I can move 'O' from (5,1) to (4,1) and then to (3,1) to match your moves. Let's proceed with this plan and confirm with each other after each move to ensure we're synchronized."} \\
\\ 
\\ 
\texttt{What is your next command?} \\
            }
        }
    }
    & \\ \\

    \theutterance \stepcounter{utterance}  
    & \multicolumn{4}{p{0.6\linewidth}}{
        \cellcolor[rgb]{0.9,0.9,0.9}{
            \makecell[{{p{\linewidth}}}]{
                \texttt{\tiny{[P1$\rangle$GM]}}
                \texttt{MOVE: O, (4, 1)} \\
            }
        }
    }
    & & \\ \\

    \theutterance \stepcounter{utterance}  
    & & & \multicolumn{2}{p{0.3\linewidth}}{
        \cellcolor[rgb]{0.9,0.9,0.9}{
            \makecell[{{p{\linewidth}}}]{
                \texttt{\tiny{[GM$|$GM]}}
                \texttt{MOVE: O, (4, 1)} \\
            }
        }
    }
    & & \\ \\

    \theutterance \stepcounter{utterance}  
    & & & \multicolumn{2}{p{0.3\linewidth}}{
        \cellcolor[rgb]{0.9,0.9,0.9}{
            \makecell[{{p{\linewidth}}}]{
                \texttt{\tiny{[GM$|$GM]}}
                \texttt{Moved 'O' to (4,1) successfully. Your updated grid looks like this:} \\
\texttt{```} \\
\texttt{ 1234567} \\
\texttt{╔═╤═══╤═╗} \\
\texttt{║◌│◌O◌│◌║ 1} \\
\texttt{╟─┼──┬┘◌║ 2} \\
\texttt{║◌│◌◌│S◌║ 3} \\
\texttt{╟─┴──┘DT║ 4} \\
\texttt{║◌◌◌◌◌◌◌║ 5} \\
\texttt{║◌◌◌◌◌◌◌║ 6} \\
\texttt{║◌◌APM◌◌║ 7} \\
\texttt{╚═══════╝} \\
\texttt{```} \\
            }
        }
    }
    & & \\ \\

    \theutterance \stepcounter{utterance}  
    & & \multicolumn{4}{p{0.6\linewidth}}{
        \cellcolor[rgb]{0.9,0.9,0.9}{
            \makecell[{{p{\linewidth}}}]{
                \texttt{\tiny{[GM$\rangle$P2]}}
                \texttt{Your message has been relayed to the other player.} \\
\\ 
\\ 
\texttt{You are currently playing round 18 of maximum 28.} \\
\texttt{You have collectively accumulated 7 of 16 penalties.} \\
\texttt{The other player moved an object on their grid.} \\
\\ 
\\ 
\texttt{What is your next command?} \\
            }
        }
    }
    & \\ \\

    \theutterance \stepcounter{utterance}  
    & & & \multicolumn{4}{p{0.6\linewidth}}{
        \cellcolor[rgb]{0.9,0.9,0.9}{
            \makecell[{{p{\linewidth}}}]{
                \texttt{\tiny{[GM$\langle$P2]}}
                \texttt{MOVE: O, (4,1)} \\
            }
        }
    }
     \\ \\

    \theutterance \stepcounter{utterance}  
    & & & \multicolumn{2}{p{0.3\linewidth}}{
        \cellcolor[rgb]{0.9,0.9,0.9}{
            \makecell[{{p{\linewidth}}}]{
                \texttt{\tiny{[GM$|$GM]}}
                \texttt{MOVE: O, (4,1)} \\
            }
        }
    }
    & & \\ \\

    \theutterance \stepcounter{utterance}  
    & & & \multicolumn{2}{p{0.3\linewidth}}{
        \cellcolor[rgb]{0.9,0.9,0.9}{
            \makecell[{{p{\linewidth}}}]{
                \texttt{\tiny{[GM$|$GM]}}
                \texttt{Moved 'O' to (4,1) successfully. Your updated grid looks like this:} \\
\texttt{```} \\
\texttt{ 1234567} \\
\texttt{╔═╤═══╤═╗} \\
\texttt{║◌│◌O◌│◌║ 1} \\
\texttt{╟─┼──┬┘◌║ 2} \\
\texttt{║D│◌◌│◌T║ 3} \\
\texttt{╟─┴──┘P◌║ 4} \\
\texttt{║◌◌◌◌◌A◌║ 5} \\
\texttt{║◌◌◌◌◌◌◌║ 6} \\
\texttt{║◌◌S◌M◌◌║ 7} \\
\texttt{╚═══════╝} \\
\texttt{```} \\
            }
        }
    }
    & & \\ \\

    \theutterance \stepcounter{utterance}  
    & & \multicolumn{4}{p{0.6\linewidth}}{
        \cellcolor[rgb]{0.9,0.9,0.9}{
            \makecell[{{p{\linewidth}}}]{
                \texttt{\tiny{[P1$\langle$GM]}}
                \texttt{Moved 'O' to (4,1) successfully. Your updated grid looks like this:} \\
\texttt{```} \\
\texttt{ 1234567} \\
\texttt{╔═╤═══╤═╗} \\
\texttt{║◌│◌O◌│◌║ 1} \\
\texttt{╟─┼──┬┘◌║ 2} \\
\texttt{║◌│◌◌│S◌║ 3} \\
\texttt{╟─┴──┘DT║ 4} \\
\texttt{║◌◌◌◌◌◌◌║ 5} \\
\texttt{║◌◌◌◌◌◌◌║ 6} \\
\texttt{║◌◌APM◌◌║ 7} \\
\texttt{╚═══════╝} \\
\texttt{```} \\
\texttt{You are currently playing round 18 of maximum 28.} \\
\texttt{You have collectively accumulated 7 of 16 penalties.} \\
\texttt{The other player moved an object on their grid.} \\
\\ 
\\ 
\texttt{What is your next command?} \\
            }
        }
    }
    & \\ \\

    \theutterance \stepcounter{utterance}  
    & \multicolumn{4}{p{0.6\linewidth}}{
        \cellcolor[rgb]{0.9,0.9,0.9}{
            \makecell[{{p{\linewidth}}}]{
                \texttt{\tiny{[P1$\rangle$GM]}}
                \texttt{SAY: I've moved 'O' to (4,1) on my board. Now, I'll move it to (3,1) to reach the final position. Please confirm that you've also moved 'O' to (4,1) on your board and are ready to move it to (3,1) as well, so we can synchronize our final move for 'O'.} \\
            }
        }
    }
    & & \\ \\

    \theutterance \stepcounter{utterance}  
    & & & \multicolumn{2}{p{0.3\linewidth}}{
        \cellcolor[rgb]{0.9,0.9,0.9}{
            \makecell[{{p{\linewidth}}}]{
                \texttt{\tiny{[GM$|$GM]}}
                \texttt{SAY: I've moved 'O' to (4,1) on my board. Now, I'll move it to (3,1) to reach the final position. Please confirm that you've also moved 'O' to (4,1) on your board and are ready to move it to (3,1) as well, so we can synchronize our final move for 'O'.} \\
            }
        }
    }
    & & \\ \\

    \theutterance \stepcounter{utterance}  
    & & \multicolumn{4}{p{0.6\linewidth}}{
        \cellcolor[rgb]{0.9,0.9,0.9}{
            \makecell[{{p{\linewidth}}}]{
                \texttt{\tiny{[GM$\rangle$P2]}}
                \texttt{Moved 'O' to (4,1) successfully. Your updated grid looks like this:} \\
\texttt{```} \\
\texttt{ 1234567} \\
\texttt{╔═╤═══╤═╗} \\
\texttt{║◌│◌O◌│◌║ 1} \\
\texttt{╟─┼──┬┘◌║ 2} \\
\texttt{║D│◌◌│◌T║ 3} \\
\texttt{╟─┴──┘P◌║ 4} \\
\texttt{║◌◌◌◌◌A◌║ 5} \\
\texttt{║◌◌◌◌◌◌◌║ 6} \\
\texttt{║◌◌S◌M◌◌║ 7} \\
\texttt{╚═══════╝} \\
\texttt{```} \\
\texttt{You are currently playing round 19 of maximum 28.} \\
\texttt{You have collectively accumulated 7 of 16 penalties.} \\
\texttt{The other player sent this message:} \\
\\ 
\\ 
\texttt{"I've moved 'O' to (4,1) on my board. Now, I'll move it to (3,1) to reach the final position. Please confirm that you've also moved 'O' to (4,1) on your board and are ready to move it to (3,1) as well, so we can synchronize our final move for 'O'."} \\
\\ 
\\ 
\texttt{What is your next command?} \\
            }
        }
    }
    & \\ \\

    \theutterance \stepcounter{utterance}  
    & & & \multicolumn{4}{p{0.6\linewidth}}{
        \cellcolor[rgb]{0.9,0.9,0.9}{
            \makecell[{{p{\linewidth}}}]{
                \texttt{\tiny{[GM$\langle$P2]}}
                \texttt{SAY: Yes, I've moved 'O' to (4,1) on my board, and I'm ready to move it to (3,1) to match the final position on your board. I'll wait for your confirmation that you've moved 'O' to (3,1) before I make the final move, to ensure we're perfectly synchronized. Please proceed with moving 'O' to (3,1) on your board.} \\
            }
        }
    }
     \\ \\

    \theutterance \stepcounter{utterance}  
    & & & \multicolumn{2}{p{0.3\linewidth}}{
        \cellcolor[rgb]{0.9,0.9,0.9}{
            \makecell[{{p{\linewidth}}}]{
                \texttt{\tiny{[GM$|$GM]}}
                \texttt{SAY: Yes, I've moved 'O' to (4,1) on my board, and I'm ready to move it to (3,1) to match the final position on your board. I'll wait for your confirmation that you've moved 'O' to (3,1) before I make the final move, to ensure we're perfectly synchronized. Please proceed with moving 'O' to (3,1) on your board.} \\
            }
        }
    }
    & & \\ \\

    \theutterance \stepcounter{utterance}  
    & & \multicolumn{4}{p{0.6\linewidth}}{
        \cellcolor[rgb]{0.9,0.9,0.9}{
            \makecell[{{p{\linewidth}}}]{
                \texttt{\tiny{[P1$\langle$GM]}}
                \texttt{Your message has been relayed to the other player.} \\
\\ 
\\ 
\texttt{You are currently playing round 19 of maximum 28.} \\
\texttt{You have collectively accumulated 7 of 16 penalties.} \\
\texttt{The other player sent this message:} \\
\\ 
\\ 
\texttt{"Yes, I've moved 'O' to (4,1) on my board, and I'm ready to move it to (3,1) to match the final position on your board. I'll wait for your confirmation that you've moved 'O' to (3,1) before I make the final move, to ensure we're perfectly synchronized. Please proceed with moving 'O' to (3,1) on your board."} \\
\\ 
\\ 
\texttt{What is your next command?} \\
            }
        }
    }
    & \\ \\

    \theutterance \stepcounter{utterance}  
    & \multicolumn{4}{p{0.6\linewidth}}{
        \cellcolor[rgb]{0.9,0.9,0.9}{
            \makecell[{{p{\linewidth}}}]{
                \texttt{\tiny{[P1$\rangle$GM]}}
                \texttt{MOVE: O, (3, 1)} \\
            }
        }
    }
    & & \\ \\

    \theutterance \stepcounter{utterance}  
    & & & \multicolumn{2}{p{0.3\linewidth}}{
        \cellcolor[rgb]{0.9,0.9,0.9}{
            \makecell[{{p{\linewidth}}}]{
                \texttt{\tiny{[GM$|$GM]}}
                \texttt{MOVE: O, (3, 1)} \\
            }
        }
    }
    & & \\ \\

    \theutterance \stepcounter{utterance}  
    & & & \multicolumn{2}{p{0.3\linewidth}}{
        \cellcolor[rgb]{0.9,0.9,0.9}{
            \makecell[{{p{\linewidth}}}]{
                \texttt{\tiny{[GM$|$GM]}}
                \texttt{Moved 'O' to (3,1) successfully. Your updated grid looks like this:} \\
\texttt{```} \\
\texttt{ 1234567} \\
\texttt{╔═╤═══╤═╗} \\
\texttt{║◌│O◌◌│◌║ 1} \\
\texttt{╟─┼──┬┘◌║ 2} \\
\texttt{║◌│◌◌│S◌║ 3} \\
\texttt{╟─┴──┘DT║ 4} \\
\texttt{║◌◌◌◌◌◌◌║ 5} \\
\texttt{║◌◌◌◌◌◌◌║ 6} \\
\texttt{║◌◌APM◌◌║ 7} \\
\texttt{╚═══════╝} \\
\texttt{```} \\
            }
        }
    }
    & & \\ \\

    \theutterance \stepcounter{utterance}  
    & & \multicolumn{4}{p{0.6\linewidth}}{
        \cellcolor[rgb]{0.9,0.9,0.9}{
            \makecell[{{p{\linewidth}}}]{
                \texttt{\tiny{[GM$\rangle$P2]}}
                \texttt{Your message has been relayed to the other player.} \\
\\ 
\\ 
\texttt{You are currently playing round 20 of maximum 28.} \\
\texttt{You have collectively accumulated 7 of 16 penalties.} \\
\texttt{The other player moved an object on their grid.} \\
\\ 
\\ 
\texttt{What is your next command?} \\
            }
        }
    }
    & \\ \\

    \theutterance \stepcounter{utterance}  
    & & & \multicolumn{4}{p{0.6\linewidth}}{
        \cellcolor[rgb]{0.9,0.9,0.9}{
            \makecell[{{p{\linewidth}}}]{
                \texttt{\tiny{[GM$\langle$P2]}}
                \texttt{MOVE: O, (3,1)} \\
            }
        }
    }
     \\ \\

    \theutterance \stepcounter{utterance}  
    & & & \multicolumn{2}{p{0.3\linewidth}}{
        \cellcolor[rgb]{0.9,0.9,0.9}{
            \makecell[{{p{\linewidth}}}]{
                \texttt{\tiny{[GM$|$GM]}}
                \texttt{MOVE: O, (3,1)} \\
            }
        }
    }
    & & \\ \\

    \theutterance \stepcounter{utterance}  
    & & & \multicolumn{2}{p{0.3\linewidth}}{
        \cellcolor[rgb]{0.9,0.9,0.9}{
            \makecell[{{p{\linewidth}}}]{
                \texttt{\tiny{[GM$|$GM]}}
                \texttt{Moved 'O' to (3,1) successfully. Your updated grid looks like this:} \\
\texttt{```} \\
\texttt{ 1234567} \\
\texttt{╔═╤═══╤═╗} \\
\texttt{║◌│O◌◌│◌║ 1} \\
\texttt{╟─┼──┬┘◌║ 2} \\
\texttt{║D│◌◌│◌T║ 3} \\
\texttt{╟─┴──┘P◌║ 4} \\
\texttt{║◌◌◌◌◌A◌║ 5} \\
\texttt{║◌◌◌◌◌◌◌║ 6} \\
\texttt{║◌◌S◌M◌◌║ 7} \\
\texttt{╚═══════╝} \\
\texttt{```} \\
            }
        }
    }
    & & \\ \\

    \theutterance \stepcounter{utterance}  
    & & \multicolumn{4}{p{0.6\linewidth}}{
        \cellcolor[rgb]{0.9,0.9,0.9}{
            \makecell[{{p{\linewidth}}}]{
                \texttt{\tiny{[P1$\langle$GM]}}
                \texttt{Moved 'O' to (3,1) successfully. Your updated grid looks like this:} \\
\texttt{```} \\
\texttt{ 1234567} \\
\texttt{╔═╤═══╤═╗} \\
\texttt{║◌│O◌◌│◌║ 1} \\
\texttt{╟─┼──┬┘◌║ 2} \\
\texttt{║◌│◌◌│S◌║ 3} \\
\texttt{╟─┴──┘DT║ 4} \\
\texttt{║◌◌◌◌◌◌◌║ 5} \\
\texttt{║◌◌◌◌◌◌◌║ 6} \\
\texttt{║◌◌APM◌◌║ 7} \\
\texttt{╚═══════╝} \\
\texttt{```} \\
\texttt{You are currently playing round 20 of maximum 28.} \\
\texttt{You have collectively accumulated 7 of 16 penalties.} \\
\texttt{The other player moved an object on their grid.} \\
\\ 
\\ 
\texttt{What is your next command?} \\
            }
        }
    }
    & \\ \\

    \theutterance \stepcounter{utterance}  
    & \multicolumn{4}{p{0.6\linewidth}}{
        \cellcolor[rgb]{0.9,0.9,0.9}{
            \makecell[{{p{\linewidth}}}]{
                \texttt{\tiny{[P1$\rangle$GM]}}
                \texttt{SAY: I've moved 'O' to (3,1) on my board. Now that 'O' is in its final position, let's review the current state of both boards and plan the next steps for the remaining objects. Please share the updated position of 'O' on your board and the status of other objects, so we can discuss the next moves to align them.} \\
            }
        }
    }
    & & \\ \\

    \theutterance \stepcounter{utterance}  
    & & & \multicolumn{2}{p{0.3\linewidth}}{
        \cellcolor[rgb]{0.9,0.9,0.9}{
            \makecell[{{p{\linewidth}}}]{
                \texttt{\tiny{[GM$|$GM]}}
                \texttt{SAY: I've moved 'O' to (3,1) on my board. Now that 'O' is in its final position, let's review the current state of both boards and plan the next steps for the remaining objects. Please share the updated position of 'O' on your board and the status of other objects, so we can discuss the next moves to align them.} \\
            }
        }
    }
    & & \\ \\

    \theutterance \stepcounter{utterance}  
    & & \multicolumn{4}{p{0.6\linewidth}}{
        \cellcolor[rgb]{0.9,0.9,0.9}{
            \makecell[{{p{\linewidth}}}]{
                \texttt{\tiny{[GM$\rangle$P2]}}
                \texttt{Moved 'O' to (3,1) successfully. Your updated grid looks like this:} \\
\texttt{```} \\
\texttt{ 1234567} \\
\texttt{╔═╤═══╤═╗} \\
\texttt{║◌│O◌◌│◌║ 1} \\
\texttt{╟─┼──┬┘◌║ 2} \\
\texttt{║D│◌◌│◌T║ 3} \\
\texttt{╟─┴──┘P◌║ 4} \\
\texttt{║◌◌◌◌◌A◌║ 5} \\
\texttt{║◌◌◌◌◌◌◌║ 6} \\
\texttt{║◌◌S◌M◌◌║ 7} \\
\texttt{╚═══════╝} \\
\texttt{```} \\
\texttt{You are currently playing round 21 of maximum 28.} \\
\texttt{You have collectively accumulated 7 of 16 penalties.} \\
\texttt{The other player sent this message:} \\
\\ 
\\ 
\texttt{"I've moved 'O' to (3,1) on my board. Now that 'O' is in its final position, let's review the current state of both boards and plan the next steps for the remaining objects. Please share the updated position of 'O' on your board and the status of other objects, so we can discuss the next moves to align them."} \\
\\ 
\\ 
\texttt{What is your next command?} \\
            }
        }
    }
    & \\ \\

    \theutterance \stepcounter{utterance}  
    & & & \multicolumn{4}{p{0.6\linewidth}}{
        \cellcolor[rgb]{0.9,0.9,0.9}{
            \makecell[{{p{\linewidth}}}]{
                \texttt{\tiny{[GM$\langle$P2]}}
                \texttt{SAY: I've successfully moved 'O' to (3,1) on my board, as you can see from the updated grid. The current positions of the objects on my board are: 'O' at (3,1), 'D' at (1,3), 'P' at (4,4), 'T' at (3,4) (but I see it at (4,3) on the grid), 'S' at (3,7), 'A' at (5,5), and 'M' at (5,7). Now that 'O' is aligned, let's discuss the next steps for the remaining objects, focusing on 'D', 'P', 'T', 'S', 'A', and 'M'. Which object do you think we should prioritize next to move to its assigned position?} \\
            }
        }
    }
     \\ \\

    \theutterance \stepcounter{utterance}  
    & & & \multicolumn{2}{p{0.3\linewidth}}{
        \cellcolor[rgb]{0.9,0.9,0.9}{
            \makecell[{{p{\linewidth}}}]{
                \texttt{\tiny{[GM$|$GM]}}
                \texttt{SAY: I've successfully moved 'O' to (3,1) on my board, as you can see from the updated grid. The current positions of the objects on my board are: 'O' at (3,1), 'D' at (1,3), 'P' at (4,4), 'T' at (3,4) (but I see it at (4,3) on the grid), 'S' at (3,7), 'A' at (5,5), and 'M' at (5,7). Now that 'O' is aligned, let's discuss the next steps for the remaining objects, focusing on 'D', 'P', 'T', 'S', 'A', and 'M'. Which object do you think we should prioritize next to move to its assigned position?} \\
            }
        }
    }
    & & \\ \\

    \theutterance \stepcounter{utterance}  
    & & \multicolumn{4}{p{0.6\linewidth}}{
        \cellcolor[rgb]{0.9,0.9,0.9}{
            \makecell[{{p{\linewidth}}}]{
                \texttt{\tiny{[P1$\langle$GM]}}
                \texttt{Your message has been relayed to the other player.} \\
\\ 
\\ 
\texttt{You are currently playing round 21 of maximum 28.} \\
\texttt{You have collectively accumulated 7 of 16 penalties.} \\
\texttt{The other player sent this message:} \\
\\ 
\\ 
\texttt{"I've successfully moved 'O' to (3,1) on my board, as you can see from the updated grid. The current positions of the objects on my board are: 'O' at (3,1), 'D' at (1,3), 'P' at (4,4), 'T' at (3,4) (but I see it at (4,3) on the grid), 'S' at (3,7), 'A' at (5,5), and 'M' at (5,7). Now that 'O' is aligned, let's discuss the next steps for the remaining objects, focusing on 'D', 'P', 'T', 'S', 'A', and 'M'. Which object do you think we should prioritize next to move to its assigned position?"} \\
\\ 
\\ 
\texttt{What is your next command?} \\
            }
        }
    }
    & \\ \\

    \theutterance \stepcounter{utterance}  
    & \multicolumn{4}{p{0.6\linewidth}}{
        \cellcolor[rgb]{0.9,0.9,0.9}{
            \makecell[{{p{\linewidth}}}]{
                \texttt{\tiny{[P1$\rangle$GM]}}
                \texttt{SAY: I'm glad we have 'O' aligned. Let's review the assigned positions: 'P' at (1,1), 'O' at (3,1), 'T' at (5,1), 'S' at (2,2), 'D' at (4,2), 'A' at (3,7), and 'M' at (5,7). Considering the current positions, I think we should prioritize moving 'P' to its assigned position (1,1), as it seems to be the furthest from its goal on both boards. What are your thoughts on this, and do you have any suggestions for the next steps?} \\
            }
        }
    }
    & & \\ \\

    \theutterance \stepcounter{utterance}  
    & & & \multicolumn{2}{p{0.3\linewidth}}{
        \cellcolor[rgb]{0.9,0.9,0.9}{
            \makecell[{{p{\linewidth}}}]{
                \texttt{\tiny{[GM$|$GM]}}
                \texttt{SAY: I'm glad we have 'O' aligned. Let's review the assigned positions: 'P' at (1,1), 'O' at (3,1), 'T' at (5,1), 'S' at (2,2), 'D' at (4,2), 'A' at (3,7), and 'M' at (5,7). Considering the current positions, I think we should prioritize moving 'P' to its assigned position (1,1), as it seems to be the furthest from its goal on both boards. What are your thoughts on this, and do you have any suggestions for the next steps?} \\
            }
        }
    }
    & & \\ \\

    \theutterance \stepcounter{utterance}  
    & & \multicolumn{4}{p{0.6\linewidth}}{
        \cellcolor[rgb]{0.9,0.9,0.9}{
            \makecell[{{p{\linewidth}}}]{
                \texttt{\tiny{[GM$\rangle$P2]}}
                \texttt{Your message has been relayed to the other player.} \\
\\ 
\\ 
\texttt{You are currently playing round 22 of maximum 28.} \\
\texttt{You have collectively accumulated 7 of 16 penalties.} \\
\texttt{The other player sent this message:} \\
\\ 
\\ 
\texttt{"I'm glad we have 'O' aligned. Let's review the assigned positions: 'P' at (1,1), 'O' at (3,1), 'T' at (5,1), 'S' at (2,2), 'D' at (4,2), 'A' at (3,7), and 'M' at (5,7). Considering the current positions, I think we should prioritize moving 'P' to its assigned position (1,1), as it seems to be the furthest from its goal on both boards. What are your thoughts on this, and do you have any suggestions for the next steps?"} \\
\\ 
\\ 
\texttt{What is your next command?} \\
            }
        }
    }
    & \\ \\

    \theutterance \stepcounter{utterance}  
    & & & \multicolumn{4}{p{0.6\linewidth}}{
        \cellcolor[rgb]{0.9,0.9,0.9}{
            \makecell[{{p{\linewidth}}}]{
                \texttt{\tiny{[GM$\langle$P2]}}
                \texttt{SAY: Prioritizing 'P' to move to (1,1) makes sense, given its current position at (4,4) on my board. I agree that 'P' is one of the objects furthest from its goal. To move 'P' to (1,1), I can start by moving it to an intermediate position, such as (1,4) or (4,1), and then adjust from there. On your board, what is the current position of 'P', and what intermediate steps do you propose to move it to (1,1)? Let's coordinate our moves to ensure 'P' reaches its assigned position on both boards efficiently.} \\
            }
        }
    }
     \\ \\

    \theutterance \stepcounter{utterance}  
    & & & \multicolumn{2}{p{0.3\linewidth}}{
        \cellcolor[rgb]{0.9,0.9,0.9}{
            \makecell[{{p{\linewidth}}}]{
                \texttt{\tiny{[GM$|$GM]}}
                \texttt{SAY: Prioritizing 'P' to move to (1,1) makes sense, given its current position at (4,4) on my board. I agree that 'P' is one of the objects furthest from its goal. To move 'P' to (1,1), I can start by moving it to an intermediate position, such as (1,4) or (4,1), and then adjust from there. On your board, what is the current position of 'P', and what intermediate steps do you propose to move it to (1,1)? Let's coordinate our moves to ensure 'P' reaches its assigned position on both boards efficiently.} \\
            }
        }
    }
    & & \\ \\

    \theutterance \stepcounter{utterance}  
    & & \multicolumn{4}{p{0.6\linewidth}}{
        \cellcolor[rgb]{0.9,0.9,0.9}{
            \makecell[{{p{\linewidth}}}]{
                \texttt{\tiny{[P1$\langle$GM]}}
                \texttt{Your message has been relayed to the other player.} \\
\\ 
\\ 
\texttt{You are currently playing round 22 of maximum 28.} \\
\texttt{You have collectively accumulated 7 of 16 penalties.} \\
\texttt{The other player sent this message:} \\
\\ 
\\ 
\texttt{"Prioritizing 'P' to move to (1,1) makes sense, given its current position at (4,4) on my board. I agree that 'P' is one of the objects furthest from its goal. To move 'P' to (1,1), I can start by moving it to an intermediate position, such as (1,4) or (4,1), and then adjust from there. On your board, what is the current position of 'P', and what intermediate steps do you propose to move it to (1,1)? Let's coordinate our moves to ensure 'P' reaches its assigned position on both boards efficiently."} \\
\\ 
\\ 
\texttt{What is your next command?} \\
            }
        }
    }
    & \\ \\

    \theutterance \stepcounter{utterance}  
    & \multicolumn{4}{p{0.6\linewidth}}{
        \cellcolor[rgb]{0.9,0.9,0.9}{
            \makecell[{{p{\linewidth}}}]{
                \texttt{\tiny{[P1$\rangle$GM]}}
                \texttt{SAY: On my board, 'P' is currently at (1,7). To move it to (1,1), I can move it up six rows directly. Since we want to synchronize our moves, let's consider moving 'P' to an intermediate position that is closer to (1,1) on both boards. How about we both move 'P' to (1,4) as an intermediate step, and then reassess our next moves to get 'P' to (1,1)?} \\
            }
        }
    }
    & & \\ \\

    \theutterance \stepcounter{utterance}  
    & & & \multicolumn{2}{p{0.3\linewidth}}{
        \cellcolor[rgb]{0.9,0.9,0.9}{
            \makecell[{{p{\linewidth}}}]{
                \texttt{\tiny{[GM$|$GM]}}
                \texttt{SAY: On my board, 'P' is currently at (1,7). To move it to (1,1), I can move it up six rows directly. Since we want to synchronize our moves, let's consider moving 'P' to an intermediate position that is closer to (1,1) on both boards. How about we both move 'P' to (1,4) as an intermediate step, and then reassess our next moves to get 'P' to (1,1)?} \\
            }
        }
    }
    & & \\ \\

    \theutterance \stepcounter{utterance}  
    & & \multicolumn{4}{p{0.6\linewidth}}{
        \cellcolor[rgb]{0.9,0.9,0.9}{
            \makecell[{{p{\linewidth}}}]{
                \texttt{\tiny{[GM$\rangle$P2]}}
                \texttt{Your message has been relayed to the other player.} \\
\\ 
\\ 
\texttt{You are currently playing round 23 of maximum 28.} \\
\texttt{You have collectively accumulated 7 of 16 penalties.} \\
\texttt{The other player sent this message:} \\
\\ 
\\ 
\texttt{"On my board, 'P' is currently at (1,7). To move it to (1,1), I can move it up six rows directly. Since we want to synchronize our moves, let's consider moving 'P' to an intermediate position that is closer to (1,1) on both boards. How about we both move 'P' to (1,4) as an intermediate step, and then reassess our next moves to get 'P' to (1,1)?"} \\
\\ 
\\ 
\texttt{What is your next command?} \\
            }
        }
    }
    & \\ \\

    \theutterance \stepcounter{utterance}  
    & & & \multicolumn{4}{p{0.6\linewidth}}{
        \cellcolor[rgb]{0.9,0.9,0.9}{
            \makecell[{{p{\linewidth}}}]{
                \texttt{\tiny{[GM$\langle$P2]}}
                \texttt{MOVE: P, (1,4)} \\
            }
        }
    }
     \\ \\

    \theutterance \stepcounter{utterance}  
    & & & \multicolumn{2}{p{0.3\linewidth}}{
        \cellcolor[rgb]{0.9,0.9,0.9}{
            \makecell[{{p{\linewidth}}}]{
                \texttt{\tiny{[GM$|$GM]}}
                \texttt{MOVE: P, (1,4)} \\
            }
        }
    }
    & & \\ \\

    \theutterance \stepcounter{utterance}  
    & & & \multicolumn{2}{p{0.3\linewidth}}{
        \cellcolor[rgb]{0.9,0.9,0.9}{
            \makecell[{{p{\linewidth}}}]{
                \texttt{\tiny{[GM$|$GM]}}
                \texttt{Penalty: (1,4) is not empty, but contains '─'.} \\
\\ 
\texttt{You have collectively accumulated 8 of 16 penalties.} \\
\texttt{Please try again!} \\
            }
        }
    }
    & & \\ \\

    \theutterance \stepcounter{utterance}  
    & & \multicolumn{4}{p{0.6\linewidth}}{
        \cellcolor[rgb]{0.9,0.9,0.9}{
            \makecell[{{p{\linewidth}}}]{
                \texttt{\tiny{[GM$\rangle$P2]}}
                \texttt{Penalty: (1,4) is not empty, but contains '─'.} \\
\\ 
\texttt{You have collectively accumulated 8 of 16 penalties.} \\
\texttt{Please try again!} \\
            }
        }
    }
    & \\ \\

    \theutterance \stepcounter{utterance}  
    & & & \multicolumn{4}{p{0.6\linewidth}}{
        \cellcolor[rgb]{0.9,0.9,0.9}{
            \makecell[{{p{\linewidth}}}]{
                \texttt{\tiny{[GM$\langle$P2]}}
                \texttt{MOVE: P, (4,1)} \\
            }
        }
    }
     \\ \\

    \theutterance \stepcounter{utterance}  
    & & & \multicolumn{2}{p{0.3\linewidth}}{
        \cellcolor[rgb]{0.9,0.9,0.9}{
            \makecell[{{p{\linewidth}}}]{
                \texttt{\tiny{[GM$|$GM]}}
                \texttt{MOVE: P, (4,1)} \\
            }
        }
    }
    & & \\ \\

    \theutterance \stepcounter{utterance}  
    & & & \multicolumn{2}{p{0.3\linewidth}}{
        \cellcolor[rgb]{0.9,0.9,0.9}{
            \makecell[{{p{\linewidth}}}]{
                \texttt{\tiny{[GM$|$GM]}}
                \texttt{Moved 'P' to (4,1) successfully. Your updated grid looks like this:} \\
\texttt{```} \\
\texttt{ 1234567} \\
\texttt{╔═╤═══╤═╗} \\
\texttt{║◌│OP◌│◌║ 1} \\
\texttt{╟─┼──┬┘◌║ 2} \\
\texttt{║D│◌◌│◌T║ 3} \\
\texttt{╟─┴──┘◌◌║ 4} \\
\texttt{║◌◌◌◌◌A◌║ 5} \\
\texttt{║◌◌◌◌◌◌◌║ 6} \\
\texttt{║◌◌S◌M◌◌║ 7} \\
\texttt{╚═══════╝} \\
\texttt{```} \\
            }
        }
    }
    & & \\ \\

    \theutterance \stepcounter{utterance}  
    & & \multicolumn{4}{p{0.6\linewidth}}{
        \cellcolor[rgb]{0.9,0.9,0.9}{
            \makecell[{{p{\linewidth}}}]{
                \texttt{\tiny{[P1$\langle$GM]}}
                \texttt{Your message has been relayed to the other player.} \\
\\ 
\\ 
\texttt{You are currently playing round 23 of maximum 28.} \\
\texttt{You have collectively accumulated 8 of 16 penalties.} \\
\texttt{The other player moved an object on their grid.} \\
\\ 
\\ 
\texttt{What is your next command?} \\
            }
        }
    }
    & \\ \\

    \theutterance \stepcounter{utterance}  
    & \multicolumn{4}{p{0.6\linewidth}}{
        \cellcolor[rgb]{0.9,0.9,0.9}{
            \makecell[{{p{\linewidth}}}]{
                \texttt{\tiny{[P1$\rangle$GM]}}
                \texttt{MOVE: P, (1, 4)} \\
            }
        }
    }
    & & \\ \\

    \theutterance \stepcounter{utterance}  
    & & & \multicolumn{2}{p{0.3\linewidth}}{
        \cellcolor[rgb]{0.9,0.9,0.9}{
            \makecell[{{p{\linewidth}}}]{
                \texttt{\tiny{[GM$|$GM]}}
                \texttt{MOVE: P, (1, 4)} \\
            }
        }
    }
    & & \\ \\

    \theutterance \stepcounter{utterance}  
    & & & \multicolumn{2}{p{0.3\linewidth}}{
        \cellcolor[rgb]{0.9,0.9,0.9}{
            \makecell[{{p{\linewidth}}}]{
                \texttt{\tiny{[GM$|$GM]}}
                \texttt{Penalty: (1,4) is not empty, but contains '─'.} \\
\\ 
\texttt{You have collectively accumulated 9 of 16 penalties.} \\
\texttt{Please try again!} \\
            }
        }
    }
    & & \\ \\

    \theutterance \stepcounter{utterance}  
    & & \multicolumn{4}{p{0.6\linewidth}}{
        \cellcolor[rgb]{0.9,0.9,0.9}{
            \makecell[{{p{\linewidth}}}]{
                \texttt{\tiny{[P1$\langle$GM]}}
                \texttt{Penalty: (1,4) is not empty, but contains '─'.} \\
\\ 
\texttt{You have collectively accumulated 9 of 16 penalties.} \\
\texttt{Please try again!} \\
            }
        }
    }
    & \\ \\

    \theutterance \stepcounter{utterance}  
    & \multicolumn{4}{p{0.6\linewidth}}{
        \cellcolor[rgb]{0.9,0.9,0.9}{
            \makecell[{{p{\linewidth}}}]{
                \texttt{\tiny{[P1$\rangle$GM]}}
                \texttt{SAY: It seems that I need to find an alternative intermediate position for 'P' to move to, since (1,4) is occupied by '─'. Let's discuss other possible intermediate positions for 'P' that could help us synchronize our moves and eventually get 'P' to (1,1) on both boards.} \\
            }
        }
    }
    & & \\ \\

    \theutterance \stepcounter{utterance}  
    & & & \multicolumn{2}{p{0.3\linewidth}}{
        \cellcolor[rgb]{0.9,0.9,0.9}{
            \makecell[{{p{\linewidth}}}]{
                \texttt{\tiny{[GM$|$GM]}}
                \texttt{SAY: It seems that I need to find an alternative intermediate position for 'P' to move to, since (1,4) is occupied by '─'. Let's discuss other possible intermediate positions for 'P' that could help us synchronize our moves and eventually get 'P' to (1,1) on both boards.} \\
            }
        }
    }
    & & \\ \\

    \theutterance \stepcounter{utterance}  
    & & \multicolumn{4}{p{0.6\linewidth}}{
        \cellcolor[rgb]{0.9,0.9,0.9}{
            \makecell[{{p{\linewidth}}}]{
                \texttt{\tiny{[GM$\rangle$P2]}}
                \texttt{Moved 'P' to (4,1) successfully. Your updated grid looks like this:} \\
\texttt{```} \\
\texttt{ 1234567} \\
\texttt{╔═╤═══╤═╗} \\
\texttt{║◌│OP◌│◌║ 1} \\
\texttt{╟─┼──┬┘◌║ 2} \\
\texttt{║D│◌◌│◌T║ 3} \\
\texttt{╟─┴──┘◌◌║ 4} \\
\texttt{║◌◌◌◌◌A◌║ 5} \\
\texttt{║◌◌◌◌◌◌◌║ 6} \\
\texttt{║◌◌S◌M◌◌║ 7} \\
\texttt{╚═══════╝} \\
\texttt{```} \\
\texttt{You are currently playing round 24 of maximum 28.} \\
\texttt{You have collectively accumulated 9 of 16 penalties.} \\
\texttt{The other player sent this message:} \\
\\ 
\\ 
\texttt{"It seems that I need to find an alternative intermediate position for 'P' to move to, since (1,4) is occupied by '─'. Let's discuss other possible intermediate positions for 'P' that could help us synchronize our moves and eventually get 'P' to (1,1) on both boards."} \\
\\ 
\\ 
\texttt{What is your next command?} \\
            }
        }
    }
    & \\ \\

    \theutterance \stepcounter{utterance}  
    & & & \multicolumn{4}{p{0.6\linewidth}}{
        \cellcolor[rgb]{0.9,0.9,0.9}{
            \makecell[{{p{\linewidth}}}]{
                \texttt{\tiny{[GM$\langle$P2]}}
                \texttt{SAY: Now that 'P' is at (4,1) on my board, we can reassess the possible intermediate positions. Considering the obstacles and the goal of getting 'P' to (1,1), how about we move 'P' to (1,1) directly from its current positions on both boards? On my board, 'P' would need to move from (4,1) to (1,1), and on your board, 'P' would move from its current position to (1,1). This direct approach could simplify our moves and reduce the risk of further penalties. What are your thoughts on moving 'P' directly to (1,1) on both boards?} \\
            }
        }
    }
     \\ \\

    \theutterance \stepcounter{utterance}  
    & & & \multicolumn{2}{p{0.3\linewidth}}{
        \cellcolor[rgb]{0.9,0.9,0.9}{
            \makecell[{{p{\linewidth}}}]{
                \texttt{\tiny{[GM$|$GM]}}
                \texttt{SAY: Now that 'P' is at (4,1) on my board, we can reassess the possible intermediate positions. Considering the obstacles and the goal of getting 'P' to (1,1), how about we move 'P' to (1,1) directly from its current positions on both boards? On my board, 'P' would need to move from (4,1) to (1,1), and on your board, 'P' would move from its current position to (1,1). This direct approach could simplify our moves and reduce the risk of further penalties. What are your thoughts on moving 'P' directly to (1,1) on both boards?} \\
            }
        }
    }
    & & \\ \\

    \theutterance \stepcounter{utterance}  
    & & \multicolumn{4}{p{0.6\linewidth}}{
        \cellcolor[rgb]{0.9,0.9,0.9}{
            \makecell[{{p{\linewidth}}}]{
                \texttt{\tiny{[P1$\langle$GM]}}
                \texttt{Your message has been relayed to the other player.} \\
\\ 
\\ 
\texttt{You are currently playing round 24 of maximum 28.} \\
\texttt{You have collectively accumulated 9 of 16 penalties.} \\
\texttt{The other player sent this message:} \\
\\ 
\\ 
\texttt{"Now that 'P' is at (4,1) on my board, we can reassess the possible intermediate positions. Considering the obstacles and the goal of getting 'P' to (1,1), how about we move 'P' to (1,1) directly from its current positions on both boards? On my board, 'P' would need to move from (4,1) to (1,1), and on your board, 'P' would move from its current position to (1,1). This direct approach could simplify our moves and reduce the risk of further penalties. What are your thoughts on moving 'P' directly to (1,1) on both boards?"} \\
\\ 
\\ 
\texttt{What is your next command?} \\
            }
        }
    }
    & \\ \\

    \theutterance \stepcounter{utterance}  
    & \multicolumn{4}{p{0.6\linewidth}}{
        \cellcolor[rgb]{0.9,0.9,0.9}{
            \makecell[{{p{\linewidth}}}]{
                \texttt{\tiny{[P1$\rangle$GM]}}
                \texttt{SAY: Moving 'P' directly to (1,1) from its current positions on both boards seems like a simplified and efficient approach. On my board, 'P' is currently at (1,7), so moving it directly to (1,1) would involve moving it up six rows. I agree that this direct approach could reduce the risk of further penalties. Let's proceed with moving 'P' directly to (1,1) on both boards. I'll make the move on my board. MOVE: P, (1, 1)} \\
            }
        }
    }
    & & \\ \\

    \theutterance \stepcounter{utterance}  
    & & & \multicolumn{2}{p{0.3\linewidth}}{
        \cellcolor[rgb]{0.9,0.9,0.9}{
            \makecell[{{p{\linewidth}}}]{
                \texttt{\tiny{[GM$|$GM]}}
                \texttt{SAY: Moving 'P' directly to (1,1) from its current positions on both boards seems like a simplified and efficient approach. On my board, 'P' is currently at (1,7), so moving it directly to (1,1) would involve moving it up six rows. I agree that this direct approach could reduce the risk of further penalties. Let's proceed with moving 'P' directly to (1,1) on both boards. I'll make the move on my board. MOVE: P, (1, 1)} \\
            }
        }
    }
    & & \\ \\

    \theutterance \stepcounter{utterance}  
    & & & \multicolumn{2}{p{0.3\linewidth}}{
        \cellcolor[rgb]{0.9,0.9,0.9}{
            \makecell[{{p{\linewidth}}}]{
                \texttt{\tiny{[GM$|$GM]}}
                \texttt{Invalid response format: SAY: Moving 'P' directly to (1,1) from its current positions on both boards seems like a simplified and efficient approach. On my board, 'P' is currently at (1,7), so moving it directly to (1,1) would involve moving it up six rows. I agree that this direct approach could reduce the risk of further penalties. Let's proceed with moving 'P' directly to (1,1) on both boards. I'll make the move on my board. MOVE: P, (1, 1)} \\
            }
        }
    }
    & & \\ \\

    \theutterance \stepcounter{utterance}  
    & & \multicolumn{4}{p{0.6\linewidth}}{
        \cellcolor[rgb]{0.9,0.9,0.9}{
            \makecell[{{p{\linewidth}}}]{
                \texttt{\tiny{[P1$\langle$GM]}}
                \texttt{Penalty: Your message contains more than one command!} \\
\\ 
\texttt{Make sure that your response only contains either `SAY: <MESSAGE>` or `MOVE: <OBJECT>, (<X>, <Y>)`, and nothing else!} \\
\\ 
\texttt{You have collectively accumulated 10 of 16 penalties.} \\
\texttt{Please try again!} \\
            }
        }
    }
    & \\ \\

    \theutterance \stepcounter{utterance}  
    & \multicolumn{4}{p{0.6\linewidth}}{
        \cellcolor[rgb]{0.9,0.9,0.9}{
            \makecell[{{p{\linewidth}}}]{
                \texttt{\tiny{[P1$\rangle$GM]}}
                \texttt{SAY: Moving 'P' directly to (1,1) from its current positions on both boards seems like a simplified and efficient approach. On my board, 'P' is currently at (1,7), so moving it directly to (1,1) would involve moving it up six rows. I agree that this direct approach could reduce the risk of further penalties. Let's proceed with moving 'P' directly to (1,1) on both boards. I'll make the move on my board. Please confirm when you're ready to make the move on your board.} \\
            }
        }
    }
    & & \\ \\

    \theutterance \stepcounter{utterance}  
    & & & \multicolumn{2}{p{0.3\linewidth}}{
        \cellcolor[rgb]{0.9,0.9,0.9}{
            \makecell[{{p{\linewidth}}}]{
                \texttt{\tiny{[GM$|$GM]}}
                \texttt{SAY: Moving 'P' directly to (1,1) from its current positions on both boards seems like a simplified and efficient approach. On my board, 'P' is currently at (1,7), so moving it directly to (1,1) would involve moving it up six rows. I agree that this direct approach could reduce the risk of further penalties. Let's proceed with moving 'P' directly to (1,1) on both boards. I'll make the move on my board. Please confirm when you're ready to make the move on your board.} \\
            }
        }
    }
    & & \\ \\

    \theutterance \stepcounter{utterance}  
    & & \multicolumn{4}{p{0.6\linewidth}}{
        \cellcolor[rgb]{0.9,0.9,0.9}{
            \makecell[{{p{\linewidth}}}]{
                \texttt{\tiny{[GM$\rangle$P2]}}
                \texttt{Your message has been relayed to the other player.} \\
\\ 
\\ 
\texttt{You are currently playing round 25 of maximum 28.} \\
\texttt{You have collectively accumulated 10 of 16 penalties.} \\
\texttt{The other player sent this message:} \\
\\ 
\\ 
\texttt{"Moving 'P' directly to (1,1) from its current positions on both boards seems like a simplified and efficient approach. On my board, 'P' is currently at (1,7), so moving it directly to (1,1) would involve moving it up six rows. I agree that this direct approach could reduce the risk of further penalties. Let's proceed with moving 'P' directly to (1,1) on both boards. I'll make the move on my board. Please confirm when you're ready to make the move on your board."} \\
\\ 
\\ 
\texttt{What is your next command?} \\
            }
        }
    }
    & \\ \\

    \theutterance \stepcounter{utterance}  
    & & & \multicolumn{4}{p{0.6\linewidth}}{
        \cellcolor[rgb]{0.9,0.9,0.9}{
            \makecell[{{p{\linewidth}}}]{
                \texttt{\tiny{[GM$\langle$P2]}}
                \texttt{MOVE: P, (1,1)} \\
            }
        }
    }
     \\ \\

    \theutterance \stepcounter{utterance}  
    & & & \multicolumn{2}{p{0.3\linewidth}}{
        \cellcolor[rgb]{0.9,0.9,0.9}{
            \makecell[{{p{\linewidth}}}]{
                \texttt{\tiny{[GM$|$GM]}}
                \texttt{MOVE: P, (1,1)} \\
            }
        }
    }
    & & \\ \\

    \theutterance \stepcounter{utterance}  
    & & & \multicolumn{2}{p{0.3\linewidth}}{
        \cellcolor[rgb]{0.9,0.9,0.9}{
            \makecell[{{p{\linewidth}}}]{
                \texttt{\tiny{[GM$|$GM]}}
                \texttt{Moved 'P' to (1,1) successfully. Your updated grid looks like this:} \\
\texttt{```} \\
\texttt{ 1234567} \\
\texttt{╔═╤═══╤═╗} \\
\texttt{║P│O◌◌│◌║ 1} \\
\texttt{╟─┼──┬┘◌║ 2} \\
\texttt{║D│◌◌│◌T║ 3} \\
\texttt{╟─┴──┘◌◌║ 4} \\
\texttt{║◌◌◌◌◌A◌║ 5} \\
\texttt{║◌◌◌◌◌◌◌║ 6} \\
\texttt{║◌◌S◌M◌◌║ 7} \\
\texttt{╚═══════╝} \\
\texttt{```} \\
            }
        }
    }
    & & \\ \\

    \theutterance \stepcounter{utterance}  
    & & \multicolumn{4}{p{0.6\linewidth}}{
        \cellcolor[rgb]{0.9,0.9,0.9}{
            \makecell[{{p{\linewidth}}}]{
                \texttt{\tiny{[P1$\langle$GM]}}
                \texttt{Your message has been relayed to the other player.} \\
\\ 
\\ 
\texttt{You are currently playing round 25 of maximum 28.} \\
\texttt{You have collectively accumulated 10 of 16 penalties.} \\
\texttt{The other player moved an object on their grid.} \\
\\ 
\\ 
\texttt{What is your next command?} \\
            }
        }
    }
    & \\ \\

    \theutterance \stepcounter{utterance}  
    & \multicolumn{4}{p{0.6\linewidth}}{
        \cellcolor[rgb]{0.9,0.9,0.9}{
            \makecell[{{p{\linewidth}}}]{
                \texttt{\tiny{[P1$\rangle$GM]}}
                \texttt{MOVE: P, (1, 1)} \\
            }
        }
    }
    & & \\ \\

    \theutterance \stepcounter{utterance}  
    & & & \multicolumn{2}{p{0.3\linewidth}}{
        \cellcolor[rgb]{0.9,0.9,0.9}{
            \makecell[{{p{\linewidth}}}]{
                \texttt{\tiny{[GM$|$GM]}}
                \texttt{MOVE: P, (1, 1)} \\
            }
        }
    }
    & & \\ \\

    \theutterance \stepcounter{utterance}  
    & & & \multicolumn{2}{p{0.3\linewidth}}{
        \cellcolor[rgb]{0.9,0.9,0.9}{
            \makecell[{{p{\linewidth}}}]{
                \texttt{\tiny{[GM$|$GM]}}
                \texttt{Moved 'P' to (1,1) successfully. Your updated grid looks like this:} \\
\texttt{```} \\
\texttt{ 1234567} \\
\texttt{╔═╤═══╤═╗} \\
\texttt{║P│O◌◌│◌║ 1} \\
\texttt{╟─┼──┬┘◌║ 2} \\
\texttt{║◌│◌◌│S◌║ 3} \\
\texttt{╟─┴──┘DT║ 4} \\
\texttt{║◌◌◌◌◌◌◌║ 5} \\
\texttt{║◌◌◌◌◌◌◌║ 6} \\
\texttt{║◌◌A◌M◌◌║ 7} \\
\texttt{╚═══════╝} \\
\texttt{```} \\
            }
        }
    }
    & & \\ \\

    \theutterance \stepcounter{utterance}  
    & & \multicolumn{4}{p{0.6\linewidth}}{
        \cellcolor[rgb]{0.9,0.9,0.9}{
            \makecell[{{p{\linewidth}}}]{
                \texttt{\tiny{[GM$\rangle$P2]}}
                \texttt{Moved 'P' to (1,1) successfully. Your updated grid looks like this:} \\
\texttt{```} \\
\texttt{ 1234567} \\
\texttt{╔═╤═══╤═╗} \\
\texttt{║P│O◌◌│◌║ 1} \\
\texttt{╟─┼──┬┘◌║ 2} \\
\texttt{║D│◌◌│◌T║ 3} \\
\texttt{╟─┴──┘◌◌║ 4} \\
\texttt{║◌◌◌◌◌A◌║ 5} \\
\texttt{║◌◌◌◌◌◌◌║ 6} \\
\texttt{║◌◌S◌M◌◌║ 7} \\
\texttt{╚═══════╝} \\
\texttt{```} \\
\texttt{You are currently playing round 26 of maximum 28.} \\
\texttt{You have collectively accumulated 10 of 16 penalties.} \\
\texttt{The other player moved an object on their grid.} \\
\\ 
\\ 
\texttt{What is your next command?} \\
            }
        }
    }
    & \\ \\

    \theutterance \stepcounter{utterance}  
    & & & \multicolumn{4}{p{0.6\linewidth}}{
        \cellcolor[rgb]{0.9,0.9,0.9}{
            \makecell[{{p{\linewidth}}}]{
                \texttt{\tiny{[GM$\langle$P2]}}
                \texttt{SAY: Now that 'P' is at (1,1) on both boards, we've made significant progress. Let's review the remaining objects and their assigned positions: 'O' is already at (3,1), 'T' needs to move to (5,1), 'S' needs to move to (2,2), 'D' needs to move to (4,2), and 'A' and 'M' are already close to their assigned positions at (3,7) and (5,7) respectively. Which object do you suggest we prioritize next to move to its assigned position, and what intermediate steps should we consider?} \\
            }
        }
    }
     \\ \\

    \theutterance \stepcounter{utterance}  
    & & & \multicolumn{2}{p{0.3\linewidth}}{
        \cellcolor[rgb]{0.9,0.9,0.9}{
            \makecell[{{p{\linewidth}}}]{
                \texttt{\tiny{[GM$|$GM]}}
                \texttt{SAY: Now that 'P' is at (1,1) on both boards, we've made significant progress. Let's review the remaining objects and their assigned positions: 'O' is already at (3,1), 'T' needs to move to (5,1), 'S' needs to move to (2,2), 'D' needs to move to (4,2), and 'A' and 'M' are already close to their assigned positions at (3,7) and (5,7) respectively. Which object do you suggest we prioritize next to move to its assigned position, and what intermediate steps should we consider?} \\
            }
        }
    }
    & & \\ \\

    \theutterance \stepcounter{utterance}  
    & & \multicolumn{4}{p{0.6\linewidth}}{
        \cellcolor[rgb]{0.9,0.9,0.9}{
            \makecell[{{p{\linewidth}}}]{
                \texttt{\tiny{[P1$\langle$GM]}}
                \texttt{Moved 'P' to (1,1) successfully. Your updated grid looks like this:} \\
\texttt{```} \\
\texttt{ 1234567} \\
\texttt{╔═╤═══╤═╗} \\
\texttt{║P│O◌◌│◌║ 1} \\
\texttt{╟─┼──┬┘◌║ 2} \\
\texttt{║◌│◌◌│S◌║ 3} \\
\texttt{╟─┴──┘DT║ 4} \\
\texttt{║◌◌◌◌◌◌◌║ 5} \\
\texttt{║◌◌◌◌◌◌◌║ 6} \\
\texttt{║◌◌A◌M◌◌║ 7} \\
\texttt{╚═══════╝} \\
\texttt{```} \\
\texttt{You are currently playing round 26 of maximum 28.} \\
\texttt{You have collectively accumulated 10 of 16 penalties.} \\
\texttt{The other player sent this message:} \\
\\ 
\\ 
\texttt{"Now that 'P' is at (1,1) on both boards, we've made significant progress. Let's review the remaining objects and their assigned positions: 'O' is already at (3,1), 'T' needs to move to (5,1), 'S' needs to move to (2,2), 'D' needs to move to (4,2), and 'A' and 'M' are already close to their assigned positions at (3,7) and (5,7) respectively. Which object do you suggest we prioritize next to move to its assigned position, and what intermediate steps should we consider?"} \\
\\ 
\\ 
\texttt{What is your next command?} \\
            }
        }
    }
    & \\ \\

    \theutterance \stepcounter{utterance}  
    & \multicolumn{4}{p{0.6\linewidth}}{
        \cellcolor[rgb]{0.9,0.9,0.9}{
            \makecell[{{p{\linewidth}}}]{
                \texttt{\tiny{[P1$\rangle$GM]}}
                \texttt{SAY: Now that 'P' is in place, let's focus on 'T' next, as it's one of the objects that needs to move to its assigned position (5,1). On my board, I'll need to locate 'T' and plan its move. Once I find 'T', I can determine the best intermediate steps to get it to (5,1). Can you please tell me the current position of 'T' on your board and suggest any intermediate positions that could help us synchronize our moves for 'T'?} \\
            }
        }
    }
    & & \\ \\

    \theutterance \stepcounter{utterance}  
    & & & \multicolumn{2}{p{0.3\linewidth}}{
        \cellcolor[rgb]{0.9,0.9,0.9}{
            \makecell[{{p{\linewidth}}}]{
                \texttt{\tiny{[GM$|$GM]}}
                \texttt{SAY: Now that 'P' is in place, let's focus on 'T' next, as it's one of the objects that needs to move to its assigned position (5,1). On my board, I'll need to locate 'T' and plan its move. Once I find 'T', I can determine the best intermediate steps to get it to (5,1). Can you please tell me the current position of 'T' on your board and suggest any intermediate positions that could help us synchronize our moves for 'T'?} \\
            }
        }
    }
    & & \\ \\

    \theutterance \stepcounter{utterance}  
    & & \multicolumn{4}{p{0.6\linewidth}}{
        \cellcolor[rgb]{0.9,0.9,0.9}{
            \makecell[{{p{\linewidth}}}]{
                \texttt{\tiny{[GM$\rangle$P2]}}
                \texttt{Your message has been relayed to the other player.} \\
\\ 
\\ 
\texttt{You are currently playing round 27 of maximum 28.} \\
\texttt{You have collectively accumulated 10 of 16 penalties.} \\
\texttt{The other player sent this message:} \\
\\ 
\\ 
\texttt{"Now that 'P' is in place, let's focus on 'T' next, as it's one of the objects that needs to move to its assigned position (5,1). On my board, I'll need to locate 'T' and plan its move. Once I find 'T', I can determine the best intermediate steps to get it to (5,1). Can you please tell me the current position of 'T' on your board and suggest any intermediate positions that could help us synchronize our moves for 'T'?"} \\
\\ 
\\ 
\texttt{What is your next command?} \\
            }
        }
    }
    & \\ \\

    \theutterance \stepcounter{utterance}  
    & & & \multicolumn{4}{p{0.6\linewidth}}{
        \cellcolor[rgb]{0.9,0.9,0.9}{
            \makecell[{{p{\linewidth}}}]{
                \texttt{\tiny{[GM$\langle$P2]}}
                \texttt{SAY: On my board, 'T' is currently at (4,3). To move 'T' to its assigned position (5,1), I can suggest moving it to an intermediate position such as (4,1) or (5,3) before finally moving it to (5,1). Since we want to synchronize our moves, please let me know the current position of 'T' on your board and any intermediate positions you think would be helpful. This will allow us to coordinate our moves effectively and get 'T' to (5,1) on both boards.} \\
            }
        }
    }
     \\ \\

    \theutterance \stepcounter{utterance}  
    & & & \multicolumn{2}{p{0.3\linewidth}}{
        \cellcolor[rgb]{0.9,0.9,0.9}{
            \makecell[{{p{\linewidth}}}]{
                \texttt{\tiny{[GM$|$GM]}}
                \texttt{SAY: On my board, 'T' is currently at (4,3). To move 'T' to its assigned position (5,1), I can suggest moving it to an intermediate position such as (4,1) or (5,3) before finally moving it to (5,1). Since we want to synchronize our moves, please let me know the current position of 'T' on your board and any intermediate positions you think would be helpful. This will allow us to coordinate our moves effectively and get 'T' to (5,1) on both boards.} \\
            }
        }
    }
    & & \\ \\

    \theutterance \stepcounter{utterance}  
    & & \multicolumn{4}{p{0.6\linewidth}}{
        \cellcolor[rgb]{0.9,0.9,0.9}{
            \makecell[{{p{\linewidth}}}]{
                \texttt{\tiny{[P1$\langle$GM]}}
                \texttt{Your message has been relayed to the other player.} \\
\\ 
\\ 
\texttt{You are currently playing round 27 of maximum 28.} \\
\texttt{You have collectively accumulated 10 of 16 penalties.} \\
\texttt{The other player sent this message:} \\
\\ 
\\ 
\texttt{"On my board, 'T' is currently at (4,3). To move 'T' to its assigned position (5,1), I can suggest moving it to an intermediate position such as (4,1) or (5,3) before finally moving it to (5,1). Since we want to synchronize our moves, please let me know the current position of 'T' on your board and any intermediate positions you think would be helpful. This will allow us to coordinate our moves effectively and get 'T' to (5,1) on both boards."} \\
\\ 
\\ 
\texttt{What is your next command?} \\
            }
        }
    }
    & \\ \\

    \theutterance \stepcounter{utterance}  
    & \multicolumn{4}{p{0.6\linewidth}}{
        \cellcolor[rgb]{0.9,0.9,0.9}{
            \makecell[{{p{\linewidth}}}]{
                \texttt{\tiny{[P1$\rangle$GM]}}
                \texttt{SAY: Unfortunately, I don't have the current position of 'T' on my board, as it's not visible in the provided grid. However, I can try to move 'T' to its assigned position (5,1) based on our previous discussions. Assuming 'T' is not at its assigned position, I suggest we both try to move 'T' to (5,1) directly, and if that's not possible, we can discuss alternative intermediate positions. Please try to move 'T' to (5,1) on your board, and I'll attempt to do the same on my board. If we encounter any issues, we can reassess and adjust our strategy accordingly.} \\
            }
        }
    }
    & & \\ \\

    \theutterance \stepcounter{utterance}  
    & & & \multicolumn{2}{p{0.3\linewidth}}{
        \cellcolor[rgb]{0.9,0.9,0.9}{
            \makecell[{{p{\linewidth}}}]{
                \texttt{\tiny{[GM$|$GM]}}
                \texttt{SAY: Unfortunately, I don't have the current position of 'T' on my board, as it's not visible in the provided grid. However, I can try to move 'T' to its assigned position (5,1) based on our previous discussions. Assuming 'T' is not at its assigned position, I suggest we both try to move 'T' to (5,1) directly, and if that's not possible, we can discuss alternative intermediate positions. Please try to move 'T' to (5,1) on your board, and I'll attempt to do the same on my board. If we encounter any issues, we can reassess and adjust our strategy accordingly.} \\
            }
        }
    }
    & & \\ \\

    \theutterance \stepcounter{utterance}  
    & & \multicolumn{4}{p{0.6\linewidth}}{
        \cellcolor[rgb]{0.9,0.9,0.9}{
            \makecell[{{p{\linewidth}}}]{
                \texttt{\tiny{[GM$\rangle$P2]}}
                \texttt{Your message has been relayed to the other player.} \\
\\ 
\\ 
\texttt{You are currently playing round 28 of maximum 28.} \\
\texttt{You have collectively accumulated 10 of 16 penalties.} \\
\texttt{The other player sent this message:} \\
\\ 
\\ 
\texttt{"Unfortunately, I don't have the current position of 'T' on my board, as it's not visible in the provided grid. However, I can try to move 'T' to its assigned position (5,1) based on our previous discussions. Assuming 'T' is not at its assigned position, I suggest we both try to move 'T' to (5,1) directly, and if that's not possible, we can discuss alternative intermediate positions. Please try to move 'T' to (5,1) on your board, and I'll attempt to do the same on my board. If we encounter any issues, we can reassess and adjust our strategy accordingly."} \\
\\ 
\\ 
\texttt{What is your next command?} \\
            }
        }
    }
    & \\ \\

    \theutterance \stepcounter{utterance}  
    & & & \multicolumn{4}{p{0.6\linewidth}}{
        \cellcolor[rgb]{0.9,0.9,0.9}{
            \makecell[{{p{\linewidth}}}]{
                \texttt{\tiny{[GM$\langle$P2]}}
                \texttt{MOVE: T, (5,1)} \\
            }
        }
    }
     \\ \\

    \theutterance \stepcounter{utterance}  
    & & & \multicolumn{2}{p{0.3\linewidth}}{
        \cellcolor[rgb]{0.9,0.9,0.9}{
            \makecell[{{p{\linewidth}}}]{
                \texttt{\tiny{[GM$|$GM]}}
                \texttt{MOVE: T, (5,1)} \\
            }
        }
    }
    & & \\ \\

    \theutterance \stepcounter{utterance}  
    & & & \multicolumn{2}{p{0.3\linewidth}}{
        \cellcolor[rgb]{0.9,0.9,0.9}{
            \makecell[{{p{\linewidth}}}]{
                \texttt{\tiny{[GM$|$GM]}}
                \texttt{Moved 'T' to (5,1) successfully. Your updated grid looks like this:} \\
\texttt{```} \\
\texttt{ 1234567} \\
\texttt{╔═╤═══╤═╗} \\
\texttt{║P│O◌T│◌║ 1} \\
\texttt{╟─┼──┬┘◌║ 2} \\
\texttt{║D│◌◌│◌◌║ 3} \\
\texttt{╟─┴──┘◌◌║ 4} \\
\texttt{║◌◌◌◌◌A◌║ 5} \\
\texttt{║◌◌◌◌◌◌◌║ 6} \\
\texttt{║◌◌S◌M◌◌║ 7} \\
\texttt{╚═══════╝} \\
\texttt{```} \\
            }
        }
    }
    & & \\ \\

    \theutterance \stepcounter{utterance}  
    & & & \multicolumn{2}{p{0.3\linewidth}}{
        \cellcolor[rgb]{0.9,0.9,0.9}{
            \makecell[{{p{\linewidth}}}]{
                \texttt{\tiny{[GM$|$GM]}}
                \texttt{Maximum number of rounds reached} \\
            }
        }
    }
    & & \\ \\

    \theutterance \stepcounter{utterance}  
    & & & \multicolumn{2}{p{0.3\linewidth}}{
        \cellcolor[rgb]{0.9,0.9,0.9}{
            \makecell[{{p{\linewidth}}}]{
                \texttt{\tiny{[GM$|$GM]}}
                \texttt{* success: True} \\
\texttt{* lose: False} \\
\texttt{* aborted: False} \\
\texttt{{-}{-}{-}{-}{-}{-}{-}} \\
\texttt{* Shifts: 5.00} \\
\texttt{* Max Shifts: 12.00} \\
\texttt{* Min Shifts: 6.00} \\
\texttt{* End Distance Sum: 17.31} \\
\texttt{* Init Distance Sum: 25.60} \\
\texttt{* Expected Distance Sum: 29.33} \\
\texttt{* Penalties: 10.00} \\
\texttt{* Max Penalties: 16.00} \\
\texttt{* Rounds: 28.00} \\
\texttt{* Max Rounds: 28.00} \\
\texttt{* Object Count: 7.00} \\
            }
        }
    }
    & & \\ \\

    \theutterance \stepcounter{utterance}  
    & & & \multicolumn{2}{p{0.3\linewidth}}{
        \cellcolor[rgb]{0.9,0.9,0.9}{
            \makecell[{{p{\linewidth}}}]{
                \texttt{\tiny{[GM$|$GM]}}
                \texttt{Initial Grids:} \\
\texttt{```} \\
\texttt{Player 1:     Player 2:} \\
\texttt{ 1234567       1234567} \\
\texttt{╔═╤═══╤═╗     ╔═╤═══╤═╗} \\
\texttt{║◌│◌◌◌│◌║ 1   ║◌│◌◌◌│◌║ 1} \\
\texttt{╟─┼──┬┘◌║ 2   ╟─┼──┬┘◌║ 2} \\
\texttt{║◌│◌◌│S◌║ 3   ║◌│◌M│◌T║ 3} \\
\texttt{╟─┴──┘DT║ 4   ╟─┴──┘P◌║ 4} \\
\texttt{║◌O◌◌◌◌◌║ 5   ║D◌◌◌◌A◌║ 5} \\
\texttt{║◌◌◌◌◌◌◌║ 6   ║◌◌◌◌O◌◌║ 6} \\
\texttt{║◌◌APM◌◌║ 7   ║◌◌S◌◌◌◌║ 7} \\
\texttt{╚═══════╝     ╚═══════╝} \\
\texttt{```} \\
\\ 
\texttt{Final Grids:} \\
\texttt{```} \\
\texttt{Player 1:     Player 2:} \\
\texttt{ 1234567       1234567} \\
\texttt{╔═╤═══╤═╗     ╔═╤═══╤═╗} \\
\texttt{║P│O◌◌│◌║ 1   ║P│O◌T│◌║ 1} \\
\texttt{╟─┼──┬┘◌║ 2   ╟─┼──┬┘◌║ 2} \\
\texttt{║◌│◌◌│S◌║ 3   ║D│◌◌│◌◌║ 3} \\
\texttt{╟─┴──┘DT║ 4   ╟─┴──┘◌◌║ 4} \\
\texttt{║◌◌◌◌◌◌◌║ 5   ║◌◌◌◌◌A◌║ 5} \\
\texttt{║◌◌◌◌◌◌◌║ 6   ║◌◌◌◌◌◌◌║ 6} \\
\texttt{║◌◌A◌M◌◌║ 7   ║◌◌S◌M◌◌║ 7} \\
\texttt{╚═══════╝     ╚═══════╝} \\
\texttt{```} \\
            }
        }
    }
    & & \\ \\

\end{supertabular}
}

\end{document}
