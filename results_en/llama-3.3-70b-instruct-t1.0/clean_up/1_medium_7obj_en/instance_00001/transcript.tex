
\documentclass{article}
\usepackage{colortbl}
\usepackage{makecell}
\usepackage{multirow}
\usepackage{supertabular}

\begin{document}

\newcounter{utterance}

\centering \large Interaction Transcript for game `clean\_up', experiment `1\_medium\_7obj\_en', episode 1 with llama{-}3.3{-}70b{-}instruct{-}t1.0.
\vspace{24pt}

{ \footnotesize  \setcounter{utterance}{1}
\setlength{\tabcolsep}{0pt}
\begin{supertabular}{c@{$\;$}|p{.15\linewidth}@{}p{.15\linewidth}p{.15\linewidth}p{.15\linewidth}p{.15\linewidth}p{.15\linewidth}}
    \# & $\;$A & \multicolumn{4}{c}{Game Master} & $\;\:$B\\
    \hline

    \theutterance \stepcounter{utterance}  
    & & \multicolumn{4}{p{0.6\linewidth}}{
        \cellcolor[rgb]{0.9,0.9,0.9}{
            \makecell[{{p{\linewidth}}}]{
                \texttt{\tiny{[P1$\langle$GM]}}
                \texttt{I am your game master, and you are playing a collaborative game with the following grid as game board:} \\
\\ 
\texttt{```} \\
\texttt{ 1234567} \\
\texttt{╔═══╤═╤═╗} \\
\texttt{║◌◌◌│◌│◌║ 1} \\
\texttt{║◌◌M│◌├─╢ 2} \\
\texttt{║AT◌├─┘◌║ 3} \\
\texttt{║◌◌P│◌◌◌║ 4} \\
\texttt{╟─┬─┼───╢ 5} \\
\texttt{║◌│◌│S◌O║ 6} \\
\texttt{║◌│◌│◌D◌║ 7} \\
\texttt{╚═╧═╧═══╝} \\
\\ 
\texttt{```} \\
\\ 
\texttt{* The upper edge displays x{-}coordinates increasing to the right, and the right edge y{-}coordinates increasing downward.} \\
\texttt{* The following objects are randomly placed on your grid: 'P', 'O', 'T', 'S', 'D', 'A', 'M'.} \\
\\ 
\\ 
\texttt{The other player sees a variation of the game board, where the objects are placed at different random locations. You cannot see the other player's board, and they cannot see yours.} \\
\\ 
\\ 
\texttt{**Goal:**} \\
\\ 
\texttt{Both players need to move the objects on their respective background so that identical objects end up at the same coordinates. You have to communicate with the other player to agree upon a common goal configuration.} \\
\\ 
\\ 
\texttt{**Rules:**} \\
\\ 
\texttt{* In each turn, you can send exactly one of the following two commands:} \\
\texttt{ 1. `SAY: <MESSAGE>`: to send a message (everything up until the next line break) to the other player. I will forward it to your partner.} \\
\texttt{2. `MOVE: <OBJECT>, (<X>, <Y>)`: to move an object to a new position, where `<X>` is the column and `<Y>` is the row. I will inform you if your move was successful or not.} \\
\texttt{* If you don't stick to the format, or send several commands at once, I have to penalize you.} \\
\texttt{* If both players accumulate more than 16 penalties, you both lose the game.} \\
\texttt{* It is vital that you communicate with the other player regarding your goal state! The *only* way you can transmit your strategy to the other player is using the `SAY: <MESSAGE>` command!} \\
\\ 
\\ 
\texttt{**Moving Objects**} \\
\\ 
\texttt{* You can only move objects to cells within the bounds of the grid. The target cell must be empty, i.e., it must only contain the symbol '◌'.} \\
\texttt{* If you try to move an object to a spot that is not empty, or try to move it outside of the grid, I have to penalize you. You get another try.} \\
\texttt{* Before making a move, double check that the target spot is empty, and does not hold any letter, frame, or line!} \\
\\ 
\\ 
\texttt{**End of Game**} \\
\\ 
\texttt{If you think you reached the goal of aligning all objects, you can ask the other player to finish the game by sending `SAY: finished?`. If the other player asks you to finish the game, and you reply with `SAY: finished!`, the game will end.} \\
\\ 
\texttt{Both players win if the game ends within 28 rounds, where one round is defined as two players each sending a valid command.} \\
\\ 
\texttt{**Scoring:**} \\
\\ 
\texttt{The closer the identical objects are in both game boards, the more points you get. Penalties reduce your points. Can you beat the record?} \\
\\ 
\\ 
\texttt{Please send a message to the other player to start the game!} \\
            }
        }
    }
    & \\ \\

    \theutterance \stepcounter{utterance}  
    & \multicolumn{4}{p{0.6\linewidth}}{
        \cellcolor[rgb]{0.9,0.9,0.9}{
            \makecell[{{p{\linewidth}}}]{
                \texttt{\tiny{[P1$\rangle$GM]}}
                \texttt{SAY: Hello, let's start by describing our initial boards. I'll describe mine by listing the coordinates of each object. 'P' is at (4,4), 'O' is at (6,6), 'T' is missing, 'S' is at (6,5), 'D' is at (7,7), 'A' is at (1,3), 'M' is at (3,2). Please respond with the coordinates of the objects on your board.} \\
            }
        }
    }
    & & \\ \\

    \theutterance \stepcounter{utterance}  
    & & & \multicolumn{2}{p{0.3\linewidth}}{
        \cellcolor[rgb]{0.9,0.9,0.9}{
            \makecell[{{p{\linewidth}}}]{
                \texttt{\tiny{[GM$|$GM]}}
                \texttt{SAY: Hello, let's start by describing our initial boards. I'll describe mine by listing the coordinates of each object. 'P' is at (4,4), 'O' is at (6,6), 'T' is missing, 'S' is at (6,5), 'D' is at (7,7), 'A' is at (1,3), 'M' is at (3,2). Please respond with the coordinates of the objects on your board.} \\
            }
        }
    }
    & & \\ \\

    \theutterance \stepcounter{utterance}  
    & & \multicolumn{4}{p{0.6\linewidth}}{
        \cellcolor[rgb]{0.9,0.9,0.9}{
            \makecell[{{p{\linewidth}}}]{
                \texttt{\tiny{[GM$\rangle$P2]}}
                \texttt{I am your game master, and you are playing a collaborative game with the following grid as game board:} \\
\\ 
\texttt{```} \\
\texttt{ 1234567} \\
\texttt{╔═══╤═╤═╗} \\
\texttt{║◌MS│◌│◌║ 1} \\
\texttt{║◌O◌│◌├─╢ 2} \\
\texttt{║◌◌◌├─┘◌║ 3} \\
\texttt{║◌◌◌│◌◌◌║ 4} \\
\texttt{╟─┬─┼───╢ 5} \\
\texttt{║◌│◌│◌◌D║ 6} \\
\texttt{║T│◌│◌PA║ 7} \\
\texttt{╚═╧═╧═══╝} \\
\\ 
\texttt{```} \\
\\ 
\texttt{* The upper edge displays x{-}coordinates increasing to the right, and the right edge y{-}coordinates increasing downward.} \\
\texttt{* The following objects are randomly placed on your grid: 'P', 'O', 'T', 'S', 'D', 'A', 'M'.} \\
\\ 
\\ 
\texttt{The other player sees a variation of the game board, where the objects are placed at different random locations. You cannot see the other player's board, and they cannot see yours.} \\
\\ 
\\ 
\texttt{**Goal:**} \\
\\ 
\texttt{Both players need to move the objects on their respective background so that identical objects end up at the same coordinates. You have to communicate with the other player to agree upon a common goal configuration.} \\
\\ 
\\ 
\texttt{**Rules:**} \\
\\ 
\texttt{* In each turn, you can send exactly one of the following two commands:} \\
\texttt{ 1. `SAY: <MESSAGE>`: to send a message (everything up until the next line break) to the other player. I will forward it to your partner.} \\
\texttt{2. `MOVE: <OBJECT>, (<X>, <Y>)`: to move an object to a new position, where `<X>` is the column and `<Y>` is the row. I will inform you if your move was successful or not.} \\
\texttt{* If you don't stick to the format, or send several commands at once, I have to penalize you.} \\
\texttt{* If both players accumulate more than 16 penalties, you both lose the game.} \\
\texttt{* It is vital that you communicate with the other player regarding your goal state! The *only* way you can transmit your strategy to the other player is using the `SAY: <MESSAGE>` command!} \\
\\ 
\\ 
\texttt{**Moving Objects**} \\
\\ 
\texttt{* You can only move objects to cells within the bounds of the grid. The target cell must be empty, i.e., it must only contain the symbol '◌'.} \\
\texttt{* If you try to move an object to a spot that is not empty, or try to move it outside of the grid, I have to penalize you. You get another try.} \\
\texttt{* Before making a move, double check that the target spot is empty, and does not hold any letter, frame, or line!} \\
\\ 
\\ 
\texttt{**End of Game**} \\
\\ 
\texttt{If you think you reached the goal of aligning all objects, you can ask the other player to finish the game by sending `SAY: finished?`. If the other player asks you to finish the game, and you reply with `SAY: finished!`, the game will end.} \\
\\ 
\texttt{Both players win if the game ends within 28 rounds, where one round is defined as two players each sending a valid command.} \\
\\ 
\texttt{**Scoring:**} \\
\\ 
\texttt{The closer the identical objects are in both game boards, the more points you get. Penalties reduce your points. Can you beat the record?} \\
\\ 
\texttt{The other player started the game by sending this message:} \\
\\ 
\texttt{"Hello, let's start by describing our initial boards. I'll describe mine by listing the coordinates of each object. 'P' is at (4,4), 'O' is at (6,6), 'T' is missing, 'S' is at (6,5), 'D' is at (7,7), 'A' is at (1,3), 'M' is at (3,2). Please respond with the coordinates of the objects on your board."} \\
\\ 
\texttt{What is your first command?} \\
            }
        }
    }
    & \\ \\

    \theutterance \stepcounter{utterance}  
    & & & \multicolumn{4}{p{0.6\linewidth}}{
        \cellcolor[rgb]{0.9,0.9,0.9}{
            \makecell[{{p{\linewidth}}}]{
                \texttt{\tiny{[GM$\langle$P2]}}
                \texttt{Since the other player has sent a message describing their initial board and asking for the coordinates of the objects on my board, I will respond with the coordinates of the objects on my board.} \\
\\ 
\texttt{The objects on my board are: 'P' is at (6,7), 'O' is at (2,2), 'T' is at (1,7), 'S' is at (1,1), 'D' is at (7,6), 'A' is at (7,7), 'M' is at (1,1).} \\
\\ 
\texttt{My command is: } \\
\texttt{SAY: 'P' is at (6,7), 'O' is at (2,2), 'T' is at (1,7), 'S' is at (1,1), 'D' is at (7,6), 'A' is at (7,7), 'M' is at (1,1)} \\
            }
        }
    }
     \\ \\

    \theutterance \stepcounter{utterance}  
    & & & \multicolumn{2}{p{0.3\linewidth}}{
        \cellcolor[rgb]{0.9,0.9,0.9}{
            \makecell[{{p{\linewidth}}}]{
                \texttt{\tiny{[GM$|$GM]}}
                \texttt{Since the other player has sent a message describing their initial board and asking for the coordinates of the objects on my board, I will respond with the coordinates of the objects on my board.} \\
\\ 
\texttt{The objects on my board are: 'P' is at (6,7), 'O' is at (2,2), 'T' is at (1,7), 'S' is at (1,1), 'D' is at (7,6), 'A' is at (7,7), 'M' is at (1,1).} \\
\\ 
\texttt{My command is: } \\
\texttt{SAY: 'P' is at (6,7), 'O' is at (2,2), 'T' is at (1,7), 'S' is at (1,1), 'D' is at (7,6), 'A' is at (7,7), 'M' is at (1,1)} \\
            }
        }
    }
    & & \\ \\

    \theutterance \stepcounter{utterance}  
    & & & \multicolumn{2}{p{0.3\linewidth}}{
        \cellcolor[rgb]{0.9,0.9,0.9}{
            \makecell[{{p{\linewidth}}}]{
                \texttt{\tiny{[GM$|$GM]}}
                \texttt{Invalid format: head is not empty: 'Since the other player has sent a message describing their initial board and asking for the coordinates of the objects on my board, I will respond with the coordinates of the objects on my board.} \\
\\ 
\texttt{The objects on my board are: 'P' is at (6,7), 'O' is at (2,2), 'T' is at (1,7), 'S' is at (1,1), 'D' is at (7,6), 'A' is at (7,7), 'M' is at (1,1).} \\
\\ 
\texttt{My command is: } \\
\texttt{'} \\
            }
        }
    }
    & & \\ \\

    \theutterance \stepcounter{utterance}  
    & & \multicolumn{4}{p{0.6\linewidth}}{
        \cellcolor[rgb]{0.9,0.9,0.9}{
            \makecell[{{p{\linewidth}}}]{
                \texttt{\tiny{[GM$\rangle$P2]}}
                \texttt{Penalty: Your message must not contain anything before the command!} \\
\\ 
\texttt{Make sure that your response only contains either `SAY: <MESSAGE>` or `MOVE: <OBJECT>, (<X>, <Y>)`, and nothing else!} \\
\\ 
\texttt{You have collectively accumulated 1 of 16 penalties.} \\
\texttt{Please try again!} \\
            }
        }
    }
    & \\ \\

    \theutterance \stepcounter{utterance}  
    & & & \multicolumn{4}{p{0.6\linewidth}}{
        \cellcolor[rgb]{0.9,0.9,0.9}{
            \makecell[{{p{\linewidth}}}]{
                \texttt{\tiny{[GM$\langle$P2]}}
                \texttt{SAY: 'P' is at (6,7), 'O' is at (2,2), 'T' is at (1,7), 'S' is at (1,1), 'D' is at (7,6), 'A' is at (7,7), 'M' is at (1,1)} \\
            }
        }
    }
     \\ \\

    \theutterance \stepcounter{utterance}  
    & & & \multicolumn{2}{p{0.3\linewidth}}{
        \cellcolor[rgb]{0.9,0.9,0.9}{
            \makecell[{{p{\linewidth}}}]{
                \texttt{\tiny{[GM$|$GM]}}
                \texttt{SAY: 'P' is at (6,7), 'O' is at (2,2), 'T' is at (1,7), 'S' is at (1,1), 'D' is at (7,6), 'A' is at (7,7), 'M' is at (1,1)} \\
            }
        }
    }
    & & \\ \\

    \theutterance \stepcounter{utterance}  
    & & \multicolumn{4}{p{0.6\linewidth}}{
        \cellcolor[rgb]{0.9,0.9,0.9}{
            \makecell[{{p{\linewidth}}}]{
                \texttt{\tiny{[P1$\langle$GM]}}
                \texttt{Your message has been relayed to the other player.} \\
\\ 
\\ 
\texttt{You are currently playing round 1 of maximum 28.} \\
\texttt{You have collectively accumulated 1 of 16 penalties.} \\
\texttt{The other player sent this message:} \\
\\ 
\\ 
\texttt{"'P' is at (6,7), 'O' is at (2,2), 'T' is at (1,7), 'S' is at (1,1), 'D' is at (7,6), 'A' is at (7,7), 'M' is at (1,1)"} \\
\\ 
\\ 
\texttt{What is your next command?} \\
            }
        }
    }
    & \\ \\

    \theutterance \stepcounter{utterance}  
    & \multicolumn{4}{p{0.6\linewidth}}{
        \cellcolor[rgb]{0.9,0.9,0.9}{
            \makecell[{{p{\linewidth}}}]{
                \texttt{\tiny{[P1$\rangle$GM]}}
                \texttt{SAY: It seems our boards are quite different. Let's try to find a common goal configuration. I propose we aim to place each object in a unique row, with 'P' in row 1, 'O' in row 2, 'T' in row 3, 'S' in row 4, 'D' in row 5, 'A' in row 6, and 'M' in row 7. We can discuss the column placements later. What do you think?} \\
            }
        }
    }
    & & \\ \\

    \theutterance \stepcounter{utterance}  
    & & & \multicolumn{2}{p{0.3\linewidth}}{
        \cellcolor[rgb]{0.9,0.9,0.9}{
            \makecell[{{p{\linewidth}}}]{
                \texttt{\tiny{[GM$|$GM]}}
                \texttt{SAY: It seems our boards are quite different. Let's try to find a common goal configuration. I propose we aim to place each object in a unique row, with 'P' in row 1, 'O' in row 2, 'T' in row 3, 'S' in row 4, 'D' in row 5, 'A' in row 6, and 'M' in row 7. We can discuss the column placements later. What do you think?} \\
            }
        }
    }
    & & \\ \\

    \theutterance \stepcounter{utterance}  
    & & \multicolumn{4}{p{0.6\linewidth}}{
        \cellcolor[rgb]{0.9,0.9,0.9}{
            \makecell[{{p{\linewidth}}}]{
                \texttt{\tiny{[GM$\rangle$P2]}}
                \texttt{Your message has been relayed to the other player.} \\
\\ 
\\ 
\texttt{You are currently playing round 2 of maximum 28.} \\
\texttt{You have collectively accumulated 1 of 16 penalties.} \\
\texttt{The other player sent this message:} \\
\\ 
\\ 
\texttt{"It seems our boards are quite different. Let's try to find a common goal configuration. I propose we aim to place each object in a unique row, with 'P' in row 1, 'O' in row 2, 'T' in row 3, 'S' in row 4, 'D' in row 5, 'A' in row 6, and 'M' in row 7. We can discuss the column placements later. What do you think?"} \\
\\ 
\\ 
\texttt{What is your next command?} \\
            }
        }
    }
    & \\ \\

    \theutterance \stepcounter{utterance}  
    & & & \multicolumn{4}{p{0.6\linewidth}}{
        \cellcolor[rgb]{0.9,0.9,0.9}{
            \makecell[{{p{\linewidth}}}]{
                \texttt{\tiny{[GM$\langle$P2]}}
                \texttt{SAY: That sounds like a good starting point. I agree with the row placements you proposed. For the columns, how about we place each object in the column that matches its alphabetical order, i.e., 'A' in column 1, 'D' in column 4, 'M' in column 13 (but since our grid only has 7 columns, let's place 'M' in column 7, and adjust the others accordingly), 'O' in column 2, 'P' in column 3, 'S' in column 4, and 'T' in column 5? Alternatively, we could discuss other column arrangements that might be more suitable.} \\
            }
        }
    }
     \\ \\

    \theutterance \stepcounter{utterance}  
    & & & \multicolumn{2}{p{0.3\linewidth}}{
        \cellcolor[rgb]{0.9,0.9,0.9}{
            \makecell[{{p{\linewidth}}}]{
                \texttt{\tiny{[GM$|$GM]}}
                \texttt{SAY: That sounds like a good starting point. I agree with the row placements you proposed. For the columns, how about we place each object in the column that matches its alphabetical order, i.e., 'A' in column 1, 'D' in column 4, 'M' in column 13 (but since our grid only has 7 columns, let's place 'M' in column 7, and adjust the others accordingly), 'O' in column 2, 'P' in column 3, 'S' in column 4, and 'T' in column 5? Alternatively, we could discuss other column arrangements that might be more suitable.} \\
            }
        }
    }
    & & \\ \\

    \theutterance \stepcounter{utterance}  
    & & \multicolumn{4}{p{0.6\linewidth}}{
        \cellcolor[rgb]{0.9,0.9,0.9}{
            \makecell[{{p{\linewidth}}}]{
                \texttt{\tiny{[P1$\langle$GM]}}
                \texttt{Your message has been relayed to the other player.} \\
\\ 
\\ 
\texttt{You are currently playing round 2 of maximum 28.} \\
\texttt{You have collectively accumulated 1 of 16 penalties.} \\
\texttt{The other player sent this message:} \\
\\ 
\\ 
\texttt{"That sounds like a good starting point. I agree with the row placements you proposed. For the columns, how about we place each object in the column that matches its alphabetical order, i.e., 'A' in column 1, 'D' in column 4, 'M' in column 13 (but since our grid only has 7 columns, let's place 'M' in column 7, and adjust the others accordingly), 'O' in column 2, 'P' in column 3, 'S' in column 4, and 'T' in column 5? Alternatively, we could discuss other column arrangements that might be more suitable."} \\
\\ 
\\ 
\texttt{What is your next command?} \\
            }
        }
    }
    & \\ \\

    \theutterance \stepcounter{utterance}  
    & \multicolumn{4}{p{0.6\linewidth}}{
        \cellcolor[rgb]{0.9,0.9,0.9}{
            \makecell[{{p{\linewidth}}}]{
                \texttt{\tiny{[P1$\rangle$GM]}}
                \texttt{SAY: I like the idea of using alphabetical order to determine the column placements. However, since our grid only has 7 columns, we'll need to adjust the placements for 'M' and any other letters that would fall outside the grid. Let's revise the column placements as follows: 'A' in column 1, 'D' in column 4, 'M' in column 7 (as you mentioned), 'O' in column 2, 'P' in column 3, 'S' in column 4 (but we already have 'D' in column 4, so let's place 'S' in column 5 and 'T' in column 6). This way, we can maintain a consistent ordering while accommodating the grid's limitations. What do you think of this revised arrangement?} \\
            }
        }
    }
    & & \\ \\

    \theutterance \stepcounter{utterance}  
    & & & \multicolumn{2}{p{0.3\linewidth}}{
        \cellcolor[rgb]{0.9,0.9,0.9}{
            \makecell[{{p{\linewidth}}}]{
                \texttt{\tiny{[GM$|$GM]}}
                \texttt{SAY: I like the idea of using alphabetical order to determine the column placements. However, since our grid only has 7 columns, we'll need to adjust the placements for 'M' and any other letters that would fall outside the grid. Let's revise the column placements as follows: 'A' in column 1, 'D' in column 4, 'M' in column 7 (as you mentioned), 'O' in column 2, 'P' in column 3, 'S' in column 4 (but we already have 'D' in column 4, so let's place 'S' in column 5 and 'T' in column 6). This way, we can maintain a consistent ordering while accommodating the grid's limitations. What do you think of this revised arrangement?} \\
            }
        }
    }
    & & \\ \\

    \theutterance \stepcounter{utterance}  
    & & \multicolumn{4}{p{0.6\linewidth}}{
        \cellcolor[rgb]{0.9,0.9,0.9}{
            \makecell[{{p{\linewidth}}}]{
                \texttt{\tiny{[GM$\rangle$P2]}}
                \texttt{Your message has been relayed to the other player.} \\
\\ 
\\ 
\texttt{You are currently playing round 3 of maximum 28.} \\
\texttt{You have collectively accumulated 1 of 16 penalties.} \\
\texttt{The other player sent this message:} \\
\\ 
\\ 
\texttt{"I like the idea of using alphabetical order to determine the column placements. However, since our grid only has 7 columns, we'll need to adjust the placements for 'M' and any other letters that would fall outside the grid. Let's revise the column placements as follows: 'A' in column 1, 'D' in column 4, 'M' in column 7 (as you mentioned), 'O' in column 2, 'P' in column 3, 'S' in column 4 (but we already have 'D' in column 4, so let's place 'S' in column 5 and 'T' in column 6). This way, we can maintain a consistent ordering while accommodating the grid's limitations. What do you think of this revised arrangement?"} \\
\\ 
\\ 
\texttt{What is your next command?} \\
            }
        }
    }
    & \\ \\

    \theutterance \stepcounter{utterance}  
    & & & \multicolumn{4}{p{0.6\linewidth}}{
        \cellcolor[rgb]{0.9,0.9,0.9}{
            \makecell[{{p{\linewidth}}}]{
                \texttt{\tiny{[GM$\langle$P2]}}
                \texttt{SAY: I think the revised arrangement you proposed is a good compromise, given the grid's limitations. I agree with the placements: 'A' in column 1, 'M' in column 7, 'O' in column 2, 'P' in column 3, 'D' in column 4, 'S' in column 5, and 'T' in column 6. Before we start moving objects, let's confirm that we both understand the goal configuration: each object in a unique row, with the rows ordered as initially proposed ('P' in row 1, 'O' in row 2, 'T' in row 3, 'S' in row 4, 'D' in row 5, 'A' in row 6, 'M' in row 7), and the columns as we just discussed. Shall we start moving objects to achieve this configuration?} \\
            }
        }
    }
     \\ \\

    \theutterance \stepcounter{utterance}  
    & & & \multicolumn{2}{p{0.3\linewidth}}{
        \cellcolor[rgb]{0.9,0.9,0.9}{
            \makecell[{{p{\linewidth}}}]{
                \texttt{\tiny{[GM$|$GM]}}
                \texttt{SAY: I think the revised arrangement you proposed is a good compromise, given the grid's limitations. I agree with the placements: 'A' in column 1, 'M' in column 7, 'O' in column 2, 'P' in column 3, 'D' in column 4, 'S' in column 5, and 'T' in column 6. Before we start moving objects, let's confirm that we both understand the goal configuration: each object in a unique row, with the rows ordered as initially proposed ('P' in row 1, 'O' in row 2, 'T' in row 3, 'S' in row 4, 'D' in row 5, 'A' in row 6, 'M' in row 7), and the columns as we just discussed. Shall we start moving objects to achieve this configuration?} \\
            }
        }
    }
    & & \\ \\

    \theutterance \stepcounter{utterance}  
    & & \multicolumn{4}{p{0.6\linewidth}}{
        \cellcolor[rgb]{0.9,0.9,0.9}{
            \makecell[{{p{\linewidth}}}]{
                \texttt{\tiny{[P1$\langle$GM]}}
                \texttt{Your message has been relayed to the other player.} \\
\\ 
\\ 
\texttt{You are currently playing round 3 of maximum 28.} \\
\texttt{You have collectively accumulated 1 of 16 penalties.} \\
\texttt{The other player sent this message:} \\
\\ 
\\ 
\texttt{"I think the revised arrangement you proposed is a good compromise, given the grid's limitations. I agree with the placements: 'A' in column 1, 'M' in column 7, 'O' in column 2, 'P' in column 3, 'D' in column 4, 'S' in column 5, and 'T' in column 6. Before we start moving objects, let's confirm that we both understand the goal configuration: each object in a unique row, with the rows ordered as initially proposed ('P' in row 1, 'O' in row 2, 'T' in row 3, 'S' in row 4, 'D' in row 5, 'A' in row 6, 'M' in row 7), and the columns as we just discussed. Shall we start moving objects to achieve this configuration?"} \\
\\ 
\\ 
\texttt{What is your next command?} \\
            }
        }
    }
    & \\ \\

    \theutterance \stepcounter{utterance}  
    & \multicolumn{4}{p{0.6\linewidth}}{
        \cellcolor[rgb]{0.9,0.9,0.9}{
            \makecell[{{p{\linewidth}}}]{
                \texttt{\tiny{[P1$\rangle$GM]}}
                \texttt{SAY: I'm glad we have a agreed upon goal configuration. To confirm, the target configuration is: 'P' in (3,1), 'O' in (2,2), 'T' in (6,3), 'S' in (5,4), 'D' in (4,5), 'A' in (1,6), and 'M' in (7,7). I've double{-}checked the rows and columns, and everything looks correct. Before we start moving objects, I'd like to suggest that we prioritize moving objects that are currently farthest from their target positions. This might help us make the most progress in the fewest number of moves. What's your thoughts on this approach?} \\
            }
        }
    }
    & & \\ \\

    \theutterance \stepcounter{utterance}  
    & & & \multicolumn{2}{p{0.3\linewidth}}{
        \cellcolor[rgb]{0.9,0.9,0.9}{
            \makecell[{{p{\linewidth}}}]{
                \texttt{\tiny{[GM$|$GM]}}
                \texttt{SAY: I'm glad we have a agreed upon goal configuration. To confirm, the target configuration is: 'P' in (3,1), 'O' in (2,2), 'T' in (6,3), 'S' in (5,4), 'D' in (4,5), 'A' in (1,6), and 'M' in (7,7). I've double{-}checked the rows and columns, and everything looks correct. Before we start moving objects, I'd like to suggest that we prioritize moving objects that are currently farthest from their target positions. This might help us make the most progress in the fewest number of moves. What's your thoughts on this approach?} \\
            }
        }
    }
    & & \\ \\

    \theutterance \stepcounter{utterance}  
    & & \multicolumn{4}{p{0.6\linewidth}}{
        \cellcolor[rgb]{0.9,0.9,0.9}{
            \makecell[{{p{\linewidth}}}]{
                \texttt{\tiny{[GM$\rangle$P2]}}
                \texttt{Your message has been relayed to the other player.} \\
\\ 
\\ 
\texttt{You are currently playing round 4 of maximum 28.} \\
\texttt{You have collectively accumulated 1 of 16 penalties.} \\
\texttt{The other player sent this message:} \\
\\ 
\\ 
\texttt{"I'm glad we have a agreed upon goal configuration. To confirm, the target configuration is: 'P' in (3,1), 'O' in (2,2), 'T' in (6,3), 'S' in (5,4), 'D' in (4,5), 'A' in (1,6), and 'M' in (7,7). I've double{-}checked the rows and columns, and everything looks correct. Before we start moving objects, I'd like to suggest that we prioritize moving objects that are currently farthest from their target positions. This might help us make the most progress in the fewest number of moves. What's your thoughts on this approach?"} \\
\\ 
\\ 
\texttt{What is your next command?} \\
            }
        }
    }
    & \\ \\

    \theutterance \stepcounter{utterance}  
    & & & \multicolumn{4}{p{0.6\linewidth}}{
        \cellcolor[rgb]{0.9,0.9,0.9}{
            \makecell[{{p{\linewidth}}}]{
                \texttt{\tiny{[GM$\langle$P2]}}
                \texttt{SAY: That's a great approach! Prioritizing objects that are farthest from their target positions makes sense, as it will help us make the most efficient moves. I'll review my board to identify which objects need to be moved the most. Based on our goal configuration, I'll start by identifying the objects that are farthest from their target positions and plan my moves accordingly. Shall we start moving objects now, and after each move, we can inform each other of the new positions of the objects on our respective boards?} \\
            }
        }
    }
     \\ \\

    \theutterance \stepcounter{utterance}  
    & & & \multicolumn{2}{p{0.3\linewidth}}{
        \cellcolor[rgb]{0.9,0.9,0.9}{
            \makecell[{{p{\linewidth}}}]{
                \texttt{\tiny{[GM$|$GM]}}
                \texttt{SAY: That's a great approach! Prioritizing objects that are farthest from their target positions makes sense, as it will help us make the most efficient moves. I'll review my board to identify which objects need to be moved the most. Based on our goal configuration, I'll start by identifying the objects that are farthest from their target positions and plan my moves accordingly. Shall we start moving objects now, and after each move, we can inform each other of the new positions of the objects on our respective boards?} \\
            }
        }
    }
    & & \\ \\

    \theutterance \stepcounter{utterance}  
    & & \multicolumn{4}{p{0.6\linewidth}}{
        \cellcolor[rgb]{0.9,0.9,0.9}{
            \makecell[{{p{\linewidth}}}]{
                \texttt{\tiny{[P1$\langle$GM]}}
                \texttt{Your message has been relayed to the other player.} \\
\\ 
\\ 
\texttt{You are currently playing round 4 of maximum 28.} \\
\texttt{You have collectively accumulated 1 of 16 penalties.} \\
\texttt{The other player sent this message:} \\
\\ 
\\ 
\texttt{"That's a great approach! Prioritizing objects that are farthest from their target positions makes sense, as it will help us make the most efficient moves. I'll review my board to identify which objects need to be moved the most. Based on our goal configuration, I'll start by identifying the objects that are farthest from their target positions and plan my moves accordingly. Shall we start moving objects now, and after each move, we can inform each other of the new positions of the objects on our respective boards?"} \\
\\ 
\\ 
\texttt{What is your next command?} \\
            }
        }
    }
    & \\ \\

    \theutterance \stepcounter{utterance}  
    & \multicolumn{4}{p{0.6\linewidth}}{
        \cellcolor[rgb]{0.9,0.9,0.9}{
            \makecell[{{p{\linewidth}}}]{
                \texttt{\tiny{[P1$\rangle$GM]}}
                \texttt{SAY: Sounds like a plan! I've reviewed my board and identified the objects that need to be moved. I'll start by moving one of the objects that is farthest from its target position. After I make my move, I'll let you know the new position of the object on my board. Please do the same after you make your move, so we can keep track of the changes on both boards. To ensure we're on the same page, let's agree to report the moves in the format "moved <object> to (<x>, <y>)". I'm ready to make my first move.} \\
            }
        }
    }
    & & \\ \\

    \theutterance \stepcounter{utterance}  
    & & & \multicolumn{2}{p{0.3\linewidth}}{
        \cellcolor[rgb]{0.9,0.9,0.9}{
            \makecell[{{p{\linewidth}}}]{
                \texttt{\tiny{[GM$|$GM]}}
                \texttt{SAY: Sounds like a plan! I've reviewed my board and identified the objects that need to be moved. I'll start by moving one of the objects that is farthest from its target position. After I make my move, I'll let you know the new position of the object on my board. Please do the same after you make your move, so we can keep track of the changes on both boards. To ensure we're on the same page, let's agree to report the moves in the format "moved <object> to (<x>, <y>)". I'm ready to make my first move.} \\
            }
        }
    }
    & & \\ \\

    \theutterance \stepcounter{utterance}  
    & & \multicolumn{4}{p{0.6\linewidth}}{
        \cellcolor[rgb]{0.9,0.9,0.9}{
            \makecell[{{p{\linewidth}}}]{
                \texttt{\tiny{[GM$\rangle$P2]}}
                \texttt{Your message has been relayed to the other player.} \\
\\ 
\\ 
\texttt{You are currently playing round 5 of maximum 28.} \\
\texttt{You have collectively accumulated 1 of 16 penalties.} \\
\texttt{The other player sent this message:} \\
\\ 
\\ 
\texttt{"Sounds like a plan! I've reviewed my board and identified the objects that need to be moved. I'll start by moving one of the objects that is farthest from its target position. After I make my move, I'll let you know the new position of the object on my board. Please do the same after you make your move, so we can keep track of the changes on both boards. To ensure we're on the same page, let's agree to report the moves in the format "moved <object> to (<x>, <y>)". I'm ready to make my first move."} \\
\\ 
\\ 
\texttt{What is your next command?} \\
            }
        }
    }
    & \\ \\

    \theutterance \stepcounter{utterance}  
    & & & \multicolumn{4}{p{0.6\linewidth}}{
        \cellcolor[rgb]{0.9,0.9,0.9}{
            \makecell[{{p{\linewidth}}}]{
                \texttt{\tiny{[GM$\langle$P2]}}
                \texttt{SAY: I'm ready to make my first move as well. To confirm, we will report our moves in the format "moved <object> to (<x>, <y>)" and also inform each other of the new positions of the objects on our respective boards after each move. I'll make my move and then report it back to you. Let's get started!} \\
            }
        }
    }
     \\ \\

    \theutterance \stepcounter{utterance}  
    & & & \multicolumn{2}{p{0.3\linewidth}}{
        \cellcolor[rgb]{0.9,0.9,0.9}{
            \makecell[{{p{\linewidth}}}]{
                \texttt{\tiny{[GM$|$GM]}}
                \texttt{SAY: I'm ready to make my first move as well. To confirm, we will report our moves in the format "moved <object> to (<x>, <y>)" and also inform each other of the new positions of the objects on our respective boards after each move. I'll make my move and then report it back to you. Let's get started!} \\
            }
        }
    }
    & & \\ \\

    \theutterance \stepcounter{utterance}  
    & & \multicolumn{4}{p{0.6\linewidth}}{
        \cellcolor[rgb]{0.9,0.9,0.9}{
            \makecell[{{p{\linewidth}}}]{
                \texttt{\tiny{[P1$\langle$GM]}}
                \texttt{Your message has been relayed to the other player.} \\
\\ 
\\ 
\texttt{You are currently playing round 5 of maximum 28.} \\
\texttt{You have collectively accumulated 1 of 16 penalties.} \\
\texttt{The other player sent this message:} \\
\\ 
\\ 
\texttt{"I'm ready to make my first move as well. To confirm, we will report our moves in the format "moved <object> to (<x>, <y>)" and also inform each other of the new positions of the objects on our respective boards after each move. I'll make my move and then report it back to you. Let's get started!"} \\
\\ 
\\ 
\texttt{What is your next command?} \\
            }
        }
    }
    & \\ \\

    \theutterance \stepcounter{utterance}  
    & \multicolumn{4}{p{0.6\linewidth}}{
        \cellcolor[rgb]{0.9,0.9,0.9}{
            \makecell[{{p{\linewidth}}}]{
                \texttt{\tiny{[P1$\rangle$GM]}}
                \texttt{MOVE: A, (1, 6)} \\
            }
        }
    }
    & & \\ \\

    \theutterance \stepcounter{utterance}  
    & & & \multicolumn{2}{p{0.3\linewidth}}{
        \cellcolor[rgb]{0.9,0.9,0.9}{
            \makecell[{{p{\linewidth}}}]{
                \texttt{\tiny{[GM$|$GM]}}
                \texttt{MOVE: A, (1, 6)} \\
            }
        }
    }
    & & \\ \\

    \theutterance \stepcounter{utterance}  
    & & & \multicolumn{2}{p{0.3\linewidth}}{
        \cellcolor[rgb]{0.9,0.9,0.9}{
            \makecell[{{p{\linewidth}}}]{
                \texttt{\tiny{[GM$|$GM]}}
                \texttt{Moved 'A' to (1,6) successfully. Your updated grid looks like this:} \\
\texttt{```} \\
\texttt{ 1234567} \\
\texttt{╔═══╤═╤═╗} \\
\texttt{║◌◌◌│◌│◌║ 1} \\
\texttt{║◌◌M│◌├─╢ 2} \\
\texttt{║◌T◌├─┘◌║ 3} \\
\texttt{║◌◌P│◌◌◌║ 4} \\
\texttt{╟─┬─┼───╢ 5} \\
\texttt{║A│◌│S◌O║ 6} \\
\texttt{║◌│◌│◌D◌║ 7} \\
\texttt{╚═╧═╧═══╝} \\
\texttt{```} \\
            }
        }
    }
    & & \\ \\

    \theutterance \stepcounter{utterance}  
    & & \multicolumn{4}{p{0.6\linewidth}}{
        \cellcolor[rgb]{0.9,0.9,0.9}{
            \makecell[{{p{\linewidth}}}]{
                \texttt{\tiny{[GM$\rangle$P2]}}
                \texttt{Your message has been relayed to the other player.} \\
\\ 
\\ 
\texttt{You are currently playing round 6 of maximum 28.} \\
\texttt{You have collectively accumulated 1 of 16 penalties.} \\
\texttt{The other player moved an object on their grid.} \\
\\ 
\\ 
\texttt{What is your next command?} \\
            }
        }
    }
    & \\ \\

    \theutterance \stepcounter{utterance}  
    & & & \multicolumn{4}{p{0.6\linewidth}}{
        \cellcolor[rgb]{0.9,0.9,0.9}{
            \makecell[{{p{\linewidth}}}]{
                \texttt{\tiny{[GM$\langle$P2]}}
                \texttt{MOVE: T, (6,3)} \\
            }
        }
    }
     \\ \\

    \theutterance \stepcounter{utterance}  
    & & & \multicolumn{2}{p{0.3\linewidth}}{
        \cellcolor[rgb]{0.9,0.9,0.9}{
            \makecell[{{p{\linewidth}}}]{
                \texttt{\tiny{[GM$|$GM]}}
                \texttt{MOVE: T, (6,3)} \\
            }
        }
    }
    & & \\ \\

    \theutterance \stepcounter{utterance}  
    & & & \multicolumn{2}{p{0.3\linewidth}}{
        \cellcolor[rgb]{0.9,0.9,0.9}{
            \makecell[{{p{\linewidth}}}]{
                \texttt{\tiny{[GM$|$GM]}}
                \texttt{Penalty: (6,3) is not empty, but contains '┘'.} \\
\\ 
\texttt{You have collectively accumulated 2 of 16 penalties.} \\
\texttt{Please try again!} \\
            }
        }
    }
    & & \\ \\

    \theutterance \stepcounter{utterance}  
    & & \multicolumn{4}{p{0.6\linewidth}}{
        \cellcolor[rgb]{0.9,0.9,0.9}{
            \makecell[{{p{\linewidth}}}]{
                \texttt{\tiny{[GM$\rangle$P2]}}
                \texttt{Penalty: (6,3) is not empty, but contains '┘'.} \\
\\ 
\texttt{You have collectively accumulated 2 of 16 penalties.} \\
\texttt{Please try again!} \\
            }
        }
    }
    & \\ \\

    \theutterance \stepcounter{utterance}  
    & & & \multicolumn{4}{p{0.6\linewidth}}{
        \cellcolor[rgb]{0.9,0.9,0.9}{
            \makecell[{{p{\linewidth}}}]{
                \texttt{\tiny{[GM$\langle$P2]}}
                \texttt{SAY: moved T to (6,3) didn't work for me, my target (6,3) is occupied. Let me try to move a different object, or to a different location. Before I try again, could you please tell me which object you moved and where you moved it to? This might help me plan my next move.} \\
            }
        }
    }
     \\ \\

    \theutterance \stepcounter{utterance}  
    & & & \multicolumn{2}{p{0.3\linewidth}}{
        \cellcolor[rgb]{0.9,0.9,0.9}{
            \makecell[{{p{\linewidth}}}]{
                \texttt{\tiny{[GM$|$GM]}}
                \texttt{SAY: moved T to (6,3) didn't work for me, my target (6,3) is occupied. Let me try to move a different object, or to a different location. Before I try again, could you please tell me which object you moved and where you moved it to? This might help me plan my next move.} \\
            }
        }
    }
    & & \\ \\

    \theutterance \stepcounter{utterance}  
    & & \multicolumn{4}{p{0.6\linewidth}}{
        \cellcolor[rgb]{0.9,0.9,0.9}{
            \makecell[{{p{\linewidth}}}]{
                \texttt{\tiny{[P1$\langle$GM]}}
                \texttt{Moved 'A' to (1,6) successfully. Your updated grid looks like this:} \\
\texttt{```} \\
\texttt{ 1234567} \\
\texttt{╔═══╤═╤═╗} \\
\texttt{║◌◌◌│◌│◌║ 1} \\
\texttt{║◌◌M│◌├─╢ 2} \\
\texttt{║◌T◌├─┘◌║ 3} \\
\texttt{║◌◌P│◌◌◌║ 4} \\
\texttt{╟─┬─┼───╢ 5} \\
\texttt{║A│◌│S◌O║ 6} \\
\texttt{║◌│◌│◌D◌║ 7} \\
\texttt{╚═╧═╧═══╝} \\
\texttt{```} \\
\texttt{You are currently playing round 6 of maximum 28.} \\
\texttt{You have collectively accumulated 2 of 16 penalties.} \\
\texttt{The other player sent this message:} \\
\\ 
\\ 
\texttt{"moved T to (6,3) didn't work for me, my target (6,3) is occupied. Let me try to move a different object, or to a different location. Before I try again, could you please tell me which object you moved and where you moved it to? This might help me plan my next move."} \\
\\ 
\\ 
\texttt{What is your next command?} \\
            }
        }
    }
    & \\ \\

    \theutterance \stepcounter{utterance}  
    & \multicolumn{4}{p{0.6\linewidth}}{
        \cellcolor[rgb]{0.9,0.9,0.9}{
            \makecell[{{p{\linewidth}}}]{
                \texttt{\tiny{[P1$\rangle$GM]}}
                \texttt{SAY: I moved 'A' to (1,6). On my updated board, 'A' is now in its target position. The current state of my board is: 'P' is at (4,4), 'O' is at (6,6), 'T' is at (3,3), 'S' is at (6,5), 'D' is at (7,7), 'A' is at (1,6), and 'M' is at (3,2). Please let me know when you're ready to make your next move, and don't forget to report the new position of the object you move.} \\
            }
        }
    }
    & & \\ \\

    \theutterance \stepcounter{utterance}  
    & & & \multicolumn{2}{p{0.3\linewidth}}{
        \cellcolor[rgb]{0.9,0.9,0.9}{
            \makecell[{{p{\linewidth}}}]{
                \texttt{\tiny{[GM$|$GM]}}
                \texttt{SAY: I moved 'A' to (1,6). On my updated board, 'A' is now in its target position. The current state of my board is: 'P' is at (4,4), 'O' is at (6,6), 'T' is at (3,3), 'S' is at (6,5), 'D' is at (7,7), 'A' is at (1,6), and 'M' is at (3,2). Please let me know when you're ready to make your next move, and don't forget to report the new position of the object you move.} \\
            }
        }
    }
    & & \\ \\

    \theutterance \stepcounter{utterance}  
    & & \multicolumn{4}{p{0.6\linewidth}}{
        \cellcolor[rgb]{0.9,0.9,0.9}{
            \makecell[{{p{\linewidth}}}]{
                \texttt{\tiny{[GM$\rangle$P2]}}
                \texttt{Your message has been relayed to the other player.} \\
\\ 
\\ 
\texttt{You are currently playing round 7 of maximum 28.} \\
\texttt{You have collectively accumulated 2 of 16 penalties.} \\
\texttt{The other player sent this message:} \\
\\ 
\\ 
\texttt{"I moved 'A' to (1,6). On my updated board, 'A' is now in its target position. The current state of my board is: 'P' is at (4,4), 'O' is at (6,6), 'T' is at (3,3), 'S' is at (6,5), 'D' is at (7,7), 'A' is at (1,6), and 'M' is at (3,2). Please let me know when you're ready to make your next move, and don't forget to report the new position of the object you move."} \\
\\ 
\\ 
\texttt{What is your next command?} \\
            }
        }
    }
    & \\ \\

    \theutterance \stepcounter{utterance}  
    & & & \multicolumn{4}{p{0.6\linewidth}}{
        \cellcolor[rgb]{0.9,0.9,0.9}{
            \makecell[{{p{\linewidth}}}]{
                \texttt{\tiny{[GM$\langle$P2]}}
                \texttt{MOVE: A, (1,6)} \\
            }
        }
    }
     \\ \\

    \theutterance \stepcounter{utterance}  
    & & & \multicolumn{2}{p{0.3\linewidth}}{
        \cellcolor[rgb]{0.9,0.9,0.9}{
            \makecell[{{p{\linewidth}}}]{
                \texttt{\tiny{[GM$|$GM]}}
                \texttt{MOVE: A, (1,6)} \\
            }
        }
    }
    & & \\ \\

    \theutterance \stepcounter{utterance}  
    & & & \multicolumn{2}{p{0.3\linewidth}}{
        \cellcolor[rgb]{0.9,0.9,0.9}{
            \makecell[{{p{\linewidth}}}]{
                \texttt{\tiny{[GM$|$GM]}}
                \texttt{Moved 'A' to (1,6) successfully. Your updated grid looks like this:} \\
\texttt{```} \\
\texttt{ 1234567} \\
\texttt{╔═══╤═╤═╗} \\
\texttt{║◌MS│◌│◌║ 1} \\
\texttt{║◌O◌│◌├─╢ 2} \\
\texttt{║◌◌◌├─┘◌║ 3} \\
\texttt{║◌◌◌│◌◌◌║ 4} \\
\texttt{╟─┬─┼───╢ 5} \\
\texttt{║A│◌│◌◌D║ 6} \\
\texttt{║T│◌│◌P◌║ 7} \\
\texttt{╚═╧═╧═══╝} \\
\texttt{```} \\
            }
        }
    }
    & & \\ \\

    \theutterance \stepcounter{utterance}  
    & & \multicolumn{4}{p{0.6\linewidth}}{
        \cellcolor[rgb]{0.9,0.9,0.9}{
            \makecell[{{p{\linewidth}}}]{
                \texttt{\tiny{[P1$\langle$GM]}}
                \texttt{Your message has been relayed to the other player.} \\
\\ 
\\ 
\texttt{You are currently playing round 7 of maximum 28.} \\
\texttt{You have collectively accumulated 2 of 16 penalties.} \\
\texttt{The other player moved an object on their grid.} \\
\\ 
\\ 
\texttt{What is your next command?} \\
            }
        }
    }
    & \\ \\

    \theutterance \stepcounter{utterance}  
    & \multicolumn{4}{p{0.6\linewidth}}{
        \cellcolor[rgb]{0.9,0.9,0.9}{
            \makecell[{{p{\linewidth}}}]{
                \texttt{\tiny{[P1$\rangle$GM]}}
                \texttt{SAY: You mentioned that you tried to move 'T' to (6,3) but the target was occupied. You then moved a different object, but you didn't specify which one. Could you please tell me which object you moved and where you moved it to? This will help me understand the current state of your board and plan my next move accordingly.} \\
            }
        }
    }
    & & \\ \\

    \theutterance \stepcounter{utterance}  
    & & & \multicolumn{2}{p{0.3\linewidth}}{
        \cellcolor[rgb]{0.9,0.9,0.9}{
            \makecell[{{p{\linewidth}}}]{
                \texttt{\tiny{[GM$|$GM]}}
                \texttt{SAY: You mentioned that you tried to move 'T' to (6,3) but the target was occupied. You then moved a different object, but you didn't specify which one. Could you please tell me which object you moved and where you moved it to? This will help me understand the current state of your board and plan my next move accordingly.} \\
            }
        }
    }
    & & \\ \\

    \theutterance \stepcounter{utterance}  
    & & \multicolumn{4}{p{0.6\linewidth}}{
        \cellcolor[rgb]{0.9,0.9,0.9}{
            \makecell[{{p{\linewidth}}}]{
                \texttt{\tiny{[GM$\rangle$P2]}}
                \texttt{Moved 'A' to (1,6) successfully. Your updated grid looks like this:} \\
\texttt{```} \\
\texttt{ 1234567} \\
\texttt{╔═══╤═╤═╗} \\
\texttt{║◌MS│◌│◌║ 1} \\
\texttt{║◌O◌│◌├─╢ 2} \\
\texttt{║◌◌◌├─┘◌║ 3} \\
\texttt{║◌◌◌│◌◌◌║ 4} \\
\texttt{╟─┬─┼───╢ 5} \\
\texttt{║A│◌│◌◌D║ 6} \\
\texttt{║T│◌│◌P◌║ 7} \\
\texttt{╚═╧═╧═══╝} \\
\texttt{```} \\
\texttt{You are currently playing round 8 of maximum 28.} \\
\texttt{You have collectively accumulated 2 of 16 penalties.} \\
\texttt{The other player sent this message:} \\
\\ 
\\ 
\texttt{"You mentioned that you tried to move 'T' to (6,3) but the target was occupied. You then moved a different object, but you didn't specify which one. Could you please tell me which object you moved and where you moved it to? This will help me understand the current state of your board and plan my next move accordingly."} \\
\\ 
\\ 
\texttt{What is your next command?} \\
            }
        }
    }
    & \\ \\

    \theutterance \stepcounter{utterance}  
    & & & \multicolumn{4}{p{0.6\linewidth}}{
        \cellcolor[rgb]{0.9,0.9,0.9}{
            \makecell[{{p{\linewidth}}}]{
                \texttt{\tiny{[GM$\langle$P2]}}
                \texttt{SAY: I moved 'A' to (1,6). The current state of my board is: 'P' is at (6,7), 'O' is at (2,2), 'T' is at (1,7), 'S' is at (1,1), 'D' is at (7,6), 'A' is at (1,6), and 'M' is at (1,1). I'll make sure to report my moves clearly from now on. Please let me know your next move!} \\
            }
        }
    }
     \\ \\

    \theutterance \stepcounter{utterance}  
    & & & \multicolumn{2}{p{0.3\linewidth}}{
        \cellcolor[rgb]{0.9,0.9,0.9}{
            \makecell[{{p{\linewidth}}}]{
                \texttt{\tiny{[GM$|$GM]}}
                \texttt{SAY: I moved 'A' to (1,6). The current state of my board is: 'P' is at (6,7), 'O' is at (2,2), 'T' is at (1,7), 'S' is at (1,1), 'D' is at (7,6), 'A' is at (1,6), and 'M' is at (1,1). I'll make sure to report my moves clearly from now on. Please let me know your next move!} \\
            }
        }
    }
    & & \\ \\

    \theutterance \stepcounter{utterance}  
    & & \multicolumn{4}{p{0.6\linewidth}}{
        \cellcolor[rgb]{0.9,0.9,0.9}{
            \makecell[{{p{\linewidth}}}]{
                \texttt{\tiny{[P1$\langle$GM]}}
                \texttt{Your message has been relayed to the other player.} \\
\\ 
\\ 
\texttt{You are currently playing round 8 of maximum 28.} \\
\texttt{You have collectively accumulated 2 of 16 penalties.} \\
\texttt{The other player sent this message:} \\
\\ 
\\ 
\texttt{"I moved 'A' to (1,6). The current state of my board is: 'P' is at (6,7), 'O' is at (2,2), 'T' is at (1,7), 'S' is at (1,1), 'D' is at (7,6), 'A' is at (1,6), and 'M' is at (1,1). I'll make sure to report my moves clearly from now on. Please let me know your next move!"} \\
\\ 
\\ 
\texttt{What is your next command?} \\
            }
        }
    }
    & \\ \\

    \theutterance \stepcounter{utterance}  
    & \multicolumn{4}{p{0.6\linewidth}}{
        \cellcolor[rgb]{0.9,0.9,0.9}{
            \makecell[{{p{\linewidth}}}]{
                \texttt{\tiny{[P1$\rangle$GM]}}
                \texttt{SAY: Thanks for the update. It looks like we've both moved 'A' to (1,6), which is consistent with our goal configuration. For my next move, I'm considering moving 'P' to its target position. Before I do, I'd like to ask: have you made any progress in clearing the target positions for 'P', 'O', 'T', 'S', and 'D' on your board? Are there any objects that are currently occupying those positions?} \\
            }
        }
    }
    & & \\ \\

    \theutterance \stepcounter{utterance}  
    & & & \multicolumn{2}{p{0.3\linewidth}}{
        \cellcolor[rgb]{0.9,0.9,0.9}{
            \makecell[{{p{\linewidth}}}]{
                \texttt{\tiny{[GM$|$GM]}}
                \texttt{SAY: Thanks for the update. It looks like we've both moved 'A' to (1,6), which is consistent with our goal configuration. For my next move, I'm considering moving 'P' to its target position. Before I do, I'd like to ask: have you made any progress in clearing the target positions for 'P', 'O', 'T', 'S', and 'D' on your board? Are there any objects that are currently occupying those positions?} \\
            }
        }
    }
    & & \\ \\

    \theutterance \stepcounter{utterance}  
    & & \multicolumn{4}{p{0.6\linewidth}}{
        \cellcolor[rgb]{0.9,0.9,0.9}{
            \makecell[{{p{\linewidth}}}]{
                \texttt{\tiny{[GM$\rangle$P2]}}
                \texttt{Your message has been relayed to the other player.} \\
\\ 
\\ 
\texttt{You are currently playing round 9 of maximum 28.} \\
\texttt{You have collectively accumulated 2 of 16 penalties.} \\
\texttt{The other player sent this message:} \\
\\ 
\\ 
\texttt{"Thanks for the update. It looks like we've both moved 'A' to (1,6), which is consistent with our goal configuration. For my next move, I'm considering moving 'P' to its target position. Before I do, I'd like to ask: have you made any progress in clearing the target positions for 'P', 'O', 'T', 'S', and 'D' on your board? Are there any objects that are currently occupying those positions?"} \\
\\ 
\\ 
\texttt{What is your next command?} \\
            }
        }
    }
    & \\ \\

    \theutterance \stepcounter{utterance}  
    & & & \multicolumn{4}{p{0.6\linewidth}}{
        \cellcolor[rgb]{0.9,0.9,0.9}{
            \makecell[{{p{\linewidth}}}]{
                \texttt{\tiny{[GM$\langle$P2]}}
                \texttt{SAY: On my board, the target positions are currently occupied as follows: 'P' (3,1) is occupied by 'M' and 'S', 'O' (2,2) is occupied by 'O' (which is actually in its correct column, but not its correct row), 'T' (6,3) is occupied by '┘', 'S' (5,4) is empty, 'D' (4,5) is empty, and 'M' (7,7) is empty but 'M' is currently at (1,1). I'll try to clear these positions to make way for the correct objects. Let me know when you're ready to make your next move!} \\
            }
        }
    }
     \\ \\

    \theutterance \stepcounter{utterance}  
    & & & \multicolumn{2}{p{0.3\linewidth}}{
        \cellcolor[rgb]{0.9,0.9,0.9}{
            \makecell[{{p{\linewidth}}}]{
                \texttt{\tiny{[GM$|$GM]}}
                \texttt{SAY: On my board, the target positions are currently occupied as follows: 'P' (3,1) is occupied by 'M' and 'S', 'O' (2,2) is occupied by 'O' (which is actually in its correct column, but not its correct row), 'T' (6,3) is occupied by '┘', 'S' (5,4) is empty, 'D' (4,5) is empty, and 'M' (7,7) is empty but 'M' is currently at (1,1). I'll try to clear these positions to make way for the correct objects. Let me know when you're ready to make your next move!} \\
            }
        }
    }
    & & \\ \\

    \theutterance \stepcounter{utterance}  
    & & \multicolumn{4}{p{0.6\linewidth}}{
        \cellcolor[rgb]{0.9,0.9,0.9}{
            \makecell[{{p{\linewidth}}}]{
                \texttt{\tiny{[P1$\langle$GM]}}
                \texttt{Your message has been relayed to the other player.} \\
\\ 
\\ 
\texttt{You are currently playing round 9 of maximum 28.} \\
\texttt{You have collectively accumulated 2 of 16 penalties.} \\
\texttt{The other player sent this message:} \\
\\ 
\\ 
\texttt{"On my board, the target positions are currently occupied as follows: 'P' (3,1) is occupied by 'M' and 'S', 'O' (2,2) is occupied by 'O' (which is actually in its correct column, but not its correct row), 'T' (6,3) is occupied by '┘', 'S' (5,4) is empty, 'D' (4,5) is empty, and 'M' (7,7) is empty but 'M' is currently at (1,1). I'll try to clear these positions to make way for the correct objects. Let me know when you're ready to make your next move!"} \\
\\ 
\\ 
\texttt{What is your next command?} \\
            }
        }
    }
    & \\ \\

    \theutterance \stepcounter{utterance}  
    & \multicolumn{4}{p{0.6\linewidth}}{
        \cellcolor[rgb]{0.9,0.9,0.9}{
            \makecell[{{p{\linewidth}}}]{
                \texttt{\tiny{[P1$\rangle$GM]}}
                \texttt{MOVE: P, (3,1)} \\
            }
        }
    }
    & & \\ \\

    \theutterance \stepcounter{utterance}  
    & & & \multicolumn{2}{p{0.3\linewidth}}{
        \cellcolor[rgb]{0.9,0.9,0.9}{
            \makecell[{{p{\linewidth}}}]{
                \texttt{\tiny{[GM$|$GM]}}
                \texttt{MOVE: P, (3,1)} \\
            }
        }
    }
    & & \\ \\

    \theutterance \stepcounter{utterance}  
    & & & \multicolumn{2}{p{0.3\linewidth}}{
        \cellcolor[rgb]{0.9,0.9,0.9}{
            \makecell[{{p{\linewidth}}}]{
                \texttt{\tiny{[GM$|$GM]}}
                \texttt{Moved 'P' to (3,1) successfully. Your updated grid looks like this:} \\
\texttt{```} \\
\texttt{ 1234567} \\
\texttt{╔═══╤═╤═╗} \\
\texttt{║◌◌P│◌│◌║ 1} \\
\texttt{║◌◌M│◌├─╢ 2} \\
\texttt{║◌T◌├─┘◌║ 3} \\
\texttt{║◌◌◌│◌◌◌║ 4} \\
\texttt{╟─┬─┼───╢ 5} \\
\texttt{║A│◌│S◌O║ 6} \\
\texttt{║◌│◌│◌D◌║ 7} \\
\texttt{╚═╧═╧═══╝} \\
\texttt{```} \\
            }
        }
    }
    & & \\ \\

    \theutterance \stepcounter{utterance}  
    & & \multicolumn{4}{p{0.6\linewidth}}{
        \cellcolor[rgb]{0.9,0.9,0.9}{
            \makecell[{{p{\linewidth}}}]{
                \texttt{\tiny{[GM$\rangle$P2]}}
                \texttt{Your message has been relayed to the other player.} \\
\\ 
\\ 
\texttt{You are currently playing round 10 of maximum 28.} \\
\texttt{You have collectively accumulated 2 of 16 penalties.} \\
\texttt{The other player moved an object on their grid.} \\
\\ 
\\ 
\texttt{What is your next command?} \\
            }
        }
    }
    & \\ \\

    \theutterance \stepcounter{utterance}  
    & & & \multicolumn{4}{p{0.6\linewidth}}{
        \cellcolor[rgb]{0.9,0.9,0.9}{
            \makecell[{{p{\linewidth}}}]{
                \texttt{\tiny{[GM$\langle$P2]}}
                \texttt{MOVE: M, (7,7)} \\
            }
        }
    }
     \\ \\

    \theutterance \stepcounter{utterance}  
    & & & \multicolumn{2}{p{0.3\linewidth}}{
        \cellcolor[rgb]{0.9,0.9,0.9}{
            \makecell[{{p{\linewidth}}}]{
                \texttt{\tiny{[GM$|$GM]}}
                \texttt{MOVE: M, (7,7)} \\
            }
        }
    }
    & & \\ \\

    \theutterance \stepcounter{utterance}  
    & & & \multicolumn{2}{p{0.3\linewidth}}{
        \cellcolor[rgb]{0.9,0.9,0.9}{
            \makecell[{{p{\linewidth}}}]{
                \texttt{\tiny{[GM$|$GM]}}
                \texttt{Moved 'M' to (7,7) successfully. Your updated grid looks like this:} \\
\texttt{```} \\
\texttt{ 1234567} \\
\texttt{╔═══╤═╤═╗} \\
\texttt{║◌◌S│◌│◌║ 1} \\
\texttt{║◌O◌│◌├─╢ 2} \\
\texttt{║◌◌◌├─┘◌║ 3} \\
\texttt{║◌◌◌│◌◌◌║ 4} \\
\texttt{╟─┬─┼───╢ 5} \\
\texttt{║A│◌│◌◌D║ 6} \\
\texttt{║T│◌│◌PM║ 7} \\
\texttt{╚═╧═╧═══╝} \\
\texttt{```} \\
            }
        }
    }
    & & \\ \\

    \theutterance \stepcounter{utterance}  
    & & \multicolumn{4}{p{0.6\linewidth}}{
        \cellcolor[rgb]{0.9,0.9,0.9}{
            \makecell[{{p{\linewidth}}}]{
                \texttt{\tiny{[P1$\langle$GM]}}
                \texttt{Moved 'P' to (3,1) successfully. Your updated grid looks like this:} \\
\texttt{```} \\
\texttt{ 1234567} \\
\texttt{╔═══╤═╤═╗} \\
\texttt{║◌◌P│◌│◌║ 1} \\
\texttt{║◌◌M│◌├─╢ 2} \\
\texttt{║◌T◌├─┘◌║ 3} \\
\texttt{║◌◌◌│◌◌◌║ 4} \\
\texttt{╟─┬─┼───╢ 5} \\
\texttt{║A│◌│S◌O║ 6} \\
\texttt{║◌│◌│◌D◌║ 7} \\
\texttt{╚═╧═╧═══╝} \\
\texttt{```} \\
\texttt{You are currently playing round 10 of maximum 28.} \\
\texttt{You have collectively accumulated 2 of 16 penalties.} \\
\texttt{The other player moved an object on their grid.} \\
\\ 
\\ 
\texttt{What is your next command?} \\
            }
        }
    }
    & \\ \\

    \theutterance \stepcounter{utterance}  
    & \multicolumn{4}{p{0.6\linewidth}}{
        \cellcolor[rgb]{0.9,0.9,0.9}{
            \makecell[{{p{\linewidth}}}]{
                \texttt{\tiny{[P1$\rangle$GM]}}
                \texttt{SAY: I've moved 'P' to its target position (3,1). Please let me know which object you moved on your grid and where you moved it to. This will help me understand the current state of your board and plan my next move accordingly. Also, I'm curious to know if you've made any progress in clearing the target positions for 'O', 'T', 'S', 'D', and 'M' on your board.} \\
            }
        }
    }
    & & \\ \\

    \theutterance \stepcounter{utterance}  
    & & & \multicolumn{2}{p{0.3\linewidth}}{
        \cellcolor[rgb]{0.9,0.9,0.9}{
            \makecell[{{p{\linewidth}}}]{
                \texttt{\tiny{[GM$|$GM]}}
                \texttt{SAY: I've moved 'P' to its target position (3,1). Please let me know which object you moved on your grid and where you moved it to. This will help me understand the current state of your board and plan my next move accordingly. Also, I'm curious to know if you've made any progress in clearing the target positions for 'O', 'T', 'S', 'D', and 'M' on your board.} \\
            }
        }
    }
    & & \\ \\

    \theutterance \stepcounter{utterance}  
    & & \multicolumn{4}{p{0.6\linewidth}}{
        \cellcolor[rgb]{0.9,0.9,0.9}{
            \makecell[{{p{\linewidth}}}]{
                \texttt{\tiny{[GM$\rangle$P2]}}
                \texttt{Moved 'M' to (7,7) successfully. Your updated grid looks like this:} \\
\texttt{```} \\
\texttt{ 1234567} \\
\texttt{╔═══╤═╤═╗} \\
\texttt{║◌◌S│◌│◌║ 1} \\
\texttt{║◌O◌│◌├─╢ 2} \\
\texttt{║◌◌◌├─┘◌║ 3} \\
\texttt{║◌◌◌│◌◌◌║ 4} \\
\texttt{╟─┬─┼───╢ 5} \\
\texttt{║A│◌│◌◌D║ 6} \\
\texttt{║T│◌│◌PM║ 7} \\
\texttt{╚═╧═╧═══╝} \\
\texttt{```} \\
\texttt{You are currently playing round 11 of maximum 28.} \\
\texttt{You have collectively accumulated 2 of 16 penalties.} \\
\texttt{The other player sent this message:} \\
\\ 
\\ 
\texttt{"I've moved 'P' to its target position (3,1). Please let me know which object you moved on your grid and where you moved it to. This will help me understand the current state of your board and plan my next move accordingly. Also, I'm curious to know if you've made any progress in clearing the target positions for 'O', 'T', 'S', 'D', and 'M' on your board."} \\
\\ 
\\ 
\texttt{What is your next command?} \\
            }
        }
    }
    & \\ \\

    \theutterance \stepcounter{utterance}  
    & & & \multicolumn{4}{p{0.6\linewidth}}{
        \cellcolor[rgb]{0.9,0.9,0.9}{
            \makecell[{{p{\linewidth}}}]{
                \texttt{\tiny{[GM$\langle$P2]}}
                \texttt{SAY: I moved 'M' to (7,7). The current state of my board is: 'P' is at (6,7), 'O' is at (2,2), 'T' is at (1,7), 'S' is at (1,1), 'D' is at (7,6), 'A' is at (1,6), and 'M' is at (7,7). I've made some progress in clearing the target positions, but there's still work to be done. 'S' is currently occupying the target position for 'P', and 'T' is not in its correct row. I'll try to address these issues in my next moves. Could you please tell me more about the move you made, and what your board looks like now?} \\
            }
        }
    }
     \\ \\

    \theutterance \stepcounter{utterance}  
    & & & \multicolumn{2}{p{0.3\linewidth}}{
        \cellcolor[rgb]{0.9,0.9,0.9}{
            \makecell[{{p{\linewidth}}}]{
                \texttt{\tiny{[GM$|$GM]}}
                \texttt{SAY: I moved 'M' to (7,7). The current state of my board is: 'P' is at (6,7), 'O' is at (2,2), 'T' is at (1,7), 'S' is at (1,1), 'D' is at (7,6), 'A' is at (1,6), and 'M' is at (7,7). I've made some progress in clearing the target positions, but there's still work to be done. 'S' is currently occupying the target position for 'P', and 'T' is not in its correct row. I'll try to address these issues in my next moves. Could you please tell me more about the move you made, and what your board looks like now?} \\
            }
        }
    }
    & & \\ \\

    \theutterance \stepcounter{utterance}  
    & & \multicolumn{4}{p{0.6\linewidth}}{
        \cellcolor[rgb]{0.9,0.9,0.9}{
            \makecell[{{p{\linewidth}}}]{
                \texttt{\tiny{[P1$\langle$GM]}}
                \texttt{Your message has been relayed to the other player.} \\
\\ 
\\ 
\texttt{You are currently playing round 11 of maximum 28.} \\
\texttt{You have collectively accumulated 2 of 16 penalties.} \\
\texttt{The other player sent this message:} \\
\\ 
\\ 
\texttt{"I moved 'M' to (7,7). The current state of my board is: 'P' is at (6,7), 'O' is at (2,2), 'T' is at (1,7), 'S' is at (1,1), 'D' is at (7,6), 'A' is at (1,6), and 'M' is at (7,7). I've made some progress in clearing the target positions, but there's still work to be done. 'S' is currently occupying the target position for 'P', and 'T' is not in its correct row. I'll try to address these issues in my next moves. Could you please tell me more about the move you made, and what your board looks like now?"} \\
\\ 
\\ 
\texttt{What is your next command?} \\
            }
        }
    }
    & \\ \\

    \theutterance \stepcounter{utterance}  
    & \multicolumn{4}{p{0.6\linewidth}}{
        \cellcolor[rgb]{0.9,0.9,0.9}{
            \makecell[{{p{\linewidth}}}]{
                \texttt{\tiny{[P1$\rangle$GM]}}
                \texttt{SAY: I moved 'P' to (3,1), which is its target position. My current board state is: 'P' is at (3,1), 'O' is at (6,6), 'T' is at (3,3), 'S' is at (6,5), 'D' is at (7,7), 'A' is at (1,6), and 'M' is at (3,2). Now that 'P' is in its correct position, my next priority is to move 'O' to its target position (2,2), but I notice that 'O' is currently at (6,6) on my board, and the target position (2,2) is occupied by 'M' is not, but there is an object 'O' is actually at the correct column but wrong row on your board. I'd like to discuss our next moves to ensure we're not duplicating efforts or causing conflicts. Should we focus on moving 'O' and 'T' to their correct positions next?} \\
            }
        }
    }
    & & \\ \\

    \theutterance \stepcounter{utterance}  
    & & & \multicolumn{2}{p{0.3\linewidth}}{
        \cellcolor[rgb]{0.9,0.9,0.9}{
            \makecell[{{p{\linewidth}}}]{
                \texttt{\tiny{[GM$|$GM]}}
                \texttt{SAY: I moved 'P' to (3,1), which is its target position. My current board state is: 'P' is at (3,1), 'O' is at (6,6), 'T' is at (3,3), 'S' is at (6,5), 'D' is at (7,7), 'A' is at (1,6), and 'M' is at (3,2). Now that 'P' is in its correct position, my next priority is to move 'O' to its target position (2,2), but I notice that 'O' is currently at (6,6) on my board, and the target position (2,2) is occupied by 'M' is not, but there is an object 'O' is actually at the correct column but wrong row on your board. I'd like to discuss our next moves to ensure we're not duplicating efforts or causing conflicts. Should we focus on moving 'O' and 'T' to their correct positions next?} \\
            }
        }
    }
    & & \\ \\

    \theutterance \stepcounter{utterance}  
    & & \multicolumn{4}{p{0.6\linewidth}}{
        \cellcolor[rgb]{0.9,0.9,0.9}{
            \makecell[{{p{\linewidth}}}]{
                \texttt{\tiny{[GM$\rangle$P2]}}
                \texttt{Your message has been relayed to the other player.} \\
\\ 
\\ 
\texttt{You are currently playing round 12 of maximum 28.} \\
\texttt{You have collectively accumulated 2 of 16 penalties.} \\
\texttt{The other player sent this message:} \\
\\ 
\\ 
\texttt{"I moved 'P' to (3,1), which is its target position. My current board state is: 'P' is at (3,1), 'O' is at (6,6), 'T' is at (3,3), 'S' is at (6,5), 'D' is at (7,7), 'A' is at (1,6), and 'M' is at (3,2). Now that 'P' is in its correct position, my next priority is to move 'O' to its target position (2,2), but I notice that 'O' is currently at (6,6) on my board, and the target position (2,2) is occupied by 'M' is not, but there is an object 'O' is actually at the correct column but wrong row on your board. I'd like to discuss our next moves to ensure we're not duplicating efforts or causing conflicts. Should we focus on moving 'O' and 'T' to their correct positions next?"} \\
\\ 
\\ 
\texttt{What is your next command?} \\
            }
        }
    }
    & \\ \\

    \theutterance \stepcounter{utterance}  
    & & & \multicolumn{4}{p{0.6\linewidth}}{
        \cellcolor[rgb]{0.9,0.9,0.9}{
            \makecell[{{p{\linewidth}}}]{
                \texttt{\tiny{[GM$\langle$P2]}}
                \texttt{SAY: Yes, let's focus on moving 'O' and 'T' to their correct positions next. On my board, 'O' is currently at (2,2), which is its correct column but not its correct row. I can try to move it to (2,2) is not needed, but I can move 'O' to make space for other objects or move it to its target row. 'T' is at (1,7) on my board, and its target position is (6,3). However, as I mentioned earlier, (6,3) is occupied by '┘' on my board. We should discuss how to handle this obstacle. Perhaps we can move objects around to create a path for 'T' to reach its target position. What are your thoughts on this? Should we try to move 'O' first, and then focus on 'T'?} \\
            }
        }
    }
     \\ \\

    \theutterance \stepcounter{utterance}  
    & & & \multicolumn{2}{p{0.3\linewidth}}{
        \cellcolor[rgb]{0.9,0.9,0.9}{
            \makecell[{{p{\linewidth}}}]{
                \texttt{\tiny{[GM$|$GM]}}
                \texttt{SAY: Yes, let's focus on moving 'O' and 'T' to their correct positions next. On my board, 'O' is currently at (2,2), which is its correct column but not its correct row. I can try to move it to (2,2) is not needed, but I can move 'O' to make space for other objects or move it to its target row. 'T' is at (1,7) on my board, and its target position is (6,3). However, as I mentioned earlier, (6,3) is occupied by '┘' on my board. We should discuss how to handle this obstacle. Perhaps we can move objects around to create a path for 'T' to reach its target position. What are your thoughts on this? Should we try to move 'O' first, and then focus on 'T'?} \\
            }
        }
    }
    & & \\ \\

    \theutterance \stepcounter{utterance}  
    & & \multicolumn{4}{p{0.6\linewidth}}{
        \cellcolor[rgb]{0.9,0.9,0.9}{
            \makecell[{{p{\linewidth}}}]{
                \texttt{\tiny{[P1$\langle$GM]}}
                \texttt{Your message has been relayed to the other player.} \\
\\ 
\\ 
\texttt{You are currently playing round 12 of maximum 28.} \\
\texttt{You have collectively accumulated 2 of 16 penalties.} \\
\texttt{The other player sent this message:} \\
\\ 
\\ 
\texttt{"Yes, let's focus on moving 'O' and 'T' to their correct positions next. On my board, 'O' is currently at (2,2), which is its correct column but not its correct row. I can try to move it to (2,2) is not needed, but I can move 'O' to make space for other objects or move it to its target row. 'T' is at (1,7) on my board, and its target position is (6,3). However, as I mentioned earlier, (6,3) is occupied by '┘' on my board. We should discuss how to handle this obstacle. Perhaps we can move objects around to create a path for 'T' to reach its target position. What are your thoughts on this? Should we try to move 'O' first, and then focus on 'T'?"} \\
\\ 
\\ 
\texttt{What is your next command?} \\
            }
        }
    }
    & \\ \\

    \theutterance \stepcounter{utterance}  
    & \multicolumn{4}{p{0.6\linewidth}}{
        \cellcolor[rgb]{0.9,0.9,0.9}{
            \makecell[{{p{\linewidth}}}]{
                \texttt{\tiny{[P1$\rangle$GM]}}
                \texttt{SAY: I think it's a good idea to move 'O' first, as it's already in the correct column on both our boards. On my board, 'O' is at (6,6), and its target position is (2,2). I can try to move 'O' to a temporary position to free up space, and then you can try to move 'O' on your board to its target row. Once 'O' is in its correct position, we can focus on moving 'T' to its target position (6,3). Regarding the obstacle at (6,3) on your board, I think we should try to move the objects around '┘' to create a path for 'T'. Perhaps we can move 'S' or other objects to adjacent positions to make way for 'T'. Let's coordinate our moves to ensure we're creating a clear path for 'T'. Should we try to move 'O' on both our boards to their target rows, and then reassess the situation for 'T'?} \\
            }
        }
    }
    & & \\ \\

    \theutterance \stepcounter{utterance}  
    & & & \multicolumn{2}{p{0.3\linewidth}}{
        \cellcolor[rgb]{0.9,0.9,0.9}{
            \makecell[{{p{\linewidth}}}]{
                \texttt{\tiny{[GM$|$GM]}}
                \texttt{SAY: I think it's a good idea to move 'O' first, as it's already in the correct column on both our boards. On my board, 'O' is at (6,6), and its target position is (2,2). I can try to move 'O' to a temporary position to free up space, and then you can try to move 'O' on your board to its target row. Once 'O' is in its correct position, we can focus on moving 'T' to its target position (6,3). Regarding the obstacle at (6,3) on your board, I think we should try to move the objects around '┘' to create a path for 'T'. Perhaps we can move 'S' or other objects to adjacent positions to make way for 'T'. Let's coordinate our moves to ensure we're creating a clear path for 'T'. Should we try to move 'O' on both our boards to their target rows, and then reassess the situation for 'T'?} \\
            }
        }
    }
    & & \\ \\

    \theutterance \stepcounter{utterance}  
    & & \multicolumn{4}{p{0.6\linewidth}}{
        \cellcolor[rgb]{0.9,0.9,0.9}{
            \makecell[{{p{\linewidth}}}]{
                \texttt{\tiny{[GM$\rangle$P2]}}
                \texttt{Your message has been relayed to the other player.} \\
\\ 
\\ 
\texttt{You are currently playing round 13 of maximum 28.} \\
\texttt{You have collectively accumulated 2 of 16 penalties.} \\
\texttt{The other player sent this message:} \\
\\ 
\\ 
\texttt{"I think it's a good idea to move 'O' first, as it's already in the correct column on both our boards. On my board, 'O' is at (6,6), and its target position is (2,2). I can try to move 'O' to a temporary position to free up space, and then you can try to move 'O' on your board to its target row. Once 'O' is in its correct position, we can focus on moving 'T' to its target position (6,3). Regarding the obstacle at (6,3) on your board, I think we should try to move the objects around '┘' to create a path for 'T'. Perhaps we can move 'S' or other objects to adjacent positions to make way for 'T'. Let's coordinate our moves to ensure we're creating a clear path for 'T'. Should we try to move 'O' on both our boards to their target rows, and then reassess the situation for 'T'?"} \\
\\ 
\\ 
\texttt{What is your next command?} \\
            }
        }
    }
    & \\ \\

    \theutterance \stepcounter{utterance}  
    & & & \multicolumn{4}{p{0.6\linewidth}}{
        \cellcolor[rgb]{0.9,0.9,0.9}{
            \makecell[{{p{\linewidth}}}]{
                \texttt{\tiny{[GM$\langle$P2]}}
                \texttt{SAY: That sounds like a good plan. On my board, 'O' is already at (2,2), but it's not in its correct row. I'll try to move 'O' to its target row, which is row 2. Since 'O' is already in the correct column, I just need to move it up one row. I'll make the move and let you know the outcome. Please go ahead and try to move 'O' on your board to its target row as well. After we've both moved 'O', we can reassess the situation and plan our next moves for 'T'. Also, I'll take a closer look at the objects around '┘' on my board to see if I can create a path for 'T' as you suggested. Let me know how your move goes!} \\
            }
        }
    }
     \\ \\

    \theutterance \stepcounter{utterance}  
    & & & \multicolumn{2}{p{0.3\linewidth}}{
        \cellcolor[rgb]{0.9,0.9,0.9}{
            \makecell[{{p{\linewidth}}}]{
                \texttt{\tiny{[GM$|$GM]}}
                \texttt{SAY: That sounds like a good plan. On my board, 'O' is already at (2,2), but it's not in its correct row. I'll try to move 'O' to its target row, which is row 2. Since 'O' is already in the correct column, I just need to move it up one row. I'll make the move and let you know the outcome. Please go ahead and try to move 'O' on your board to its target row as well. After we've both moved 'O', we can reassess the situation and plan our next moves for 'T'. Also, I'll take a closer look at the objects around '┘' on my board to see if I can create a path for 'T' as you suggested. Let me know how your move goes!} \\
            }
        }
    }
    & & \\ \\

    \theutterance \stepcounter{utterance}  
    & & \multicolumn{4}{p{0.6\linewidth}}{
        \cellcolor[rgb]{0.9,0.9,0.9}{
            \makecell[{{p{\linewidth}}}]{
                \texttt{\tiny{[P1$\langle$GM]}}
                \texttt{Your message has been relayed to the other player.} \\
\\ 
\\ 
\texttt{You are currently playing round 13 of maximum 28.} \\
\texttt{You have collectively accumulated 2 of 16 penalties.} \\
\texttt{The other player sent this message:} \\
\\ 
\\ 
\texttt{"That sounds like a good plan. On my board, 'O' is already at (2,2), but it's not in its correct row. I'll try to move 'O' to its target row, which is row 2. Since 'O' is already in the correct column, I just need to move it up one row. I'll make the move and let you know the outcome. Please go ahead and try to move 'O' on your board to its target row as well. After we've both moved 'O', we can reassess the situation and plan our next moves for 'T'. Also, I'll take a closer look at the objects around '┘' on my board to see if I can create a path for 'T' as you suggested. Let me know how your move goes!"} \\
\\ 
\\ 
\texttt{What is your next command?} \\
            }
        }
    }
    & \\ \\

    \theutterance \stepcounter{utterance}  
    & \multicolumn{4}{p{0.6\linewidth}}{
        \cellcolor[rgb]{0.9,0.9,0.9}{
            \makecell[{{p{\linewidth}}}]{
                \texttt{\tiny{[P1$\rangle$GM]}}
                \texttt{MOVE: O, (2,2)} \\
            }
        }
    }
    & & \\ \\

    \theutterance \stepcounter{utterance}  
    & & & \multicolumn{2}{p{0.3\linewidth}}{
        \cellcolor[rgb]{0.9,0.9,0.9}{
            \makecell[{{p{\linewidth}}}]{
                \texttt{\tiny{[GM$|$GM]}}
                \texttt{MOVE: O, (2,2)} \\
            }
        }
    }
    & & \\ \\

    \theutterance \stepcounter{utterance}  
    & & & \multicolumn{2}{p{0.3\linewidth}}{
        \cellcolor[rgb]{0.9,0.9,0.9}{
            \makecell[{{p{\linewidth}}}]{
                \texttt{\tiny{[GM$|$GM]}}
                \texttt{Moved 'O' to (2,2) successfully. Your updated grid looks like this:} \\
\texttt{```} \\
\texttt{ 1234567} \\
\texttt{╔═══╤═╤═╗} \\
\texttt{║◌◌P│◌│◌║ 1} \\
\texttt{║◌OM│◌├─╢ 2} \\
\texttt{║◌T◌├─┘◌║ 3} \\
\texttt{║◌◌◌│◌◌◌║ 4} \\
\texttt{╟─┬─┼───╢ 5} \\
\texttt{║A│◌│S◌◌║ 6} \\
\texttt{║◌│◌│◌D◌║ 7} \\
\texttt{╚═╧═╧═══╝} \\
\texttt{```} \\
            }
        }
    }
    & & \\ \\

    \theutterance \stepcounter{utterance}  
    & & \multicolumn{4}{p{0.6\linewidth}}{
        \cellcolor[rgb]{0.9,0.9,0.9}{
            \makecell[{{p{\linewidth}}}]{
                \texttt{\tiny{[GM$\rangle$P2]}}
                \texttt{Your message has been relayed to the other player.} \\
\\ 
\\ 
\texttt{You are currently playing round 14 of maximum 28.} \\
\texttt{You have collectively accumulated 2 of 16 penalties.} \\
\texttt{The other player moved an object on their grid.} \\
\\ 
\\ 
\texttt{What is your next command?} \\
            }
        }
    }
    & \\ \\

    \theutterance \stepcounter{utterance}  
    & & & \multicolumn{4}{p{0.6\linewidth}}{
        \cellcolor[rgb]{0.9,0.9,0.9}{
            \makecell[{{p{\linewidth}}}]{
                \texttt{\tiny{[GM$\langle$P2]}}
                \texttt{MOVE: O, (2,2)} \\
            }
        }
    }
     \\ \\

    \theutterance \stepcounter{utterance}  
    & & & \multicolumn{2}{p{0.3\linewidth}}{
        \cellcolor[rgb]{0.9,0.9,0.9}{
            \makecell[{{p{\linewidth}}}]{
                \texttt{\tiny{[GM$|$GM]}}
                \texttt{MOVE: O, (2,2)} \\
            }
        }
    }
    & & \\ \\

    \theutterance \stepcounter{utterance}  
    & & & \multicolumn{2}{p{0.3\linewidth}}{
        \cellcolor[rgb]{0.9,0.9,0.9}{
            \makecell[{{p{\linewidth}}}]{
                \texttt{\tiny{[GM$|$GM]}}
                \texttt{Penalty: (2,2) is not empty, but contains 'O'.} \\
\\ 
\texttt{You have collectively accumulated 3 of 16 penalties.} \\
\texttt{Please try again!} \\
            }
        }
    }
    & & \\ \\

    \theutterance \stepcounter{utterance}  
    & & \multicolumn{4}{p{0.6\linewidth}}{
        \cellcolor[rgb]{0.9,0.9,0.9}{
            \makecell[{{p{\linewidth}}}]{
                \texttt{\tiny{[GM$\rangle$P2]}}
                \texttt{Penalty: (2,2) is not empty, but contains 'O'.} \\
\\ 
\texttt{You have collectively accumulated 3 of 16 penalties.} \\
\texttt{Please try again!} \\
            }
        }
    }
    & \\ \\

    \theutterance \stepcounter{utterance}  
    & & & \multicolumn{4}{p{0.6\linewidth}}{
        \cellcolor[rgb]{0.9,0.9,0.9}{
            \makecell[{{p{\linewidth}}}]{
                \texttt{\tiny{[GM$\langle$P2]}}
                \texttt{SAY: It seems I tried to move 'O' to a position where it already exists. Let me reassess the board and try a different move. Since 'O' is already at (2,2), I'll think of a different object to move. Perhaps I can move 'S' to create space for other objects or to make way for 'T'. I'll consider the current state of my board and plan a new move. Can you please remind me of the current state of your board and the moves you've made so far? This will help me coordinate my next move.} \\
            }
        }
    }
     \\ \\

    \theutterance \stepcounter{utterance}  
    & & & \multicolumn{2}{p{0.3\linewidth}}{
        \cellcolor[rgb]{0.9,0.9,0.9}{
            \makecell[{{p{\linewidth}}}]{
                \texttt{\tiny{[GM$|$GM]}}
                \texttt{SAY: It seems I tried to move 'O' to a position where it already exists. Let me reassess the board and try a different move. Since 'O' is already at (2,2), I'll think of a different object to move. Perhaps I can move 'S' to create space for other objects or to make way for 'T'. I'll consider the current state of my board and plan a new move. Can you please remind me of the current state of your board and the moves you've made so far? This will help me coordinate my next move.} \\
            }
        }
    }
    & & \\ \\

    \theutterance \stepcounter{utterance}  
    & & \multicolumn{4}{p{0.6\linewidth}}{
        \cellcolor[rgb]{0.9,0.9,0.9}{
            \makecell[{{p{\linewidth}}}]{
                \texttt{\tiny{[P1$\langle$GM]}}
                \texttt{Moved 'O' to (2,2) successfully. Your updated grid looks like this:} \\
\texttt{```} \\
\texttt{ 1234567} \\
\texttt{╔═══╤═╤═╗} \\
\texttt{║◌◌P│◌│◌║ 1} \\
\texttt{║◌OM│◌├─╢ 2} \\
\texttt{║◌T◌├─┘◌║ 3} \\
\texttt{║◌◌◌│◌◌◌║ 4} \\
\texttt{╟─┬─┼───╢ 5} \\
\texttt{║A│◌│S◌◌║ 6} \\
\texttt{║◌│◌│◌D◌║ 7} \\
\texttt{╚═╧═╧═══╝} \\
\texttt{```} \\
\texttt{You are currently playing round 14 of maximum 28.} \\
\texttt{You have collectively accumulated 3 of 16 penalties.} \\
\texttt{The other player sent this message:} \\
\\ 
\\ 
\texttt{"It seems I tried to move 'O' to a position where it already exists. Let me reassess the board and try a different move. Since 'O' is already at (2,2), I'll think of a different object to move. Perhaps I can move 'S' to create space for other objects or to make way for 'T'. I'll consider the current state of my board and plan a new move. Can you please remind me of the current state of your board and the moves you've made so far? This will help me coordinate my next move."} \\
\\ 
\\ 
\texttt{What is your next command?} \\
            }
        }
    }
    & \\ \\

    \theutterance \stepcounter{utterance}  
    & \multicolumn{4}{p{0.6\linewidth}}{
        \cellcolor[rgb]{0.9,0.9,0.9}{
            \makecell[{{p{\linewidth}}}]{
                \texttt{\tiny{[P1$\rangle$GM]}}
                \texttt{SAY: My current board state is: 'P' is at (3,1), 'O' is at (2,2), 'T' is at (3,3), 'S' is at (6,5), 'D' is at (7,7), 'A' is at (1,6), and 'M' is at (3,2). I've made the following moves so far: I moved 'A' to (1,6) and then moved 'P' to (3,1), and finally moved 'O' to (2,2). Let me know what you decide to move on your board, and we can continue to coordinate our moves to achieve our goal configuration. Also, please keep me updated on your progress with creating a path for 'T' to reach its target position (6,3).} \\
            }
        }
    }
    & & \\ \\

    \theutterance \stepcounter{utterance}  
    & & & \multicolumn{2}{p{0.3\linewidth}}{
        \cellcolor[rgb]{0.9,0.9,0.9}{
            \makecell[{{p{\linewidth}}}]{
                \texttt{\tiny{[GM$|$GM]}}
                \texttt{SAY: My current board state is: 'P' is at (3,1), 'O' is at (2,2), 'T' is at (3,3), 'S' is at (6,5), 'D' is at (7,7), 'A' is at (1,6), and 'M' is at (3,2). I've made the following moves so far: I moved 'A' to (1,6) and then moved 'P' to (3,1), and finally moved 'O' to (2,2). Let me know what you decide to move on your board, and we can continue to coordinate our moves to achieve our goal configuration. Also, please keep me updated on your progress with creating a path for 'T' to reach its target position (6,3).} \\
            }
        }
    }
    & & \\ \\

    \theutterance \stepcounter{utterance}  
    & & \multicolumn{4}{p{0.6\linewidth}}{
        \cellcolor[rgb]{0.9,0.9,0.9}{
            \makecell[{{p{\linewidth}}}]{
                \texttt{\tiny{[GM$\rangle$P2]}}
                \texttt{Your message has been relayed to the other player.} \\
\\ 
\\ 
\texttt{You are currently playing round 15 of maximum 28.} \\
\texttt{You have collectively accumulated 3 of 16 penalties.} \\
\texttt{The other player sent this message:} \\
\\ 
\\ 
\texttt{"My current board state is: 'P' is at (3,1), 'O' is at (2,2), 'T' is at (3,3), 'S' is at (6,5), 'D' is at (7,7), 'A' is at (1,6), and 'M' is at (3,2). I've made the following moves so far: I moved 'A' to (1,6) and then moved 'P' to (3,1), and finally moved 'O' to (2,2). Let me know what you decide to move on your board, and we can continue to coordinate our moves to achieve our goal configuration. Also, please keep me updated on your progress with creating a path for 'T' to reach its target position (6,3)."} \\
\\ 
\\ 
\texttt{What is your next command?} \\
            }
        }
    }
    & \\ \\

    \theutterance \stepcounter{utterance}  
    & & & \multicolumn{4}{p{0.6\linewidth}}{
        \cellcolor[rgb]{0.9,0.9,0.9}{
            \makecell[{{p{\linewidth}}}]{
                \texttt{\tiny{[GM$\langle$P2]}}
                \texttt{SAY: Thank you for the update. I'll review my board and plan my next move. Since 'O' is already in its correct position on my board, I'll focus on creating a path for 'T'. I notice that 'S' is currently at (1,1) on my board, and moving it could potentially create space for other objects. I'll consider moving 'S' to a different position to make way for 'T'. Before I make my move, I'd like to ask: have you considered moving 'T' on your board to its target position (6,3), or is there an obstacle on your board that's preventing you from doing so? Additionally, are there any objects on your board that you think I should be aware of when planning my next move?} \\
            }
        }
    }
     \\ \\

    \theutterance \stepcounter{utterance}  
    & & & \multicolumn{2}{p{0.3\linewidth}}{
        \cellcolor[rgb]{0.9,0.9,0.9}{
            \makecell[{{p{\linewidth}}}]{
                \texttt{\tiny{[GM$|$GM]}}
                \texttt{SAY: Thank you for the update. I'll review my board and plan my next move. Since 'O' is already in its correct position on my board, I'll focus on creating a path for 'T'. I notice that 'S' is currently at (1,1) on my board, and moving it could potentially create space for other objects. I'll consider moving 'S' to a different position to make way for 'T'. Before I make my move, I'd like to ask: have you considered moving 'T' on your board to its target position (6,3), or is there an obstacle on your board that's preventing you from doing so? Additionally, are there any objects on your board that you think I should be aware of when planning my next move?} \\
            }
        }
    }
    & & \\ \\

    \theutterance \stepcounter{utterance}  
    & & \multicolumn{4}{p{0.6\linewidth}}{
        \cellcolor[rgb]{0.9,0.9,0.9}{
            \makecell[{{p{\linewidth}}}]{
                \texttt{\tiny{[P1$\langle$GM]}}
                \texttt{Your message has been relayed to the other player.} \\
\\ 
\\ 
\texttt{You are currently playing round 15 of maximum 28.} \\
\texttt{You have collectively accumulated 3 of 16 penalties.} \\
\texttt{The other player sent this message:} \\
\\ 
\\ 
\texttt{"Thank you for the update. I'll review my board and plan my next move. Since 'O' is already in its correct position on my board, I'll focus on creating a path for 'T'. I notice that 'S' is currently at (1,1) on my board, and moving it could potentially create space for other objects. I'll consider moving 'S' to a different position to make way for 'T'. Before I make my move, I'd like to ask: have you considered moving 'T' on your board to its target position (6,3), or is there an obstacle on your board that's preventing you from doing so? Additionally, are there any objects on your board that you think I should be aware of when planning my next move?"} \\
\\ 
\\ 
\texttt{What is your next command?} \\
            }
        }
    }
    & \\ \\

    \theutterance \stepcounter{utterance}  
    & \multicolumn{4}{p{0.6\linewidth}}{
        \cellcolor[rgb]{0.9,0.9,0.9}{
            \makecell[{{p{\linewidth}}}]{
                \texttt{\tiny{[P1$\rangle$GM]}}
                \texttt{SAY: On my board, 'T' is currently at (3,3), and its target position is indeed (6,3). However, the target position (6,3) on my board is occupied by '└' (a frame piece), not an object that can be moved. This means that I won't be able to move 'T' to its target position until we can somehow remove or move the frame piece. As for objects on my board that you should be aware of, I think it's worth noting that 'S' is currently at (6,5) on my board, which is close to the target position for 'T'. If you're planning to move 'S' on your board, it might be helpful to coordinate our moves to ensure that we're not creating conflicts or duplicating efforts. Let me know your thoughts on how to proceed with moving 'T' and creating a path for it to reach its target position.} \\
            }
        }
    }
    & & \\ \\

    \theutterance \stepcounter{utterance}  
    & & & \multicolumn{2}{p{0.3\linewidth}}{
        \cellcolor[rgb]{0.9,0.9,0.9}{
            \makecell[{{p{\linewidth}}}]{
                \texttt{\tiny{[GM$|$GM]}}
                \texttt{SAY: On my board, 'T' is currently at (3,3), and its target position is indeed (6,3). However, the target position (6,3) on my board is occupied by '└' (a frame piece), not an object that can be moved. This means that I won't be able to move 'T' to its target position until we can somehow remove or move the frame piece. As for objects on my board that you should be aware of, I think it's worth noting that 'S' is currently at (6,5) on my board, which is close to the target position for 'T'. If you're planning to move 'S' on your board, it might be helpful to coordinate our moves to ensure that we're not creating conflicts or duplicating efforts. Let me know your thoughts on how to proceed with moving 'T' and creating a path for it to reach its target position.} \\
            }
        }
    }
    & & \\ \\

    \theutterance \stepcounter{utterance}  
    & & \multicolumn{4}{p{0.6\linewidth}}{
        \cellcolor[rgb]{0.9,0.9,0.9}{
            \makecell[{{p{\linewidth}}}]{
                \texttt{\tiny{[GM$\rangle$P2]}}
                \texttt{Your message has been relayed to the other player.} \\
\\ 
\\ 
\texttt{You are currently playing round 16 of maximum 28.} \\
\texttt{You have collectively accumulated 3 of 16 penalties.} \\
\texttt{The other player sent this message:} \\
\\ 
\\ 
\texttt{"On my board, 'T' is currently at (3,3), and its target position is indeed (6,3). However, the target position (6,3) on my board is occupied by '└' (a frame piece), not an object that can be moved. This means that I won't be able to move 'T' to its target position until we can somehow remove or move the frame piece. As for objects on my board that you should be aware of, I think it's worth noting that 'S' is currently at (6,5) on my board, which is close to the target position for 'T'. If you're planning to move 'S' on your board, it might be helpful to coordinate our moves to ensure that we're not creating conflicts or duplicating efforts. Let me know your thoughts on how to proceed with moving 'T' and creating a path for it to reach its target position."} \\
\\ 
\\ 
\texttt{What is your next command?} \\
            }
        }
    }
    & \\ \\

    \theutterance \stepcounter{utterance}  
    & & & \multicolumn{4}{p{0.6\linewidth}}{
        \cellcolor[rgb]{0.9,0.9,0.9}{
            \makecell[{{p{\linewidth}}}]{
                \texttt{\tiny{[GM$\langle$P2]}}
                \texttt{SAY: That's a good point about the frame piece on your board. Since we can't move the frame pieces, we'll have to work around them. On my board, I have a similar issue with the '┘' frame piece at (6,3), which is also blocking the target position for 'T'. Considering our goal configuration, I think we should focus on moving the objects that are currently occupying the target rows for 'T', rather than trying to move the frame pieces. Consider the objects that are currently in the target row for 'T' {-} can we move any of those objects to create space and allow 'T' to move into its target row? Let's review our boards and see if there are any objects that we can move to make way for 'T'. I'll take another look at my board and see what options I have.} \\
            }
        }
    }
     \\ \\

    \theutterance \stepcounter{utterance}  
    & & & \multicolumn{2}{p{0.3\linewidth}}{
        \cellcolor[rgb]{0.9,0.9,0.9}{
            \makecell[{{p{\linewidth}}}]{
                \texttt{\tiny{[GM$|$GM]}}
                \texttt{SAY: That's a good point about the frame piece on your board. Since we can't move the frame pieces, we'll have to work around them. On my board, I have a similar issue with the '┘' frame piece at (6,3), which is also blocking the target position for 'T'. Considering our goal configuration, I think we should focus on moving the objects that are currently occupying the target rows for 'T', rather than trying to move the frame pieces. Consider the objects that are currently in the target row for 'T' {-} can we move any of those objects to create space and allow 'T' to move into its target row? Let's review our boards and see if there are any objects that we can move to make way for 'T'. I'll take another look at my board and see what options I have.} \\
            }
        }
    }
    & & \\ \\

    \theutterance \stepcounter{utterance}  
    & & \multicolumn{4}{p{0.6\linewidth}}{
        \cellcolor[rgb]{0.9,0.9,0.9}{
            \makecell[{{p{\linewidth}}}]{
                \texttt{\tiny{[P1$\langle$GM]}}
                \texttt{Your message has been relayed to the other player.} \\
\\ 
\\ 
\texttt{You are currently playing round 16 of maximum 28.} \\
\texttt{You have collectively accumulated 3 of 16 penalties.} \\
\texttt{The other player sent this message:} \\
\\ 
\\ 
\texttt{"That's a good point about the frame piece on your board. Since we can't move the frame pieces, we'll have to work around them. On my board, I have a similar issue with the '┘' frame piece at (6,3), which is also blocking the target position for 'T'. Considering our goal configuration, I think we should focus on moving the objects that are currently occupying the target rows for 'T', rather than trying to move the frame pieces. Consider the objects that are currently in the target row for 'T' {-} can we move any of those objects to create space and allow 'T' to move into its target row? Let's review our boards and see if there are any objects that we can move to make way for 'T'. I'll take another look at my board and see what options I have."} \\
\\ 
\\ 
\texttt{What is your next command?} \\
            }
        }
    }
    & \\ \\

    \theutterance \stepcounter{utterance}  
    & \multicolumn{4}{p{0.6\linewidth}}{
        \cellcolor[rgb]{0.9,0.9,0.9}{
            \makecell[{{p{\linewidth}}}]{
                \texttt{\tiny{[P1$\rangle$GM]}}
                \texttt{SAY: That's a good approach. Let's review our boards and see if we can move any objects to create space for 'T'. On my board, the target row for 'T' is row 3, and currently, 'T' is at (3,3), but its target position is (6,3). However, as we discussed, the frame piece '└' is occupying the target position. But we can try to move 'T' to a different column in row 3, and then see if we can move other objects to create space. I notice that 'M' is currently at (3,2) on my board, which is close to the target row for 'T'. If I move 'M' down to row 3, it could potentially create space for 'T' to move into its target row. What do you think? Should we try to move objects like 'M' on our boards to create space for 'T'?} \\
            }
        }
    }
    & & \\ \\

    \theutterance \stepcounter{utterance}  
    & & & \multicolumn{2}{p{0.3\linewidth}}{
        \cellcolor[rgb]{0.9,0.9,0.9}{
            \makecell[{{p{\linewidth}}}]{
                \texttt{\tiny{[GM$|$GM]}}
                \texttt{SAY: That's a good approach. Let's review our boards and see if we can move any objects to create space for 'T'. On my board, the target row for 'T' is row 3, and currently, 'T' is at (3,3), but its target position is (6,3). However, as we discussed, the frame piece '└' is occupying the target position. But we can try to move 'T' to a different column in row 3, and then see if we can move other objects to create space. I notice that 'M' is currently at (3,2) on my board, which is close to the target row for 'T'. If I move 'M' down to row 3, it could potentially create space for 'T' to move into its target row. What do you think? Should we try to move objects like 'M' on our boards to create space for 'T'?} \\
            }
        }
    }
    & & \\ \\

    \theutterance \stepcounter{utterance}  
    & & \multicolumn{4}{p{0.6\linewidth}}{
        \cellcolor[rgb]{0.9,0.9,0.9}{
            \makecell[{{p{\linewidth}}}]{
                \texttt{\tiny{[GM$\rangle$P2]}}
                \texttt{Your message has been relayed to the other player.} \\
\\ 
\\ 
\texttt{You are currently playing round 17 of maximum 28.} \\
\texttt{You have collectively accumulated 3 of 16 penalties.} \\
\texttt{The other player sent this message:} \\
\\ 
\\ 
\texttt{"That's a good approach. Let's review our boards and see if we can move any objects to create space for 'T'. On my board, the target row for 'T' is row 3, and currently, 'T' is at (3,3), but its target position is (6,3). However, as we discussed, the frame piece '└' is occupying the target position. But we can try to move 'T' to a different column in row 3, and then see if we can move other objects to create space. I notice that 'M' is currently at (3,2) on my board, which is close to the target row for 'T'. If I move 'M' down to row 3, it could potentially create space for 'T' to move into its target row. What do you think? Should we try to move objects like 'M' on our boards to create space for 'T'?"} \\
\\ 
\\ 
\texttt{What is your next command?} \\
            }
        }
    }
    & \\ \\

    \theutterance \stepcounter{utterance}  
    & & & \multicolumn{4}{p{0.6\linewidth}}{
        \cellcolor[rgb]{0.9,0.9,0.9}{
            \makecell[{{p{\linewidth}}}]{
                \texttt{\tiny{[GM$\langle$P2]}}
                \texttt{SAY: That's a good idea to move 'M' down to row 3 on your board to create space for 'T'. On my board, 'M' is currently at (7,7), but I have an 'M' and an 'S' at (1,1). If I move 'S' from (1,1) to a different position, it could potentially create space for other objects to move. I'll consider moving 'S' to create a chain reaction that could ultimately help 'T' move into its target position. Before I make my move, I'd like to confirm: if you move 'M' down to row 3 on your board, will that create enough space for 'T' to move into its target row, or are there other objects that need to be moved as well?} \\
            }
        }
    }
     \\ \\

    \theutterance \stepcounter{utterance}  
    & & & \multicolumn{2}{p{0.3\linewidth}}{
        \cellcolor[rgb]{0.9,0.9,0.9}{
            \makecell[{{p{\linewidth}}}]{
                \texttt{\tiny{[GM$|$GM]}}
                \texttt{SAY: That's a good idea to move 'M' down to row 3 on your board to create space for 'T'. On my board, 'M' is currently at (7,7), but I have an 'M' and an 'S' at (1,1). If I move 'S' from (1,1) to a different position, it could potentially create space for other objects to move. I'll consider moving 'S' to create a chain reaction that could ultimately help 'T' move into its target position. Before I make my move, I'd like to confirm: if you move 'M' down to row 3 on your board, will that create enough space for 'T' to move into its target row, or are there other objects that need to be moved as well?} \\
            }
        }
    }
    & & \\ \\

    \theutterance \stepcounter{utterance}  
    & & \multicolumn{4}{p{0.6\linewidth}}{
        \cellcolor[rgb]{0.9,0.9,0.9}{
            \makecell[{{p{\linewidth}}}]{
                \texttt{\tiny{[P1$\langle$GM]}}
                \texttt{Your message has been relayed to the other player.} \\
\\ 
\\ 
\texttt{You are currently playing round 17 of maximum 28.} \\
\texttt{You have collectively accumulated 3 of 16 penalties.} \\
\texttt{The other player sent this message:} \\
\\ 
\\ 
\texttt{"That's a good idea to move 'M' down to row 3 on your board to create space for 'T'. On my board, 'M' is currently at (7,7), but I have an 'M' and an 'S' at (1,1). If I move 'S' from (1,1) to a different position, it could potentially create space for other objects to move. I'll consider moving 'S' to create a chain reaction that could ultimately help 'T' move into its target position. Before I make my move, I'd like to confirm: if you move 'M' down to row 3 on your board, will that create enough space for 'T' to move into its target row, or are there other objects that need to be moved as well?"} \\
\\ 
\\ 
\texttt{What is your next command?} \\
            }
        }
    }
    & \\ \\

    \theutterance \stepcounter{utterance}  
    & \multicolumn{4}{p{0.6\linewidth}}{
        \cellcolor[rgb]{0.9,0.9,0.9}{
            \makecell[{{p{\linewidth}}}]{
                \texttt{\tiny{[P1$\rangle$GM]}}
                \texttt{SAY: If I move 'M' down to row 3 on my board, it will create some space, but it's not a direct solution for 'T' to move into its target position (6,3). As I mentioned earlier, the frame piece '└' is occupying the target position, so we'll need to find another way to get 'T' into its target row. However, moving 'M' down to row 3 could be a useful step in creating a chain reaction that ultimately allows 'T' to move into its target row. I'll need to consider the next steps carefully and potentially move other objects to create a clear path for 'T'. It might be helpful if we could coordinate our moves to ensure that we're creating space and opportunities for each other's boards. What if we try to move 'S' on both our boards to create some space and see how that affects the overall situation?} \\
            }
        }
    }
    & & \\ \\

    \theutterance \stepcounter{utterance}  
    & & & \multicolumn{2}{p{0.3\linewidth}}{
        \cellcolor[rgb]{0.9,0.9,0.9}{
            \makecell[{{p{\linewidth}}}]{
                \texttt{\tiny{[GM$|$GM]}}
                \texttt{SAY: If I move 'M' down to row 3 on my board, it will create some space, but it's not a direct solution for 'T' to move into its target position (6,3). As I mentioned earlier, the frame piece '└' is occupying the target position, so we'll need to find another way to get 'T' into its target row. However, moving 'M' down to row 3 could be a useful step in creating a chain reaction that ultimately allows 'T' to move into its target row. I'll need to consider the next steps carefully and potentially move other objects to create a clear path for 'T'. It might be helpful if we could coordinate our moves to ensure that we're creating space and opportunities for each other's boards. What if we try to move 'S' on both our boards to create some space and see how that affects the overall situation?} \\
            }
        }
    }
    & & \\ \\

    \theutterance \stepcounter{utterance}  
    & & \multicolumn{4}{p{0.6\linewidth}}{
        \cellcolor[rgb]{0.9,0.9,0.9}{
            \makecell[{{p{\linewidth}}}]{
                \texttt{\tiny{[GM$\rangle$P2]}}
                \texttt{Your message has been relayed to the other player.} \\
\\ 
\\ 
\texttt{You are currently playing round 18 of maximum 28.} \\
\texttt{You have collectively accumulated 3 of 16 penalties.} \\
\texttt{The other player sent this message:} \\
\\ 
\\ 
\texttt{"If I move 'M' down to row 3 on my board, it will create some space, but it's not a direct solution for 'T' to move into its target position (6,3). As I mentioned earlier, the frame piece '└' is occupying the target position, so we'll need to find another way to get 'T' into its target row. However, moving 'M' down to row 3 could be a useful step in creating a chain reaction that ultimately allows 'T' to move into its target row. I'll need to consider the next steps carefully and potentially move other objects to create a clear path for 'T'. It might be helpful if we could coordinate our moves to ensure that we're creating space and opportunities for each other's boards. What if we try to move 'S' on both our boards to create some space and see how that affects the overall situation?"} \\
\\ 
\\ 
\texttt{What is your next command?} \\
            }
        }
    }
    & \\ \\

    \theutterance \stepcounter{utterance}  
    & & & \multicolumn{4}{p{0.6\linewidth}}{
        \cellcolor[rgb]{0.9,0.9,0.9}{
            \makecell[{{p{\linewidth}}}]{
                \texttt{\tiny{[GM$\langle$P2]}}
                \texttt{SAY: Moving 'S' on both our boards could be a good idea. On my board, 'S' is currently at (1,1), and moving it could create some space. However, I need to consider where to move 'S' to ensure that it's creating space for other objects and not blocking any potential paths. Since 'S' is supposed to end up in column 5, row 4, I could try to move it in that general direction. But before I make my move, I'd like to ask: where do you think you could move 'S' on your board to create space and potentially help 'T' move into its target row? Are there any specific columns or rows that you think would be beneficial to target?} \\
            }
        }
    }
     \\ \\

    \theutterance \stepcounter{utterance}  
    & & & \multicolumn{2}{p{0.3\linewidth}}{
        \cellcolor[rgb]{0.9,0.9,0.9}{
            \makecell[{{p{\linewidth}}}]{
                \texttt{\tiny{[GM$|$GM]}}
                \texttt{SAY: Moving 'S' on both our boards could be a good idea. On my board, 'S' is currently at (1,1), and moving it could create some space. However, I need to consider where to move 'S' to ensure that it's creating space for other objects and not blocking any potential paths. Since 'S' is supposed to end up in column 5, row 4, I could try to move it in that general direction. But before I make my move, I'd like to ask: where do you think you could move 'S' on your board to create space and potentially help 'T' move into its target row? Are there any specific columns or rows that you think would be beneficial to target?} \\
            }
        }
    }
    & & \\ \\

    \theutterance \stepcounter{utterance}  
    & & \multicolumn{4}{p{0.6\linewidth}}{
        \cellcolor[rgb]{0.9,0.9,0.9}{
            \makecell[{{p{\linewidth}}}]{
                \texttt{\tiny{[P1$\langle$GM]}}
                \texttt{Your message has been relayed to the other player.} \\
\\ 
\\ 
\texttt{You are currently playing round 18 of maximum 28.} \\
\texttt{You have collectively accumulated 3 of 16 penalties.} \\
\texttt{The other player sent this message:} \\
\\ 
\\ 
\texttt{"Moving 'S' on both our boards could be a good idea. On my board, 'S' is currently at (1,1), and moving it could create some space. However, I need to consider where to move 'S' to ensure that it's creating space for other objects and not blocking any potential paths. Since 'S' is supposed to end up in column 5, row 4, I could try to move it in that general direction. But before I make my move, I'd like to ask: where do you think you could move 'S' on your board to create space and potentially help 'T' move into its target row? Are there any specific columns or rows that you think would be beneficial to target?"} \\
\\ 
\\ 
\texttt{What is your next command?} \\
            }
        }
    }
    & \\ \\

    \theutterance \stepcounter{utterance}  
    & \multicolumn{4}{p{0.6\linewidth}}{
        \cellcolor[rgb]{0.9,0.9,0.9}{
            \makecell[{{p{\linewidth}}}]{
                \texttt{\tiny{[P1$\rangle$GM]}}
                \texttt{SAY: On my board, 'S' is currently at (6,5), which is close to its target position (5,4). If I move 'S' to its target position, it could create some space and potentially help 'T' move into its target row. Alternatively, I could try to move 'S' to a temporary position to create more space, but I'd need to be careful not to block any potential paths. Considering the layout of my board, I think it might be beneficial to move 'S' to column 5, either in row 4 or row 5, to create some space and flexibility for future moves. What if we both try to move 'S' to column 5 on our respective boards, and then reassess the situation to see how it affects our progress towards the goal configuration?} \\
            }
        }
    }
    & & \\ \\

    \theutterance \stepcounter{utterance}  
    & & & \multicolumn{2}{p{0.3\linewidth}}{
        \cellcolor[rgb]{0.9,0.9,0.9}{
            \makecell[{{p{\linewidth}}}]{
                \texttt{\tiny{[GM$|$GM]}}
                \texttt{SAY: On my board, 'S' is currently at (6,5), which is close to its target position (5,4). If I move 'S' to its target position, it could create some space and potentially help 'T' move into its target row. Alternatively, I could try to move 'S' to a temporary position to create more space, but I'd need to be careful not to block any potential paths. Considering the layout of my board, I think it might be beneficial to move 'S' to column 5, either in row 4 or row 5, to create some space and flexibility for future moves. What if we both try to move 'S' to column 5 on our respective boards, and then reassess the situation to see how it affects our progress towards the goal configuration?} \\
            }
        }
    }
    & & \\ \\

    \theutterance \stepcounter{utterance}  
    & & \multicolumn{4}{p{0.6\linewidth}}{
        \cellcolor[rgb]{0.9,0.9,0.9}{
            \makecell[{{p{\linewidth}}}]{
                \texttt{\tiny{[GM$\rangle$P2]}}
                \texttt{Your message has been relayed to the other player.} \\
\\ 
\\ 
\texttt{You are currently playing round 19 of maximum 28.} \\
\texttt{You have collectively accumulated 3 of 16 penalties.} \\
\texttt{The other player sent this message:} \\
\\ 
\\ 
\texttt{"On my board, 'S' is currently at (6,5), which is close to its target position (5,4). If I move 'S' to its target position, it could create some space and potentially help 'T' move into its target row. Alternatively, I could try to move 'S' to a temporary position to create more space, but I'd need to be careful not to block any potential paths. Considering the layout of my board, I think it might be beneficial to move 'S' to column 5, either in row 4 or row 5, to create some space and flexibility for future moves. What if we both try to move 'S' to column 5 on our respective boards, and then reassess the situation to see how it affects our progress towards the goal configuration?"} \\
\\ 
\\ 
\texttt{What is your next command?} \\
            }
        }
    }
    & \\ \\

    \theutterance \stepcounter{utterance}  
    & & & \multicolumn{4}{p{0.6\linewidth}}{
        \cellcolor[rgb]{0.9,0.9,0.9}{
            \makecell[{{p{\linewidth}}}]{
                \texttt{\tiny{[GM$\langle$P2]}}
                \texttt{SAY: Moving 'S' to column 5 on both our boards could be a good strategy. On my board, 'S' is currently at (1,1), so I'll try to move it to column 5. Since 'S' is supposed to end up in row 4, I'll aim to move it to (5,1) for now, and then potentially move it to row 4 later. I'll make the move and let you know the outcome. Please go ahead and try to move 'S' to column 5 on your board as well, either to its target position (5,4) or to a temporary position in row 5. After we've both made our moves, we can reassess the situation and plan our next steps. Let me know how your move goes!} \\
            }
        }
    }
     \\ \\

    \theutterance \stepcounter{utterance}  
    & & & \multicolumn{2}{p{0.3\linewidth}}{
        \cellcolor[rgb]{0.9,0.9,0.9}{
            \makecell[{{p{\linewidth}}}]{
                \texttt{\tiny{[GM$|$GM]}}
                \texttt{SAY: Moving 'S' to column 5 on both our boards could be a good strategy. On my board, 'S' is currently at (1,1), so I'll try to move it to column 5. Since 'S' is supposed to end up in row 4, I'll aim to move it to (5,1) for now, and then potentially move it to row 4 later. I'll make the move and let you know the outcome. Please go ahead and try to move 'S' to column 5 on your board as well, either to its target position (5,4) or to a temporary position in row 5. After we've both made our moves, we can reassess the situation and plan our next steps. Let me know how your move goes!} \\
            }
        }
    }
    & & \\ \\

    \theutterance \stepcounter{utterance}  
    & & \multicolumn{4}{p{0.6\linewidth}}{
        \cellcolor[rgb]{0.9,0.9,0.9}{
            \makecell[{{p{\linewidth}}}]{
                \texttt{\tiny{[P1$\langle$GM]}}
                \texttt{Your message has been relayed to the other player.} \\
\\ 
\\ 
\texttt{You are currently playing round 19 of maximum 28.} \\
\texttt{You have collectively accumulated 3 of 16 penalties.} \\
\texttt{The other player sent this message:} \\
\\ 
\\ 
\texttt{"Moving 'S' to column 5 on both our boards could be a good strategy. On my board, 'S' is currently at (1,1), so I'll try to move it to column 5. Since 'S' is supposed to end up in row 4, I'll aim to move it to (5,1) for now, and then potentially move it to row 4 later. I'll make the move and let you know the outcome. Please go ahead and try to move 'S' to column 5 on your board as well, either to its target position (5,4) or to a temporary position in row 5. After we've both made our moves, we can reassess the situation and plan our next steps. Let me know how your move goes!"} \\
\\ 
\\ 
\texttt{What is your next command?} \\
            }
        }
    }
    & \\ \\

    \theutterance \stepcounter{utterance}  
    & \multicolumn{4}{p{0.6\linewidth}}{
        \cellcolor[rgb]{0.9,0.9,0.9}{
            \makecell[{{p{\linewidth}}}]{
                \texttt{\tiny{[P1$\rangle$GM]}}
                \texttt{MOVE: S, (5,4)} \\
            }
        }
    }
    & & \\ \\

    \theutterance \stepcounter{utterance}  
    & & & \multicolumn{2}{p{0.3\linewidth}}{
        \cellcolor[rgb]{0.9,0.9,0.9}{
            \makecell[{{p{\linewidth}}}]{
                \texttt{\tiny{[GM$|$GM]}}
                \texttt{MOVE: S, (5,4)} \\
            }
        }
    }
    & & \\ \\

    \theutterance \stepcounter{utterance}  
    & & & \multicolumn{2}{p{0.3\linewidth}}{
        \cellcolor[rgb]{0.9,0.9,0.9}{
            \makecell[{{p{\linewidth}}}]{
                \texttt{\tiny{[GM$|$GM]}}
                \texttt{Moved 'S' to (5,4) successfully. Your updated grid looks like this:} \\
\texttt{```} \\
\texttt{ 1234567} \\
\texttt{╔═══╤═╤═╗} \\
\texttt{║◌◌P│◌│◌║ 1} \\
\texttt{║◌OM│◌├─╢ 2} \\
\texttt{║◌T◌├─┘◌║ 3} \\
\texttt{║◌◌◌│S◌◌║ 4} \\
\texttt{╟─┬─┼───╢ 5} \\
\texttt{║A│◌│◌◌◌║ 6} \\
\texttt{║◌│◌│◌D◌║ 7} \\
\texttt{╚═╧═╧═══╝} \\
\texttt{```} \\
            }
        }
    }
    & & \\ \\

    \theutterance \stepcounter{utterance}  
    & & \multicolumn{4}{p{0.6\linewidth}}{
        \cellcolor[rgb]{0.9,0.9,0.9}{
            \makecell[{{p{\linewidth}}}]{
                \texttt{\tiny{[GM$\rangle$P2]}}
                \texttt{Your message has been relayed to the other player.} \\
\\ 
\\ 
\texttt{You are currently playing round 20 of maximum 28.} \\
\texttt{You have collectively accumulated 3 of 16 penalties.} \\
\texttt{The other player moved an object on their grid.} \\
\\ 
\\ 
\texttt{What is your next command?} \\
            }
        }
    }
    & \\ \\

    \theutterance \stepcounter{utterance}  
    & & & \multicolumn{4}{p{0.6\linewidth}}{
        \cellcolor[rgb]{0.9,0.9,0.9}{
            \makecell[{{p{\linewidth}}}]{
                \texttt{\tiny{[GM$\langle$P2]}}
                \texttt{MOVE: S, (5,1)} \\
            }
        }
    }
     \\ \\

    \theutterance \stepcounter{utterance}  
    & & & \multicolumn{2}{p{0.3\linewidth}}{
        \cellcolor[rgb]{0.9,0.9,0.9}{
            \makecell[{{p{\linewidth}}}]{
                \texttt{\tiny{[GM$|$GM]}}
                \texttt{MOVE: S, (5,1)} \\
            }
        }
    }
    & & \\ \\

    \theutterance \stepcounter{utterance}  
    & & & \multicolumn{2}{p{0.3\linewidth}}{
        \cellcolor[rgb]{0.9,0.9,0.9}{
            \makecell[{{p{\linewidth}}}]{
                \texttt{\tiny{[GM$|$GM]}}
                \texttt{Moved 'S' to (5,1) successfully. Your updated grid looks like this:} \\
\texttt{```} \\
\texttt{ 1234567} \\
\texttt{╔═══╤═╤═╗} \\
\texttt{║◌◌◌│S│◌║ 1} \\
\texttt{║◌O◌│◌├─╢ 2} \\
\texttt{║◌◌◌├─┘◌║ 3} \\
\texttt{║◌◌◌│◌◌◌║ 4} \\
\texttt{╟─┬─┼───╢ 5} \\
\texttt{║A│◌│◌◌D║ 6} \\
\texttt{║T│◌│◌PM║ 7} \\
\texttt{╚═╧═╧═══╝} \\
\texttt{```} \\
            }
        }
    }
    & & \\ \\

    \theutterance \stepcounter{utterance}  
    & & \multicolumn{4}{p{0.6\linewidth}}{
        \cellcolor[rgb]{0.9,0.9,0.9}{
            \makecell[{{p{\linewidth}}}]{
                \texttt{\tiny{[P1$\langle$GM]}}
                \texttt{Moved 'S' to (5,4) successfully. Your updated grid looks like this:} \\
\texttt{```} \\
\texttt{ 1234567} \\
\texttt{╔═══╤═╤═╗} \\
\texttt{║◌◌P│◌│◌║ 1} \\
\texttt{║◌OM│◌├─╢ 2} \\
\texttt{║◌T◌├─┘◌║ 3} \\
\texttt{║◌◌◌│S◌◌║ 4} \\
\texttt{╟─┬─┼───╢ 5} \\
\texttt{║A│◌│◌◌◌║ 6} \\
\texttt{║◌│◌│◌D◌║ 7} \\
\texttt{╚═╧═╧═══╝} \\
\texttt{```} \\
\texttt{You are currently playing round 20 of maximum 28.} \\
\texttt{You have collectively accumulated 3 of 16 penalties.} \\
\texttt{The other player moved an object on their grid.} \\
\\ 
\\ 
\texttt{What is your next command?} \\
            }
        }
    }
    & \\ \\

    \theutterance \stepcounter{utterance}  
    & \multicolumn{4}{p{0.6\linewidth}}{
        \cellcolor[rgb]{0.9,0.9,0.9}{
            \makecell[{{p{\linewidth}}}]{
                \texttt{\tiny{[P1$\rangle$GM]}}
                \texttt{SAY: I've moved 'S' to its target position (5,4). Please let me know which object you moved on your grid and where you moved it to. Also, please confirm whether 'S' is now in column 5 on your grid as we planned. This will help me understand the current state of your board and plan my next move accordingly. Additionally, now that 'S' is in its target position on my board, I'll start thinking about how to move 'T' to its target position (6,3). If you have any suggestions or ideas, feel free to share!} \\
            }
        }
    }
    & & \\ \\

    \theutterance \stepcounter{utterance}  
    & & & \multicolumn{2}{p{0.3\linewidth}}{
        \cellcolor[rgb]{0.9,0.9,0.9}{
            \makecell[{{p{\linewidth}}}]{
                \texttt{\tiny{[GM$|$GM]}}
                \texttt{SAY: I've moved 'S' to its target position (5,4). Please let me know which object you moved on your grid and where you moved it to. Also, please confirm whether 'S' is now in column 5 on your grid as we planned. This will help me understand the current state of your board and plan my next move accordingly. Additionally, now that 'S' is in its target position on my board, I'll start thinking about how to move 'T' to its target position (6,3). If you have any suggestions or ideas, feel free to share!} \\
            }
        }
    }
    & & \\ \\

    \theutterance \stepcounter{utterance}  
    & & \multicolumn{4}{p{0.6\linewidth}}{
        \cellcolor[rgb]{0.9,0.9,0.9}{
            \makecell[{{p{\linewidth}}}]{
                \texttt{\tiny{[GM$\rangle$P2]}}
                \texttt{Moved 'S' to (5,1) successfully. Your updated grid looks like this:} \\
\texttt{```} \\
\texttt{ 1234567} \\
\texttt{╔═══╤═╤═╗} \\
\texttt{║◌◌◌│S│◌║ 1} \\
\texttt{║◌O◌│◌├─╢ 2} \\
\texttt{║◌◌◌├─┘◌║ 3} \\
\texttt{║◌◌◌│◌◌◌║ 4} \\
\texttt{╟─┬─┼───╢ 5} \\
\texttt{║A│◌│◌◌D║ 6} \\
\texttt{║T│◌│◌PM║ 7} \\
\texttt{╚═╧═╧═══╝} \\
\texttt{```} \\
\texttt{You are currently playing round 21 of maximum 28.} \\
\texttt{You have collectively accumulated 3 of 16 penalties.} \\
\texttt{The other player sent this message:} \\
\\ 
\\ 
\texttt{"I've moved 'S' to its target position (5,4). Please let me know which object you moved on your grid and where you moved it to. Also, please confirm whether 'S' is now in column 5 on your grid as we planned. This will help me understand the current state of your board and plan my next move accordingly. Additionally, now that 'S' is in its target position on my board, I'll start thinking about how to move 'T' to its target position (6,3). If you have any suggestions or ideas, feel free to share!"} \\
\\ 
\\ 
\texttt{What is your next command?} \\
            }
        }
    }
    & \\ \\

    \theutterance \stepcounter{utterance}  
    & & & \multicolumn{4}{p{0.6\linewidth}}{
        \cellcolor[rgb]{0.9,0.9,0.9}{
            \makecell[{{p{\linewidth}}}]{
                \texttt{\tiny{[GM$\langle$P2]}}
                \texttt{SAY: I moved 'S' to (5,1) on my grid, which puts it in column 5 as we planned. I'll need to move it to row 4 eventually, but for now, it's in the correct column. Congratulations on moving 'S' to its target position (5,4) on your board! That's a big step towards our goal configuration. As for moving 'T' to its target position (6,3), I think we should continue to focus on creating space and clearing a path for 'T'. On my board, I'll look for opportunities to move objects out of the way and create a chain reaction that could help 'T' move into its target row. Do you have any specific ideas for how to move 'T' on your board, or would you like to coordinate our next moves to ensure we're working together effectively?} \\
            }
        }
    }
     \\ \\

    \theutterance \stepcounter{utterance}  
    & & & \multicolumn{2}{p{0.3\linewidth}}{
        \cellcolor[rgb]{0.9,0.9,0.9}{
            \makecell[{{p{\linewidth}}}]{
                \texttt{\tiny{[GM$|$GM]}}
                \texttt{SAY: I moved 'S' to (5,1) on my grid, which puts it in column 5 as we planned. I'll need to move it to row 4 eventually, but for now, it's in the correct column. Congratulations on moving 'S' to its target position (5,4) on your board! That's a big step towards our goal configuration. As for moving 'T' to its target position (6,3), I think we should continue to focus on creating space and clearing a path for 'T'. On my board, I'll look for opportunities to move objects out of the way and create a chain reaction that could help 'T' move into its target row. Do you have any specific ideas for how to move 'T' on your board, or would you like to coordinate our next moves to ensure we're working together effectively?} \\
            }
        }
    }
    & & \\ \\

    \theutterance \stepcounter{utterance}  
    & & \multicolumn{4}{p{0.6\linewidth}}{
        \cellcolor[rgb]{0.9,0.9,0.9}{
            \makecell[{{p{\linewidth}}}]{
                \texttt{\tiny{[P1$\langle$GM]}}
                \texttt{Your message has been relayed to the other player.} \\
\\ 
\\ 
\texttt{You are currently playing round 21 of maximum 28.} \\
\texttt{You have collectively accumulated 3 of 16 penalties.} \\
\texttt{The other player sent this message:} \\
\\ 
\\ 
\texttt{"I moved 'S' to (5,1) on my grid, which puts it in column 5 as we planned. I'll need to move it to row 4 eventually, but for now, it's in the correct column. Congratulations on moving 'S' to its target position (5,4) on your board! That's a big step towards our goal configuration. As for moving 'T' to its target position (6,3), I think we should continue to focus on creating space and clearing a path for 'T'. On my board, I'll look for opportunities to move objects out of the way and create a chain reaction that could help 'T' move into its target row. Do you have any specific ideas for how to move 'T' on your board, or would you like to coordinate our next moves to ensure we're working together effectively?"} \\
\\ 
\\ 
\texttt{What is your next command?} \\
            }
        }
    }
    & \\ \\

    \theutterance \stepcounter{utterance}  
    & \multicolumn{4}{p{0.6\linewidth}}{
        \cellcolor[rgb]{0.9,0.9,0.9}{
            \makecell[{{p{\linewidth}}}]{
                \texttt{\tiny{[P1$\rangle$GM]}}
                \texttt{SAY: Thanks for the update! I'm glad we were able to move 'S' to column 5 on both our boards. Now, let's focus on moving 'T' to its target position (6,3). On my board, I think I can try to move 'T' to a different column in row 3, to get it closer to its target position. However, as we previously discussed, the frame piece '└' is occupying the target position, so we'll need to find a way to work around it. I'd like to propose that we try to move 'D' on both our boards to create some space and potentially clear a path for 'T'. Since 'D' is supposed to end up in row 5, column 4, maybe we can try to move it to a position that will ultimately help 'T' move into its target row. What do you think? Should we try to move 'D' on both our boards and see how it affects the overall situation?} \\
            }
        }
    }
    & & \\ \\

    \theutterance \stepcounter{utterance}  
    & & & \multicolumn{2}{p{0.3\linewidth}}{
        \cellcolor[rgb]{0.9,0.9,0.9}{
            \makecell[{{p{\linewidth}}}]{
                \texttt{\tiny{[GM$|$GM]}}
                \texttt{SAY: Thanks for the update! I'm glad we were able to move 'S' to column 5 on both our boards. Now, let's focus on moving 'T' to its target position (6,3). On my board, I think I can try to move 'T' to a different column in row 3, to get it closer to its target position. However, as we previously discussed, the frame piece '└' is occupying the target position, so we'll need to find a way to work around it. I'd like to propose that we try to move 'D' on both our boards to create some space and potentially clear a path for 'T'. Since 'D' is supposed to end up in row 5, column 4, maybe we can try to move it to a position that will ultimately help 'T' move into its target row. What do you think? Should we try to move 'D' on both our boards and see how it affects the overall situation?} \\
            }
        }
    }
    & & \\ \\

    \theutterance \stepcounter{utterance}  
    & & \multicolumn{4}{p{0.6\linewidth}}{
        \cellcolor[rgb]{0.9,0.9,0.9}{
            \makecell[{{p{\linewidth}}}]{
                \texttt{\tiny{[GM$\rangle$P2]}}
                \texttt{Your message has been relayed to the other player.} \\
\\ 
\\ 
\texttt{You are currently playing round 22 of maximum 28.} \\
\texttt{You have collectively accumulated 3 of 16 penalties.} \\
\texttt{The other player sent this message:} \\
\\ 
\\ 
\texttt{"Thanks for the update! I'm glad we were able to move 'S' to column 5 on both our boards. Now, let's focus on moving 'T' to its target position (6,3). On my board, I think I can try to move 'T' to a different column in row 3, to get it closer to its target position. However, as we previously discussed, the frame piece '└' is occupying the target position, so we'll need to find a way to work around it. I'd like to propose that we try to move 'D' on both our boards to create some space and potentially clear a path for 'T'. Since 'D' is supposed to end up in row 5, column 4, maybe we can try to move it to a position that will ultimately help 'T' move into its target row. What do you think? Should we try to move 'D' on both our boards and see how it affects the overall situation?"} \\
\\ 
\\ 
\texttt{What is your next command?} \\
            }
        }
    }
    & \\ \\

    \theutterance \stepcounter{utterance}  
    & & & \multicolumn{4}{p{0.6\linewidth}}{
        \cellcolor[rgb]{0.9,0.9,0.9}{
            \makecell[{{p{\linewidth}}}]{
                \texttt{\tiny{[GM$\langle$P2]}}
                \texttt{SAY: Moving 'D' on both our boards could be a good idea. On my board, 'D' is currently at (7,6), so I'll try to move it to a position that will help create space and potentially clear a path for 'T'. Since 'D' is supposed to end up in row 5, column 4, I'll aim to move it to a position that will get it closer to its target position. Maybe I can move 'D' to column 4, either in row 5 or a row above it, to create some space and flexibility for future moves. Please go ahead and try to move 'D' on your board as well, and let's see how it affects the overall situation. After we've both made our moves, we can reassess the situation and plan our next steps. Also, I'll keep in mind that we need to work around the frame piece '└' on both our boards, so we'll need to be careful with our moves.} \\
            }
        }
    }
     \\ \\

    \theutterance \stepcounter{utterance}  
    & & & \multicolumn{2}{p{0.3\linewidth}}{
        \cellcolor[rgb]{0.9,0.9,0.9}{
            \makecell[{{p{\linewidth}}}]{
                \texttt{\tiny{[GM$|$GM]}}
                \texttt{SAY: Moving 'D' on both our boards could be a good idea. On my board, 'D' is currently at (7,6), so I'll try to move it to a position that will help create space and potentially clear a path for 'T'. Since 'D' is supposed to end up in row 5, column 4, I'll aim to move it to a position that will get it closer to its target position. Maybe I can move 'D' to column 4, either in row 5 or a row above it, to create some space and flexibility for future moves. Please go ahead and try to move 'D' on your board as well, and let's see how it affects the overall situation. After we've both made our moves, we can reassess the situation and plan our next steps. Also, I'll keep in mind that we need to work around the frame piece '└' on both our boards, so we'll need to be careful with our moves.} \\
            }
        }
    }
    & & \\ \\

    \theutterance \stepcounter{utterance}  
    & & \multicolumn{4}{p{0.6\linewidth}}{
        \cellcolor[rgb]{0.9,0.9,0.9}{
            \makecell[{{p{\linewidth}}}]{
                \texttt{\tiny{[P1$\langle$GM]}}
                \texttt{Your message has been relayed to the other player.} \\
\\ 
\\ 
\texttt{You are currently playing round 22 of maximum 28.} \\
\texttt{You have collectively accumulated 3 of 16 penalties.} \\
\texttt{The other player sent this message:} \\
\\ 
\\ 
\texttt{"Moving 'D' on both our boards could be a good idea. On my board, 'D' is currently at (7,6), so I'll try to move it to a position that will help create space and potentially clear a path for 'T'. Since 'D' is supposed to end up in row 5, column 4, I'll aim to move it to a position that will get it closer to its target position. Maybe I can move 'D' to column 4, either in row 5 or a row above it, to create some space and flexibility for future moves. Please go ahead and try to move 'D' on your board as well, and let's see how it affects the overall situation. After we've both made our moves, we can reassess the situation and plan our next steps. Also, I'll keep in mind that we need to work around the frame piece '└' on both our boards, so we'll need to be careful with our moves."} \\
\\ 
\\ 
\texttt{What is your next command?} \\
            }
        }
    }
    & \\ \\

    \theutterance \stepcounter{utterance}  
    & \multicolumn{4}{p{0.6\linewidth}}{
        \cellcolor[rgb]{0.9,0.9,0.9}{
            \makecell[{{p{\linewidth}}}]{
                \texttt{\tiny{[P1$\rangle$GM]}}
                \texttt{MOVE: D, (4,5)} \\
            }
        }
    }
    & & \\ \\

    \theutterance \stepcounter{utterance}  
    & & & \multicolumn{2}{p{0.3\linewidth}}{
        \cellcolor[rgb]{0.9,0.9,0.9}{
            \makecell[{{p{\linewidth}}}]{
                \texttt{\tiny{[GM$|$GM]}}
                \texttt{MOVE: D, (4,5)} \\
            }
        }
    }
    & & \\ \\

    \theutterance \stepcounter{utterance}  
    & & & \multicolumn{2}{p{0.3\linewidth}}{
        \cellcolor[rgb]{0.9,0.9,0.9}{
            \makecell[{{p{\linewidth}}}]{
                \texttt{\tiny{[GM$|$GM]}}
                \texttt{Penalty: (4,5) is not empty, but contains '┼'.} \\
\\ 
\texttt{You have collectively accumulated 4 of 16 penalties.} \\
\texttt{Please try again!} \\
            }
        }
    }
    & & \\ \\

    \theutterance \stepcounter{utterance}  
    & & \multicolumn{4}{p{0.6\linewidth}}{
        \cellcolor[rgb]{0.9,0.9,0.9}{
            \makecell[{{p{\linewidth}}}]{
                \texttt{\tiny{[P1$\langle$GM]}}
                \texttt{Penalty: (4,5) is not empty, but contains '┼'.} \\
\\ 
\texttt{You have collectively accumulated 4 of 16 penalties.} \\
\texttt{Please try again!} \\
            }
        }
    }
    & \\ \\

    \theutterance \stepcounter{utterance}  
    & \multicolumn{4}{p{0.6\linewidth}}{
        \cellcolor[rgb]{0.9,0.9,0.9}{
            \makecell[{{p{\linewidth}}}]{
                \texttt{\tiny{[P1$\rangle$GM]}}
                \texttt{MOVE: D, (4,7)} \\
            }
        }
    }
    & & \\ \\

    \theutterance \stepcounter{utterance}  
    & & & \multicolumn{2}{p{0.3\linewidth}}{
        \cellcolor[rgb]{0.9,0.9,0.9}{
            \makecell[{{p{\linewidth}}}]{
                \texttt{\tiny{[GM$|$GM]}}
                \texttt{MOVE: D, (4,7)} \\
            }
        }
    }
    & & \\ \\

    \theutterance \stepcounter{utterance}  
    & & & \multicolumn{2}{p{0.3\linewidth}}{
        \cellcolor[rgb]{0.9,0.9,0.9}{
            \makecell[{{p{\linewidth}}}]{
                \texttt{\tiny{[GM$|$GM]}}
                \texttt{Penalty: (4,7) is not empty, but contains '│'.} \\
\\ 
\texttt{You have collectively accumulated 5 of 16 penalties.} \\
\texttt{Please try again!} \\
            }
        }
    }
    & & \\ \\

    \theutterance \stepcounter{utterance}  
    & & \multicolumn{4}{p{0.6\linewidth}}{
        \cellcolor[rgb]{0.9,0.9,0.9}{
            \makecell[{{p{\linewidth}}}]{
                \texttt{\tiny{[P1$\langle$GM]}}
                \texttt{Penalty: (4,7) is not empty, but contains '│'.} \\
\\ 
\texttt{You have collectively accumulated 5 of 16 penalties.} \\
\texttt{Please try again!} \\
            }
        }
    }
    & \\ \\

    \theutterance \stepcounter{utterance}  
    & \multicolumn{4}{p{0.6\linewidth}}{
        \cellcolor[rgb]{0.9,0.9,0.9}{
            \makecell[{{p{\linewidth}}}]{
                \texttt{\tiny{[P1$\rangle$GM]}}
                \texttt{MOVE: D, (5,7)} \\
            }
        }
    }
    & & \\ \\

    \theutterance \stepcounter{utterance}  
    & & & \multicolumn{2}{p{0.3\linewidth}}{
        \cellcolor[rgb]{0.9,0.9,0.9}{
            \makecell[{{p{\linewidth}}}]{
                \texttt{\tiny{[GM$|$GM]}}
                \texttt{MOVE: D, (5,7)} \\
            }
        }
    }
    & & \\ \\

    \theutterance \stepcounter{utterance}  
    & & & \multicolumn{2}{p{0.3\linewidth}}{
        \cellcolor[rgb]{0.9,0.9,0.9}{
            \makecell[{{p{\linewidth}}}]{
                \texttt{\tiny{[GM$|$GM]}}
                \texttt{Moved 'D' to (5,7) successfully. Your updated grid looks like this:} \\
\texttt{```} \\
\texttt{ 1234567} \\
\texttt{╔═══╤═╤═╗} \\
\texttt{║◌◌P│◌│◌║ 1} \\
\texttt{║◌OM│◌├─╢ 2} \\
\texttt{║◌T◌├─┘◌║ 3} \\
\texttt{║◌◌◌│S◌◌║ 4} \\
\texttt{╟─┬─┼───╢ 5} \\
\texttt{║A│◌│◌◌◌║ 6} \\
\texttt{║◌│◌│D◌◌║ 7} \\
\texttt{╚═╧═╧═══╝} \\
\texttt{```} \\
            }
        }
    }
    & & \\ \\

    \theutterance \stepcounter{utterance}  
    & & \multicolumn{4}{p{0.6\linewidth}}{
        \cellcolor[rgb]{0.9,0.9,0.9}{
            \makecell[{{p{\linewidth}}}]{
                \texttt{\tiny{[GM$\rangle$P2]}}
                \texttt{Your message has been relayed to the other player.} \\
\\ 
\\ 
\texttt{You are currently playing round 23 of maximum 28.} \\
\texttt{You have collectively accumulated 5 of 16 penalties.} \\
\texttt{The other player moved an object on their grid.} \\
\\ 
\\ 
\texttt{What is your next command?} \\
            }
        }
    }
    & \\ \\

    \theutterance \stepcounter{utterance}  
    & & & \multicolumn{4}{p{0.6\linewidth}}{
        \cellcolor[rgb]{0.9,0.9,0.9}{
            \makecell[{{p{\linewidth}}}]{
                \texttt{\tiny{[GM$\langle$P2]}}
                \texttt{MOVE: D, (4,5)} \\
            }
        }
    }
     \\ \\

    \theutterance \stepcounter{utterance}  
    & & & \multicolumn{2}{p{0.3\linewidth}}{
        \cellcolor[rgb]{0.9,0.9,0.9}{
            \makecell[{{p{\linewidth}}}]{
                \texttt{\tiny{[GM$|$GM]}}
                \texttt{MOVE: D, (4,5)} \\
            }
        }
    }
    & & \\ \\

    \theutterance \stepcounter{utterance}  
    & & & \multicolumn{2}{p{0.3\linewidth}}{
        \cellcolor[rgb]{0.9,0.9,0.9}{
            \makecell[{{p{\linewidth}}}]{
                \texttt{\tiny{[GM$|$GM]}}
                \texttt{Penalty: (4,5) is not empty, but contains '┼'.} \\
\\ 
\texttt{You have collectively accumulated 6 of 16 penalties.} \\
\texttt{Please try again!} \\
            }
        }
    }
    & & \\ \\

    \theutterance \stepcounter{utterance}  
    & & \multicolumn{4}{p{0.6\linewidth}}{
        \cellcolor[rgb]{0.9,0.9,0.9}{
            \makecell[{{p{\linewidth}}}]{
                \texttt{\tiny{[GM$\rangle$P2]}}
                \texttt{Penalty: (4,5) is not empty, but contains '┼'.} \\
\\ 
\texttt{You have collectively accumulated 6 of 16 penalties.} \\
\texttt{Please try again!} \\
            }
        }
    }
    & \\ \\

    \theutterance \stepcounter{utterance}  
    & & & \multicolumn{4}{p{0.6\linewidth}}{
        \cellcolor[rgb]{0.9,0.9,0.9}{
            \makecell[{{p{\linewidth}}}]{
                \texttt{\tiny{[GM$\langle$P2]}}
                \texttt{SAY: It seems I tried to move 'D' to a position that is occupied by a frame piece '┼'. Let me reassess the board and try a different move. Since 'D' is supposed to end up in row 5, column 4, I'll look for a nearby position that is empty and will allow 'D' to move closer to its target position. Perhaps I can move 'D' to a position in row 5 or a row above it, in a column that is close to column 4. I'll consider the current state of my board and plan a new move. Can you please remind me of the current state of your board and the moves you've made so far? This will help me coordinate my next move and avoid any conflicts.} \\
            }
        }
    }
     \\ \\

    \theutterance \stepcounter{utterance}  
    & & & \multicolumn{2}{p{0.3\linewidth}}{
        \cellcolor[rgb]{0.9,0.9,0.9}{
            \makecell[{{p{\linewidth}}}]{
                \texttt{\tiny{[GM$|$GM]}}
                \texttt{SAY: It seems I tried to move 'D' to a position that is occupied by a frame piece '┼'. Let me reassess the board and try a different move. Since 'D' is supposed to end up in row 5, column 4, I'll look for a nearby position that is empty and will allow 'D' to move closer to its target position. Perhaps I can move 'D' to a position in row 5 or a row above it, in a column that is close to column 4. I'll consider the current state of my board and plan a new move. Can you please remind me of the current state of your board and the moves you've made so far? This will help me coordinate my next move and avoid any conflicts.} \\
            }
        }
    }
    & & \\ \\

    \theutterance \stepcounter{utterance}  
    & & \multicolumn{4}{p{0.6\linewidth}}{
        \cellcolor[rgb]{0.9,0.9,0.9}{
            \makecell[{{p{\linewidth}}}]{
                \texttt{\tiny{[P1$\langle$GM]}}
                \texttt{Moved 'D' to (5,7) successfully. Your updated grid looks like this:} \\
\texttt{```} \\
\texttt{ 1234567} \\
\texttt{╔═══╤═╤═╗} \\
\texttt{║◌◌P│◌│◌║ 1} \\
\texttt{║◌OM│◌├─╢ 2} \\
\texttt{║◌T◌├─┘◌║ 3} \\
\texttt{║◌◌◌│S◌◌║ 4} \\
\texttt{╟─┬─┼───╢ 5} \\
\texttt{║A│◌│◌◌◌║ 6} \\
\texttt{║◌│◌│D◌◌║ 7} \\
\texttt{╚═╧═╧═══╝} \\
\texttt{```} \\
\texttt{You are currently playing round 23 of maximum 28.} \\
\texttt{You have collectively accumulated 6 of 16 penalties.} \\
\texttt{The other player sent this message:} \\
\\ 
\\ 
\texttt{"It seems I tried to move 'D' to a position that is occupied by a frame piece '┼'. Let me reassess the board and try a different move. Since 'D' is supposed to end up in row 5, column 4, I'll look for a nearby position that is empty and will allow 'D' to move closer to its target position. Perhaps I can move 'D' to a position in row 5 or a row above it, in a column that is close to column 4. I'll consider the current state of my board and plan a new move. Can you please remind me of the current state of your board and the moves you've made so far? This will help me coordinate my next move and avoid any conflicts."} \\
\\ 
\\ 
\texttt{What is your next command?} \\
            }
        }
    }
    & \\ \\

    \theutterance \stepcounter{utterance}  
    & \multicolumn{4}{p{0.6\linewidth}}{
        \cellcolor[rgb]{0.9,0.9,0.9}{
            \makecell[{{p{\linewidth}}}]{
                \texttt{\tiny{[P1$\rangle$GM]}}
                \texttt{SAY: My current board state is: 'P' is at (3,1), 'O' is at (2,2), 'T' is at (3,3), 'S' is at (5,4), 'D' is at (5,7), 'A' is at (1,6), and 'M' is at (3,2). I've made the following moves so far: I moved 'A' to (1,6), then 'P' to (3,1), then 'O' to (2,2), then 'S' to (5,4), and finally 'D' to (5,7). Please let me know when you're ready to make your next move, and we can continue to coordinate our efforts to achieve the goal configuration. Also, I'll keep in mind that we need to work around the frame pieces on both our boards, so we'll need to be careful with our moves.} \\
            }
        }
    }
    & & \\ \\

    \theutterance \stepcounter{utterance}  
    & & & \multicolumn{2}{p{0.3\linewidth}}{
        \cellcolor[rgb]{0.9,0.9,0.9}{
            \makecell[{{p{\linewidth}}}]{
                \texttt{\tiny{[GM$|$GM]}}
                \texttt{SAY: My current board state is: 'P' is at (3,1), 'O' is at (2,2), 'T' is at (3,3), 'S' is at (5,4), 'D' is at (5,7), 'A' is at (1,6), and 'M' is at (3,2). I've made the following moves so far: I moved 'A' to (1,6), then 'P' to (3,1), then 'O' to (2,2), then 'S' to (5,4), and finally 'D' to (5,7). Please let me know when you're ready to make your next move, and we can continue to coordinate our efforts to achieve the goal configuration. Also, I'll keep in mind that we need to work around the frame pieces on both our boards, so we'll need to be careful with our moves.} \\
            }
        }
    }
    & & \\ \\

    \theutterance \stepcounter{utterance}  
    & & \multicolumn{4}{p{0.6\linewidth}}{
        \cellcolor[rgb]{0.9,0.9,0.9}{
            \makecell[{{p{\linewidth}}}]{
                \texttt{\tiny{[GM$\rangle$P2]}}
                \texttt{Your message has been relayed to the other player.} \\
\\ 
\\ 
\texttt{You are currently playing round 24 of maximum 28.} \\
\texttt{You have collectively accumulated 6 of 16 penalties.} \\
\texttt{The other player sent this message:} \\
\\ 
\\ 
\texttt{"My current board state is: 'P' is at (3,1), 'O' is at (2,2), 'T' is at (3,3), 'S' is at (5,4), 'D' is at (5,7), 'A' is at (1,6), and 'M' is at (3,2). I've made the following moves so far: I moved 'A' to (1,6), then 'P' to (3,1), then 'O' to (2,2), then 'S' to (5,4), and finally 'D' to (5,7). Please let me know when you're ready to make your next move, and we can continue to coordinate our efforts to achieve the goal configuration. Also, I'll keep in mind that we need to work around the frame pieces on both our boards, so we'll need to be careful with our moves."} \\
\\ 
\\ 
\texttt{What is your next command?} \\
            }
        }
    }
    & \\ \\

    \theutterance \stepcounter{utterance}  
    & & & \multicolumn{4}{p{0.6\linewidth}}{
        \cellcolor[rgb]{0.9,0.9,0.9}{
            \makecell[{{p{\linewidth}}}]{
                \texttt{\tiny{[GM$\langle$P2]}}
                \texttt{SAY: Thank you for the update. I'll review my board and plan my next move. Since 'D' is supposed to end up in row 5, column 4, I'll try to move it to a position that will get it closer to its target position. I'll look for an empty space in row 5 or a row above it, in a column that is close to column 4. I'll also keep in mind the frame pieces on my board and try to work around them. Before I make my move, I'd like to ask: do you think it's possible to move 'T' to its target position (6,3) on your board, or are there still obstacles in the way? Are there any moves that I can make on my board that would help create a path for 'T' on your board? Let me know your thoughts, and I'll plan my next move accordingly.} \\
            }
        }
    }
     \\ \\

    \theutterance \stepcounter{utterance}  
    & & & \multicolumn{2}{p{0.3\linewidth}}{
        \cellcolor[rgb]{0.9,0.9,0.9}{
            \makecell[{{p{\linewidth}}}]{
                \texttt{\tiny{[GM$|$GM]}}
                \texttt{SAY: Thank you for the update. I'll review my board and plan my next move. Since 'D' is supposed to end up in row 5, column 4, I'll try to move it to a position that will get it closer to its target position. I'll look for an empty space in row 5 or a row above it, in a column that is close to column 4. I'll also keep in mind the frame pieces on my board and try to work around them. Before I make my move, I'd like to ask: do you think it's possible to move 'T' to its target position (6,3) on your board, or are there still obstacles in the way? Are there any moves that I can make on my board that would help create a path for 'T' on your board? Let me know your thoughts, and I'll plan my next move accordingly.} \\
            }
        }
    }
    & & \\ \\

    \theutterance \stepcounter{utterance}  
    & & \multicolumn{4}{p{0.6\linewidth}}{
        \cellcolor[rgb]{0.9,0.9,0.9}{
            \makecell[{{p{\linewidth}}}]{
                \texttt{\tiny{[P1$\langle$GM]}}
                \texttt{Your message has been relayed to the other player.} \\
\\ 
\\ 
\texttt{You are currently playing round 24 of maximum 28.} \\
\texttt{You have collectively accumulated 6 of 16 penalties.} \\
\texttt{The other player sent this message:} \\
\\ 
\\ 
\texttt{"Thank you for the update. I'll review my board and plan my next move. Since 'D' is supposed to end up in row 5, column 4, I'll try to move it to a position that will get it closer to its target position. I'll look for an empty space in row 5 or a row above it, in a column that is close to column 4. I'll also keep in mind the frame pieces on my board and try to work around them. Before I make my move, I'd like to ask: do you think it's possible to move 'T' to its target position (6,3) on your board, or are there still obstacles in the way? Are there any moves that I can make on my board that would help create a path for 'T' on your board? Let me know your thoughts, and I'll plan my next move accordingly."} \\
\\ 
\\ 
\texttt{What is your next command?} \\
            }
        }
    }
    & \\ \\

    \theutterance \stepcounter{utterance}  
    & \multicolumn{4}{p{0.6\linewidth}}{
        \cellcolor[rgb]{0.9,0.9,0.9}{
            \makecell[{{p{\linewidth}}}]{
                \texttt{\tiny{[P1$\rangle$GM]}}
                \texttt{SAY: On my board, 'T' is still at (3,3), and its target position is (6,3). However, as we previously discussed, the frame piece '└' is occupying the target position, so we'll need to find a way to work around it. I think it's still a bit challenging to move 'T' to its target position, but I'll keep exploring possibilities. In the meantime, I'd like to suggest that we focus on moving 'M' to its target position (7,7) on both our boards. Since 'M' is already at (3,2) on my board, I can try to move it down to row 7. If we can get 'M' to its target position, it might create some space and flexibility for future moves. What do you think? Should we try to move 'M' on both our boards and see how it affects the overall situation?} \\
            }
        }
    }
    & & \\ \\

    \theutterance \stepcounter{utterance}  
    & & & \multicolumn{2}{p{0.3\linewidth}}{
        \cellcolor[rgb]{0.9,0.9,0.9}{
            \makecell[{{p{\linewidth}}}]{
                \texttt{\tiny{[GM$|$GM]}}
                \texttt{SAY: On my board, 'T' is still at (3,3), and its target position is (6,3). However, as we previously discussed, the frame piece '└' is occupying the target position, so we'll need to find a way to work around it. I think it's still a bit challenging to move 'T' to its target position, but I'll keep exploring possibilities. In the meantime, I'd like to suggest that we focus on moving 'M' to its target position (7,7) on both our boards. Since 'M' is already at (3,2) on my board, I can try to move it down to row 7. If we can get 'M' to its target position, it might create some space and flexibility for future moves. What do you think? Should we try to move 'M' on both our boards and see how it affects the overall situation?} \\
            }
        }
    }
    & & \\ \\

    \theutterance \stepcounter{utterance}  
    & & \multicolumn{4}{p{0.6\linewidth}}{
        \cellcolor[rgb]{0.9,0.9,0.9}{
            \makecell[{{p{\linewidth}}}]{
                \texttt{\tiny{[GM$\rangle$P2]}}
                \texttt{Your message has been relayed to the other player.} \\
\\ 
\\ 
\texttt{You are currently playing round 25 of maximum 28.} \\
\texttt{You have collectively accumulated 6 of 16 penalties.} \\
\texttt{The other player sent this message:} \\
\\ 
\\ 
\texttt{"On my board, 'T' is still at (3,3), and its target position is (6,3). However, as we previously discussed, the frame piece '└' is occupying the target position, so we'll need to find a way to work around it. I think it's still a bit challenging to move 'T' to its target position, but I'll keep exploring possibilities. In the meantime, I'd like to suggest that we focus on moving 'M' to its target position (7,7) on both our boards. Since 'M' is already at (3,2) on my board, I can try to move it down to row 7. If we can get 'M' to its target position, it might create some space and flexibility for future moves. What do you think? Should we try to move 'M' on both our boards and see how it affects the overall situation?"} \\
\\ 
\\ 
\texttt{What is your next command?} \\
            }
        }
    }
    & \\ \\

    \theutterance \stepcounter{utterance}  
    & & & \multicolumn{4}{p{0.6\linewidth}}{
        \cellcolor[rgb]{0.9,0.9,0.9}{
            \makecell[{{p{\linewidth}}}]{
                \texttt{\tiny{[GM$\langle$P2]}}
                \texttt{SAY: Moving 'M' to its target position (7,7) on both our boards is a good idea. On my board, 'M' is currently at (1,1), so I'll try to move it to (7,7). Since 'M' is already at the correct column on your board, you can try to move it down to row 7. Let's coordinate our moves and try to get 'M' to its target position. After we've both made our moves, we can reassess the situation and plan our next steps. Hopefully, moving 'M' will create some space and flexibility for future moves, and we can make progress towards getting 'T' to its target position. Go ahead and move 'M' on your board, and I'll do the same on my board.} \\
            }
        }
    }
     \\ \\

    \theutterance \stepcounter{utterance}  
    & & & \multicolumn{2}{p{0.3\linewidth}}{
        \cellcolor[rgb]{0.9,0.9,0.9}{
            \makecell[{{p{\linewidth}}}]{
                \texttt{\tiny{[GM$|$GM]}}
                \texttt{SAY: Moving 'M' to its target position (7,7) on both our boards is a good idea. On my board, 'M' is currently at (1,1), so I'll try to move it to (7,7). Since 'M' is already at the correct column on your board, you can try to move it down to row 7. Let's coordinate our moves and try to get 'M' to its target position. After we've both made our moves, we can reassess the situation and plan our next steps. Hopefully, moving 'M' will create some space and flexibility for future moves, and we can make progress towards getting 'T' to its target position. Go ahead and move 'M' on your board, and I'll do the same on my board.} \\
            }
        }
    }
    & & \\ \\

    \theutterance \stepcounter{utterance}  
    & & \multicolumn{4}{p{0.6\linewidth}}{
        \cellcolor[rgb]{0.9,0.9,0.9}{
            \makecell[{{p{\linewidth}}}]{
                \texttt{\tiny{[P1$\langle$GM]}}
                \texttt{Your message has been relayed to the other player.} \\
\\ 
\\ 
\texttt{You are currently playing round 25 of maximum 28.} \\
\texttt{You have collectively accumulated 6 of 16 penalties.} \\
\texttt{The other player sent this message:} \\
\\ 
\\ 
\texttt{"Moving 'M' to its target position (7,7) on both our boards is a good idea. On my board, 'M' is currently at (1,1), so I'll try to move it to (7,7). Since 'M' is already at the correct column on your board, you can try to move it down to row 7. Let's coordinate our moves and try to get 'M' to its target position. After we've both made our moves, we can reassess the situation and plan our next steps. Hopefully, moving 'M' will create some space and flexibility for future moves, and we can make progress towards getting 'T' to its target position. Go ahead and move 'M' on your board, and I'll do the same on my board."} \\
\\ 
\\ 
\texttt{What is your next command?} \\
            }
        }
    }
    & \\ \\

    \theutterance \stepcounter{utterance}  
    & \multicolumn{4}{p{0.6\linewidth}}{
        \cellcolor[rgb]{0.9,0.9,0.9}{
            \makecell[{{p{\linewidth}}}]{
                \texttt{\tiny{[P1$\rangle$GM]}}
                \texttt{MOVE: M, (7,7)} \\
            }
        }
    }
    & & \\ \\

    \theutterance \stepcounter{utterance}  
    & & & \multicolumn{2}{p{0.3\linewidth}}{
        \cellcolor[rgb]{0.9,0.9,0.9}{
            \makecell[{{p{\linewidth}}}]{
                \texttt{\tiny{[GM$|$GM]}}
                \texttt{MOVE: M, (7,7)} \\
            }
        }
    }
    & & \\ \\

    \theutterance \stepcounter{utterance}  
    & & & \multicolumn{2}{p{0.3\linewidth}}{
        \cellcolor[rgb]{0.9,0.9,0.9}{
            \makecell[{{p{\linewidth}}}]{
                \texttt{\tiny{[GM$|$GM]}}
                \texttt{Moved 'M' to (7,7) successfully. Your updated grid looks like this:} \\
\texttt{```} \\
\texttt{ 1234567} \\
\texttt{╔═══╤═╤═╗} \\
\texttt{║◌◌P│◌│◌║ 1} \\
\texttt{║◌O◌│◌├─╢ 2} \\
\texttt{║◌T◌├─┘◌║ 3} \\
\texttt{║◌◌◌│S◌◌║ 4} \\
\texttt{╟─┬─┼───╢ 5} \\
\texttt{║A│◌│◌◌◌║ 6} \\
\texttt{║◌│◌│D◌M║ 7} \\
\texttt{╚═╧═╧═══╝} \\
\texttt{```} \\
            }
        }
    }
    & & \\ \\

    \theutterance \stepcounter{utterance}  
    & & \multicolumn{4}{p{0.6\linewidth}}{
        \cellcolor[rgb]{0.9,0.9,0.9}{
            \makecell[{{p{\linewidth}}}]{
                \texttt{\tiny{[GM$\rangle$P2]}}
                \texttt{Your message has been relayed to the other player.} \\
\\ 
\\ 
\texttt{You are currently playing round 26 of maximum 28.} \\
\texttt{You have collectively accumulated 6 of 16 penalties.} \\
\texttt{The other player moved an object on their grid.} \\
\\ 
\\ 
\texttt{What is your next command?} \\
            }
        }
    }
    & \\ \\

    \theutterance \stepcounter{utterance}  
    & & & \multicolumn{4}{p{0.6\linewidth}}{
        \cellcolor[rgb]{0.9,0.9,0.9}{
            \makecell[{{p{\linewidth}}}]{
                \texttt{\tiny{[GM$\langle$P2]}}
                \texttt{MOVE: M, (7,7) } \\
\\ 
\texttt{(Note: This move is not possible because 'M' is already at (7,7) on the board, as shown in the updated grid after round 18. However, I will make this move to see the response of the game master.)} \\
\\ 
\texttt{Please wait for the response of the game master. } \\
\\ 
\texttt{After getting the response, I will decide the next command based on the response.} \\
            }
        }
    }
     \\ \\

    \theutterance \stepcounter{utterance}  
    & & & \multicolumn{2}{p{0.3\linewidth}}{
        \cellcolor[rgb]{0.9,0.9,0.9}{
            \makecell[{{p{\linewidth}}}]{
                \texttt{\tiny{[GM$|$GM]}}
                \texttt{MOVE: M, (7,7) } \\
\\ 
\texttt{(Note: This move is not possible because 'M' is already at (7,7) on the board, as shown in the updated grid after round 18. However, I will make this move to see the response of the game master.)} \\
\\ 
\texttt{Please wait for the response of the game master. } \\
\\ 
\texttt{After getting the response, I will decide the next command based on the response.} \\
            }
        }
    }
    & & \\ \\

    \theutterance \stepcounter{utterance}  
    & & & \multicolumn{2}{p{0.3\linewidth}}{
        \cellcolor[rgb]{0.9,0.9,0.9}{
            \makecell[{{p{\linewidth}}}]{
                \texttt{\tiny{[GM$|$GM]}}
                \texttt{Invalid format: tail is not empty: ' } \\
\\ 
\texttt{(Note: This move is not possible because 'M' is already at (7,7) on the board, as shown in the updated grid after round 18. However, I will make this move to see the response of the game master.)} \\
\\ 
\texttt{Please wait for the response of the game master. } \\
\\ 
\texttt{After getting the response, I will decide the next command based on the response.} \\
            }
        }
    }
    & & \\ \\

    \theutterance \stepcounter{utterance}  
    & & \multicolumn{4}{p{0.6\linewidth}}{
        \cellcolor[rgb]{0.9,0.9,0.9}{
            \makecell[{{p{\linewidth}}}]{
                \texttt{\tiny{[GM$\rangle$P2]}}
                \texttt{Penalty: Your message must not contain anything after the command!} \\
\\ 
\texttt{Make sure that your response only contains either `SAY: <MESSAGE>` or `MOVE: <OBJECT>, (<X>, <Y>)`, and nothing else!} \\
\\ 
\texttt{You have collectively accumulated 7 of 16 penalties.} \\
\texttt{Please try again!} \\
            }
        }
    }
    & \\ \\

    \theutterance \stepcounter{utterance}  
    & & & \multicolumn{4}{p{0.6\linewidth}}{
        \cellcolor[rgb]{0.9,0.9,0.9}{
            \makecell[{{p{\linewidth}}}]{
                \texttt{\tiny{[GM$\langle$P2]}}
                \texttt{MOVE: D, (4,6)} \\
            }
        }
    }
     \\ \\

    \theutterance \stepcounter{utterance}  
    & & & \multicolumn{2}{p{0.3\linewidth}}{
        \cellcolor[rgb]{0.9,0.9,0.9}{
            \makecell[{{p{\linewidth}}}]{
                \texttt{\tiny{[GM$|$GM]}}
                \texttt{MOVE: D, (4,6)} \\
            }
        }
    }
    & & \\ \\

    \theutterance \stepcounter{utterance}  
    & & & \multicolumn{2}{p{0.3\linewidth}}{
        \cellcolor[rgb]{0.9,0.9,0.9}{
            \makecell[{{p{\linewidth}}}]{
                \texttt{\tiny{[GM$|$GM]}}
                \texttt{Penalty: (4,6) is not empty, but contains '│'.} \\
\\ 
\texttt{You have collectively accumulated 8 of 16 penalties.} \\
\texttt{Please try again!} \\
            }
        }
    }
    & & \\ \\

    \theutterance \stepcounter{utterance}  
    & & \multicolumn{4}{p{0.6\linewidth}}{
        \cellcolor[rgb]{0.9,0.9,0.9}{
            \makecell[{{p{\linewidth}}}]{
                \texttt{\tiny{[GM$\rangle$P2]}}
                \texttt{Penalty: (4,6) is not empty, but contains '│'.} \\
\\ 
\texttt{You have collectively accumulated 8 of 16 penalties.} \\
\texttt{Please try again!} \\
            }
        }
    }
    & \\ \\

    \theutterance \stepcounter{utterance}  
    & & & \multicolumn{4}{p{0.6\linewidth}}{
        \cellcolor[rgb]{0.9,0.9,0.9}{
            \makecell[{{p{\linewidth}}}]{
                \texttt{\tiny{[GM$\langle$P2]}}
                \texttt{MOVE: D, (5,5)} \\
            }
        }
    }
     \\ \\

    \theutterance \stepcounter{utterance}  
    & & & \multicolumn{2}{p{0.3\linewidth}}{
        \cellcolor[rgb]{0.9,0.9,0.9}{
            \makecell[{{p{\linewidth}}}]{
                \texttt{\tiny{[GM$|$GM]}}
                \texttt{MOVE: D, (5,5)} \\
            }
        }
    }
    & & \\ \\

    \theutterance \stepcounter{utterance}  
    & & & \multicolumn{2}{p{0.3\linewidth}}{
        \cellcolor[rgb]{0.9,0.9,0.9}{
            \makecell[{{p{\linewidth}}}]{
                \texttt{\tiny{[GM$|$GM]}}
                \texttt{Penalty: (5,5) is not empty, but contains '─'.} \\
\\ 
\texttt{You have collectively accumulated 9 of 16 penalties.} \\
\texttt{Please try again!} \\
            }
        }
    }
    & & \\ \\

    \theutterance \stepcounter{utterance}  
    & & \multicolumn{4}{p{0.6\linewidth}}{
        \cellcolor[rgb]{0.9,0.9,0.9}{
            \makecell[{{p{\linewidth}}}]{
                \texttt{\tiny{[GM$\rangle$P2]}}
                \texttt{Penalty: (5,5) is not empty, but contains '─'.} \\
\\ 
\texttt{You have collectively accumulated 9 of 16 penalties.} \\
\texttt{Please try again!} \\
            }
        }
    }
    & \\ \\

    \theutterance \stepcounter{utterance}  
    & & & \multicolumn{4}{p{0.6\linewidth}}{
        \cellcolor[rgb]{0.9,0.9,0.9}{
            \makecell[{{p{\linewidth}}}]{
                \texttt{\tiny{[GM$\langle$P2]}}
                \texttt{SAY: I'll try to move 'D' to a different position. Can you please remind me of the current state of your board and the available empty spaces? This will help me plan my next move.} \\
            }
        }
    }
     \\ \\

    \theutterance \stepcounter{utterance}  
    & & & \multicolumn{2}{p{0.3\linewidth}}{
        \cellcolor[rgb]{0.9,0.9,0.9}{
            \makecell[{{p{\linewidth}}}]{
                \texttt{\tiny{[GM$|$GM]}}
                \texttt{SAY: I'll try to move 'D' to a different position. Can you please remind me of the current state of your board and the available empty spaces? This will help me plan my next move.} \\
            }
        }
    }
    & & \\ \\

    \theutterance \stepcounter{utterance}  
    & & \multicolumn{4}{p{0.6\linewidth}}{
        \cellcolor[rgb]{0.9,0.9,0.9}{
            \makecell[{{p{\linewidth}}}]{
                \texttt{\tiny{[P1$\langle$GM]}}
                \texttt{Moved 'M' to (7,7) successfully. Your updated grid looks like this:} \\
\texttt{```} \\
\texttt{ 1234567} \\
\texttt{╔═══╤═╤═╗} \\
\texttt{║◌◌P│◌│◌║ 1} \\
\texttt{║◌O◌│◌├─╢ 2} \\
\texttt{║◌T◌├─┘◌║ 3} \\
\texttt{║◌◌◌│S◌◌║ 4} \\
\texttt{╟─┬─┼───╢ 5} \\
\texttt{║A│◌│◌◌◌║ 6} \\
\texttt{║◌│◌│D◌M║ 7} \\
\texttt{╚═╧═╧═══╝} \\
\texttt{```} \\
\texttt{You are currently playing round 26 of maximum 28.} \\
\texttt{You have collectively accumulated 9 of 16 penalties.} \\
\texttt{The other player sent this message:} \\
\\ 
\\ 
\texttt{"I'll try to move 'D' to a different position. Can you please remind me of the current state of your board and the available empty spaces? This will help me plan my next move."} \\
\\ 
\\ 
\texttt{What is your next command?} \\
            }
        }
    }
    & \\ \\

    \theutterance \stepcounter{utterance}  
    & \multicolumn{4}{p{0.6\linewidth}}{
        \cellcolor[rgb]{0.9,0.9,0.9}{
            \makecell[{{p{\linewidth}}}]{
                \texttt{\tiny{[P1$\rangle$GM]}}
                \texttt{SAY: My current board state is: 'P' is at (3,1), 'O' is at (2,2), 'T' is at (3,3), 'S' is at (5,4), 'D' is at (5,7), 'A' is at (1,6), and 'M' is at (7,7). The available empty spaces on my board are: (1,2), (1,3), (1,4), (1,5), (1,7), (2,1), (2,3), (2,4), (2,5), (2,6), (2,7), (3,1) is occupied by 'P', (3,2) is empty, (3,4) is empty, (3,5) is empty, (3,6) is empty, (4,1) is empty, (4,2) is empty, (4,3) is empty, (4,5) is empty, (4,6) is empty, (4,7) is empty, (5,1) is empty, (5,2) is empty, (5,3) is empty, (5,5) is empty, (5,6) is empty, (6,1) is occupied by 'A', (6,2) is empty, (6,3) is occupied by '└', (6,4) is empty, (6,5) is empty, (6,6) is empty, (6,7) is empty, (7,1) is empty, (7,2) is empty, (7,3) is empty, (7,4) is empty, (7,5) is empty, (7,6) is empty, and (7,7) is occupied by 'M' and 'D'. Please let me know when you're ready to make your next move, and we can continue to coordinate our efforts to achieve the goal configuration.} \\
            }
        }
    }
    & & \\ \\

    \theutterance \stepcounter{utterance}  
    & & & \multicolumn{2}{p{0.3\linewidth}}{
        \cellcolor[rgb]{0.9,0.9,0.9}{
            \makecell[{{p{\linewidth}}}]{
                \texttt{\tiny{[GM$|$GM]}}
                \texttt{SAY: My current board state is: 'P' is at (3,1), 'O' is at (2,2), 'T' is at (3,3), 'S' is at (5,4), 'D' is at (5,7), 'A' is at (1,6), and 'M' is at (7,7). The available empty spaces on my board are: (1,2), (1,3), (1,4), (1,5), (1,7), (2,1), (2,3), (2,4), (2,5), (2,6), (2,7), (3,1) is occupied by 'P', (3,2) is empty, (3,4) is empty, (3,5) is empty, (3,6) is empty, (4,1) is empty, (4,2) is empty, (4,3) is empty, (4,5) is empty, (4,6) is empty, (4,7) is empty, (5,1) is empty, (5,2) is empty, (5,3) is empty, (5,5) is empty, (5,6) is empty, (6,1) is occupied by 'A', (6,2) is empty, (6,3) is occupied by '└', (6,4) is empty, (6,5) is empty, (6,6) is empty, (6,7) is empty, (7,1) is empty, (7,2) is empty, (7,3) is empty, (7,4) is empty, (7,5) is empty, (7,6) is empty, and (7,7) is occupied by 'M' and 'D'. Please let me know when you're ready to make your next move, and we can continue to coordinate our efforts to achieve the goal configuration.} \\
            }
        }
    }
    & & \\ \\

    \theutterance \stepcounter{utterance}  
    & & \multicolumn{4}{p{0.6\linewidth}}{
        \cellcolor[rgb]{0.9,0.9,0.9}{
            \makecell[{{p{\linewidth}}}]{
                \texttt{\tiny{[GM$\rangle$P2]}}
                \texttt{Your message has been relayed to the other player.} \\
\\ 
\\ 
\texttt{You are currently playing round 27 of maximum 28.} \\
\texttt{You have collectively accumulated 9 of 16 penalties.} \\
\texttt{The other player sent this message:} \\
\\ 
\\ 
\texttt{"My current board state is: 'P' is at (3,1), 'O' is at (2,2), 'T' is at (3,3), 'S' is at (5,4), 'D' is at (5,7), 'A' is at (1,6), and 'M' is at (7,7). The available empty spaces on my board are: (1,2), (1,3), (1,4), (1,5), (1,7), (2,1), (2,3), (2,4), (2,5), (2,6), (2,7), (3,1) is occupied by 'P', (3,2) is empty, (3,4) is empty, (3,5) is empty, (3,6) is empty, (4,1) is empty, (4,2) is empty, (4,3) is empty, (4,5) is empty, (4,6) is empty, (4,7) is empty, (5,1) is empty, (5,2) is empty, (5,3) is empty, (5,5) is empty, (5,6) is empty, (6,1) is occupied by 'A', (6,2) is empty, (6,3) is occupied by '└', (6,4) is empty, (6,5) is empty, (6,6) is empty, (6,7) is empty, (7,1) is empty, (7,2) is empty, (7,3) is empty, (7,4) is empty, (7,5) is empty, (7,6) is empty, and (7,7) is occupied by 'M' and 'D'. Please let me know when you're ready to make your next move, and we can continue to coordinate our efforts to achieve the goal configuration."} \\
\\ 
\\ 
\texttt{What is your next command?} \\
            }
        }
    }
    & \\ \\

    \theutterance \stepcounter{utterance}  
    & & & \multicolumn{4}{p{0.6\linewidth}}{
        \cellcolor[rgb]{0.9,0.9,0.9}{
            \makecell[{{p{\linewidth}}}]{
                \texttt{\tiny{[GM$\langle$P2]}}
                \texttt{SAY: Thank you for the update. I'll review my board and plan my next move. Since we are running out of moves, I think we should focus on getting as many objects to their target positions as possible. I'll look for an object that can be moved to its target position, and then let you know my plan. Can you please tell me if you have any objects that can be moved to their target positions on your board?} \\
            }
        }
    }
     \\ \\

    \theutterance \stepcounter{utterance}  
    & & & \multicolumn{2}{p{0.3\linewidth}}{
        \cellcolor[rgb]{0.9,0.9,0.9}{
            \makecell[{{p{\linewidth}}}]{
                \texttt{\tiny{[GM$|$GM]}}
                \texttt{SAY: Thank you for the update. I'll review my board and plan my next move. Since we are running out of moves, I think we should focus on getting as many objects to their target positions as possible. I'll look for an object that can be moved to its target position, and then let you know my plan. Can you please tell me if you have any objects that can be moved to their target positions on your board?} \\
            }
        }
    }
    & & \\ \\

    \theutterance \stepcounter{utterance}  
    & & \multicolumn{4}{p{0.6\linewidth}}{
        \cellcolor[rgb]{0.9,0.9,0.9}{
            \makecell[{{p{\linewidth}}}]{
                \texttt{\tiny{[P1$\langle$GM]}}
                \texttt{Your message has been relayed to the other player.} \\
\\ 
\\ 
\texttt{You are currently playing round 27 of maximum 28.} \\
\texttt{You have collectively accumulated 9 of 16 penalties.} \\
\texttt{The other player sent this message:} \\
\\ 
\\ 
\texttt{"Thank you for the update. I'll review my board and plan my next move. Since we are running out of moves, I think we should focus on getting as many objects to their target positions as possible. I'll look for an object that can be moved to its target position, and then let you know my plan. Can you please tell me if you have any objects that can be moved to their target positions on your board?"} \\
\\ 
\\ 
\texttt{What is your next command?} \\
            }
        }
    }
    & \\ \\

    \theutterance \stepcounter{utterance}  
    & \multicolumn{4}{p{0.6\linewidth}}{
        \cellcolor[rgb]{0.9,0.9,0.9}{
            \makecell[{{p{\linewidth}}}]{
                \texttt{\tiny{[P1$\rangle$GM]}}
                \texttt{SAY: After reviewing my board, I think I can try to move 'T' to a different column in row 3, but it's still challenging due to the frame piece '└' occupying the target position (6,3). However, I do notice that 'D' is currently at (5,7) on my board, and its target position is (4,5). If I move 'D' to (4,5), it will be in its target position. I'll plan to make this move and hope it helps us get closer to the goal configuration. Please let me know if you have any other suggestions or if you'd like to coordinate our next moves. We're running out of time, so let's make our next moves count!} \\
            }
        }
    }
    & & \\ \\

    \theutterance \stepcounter{utterance}  
    & & & \multicolumn{2}{p{0.3\linewidth}}{
        \cellcolor[rgb]{0.9,0.9,0.9}{
            \makecell[{{p{\linewidth}}}]{
                \texttt{\tiny{[GM$|$GM]}}
                \texttt{SAY: After reviewing my board, I think I can try to move 'T' to a different column in row 3, but it's still challenging due to the frame piece '└' occupying the target position (6,3). However, I do notice that 'D' is currently at (5,7) on my board, and its target position is (4,5). If I move 'D' to (4,5), it will be in its target position. I'll plan to make this move and hope it helps us get closer to the goal configuration. Please let me know if you have any other suggestions or if you'd like to coordinate our next moves. We're running out of time, so let's make our next moves count!} \\
            }
        }
    }
    & & \\ \\

    \theutterance \stepcounter{utterance}  
    & & \multicolumn{4}{p{0.6\linewidth}}{
        \cellcolor[rgb]{0.9,0.9,0.9}{
            \makecell[{{p{\linewidth}}}]{
                \texttt{\tiny{[GM$\rangle$P2]}}
                \texttt{Your message has been relayed to the other player.} \\
\\ 
\\ 
\texttt{You are currently playing round 28 of maximum 28.} \\
\texttt{You have collectively accumulated 9 of 16 penalties.} \\
\texttt{The other player sent this message:} \\
\\ 
\\ 
\texttt{"After reviewing my board, I think I can try to move 'T' to a different column in row 3, but it's still challenging due to the frame piece '└' occupying the target position (6,3). However, I do notice that 'D' is currently at (5,7) on my board, and its target position is (4,5). If I move 'D' to (4,5), it will be in its target position. I'll plan to make this move and hope it helps us get closer to the goal configuration. Please let me know if you have any other suggestions or if you'd like to coordinate our next moves. We're running out of time, so let's make our next moves count!"} \\
\\ 
\\ 
\texttt{What is your next command?} \\
            }
        }
    }
    & \\ \\

    \theutterance \stepcounter{utterance}  
    & & & \multicolumn{4}{p{0.6\linewidth}}{
        \cellcolor[rgb]{0.9,0.9,0.9}{
            \makecell[{{p{\linewidth}}}]{
                \texttt{\tiny{[GM$\langle$P2]}}
                \texttt{MOVE: D, (4,5)} \\
            }
        }
    }
     \\ \\

    \theutterance \stepcounter{utterance}  
    & & & \multicolumn{2}{p{0.3\linewidth}}{
        \cellcolor[rgb]{0.9,0.9,0.9}{
            \makecell[{{p{\linewidth}}}]{
                \texttt{\tiny{[GM$|$GM]}}
                \texttt{MOVE: D, (4,5)} \\
            }
        }
    }
    & & \\ \\

    \theutterance \stepcounter{utterance}  
    & & & \multicolumn{2}{p{0.3\linewidth}}{
        \cellcolor[rgb]{0.9,0.9,0.9}{
            \makecell[{{p{\linewidth}}}]{
                \texttt{\tiny{[GM$|$GM]}}
                \texttt{Penalty: (4,5) is not empty, but contains '┼'.} \\
\\ 
\texttt{You have collectively accumulated 10 of 16 penalties.} \\
\texttt{Please try again!} \\
            }
        }
    }
    & & \\ \\

    \theutterance \stepcounter{utterance}  
    & & \multicolumn{4}{p{0.6\linewidth}}{
        \cellcolor[rgb]{0.9,0.9,0.9}{
            \makecell[{{p{\linewidth}}}]{
                \texttt{\tiny{[GM$\rangle$P2]}}
                \texttt{Penalty: (4,5) is not empty, but contains '┼'.} \\
\\ 
\texttt{You have collectively accumulated 10 of 16 penalties.} \\
\texttt{Please try again!} \\
            }
        }
    }
    & \\ \\

    \theutterance \stepcounter{utterance}  
    & & & \multicolumn{4}{p{0.6\linewidth}}{
        \cellcolor[rgb]{0.9,0.9,0.9}{
            \makecell[{{p{\linewidth}}}]{
                \texttt{\tiny{[GM$\langle$P2]}}
                \texttt{SAY: finished?} \\
            }
        }
    }
     \\ \\

    \theutterance \stepcounter{utterance}  
    & & & \multicolumn{2}{p{0.3\linewidth}}{
        \cellcolor[rgb]{0.9,0.9,0.9}{
            \makecell[{{p{\linewidth}}}]{
                \texttt{\tiny{[GM$|$GM]}}
                \texttt{SAY: finished?} \\
            }
        }
    }
    & & \\ \\

    \theutterance \stepcounter{utterance}  
    & & & \multicolumn{2}{p{0.3\linewidth}}{
        \cellcolor[rgb]{0.9,0.9,0.9}{
            \makecell[{{p{\linewidth}}}]{
                \texttt{\tiny{[GM$|$GM]}}
                \texttt{Maximum number of rounds reached} \\
            }
        }
    }
    & & \\ \\

    \theutterance \stepcounter{utterance}  
    & & & \multicolumn{2}{p{0.3\linewidth}}{
        \cellcolor[rgb]{0.9,0.9,0.9}{
            \makecell[{{p{\linewidth}}}]{
                \texttt{\tiny{[GM$|$GM]}}
                \texttt{* success: True} \\
\texttt{* lose: False} \\
\texttt{* aborted: False} \\
\texttt{{-}{-}{-}{-}{-}{-}{-}} \\
\texttt{* Shifts: 6.00} \\
\texttt{* Max Shifts: 12.00} \\
\texttt{* Min Shifts: 6.00} \\
\texttt{* End Distance Sum: 16.07} \\
\texttt{* Init Distance Sum: 30.19} \\
\texttt{* Expected Distance Sum: 29.33} \\
\texttt{* Penalties: 10.00} \\
\texttt{* Max Penalties: 16.00} \\
\texttt{* Rounds: 28.00} \\
\texttt{* Max Rounds: 28.00} \\
\texttt{* Object Count: 7.00} \\
            }
        }
    }
    & & \\ \\

    \theutterance \stepcounter{utterance}  
    & & & \multicolumn{2}{p{0.3\linewidth}}{
        \cellcolor[rgb]{0.9,0.9,0.9}{
            \makecell[{{p{\linewidth}}}]{
                \texttt{\tiny{[GM$|$GM]}}
                \texttt{Initial Grids:} \\
\texttt{```} \\
\texttt{Player 1:     Player 2:} \\
\texttt{ 1234567       1234567} \\
\texttt{╔═══╤═╤═╗     ╔═══╤═╤═╗} \\
\texttt{║◌◌◌│◌│◌║ 1   ║◌MS│◌│◌║ 1} \\
\texttt{║◌◌M│◌├─╢ 2   ║◌O◌│◌├─╢ 2} \\
\texttt{║AT◌├─┘◌║ 3   ║◌◌◌├─┘◌║ 3} \\
\texttt{║◌◌P│◌◌◌║ 4   ║◌◌◌│◌◌◌║ 4} \\
\texttt{╟─┬─┼───╢ 5   ╟─┬─┼───╢ 5} \\
\texttt{║◌│◌│S◌O║ 6   ║◌│◌│◌◌D║ 6} \\
\texttt{║◌│◌│◌D◌║ 7   ║T│◌│◌PA║ 7} \\
\texttt{╚═╧═╧═══╝     ╚═╧═╧═══╝} \\
\texttt{```} \\
\\ 
\texttt{Final Grids:} \\
\texttt{```} \\
\texttt{Player 1:     Player 2:} \\
\texttt{ 1234567       1234567} \\
\texttt{╔═══╤═╤═╗     ╔═══╤═╤═╗} \\
\texttt{║◌◌P│◌│◌║ 1   ║◌◌◌│S│◌║ 1} \\
\texttt{║◌O◌│◌├─╢ 2   ║◌O◌│◌├─╢ 2} \\
\texttt{║◌T◌├─┘◌║ 3   ║◌◌◌├─┘◌║ 3} \\
\texttt{║◌◌◌│S◌◌║ 4   ║◌◌◌│◌◌◌║ 4} \\
\texttt{╟─┬─┼───╢ 5   ╟─┬─┼───╢ 5} \\
\texttt{║A│◌│◌◌◌║ 6   ║A│◌│◌◌D║ 6} \\
\texttt{║◌│◌│D◌M║ 7   ║T│◌│◌PM║ 7} \\
\texttt{╚═╧═╧═══╝     ╚═╧═╧═══╝} \\
\texttt{```} \\
            }
        }
    }
    & & \\ \\

\end{supertabular}
}

\end{document}
