
\documentclass{article}
\usepackage{colortbl}
\usepackage{makecell}
\usepackage{multirow}
\usepackage{supertabular}

\begin{document}

\newcounter{utterance}

\centering \large Interaction Transcript for game `clean\_up', experiment `1\_medium\_7obj\_en', episode 0 with llama{-}3.3{-}70b{-}instruct{-}t1.0.
\vspace{24pt}

{ \footnotesize  \setcounter{utterance}{1}
\setlength{\tabcolsep}{0pt}
\begin{supertabular}{c@{$\;$}|p{.15\linewidth}@{}p{.15\linewidth}p{.15\linewidth}p{.15\linewidth}p{.15\linewidth}p{.15\linewidth}}
    \# & $\;$A & \multicolumn{4}{c}{Game Master} & $\;\:$B\\
    \hline

    \theutterance \stepcounter{utterance}  
    & & \multicolumn{4}{p{0.6\linewidth}}{
        \cellcolor[rgb]{0.9,0.9,0.9}{
            \makecell[{{p{\linewidth}}}]{
                \texttt{\tiny{[P1$\langle$GM]}}
                \texttt{I am your game master, and you are playing a collaborative game with the following grid as game board:} \\
\\ 
\texttt{```} \\
\texttt{ 1234567} \\
\texttt{╔═╤═╤═╤═╗} \\
\texttt{║◌│◌│◌│◌║ 1} \\
\texttt{║P└─┼─┴─╢ 2} \\
\texttt{║◌SA│◌◌◌║ 3} \\
\texttt{╟──┬┴───╢ 4} \\
\texttt{║◌◌│◌◌O◌║ 5} \\
\texttt{║◌◌│◌◌MD║ 6} \\
\texttt{║◌◌│◌◌◌T║ 7} \\
\texttt{╚══╧════╝} \\
\\ 
\texttt{```} \\
\\ 
\texttt{* The upper edge displays x{-}coordinates increasing to the right, and the right edge y{-}coordinates increasing downward.} \\
\texttt{* The following objects are randomly placed on your grid: 'P', 'O', 'T', 'S', 'D', 'A', 'M'.} \\
\\ 
\\ 
\texttt{The other player sees a variation of the game board, where the objects are placed at different random locations. You cannot see the other player's board, and they cannot see yours.} \\
\\ 
\\ 
\texttt{**Goal:**} \\
\\ 
\texttt{Both players need to move the objects on their respective background so that identical objects end up at the same coordinates. You have to communicate with the other player to agree upon a common goal configuration.} \\
\\ 
\\ 
\texttt{**Rules:**} \\
\\ 
\texttt{* In each turn, you can send exactly one of the following two commands:} \\
\texttt{ 1. `SAY: <MESSAGE>`: to send a message (everything up until the next line break) to the other player. I will forward it to your partner.} \\
\texttt{2. `MOVE: <OBJECT>, (<X>, <Y>)`: to move an object to a new position, where `<X>` is the column and `<Y>` is the row. I will inform you if your move was successful or not.} \\
\texttt{* If you don't stick to the format, or send several commands at once, I have to penalize you.} \\
\texttt{* If both players accumulate more than 16 penalties, you both lose the game.} \\
\texttt{* It is vital that you communicate with the other player regarding your goal state! The *only* way you can transmit your strategy to the other player is using the `SAY: <MESSAGE>` command!} \\
\\ 
\\ 
\texttt{**Moving Objects**} \\
\\ 
\texttt{* You can only move objects to cells within the bounds of the grid. The target cell must be empty, i.e., it must only contain the symbol '◌'.} \\
\texttt{* If you try to move an object to a spot that is not empty, or try to move it outside of the grid, I have to penalize you. You get another try.} \\
\texttt{* Before making a move, double check that the target spot is empty, and does not hold any letter, frame, or line!} \\
\\ 
\\ 
\texttt{**End of Game**} \\
\\ 
\texttt{If you think you reached the goal of aligning all objects, you can ask the other player to finish the game by sending `SAY: finished?`. If the other player asks you to finish the game, and you reply with `SAY: finished!`, the game will end.} \\
\\ 
\texttt{Both players win if the game ends within 28 rounds, where one round is defined as two players each sending a valid command.} \\
\\ 
\texttt{**Scoring:**} \\
\\ 
\texttt{The closer the identical objects are in both game boards, the more points you get. Penalties reduce your points. Can you beat the record?} \\
\\ 
\\ 
\texttt{Please send a message to the other player to start the game!} \\
            }
        }
    }
    & \\ \\

    \theutterance \stepcounter{utterance}  
    & \multicolumn{4}{p{0.6\linewidth}}{
        \cellcolor[rgb]{0.9,0.9,0.9}{
            \makecell[{{p{\linewidth}}}]{
                \texttt{\tiny{[P1$\rangle$GM]}}
                \texttt{SAY: Hello, let's start by describing our boards. I'll send you a description of my board, and you send me a description of yours. We can then discuss a strategy to align the objects. My board has the following objects: 'P' at (1,2), 'O' at (4,5), 'T' at (4,7), 'S' at (1,3), 'D' at (4,6), 'A' at (2,3), 'M' at (3,6). Please describe your board.} \\
            }
        }
    }
    & & \\ \\

    \theutterance \stepcounter{utterance}  
    & & & \multicolumn{2}{p{0.3\linewidth}}{
        \cellcolor[rgb]{0.9,0.9,0.9}{
            \makecell[{{p{\linewidth}}}]{
                \texttt{\tiny{[GM$|$GM]}}
                \texttt{SAY: Hello, let's start by describing our boards. I'll send you a description of my board, and you send me a description of yours. We can then discuss a strategy to align the objects. My board has the following objects: 'P' at (1,2), 'O' at (4,5), 'T' at (4,7), 'S' at (1,3), 'D' at (4,6), 'A' at (2,3), 'M' at (3,6). Please describe your board.} \\
            }
        }
    }
    & & \\ \\

    \theutterance \stepcounter{utterance}  
    & & \multicolumn{4}{p{0.6\linewidth}}{
        \cellcolor[rgb]{0.9,0.9,0.9}{
            \makecell[{{p{\linewidth}}}]{
                \texttt{\tiny{[GM$\rangle$P2]}}
                \texttt{I am your game master, and you are playing a collaborative game with the following grid as game board:} \\
\\ 
\texttt{```} \\
\texttt{ 1234567} \\
\texttt{╔═╤═╤═╤═╗} \\
\texttt{║◌│◌│◌│P║ 1} \\
\texttt{║T└─┼─┴─╢ 2} \\
\texttt{║◌◌◌│M◌◌║ 3} \\
\texttt{╟──┬┴───╢ 4} \\
\texttt{║◌D│◌◌O◌║ 5} \\
\texttt{║◌◌│◌S◌◌║ 6} \\
\texttt{║A◌│◌◌◌◌║ 7} \\
\texttt{╚══╧════╝} \\
\\ 
\texttt{```} \\
\\ 
\texttt{* The upper edge displays x{-}coordinates increasing to the right, and the right edge y{-}coordinates increasing downward.} \\
\texttt{* The following objects are randomly placed on your grid: 'P', 'O', 'T', 'S', 'D', 'A', 'M'.} \\
\\ 
\\ 
\texttt{The other player sees a variation of the game board, where the objects are placed at different random locations. You cannot see the other player's board, and they cannot see yours.} \\
\\ 
\\ 
\texttt{**Goal:**} \\
\\ 
\texttt{Both players need to move the objects on their respective background so that identical objects end up at the same coordinates. You have to communicate with the other player to agree upon a common goal configuration.} \\
\\ 
\\ 
\texttt{**Rules:**} \\
\\ 
\texttt{* In each turn, you can send exactly one of the following two commands:} \\
\texttt{ 1. `SAY: <MESSAGE>`: to send a message (everything up until the next line break) to the other player. I will forward it to your partner.} \\
\texttt{2. `MOVE: <OBJECT>, (<X>, <Y>)`: to move an object to a new position, where `<X>` is the column and `<Y>` is the row. I will inform you if your move was successful or not.} \\
\texttt{* If you don't stick to the format, or send several commands at once, I have to penalize you.} \\
\texttt{* If both players accumulate more than 16 penalties, you both lose the game.} \\
\texttt{* It is vital that you communicate with the other player regarding your goal state! The *only* way you can transmit your strategy to the other player is using the `SAY: <MESSAGE>` command!} \\
\\ 
\\ 
\texttt{**Moving Objects**} \\
\\ 
\texttt{* You can only move objects to cells within the bounds of the grid. The target cell must be empty, i.e., it must only contain the symbol '◌'.} \\
\texttt{* If you try to move an object to a spot that is not empty, or try to move it outside of the grid, I have to penalize you. You get another try.} \\
\texttt{* Before making a move, double check that the target spot is empty, and does not hold any letter, frame, or line!} \\
\\ 
\\ 
\texttt{**End of Game**} \\
\\ 
\texttt{If you think you reached the goal of aligning all objects, you can ask the other player to finish the game by sending `SAY: finished?`. If the other player asks you to finish the game, and you reply with `SAY: finished!`, the game will end.} \\
\\ 
\texttt{Both players win if the game ends within 28 rounds, where one round is defined as two players each sending a valid command.} \\
\\ 
\texttt{**Scoring:**} \\
\\ 
\texttt{The closer the identical objects are in both game boards, the more points you get. Penalties reduce your points. Can you beat the record?} \\
\\ 
\texttt{The other player started the game by sending this message:} \\
\\ 
\texttt{"Hello, let's start by describing our boards. I'll send you a description of my board, and you send me a description of yours. We can then discuss a strategy to align the objects. My board has the following objects: 'P' at (1,2), 'O' at (4,5), 'T' at (4,7), 'S' at (1,3), 'D' at (4,6), 'A' at (2,3), 'M' at (3,6). Please describe your board."} \\
\\ 
\texttt{What is your first command?} \\
            }
        }
    }
    & \\ \\

    \theutterance \stepcounter{utterance}  
    & & & \multicolumn{4}{p{0.6\linewidth}}{
        \cellcolor[rgb]{0.9,0.9,0.9}{
            \makecell[{{p{\linewidth}}}]{
                \texttt{\tiny{[GM$\langle$P2]}}
                \texttt{Since the other player has sent a message describing their board and asking for a description of my board, my first command will be to send a description of my board.} \\
\\ 
\texttt{SAY: Hello, thanks for starting. My board has the following objects: 'P' at (4,1), 'O' at (5,5), 'T' at (1,2), 'S' at (5,6), 'D' at (2,5), 'A' at (1,7), 'M' at (3,3). Let's discuss a strategy to align the objects.} \\
            }
        }
    }
     \\ \\

    \theutterance \stepcounter{utterance}  
    & & & \multicolumn{2}{p{0.3\linewidth}}{
        \cellcolor[rgb]{0.9,0.9,0.9}{
            \makecell[{{p{\linewidth}}}]{
                \texttt{\tiny{[GM$|$GM]}}
                \texttt{Since the other player has sent a message describing their board and asking for a description of my board, my first command will be to send a description of my board.} \\
\\ 
\texttt{SAY: Hello, thanks for starting. My board has the following objects: 'P' at (4,1), 'O' at (5,5), 'T' at (1,2), 'S' at (5,6), 'D' at (2,5), 'A' at (1,7), 'M' at (3,3). Let's discuss a strategy to align the objects.} \\
            }
        }
    }
    & & \\ \\

    \theutterance \stepcounter{utterance}  
    & & & \multicolumn{2}{p{0.3\linewidth}}{
        \cellcolor[rgb]{0.9,0.9,0.9}{
            \makecell[{{p{\linewidth}}}]{
                \texttt{\tiny{[GM$|$GM]}}
                \texttt{Invalid format: head is not empty: 'Since the other player has sent a message describing their board and asking for a description of my board, my first command will be to send a description of my board.} \\
\\ 
\texttt{'} \\
            }
        }
    }
    & & \\ \\

    \theutterance \stepcounter{utterance}  
    & & \multicolumn{4}{p{0.6\linewidth}}{
        \cellcolor[rgb]{0.9,0.9,0.9}{
            \makecell[{{p{\linewidth}}}]{
                \texttt{\tiny{[GM$\rangle$P2]}}
                \texttt{Penalty: Your message must not contain anything before the command!} \\
\\ 
\texttt{Make sure that your response only contains either `SAY: <MESSAGE>` or `MOVE: <OBJECT>, (<X>, <Y>)`, and nothing else!} \\
\\ 
\texttt{You have collectively accumulated 1 of 16 penalties.} \\
\texttt{Please try again!} \\
            }
        }
    }
    & \\ \\

    \theutterance \stepcounter{utterance}  
    & & & \multicolumn{4}{p{0.6\linewidth}}{
        \cellcolor[rgb]{0.9,0.9,0.9}{
            \makecell[{{p{\linewidth}}}]{
                \texttt{\tiny{[GM$\langle$P2]}}
                \texttt{SAY: Hello, thanks for starting. My board has the following objects: 'P' at (4,1), 'O' at (5,5), 'T' at (1,2), 'S' at (5,6), 'D' at (2,5), 'A' at (1,7), 'M' at (3,3). Let's discuss a strategy to align the objects.} \\
            }
        }
    }
     \\ \\

    \theutterance \stepcounter{utterance}  
    & & & \multicolumn{2}{p{0.3\linewidth}}{
        \cellcolor[rgb]{0.9,0.9,0.9}{
            \makecell[{{p{\linewidth}}}]{
                \texttt{\tiny{[GM$|$GM]}}
                \texttt{SAY: Hello, thanks for starting. My board has the following objects: 'P' at (4,1), 'O' at (5,5), 'T' at (1,2), 'S' at (5,6), 'D' at (2,5), 'A' at (1,7), 'M' at (3,3). Let's discuss a strategy to align the objects.} \\
            }
        }
    }
    & & \\ \\

    \theutterance \stepcounter{utterance}  
    & & \multicolumn{4}{p{0.6\linewidth}}{
        \cellcolor[rgb]{0.9,0.9,0.9}{
            \makecell[{{p{\linewidth}}}]{
                \texttt{\tiny{[P1$\langle$GM]}}
                \texttt{Your message has been relayed to the other player.} \\
\\ 
\\ 
\texttt{You are currently playing round 1 of maximum 28.} \\
\texttt{You have collectively accumulated 1 of 16 penalties.} \\
\texttt{The other player sent this message:} \\
\\ 
\\ 
\texttt{"Hello, thanks for starting. My board has the following objects: 'P' at (4,1), 'O' at (5,5), 'T' at (1,2), 'S' at (5,6), 'D' at (2,5), 'A' at (1,7), 'M' at (3,3). Let's discuss a strategy to align the objects."} \\
\\ 
\\ 
\texttt{What is your next command?} \\
            }
        }
    }
    & \\ \\

    \theutterance \stepcounter{utterance}  
    & \multicolumn{4}{p{0.6\linewidth}}{
        \cellcolor[rgb]{0.9,0.9,0.9}{
            \makecell[{{p{\linewidth}}}]{
                \texttt{\tiny{[P1$\rangle$GM]}}
                \texttt{SAY: Thanks for the description. It seems like our boards are quite different. To start aligning the objects, let's focus on one object at a time. How about we start with 'P'? On my board, 'P' is at (1,2). On your board, 'P' is at (4,1). Do you think we should try to move 'P' to a specific coordinate, like (1,1) or (4,4), or do you have a different suggestion?} \\
            }
        }
    }
    & & \\ \\

    \theutterance \stepcounter{utterance}  
    & & & \multicolumn{2}{p{0.3\linewidth}}{
        \cellcolor[rgb]{0.9,0.9,0.9}{
            \makecell[{{p{\linewidth}}}]{
                \texttt{\tiny{[GM$|$GM]}}
                \texttt{SAY: Thanks for the description. It seems like our boards are quite different. To start aligning the objects, let's focus on one object at a time. How about we start with 'P'? On my board, 'P' is at (1,2). On your board, 'P' is at (4,1). Do you think we should try to move 'P' to a specific coordinate, like (1,1) or (4,4), or do you have a different suggestion?} \\
            }
        }
    }
    & & \\ \\

    \theutterance \stepcounter{utterance}  
    & & \multicolumn{4}{p{0.6\linewidth}}{
        \cellcolor[rgb]{0.9,0.9,0.9}{
            \makecell[{{p{\linewidth}}}]{
                \texttt{\tiny{[GM$\rangle$P2]}}
                \texttt{Your message has been relayed to the other player.} \\
\\ 
\\ 
\texttt{You are currently playing round 2 of maximum 28.} \\
\texttt{You have collectively accumulated 1 of 16 penalties.} \\
\texttt{The other player sent this message:} \\
\\ 
\\ 
\texttt{"Thanks for the description. It seems like our boards are quite different. To start aligning the objects, let's focus on one object at a time. How about we start with 'P'? On my board, 'P' is at (1,2). On your board, 'P' is at (4,1). Do you think we should try to move 'P' to a specific coordinate, like (1,1) or (4,4), or do you have a different suggestion?"} \\
\\ 
\\ 
\texttt{What is your next command?} \\
            }
        }
    }
    & \\ \\

    \theutterance \stepcounter{utterance}  
    & & & \multicolumn{4}{p{0.6\linewidth}}{
        \cellcolor[rgb]{0.9,0.9,0.9}{
            \makecell[{{p{\linewidth}}}]{
                \texttt{\tiny{[GM$\langle$P2]}}
                \texttt{SAY: I think moving 'P' to (1,1) could be a good starting point, as it's a corner and gives us a reference point for other objects. What do you think about moving 'P' to (1,1) on both boards, and then discuss the next object to align?} \\
            }
        }
    }
     \\ \\

    \theutterance \stepcounter{utterance}  
    & & & \multicolumn{2}{p{0.3\linewidth}}{
        \cellcolor[rgb]{0.9,0.9,0.9}{
            \makecell[{{p{\linewidth}}}]{
                \texttt{\tiny{[GM$|$GM]}}
                \texttt{SAY: I think moving 'P' to (1,1) could be a good starting point, as it's a corner and gives us a reference point for other objects. What do you think about moving 'P' to (1,1) on both boards, and then discuss the next object to align?} \\
            }
        }
    }
    & & \\ \\

    \theutterance \stepcounter{utterance}  
    & & \multicolumn{4}{p{0.6\linewidth}}{
        \cellcolor[rgb]{0.9,0.9,0.9}{
            \makecell[{{p{\linewidth}}}]{
                \texttt{\tiny{[P1$\langle$GM]}}
                \texttt{Your message has been relayed to the other player.} \\
\\ 
\\ 
\texttt{You are currently playing round 2 of maximum 28.} \\
\texttt{You have collectively accumulated 1 of 16 penalties.} \\
\texttt{The other player sent this message:} \\
\\ 
\\ 
\texttt{"I think moving 'P' to (1,1) could be a good starting point, as it's a corner and gives us a reference point for other objects. What do you think about moving 'P' to (1,1) on both boards, and then discuss the next object to align?"} \\
\\ 
\\ 
\texttt{What is your next command?} \\
            }
        }
    }
    & \\ \\

    \theutterance \stepcounter{utterance}  
    & \multicolumn{4}{p{0.6\linewidth}}{
        \cellcolor[rgb]{0.9,0.9,0.9}{
            \makecell[{{p{\linewidth}}}]{
                \texttt{\tiny{[P1$\rangle$GM]}}
                \texttt{SAY: Moving 'P' to (1,1) sounds like a great plan. I'll try to move 'P' to (1,1) on my board. Before I do that, I just want to confirm that we're on the same page: if the move is successful, 'P' will be at (1,1) on both boards, right? Also, after 'P' is moved, we can discuss the next object to align. Should we prioritize objects that are currently close to each other or try to move objects that are farthest apart?} \\
            }
        }
    }
    & & \\ \\

    \theutterance \stepcounter{utterance}  
    & & & \multicolumn{2}{p{0.3\linewidth}}{
        \cellcolor[rgb]{0.9,0.9,0.9}{
            \makecell[{{p{\linewidth}}}]{
                \texttt{\tiny{[GM$|$GM]}}
                \texttt{SAY: Moving 'P' to (1,1) sounds like a great plan. I'll try to move 'P' to (1,1) on my board. Before I do that, I just want to confirm that we're on the same page: if the move is successful, 'P' will be at (1,1) on both boards, right? Also, after 'P' is moved, we can discuss the next object to align. Should we prioritize objects that are currently close to each other or try to move objects that are farthest apart?} \\
            }
        }
    }
    & & \\ \\

    \theutterance \stepcounter{utterance}  
    & & \multicolumn{4}{p{0.6\linewidth}}{
        \cellcolor[rgb]{0.9,0.9,0.9}{
            \makecell[{{p{\linewidth}}}]{
                \texttt{\tiny{[GM$\rangle$P2]}}
                \texttt{Your message has been relayed to the other player.} \\
\\ 
\\ 
\texttt{You are currently playing round 3 of maximum 28.} \\
\texttt{You have collectively accumulated 1 of 16 penalties.} \\
\texttt{The other player sent this message:} \\
\\ 
\\ 
\texttt{"Moving 'P' to (1,1) sounds like a great plan. I'll try to move 'P' to (1,1) on my board. Before I do that, I just want to confirm that we're on the same page: if the move is successful, 'P' will be at (1,1) on both boards, right? Also, after 'P' is moved, we can discuss the next object to align. Should we prioritize objects that are currently close to each other or try to move objects that are farthest apart?"} \\
\\ 
\\ 
\texttt{What is your next command?} \\
            }
        }
    }
    & \\ \\

    \theutterance \stepcounter{utterance}  
    & & & \multicolumn{4}{p{0.6\linewidth}}{
        \cellcolor[rgb]{0.9,0.9,0.9}{
            \makecell[{{p{\linewidth}}}]{
                \texttt{\tiny{[GM$\langle$P2]}}
                \texttt{SAY: Yes, that's correct, if the move is successful, 'P' will be at (1,1) on both boards. After 'P' is moved, I think we should prioritize objects that are currently farthest apart, as it will give us a better overall sense of the board's layout and make it easier to align the remaining objects. Let's wait for your confirmation that 'P' has been moved to (1,1) on your board, and then we can discuss the next steps.} \\
            }
        }
    }
     \\ \\

    \theutterance \stepcounter{utterance}  
    & & & \multicolumn{2}{p{0.3\linewidth}}{
        \cellcolor[rgb]{0.9,0.9,0.9}{
            \makecell[{{p{\linewidth}}}]{
                \texttt{\tiny{[GM$|$GM]}}
                \texttt{SAY: Yes, that's correct, if the move is successful, 'P' will be at (1,1) on both boards. After 'P' is moved, I think we should prioritize objects that are currently farthest apart, as it will give us a better overall sense of the board's layout and make it easier to align the remaining objects. Let's wait for your confirmation that 'P' has been moved to (1,1) on your board, and then we can discuss the next steps.} \\
            }
        }
    }
    & & \\ \\

    \theutterance \stepcounter{utterance}  
    & & \multicolumn{4}{p{0.6\linewidth}}{
        \cellcolor[rgb]{0.9,0.9,0.9}{
            \makecell[{{p{\linewidth}}}]{
                \texttt{\tiny{[P1$\langle$GM]}}
                \texttt{Your message has been relayed to the other player.} \\
\\ 
\\ 
\texttt{You are currently playing round 3 of maximum 28.} \\
\texttt{You have collectively accumulated 1 of 16 penalties.} \\
\texttt{The other player sent this message:} \\
\\ 
\\ 
\texttt{"Yes, that's correct, if the move is successful, 'P' will be at (1,1) on both boards. After 'P' is moved, I think we should prioritize objects that are currently farthest apart, as it will give us a better overall sense of the board's layout and make it easier to align the remaining objects. Let's wait for your confirmation that 'P' has been moved to (1,1) on your board, and then we can discuss the next steps."} \\
\\ 
\\ 
\texttt{What is your next command?} \\
            }
        }
    }
    & \\ \\

    \theutterance \stepcounter{utterance}  
    & \multicolumn{4}{p{0.6\linewidth}}{
        \cellcolor[rgb]{0.9,0.9,0.9}{
            \makecell[{{p{\linewidth}}}]{
                \texttt{\tiny{[P1$\rangle$GM]}}
                \texttt{MOVE: P, (1, 1)} \\
            }
        }
    }
    & & \\ \\

    \theutterance \stepcounter{utterance}  
    & & & \multicolumn{2}{p{0.3\linewidth}}{
        \cellcolor[rgb]{0.9,0.9,0.9}{
            \makecell[{{p{\linewidth}}}]{
                \texttt{\tiny{[GM$|$GM]}}
                \texttt{MOVE: P, (1, 1)} \\
            }
        }
    }
    & & \\ \\

    \theutterance \stepcounter{utterance}  
    & & & \multicolumn{2}{p{0.3\linewidth}}{
        \cellcolor[rgb]{0.9,0.9,0.9}{
            \makecell[{{p{\linewidth}}}]{
                \texttt{\tiny{[GM$|$GM]}}
                \texttt{Moved 'P' to (1,1) successfully. Your updated grid looks like this:} \\
\texttt{```} \\
\texttt{ 1234567} \\
\texttt{╔═╤═╤═╤═╗} \\
\texttt{║P│◌│◌│◌║ 1} \\
\texttt{║◌└─┼─┴─╢ 2} \\
\texttt{║◌SA│◌◌◌║ 3} \\
\texttt{╟──┬┴───╢ 4} \\
\texttt{║◌◌│◌◌O◌║ 5} \\
\texttt{║◌◌│◌◌MD║ 6} \\
\texttt{║◌◌│◌◌◌T║ 7} \\
\texttt{╚══╧════╝} \\
\texttt{```} \\
            }
        }
    }
    & & \\ \\

    \theutterance \stepcounter{utterance}  
    & & \multicolumn{4}{p{0.6\linewidth}}{
        \cellcolor[rgb]{0.9,0.9,0.9}{
            \makecell[{{p{\linewidth}}}]{
                \texttt{\tiny{[GM$\rangle$P2]}}
                \texttt{Your message has been relayed to the other player.} \\
\\ 
\\ 
\texttt{You are currently playing round 4 of maximum 28.} \\
\texttt{You have collectively accumulated 1 of 16 penalties.} \\
\texttt{The other player moved an object on their grid.} \\
\\ 
\\ 
\texttt{What is your next command?} \\
            }
        }
    }
    & \\ \\

    \theutterance \stepcounter{utterance}  
    & & & \multicolumn{4}{p{0.6\linewidth}}{
        \cellcolor[rgb]{0.9,0.9,0.9}{
            \makecell[{{p{\linewidth}}}]{
                \texttt{\tiny{[GM$\langle$P2]}}
                \texttt{MOVE: P, (1,1)} \\
            }
        }
    }
     \\ \\

    \theutterance \stepcounter{utterance}  
    & & & \multicolumn{2}{p{0.3\linewidth}}{
        \cellcolor[rgb]{0.9,0.9,0.9}{
            \makecell[{{p{\linewidth}}}]{
                \texttt{\tiny{[GM$|$GM]}}
                \texttt{MOVE: P, (1,1)} \\
            }
        }
    }
    & & \\ \\

    \theutterance \stepcounter{utterance}  
    & & & \multicolumn{2}{p{0.3\linewidth}}{
        \cellcolor[rgb]{0.9,0.9,0.9}{
            \makecell[{{p{\linewidth}}}]{
                \texttt{\tiny{[GM$|$GM]}}
                \texttt{Moved 'P' to (1,1) successfully. Your updated grid looks like this:} \\
\texttt{```} \\
\texttt{ 1234567} \\
\texttt{╔═╤═╤═╤═╗} \\
\texttt{║P│◌│◌│◌║ 1} \\
\texttt{║T└─┼─┴─╢ 2} \\
\texttt{║◌◌◌│M◌◌║ 3} \\
\texttt{╟──┬┴───╢ 4} \\
\texttt{║◌D│◌◌O◌║ 5} \\
\texttt{║◌◌│◌S◌◌║ 6} \\
\texttt{║A◌│◌◌◌◌║ 7} \\
\texttt{╚══╧════╝} \\
\texttt{```} \\
            }
        }
    }
    & & \\ \\

    \theutterance \stepcounter{utterance}  
    & & \multicolumn{4}{p{0.6\linewidth}}{
        \cellcolor[rgb]{0.9,0.9,0.9}{
            \makecell[{{p{\linewidth}}}]{
                \texttt{\tiny{[P1$\langle$GM]}}
                \texttt{Moved 'P' to (1,1) successfully. Your updated grid looks like this:} \\
\texttt{```} \\
\texttt{ 1234567} \\
\texttt{╔═╤═╤═╤═╗} \\
\texttt{║P│◌│◌│◌║ 1} \\
\texttt{║◌└─┼─┴─╢ 2} \\
\texttt{║◌SA│◌◌◌║ 3} \\
\texttt{╟──┬┴───╢ 4} \\
\texttt{║◌◌│◌◌O◌║ 5} \\
\texttt{║◌◌│◌◌MD║ 6} \\
\texttt{║◌◌│◌◌◌T║ 7} \\
\texttt{╚══╧════╝} \\
\texttt{```} \\
\texttt{You are currently playing round 4 of maximum 28.} \\
\texttt{You have collectively accumulated 1 of 16 penalties.} \\
\texttt{The other player moved an object on their grid.} \\
\\ 
\\ 
\texttt{What is your next command?} \\
            }
        }
    }
    & \\ \\

    \theutterance \stepcounter{utterance}  
    & \multicolumn{4}{p{0.6\linewidth}}{
        \cellcolor[rgb]{0.9,0.9,0.9}{
            \makecell[{{p{\linewidth}}}]{
                \texttt{\tiny{[P1$\rangle$GM]}}
                \texttt{SAY: I've successfully moved 'P' to (1,1) on my board. Now that 'P' is in place, let's discuss the next object to move. As we agreed, we should prioritize objects that are currently farthest apart. On my board, 'T' is at (4,7) and on your board, 'T' was at (1,2). That's a big distance. Should we try to move 'T' to a more central location, like (4,4), or do you have a different suggestion?} \\
            }
        }
    }
    & & \\ \\

    \theutterance \stepcounter{utterance}  
    & & & \multicolumn{2}{p{0.3\linewidth}}{
        \cellcolor[rgb]{0.9,0.9,0.9}{
            \makecell[{{p{\linewidth}}}]{
                \texttt{\tiny{[GM$|$GM]}}
                \texttt{SAY: I've successfully moved 'P' to (1,1) on my board. Now that 'P' is in place, let's discuss the next object to move. As we agreed, we should prioritize objects that are currently farthest apart. On my board, 'T' is at (4,7) and on your board, 'T' was at (1,2). That's a big distance. Should we try to move 'T' to a more central location, like (4,4), or do you have a different suggestion?} \\
            }
        }
    }
    & & \\ \\

    \theutterance \stepcounter{utterance}  
    & & \multicolumn{4}{p{0.6\linewidth}}{
        \cellcolor[rgb]{0.9,0.9,0.9}{
            \makecell[{{p{\linewidth}}}]{
                \texttt{\tiny{[GM$\rangle$P2]}}
                \texttt{Moved 'P' to (1,1) successfully. Your updated grid looks like this:} \\
\texttt{```} \\
\texttt{ 1234567} \\
\texttt{╔═╤═╤═╤═╗} \\
\texttt{║P│◌│◌│◌║ 1} \\
\texttt{║T└─┼─┴─╢ 2} \\
\texttt{║◌◌◌│M◌◌║ 3} \\
\texttt{╟──┬┴───╢ 4} \\
\texttt{║◌D│◌◌O◌║ 5} \\
\texttt{║◌◌│◌S◌◌║ 6} \\
\texttt{║A◌│◌◌◌◌║ 7} \\
\texttt{╚══╧════╝} \\
\texttt{```} \\
\texttt{You are currently playing round 5 of maximum 28.} \\
\texttt{You have collectively accumulated 1 of 16 penalties.} \\
\texttt{The other player sent this message:} \\
\\ 
\\ 
\texttt{"I've successfully moved 'P' to (1,1) on my board. Now that 'P' is in place, let's discuss the next object to move. As we agreed, we should prioritize objects that are currently farthest apart. On my board, 'T' is at (4,7) and on your board, 'T' was at (1,2). That's a big distance. Should we try to move 'T' to a more central location, like (4,4), or do you have a different suggestion?"} \\
\\ 
\\ 
\texttt{What is your next command?} \\
            }
        }
    }
    & \\ \\

    \theutterance \stepcounter{utterance}  
    & & & \multicolumn{4}{p{0.6\linewidth}}{
        \cellcolor[rgb]{0.9,0.9,0.9}{
            \makecell[{{p{\linewidth}}}]{
                \texttt{\tiny{[GM$\langle$P2]}}
                \texttt{SAY: Moving 'T' to a central location like (4,4) sounds like a good idea. Before we do that, I'd like to confirm that we're considering the distance between the objects on both boards. Since 'T' is at (1,2) on my board, moving it to (4,4) would be a significant move. What if we aim to move 'T' to (3,4) instead, as it might be a more manageable step and bring the 'T's closer together on both boards?} \\
            }
        }
    }
     \\ \\

    \theutterance \stepcounter{utterance}  
    & & & \multicolumn{2}{p{0.3\linewidth}}{
        \cellcolor[rgb]{0.9,0.9,0.9}{
            \makecell[{{p{\linewidth}}}]{
                \texttt{\tiny{[GM$|$GM]}}
                \texttt{SAY: Moving 'T' to a central location like (4,4) sounds like a good idea. Before we do that, I'd like to confirm that we're considering the distance between the objects on both boards. Since 'T' is at (1,2) on my board, moving it to (4,4) would be a significant move. What if we aim to move 'T' to (3,4) instead, as it might be a more manageable step and bring the 'T's closer together on both boards?} \\
            }
        }
    }
    & & \\ \\

    \theutterance \stepcounter{utterance}  
    & & \multicolumn{4}{p{0.6\linewidth}}{
        \cellcolor[rgb]{0.9,0.9,0.9}{
            \makecell[{{p{\linewidth}}}]{
                \texttt{\tiny{[P1$\langle$GM]}}
                \texttt{Your message has been relayed to the other player.} \\
\\ 
\\ 
\texttt{You are currently playing round 5 of maximum 28.} \\
\texttt{You have collectively accumulated 1 of 16 penalties.} \\
\texttt{The other player sent this message:} \\
\\ 
\\ 
\texttt{"Moving 'T' to a central location like (4,4) sounds like a good idea. Before we do that, I'd like to confirm that we're considering the distance between the objects on both boards. Since 'T' is at (1,2) on my board, moving it to (4,4) would be a significant move. What if we aim to move 'T' to (3,4) instead, as it might be a more manageable step and bring the 'T's closer together on both boards?"} \\
\\ 
\\ 
\texttt{What is your next command?} \\
            }
        }
    }
    & \\ \\

    \theutterance \stepcounter{utterance}  
    & \multicolumn{4}{p{0.6\linewidth}}{
        \cellcolor[rgb]{0.9,0.9,0.9}{
            \makecell[{{p{\linewidth}}}]{
                \texttt{\tiny{[P1$\rangle$GM]}}
                \texttt{SAY: Moving 'T' to (3,4) is a great suggestion, as it's a more incremental step and could help us gauge the progress. I agree that (3,4) is a good target for 'T'. Before we make the move, I'd like to ask: have you moved any objects on your board since we last discussed, apart from the one you mentioned earlier? I want to make sure we're still on the same page regarding the board states.} \\
            }
        }
    }
    & & \\ \\

    \theutterance \stepcounter{utterance}  
    & & & \multicolumn{2}{p{0.3\linewidth}}{
        \cellcolor[rgb]{0.9,0.9,0.9}{
            \makecell[{{p{\linewidth}}}]{
                \texttt{\tiny{[GM$|$GM]}}
                \texttt{SAY: Moving 'T' to (3,4) is a great suggestion, as it's a more incremental step and could help us gauge the progress. I agree that (3,4) is a good target for 'T'. Before we make the move, I'd like to ask: have you moved any objects on your board since we last discussed, apart from the one you mentioned earlier? I want to make sure we're still on the same page regarding the board states.} \\
            }
        }
    }
    & & \\ \\

    \theutterance \stepcounter{utterance}  
    & & \multicolumn{4}{p{0.6\linewidth}}{
        \cellcolor[rgb]{0.9,0.9,0.9}{
            \makecell[{{p{\linewidth}}}]{
                \texttt{\tiny{[GM$\rangle$P2]}}
                \texttt{Your message has been relayed to the other player.} \\
\\ 
\\ 
\texttt{You are currently playing round 6 of maximum 28.} \\
\texttt{You have collectively accumulated 1 of 16 penalties.} \\
\texttt{The other player sent this message:} \\
\\ 
\\ 
\texttt{"Moving 'T' to (3,4) is a great suggestion, as it's a more incremental step and could help us gauge the progress. I agree that (3,4) is a good target for 'T'. Before we make the move, I'd like to ask: have you moved any objects on your board since we last discussed, apart from the one you mentioned earlier? I want to make sure we're still on the same page regarding the board states."} \\
\\ 
\\ 
\texttt{What is your next command?} \\
            }
        }
    }
    & \\ \\

    \theutterance \stepcounter{utterance}  
    & & & \multicolumn{4}{p{0.6\linewidth}}{
        \cellcolor[rgb]{0.9,0.9,0.9}{
            \makecell[{{p{\linewidth}}}]{
                \texttt{\tiny{[GM$\langle$P2]}}
                \texttt{SAY: No, I haven't moved any objects on my board apart from moving 'P' to (1,1) as we discussed earlier. My board still looks like the updated grid you saw after the 'P' move, with 'T' still at (1,2). I'm ready to move 'T' to (3,4) as we agreed, but I'll wait for your confirmation that you're ready to move 'T' to (3,4) on your board as well.} \\
            }
        }
    }
     \\ \\

    \theutterance \stepcounter{utterance}  
    & & & \multicolumn{2}{p{0.3\linewidth}}{
        \cellcolor[rgb]{0.9,0.9,0.9}{
            \makecell[{{p{\linewidth}}}]{
                \texttt{\tiny{[GM$|$GM]}}
                \texttt{SAY: No, I haven't moved any objects on my board apart from moving 'P' to (1,1) as we discussed earlier. My board still looks like the updated grid you saw after the 'P' move, with 'T' still at (1,2). I'm ready to move 'T' to (3,4) as we agreed, but I'll wait for your confirmation that you're ready to move 'T' to (3,4) on your board as well.} \\
            }
        }
    }
    & & \\ \\

    \theutterance \stepcounter{utterance}  
    & & \multicolumn{4}{p{0.6\linewidth}}{
        \cellcolor[rgb]{0.9,0.9,0.9}{
            \makecell[{{p{\linewidth}}}]{
                \texttt{\tiny{[P1$\langle$GM]}}
                \texttt{Your message has been relayed to the other player.} \\
\\ 
\\ 
\texttt{You are currently playing round 6 of maximum 28.} \\
\texttt{You have collectively accumulated 1 of 16 penalties.} \\
\texttt{The other player sent this message:} \\
\\ 
\\ 
\texttt{"No, I haven't moved any objects on my board apart from moving 'P' to (1,1) as we discussed earlier. My board still looks like the updated grid you saw after the 'P' move, with 'T' still at (1,2). I'm ready to move 'T' to (3,4) as we agreed, but I'll wait for your confirmation that you're ready to move 'T' to (3,4) on your board as well."} \\
\\ 
\\ 
\texttt{What is your next command?} \\
            }
        }
    }
    & \\ \\

    \theutterance \stepcounter{utterance}  
    & \multicolumn{4}{p{0.6\linewidth}}{
        \cellcolor[rgb]{0.9,0.9,0.9}{
            \makecell[{{p{\linewidth}}}]{
                \texttt{\tiny{[P1$\rangle$GM]}}
                \texttt{SAY: I'm ready to move 'T' to (3,4) on my board. Please go ahead and move 'T' to (3,4) on your board as well. After we both make this move, we can compare our boards again and discuss the next steps. Also, just to confirm, after this move, 'T' will be at (3,4) on both boards, and 'P' will still be at (1,1) on both boards, right?} \\
            }
        }
    }
    & & \\ \\

    \theutterance \stepcounter{utterance}  
    & & & \multicolumn{2}{p{0.3\linewidth}}{
        \cellcolor[rgb]{0.9,0.9,0.9}{
            \makecell[{{p{\linewidth}}}]{
                \texttt{\tiny{[GM$|$GM]}}
                \texttt{SAY: I'm ready to move 'T' to (3,4) on my board. Please go ahead and move 'T' to (3,4) on your board as well. After we both make this move, we can compare our boards again and discuss the next steps. Also, just to confirm, after this move, 'T' will be at (3,4) on both boards, and 'P' will still be at (1,1) on both boards, right?} \\
            }
        }
    }
    & & \\ \\

    \theutterance \stepcounter{utterance}  
    & & \multicolumn{4}{p{0.6\linewidth}}{
        \cellcolor[rgb]{0.9,0.9,0.9}{
            \makecell[{{p{\linewidth}}}]{
                \texttt{\tiny{[GM$\rangle$P2]}}
                \texttt{Your message has been relayed to the other player.} \\
\\ 
\\ 
\texttt{You are currently playing round 7 of maximum 28.} \\
\texttt{You have collectively accumulated 1 of 16 penalties.} \\
\texttt{The other player sent this message:} \\
\\ 
\\ 
\texttt{"I'm ready to move 'T' to (3,4) on my board. Please go ahead and move 'T' to (3,4) on your board as well. After we both make this move, we can compare our boards again and discuss the next steps. Also, just to confirm, after this move, 'T' will be at (3,4) on both boards, and 'P' will still be at (1,1) on both boards, right?"} \\
\\ 
\\ 
\texttt{What is your next command?} \\
            }
        }
    }
    & \\ \\

    \theutterance \stepcounter{utterance}  
    & & & \multicolumn{4}{p{0.6\linewidth}}{
        \cellcolor[rgb]{0.9,0.9,0.9}{
            \makecell[{{p{\linewidth}}}]{
                \texttt{\tiny{[GM$\langle$P2]}}
                \texttt{SAY: That's correct, after we both move 'T' to (3,4), 'T' will be at (3,4) on both boards, and 'P' will still be at (1,1) on both boards. I'll go ahead and move 'T' to (3,4) on my board now.} \\
            }
        }
    }
     \\ \\

    \theutterance \stepcounter{utterance}  
    & & & \multicolumn{2}{p{0.3\linewidth}}{
        \cellcolor[rgb]{0.9,0.9,0.9}{
            \makecell[{{p{\linewidth}}}]{
                \texttt{\tiny{[GM$|$GM]}}
                \texttt{SAY: That's correct, after we both move 'T' to (3,4), 'T' will be at (3,4) on both boards, and 'P' will still be at (1,1) on both boards. I'll go ahead and move 'T' to (3,4) on my board now.} \\
            }
        }
    }
    & & \\ \\

    \theutterance \stepcounter{utterance}  
    & & \multicolumn{4}{p{0.6\linewidth}}{
        \cellcolor[rgb]{0.9,0.9,0.9}{
            \makecell[{{p{\linewidth}}}]{
                \texttt{\tiny{[P1$\langle$GM]}}
                \texttt{Your message has been relayed to the other player.} \\
\\ 
\\ 
\texttt{You are currently playing round 7 of maximum 28.} \\
\texttt{You have collectively accumulated 1 of 16 penalties.} \\
\texttt{The other player sent this message:} \\
\\ 
\\ 
\texttt{"That's correct, after we both move 'T' to (3,4), 'T' will be at (3,4) on both boards, and 'P' will still be at (1,1) on both boards. I'll go ahead and move 'T' to (3,4) on my board now."} \\
\\ 
\\ 
\texttt{What is your next command?} \\
            }
        }
    }
    & \\ \\

    \theutterance \stepcounter{utterance}  
    & \multicolumn{4}{p{0.6\linewidth}}{
        \cellcolor[rgb]{0.9,0.9,0.9}{
            \makecell[{{p{\linewidth}}}]{
                \texttt{\tiny{[P1$\rangle$GM]}}
                \texttt{MOVE: T, (3, 4)} \\
            }
        }
    }
    & & \\ \\

    \theutterance \stepcounter{utterance}  
    & & & \multicolumn{2}{p{0.3\linewidth}}{
        \cellcolor[rgb]{0.9,0.9,0.9}{
            \makecell[{{p{\linewidth}}}]{
                \texttt{\tiny{[GM$|$GM]}}
                \texttt{MOVE: T, (3, 4)} \\
            }
        }
    }
    & & \\ \\

    \theutterance \stepcounter{utterance}  
    & & & \multicolumn{2}{p{0.3\linewidth}}{
        \cellcolor[rgb]{0.9,0.9,0.9}{
            \makecell[{{p{\linewidth}}}]{
                \texttt{\tiny{[GM$|$GM]}}
                \texttt{Penalty: (3,4) is not empty, but contains '┬'.} \\
\\ 
\texttt{You have collectively accumulated 2 of 16 penalties.} \\
\texttt{Please try again!} \\
            }
        }
    }
    & & \\ \\

    \theutterance \stepcounter{utterance}  
    & & \multicolumn{4}{p{0.6\linewidth}}{
        \cellcolor[rgb]{0.9,0.9,0.9}{
            \makecell[{{p{\linewidth}}}]{
                \texttt{\tiny{[P1$\langle$GM]}}
                \texttt{Penalty: (3,4) is not empty, but contains '┬'.} \\
\\ 
\texttt{You have collectively accumulated 2 of 16 penalties.} \\
\texttt{Please try again!} \\
            }
        }
    }
    & \\ \\

    \theutterance \stepcounter{utterance}  
    & \multicolumn{4}{p{0.6\linewidth}}{
        \cellcolor[rgb]{0.9,0.9,0.9}{
            \makecell[{{p{\linewidth}}}]{
                \texttt{\tiny{[P1$\rangle$GM]}}
                \texttt{SAY: It seems that the move I tried to make wasn't possible because (3,4) is not empty on my board. The cell (3,4) contains '┬'. I'll need to choose a different target cell for 'T'. Can we discuss an alternative location for 'T' that is empty on both our boards? Maybe we can try to move 'T' to a cell that is currently empty on my board, such as (2,4) or (3,5)?} \\
            }
        }
    }
    & & \\ \\

    \theutterance \stepcounter{utterance}  
    & & & \multicolumn{2}{p{0.3\linewidth}}{
        \cellcolor[rgb]{0.9,0.9,0.9}{
            \makecell[{{p{\linewidth}}}]{
                \texttt{\tiny{[GM$|$GM]}}
                \texttt{SAY: It seems that the move I tried to make wasn't possible because (3,4) is not empty on my board. The cell (3,4) contains '┬'. I'll need to choose a different target cell for 'T'. Can we discuss an alternative location for 'T' that is empty on both our boards? Maybe we can try to move 'T' to a cell that is currently empty on my board, such as (2,4) or (3,5)?} \\
            }
        }
    }
    & & \\ \\

    \theutterance \stepcounter{utterance}  
    & & \multicolumn{4}{p{0.6\linewidth}}{
        \cellcolor[rgb]{0.9,0.9,0.9}{
            \makecell[{{p{\linewidth}}}]{
                \texttt{\tiny{[GM$\rangle$P2]}}
                \texttt{Your message has been relayed to the other player.} \\
\\ 
\\ 
\texttt{You are currently playing round 8 of maximum 28.} \\
\texttt{You have collectively accumulated 2 of 16 penalties.} \\
\texttt{The other player sent this message:} \\
\\ 
\\ 
\texttt{"It seems that the move I tried to make wasn't possible because (3,4) is not empty on my board. The cell (3,4) contains '┬'. I'll need to choose a different target cell for 'T'. Can we discuss an alternative location for 'T' that is empty on both our boards? Maybe we can try to move 'T' to a cell that is currently empty on my board, such as (2,4) or (3,5)?"} \\
\\ 
\\ 
\texttt{What is your next command?} \\
            }
        }
    }
    & \\ \\

    \theutterance \stepcounter{utterance}  
    & & & \multicolumn{4}{p{0.6\linewidth}}{
        \cellcolor[rgb]{0.9,0.9,0.9}{
            \makecell[{{p{\linewidth}}}]{
                \texttt{\tiny{[GM$\langle$P2]}}
                \texttt{SAY: Sorry to hear that the move didn't work out. Let me check my board for the cells you mentioned. Since (2,4) and (3,5) are potential targets, I'll check if they are empty on my board. On my board, (2,4) is empty, but I'll need to double{-}check (3,5). If (2,4) is empty on both boards, that could be a good alternative location for 'T'. What do you think about moving 'T' to (2,4) on both boards?} \\
            }
        }
    }
     \\ \\

    \theutterance \stepcounter{utterance}  
    & & & \multicolumn{2}{p{0.3\linewidth}}{
        \cellcolor[rgb]{0.9,0.9,0.9}{
            \makecell[{{p{\linewidth}}}]{
                \texttt{\tiny{[GM$|$GM]}}
                \texttt{SAY: Sorry to hear that the move didn't work out. Let me check my board for the cells you mentioned. Since (2,4) and (3,5) are potential targets, I'll check if they are empty on my board. On my board, (2,4) is empty, but I'll need to double{-}check (3,5). If (2,4) is empty on both boards, that could be a good alternative location for 'T'. What do you think about moving 'T' to (2,4) on both boards?} \\
            }
        }
    }
    & & \\ \\

    \theutterance \stepcounter{utterance}  
    & & \multicolumn{4}{p{0.6\linewidth}}{
        \cellcolor[rgb]{0.9,0.9,0.9}{
            \makecell[{{p{\linewidth}}}]{
                \texttt{\tiny{[P1$\langle$GM]}}
                \texttt{Your message has been relayed to the other player.} \\
\\ 
\\ 
\texttt{You are currently playing round 8 of maximum 28.} \\
\texttt{You have collectively accumulated 2 of 16 penalties.} \\
\texttt{The other player sent this message:} \\
\\ 
\\ 
\texttt{"Sorry to hear that the move didn't work out. Let me check my board for the cells you mentioned. Since (2,4) and (3,5) are potential targets, I'll check if they are empty on my board. On my board, (2,4) is empty, but I'll need to double{-}check (3,5). If (2,4) is empty on both boards, that could be a good alternative location for 'T'. What do you think about moving 'T' to (2,4) on both boards?"} \\
\\ 
\\ 
\texttt{What is your next command?} \\
            }
        }
    }
    & \\ \\

    \theutterance \stepcounter{utterance}  
    & \multicolumn{4}{p{0.6\linewidth}}{
        \cellcolor[rgb]{0.9,0.9,0.9}{
            \makecell[{{p{\linewidth}}}]{
                \texttt{\tiny{[P1$\rangle$GM]}}
                \texttt{SAY: Moving 'T' to (2,4) sounds like a good alternative. I've checked my board, and (2,4) is empty on my board as well. So, it seems that (2,4) is a viable target for 'T' on both boards. Before we make the move, I just want to confirm that we're both on the same page: if we move 'T' to (2,4), it will be at (2,4) on both boards, and 'P' will still be at (1,1) on both boards, right? Also, are you ready to move 'T' to (2,4) on your board?} \\
            }
        }
    }
    & & \\ \\

    \theutterance \stepcounter{utterance}  
    & & & \multicolumn{2}{p{0.3\linewidth}}{
        \cellcolor[rgb]{0.9,0.9,0.9}{
            \makecell[{{p{\linewidth}}}]{
                \texttt{\tiny{[GM$|$GM]}}
                \texttt{SAY: Moving 'T' to (2,4) sounds like a good alternative. I've checked my board, and (2,4) is empty on my board as well. So, it seems that (2,4) is a viable target for 'T' on both boards. Before we make the move, I just want to confirm that we're both on the same page: if we move 'T' to (2,4), it will be at (2,4) on both boards, and 'P' will still be at (1,1) on both boards, right? Also, are you ready to move 'T' to (2,4) on your board?} \\
            }
        }
    }
    & & \\ \\

    \theutterance \stepcounter{utterance}  
    & & \multicolumn{4}{p{0.6\linewidth}}{
        \cellcolor[rgb]{0.9,0.9,0.9}{
            \makecell[{{p{\linewidth}}}]{
                \texttt{\tiny{[GM$\rangle$P2]}}
                \texttt{Your message has been relayed to the other player.} \\
\\ 
\\ 
\texttt{You are currently playing round 9 of maximum 28.} \\
\texttt{You have collectively accumulated 2 of 16 penalties.} \\
\texttt{The other player sent this message:} \\
\\ 
\\ 
\texttt{"Moving 'T' to (2,4) sounds like a good alternative. I've checked my board, and (2,4) is empty on my board as well. So, it seems that (2,4) is a viable target for 'T' on both boards. Before we make the move, I just want to confirm that we're both on the same page: if we move 'T' to (2,4), it will be at (2,4) on both boards, and 'P' will still be at (1,1) on both boards, right? Also, are you ready to move 'T' to (2,4) on your board?"} \\
\\ 
\\ 
\texttt{What is your next command?} \\
            }
        }
    }
    & \\ \\

    \theutterance \stepcounter{utterance}  
    & & & \multicolumn{4}{p{0.6\linewidth}}{
        \cellcolor[rgb]{0.9,0.9,0.9}{
            \makecell[{{p{\linewidth}}}]{
                \texttt{\tiny{[GM$\langle$P2]}}
                \texttt{MOVE: T, (2,4)} \\
            }
        }
    }
     \\ \\

    \theutterance \stepcounter{utterance}  
    & & & \multicolumn{2}{p{0.3\linewidth}}{
        \cellcolor[rgb]{0.9,0.9,0.9}{
            \makecell[{{p{\linewidth}}}]{
                \texttt{\tiny{[GM$|$GM]}}
                \texttt{MOVE: T, (2,4)} \\
            }
        }
    }
    & & \\ \\

    \theutterance \stepcounter{utterance}  
    & & & \multicolumn{2}{p{0.3\linewidth}}{
        \cellcolor[rgb]{0.9,0.9,0.9}{
            \makecell[{{p{\linewidth}}}]{
                \texttt{\tiny{[GM$|$GM]}}
                \texttt{Penalty: (2,4) is not empty, but contains '─'.} \\
\\ 
\texttt{You have collectively accumulated 3 of 16 penalties.} \\
\texttt{Please try again!} \\
            }
        }
    }
    & & \\ \\

    \theutterance \stepcounter{utterance}  
    & & \multicolumn{4}{p{0.6\linewidth}}{
        \cellcolor[rgb]{0.9,0.9,0.9}{
            \makecell[{{p{\linewidth}}}]{
                \texttt{\tiny{[GM$\rangle$P2]}}
                \texttt{Penalty: (2,4) is not empty, but contains '─'.} \\
\\ 
\texttt{You have collectively accumulated 3 of 16 penalties.} \\
\texttt{Please try again!} \\
            }
        }
    }
    & \\ \\

    \theutterance \stepcounter{utterance}  
    & & & \multicolumn{4}{p{0.6\linewidth}}{
        \cellcolor[rgb]{0.9,0.9,0.9}{
            \makecell[{{p{\linewidth}}}]{
                \texttt{\tiny{[GM$\langle$P2]}}
                \texttt{SAY: It seems that I made an error, (2,4) is not empty on my board. Let me re{-}check the board for an alternative location for 'T'. Since (2,4) didn't work out, what about (3,5)? Is (3,5) empty on your board?} \\
            }
        }
    }
     \\ \\

    \theutterance \stepcounter{utterance}  
    & & & \multicolumn{2}{p{0.3\linewidth}}{
        \cellcolor[rgb]{0.9,0.9,0.9}{
            \makecell[{{p{\linewidth}}}]{
                \texttt{\tiny{[GM$|$GM]}}
                \texttt{SAY: It seems that I made an error, (2,4) is not empty on my board. Let me re{-}check the board for an alternative location for 'T'. Since (2,4) didn't work out, what about (3,5)? Is (3,5) empty on your board?} \\
            }
        }
    }
    & & \\ \\

    \theutterance \stepcounter{utterance}  
    & & \multicolumn{4}{p{0.6\linewidth}}{
        \cellcolor[rgb]{0.9,0.9,0.9}{
            \makecell[{{p{\linewidth}}}]{
                \texttt{\tiny{[P1$\langle$GM]}}
                \texttt{Your message has been relayed to the other player.} \\
\\ 
\\ 
\texttt{You are currently playing round 9 of maximum 28.} \\
\texttt{You have collectively accumulated 3 of 16 penalties.} \\
\texttt{The other player sent this message:} \\
\\ 
\\ 
\texttt{"It seems that I made an error, (2,4) is not empty on my board. Let me re{-}check the board for an alternative location for 'T'. Since (2,4) didn't work out, what about (3,5)? Is (3,5) empty on your board?"} \\
\\ 
\\ 
\texttt{What is your next command?} \\
            }
        }
    }
    & \\ \\

    \theutterance \stepcounter{utterance}  
    & \multicolumn{4}{p{0.6\linewidth}}{
        \cellcolor[rgb]{0.9,0.9,0.9}{
            \makecell[{{p{\linewidth}}}]{
                \texttt{\tiny{[P1$\rangle$GM]}}
                \texttt{SAY: No worries, mistakes happen. Let me check (3,5) on my board. According to my board, (3,5) is empty, it contains '◌'. So, (3,5) could be a possible target for 'T'. If we move 'T' to (3,5) on both boards, it would be at (3,5) on both boards, and 'P' would still be at (1,1) on both boards. What do you think? Should we try to move 'T' to (3,5) on both boards?} \\
            }
        }
    }
    & & \\ \\

    \theutterance \stepcounter{utterance}  
    & & & \multicolumn{2}{p{0.3\linewidth}}{
        \cellcolor[rgb]{0.9,0.9,0.9}{
            \makecell[{{p{\linewidth}}}]{
                \texttt{\tiny{[GM$|$GM]}}
                \texttt{SAY: No worries, mistakes happen. Let me check (3,5) on my board. According to my board, (3,5) is empty, it contains '◌'. So, (3,5) could be a possible target for 'T'. If we move 'T' to (3,5) on both boards, it would be at (3,5) on both boards, and 'P' would still be at (1,1) on both boards. What do you think? Should we try to move 'T' to (3,5) on both boards?} \\
            }
        }
    }
    & & \\ \\

    \theutterance \stepcounter{utterance}  
    & & \multicolumn{4}{p{0.6\linewidth}}{
        \cellcolor[rgb]{0.9,0.9,0.9}{
            \makecell[{{p{\linewidth}}}]{
                \texttt{\tiny{[GM$\rangle$P2]}}
                \texttt{Your message has been relayed to the other player.} \\
\\ 
\\ 
\texttt{You are currently playing round 10 of maximum 28.} \\
\texttt{You have collectively accumulated 3 of 16 penalties.} \\
\texttt{The other player sent this message:} \\
\\ 
\\ 
\texttt{"No worries, mistakes happen. Let me check (3,5) on my board. According to my board, (3,5) is empty, it contains '◌'. So, (3,5) could be a possible target for 'T'. If we move 'T' to (3,5) on both boards, it would be at (3,5) on both boards, and 'P' would still be at (1,1) on both boards. What do you think? Should we try to move 'T' to (3,5) on both boards?"} \\
\\ 
\\ 
\texttt{What is your next command?} \\
            }
        }
    }
    & \\ \\

    \theutterance \stepcounter{utterance}  
    & & & \multicolumn{4}{p{0.6\linewidth}}{
        \cellcolor[rgb]{0.9,0.9,0.9}{
            \makecell[{{p{\linewidth}}}]{
                \texttt{\tiny{[GM$\langle$P2]}}
                \texttt{MOVE: T, (3,5)} \\
            }
        }
    }
     \\ \\

    \theutterance \stepcounter{utterance}  
    & & & \multicolumn{2}{p{0.3\linewidth}}{
        \cellcolor[rgb]{0.9,0.9,0.9}{
            \makecell[{{p{\linewidth}}}]{
                \texttt{\tiny{[GM$|$GM]}}
                \texttt{MOVE: T, (3,5)} \\
            }
        }
    }
    & & \\ \\

    \theutterance \stepcounter{utterance}  
    & & & \multicolumn{2}{p{0.3\linewidth}}{
        \cellcolor[rgb]{0.9,0.9,0.9}{
            \makecell[{{p{\linewidth}}}]{
                \texttt{\tiny{[GM$|$GM]}}
                \texttt{Penalty: (3,5) is not empty, but contains '│'.} \\
\\ 
\texttt{You have collectively accumulated 4 of 16 penalties.} \\
\texttt{Please try again!} \\
            }
        }
    }
    & & \\ \\

    \theutterance \stepcounter{utterance}  
    & & \multicolumn{4}{p{0.6\linewidth}}{
        \cellcolor[rgb]{0.9,0.9,0.9}{
            \makecell[{{p{\linewidth}}}]{
                \texttt{\tiny{[GM$\rangle$P2]}}
                \texttt{Penalty: (3,5) is not empty, but contains '│'.} \\
\\ 
\texttt{You have collectively accumulated 4 of 16 penalties.} \\
\texttt{Please try again!} \\
            }
        }
    }
    & \\ \\

    \theutterance \stepcounter{utterance}  
    & & & \multicolumn{4}{p{0.6\linewidth}}{
        \cellcolor[rgb]{0.9,0.9,0.9}{
            \makecell[{{p{\linewidth}}}]{
                \texttt{\tiny{[GM$\langle$P2]}}
                \texttt{SAY: It seems that (3,5) is not empty on my board either. Let me check again for an empty cell. What about (2,5)? Is (2,5) empty on your board, and could it be a possible target for 'T'?} \\
            }
        }
    }
     \\ \\

    \theutterance \stepcounter{utterance}  
    & & & \multicolumn{2}{p{0.3\linewidth}}{
        \cellcolor[rgb]{0.9,0.9,0.9}{
            \makecell[{{p{\linewidth}}}]{
                \texttt{\tiny{[GM$|$GM]}}
                \texttt{SAY: It seems that (3,5) is not empty on my board either. Let me check again for an empty cell. What about (2,5)? Is (2,5) empty on your board, and could it be a possible target for 'T'?} \\
            }
        }
    }
    & & \\ \\

    \theutterance \stepcounter{utterance}  
    & & \multicolumn{4}{p{0.6\linewidth}}{
        \cellcolor[rgb]{0.9,0.9,0.9}{
            \makecell[{{p{\linewidth}}}]{
                \texttt{\tiny{[P1$\langle$GM]}}
                \texttt{Your message has been relayed to the other player.} \\
\\ 
\\ 
\texttt{You are currently playing round 10 of maximum 28.} \\
\texttt{You have collectively accumulated 4 of 16 penalties.} \\
\texttt{The other player sent this message:} \\
\\ 
\\ 
\texttt{"It seems that (3,5) is not empty on my board either. Let me check again for an empty cell. What about (2,5)? Is (2,5) empty on your board, and could it be a possible target for 'T'?"} \\
\\ 
\\ 
\texttt{What is your next command?} \\
            }
        }
    }
    & \\ \\

    \theutterance \stepcounter{utterance}  
    & \multicolumn{4}{p{0.6\linewidth}}{
        \cellcolor[rgb]{0.9,0.9,0.9}{
            \makecell[{{p{\linewidth}}}]{
                \texttt{\tiny{[P1$\rangle$GM]}}
                \texttt{SAY: Let me check (2,5) on my board. According to my board, (2,5) is empty, it contains '◌'. So, (2,5) could be a possible target for 'T'. Before we make the move, I'd like to confirm that (2,5) is empty on your board as well. If it is, we can try to move 'T' to (2,5) on both boards. If we do that, 'T' will be at (2,5) on both boards, and 'P' will still be at (1,1) on both boards. Please confirm if (2,5) is empty on your board.} \\
            }
        }
    }
    & & \\ \\

    \theutterance \stepcounter{utterance}  
    & & & \multicolumn{2}{p{0.3\linewidth}}{
        \cellcolor[rgb]{0.9,0.9,0.9}{
            \makecell[{{p{\linewidth}}}]{
                \texttt{\tiny{[GM$|$GM]}}
                \texttt{SAY: Let me check (2,5) on my board. According to my board, (2,5) is empty, it contains '◌'. So, (2,5) could be a possible target for 'T'. Before we make the move, I'd like to confirm that (2,5) is empty on your board as well. If it is, we can try to move 'T' to (2,5) on both boards. If we do that, 'T' will be at (2,5) on both boards, and 'P' will still be at (1,1) on both boards. Please confirm if (2,5) is empty on your board.} \\
            }
        }
    }
    & & \\ \\

    \theutterance \stepcounter{utterance}  
    & & \multicolumn{4}{p{0.6\linewidth}}{
        \cellcolor[rgb]{0.9,0.9,0.9}{
            \makecell[{{p{\linewidth}}}]{
                \texttt{\tiny{[GM$\rangle$P2]}}
                \texttt{Your message has been relayed to the other player.} \\
\\ 
\\ 
\texttt{You are currently playing round 11 of maximum 28.} \\
\texttt{You have collectively accumulated 4 of 16 penalties.} \\
\texttt{The other player sent this message:} \\
\\ 
\\ 
\texttt{"Let me check (2,5) on my board. According to my board, (2,5) is empty, it contains '◌'. So, (2,5) could be a possible target for 'T'. Before we make the move, I'd like to confirm that (2,5) is empty on your board as well. If it is, we can try to move 'T' to (2,5) on both boards. If we do that, 'T' will be at (2,5) on both boards, and 'P' will still be at (1,1) on both boards. Please confirm if (2,5) is empty on your board."} \\
\\ 
\\ 
\texttt{What is your next command?} \\
            }
        }
    }
    & \\ \\

    \theutterance \stepcounter{utterance}  
    & & & \multicolumn{4}{p{0.6\linewidth}}{
        \cellcolor[rgb]{0.9,0.9,0.9}{
            \makecell[{{p{\linewidth}}}]{
                \texttt{\tiny{[GM$\langle$P2]}}
                \texttt{SAY: Yes, (2,5) is empty on my board, it contains '◌'. We can proceed with moving 'T' to (2,5) on both boards. 'T' will be at (2,5) on both boards, and 'P' will still be at (1,1) on both boards. I'm ready to make the move. Please confirm when you're ready, and we can make the move simultaneously.} \\
            }
        }
    }
     \\ \\

    \theutterance \stepcounter{utterance}  
    & & & \multicolumn{2}{p{0.3\linewidth}}{
        \cellcolor[rgb]{0.9,0.9,0.9}{
            \makecell[{{p{\linewidth}}}]{
                \texttt{\tiny{[GM$|$GM]}}
                \texttt{SAY: Yes, (2,5) is empty on my board, it contains '◌'. We can proceed with moving 'T' to (2,5) on both boards. 'T' will be at (2,5) on both boards, and 'P' will still be at (1,1) on both boards. I'm ready to make the move. Please confirm when you're ready, and we can make the move simultaneously.} \\
            }
        }
    }
    & & \\ \\

    \theutterance \stepcounter{utterance}  
    & & \multicolumn{4}{p{0.6\linewidth}}{
        \cellcolor[rgb]{0.9,0.9,0.9}{
            \makecell[{{p{\linewidth}}}]{
                \texttt{\tiny{[P1$\langle$GM]}}
                \texttt{Your message has been relayed to the other player.} \\
\\ 
\\ 
\texttt{You are currently playing round 11 of maximum 28.} \\
\texttt{You have collectively accumulated 4 of 16 penalties.} \\
\texttt{The other player sent this message:} \\
\\ 
\\ 
\texttt{"Yes, (2,5) is empty on my board, it contains '◌'. We can proceed with moving 'T' to (2,5) on both boards. 'T' will be at (2,5) on both boards, and 'P' will still be at (1,1) on both boards. I'm ready to make the move. Please confirm when you're ready, and we can make the move simultaneously."} \\
\\ 
\\ 
\texttt{What is your next command?} \\
            }
        }
    }
    & \\ \\

    \theutterance \stepcounter{utterance}  
    & \multicolumn{4}{p{0.6\linewidth}}{
        \cellcolor[rgb]{0.9,0.9,0.9}{
            \makecell[{{p{\linewidth}}}]{
                \texttt{\tiny{[P1$\rangle$GM]}}
                \texttt{MOVE: T, (2, 5)} \\
            }
        }
    }
    & & \\ \\

    \theutterance \stepcounter{utterance}  
    & & & \multicolumn{2}{p{0.3\linewidth}}{
        \cellcolor[rgb]{0.9,0.9,0.9}{
            \makecell[{{p{\linewidth}}}]{
                \texttt{\tiny{[GM$|$GM]}}
                \texttt{MOVE: T, (2, 5)} \\
            }
        }
    }
    & & \\ \\

    \theutterance \stepcounter{utterance}  
    & & & \multicolumn{2}{p{0.3\linewidth}}{
        \cellcolor[rgb]{0.9,0.9,0.9}{
            \makecell[{{p{\linewidth}}}]{
                \texttt{\tiny{[GM$|$GM]}}
                \texttt{Moved 'T' to (2,5) successfully. Your updated grid looks like this:} \\
\texttt{```} \\
\texttt{ 1234567} \\
\texttt{╔═╤═╤═╤═╗} \\
\texttt{║P│◌│◌│◌║ 1} \\
\texttt{║◌└─┼─┴─╢ 2} \\
\texttt{║◌SA│◌◌◌║ 3} \\
\texttt{╟──┬┴───╢ 4} \\
\texttt{║◌T│◌◌O◌║ 5} \\
\texttt{║◌◌│◌◌MD║ 6} \\
\texttt{║◌◌│◌◌◌◌║ 7} \\
\texttt{╚══╧════╝} \\
\texttt{```} \\
            }
        }
    }
    & & \\ \\

    \theutterance \stepcounter{utterance}  
    & & \multicolumn{4}{p{0.6\linewidth}}{
        \cellcolor[rgb]{0.9,0.9,0.9}{
            \makecell[{{p{\linewidth}}}]{
                \texttt{\tiny{[GM$\rangle$P2]}}
                \texttt{Your message has been relayed to the other player.} \\
\\ 
\\ 
\texttt{You are currently playing round 12 of maximum 28.} \\
\texttt{You have collectively accumulated 4 of 16 penalties.} \\
\texttt{The other player moved an object on their grid.} \\
\\ 
\\ 
\texttt{What is your next command?} \\
            }
        }
    }
    & \\ \\

    \theutterance \stepcounter{utterance}  
    & & & \multicolumn{4}{p{0.6\linewidth}}{
        \cellcolor[rgb]{0.9,0.9,0.9}{
            \makecell[{{p{\linewidth}}}]{
                \texttt{\tiny{[GM$\langle$P2]}}
                \texttt{MOVE: T, (2,5)} \\
            }
        }
    }
     \\ \\

    \theutterance \stepcounter{utterance}  
    & & & \multicolumn{2}{p{0.3\linewidth}}{
        \cellcolor[rgb]{0.9,0.9,0.9}{
            \makecell[{{p{\linewidth}}}]{
                \texttt{\tiny{[GM$|$GM]}}
                \texttt{MOVE: T, (2,5)} \\
            }
        }
    }
    & & \\ \\

    \theutterance \stepcounter{utterance}  
    & & & \multicolumn{2}{p{0.3\linewidth}}{
        \cellcolor[rgb]{0.9,0.9,0.9}{
            \makecell[{{p{\linewidth}}}]{
                \texttt{\tiny{[GM$|$GM]}}
                \texttt{Penalty: (2,5) is not empty, but contains 'D'.} \\
\\ 
\texttt{You have collectively accumulated 5 of 16 penalties.} \\
\texttt{Please try again!} \\
            }
        }
    }
    & & \\ \\

    \theutterance \stepcounter{utterance}  
    & & \multicolumn{4}{p{0.6\linewidth}}{
        \cellcolor[rgb]{0.9,0.9,0.9}{
            \makecell[{{p{\linewidth}}}]{
                \texttt{\tiny{[GM$\rangle$P2]}}
                \texttt{Penalty: (2,5) is not empty, but contains 'D'.} \\
\\ 
\texttt{You have collectively accumulated 5 of 16 penalties.} \\
\texttt{Please try again!} \\
            }
        }
    }
    & \\ \\

    \theutterance \stepcounter{utterance}  
    & & & \multicolumn{4}{p{0.6\linewidth}}{
        \cellcolor[rgb]{0.9,0.9,0.9}{
            \makecell[{{p{\linewidth}}}]{
                \texttt{\tiny{[GM$\langle$P2]}}
                \texttt{SAY: It seems that (2,5) is not empty on my board, it contains 'D'. Let me re{-}evaluate the board and find an alternative location for 'T'. Can we discuss other possible empty cells on both boards that 'T' could be moved to?} \\
            }
        }
    }
     \\ \\

    \theutterance \stepcounter{utterance}  
    & & & \multicolumn{2}{p{0.3\linewidth}}{
        \cellcolor[rgb]{0.9,0.9,0.9}{
            \makecell[{{p{\linewidth}}}]{
                \texttt{\tiny{[GM$|$GM]}}
                \texttt{SAY: It seems that (2,5) is not empty on my board, it contains 'D'. Let me re{-}evaluate the board and find an alternative location for 'T'. Can we discuss other possible empty cells on both boards that 'T' could be moved to?} \\
            }
        }
    }
    & & \\ \\

    \theutterance \stepcounter{utterance}  
    & & \multicolumn{4}{p{0.6\linewidth}}{
        \cellcolor[rgb]{0.9,0.9,0.9}{
            \makecell[{{p{\linewidth}}}]{
                \texttt{\tiny{[P1$\langle$GM]}}
                \texttt{Moved 'T' to (2,5) successfully. Your updated grid looks like this:} \\
\texttt{```} \\
\texttt{ 1234567} \\
\texttt{╔═╤═╤═╤═╗} \\
\texttt{║P│◌│◌│◌║ 1} \\
\texttt{║◌└─┼─┴─╢ 2} \\
\texttt{║◌SA│◌◌◌║ 3} \\
\texttt{╟──┬┴───╢ 4} \\
\texttt{║◌T│◌◌O◌║ 5} \\
\texttt{║◌◌│◌◌MD║ 6} \\
\texttt{║◌◌│◌◌◌◌║ 7} \\
\texttt{╚══╧════╝} \\
\texttt{```} \\
\texttt{You are currently playing round 12 of maximum 28.} \\
\texttt{You have collectively accumulated 5 of 16 penalties.} \\
\texttt{The other player sent this message:} \\
\\ 
\\ 
\texttt{"It seems that (2,5) is not empty on my board, it contains 'D'. Let me re{-}evaluate the board and find an alternative location for 'T'. Can we discuss other possible empty cells on both boards that 'T' could be moved to?"} \\
\\ 
\\ 
\texttt{What is your next command?} \\
            }
        }
    }
    & \\ \\

    \theutterance \stepcounter{utterance}  
    & \multicolumn{4}{p{0.6\linewidth}}{
        \cellcolor[rgb]{0.9,0.9,0.9}{
            \makecell[{{p{\linewidth}}}]{
                \texttt{\tiny{[P1$\rangle$GM]}}
                \texttt{SAY: It seems that there was a mismatch in our previous move. Let's start fresh and re{-}evaluate the board. Since 'T' is now at (2,5) on my board, I'll look for another empty cell that could be a possible target for 'T' on both boards. How about (3,6)? Is (3,6) empty on your board? On my board, (3,6) contains 'M', but I could potentially move 'M' to make space for 'T' if needed. What do you think?} \\
            }
        }
    }
    & & \\ \\

    \theutterance \stepcounter{utterance}  
    & & & \multicolumn{2}{p{0.3\linewidth}}{
        \cellcolor[rgb]{0.9,0.9,0.9}{
            \makecell[{{p{\linewidth}}}]{
                \texttt{\tiny{[GM$|$GM]}}
                \texttt{SAY: It seems that there was a mismatch in our previous move. Let's start fresh and re{-}evaluate the board. Since 'T' is now at (2,5) on my board, I'll look for another empty cell that could be a possible target for 'T' on both boards. How about (3,6)? Is (3,6) empty on your board? On my board, (3,6) contains 'M', but I could potentially move 'M' to make space for 'T' if needed. What do you think?} \\
            }
        }
    }
    & & \\ \\

    \theutterance \stepcounter{utterance}  
    & & \multicolumn{4}{p{0.6\linewidth}}{
        \cellcolor[rgb]{0.9,0.9,0.9}{
            \makecell[{{p{\linewidth}}}]{
                \texttt{\tiny{[GM$\rangle$P2]}}
                \texttt{Your message has been relayed to the other player.} \\
\\ 
\\ 
\texttt{You are currently playing round 13 of maximum 28.} \\
\texttt{You have collectively accumulated 5 of 16 penalties.} \\
\texttt{The other player sent this message:} \\
\\ 
\\ 
\texttt{"It seems that there was a mismatch in our previous move. Let's start fresh and re{-}evaluate the board. Since 'T' is now at (2,5) on my board, I'll look for another empty cell that could be a possible target for 'T' on both boards. How about (3,6)? Is (3,6) empty on your board? On my board, (3,6) contains 'M', but I could potentially move 'M' to make space for 'T' if needed. What do you think?"} \\
\\ 
\\ 
\texttt{What is your next command?} \\
            }
        }
    }
    & \\ \\

    \theutterance \stepcounter{utterance}  
    & & & \multicolumn{4}{p{0.6\linewidth}}{
        \cellcolor[rgb]{0.9,0.9,0.9}{
            \makecell[{{p{\linewidth}}}]{
                \texttt{\tiny{[GM$\langle$P2]}}
                \texttt{SAY: On my board, (3,6) is not empty, it contains 'S'. However, I could potentially move 'S' to make space for 'T' if needed. But before we consider that, let me check if there are any other empty cells on my board that could be a possible target for 'T'. Alternatively, if you could move 'M' from (3,6) on your board, that could be an option. What do you think about moving 'M' on your board to free up (3,6) for 'T'?} \\
            }
        }
    }
     \\ \\

    \theutterance \stepcounter{utterance}  
    & & & \multicolumn{2}{p{0.3\linewidth}}{
        \cellcolor[rgb]{0.9,0.9,0.9}{
            \makecell[{{p{\linewidth}}}]{
                \texttt{\tiny{[GM$|$GM]}}
                \texttt{SAY: On my board, (3,6) is not empty, it contains 'S'. However, I could potentially move 'S' to make space for 'T' if needed. But before we consider that, let me check if there are any other empty cells on my board that could be a possible target for 'T'. Alternatively, if you could move 'M' from (3,6) on your board, that could be an option. What do you think about moving 'M' on your board to free up (3,6) for 'T'?} \\
            }
        }
    }
    & & \\ \\

    \theutterance \stepcounter{utterance}  
    & & \multicolumn{4}{p{0.6\linewidth}}{
        \cellcolor[rgb]{0.9,0.9,0.9}{
            \makecell[{{p{\linewidth}}}]{
                \texttt{\tiny{[P1$\langle$GM]}}
                \texttt{Your message has been relayed to the other player.} \\
\\ 
\\ 
\texttt{You are currently playing round 13 of maximum 28.} \\
\texttt{You have collectively accumulated 5 of 16 penalties.} \\
\texttt{The other player sent this message:} \\
\\ 
\\ 
\texttt{"On my board, (3,6) is not empty, it contains 'S'. However, I could potentially move 'S' to make space for 'T' if needed. But before we consider that, let me check if there are any other empty cells on my board that could be a possible target for 'T'. Alternatively, if you could move 'M' from (3,6) on your board, that could be an option. What do you think about moving 'M' on your board to free up (3,6) for 'T'?"} \\
\\ 
\\ 
\texttt{What is your next command?} \\
            }
        }
    }
    & \\ \\

    \theutterance \stepcounter{utterance}  
    & \multicolumn{4}{p{0.6\linewidth}}{
        \cellcolor[rgb]{0.9,0.9,0.9}{
            \makecell[{{p{\linewidth}}}]{
                \texttt{\tiny{[P1$\rangle$GM]}}
                \texttt{SAY: Moving 'M' from (3,6) on my board to free up the space for 'T' is a good idea. Before I do that, I'd like to discuss a potential target cell for 'M'. On my board, 'M' is currently at (3,6). If I move 'M', I could potentially move it to a cell that is empty on both boards. How about (7,6)? Is (7,6) empty on your board? If it is, I could move 'M' to (7,6) and then move 'T' to (3,6). What do you think?} \\
            }
        }
    }
    & & \\ \\

    \theutterance \stepcounter{utterance}  
    & & & \multicolumn{2}{p{0.3\linewidth}}{
        \cellcolor[rgb]{0.9,0.9,0.9}{
            \makecell[{{p{\linewidth}}}]{
                \texttt{\tiny{[GM$|$GM]}}
                \texttt{SAY: Moving 'M' from (3,6) on my board to free up the space for 'T' is a good idea. Before I do that, I'd like to discuss a potential target cell for 'M'. On my board, 'M' is currently at (3,6). If I move 'M', I could potentially move it to a cell that is empty on both boards. How about (7,6)? Is (7,6) empty on your board? If it is, I could move 'M' to (7,6) and then move 'T' to (3,6). What do you think?} \\
            }
        }
    }
    & & \\ \\

    \theutterance \stepcounter{utterance}  
    & & \multicolumn{4}{p{0.6\linewidth}}{
        \cellcolor[rgb]{0.9,0.9,0.9}{
            \makecell[{{p{\linewidth}}}]{
                \texttt{\tiny{[GM$\rangle$P2]}}
                \texttt{Your message has been relayed to the other player.} \\
\\ 
\\ 
\texttt{You are currently playing round 14 of maximum 28.} \\
\texttt{You have collectively accumulated 5 of 16 penalties.} \\
\texttt{The other player sent this message:} \\
\\ 
\\ 
\texttt{"Moving 'M' from (3,6) on my board to free up the space for 'T' is a good idea. Before I do that, I'd like to discuss a potential target cell for 'M'. On my board, 'M' is currently at (3,6). If I move 'M', I could potentially move it to a cell that is empty on both boards. How about (7,6)? Is (7,6) empty on your board? If it is, I could move 'M' to (7,6) and then move 'T' to (3,6). What do you think?"} \\
\\ 
\\ 
\texttt{What is your next command?} \\
            }
        }
    }
    & \\ \\

    \theutterance \stepcounter{utterance}  
    & & & \multicolumn{4}{p{0.6\linewidth}}{
        \cellcolor[rgb]{0.9,0.9,0.9}{
            \makecell[{{p{\linewidth}}}]{
                \texttt{\tiny{[GM$\langle$P2]}}
                \texttt{SAY: On my board, (7,6) is not a valid cell, as the grid only goes up to column 4 and row 7, and the right edge is at column 7, but there's no column 7 with a cell that can be occupied by an object, only the right frame. However, I can check other cells. How about (4,7) or another cell? Is there another empty cell on your board where 'M' could be moved to free up (3,6) for 'T'?} \\
            }
        }
    }
     \\ \\

    \theutterance \stepcounter{utterance}  
    & & & \multicolumn{2}{p{0.3\linewidth}}{
        \cellcolor[rgb]{0.9,0.9,0.9}{
            \makecell[{{p{\linewidth}}}]{
                \texttt{\tiny{[GM$|$GM]}}
                \texttt{SAY: On my board, (7,6) is not a valid cell, as the grid only goes up to column 4 and row 7, and the right edge is at column 7, but there's no column 7 with a cell that can be occupied by an object, only the right frame. However, I can check other cells. How about (4,7) or another cell? Is there another empty cell on your board where 'M' could be moved to free up (3,6) for 'T'?} \\
            }
        }
    }
    & & \\ \\

    \theutterance \stepcounter{utterance}  
    & & \multicolumn{4}{p{0.6\linewidth}}{
        \cellcolor[rgb]{0.9,0.9,0.9}{
            \makecell[{{p{\linewidth}}}]{
                \texttt{\tiny{[P1$\langle$GM]}}
                \texttt{Your message has been relayed to the other player.} \\
\\ 
\\ 
\texttt{You are currently playing round 14 of maximum 28.} \\
\texttt{You have collectively accumulated 5 of 16 penalties.} \\
\texttt{The other player sent this message:} \\
\\ 
\\ 
\texttt{"On my board, (7,6) is not a valid cell, as the grid only goes up to column 4 and row 7, and the right edge is at column 7, but there's no column 7 with a cell that can be occupied by an object, only the right frame. However, I can check other cells. How about (4,7) or another cell? Is there another empty cell on your board where 'M' could be moved to free up (3,6) for 'T'?"} \\
\\ 
\\ 
\texttt{What is your next command?} \\
            }
        }
    }
    & \\ \\

    \theutterance \stepcounter{utterance}  
    & \multicolumn{4}{p{0.6\linewidth}}{
        \cellcolor[rgb]{0.9,0.9,0.9}{
            \makecell[{{p{\linewidth}}}]{
                \texttt{\tiny{[P1$\rangle$GM]}}
                \texttt{SAY: You are correct that the grid only goes up to column 4 with occupiable cells, and column 7 is the right frame. Let me check my board again for an alternative cell to move 'M' to. How about (4,7)? On my board, (4,7) is empty, as 'T' was previously there but has been moved to (2,5). If I move 'M' to (4,7), that would free up (3,6) for 'T'. Would that work for you? Could you potentially move 'S' from (3,6) on your board to make space for 'T' as well?} \\
            }
        }
    }
    & & \\ \\

    \theutterance \stepcounter{utterance}  
    & & & \multicolumn{2}{p{0.3\linewidth}}{
        \cellcolor[rgb]{0.9,0.9,0.9}{
            \makecell[{{p{\linewidth}}}]{
                \texttt{\tiny{[GM$|$GM]}}
                \texttt{SAY: You are correct that the grid only goes up to column 4 with occupiable cells, and column 7 is the right frame. Let me check my board again for an alternative cell to move 'M' to. How about (4,7)? On my board, (4,7) is empty, as 'T' was previously there but has been moved to (2,5). If I move 'M' to (4,7), that would free up (3,6) for 'T'. Would that work for you? Could you potentially move 'S' from (3,6) on your board to make space for 'T' as well?} \\
            }
        }
    }
    & & \\ \\

    \theutterance \stepcounter{utterance}  
    & & \multicolumn{4}{p{0.6\linewidth}}{
        \cellcolor[rgb]{0.9,0.9,0.9}{
            \makecell[{{p{\linewidth}}}]{
                \texttt{\tiny{[GM$\rangle$P2]}}
                \texttt{Your message has been relayed to the other player.} \\
\\ 
\\ 
\texttt{You are currently playing round 15 of maximum 28.} \\
\texttt{You have collectively accumulated 5 of 16 penalties.} \\
\texttt{The other player sent this message:} \\
\\ 
\\ 
\texttt{"You are correct that the grid only goes up to column 4 with occupiable cells, and column 7 is the right frame. Let me check my board again for an alternative cell to move 'M' to. How about (4,7)? On my board, (4,7) is empty, as 'T' was previously there but has been moved to (2,5). If I move 'M' to (4,7), that would free up (3,6) for 'T'. Would that work for you? Could you potentially move 'S' from (3,6) on your board to make space for 'T' as well?"} \\
\\ 
\\ 
\texttt{What is your next command?} \\
            }
        }
    }
    & \\ \\

    \theutterance \stepcounter{utterance}  
    & & & \multicolumn{4}{p{0.6\linewidth}}{
        \cellcolor[rgb]{0.9,0.9,0.9}{
            \makecell[{{p{\linewidth}}}]{
                \texttt{\tiny{[GM$\langle$P2]}}
                \texttt{SAY: On my board, (4,7) is not empty, as it contains 'A'. But I can consider moving 'S' from (5,6) to make space for 'T' at (3,6) or another cell. However, I think we should focus on moving 'M' to (4,7) on your board and then moving 'T' to (3,6) on your board. After that, we can discuss how to align 'T' on both boards. Please go ahead and move 'M' to (4,7) on your board, and then we'll discuss the next steps.} \\
            }
        }
    }
     \\ \\

    \theutterance \stepcounter{utterance}  
    & & & \multicolumn{2}{p{0.3\linewidth}}{
        \cellcolor[rgb]{0.9,0.9,0.9}{
            \makecell[{{p{\linewidth}}}]{
                \texttt{\tiny{[GM$|$GM]}}
                \texttt{SAY: On my board, (4,7) is not empty, as it contains 'A'. But I can consider moving 'S' from (5,6) to make space for 'T' at (3,6) or another cell. However, I think we should focus on moving 'M' to (4,7) on your board and then moving 'T' to (3,6) on your board. After that, we can discuss how to align 'T' on both boards. Please go ahead and move 'M' to (4,7) on your board, and then we'll discuss the next steps.} \\
            }
        }
    }
    & & \\ \\

    \theutterance \stepcounter{utterance}  
    & & \multicolumn{4}{p{0.6\linewidth}}{
        \cellcolor[rgb]{0.9,0.9,0.9}{
            \makecell[{{p{\linewidth}}}]{
                \texttt{\tiny{[P1$\langle$GM]}}
                \texttt{Your message has been relayed to the other player.} \\
\\ 
\\ 
\texttt{You are currently playing round 15 of maximum 28.} \\
\texttt{You have collectively accumulated 5 of 16 penalties.} \\
\texttt{The other player sent this message:} \\
\\ 
\\ 
\texttt{"On my board, (4,7) is not empty, as it contains 'A'. But I can consider moving 'S' from (5,6) to make space for 'T' at (3,6) or another cell. However, I think we should focus on moving 'M' to (4,7) on your board and then moving 'T' to (3,6) on your board. After that, we can discuss how to align 'T' on both boards. Please go ahead and move 'M' to (4,7) on your board, and then we'll discuss the next steps."} \\
\\ 
\\ 
\texttt{What is your next command?} \\
            }
        }
    }
    & \\ \\

    \theutterance \stepcounter{utterance}  
    & \multicolumn{4}{p{0.6\linewidth}}{
        \cellcolor[rgb]{0.9,0.9,0.9}{
            \makecell[{{p{\linewidth}}}]{
                \texttt{\tiny{[P1$\rangle$GM]}}
                \texttt{MOVE: M, (4, 7)} \\
            }
        }
    }
    & & \\ \\

    \theutterance \stepcounter{utterance}  
    & & & \multicolumn{2}{p{0.3\linewidth}}{
        \cellcolor[rgb]{0.9,0.9,0.9}{
            \makecell[{{p{\linewidth}}}]{
                \texttt{\tiny{[GM$|$GM]}}
                \texttt{MOVE: M, (4, 7)} \\
            }
        }
    }
    & & \\ \\

    \theutterance \stepcounter{utterance}  
    & & & \multicolumn{2}{p{0.3\linewidth}}{
        \cellcolor[rgb]{0.9,0.9,0.9}{
            \makecell[{{p{\linewidth}}}]{
                \texttt{\tiny{[GM$|$GM]}}
                \texttt{Moved 'M' to (4,7) successfully. Your updated grid looks like this:} \\
\texttt{```} \\
\texttt{ 1234567} \\
\texttt{╔═╤═╤═╤═╗} \\
\texttt{║P│◌│◌│◌║ 1} \\
\texttt{║◌└─┼─┴─╢ 2} \\
\texttt{║◌SA│◌◌◌║ 3} \\
\texttt{╟──┬┴───╢ 4} \\
\texttt{║◌T│◌◌O◌║ 5} \\
\texttt{║◌◌│◌◌◌D║ 6} \\
\texttt{║◌◌│M◌◌◌║ 7} \\
\texttt{╚══╧════╝} \\
\texttt{```} \\
            }
        }
    }
    & & \\ \\

    \theutterance \stepcounter{utterance}  
    & & \multicolumn{4}{p{0.6\linewidth}}{
        \cellcolor[rgb]{0.9,0.9,0.9}{
            \makecell[{{p{\linewidth}}}]{
                \texttt{\tiny{[GM$\rangle$P2]}}
                \texttt{Your message has been relayed to the other player.} \\
\\ 
\\ 
\texttt{You are currently playing round 16 of maximum 28.} \\
\texttt{You have collectively accumulated 5 of 16 penalties.} \\
\texttt{The other player moved an object on their grid.} \\
\\ 
\\ 
\texttt{What is your next command?} \\
            }
        }
    }
    & \\ \\

    \theutterance \stepcounter{utterance}  
    & & & \multicolumn{4}{p{0.6\linewidth}}{
        \cellcolor[rgb]{0.9,0.9,0.9}{
            \makecell[{{p{\linewidth}}}]{
                \texttt{\tiny{[GM$\langle$P2]}}
                \texttt{SAY: Now that you've moved 'M' to (4,7) on your board, I'll check my board again. Since we discussed moving 'T' to (3,6) on your board, I'll try to move 'S' from (5,6) to make space for 'T' at a different location. However, before I do that, I'd like to confirm the next steps. Should I try to move 'T' to (3,6) or a different location on my board to align it with your board?} \\
            }
        }
    }
     \\ \\

    \theutterance \stepcounter{utterance}  
    & & & \multicolumn{2}{p{0.3\linewidth}}{
        \cellcolor[rgb]{0.9,0.9,0.9}{
            \makecell[{{p{\linewidth}}}]{
                \texttt{\tiny{[GM$|$GM]}}
                \texttt{SAY: Now that you've moved 'M' to (4,7) on your board, I'll check my board again. Since we discussed moving 'T' to (3,6) on your board, I'll try to move 'S' from (5,6) to make space for 'T' at a different location. However, before I do that, I'd like to confirm the next steps. Should I try to move 'T' to (3,6) or a different location on my board to align it with your board?} \\
            }
        }
    }
    & & \\ \\

    \theutterance \stepcounter{utterance}  
    & & \multicolumn{4}{p{0.6\linewidth}}{
        \cellcolor[rgb]{0.9,0.9,0.9}{
            \makecell[{{p{\linewidth}}}]{
                \texttt{\tiny{[P1$\langle$GM]}}
                \texttt{Moved 'M' to (4,7) successfully. Your updated grid looks like this:} \\
\texttt{```} \\
\texttt{ 1234567} \\
\texttt{╔═╤═╤═╤═╗} \\
\texttt{║P│◌│◌│◌║ 1} \\
\texttt{║◌└─┼─┴─╢ 2} \\
\texttt{║◌SA│◌◌◌║ 3} \\
\texttt{╟──┬┴───╢ 4} \\
\texttt{║◌T│◌◌O◌║ 5} \\
\texttt{║◌◌│◌◌◌D║ 6} \\
\texttt{║◌◌│M◌◌◌║ 7} \\
\texttt{╚══╧════╝} \\
\texttt{```} \\
\texttt{You are currently playing round 16 of maximum 28.} \\
\texttt{You have collectively accumulated 5 of 16 penalties.} \\
\texttt{The other player sent this message:} \\
\\ 
\\ 
\texttt{"Now that you've moved 'M' to (4,7) on your board, I'll check my board again. Since we discussed moving 'T' to (3,6) on your board, I'll try to move 'S' from (5,6) to make space for 'T' at a different location. However, before I do that, I'd like to confirm the next steps. Should I try to move 'T' to (3,6) or a different location on my board to align it with your board?"} \\
\\ 
\\ 
\texttt{What is your next command?} \\
            }
        }
    }
    & \\ \\

    \theutterance \stepcounter{utterance}  
    & \multicolumn{4}{p{0.6\linewidth}}{
        \cellcolor[rgb]{0.9,0.9,0.9}{
            \makecell[{{p{\linewidth}}}]{
                \texttt{\tiny{[P1$\rangle$GM]}}
                \texttt{SAY: Now that 'M' is at (4,7) on my board, the next step would be to move 'T' to (3,6) on my board. As for your board, since 'T' is currently at (2,5) on my board, and we want to align it with your board, could you please try to move 'T' to (3,6) on your board as well? If that's not possible, we can discuss alternative locations. But ideally, we'd like to have 'T' at (3,6) on both boards to maintain consistency. Please let me know if that's feasible on your end.} \\
            }
        }
    }
    & & \\ \\

    \theutterance \stepcounter{utterance}  
    & & & \multicolumn{2}{p{0.3\linewidth}}{
        \cellcolor[rgb]{0.9,0.9,0.9}{
            \makecell[{{p{\linewidth}}}]{
                \texttt{\tiny{[GM$|$GM]}}
                \texttt{SAY: Now that 'M' is at (4,7) on my board, the next step would be to move 'T' to (3,6) on my board. As for your board, since 'T' is currently at (2,5) on my board, and we want to align it with your board, could you please try to move 'T' to (3,6) on your board as well? If that's not possible, we can discuss alternative locations. But ideally, we'd like to have 'T' at (3,6) on both boards to maintain consistency. Please let me know if that's feasible on your end.} \\
            }
        }
    }
    & & \\ \\

    \theutterance \stepcounter{utterance}  
    & & \multicolumn{4}{p{0.6\linewidth}}{
        \cellcolor[rgb]{0.9,0.9,0.9}{
            \makecell[{{p{\linewidth}}}]{
                \texttt{\tiny{[GM$\rangle$P2]}}
                \texttt{Your message has been relayed to the other player.} \\
\\ 
\\ 
\texttt{You are currently playing round 17 of maximum 28.} \\
\texttt{You have collectively accumulated 5 of 16 penalties.} \\
\texttt{The other player sent this message:} \\
\\ 
\\ 
\texttt{"Now that 'M' is at (4,7) on my board, the next step would be to move 'T' to (3,6) on my board. As for your board, since 'T' is currently at (2,5) on my board, and we want to align it with your board, could you please try to move 'T' to (3,6) on your board as well? If that's not possible, we can discuss alternative locations. But ideally, we'd like to have 'T' at (3,6) on both boards to maintain consistency. Please let me know if that's feasible on your end."} \\
\\ 
\\ 
\texttt{What is your next command?} \\
            }
        }
    }
    & \\ \\

    \theutterance \stepcounter{utterance}  
    & & & \multicolumn{4}{p{0.6\linewidth}}{
        \cellcolor[rgb]{0.9,0.9,0.9}{
            \makecell[{{p{\linewidth}}}]{
                \texttt{\tiny{[GM$\langle$P2]}}
                \texttt{MOVE: T, (3,6)} \\
            }
        }
    }
     \\ \\

    \theutterance \stepcounter{utterance}  
    & & & \multicolumn{2}{p{0.3\linewidth}}{
        \cellcolor[rgb]{0.9,0.9,0.9}{
            \makecell[{{p{\linewidth}}}]{
                \texttt{\tiny{[GM$|$GM]}}
                \texttt{MOVE: T, (3,6)} \\
            }
        }
    }
    & & \\ \\

    \theutterance \stepcounter{utterance}  
    & & & \multicolumn{2}{p{0.3\linewidth}}{
        \cellcolor[rgb]{0.9,0.9,0.9}{
            \makecell[{{p{\linewidth}}}]{
                \texttt{\tiny{[GM$|$GM]}}
                \texttt{Penalty: (3,6) is not empty, but contains '│'.} \\
\\ 
\texttt{You have collectively accumulated 6 of 16 penalties.} \\
\texttt{Please try again!} \\
            }
        }
    }
    & & \\ \\

    \theutterance \stepcounter{utterance}  
    & & \multicolumn{4}{p{0.6\linewidth}}{
        \cellcolor[rgb]{0.9,0.9,0.9}{
            \makecell[{{p{\linewidth}}}]{
                \texttt{\tiny{[GM$\rangle$P2]}}
                \texttt{Penalty: (3,6) is not empty, but contains '│'.} \\
\\ 
\texttt{You have collectively accumulated 6 of 16 penalties.} \\
\texttt{Please try again!} \\
            }
        }
    }
    & \\ \\

    \theutterance \stepcounter{utterance}  
    & & & \multicolumn{4}{p{0.6\linewidth}}{
        \cellcolor[rgb]{0.9,0.9,0.9}{
            \makecell[{{p{\linewidth}}}]{
                \texttt{\tiny{[GM$\langle$P2]}}
                \texttt{SAY: It seems that (3,6) is not empty on my board, it contains '│'. Let me check my board again for an alternative location for 'T'. Since 'T' needs to be aligned with your board, I'll look for another empty cell that could be a possible target for 'T'. Could we discuss other possible locations for 'T' that might be empty on both boards? Alternatively, I can try to move 'S' from (5,6) to make space for 'T' at a different location. What do you think?} \\
            }
        }
    }
     \\ \\

    \theutterance \stepcounter{utterance}  
    & & & \multicolumn{2}{p{0.3\linewidth}}{
        \cellcolor[rgb]{0.9,0.9,0.9}{
            \makecell[{{p{\linewidth}}}]{
                \texttt{\tiny{[GM$|$GM]}}
                \texttt{SAY: It seems that (3,6) is not empty on my board, it contains '│'. Let me check my board again for an alternative location for 'T'. Since 'T' needs to be aligned with your board, I'll look for another empty cell that could be a possible target for 'T'. Could we discuss other possible locations for 'T' that might be empty on both boards? Alternatively, I can try to move 'S' from (5,6) to make space for 'T' at a different location. What do you think?} \\
            }
        }
    }
    & & \\ \\

    \theutterance \stepcounter{utterance}  
    & & \multicolumn{4}{p{0.6\linewidth}}{
        \cellcolor[rgb]{0.9,0.9,0.9}{
            \makecell[{{p{\linewidth}}}]{
                \texttt{\tiny{[P1$\langle$GM]}}
                \texttt{Your message has been relayed to the other player.} \\
\\ 
\\ 
\texttt{You are currently playing round 17 of maximum 28.} \\
\texttt{You have collectively accumulated 6 of 16 penalties.} \\
\texttt{The other player sent this message:} \\
\\ 
\\ 
\texttt{"It seems that (3,6) is not empty on my board, it contains '│'. Let me check my board again for an alternative location for 'T'. Since 'T' needs to be aligned with your board, I'll look for another empty cell that could be a possible target for 'T'. Could we discuss other possible locations for 'T' that might be empty on both boards? Alternatively, I can try to move 'S' from (5,6) to make space for 'T' at a different location. What do you think?"} \\
\\ 
\\ 
\texttt{What is your next command?} \\
            }
        }
    }
    & \\ \\

    \theutterance \stepcounter{utterance}  
    & \multicolumn{4}{p{0.6\linewidth}}{
        \cellcolor[rgb]{0.9,0.9,0.9}{
            \makecell[{{p{\linewidth}}}]{
                \texttt{\tiny{[P1$\rangle$GM]}}
                \texttt{SAY: Since (3,6) is not empty on your board, let's explore other options. On my board, I can move 'T' to (3,6) as planned, but we need to find a corresponding location on your board. How about (3,5)? Is (3,5) empty on your board? If it is, we could move 'T' to (3,5) on your board and (3,6) on my board, and then discuss further adjustments to align the objects. Alternatively, if moving 'S' from (5,6) can create space for 'T', that's also an option. What do you think about trying (3,5) or moving 'S' to create space?} \\
            }
        }
    }
    & & \\ \\

    \theutterance \stepcounter{utterance}  
    & & & \multicolumn{2}{p{0.3\linewidth}}{
        \cellcolor[rgb]{0.9,0.9,0.9}{
            \makecell[{{p{\linewidth}}}]{
                \texttt{\tiny{[GM$|$GM]}}
                \texttt{SAY: Since (3,6) is not empty on your board, let's explore other options. On my board, I can move 'T' to (3,6) as planned, but we need to find a corresponding location on your board. How about (3,5)? Is (3,5) empty on your board? If it is, we could move 'T' to (3,5) on your board and (3,6) on my board, and then discuss further adjustments to align the objects. Alternatively, if moving 'S' from (5,6) can create space for 'T', that's also an option. What do you think about trying (3,5) or moving 'S' to create space?} \\
            }
        }
    }
    & & \\ \\

    \theutterance \stepcounter{utterance}  
    & & \multicolumn{4}{p{0.6\linewidth}}{
        \cellcolor[rgb]{0.9,0.9,0.9}{
            \makecell[{{p{\linewidth}}}]{
                \texttt{\tiny{[GM$\rangle$P2]}}
                \texttt{Your message has been relayed to the other player.} \\
\\ 
\\ 
\texttt{You are currently playing round 18 of maximum 28.} \\
\texttt{You have collectively accumulated 6 of 16 penalties.} \\
\texttt{The other player sent this message:} \\
\\ 
\\ 
\texttt{"Since (3,6) is not empty on your board, let's explore other options. On my board, I can move 'T' to (3,6) as planned, but we need to find a corresponding location on your board. How about (3,5)? Is (3,5) empty on your board? If it is, we could move 'T' to (3,5) on your board and (3,6) on my board, and then discuss further adjustments to align the objects. Alternatively, if moving 'S' from (5,6) can create space for 'T', that's also an option. What do you think about trying (3,5) or moving 'S' to create space?"} \\
\\ 
\\ 
\texttt{What is your next command?} \\
            }
        }
    }
    & \\ \\

    \theutterance \stepcounter{utterance}  
    & & & \multicolumn{4}{p{0.6\linewidth}}{
        \cellcolor[rgb]{0.9,0.9,0.9}{
            \makecell[{{p{\linewidth}}}]{
                \texttt{\tiny{[GM$\langle$P2]}}
                \texttt{SAY: On my board, (3,5) is empty, it contains '◌'. Moving 'T' to (3,5) on my board could be a good option. Since 'T' will be at (3,6) on your board, we can discuss further adjustments to align the objects after this move. I'm ready to move 'T' to (3,5) on my board. Please confirm when you're ready to move 'T' to (3,6) on your board.} \\
            }
        }
    }
     \\ \\

    \theutterance \stepcounter{utterance}  
    & & & \multicolumn{2}{p{0.3\linewidth}}{
        \cellcolor[rgb]{0.9,0.9,0.9}{
            \makecell[{{p{\linewidth}}}]{
                \texttt{\tiny{[GM$|$GM]}}
                \texttt{SAY: On my board, (3,5) is empty, it contains '◌'. Moving 'T' to (3,5) on my board could be a good option. Since 'T' will be at (3,6) on your board, we can discuss further adjustments to align the objects after this move. I'm ready to move 'T' to (3,5) on my board. Please confirm when you're ready to move 'T' to (3,6) on your board.} \\
            }
        }
    }
    & & \\ \\

    \theutterance \stepcounter{utterance}  
    & & \multicolumn{4}{p{0.6\linewidth}}{
        \cellcolor[rgb]{0.9,0.9,0.9}{
            \makecell[{{p{\linewidth}}}]{
                \texttt{\tiny{[P1$\langle$GM]}}
                \texttt{Your message has been relayed to the other player.} \\
\\ 
\\ 
\texttt{You are currently playing round 18 of maximum 28.} \\
\texttt{You have collectively accumulated 6 of 16 penalties.} \\
\texttt{The other player sent this message:} \\
\\ 
\\ 
\texttt{"On my board, (3,5) is empty, it contains '◌'. Moving 'T' to (3,5) on my board could be a good option. Since 'T' will be at (3,6) on your board, we can discuss further adjustments to align the objects after this move. I'm ready to move 'T' to (3,5) on my board. Please confirm when you're ready to move 'T' to (3,6) on your board."} \\
\\ 
\\ 
\texttt{What is your next command?} \\
            }
        }
    }
    & \\ \\

    \theutterance \stepcounter{utterance}  
    & \multicolumn{4}{p{0.6\linewidth}}{
        \cellcolor[rgb]{0.9,0.9,0.9}{
            \makecell[{{p{\linewidth}}}]{
                \texttt{\tiny{[P1$\rangle$GM]}}
                \texttt{MOVE: T, (3, 6)} \\
            }
        }
    }
    & & \\ \\

    \theutterance \stepcounter{utterance}  
    & & & \multicolumn{2}{p{0.3\linewidth}}{
        \cellcolor[rgb]{0.9,0.9,0.9}{
            \makecell[{{p{\linewidth}}}]{
                \texttt{\tiny{[GM$|$GM]}}
                \texttt{MOVE: T, (3, 6)} \\
            }
        }
    }
    & & \\ \\

    \theutterance \stepcounter{utterance}  
    & & & \multicolumn{2}{p{0.3\linewidth}}{
        \cellcolor[rgb]{0.9,0.9,0.9}{
            \makecell[{{p{\linewidth}}}]{
                \texttt{\tiny{[GM$|$GM]}}
                \texttt{Penalty: (3,6) is not empty, but contains '│'.} \\
\\ 
\texttt{You have collectively accumulated 7 of 16 penalties.} \\
\texttt{Please try again!} \\
            }
        }
    }
    & & \\ \\

    \theutterance \stepcounter{utterance}  
    & & \multicolumn{4}{p{0.6\linewidth}}{
        \cellcolor[rgb]{0.9,0.9,0.9}{
            \makecell[{{p{\linewidth}}}]{
                \texttt{\tiny{[P1$\langle$GM]}}
                \texttt{Penalty: (3,6) is not empty, but contains '│'.} \\
\\ 
\texttt{You have collectively accumulated 7 of 16 penalties.} \\
\texttt{Please try again!} \\
            }
        }
    }
    & \\ \\

    \theutterance \stepcounter{utterance}  
    & \multicolumn{4}{p{0.6\linewidth}}{
        \cellcolor[rgb]{0.9,0.9,0.9}{
            \makecell[{{p{\linewidth}}}]{
                \texttt{\tiny{[P1$\rangle$GM]}}
                \texttt{SAY: It seems that I made an error in my previous move. Since (3,6) is not empty on my board, I'll need to choose a different target cell for 'T'. However, I'd like to discuss this with you first. Considering that you're moving 'T' to (3,5) on your board, I could try to move 'D' from (4,6) to make space for 'T' at a different location. Alternatively, I could try to move 'T' to a different cell that is empty on my board. What do you think? Should I try to move 'D' or explore other options for 'T'?} \\
            }
        }
    }
    & & \\ \\

    \theutterance \stepcounter{utterance}  
    & & & \multicolumn{2}{p{0.3\linewidth}}{
        \cellcolor[rgb]{0.9,0.9,0.9}{
            \makecell[{{p{\linewidth}}}]{
                \texttt{\tiny{[GM$|$GM]}}
                \texttt{SAY: It seems that I made an error in my previous move. Since (3,6) is not empty on my board, I'll need to choose a different target cell for 'T'. However, I'd like to discuss this with you first. Considering that you're moving 'T' to (3,5) on your board, I could try to move 'D' from (4,6) to make space for 'T' at a different location. Alternatively, I could try to move 'T' to a different cell that is empty on my board. What do you think? Should I try to move 'D' or explore other options for 'T'?} \\
            }
        }
    }
    & & \\ \\

    \theutterance \stepcounter{utterance}  
    & & \multicolumn{4}{p{0.6\linewidth}}{
        \cellcolor[rgb]{0.9,0.9,0.9}{
            \makecell[{{p{\linewidth}}}]{
                \texttt{\tiny{[GM$\rangle$P2]}}
                \texttt{Your message has been relayed to the other player.} \\
\\ 
\\ 
\texttt{You are currently playing round 19 of maximum 28.} \\
\texttt{You have collectively accumulated 7 of 16 penalties.} \\
\texttt{The other player sent this message:} \\
\\ 
\\ 
\texttt{"It seems that I made an error in my previous move. Since (3,6) is not empty on my board, I'll need to choose a different target cell for 'T'. However, I'd like to discuss this with you first. Considering that you're moving 'T' to (3,5) on your board, I could try to move 'D' from (4,6) to make space for 'T' at a different location. Alternatively, I could try to move 'T' to a different cell that is empty on my board. What do you think? Should I try to move 'D' or explore other options for 'T'?"} \\
\\ 
\\ 
\texttt{What is your next command?} \\
            }
        }
    }
    & \\ \\

    \theutterance \stepcounter{utterance}  
    & & & \multicolumn{4}{p{0.6\linewidth}}{
        \cellcolor[rgb]{0.9,0.9,0.9}{
            \makecell[{{p{\linewidth}}}]{
                \texttt{\tiny{[GM$\langle$P2]}}
                \texttt{SAY: Given that I'm moving 'T' to (3,5) on my board, it would be ideal to find a corresponding location on your board that is empty. Since (3,6) is not empty on your board, let's explore other options. If you could move 'D' from (4,6) to make space for 'T', that could be a good option. Alternatively, you could try to move 'T' to a different cell that is empty on your board. Could you please check your board for any empty cells that could be a potential target for 'T'? Maybe we can find a cell that is empty on both boards or close to (3,5) on my board.} \\
            }
        }
    }
     \\ \\

    \theutterance \stepcounter{utterance}  
    & & & \multicolumn{2}{p{0.3\linewidth}}{
        \cellcolor[rgb]{0.9,0.9,0.9}{
            \makecell[{{p{\linewidth}}}]{
                \texttt{\tiny{[GM$|$GM]}}
                \texttt{SAY: Given that I'm moving 'T' to (3,5) on my board, it would be ideal to find a corresponding location on your board that is empty. Since (3,6) is not empty on your board, let's explore other options. If you could move 'D' from (4,6) to make space for 'T', that could be a good option. Alternatively, you could try to move 'T' to a different cell that is empty on your board. Could you please check your board for any empty cells that could be a potential target for 'T'? Maybe we can find a cell that is empty on both boards or close to (3,5) on my board.} \\
            }
        }
    }
    & & \\ \\

    \theutterance \stepcounter{utterance}  
    & & \multicolumn{4}{p{0.6\linewidth}}{
        \cellcolor[rgb]{0.9,0.9,0.9}{
            \makecell[{{p{\linewidth}}}]{
                \texttt{\tiny{[P1$\langle$GM]}}
                \texttt{Your message has been relayed to the other player.} \\
\\ 
\\ 
\texttt{You are currently playing round 19 of maximum 28.} \\
\texttt{You have collectively accumulated 7 of 16 penalties.} \\
\texttt{The other player sent this message:} \\
\\ 
\\ 
\texttt{"Given that I'm moving 'T' to (3,5) on my board, it would be ideal to find a corresponding location on your board that is empty. Since (3,6) is not empty on your board, let's explore other options. If you could move 'D' from (4,6) to make space for 'T', that could be a good option. Alternatively, you could try to move 'T' to a different cell that is empty on your board. Could you please check your board for any empty cells that could be a potential target for 'T'? Maybe we can find a cell that is empty on both boards or close to (3,5) on my board."} \\
\\ 
\\ 
\texttt{What is your next command?} \\
            }
        }
    }
    & \\ \\

    \theutterance \stepcounter{utterance}  
    & \multicolumn{4}{p{0.6\linewidth}}{
        \cellcolor[rgb]{0.9,0.9,0.9}{
            \makecell[{{p{\linewidth}}}]{
                \texttt{\tiny{[P1$\rangle$GM]}}
                \texttt{SAY: I've checked my board, and I found a potential target cell for 'T'. How about (3,5)? Is (3,5) empty on your board? On my board, (3,5) contains '◌', which is empty. If we can sync 'T' to (3,5) on both boards, that would be ideal. If not, I can try to move 'D' from (4,6) to make space for 'T' at a different location, as we discussed earlier. Please let me know if (3,5) is a viable option on your board.} \\
            }
        }
    }
    & & \\ \\

    \theutterance \stepcounter{utterance}  
    & & & \multicolumn{2}{p{0.3\linewidth}}{
        \cellcolor[rgb]{0.9,0.9,0.9}{
            \makecell[{{p{\linewidth}}}]{
                \texttt{\tiny{[GM$|$GM]}}
                \texttt{SAY: I've checked my board, and I found a potential target cell for 'T'. How about (3,5)? Is (3,5) empty on your board? On my board, (3,5) contains '◌', which is empty. If we can sync 'T' to (3,5) on both boards, that would be ideal. If not, I can try to move 'D' from (4,6) to make space for 'T' at a different location, as we discussed earlier. Please let me know if (3,5) is a viable option on your board.} \\
            }
        }
    }
    & & \\ \\

    \theutterance \stepcounter{utterance}  
    & & \multicolumn{4}{p{0.6\linewidth}}{
        \cellcolor[rgb]{0.9,0.9,0.9}{
            \makecell[{{p{\linewidth}}}]{
                \texttt{\tiny{[GM$\rangle$P2]}}
                \texttt{Your message has been relayed to the other player.} \\
\\ 
\\ 
\texttt{You are currently playing round 20 of maximum 28.} \\
\texttt{You have collectively accumulated 7 of 16 penalties.} \\
\texttt{The other player sent this message:} \\
\\ 
\\ 
\texttt{"I've checked my board, and I found a potential target cell for 'T'. How about (3,5)? Is (3,5) empty on your board? On my board, (3,5) contains '◌', which is empty. If we can sync 'T' to (3,5) on both boards, that would be ideal. If not, I can try to move 'D' from (4,6) to make space for 'T' at a different location, as we discussed earlier. Please let me know if (3,5) is a viable option on your board."} \\
\\ 
\\ 
\texttt{What is your next command?} \\
            }
        }
    }
    & \\ \\

    \theutterance \stepcounter{utterance}  
    & & & \multicolumn{4}{p{0.6\linewidth}}{
        \cellcolor[rgb]{0.9,0.9,0.9}{
            \makecell[{{p{\linewidth}}}]{
                \texttt{\tiny{[GM$\langle$P2]}}
                \texttt{MOVE: T, (3,5)} \\
            }
        }
    }
     \\ \\

    \theutterance \stepcounter{utterance}  
    & & & \multicolumn{2}{p{0.3\linewidth}}{
        \cellcolor[rgb]{0.9,0.9,0.9}{
            \makecell[{{p{\linewidth}}}]{
                \texttt{\tiny{[GM$|$GM]}}
                \texttt{MOVE: T, (3,5)} \\
            }
        }
    }
    & & \\ \\

    \theutterance \stepcounter{utterance}  
    & & & \multicolumn{2}{p{0.3\linewidth}}{
        \cellcolor[rgb]{0.9,0.9,0.9}{
            \makecell[{{p{\linewidth}}}]{
                \texttt{\tiny{[GM$|$GM]}}
                \texttt{Penalty: (3,5) is not empty, but contains '│'.} \\
\\ 
\texttt{You have collectively accumulated 8 of 16 penalties.} \\
\texttt{Please try again!} \\
            }
        }
    }
    & & \\ \\

    \theutterance \stepcounter{utterance}  
    & & \multicolumn{4}{p{0.6\linewidth}}{
        \cellcolor[rgb]{0.9,0.9,0.9}{
            \makecell[{{p{\linewidth}}}]{
                \texttt{\tiny{[GM$\rangle$P2]}}
                \texttt{Penalty: (3,5) is not empty, but contains '│'.} \\
\\ 
\texttt{You have collectively accumulated 8 of 16 penalties.} \\
\texttt{Please try again!} \\
            }
        }
    }
    & \\ \\

    \theutterance \stepcounter{utterance}  
    & & & \multicolumn{4}{p{0.6\linewidth}}{
        \cellcolor[rgb]{0.9,0.9,0.9}{
            \makecell[{{p{\linewidth}}}]{
                \texttt{\tiny{[GM$\langle$P2]}}
                \texttt{SAY: It seems that (3,5) is not empty on my board, it contains '│'. I remember that I previously said (3,5) was empty, but that was incorrect. Let me re{-}check my board for an alternative location for 'T'. Could we discuss other possible locations for 'T' that might be empty on both boards? Alternatively, you can try to move 'D' from (4,6) to make space for 'T' at a different location, as we discussed earlier. What do you think?} \\
            }
        }
    }
     \\ \\

    \theutterance \stepcounter{utterance}  
    & & & \multicolumn{2}{p{0.3\linewidth}}{
        \cellcolor[rgb]{0.9,0.9,0.9}{
            \makecell[{{p{\linewidth}}}]{
                \texttt{\tiny{[GM$|$GM]}}
                \texttt{SAY: It seems that (3,5) is not empty on my board, it contains '│'. I remember that I previously said (3,5) was empty, but that was incorrect. Let me re{-}check my board for an alternative location for 'T'. Could we discuss other possible locations for 'T' that might be empty on both boards? Alternatively, you can try to move 'D' from (4,6) to make space for 'T' at a different location, as we discussed earlier. What do you think?} \\
            }
        }
    }
    & & \\ \\

    \theutterance \stepcounter{utterance}  
    & & \multicolumn{4}{p{0.6\linewidth}}{
        \cellcolor[rgb]{0.9,0.9,0.9}{
            \makecell[{{p{\linewidth}}}]{
                \texttt{\tiny{[P1$\langle$GM]}}
                \texttt{Your message has been relayed to the other player.} \\
\\ 
\\ 
\texttt{You are currently playing round 20 of maximum 28.} \\
\texttt{You have collectively accumulated 8 of 16 penalties.} \\
\texttt{The other player sent this message:} \\
\\ 
\\ 
\texttt{"It seems that (3,5) is not empty on my board, it contains '│'. I remember that I previously said (3,5) was empty, but that was incorrect. Let me re{-}check my board for an alternative location for 'T'. Could we discuss other possible locations for 'T' that might be empty on both boards? Alternatively, you can try to move 'D' from (4,6) to make space for 'T' at a different location, as we discussed earlier. What do you think?"} \\
\\ 
\\ 
\texttt{What is your next command?} \\
            }
        }
    }
    & \\ \\

    \theutterance \stepcounter{utterance}  
    & \multicolumn{4}{p{0.6\linewidth}}{
        \cellcolor[rgb]{0.9,0.9,0.9}{
            \makecell[{{p{\linewidth}}}]{
                \texttt{\tiny{[P1$\rangle$GM]}}
                \texttt{SAY: Given that (3,5) is not empty on your board, let's explore other options. I can try to move 'D' from (4,6) to make space for 'T' at a different location. Before I do that, I'd like to confirm that moving 'D' will create a viable path for 'T'. On my board, if I move 'D' from (4,6), I could potentially move 'T' to (4,6). Would that be a good option? Could you please check your board and see if (4,6) is empty on your board as well? If it is, we could try to move 'T' to (4,6) on both boards.} \\
            }
        }
    }
    & & \\ \\

    \theutterance \stepcounter{utterance}  
    & & & \multicolumn{2}{p{0.3\linewidth}}{
        \cellcolor[rgb]{0.9,0.9,0.9}{
            \makecell[{{p{\linewidth}}}]{
                \texttt{\tiny{[GM$|$GM]}}
                \texttt{SAY: Given that (3,5) is not empty on your board, let's explore other options. I can try to move 'D' from (4,6) to make space for 'T' at a different location. Before I do that, I'd like to confirm that moving 'D' will create a viable path for 'T'. On my board, if I move 'D' from (4,6), I could potentially move 'T' to (4,6). Would that be a good option? Could you please check your board and see if (4,6) is empty on your board as well? If it is, we could try to move 'T' to (4,6) on both boards.} \\
            }
        }
    }
    & & \\ \\

    \theutterance \stepcounter{utterance}  
    & & \multicolumn{4}{p{0.6\linewidth}}{
        \cellcolor[rgb]{0.9,0.9,0.9}{
            \makecell[{{p{\linewidth}}}]{
                \texttt{\tiny{[GM$\rangle$P2]}}
                \texttt{Your message has been relayed to the other player.} \\
\\ 
\\ 
\texttt{You are currently playing round 21 of maximum 28.} \\
\texttt{You have collectively accumulated 8 of 16 penalties.} \\
\texttt{The other player sent this message:} \\
\\ 
\\ 
\texttt{"Given that (3,5) is not empty on your board, let's explore other options. I can try to move 'D' from (4,6) to make space for 'T' at a different location. Before I do that, I'd like to confirm that moving 'D' will create a viable path for 'T'. On my board, if I move 'D' from (4,6), I could potentially move 'T' to (4,6). Would that be a good option? Could you please check your board and see if (4,6) is empty on your board as well? If it is, we could try to move 'T' to (4,6) on both boards."} \\
\\ 
\\ 
\texttt{What is your next command?} \\
            }
        }
    }
    & \\ \\

    \theutterance \stepcounter{utterance}  
    & & & \multicolumn{4}{p{0.6\linewidth}}{
        \cellcolor[rgb]{0.9,0.9,0.9}{
            \makecell[{{p{\linewidth}}}]{
                \texttt{\tiny{[GM$\langle$P2]}}
                \texttt{SAY: On my board, (4,6) is not empty, it contains 'S'. However, I can try to move 'S' to make space for 'T' at (4,6) if needed. But before we do that, let me check if there are any other empty cells on my board that could be a possible target for 'T'. Could we discuss other possible locations for 'T' that might be empty on both boards? Alternatively, if moving 'S' from (5,6) to create space for 'T' at (4,6) is the best option, I can try to do that. What do you think?} \\
            }
        }
    }
     \\ \\

    \theutterance \stepcounter{utterance}  
    & & & \multicolumn{2}{p{0.3\linewidth}}{
        \cellcolor[rgb]{0.9,0.9,0.9}{
            \makecell[{{p{\linewidth}}}]{
                \texttt{\tiny{[GM$|$GM]}}
                \texttt{SAY: On my board, (4,6) is not empty, it contains 'S'. However, I can try to move 'S' to make space for 'T' at (4,6) if needed. But before we do that, let me check if there are any other empty cells on my board that could be a possible target for 'T'. Could we discuss other possible locations for 'T' that might be empty on both boards? Alternatively, if moving 'S' from (5,6) to create space for 'T' at (4,6) is the best option, I can try to do that. What do you think?} \\
            }
        }
    }
    & & \\ \\

    \theutterance \stepcounter{utterance}  
    & & \multicolumn{4}{p{0.6\linewidth}}{
        \cellcolor[rgb]{0.9,0.9,0.9}{
            \makecell[{{p{\linewidth}}}]{
                \texttt{\tiny{[P1$\langle$GM]}}
                \texttt{Your message has been relayed to the other player.} \\
\\ 
\\ 
\texttt{You are currently playing round 21 of maximum 28.} \\
\texttt{You have collectively accumulated 8 of 16 penalties.} \\
\texttt{The other player sent this message:} \\
\\ 
\\ 
\texttt{"On my board, (4,6) is not empty, it contains 'S'. However, I can try to move 'S' to make space for 'T' at (4,6) if needed. But before we do that, let me check if there are any other empty cells on my board that could be a possible target for 'T'. Could we discuss other possible locations for 'T' that might be empty on both boards? Alternatively, if moving 'S' from (5,6) to create space for 'T' at (4,6) is the best option, I can try to do that. What do you think?"} \\
\\ 
\\ 
\texttt{What is your next command?} \\
            }
        }
    }
    & \\ \\

    \theutterance \stepcounter{utterance}  
    & \multicolumn{4}{p{0.6\linewidth}}{
        \cellcolor[rgb]{0.9,0.9,0.9}{
            \makecell[{{p{\linewidth}}}]{
                \texttt{\tiny{[P1$\rangle$GM]}}
                \texttt{SAY: It seems that (4,6) is not empty on your board, and it contains 'S'. Moving 'S' to create space for 'T' at (4,6) could be a viable option. Before we proceed, I'd like to confirm that moving 'D' from (4,6) on my board will create space for 'T'. If I move 'D', I can then move 'T' to (4,6) on my board. If you move 'S' from (5,6) or (4,6) on your board, we can try to align 'T' at (4,6) on both boards. Let's try to coordinate our moves to achieve this. Shall we proceed with moving 'D' on my board and 'S' on your board to make space for 'T' at (4,6)?} \\
            }
        }
    }
    & & \\ \\

    \theutterance \stepcounter{utterance}  
    & & & \multicolumn{2}{p{0.3\linewidth}}{
        \cellcolor[rgb]{0.9,0.9,0.9}{
            \makecell[{{p{\linewidth}}}]{
                \texttt{\tiny{[GM$|$GM]}}
                \texttt{SAY: It seems that (4,6) is not empty on your board, and it contains 'S'. Moving 'S' to create space for 'T' at (4,6) could be a viable option. Before we proceed, I'd like to confirm that moving 'D' from (4,6) on my board will create space for 'T'. If I move 'D', I can then move 'T' to (4,6) on my board. If you move 'S' from (5,6) or (4,6) on your board, we can try to align 'T' at (4,6) on both boards. Let's try to coordinate our moves to achieve this. Shall we proceed with moving 'D' on my board and 'S' on your board to make space for 'T' at (4,6)?} \\
            }
        }
    }
    & & \\ \\

    \theutterance \stepcounter{utterance}  
    & & \multicolumn{4}{p{0.6\linewidth}}{
        \cellcolor[rgb]{0.9,0.9,0.9}{
            \makecell[{{p{\linewidth}}}]{
                \texttt{\tiny{[GM$\rangle$P2]}}
                \texttt{Your message has been relayed to the other player.} \\
\\ 
\\ 
\texttt{You are currently playing round 22 of maximum 28.} \\
\texttt{You have collectively accumulated 8 of 16 penalties.} \\
\texttt{The other player sent this message:} \\
\\ 
\\ 
\texttt{"It seems that (4,6) is not empty on your board, and it contains 'S'. Moving 'S' to create space for 'T' at (4,6) could be a viable option. Before we proceed, I'd like to confirm that moving 'D' from (4,6) on my board will create space for 'T'. If I move 'D', I can then move 'T' to (4,6) on my board. If you move 'S' from (5,6) or (4,6) on your board, we can try to align 'T' at (4,6) on both boards. Let's try to coordinate our moves to achieve this. Shall we proceed with moving 'D' on my board and 'S' on your board to make space for 'T' at (4,6)?"} \\
\\ 
\\ 
\texttt{What is your next command?} \\
            }
        }
    }
    & \\ \\

    \theutterance \stepcounter{utterance}  
    & & & \multicolumn{4}{p{0.6\linewidth}}{
        \cellcolor[rgb]{0.9,0.9,0.9}{
            \makecell[{{p{\linewidth}}}]{
                \texttt{\tiny{[GM$\langle$P2]}}
                \texttt{SAY: Yes, let's coordinate our moves. You can move 'D' from (4,6) on your board, and I'll move 'S' from (5,6) on my board to make space for 'T' at (4,6). After we make these moves, we can try to move 'T' to (4,6) on both boards. Please go ahead and move 'D' on your board, and I'll prepare to move 'S' on my board.} \\
            }
        }
    }
     \\ \\

    \theutterance \stepcounter{utterance}  
    & & & \multicolumn{2}{p{0.3\linewidth}}{
        \cellcolor[rgb]{0.9,0.9,0.9}{
            \makecell[{{p{\linewidth}}}]{
                \texttt{\tiny{[GM$|$GM]}}
                \texttt{SAY: Yes, let's coordinate our moves. You can move 'D' from (4,6) on your board, and I'll move 'S' from (5,6) on my board to make space for 'T' at (4,6). After we make these moves, we can try to move 'T' to (4,6) on both boards. Please go ahead and move 'D' on your board, and I'll prepare to move 'S' on my board.} \\
            }
        }
    }
    & & \\ \\

    \theutterance \stepcounter{utterance}  
    & & \multicolumn{4}{p{0.6\linewidth}}{
        \cellcolor[rgb]{0.9,0.9,0.9}{
            \makecell[{{p{\linewidth}}}]{
                \texttt{\tiny{[P1$\langle$GM]}}
                \texttt{Your message has been relayed to the other player.} \\
\\ 
\\ 
\texttt{You are currently playing round 22 of maximum 28.} \\
\texttt{You have collectively accumulated 8 of 16 penalties.} \\
\texttt{The other player sent this message:} \\
\\ 
\\ 
\texttt{"Yes, let's coordinate our moves. You can move 'D' from (4,6) on your board, and I'll move 'S' from (5,6) on my board to make space for 'T' at (4,6). After we make these moves, we can try to move 'T' to (4,6) on both boards. Please go ahead and move 'D' on your board, and I'll prepare to move 'S' on my board."} \\
\\ 
\\ 
\texttt{What is your next command?} \\
            }
        }
    }
    & \\ \\

    \theutterance \stepcounter{utterance}  
    & \multicolumn{4}{p{0.6\linewidth}}{
        \cellcolor[rgb]{0.9,0.9,0.9}{
            \makecell[{{p{\linewidth}}}]{
                \texttt{\tiny{[P1$\rangle$GM]}}
                \texttt{MOVE: D, (7, 6)} \\
            }
        }
    }
    & & \\ \\

    \theutterance \stepcounter{utterance}  
    & & & \multicolumn{2}{p{0.3\linewidth}}{
        \cellcolor[rgb]{0.9,0.9,0.9}{
            \makecell[{{p{\linewidth}}}]{
                \texttt{\tiny{[GM$|$GM]}}
                \texttt{MOVE: D, (7, 6)} \\
            }
        }
    }
    & & \\ \\

    \theutterance \stepcounter{utterance}  
    & & & \multicolumn{2}{p{0.3\linewidth}}{
        \cellcolor[rgb]{0.9,0.9,0.9}{
            \makecell[{{p{\linewidth}}}]{
                \texttt{\tiny{[GM$|$GM]}}
                \texttt{Penalty: (7,6) is not empty, but contains 'D'.} \\
\\ 
\texttt{You have collectively accumulated 9 of 16 penalties.} \\
\texttt{Please try again!} \\
            }
        }
    }
    & & \\ \\

    \theutterance \stepcounter{utterance}  
    & & \multicolumn{4}{p{0.6\linewidth}}{
        \cellcolor[rgb]{0.9,0.9,0.9}{
            \makecell[{{p{\linewidth}}}]{
                \texttt{\tiny{[P1$\langle$GM]}}
                \texttt{Penalty: (7,6) is not empty, but contains 'D'.} \\
\\ 
\texttt{You have collectively accumulated 9 of 16 penalties.} \\
\texttt{Please try again!} \\
            }
        }
    }
    & \\ \\

    \theutterance \stepcounter{utterance}  
    & \multicolumn{4}{p{0.6\linewidth}}{
        \cellcolor[rgb]{0.9,0.9,0.9}{
            \makecell[{{p{\linewidth}}}]{
                \texttt{\tiny{[P1$\rangle$GM]}}
                \texttt{SAY: It seems that I tried to move 'D' to a cell that already contains 'D'. Let me try again. How about I move 'D' to (7,7) instead? Is (7,7) empty on both boards? If it is, I can move 'D' to (7,7) and then we can proceed with moving 'T' to (4,6) on both boards. Please let me know if (7,7) is a viable option.} \\
            }
        }
    }
    & & \\ \\

    \theutterance \stepcounter{utterance}  
    & & & \multicolumn{2}{p{0.3\linewidth}}{
        \cellcolor[rgb]{0.9,0.9,0.9}{
            \makecell[{{p{\linewidth}}}]{
                \texttt{\tiny{[GM$|$GM]}}
                \texttt{SAY: It seems that I tried to move 'D' to a cell that already contains 'D'. Let me try again. How about I move 'D' to (7,7) instead? Is (7,7) empty on both boards? If it is, I can move 'D' to (7,7) and then we can proceed with moving 'T' to (4,6) on both boards. Please let me know if (7,7) is a viable option.} \\
            }
        }
    }
    & & \\ \\

    \theutterance \stepcounter{utterance}  
    & & \multicolumn{4}{p{0.6\linewidth}}{
        \cellcolor[rgb]{0.9,0.9,0.9}{
            \makecell[{{p{\linewidth}}}]{
                \texttt{\tiny{[GM$\rangle$P2]}}
                \texttt{Your message has been relayed to the other player.} \\
\\ 
\\ 
\texttt{You are currently playing round 23 of maximum 28.} \\
\texttt{You have collectively accumulated 9 of 16 penalties.} \\
\texttt{The other player sent this message:} \\
\\ 
\\ 
\texttt{"It seems that I tried to move 'D' to a cell that already contains 'D'. Let me try again. How about I move 'D' to (7,7) instead? Is (7,7) empty on both boards? If it is, I can move 'D' to (7,7) and then we can proceed with moving 'T' to (4,6) on both boards. Please let me know if (7,7) is a viable option."} \\
\\ 
\\ 
\texttt{What is your next command?} \\
            }
        }
    }
    & \\ \\

    \theutterance \stepcounter{utterance}  
    & & & \multicolumn{4}{p{0.6\linewidth}}{
        \cellcolor[rgb]{0.9,0.9,0.9}{
            \makecell[{{p{\linewidth}}}]{
                \texttt{\tiny{[GM$\langle$P2]}}
                \texttt{SAY: On my board, (7,7) is not a valid cell, as the grid only goes up to column 4 with occupiable cells, and column 7 is the right frame, and row 7 only has cells up to column 4 and 'A' is at (1,7). However, I can try to help you find an alternative location for 'D'. Could you please check your board again for an empty cell that could be a possible target for 'D'? Maybe we can find a cell that is empty on both boards or close to the current location of 'D'.} \\
            }
        }
    }
     \\ \\

    \theutterance \stepcounter{utterance}  
    & & & \multicolumn{2}{p{0.3\linewidth}}{
        \cellcolor[rgb]{0.9,0.9,0.9}{
            \makecell[{{p{\linewidth}}}]{
                \texttt{\tiny{[GM$|$GM]}}
                \texttt{SAY: On my board, (7,7) is not a valid cell, as the grid only goes up to column 4 with occupiable cells, and column 7 is the right frame, and row 7 only has cells up to column 4 and 'A' is at (1,7). However, I can try to help you find an alternative location for 'D'. Could you please check your board again for an empty cell that could be a possible target for 'D'? Maybe we can find a cell that is empty on both boards or close to the current location of 'D'.} \\
            }
        }
    }
    & & \\ \\

    \theutterance \stepcounter{utterance}  
    & & \multicolumn{4}{p{0.6\linewidth}}{
        \cellcolor[rgb]{0.9,0.9,0.9}{
            \makecell[{{p{\linewidth}}}]{
                \texttt{\tiny{[P1$\langle$GM]}}
                \texttt{Your message has been relayed to the other player.} \\
\\ 
\\ 
\texttt{You are currently playing round 23 of maximum 28.} \\
\texttt{You have collectively accumulated 9 of 16 penalties.} \\
\texttt{The other player sent this message:} \\
\\ 
\\ 
\texttt{"On my board, (7,7) is not a valid cell, as the grid only goes up to column 4 with occupiable cells, and column 7 is the right frame, and row 7 only has cells up to column 4 and 'A' is at (1,7). However, I can try to help you find an alternative location for 'D'. Could you please check your board again for an empty cell that could be a possible target for 'D'? Maybe we can find a cell that is empty on both boards or close to the current location of 'D'."} \\
\\ 
\\ 
\texttt{What is your next command?} \\
            }
        }
    }
    & \\ \\

    \theutterance \stepcounter{utterance}  
    & \multicolumn{4}{p{0.6\linewidth}}{
        \cellcolor[rgb]{0.9,0.9,0.9}{
            \makecell[{{p{\linewidth}}}]{
                \texttt{\tiny{[P1$\rangle$GM]}}
                \texttt{SAY: Let me re{-}examine my board to find a suitable location for 'D'. Since 'D' is currently at (4,6), I'll look for an adjacent cell that is empty. How about (3,7)? No, that's not a valid option since the grid only goes up to column 4 with occupiable cells. Let me think... Ah, yes! How about (5,6)? No, that's not empty either. Wait, I have an idea! What if I move 'O' from (4,5) to make space for 'D' at (4,5)? Would that be a viable option? Please let me know if (4,5) is empty on your board or if you have any other suggestions.} \\
            }
        }
    }
    & & \\ \\

    \theutterance \stepcounter{utterance}  
    & & & \multicolumn{2}{p{0.3\linewidth}}{
        \cellcolor[rgb]{0.9,0.9,0.9}{
            \makecell[{{p{\linewidth}}}]{
                \texttt{\tiny{[GM$|$GM]}}
                \texttt{SAY: Let me re{-}examine my board to find a suitable location for 'D'. Since 'D' is currently at (4,6), I'll look for an adjacent cell that is empty. How about (3,7)? No, that's not a valid option since the grid only goes up to column 4 with occupiable cells. Let me think... Ah, yes! How about (5,6)? No, that's not empty either. Wait, I have an idea! What if I move 'O' from (4,5) to make space for 'D' at (4,5)? Would that be a viable option? Please let me know if (4,5) is empty on your board or if you have any other suggestions.} \\
            }
        }
    }
    & & \\ \\

    \theutterance \stepcounter{utterance}  
    & & \multicolumn{4}{p{0.6\linewidth}}{
        \cellcolor[rgb]{0.9,0.9,0.9}{
            \makecell[{{p{\linewidth}}}]{
                \texttt{\tiny{[GM$\rangle$P2]}}
                \texttt{Your message has been relayed to the other player.} \\
\\ 
\\ 
\texttt{You are currently playing round 24 of maximum 28.} \\
\texttt{You have collectively accumulated 9 of 16 penalties.} \\
\texttt{The other player sent this message:} \\
\\ 
\\ 
\texttt{"Let me re{-}examine my board to find a suitable location for 'D'. Since 'D' is currently at (4,6), I'll look for an adjacent cell that is empty. How about (3,7)? No, that's not a valid option since the grid only goes up to column 4 with occupiable cells. Let me think... Ah, yes! How about (5,6)? No, that's not empty either. Wait, I have an idea! What if I move 'O' from (4,5) to make space for 'D' at (4,5)? Would that be a viable option? Please let me know if (4,5) is empty on your board or if you have any other suggestions."} \\
\\ 
\\ 
\texttt{What is your next command?} \\
            }
        }
    }
    & \\ \\

    \theutterance \stepcounter{utterance}  
    & & & \multicolumn{4}{p{0.6\linewidth}}{
        \cellcolor[rgb]{0.9,0.9,0.9}{
            \makecell[{{p{\linewidth}}}]{
                \texttt{\tiny{[GM$\langle$P2]}}
                \texttt{SAY: On my board, (4,5) is not empty, it contains 'D'. However, I think I can help you with your idea of moving 'O' from (4,5) on your board. If you move 'O' from (4,5), you can then move 'D' to (4,5) on your board. As for my board, I can try to move 'D' from (2,5) to a different location to align with your board. Let's focus on moving 'O' on your board first, and then we can discuss the next steps for 'D' on both boards.} \\
            }
        }
    }
     \\ \\

    \theutterance \stepcounter{utterance}  
    & & & \multicolumn{2}{p{0.3\linewidth}}{
        \cellcolor[rgb]{0.9,0.9,0.9}{
            \makecell[{{p{\linewidth}}}]{
                \texttt{\tiny{[GM$|$GM]}}
                \texttt{SAY: On my board, (4,5) is not empty, it contains 'D'. However, I think I can help you with your idea of moving 'O' from (4,5) on your board. If you move 'O' from (4,5), you can then move 'D' to (4,5) on your board. As for my board, I can try to move 'D' from (2,5) to a different location to align with your board. Let's focus on moving 'O' on your board first, and then we can discuss the next steps for 'D' on both boards.} \\
            }
        }
    }
    & & \\ \\

    \theutterance \stepcounter{utterance}  
    & & \multicolumn{4}{p{0.6\linewidth}}{
        \cellcolor[rgb]{0.9,0.9,0.9}{
            \makecell[{{p{\linewidth}}}]{
                \texttt{\tiny{[P1$\langle$GM]}}
                \texttt{Your message has been relayed to the other player.} \\
\\ 
\\ 
\texttt{You are currently playing round 24 of maximum 28.} \\
\texttt{You have collectively accumulated 9 of 16 penalties.} \\
\texttt{The other player sent this message:} \\
\\ 
\\ 
\texttt{"On my board, (4,5) is not empty, it contains 'D'. However, I think I can help you with your idea of moving 'O' from (4,5) on your board. If you move 'O' from (4,5), you can then move 'D' to (4,5) on your board. As for my board, I can try to move 'D' from (2,5) to a different location to align with your board. Let's focus on moving 'O' on your board first, and then we can discuss the next steps for 'D' on both boards."} \\
\\ 
\\ 
\texttt{What is your next command?} \\
            }
        }
    }
    & \\ \\

    \theutterance \stepcounter{utterance}  
    & \multicolumn{4}{p{0.6\linewidth}}{
        \cellcolor[rgb]{0.9,0.9,0.9}{
            \makecell[{{p{\linewidth}}}]{
                \texttt{\tiny{[P1$\rangle$GM]}}
                \texttt{MOVE: O, (5, 5)} \\
            }
        }
    }
    & & \\ \\

    \theutterance \stepcounter{utterance}  
    & & & \multicolumn{2}{p{0.3\linewidth}}{
        \cellcolor[rgb]{0.9,0.9,0.9}{
            \makecell[{{p{\linewidth}}}]{
                \texttt{\tiny{[GM$|$GM]}}
                \texttt{MOVE: O, (5, 5)} \\
            }
        }
    }
    & & \\ \\

    \theutterance \stepcounter{utterance}  
    & & & \multicolumn{2}{p{0.3\linewidth}}{
        \cellcolor[rgb]{0.9,0.9,0.9}{
            \makecell[{{p{\linewidth}}}]{
                \texttt{\tiny{[GM$|$GM]}}
                \texttt{Moved 'O' to (5,5) successfully. Your updated grid looks like this:} \\
\texttt{```} \\
\texttt{ 1234567} \\
\texttt{╔═╤═╤═╤═╗} \\
\texttt{║P│◌│◌│◌║ 1} \\
\texttt{║◌└─┼─┴─╢ 2} \\
\texttt{║◌SA│◌◌◌║ 3} \\
\texttt{╟──┬┴───╢ 4} \\
\texttt{║◌T│◌O◌◌║ 5} \\
\texttt{║◌◌│◌◌◌D║ 6} \\
\texttt{║◌◌│M◌◌◌║ 7} \\
\texttt{╚══╧════╝} \\
\texttt{```} \\
            }
        }
    }
    & & \\ \\

    \theutterance \stepcounter{utterance}  
    & & \multicolumn{4}{p{0.6\linewidth}}{
        \cellcolor[rgb]{0.9,0.9,0.9}{
            \makecell[{{p{\linewidth}}}]{
                \texttt{\tiny{[GM$\rangle$P2]}}
                \texttt{Your message has been relayed to the other player.} \\
\\ 
\\ 
\texttt{You are currently playing round 25 of maximum 28.} \\
\texttt{You have collectively accumulated 9 of 16 penalties.} \\
\texttt{The other player moved an object on their grid.} \\
\\ 
\\ 
\texttt{What is your next command?} \\
            }
        }
    }
    & \\ \\

    \theutterance \stepcounter{utterance}  
    & & & \multicolumn{4}{p{0.6\linewidth}}{
        \cellcolor[rgb]{0.9,0.9,0.9}{
            \makecell[{{p{\linewidth}}}]{
                \texttt{\tiny{[GM$\langle$P2]}}
                \texttt{SAY: Now that you've moved an object on your grid, I'll wait for your confirmation on the new position of 'O' and the status of 'D' on your board. Please let me know what move you made and what the current state of your board is, so we can discuss the next steps to align the objects on both boards.} \\
            }
        }
    }
     \\ \\

    \theutterance \stepcounter{utterance}  
    & & & \multicolumn{2}{p{0.3\linewidth}}{
        \cellcolor[rgb]{0.9,0.9,0.9}{
            \makecell[{{p{\linewidth}}}]{
                \texttt{\tiny{[GM$|$GM]}}
                \texttt{SAY: Now that you've moved an object on your grid, I'll wait for your confirmation on the new position of 'O' and the status of 'D' on your board. Please let me know what move you made and what the current state of your board is, so we can discuss the next steps to align the objects on both boards.} \\
            }
        }
    }
    & & \\ \\

    \theutterance \stepcounter{utterance}  
    & & \multicolumn{4}{p{0.6\linewidth}}{
        \cellcolor[rgb]{0.9,0.9,0.9}{
            \makecell[{{p{\linewidth}}}]{
                \texttt{\tiny{[P1$\langle$GM]}}
                \texttt{Moved 'O' to (5,5) successfully. Your updated grid looks like this:} \\
\texttt{```} \\
\texttt{ 1234567} \\
\texttt{╔═╤═╤═╤═╗} \\
\texttt{║P│◌│◌│◌║ 1} \\
\texttt{║◌└─┼─┴─╢ 2} \\
\texttt{║◌SA│◌◌◌║ 3} \\
\texttt{╟──┬┴───╢ 4} \\
\texttt{║◌T│◌O◌◌║ 5} \\
\texttt{║◌◌│◌◌◌D║ 6} \\
\texttt{║◌◌│M◌◌◌║ 7} \\
\texttt{╚══╧════╝} \\
\texttt{```} \\
\texttt{You are currently playing round 25 of maximum 28.} \\
\texttt{You have collectively accumulated 9 of 16 penalties.} \\
\texttt{The other player sent this message:} \\
\\ 
\\ 
\texttt{"Now that you've moved an object on your grid, I'll wait for your confirmation on the new position of 'O' and the status of 'D' on your board. Please let me know what move you made and what the current state of your board is, so we can discuss the next steps to align the objects on both boards."} \\
\\ 
\\ 
\texttt{What is your next command?} \\
            }
        }
    }
    & \\ \\

    \theutterance \stepcounter{utterance}  
    & \multicolumn{4}{p{0.6\linewidth}}{
        \cellcolor[rgb]{0.9,0.9,0.9}{
            \makecell[{{p{\linewidth}}}]{
                \texttt{\tiny{[P1$\rangle$GM]}}
                \texttt{SAY: I've moved 'O' to (5,5) on my board, which has created space for 'D' at (4,5). Now that 'O' is out of the way, I can move 'D' to (4,5) on my board. This will allow us to potentially move 'T' to (4,6) on both boards, as we previously discussed. Please confirm that you're ready to proceed with moving 'D' on your board and 'T' to (4,6) on both boards.} \\
            }
        }
    }
    & & \\ \\

    \theutterance \stepcounter{utterance}  
    & & & \multicolumn{2}{p{0.3\linewidth}}{
        \cellcolor[rgb]{0.9,0.9,0.9}{
            \makecell[{{p{\linewidth}}}]{
                \texttt{\tiny{[GM$|$GM]}}
                \texttt{SAY: I've moved 'O' to (5,5) on my board, which has created space for 'D' at (4,5). Now that 'O' is out of the way, I can move 'D' to (4,5) on my board. This will allow us to potentially move 'T' to (4,6) on both boards, as we previously discussed. Please confirm that you're ready to proceed with moving 'D' on your board and 'T' to (4,6) on both boards.} \\
            }
        }
    }
    & & \\ \\

    \theutterance \stepcounter{utterance}  
    & & \multicolumn{4}{p{0.6\linewidth}}{
        \cellcolor[rgb]{0.9,0.9,0.9}{
            \makecell[{{p{\linewidth}}}]{
                \texttt{\tiny{[GM$\rangle$P2]}}
                \texttt{Your message has been relayed to the other player.} \\
\\ 
\\ 
\texttt{You are currently playing round 26 of maximum 28.} \\
\texttt{You have collectively accumulated 9 of 16 penalties.} \\
\texttt{The other player sent this message:} \\
\\ 
\\ 
\texttt{"I've moved 'O' to (5,5) on my board, which has created space for 'D' at (4,5). Now that 'O' is out of the way, I can move 'D' to (4,5) on my board. This will allow us to potentially move 'T' to (4,6) on both boards, as we previously discussed. Please confirm that you're ready to proceed with moving 'D' on your board and 'T' to (4,6) on both boards."} \\
\\ 
\\ 
\texttt{What is your next command?} \\
            }
        }
    }
    & \\ \\

    \theutterance \stepcounter{utterance}  
    & & & \multicolumn{4}{p{0.6\linewidth}}{
        \cellcolor[rgb]{0.9,0.9,0.9}{
            \makecell[{{p{\linewidth}}}]{
                \texttt{\tiny{[GM$\langle$P2]}}
                \texttt{SAY: Yes, I'm ready to proceed. Since you've moved 'O' to (5,5) on your board and created space for 'D' at (4,5), I'll check my board to see if I can move 'D' to a corresponding location. On my board, 'D' is currently at (2,5). I can try to move 'D' to (4,5) on my board, but first, I need to make sure that (4,5) is empty. If it is, I can move 'D' to (4,5) and then we can discuss moving 'T' to (4,6) on both boards. Let me check my board again.} \\
            }
        }
    }
     \\ \\

    \theutterance \stepcounter{utterance}  
    & & & \multicolumn{2}{p{0.3\linewidth}}{
        \cellcolor[rgb]{0.9,0.9,0.9}{
            \makecell[{{p{\linewidth}}}]{
                \texttt{\tiny{[GM$|$GM]}}
                \texttt{SAY: Yes, I'm ready to proceed. Since you've moved 'O' to (5,5) on your board and created space for 'D' at (4,5), I'll check my board to see if I can move 'D' to a corresponding location. On my board, 'D' is currently at (2,5). I can try to move 'D' to (4,5) on my board, but first, I need to make sure that (4,5) is empty. If it is, I can move 'D' to (4,5) and then we can discuss moving 'T' to (4,6) on both boards. Let me check my board again.} \\
            }
        }
    }
    & & \\ \\

    \theutterance \stepcounter{utterance}  
    & & \multicolumn{4}{p{0.6\linewidth}}{
        \cellcolor[rgb]{0.9,0.9,0.9}{
            \makecell[{{p{\linewidth}}}]{
                \texttt{\tiny{[P1$\langle$GM]}}
                \texttt{Your message has been relayed to the other player.} \\
\\ 
\\ 
\texttt{You are currently playing round 26 of maximum 28.} \\
\texttt{You have collectively accumulated 9 of 16 penalties.} \\
\texttt{The other player sent this message:} \\
\\ 
\\ 
\texttt{"Yes, I'm ready to proceed. Since you've moved 'O' to (5,5) on your board and created space for 'D' at (4,5), I'll check my board to see if I can move 'D' to a corresponding location. On my board, 'D' is currently at (2,5). I can try to move 'D' to (4,5) on my board, but first, I need to make sure that (4,5) is empty. If it is, I can move 'D' to (4,5) and then we can discuss moving 'T' to (4,6) on both boards. Let me check my board again."} \\
\\ 
\\ 
\texttt{What is your next command?} \\
            }
        }
    }
    & \\ \\

    \theutterance \stepcounter{utterance}  
    & \multicolumn{4}{p{0.6\linewidth}}{
        \cellcolor[rgb]{0.9,0.9,0.9}{
            \makecell[{{p{\linewidth}}}]{
                \texttt{\tiny{[P1$\rangle$GM]}}
                \texttt{SAY: Please go ahead and check your board to confirm if (4,5) is empty. If it is, you can move 'D' to (4,5) on your board. Meanwhile, I'll move 'D' to (4,5) on my board. After we've both moved 'D', we can proceed with moving 'T' to (4,6) on both boards. Let's coordinate our next steps to ensure we're aligned. Please confirm if (4,5) is empty on your board and if you're ready to move 'D'.} \\
            }
        }
    }
    & & \\ \\

    \theutterance \stepcounter{utterance}  
    & & & \multicolumn{2}{p{0.3\linewidth}}{
        \cellcolor[rgb]{0.9,0.9,0.9}{
            \makecell[{{p{\linewidth}}}]{
                \texttt{\tiny{[GM$|$GM]}}
                \texttt{SAY: Please go ahead and check your board to confirm if (4,5) is empty. If it is, you can move 'D' to (4,5) on your board. Meanwhile, I'll move 'D' to (4,5) on my board. After we've both moved 'D', we can proceed with moving 'T' to (4,6) on both boards. Let's coordinate our next steps to ensure we're aligned. Please confirm if (4,5) is empty on your board and if you're ready to move 'D'.} \\
            }
        }
    }
    & & \\ \\

    \theutterance \stepcounter{utterance}  
    & & \multicolumn{4}{p{0.6\linewidth}}{
        \cellcolor[rgb]{0.9,0.9,0.9}{
            \makecell[{{p{\linewidth}}}]{
                \texttt{\tiny{[GM$\rangle$P2]}}
                \texttt{Your message has been relayed to the other player.} \\
\\ 
\\ 
\texttt{You are currently playing round 27 of maximum 28.} \\
\texttt{You have collectively accumulated 9 of 16 penalties.} \\
\texttt{The other player sent this message:} \\
\\ 
\\ 
\texttt{"Please go ahead and check your board to confirm if (4,5) is empty. If it is, you can move 'D' to (4,5) on your board. Meanwhile, I'll move 'D' to (4,5) on my board. After we've both moved 'D', we can proceed with moving 'T' to (4,6) on both boards. Let's coordinate our next steps to ensure we're aligned. Please confirm if (4,5) is empty on your board and if you're ready to move 'D'."} \\
\\ 
\\ 
\texttt{What is your next command?} \\
            }
        }
    }
    & \\ \\

    \theutterance \stepcounter{utterance}  
    & & & \multicolumn{4}{p{0.6\linewidth}}{
        \cellcolor[rgb]{0.9,0.9,0.9}{
            \makecell[{{p{\linewidth}}}]{
                \texttt{\tiny{[GM$\langle$P2]}}
                \texttt{MOVE: D, (4,5)} \\
            }
        }
    }
     \\ \\

    \theutterance \stepcounter{utterance}  
    & & & \multicolumn{2}{p{0.3\linewidth}}{
        \cellcolor[rgb]{0.9,0.9,0.9}{
            \makecell[{{p{\linewidth}}}]{
                \texttt{\tiny{[GM$|$GM]}}
                \texttt{MOVE: D, (4,5)} \\
            }
        }
    }
    & & \\ \\

    \theutterance \stepcounter{utterance}  
    & & & \multicolumn{2}{p{0.3\linewidth}}{
        \cellcolor[rgb]{0.9,0.9,0.9}{
            \makecell[{{p{\linewidth}}}]{
                \texttt{\tiny{[GM$|$GM]}}
                \texttt{Moved 'D' to (4,5) successfully. Your updated grid looks like this:} \\
\texttt{```} \\
\texttt{ 1234567} \\
\texttt{╔═╤═╤═╤═╗} \\
\texttt{║P│◌│◌│◌║ 1} \\
\texttt{║T└─┼─┴─╢ 2} \\
\texttt{║◌◌◌│M◌◌║ 3} \\
\texttt{╟──┬┴───╢ 4} \\
\texttt{║◌◌│D◌O◌║ 5} \\
\texttt{║◌◌│◌S◌◌║ 6} \\
\texttt{║A◌│◌◌◌◌║ 7} \\
\texttt{╚══╧════╝} \\
\texttt{```} \\
            }
        }
    }
    & & \\ \\

    \theutterance \stepcounter{utterance}  
    & & \multicolumn{4}{p{0.6\linewidth}}{
        \cellcolor[rgb]{0.9,0.9,0.9}{
            \makecell[{{p{\linewidth}}}]{
                \texttt{\tiny{[P1$\langle$GM]}}
                \texttt{Your message has been relayed to the other player.} \\
\\ 
\\ 
\texttt{You are currently playing round 27 of maximum 28.} \\
\texttt{You have collectively accumulated 9 of 16 penalties.} \\
\texttt{The other player moved an object on their grid.} \\
\\ 
\\ 
\texttt{What is your next command?} \\
            }
        }
    }
    & \\ \\

    \theutterance \stepcounter{utterance}  
    & \multicolumn{4}{p{0.6\linewidth}}{
        \cellcolor[rgb]{0.9,0.9,0.9}{
            \makecell[{{p{\linewidth}}}]{
                \texttt{\tiny{[P1$\rangle$GM]}}
                \texttt{MOVE: D, (4, 5)} \\
            }
        }
    }
    & & \\ \\

    \theutterance \stepcounter{utterance}  
    & & & \multicolumn{2}{p{0.3\linewidth}}{
        \cellcolor[rgb]{0.9,0.9,0.9}{
            \makecell[{{p{\linewidth}}}]{
                \texttt{\tiny{[GM$|$GM]}}
                \texttt{MOVE: D, (4, 5)} \\
            }
        }
    }
    & & \\ \\

    \theutterance \stepcounter{utterance}  
    & & & \multicolumn{2}{p{0.3\linewidth}}{
        \cellcolor[rgb]{0.9,0.9,0.9}{
            \makecell[{{p{\linewidth}}}]{
                \texttt{\tiny{[GM$|$GM]}}
                \texttt{Moved 'D' to (4,5) successfully. Your updated grid looks like this:} \\
\texttt{```} \\
\texttt{ 1234567} \\
\texttt{╔═╤═╤═╤═╗} \\
\texttt{║P│◌│◌│◌║ 1} \\
\texttt{║◌└─┼─┴─╢ 2} \\
\texttt{║◌SA│◌◌◌║ 3} \\
\texttt{╟──┬┴───╢ 4} \\
\texttt{║◌T│DO◌◌║ 5} \\
\texttt{║◌◌│◌◌◌◌║ 6} \\
\texttt{║◌◌│M◌◌◌║ 7} \\
\texttt{╚══╧════╝} \\
\texttt{```} \\
            }
        }
    }
    & & \\ \\

    \theutterance \stepcounter{utterance}  
    & & \multicolumn{4}{p{0.6\linewidth}}{
        \cellcolor[rgb]{0.9,0.9,0.9}{
            \makecell[{{p{\linewidth}}}]{
                \texttt{\tiny{[GM$\rangle$P2]}}
                \texttt{Moved 'D' to (4,5) successfully. Your updated grid looks like this:} \\
\texttt{```} \\
\texttt{ 1234567} \\
\texttt{╔═╤═╤═╤═╗} \\
\texttt{║P│◌│◌│◌║ 1} \\
\texttt{║T└─┼─┴─╢ 2} \\
\texttt{║◌◌◌│M◌◌║ 3} \\
\texttt{╟──┬┴───╢ 4} \\
\texttt{║◌◌│D◌O◌║ 5} \\
\texttt{║◌◌│◌S◌◌║ 6} \\
\texttt{║A◌│◌◌◌◌║ 7} \\
\texttt{╚══╧════╝} \\
\texttt{```} \\
\texttt{You are currently playing round 28 of maximum 28.} \\
\texttt{You have collectively accumulated 9 of 16 penalties.} \\
\texttt{The other player moved an object on their grid.} \\
\\ 
\\ 
\texttt{What is your next command?} \\
            }
        }
    }
    & \\ \\

    \theutterance \stepcounter{utterance}  
    & & & \multicolumn{4}{p{0.6\linewidth}}{
        \cellcolor[rgb]{0.9,0.9,0.9}{
            \makecell[{{p{\linewidth}}}]{
                \texttt{\tiny{[GM$\langle$P2]}}
                \texttt{SAY: finished?} \\
            }
        }
    }
     \\ \\

    \theutterance \stepcounter{utterance}  
    & & & \multicolumn{2}{p{0.3\linewidth}}{
        \cellcolor[rgb]{0.9,0.9,0.9}{
            \makecell[{{p{\linewidth}}}]{
                \texttt{\tiny{[GM$|$GM]}}
                \texttt{SAY: finished?} \\
            }
        }
    }
    & & \\ \\

    \theutterance \stepcounter{utterance}  
    & & & \multicolumn{2}{p{0.3\linewidth}}{
        \cellcolor[rgb]{0.9,0.9,0.9}{
            \makecell[{{p{\linewidth}}}]{
                \texttt{\tiny{[GM$|$GM]}}
                \texttt{Maximum number of rounds reached} \\
            }
        }
    }
    & & \\ \\

    \theutterance \stepcounter{utterance}  
    & & & \multicolumn{2}{p{0.3\linewidth}}{
        \cellcolor[rgb]{0.9,0.9,0.9}{
            \makecell[{{p{\linewidth}}}]{
                \texttt{\tiny{[GM$|$GM]}}
                \texttt{* success: True} \\
\texttt{* lose: False} \\
\texttt{* aborted: False} \\
\texttt{{-}{-}{-}{-}{-}{-}{-}} \\
\texttt{* Shifts: 4.00} \\
\texttt{* Max Shifts: 12.00} \\
\texttt{* Min Shifts: 6.00} \\
\texttt{* End Distance Sum: 17.00} \\
\texttt{* Init Distance Sum: 30.87} \\
\texttt{* Expected Distance Sum: 29.33} \\
\texttt{* Penalties: 9.00} \\
\texttt{* Max Penalties: 16.00} \\
\texttt{* Rounds: 28.00} \\
\texttt{* Max Rounds: 28.00} \\
\texttt{* Object Count: 7.00} \\
            }
        }
    }
    & & \\ \\

    \theutterance \stepcounter{utterance}  
    & & & \multicolumn{2}{p{0.3\linewidth}}{
        \cellcolor[rgb]{0.9,0.9,0.9}{
            \makecell[{{p{\linewidth}}}]{
                \texttt{\tiny{[GM$|$GM]}}
                \texttt{Initial Grids:} \\
\texttt{```} \\
\texttt{Player 1:     Player 2:} \\
\texttt{ 1234567       1234567} \\
\texttt{╔═╤═╤═╤═╗     ╔═╤═╤═╤═╗} \\
\texttt{║◌│◌│◌│◌║ 1   ║◌│◌│◌│P║ 1} \\
\texttt{║P└─┼─┴─╢ 2   ║T└─┼─┴─╢ 2} \\
\texttt{║◌SA│◌◌◌║ 3   ║◌◌◌│M◌◌║ 3} \\
\texttt{╟──┬┴───╢ 4   ╟──┬┴───╢ 4} \\
\texttt{║◌◌│◌◌O◌║ 5   ║◌D│◌◌O◌║ 5} \\
\texttt{║◌◌│◌◌MD║ 6   ║◌◌│◌S◌◌║ 6} \\
\texttt{║◌◌│◌◌◌T║ 7   ║A◌│◌◌◌◌║ 7} \\
\texttt{╚══╧════╝     ╚══╧════╝} \\
\texttt{```} \\
\\ 
\texttt{Final Grids:} \\
\texttt{```} \\
\texttt{Player 1:     Player 2:} \\
\texttt{ 1234567       1234567} \\
\texttt{╔═╤═╤═╤═╗     ╔═╤═╤═╤═╗} \\
\texttt{║P│◌│◌│◌║ 1   ║P│◌│◌│◌║ 1} \\
\texttt{║◌└─┼─┴─╢ 2   ║T└─┼─┴─╢ 2} \\
\texttt{║◌SA│◌◌◌║ 3   ║◌◌◌│M◌◌║ 3} \\
\texttt{╟──┬┴───╢ 4   ╟──┬┴───╢ 4} \\
\texttt{║◌T│DO◌◌║ 5   ║◌◌│D◌O◌║ 5} \\
\texttt{║◌◌│◌◌◌◌║ 6   ║◌◌│◌S◌◌║ 6} \\
\texttt{║◌◌│M◌◌◌║ 7   ║A◌│◌◌◌◌║ 7} \\
\texttt{╚══╧════╝     ╚══╧════╝} \\
\texttt{```} \\
            }
        }
    }
    & & \\ \\

\end{supertabular}
}

\end{document}
