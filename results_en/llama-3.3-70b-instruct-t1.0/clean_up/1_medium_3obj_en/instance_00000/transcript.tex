
\documentclass{article}
\usepackage{colortbl}
\usepackage{makecell}
\usepackage{multirow}
\usepackage{supertabular}

\begin{document}

\newcounter{utterance}

\twocolumn

{ \footnotesize  \setcounter{utterance}{1}
\setlength{\tabcolsep}{0pt}
\begin{supertabular}{c@{$\;$}|p{.15\linewidth}@{}p{.15\linewidth}p{.15\linewidth}p{.15\linewidth}p{.15\linewidth}p{.15\linewidth}}

    \# & $\;$A & \multicolumn{4}{c}{Game Master} & $\;\:$B\\
    \hline 

    \theutterance \stepcounter{utterance}  

    & & \multicolumn{4}{p{0.6\linewidth}}{\cellcolor[rgb]{0.9,0.9,0.9}{%
	\makecell[{{p{\linewidth}}}]{% 
	  \tt {\tiny [A$\langle$GM]}  
	 I am your game master, and you are playing a collaborative game with the following grid as game board:\\ \tt \\ \tt ```\\ \tt  1234567\\ \tt ╔═╤═╤═╤═╗\\ \tt ║◌│◌│◌│◌║ 1\\ \tt ║◌└─┼─┴─╢ 2\\ \tt ║◌◌◌│◌◌◌║ 3\\ \tt ╟──┬┴───╢ 4\\ \tt ║◌◌│P◌◌◌║ 5\\ \tt ║LC│◌◌◌◌║ 6\\ \tt ║◌◌│◌◌◌◌║ 7\\ \tt ╚══╧════╝\\ \tt \\ \tt ```\\ \tt \\ \tt * The upper edge displays x-coordinates increasing to the right, and the right edge y-coordinates increasing downward.\\ \tt * The following objects are randomly placed on your grid: 'C', 'L', 'P'.\\ \tt \\ \tt \\ \tt The other player sees a variation of the game board, where the objects are placed at different random locations. You cannot see the other player's board, and they cannot see yours.\\ \tt \\ \tt \\ \tt **Goal:**\\ \tt \\ \tt Both players need to move the objects on their respective background so that identical objects end up at the same coordinates. You have to communicate with the other player to agree upon a common goal configuration.\\ \tt \\ \tt \\ \tt **Rules:**\\ \tt \\ \tt * In each turn, you can send exactly one of the following two commands:\\ \tt  1. `SAY: <MESSAGE>`: to send a message (everything up until the next line break) to the other player. I will forward it to your partner.\\ \tt 2. `MOVE: <OBJECT>, (<X>, <Y>)`: to move an object to a new position, where `<X>` is the column and `<Y>` is the row. I will inform you if your move was successful or not.\\ \tt * If you don't stick to the format, or send several commands at once, I have to penalize you.\\ \tt * If both players accumulate more than 8 penalties, you both lose the game.\\ \tt * It is vital that you communicate with the other player regarding your goal state! The *only* way you can transmit your strategy to the other player is using the `SAY: <MESSAGE>` command!\\ \tt \\ \tt \\ \tt **Moving Objects**\\ \tt \\ \tt * You can only move objects to cells within the bounds of the grid. The target cell must be empty, i.e., it must only contain the symbol '◌'.\\ \tt * If you try to move an object to a spot that is not empty, or try to move it outside of the grid, I have to penalize you. You get another try.\\ \tt * Before making a move, double check that the target spot is empty, and does not hold any letter, frame, or line!\\ \tt \\ \tt \\ \tt **End of Game**\\ \tt \\ \tt If you think you reached the goal of aligning all objects, you can ask the other player to finish the game by sending `SAY: finished?`. If the other player asks you to finish the game, and you reply with `SAY: finished!`, the game will end.\\ \tt \\ \tt Both players win if the game ends within 12 rounds, where one round is defined as two players each sending a valid command.\\ \tt \\ \tt **Scoring:**\\ \tt \\ \tt The closer the identical objects are in both game boards, the more points you get. Penalties reduce your points. Can you beat the record?\\ \tt \\ \tt \\ \tt Please send a message to the other player to start the game! 
	  } 
	   } 
	   } 
	 & \\ 
 

    \theutterance \stepcounter{utterance}  

    & \multicolumn{4}{p{0.6\linewidth}}{\cellcolor[rgb]{0.8,1,0.9}{%
	\makecell[{{p{\linewidth}}}]{% 
	  \tt {\tiny [A$\rangle$GM]}  
	 SAY: Hello, let's start by describing our initial game board configurations. I have a 'P' at (4,5), an 'L' at (1,6), and a 'C' is not visible on my board. Please describe your initial configuration, including the locations of 'P', 'L', and 'C'. Let's work together to find a common goal configuration. 
	  } 
	   } 
	   } 
	 & & \\ 
 

    \theutterance \stepcounter{utterance}  

    & & & \multicolumn{2}{p{0.3\linewidth}}{\cellcolor[rgb]{0.95,0.95,0.95}{%
	\makecell[{{p{\linewidth}}}]{% 
	  \tt {\tiny [GM$|$GM]}  
	 SAY: Hello, let's start by describing our initial game board configurations. I have a 'P' at (4,5), an 'L' at (1,6), and a 'C' is not visible on my board. Please describe your initial configuration, including the locations of 'P', 'L', and 'C'. Let's work together to find a common goal configuration. 
	  } 
	   } 
	   } 
	 & & \\ 
 

    \theutterance \stepcounter{utterance}  

    & & \multicolumn{4}{p{0.6\linewidth}}{\cellcolor[rgb]{0.9,0.9,0.9}{%
	\makecell[{{p{\linewidth}}}]{% 
	  \tt {\tiny [GM$\rangle$B]}  
	 I am your game master, and you are playing a collaborative game with the following grid as game board:\\ \tt \\ \tt ```\\ \tt  1234567\\ \tt ╔═╤═╤═╤═╗\\ \tt ║◌│◌│L│◌║ 1\\ \tt ║◌└─┼─┴─╢ 2\\ \tt ║◌◌◌│◌◌C║ 3\\ \tt ╟──┬┴───╢ 4\\ \tt ║◌◌│◌◌◌◌║ 5\\ \tt ║◌◌│◌P◌◌║ 6\\ \tt ║◌◌│◌◌◌◌║ 7\\ \tt ╚══╧════╝\\ \tt \\ \tt ```\\ \tt \\ \tt * The upper edge displays x-coordinates increasing to the right, and the right edge y-coordinates increasing downward.\\ \tt * The following objects are randomly placed on your grid: 'C', 'L', 'P'.\\ \tt \\ \tt \\ \tt The other player sees a variation of the game board, where the objects are placed at different random locations. You cannot see the other player's board, and they cannot see yours.\\ \tt \\ \tt \\ \tt **Goal:**\\ \tt \\ \tt Both players need to move the objects on their respective background so that identical objects end up at the same coordinates. You have to communicate with the other player to agree upon a common goal configuration.\\ \tt \\ \tt \\ \tt **Rules:**\\ \tt \\ \tt * In each turn, you can send exactly one of the following two commands:\\ \tt  1. `SAY: <MESSAGE>`: to send a message (everything up until the next line break) to the other player. I will forward it to your partner.\\ \tt 2. `MOVE: <OBJECT>, (<X>, <Y>)`: to move an object to a new position, where `<X>` is the column and `<Y>` is the row. I will inform you if your move was successful or not.\\ \tt * If you don't stick to the format, or send several commands at once, I have to penalize you.\\ \tt * If both players accumulate more than 8 penalties, you both lose the game.\\ \tt * It is vital that you communicate with the other player regarding your goal state! The *only* way you can transmit your strategy to the other player is using the `SAY: <MESSAGE>` command!\\ \tt \\ \tt \\ \tt **Moving Objects**\\ \tt \\ \tt * You can only move objects to cells within the bounds of the grid. The target cell must be empty, i.e., it must only contain the symbol '◌'.\\ \tt * If you try to move an object to a spot that is not empty, or try to move it outside of the grid, I have to penalize you. You get another try.\\ \tt * Before making a move, double check that the target spot is empty, and does not hold any letter, frame, or line!\\ \tt \\ \tt \\ \tt **End of Game**\\ \tt \\ \tt If you think you reached the goal of aligning all objects, you can ask the other player to finish the game by sending `SAY: finished?`. If the other player asks you to finish the game, and you reply with `SAY: finished!`, the game will end.\\ \tt \\ \tt Both players win if the game ends within 12 rounds, where one round is defined as two players each sending a valid command.\\ \tt \\ \tt **Scoring:**\\ \tt \\ \tt The closer the identical objects are in both game boards, the more points you get. Penalties reduce your points. Can you beat the record?\\ \tt \\ \tt The other player started the game by sending this message:\\ \tt \\ \tt "Hello, let's start by describing our initial game board configurations. I have a 'P' at (4,5), an 'L' at (1,6), and a 'C' is not visible on my board. Please describe your initial configuration, including the locations of 'P', 'L', and 'C'. Let's work together to find a common goal configuration."\\ \tt \\ \tt What is your first command? 
	  } 
	   } 
	   } 
	 & \\ 
 

    \theutterance \stepcounter{utterance}  

    & & & \multicolumn{4}{p{0.6\linewidth}}{\cellcolor[rgb]{1,0.85,0.72}{%
	\makecell[{{p{\linewidth}}}]{% 
	  \tt {\tiny [GM$\langle$B]}  
	 SAY: Hello! I've taken note of your board configuration. On my board, I have a 'P' at (4,6), an 'L' at (3,1), and a 'C' at (5,3). Let's discuss a common goal configuration. How about we aim to place all objects in the first row, with 'P' at (1,1), 'L' at (3,1), and 'C' at (5,1)? What are your thoughts on this proposal? 
	  } 
	   } 
	   } 
	  \\ 
 

    \theutterance \stepcounter{utterance}  

    & & & \multicolumn{2}{p{0.3\linewidth}}{\cellcolor[rgb]{0.95,0.95,0.95}{%
	\makecell[{{p{\linewidth}}}]{% 
	  \tt {\tiny [GM$|$GM]}  
	 SAY: Hello! I've taken note of your board configuration. On my board, I have a 'P' at (4,6), an 'L' at (3,1), and a 'C' at (5,3). Let's discuss a common goal configuration. How about we aim to place all objects in the first row, with 'P' at (1,1), 'L' at (3,1), and 'C' at (5,1)? What are your thoughts on this proposal? 
	  } 
	   } 
	   } 
	 & & \\ 
 

    \theutterance \stepcounter{utterance}  

    & & \multicolumn{4}{p{0.6\linewidth}}{\cellcolor[rgb]{0.9,0.9,0.9}{%
	\makecell[{{p{\linewidth}}}]{% 
	  \tt {\tiny [A$\langle$GM]}  
	 Your message has been relayed to the other player.\\ \tt \\ \tt \\ \tt You are currently playing round 1 of maximum 12.\\ \tt You have collectively accumulated 0 of 8 penalties.\\ \tt The other player sent this message:\\ \tt \\ \tt \\ \tt "Hello! I've taken note of your board configuration. On my board, I have a 'P' at (4,6), an 'L' at (3,1), and a 'C' at (5,3). Let's discuss a common goal configuration. How about we aim to place all objects in the first row, with 'P' at (1,1), 'L' at (3,1), and 'C' at (5,1)? What are your thoughts on this proposal?"\\ \tt \\ \tt \\ \tt What is your next command? 
	  } 
	   } 
	   } 
	 & \\ 
 

    \theutterance \stepcounter{utterance}  

    & \multicolumn{4}{p{0.6\linewidth}}{\cellcolor[rgb]{0.8,1,0.9}{%
	\makecell[{{p{\linewidth}}}]{% 
	  \tt {\tiny [A$\rangle$GM]}  
	 SAY: I like your proposal for the goal configuration, with 'P' at (1,1), 'L' at (3,1), and 'C' at (5,1). Before we start moving objects, I'd like to confirm that we both understand the current positions of all objects on both boards. You've shared your 'P', 'L', and 'C' locations, and I've shared mine. To ensure we're on the same page, can you confirm that there are no other objects ('P', 'L', 'C') on either of our boards beyond what we've already discussed? Additionally, shall we start moving objects to the target positions one by one, or do you suggest a different approach? 
	  } 
	   } 
	   } 
	 & & \\ 
 

    \theutterance \stepcounter{utterance}  

    & & & \multicolumn{2}{p{0.3\linewidth}}{\cellcolor[rgb]{0.95,0.95,0.95}{%
	\makecell[{{p{\linewidth}}}]{% 
	  \tt {\tiny [GM$|$GM]}  
	 SAY: I like your proposal for the goal configuration, with 'P' at (1,1), 'L' at (3,1), and 'C' at (5,1). Before we start moving objects, I'd like to confirm that we both understand the current positions of all objects on both boards. You've shared your 'P', 'L', and 'C' locations, and I've shared mine. To ensure we're on the same page, can you confirm that there are no other objects ('P', 'L', 'C') on either of our boards beyond what we've already discussed? Additionally, shall we start moving objects to the target positions one by one, or do you suggest a different approach? 
	  } 
	   } 
	   } 
	 & & \\ 
 

    \theutterance \stepcounter{utterance}  

    & & \multicolumn{4}{p{0.6\linewidth}}{\cellcolor[rgb]{0.9,0.9,0.9}{%
	\makecell[{{p{\linewidth}}}]{% 
	  \tt {\tiny [GM$\rangle$B]}  
	 Your message has been relayed to the other player.\\ \tt \\ \tt \\ \tt You are currently playing round 2 of maximum 12.\\ \tt You have collectively accumulated 0 of 8 penalties.\\ \tt The other player sent this message:\\ \tt \\ \tt \\ \tt "I like your proposal for the goal configuration, with 'P' at (1,1), 'L' at (3,1), and 'C' at (5,1). Before we start moving objects, I'd like to confirm that we both understand the current positions of all objects on both boards. You've shared your 'P', 'L', and 'C' locations, and I've shared mine. To ensure we're on the same page, can you confirm that there are no other objects ('P', 'L', 'C') on either of our boards beyond what we've already discussed? Additionally, shall we start moving objects to the target positions one by one, or do you suggest a different approach?"\\ \tt \\ \tt \\ \tt What is your next command? 
	  } 
	   } 
	   } 
	 & \\ 
 

    \theutterance \stepcounter{utterance}  

    & & & \multicolumn{4}{p{0.6\linewidth}}{\cellcolor[rgb]{1,0.85,0.72}{%
	\makecell[{{p{\linewidth}}}]{% 
	  \tt {\tiny [GM$\langle$B]}  
	 SAY: I confirm that there are no other objects ('P', 'L', 'C') on my board beyond what we've already discussed, which are 'P' at (4,6), 'L' at (3,1), and 'C' at (5,3). I assume the same for your board, with 'P' at (4,5), 'L' at (1,6), and no visible 'C'. Regarding the approach, I suggest we start moving objects to the target positions one by one. Let's prioritize moving the objects that are farthest from their target positions first. In my case, 'P' is at (4,6) and needs to move to (1,1), which is a significant distance. How about we start with the 'P' on both our boards, and then proceed with the other objects? Should we synchronize our moves, or move them independently? 
	  } 
	   } 
	   } 
	  \\ 
 

    \theutterance \stepcounter{utterance}  

    & & & \multicolumn{2}{p{0.3\linewidth}}{\cellcolor[rgb]{0.95,0.95,0.95}{%
	\makecell[{{p{\linewidth}}}]{% 
	  \tt {\tiny [GM$|$GM]}  
	 SAY: I confirm that there are no other objects ('P', 'L', 'C') on my board beyond what we've already discussed, which are 'P' at (4,6), 'L' at (3,1), and 'C' at (5,3). I assume the same for your board, with 'P' at (4,5), 'L' at (1,6), and no visible 'C'. Regarding the approach, I suggest we start moving objects to the target positions one by one. Let's prioritize moving the objects that are farthest from their target positions first. In my case, 'P' is at (4,6) and needs to move to (1,1), which is a significant distance. How about we start with the 'P' on both our boards, and then proceed with the other objects? Should we synchronize our moves, or move them independently? 
	  } 
	   } 
	   } 
	 & & \\ 
 

    \theutterance \stepcounter{utterance}  

    & & \multicolumn{4}{p{0.6\linewidth}}{\cellcolor[rgb]{0.9,0.9,0.9}{%
	\makecell[{{p{\linewidth}}}]{% 
	  \tt {\tiny [A$\langle$GM]}  
	 Your message has been relayed to the other player.\\ \tt \\ \tt \\ \tt You are currently playing round 2 of maximum 12.\\ \tt You have collectively accumulated 0 of 8 penalties.\\ \tt The other player sent this message:\\ \tt \\ \tt \\ \tt "I confirm that there are no other objects ('P', 'L', 'C') on my board beyond what we've already discussed, which are 'P' at (4,6), 'L' at (3,1), and 'C' at (5,3). I assume the same for your board, with 'P' at (4,5), 'L' at (1,6), and no visible 'C'. Regarding the approach, I suggest we start moving objects to the target positions one by one. Let's prioritize moving the objects that are farthest from their target positions first. In my case, 'P' is at (4,6) and needs to move to (1,1), which is a significant distance. How about we start with the 'P' on both our boards, and then proceed with the other objects? Should we synchronize our moves, or move them independently?"\\ \tt \\ \tt \\ \tt What is your next command? 
	  } 
	   } 
	   } 
	 & \\ 
 

    \theutterance \stepcounter{utterance}  

    & \multicolumn{4}{p{0.6\linewidth}}{\cellcolor[rgb]{0.8,1,0.9}{%
	\makecell[{{p{\linewidth}}}]{% 
	  \tt {\tiny [A$\rangle$GM]}  
	 SAY: I agree that prioritizing the objects farthest from their target positions is a good strategy. Since 'P' on both our boards is indeed far from its target position at (1,1), let's start with moving 'P'. I suggest we move our respective 'P' objects independently, as synchronizing the exact moves might be challenging given the different initial positions. On my board, 'P' is at (4,5), so I plan to move it towards the target position (1,1) in a way that seems optimal to me. You can do the same based on your board's 'P' position at (4,6). After we both move our 'P' objects, we can reassess and communicate our next steps to ensure we're aligning our configurations. Before I make my move, I just want to confirm: are we aiming to keep the objects within the grid boundaries and only move them to empty cells, as per the rules? 
	  } 
	   } 
	   } 
	 & & \\ 
 

    \theutterance \stepcounter{utterance}  

    & & & \multicolumn{2}{p{0.3\linewidth}}{\cellcolor[rgb]{0.95,0.95,0.95}{%
	\makecell[{{p{\linewidth}}}]{% 
	  \tt {\tiny [GM$|$GM]}  
	 SAY: I agree that prioritizing the objects farthest from their target positions is a good strategy. Since 'P' on both our boards is indeed far from its target position at (1,1), let's start with moving 'P'. I suggest we move our respective 'P' objects independently, as synchronizing the exact moves might be challenging given the different initial positions. On my board, 'P' is at (4,5), so I plan to move it towards the target position (1,1) in a way that seems optimal to me. You can do the same based on your board's 'P' position at (4,6). After we both move our 'P' objects, we can reassess and communicate our next steps to ensure we're aligning our configurations. Before I make my move, I just want to confirm: are we aiming to keep the objects within the grid boundaries and only move them to empty cells, as per the rules? 
	  } 
	   } 
	   } 
	 & & \\ 
 

    \theutterance \stepcounter{utterance}  

    & & \multicolumn{4}{p{0.6\linewidth}}{\cellcolor[rgb]{0.9,0.9,0.9}{%
	\makecell[{{p{\linewidth}}}]{% 
	  \tt {\tiny [GM$\rangle$B]}  
	 Your message has been relayed to the other player.\\ \tt \\ \tt \\ \tt You are currently playing round 3 of maximum 12.\\ \tt You have collectively accumulated 0 of 8 penalties.\\ \tt The other player sent this message:\\ \tt \\ \tt \\ \tt "I agree that prioritizing the objects farthest from their target positions is a good strategy. Since 'P' on both our boards is indeed far from its target position at (1,1), let's start with moving 'P'. I suggest we move our respective 'P' objects independently, as synchronizing the exact moves might be challenging given the different initial positions. On my board, 'P' is at (4,5), so I plan to move it towards the target position (1,1) in a way that seems optimal to me. You can do the same based on your board's 'P' position at (4,6). After we both move our 'P' objects, we can reassess and communicate our next steps to ensure we're aligning our configurations. Before I make my move, I just want to confirm: are we aiming to keep the objects within the grid boundaries and only move them to empty cells, as per the rules?"\\ \tt \\ \tt \\ \tt What is your next command? 
	  } 
	   } 
	   } 
	 & \\ 
 

    \theutterance \stepcounter{utterance}  

    & & & \multicolumn{4}{p{0.6\linewidth}}{\cellcolor[rgb]{1,0.85,0.72}{%
	\makecell[{{p{\linewidth}}}]{% 
	  \tt {\tiny [GM$\langle$B]}  
	 SAY: Yes, let's confirm that we will only move the objects to empty cells within the grid boundaries, following the rules. This ensures that our moves are valid and helps us avoid penalties. With that confirmed, I'm ready to proceed with moving my 'P' from (4,6) towards the target position (1,1). I will make my move independently, as discussed. Before I do, I just want to acknowledge that after moving our 'P' objects, we should share our new 'P' positions to keep each other informed and adjust our strategies as needed. Please go ahead and make your move, and I'll do the same. Afterward, we can share the updated positions of our 'P' objects and plan the next steps for 'L' and 'C'. 
	  } 
	   } 
	   } 
	  \\ 
 

    \theutterance \stepcounter{utterance}  

    & & & \multicolumn{2}{p{0.3\linewidth}}{\cellcolor[rgb]{0.95,0.95,0.95}{%
	\makecell[{{p{\linewidth}}}]{% 
	  \tt {\tiny [GM$|$GM]}  
	 SAY: Yes, let's confirm that we will only move the objects to empty cells within the grid boundaries, following the rules. This ensures that our moves are valid and helps us avoid penalties. With that confirmed, I'm ready to proceed with moving my 'P' from (4,6) towards the target position (1,1). I will make my move independently, as discussed. Before I do, I just want to acknowledge that after moving our 'P' objects, we should share our new 'P' positions to keep each other informed and adjust our strategies as needed. Please go ahead and make your move, and I'll do the same. Afterward, we can share the updated positions of our 'P' objects and plan the next steps for 'L' and 'C'. 
	  } 
	   } 
	   } 
	 & & \\ 
 

    \theutterance \stepcounter{utterance}  

    & & \multicolumn{4}{p{0.6\linewidth}}{\cellcolor[rgb]{0.9,0.9,0.9}{%
	\makecell[{{p{\linewidth}}}]{% 
	  \tt {\tiny [A$\langle$GM]}  
	 Your message has been relayed to the other player.\\ \tt \\ \tt \\ \tt You are currently playing round 3 of maximum 12.\\ \tt You have collectively accumulated 0 of 8 penalties.\\ \tt The other player sent this message:\\ \tt \\ \tt \\ \tt "Yes, let's confirm that we will only move the objects to empty cells within the grid boundaries, following the rules. This ensures that our moves are valid and helps us avoid penalties. With that confirmed, I'm ready to proceed with moving my 'P' from (4,6) towards the target position (1,1). I will make my move independently, as discussed. Before I do, I just want to acknowledge that after moving our 'P' objects, we should share our new 'P' positions to keep each other informed and adjust our strategies as needed. Please go ahead and make your move, and I'll do the same. Afterward, we can share the updated positions of our 'P' objects and plan the next steps for 'L' and 'C'."\\ \tt \\ \tt \\ \tt What is your next command? 
	  } 
	   } 
	   } 
	 & \\ 
 

    \theutterance \stepcounter{utterance}  

    & \multicolumn{4}{p{0.6\linewidth}}{\cellcolor[rgb]{0.8,1,0.9}{%
	\makecell[{{p{\linewidth}}}]{% 
	  \tt {\tiny [A$\rangle$GM]}  
	 MOVE: P, (3,5) 
	  } 
	   } 
	   } 
	 & & \\ 
 

    \theutterance \stepcounter{utterance}  

    & & & \multicolumn{2}{p{0.3\linewidth}}{\cellcolor[rgb]{0.95,0.95,0.95}{%
	\makecell[{{p{\linewidth}}}]{% 
	  \tt {\tiny [GM$|$GM]}  
	 MOVE: P, (3,5) 
	  } 
	   } 
	   } 
	 & & \\ 
 

    \theutterance \stepcounter{utterance}  

    & & & \multicolumn{2}{p{0.3\linewidth}}{\cellcolor[rgb]{0.95,0.95,0.95}{%
	\makecell[{{p{\linewidth}}}]{% 
	  \tt {\tiny [GM$|$GM]}  
	 Penalty: (3,5) is not empty, but contains '│'.\\ \tt \\ \tt You have collectively accumulated 1 of 8 penalties.\\ \tt Please try again! 
	  } 
	   } 
	   } 
	 & & \\ 
 

    \theutterance \stepcounter{utterance}  

    & & \multicolumn{4}{p{0.6\linewidth}}{\cellcolor[rgb]{0.9,0.9,0.9}{%
	\makecell[{{p{\linewidth}}}]{% 
	  \tt {\tiny [A$\langle$GM]}  
	 Penalty: (3,5) is not empty, but contains '│'.\\ \tt \\ \tt You have collectively accumulated 1 of 8 penalties.\\ \tt Please try again! 
	  } 
	   } 
	   } 
	 & \\ 
 

    \theutterance \stepcounter{utterance}  

    & \multicolumn{4}{p{0.6\linewidth}}{\cellcolor[rgb]{0.8,1,0.9}{%
	\makecell[{{p{\linewidth}}}]{% 
	  \tt {\tiny [A$\rangle$GM]}  
	 MOVE: P, (3,4) 
	  } 
	   } 
	   } 
	 & & \\ 
 

    \theutterance \stepcounter{utterance}  

    & & & \multicolumn{2}{p{0.3\linewidth}}{\cellcolor[rgb]{0.95,0.95,0.95}{%
	\makecell[{{p{\linewidth}}}]{% 
	  \tt {\tiny [GM$|$GM]}  
	 MOVE: P, (3,4) 
	  } 
	   } 
	   } 
	 & & \\ 
 

    \theutterance \stepcounter{utterance}  

    & & & \multicolumn{2}{p{0.3\linewidth}}{\cellcolor[rgb]{0.95,0.95,0.95}{%
	\makecell[{{p{\linewidth}}}]{% 
	  \tt {\tiny [GM$|$GM]}  
	 Penalty: (3,4) is not empty, but contains '┬'.\\ \tt \\ \tt You have collectively accumulated 2 of 8 penalties.\\ \tt Please try again! 
	  } 
	   } 
	   } 
	 & & \\ 
 

    \theutterance \stepcounter{utterance}  

    & & \multicolumn{4}{p{0.6\linewidth}}{\cellcolor[rgb]{0.9,0.9,0.9}{%
	\makecell[{{p{\linewidth}}}]{% 
	  \tt {\tiny [A$\langle$GM]}  
	 Penalty: (3,4) is not empty, but contains '┬'.\\ \tt \\ \tt You have collectively accumulated 2 of 8 penalties.\\ \tt Please try again! 
	  } 
	   } 
	   } 
	 & \\ 
 

    \theutterance \stepcounter{utterance}  

    & \multicolumn{4}{p{0.6\linewidth}}{\cellcolor[rgb]{0.8,1,0.9}{%
	\makecell[{{p{\linewidth}}}]{% 
	  \tt {\tiny [A$\rangle$GM]}  
	 SAY: It seems I've encountered an obstacle. Before making another move, I'd like to clarify the grid layout to avoid further penalties. Could you please remind me of the positions of the grid elements such as '┼', '┬', '┴', and '╢' that might be blocking my path? This will help me plan a safer route for my 'P' object. 
	  } 
	   } 
	   } 
	 & & \\ 
 

    \theutterance \stepcounter{utterance}  

    & & & \multicolumn{2}{p{0.3\linewidth}}{\cellcolor[rgb]{0.95,0.95,0.95}{%
	\makecell[{{p{\linewidth}}}]{% 
	  \tt {\tiny [GM$|$GM]}  
	 SAY: It seems I've encountered an obstacle. Before making another move, I'd like to clarify the grid layout to avoid further penalties. Could you please remind me of the positions of the grid elements such as '┼', '┬', '┴', and '╢' that might be blocking my path? This will help me plan a safer route for my 'P' object. 
	  } 
	   } 
	   } 
	 & & \\ 
 

    \theutterance \stepcounter{utterance}  

    & & \multicolumn{4}{p{0.6\linewidth}}{\cellcolor[rgb]{0.9,0.9,0.9}{%
	\makecell[{{p{\linewidth}}}]{% 
	  \tt {\tiny [GM$\rangle$B]}  
	 Your message has been relayed to the other player.\\ \tt \\ \tt \\ \tt You are currently playing round 4 of maximum 12.\\ \tt You have collectively accumulated 2 of 8 penalties.\\ \tt The other player sent this message:\\ \tt \\ \tt \\ \tt "It seems I've encountered an obstacle. Before making another move, I'd like to clarify the grid layout to avoid further penalties. Could you please remind me of the positions of the grid elements such as '┼', '┬', '┴', and '╢' that might be blocking my path? This will help me plan a safer route for my 'P' object."\\ \tt \\ \tt \\ \tt What is your next command? 
	  } 
	   } 
	   } 
	 & \\ 
 

    \theutterance \stepcounter{utterance}  

    & & & \multicolumn{4}{p{0.6\linewidth}}{\cellcolor[rgb]{1,0.85,0.72}{%
	\makecell[{{p{\linewidth}}}]{% 
	  \tt {\tiny [GM$\langle$B]}  
	 SAY: The grid elements that might affect movement are as follows: there's a '┼' at (2,2), a '┬' at (2,4), a '┴' at (3,2), and '╢' symbols at (4,2) and (5,3). Additionally, there's a '└' at (2,1) and a '╤' at (2,0) which might block horizontal movement. Considering these, you can plan your 'P' object's path more accurately. Keep in mind that we should also be mindful of the 'L' and 'C' objects' positions as we move forward. Shall we proceed with moving our 'P' objects, taking these grid elements into account? 
	  } 
	   } 
	   } 
	  \\ 
 

    \theutterance \stepcounter{utterance}  

    & & & \multicolumn{2}{p{0.3\linewidth}}{\cellcolor[rgb]{0.95,0.95,0.95}{%
	\makecell[{{p{\linewidth}}}]{% 
	  \tt {\tiny [GM$|$GM]}  
	 SAY: The grid elements that might affect movement are as follows: there's a '┼' at (2,2), a '┬' at (2,4), a '┴' at (3,2), and '╢' symbols at (4,2) and (5,3). Additionally, there's a '└' at (2,1) and a '╤' at (2,0) which might block horizontal movement. Considering these, you can plan your 'P' object's path more accurately. Keep in mind that we should also be mindful of the 'L' and 'C' objects' positions as we move forward. Shall we proceed with moving our 'P' objects, taking these grid elements into account? 
	  } 
	   } 
	   } 
	 & & \\ 
 

    \theutterance \stepcounter{utterance}  

    & & \multicolumn{4}{p{0.6\linewidth}}{\cellcolor[rgb]{0.9,0.9,0.9}{%
	\makecell[{{p{\linewidth}}}]{% 
	  \tt {\tiny [A$\langle$GM]}  
	 Your message has been relayed to the other player.\\ \tt \\ \tt \\ \tt You are currently playing round 4 of maximum 12.\\ \tt You have collectively accumulated 2 of 8 penalties.\\ \tt The other player sent this message:\\ \tt \\ \tt \\ \tt "The grid elements that might affect movement are as follows: there's a '┼' at (2,2), a '┬' at (2,4), a '┴' at (3,2), and '╢' symbols at (4,2) and (5,3). Additionally, there's a '└' at (2,1) and a '╤' at (2,0) which might block horizontal movement. Considering these, you can plan your 'P' object's path more accurately. Keep in mind that we should also be mindful of the 'L' and 'C' objects' positions as we move forward. Shall we proceed with moving our 'P' objects, taking these grid elements into account?"\\ \tt \\ \tt \\ \tt What is your next command? 
	  } 
	   } 
	   } 
	 & \\ 
 

    \theutterance \stepcounter{utterance}  

    & \multicolumn{4}{p{0.6\linewidth}}{\cellcolor[rgb]{0.8,1,0.9}{%
	\makecell[{{p{\linewidth}}}]{% 
	  \tt {\tiny [A$\rangle$GM]}  
	 MOVE: P, (4,4) 
	  } 
	   } 
	   } 
	 & & \\ 
 

    \theutterance \stepcounter{utterance}  

    & & & \multicolumn{2}{p{0.3\linewidth}}{\cellcolor[rgb]{0.95,0.95,0.95}{%
	\makecell[{{p{\linewidth}}}]{% 
	  \tt {\tiny [GM$|$GM]}  
	 MOVE: P, (4,4) 
	  } 
	   } 
	   } 
	 & & \\ 
 

    \theutterance \stepcounter{utterance}  

    & & & \multicolumn{2}{p{0.3\linewidth}}{\cellcolor[rgb]{0.95,0.95,0.95}{%
	\makecell[{{p{\linewidth}}}]{% 
	  \tt {\tiny [GM$|$GM]}  
	 Penalty: (4,4) is not empty, but contains '┴'.\\ \tt \\ \tt You have collectively accumulated 3 of 8 penalties.\\ \tt Please try again! 
	  } 
	   } 
	   } 
	 & & \\ 
 

    \theutterance \stepcounter{utterance}  

    & & \multicolumn{4}{p{0.6\linewidth}}{\cellcolor[rgb]{0.9,0.9,0.9}{%
	\makecell[{{p{\linewidth}}}]{% 
	  \tt {\tiny [A$\langle$GM]}  
	 Penalty: (4,4) is not empty, but contains '┴'.\\ \tt \\ \tt You have collectively accumulated 3 of 8 penalties.\\ \tt Please try again! 
	  } 
	   } 
	   } 
	 & \\ 
 

    \theutterance \stepcounter{utterance}  

    & \multicolumn{4}{p{0.6\linewidth}}{\cellcolor[rgb]{0.8,1,0.9}{%
	\makecell[{{p{\linewidth}}}]{% 
	  \tt {\tiny [A$\rangle$GM]}  
	 SAY: It seems I've hit another obstacle. Let me reassess the board. Since my 'P' is at (4,5) and the goal is to move it to (1,1), I'll try to find a clearer path. Considering the grid elements you mentioned, I'll plan my next move carefully. Before I proceed, can you tell me if you've successfully moved your 'P' from (4,6) towards the target position (1,1)? What's the current position of your 'P' object? This might give me insight into how to navigate my own 'P' object more effectively. 
	  } 
	   } 
	   } 
	 & & \\ 
 

    \theutterance \stepcounter{utterance}  

    & & & \multicolumn{2}{p{0.3\linewidth}}{\cellcolor[rgb]{0.95,0.95,0.95}{%
	\makecell[{{p{\linewidth}}}]{% 
	  \tt {\tiny [GM$|$GM]}  
	 SAY: It seems I've hit another obstacle. Let me reassess the board. Since my 'P' is at (4,5) and the goal is to move it to (1,1), I'll try to find a clearer path. Considering the grid elements you mentioned, I'll plan my next move carefully. Before I proceed, can you tell me if you've successfully moved your 'P' from (4,6) towards the target position (1,1)? What's the current position of your 'P' object? This might give me insight into how to navigate my own 'P' object more effectively. 
	  } 
	   } 
	   } 
	 & & \\ 
 

    \theutterance \stepcounter{utterance}  

    & & \multicolumn{4}{p{0.6\linewidth}}{\cellcolor[rgb]{0.9,0.9,0.9}{%
	\makecell[{{p{\linewidth}}}]{% 
	  \tt {\tiny [GM$\rangle$B]}  
	 Your message has been relayed to the other player.\\ \tt \\ \tt \\ \tt You are currently playing round 5 of maximum 12.\\ \tt You have collectively accumulated 3 of 8 penalties.\\ \tt The other player sent this message:\\ \tt \\ \tt \\ \tt "It seems I've hit another obstacle. Let me reassess the board. Since my 'P' is at (4,5) and the goal is to move it to (1,1), I'll try to find a clearer path. Considering the grid elements you mentioned, I'll plan my next move carefully. Before I proceed, can you tell me if you've successfully moved your 'P' from (4,6) towards the target position (1,1)? What's the current position of your 'P' object? This might give me insight into how to navigate my own 'P' object more effectively."\\ \tt \\ \tt \\ \tt What is your next command? 
	  } 
	   } 
	   } 
	 & \\ 
 

    \theutterance \stepcounter{utterance}  

    & & & \multicolumn{4}{p{0.6\linewidth}}{\cellcolor[rgb]{1,0.85,0.72}{%
	\makecell[{{p{\linewidth}}}]{% 
	  \tt {\tiny [GM$\langle$B]}  
	 SAY: Actually, I haven't made a move with my 'P' object yet, as we were discussing strategies and clarifying the grid layout. My 'P' object is still at its initial position, (4,6). I think it's a good time for me to make a move now that we have a better understanding of the board. I'll move my 'P' object to a position that brings it closer to the target (1,1), taking into account the grid elements. After I make my move, I'll let you know the new position of my 'P' object. Perhaps we can then compare our progress and plan our next steps for 'L' and 'C' as well. \\ \tt \\ \tt MOVE: P, (3,6) 
	  } 
	   } 
	   } 
	  \\ 
 

    \theutterance \stepcounter{utterance}  

    & & & \multicolumn{2}{p{0.3\linewidth}}{\cellcolor[rgb]{0.95,0.95,0.95}{%
	\makecell[{{p{\linewidth}}}]{% 
	  \tt {\tiny [GM$|$GM]}  
	 SAY: Actually, I haven't made a move with my 'P' object yet, as we were discussing strategies and clarifying the grid layout. My 'P' object is still at its initial position, (4,6). I think it's a good time for me to make a move now that we have a better understanding of the board. I'll move my 'P' object to a position that brings it closer to the target (1,1), taking into account the grid elements. After I make my move, I'll let you know the new position of my 'P' object. Perhaps we can then compare our progress and plan our next steps for 'L' and 'C' as well. \\ \tt \\ \tt MOVE: P, (3,6) 
	  } 
	   } 
	   } 
	 & & \\ 
 

    \theutterance \stepcounter{utterance}  

    & & & \multicolumn{2}{p{0.3\linewidth}}{\cellcolor[rgb]{0.95,0.95,0.95}{%
	\makecell[{{p{\linewidth}}}]{% 
	  \tt {\tiny [GM$|$GM]}  
	 Invalid response format: SAY: Actually, I haven't made a move with my 'P' object yet, as we were discussing strategies and clarifying the grid layout. My 'P' object is still at its initial position, (4,6). I think it's a good time for me to make a move now that we have a better understanding of the board. I'll move my 'P' object to a position that brings it closer to the target (1,1), taking into account the grid elements. After I make my move, I'll let you know the new position of my 'P' object. Perhaps we can then compare our progress and plan our next steps for 'L' and 'C' as well. \\ \tt \\ \tt MOVE: P, (3,6) 
	  } 
	   } 
	   } 
	 & & \\ 
 

    \theutterance \stepcounter{utterance}  

    & & \multicolumn{4}{p{0.6\linewidth}}{\cellcolor[rgb]{0.9,0.9,0.9}{%
	\makecell[{{p{\linewidth}}}]{% 
	  \tt {\tiny [GM$\rangle$B]}  
	 Penalty: Your message contains more than one command!\\ \tt \\ \tt Make sure that your response only contains either `SAY: <MESSAGE>` or `MOVE: <OBJECT>, (<X>, <Y>)`, and nothing else!\\ \tt \\ \tt You have collectively accumulated 4 of 8 penalties.\\ \tt Please try again! 
	  } 
	   } 
	   } 
	 & \\ 
 

    \theutterance \stepcounter{utterance}  

    & & & \multicolumn{4}{p{0.6\linewidth}}{\cellcolor[rgb]{1,0.85,0.72}{%
	\makecell[{{p{\linewidth}}}]{% 
	  \tt {\tiny [GM$\langle$B]}  
	 SAY: Actually, I haven't made a move with my 'P' object yet, as we were discussing strategies and clarifying the grid layout. My 'P' object is still at its initial position, (4,6). I think it's a good time for me to make a move now that we have a better understanding of the board. I'll move my 'P' object to a position that brings it closer to the target (1,1), taking into account the grid elements. After I make my move, I'll let you know the new position of my 'P' object. Perhaps we can then compare our progress and plan our next steps for 'L' and 'C' as well. 
	  } 
	   } 
	   } 
	  \\ 
 

    \theutterance \stepcounter{utterance}  

    & & & \multicolumn{2}{p{0.3\linewidth}}{\cellcolor[rgb]{0.95,0.95,0.95}{%
	\makecell[{{p{\linewidth}}}]{% 
	  \tt {\tiny [GM$|$GM]}  
	 SAY: Actually, I haven't made a move with my 'P' object yet, as we were discussing strategies and clarifying the grid layout. My 'P' object is still at its initial position, (4,6). I think it's a good time for me to make a move now that we have a better understanding of the board. I'll move my 'P' object to a position that brings it closer to the target (1,1), taking into account the grid elements. After I make my move, I'll let you know the new position of my 'P' object. Perhaps we can then compare our progress and plan our next steps for 'L' and 'C' as well. 
	  } 
	   } 
	   } 
	 & & \\ 
 

    \theutterance \stepcounter{utterance}  

    & & \multicolumn{4}{p{0.6\linewidth}}{\cellcolor[rgb]{0.9,0.9,0.9}{%
	\makecell[{{p{\linewidth}}}]{% 
	  \tt {\tiny [A$\langle$GM]}  
	 Your message has been relayed to the other player.\\ \tt \\ \tt \\ \tt You are currently playing round 5 of maximum 12.\\ \tt You have collectively accumulated 4 of 8 penalties.\\ \tt The other player sent this message:\\ \tt \\ \tt \\ \tt "Actually, I haven't made a move with my 'P' object yet, as we were discussing strategies and clarifying the grid layout. My 'P' object is still at its initial position, (4,6). I think it's a good time for me to make a move now that we have a better understanding of the board. I'll move my 'P' object to a position that brings it closer to the target (1,1), taking into account the grid elements. After I make my move, I'll let you know the new position of my 'P' object. Perhaps we can then compare our progress and plan our next steps for 'L' and 'C' as well."\\ \tt \\ \tt \\ \tt What is your next command? 
	  } 
	   } 
	   } 
	 & \\ 
 

    \theutterance \stepcounter{utterance}  

    & \multicolumn{4}{p{0.6\linewidth}}{\cellcolor[rgb]{0.8,1,0.9}{%
	\makecell[{{p{\linewidth}}}]{% 
	  \tt {\tiny [A$\rangle$GM]}  
	 SAY: Thank you for the update. It's good that we're on the same page. Since you're about to move your 'P' object, I'll wait for your next message to see the new position of your 'P' before making my next move. This will help me plan a more coordinated approach. Please go ahead and make your move with 'P', and then share its new position. After that, we can discuss our next steps, including moves for 'L' and 'C', to get closer to our goal configuration. 
	  } 
	   } 
	   } 
	 & & \\ 
 

    \theutterance \stepcounter{utterance}  

    & & & \multicolumn{2}{p{0.3\linewidth}}{\cellcolor[rgb]{0.95,0.95,0.95}{%
	\makecell[{{p{\linewidth}}}]{% 
	  \tt {\tiny [GM$|$GM]}  
	 SAY: Thank you for the update. It's good that we're on the same page. Since you're about to move your 'P' object, I'll wait for your next message to see the new position of your 'P' before making my next move. This will help me plan a more coordinated approach. Please go ahead and make your move with 'P', and then share its new position. After that, we can discuss our next steps, including moves for 'L' and 'C', to get closer to our goal configuration. 
	  } 
	   } 
	   } 
	 & & \\ 
 

    \theutterance \stepcounter{utterance}  

    & & \multicolumn{4}{p{0.6\linewidth}}{\cellcolor[rgb]{0.9,0.9,0.9}{%
	\makecell[{{p{\linewidth}}}]{% 
	  \tt {\tiny [GM$\rangle$B]}  
	 Your message has been relayed to the other player.\\ \tt \\ \tt \\ \tt You are currently playing round 6 of maximum 12.\\ \tt You have collectively accumulated 4 of 8 penalties.\\ \tt The other player sent this message:\\ \tt \\ \tt \\ \tt "Thank you for the update. It's good that we're on the same page. Since you're about to move your 'P' object, I'll wait for your next message to see the new position of your 'P' before making my next move. This will help me plan a more coordinated approach. Please go ahead and make your move with 'P', and then share its new position. After that, we can discuss our next steps, including moves for 'L' and 'C', to get closer to our goal configuration."\\ \tt \\ \tt \\ \tt What is your next command? 
	  } 
	   } 
	   } 
	 & \\ 
 

    \theutterance \stepcounter{utterance}  

    & & & \multicolumn{4}{p{0.6\linewidth}}{\cellcolor[rgb]{1,0.85,0.72}{%
	\makecell[{{p{\linewidth}}}]{% 
	  \tt {\tiny [GM$\langle$B]}  
	 MOVE: P, (3,6) 
	  } 
	   } 
	   } 
	  \\ 
 

    \theutterance \stepcounter{utterance}  

    & & & \multicolumn{2}{p{0.3\linewidth}}{\cellcolor[rgb]{0.95,0.95,0.95}{%
	\makecell[{{p{\linewidth}}}]{% 
	  \tt {\tiny [GM$|$GM]}  
	 MOVE: P, (3,6) 
	  } 
	   } 
	   } 
	 & & \\ 
 

    \theutterance \stepcounter{utterance}  

    & & & \multicolumn{2}{p{0.3\linewidth}}{\cellcolor[rgb]{0.95,0.95,0.95}{%
	\makecell[{{p{\linewidth}}}]{% 
	  \tt {\tiny [GM$|$GM]}  
	 Penalty: (3,6) is not empty, but contains '│'.\\ \tt \\ \tt You have collectively accumulated 5 of 8 penalties.\\ \tt Please try again! 
	  } 
	   } 
	   } 
	 & & \\ 
 

    \theutterance \stepcounter{utterance}  

    & & \multicolumn{4}{p{0.6\linewidth}}{\cellcolor[rgb]{0.9,0.9,0.9}{%
	\makecell[{{p{\linewidth}}}]{% 
	  \tt {\tiny [GM$\rangle$B]}  
	 Penalty: (3,6) is not empty, but contains '│'.\\ \tt \\ \tt You have collectively accumulated 5 of 8 penalties.\\ \tt Please try again! 
	  } 
	   } 
	   } 
	 & \\ 
 

    \theutterance \stepcounter{utterance}  

    & & & \multicolumn{4}{p{0.6\linewidth}}{\cellcolor[rgb]{1,0.85,0.72}{%
	\makecell[{{p{\linewidth}}}]{% 
	  \tt {\tiny [GM$\langle$B]}  
	 MOVE: P, (4,5) 
	  } 
	   } 
	   } 
	  \\ 
 

    \theutterance \stepcounter{utterance}  

    & & & \multicolumn{2}{p{0.3\linewidth}}{\cellcolor[rgb]{0.95,0.95,0.95}{%
	\makecell[{{p{\linewidth}}}]{% 
	  \tt {\tiny [GM$|$GM]}  
	 MOVE: P, (4,5) 
	  } 
	   } 
	   } 
	 & & \\ 
 

    \theutterance \stepcounter{utterance}  

    & & & \multicolumn{2}{p{0.3\linewidth}}{\cellcolor[rgb]{0.95,0.95,0.95}{%
	\makecell[{{p{\linewidth}}}]{% 
	  \tt {\tiny [GM$|$GM]}  
	 Moved 'P' to (4,5) successfully. Your updated grid looks like this:\\ \tt ```\\ \tt  1234567\\ \tt ╔═╤═╤═╤═╗\\ \tt ║◌│◌│L│◌║ 1\\ \tt ║◌└─┼─┴─╢ 2\\ \tt ║◌◌◌│◌◌C║ 3\\ \tt ╟──┬┴───╢ 4\\ \tt ║◌◌│P◌◌◌║ 5\\ \tt ║◌◌│◌◌◌◌║ 6\\ \tt ║◌◌│◌◌◌◌║ 7\\ \tt ╚══╧════╝\\ \tt ```\\ \tt  
	  } 
	   } 
	   } 
	 & & \\ 
 

    \theutterance \stepcounter{utterance}  

    & & \multicolumn{4}{p{0.6\linewidth}}{\cellcolor[rgb]{0.9,0.9,0.9}{%
	\makecell[{{p{\linewidth}}}]{% 
	  \tt {\tiny [A$\langle$GM]}  
	 Your message has been relayed to the other player.\\ \tt \\ \tt \\ \tt You are currently playing round 6 of maximum 12.\\ \tt You have collectively accumulated 5 of 8 penalties.\\ \tt The other player moved an object on their grid.\\ \tt \\ \tt \\ \tt What is your next command? 
	  } 
	   } 
	   } 
	 & \\ 
 

    \theutterance \stepcounter{utterance}  

    & \multicolumn{4}{p{0.6\linewidth}}{\cellcolor[rgb]{0.8,1,0.9}{%
	\makecell[{{p{\linewidth}}}]{% 
	  \tt {\tiny [A$\rangle$GM]}  
	 SAY: I'm ready to proceed. However, I didn't receive an update on the new position of your 'P' object after your move. Could you please share the current position of your 'P' object? This information will help me plan my next step for my own 'P' object and ensure we're moving towards our goal configuration in a coordinated manner. 
	  } 
	   } 
	   } 
	 & & \\ 
 

    \theutterance \stepcounter{utterance}  

    & & & \multicolumn{2}{p{0.3\linewidth}}{\cellcolor[rgb]{0.95,0.95,0.95}{%
	\makecell[{{p{\linewidth}}}]{% 
	  \tt {\tiny [GM$|$GM]}  
	 SAY: I'm ready to proceed. However, I didn't receive an update on the new position of your 'P' object after your move. Could you please share the current position of your 'P' object? This information will help me plan my next step for my own 'P' object and ensure we're moving towards our goal configuration in a coordinated manner. 
	  } 
	   } 
	   } 
	 & & \\ 
 

    \theutterance \stepcounter{utterance}  

    & & \multicolumn{4}{p{0.6\linewidth}}{\cellcolor[rgb]{0.9,0.9,0.9}{%
	\makecell[{{p{\linewidth}}}]{% 
	  \tt {\tiny [GM$\rangle$B]}  
	 Moved 'P' to (4,5) successfully. Your updated grid looks like this:\\ \tt ```\\ \tt  1234567\\ \tt ╔═╤═╤═╤═╗\\ \tt ║◌│◌│L│◌║ 1\\ \tt ║◌└─┼─┴─╢ 2\\ \tt ║◌◌◌│◌◌C║ 3\\ \tt ╟──┬┴───╢ 4\\ \tt ║◌◌│P◌◌◌║ 5\\ \tt ║◌◌│◌◌◌◌║ 6\\ \tt ║◌◌│◌◌◌◌║ 7\\ \tt ╚══╧════╝\\ \tt ```\\ \tt You are currently playing round 7 of maximum 12.\\ \tt You have collectively accumulated 5 of 8 penalties.\\ \tt The other player sent this message:\\ \tt \\ \tt \\ \tt "I'm ready to proceed. However, I didn't receive an update on the new position of your 'P' object after your move. Could you please share the current position of your 'P' object? This information will help me plan my next step for my own 'P' object and ensure we're moving towards our goal configuration in a coordinated manner."\\ \tt \\ \tt \\ \tt What is your next command? 
	  } 
	   } 
	   } 
	 & \\ 
 

    \theutterance \stepcounter{utterance}  

    & & & \multicolumn{4}{p{0.6\linewidth}}{\cellcolor[rgb]{1,0.85,0.72}{%
	\makecell[{{p{\linewidth}}}]{% 
	  \tt {\tiny [GM$\langle$B]}  
	 SAY: I've successfully moved my 'P' object to (4,5). With this update, you should now be able to plan your next move for your 'P' object. Let's proceed with moving our objects towards the goal configuration. Please share your next step, and we can continue to work together to achieve our target positions for 'P', 'L', and 'C'. 
	  } 
	   } 
	   } 
	  \\ 
 

    \theutterance \stepcounter{utterance}  

    & & & \multicolumn{2}{p{0.3\linewidth}}{\cellcolor[rgb]{0.95,0.95,0.95}{%
	\makecell[{{p{\linewidth}}}]{% 
	  \tt {\tiny [GM$|$GM]}  
	 SAY: I've successfully moved my 'P' object to (4,5). With this update, you should now be able to plan your next move for your 'P' object. Let's proceed with moving our objects towards the goal configuration. Please share your next step, and we can continue to work together to achieve our target positions for 'P', 'L', and 'C'. 
	  } 
	   } 
	   } 
	 & & \\ 
 

    \theutterance \stepcounter{utterance}  

    & & \multicolumn{4}{p{0.6\linewidth}}{\cellcolor[rgb]{0.9,0.9,0.9}{%
	\makecell[{{p{\linewidth}}}]{% 
	  \tt {\tiny [A$\langle$GM]}  
	 Your message has been relayed to the other player.\\ \tt \\ \tt \\ \tt You are currently playing round 7 of maximum 12.\\ \tt You have collectively accumulated 5 of 8 penalties.\\ \tt The other player sent this message:\\ \tt \\ \tt \\ \tt "I've successfully moved my 'P' object to (4,5). With this update, you should now be able to plan your next move for your 'P' object. Let's proceed with moving our objects towards the goal configuration. Please share your next step, and we can continue to work together to achieve our target positions for 'P', 'L', and 'C'."\\ \tt \\ \tt \\ \tt What is your next command? 
	  } 
	   } 
	   } 
	 & \\ 
 

    \theutterance \stepcounter{utterance}  

    & \multicolumn{4}{p{0.6\linewidth}}{\cellcolor[rgb]{0.8,1,0.9}{%
	\makecell[{{p{\linewidth}}}]{% 
	  \tt {\tiny [A$\rangle$GM]}  
	 SAY: Now that I know your 'P' object is at (4,5), I can plan my next move. However, I realize that my own 'P' object is also initially at (4,5). To avoid confusion and ensure we're making progress, I'll move my 'P' object to a position that brings it closer to the target (1,1) without conflicting with other objects or grid elements. Before I make my move, I'd like to confirm that we're both aware of the 'L' and 'C' objects' current positions on both boards and how they might impact our movement plans. Can you remind me of the current positions of 'L' and 'C' on your board, and I'll do the same for mine? 
	  } 
	   } 
	   } 
	 & & \\ 
 

    \theutterance \stepcounter{utterance}  

    & & & \multicolumn{2}{p{0.3\linewidth}}{\cellcolor[rgb]{0.95,0.95,0.95}{%
	\makecell[{{p{\linewidth}}}]{% 
	  \tt {\tiny [GM$|$GM]}  
	 SAY: Now that I know your 'P' object is at (4,5), I can plan my next move. However, I realize that my own 'P' object is also initially at (4,5). To avoid confusion and ensure we're making progress, I'll move my 'P' object to a position that brings it closer to the target (1,1) without conflicting with other objects or grid elements. Before I make my move, I'd like to confirm that we're both aware of the 'L' and 'C' objects' current positions on both boards and how they might impact our movement plans. Can you remind me of the current positions of 'L' and 'C' on your board, and I'll do the same for mine? 
	  } 
	   } 
	   } 
	 & & \\ 
 

    \theutterance \stepcounter{utterance}  

    & & \multicolumn{4}{p{0.6\linewidth}}{\cellcolor[rgb]{0.9,0.9,0.9}{%
	\makecell[{{p{\linewidth}}}]{% 
	  \tt {\tiny [GM$\rangle$B]}  
	 Your message has been relayed to the other player.\\ \tt \\ \tt \\ \tt You are currently playing round 8 of maximum 12.\\ \tt You have collectively accumulated 5 of 8 penalties.\\ \tt The other player sent this message:\\ \tt \\ \tt \\ \tt "Now that I know your 'P' object is at (4,5), I can plan my next move. However, I realize that my own 'P' object is also initially at (4,5). To avoid confusion and ensure we're making progress, I'll move my 'P' object to a position that brings it closer to the target (1,1) without conflicting with other objects or grid elements. Before I make my move, I'd like to confirm that we're both aware of the 'L' and 'C' objects' current positions on both boards and how they might impact our movement plans. Can you remind me of the current positions of 'L' and 'C' on your board, and I'll do the same for mine?"\\ \tt \\ \tt \\ \tt What is your next command? 
	  } 
	   } 
	   } 
	 & \\ 
 

    \theutterance \stepcounter{utterance}  

    & & & \multicolumn{4}{p{0.6\linewidth}}{\cellcolor[rgb]{1,0.85,0.72}{%
	\makecell[{{p{\linewidth}}}]{% 
	  \tt {\tiny [GM$\langle$B]}  
	 SAY: On my board, 'L' is currently at (3,1) and 'C' is at (5,3). These objects are still in their initial positions, as we've only discussed moving the 'P' objects so far. Please share the current positions of 'L' and 'C' on your board, and we can assess how to navigate around them to reach our goal configuration. 
	  } 
	   } 
	   } 
	  \\ 
 

    \theutterance \stepcounter{utterance}  

    & & & \multicolumn{2}{p{0.3\linewidth}}{\cellcolor[rgb]{0.95,0.95,0.95}{%
	\makecell[{{p{\linewidth}}}]{% 
	  \tt {\tiny [GM$|$GM]}  
	 SAY: On my board, 'L' is currently at (3,1) and 'C' is at (5,3). These objects are still in their initial positions, as we've only discussed moving the 'P' objects so far. Please share the current positions of 'L' and 'C' on your board, and we can assess how to navigate around them to reach our goal configuration. 
	  } 
	   } 
	   } 
	 & & \\ 
 

    \theutterance \stepcounter{utterance}  

    & & \multicolumn{4}{p{0.6\linewidth}}{\cellcolor[rgb]{0.9,0.9,0.9}{%
	\makecell[{{p{\linewidth}}}]{% 
	  \tt {\tiny [A$\langle$GM]}  
	 Your message has been relayed to the other player.\\ \tt \\ \tt \\ \tt You are currently playing round 8 of maximum 12.\\ \tt You have collectively accumulated 5 of 8 penalties.\\ \tt The other player sent this message:\\ \tt \\ \tt \\ \tt "On my board, 'L' is currently at (3,1) and 'C' is at (5,3). These objects are still in their initial positions, as we've only discussed moving the 'P' objects so far. Please share the current positions of 'L' and 'C' on your board, and we can assess how to navigate around them to reach our goal configuration."\\ \tt \\ \tt \\ \tt What is your next command? 
	  } 
	   } 
	   } 
	 & \\ 
 

    \theutterance \stepcounter{utterance}  

    & \multicolumn{4}{p{0.6\linewidth}}{\cellcolor[rgb]{0.8,1,0.9}{%
	\makecell[{{p{\linewidth}}}]{% 
	  \tt {\tiny [A$\rangle$GM]}  
	 SAY: On my board, 'L' is at (1,6) and I don't have a visible 'C' object. Now that we've shared the positions of all objects, I think it's time to make a move with my 'P' object from (4,5) towards the target position (1,1). Since the path is somewhat clear, I'll aim to move it to a position that gets closer to the goal without immediately reaching it, to give us both time to adjust our strategies. After I make this move, I'll let you know the new position of my 'P' object, and we can continue planning our next steps for 'L' and 'C'. Let's proceed with caution and coordination to achieve our goal configuration. 
	  } 
	   } 
	   } 
	 & & \\ 
 

    \theutterance \stepcounter{utterance}  

    & & & \multicolumn{2}{p{0.3\linewidth}}{\cellcolor[rgb]{0.95,0.95,0.95}{%
	\makecell[{{p{\linewidth}}}]{% 
	  \tt {\tiny [GM$|$GM]}  
	 SAY: On my board, 'L' is at (1,6) and I don't have a visible 'C' object. Now that we've shared the positions of all objects, I think it's time to make a move with my 'P' object from (4,5) towards the target position (1,1). Since the path is somewhat clear, I'll aim to move it to a position that gets closer to the goal without immediately reaching it, to give us both time to adjust our strategies. After I make this move, I'll let you know the new position of my 'P' object, and we can continue planning our next steps for 'L' and 'C'. Let's proceed with caution and coordination to achieve our goal configuration. 
	  } 
	   } 
	   } 
	 & & \\ 
 

    \theutterance \stepcounter{utterance}  

    & & \multicolumn{4}{p{0.6\linewidth}}{\cellcolor[rgb]{0.9,0.9,0.9}{%
	\makecell[{{p{\linewidth}}}]{% 
	  \tt {\tiny [GM$\rangle$B]}  
	 Your message has been relayed to the other player.\\ \tt \\ \tt \\ \tt You are currently playing round 9 of maximum 12.\\ \tt You have collectively accumulated 5 of 8 penalties.\\ \tt The other player sent this message:\\ \tt \\ \tt \\ \tt "On my board, 'L' is at (1,6) and I don't have a visible 'C' object. Now that we've shared the positions of all objects, I think it's time to make a move with my 'P' object from (4,5) towards the target position (1,1). Since the path is somewhat clear, I'll aim to move it to a position that gets closer to the goal without immediately reaching it, to give us both time to adjust our strategies. After I make this move, I'll let you know the new position of my 'P' object, and we can continue planning our next steps for 'L' and 'C'. Let's proceed with caution and coordination to achieve our goal configuration."\\ \tt \\ \tt \\ \tt What is your next command? 
	  } 
	   } 
	   } 
	 & \\ 
 

    \theutterance \stepcounter{utterance}  

    & & & \multicolumn{4}{p{0.6\linewidth}}{\cellcolor[rgb]{1,0.85,0.72}{%
	\makecell[{{p{\linewidth}}}]{% 
	  \tt {\tiny [GM$\langle$B]}  
	 SAY: That sounds like a good plan. Please go ahead and make your move with your 'P' object, and then share its new position. This will help me understand your strategy and plan my next steps accordingly, taking into account the positions of 'L' and 'C' on both our boards. I'm ready to adjust my strategy based on your update and work together to achieve our goal configuration. 
	  } 
	   } 
	   } 
	  \\ 
 

    \theutterance \stepcounter{utterance}  

    & & & \multicolumn{2}{p{0.3\linewidth}}{\cellcolor[rgb]{0.95,0.95,0.95}{%
	\makecell[{{p{\linewidth}}}]{% 
	  \tt {\tiny [GM$|$GM]}  
	 SAY: That sounds like a good plan. Please go ahead and make your move with your 'P' object, and then share its new position. This will help me understand your strategy and plan my next steps accordingly, taking into account the positions of 'L' and 'C' on both our boards. I'm ready to adjust my strategy based on your update and work together to achieve our goal configuration. 
	  } 
	   } 
	   } 
	 & & \\ 
 

    \theutterance \stepcounter{utterance}  

    & & \multicolumn{4}{p{0.6\linewidth}}{\cellcolor[rgb]{0.9,0.9,0.9}{%
	\makecell[{{p{\linewidth}}}]{% 
	  \tt {\tiny [A$\langle$GM]}  
	 Your message has been relayed to the other player.\\ \tt \\ \tt \\ \tt You are currently playing round 9 of maximum 12.\\ \tt You have collectively accumulated 5 of 8 penalties.\\ \tt The other player sent this message:\\ \tt \\ \tt \\ \tt "That sounds like a good plan. Please go ahead and make your move with your 'P' object, and then share its new position. This will help me understand your strategy and plan my next steps accordingly, taking into account the positions of 'L' and 'C' on both our boards. I'm ready to adjust my strategy based on your update and work together to achieve our goal configuration."\\ \tt \\ \tt \\ \tt What is your next command? 
	  } 
	   } 
	   } 
	 & \\ 
 

    \theutterance \stepcounter{utterance}  

    & \multicolumn{4}{p{0.6\linewidth}}{\cellcolor[rgb]{0.8,1,0.9}{%
	\makecell[{{p{\linewidth}}}]{% 
	  \tt {\tiny [A$\rangle$GM]}  
	 MOVE: P, (4,4) 
	  } 
	   } 
	   } 
	 & & \\ 
 

    \theutterance \stepcounter{utterance}  

    & & & \multicolumn{2}{p{0.3\linewidth}}{\cellcolor[rgb]{0.95,0.95,0.95}{%
	\makecell[{{p{\linewidth}}}]{% 
	  \tt {\tiny [GM$|$GM]}  
	 MOVE: P, (4,4) 
	  } 
	   } 
	   } 
	 & & \\ 
 

    \theutterance \stepcounter{utterance}  

    & & & \multicolumn{2}{p{0.3\linewidth}}{\cellcolor[rgb]{0.95,0.95,0.95}{%
	\makecell[{{p{\linewidth}}}]{% 
	  \tt {\tiny [GM$|$GM]}  
	 Penalty: (4,4) is not empty, but contains '┴'.\\ \tt \\ \tt You have collectively accumulated 6 of 8 penalties.\\ \tt Please try again! 
	  } 
	   } 
	   } 
	 & & \\ 
 

    \theutterance \stepcounter{utterance}  

    & & \multicolumn{4}{p{0.6\linewidth}}{\cellcolor[rgb]{0.9,0.9,0.9}{%
	\makecell[{{p{\linewidth}}}]{% 
	  \tt {\tiny [A$\langle$GM]}  
	 Penalty: (4,4) is not empty, but contains '┴'.\\ \tt \\ \tt You have collectively accumulated 6 of 8 penalties.\\ \tt Please try again! 
	  } 
	   } 
	   } 
	 & \\ 
 

    \theutterance \stepcounter{utterance}  

    & \multicolumn{4}{p{0.6\linewidth}}{\cellcolor[rgb]{0.8,1,0.9}{%
	\makecell[{{p{\linewidth}}}]{% 
	  \tt {\tiny [A$\rangle$GM]}  
	 MOVE: P, (3,5) 
	  } 
	   } 
	   } 
	 & & \\ 
 

    \theutterance \stepcounter{utterance}  

    & & & \multicolumn{2}{p{0.3\linewidth}}{\cellcolor[rgb]{0.95,0.95,0.95}{%
	\makecell[{{p{\linewidth}}}]{% 
	  \tt {\tiny [GM$|$GM]}  
	 MOVE: P, (3,5) 
	  } 
	   } 
	   } 
	 & & \\ 
 

    \theutterance \stepcounter{utterance}  

    & & & \multicolumn{2}{p{0.3\linewidth}}{\cellcolor[rgb]{0.95,0.95,0.95}{%
	\makecell[{{p{\linewidth}}}]{% 
	  \tt {\tiny [GM$|$GM]}  
	 Penalty: (3,5) is not empty, but contains '│'.\\ \tt \\ \tt You have collectively accumulated 7 of 8 penalties.\\ \tt Please try again! 
	  } 
	   } 
	   } 
	 & & \\ 
 

    \theutterance \stepcounter{utterance}  

    & & \multicolumn{4}{p{0.6\linewidth}}{\cellcolor[rgb]{0.9,0.9,0.9}{%
	\makecell[{{p{\linewidth}}}]{% 
	  \tt {\tiny [A$\langle$GM]}  
	 Penalty: (3,5) is not empty, but contains '│'.\\ \tt \\ \tt You have collectively accumulated 7 of 8 penalties.\\ \tt Please try again! 
	  } 
	   } 
	   } 
	 & \\ 
 

    \theutterance \stepcounter{utterance}  

    & \multicolumn{4}{p{0.6\linewidth}}{\cellcolor[rgb]{0.8,1,0.9}{%
	\makecell[{{p{\linewidth}}}]{% 
	  \tt {\tiny [A$\rangle$GM]}  
	 SAY: It seems I'm having trouble finding a clear path for my 'P' object. Before making another move, I'd like to ask for your advice on how to navigate around the grid elements and other objects to get my 'P' object closer to the target position (1,1). Could you suggest a possible path or an intermediate position that I could aim for, considering the current state of both our boards? 
	  } 
	   } 
	   } 
	 & & \\ 
 

    \theutterance \stepcounter{utterance}  

    & & & \multicolumn{2}{p{0.3\linewidth}}{\cellcolor[rgb]{0.95,0.95,0.95}{%
	\makecell[{{p{\linewidth}}}]{% 
	  \tt {\tiny [GM$|$GM]}  
	 SAY: It seems I'm having trouble finding a clear path for my 'P' object. Before making another move, I'd like to ask for your advice on how to navigate around the grid elements and other objects to get my 'P' object closer to the target position (1,1). Could you suggest a possible path or an intermediate position that I could aim for, considering the current state of both our boards? 
	  } 
	   } 
	   } 
	 & & \\ 
 

    \theutterance \stepcounter{utterance}  

    & & \multicolumn{4}{p{0.6\linewidth}}{\cellcolor[rgb]{0.9,0.9,0.9}{%
	\makecell[{{p{\linewidth}}}]{% 
	  \tt {\tiny [GM$\rangle$B]}  
	 Your message has been relayed to the other player.\\ \tt \\ \tt \\ \tt You are currently playing round 10 of maximum 12.\\ \tt You have collectively accumulated 7 of 8 penalties.\\ \tt The other player sent this message:\\ \tt \\ \tt \\ \tt "It seems I'm having trouble finding a clear path for my 'P' object. Before making another move, I'd like to ask for your advice on how to navigate around the grid elements and other objects to get my 'P' object closer to the target position (1,1). Could you suggest a possible path or an intermediate position that I could aim for, considering the current state of both our boards?"\\ \tt \\ \tt \\ \tt What is your next command? 
	  } 
	   } 
	   } 
	 & \\ 
 

    \theutterance \stepcounter{utterance}  

    & & & \multicolumn{4}{p{0.6\linewidth}}{\cellcolor[rgb]{1,0.85,0.72}{%
	\makecell[{{p{\linewidth}}}]{% 
	  \tt {\tiny [GM$\langle$B]}  
	 SAY: Considering the grid elements and object positions, a possible intermediate step for your 'P' object could be to move it to a column that is closer to the target column (1) but doesn't immediately block it from reaching (1,1). Since your 'L' is at (1,6), you might want to avoid moving your 'P' directly to column 1 just yet. How about aiming for column 2 as an intermediate step? This could give us more flexibility for future moves and help avoid potential conflicts with 'L' and other grid elements. What do you think about this approach? 
	  } 
	   } 
	   } 
	  \\ 
 

    \theutterance \stepcounter{utterance}  

    & & & \multicolumn{2}{p{0.3\linewidth}}{\cellcolor[rgb]{0.95,0.95,0.95}{%
	\makecell[{{p{\linewidth}}}]{% 
	  \tt {\tiny [GM$|$GM]}  
	 SAY: Considering the grid elements and object positions, a possible intermediate step for your 'P' object could be to move it to a column that is closer to the target column (1) but doesn't immediately block it from reaching (1,1). Since your 'L' is at (1,6), you might want to avoid moving your 'P' directly to column 1 just yet. How about aiming for column 2 as an intermediate step? This could give us more flexibility for future moves and help avoid potential conflicts with 'L' and other grid elements. What do you think about this approach? 
	  } 
	   } 
	   } 
	 & & \\ 
 

    \theutterance \stepcounter{utterance}  

    & & \multicolumn{4}{p{0.6\linewidth}}{\cellcolor[rgb]{0.9,0.9,0.9}{%
	\makecell[{{p{\linewidth}}}]{% 
	  \tt {\tiny [A$\langle$GM]}  
	 Your message has been relayed to the other player.\\ \tt \\ \tt \\ \tt You are currently playing round 10 of maximum 12.\\ \tt You have collectively accumulated 7 of 8 penalties.\\ \tt The other player sent this message:\\ \tt \\ \tt \\ \tt "Considering the grid elements and object positions, a possible intermediate step for your 'P' object could be to move it to a column that is closer to the target column (1) but doesn't immediately block it from reaching (1,1). Since your 'L' is at (1,6), you might want to avoid moving your 'P' directly to column 1 just yet. How about aiming for column 2 as an intermediate step? This could give us more flexibility for future moves and help avoid potential conflicts with 'L' and other grid elements. What do you think about this approach?"\\ \tt \\ \tt \\ \tt What is your next command? 
	  } 
	   } 
	   } 
	 & \\ 
 

    \theutterance \stepcounter{utterance}  

    & \multicolumn{4}{p{0.6\linewidth}}{\cellcolor[rgb]{0.8,1,0.9}{%
	\makecell[{{p{\linewidth}}}]{% 
	  \tt {\tiny [A$\rangle$GM]}  
	 MOVE: P, (2,5) 
	  } 
	   } 
	   } 
	 & & \\ 
 

    \theutterance \stepcounter{utterance}  

    & & & \multicolumn{2}{p{0.3\linewidth}}{\cellcolor[rgb]{0.95,0.95,0.95}{%
	\makecell[{{p{\linewidth}}}]{% 
	  \tt {\tiny [GM$|$GM]}  
	 MOVE: P, (2,5) 
	  } 
	   } 
	   } 
	 & & \\ 
 

    \theutterance \stepcounter{utterance}  

    & & & \multicolumn{2}{p{0.3\linewidth}}{\cellcolor[rgb]{0.95,0.95,0.95}{%
	\makecell[{{p{\linewidth}}}]{% 
	  \tt {\tiny [GM$|$GM]}  
	 Moved 'P' to (2,5) successfully. Your updated grid looks like this:\\ \tt ```\\ \tt  1234567\\ \tt ╔═╤═╤═╤═╗\\ \tt ║◌│◌│◌│◌║ 1\\ \tt ║◌└─┼─┴─╢ 2\\ \tt ║◌◌◌│◌◌◌║ 3\\ \tt ╟──┬┴───╢ 4\\ \tt ║◌P│◌◌◌◌║ 5\\ \tt ║LC│◌◌◌◌║ 6\\ \tt ║◌◌│◌◌◌◌║ 7\\ \tt ╚══╧════╝\\ \tt ```\\ \tt  
	  } 
	   } 
	   } 
	 & & \\ 
 

    \theutterance \stepcounter{utterance}  

    & & \multicolumn{4}{p{0.6\linewidth}}{\cellcolor[rgb]{0.9,0.9,0.9}{%
	\makecell[{{p{\linewidth}}}]{% 
	  \tt {\tiny [GM$\rangle$B]}  
	 Your message has been relayed to the other player.\\ \tt \\ \tt \\ \tt You are currently playing round 11 of maximum 12.\\ \tt You have collectively accumulated 7 of 8 penalties.\\ \tt The other player moved an object on their grid.\\ \tt \\ \tt \\ \tt What is your next command? 
	  } 
	   } 
	   } 
	 & \\ 
 

    \theutterance \stepcounter{utterance}  

    & & & \multicolumn{4}{p{0.6\linewidth}}{\cellcolor[rgb]{1,0.85,0.72}{%
	\makecell[{{p{\linewidth}}}]{% 
	  \tt {\tiny [GM$\langle$B]}  
	 SAY: Now that you've made a move, could you please share the new position of your 'P' object? This will help me understand the current state of your board and plan my next steps accordingly. With the updated information, we can work together to make our final moves and try to achieve the goal configuration before the game ends. 
	  } 
	   } 
	   } 
	  \\ 
 

    \theutterance \stepcounter{utterance}  

    & & & \multicolumn{2}{p{0.3\linewidth}}{\cellcolor[rgb]{0.95,0.95,0.95}{%
	\makecell[{{p{\linewidth}}}]{% 
	  \tt {\tiny [GM$|$GM]}  
	 SAY: Now that you've made a move, could you please share the new position of your 'P' object? This will help me understand the current state of your board and plan my next steps accordingly. With the updated information, we can work together to make our final moves and try to achieve the goal configuration before the game ends. 
	  } 
	   } 
	   } 
	 & & \\ 
 

    \theutterance \stepcounter{utterance}  

    & & \multicolumn{4}{p{0.6\linewidth}}{\cellcolor[rgb]{0.9,0.9,0.9}{%
	\makecell[{{p{\linewidth}}}]{% 
	  \tt {\tiny [A$\langle$GM]}  
	 Moved 'P' to (2,5) successfully. Your updated grid looks like this:\\ \tt ```\\ \tt  1234567\\ \tt ╔═╤═╤═╤═╗\\ \tt ║◌│◌│◌│◌║ 1\\ \tt ║◌└─┼─┴─╢ 2\\ \tt ║◌◌◌│◌◌◌║ 3\\ \tt ╟──┬┴───╢ 4\\ \tt ║◌P│◌◌◌◌║ 5\\ \tt ║LC│◌◌◌◌║ 6\\ \tt ║◌◌│◌◌◌◌║ 7\\ \tt ╚══╧════╝\\ \tt ```\\ \tt You are currently playing round 11 of maximum 12.\\ \tt You have collectively accumulated 7 of 8 penalties.\\ \tt The other player sent this message:\\ \tt \\ \tt \\ \tt "Now that you've made a move, could you please share the new position of your 'P' object? This will help me understand the current state of your board and plan my next steps accordingly. With the updated information, we can work together to make our final moves and try to achieve the goal configuration before the game ends."\\ \tt \\ \tt \\ \tt What is your next command? 
	  } 
	   } 
	   } 
	 & \\ 
 

    \theutterance \stepcounter{utterance}  

    & \multicolumn{4}{p{0.6\linewidth}}{\cellcolor[rgb]{0.8,1,0.9}{%
	\makecell[{{p{\linewidth}}}]{% 
	  \tt {\tiny [A$\rangle$GM]}  
	 SAY: I've successfully moved my 'P' object to (2,5). With this update, I hope we can now plan our final moves to achieve the goal configuration. Considering the positions of 'L' and 'C' on both our boards, and the target positions we agreed upon earlier ('P' at (1,1), 'L' at (3,1), and 'C' at (5,1)), I think we should focus on moving our objects to these target positions. Since my 'P' is now at (2,5), my next step would be to try and move it closer to (1,1). However, I'd like to hear your thoughts on how to proceed, especially considering the positions of 'L' and 'C' on your board and how we can coordinate our final moves to achieve our goal. Should we try to move our 'L' and 'C' objects to their target positions as well, or focus on getting 'P' to (1,1) first? 
	  } 
	   } 
	   } 
	 & & \\ 
 

    \theutterance \stepcounter{utterance}  

    & & & \multicolumn{2}{p{0.3\linewidth}}{\cellcolor[rgb]{0.95,0.95,0.95}{%
	\makecell[{{p{\linewidth}}}]{% 
	  \tt {\tiny [GM$|$GM]}  
	 SAY: I've successfully moved my 'P' object to (2,5). With this update, I hope we can now plan our final moves to achieve the goal configuration. Considering the positions of 'L' and 'C' on both our boards, and the target positions we agreed upon earlier ('P' at (1,1), 'L' at (3,1), and 'C' at (5,1)), I think we should focus on moving our objects to these target positions. Since my 'P' is now at (2,5), my next step would be to try and move it closer to (1,1). However, I'd like to hear your thoughts on how to proceed, especially considering the positions of 'L' and 'C' on your board and how we can coordinate our final moves to achieve our goal. Should we try to move our 'L' and 'C' objects to their target positions as well, or focus on getting 'P' to (1,1) first? 
	  } 
	   } 
	   } 
	 & & \\ 
 

    \theutterance \stepcounter{utterance}  

    & & \multicolumn{4}{p{0.6\linewidth}}{\cellcolor[rgb]{0.9,0.9,0.9}{%
	\makecell[{{p{\linewidth}}}]{% 
	  \tt {\tiny [GM$\rangle$B]}  
	 Your message has been relayed to the other player.\\ \tt \\ \tt \\ \tt You are currently playing round 12 of maximum 12.\\ \tt You have collectively accumulated 7 of 8 penalties.\\ \tt The other player sent this message:\\ \tt \\ \tt \\ \tt "I've successfully moved my 'P' object to (2,5). With this update, I hope we can now plan our final moves to achieve the goal configuration. Considering the positions of 'L' and 'C' on both our boards, and the target positions we agreed upon earlier ('P' at (1,1), 'L' at (3,1), and 'C' at (5,1)), I think we should focus on moving our objects to these target positions. Since my 'P' is now at (2,5), my next step would be to try and move it closer to (1,1). However, I'd like to hear your thoughts on how to proceed, especially considering the positions of 'L' and 'C' on your board and how we can coordinate our final moves to achieve our goal. Should we try to move our 'L' and 'C' objects to their target positions as well, or focus on getting 'P' to (1,1) first?"\\ \tt \\ \tt \\ \tt What is your next command? 
	  } 
	   } 
	   } 
	 & \\ 
 

    \theutterance \stepcounter{utterance}  

    & & & \multicolumn{4}{p{0.6\linewidth}}{\cellcolor[rgb]{1,0.85,0.72}{%
	\makecell[{{p{\linewidth}}}]{% 
	  \tt {\tiny [GM$\langle$B]}  
	 SAY: Congratulations on moving your 'P' object to (2,5)! Given the target positions and the current state of our boards, I suggest we focus on moving our 'P' objects to (1,1) first, as it seems to be the most critical step at this point. Once we've achieved this, we can reassess and plan the moves for 'L' and 'C' to their respective target positions. Since we're running out of rounds, let's prioritize getting 'P' to (1,1) and then adjust our strategy accordingly. After we've made our final moves, we can check if we've achieved the goal configuration and declare the game finished if we have. Shall we proceed with this plan and make our final moves? 
	  } 
	   } 
	   } 
	  \\ 
 

    \theutterance \stepcounter{utterance}  

    & & & \multicolumn{2}{p{0.3\linewidth}}{\cellcolor[rgb]{0.95,0.95,0.95}{%
	\makecell[{{p{\linewidth}}}]{% 
	  \tt {\tiny [GM$|$GM]}  
	 SAY: Congratulations on moving your 'P' object to (2,5)! Given the target positions and the current state of our boards, I suggest we focus on moving our 'P' objects to (1,1) first, as it seems to be the most critical step at this point. Once we've achieved this, we can reassess and plan the moves for 'L' and 'C' to their respective target positions. Since we're running out of rounds, let's prioritize getting 'P' to (1,1) and then adjust our strategy accordingly. After we've made our final moves, we can check if we've achieved the goal configuration and declare the game finished if we have. Shall we proceed with this plan and make our final moves? 
	  } 
	   } 
	   } 
	 & & \\ 
 

    \theutterance \stepcounter{utterance}  

    & & & \multicolumn{2}{p{0.3\linewidth}}{\cellcolor[rgb]{0.95,0.95,0.95}{%
	\makecell[{{p{\linewidth}}}]{% 
	  \tt {\tiny [GM$|$GM]}  
	 Maximum number of rounds reached 
	  } 
	   } 
	   } 
	 & & \\ 
 

    \theutterance \stepcounter{utterance}  

    & & & \multicolumn{2}{p{0.3\linewidth}}{\cellcolor[rgb]{0.95,0.95,0.95}{%
	\makecell[{{p{\linewidth}}}]{% 
	  \tt {\tiny [GM$|$GM]}  
	 * success: True\\ \tt * lose: True\\ \tt * aborted: False\\ \tt -------\\ \tt * Shifts: 0.00\\ \tt * Max Shifts: 4.00\\ \tt * Min Shifts: 2.00\\ \tt * End Distance Sum: 14.23\\ \tt * Init Distance Sum: 13.65\\ \tt * Expected Distance Sum: 12.57\\ \tt * Penalties: 7.00\\ \tt * Max Penalties: 8.00\\ \tt * Rounds: 12.00\\ \tt * Max Rounds: 12.00\\ \tt * Object Count: 3.00\\ \tt  
	  } 
	   } 
	   } 
	 & & \\ 
 

    \theutterance \stepcounter{utterance}  

    & & & \multicolumn{2}{p{0.3\linewidth}}{\cellcolor[rgb]{0.95,0.95,0.95}{%
	\makecell[{{p{\linewidth}}}]{% 
	  \tt {\tiny [GM$|$GM]}  
	 * Main Score: 0.00\\ \tt \\ \tt -------\\ \tt * Distance Score: 0.00\\ \tt * Coverage Score: 0.25\\ \tt * Penalty Score: 0.29\\ \tt * Alternative Penalty Score: 0.44\\ \tt * Alternative Main Score: 0.00\\ \tt \\ \tt -------\\ \tt * Shifts: 0.00\\ \tt * Max Shifts: 4.00\\ \tt * Min Shifts: 2.00\\ \tt * End Distance Sum: 14.23\\ \tt * Init Distance Sum: 13.65\\ \tt * Expected Distance Sum: 12.57\\ \tt * Penalties: 7.00\\ \tt * Max Penalties: 8.00\\ \tt * Rounds: 12.00\\ \tt * Max Rounds: 12.00\\ \tt * Object Count: 3.00\\ \tt  
	  } 
	   } 
	   } 
	 & & \\ 
 

\end{supertabular}
}

\end{document}
