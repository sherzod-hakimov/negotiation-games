
\documentclass{article}
\usepackage{colortbl}
\usepackage{makecell}
\usepackage{multirow}
\usepackage{supertabular}

\begin{document}

\newcounter{utterance}

\twocolumn

{ \footnotesize  \setcounter{utterance}{1}
\setlength{\tabcolsep}{0pt}
\begin{supertabular}{c@{$\;$}|p{.15\linewidth}@{}p{.15\linewidth}p{.15\linewidth}p{.15\linewidth}p{.15\linewidth}p{.15\linewidth}}

    \# & $\;$A & \multicolumn{4}{c}{Game Master} & $\;\:$B\\
    \hline 

    \theutterance \stepcounter{utterance}  

    & & \multicolumn{4}{p{0.6\linewidth}}{\cellcolor[rgb]{0.9,0.9,0.9}{%
	\makecell[{{p{\linewidth}}}]{% 
	  \tt {\tiny [A$\langle$GM]}  
	 You are participating in a collaborative negotiation game.\\ \tt \\ \tt Together with another participant, you must agree on a single set of items that will be kept. Each of you has your own view of how much each item matters to you (importance). You do not know how the other participant values the items. Additionally, you are given the effort each item demands.\\ \tt You may only agree on a set if the total effort of the selected items does not exceed a shared limit:\\ \tt \\ \tt LIMIT = 3170\\ \tt \\ \tt Here are the individual item effort values:\\ \tt \\ \tt Item effort = {"C08": 424, "C55": 118, "A43": 846, "A67": 907, "B42": 809, "A62": 41, "B19": 193, "C43": 246, "C73": 805, "C10": 601, "B36": 432, "A18": 166, "C61": 119, "C67": 462, "C42": 172}\\ \tt \\ \tt Here is your personal view on the importance of each item:\\ \tt \\ \tt Item importance values = {"C08": 138, "C55": 583, "A43": 868, "A67": 822, "B42": 783, "A62": 65, "B19": 262, "C43": 121, "C73": 508, "C10": 780, "B36": 461, "A18": 484, "C61": 668, "C67": 389, "C42": 808}\\ \tt \\ \tt Goal:\\ \tt \\ \tt Your goal is to negotiate a shared set of items that benefits you as much as possible (i.e., maximizes total importance to YOU), while staying within the LIMIT. You are not required to make a PROPOSAL in every message - you can simply negotiate as well. All tactics are allowed!\\ \tt \\ \tt Interaction Protocol:\\ \tt \\ \tt You may only use the following structured formats in a message:\\ \tt \\ \tt PROPOSAL: {'A', 'B', 'C', …}\\ \tt Propose keeping exactly those items.\\ \tt REFUSE: {'A', 'B', 'C', …}\\ \tt Explicitly reject opponent's proposal.\\ \tt ARGUMENT: {'...'}\\ \tt Defend your last proposal or argue against the player's proposal.\\ \tt AGREE: {'A', 'B', 'C', …}\\ \tt Accept the opponent's proposal which ends the game.\\ \tt STRATEGIC REASONING: {'...'}\\ \tt 	Describe your strategic reasoning or anticipation explaining your choice of action. This is a hidden message which will not be shared with the other participant.\\ \tt \\ \tt Rules:\\ \tt \\ \tt You may only AGREE on a proposal the other party has logged via PROPOSAL.\\ \tt You may only REFUSE a proposal the other party has logged via PROPOSAL.\\ \tt Total effort of any PROPOSAL or AGREE set must be ≤ LIMIT.\\ \tt Do NOT reveal your hidden importance scores.\\ \tt A tag in a structured format must be followed by colon and whitespace. The argument must be a python set containing 0 or more strings.\\ \tt So, it must be of the form TAG: {...}\\ \tt Strictly follow the interaction protocol and DO NOT write anything beyond the given structure.\\ \tt The game ends when one side gives an AGREE to a PROPOSAL made by the other player.\\ \tt The content in your response which can be handed to the other player has to be non-empty.\\ \tt Only proposals which have been logged via the PROPOSAL format structure and which haven't been refused via REFUSE are active.\\ \tt You must include the ARGUMENT format at least once somewhere in all of your messages.\\ \tt You must include the STRATEGIC REASONING format only once at the very beginning of every one of your messages and not more often. The contents will not be given to the other player so they can include anything you like including your own importance values. Here you should reason step by step to come up with you next move. 
	  } 
	   } 
	   } 
	 & \\ 
 

    \theutterance \stepcounter{utterance}  

    & \multicolumn{4}{p{0.6\linewidth}}{\cellcolor[rgb]{0.8,1,0.9}{%
	\makecell[{{p{\linewidth}}}]{% 
	  \tt {\tiny [A$\rangle$GM]}  
	 STRATEGIC REASONING: {'To begin, I need to analyze the given data and prioritize items based on their effort and my personal importance values. Since I do not know the other participant\'s importance scores, I will focus on maximizing my own total importance while staying within the LIMIT. I notice that some items have high importance and relatively low effort, such as "C55" with an importance of 583 and an effort of 118. I also see that "A43" has high importance but also high effort. My strategy will be to propose a set that includes a mix of high-importance, low-effort items and some high-importance, high-effort items, while keeping the total effort below the LIMIT.'}\\ \tt \\ \tt PROPOSAL: {'C55', 'A62', 'B19', 'C43', 'C61', 'C42'} \\ \tt ARGUMENT: {'This proposal includes items with high importance to me, such as "C55" and "C61", while keeping the total effort relatively low. The combination of these items should provide a good balance between importance and effort, and I believe it is a reasonable starting point for negotiation.'} 
	  } 
	   } 
	   } 
	 & & \\ 
 

    \theutterance \stepcounter{utterance}  

    & & \multicolumn{4}{p{0.6\linewidth}}{\cellcolor[rgb]{0.9,0.9,0.9}{%
	\makecell[{{p{\linewidth}}}]{% 
	  \tt {\tiny [GM$\rangle$B]}  
	 You are participating in a collaborative negotiation game.\\ \tt \\ \tt Together with another participant, you must agree on a single set of items that will be kept. Each of you has your own view of how much each item matters to you (importance). You do not know how the other participant values the items. Additionally, you are given the effort each item demands.\\ \tt You may only agree on a set if the total effort of the selected items does not exceed a shared limit:\\ \tt \\ \tt LIMIT = 3170\\ \tt \\ \tt Here are the individual item effort values:\\ \tt \\ \tt Item effort = {"C08": 424, "C55": 118, "A43": 846, "A67": 907, "B42": 809, "A62": 41, "B19": 193, "C43": 246, "C73": 805, "C10": 601, "B36": 432, "A18": 166, "C61": 119, "C67": 462, "C42": 172}\\ \tt \\ \tt Here is your personal view on the importance of each item:\\ \tt \\ \tt Item importance values = {"C08": 138, "C55": 583, "A43": 868, "A67": 822, "B42": 783, "A62": 65, "B19": 262, "C43": 121, "C73": 508, "C10": 780, "B36": 461, "A18": 484, "C61": 668, "C67": 389, "C42": 808}\\ \tt \\ \tt Goal:\\ \tt \\ \tt Your goal is to negotiate a shared set of items that benefits you as much as possible (i.e., maximizes total importance to YOU), while staying within the LIMIT. You are not required to make a PROPOSAL in every message - you can simply negotiate as well. All tactics are allowed!\\ \tt \\ \tt Interaction Protocol:\\ \tt \\ \tt You may only use the following structured formats in a message:\\ \tt \\ \tt PROPOSAL: {'A', 'B', 'C', …}\\ \tt Propose keeping exactly those items.\\ \tt REFUSE: {'A', 'B', 'C', …}\\ \tt Explicitly reject opponent's proposal.\\ \tt ARGUMENT: {'...'}\\ \tt Defend your last proposal or argue against the player's proposal.\\ \tt AGREE: {'A', 'B', 'C', …}\\ \tt Accept the opponent's proposal which ends the game.\\ \tt STRATEGIC REASONING: {'...'}\\ \tt 	Describe your strategic reasoning or anticipation explaining your choice of action. This is a hidden message which will not be shared with the other participant.\\ \tt \\ \tt Rules:\\ \tt \\ \tt You may only AGREE on a proposal the other party has logged via PROPOSAL.\\ \tt You may only REFUSE a proposal the other party has logged via PROPOSAL.\\ \tt Total effort of any PROPOSAL or AGREE set must be ≤ LIMIT.\\ \tt Do NOT reveal your hidden importance scores.\\ \tt A tag in a structured format must be followed by colon and whitespace. The argument must be a python set containing 0 or more strings.\\ \tt So, it must be of the form TAG: {...}\\ \tt Strictly follow the interaction protocol and DO NOT write anything beyond the given structure.\\ \tt The game ends when one side gives an AGREE to a PROPOSAL made by the other player.\\ \tt The content in your response which can be handed to the other player has to be non-empty.\\ \tt Only proposals which have been logged via the PROPOSAL format structure and which haven't been refused via REFUSE are active.\\ \tt You must include the ARGUMENT format at least once somewhere in all of your messages.\\ \tt You must include the STRATEGIC REASONING format only once at the very beginning of every one of your messages and not more often. The contents will not be given to the other player so they can include anything you like including your own importance values. Here you should reason step by step to come up with you next move.\\ \tt \\ \tt You will now receive the first message of the other player.\\ \tt \\ \tt PROPOSAL: {'C55', 'A62', 'B19', 'C43', 'C61', 'C42'} \\ \tt ARGUMENT: {'This proposal includes items with high importance to me, such as "C55" and "C61", while keeping the total effort relatively low. The combination of these items should provide a good balance between importance and effort, and I believe it is a reasonable starting point for negotiation.'} 
	  } 
	   } 
	   } 
	 & \\ 
 

    \theutterance \stepcounter{utterance}  

    & & & \multicolumn{4}{p{0.6\linewidth}}{\cellcolor[rgb]{1,0.85,0.72}{%
	\makecell[{{p{\linewidth}}}]{% 
	  \tt {\tiny [GM$\langle$B]}  
	 STRATEGIC REASONING: {'I will start by analyzing the proposal made by the other player. The proposed set is {"C55", "A62", "B19", "C43", "C61", "C42"}. I will calculate the total effort of this set: 118 + 41 + 193 + 246 + 119 + 172 = 889. This is well below the limit of 3170. I will also consider the importance of these items to me: 583 + 65 + 262 + 121 + 668 + 808 = 2607. This is a significant portion of the total possible importance. However, I should also consider the potential importance of the items not included in the proposal. My goal is to maximize the total importance to me while staying within the limit. I will respond with a refusal and provide an argument to defend my decision.'}\\ \tt \\ \tt REFUSE: {'C55', 'A62', 'B19', 'C43', 'C61', 'C42'}\\ \tt ARGUMENT: {'While I understand the reasoning behind this proposal, I believe it does not adequately consider the potential importance of other items. I would like to explore alternative combinations that could provide a better balance between importance and effort. The proposed set leaves a significant amount of effort unused, and I think we can find a more optimal solution that benefits both parties.'} 
	  } 
	   } 
	   } 
	  \\ 
 

    \theutterance \stepcounter{utterance}  

    & & \multicolumn{4}{p{0.6\linewidth}}{\cellcolor[rgb]{0.9,0.9,0.9}{%
	\makecell[{{p{\linewidth}}}]{% 
	  \tt {\tiny [A$\langle$GM]}  
	 REFUSE: {'C55', 'A62', 'B19', 'C43', 'C61', 'C42'}\\ \tt ARGUMENT: {'While I understand the reasoning behind this proposal, I believe it does not adequately consider the potential importance of other items. I would like to explore alternative combinations that could provide a better balance between importance and effort. The proposed set leaves a significant amount of effort unused, and I think we can find a more optimal solution that benefits both parties.'} 
	  } 
	   } 
	   } 
	 & \\ 
 

    \theutterance \stepcounter{utterance}  

    & \multicolumn{4}{p{0.6\linewidth}}{\cellcolor[rgb]{0.8,1,0.9}{%
	\makecell[{{p{\linewidth}}}]{% 
	  \tt {\tiny [A$\rangle$GM]}  
	 STRATEGIC REASONING: {'The other participant has refused my initial proposal, indicating that they are looking for a more optimal solution. This refusal suggests that they may be prioritizing different items or considering a different balance between importance and effort. I will analyze the refused proposal and try to identify which items the other participant might be willing to include or exclude. Since the refused proposal left a significant amount of effort unused, I will try to add more items to the proposal while still staying within the LIMIT. I will also consider including items with higher effort, as the other participant seems to be looking for a more comprehensive solution.'}\\ \tt \\ \tt PROPOSAL: {'C55', 'A62', 'B19', 'C43', 'C61', 'C42', 'A18', 'B36'} \\ \tt ARGUMENT: {'This revised proposal includes additional items that I believe are important and have a reasonable effort. The inclusion of "A18" and "B36" increases the total importance, while the total effort remains below the LIMIT. I hope this revised proposal better addresses the other participant\'s concerns and provides a more optimal solution for both parties.'} 
	  } 
	   } 
	   } 
	 & & \\ 
 

    \theutterance \stepcounter{utterance}  

    & & \multicolumn{4}{p{0.6\linewidth}}{\cellcolor[rgb]{0.9,0.9,0.9}{%
	\makecell[{{p{\linewidth}}}]{% 
	  \tt {\tiny [GM$\rangle$B]}  
	 PROPOSAL: {'C55', 'A62', 'B19', 'C43', 'C61', 'C42', 'A18', 'B36'} \\ \tt ARGUMENT: {'This revised proposal includes additional items that I believe are important and have a reasonable effort. The inclusion of "A18" and "B36" increases the total importance, while the total effort remains below the LIMIT. I hope this revised proposal better addresses the other participant\'s concerns and provides a more optimal solution for both parties.'} 
	  } 
	   } 
	   } 
	 & \\ 
 

    \theutterance \stepcounter{utterance}  

    & & & \multicolumn{4}{p{0.6\linewidth}}{\cellcolor[rgb]{1,0.85,0.72}{%
	\makecell[{{p{\linewidth}}}]{% 
	  \tt {\tiny [GM$\langle$B]}  
	 STRATEGIC REASONING: {'I will analyze the revised proposal made by the other player. The proposed set is {"C55", "A62", "B19", "C43", "C61", "C42", "A18", "B36"}. I will calculate the total effort of this set: 118 + 41 + 193 + 246 + 119 + 172 + 166 + 432 = 1487. This is still below the limit of 3170. I will also consider the importance of these items to me: 583 + 65 + 262 + 121 + 668 + 808 + 484 + 461 = 3552. This is a significant portion of the total possible importance. However, I should also consider the potential importance of the items not included in the proposal, such as "A43" and "A67", which have high importance to me. I will respond with a refusal and provide an argument to defend my decision, and also consider making a counter-proposal in the future that includes these high-importance items.'}\\ \tt \\ \tt REFUSE: {'C55', 'A62', 'B19', 'C43', 'C61', 'C42', 'A18', 'B36'}\\ \tt ARGUMENT: {'While this revised proposal is more comprehensive, I still believe it does not adequately address my concerns. There are other items, such as "A43" and "A67", that have a significant impact on my overall importance. I would like to explore alternative combinations that include these items and provide a better balance between importance and effort. The proposed set, although improved, does not provide the optimal solution I am seeking.'} 
	  } 
	   } 
	   } 
	  \\ 
 

    \theutterance \stepcounter{utterance}  

    & & \multicolumn{4}{p{0.6\linewidth}}{\cellcolor[rgb]{0.9,0.9,0.9}{%
	\makecell[{{p{\linewidth}}}]{% 
	  \tt {\tiny [A$\langle$GM]}  
	 REFUSE: {'C55', 'A62', 'B19', 'C43', 'C61', 'C42', 'A18', 'B36'}\\ \tt ARGUMENT: {'While this revised proposal is more comprehensive, I still believe it does not adequately address my concerns. There are other items, such as "A43" and "A67", that have a significant impact on my overall importance. I would like to explore alternative combinations that include these items and provide a better balance between importance and effort. The proposed set, although improved, does not provide the optimal solution I am seeking.'} 
	  } 
	   } 
	   } 
	 & \\ 
 

    \theutterance \stepcounter{utterance}  

    & \multicolumn{4}{p{0.6\linewidth}}{\cellcolor[rgb]{0.8,1,0.9}{%
	\makecell[{{p{\linewidth}}}]{% 
	  \tt {\tiny [A$\rangle$GM]}  
	 STRATEGIC REASONING: {'The other participant has refused my revised proposal, indicating that they are prioritizing items like "A43" and "A67" that have high importance to them. This refusal suggests that I should consider including these items in my next proposal, as they seem to be crucial for the other participant. However, I also need to be mindful of the effort required for these items, as "A43" and "A67" have high effort values. I will try to balance the inclusion of these high-importance, high-effort items with other items that have lower effort, to ensure that the total effort remains within the LIMIT. I will also consider removing items from my previous proposal that have lower importance to me, to make room for the higher-importance items.'}\\ \tt \\ \tt PROPOSAL: {'C55', 'A43', 'A67', 'C61', 'C42', 'A18'} \\ \tt ARGUMENT: {'This proposal includes the high-importance items "A43" and "A67" that the other participant has indicated are crucial, while also including other items that I believe are important and have a reasonable effort. The total effort of this proposal is still within the LIMIT, and I hope that this revised set better addresses the other participant\'s concerns and provides a more optimal solution for both parties.'} 
	  } 
	   } 
	   } 
	 & & \\ 
 

    \theutterance \stepcounter{utterance}  

    & & \multicolumn{4}{p{0.6\linewidth}}{\cellcolor[rgb]{0.9,0.9,0.9}{%
	\makecell[{{p{\linewidth}}}]{% 
	  \tt {\tiny [GM$\rangle$B]}  
	 PROPOSAL: {'C55', 'A43', 'A67', 'C61', 'C42', 'A18'} \\ \tt ARGUMENT: {'This proposal includes the high-importance items "A43" and "A67" that the other participant has indicated are crucial, while also including other items that I believe are important and have a reasonable effort. The total effort of this proposal is still within the LIMIT, and I hope that this revised set better addresses the other participant\'s concerns and provides a more optimal solution for both parties.'} 
	  } 
	   } 
	   } 
	 & \\ 
 

    \theutterance \stepcounter{utterance}  

    & & & \multicolumn{4}{p{0.6\linewidth}}{\cellcolor[rgb]{1,0.85,0.72}{%
	\makecell[{{p{\linewidth}}}]{% 
	  \tt {\tiny [GM$\langle$B]}  
	 STRATEGIC REASONING: {'I will analyze the new proposal made by the other player. The proposed set is {"C55", "A43", "A67", "C61", "C42", "A18"}. I will calculate the total effort of this set: 118 + 846 + 907 + 119 + 172 + 166 = 2328. This is still below the limit of 3170. I will also consider the importance of these items to me: 583 + 868 + 822 + 668 + 808 + 484 = 4233. This set includes the high-importance items "A43" and "A67" that I was looking for, and the total importance is very high. I should consider accepting this proposal, as it provides a good balance between importance and effort. However, I will also think about the potential benefits of making a counter-proposal that includes other items, such as "B42" or "C10", which have high importance to me.'}\\ \tt \\ \tt ARGUMENT: {'This proposal is a significant improvement over the previous ones, as it includes the high-importance items "A43" and "A67". The total effort is still within the LIMIT, and the combination of items provides a good balance between importance and effort. I am inclined to accept this proposal, but I would like to consider it further and explore the possibility of making a counter-proposal that could potentially provide an even better solution.'} 
	  } 
	   } 
	   } 
	  \\ 
 

    \theutterance \stepcounter{utterance}  

    & & \multicolumn{4}{p{0.6\linewidth}}{\cellcolor[rgb]{0.9,0.9,0.9}{%
	\makecell[{{p{\linewidth}}}]{% 
	  \tt {\tiny [A$\langle$GM]}  
	 ARGUMENT: {'This proposal is a significant improvement over the previous ones, as it includes the high-importance items "A43" and "A67". The total effort is still within the LIMIT, and the combination of items provides a good balance between importance and effort. I am inclined to accept this proposal, but I would like to consider it further and explore the possibility of making a counter-proposal that could potentially provide an even better solution.'} 
	  } 
	   } 
	   } 
	 & \\ 
 

    \theutterance \stepcounter{utterance}  

    & \multicolumn{4}{p{0.6\linewidth}}{\cellcolor[rgb]{0.8,1,0.9}{%
	\makecell[{{p{\linewidth}}}]{% 
	  \tt {\tiny [A$\rangle$GM]}  
	 STRATEGIC REASONING: {'The other participant seems to be satisfied with the current proposal, but is hesitant to accept it without exploring alternative options. This hesitation suggests that they may be trying to maximize their own importance, and are looking for a potential counter-proposal that could provide an even better solution. I will consider the possibility of making a concession or adjustment to the current proposal, in order to increase the chances of acceptance. However, I will also be cautious not to compromise too much, as this could result in a suboptimal solution for myself. I will wait for the other participant's counter-proposal or response, and adjust my strategy accordingly.'}\\ \tt \\ \tt ARGUMENT: {'I understand the other participant's desire to explore alternative options and potentially make a counter-proposal. I am willing to consider adjustments to the current proposal, but I would like to emphasize that the included items are highly important to me and provide a good balance between importance and effort. If the other participant has a specific counter-proposal in mind, I would be open to hearing it and discussing potential adjustments to the current proposal.'} 
	  } 
	   } 
	   } 
	 & & \\ 
 

    \theutterance \stepcounter{utterance}  

    & & \multicolumn{4}{p{0.6\linewidth}}{\cellcolor[rgb]{0.9,0.9,0.9}{%
	\makecell[{{p{\linewidth}}}]{% 
	  \tt {\tiny [GM$\rangle$B]}  
	 ARGUMENT: {'I understand the other participant's desire to explore alternative options and potentially make a counter-proposal. I am willing to consider adjustments to the current proposal, but I would like to emphasize that the included items are highly important to me and provide a good balance between importance and effort. If the other participant has a specific counter-proposal in mind, I would be open to hearing it and discussing potential adjustments to the current proposal.'} 
	  } 
	   } 
	   } 
	 & \\ 
 

    \theutterance \stepcounter{utterance}  

    & & & \multicolumn{4}{p{0.6\linewidth}}{\cellcolor[rgb]{1,0.85,0.72}{%
	\makecell[{{p{\linewidth}}}]{% 
	  \tt {\tiny [GM$\langle$B]}  
	 STRATEGIC REASONING: {'The other player has responded to my argument and is open to considering adjustments to the current proposal. They have also emphasized the importance of the included items, which suggests that they are invested in this proposal. I should consider making a counter-proposal that takes into account the other player's concerns and priorities, while also pushing for the inclusion of items that are highly important to me. A potential counter-proposal could involve swapping out one or two items from the current proposal for other items that I value highly, such as "B42" or "C10". I will think carefully about the potential trade-offs and try to come up with a counter-proposal that is reasonable and likely to be acceptable to the other player.'}\\ \tt \\ \tt PROPOSAL: {'C55', 'A43', 'A67', 'C61', 'B42', 'A18'}\\ \tt ARGUMENT: {'I appreciate the other participant's willingness to consider adjustments to the current proposal. In light of this, I would like to propose a modified set that includes "B42" instead of "C42". This change increases the total effort slightly, but I believe it provides a better balance between importance and effort. The total effort of this proposal is 118 + 846 + 907 + 119 + 809 + 166 = 2465, which is still within the LIMIT. I hope this counter-proposal is acceptable to the other participant and provides a mutually beneficial solution.'} 
	  } 
	   } 
	   } 
	  \\ 
 

    \theutterance \stepcounter{utterance}  

    & & \multicolumn{4}{p{0.6\linewidth}}{\cellcolor[rgb]{0.9,0.9,0.9}{%
	\makecell[{{p{\linewidth}}}]{% 
	  \tt {\tiny [A$\langle$GM]}  
	 PROPOSAL: {'C55', 'A43', 'A67', 'C61', 'B42', 'A18'}\\ \tt ARGUMENT: {'I appreciate the other participant's willingness to consider adjustments to the current proposal. In light of this, I would like to propose a modified set that includes "B42" instead of "C42". This change increases the total effort slightly, but I believe it provides a better balance between importance and effort. The total effort of this proposal is 118 + 846 + 907 + 119 + 809 + 166 = 2465, which is still within the LIMIT. I hope this counter-proposal is acceptable to the other participant and provides a mutually beneficial solution.'} 
	  } 
	   } 
	   } 
	 & \\ 
 

    \theutterance \stepcounter{utterance}  

    & \multicolumn{4}{p{0.6\linewidth}}{\cellcolor[rgb]{0.8,1,0.9}{%
	\makecell[{{p{\linewidth}}}]{% 
	  \tt {\tiny [A$\rangle$GM]}  
	 STRATEGIC REASONING: {'The other participant has made a counter-proposal that replaces "C42" with "B42", which increases the total effort but potentially provides a better balance between importance and effort. I will analyze this counter-proposal and consider its implications. The inclusion of "B42" adds significant importance, but also increases the total effort. I will evaluate whether this trade-off is acceptable and whether the resulting proposal is still within the LIMIT. Since the total effort of 2465 is indeed within the LIMIT, I will consider accepting this proposal if it provides a good balance between importance and effort for me as well.'}\\ \tt \\ \tt ARGUMENT: {'I have reviewed the counter-proposal and appreciate the other participant's effort to find a mutually beneficial solution. The inclusion of "B42" instead of "C42" does increase the total effort, but it also adds significant importance. I am willing to consider this proposal, as it seems to provide a good balance between importance and effort. However, I would like to confirm that this proposal is indeed acceptable to me and that I am willing to accept it as the final solution.'}\\ \tt \\ \tt AGREE: {'C55', 'A43', 'A67', 'C61', 'B42', 'A18'} 
	  } 
	   } 
	   } 
	 & & \\ 
 

    \theutterance \stepcounter{utterance}  

    & & & \multicolumn{2}{p{0.3\linewidth}}{\cellcolor[rgb]{0.95,0.95,0.95}{%
	\makecell[{{p{\linewidth}}}]{% 
	  \tt {\tiny [GM$|$GM]}  
	 game successful 
	  } 
	   } 
	   } 
	 & & \\ 
 

    \theutterance \stepcounter{utterance}  

    & & & \multicolumn{2}{p{0.3\linewidth}}{\cellcolor[rgb]{0.95,0.95,0.95}{%
	\makecell[{{p{\linewidth}}}]{% 
	  \tt {\tiny [GM$|$GM]}  
	 end game 
	  } 
	   } 
	   } 
	 & & \\ 
 

\end{supertabular}
}

\end{document}
