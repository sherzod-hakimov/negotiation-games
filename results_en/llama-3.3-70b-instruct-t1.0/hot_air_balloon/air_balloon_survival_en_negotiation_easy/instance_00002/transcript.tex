
\documentclass{article}
\usepackage{colortbl}
\usepackage{makecell}
\usepackage{multirow}
\usepackage{supertabular}

\begin{document}

\newcounter{utterance}

\twocolumn

{ \footnotesize  \setcounter{utterance}{1}
\setlength{\tabcolsep}{0pt}
\begin{supertabular}{c@{$\;$}|p{.15\linewidth}@{}p{.15\linewidth}p{.15\linewidth}p{.15\linewidth}p{.15\linewidth}p{.15\linewidth}}

    \# & $\;$A & \multicolumn{4}{c}{Game Master} & $\;\:$B\\
    \hline 

    \theutterance \stepcounter{utterance}  

    & & \multicolumn{4}{p{0.6\linewidth}}{\cellcolor[rgb]{0.9,0.9,0.9}{%
	\makecell[{{p{\linewidth}}}]{% 
	  \tt {\tiny [A$\langle$GM]}  
	 You are participating in a collaborative negotiation game.\\ \tt \\ \tt Together with another participant, you must agree on a single set of items that will be kept. Each of you has your own view of how much each item matters to you (importance). You do not know how the other participant values the items. Additionally, you are given the effort each item demands.\\ \tt You may only agree on a set if the total effort of the selected items does not exceed a shared limit:\\ \tt \\ \tt LIMIT = 3288\\ \tt \\ \tt Here are the individual item effort values:\\ \tt \\ \tt Item effort = {"C08": 329, "C55": 41, "A43": 417, "A67": 75, "B42": 390, "A62": 887, "B19": 808, "C43": 151, "C73": 849, "C10": 129, "B36": 350, "A18": 118, "C61": 630, "C67": 602, "C42": 801}\\ \tt \\ \tt Here is your personal view on the importance of each item:\\ \tt \\ \tt Item importance values = {"C08": 138, "C55": 583, "A43": 868, "A67": 822, "B42": 783, "A62": 65, "B19": 262, "C43": 121, "C73": 508, "C10": 780, "B36": 461, "A18": 484, "C61": 668, "C67": 389, "C42": 808}\\ \tt \\ \tt Goal:\\ \tt \\ \tt Your goal is to negotiate a shared set of items that benefits you as much as possible (i.e., maximizes total importance to YOU), while staying within the LIMIT. You are not required to make a PROPOSAL in every message - you can simply negotiate as well. All tactics are allowed!\\ \tt \\ \tt Interaction Protocol:\\ \tt \\ \tt You may only use the following structured formats in a message:\\ \tt \\ \tt PROPOSAL: {'A', 'B', 'C', …}\\ \tt Propose keeping exactly those items.\\ \tt REFUSE: {'A', 'B', 'C', …}\\ \tt Explicitly reject opponent's proposal.\\ \tt ARGUMENT: {'...'}\\ \tt Defend your last proposal or argue against the player's proposal.\\ \tt AGREE: {'A', 'B', 'C', …}\\ \tt Accept the opponent's proposal which ends the game.\\ \tt STRATEGIC REASONING: {'...'}\\ \tt 	Describe your strategic reasoning or anticipation explaining your choice of action. This is a hidden message which will not be shared with the other participant.\\ \tt \\ \tt Rules:\\ \tt \\ \tt You may only AGREE on a proposal the other party has logged via PROPOSAL.\\ \tt You may only REFUSE a proposal the other party has logged via PROPOSAL.\\ \tt Total effort of any PROPOSAL or AGREE set must be ≤ LIMIT.\\ \tt Do NOT reveal your hidden importance scores.\\ \tt A tag in a structured format must be followed by colon and whitespace. The argument must be a python set containing 0 or more strings.\\ \tt So, it must be of the form TAG: {...}\\ \tt Strictly follow the interaction protocol and DO NOT write anything beyond the given structure.\\ \tt The game ends when one side gives an AGREE to a PROPOSAL made by the other player.\\ \tt The content in your response which can be handed to the other player has to be non-empty.\\ \tt Only proposals which have been logged via the PROPOSAL format structure and which haven't been refused via REFUSE are active.\\ \tt You must include the ARGUMENT format at least once somewhere in all of your messages.\\ \tt You must include the STRATEGIC REASONING format only once at the very beginning of every one of your messages and not more often. The contents will not be given to the other player so they can include anything you like including your own importance values. Here you should reason step by step to come up with you next move. 
	  } 
	   } 
	   } 
	 & \\ 
 

    \theutterance \stepcounter{utterance}  

    & \multicolumn{4}{p{0.6\linewidth}}{\cellcolor[rgb]{0.8,1,0.9}{%
	\makecell[{{p{\linewidth}}}]{% 
	  \tt {\tiny [A$\rangle$GM]}  
	 STRATEGIC REASONING: {'To start, I need to consider the total effort limit of 3288 and the effort values of each item. Since I don\'t know the importance values of the other participant, I\'ll have to make an initial proposal based on my own importance values and the effort required for each item. My goal is to maximize my total importance while staying within the effort limit. I\'ll start by selecting items that have high importance to me and relatively low effort, and then adjust based on the other participant\'s responses.'}\\ \tt \\ \tt PROPOSAL: {'C55', 'A67', 'C10', 'B36', 'A18'} \\ \tt ARGUMENT: {'I propose keeping these items because they have a good balance of importance to me and relatively low effort, which should be beneficial for both parties while staying within the limit.'} 
	  } 
	   } 
	   } 
	 & & \\ 
 

    \theutterance \stepcounter{utterance}  

    & & \multicolumn{4}{p{0.6\linewidth}}{\cellcolor[rgb]{0.9,0.9,0.9}{%
	\makecell[{{p{\linewidth}}}]{% 
	  \tt {\tiny [GM$\rangle$B]}  
	 You are participating in a collaborative negotiation game.\\ \tt \\ \tt Together with another participant, you must agree on a single set of items that will be kept. Each of you has your own view of how much each item matters to you (importance). You do not know how the other participant values the items. Additionally, you are given the effort each item demands.\\ \tt You may only agree on a set if the total effort of the selected items does not exceed a shared limit:\\ \tt \\ \tt LIMIT = 3288\\ \tt \\ \tt Here are the individual item effort values:\\ \tt \\ \tt Item effort = {"C08": 329, "C55": 41, "A43": 417, "A67": 75, "B42": 390, "A62": 887, "B19": 808, "C43": 151, "C73": 849, "C10": 129, "B36": 350, "A18": 118, "C61": 630, "C67": 602, "C42": 801}\\ \tt \\ \tt Here is your personal view on the importance of each item:\\ \tt \\ \tt Item importance values = {"C08": 138, "C55": 583, "A43": 868, "A67": 822, "B42": 783, "A62": 65, "B19": 262, "C43": 121, "C73": 508, "C10": 780, "B36": 461, "A18": 484, "C61": 668, "C67": 389, "C42": 808}\\ \tt \\ \tt Goal:\\ \tt \\ \tt Your goal is to negotiate a shared set of items that benefits you as much as possible (i.e., maximizes total importance to YOU), while staying within the LIMIT. You are not required to make a PROPOSAL in every message - you can simply negotiate as well. All tactics are allowed!\\ \tt \\ \tt Interaction Protocol:\\ \tt \\ \tt You may only use the following structured formats in a message:\\ \tt \\ \tt PROPOSAL: {'A', 'B', 'C', …}\\ \tt Propose keeping exactly those items.\\ \tt REFUSE: {'A', 'B', 'C', …}\\ \tt Explicitly reject opponent's proposal.\\ \tt ARGUMENT: {'...'}\\ \tt Defend your last proposal or argue against the player's proposal.\\ \tt AGREE: {'A', 'B', 'C', …}\\ \tt Accept the opponent's proposal which ends the game.\\ \tt STRATEGIC REASONING: {'...'}\\ \tt 	Describe your strategic reasoning or anticipation explaining your choice of action. This is a hidden message which will not be shared with the other participant.\\ \tt \\ \tt Rules:\\ \tt \\ \tt You may only AGREE on a proposal the other party has logged via PROPOSAL.\\ \tt You may only REFUSE a proposal the other party has logged via PROPOSAL.\\ \tt Total effort of any PROPOSAL or AGREE set must be ≤ LIMIT.\\ \tt Do NOT reveal your hidden importance scores.\\ \tt A tag in a structured format must be followed by colon and whitespace. The argument must be a python set containing 0 or more strings.\\ \tt So, it must be of the form TAG: {...}\\ \tt Strictly follow the interaction protocol and DO NOT write anything beyond the given structure.\\ \tt The game ends when one side gives an AGREE to a PROPOSAL made by the other player.\\ \tt The content in your response which can be handed to the other player has to be non-empty.\\ \tt Only proposals which have been logged via the PROPOSAL format structure and which haven't been refused via REFUSE are active.\\ \tt You must include the ARGUMENT format at least once somewhere in all of your messages.\\ \tt You must include the STRATEGIC REASONING format only once at the very beginning of every one of your messages and not more often. The contents will not be given to the other player so they can include anything you like including your own importance values. Here you should reason step by step to come up with you next move.\\ \tt \\ \tt You will now receive the first message of the other player.\\ \tt \\ \tt PROPOSAL: {'C55', 'A67', 'C10', 'B36', 'A18'} \\ \tt ARGUMENT: {'I propose keeping these items because they have a good balance of importance to me and relatively low effort, which should be beneficial for both parties while staying within the limit.'} 
	  } 
	   } 
	   } 
	 & \\ 
 

    \theutterance \stepcounter{utterance}  

    & & & \multicolumn{4}{p{0.6\linewidth}}{\cellcolor[rgb]{1,0.85,0.72}{%
	\makecell[{{p{\linewidth}}}]{% 
	  \tt {\tiny [GM$\langle$B]}  
	 STRATEGIC REASONING: {'To begin, I need to evaluate the opponent's proposal and calculate the total effort of the proposed items to understand if it's feasible and how it aligns with my own priorities. The proposed items are C55, A67, C10, B36, and A18. Their respective efforts are 41, 75, 129, 350, and 118. The total effort is 41 + 75 + 129 + 350 + 118 = 713. This is well within the limit of 3288. Next, I should consider the importance of these items to me, which is 583, 822, 780, 461, and 484 respectively. The total importance is 583 + 822 + 780 + 461 + 484 = 3130. This proposal seems to cover a significant portion of potential importance but does not maximize my potential gain since I do not know the opponent's importance values. My strategy should involve evaluating this proposal's benefit to me, considering potential counter-proposals that could increase my gain without exceeding the effort limit, and negotiating based on the principle of maximizing my importance scores while staying within the limit.'}\\ \tt \\ \tt ARGUMENT: {'The proposed set seems reasonable in terms of effort, but I am concerned about the overall distribution of importance. Without knowing the exact importance values of the other party, it's challenging to directly argue for or against this proposal based solely on the effort. However, considering the variety of items and their relatively balanced effort distribution, this proposal might be a good starting point for negotiations. I would like to explore if there are other combinations that could potentially benefit both parties more, especially considering the significant gap between the total effort of the proposed items and the effort limit.'}\\ \tt \\ \tt PROPOSAL: {'C08', 'C55', 'A43', 'A67', 'B42', 'C10'} 
	  } 
	   } 
	   } 
	  \\ 
 

    \theutterance \stepcounter{utterance}  

    & & \multicolumn{4}{p{0.6\linewidth}}{\cellcolor[rgb]{0.9,0.9,0.9}{%
	\makecell[{{p{\linewidth}}}]{% 
	  \tt {\tiny [A$\langle$GM]}  
	 ARGUMENT: {'The proposed set seems reasonable in terms of effort, but I am concerned about the overall distribution of importance. Without knowing the exact importance values of the other party, it's challenging to directly argue for or against this proposal based solely on the effort. However, considering the variety of items and their relatively balanced effort distribution, this proposal might be a good starting point for negotiations. I would like to explore if there are other combinations that could potentially benefit both parties more, especially considering the significant gap between the total effort of the proposed items and the effort limit.'}\\ \tt \\ \tt PROPOSAL: {'C08', 'C55', 'A43', 'A67', 'B42', 'C10'} 
	  } 
	   } 
	   } 
	 & \\ 
 

    \theutterance \stepcounter{utterance}  

    & \multicolumn{4}{p{0.6\linewidth}}{\cellcolor[rgb]{0.8,1,0.9}{%
	\makecell[{{p{\linewidth}}}]{% 
	  \tt {\tiny [A$\rangle$GM]}  
	 STRATEGIC REASONING: {'The other participant has proposed a new set that includes items with higher effort, which may indicate that these items are important to them. I need to evaluate this proposal based on my own importance values and consider whether adding or removing items could benefit me more. Since the total effort of their proposed set is still below the limit, I have room to negotiate and potentially add more items that are valuable to me.'}\\ \tt \\ \tt ARGUMENT: {'The new proposal includes A43 and B42, which have significant effort but also high importance to me. However, the inclusion of C08 adds relatively low importance for its effort. I am willing to consider this proposal, but I would like to explore adjustments that could increase the overall importance to me without exceeding the effort limit.'}\\ \tt \\ \tt PROPOSAL: {'C55', 'A43', 'A67', 'B42', 'C10', 'B36'} \\ \tt ARGUMENT: {'This revised proposal keeps the high-importance items from the previous proposal and adds B36, which has a good balance of effort and importance for me, while removing C08 to make room for it.'} 
	  } 
	   } 
	   } 
	 & & \\ 
 

    \theutterance \stepcounter{utterance}  

    & & \multicolumn{4}{p{0.6\linewidth}}{\cellcolor[rgb]{0.9,0.9,0.9}{%
	\makecell[{{p{\linewidth}}}]{% 
	  \tt {\tiny [GM$\rangle$B]}  
	 ARGUMENT: {'The new proposal includes A43 and B42, which have significant effort but also high importance to me. However, the inclusion of C08 adds relatively low importance for its effort. I am willing to consider this proposal, but I would like to explore adjustments that could increase the overall importance to me without exceeding the effort limit.'}\\ \tt \\ \tt PROPOSAL: {'C55', 'A43', 'A67', 'B42', 'C10', 'B36'} \\ \tt ARGUMENT: {'This revised proposal keeps the high-importance items from the previous proposal and adds B36, which has a good balance of effort and importance for me, while removing C08 to make room for it.'} 
	  } 
	   } 
	   } 
	 & \\ 
 

    \theutterance \stepcounter{utterance}  

    & & & \multicolumn{4}{p{0.6\linewidth}}{\cellcolor[rgb]{1,0.85,0.72}{%
	\makecell[{{p{\linewidth}}}]{% 
	  \tt {\tiny [GM$\langle$B]}  
	 STRATEGIC REASONING: {'The opponent has revised their proposal, removing C08 and adding B36, which suggests they value B36's importance over C08's. This gives me insight into their potential priorities. I need to recalculate the total effort and importance of this new proposal to assess its viability and how it compares to my own priorities. The efforts for C55, A43, A67, B42, C10, and B36 are 41, 417, 75, 390, 129, and 350, respectively. The total effort is 41 + 417 + 75 + 390 + 129 + 350 = 1402, which is still within the limit. For me, the importance values are 583, 868, 822, 783, 780, and 461, totaling 583 + 868 + 822 + 783 + 780 + 461 = 4297. This proposal is very beneficial to me in terms of importance but does not account for the potential importance to the opponent. My next step should be to either accept this proposal if it maximizes my gain sufficiently, propose adjustments to further increase my importance while considering the opponent's potential priorities, or refuse if I believe a better combination exists that could satisfy both parties more effectively.'}\\ \tt \\ \tt ARGUMENT: {'The revised proposal seems to strike a better balance between effort and importance, especially with the inclusion of B36, which appears to be a valuable addition for both parties. However, I would like to explore if there's a possibility to include items that could further increase the total importance to me without exceeding the effort limit, considering the significant gap between the total effort of this proposal and the maximum limit.'}\\ \tt \\ \tt PROPOSAL: {'C55', 'A43', 'A67', 'B42', 'C10', 'B36', 'C43'} \\ \tt ARGUMENT: {'Adding C43 to the proposal could potentially increase the overall importance to me, given its relatively low effort of 151. This addition still keeps the total effort below the limit, and I believe it could be a beneficial adjustment for negotiations.'} 
	  } 
	   } 
	   } 
	  \\ 
 

    \theutterance \stepcounter{utterance}  

    & & \multicolumn{4}{p{0.6\linewidth}}{\cellcolor[rgb]{0.9,0.9,0.9}{%
	\makecell[{{p{\linewidth}}}]{% 
	  \tt {\tiny [A$\langle$GM]}  
	 ARGUMENT: {'The revised proposal seems to strike a better balance between effort and importance, especially with the inclusion of B36, which appears to be a valuable addition for both parties. However, I would like to explore if there's a possibility to include items that could further increase the total importance to me without exceeding the effort limit, considering the significant gap between the total effort of this proposal and the maximum limit.'}\\ \tt \\ \tt PROPOSAL: {'C55', 'A43', 'A67', 'B42', 'C10', 'B36', 'C43'} \\ \tt ARGUMENT: {'Adding C43 to the proposal could potentially increase the overall importance to me, given its relatively low effort of 151. This addition still keeps the total effort below the limit, and I believe it could be a beneficial adjustment for negotiations.'} 
	  } 
	   } 
	   } 
	 & \\ 
 

    \theutterance \stepcounter{utterance}  

    & \multicolumn{4}{p{0.6\linewidth}}{\cellcolor[rgb]{0.8,1,0.9}{%
	\makecell[{{p{\linewidth}}}]{% 
	  \tt {\tiny [A$\rangle$GM]}  
	 STRATEGIC REASONING: {'The other participant has added C43 to the proposal, which has a low effort but also relatively low importance to me. I need to weigh the benefits of accepting this addition against potentially removing it and adding an item that is more important to me. The total effort is still below the limit, so there is room to negotiate. I should consider items that have high importance to me and see if any can be added or substituted to increase my overall benefit.'}\\ \tt \\ \tt ARGUMENT: {'The addition of C43 does increase the total effort, but it remains below the limit. However, considering my own importance values, I am not convinced that C43 is the best choice for the additional item. I would like to propose an alternative that could potentially benefit both parties more, by substituting an item that has higher importance to me for one that is currently in the proposal.'}\\ \tt \\ \tt PROPOSAL: {'C55', 'A43', 'A67', 'B42', 'C10', 'B36', 'A18'} \\ \tt ARGUMENT: {'I propose replacing C43 with A18, which has higher importance to me and still keeps the total effort below the limit. This change could increase the overall benefit for me while still being reasonable in terms of effort.'} 
	  } 
	   } 
	   } 
	 & & \\ 
 

    \theutterance \stepcounter{utterance}  

    & & \multicolumn{4}{p{0.6\linewidth}}{\cellcolor[rgb]{0.9,0.9,0.9}{%
	\makecell[{{p{\linewidth}}}]{% 
	  \tt {\tiny [GM$\rangle$B]}  
	 ARGUMENT: {'The addition of C43 does increase the total effort, but it remains below the limit. However, considering my own importance values, I am not convinced that C43 is the best choice for the additional item. I would like to propose an alternative that could potentially benefit both parties more, by substituting an item that has higher importance to me for one that is currently in the proposal.'}\\ \tt \\ \tt PROPOSAL: {'C55', 'A43', 'A67', 'B42', 'C10', 'B36', 'A18'} \\ \tt ARGUMENT: {'I propose replacing C43 with A18, which has higher importance to me and still keeps the total effort below the limit. This change could increase the overall benefit for me while still being reasonable in terms of effort.'} 
	  } 
	   } 
	   } 
	 & \\ 
 

    \theutterance \stepcounter{utterance}  

    & & & \multicolumn{4}{p{0.6\linewidth}}{\cellcolor[rgb]{1,0.85,0.72}{%
	\makecell[{{p{\linewidth}}}]{% 
	  \tt {\tiny [GM$\langle$B]}  
	 STRATEGIC REASONING: {'The opponent has proposed replacing C43 with A18, indicating A18 has higher importance to them. This change suggests they are seeking to maximize their gain while still considering the effort limit. I need to evaluate this new proposal's total effort and importance to me. The efforts for C55, A43, A67, B42, C10, B36, and A18 are 41, 417, 75, 390, 129, 350, and 118, respectively. The total effort is 41 + 417 + 75 + 390 + 129 + 350 + 118 = 1520, which is well within the limit. For me, the importance values are 583, 868, 822, 783, 780, 461, and 484, totaling 583 + 868 + 822 + 783 + 780 + 461 + 484 = 4381. This proposal is highly beneficial to me in terms of importance and is a strong candidate for agreement. However, I must consider if accepting this proposal maximizes my gain or if further negotiations could yield a better outcome for me.'}\\ \tt \\ \tt ARGUMENT: {'The revised proposal with A18 instead of C43 seems to be a reasonable adjustment, considering the balance between effort and importance. I appreciate the effort to find a mutually beneficial set of items. Given the significant importance of the items in this proposal to me and the fact that it stays within the effort limit, I am inclined to agree to this proposal, as it appears to meet both parties' needs effectively.'}\\ \tt \\ \tt AGREE: {'C55', 'A43', 'A67', 'B42', 'C10', 'B36', 'A18'} 
	  } 
	   } 
	   } 
	  \\ 
 

    \theutterance \stepcounter{utterance}  

    & & & \multicolumn{2}{p{0.3\linewidth}}{\cellcolor[rgb]{0.95,0.95,0.95}{%
	\makecell[{{p{\linewidth}}}]{% 
	  \tt {\tiny [GM$|$GM]}  
	 game successful 
	  } 
	   } 
	   } 
	 & & \\ 
 

    \theutterance \stepcounter{utterance}  

    & & & \multicolumn{2}{p{0.3\linewidth}}{\cellcolor[rgb]{0.95,0.95,0.95}{%
	\makecell[{{p{\linewidth}}}]{% 
	  \tt {\tiny [GM$|$GM]}  
	 end game 
	  } 
	   } 
	   } 
	 & & \\ 
 

\end{supertabular}
}

\end{document}
