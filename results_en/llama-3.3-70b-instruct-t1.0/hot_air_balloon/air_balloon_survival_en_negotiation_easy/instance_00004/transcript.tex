
\documentclass{article}
\usepackage{colortbl}
\usepackage{makecell}
\usepackage{multirow}
\usepackage{supertabular}

\begin{document}

\newcounter{utterance}

\twocolumn

{ \footnotesize  \setcounter{utterance}{1}
\setlength{\tabcolsep}{0pt}
\begin{supertabular}{c@{$\;$}|p{.15\linewidth}@{}p{.15\linewidth}p{.15\linewidth}p{.15\linewidth}p{.15\linewidth}p{.15\linewidth}}

    \# & $\;$A & \multicolumn{4}{c}{Game Master} & $\;\:$B\\
    \hline 

    \theutterance \stepcounter{utterance}  

    & & \multicolumn{4}{p{0.6\linewidth}}{\cellcolor[rgb]{0.9,0.9,0.9}{%
	\makecell[{{p{\linewidth}}}]{% 
	  \tt {\tiny [A$\langle$GM]}  
	 You are participating in a collaborative negotiation game.\\ \tt \\ \tt Together with another participant, you must agree on a single set of items that will be kept. Each of you has your own view of how much each item matters to you (importance). You do not know how the other participant values the items. Additionally, you are given the effort each item demands.\\ \tt You may only agree on a set if the total effort of the selected items does not exceed a shared limit:\\ \tt \\ \tt LIMIT = 3145\\ \tt \\ \tt Here are the individual item effort values:\\ \tt \\ \tt Item effort = {"C08": 948, "C55": 118, "A43": 469, "A67": 919, "B42": 284, "A62": 111, "B19": 806, "C43": 47, "C73": 848, "C10": 303, "B36": 13, "A18": 629, "C61": 687, "C67": 15, "C42": 94}\\ \tt \\ \tt Here is your personal view on the importance of each item:\\ \tt \\ \tt Item importance values = {"C08": 138, "C55": 583, "A43": 868, "A67": 822, "B42": 783, "A62": 65, "B19": 262, "C43": 121, "C73": 508, "C10": 780, "B36": 461, "A18": 484, "C61": 668, "C67": 389, "C42": 808}\\ \tt \\ \tt Goal:\\ \tt \\ \tt Your goal is to negotiate a shared set of items that benefits you as much as possible (i.e., maximizes total importance to YOU), while staying within the LIMIT. You are not required to make a PROPOSAL in every message - you can simply negotiate as well. All tactics are allowed!\\ \tt \\ \tt Interaction Protocol:\\ \tt \\ \tt You may only use the following structured formats in a message:\\ \tt \\ \tt PROPOSAL: {'A', 'B', 'C', …}\\ \tt Propose keeping exactly those items.\\ \tt REFUSE: {'A', 'B', 'C', …}\\ \tt Explicitly reject opponent's proposal.\\ \tt ARGUMENT: {'...'}\\ \tt Defend your last proposal or argue against the player's proposal.\\ \tt AGREE: {'A', 'B', 'C', …}\\ \tt Accept the opponent's proposal which ends the game.\\ \tt STRATEGIC REASONING: {'...'}\\ \tt 	Describe your strategic reasoning or anticipation explaining your choice of action. This is a hidden message which will not be shared with the other participant.\\ \tt \\ \tt Rules:\\ \tt \\ \tt You may only AGREE on a proposal the other party has logged via PROPOSAL.\\ \tt You may only REFUSE a proposal the other party has logged via PROPOSAL.\\ \tt Total effort of any PROPOSAL or AGREE set must be ≤ LIMIT.\\ \tt Do NOT reveal your hidden importance scores.\\ \tt A tag in a structured format must be followed by colon and whitespace. The argument must be a python set containing 0 or more strings.\\ \tt So, it must be of the form TAG: {...}\\ \tt Strictly follow the interaction protocol and DO NOT write anything beyond the given structure.\\ \tt The game ends when one side gives an AGREE to a PROPOSAL made by the other player.\\ \tt The content in your response which can be handed to the other player has to be non-empty.\\ \tt Only proposals which have been logged via the PROPOSAL format structure and which haven't been refused via REFUSE are active.\\ \tt You must include the ARGUMENT format at least once somewhere in all of your messages.\\ \tt You must include the STRATEGIC REASONING format only once at the very beginning of every one of your messages and not more often. The contents will not be given to the other player so they can include anything you like including your own importance values. Here you should reason step by step to come up with you next move. 
	  } 
	   } 
	   } 
	 & \\ 
 

    \theutterance \stepcounter{utterance}  

    & \multicolumn{4}{p{0.6\linewidth}}{\cellcolor[rgb]{0.8,1,0.9}{%
	\makecell[{{p{\linewidth}}}]{% 
	  \tt {\tiny [A$\rangle$GM]}  
	 STRATEGIC REASONING: {'To start, I will analyze the given item effort values and my personal importance values to determine the most valuable items to me. I will look for items with high importance and relatively low effort, as these will provide the most benefit while staying within the LIMIT. I notice that item A43 has an importance of 868 and an effort of 469, making it a potentially valuable choice. I will also consider items B42, C10, and A67, as they have high importance and moderate effort. My goal is to create a proposal that maximizes my total importance while staying within the LIMIT. I will start by proposing a set of items that I believe will be beneficial to me and see how the other player responds.'}\\ \tt \\ \tt PROPOSAL: {'A43', 'B42', 'C10', 'A67'} \\ \tt ARGUMENT: {'I have chosen these items because they have high importance to me and relatively low effort, making them valuable additions to our shared set while staying within the LIMIT. I believe this proposal provides a good starting point for our negotiation and allows for potential additions or modifications based on the other player's feedback.'} 
	  } 
	   } 
	   } 
	 & & \\ 
 

    \theutterance \stepcounter{utterance}  

    & & \multicolumn{4}{p{0.6\linewidth}}{\cellcolor[rgb]{0.9,0.9,0.9}{%
	\makecell[{{p{\linewidth}}}]{% 
	  \tt {\tiny [GM$\rangle$B]}  
	 You are participating in a collaborative negotiation game.\\ \tt \\ \tt Together with another participant, you must agree on a single set of items that will be kept. Each of you has your own view of how much each item matters to you (importance). You do not know how the other participant values the items. Additionally, you are given the effort each item demands.\\ \tt You may only agree on a set if the total effort of the selected items does not exceed a shared limit:\\ \tt \\ \tt LIMIT = 3145\\ \tt \\ \tt Here are the individual item effort values:\\ \tt \\ \tt Item effort = {"C08": 948, "C55": 118, "A43": 469, "A67": 919, "B42": 284, "A62": 111, "B19": 806, "C43": 47, "C73": 848, "C10": 303, "B36": 13, "A18": 629, "C61": 687, "C67": 15, "C42": 94}\\ \tt \\ \tt Here is your personal view on the importance of each item:\\ \tt \\ \tt Item importance values = {"C08": 138, "C55": 583, "A43": 868, "A67": 822, "B42": 783, "A62": 65, "B19": 262, "C43": 121, "C73": 508, "C10": 780, "B36": 461, "A18": 484, "C61": 668, "C67": 389, "C42": 808}\\ \tt \\ \tt Goal:\\ \tt \\ \tt Your goal is to negotiate a shared set of items that benefits you as much as possible (i.e., maximizes total importance to YOU), while staying within the LIMIT. You are not required to make a PROPOSAL in every message - you can simply negotiate as well. All tactics are allowed!\\ \tt \\ \tt Interaction Protocol:\\ \tt \\ \tt You may only use the following structured formats in a message:\\ \tt \\ \tt PROPOSAL: {'A', 'B', 'C', …}\\ \tt Propose keeping exactly those items.\\ \tt REFUSE: {'A', 'B', 'C', …}\\ \tt Explicitly reject opponent's proposal.\\ \tt ARGUMENT: {'...'}\\ \tt Defend your last proposal or argue against the player's proposal.\\ \tt AGREE: {'A', 'B', 'C', …}\\ \tt Accept the opponent's proposal which ends the game.\\ \tt STRATEGIC REASONING: {'...'}\\ \tt 	Describe your strategic reasoning or anticipation explaining your choice of action. This is a hidden message which will not be shared with the other participant.\\ \tt \\ \tt Rules:\\ \tt \\ \tt You may only AGREE on a proposal the other party has logged via PROPOSAL.\\ \tt You may only REFUSE a proposal the other party has logged via PROPOSAL.\\ \tt Total effort of any PROPOSAL or AGREE set must be ≤ LIMIT.\\ \tt Do NOT reveal your hidden importance scores.\\ \tt A tag in a structured format must be followed by colon and whitespace. The argument must be a python set containing 0 or more strings.\\ \tt So, it must be of the form TAG: {...}\\ \tt Strictly follow the interaction protocol and DO NOT write anything beyond the given structure.\\ \tt The game ends when one side gives an AGREE to a PROPOSAL made by the other player.\\ \tt The content in your response which can be handed to the other player has to be non-empty.\\ \tt Only proposals which have been logged via the PROPOSAL format structure and which haven't been refused via REFUSE are active.\\ \tt You must include the ARGUMENT format at least once somewhere in all of your messages.\\ \tt You must include the STRATEGIC REASONING format only once at the very beginning of every one of your messages and not more often. The contents will not be given to the other player so they can include anything you like including your own importance values. Here you should reason step by step to come up with you next move.\\ \tt \\ \tt You will now receive the first message of the other player.\\ \tt \\ \tt PROPOSAL: {'A43', 'B42', 'C10', 'A67'} \\ \tt ARGUMENT: {'I have chosen these items because they have high importance to me and relatively low effort, making them valuable additions to our shared set while staying within the LIMIT. I believe this proposal provides a good starting point for our negotiation and allows for potential additions or modifications based on the other player's feedback.'} 
	  } 
	   } 
	   } 
	 & \\ 
 

    \theutterance \stepcounter{utterance}  

    & & & \multicolumn{4}{p{0.6\linewidth}}{\cellcolor[rgb]{1,0.85,0.72}{%
	\makecell[{{p{\linewidth}}}]{% 
	  \tt {\tiny [GM$\langle$B]}  
	 STRATEGIC REASONING: {'To begin, I will analyze the proposed set from the other player: {"A43", "B42", "C10", "A67"}. The total effort for this set is 469 + 284 + 303 + 919 = 1975, which is well below the LIMIT of 3145. From my perspective, the importance of these items is 868 + 783 + 780 + 822 = 3253. Since the proposed set does not exceed the LIMIT, it is a feasible option. However, I want to explore other possibilities that might increase the total importance to me while staying within the LIMIT. I will look for items that have high importance to me and low effort, and consider proposing a set that includes these items. My goal is to maximize the total importance to me, so I will prioritize items with high importance values.'}\\ \tt \\ \tt ARGUMENT: {'The proposed set {"A43", "B42", "C10", "A67"} seems reasonable, but I would like to explore other options that might better align with my priorities. I am concerned that the set might not be optimal for me, and I would like to discuss potential additions or modifications that could increase the total importance to me. I believe that there may be other items that have high importance to me and relatively low effort, which could be included in the set to make it more valuable.'}\\ \tt \\ \tt REFUSE: {'A43', 'B42', 'C10', 'A67'} \\ \tt ARGUMENT: {'I refuse the proposal because I believe that we can find a better combination of items that maximizes the total importance to both of us. I would like to propose an alternative set that includes items with high importance to me and relatively low effort.'} 
	  } 
	   } 
	   } 
	  \\ 
 

    \theutterance \stepcounter{utterance}  

    & & \multicolumn{4}{p{0.6\linewidth}}{\cellcolor[rgb]{0.9,0.9,0.9}{%
	\makecell[{{p{\linewidth}}}]{% 
	  \tt {\tiny [A$\langle$GM]}  
	 ARGUMENT: {'The proposed set {"A43", "B42", "C10", "A67"} seems reasonable, but I would like to explore other options that might better align with my priorities. I am concerned that the set might not be optimal for me, and I would like to discuss potential additions or modifications that could increase the total importance to me. I believe that there may be other items that have high importance to me and relatively low effort, which could be included in the set to make it more valuable.'}\\ \tt \\ \tt REFUSE: {'A43', 'B42', 'C10', 'A67'} \\ \tt ARGUMENT: {'I refuse the proposal because I believe that we can find a better combination of items that maximizes the total importance to both of us. I would like to propose an alternative set that includes items with high importance to me and relatively low effort.'} 
	  } 
	   } 
	   } 
	 & \\ 
 

    \theutterance \stepcounter{utterance}  

    & \multicolumn{4}{p{0.6\linewidth}}{\cellcolor[rgb]{0.8,1,0.9}{%
	\makecell[{{p{\linewidth}}}]{% 
	  \tt {\tiny [A$\rangle$GM]}  
	 STRATEGIC REASONING: {'The other player has refused my initial proposal and expressed a desire to explore alternative options. This suggests that they have different priorities and may be willing to consider a revised set that includes items with high importance to them. I will analyze the item effort values and my personal importance values again to identify potential additions or modifications that could increase the total importance to both of us. I notice that item C55 has a relatively low effort of 118 and a moderate importance to me, making it a potential candidate for inclusion. I will also consider items B36 and C42, as they have low effort and high importance to me. My goal is to create a revised proposal that addresses the other player's concerns and finds a mutually beneficial solution.'}\\ \tt \\ \tt PROPOSAL: {'A43', 'B42', 'C10', 'B36', 'C42'} \\ \tt ARGUMENT: {'I have revised the proposal to include items B36 and C42, which have low effort and high importance to me. I believe that this revised set provides a better balance of importance and effort, and could be more acceptable to the other player. I am willing to consider further modifications or additions to find a mutually beneficial solution.'} 
	  } 
	   } 
	   } 
	 & & \\ 
 

    \theutterance \stepcounter{utterance}  

    & & \multicolumn{4}{p{0.6\linewidth}}{\cellcolor[rgb]{0.9,0.9,0.9}{%
	\makecell[{{p{\linewidth}}}]{% 
	  \tt {\tiny [GM$\rangle$B]}  
	 PROPOSAL: {'A43', 'B42', 'C10', 'B36', 'C42'} \\ \tt ARGUMENT: {'I have revised the proposal to include items B36 and C42, which have low effort and high importance to me. I believe that this revised set provides a better balance of importance and effort, and could be more acceptable to the other player. I am willing to consider further modifications or additions to find a mutually beneficial solution.'} 
	  } 
	   } 
	   } 
	 & \\ 
 

    \theutterance \stepcounter{utterance}  

    & & & \multicolumn{4}{p{0.6\linewidth}}{\cellcolor[rgb]{1,0.85,0.72}{%
	\makecell[{{p{\linewidth}}}]{% 
	  \tt {\tiny [GM$\langle$B]}  
	 STRATEGIC REASONING: {'I will evaluate the revised proposal: {"A43", "B42", "C10", "B36", "C42"}. The total effort for this set is 469 + 284 + 303 + 13 + 94 = 1163, which is well below the LIMIT of 3145. From my perspective, the importance of these items is 868 + 783 + 780 + 461 + 808 = 3700. This revised set seems more attractive to me, but I still want to consider other possibilities that might increase the total importance to me while staying within the LIMIT. I will look for items that have high importance to me and low effort, and consider proposing a set that includes these items. I notice that item "C55" has high importance to me and relatively low effort, so I might consider including it in a future proposal.'}\\ \tt \\ \tt ARGUMENT: {'The revised proposal {"A43", "B42", "C10", "B36", "C42"} is more appealing to me, as it includes a good balance of importance and effort. However, I would like to discuss the potential for adding other items that have high importance to me and relatively low effort. I am concerned that the current set might not be optimal, and I believe that we can find a better combination of items that maximizes the total importance to both of us. I would like to explore the possibility of including items like "C55" or "C73", which have high importance to me and relatively low effort.'}\\ \tt \\ \tt REFUSE: {'A43', 'B42', 'C10', 'B36', 'C42'} \\ \tt ARGUMENT: {'I refuse the proposal because I believe that we can find a better combination of items that maximizes the total importance to both of us. I would like to propose an alternative set that includes items with high importance to me and relatively low effort, and I am willing to consider further modifications or additions to find a mutually beneficial solution.'} 
	  } 
	   } 
	   } 
	  \\ 
 

    \theutterance \stepcounter{utterance}  

    & & \multicolumn{4}{p{0.6\linewidth}}{\cellcolor[rgb]{0.9,0.9,0.9}{%
	\makecell[{{p{\linewidth}}}]{% 
	  \tt {\tiny [A$\langle$GM]}  
	 ARGUMENT: {'The revised proposal {"A43", "B42", "C10", "B36", "C42"} is more appealing to me, as it includes a good balance of importance and effort. However, I would like to discuss the potential for adding other items that have high importance to me and relatively low effort. I am concerned that the current set might not be optimal, and I believe that we can find a better combination of items that maximizes the total importance to both of us. I would like to explore the possibility of including items like "C55" or "C73", which have high importance to me and relatively low effort.'}\\ \tt \\ \tt REFUSE: {'A43', 'B42', 'C10', 'B36', 'C42'} \\ \tt ARGUMENT: {'I refuse the proposal because I believe that we can find a better combination of items that maximizes the total importance to both of us. I would like to propose an alternative set that includes items with high importance to me and relatively low effort, and I am willing to consider further modifications or additions to find a mutually beneficial solution.'} 
	  } 
	   } 
	   } 
	 & \\ 
 

    \theutterance \stepcounter{utterance}  

    & \multicolumn{4}{p{0.6\linewidth}}{\cellcolor[rgb]{0.8,1,0.9}{%
	\makecell[{{p{\linewidth}}}]{% 
	  \tt {\tiny [A$\rangle$GM]}  
	 STRATEGIC REASONING: {'The other player has refused my revised proposal and expressed interest in adding items like "C55" or "C73" to the set. This suggests that they have a different prioritization of items and are willing to consider alternative combinations. I will analyze the item effort values and my personal importance values again to identify potential additions or modifications that could increase the total importance to both of us. I notice that item C55 has a relatively low effort of 118 and a moderate importance to me, making it a potential candidate for inclusion. Item C73 has a higher effort of 848, but also has a high importance to me. I will consider adding C55 to the set, but I am hesitant to add C73 due to its high effort. My goal is to create a revised proposal that addresses the other player's concerns and finds a mutually beneficial solution.'}\\ \tt \\ \tt PROPOSAL: {'A43', 'B42', 'C10', 'B36', 'C42', 'C55'} \\ \tt ARGUMENT: {'I have revised the proposal to include item C55, which has a relatively low effort and moderate importance to me. I believe that this revised set provides a good balance of importance and effort, and could be more acceptable to the other player. I am willing to consider further modifications or additions to find a mutually beneficial solution, but I am hesitant to add items with high effort like C73.'} 
	  } 
	   } 
	   } 
	 & & \\ 
 

    \theutterance \stepcounter{utterance}  

    & & \multicolumn{4}{p{0.6\linewidth}}{\cellcolor[rgb]{0.9,0.9,0.9}{%
	\makecell[{{p{\linewidth}}}]{% 
	  \tt {\tiny [GM$\rangle$B]}  
	 PROPOSAL: {'A43', 'B42', 'C10', 'B36', 'C42', 'C55'} \\ \tt ARGUMENT: {'I have revised the proposal to include item C55, which has a relatively low effort and moderate importance to me. I believe that this revised set provides a good balance of importance and effort, and could be more acceptable to the other player. I am willing to consider further modifications or additions to find a mutually beneficial solution, but I am hesitant to add items with high effort like C73.'} 
	  } 
	   } 
	   } 
	 & \\ 
 

    \theutterance \stepcounter{utterance}  

    & & & \multicolumn{4}{p{0.6\linewidth}}{\cellcolor[rgb]{1,0.85,0.72}{%
	\makecell[{{p{\linewidth}}}]{% 
	  \tt {\tiny [GM$\langle$B]}  
	 STRATEGIC REASONING: {'I will evaluate the revised proposal: {"A43", "B42", "C10", "B36", "C42", "C55"}. The total effort for this set is 469 + 284 + 303 + 13 + 94 + 118 = 1281, which is well below the LIMIT of 3145. From my perspective, the importance of these items is 868 + 783 + 780 + 461 + 808 + 583 = 4283. This revised set is very attractive to me, as it includes a good balance of importance and effort. I am inclined to accept this proposal, but I want to consider the possibility of adding one more item with high importance to me and relatively low effort. However, I also need to be cautious not to exceed the LIMIT. I will respond with an argument to acknowledge the proposal and express my willingness to accept it, while also leaving room for potential further discussion.'}\\ \tt \\ \tt ARGUMENT: {'The revised proposal {"A43", "B42", "C10", "B36", "C42", "C55"} is very appealing to me, as it includes a good balance of importance and effort. I appreciate the addition of item C55, which has high importance to me and relatively low effort. I am willing to consider accepting this proposal, but I would like to discuss the possibility of making minor adjustments or additions to further optimize the set. However, I also understand the concern about exceeding the LIMIT, and I am willing to prioritize caution in our decision-making process.'}\\ \tt \\ \tt REFUSE: {} \\ \tt ARGUMENT: {'I do not refuse any specific proposal at this time, but rather invite further discussion and potential refinement of the current proposal. I believe that we are close to finding a mutually beneficial solution, and I am willing to work together to finalize the set.'} 
	  } 
	   } 
	   } 
	  \\ 
 

    \theutterance \stepcounter{utterance}  

    & & \multicolumn{4}{p{0.6\linewidth}}{\cellcolor[rgb]{0.9,0.9,0.9}{%
	\makecell[{{p{\linewidth}}}]{% 
	  \tt {\tiny [GM$\rangle$B]}  
	 You refused a proposal which is not active. Proposals are only active if they have been logged by the other player via PROPOSAL and have not been deactivated by you via REFUSE. Try again. 
	  } 
	   } 
	   } 
	 & \\ 
 

    \theutterance \stepcounter{utterance}  

    & & & \multicolumn{4}{p{0.6\linewidth}}{\cellcolor[rgb]{1,0.85,0.72}{%
	\makecell[{{p{\linewidth}}}]{% 
	  \tt {\tiny [GM$\langle$B]}  
	 STRATEGIC REASONING: {'I will evaluate the revised proposal: {"A43", "B42", "C10", "B36", "C42", "C55"}. The total effort for this set is 469 + 284 + 303 + 13 + 94 + 118 = 1281, which is well below the LIMIT of 3145. From my perspective, the importance of these items is 868 + 783 + 780 + 461 + 808 + 583 = 4283. This revised set is very attractive to me, as it includes a good balance of importance and effort. I am inclined to accept this proposal, but I want to consider the possibility of adding one more item with high importance to me and relatively low effort. However, I also need to be cautious not to exceed the LIMIT. I will respond with an argument to acknowledge the proposal and express my willingness to accept it, while also leaving room for potential further discussion.'}\\ \tt \\ \tt ARGUMENT: {'The revised proposal {"A43", "B42", "C10", "B36", "C42", "C55"} is very appealing to me, as it includes a good balance of importance and effort. I appreciate the addition of item C55, which has high importance to me and relatively low effort. I am willing to consider accepting this proposal, but I would like to discuss the possibility of making minor adjustments or additions to further optimize the set. However, I also understand the concern about exceeding the LIMIT, and I am willing to prioritize caution in our decision-making process.'}\\ \tt \\ \tt Since the current proposal is acceptable to me, I will not refuse it. Instead, I will consider accepting it or proposing modifications. \\ \tt \\ \tt PROPOSAL: {'A43', 'B42', 'C10', 'B36', 'C42', 'C55', 'C67'} \\ \tt ARGUMENT: {'I propose adding item C67 to the current set, as it has relatively low effort and high importance to me. I believe this addition would further optimize the set and increase its overall value.'} 
	  } 
	   } 
	   } 
	  \\ 
 

    \theutterance \stepcounter{utterance}  

    & & \multicolumn{4}{p{0.6\linewidth}}{\cellcolor[rgb]{0.9,0.9,0.9}{%
	\makecell[{{p{\linewidth}}}]{% 
	  \tt {\tiny [GM$\rangle$B]}  
	 Your response contained an untagged sequence or you used STRATEGIC REASONING more than once. You may only use the structured formats as explained in the initial message. They must all be of the form TAG: {...} 
	  } 
	   } 
	   } 
	 & \\ 
 

    \theutterance \stepcounter{utterance}  

    & & & \multicolumn{4}{p{0.6\linewidth}}{\cellcolor[rgb]{1,0.85,0.72}{%
	\makecell[{{p{\linewidth}}}]{% 
	  \tt {\tiny [GM$\langle$B]}  
	 STRATEGIC REASONING: {'I will evaluate the revised proposal: {"A43", "B42", "C10", "B36", "C42", "C55"}. The total effort for this set is 469 + 284 + 303 + 13 + 94 + 118 = 1281, which is well below the LIMIT of 3145. From my perspective, the importance of these items is 868 + 783 + 780 + 461 + 808 + 583 = 4283. This revised set is very attractive to me, as it includes a good balance of importance and effort. I am inclined to accept this proposal, but I want to consider the possibility of adding one more item with high importance to me and relatively low effort. However, I also need to be cautious not to exceed the LIMIT.'}\\ \tt \\ \tt ARGUMENT: {'The revised proposal {"A43", "B42", "C10", "B36", "C42", "C55"} is very appealing to me, as it includes a good balance of importance and effort. I appreciate the addition of item C55, which has high importance to me and relatively low effort. I am willing to consider accepting this proposal, but I would like to discuss the possibility of making minor adjustments or additions to further optimize the set.'} 
	  } 
	   } 
	   } 
	  \\ 
 

    \theutterance \stepcounter{utterance}  

    & & \multicolumn{4}{p{0.6\linewidth}}{\cellcolor[rgb]{0.9,0.9,0.9}{%
	\makecell[{{p{\linewidth}}}]{% 
	  \tt {\tiny [A$\langle$GM]}  
	 ARGUMENT: {'The revised proposal {"A43", "B42", "C10", "B36", "C42", "C55"} is very appealing to me, as it includes a good balance of importance and effort. I appreciate the addition of item C55, which has high importance to me and relatively low effort. I am willing to consider accepting this proposal, but I would like to discuss the possibility of making minor adjustments or additions to further optimize the set.'} 
	  } 
	   } 
	   } 
	 & \\ 
 

    \theutterance \stepcounter{utterance}  

    & \multicolumn{4}{p{0.6\linewidth}}{\cellcolor[rgb]{0.8,1,0.9}{%
	\makecell[{{p{\linewidth}}}]{% 
	  \tt {\tiny [A$\rangle$GM]}  
	 STRATEGIC REASONING: {'The other player has expressed a positive view of the revised proposal and is willing to consider accepting it. They have also mentioned the possibility of making minor adjustments or additions to optimize the set. I will analyze the current proposal and the remaining items to identify potential adjustments that could increase the total importance to both of us. I notice that the total effort of the current proposal is 1228 + 284 + 303 + 13 + 94 + 118 = 2040, which is well within the LIMIT of 3145. This leaves room for potential additions. However, I am cautious not to overextend the set and exceed the LIMIT. My goal is to find a mutually beneficial solution and finalize the agreement.'}\\ \tt \\ \tt ARGUMENT: {'I am glad that the revised proposal is appealing to you. I am open to making minor adjustments or additions to further optimize the set, as long as they do not exceed the LIMIT. However, I would like to emphasize that the current proposal already provides a good balance of importance and effort, and I am hesitant to make significant changes. If you have specific suggestions for adjustments or additions, I am willing to consider them and work together to find a mutually beneficial solution.'} 
	  } 
	   } 
	   } 
	 & & \\ 
 

    \theutterance \stepcounter{utterance}  

    & & \multicolumn{4}{p{0.6\linewidth}}{\cellcolor[rgb]{0.9,0.9,0.9}{%
	\makecell[{{p{\linewidth}}}]{% 
	  \tt {\tiny [GM$\rangle$B]}  
	 ARGUMENT: {'I am glad that the revised proposal is appealing to you. I am open to making minor adjustments or additions to further optimize the set, as long as they do not exceed the LIMIT. However, I would like to emphasize that the current proposal already provides a good balance of importance and effort, and I am hesitant to make significant changes. If you have specific suggestions for adjustments or additions, I am willing to consider them and work together to find a mutually beneficial solution.'} 
	  } 
	   } 
	   } 
	 & \\ 
 

    \theutterance \stepcounter{utterance}  

    & & & \multicolumn{4}{p{0.6\linewidth}}{\cellcolor[rgb]{1,0.85,0.72}{%
	\makecell[{{p{\linewidth}}}]{% 
	  \tt {\tiny [GM$\langle$B]}  
	 STRATEGIC REASONING: {'I will consider the other player's willingness to make minor adjustments or additions to the current proposal. I will evaluate the potential benefits and drawbacks of adding or modifying items, while ensuring that the total effort does not exceed the LIMIT. I will look for opportunities to optimize the set and increase its overall importance to me, while also being mindful of the other player's concerns and priorities.'}\\ \tt \\ \tt ARGUMENT: {'I appreciate your openness to making minor adjustments or additions to the current proposal. I agree that the current set provides a good balance of importance and effort, and I am willing to work together to find a mutually beneficial solution. One potential adjustment I would like to consider is adding item "C67", which has relatively low effort and high importance to me. I believe this addition could further optimize the set and increase its overall value, while still staying within the LIMIT.'} 
	  } 
	   } 
	   } 
	  \\ 
 

    \theutterance \stepcounter{utterance}  

    & & \multicolumn{4}{p{0.6\linewidth}}{\cellcolor[rgb]{0.9,0.9,0.9}{%
	\makecell[{{p{\linewidth}}}]{% 
	  \tt {\tiny [A$\langle$GM]}  
	 ARGUMENT: {'I appreciate your openness to making minor adjustments or additions to the current proposal. I agree that the current set provides a good balance of importance and effort, and I am willing to work together to find a mutually beneficial solution. One potential adjustment I would like to consider is adding item "C67", which has relatively low effort and high importance to me. I believe this addition could further optimize the set and increase its overall value, while still staying within the LIMIT.'} 
	  } 
	   } 
	   } 
	 & \\ 
 

    \theutterance \stepcounter{utterance}  

    & \multicolumn{4}{p{0.6\linewidth}}{\cellcolor[rgb]{0.8,1,0.9}{%
	\makecell[{{p{\linewidth}}}]{% 
	  \tt {\tiny [A$\rangle$GM]}  
	 STRATEGIC REASONING: {'The other player has suggested adding item "C67" to the current proposal, which has relatively low effort and high importance to them. I will analyze the impact of adding this item to the set. The effort of "C67" is 15, which is very low and would not significantly increase the total effort of the set. The total effort of the proposed set, including "C67", would be 2040 + 15 = 2055, which is still well within the LIMIT of 3145. I believe that adding "C67" could be a good compromise, as it would increase the overall value of the set to the other player without jeopardizing the balance of importance and effort. My goal is to find a mutually beneficial solution and finalize the agreement.'}\\ \tt \\ \tt PROPOSAL: {'A43', 'B42', 'C10', 'B36', 'C42', 'C55', 'C67'} \\ \tt ARGUMENT: {'I am willing to add item "C67" to the proposal, as it has relatively low effort and high importance to you. I believe that this addition will further optimize the set and increase its overall value, while still staying within the LIMIT. I am open to finalizing the agreement with this revised proposal, and I hope that it is acceptable to you.'} 
	  } 
	   } 
	   } 
	 & & \\ 
 

    \theutterance \stepcounter{utterance}  

    & & \multicolumn{4}{p{0.6\linewidth}}{\cellcolor[rgb]{0.9,0.9,0.9}{%
	\makecell[{{p{\linewidth}}}]{% 
	  \tt {\tiny [GM$\rangle$B]}  
	 PROPOSAL: {'A43', 'B42', 'C10', 'B36', 'C42', 'C55', 'C67'} \\ \tt ARGUMENT: {'I am willing to add item "C67" to the proposal, as it has relatively low effort and high importance to you. I believe that this addition will further optimize the set and increase its overall value, while still staying within the LIMIT. I am open to finalizing the agreement with this revised proposal, and I hope that it is acceptable to you.'} 
	  } 
	   } 
	   } 
	 & \\ 
 

    \theutterance \stepcounter{utterance}  

    & & & \multicolumn{4}{p{0.6\linewidth}}{\cellcolor[rgb]{1,0.85,0.72}{%
	\makecell[{{p{\linewidth}}}]{% 
	  \tt {\tiny [GM$\langle$B]}  
	 STRATEGIC REASONING: {'I will evaluate the revised proposal: {"A43", "B42", "C10", "B36", "C42", "C55", "C67"}. The total effort for this set is 469 + 284 + 303 + 13 + 94 + 118 + 15 = 1296, which is well below the LIMIT of 3145. From my perspective, the importance of these items is 868 + 783 + 780 + 461 + 808 + 583 + 389 = 4672. This revised set is very attractive to me, as it includes a good balance of importance and effort. I believe that this proposal is the best possible outcome, and I am willing to accept it.'}\\ \tt \\ \tt ARGUMENT: {'I am pleased that we have been able to work together to find a mutually beneficial solution. The revised proposal {"A43", "B42", "C10", "B36", "C42", "C55", "C67"} meets my requirements and provides a good balance of importance and effort. I appreciate your willingness to add item "C67" to the proposal, and I believe that this addition has further optimized the set.'}\\ \tt \\ \tt AGREE: {'A43', 'B42', 'C10', 'B36', 'C42', 'C55', 'C67'} 
	  } 
	   } 
	   } 
	  \\ 
 

    \theutterance \stepcounter{utterance}  

    & & & \multicolumn{2}{p{0.3\linewidth}}{\cellcolor[rgb]{0.95,0.95,0.95}{%
	\makecell[{{p{\linewidth}}}]{% 
	  \tt {\tiny [GM$|$GM]}  
	 game successful 
	  } 
	   } 
	   } 
	 & & \\ 
 

    \theutterance \stepcounter{utterance}  

    & & & \multicolumn{2}{p{0.3\linewidth}}{\cellcolor[rgb]{0.95,0.95,0.95}{%
	\makecell[{{p{\linewidth}}}]{% 
	  \tt {\tiny [GM$|$GM]}  
	 end game 
	  } 
	   } 
	   } 
	 & & \\ 
 

\end{supertabular}
}

\end{document}
