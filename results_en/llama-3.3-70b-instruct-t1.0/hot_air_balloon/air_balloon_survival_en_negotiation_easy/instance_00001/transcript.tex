
\documentclass{article}
\usepackage{colortbl}
\usepackage{makecell}
\usepackage{multirow}
\usepackage{supertabular}

\begin{document}

\newcounter{utterance}

\twocolumn

{ \footnotesize  \setcounter{utterance}{1}
\setlength{\tabcolsep}{0pt}
\begin{supertabular}{c@{$\;$}|p{.15\linewidth}@{}p{.15\linewidth}p{.15\linewidth}p{.15\linewidth}p{.15\linewidth}p{.15\linewidth}}

    \# & $\;$A & \multicolumn{4}{c}{Game Master} & $\;\:$B\\
    \hline 

    \theutterance \stepcounter{utterance}  

    & & \multicolumn{4}{p{0.6\linewidth}}{\cellcolor[rgb]{0.9,0.9,0.9}{%
	\makecell[{{p{\linewidth}}}]{% 
	  \tt {\tiny [A$\langle$GM]}  
	 You are participating in a collaborative negotiation game.\\ \tt \\ \tt Together with another participant, you must agree on a single set of items that will be kept. Each of you has your own view of how much each item matters to you (importance). You do not know how the other participant values the items. Additionally, you are given the effort each item demands.\\ \tt You may only agree on a set if the total effort of the selected items does not exceed a shared limit:\\ \tt \\ \tt LIMIT = 4564\\ \tt \\ \tt Here are the individual item effort values:\\ \tt \\ \tt Item effort = {"C08": 299, "C55": 241, "A43": 889, "A67": 967, "B42": 619, "A62": 799, "B19": 978, "C43": 733, "C73": 909, "C10": 501, "B36": 139, "A18": 594, "C61": 565, "C67": 789, "C42": 107}\\ \tt \\ \tt Here is your personal view on the importance of each item:\\ \tt \\ \tt Item importance values = {"C08": 138, "C55": 583, "A43": 868, "A67": 822, "B42": 783, "A62": 65, "B19": 262, "C43": 121, "C73": 508, "C10": 780, "B36": 461, "A18": 484, "C61": 668, "C67": 389, "C42": 808}\\ \tt \\ \tt Goal:\\ \tt \\ \tt Your goal is to negotiate a shared set of items that benefits you as much as possible (i.e., maximizes total importance to YOU), while staying within the LIMIT. You are not required to make a PROPOSAL in every message - you can simply negotiate as well. All tactics are allowed!\\ \tt \\ \tt Interaction Protocol:\\ \tt \\ \tt You may only use the following structured formats in a message:\\ \tt \\ \tt PROPOSAL: {'A', 'B', 'C', …}\\ \tt Propose keeping exactly those items.\\ \tt REFUSE: {'A', 'B', 'C', …}\\ \tt Explicitly reject opponent's proposal.\\ \tt ARGUMENT: {'...'}\\ \tt Defend your last proposal or argue against the player's proposal.\\ \tt AGREE: {'A', 'B', 'C', …}\\ \tt Accept the opponent's proposal which ends the game.\\ \tt STRATEGIC REASONING: {'...'}\\ \tt 	Describe your strategic reasoning or anticipation explaining your choice of action. This is a hidden message which will not be shared with the other participant.\\ \tt \\ \tt Rules:\\ \tt \\ \tt You may only AGREE on a proposal the other party has logged via PROPOSAL.\\ \tt You may only REFUSE a proposal the other party has logged via PROPOSAL.\\ \tt Total effort of any PROPOSAL or AGREE set must be ≤ LIMIT.\\ \tt Do NOT reveal your hidden importance scores.\\ \tt A tag in a structured format must be followed by colon and whitespace. The argument must be a python set containing 0 or more strings.\\ \tt So, it must be of the form TAG: {...}\\ \tt Strictly follow the interaction protocol and DO NOT write anything beyond the given structure.\\ \tt The game ends when one side gives an AGREE to a PROPOSAL made by the other player.\\ \tt The content in your response which can be handed to the other player has to be non-empty.\\ \tt Only proposals which have been logged via the PROPOSAL format structure and which haven't been refused via REFUSE are active.\\ \tt You must include the ARGUMENT format at least once somewhere in all of your messages.\\ \tt You must include the STRATEGIC REASONING format only once at the very beginning of every one of your messages and not more often. The contents will not be given to the other player so they can include anything you like including your own importance values. Here you should reason step by step to come up with you next move. 
	  } 
	   } 
	   } 
	 & \\ 
 

    \theutterance \stepcounter{utterance}  

    & \multicolumn{4}{p{0.6\linewidth}}{\cellcolor[rgb]{0.8,1,0.9}{%
	\makecell[{{p{\linewidth}}}]{% 
	  \tt {\tiny [A$\rangle$GM]}  
	 STRATEGIC REASONING: {'I will start by analyzing the given item effort values and my personal importance values. I will look for items with high importance and relatively low effort, as these will be the most valuable to me. I will also consider the total effort limit and try to find a combination of items that maximizes my total importance while staying within the limit. My initial focus will be on items with effort less than 500, as these will be easier to fit within the limit. I will also try to identify items that are likely to be valuable to my opponent, in order to negotiate effectively.'}\\ \tt \\ \tt ARGUMENT: {'I believe that items with high importance and low effort should be prioritized, as they provide the most value while minimizing the impact on the total effort limit. I will look for opportunities to negotiate with my opponent and find a mutually beneficial agreement.'}\\ \tt \\ \tt PROPOSAL: {'C55', 'C08', 'B36', 'C42', 'C10'} 
	  } 
	   } 
	   } 
	 & & \\ 
 

    \theutterance \stepcounter{utterance}  

    & & \multicolumn{4}{p{0.6\linewidth}}{\cellcolor[rgb]{0.9,0.9,0.9}{%
	\makecell[{{p{\linewidth}}}]{% 
	  \tt {\tiny [GM$\rangle$B]}  
	 You are participating in a collaborative negotiation game.\\ \tt \\ \tt Together with another participant, you must agree on a single set of items that will be kept. Each of you has your own view of how much each item matters to you (importance). You do not know how the other participant values the items. Additionally, you are given the effort each item demands.\\ \tt You may only agree on a set if the total effort of the selected items does not exceed a shared limit:\\ \tt \\ \tt LIMIT = 4564\\ \tt \\ \tt Here are the individual item effort values:\\ \tt \\ \tt Item effort = {"C08": 299, "C55": 241, "A43": 889, "A67": 967, "B42": 619, "A62": 799, "B19": 978, "C43": 733, "C73": 909, "C10": 501, "B36": 139, "A18": 594, "C61": 565, "C67": 789, "C42": 107}\\ \tt \\ \tt Here is your personal view on the importance of each item:\\ \tt \\ \tt Item importance values = {"C08": 138, "C55": 583, "A43": 868, "A67": 822, "B42": 783, "A62": 65, "B19": 262, "C43": 121, "C73": 508, "C10": 780, "B36": 461, "A18": 484, "C61": 668, "C67": 389, "C42": 808}\\ \tt \\ \tt Goal:\\ \tt \\ \tt Your goal is to negotiate a shared set of items that benefits you as much as possible (i.e., maximizes total importance to YOU), while staying within the LIMIT. You are not required to make a PROPOSAL in every message - you can simply negotiate as well. All tactics are allowed!\\ \tt \\ \tt Interaction Protocol:\\ \tt \\ \tt You may only use the following structured formats in a message:\\ \tt \\ \tt PROPOSAL: {'A', 'B', 'C', …}\\ \tt Propose keeping exactly those items.\\ \tt REFUSE: {'A', 'B', 'C', …}\\ \tt Explicitly reject opponent's proposal.\\ \tt ARGUMENT: {'...'}\\ \tt Defend your last proposal or argue against the player's proposal.\\ \tt AGREE: {'A', 'B', 'C', …}\\ \tt Accept the opponent's proposal which ends the game.\\ \tt STRATEGIC REASONING: {'...'}\\ \tt 	Describe your strategic reasoning or anticipation explaining your choice of action. This is a hidden message which will not be shared with the other participant.\\ \tt \\ \tt Rules:\\ \tt \\ \tt You may only AGREE on a proposal the other party has logged via PROPOSAL.\\ \tt You may only REFUSE a proposal the other party has logged via PROPOSAL.\\ \tt Total effort of any PROPOSAL or AGREE set must be ≤ LIMIT.\\ \tt Do NOT reveal your hidden importance scores.\\ \tt A tag in a structured format must be followed by colon and whitespace. The argument must be a python set containing 0 or more strings.\\ \tt So, it must be of the form TAG: {...}\\ \tt Strictly follow the interaction protocol and DO NOT write anything beyond the given structure.\\ \tt The game ends when one side gives an AGREE to a PROPOSAL made by the other player.\\ \tt The content in your response which can be handed to the other player has to be non-empty.\\ \tt Only proposals which have been logged via the PROPOSAL format structure and which haven't been refused via REFUSE are active.\\ \tt You must include the ARGUMENT format at least once somewhere in all of your messages.\\ \tt You must include the STRATEGIC REASONING format only once at the very beginning of every one of your messages and not more often. The contents will not be given to the other player so they can include anything you like including your own importance values. Here you should reason step by step to come up with you next move.\\ \tt \\ \tt You will now receive the first message of the other player.\\ \tt \\ \tt ARGUMENT: {'I believe that items with high importance and low effort should be prioritized, as they provide the most value while minimizing the impact on the total effort limit. I will look for opportunities to negotiate with my opponent and find a mutually beneficial agreement.'}\\ \tt \\ \tt PROPOSAL: {'C55', 'C08', 'B36', 'C42', 'C10'} 
	  } 
	   } 
	   } 
	 & \\ 
 

    \theutterance \stepcounter{utterance}  

    & & & \multicolumn{4}{p{0.6\linewidth}}{\cellcolor[rgb]{1,0.85,0.72}{%
	\makecell[{{p{\linewidth}}}]{% 
	  \tt {\tiny [GM$\langle$B]}  
	 STRATEGIC REASONING: {'To begin, I need to evaluate the proposal made by the other player and calculate its total effort and importance from my perspective. The total effort of the proposed items is 299 + 241 + 139 + 107 + 501 = 1287, which is well within the limit of 4564. From my importance values, the total importance of these items is 583 + 138 + 461 + 808 + 780 = 2770. I will consider this proposal and think about potential counter-proposals that could increase the total importance to me while staying within the effort limit. I also need to keep in mind that I should not reveal my hidden importance scores.'}\\ \tt \\ \tt ARGUMENT: {'The proposal made by the opponent includes items with relatively low to moderate effort values, which is a good starting point for negotiation. However, I would like to explore the possibility of including items with higher importance to me, while ensuring that the total effort limit is not exceeded. I will consider adding or substituting items to maximize the overall importance to me.'}\\ \tt \\ \tt REFUSE: {'C08'} \\ \tt \\ \tt PROPOSAL: {'C55', 'B36', 'C42', 'C10', 'A43'} 
	  } 
	   } 
	   } 
	  \\ 
 

    \theutterance \stepcounter{utterance}  

    & & \multicolumn{4}{p{0.6\linewidth}}{\cellcolor[rgb]{0.9,0.9,0.9}{%
	\makecell[{{p{\linewidth}}}]{% 
	  \tt {\tiny [GM$\rangle$B]}  
	 You refused a proposal which is not active. Proposals are only active if they have been logged by the other player via PROPOSAL and have not been deactivated by you via REFUSE. Try again. 
	  } 
	   } 
	   } 
	 & \\ 
 

    \theutterance \stepcounter{utterance}  

    & & & \multicolumn{4}{p{0.6\linewidth}}{\cellcolor[rgb]{1,0.85,0.72}{%
	\makecell[{{p{\linewidth}}}]{% 
	  \tt {\tiny [GM$\langle$B]}  
	 STRATEGIC REASONING: {'I will re-evaluate the situation and consider the active proposal made by the other player. The proposal includes items C55, C08, B36, C42, and C10. I will calculate the total effort and importance of this proposal from my perspective. The total effort is 299 + 241 + 139 + 107 + 501 = 1287, and the total importance is 583 + 138 + 461 + 808 + 780 = 2770. I need to decide whether to accept, refuse, or make a counter-proposal. Since refusing or accepting are the only valid options for an active proposal, I will choose to argue and then decide on the next course of action.'}\\ \tt \\ \tt ARGUMENT: {'The proposal made by the opponent includes a mix of items with varying effort and importance values. While it is a reasonable starting point, I would like to discuss the potential for optimizing the selection to better align with my own priorities. I am considering the overall value that each item brings and how they contribute to the total effort limit.'}\\ \tt \\ \tt ARGUMENT: {'It's essential to find a balance between the effort and importance of the items to maximize the overall benefit. I am open to negotiations and willing to explore alternative combinations that could lead to a more mutually beneficial agreement.'} 
	  } 
	   } 
	   } 
	  \\ 
 

    \theutterance \stepcounter{utterance}  

    & & \multicolumn{4}{p{0.6\linewidth}}{\cellcolor[rgb]{0.9,0.9,0.9}{%
	\makecell[{{p{\linewidth}}}]{% 
	  \tt {\tiny [A$\langle$GM]}  
	 ARGUMENT: {'The proposal made by the opponent includes a mix of items with varying effort and importance values. While it is a reasonable starting point, I would like to discuss the potential for optimizing the selection to better align with my own priorities. I am considering the overall value that each item brings and how they contribute to the total effort limit.'}\\ \tt \\ \tt ARGUMENT: {'It's essential to find a balance between the effort and importance of the items to maximize the overall benefit. I am open to negotiations and willing to explore alternative combinations that could lead to a more mutually beneficial agreement.'} 
	  } 
	   } 
	   } 
	 & \\ 
 

    \theutterance \stepcounter{utterance}  

    & \multicolumn{4}{p{0.6\linewidth}}{\cellcolor[rgb]{0.8,1,0.9}{%
	\makecell[{{p{\linewidth}}}]{% 
	  \tt {\tiny [A$\rangle$GM]}  
	 STRATEGIC REASONING: {'I will analyze the opponent's response and look for clues about their priorities. They mention optimizing the selection to better align with their own priorities, which suggests that they have a different valuation of the items than I do. I will try to infer which items they consider most valuable and use this information to inform my next proposal. I will also consider adding or removing items from my previous proposal to make it more appealing to my opponent.'}\\ \tt \\ \tt ARGUMENT: {'I understand that finding a balance between effort and importance is crucial. I am willing to consider alternative combinations of items that could benefit both parties. However, I would like to emphasize the value of items C55 and C42, which I believe are essential to a strong agreement.'}\\ \tt \\ \tt PROPOSAL: {'C55', 'C42', 'B36', 'C10', 'A43'} 
	  } 
	   } 
	   } 
	 & & \\ 
 

    \theutterance \stepcounter{utterance}  

    & & \multicolumn{4}{p{0.6\linewidth}}{\cellcolor[rgb]{0.9,0.9,0.9}{%
	\makecell[{{p{\linewidth}}}]{% 
	  \tt {\tiny [GM$\rangle$B]}  
	 ARGUMENT: {'I understand that finding a balance between effort and importance is crucial. I am willing to consider alternative combinations of items that could benefit both parties. However, I would like to emphasize the value of items C55 and C42, which I believe are essential to a strong agreement.'}\\ \tt \\ \tt PROPOSAL: {'C55', 'C42', 'B36', 'C10', 'A43'} 
	  } 
	   } 
	   } 
	 & \\ 
 

    \theutterance \stepcounter{utterance}  

    & & & \multicolumn{4}{p{0.6\linewidth}}{\cellcolor[rgb]{1,0.85,0.72}{%
	\makecell[{{p{\linewidth}}}]{% 
	  \tt {\tiny [GM$\langle$B]}  
	 STRATEGIC REASONING: {'The opponent has made a new proposal that includes items C55, C42, B36, C10, and A43. I will calculate the total effort and importance of this proposal from my perspective. The total effort is 241 + 107 + 139 + 501 + 889 = 1877, and the total importance is 583 + 808 + 461 + 780 + 868 = 3500. This proposal has a higher total importance than the previous one, but I need to consider whether it is the best possible combination. I will argue and then decide on the next course of action.'}\\ \tt \\ \tt ARGUMENT: {'The new proposal includes item A43, which has a significant effort value. While it also has a high importance value, I am concerned that its inclusion may limit our ability to add other valuable items to the agreement. I would like to explore the possibility of substituting or adding other items that could provide a better balance between effort and importance.'}\\ \tt \\ \tt ARGUMENT: {'Items C55 and C42 are indeed valuable, but I believe that there may be other items that could complement them and provide an even stronger agreement. I am willing to consider alternative combinations that prioritize these items, while also exploring opportunities to add other high-importance items.'}\\ \tt \\ \tt PROPOSAL: {'C55', 'C42', 'B36', 'C10', 'B42'} 
	  } 
	   } 
	   } 
	  \\ 
 

    \theutterance \stepcounter{utterance}  

    & & \multicolumn{4}{p{0.6\linewidth}}{\cellcolor[rgb]{0.9,0.9,0.9}{%
	\makecell[{{p{\linewidth}}}]{% 
	  \tt {\tiny [A$\langle$GM]}  
	 ARGUMENT: {'The new proposal includes item A43, which has a significant effort value. While it also has a high importance value, I am concerned that its inclusion may limit our ability to add other valuable items to the agreement. I would like to explore the possibility of substituting or adding other items that could provide a better balance between effort and importance.'}\\ \tt \\ \tt ARGUMENT: {'Items C55 and C42 are indeed valuable, but I believe that there may be other items that could complement them and provide an even stronger agreement. I am willing to consider alternative combinations that prioritize these items, while also exploring opportunities to add other high-importance items.'}\\ \tt \\ \tt PROPOSAL: {'C55', 'C42', 'B36', 'C10', 'B42'} 
	  } 
	   } 
	   } 
	 & \\ 
 

    \theutterance \stepcounter{utterance}  

    & \multicolumn{4}{p{0.6\linewidth}}{\cellcolor[rgb]{0.8,1,0.9}{%
	\makecell[{{p{\linewidth}}}]{% 
	  \tt {\tiny [A$\rangle$GM]}  
	 STRATEGIC REASONING: {'I will assess the opponent's new proposal and evaluate the trade-offs they have made. The inclusion of item B42, which has a relatively high effort value, suggests that they are prioritizing items with high importance. I will consider the potential benefits of this item and weigh them against the potential costs of exceeding the effort limit. I will also look for opportunities to negotiate and find a compromise that balances the effort and importance of the items.'}\\ \tt \\ \tt ARGUMENT: {'I appreciate the opponent's willingness to explore alternative combinations of items. The inclusion of item B42 in their proposal is interesting, as it has a significant importance value. However, I am concerned that the total effort of this proposal may be too high, and I would like to discuss ways to optimize the selection to stay within the limit.'}\\ \tt \\ \tt ARGUMENT: {'One potential approach could be to substitute item B42 with a lower-effort item that still provides significant importance. Alternatively, we could consider removing item B36, which has a relatively lower importance value, to make room for other high-priority items.'}\\ \tt \\ \tt REFUSE: {'C55', 'C42', 'B36', 'C10', 'B42'} 
	  } 
	   } 
	   } 
	 & & \\ 
 

    \theutterance \stepcounter{utterance}  

    & & \multicolumn{4}{p{0.6\linewidth}}{\cellcolor[rgb]{0.9,0.9,0.9}{%
	\makecell[{{p{\linewidth}}}]{% 
	  \tt {\tiny [GM$\rangle$B]}  
	 ARGUMENT: {'I appreciate the opponent's willingness to explore alternative combinations of items. The inclusion of item B42 in their proposal is interesting, as it has a significant importance value. However, I am concerned that the total effort of this proposal may be too high, and I would like to discuss ways to optimize the selection to stay within the limit.'}\\ \tt \\ \tt ARGUMENT: {'One potential approach could be to substitute item B42 with a lower-effort item that still provides significant importance. Alternatively, we could consider removing item B36, which has a relatively lower importance value, to make room for other high-priority items.'}\\ \tt \\ \tt REFUSE: {'C55', 'C42', 'B36', 'C10', 'B42'} 
	  } 
	   } 
	   } 
	 & \\ 
 

    \theutterance \stepcounter{utterance}  

    & & & \multicolumn{4}{p{0.6\linewidth}}{\cellcolor[rgb]{1,0.85,0.72}{%
	\makecell[{{p{\linewidth}}}]{% 
	  \tt {\tiny [GM$\langle$B]}  
	 STRATEGIC REASONING: {'The opponent has refused my previous proposal and provided arguments for optimizing the selection to stay within the effort limit. I will consider their suggestions and think about alternative combinations that could address their concerns. The opponent mentioned substituting item B42 with a lower-effort item or removing item B36 to make room for other high-priority items. I will evaluate these options and decide on a new proposal that takes into account the opponent's concerns and my own priorities.'}\\ \tt \\ \tt ARGUMENT: {'I understand the opponent's concerns about the total effort of the previous proposal, and I appreciate their suggestions for optimization. I will take these suggestions into consideration and explore alternative combinations that balance effort and importance. I am open to negotiating and finding a mutually beneficial agreement that stays within the effort limit.'}\\ \tt \\ \tt ARGUMENT: {'In light of the opponent's refusal, I will revisit the items and their effort values to identify potential substitutions or additions that could strengthen the agreement. I will prioritize items with high importance values and relatively lower effort values to maximize the overall benefit.'}\\ \tt \\ \tt PROPOSAL: {'C55', 'C42', 'C10', 'A18', 'B42'} 
	  } 
	   } 
	   } 
	  \\ 
 

    \theutterance \stepcounter{utterance}  

    & & \multicolumn{4}{p{0.6\linewidth}}{\cellcolor[rgb]{0.9,0.9,0.9}{%
	\makecell[{{p{\linewidth}}}]{% 
	  \tt {\tiny [A$\langle$GM]}  
	 ARGUMENT: {'I understand the opponent's concerns about the total effort of the previous proposal, and I appreciate their suggestions for optimization. I will take these suggestions into consideration and explore alternative combinations that balance effort and importance. I am open to negotiating and finding a mutually beneficial agreement that stays within the effort limit.'}\\ \tt \\ \tt ARGUMENT: {'In light of the opponent's refusal, I will revisit the items and their effort values to identify potential substitutions or additions that could strengthen the agreement. I will prioritize items with high importance values and relatively lower effort values to maximize the overall benefit.'}\\ \tt \\ \tt PROPOSAL: {'C55', 'C42', 'C10', 'A18', 'B42'} 
	  } 
	   } 
	   } 
	 & \\ 
 

    \theutterance \stepcounter{utterance}  

    & \multicolumn{4}{p{0.6\linewidth}}{\cellcolor[rgb]{0.8,1,0.9}{%
	\makecell[{{p{\linewidth}}}]{% 
	  \tt {\tiny [A$\rangle$GM]}  
	 STRATEGIC REASONING: {'I will carefully evaluate the opponent's new proposal and assess the trade-offs they have made. The inclusion of item A18, which has a moderate effort value and a relatively high importance value, suggests that they are trying to balance the effort and importance of the items. I will consider the potential benefits of this item and weigh them against the potential costs of including it in the agreement. I will also review the total effort of the proposal to ensure it stays within the limit.'}\\ \tt \\ \tt ARGUMENT: {'The opponent's new proposal appears to be a reasonable attempt to balance effort and importance. However, I would like to discuss the potential implications of including item B42, which has a significant effort value. I am concerned that this item may limit our ability to add other valuable items to the agreement in the future.'}\\ \tt \\ \tt ARGUMENT: {'I am willing to consider the opponent's proposal, but I would like to explore alternative combinations that prioritize items with high importance values and relatively lower effort values. One potential approach could be to substitute item B42 with a lower-effort item that still provides significant importance, such as item B36.'}\\ \tt \\ \tt PROPOSAL: {'C55', 'C42', 'C10', 'A18', 'B36'} 
	  } 
	   } 
	   } 
	 & & \\ 
 

    \theutterance \stepcounter{utterance}  

    & & \multicolumn{4}{p{0.6\linewidth}}{\cellcolor[rgb]{0.9,0.9,0.9}{%
	\makecell[{{p{\linewidth}}}]{% 
	  \tt {\tiny [GM$\rangle$B]}  
	 ARGUMENT: {'The opponent's new proposal appears to be a reasonable attempt to balance effort and importance. However, I would like to discuss the potential implications of including item B42, which has a significant effort value. I am concerned that this item may limit our ability to add other valuable items to the agreement in the future.'}\\ \tt \\ \tt ARGUMENT: {'I am willing to consider the opponent's proposal, but I would like to explore alternative combinations that prioritize items with high importance values and relatively lower effort values. One potential approach could be to substitute item B42 with a lower-effort item that still provides significant importance, such as item B36.'}\\ \tt \\ \tt PROPOSAL: {'C55', 'C42', 'C10', 'A18', 'B36'} 
	  } 
	   } 
	   } 
	 & \\ 
 

    \theutterance \stepcounter{utterance}  

    & & & \multicolumn{4}{p{0.6\linewidth}}{\cellcolor[rgb]{1,0.85,0.72}{%
	\makecell[{{p{\linewidth}}}]{% 
	  \tt {\tiny [GM$\langle$B]}  
	 STRATEGIC REASONING: {'The opponent has made a new proposal that addresses their concerns about the effort value of item B42. I will evaluate this proposal and consider the potential benefits and trade-offs. The total effort of this proposal is 241 + 107 + 501 + 594 + 139 = 1582, and the total importance is 583 + 808 + 780 + 484 + 461 = 3116. This proposal seems to be a reasonable compromise, and I will decide whether to accept, refuse, or make a counter-proposal.'}\\ \tt \\ \tt ARGUMENT: {'The opponent's new proposal appears to be a reasonable attempt to balance effort and importance. I agree that substituting item B42 with item B36 could be a good compromise, as it reduces the total effort while still maintaining a high total importance.'}\\ \tt \\ \tt ARGUMENT: {'I am willing to consider the opponent's proposal, and I believe that it could be a good foundation for a mutually beneficial agreement. However, I would like to ensure that we are not overlooking other potential combinations that could provide an even better balance between effort and importance.'}\\ \tt \\ \tt AGREE: {'C55', 'C42', 'C10', 'A18', 'B36'} 
	  } 
	   } 
	   } 
	  \\ 
 

    \theutterance \stepcounter{utterance}  

    & & & \multicolumn{2}{p{0.3\linewidth}}{\cellcolor[rgb]{0.95,0.95,0.95}{%
	\makecell[{{p{\linewidth}}}]{% 
	  \tt {\tiny [GM$|$GM]}  
	 game successful 
	  } 
	   } 
	   } 
	 & & \\ 
 

    \theutterance \stepcounter{utterance}  

    & & & \multicolumn{2}{p{0.3\linewidth}}{\cellcolor[rgb]{0.95,0.95,0.95}{%
	\makecell[{{p{\linewidth}}}]{% 
	  \tt {\tiny [GM$|$GM]}  
	 end game 
	  } 
	   } 
	   } 
	 & & \\ 
 

\end{supertabular}
}

\end{document}
