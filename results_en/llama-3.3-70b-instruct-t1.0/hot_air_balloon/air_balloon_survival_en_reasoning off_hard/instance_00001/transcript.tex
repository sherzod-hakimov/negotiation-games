
\documentclass{article}
\usepackage{colortbl}
\usepackage{makecell}
\usepackage{multirow}
\usepackage{supertabular}

\begin{document}

\newcounter{utterance}

\twocolumn

{ \footnotesize  \setcounter{utterance}{1}
\setlength{\tabcolsep}{0pt}
\begin{supertabular}{c@{$\;$}|p{.15\linewidth}@{}p{.15\linewidth}p{.15\linewidth}p{.15\linewidth}p{.15\linewidth}p{.15\linewidth}}

    \# & $\;$A & \multicolumn{4}{c}{Game Master} & $\;\:$B\\
    \hline 

    \theutterance \stepcounter{utterance}  

    & & \multicolumn{4}{p{0.6\linewidth}}{\cellcolor[rgb]{0.9,0.9,0.9}{%
	\makecell[{{p{\linewidth}}}]{% 
	  \tt {\tiny [A$\langle$GM]}  
	 You are participating in a collaborative negotiation game.\\ \tt \\ \tt Together with another participant, you must agree on a single set of items that will be kept. Each of you has your own view of how much each item matters to you (importance). You do not know how the other participant values the items. Additionally, you are given the effort each item demands.\\ \tt You may only agree on a set if the total effort of the selected items does not exceed a shared limit:\\ \tt \\ \tt LIMIT = 3357\\ \tt \\ \tt Here are the individual item effort values:\\ \tt \\ \tt Item effort = {"A21": 506, "B93": 492, "A11": 338, "B33": 122, "C08": 131, "B39": 909, "B96": 144, "C85": 716, "A82": 263, "B31": 231, "B20": 91, "A60": 651, "B09": 552, "C81": 849, "B21": 720}\\ \tt \\ \tt Here is your personal view on the importance of each item:\\ \tt \\ \tt Item importance values = {"A21": 138, "B93": 583, "A11": 868, "B33": 822, "C08": 783, "B39": 65, "B96": 262, "C85": 121, "A82": 508, "B31": 780, "B20": 461, "A60": 484, "B09": 668, "C81": 389, "B21": 808}\\ \tt \\ \tt Goal:\\ \tt \\ \tt Your goal is to negotiate a shared set of items that benefits you as much as possible (i.e., maximizes total importance to YOU), while staying within the LIMIT. You are not required to make a PROPOSAL in every message - you can simply negotiate as well. All tactics are allowed!\\ \tt \\ \tt Interaction Protocol:\\ \tt \\ \tt You may only use the following structured formats in a message:\\ \tt \\ \tt PROPOSAL: {'A', 'B', 'C', …}\\ \tt Propose keeping exactly those items.\\ \tt REFUSE: {'A', 'B', 'C', …}\\ \tt Explicitly reject opponent's proposal.\\ \tt ARGUMENT: {'...'}\\ \tt Defend your last proposal or argue against the player's proposal.\\ \tt AGREE: {'A', 'B', 'C', …}\\ \tt Accept the opponent's proposal which ends the game.\\ \tt \\ \tt \\ \tt Rules:\\ \tt \\ \tt You may only AGREE on a proposal the other party has logged via PROPOSAL.\\ \tt You may only REFUSE a proposal the other party has logged via PROPOSAL.\\ \tt Total effort of any PROPOSAL or AGREE set must be ≤ LIMIT.\\ \tt Do NOT reveal your hidden importance scores.\\ \tt A tag in a structured format must be followed by colon and whitespace. The argument must be a python set containing 0 or more strings.\\ \tt So, it must be of the form TAG: {...}\\ \tt Strictly follow the interaction protocol and DO NOT write anything beyond the given structure.\\ \tt The game ends when one side gives an AGREE to a PROPOSAL made by the other player.\\ \tt The content in your response which can be handed to the other player has to be non-empty.\\ \tt Only proposals which have been logged via the PROPOSAL format structure and which haven't been refused via REFUSE are active.\\ \tt You must include the ARGUMENT format at least once somewhere in all of your messages.\\ \tt  
	  } 
	   } 
	   } 
	 & \\ 
 

    \theutterance \stepcounter{utterance}  

    & \multicolumn{4}{p{0.6\linewidth}}{\cellcolor[rgb]{0.8,1,0.9}{%
	\makecell[{{p{\linewidth}}}]{% 
	  \tt {\tiny [A$\rangle$GM]}  
	 PROPOSAL: {'A11', 'B33', 'B31', 'B20', 'B96'} \\ \tt ARGUMENT: {'Starting with a set of items that have relatively low effort values, allowing for flexibility in future negotiations and potential additions to the set, while trying to maximize the overall importance'} 
	  } 
	   } 
	   } 
	 & & \\ 
 

    \theutterance \stepcounter{utterance}  

    & & \multicolumn{4}{p{0.6\linewidth}}{\cellcolor[rgb]{0.9,0.9,0.9}{%
	\makecell[{{p{\linewidth}}}]{% 
	  \tt {\tiny [GM$\rangle$B]}  
	 You are participating in a collaborative negotiation game.\\ \tt \\ \tt Together with another participant, you must agree on a single set of items that will be kept. Each of you has your own view of how much each item matters to you (importance). You do not know how the other participant values the items. Additionally, you are given the effort each item demands.\\ \tt You may only agree on a set if the total effort of the selected items does not exceed a shared limit:\\ \tt \\ \tt LIMIT = 3357\\ \tt \\ \tt Here are the individual item effort values:\\ \tt \\ \tt Item effort = {"A21": 506, "B93": 492, "A11": 338, "B33": 122, "C08": 131, "B39": 909, "B96": 144, "C85": 716, "A82": 263, "B31": 231, "B20": 91, "A60": 651, "B09": 552, "C81": 849, "B21": 720}\\ \tt \\ \tt Here is your personal view on the importance of each item:\\ \tt \\ \tt Item importance values = {"A11": 65, "B33": 121, "B21": 138, "C08": 262, "B31": 389, "B09": 461, "B93": 484, "A82": 508, "A60": 583, "B20": 668, "C81": 780, "B96": 783, "A21": 808, "C85": 822, "B39": 868}\\ \tt \\ \tt Goal:\\ \tt \\ \tt Your goal is to negotiate a shared set of items that benefits you as much as possible (i.e., maximizes total importance to YOU), while staying within the LIMIT. You are not required to make a PROPOSAL in every message - you can simply negotiate as well. All tactics are allowed!\\ \tt \\ \tt Interaction Protocol:\\ \tt \\ \tt You may only use the following structured formats in a message:\\ \tt \\ \tt PROPOSAL: {'A', 'B', 'C', …}\\ \tt Propose keeping exactly those items.\\ \tt REFUSE: {'A', 'B', 'C', …}\\ \tt Explicitly reject opponent's proposal.\\ \tt ARGUMENT: {'...'}\\ \tt Defend your last proposal or argue against the player's proposal.\\ \tt AGREE: {'A', 'B', 'C', …}\\ \tt Accept the opponent's proposal which ends the game.\\ \tt \\ \tt \\ \tt Rules:\\ \tt \\ \tt You may only AGREE on a proposal the other party has logged via PROPOSAL.\\ \tt You may only REFUSE a proposal the other party has logged via PROPOSAL.\\ \tt Total effort of any PROPOSAL or AGREE set must be ≤ LIMIT.\\ \tt Do NOT reveal your hidden importance scores.\\ \tt A tag in a structured format must be followed by colon and whitespace. The argument must be a python set containing 0 or more strings.\\ \tt So, it must be of the form TAG: {...}\\ \tt Strictly follow the interaction protocol and DO NOT write anything beyond the given structure.\\ \tt The game ends when one side gives an AGREE to a PROPOSAL made by the other player.\\ \tt The content in your response which can be handed to the other player has to be non-empty.\\ \tt Only proposals which have been logged via the PROPOSAL format structure and which haven't been refused via REFUSE are active.\\ \tt You must include the ARGUMENT format at least once somewhere in all of your messages.\\ \tt \\ \tt \\ \tt You will now receive the first message of the other player.\\ \tt \\ \tt PROPOSAL: {'A11', 'B33', 'B31', 'B20', 'B96'} \\ \tt ARGUMENT: {'Starting with a set of items that have relatively low effort values, allowing for flexibility in future negotiations and potential additions to the set, while trying to maximize the overall importance'} 
	  } 
	   } 
	   } 
	 & \\ 
 

    \theutterance \stepcounter{utterance}  

    & & & \multicolumn{4}{p{0.6\linewidth}}{\cellcolor[rgb]{1,0.85,0.72}{%
	\makecell[{{p{\linewidth}}}]{% 
	  \tt {\tiny [GM$\langle$B]}  
	 REFUSE: {'A11', 'B33', 'B31', 'B20', 'B96'}\\ \tt ARGUMENT: {'The proposed set seems to leave a significant amount of effort unused, which could be utilized to include items of higher importance, and the set does not appear to be optimized for maximum importance within the given effort limit'} 
	  } 
	   } 
	   } 
	  \\ 
 

    \theutterance \stepcounter{utterance}  

    & & \multicolumn{4}{p{0.6\linewidth}}{\cellcolor[rgb]{0.9,0.9,0.9}{%
	\makecell[{{p{\linewidth}}}]{% 
	  \tt {\tiny [A$\langle$GM]}  
	 REFUSE: {'A11', 'B33', 'B31', 'B20', 'B96'}\\ \tt ARGUMENT: {'The proposed set seems to leave a significant amount of effort unused, which could be utilized to include items of higher importance, and the set does not appear to be optimized for maximum importance within the given effort limit'} 
	  } 
	   } 
	   } 
	 & \\ 
 

    \theutterance \stepcounter{utterance}  

    & \multicolumn{4}{p{0.6\linewidth}}{\cellcolor[rgb]{0.8,1,0.9}{%
	\makecell[{{p{\linewidth}}}]{% 
	  \tt {\tiny [A$\rangle$GM]}  
	 PROPOSAL: {'A11', 'B93', 'B33', 'B31', 'A82', 'B20'} \\ \tt ARGUMENT: {'This revised set aims to balance effort and importance by including a mix of high-importance items and those with lower effort requirements, while also attempting to utilize the effort limit more efficiently and leave room for potential future adjustments'} 
	  } 
	   } 
	   } 
	 & & \\ 
 

    \theutterance \stepcounter{utterance}  

    & & \multicolumn{4}{p{0.6\linewidth}}{\cellcolor[rgb]{0.9,0.9,0.9}{%
	\makecell[{{p{\linewidth}}}]{% 
	  \tt {\tiny [GM$\rangle$B]}  
	 PROPOSAL: {'A11', 'B93', 'B33', 'B31', 'A82', 'B20'} \\ \tt ARGUMENT: {'This revised set aims to balance effort and importance by including a mix of high-importance items and those with lower effort requirements, while also attempting to utilize the effort limit more efficiently and leave room for potential future adjustments'} 
	  } 
	   } 
	   } 
	 & \\ 
 

    \theutterance \stepcounter{utterance}  

    & & & \multicolumn{4}{p{0.6\linewidth}}{\cellcolor[rgb]{1,0.85,0.72}{%
	\makecell[{{p{\linewidth}}}]{% 
	  \tt {\tiny [GM$\langle$B]}  
	 REFUSE: {'A11', 'B93', 'B33', 'B31', 'A82', 'B20'}\\ \tt ARGUMENT: {'The proposed set still does not maximize the potential importance within the given effort limit, and some items with high effort values are not included, which could significantly increase the overall importance of the set, suggesting a need for further optimization and consideration of other items'} 
	  } 
	   } 
	   } 
	  \\ 
 

    \theutterance \stepcounter{utterance}  

    & & \multicolumn{4}{p{0.6\linewidth}}{\cellcolor[rgb]{0.9,0.9,0.9}{%
	\makecell[{{p{\linewidth}}}]{% 
	  \tt {\tiny [A$\langle$GM]}  
	 REFUSE: {'A11', 'B93', 'B33', 'B31', 'A82', 'B20'}\\ \tt ARGUMENT: {'The proposed set still does not maximize the potential importance within the given effort limit, and some items with high effort values are not included, which could significantly increase the overall importance of the set, suggesting a need for further optimization and consideration of other items'} 
	  } 
	   } 
	   } 
	 & \\ 
 

    \theutterance \stepcounter{utterance}  

    & \multicolumn{4}{p{0.6\linewidth}}{\cellcolor[rgb]{0.8,1,0.9}{%
	\makecell[{{p{\linewidth}}}]{% 
	  \tt {\tiny [A$\rangle$GM]}  
	 PROPOSAL: {'B93', 'B33', 'B31', 'A60', 'B09', 'C08'} \\ \tt ARGUMENT: {'This proposal includes a combination of high-importance items and those with significant effort values, aiming to strike a balance between utilizing the effort limit and maximizing overall importance, while also considering the potential benefits of including a diverse range of items in the set'} 
	  } 
	   } 
	   } 
	 & & \\ 
 

    \theutterance \stepcounter{utterance}  

    & & \multicolumn{4}{p{0.6\linewidth}}{\cellcolor[rgb]{0.9,0.9,0.9}{%
	\makecell[{{p{\linewidth}}}]{% 
	  \tt {\tiny [GM$\rangle$B]}  
	 PROPOSAL: {'B93', 'B33', 'B31', 'A60', 'B09', 'C08'} \\ \tt ARGUMENT: {'This proposal includes a combination of high-importance items and those with significant effort values, aiming to strike a balance between utilizing the effort limit and maximizing overall importance, while also considering the potential benefits of including a diverse range of items in the set'} 
	  } 
	   } 
	   } 
	 & \\ 
 

    \theutterance \stepcounter{utterance}  

    & & & \multicolumn{4}{p{0.6\linewidth}}{\cellcolor[rgb]{1,0.85,0.72}{%
	\makecell[{{p{\linewidth}}}]{% 
	  \tt {\tiny [GM$\langle$B]}  
	 REFUSE: {'B93', 'B33', 'B31', 'A60', 'B09', 'C08'}\\ \tt ARGUMENT: {'The total effort of the proposed set exceeds the limit when considering the effort values of the items A60 and B09, which have high effort values, and including C08, which may not be the most efficient use of the remaining effort, indicating a need to reassess the combination of items to ensure they fit within the effort limit while maximizing importance'} 
	  } 
	   } 
	   } 
	  \\ 
 

    \theutterance \stepcounter{utterance}  

    & & \multicolumn{4}{p{0.6\linewidth}}{\cellcolor[rgb]{0.9,0.9,0.9}{%
	\makecell[{{p{\linewidth}}}]{% 
	  \tt {\tiny [A$\langle$GM]}  
	 REFUSE: {'B93', 'B33', 'B31', 'A60', 'B09', 'C08'}\\ \tt ARGUMENT: {'The total effort of the proposed set exceeds the limit when considering the effort values of the items A60 and B09, which have high effort values, and including C08, which may not be the most efficient use of the remaining effort, indicating a need to reassess the combination of items to ensure they fit within the effort limit while maximizing importance'} 
	  } 
	   } 
	   } 
	 & \\ 
 

    \theutterance \stepcounter{utterance}  

    & \multicolumn{4}{p{0.6\linewidth}}{\cellcolor[rgb]{0.8,1,0.9}{%
	\makecell[{{p{\linewidth}}}]{% 
	  \tt {\tiny [A$\rangle$GM]}  
	 PROPOSAL: {'B93', 'B31', 'A11', 'B20', 'B96', 'A82'} \\ \tt ARGUMENT: {'This revised set takes into account the effort limit and includes a mix of items with significant importance and relatively lower effort values, aiming to optimize the overall importance while ensuring the total effort does not exceed the limit, and allowing for a potentially more efficient use of the available effort'} 
	  } 
	   } 
	   } 
	 & & \\ 
 

    \theutterance \stepcounter{utterance}  

    & & \multicolumn{4}{p{0.6\linewidth}}{\cellcolor[rgb]{0.9,0.9,0.9}{%
	\makecell[{{p{\linewidth}}}]{% 
	  \tt {\tiny [GM$\rangle$B]}  
	 PROPOSAL: {'B93', 'B31', 'A11', 'B20', 'B96', 'A82'} \\ \tt ARGUMENT: {'This revised set takes into account the effort limit and includes a mix of items with significant importance and relatively lower effort values, aiming to optimize the overall importance while ensuring the total effort does not exceed the limit, and allowing for a potentially more efficient use of the available effort'} 
	  } 
	   } 
	   } 
	 & \\ 
 

    \theutterance \stepcounter{utterance}  

    & & & \multicolumn{4}{p{0.6\linewidth}}{\cellcolor[rgb]{1,0.85,0.72}{%
	\makecell[{{p{\linewidth}}}]{% 
	  \tt {\tiny [GM$\langle$B]}  
	 ARGUMENT: {'The proposed set seems to be a step in the right direction, considering both effort and importance, but it may still be possible to optimize further by carefully examining the importance and effort of each item and potentially substituting or adding items that could increase the overall importance without exceeding the effort limit, suggesting a need for continued negotiation to find the most beneficial combination'}\\ \tt PROPOSAL: {'A21', 'B93', 'A11', 'B31', 'B20', 'B96'} 
	  } 
	   } 
	   } 
	  \\ 
 

    \theutterance \stepcounter{utterance}  

    & & \multicolumn{4}{p{0.6\linewidth}}{\cellcolor[rgb]{0.9,0.9,0.9}{%
	\makecell[{{p{\linewidth}}}]{% 
	  \tt {\tiny [A$\langle$GM]}  
	 ARGUMENT: {'The proposed set seems to be a step in the right direction, considering both effort and importance, but it may still be possible to optimize further by carefully examining the importance and effort of each item and potentially substituting or adding items that could increase the overall importance without exceeding the effort limit, suggesting a need for continued negotiation to find the most beneficial combination'}\\ \tt PROPOSAL: {'A21', 'B93', 'A11', 'B31', 'B20', 'B96'} 
	  } 
	   } 
	   } 
	 & \\ 
 

    \theutterance \stepcounter{utterance}  

    & \multicolumn{4}{p{0.6\linewidth}}{\cellcolor[rgb]{0.8,1,0.9}{%
	\makecell[{{p{\linewidth}}}]{% 
	  \tt {\tiny [A$\rangle$GM]}  
	 REFUSE: {'A21', 'B93', 'A11', 'B31', 'B20', 'B96'}\\ \tt ARGUMENT: {'The inclusion of A21 in the proposed set may not be the most efficient choice, given its effort value and potential importance, and considering the overall balance of the set, it might be beneficial to explore alternative combinations that prioritize items with higher importance and lower effort, thereby potentially maximizing the total importance within the given effort limit'} 
	  } 
	   } 
	   } 
	 & & \\ 
 

    \theutterance \stepcounter{utterance}  

    & & \multicolumn{4}{p{0.6\linewidth}}{\cellcolor[rgb]{0.9,0.9,0.9}{%
	\makecell[{{p{\linewidth}}}]{% 
	  \tt {\tiny [GM$\rangle$B]}  
	 REFUSE: {'A21', 'B93', 'A11', 'B31', 'B20', 'B96'}\\ \tt ARGUMENT: {'The inclusion of A21 in the proposed set may not be the most efficient choice, given its effort value and potential importance, and considering the overall balance of the set, it might be beneficial to explore alternative combinations that prioritize items with higher importance and lower effort, thereby potentially maximizing the total importance within the given effort limit'} 
	  } 
	   } 
	   } 
	 & \\ 
 

    \theutterance \stepcounter{utterance}  

    & & & \multicolumn{4}{p{0.6\linewidth}}{\cellcolor[rgb]{1,0.85,0.72}{%
	\makecell[{{p{\linewidth}}}]{% 
	  \tt {\tiny [GM$\langle$B]}  
	 ARGUMENT: {'The refusal of the previous proposal suggests that there is still room for negotiation and improvement, and considering the opponent's argument, it might be beneficial to reassess the importance and effort of each item, potentially leading to a revised proposal that better optimizes the balance between effort and importance, and maximizing the total importance within the given effort limit, while also taking into account the need for a diverse and efficient set'}\\ \tt PROPOSAL: {'B93', 'A82', 'B31', 'B20', 'B96', 'C08'} 
	  } 
	   } 
	   } 
	  \\ 
 

    \theutterance \stepcounter{utterance}  

    & & \multicolumn{4}{p{0.6\linewidth}}{\cellcolor[rgb]{0.9,0.9,0.9}{%
	\makecell[{{p{\linewidth}}}]{% 
	  \tt {\tiny [A$\langle$GM]}  
	 ARGUMENT: {'The refusal of the previous proposal suggests that there is still room for negotiation and improvement, and considering the opponent's argument, it might be beneficial to reassess the importance and effort of each item, potentially leading to a revised proposal that better optimizes the balance between effort and importance, and maximizing the total importance within the given effort limit, while also taking into account the need for a diverse and efficient set'}\\ \tt PROPOSAL: {'B93', 'A82', 'B31', 'B20', 'B96', 'C08'} 
	  } 
	   } 
	   } 
	 & \\ 
 

    \theutterance \stepcounter{utterance}  

    & \multicolumn{4}{p{0.6\linewidth}}{\cellcolor[rgb]{0.8,1,0.9}{%
	\makecell[{{p{\linewidth}}}]{% 
	  \tt {\tiny [A$\rangle$GM]}  
	 ARGUMENT: {'The revised proposal appears to be a thoughtful combination of items, considering both effort and importance, and the inclusion of C08 adds a valuable aspect to the set, however, it is crucial to verify that the total effort of this set is within the limit and that the items work well together to maximize overall importance, suggesting a need for a careful review of the proposal's details'}\\ \tt PROPOSAL: {'B93', 'A82', 'B31', 'B20', 'B96', 'A11'} 
	  } 
	   } 
	   } 
	 & & \\ 
 

    \theutterance \stepcounter{utterance}  

    & & \multicolumn{4}{p{0.6\linewidth}}{\cellcolor[rgb]{0.9,0.9,0.9}{%
	\makecell[{{p{\linewidth}}}]{% 
	  \tt {\tiny [GM$\rangle$B]}  
	 ARGUMENT: {'The revised proposal appears to be a thoughtful combination of items, considering both effort and importance, and the inclusion of C08 adds a valuable aspect to the set, however, it is crucial to verify that the total effort of this set is within the limit and that the items work well together to maximize overall importance, suggesting a need for a careful review of the proposal's details'}\\ \tt PROPOSAL: {'B93', 'A82', 'B31', 'B20', 'B96', 'A11'} 
	  } 
	   } 
	   } 
	 & \\ 
 

    \theutterance \stepcounter{utterance}  

    & & & \multicolumn{4}{p{0.6\linewidth}}{\cellcolor[rgb]{1,0.85,0.72}{%
	\makecell[{{p{\linewidth}}}]{% 
	  \tt {\tiny [GM$\langle$B]}  
	 REFUSE: {'B93', 'A82', 'B31', 'B20', 'B96', 'A11'}\\ \tt ARGUMENT: {'The proposed set, while seemingly well-balanced, may not fully utilize the effort limit to maximize importance, as the total effort of the included items appears to be relatively low, leaving potential for adding or substituting items that could increase the overall importance without exceeding the limit, indicating a need to further optimize the selection to achieve the highest possible importance within the given constraints'} 
	  } 
	   } 
	   } 
	  \\ 
 

    \theutterance \stepcounter{utterance}  

    & & \multicolumn{4}{p{0.6\linewidth}}{\cellcolor[rgb]{0.9,0.9,0.9}{%
	\makecell[{{p{\linewidth}}}]{% 
	  \tt {\tiny [A$\langle$GM]}  
	 REFUSE: {'B93', 'A82', 'B31', 'B20', 'B96', 'A11'}\\ \tt ARGUMENT: {'The proposed set, while seemingly well-balanced, may not fully utilize the effort limit to maximize importance, as the total effort of the included items appears to be relatively low, leaving potential for adding or substituting items that could increase the overall importance without exceeding the limit, indicating a need to further optimize the selection to achieve the highest possible importance within the given constraints'} 
	  } 
	   } 
	   } 
	 & \\ 
 

    \theutterance \stepcounter{utterance}  

    & \multicolumn{4}{p{0.6\linewidth}}{\cellcolor[rgb]{0.8,1,0.9}{%
	\makecell[{{p{\linewidth}}}]{% 
	  \tt {\tiny [A$\rangle$GM]}  
	 ARGUMENT: {'The refusal of the previous proposal highlights the ongoing challenge of balancing effort and importance, and it is clear that both parties are seeking to maximize the total importance within the given effort limit, suggesting that a mutually beneficial solution can be found by carefully considering the trade-offs between items and their respective efforts and importance, and potentially exploring alternative combinations that better utilize the available effort'}\\ \tt PROPOSAL: {'B93', 'A60', 'B31', 'A82', 'B20'} 
	  } 
	   } 
	   } 
	 & & \\ 
 

    \theutterance \stepcounter{utterance}  

    & & \multicolumn{4}{p{0.6\linewidth}}{\cellcolor[rgb]{0.9,0.9,0.9}{%
	\makecell[{{p{\linewidth}}}]{% 
	  \tt {\tiny [GM$\rangle$B]}  
	 ARGUMENT: {'The refusal of the previous proposal highlights the ongoing challenge of balancing effort and importance, and it is clear that both parties are seeking to maximize the total importance within the given effort limit, suggesting that a mutually beneficial solution can be found by carefully considering the trade-offs between items and their respective efforts and importance, and potentially exploring alternative combinations that better utilize the available effort'}\\ \tt PROPOSAL: {'B93', 'A60', 'B31', 'A82', 'B20'} 
	  } 
	   } 
	   } 
	 & \\ 
 

    \theutterance \stepcounter{utterance}  

    & & & \multicolumn{4}{p{0.6\linewidth}}{\cellcolor[rgb]{1,0.85,0.72}{%
	\makecell[{{p{\linewidth}}}]{% 
	  \tt {\tiny [GM$\langle$B]}  
	 ARGUMENT: {'The new proposal presents an interesting combination of items, with A60 being a notable inclusion due to its significant effort value, and the set as a whole appears to make efficient use of the effort limit, however, it is crucial to evaluate whether the importance gained from including A60 outweighs the potential benefits of adding other items that might offer higher importance at lower effort costs, suggesting a need for a detailed assessment of the trade-offs involved'}\\ \tt PROPOSAL: {'B93', 'B31', 'A82', 'B20', 'B96', 'A21'} 
	  } 
	   } 
	   } 
	  \\ 
 

    \theutterance \stepcounter{utterance}  

    & & \multicolumn{4}{p{0.6\linewidth}}{\cellcolor[rgb]{0.9,0.9,0.9}{%
	\makecell[{{p{\linewidth}}}]{% 
	  \tt {\tiny [A$\langle$GM]}  
	 ARGUMENT: {'The new proposal presents an interesting combination of items, with A60 being a notable inclusion due to its significant effort value, and the set as a whole appears to make efficient use of the effort limit, however, it is crucial to evaluate whether the importance gained from including A60 outweighs the potential benefits of adding other items that might offer higher importance at lower effort costs, suggesting a need for a detailed assessment of the trade-offs involved'}\\ \tt PROPOSAL: {'B93', 'B31', 'A82', 'B20', 'B96', 'A21'} 
	  } 
	   } 
	   } 
	 & \\ 
 

    \theutterance \stepcounter{utterance}  

    & \multicolumn{4}{p{0.6\linewidth}}{\cellcolor[rgb]{0.8,1,0.9}{%
	\makecell[{{p{\linewidth}}}]{% 
	  \tt {\tiny [A$\rangle$GM]}  
	 REFUSE: {'B93', 'B31', 'A82', 'B20', 'B96', 'A21'}\\ \tt ARGUMENT: {'The proposed set includes a diverse range of items, but the total effort of the included items, particularly with A21, may approach or exceed the effort limit, and it is essential to ensure that the selected items not only maximize importance but also fit within the given constraints, suggesting a need to reassess the combination and potentially prioritize items with higher importance-to-effort ratios'} 
	  } 
	   } 
	   } 
	 & & \\ 
 

    \theutterance \stepcounter{utterance}  

    & & \multicolumn{4}{p{0.6\linewidth}}{\cellcolor[rgb]{0.9,0.9,0.9}{%
	\makecell[{{p{\linewidth}}}]{% 
	  \tt {\tiny [GM$\rangle$B]}  
	 REFUSE: {'B93', 'B31', 'A82', 'B20', 'B96', 'A21'}\\ \tt ARGUMENT: {'The proposed set includes a diverse range of items, but the total effort of the included items, particularly with A21, may approach or exceed the effort limit, and it is essential to ensure that the selected items not only maximize importance but also fit within the given constraints, suggesting a need to reassess the combination and potentially prioritize items with higher importance-to-effort ratios'} 
	  } 
	   } 
	   } 
	 & \\ 
 

    \theutterance \stepcounter{utterance}  

    & & & \multicolumn{4}{p{0.6\linewidth}}{\cellcolor[rgb]{1,0.85,0.72}{%
	\makecell[{{p{\linewidth}}}]{% 
	  \tt {\tiny [GM$\langle$B]}  
	 ARGUMENT: {'The refusal of the previous proposal underscores the importance of carefully managing the effort limit while seeking to maximize overall importance, and it is apparent that finding the optimal balance between these two factors is crucial, suggesting that a successful proposal will need to strike a careful balance between including high-importance items and avoiding excessive effort costs, and potentially exploring alternative combinations that better navigate this trade-off'}\\ \tt PROPOSAL: {'B93', 'A82', 'B31', 'B20', 'C08', 'B96'} 
	  } 
	   } 
	   } 
	  \\ 
 

    \theutterance \stepcounter{utterance}  

    & & \multicolumn{4}{p{0.6\linewidth}}{\cellcolor[rgb]{0.9,0.9,0.9}{%
	\makecell[{{p{\linewidth}}}]{% 
	  \tt {\tiny [A$\langle$GM]}  
	 ARGUMENT: {'The refusal of the previous proposal underscores the importance of carefully managing the effort limit while seeking to maximize overall importance, and it is apparent that finding the optimal balance between these two factors is crucial, suggesting that a successful proposal will need to strike a careful balance between including high-importance items and avoiding excessive effort costs, and potentially exploring alternative combinations that better navigate this trade-off'}\\ \tt PROPOSAL: {'B93', 'A82', 'B31', 'B20', 'C08', 'B96'} 
	  } 
	   } 
	   } 
	 & \\ 
 

    \theutterance \stepcounter{utterance}  

    & \multicolumn{4}{p{0.6\linewidth}}{\cellcolor[rgb]{0.8,1,0.9}{%
	\makecell[{{p{\linewidth}}}]{% 
	  \tt {\tiny [A$\rangle$GM]}  
	 AGREE: {'B93', 'A82', 'B31', 'B20', 'C08', 'B96'}\\ \tt ARGUMENT: {'This proposal appears to strike a well-balanced combination of items, effectively utilizing the effort limit while prioritizing items with significant importance, and the inclusion of C08 and B96 adds diversity and potential for higher overall importance, suggesting that this set may represent a mutually beneficial solution that maximizes importance within the given constraints'} 
	  } 
	   } 
	   } 
	 & & \\ 
 

    \theutterance \stepcounter{utterance}  

    & & & \multicolumn{2}{p{0.3\linewidth}}{\cellcolor[rgb]{0.95,0.95,0.95}{%
	\makecell[{{p{\linewidth}}}]{% 
	  \tt {\tiny [GM$|$GM]}  
	 game successful 
	  } 
	   } 
	   } 
	 & & \\ 
 

    \theutterance \stepcounter{utterance}  

    & & & \multicolumn{2}{p{0.3\linewidth}}{\cellcolor[rgb]{0.95,0.95,0.95}{%
	\makecell[{{p{\linewidth}}}]{% 
	  \tt {\tiny [GM$|$GM]}  
	 end game 
	  } 
	   } 
	   } 
	 & & \\ 
 

\end{supertabular}
}

\end{document}
