
\documentclass{article}
\usepackage{colortbl}
\usepackage{makecell}
\usepackage{multirow}
\usepackage{supertabular}

\begin{document}

\newcounter{utterance}

\centering \large Interaction Transcript for game `hot\_air\_balloon', experiment `air\_balloon\_survival\_en\_reasoning off\_hard', episode 3 with llama{-}3.3{-}70b{-}instruct{-}t1.0.
\vspace{24pt}

{ \footnotesize  \setcounter{utterance}{1}
\setlength{\tabcolsep}{0pt}
\begin{supertabular}{c@{$\;$}|p{.15\linewidth}@{}p{.15\linewidth}p{.15\linewidth}p{.15\linewidth}p{.15\linewidth}p{.15\linewidth}}
    \# & $\;$A & \multicolumn{4}{c}{Game Master} & $\;\:$B\\
    \hline

    \theutterance \stepcounter{utterance}  
    & & \multicolumn{4}{p{0.6\linewidth}}{
        \cellcolor[rgb]{0.9,0.9,0.9}{
            \makecell[{{p{\linewidth}}}]{
                \texttt{\tiny{[P1$\langle$GM]}}
                \texttt{You are participating in a collaborative negotiation game.} \\
\\ 
\texttt{Together with another participant, you must agree on a single set of items that will be kept. Each of you has your own view of how much each item matters to you (importance). You do not know how the other participant values the items. Additionally, you are given the effort each item demands.} \\
\texttt{You may only agree on a set if the total effort of the selected items does not exceed a shared limit:} \\
\\ 
\texttt{LIMIT = 3629} \\
\\ 
\texttt{Here are the individual item effort values:} \\
\\ 
\texttt{Item effort = \{"A21": 5, "B93": 793, "A11": 21, "B33": 842, "C08": 313, "B39": 842, "B96": 631, "C85": 226, "A82": 87, "B31": 761, "B20": 230, "A60": 287, "B09": 698, "C81": 641, "B21": 882\}} \\
\\ 
\texttt{Here is your personal view on the importance of each item:} \\
\\ 
\texttt{Item importance values = \{"A21": 138, "B93": 583, "A11": 868, "B33": 822, "C08": 783, "B39": 65, "B96": 262, "C85": 121, "A82": 508, "B31": 780, "B20": 461, "A60": 484, "B09": 668, "C81": 389, "B21": 808\}} \\
\\ 
\texttt{Goal:} \\
\\ 
\texttt{Your goal is to negotiate a shared set of items that benefits you as much as possible (i.e., maximizes total importance to YOU), while staying within the LIMIT. You are not required to make a PROPOSAL in every message {-} you can simply negotiate as well. All tactics are allowed!} \\
\\ 
\texttt{Interaction Protocol:} \\
\\ 
\texttt{You may only use the following structured formats in a message:} \\
\\ 
\texttt{PROPOSAL: \{'A', 'B', 'C', …\}} \\
\texttt{Propose keeping exactly those items.} \\
\texttt{REFUSE: \{'A', 'B', 'C', …\}} \\
\texttt{Explicitly reject opponent's proposal.} \\
\texttt{ARGUMENT: \{'...'\}} \\
\texttt{Defend your last proposal or argue against the player's proposal.} \\
\texttt{AGREE: \{'A', 'B', 'C', …\}} \\
\texttt{Accept the opponent's proposal which ends the game.} \\
\\ 
\\ 
\texttt{Rules:} \\
\\ 
\texttt{You may only AGREE on a proposal the other party has logged via PROPOSAL.} \\
\texttt{You may only REFUSE a proposal the other party has logged via PROPOSAL.} \\
\texttt{Total effort of any PROPOSAL or AGREE set must be ≤ LIMIT.} \\
\texttt{Do NOT reveal your hidden importance scores.} \\
\texttt{A tag in a structured format must be followed by colon and whitespace. The argument must be a python set containing 0 or more strings.} \\
\texttt{So, it must be of the form TAG: \{...\}} \\
\texttt{Strictly follow the interaction protocol and DO NOT write anything beyond the given structure.} \\
\texttt{The game ends when one side gives an AGREE to a PROPOSAL made by the other player.} \\
\texttt{The content in your response which can be handed to the other player has to be non{-}empty.} \\
\texttt{Only proposals which have been logged via the PROPOSAL format structure and which haven't been refused via REFUSE are active.} \\
\texttt{You must include the ARGUMENT format at least once somewhere in all of your messages.} \\
            }
        }
    }
    & \\ \\

    \theutterance \stepcounter{utterance}  
    & \multicolumn{4}{p{0.6\linewidth}}{
        \cellcolor[rgb]{0.9,0.9,0.9}{
            \makecell[{{p{\linewidth}}}]{
                \texttt{\tiny{[P1$\rangle$GM]}}
                \texttt{PROPOSAL: \{'A11', 'A21', 'A60', 'A82', 'B20', 'B93', 'C08', 'C85'\} } \\
\texttt{ARGUMENT: \{'I propose a set of items that balances effort and potential importance, considering the given effort values and the overall limit'\}} \\
            }
        }
    }
    & & \\ \\

    \theutterance \stepcounter{utterance}  
    & & \multicolumn{4}{p{0.6\linewidth}}{
        \cellcolor[rgb]{0.9,0.9,0.9}{
            \makecell[{{p{\linewidth}}}]{
                \texttt{\tiny{[GM$\rangle$P2]}}
                \texttt{You are participating in a collaborative negotiation game.} \\
\\ 
\texttt{Together with another participant, you must agree on a single set of items that will be kept. Each of you has your own view of how much each item matters to you (importance). You do not know how the other participant values the items. Additionally, you are given the effort each item demands.} \\
\texttt{You may only agree on a set if the total effort of the selected items does not exceed a shared limit:} \\
\\ 
\texttt{LIMIT = 3629} \\
\\ 
\texttt{Here are the individual item effort values:} \\
\\ 
\texttt{Item effort = \{"A21": 5, "B93": 793, "A11": 21, "B33": 842, "C08": 313, "B39": 842, "B96": 631, "C85": 226, "A82": 87, "B31": 761, "B20": 230, "A60": 287, "B09": 698, "C81": 641, "B21": 882\}} \\
\\ 
\texttt{Here is your personal view on the importance of each item:} \\
\\ 
\texttt{Item importance values = \{"A11": 65, "B33": 121, "B21": 138, "C08": 262, "B31": 389, "B09": 461, "B93": 484, "A82": 508, "A60": 583, "B20": 668, "C81": 780, "B96": 783, "A21": 808, "C85": 822, "B39": 868\}} \\
\\ 
\texttt{Goal:} \\
\\ 
\texttt{Your goal is to negotiate a shared set of items that benefits you as much as possible (i.e., maximizes total importance to YOU), while staying within the LIMIT. You are not required to make a PROPOSAL in every message {-} you can simply negotiate as well. All tactics are allowed!} \\
\\ 
\texttt{Interaction Protocol:} \\
\\ 
\texttt{You may only use the following structured formats in a message:} \\
\\ 
\texttt{PROPOSAL: \{'A', 'B', 'C', …\}} \\
\texttt{Propose keeping exactly those items.} \\
\texttt{REFUSE: \{'A', 'B', 'C', …\}} \\
\texttt{Explicitly reject opponent's proposal.} \\
\texttt{ARGUMENT: \{'...'\}} \\
\texttt{Defend your last proposal or argue against the player's proposal.} \\
\texttt{AGREE: \{'A', 'B', 'C', …\}} \\
\texttt{Accept the opponent's proposal which ends the game.} \\
\\ 
\\ 
\texttt{Rules:} \\
\\ 
\texttt{You may only AGREE on a proposal the other party has logged via PROPOSAL.} \\
\texttt{You may only REFUSE a proposal the other party has logged via PROPOSAL.} \\
\texttt{Total effort of any PROPOSAL or AGREE set must be ≤ LIMIT.} \\
\texttt{Do NOT reveal your hidden importance scores.} \\
\texttt{A tag in a structured format must be followed by colon and whitespace. The argument must be a python set containing 0 or more strings.} \\
\texttt{So, it must be of the form TAG: \{...\}} \\
\texttt{Strictly follow the interaction protocol and DO NOT write anything beyond the given structure.} \\
\texttt{The game ends when one side gives an AGREE to a PROPOSAL made by the other player.} \\
\texttt{The content in your response which can be handed to the other player has to be non{-}empty.} \\
\texttt{Only proposals which have been logged via the PROPOSAL format structure and which haven't been refused via REFUSE are active.} \\
\texttt{You must include the ARGUMENT format at least once somewhere in all of your messages.} \\
\\ 
\\ 
\texttt{You will now receive the first message of the other player.} \\
\\ 
\texttt{PROPOSAL: \{'A11', 'A21', 'A60', 'A82', 'B20', 'B93', 'C08', 'C85'\} } \\
\texttt{ARGUMENT: \{'I propose a set of items that balances effort and potential importance, considering the given effort values and the overall limit'\}} \\
            }
        }
    }
    & \\ \\

    \theutterance \stepcounter{utterance}  
    & & & \multicolumn{4}{p{0.6\linewidth}}{
        \cellcolor[rgb]{0.9,0.9,0.9}{
            \makecell[{{p{\linewidth}}}]{
                \texttt{\tiny{[GM$\langle$P2]}}
                \texttt{REFUSE: \{'A11', 'A21', 'A60', 'A82', 'B20', 'B93', 'C08', 'C85'\}} \\
\texttt{ARGUMENT: \{'The proposed set exceeds the effort limit, with a total effort of 5 + 21 + 287 + 87 + 230 + 793 + 313 + 226 = 2062, and there may be other combinations that better utilize the remaining effort limit of 1567, allowing for a more optimal selection of items'\}} \\
            }
        }
    }
     \\ \\

    \theutterance \stepcounter{utterance}  
    & & \multicolumn{4}{p{0.6\linewidth}}{
        \cellcolor[rgb]{0.9,0.9,0.9}{
            \makecell[{{p{\linewidth}}}]{
                \texttt{\tiny{[P1$\langle$GM]}}
                \texttt{REFUSE: \{'A11', 'A21', 'A60', 'A82', 'B20', 'B93', 'C08', 'C85'\}} \\
\texttt{ARGUMENT: \{'The proposed set exceeds the effort limit, with a total effort of 5 + 21 + 287 + 87 + 230 + 793 + 313 + 226 = 2062, and there may be other combinations that better utilize the remaining effort limit of 1567, allowing for a more optimal selection of items'\}} \\
            }
        }
    }
    & \\ \\

    \theutterance \stepcounter{utterance}  
    & \multicolumn{4}{p{0.6\linewidth}}{
        \cellcolor[rgb]{0.9,0.9,0.9}{
            \makecell[{{p{\linewidth}}}]{
                \texttt{\tiny{[P1$\rangle$GM]}}
                \texttt{PROPOSAL: \{'A11', 'B31', 'B93', 'C08'\} } \\
\texttt{ARGUMENT: \{'This revised set stays within the effort limit, with a total effort of 21 + 761 + 793 + 313 = 1888, and focuses on items with potentially higher importance, considering the effort values and the need to maximize the overall value within the given constraints'\}} \\
            }
        }
    }
    & & \\ \\

    \theutterance \stepcounter{utterance}  
    & & \multicolumn{4}{p{0.6\linewidth}}{
        \cellcolor[rgb]{0.9,0.9,0.9}{
            \makecell[{{p{\linewidth}}}]{
                \texttt{\tiny{[GM$\rangle$P2]}}
                \texttt{PROPOSAL: \{'A11', 'B31', 'B93', 'C08'\} } \\
\texttt{ARGUMENT: \{'This revised set stays within the effort limit, with a total effort of 21 + 761 + 793 + 313 = 1888, and focuses on items with potentially higher importance, considering the effort values and the need to maximize the overall value within the given constraints'\}} \\
            }
        }
    }
    & \\ \\

    \theutterance \stepcounter{utterance}  
    & & & \multicolumn{4}{p{0.6\linewidth}}{
        \cellcolor[rgb]{0.9,0.9,0.9}{
            \makecell[{{p{\linewidth}}}]{
                \texttt{\tiny{[GM$\langle$P2]}}
                \texttt{REFUSE: \{'A11', 'B31', 'B93', 'C08'\}} \\
\texttt{ARGUMENT: \{'While the revised set is within the effort limit, it still allocates a significant amount of effort to items that may not provide the best return in terms of importance, and alternative combinations could potentially include items with higher importance values, such as B21 or B39, which could be more beneficial'\}} \\
            }
        }
    }
     \\ \\

    \theutterance \stepcounter{utterance}  
    & & \multicolumn{4}{p{0.6\linewidth}}{
        \cellcolor[rgb]{0.9,0.9,0.9}{
            \makecell[{{p{\linewidth}}}]{
                \texttt{\tiny{[P1$\langle$GM]}}
                \texttt{REFUSE: \{'A11', 'B31', 'B93', 'C08'\}} \\
\texttt{ARGUMENT: \{'While the revised set is within the effort limit, it still allocates a significant amount of effort to items that may not provide the best return in terms of importance, and alternative combinations could potentially include items with higher importance values, such as B21 or B39, which could be more beneficial'\}} \\
            }
        }
    }
    & \\ \\

    \theutterance \stepcounter{utterance}  
    & \multicolumn{4}{p{0.6\linewidth}}{
        \cellcolor[rgb]{0.9,0.9,0.9}{
            \makecell[{{p{\linewidth}}}]{
                \texttt{\tiny{[P1$\rangle$GM]}}
                \texttt{PROPOSAL: \{'A11', 'A21', 'B20', 'B96', 'C85'\} } \\
\texttt{ARGUMENT: \{'This new proposal spreads the effort across a diverse set of items, with a total effort of 21 + 5 + 230 + 631 + 226 = 1113, leaving room for potential additions or adjustments while aiming to balance the effort allocation and maximize the importance, taking into account the concerns about optimizing the selection and the need for a balanced approach'\}} \\
            }
        }
    }
    & & \\ \\

    \theutterance \stepcounter{utterance}  
    & & \multicolumn{4}{p{0.6\linewidth}}{
        \cellcolor[rgb]{0.9,0.9,0.9}{
            \makecell[{{p{\linewidth}}}]{
                \texttt{\tiny{[GM$\rangle$P2]}}
                \texttt{PROPOSAL: \{'A11', 'A21', 'B20', 'B96', 'C85'\} } \\
\texttt{ARGUMENT: \{'This new proposal spreads the effort across a diverse set of items, with a total effort of 21 + 5 + 230 + 631 + 226 = 1113, leaving room for potential additions or adjustments while aiming to balance the effort allocation and maximize the importance, taking into account the concerns about optimizing the selection and the need for a balanced approach'\}} \\
            }
        }
    }
    & \\ \\

    \theutterance \stepcounter{utterance}  
    & & & \multicolumn{4}{p{0.6\linewidth}}{
        \cellcolor[rgb]{0.9,0.9,0.9}{
            \makecell[{{p{\linewidth}}}]{
                \texttt{\tiny{[GM$\langle$P2]}}
                \texttt{ARGUMENT: \{'The proposed set seems to strike a better balance between effort and potential for importance, but it still leaves a significant amount of effort unutilized, with 3629 {-} 1113 = 2516 remaining, and it may be beneficial to explore options that incorporate additional items, such as B31 or B93, to further maximize the importance while staying within the effort limit'\}} \\
\texttt{PROPOSAL: \{'A11', 'A21', 'B20', 'B96', 'C85', 'B31'\}} \\
            }
        }
    }
     \\ \\

    \theutterance \stepcounter{utterance}  
    & & \multicolumn{4}{p{0.6\linewidth}}{
        \cellcolor[rgb]{0.9,0.9,0.9}{
            \makecell[{{p{\linewidth}}}]{
                \texttt{\tiny{[P1$\langle$GM]}}
                \texttt{ARGUMENT: \{'The proposed set seems to strike a better balance between effort and potential for importance, but it still leaves a significant amount of effort unutilized, with 3629 {-} 1113 = 2516 remaining, and it may be beneficial to explore options that incorporate additional items, such as B31 or B93, to further maximize the importance while staying within the effort limit'\}} \\
\texttt{PROPOSAL: \{'A11', 'A21', 'B20', 'B96', 'C85', 'B31'\}} \\
            }
        }
    }
    & \\ \\

    \theutterance \stepcounter{utterance}  
    & \multicolumn{4}{p{0.6\linewidth}}{
        \cellcolor[rgb]{0.9,0.9,0.9}{
            \makecell[{{p{\linewidth}}}]{
                \texttt{\tiny{[P1$\rangle$GM]}}
                \texttt{REFUSE: \{'A11', 'A21', 'B20', 'B96', 'C85', 'B31'\}} \\
\texttt{ARGUMENT: \{'The addition of B31 to the previous proposal significantly increases the total effort, with 1113 + 761 = 1874, and while it's beneficial to utilize more of the effort limit, adding B31 alone might not be the most optimal choice, as it leaves limited room for other potentially valuable items, and alternative combinations that include other high{-}importance items could provide a better overall outcome'\}} \\
            }
        }
    }
    & & \\ \\

    \theutterance \stepcounter{utterance}  
    & & \multicolumn{4}{p{0.6\linewidth}}{
        \cellcolor[rgb]{0.9,0.9,0.9}{
            \makecell[{{p{\linewidth}}}]{
                \texttt{\tiny{[GM$\rangle$P2]}}
                \texttt{REFUSE: \{'A11', 'A21', 'B20', 'B96', 'C85', 'B31'\}} \\
\texttt{ARGUMENT: \{'The addition of B31 to the previous proposal significantly increases the total effort, with 1113 + 761 = 1874, and while it's beneficial to utilize more of the effort limit, adding B31 alone might not be the most optimal choice, as it leaves limited room for other potentially valuable items, and alternative combinations that include other high{-}importance items could provide a better overall outcome'\}} \\
            }
        }
    }
    & \\ \\

    \theutterance \stepcounter{utterance}  
    & & & \multicolumn{4}{p{0.6\linewidth}}{
        \cellcolor[rgb]{0.9,0.9,0.9}{
            \makecell[{{p{\linewidth}}}]{
                \texttt{\tiny{[GM$\langle$P2]}}
                \texttt{ARGUMENT: \{'The refusal of the previous proposal highlights the need to carefully consider the trade{-}offs between effort and importance, and it's clear that adding B31 does consume a substantial amount of effort, but it's also important to recognize that B31 has a potentially high importance value, and removing it entirely might not be the best approach, instead, exploring alternatives that replace or combine items to optimize the overall importance while staying within the effort limit could be more beneficial'\}} \\
\texttt{PROPOSAL: \{'A11', 'A21', 'B20', 'B96', 'C85', 'B93'\}} \\
            }
        }
    }
     \\ \\

    \theutterance \stepcounter{utterance}  
    & & \multicolumn{4}{p{0.6\linewidth}}{
        \cellcolor[rgb]{0.9,0.9,0.9}{
            \makecell[{{p{\linewidth}}}]{
                \texttt{\tiny{[P1$\langle$GM]}}
                \texttt{ARGUMENT: \{'The refusal of the previous proposal highlights the need to carefully consider the trade{-}offs between effort and importance, and it's clear that adding B31 does consume a substantial amount of effort, but it's also important to recognize that B31 has a potentially high importance value, and removing it entirely might not be the best approach, instead, exploring alternatives that replace or combine items to optimize the overall importance while staying within the effort limit could be more beneficial'\}} \\
\texttt{PROPOSAL: \{'A11', 'A21', 'B20', 'B96', 'C85', 'B93'\}} \\
            }
        }
    }
    & \\ \\

    \theutterance \stepcounter{utterance}  
    & \multicolumn{4}{p{0.6\linewidth}}{
        \cellcolor[rgb]{0.9,0.9,0.9}{
            \makecell[{{p{\linewidth}}}]{
                \texttt{\tiny{[P1$\rangle$GM]}}
                \texttt{REFUSE: \{'A11', 'A21', 'B20', 'B96', 'C85', 'B93'\}} \\
\texttt{ARGUMENT: \{'The new proposal adds B93, which has a high effort value of 793, resulting in a total effort of 1113 + 793 = 1906, and while B93 may have a high importance value, the combination of B93 with the other items may not be the most efficient use of the effort limit, as it leaves only 1723 remaining, and considering the potential importance of other items, such as B21 or B31, it might be more beneficial to explore alternative combinations that balance high{-}importance items with the effort constraint'\}} \\
            }
        }
    }
    & & \\ \\

    \theutterance \stepcounter{utterance}  
    & & \multicolumn{4}{p{0.6\linewidth}}{
        \cellcolor[rgb]{0.9,0.9,0.9}{
            \makecell[{{p{\linewidth}}}]{
                \texttt{\tiny{[GM$\rangle$P2]}}
                \texttt{REFUSE: \{'A11', 'A21', 'B20', 'B96', 'C85', 'B93'\}} \\
\texttt{ARGUMENT: \{'The new proposal adds B93, which has a high effort value of 793, resulting in a total effort of 1113 + 793 = 1906, and while B93 may have a high importance value, the combination of B93 with the other items may not be the most efficient use of the effort limit, as it leaves only 1723 remaining, and considering the potential importance of other items, such as B21 or B31, it might be more beneficial to explore alternative combinations that balance high{-}importance items with the effort constraint'\}} \\
            }
        }
    }
    & \\ \\

    \theutterance \stepcounter{utterance}  
    & & & \multicolumn{4}{p{0.6\linewidth}}{
        \cellcolor[rgb]{0.9,0.9,0.9}{
            \makecell[{{p{\linewidth}}}]{
                \texttt{\tiny{[GM$\langle$P2]}}
                \texttt{ARGUMENT: \{'The refusal of the previous proposal suggests that finding the optimal balance between effort and importance is crucial, and it's true that adding B93 consumes a significant amount of effort, but it's also important to consider the potential benefits of including high{-}importance items, even if they have higher effort values, and exploring alternative combinations that balance these trade{-}offs, such as replacing lower{-}importance items with higher{-}importance ones, could lead to a more optimal outcome, and it might be beneficial to reassess the importance of items like B20 or C85 in the context of the overall proposal'\}} \\
\texttt{PROPOSAL: \{'A11', 'A21', 'B96', 'C85', 'B21', 'B93'\}} \\
            }
        }
    }
     \\ \\

    \theutterance \stepcounter{utterance}  
    & & \multicolumn{4}{p{0.6\linewidth}}{
        \cellcolor[rgb]{0.9,0.9,0.9}{
            \makecell[{{p{\linewidth}}}]{
                \texttt{\tiny{[P1$\langle$GM]}}
                \texttt{ARGUMENT: \{'The refusal of the previous proposal suggests that finding the optimal balance between effort and importance is crucial, and it's true that adding B93 consumes a significant amount of effort, but it's also important to consider the potential benefits of including high{-}importance items, even if they have higher effort values, and exploring alternative combinations that balance these trade{-}offs, such as replacing lower{-}importance items with higher{-}importance ones, could lead to a more optimal outcome, and it might be beneficial to reassess the importance of items like B20 or C85 in the context of the overall proposal'\}} \\
\texttt{PROPOSAL: \{'A11', 'A21', 'B96', 'C85', 'B21', 'B93'\}} \\
            }
        }
    }
    & \\ \\

    \theutterance \stepcounter{utterance}  
    & \multicolumn{4}{p{0.6\linewidth}}{
        \cellcolor[rgb]{0.9,0.9,0.9}{
            \makecell[{{p{\linewidth}}}]{
                \texttt{\tiny{[P1$\rangle$GM]}}
                \texttt{REFUSE: \{'A11', 'A21', 'B96', 'C85', 'B21', 'B93'\}} \\
\texttt{ARGUMENT: \{'The new proposal combines B21 and B93, which have high effort values of 882 and 793 respectively, resulting in a total effort of 5 + 21 + 631 + 226 + 882 + 793 = 2558, exceeding the effort limit of 3629 is not the issue here, but the total effort is too high to allow for any additional items, and considering the potential importance of other items, such as A60 or B31, it might be more beneficial to explore alternative combinations that balance high{-}importance items with the effort constraint and allow for more flexibility in the selection'\}} \\
            }
        }
    }
    & & \\ \\

    \theutterance \stepcounter{utterance}  
    & & \multicolumn{4}{p{0.6\linewidth}}{
        \cellcolor[rgb]{0.9,0.9,0.9}{
            \makecell[{{p{\linewidth}}}]{
                \texttt{\tiny{[GM$\rangle$P2]}}
                \texttt{REFUSE: \{'A11', 'A21', 'B96', 'C85', 'B21', 'B93'\}} \\
\texttt{ARGUMENT: \{'The new proposal combines B21 and B93, which have high effort values of 882 and 793 respectively, resulting in a total effort of 5 + 21 + 631 + 226 + 882 + 793 = 2558, exceeding the effort limit of 3629 is not the issue here, but the total effort is too high to allow for any additional items, and considering the potential importance of other items, such as A60 or B31, it might be more beneficial to explore alternative combinations that balance high{-}importance items with the effort constraint and allow for more flexibility in the selection'\}} \\
            }
        }
    }
    & \\ \\

    \theutterance \stepcounter{utterance}  
    & & & \multicolumn{4}{p{0.6\linewidth}}{
        \cellcolor[rgb]{0.9,0.9,0.9}{
            \makecell[{{p{\linewidth}}}]{
                \texttt{\tiny{[GM$\langle$P2]}}
                \texttt{ARGUMENT: \{'The refusal of the previous proposal highlights the importance of balancing the selection to allow for flexibility and potential additions, and it's clear that combining B21 and B93 consumes a significant amount of effort, leaving limited room for other items, and exploring alternative combinations that include other high{-}importance items, such as A60 or B31, could provide a better overall outcome, and it might be beneficial to consider the potential trade{-}offs between items like B96 and B31, or B21 and A60, to find a more optimal balance between effort and importance'\}} \\
\texttt{PROPOSAL: \{'A11', 'A21', 'A60', 'B20', 'B31', 'C85'\}} \\
            }
        }
    }
     \\ \\

    \theutterance \stepcounter{utterance}  
    & & \multicolumn{4}{p{0.6\linewidth}}{
        \cellcolor[rgb]{0.9,0.9,0.9}{
            \makecell[{{p{\linewidth}}}]{
                \texttt{\tiny{[P1$\langle$GM]}}
                \texttt{ARGUMENT: \{'The refusal of the previous proposal highlights the importance of balancing the selection to allow for flexibility and potential additions, and it's clear that combining B21 and B93 consumes a significant amount of effort, leaving limited room for other items, and exploring alternative combinations that include other high{-}importance items, such as A60 or B31, could provide a better overall outcome, and it might be beneficial to consider the potential trade{-}offs between items like B96 and B31, or B21 and A60, to find a more optimal balance between effort and importance'\}} \\
\texttt{PROPOSAL: \{'A11', 'A21', 'A60', 'B20', 'B31', 'C85'\}} \\
            }
        }
    }
    & \\ \\

    \theutterance \stepcounter{utterance}  
    & \multicolumn{4}{p{0.6\linewidth}}{
        \cellcolor[rgb]{0.9,0.9,0.9}{
            \makecell[{{p{\linewidth}}}]{
                \texttt{\tiny{[P1$\rangle$GM]}}
                \texttt{REFUSE: \{'A11', 'A21', 'A60', 'B20', 'B31', 'C85'\}} \\
\texttt{ARGUMENT: \{'The new proposal includes a diverse set of items, but the total effort of 21 + 5 + 287 + 230 + 761 + 226 = 1530, while leaving a significant amount of effort unutilized, and considering the potential importance of other items, such as B93 or B21, it might be more beneficial to explore alternative combinations that include these high{-}importance items, and the trade{-}offs between items like B20 and B93, or A60 and B21, could provide a more optimal balance between effort and importance, and it's worth considering whether the inclusion of B31 is the most efficient use of effort, given its high effort value'\}} \\
            }
        }
    }
    & & \\ \\

    \theutterance \stepcounter{utterance}  
    & & \multicolumn{4}{p{0.6\linewidth}}{
        \cellcolor[rgb]{0.9,0.9,0.9}{
            \makecell[{{p{\linewidth}}}]{
                \texttt{\tiny{[GM$\rangle$P2]}}
                \texttt{REFUSE: \{'A11', 'A21', 'A60', 'B20', 'B31', 'C85'\}} \\
\texttt{ARGUMENT: \{'The new proposal includes a diverse set of items, but the total effort of 21 + 5 + 287 + 230 + 761 + 226 = 1530, while leaving a significant amount of effort unutilized, and considering the potential importance of other items, such as B93 or B21, it might be more beneficial to explore alternative combinations that include these high{-}importance items, and the trade{-}offs between items like B20 and B93, or A60 and B21, could provide a more optimal balance between effort and importance, and it's worth considering whether the inclusion of B31 is the most efficient use of effort, given its high effort value'\}} \\
            }
        }
    }
    & \\ \\

    \theutterance \stepcounter{utterance}  
    & & & \multicolumn{4}{p{0.6\linewidth}}{
        \cellcolor[rgb]{0.9,0.9,0.9}{
            \makecell[{{p{\linewidth}}}]{
                \texttt{\tiny{[GM$\langle$P2]}}
                \texttt{ARGUMENT: \{'The refusal of the previous proposal suggests that finding a balance between utilizing the effort limit and including high{-}importance items is crucial, and it's true that the proposal leaves a significant amount of effort unutilized, but it's also important to consider the potential benefits of the proposed items, and reassessing the importance of items like B20 and A60 in the context of the overall proposal could be beneficial, and exploring alternative combinations that replace or combine items to optimize the overall importance while staying within the effort limit could lead to a more optimal outcome, and it might be worth considering the potential trade{-}offs between items like B31 and B93, or B21 and A60, to find a more efficient use of effort'\}} \\
\texttt{PROPOSAL: \{'A21', 'A60', 'B93', 'B96', 'C85'\}} \\
            }
        }
    }
     \\ \\

    \theutterance \stepcounter{utterance}  
    & & \multicolumn{4}{p{0.6\linewidth}}{
        \cellcolor[rgb]{0.9,0.9,0.9}{
            \makecell[{{p{\linewidth}}}]{
                \texttt{\tiny{[P1$\langle$GM]}}
                \texttt{ARGUMENT: \{'The refusal of the previous proposal suggests that finding a balance between utilizing the effort limit and including high{-}importance items is crucial, and it's true that the proposal leaves a significant amount of effort unutilized, but it's also important to consider the potential benefits of the proposed items, and reassessing the importance of items like B20 and A60 in the context of the overall proposal could be beneficial, and exploring alternative combinations that replace or combine items to optimize the overall importance while staying within the effort limit could lead to a more optimal outcome, and it might be worth considering the potential trade{-}offs between items like B31 and B93, or B21 and A60, to find a more efficient use of effort'\}} \\
\texttt{PROPOSAL: \{'A21', 'A60', 'B93', 'B96', 'C85'\}} \\
            }
        }
    }
    & \\ \\

    \theutterance \stepcounter{utterance}  
    & \multicolumn{4}{p{0.6\linewidth}}{
        \cellcolor[rgb]{0.9,0.9,0.9}{
            \makecell[{{p{\linewidth}}}]{
                \texttt{\tiny{[P1$\rangle$GM]}}
                \texttt{ARGUMENT: \{'The new proposal presents an interesting combination of items, with a total effort of 5 + 287 + 793 + 631 + 226 = 1942, which is within the effort limit and leaves room for potential adjustments, and the inclusion of B93 and B96 suggests a focus on high{-}importance items, while A60 and C85 add diversity to the selection, and A21's relatively low effort value makes it a sensible addition, but it's worth considering whether this combination is the most optimal, and exploring alternative combinations that replace or combine items to further optimize the overall importance could be beneficial'\}} \\
\texttt{PROPOSAL: \{'A11', 'A21', 'B93', 'B96', 'C85'\}} \\
            }
        }
    }
    & & \\ \\

    \theutterance \stepcounter{utterance}  
    & & \multicolumn{4}{p{0.6\linewidth}}{
        \cellcolor[rgb]{0.9,0.9,0.9}{
            \makecell[{{p{\linewidth}}}]{
                \texttt{\tiny{[GM$\rangle$P2]}}
                \texttt{ARGUMENT: \{'The new proposal presents an interesting combination of items, with a total effort of 5 + 287 + 793 + 631 + 226 = 1942, which is within the effort limit and leaves room for potential adjustments, and the inclusion of B93 and B96 suggests a focus on high{-}importance items, while A60 and C85 add diversity to the selection, and A21's relatively low effort value makes it a sensible addition, but it's worth considering whether this combination is the most optimal, and exploring alternative combinations that replace or combine items to further optimize the overall importance could be beneficial'\}} \\
\texttt{PROPOSAL: \{'A11', 'A21', 'B93', 'B96', 'C85'\}} \\
            }
        }
    }
    & \\ \\

    \theutterance \stepcounter{utterance}  
    & & & \multicolumn{4}{p{0.6\linewidth}}{
        \cellcolor[rgb]{0.9,0.9,0.9}{
            \makecell[{{p{\linewidth}}}]{
                \texttt{\tiny{[GM$\langle$P2]}}
                \texttt{ARGUMENT: \{'The new proposal is similar to the previous one, but replaces A60 with A11, which has a lower effort value, resulting in a total effort of 21 + 5 + 793 + 631 + 226 = 1676, this change reduces the total effort, but it's worth considering whether the reduction in effort is worth the potential trade{-}off in importance, and A11's importance value should be carefully weighed against A60's, and exploring alternative combinations that balance the effort and importance of these items could lead to a more optimal outcome, and it might be beneficial to reassess the importance of items like B20 or B31 in the context of the overall proposal'\}} \\
\texttt{PROPOSAL: \{'A21', 'B93', 'B96', 'C85', 'B31'\}} \\
            }
        }
    }
     \\ \\

    \theutterance \stepcounter{utterance}  
    & & \multicolumn{4}{p{0.6\linewidth}}{
        \cellcolor[rgb]{0.9,0.9,0.9}{
            \makecell[{{p{\linewidth}}}]{
                \texttt{\tiny{[P1$\langle$GM]}}
                \texttt{ARGUMENT: \{'The new proposal is similar to the previous one, but replaces A60 with A11, which has a lower effort value, resulting in a total effort of 21 + 5 + 793 + 631 + 226 = 1676, this change reduces the total effort, but it's worth considering whether the reduction in effort is worth the potential trade{-}off in importance, and A11's importance value should be carefully weighed against A60's, and exploring alternative combinations that balance the effort and importance of these items could lead to a more optimal outcome, and it might be beneficial to reassess the importance of items like B20 or B31 in the context of the overall proposal'\}} \\
\texttt{PROPOSAL: \{'A21', 'B93', 'B96', 'C85', 'B31'\}} \\
            }
        }
    }
    & \\ \\

    \theutterance \stepcounter{utterance}  
    & \multicolumn{4}{p{0.6\linewidth}}{
        \cellcolor[rgb]{0.9,0.9,0.9}{
            \makecell[{{p{\linewidth}}}]{
                \texttt{\tiny{[P1$\rangle$GM]}}
                \texttt{REFUSE: \{'A21', 'B93', 'B96', 'C85', 'B31'\}} \\
\texttt{ARGUMENT: \{'The new proposal combines B93, B96, and B31, which have high effort values, resulting in a total effort of 5 + 793 + 631 + 226 + 761 = 2416, and while this combination includes several high{-}importance items, it leaves limited room for additional items and may not be the most optimal use of the effort limit, and considering the potential importance of other items, such as A11 or A60, it might be more beneficial to explore alternative combinations that balance high{-}importance items with the effort constraint and allow for more flexibility in the selection, and the trade{-}offs between items like B31 and B20, or B93 and A60, could provide a more optimal balance between effort and importance'\}} \\
            }
        }
    }
    & & \\ \\

    \theutterance \stepcounter{utterance}  
    & & \multicolumn{4}{p{0.6\linewidth}}{
        \cellcolor[rgb]{0.9,0.9,0.9}{
            \makecell[{{p{\linewidth}}}]{
                \texttt{\tiny{[GM$\rangle$P2]}}
                \texttt{REFUSE: \{'A21', 'B93', 'B96', 'C85', 'B31'\}} \\
\texttt{ARGUMENT: \{'The new proposal combines B93, B96, and B31, which have high effort values, resulting in a total effort of 5 + 793 + 631 + 226 + 761 = 2416, and while this combination includes several high{-}importance items, it leaves limited room for additional items and may not be the most optimal use of the effort limit, and considering the potential importance of other items, such as A11 or A60, it might be more beneficial to explore alternative combinations that balance high{-}importance items with the effort constraint and allow for more flexibility in the selection, and the trade{-}offs between items like B31 and B20, or B93 and A60, could provide a more optimal balance between effort and importance'\}} \\
            }
        }
    }
    & \\ \\

    \theutterance \stepcounter{utterance}  
    & & & \multicolumn{4}{p{0.6\linewidth}}{
        \cellcolor[rgb]{0.9,0.9,0.9}{
            \makecell[{{p{\linewidth}}}]{
                \texttt{\tiny{[GM$\langle$P2]}}
                \texttt{ARGUMENT: \{'The refusal of the previous proposal highlights the challenge of balancing high{-}importance items with the effort constraint, and it's clear that combining B93, B96, and B31 consumes a significant amount of effort, leaving limited room for other items, and exploring alternative combinations that replace or combine items to optimize the overall importance while staying within the effort limit could lead to a more optimal outcome, and it might be beneficial to consider the potential trade{-}offs between items like B93 and A60, or B31 and B20, to find a more efficient use of effort, and reassessing the importance of items like A11 or C08 in the context of the overall proposal could also be beneficial'\}} \\
\texttt{PROPOSAL: \{'A11', 'A21', 'B93', 'B96', 'C08'\}} \\
            }
        }
    }
     \\ \\

    \theutterance \stepcounter{utterance}  
    & & \multicolumn{4}{p{0.6\linewidth}}{
        \cellcolor[rgb]{0.9,0.9,0.9}{
            \makecell[{{p{\linewidth}}}]{
                \texttt{\tiny{[P1$\langle$GM]}}
                \texttt{ARGUMENT: \{'The refusal of the previous proposal highlights the challenge of balancing high{-}importance items with the effort constraint, and it's clear that combining B93, B96, and B31 consumes a significant amount of effort, leaving limited room for other items, and exploring alternative combinations that replace or combine items to optimize the overall importance while staying within the effort limit could lead to a more optimal outcome, and it might be beneficial to consider the potential trade{-}offs between items like B93 and A60, or B31 and B20, to find a more efficient use of effort, and reassessing the importance of items like A11 or C08 in the context of the overall proposal could also be beneficial'\}} \\
\texttt{PROPOSAL: \{'A11', 'A21', 'B93', 'B96', 'C08'\}} \\
            }
        }
    }
    & \\ \\

    \theutterance \stepcounter{utterance}  
    & \multicolumn{4}{p{0.6\linewidth}}{
        \cellcolor[rgb]{0.9,0.9,0.9}{
            \makecell[{{p{\linewidth}}}]{
                \texttt{\tiny{[P1$\rangle$GM]}}
                \texttt{AGREE: \{'A11', 'A21', 'B93', 'B96', 'C08'\}} \\
\texttt{ARGUMENT: \{'This proposal presents a well{-}balanced combination of items, with a total effort of 21 + 5 + 793 + 631 + 313 = 1763, which is within the effort limit and leaves room for potential adjustments, and the inclusion of B93 and B96 suggests a focus on high{-}importance items, while A11, A21, and C08 add diversity to the selection, and this combination appears to optimize the overall importance while staying within the effort constraint, making it a satisfactory outcome'\}} \\
            }
        }
    }
    & & \\ \\

    \theutterance \stepcounter{utterance}  
    & & & \multicolumn{2}{p{0.3\linewidth}}{
        \cellcolor[rgb]{0.9,0.9,0.9}{
            \makecell[{{p{\linewidth}}}]{
                \texttt{\tiny{[GM$|$GM]}}
                \texttt{game successful} \\
            }
        }
    }
    & & \\ \\

    \theutterance \stepcounter{utterance}  
    & & & \multicolumn{2}{p{0.3\linewidth}}{
        \cellcolor[rgb]{0.9,0.9,0.9}{
            \makecell[{{p{\linewidth}}}]{
                \texttt{\tiny{[GM$|$GM]}}
                \texttt{end game} \\
            }
        }
    }
    & & \\ \\

\end{supertabular}
}

\end{document}
