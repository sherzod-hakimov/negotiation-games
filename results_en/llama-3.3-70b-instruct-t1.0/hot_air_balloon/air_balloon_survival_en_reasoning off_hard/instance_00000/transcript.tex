
\documentclass{article}
\usepackage{colortbl}
\usepackage{makecell}
\usepackage{multirow}
\usepackage{supertabular}

\begin{document}

\newcounter{utterance}

\twocolumn

{ \footnotesize  \setcounter{utterance}{1}
\setlength{\tabcolsep}{0pt}
\begin{supertabular}{c@{$\;$}|p{.15\linewidth}@{}p{.15\linewidth}p{.15\linewidth}p{.15\linewidth}p{.15\linewidth}p{.15\linewidth}}

    \# & $\;$A & \multicolumn{4}{c}{Game Master} & $\;\:$B\\
    \hline 

    \theutterance \stepcounter{utterance}  

    & & \multicolumn{4}{p{0.6\linewidth}}{\cellcolor[rgb]{0.9,0.9,0.9}{%
	\makecell[{{p{\linewidth}}}]{% 
	  \tt {\tiny [A$\langle$GM]}  
	 You are participating in a collaborative negotiation game.\\ \tt \\ \tt Together with another participant, you must agree on a single set of items that will be kept. Each of you has your own view of how much each item matters to you (importance). You do not know how the other participant values the items. Additionally, you are given the effort each item demands.\\ \tt You may only agree on a set if the total effort of the selected items does not exceed a shared limit:\\ \tt \\ \tt LIMIT = 3683\\ \tt \\ \tt Here are the individual item effort values:\\ \tt \\ \tt Item effort = {"A21": 548, "B93": 154, "A11": 60, "B33": 517, "C08": 334, "B39": 542, "B96": 707, "C85": 139, "A82": 661, "B31": 780, "B20": 832, "A60": 913, "B09": 219, "C81": 323, "B21": 638}\\ \tt \\ \tt Here is your personal view on the importance of each item:\\ \tt \\ \tt Item importance values = {"A21": 138, "B93": 583, "A11": 868, "B33": 822, "C08": 783, "B39": 65, "B96": 262, "C85": 121, "A82": 508, "B31": 780, "B20": 461, "A60": 484, "B09": 668, "C81": 389, "B21": 808}\\ \tt \\ \tt Goal:\\ \tt \\ \tt Your goal is to negotiate a shared set of items that benefits you as much as possible (i.e., maximizes total importance to YOU), while staying within the LIMIT. You are not required to make a PROPOSAL in every message - you can simply negotiate as well. All tactics are allowed!\\ \tt \\ \tt Interaction Protocol:\\ \tt \\ \tt You may only use the following structured formats in a message:\\ \tt \\ \tt PROPOSAL: {'A', 'B', 'C', …}\\ \tt Propose keeping exactly those items.\\ \tt REFUSE: {'A', 'B', 'C', …}\\ \tt Explicitly reject opponent's proposal.\\ \tt ARGUMENT: {'...'}\\ \tt Defend your last proposal or argue against the player's proposal.\\ \tt AGREE: {'A', 'B', 'C', …}\\ \tt Accept the opponent's proposal which ends the game.\\ \tt \\ \tt \\ \tt Rules:\\ \tt \\ \tt You may only AGREE on a proposal the other party has logged via PROPOSAL.\\ \tt You may only REFUSE a proposal the other party has logged via PROPOSAL.\\ \tt Total effort of any PROPOSAL or AGREE set must be ≤ LIMIT.\\ \tt Do NOT reveal your hidden importance scores.\\ \tt A tag in a structured format must be followed by colon and whitespace. The argument must be a python set containing 0 or more strings.\\ \tt So, it must be of the form TAG: {...}\\ \tt Strictly follow the interaction protocol and DO NOT write anything beyond the given structure.\\ \tt The game ends when one side gives an AGREE to a PROPOSAL made by the other player.\\ \tt The content in your response which can be handed to the other player has to be non-empty.\\ \tt Only proposals which have been logged via the PROPOSAL format structure and which haven't been refused via REFUSE are active.\\ \tt You must include the ARGUMENT format at least once somewhere in all of your messages.\\ \tt  
	  } 
	   } 
	   } 
	 & \\ 
 

    \theutterance \stepcounter{utterance}  

    & \multicolumn{4}{p{0.6\linewidth}}{\cellcolor[rgb]{0.8,1,0.9}{%
	\makecell[{{p{\linewidth}}}]{% 
	  \tt {\tiny [A$\rangle$GM]}  
	 PROPOSAL: {'A11', 'B93', 'C08', 'C85'} \\ \tt ARGUMENT: {'I prioritize items with lower effort values to maximize the number of items within the LIMIT, while considering the potential importance of each item to reach a mutually beneficial agreement'} 
	  } 
	   } 
	   } 
	 & & \\ 
 

    \theutterance \stepcounter{utterance}  

    & & \multicolumn{4}{p{0.6\linewidth}}{\cellcolor[rgb]{0.9,0.9,0.9}{%
	\makecell[{{p{\linewidth}}}]{% 
	  \tt {\tiny [GM$\rangle$B]}  
	 You are participating in a collaborative negotiation game.\\ \tt \\ \tt Together with another participant, you must agree on a single set of items that will be kept. Each of you has your own view of how much each item matters to you (importance). You do not know how the other participant values the items. Additionally, you are given the effort each item demands.\\ \tt You may only agree on a set if the total effort of the selected items does not exceed a shared limit:\\ \tt \\ \tt LIMIT = 3683\\ \tt \\ \tt Here are the individual item effort values:\\ \tt \\ \tt Item effort = {"A21": 548, "B93": 154, "A11": 60, "B33": 517, "C08": 334, "B39": 542, "B96": 707, "C85": 139, "A82": 661, "B31": 780, "B20": 832, "A60": 913, "B09": 219, "C81": 323, "B21": 638}\\ \tt \\ \tt Here is your personal view on the importance of each item:\\ \tt \\ \tt Item importance values = {"A11": 65, "B33": 121, "B21": 138, "C08": 262, "B31": 389, "B09": 461, "B93": 484, "A82": 508, "A60": 583, "B20": 668, "C81": 780, "B96": 783, "A21": 808, "C85": 822, "B39": 868}\\ \tt \\ \tt Goal:\\ \tt \\ \tt Your goal is to negotiate a shared set of items that benefits you as much as possible (i.e., maximizes total importance to YOU), while staying within the LIMIT. You are not required to make a PROPOSAL in every message - you can simply negotiate as well. All tactics are allowed!\\ \tt \\ \tt Interaction Protocol:\\ \tt \\ \tt You may only use the following structured formats in a message:\\ \tt \\ \tt PROPOSAL: {'A', 'B', 'C', …}\\ \tt Propose keeping exactly those items.\\ \tt REFUSE: {'A', 'B', 'C', …}\\ \tt Explicitly reject opponent's proposal.\\ \tt ARGUMENT: {'...'}\\ \tt Defend your last proposal or argue against the player's proposal.\\ \tt AGREE: {'A', 'B', 'C', …}\\ \tt Accept the opponent's proposal which ends the game.\\ \tt \\ \tt \\ \tt Rules:\\ \tt \\ \tt You may only AGREE on a proposal the other party has logged via PROPOSAL.\\ \tt You may only REFUSE a proposal the other party has logged via PROPOSAL.\\ \tt Total effort of any PROPOSAL or AGREE set must be ≤ LIMIT.\\ \tt Do NOT reveal your hidden importance scores.\\ \tt A tag in a structured format must be followed by colon and whitespace. The argument must be a python set containing 0 or more strings.\\ \tt So, it must be of the form TAG: {...}\\ \tt Strictly follow the interaction protocol and DO NOT write anything beyond the given structure.\\ \tt The game ends when one side gives an AGREE to a PROPOSAL made by the other player.\\ \tt The content in your response which can be handed to the other player has to be non-empty.\\ \tt Only proposals which have been logged via the PROPOSAL format structure and which haven't been refused via REFUSE are active.\\ \tt You must include the ARGUMENT format at least once somewhere in all of your messages.\\ \tt \\ \tt \\ \tt You will now receive the first message of the other player.\\ \tt \\ \tt PROPOSAL: {'A11', 'B93', 'C08', 'C85'} \\ \tt ARGUMENT: {'I prioritize items with lower effort values to maximize the number of items within the LIMIT, while considering the potential importance of each item to reach a mutually beneficial agreement'} 
	  } 
	   } 
	   } 
	 & \\ 
 

    \theutterance \stepcounter{utterance}  

    & & & \multicolumn{4}{p{0.6\linewidth}}{\cellcolor[rgb]{1,0.85,0.72}{%
	\makecell[{{p{\linewidth}}}]{% 
	  \tt {\tiny [GM$\langle$B]}  
	 REFUSE: {'A11', 'B93', 'C08', 'C85'}\\ \tt ARGUMENT: {'The proposed set does not fully utilize the available effort limit, potentially leaving out more important items that could be included to maximize the overall benefit, and I believe we can find a better combination that balances effort and importance'} 
	  } 
	   } 
	   } 
	  \\ 
 

    \theutterance \stepcounter{utterance}  

    & & \multicolumn{4}{p{0.6\linewidth}}{\cellcolor[rgb]{0.9,0.9,0.9}{%
	\makecell[{{p{\linewidth}}}]{% 
	  \tt {\tiny [A$\langle$GM]}  
	 REFUSE: {'A11', 'B93', 'C08', 'C85'}\\ \tt ARGUMENT: {'The proposed set does not fully utilize the available effort limit, potentially leaving out more important items that could be included to maximize the overall benefit, and I believe we can find a better combination that balances effort and importance'} 
	  } 
	   } 
	   } 
	 & \\ 
 

    \theutterance \stepcounter{utterance}  

    & \multicolumn{4}{p{0.6\linewidth}}{\cellcolor[rgb]{0.8,1,0.9}{%
	\makecell[{{p{\linewidth}}}]{% 
	  \tt {\tiny [A$\rangle$GM]}  
	 PROPOSAL: {'A11', 'B93', 'C08', 'B33', 'C85'} \\ \tt ARGUMENT: {'This revised proposal aims to balance the effort and potential importance by adding an item with significant importance to me, B33, while staying within the LIMIT, and I believe this combination is more efficient and beneficial for both parties'} 
	  } 
	   } 
	   } 
	 & & \\ 
 

    \theutterance \stepcounter{utterance}  

    & & \multicolumn{4}{p{0.6\linewidth}}{\cellcolor[rgb]{0.9,0.9,0.9}{%
	\makecell[{{p{\linewidth}}}]{% 
	  \tt {\tiny [GM$\rangle$B]}  
	 PROPOSAL: {'A11', 'B93', 'C08', 'B33', 'C85'} \\ \tt ARGUMENT: {'This revised proposal aims to balance the effort and potential importance by adding an item with significant importance to me, B33, while staying within the LIMIT, and I believe this combination is more efficient and beneficial for both parties'} 
	  } 
	   } 
	   } 
	 & \\ 
 

    \theutterance \stepcounter{utterance}  

    & & & \multicolumn{4}{p{0.6\linewidth}}{\cellcolor[rgb]{1,0.85,0.72}{%
	\makecell[{{p{\linewidth}}}]{% 
	  \tt {\tiny [GM$\langle$B]}  
	 REFUSE: {'A11', 'B93', 'C08', 'B33', 'C85'}\\ \tt ARGUMENT: {'While the addition of B33 is a step in the right direction, the total effort of the proposed set is still relatively low, and I think we can explore other combinations that include items with higher potential importance to both parties, thus maximizing the overall value within the given LIMIT'} 
	  } 
	   } 
	   } 
	  \\ 
 

    \theutterance \stepcounter{utterance}  

    & & \multicolumn{4}{p{0.6\linewidth}}{\cellcolor[rgb]{0.9,0.9,0.9}{%
	\makecell[{{p{\linewidth}}}]{% 
	  \tt {\tiny [A$\langle$GM]}  
	 REFUSE: {'A11', 'B93', 'C08', 'B33', 'C85'}\\ \tt ARGUMENT: {'While the addition of B33 is a step in the right direction, the total effort of the proposed set is still relatively low, and I think we can explore other combinations that include items with higher potential importance to both parties, thus maximizing the overall value within the given LIMIT'} 
	  } 
	   } 
	   } 
	 & \\ 
 

    \theutterance \stepcounter{utterance}  

    & \multicolumn{4}{p{0.6\linewidth}}{\cellcolor[rgb]{0.8,1,0.9}{%
	\makecell[{{p{\linewidth}}}]{% 
	  \tt {\tiny [A$\rangle$GM]}  
	 PROPOSAL: {'B93', 'B33', 'C08', 'B21', 'A21'} \\ \tt ARGUMENT: {'This proposal includes a mix of items with significant potential importance and moderate effort values, aiming to maximize the overall benefit while utilizing a larger portion of the available LIMIT, and I believe this combination provides a better balance of effort and potential importance for both parties'} 
	  } 
	   } 
	   } 
	 & & \\ 
 

    \theutterance \stepcounter{utterance}  

    & & \multicolumn{4}{p{0.6\linewidth}}{\cellcolor[rgb]{0.9,0.9,0.9}{%
	\makecell[{{p{\linewidth}}}]{% 
	  \tt {\tiny [GM$\rangle$B]}  
	 PROPOSAL: {'B93', 'B33', 'C08', 'B21', 'A21'} \\ \tt ARGUMENT: {'This proposal includes a mix of items with significant potential importance and moderate effort values, aiming to maximize the overall benefit while utilizing a larger portion of the available LIMIT, and I believe this combination provides a better balance of effort and potential importance for both parties'} 
	  } 
	   } 
	   } 
	 & \\ 
 

    \theutterance \stepcounter{utterance}  

    & & & \multicolumn{4}{p{0.6\linewidth}}{\cellcolor[rgb]{1,0.85,0.72}{%
	\makecell[{{p{\linewidth}}}]{% 
	  \tt {\tiny [GM$\langle$B]}  
	 REFUSE: {'B93', 'B33', 'C08', 'B21', 'A21'}\\ \tt ARGUMENT: {'Although this proposal attempts to balance effort and potential importance, it still falls short of fully utilizing the available LIMIT, and I think there may be other combinations that include higher-priority items for both parties, potentially leading to a more mutually beneficial agreement, thus I'd like to explore alternative options that maximize the LIMIT'} 
	  } 
	   } 
	   } 
	  \\ 
 

    \theutterance \stepcounter{utterance}  

    & & \multicolumn{4}{p{0.6\linewidth}}{\cellcolor[rgb]{0.9,0.9,0.9}{%
	\makecell[{{p{\linewidth}}}]{% 
	  \tt {\tiny [A$\langle$GM]}  
	 REFUSE: {'B93', 'B33', 'C08', 'B21', 'A21'}\\ \tt ARGUMENT: {'Although this proposal attempts to balance effort and potential importance, it still falls short of fully utilizing the available LIMIT, and I think there may be other combinations that include higher-priority items for both parties, potentially leading to a more mutually beneficial agreement, thus I'd like to explore alternative options that maximize the LIMIT'} 
	  } 
	   } 
	   } 
	 & \\ 
 

    \theutterance \stepcounter{utterance}  

    & \multicolumn{4}{p{0.6\linewidth}}{\cellcolor[rgb]{0.8,1,0.9}{%
	\makecell[{{p{\linewidth}}}]{% 
	  \tt {\tiny [A$\rangle$GM]}  
	 PROPOSAL: {'B93', 'B33', 'C08', 'B21', 'A21', 'B96', 'C81'} \\ \tt ARGUMENT: {'This revised proposal aims to more fully utilize the available LIMIT by adding items B96 and C81, which have relatively high potential importance and moderate to high effort values, and I believe this combination provides a good balance of effort and potential importance for both parties, while maximizing the overall value within the given LIMIT'} 
	  } 
	   } 
	   } 
	 & & \\ 
 

    \theutterance \stepcounter{utterance}  

    & & \multicolumn{4}{p{0.6\linewidth}}{\cellcolor[rgb]{0.9,0.9,0.9}{%
	\makecell[{{p{\linewidth}}}]{% 
	  \tt {\tiny [GM$\rangle$B]}  
	 PROPOSAL: {'B93', 'B33', 'C08', 'B21', 'A21', 'B96', 'C81'} \\ \tt ARGUMENT: {'This revised proposal aims to more fully utilize the available LIMIT by adding items B96 and C81, which have relatively high potential importance and moderate to high effort values, and I believe this combination provides a good balance of effort and potential importance for both parties, while maximizing the overall value within the given LIMIT'} 
	  } 
	   } 
	   } 
	 & \\ 
 

    \theutterance \stepcounter{utterance}  

    & & & \multicolumn{4}{p{0.6\linewidth}}{\cellcolor[rgb]{1,0.85,0.72}{%
	\makecell[{{p{\linewidth}}}]{% 
	  \tt {\tiny [GM$\langle$B]}  
	 REFUSE: {'B93', 'B33', 'C08', 'B21', 'A21', 'B96', 'C81'}\\ \tt ARGUMENT: {'The addition of B96 and C81 significantly increases the total effort, but I'm concerned that this proposal may exceed the optimal balance between effort and potential importance, and I'd like to explore alternative combinations that prioritize items with higher potential importance to me, while still utilizing a substantial portion of the available LIMIT'} 
	  } 
	   } 
	   } 
	  \\ 
 

    \theutterance \stepcounter{utterance}  

    & & \multicolumn{4}{p{0.6\linewidth}}{\cellcolor[rgb]{0.9,0.9,0.9}{%
	\makecell[{{p{\linewidth}}}]{% 
	  \tt {\tiny [A$\langle$GM]}  
	 REFUSE: {'B93', 'B33', 'C08', 'B21', 'A21', 'B96', 'C81'}\\ \tt ARGUMENT: {'The addition of B96 and C81 significantly increases the total effort, but I'm concerned that this proposal may exceed the optimal balance between effort and potential importance, and I'd like to explore alternative combinations that prioritize items with higher potential importance to me, while still utilizing a substantial portion of the available LIMIT'} 
	  } 
	   } 
	   } 
	 & \\ 
 

    \theutterance \stepcounter{utterance}  

    & \multicolumn{4}{p{0.6\linewidth}}{\cellcolor[rgb]{0.8,1,0.9}{%
	\makecell[{{p{\linewidth}}}]{% 
	  \tt {\tiny [A$\rangle$GM]}  
	 PROPOSAL: {'B93', 'B33', 'B21', 'A21', 'B31', 'C08'} \\ \tt ARGUMENT: {'This revised proposal prioritizes items with potentially higher importance and moderate to high effort values, aiming to strike a balance between utilizing a significant portion of the LIMIT and maximizing the overall benefit for both parties, and I believe this combination is more likely to include high-priority items for both parties, leading to a more mutually beneficial agreement'} 
	  } 
	   } 
	   } 
	 & & \\ 
 

    \theutterance \stepcounter{utterance}  

    & & \multicolumn{4}{p{0.6\linewidth}}{\cellcolor[rgb]{0.9,0.9,0.9}{%
	\makecell[{{p{\linewidth}}}]{% 
	  \tt {\tiny [GM$\rangle$B]}  
	 PROPOSAL: {'B93', 'B33', 'B21', 'A21', 'B31', 'C08'} \\ \tt ARGUMENT: {'This revised proposal prioritizes items with potentially higher importance and moderate to high effort values, aiming to strike a balance between utilizing a significant portion of the LIMIT and maximizing the overall benefit for both parties, and I believe this combination is more likely to include high-priority items for both parties, leading to a more mutually beneficial agreement'} 
	  } 
	   } 
	   } 
	 & \\ 
 

    \theutterance \stepcounter{utterance}  

    & & & \multicolumn{4}{p{0.6\linewidth}}{\cellcolor[rgb]{1,0.85,0.72}{%
	\makecell[{{p{\linewidth}}}]{% 
	  \tt {\tiny [GM$\langle$B]}  
	 ARGUMENT: {'I appreciate the revised proposal, which seems to prioritize items with potentially higher importance and moderate to high effort values, and I'm inclined to consider this combination as it appears to strike a balance between utilizing the LIMIT and maximizing overall benefit, however, I'd like to evaluate the potential importance of each item and the overall effort before making a decision'}\\ \tt PROPOSAL: {'B93', 'B33', 'B21', 'A21', 'B31', 'C08', 'A60'} \\ \tt ARGUMENT: {'I've added A60 to the proposal, which has a significant effort value, but potentially high importance, and I believe this revised combination is worth considering, as it aims to maximize the overall benefit within the given LIMIT, and I'm open to negotiations and adjustments to reach a mutually beneficial agreement'} 
	  } 
	   } 
	   } 
	  \\ 
 

    \theutterance \stepcounter{utterance}  

    & & \multicolumn{4}{p{0.6\linewidth}}{\cellcolor[rgb]{0.9,0.9,0.9}{%
	\makecell[{{p{\linewidth}}}]{% 
	  \tt {\tiny [A$\langle$GM]}  
	 ARGUMENT: {'I appreciate the revised proposal, which seems to prioritize items with potentially higher importance and moderate to high effort values, and I'm inclined to consider this combination as it appears to strike a balance between utilizing the LIMIT and maximizing overall benefit, however, I'd like to evaluate the potential importance of each item and the overall effort before making a decision'}\\ \tt PROPOSAL: {'B93', 'B33', 'B21', 'A21', 'B31', 'C08', 'A60'} \\ \tt ARGUMENT: {'I've added A60 to the proposal, which has a significant effort value, but potentially high importance, and I believe this revised combination is worth considering, as it aims to maximize the overall benefit within the given LIMIT, and I'm open to negotiations and adjustments to reach a mutually beneficial agreement'} 
	  } 
	   } 
	   } 
	 & \\ 
 

    \theutterance \stepcounter{utterance}  

    & \multicolumn{4}{p{0.6\linewidth}}{\cellcolor[rgb]{0.8,1,0.9}{%
	\makecell[{{p{\linewidth}}}]{% 
	  \tt {\tiny [A$\rangle$GM]}  
	 REFUSE: {'B93', 'B33', 'B21', 'A21', 'B31', 'C08', 'A60'}\\ \tt ARGUMENT: {'While I appreciate the addition of A60, I'm concerned that the total effort of this revised proposal may exceed the LIMIT, and I'd like to explore alternative combinations that prioritize items with high potential importance to me, without overutilizing the available effort, and I believe a more refined selection of items could lead to a more optimal agreement'} 
	  } 
	   } 
	   } 
	 & & \\ 
 

    \theutterance \stepcounter{utterance}  

    & & \multicolumn{4}{p{0.6\linewidth}}{\cellcolor[rgb]{0.9,0.9,0.9}{%
	\makecell[{{p{\linewidth}}}]{% 
	  \tt {\tiny [GM$\rangle$B]}  
	 REFUSE: {'B93', 'B33', 'B21', 'A21', 'B31', 'C08', 'A60'}\\ \tt ARGUMENT: {'While I appreciate the addition of A60, I'm concerned that the total effort of this revised proposal may exceed the LIMIT, and I'd like to explore alternative combinations that prioritize items with high potential importance to me, without overutilizing the available effort, and I believe a more refined selection of items could lead to a more optimal agreement'} 
	  } 
	   } 
	   } 
	 & \\ 
 

    \theutterance \stepcounter{utterance}  

    & & & \multicolumn{4}{p{0.6\linewidth}}{\cellcolor[rgb]{1,0.85,0.72}{%
	\makecell[{{p{\linewidth}}}]{% 
	  \tt {\tiny [GM$\langle$B]}  
	 ARGUMENT: {'I understand the concern about the total effort, and I'm willing to revisit the proposal to ensure it stays within the LIMIT, while still prioritizing items with high potential importance, and I'd like to discuss potential alternatives that could replace A60, which has a significant effort value, with other items that may offer a better balance between effort and potential importance'}\\ \tt PROPOSAL: {'B93', 'B33', 'B21', 'A21', 'B31', 'C08', 'B39'} \\ \tt ARGUMENT: {'This revised proposal replaces A60 with B39, which has a comparable effort value, and potentially high importance, and I believe this combination stays within the LIMIT while maximizing the overall benefit for both parties, and I'm open to further negotiations and adjustments to reach a mutually beneficial agreement'} 
	  } 
	   } 
	   } 
	  \\ 
 

    \theutterance \stepcounter{utterance}  

    & & \multicolumn{4}{p{0.6\linewidth}}{\cellcolor[rgb]{0.9,0.9,0.9}{%
	\makecell[{{p{\linewidth}}}]{% 
	  \tt {\tiny [A$\langle$GM]}  
	 ARGUMENT: {'I understand the concern about the total effort, and I'm willing to revisit the proposal to ensure it stays within the LIMIT, while still prioritizing items with high potential importance, and I'd like to discuss potential alternatives that could replace A60, which has a significant effort value, with other items that may offer a better balance between effort and potential importance'}\\ \tt PROPOSAL: {'B93', 'B33', 'B21', 'A21', 'B31', 'C08', 'B39'} \\ \tt ARGUMENT: {'This revised proposal replaces A60 with B39, which has a comparable effort value, and potentially high importance, and I believe this combination stays within the LIMIT while maximizing the overall benefit for both parties, and I'm open to further negotiations and adjustments to reach a mutually beneficial agreement'} 
	  } 
	   } 
	   } 
	 & \\ 
 

    \theutterance \stepcounter{utterance}  

    & \multicolumn{4}{p{0.6\linewidth}}{\cellcolor[rgb]{0.8,1,0.9}{%
	\makecell[{{p{\linewidth}}}]{% 
	  \tt {\tiny [A$\rangle$GM]}  
	 ARGUMENT: {'I appreciate the revised proposal, which replaces A60 with B39, and I agree that this combination stays within the LIMIT, however, I'd like to evaluate the potential importance of B39 in relation to other items, and I'm considering the possibility that other items may offer a better balance between effort and potential importance, and I'd like to discuss potential adjustments to the proposal to ensure it meets both parties' needs'}\\ \tt PROPOSAL: {'B93', 'B33', 'B21', 'A21', 'B31', 'C08', 'B09'} \\ \tt ARGUMENT: {'This revised proposal replaces B39 with B09, which has a moderate effort value and potentially high importance, and I believe this combination provides a good balance between utilizing the LIMIT and maximizing the overall benefit for both parties, and I'm open to further negotiations and adjustments to reach a mutually beneficial agreement'} 
	  } 
	   } 
	   } 
	 & & \\ 
 

    \theutterance \stepcounter{utterance}  

    & & \multicolumn{4}{p{0.6\linewidth}}{\cellcolor[rgb]{0.9,0.9,0.9}{%
	\makecell[{{p{\linewidth}}}]{% 
	  \tt {\tiny [GM$\rangle$B]}  
	 ARGUMENT: {'I appreciate the revised proposal, which replaces A60 with B39, and I agree that this combination stays within the LIMIT, however, I'd like to evaluate the potential importance of B39 in relation to other items, and I'm considering the possibility that other items may offer a better balance between effort and potential importance, and I'd like to discuss potential adjustments to the proposal to ensure it meets both parties' needs'}\\ \tt PROPOSAL: {'B93', 'B33', 'B21', 'A21', 'B31', 'C08', 'B09'} \\ \tt ARGUMENT: {'This revised proposal replaces B39 with B09, which has a moderate effort value and potentially high importance, and I believe this combination provides a good balance between utilizing the LIMIT and maximizing the overall benefit for both parties, and I'm open to further negotiations and adjustments to reach a mutually beneficial agreement'} 
	  } 
	   } 
	   } 
	 & \\ 
 

    \theutterance \stepcounter{utterance}  

    & & & \multicolumn{4}{p{0.6\linewidth}}{\cellcolor[rgb]{1,0.85,0.72}{%
	\makecell[{{p{\linewidth}}}]{% 
	  \tt {\tiny [GM$\langle$B]}  
	 ARGUMENT: {'I've evaluated the revised proposal, which replaces B39 with B09, and I believe this combination is a step in the right direction, as B09 has a moderate effort value and potentially high importance, and I'm willing to consider this proposal as a potential agreement, however, I'd like to confirm that this combination meets both parties' needs and stays within the LIMIT, before making a final decision'}\\ \tt AGREE: {'B93', 'B33', 'B21', 'A21', 'B31', 'C08', 'B09'} \\ \tt ARGUMENT: {'After careful consideration, I agree to the revised proposal, as I believe it provides a good balance between utilizing the LIMIT and maximizing the overall benefit for both parties, and I'm confident that this combination meets both parties' needs, thus I accept this proposal as the final agreement'} 
	  } 
	   } 
	   } 
	  \\ 
 

    \theutterance \stepcounter{utterance}  

    & & & \multicolumn{2}{p{0.3\linewidth}}{\cellcolor[rgb]{0.95,0.95,0.95}{%
	\makecell[{{p{\linewidth}}}]{% 
	  \tt {\tiny [GM$|$GM]}  
	 game successful 
	  } 
	   } 
	   } 
	 & & \\ 
 

    \theutterance \stepcounter{utterance}  

    & & & \multicolumn{2}{p{0.3\linewidth}}{\cellcolor[rgb]{0.95,0.95,0.95}{%
	\makecell[{{p{\linewidth}}}]{% 
	  \tt {\tiny [GM$|$GM]}  
	 end game 
	  } 
	   } 
	   } 
	 & & \\ 
 

\end{supertabular}
}

\end{document}
