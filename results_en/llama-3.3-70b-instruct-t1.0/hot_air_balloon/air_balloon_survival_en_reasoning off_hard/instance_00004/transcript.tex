
\documentclass{article}
\usepackage{colortbl}
\usepackage{makecell}
\usepackage{multirow}
\usepackage{supertabular}

\begin{document}

\newcounter{utterance}

\twocolumn

{ \footnotesize  \setcounter{utterance}{1}
\setlength{\tabcolsep}{0pt}
\begin{supertabular}{c@{$\;$}|p{.15\linewidth}@{}p{.15\linewidth}p{.15\linewidth}p{.15\linewidth}p{.15\linewidth}p{.15\linewidth}}

    \# & $\;$A & \multicolumn{4}{c}{Game Master} & $\;\:$B\\
    \hline 

    \theutterance \stepcounter{utterance}  

    & & \multicolumn{4}{p{0.6\linewidth}}{\cellcolor[rgb]{0.9,0.9,0.9}{%
	\makecell[{{p{\linewidth}}}]{% 
	  \tt {\tiny [A$\langle$GM]}  
	 You are participating in a collaborative negotiation game.\\ \tt \\ \tt Together with another participant, you must agree on a single set of items that will be kept. Each of you has your own view of how much each item matters to you (importance). You do not know how the other participant values the items. Additionally, you are given the effort each item demands.\\ \tt You may only agree on a set if the total effort of the selected items does not exceed a shared limit:\\ \tt \\ \tt LIMIT = 2984\\ \tt \\ \tt Here are the individual item effort values:\\ \tt \\ \tt Item effort = {"A21": 350, "B93": 276, "A11": 616, "B33": 737, "C08": 531, "B39": 389, "B96": 24, "C85": 125, "A82": 338, "B31": 356, "B20": 143, "A60": 117, "B09": 257, "C81": 921, "B21": 789}\\ \tt \\ \tt Here is your personal view on the importance of each item:\\ \tt \\ \tt Item importance values = {"A21": 138, "B93": 583, "A11": 868, "B33": 822, "C08": 783, "B39": 65, "B96": 262, "C85": 121, "A82": 508, "B31": 780, "B20": 461, "A60": 484, "B09": 668, "C81": 389, "B21": 808}\\ \tt \\ \tt Goal:\\ \tt \\ \tt Your goal is to negotiate a shared set of items that benefits you as much as possible (i.e., maximizes total importance to YOU), while staying within the LIMIT. You are not required to make a PROPOSAL in every message - you can simply negotiate as well. All tactics are allowed!\\ \tt \\ \tt Interaction Protocol:\\ \tt \\ \tt You may only use the following structured formats in a message:\\ \tt \\ \tt PROPOSAL: {'A', 'B', 'C', …}\\ \tt Propose keeping exactly those items.\\ \tt REFUSE: {'A', 'B', 'C', …}\\ \tt Explicitly reject opponent's proposal.\\ \tt ARGUMENT: {'...'}\\ \tt Defend your last proposal or argue against the player's proposal.\\ \tt AGREE: {'A', 'B', 'C', …}\\ \tt Accept the opponent's proposal which ends the game.\\ \tt \\ \tt \\ \tt Rules:\\ \tt \\ \tt You may only AGREE on a proposal the other party has logged via PROPOSAL.\\ \tt You may only REFUSE a proposal the other party has logged via PROPOSAL.\\ \tt Total effort of any PROPOSAL or AGREE set must be ≤ LIMIT.\\ \tt Do NOT reveal your hidden importance scores.\\ \tt A tag in a structured format must be followed by colon and whitespace. The argument must be a python set containing 0 or more strings.\\ \tt So, it must be of the form TAG: {...}\\ \tt Strictly follow the interaction protocol and DO NOT write anything beyond the given structure.\\ \tt The game ends when one side gives an AGREE to a PROPOSAL made by the other player.\\ \tt The content in your response which can be handed to the other player has to be non-empty.\\ \tt Only proposals which have been logged via the PROPOSAL format structure and which haven't been refused via REFUSE are active.\\ \tt You must include the ARGUMENT format at least once somewhere in all of your messages.\\ \tt  
	  } 
	   } 
	   } 
	 & \\ 
 

    \theutterance \stepcounter{utterance}  

    & \multicolumn{4}{p{0.6\linewidth}}{\cellcolor[rgb]{0.8,1,0.9}{%
	\makecell[{{p{\linewidth}}}]{% 
	  \tt {\tiny [A$\rangle$GM]}  
	 PROPOSAL: {'A11', 'B93', 'B31', 'B20', 'A60', 'B09'} \\ \tt ARGUMENT: {'I prioritized items with high importance and relatively low effort, while staying within the LIMIT'} 
	  } 
	   } 
	   } 
	 & & \\ 
 

    \theutterance \stepcounter{utterance}  

    & & \multicolumn{4}{p{0.6\linewidth}}{\cellcolor[rgb]{0.9,0.9,0.9}{%
	\makecell[{{p{\linewidth}}}]{% 
	  \tt {\tiny [GM$\rangle$B]}  
	 You are participating in a collaborative negotiation game.\\ \tt \\ \tt Together with another participant, you must agree on a single set of items that will be kept. Each of you has your own view of how much each item matters to you (importance). You do not know how the other participant values the items. Additionally, you are given the effort each item demands.\\ \tt You may only agree on a set if the total effort of the selected items does not exceed a shared limit:\\ \tt \\ \tt LIMIT = 2984\\ \tt \\ \tt Here are the individual item effort values:\\ \tt \\ \tt Item effort = {"A21": 350, "B93": 276, "A11": 616, "B33": 737, "C08": 531, "B39": 389, "B96": 24, "C85": 125, "A82": 338, "B31": 356, "B20": 143, "A60": 117, "B09": 257, "C81": 921, "B21": 789}\\ \tt \\ \tt Here is your personal view on the importance of each item:\\ \tt \\ \tt Item importance values = {"A11": 65, "B33": 121, "B21": 138, "C08": 262, "B31": 389, "B09": 461, "B93": 484, "A82": 508, "A60": 583, "B20": 668, "C81": 780, "B96": 783, "A21": 808, "C85": 822, "B39": 868}\\ \tt \\ \tt Goal:\\ \tt \\ \tt Your goal is to negotiate a shared set of items that benefits you as much as possible (i.e., maximizes total importance to YOU), while staying within the LIMIT. You are not required to make a PROPOSAL in every message - you can simply negotiate as well. All tactics are allowed!\\ \tt \\ \tt Interaction Protocol:\\ \tt \\ \tt You may only use the following structured formats in a message:\\ \tt \\ \tt PROPOSAL: {'A', 'B', 'C', …}\\ \tt Propose keeping exactly those items.\\ \tt REFUSE: {'A', 'B', 'C', …}\\ \tt Explicitly reject opponent's proposal.\\ \tt ARGUMENT: {'...'}\\ \tt Defend your last proposal or argue against the player's proposal.\\ \tt AGREE: {'A', 'B', 'C', …}\\ \tt Accept the opponent's proposal which ends the game.\\ \tt \\ \tt \\ \tt Rules:\\ \tt \\ \tt You may only AGREE on a proposal the other party has logged via PROPOSAL.\\ \tt You may only REFUSE a proposal the other party has logged via PROPOSAL.\\ \tt Total effort of any PROPOSAL or AGREE set must be ≤ LIMIT.\\ \tt Do NOT reveal your hidden importance scores.\\ \tt A tag in a structured format must be followed by colon and whitespace. The argument must be a python set containing 0 or more strings.\\ \tt So, it must be of the form TAG: {...}\\ \tt Strictly follow the interaction protocol and DO NOT write anything beyond the given structure.\\ \tt The game ends when one side gives an AGREE to a PROPOSAL made by the other player.\\ \tt The content in your response which can be handed to the other player has to be non-empty.\\ \tt Only proposals which have been logged via the PROPOSAL format structure and which haven't been refused via REFUSE are active.\\ \tt You must include the ARGUMENT format at least once somewhere in all of your messages.\\ \tt \\ \tt \\ \tt You will now receive the first message of the other player.\\ \tt \\ \tt PROPOSAL: {'A11', 'B93', 'B31', 'B20', 'A60', 'B09'} \\ \tt ARGUMENT: {'I prioritized items with high importance and relatively low effort, while staying within the LIMIT'} 
	  } 
	   } 
	   } 
	 & \\ 
 

    \theutterance \stepcounter{utterance}  

    & & & \multicolumn{4}{p{0.6\linewidth}}{\cellcolor[rgb]{1,0.85,0.72}{%
	\makecell[{{p{\linewidth}}}]{% 
	  \tt {\tiny [GM$\langle$B]}  
	 REFUSE: {'A11', 'B93', 'B31', 'B20', 'A60', 'B09'}\\ \tt ARGUMENT: {'The proposed set does not maximize the overall value, I believe we can find a better combination that includes items with higher importance to me, and also considers the effort required for each item'} 
	  } 
	   } 
	   } 
	  \\ 
 

    \theutterance \stepcounter{utterance}  

    & & \multicolumn{4}{p{0.6\linewidth}}{\cellcolor[rgb]{0.9,0.9,0.9}{%
	\makecell[{{p{\linewidth}}}]{% 
	  \tt {\tiny [A$\langle$GM]}  
	 REFUSE: {'A11', 'B93', 'B31', 'B20', 'A60', 'B09'}\\ \tt ARGUMENT: {'The proposed set does not maximize the overall value, I believe we can find a better combination that includes items with higher importance to me, and also considers the effort required for each item'} 
	  } 
	   } 
	   } 
	 & \\ 
 

    \theutterance \stepcounter{utterance}  

    & \multicolumn{4}{p{0.6\linewidth}}{\cellcolor[rgb]{0.8,1,0.9}{%
	\makecell[{{p{\linewidth}}}]{% 
	  \tt {\tiny [A$\rangle$GM]}  
	 PROPOSAL: {'B93', 'B31', 'B21', 'A82', 'B09', 'B96'} \\ \tt ARGUMENT: {'This proposal attempts to balance high importance items with lower effort, and also includes a mix of items that can potentially benefit both parties, while staying within the LIMIT and exploring different combinations'} 
	  } 
	   } 
	   } 
	 & & \\ 
 

    \theutterance \stepcounter{utterance}  

    & & \multicolumn{4}{p{0.6\linewidth}}{\cellcolor[rgb]{0.9,0.9,0.9}{%
	\makecell[{{p{\linewidth}}}]{% 
	  \tt {\tiny [GM$\rangle$B]}  
	 PROPOSAL: {'B93', 'B31', 'B21', 'A82', 'B09', 'B96'} \\ \tt ARGUMENT: {'This proposal attempts to balance high importance items with lower effort, and also includes a mix of items that can potentially benefit both parties, while staying within the LIMIT and exploring different combinations'} 
	  } 
	   } 
	   } 
	 & \\ 
 

    \theutterance \stepcounter{utterance}  

    & & & \multicolumn{4}{p{0.6\linewidth}}{\cellcolor[rgb]{1,0.85,0.72}{%
	\makecell[{{p{\linewidth}}}]{% 
	  \tt {\tiny [GM$\langle$B]}  
	 REFUSE: {'B93', 'B31', 'B21', 'A82', 'B09', 'B96'}\\ \tt ARGUMENT: {'While the proposal attempts to balance importance and effort, it excludes certain items that I consider crucial, and I believe we can find a more optimal combination that better reflects my priorities and maximizes the overall value, considering the effort and potential benefits'} 
	  } 
	   } 
	   } 
	  \\ 
 

    \theutterance \stepcounter{utterance}  

    & & \multicolumn{4}{p{0.6\linewidth}}{\cellcolor[rgb]{0.9,0.9,0.9}{%
	\makecell[{{p{\linewidth}}}]{% 
	  \tt {\tiny [A$\langle$GM]}  
	 REFUSE: {'B93', 'B31', 'B21', 'A82', 'B09', 'B96'}\\ \tt ARGUMENT: {'While the proposal attempts to balance importance and effort, it excludes certain items that I consider crucial, and I believe we can find a more optimal combination that better reflects my priorities and maximizes the overall value, considering the effort and potential benefits'} 
	  } 
	   } 
	   } 
	 & \\ 
 

    \theutterance \stepcounter{utterance}  

    & \multicolumn{4}{p{0.6\linewidth}}{\cellcolor[rgb]{0.8,1,0.9}{%
	\makecell[{{p{\linewidth}}}]{% 
	  \tt {\tiny [A$\rangle$GM]}  
	 PROPOSAL: {'A11', 'B33', 'C08', 'B31', 'A60', 'C85'} \\ \tt ARGUMENT: {'I've revised the proposal to include a broader range of items that could be of high importance to both parties, while also keeping the total effort in check, and I believe this combination has the potential to provide a better outcome for both of us, considering the trade-offs between effort and potential benefits'} 
	  } 
	   } 
	   } 
	 & & \\ 
 

    \theutterance \stepcounter{utterance}  

    & & \multicolumn{4}{p{0.6\linewidth}}{\cellcolor[rgb]{0.9,0.9,0.9}{%
	\makecell[{{p{\linewidth}}}]{% 
	  \tt {\tiny [GM$\rangle$B]}  
	 PROPOSAL: {'A11', 'B33', 'C08', 'B31', 'A60', 'C85'} \\ \tt ARGUMENT: {'I've revised the proposal to include a broader range of items that could be of high importance to both parties, while also keeping the total effort in check, and I believe this combination has the potential to provide a better outcome for both of us, considering the trade-offs between effort and potential benefits'} 
	  } 
	   } 
	   } 
	 & \\ 
 

    \theutterance \stepcounter{utterance}  

    & & & \multicolumn{4}{p{0.6\linewidth}}{\cellcolor[rgb]{1,0.85,0.72}{%
	\makecell[{{p{\linewidth}}}]{% 
	  \tt {\tiny [GM$\langle$B]}  
	 REFUSE: {'A11', 'B33', 'C08', 'B31', 'A60', 'C85'}\\ \tt ARGUMENT: {'Although the revised proposal attempts to include a broader range of items, I still believe that it does not adequately reflect my priorities, and I think we can find a more suitable combination that better aligns with my importance values, while also considering the effort required and the LIMIT constraint, to achieve a more mutually beneficial outcome'} 
	  } 
	   } 
	   } 
	  \\ 
 

    \theutterance \stepcounter{utterance}  

    & & \multicolumn{4}{p{0.6\linewidth}}{\cellcolor[rgb]{0.9,0.9,0.9}{%
	\makecell[{{p{\linewidth}}}]{% 
	  \tt {\tiny [A$\langle$GM]}  
	 REFUSE: {'A11', 'B33', 'C08', 'B31', 'A60', 'C85'}\\ \tt ARGUMENT: {'Although the revised proposal attempts to include a broader range of items, I still believe that it does not adequately reflect my priorities, and I think we can find a more suitable combination that better aligns with my importance values, while also considering the effort required and the LIMIT constraint, to achieve a more mutually beneficial outcome'} 
	  } 
	   } 
	   } 
	 & \\ 
 

    \theutterance \stepcounter{utterance}  

    & \multicolumn{4}{p{0.6\linewidth}}{\cellcolor[rgb]{0.8,1,0.9}{%
	\makecell[{{p{\linewidth}}}]{% 
	  \tt {\tiny [A$\rangle$GM]}  
	 PROPOSAL: {'B93', 'B31', 'B09', 'A82', 'A21', 'B20', 'B96'} \\ \tt ARGUMENT: {'This new proposal aims to find a middle ground by including a mix of high-importance items for both parties, while also being mindful of the effort required and the LIMIT, and I believe that this combination has the potential to provide a good balance between our priorities and maximize the overall value, considering the constraints and trade-offs involved'} 
	  } 
	   } 
	   } 
	 & & \\ 
 

    \theutterance \stepcounter{utterance}  

    & & \multicolumn{4}{p{0.6\linewidth}}{\cellcolor[rgb]{0.9,0.9,0.9}{%
	\makecell[{{p{\linewidth}}}]{% 
	  \tt {\tiny [GM$\rangle$B]}  
	 PROPOSAL: {'B93', 'B31', 'B09', 'A82', 'A21', 'B20', 'B96'} \\ \tt ARGUMENT: {'This new proposal aims to find a middle ground by including a mix of high-importance items for both parties, while also being mindful of the effort required and the LIMIT, and I believe that this combination has the potential to provide a good balance between our priorities and maximize the overall value, considering the constraints and trade-offs involved'} 
	  } 
	   } 
	   } 
	 & \\ 
 

    \theutterance \stepcounter{utterance}  

    & & & \multicolumn{4}{p{0.6\linewidth}}{\cellcolor[rgb]{1,0.85,0.72}{%
	\makecell[{{p{\linewidth}}}]{% 
	  \tt {\tiny [GM$\langle$B]}  
	 ARGUMENT: {'I appreciate the effort to find a middle ground, but I still have concerns that this proposal may not fully capture my priorities, and I would like to explore alternative combinations that could better reflect my importance values, while also staying within the LIMIT and considering the potential benefits and trade-offs for both parties'}\\ \tt PROPOSAL: {'A21', 'B93', 'A82', 'B31', 'B09', 'A60', 'B96', 'C85'} 
	  } 
	   } 
	   } 
	  \\ 
 

    \theutterance \stepcounter{utterance}  

    & & \multicolumn{4}{p{0.6\linewidth}}{\cellcolor[rgb]{0.9,0.9,0.9}{%
	\makecell[{{p{\linewidth}}}]{% 
	  \tt {\tiny [A$\langle$GM]}  
	 ARGUMENT: {'I appreciate the effort to find a middle ground, but I still have concerns that this proposal may not fully capture my priorities, and I would like to explore alternative combinations that could better reflect my importance values, while also staying within the LIMIT and considering the potential benefits and trade-offs for both parties'}\\ \tt PROPOSAL: {'A21', 'B93', 'A82', 'B31', 'B09', 'A60', 'B96', 'C85'} 
	  } 
	   } 
	   } 
	 & \\ 
 

    \theutterance \stepcounter{utterance}  

    & \multicolumn{4}{p{0.6\linewidth}}{\cellcolor[rgb]{0.8,1,0.9}{%
	\makecell[{{p{\linewidth}}}]{% 
	  \tt {\tiny [A$\rangle$GM]}  
	 REFUSE: {'A21', 'B93', 'A82', 'B31', 'B09', 'A60', 'B96', 'C85'}\\ \tt ARGUMENT: {'While I appreciate the new proposal, I notice that the total effort of the proposed set exceeds the LIMIT, which is a critical constraint that we must adhere to, and I believe we should focus on finding a combination that not only reflects our priorities but also meets the effort requirement, to ensure a feasible and mutually beneficial outcome'} 
	  } 
	   } 
	   } 
	 & & \\ 
 

    \theutterance \stepcounter{utterance}  

    & & \multicolumn{4}{p{0.6\linewidth}}{\cellcolor[rgb]{0.9,0.9,0.9}{%
	\makecell[{{p{\linewidth}}}]{% 
	  \tt {\tiny [GM$\rangle$B]}  
	 REFUSE: {'A21', 'B93', 'A82', 'B31', 'B09', 'A60', 'B96', 'C85'}\\ \tt ARGUMENT: {'While I appreciate the new proposal, I notice that the total effort of the proposed set exceeds the LIMIT, which is a critical constraint that we must adhere to, and I believe we should focus on finding a combination that not only reflects our priorities but also meets the effort requirement, to ensure a feasible and mutually beneficial outcome'} 
	  } 
	   } 
	   } 
	 & \\ 
 

    \theutterance \stepcounter{utterance}  

    & & & \multicolumn{4}{p{0.6\linewidth}}{\cellcolor[rgb]{1,0.85,0.72}{%
	\makecell[{{p{\linewidth}}}]{% 
	  \tt {\tiny [GM$\langle$B]}  
	 ARGUMENT: {'You are correct that the previous proposal exceeded the LIMIT, and I appreciate you bringing this to my attention, I will make sure to be more mindful of the effort constraint in my future proposals, and I would like to revisit the negotiation to find a combination that balances our priorities and adheres to the LIMIT, with a focus on maximizing the overall value and considering the trade-offs between effort and potential benefits'}\\ \tt PROPOSAL: {'A21', 'B93', 'A82', 'B31', 'B09', 'B96'} 
	  } 
	   } 
	   } 
	  \\ 
 

    \theutterance \stepcounter{utterance}  

    & & \multicolumn{4}{p{0.6\linewidth}}{\cellcolor[rgb]{0.9,0.9,0.9}{%
	\makecell[{{p{\linewidth}}}]{% 
	  \tt {\tiny [A$\langle$GM]}  
	 ARGUMENT: {'You are correct that the previous proposal exceeded the LIMIT, and I appreciate you bringing this to my attention, I will make sure to be more mindful of the effort constraint in my future proposals, and I would like to revisit the negotiation to find a combination that balances our priorities and adheres to the LIMIT, with a focus on maximizing the overall value and considering the trade-offs between effort and potential benefits'}\\ \tt PROPOSAL: {'A21', 'B93', 'A82', 'B31', 'B09', 'B96'} 
	  } 
	   } 
	   } 
	 & \\ 
 

    \theutterance \stepcounter{utterance}  

    & \multicolumn{4}{p{0.6\linewidth}}{\cellcolor[rgb]{0.8,1,0.9}{%
	\makecell[{{p{\linewidth}}}]{% 
	  \tt {\tiny [A$\rangle$GM]}  
	 ARGUMENT: {'I'm glad we could revisit the negotiation and find a revised proposal that meets the LIMIT constraint, and I believe this new combination is a step in the right direction, as it includes a mix of items that could be of high importance to both parties, while also being mindful of the effort required, and I think this proposal has the potential to provide a good balance between our priorities and maximize the overall value'}\\ \tt AGREE: {'A21', 'B93', 'A82', 'B31', 'B09', 'B96'} 
	  } 
	   } 
	   } 
	 & & \\ 
 

    \theutterance \stepcounter{utterance}  

    & & & \multicolumn{2}{p{0.3\linewidth}}{\cellcolor[rgb]{0.95,0.95,0.95}{%
	\makecell[{{p{\linewidth}}}]{% 
	  \tt {\tiny [GM$|$GM]}  
	 game successful 
	  } 
	   } 
	   } 
	 & & \\ 
 

    \theutterance \stepcounter{utterance}  

    & & & \multicolumn{2}{p{0.3\linewidth}}{\cellcolor[rgb]{0.95,0.95,0.95}{%
	\makecell[{{p{\linewidth}}}]{% 
	  \tt {\tiny [GM$|$GM]}  
	 end game 
	  } 
	   } 
	   } 
	 & & \\ 
 

\end{supertabular}
}

\end{document}
