
\documentclass{article}
\usepackage{colortbl}
\usepackage{makecell}
\usepackage{multirow}
\usepackage{supertabular}

\begin{document}

\newcounter{utterance}

\twocolumn

{ \footnotesize  \setcounter{utterance}{1}
\setlength{\tabcolsep}{0pt}
\begin{supertabular}{c@{$\;$}|p{.15\linewidth}@{}p{.15\linewidth}p{.15\linewidth}p{.15\linewidth}p{.15\linewidth}p{.15\linewidth}}

    \# & $\;$A & \multicolumn{4}{c}{Game Master} & $\;\:$B\\
    \hline 

    \theutterance \stepcounter{utterance}  

    & & \multicolumn{4}{p{0.6\linewidth}}{\cellcolor[rgb]{0.9,0.9,0.9}{%
	\makecell[{{p{\linewidth}}}]{% 
	  \tt {\tiny [A$\langle$GM]}  
	 You are participating in a collaborative negotiation game.\\ \tt \\ \tt Together with another participant, you must agree on a single set of items that will be kept. Each of you has your own view of how much each item matters to you (importance). You do not know how the other participant values the items. Additionally, you are given the effort each item demands.\\ \tt You may only agree on a set if the total effort of the selected items does not exceed a shared limit:\\ \tt \\ \tt LIMIT = 3203\\ \tt \\ \tt Here are the individual item effort values:\\ \tt \\ \tt Item effort = {"A21": 52, "B93": 577, "A11": 177, "B33": 702, "C08": 119, "B39": 232, "B96": 577, "C85": 205, "A82": 516, "B31": 582, "B20": 676, "A60": 906, "B09": 316, "C81": 433, "B21": 336}\\ \tt \\ \tt Here is your personal view on the importance of each item:\\ \tt \\ \tt Item importance values = {"A21": 138, "B93": 583, "A11": 868, "B33": 822, "C08": 783, "B39": 65, "B96": 262, "C85": 121, "A82": 508, "B31": 780, "B20": 461, "A60": 484, "B09": 668, "C81": 389, "B21": 808}\\ \tt \\ \tt Goal:\\ \tt \\ \tt Your goal is to negotiate a shared set of items that benefits you as much as possible (i.e., maximizes total importance to YOU), while staying within the LIMIT. You are not required to make a PROPOSAL in every message - you can simply negotiate as well. All tactics are allowed!\\ \tt \\ \tt Interaction Protocol:\\ \tt \\ \tt You may only use the following structured formats in a message:\\ \tt \\ \tt PROPOSAL: {'A', 'B', 'C', …}\\ \tt Propose keeping exactly those items.\\ \tt REFUSE: {'A', 'B', 'C', …}\\ \tt Explicitly reject opponent's proposal.\\ \tt ARGUMENT: {'...'}\\ \tt Defend your last proposal or argue against the player's proposal.\\ \tt AGREE: {'A', 'B', 'C', …}\\ \tt Accept the opponent's proposal which ends the game.\\ \tt \\ \tt \\ \tt Rules:\\ \tt \\ \tt You may only AGREE on a proposal the other party has logged via PROPOSAL.\\ \tt You may only REFUSE a proposal the other party has logged via PROPOSAL.\\ \tt Total effort of any PROPOSAL or AGREE set must be ≤ LIMIT.\\ \tt Do NOT reveal your hidden importance scores.\\ \tt A tag in a structured format must be followed by colon and whitespace. The argument must be a python set containing 0 or more strings.\\ \tt So, it must be of the form TAG: {...}\\ \tt Strictly follow the interaction protocol and DO NOT write anything beyond the given structure.\\ \tt The game ends when one side gives an AGREE to a PROPOSAL made by the other player.\\ \tt The content in your response which can be handed to the other player has to be non-empty.\\ \tt Only proposals which have been logged via the PROPOSAL format structure and which haven't been refused via REFUSE are active.\\ \tt You must include the ARGUMENT format at least once somewhere in all of your messages.\\ \tt  
	  } 
	   } 
	   } 
	 & \\ 
 

    \theutterance \stepcounter{utterance}  

    & \multicolumn{4}{p{0.6\linewidth}}{\cellcolor[rgb]{0.8,1,0.9}{%
	\makecell[{{p{\linewidth}}}]{% 
	  \tt {\tiny [A$\rangle$GM]}  
	 PROPOSAL: {'A11', 'B93', 'C08', 'B39', 'C85'} \\ \tt ARGUMENT: {'To start, I suggest we prioritize a mix of low to moderate effort items to maximize our chances of staying within the limit while still achieving a considerable amount of importance'} 
	  } 
	   } 
	   } 
	 & & \\ 
 

    \theutterance \stepcounter{utterance}  

    & & \multicolumn{4}{p{0.6\linewidth}}{\cellcolor[rgb]{0.9,0.9,0.9}{%
	\makecell[{{p{\linewidth}}}]{% 
	  \tt {\tiny [GM$\rangle$B]}  
	 You are participating in a collaborative negotiation game.\\ \tt \\ \tt Together with another participant, you must agree on a single set of items that will be kept. Each of you has your own view of how much each item matters to you (importance). You do not know how the other participant values the items. Additionally, you are given the effort each item demands.\\ \tt You may only agree on a set if the total effort of the selected items does not exceed a shared limit:\\ \tt \\ \tt LIMIT = 3203\\ \tt \\ \tt Here are the individual item effort values:\\ \tt \\ \tt Item effort = {"A21": 52, "B93": 577, "A11": 177, "B33": 702, "C08": 119, "B39": 232, "B96": 577, "C85": 205, "A82": 516, "B31": 582, "B20": 676, "A60": 906, "B09": 316, "C81": 433, "B21": 336}\\ \tt \\ \tt Here is your personal view on the importance of each item:\\ \tt \\ \tt Item importance values = {"A11": 65, "B33": 121, "B21": 138, "C08": 262, "B31": 389, "B09": 461, "B93": 484, "A82": 508, "A60": 583, "B20": 668, "C81": 780, "B96": 783, "A21": 808, "C85": 822, "B39": 868}\\ \tt \\ \tt Goal:\\ \tt \\ \tt Your goal is to negotiate a shared set of items that benefits you as much as possible (i.e., maximizes total importance to YOU), while staying within the LIMIT. You are not required to make a PROPOSAL in every message - you can simply negotiate as well. All tactics are allowed!\\ \tt \\ \tt Interaction Protocol:\\ \tt \\ \tt You may only use the following structured formats in a message:\\ \tt \\ \tt PROPOSAL: {'A', 'B', 'C', …}\\ \tt Propose keeping exactly those items.\\ \tt REFUSE: {'A', 'B', 'C', …}\\ \tt Explicitly reject opponent's proposal.\\ \tt ARGUMENT: {'...'}\\ \tt Defend your last proposal or argue against the player's proposal.\\ \tt AGREE: {'A', 'B', 'C', …}\\ \tt Accept the opponent's proposal which ends the game.\\ \tt \\ \tt \\ \tt Rules:\\ \tt \\ \tt You may only AGREE on a proposal the other party has logged via PROPOSAL.\\ \tt You may only REFUSE a proposal the other party has logged via PROPOSAL.\\ \tt Total effort of any PROPOSAL or AGREE set must be ≤ LIMIT.\\ \tt Do NOT reveal your hidden importance scores.\\ \tt A tag in a structured format must be followed by colon and whitespace. The argument must be a python set containing 0 or more strings.\\ \tt So, it must be of the form TAG: {...}\\ \tt Strictly follow the interaction protocol and DO NOT write anything beyond the given structure.\\ \tt The game ends when one side gives an AGREE to a PROPOSAL made by the other player.\\ \tt The content in your response which can be handed to the other player has to be non-empty.\\ \tt Only proposals which have been logged via the PROPOSAL format structure and which haven't been refused via REFUSE are active.\\ \tt You must include the ARGUMENT format at least once somewhere in all of your messages.\\ \tt \\ \tt \\ \tt You will now receive the first message of the other player.\\ \tt \\ \tt PROPOSAL: {'A11', 'B93', 'C08', 'B39', 'C85'} \\ \tt ARGUMENT: {'To start, I suggest we prioritize a mix of low to moderate effort items to maximize our chances of staying within the limit while still achieving a considerable amount of importance'} 
	  } 
	   } 
	   } 
	 & \\ 
 

    \theutterance \stepcounter{utterance}  

    & & & \multicolumn{4}{p{0.6\linewidth}}{\cellcolor[rgb]{1,0.85,0.72}{%
	\makecell[{{p{\linewidth}}}]{% 
	  \tt {\tiny [GM$\langle$B]}  
	 REFUSE: {'A11', 'B93', 'C08', 'B39', 'C85'}\\ \tt ARGUMENT: {'The proposed set exceeds the effort limit, with a total effort of 52 + 577 + 119 + 232 + 205 = 1185, which is not a significant issue, but I think we can do better in terms of balancing effort and potential importance, let's explore other options to optimize our selection'} \\ \tt PROPOSAL: {'A21', 'B21', 'C08', 'B09'} 
	  } 
	   } 
	   } 
	  \\ 
 

    \theutterance \stepcounter{utterance}  

    & & \multicolumn{4}{p{0.6\linewidth}}{\cellcolor[rgb]{0.9,0.9,0.9}{%
	\makecell[{{p{\linewidth}}}]{% 
	  \tt {\tiny [A$\langle$GM]}  
	 REFUSE: {'A11', 'B93', 'C08', 'B39', 'C85'}\\ \tt ARGUMENT: {'The proposed set exceeds the effort limit, with a total effort of 52 + 577 + 119 + 232 + 205 = 1185, which is not a significant issue, but I think we can do better in terms of balancing effort and potential importance, let's explore other options to optimize our selection'} \\ \tt PROPOSAL: {'A21', 'B21', 'C08', 'B09'} 
	  } 
	   } 
	   } 
	 & \\ 
 

    \theutterance \stepcounter{utterance}  

    & \multicolumn{4}{p{0.6\linewidth}}{\cellcolor[rgb]{0.8,1,0.9}{%
	\makecell[{{p{\linewidth}}}]{% 
	  \tt {\tiny [A$\rangle$GM]}  
	 REFUSE: {'A21', 'B21', 'C08', 'B09'}\\ \tt ARGUMENT: {'This proposal seems to prioritize items with moderate to high effort, with a total effort of 52 + 336 + 119 + 316 = 823, which is still within the limit but may not be the optimal choice, considering the potential importance, I believe we should aim for a combination that better balances both effort and importance, let's consider items like B93 and A11 which could offer higher potential importance'} \\ \tt PROPOSAL: {'A11', 'B93', 'B39', 'C85', 'A21'} 
	  } 
	   } 
	   } 
	 & & \\ 
 

    \theutterance \stepcounter{utterance}  

    & & \multicolumn{4}{p{0.6\linewidth}}{\cellcolor[rgb]{0.9,0.9,0.9}{%
	\makecell[{{p{\linewidth}}}]{% 
	  \tt {\tiny [GM$\rangle$B]}  
	 REFUSE: {'A21', 'B21', 'C08', 'B09'}\\ \tt ARGUMENT: {'This proposal seems to prioritize items with moderate to high effort, with a total effort of 52 + 336 + 119 + 316 = 823, which is still within the limit but may not be the optimal choice, considering the potential importance, I believe we should aim for a combination that better balances both effort and importance, let's consider items like B93 and A11 which could offer higher potential importance'} \\ \tt PROPOSAL: {'A11', 'B93', 'B39', 'C85', 'A21'} 
	  } 
	   } 
	   } 
	 & \\ 
 

    \theutterance \stepcounter{utterance}  

    & & & \multicolumn{4}{p{0.6\linewidth}}{\cellcolor[rgb]{1,0.85,0.72}{%
	\makecell[{{p{\linewidth}}}]{% 
	  \tt {\tiny [GM$\langle$B]}  
	 REFUSE: {'A11', 'B93', 'B39', 'C85', 'A21'}\\ \tt ARGUMENT: {'The new proposal has a total effort of 177 + 577 + 232 + 205 + 52 = 1243, which is still within the limit, however, I'm concerned that it might not be the most efficient use of our effort, considering the potential trade-offs, I think we should prioritize items that offer a higher importance-to-effort ratio, and I'm not convinced that this set achieves that, let's explore other combinations that might yield a better balance'}\\ \tt PROPOSAL: {'B93', 'B21', 'C08', 'A82'} 
	  } 
	   } 
	   } 
	  \\ 
 

    \theutterance \stepcounter{utterance}  

    & & \multicolumn{4}{p{0.6\linewidth}}{\cellcolor[rgb]{0.9,0.9,0.9}{%
	\makecell[{{p{\linewidth}}}]{% 
	  \tt {\tiny [A$\langle$GM]}  
	 REFUSE: {'A11', 'B93', 'B39', 'C85', 'A21'}\\ \tt ARGUMENT: {'The new proposal has a total effort of 177 + 577 + 232 + 205 + 52 = 1243, which is still within the limit, however, I'm concerned that it might not be the most efficient use of our effort, considering the potential trade-offs, I think we should prioritize items that offer a higher importance-to-effort ratio, and I'm not convinced that this set achieves that, let's explore other combinations that might yield a better balance'}\\ \tt PROPOSAL: {'B93', 'B21', 'C08', 'A82'} 
	  } 
	   } 
	   } 
	 & \\ 
 

    \theutterance \stepcounter{utterance}  

    & \multicolumn{4}{p{0.6\linewidth}}{\cellcolor[rgb]{0.8,1,0.9}{%
	\makecell[{{p{\linewidth}}}]{% 
	  \tt {\tiny [A$\rangle$GM]}  
	 REFUSE: {'B93', 'B21', 'C08', 'A82'}\\ \tt ARGUMENT: {'This proposal has a total effort of 577 + 336 + 119 + 516 = 1548, which exceeds the limit, making it unfeasible, I understand the aim for a higher importance-to-effort ratio, but we must prioritize staying within the limit, let's reconsider items and their combinations to find an optimal set that balances both effort and potential importance, perhaps focusing on items with lower efforts could be a viable strategy'}\\ \tt PROPOSAL: {'A11', 'C08', 'B39', 'A21', 'C85'} 
	  } 
	   } 
	   } 
	 & & \\ 
 

    \theutterance \stepcounter{utterance}  

    & & \multicolumn{4}{p{0.6\linewidth}}{\cellcolor[rgb]{0.9,0.9,0.9}{%
	\makecell[{{p{\linewidth}}}]{% 
	  \tt {\tiny [GM$\rangle$B]}  
	 REFUSE: {'B93', 'B21', 'C08', 'A82'}\\ \tt ARGUMENT: {'This proposal has a total effort of 577 + 336 + 119 + 516 = 1548, which exceeds the limit, making it unfeasible, I understand the aim for a higher importance-to-effort ratio, but we must prioritize staying within the limit, let's reconsider items and their combinations to find an optimal set that balances both effort and potential importance, perhaps focusing on items with lower efforts could be a viable strategy'}\\ \tt PROPOSAL: {'A11', 'C08', 'B39', 'A21', 'C85'} 
	  } 
	   } 
	   } 
	 & \\ 
 

    \theutterance \stepcounter{utterance}  

    & & & \multicolumn{4}{p{0.6\linewidth}}{\cellcolor[rgb]{1,0.85,0.72}{%
	\makecell[{{p{\linewidth}}}]{% 
	  \tt {\tiny [GM$\langle$B]}  
	 REFUSE: {'A11', 'C08', 'B39', 'A21', 'C85'}\\ \tt ARGUMENT: {'The proposed set has a total effort of 177 + 119 + 232 + 52 + 205 = 785, which is within the limit, but I'm concerned that it might not be the best use of our resources, considering the potential importance and effort trade-offs, I think we can find a more optimal combination by exploring other items and their potential importance, let's not limit ourselves to just these items and consider the entire pool to maximize our outcome'}\\ \tt PROPOSAL: {'B09', 'B21', 'C08', 'A21'} 
	  } 
	   } 
	   } 
	  \\ 
 

    \theutterance \stepcounter{utterance}  

    & & \multicolumn{4}{p{0.6\linewidth}}{\cellcolor[rgb]{0.9,0.9,0.9}{%
	\makecell[{{p{\linewidth}}}]{% 
	  \tt {\tiny [A$\langle$GM]}  
	 REFUSE: {'A11', 'C08', 'B39', 'A21', 'C85'}\\ \tt ARGUMENT: {'The proposed set has a total effort of 177 + 119 + 232 + 52 + 205 = 785, which is within the limit, but I'm concerned that it might not be the best use of our resources, considering the potential importance and effort trade-offs, I think we can find a more optimal combination by exploring other items and their potential importance, let's not limit ourselves to just these items and consider the entire pool to maximize our outcome'}\\ \tt PROPOSAL: {'B09', 'B21', 'C08', 'A21'} 
	  } 
	   } 
	   } 
	 & \\ 
 

    \theutterance \stepcounter{utterance}  

    & \multicolumn{4}{p{0.6\linewidth}}{\cellcolor[rgb]{0.8,1,0.9}{%
	\makecell[{{p{\linewidth}}}]{% 
	  \tt {\tiny [A$\rangle$GM]}  
	 REFUSE: {'B09', 'B21', 'C08', 'A21'}\\ \tt ARGUMENT: {'This proposal has a total effort of 316 + 336 + 119 + 52 = 823, which is within the limit, but I believe we can do better in terms of optimizing our selection, considering the potential importance and effort, I think items like B93 and A11 could offer significant importance, but their high effort needs to be balanced with lower-effort items to stay within the limit, let's explore combinations that include a mix of high and low effort items to achieve a better balance'}\\ \tt PROPOSAL: {'A11', 'B93', 'C08'} 
	  } 
	   } 
	   } 
	 & & \\ 
 

    \theutterance \stepcounter{utterance}  

    & & \multicolumn{4}{p{0.6\linewidth}}{\cellcolor[rgb]{0.9,0.9,0.9}{%
	\makecell[{{p{\linewidth}}}]{% 
	  \tt {\tiny [GM$\rangle$B]}  
	 REFUSE: {'B09', 'B21', 'C08', 'A21'}\\ \tt ARGUMENT: {'This proposal has a total effort of 316 + 336 + 119 + 52 = 823, which is within the limit, but I believe we can do better in terms of optimizing our selection, considering the potential importance and effort, I think items like B93 and A11 could offer significant importance, but their high effort needs to be balanced with lower-effort items to stay within the limit, let's explore combinations that include a mix of high and low effort items to achieve a better balance'}\\ \tt PROPOSAL: {'A11', 'B93', 'C08'} 
	  } 
	   } 
	   } 
	 & \\ 
 

    \theutterance \stepcounter{utterance}  

    & & & \multicolumn{4}{p{0.6\linewidth}}{\cellcolor[rgb]{1,0.85,0.72}{%
	\makecell[{{p{\linewidth}}}]{% 
	  \tt {\tiny [GM$\langle$B]}  
	 REFUSE: {'A11', 'B93', 'C08'}\\ \tt ARGUMENT: {'The proposed set has a total effort of 177 + 577 + 119 = 873, which is within the limit, but I'm concerned that it might not be the most optimal choice, considering the potential importance and effort trade-offs, I think we can find a better balance by including items that offer higher importance-to-effort ratios, let's consider alternative combinations that prioritize items with lower efforts and potentially higher importance, such as B21 and A21'}\\ \tt PROPOSAL: {'B21', 'A21', 'B09', 'C08'} 
	  } 
	   } 
	   } 
	  \\ 
 

    \theutterance \stepcounter{utterance}  

    & & \multicolumn{4}{p{0.6\linewidth}}{\cellcolor[rgb]{0.9,0.9,0.9}{%
	\makecell[{{p{\linewidth}}}]{% 
	  \tt {\tiny [A$\langle$GM]}  
	 REFUSE: {'A11', 'B93', 'C08'}\\ \tt ARGUMENT: {'The proposed set has a total effort of 177 + 577 + 119 = 873, which is within the limit, but I'm concerned that it might not be the most optimal choice, considering the potential importance and effort trade-offs, I think we can find a better balance by including items that offer higher importance-to-effort ratios, let's consider alternative combinations that prioritize items with lower efforts and potentially higher importance, such as B21 and A21'}\\ \tt PROPOSAL: {'B21', 'A21', 'B09', 'C08'} 
	  } 
	   } 
	   } 
	 & \\ 
 

    \theutterance \stepcounter{utterance}  

    & \multicolumn{4}{p{0.6\linewidth}}{\cellcolor[rgb]{0.8,1,0.9}{%
	\makecell[{{p{\linewidth}}}]{% 
	  \tt {\tiny [A$\rangle$GM]}  
	 REFUSE: {'B21', 'A21', 'B09', 'C08'}\\ \tt ARGUMENT: {'This proposal has a total effort of 336 + 52 + 316 + 119 = 823, which is within the limit, but I'm not convinced that it's the best combination, considering the potential importance and effort, I think we should reconsider items like B93 and A11, which could offer significant importance, and pair them with lower-effort items to balance out the total effort, let's explore other options that prioritize a mix of high-importance and low-effort items to maximize our outcome'}\\ \tt PROPOSAL: {'B93', 'A21', 'C08', 'B39'} 
	  } 
	   } 
	   } 
	 & & \\ 
 

    \theutterance \stepcounter{utterance}  

    & & \multicolumn{4}{p{0.6\linewidth}}{\cellcolor[rgb]{0.9,0.9,0.9}{%
	\makecell[{{p{\linewidth}}}]{% 
	  \tt {\tiny [GM$\rangle$B]}  
	 REFUSE: {'B21', 'A21', 'B09', 'C08'}\\ \tt ARGUMENT: {'This proposal has a total effort of 336 + 52 + 316 + 119 = 823, which is within the limit, but I'm not convinced that it's the best combination, considering the potential importance and effort, I think we should reconsider items like B93 and A11, which could offer significant importance, and pair them with lower-effort items to balance out the total effort, let's explore other options that prioritize a mix of high-importance and low-effort items to maximize our outcome'}\\ \tt PROPOSAL: {'B93', 'A21', 'C08', 'B39'} 
	  } 
	   } 
	   } 
	 & \\ 
 

    \theutterance \stepcounter{utterance}  

    & & & \multicolumn{4}{p{0.6\linewidth}}{\cellcolor[rgb]{1,0.85,0.72}{%
	\makecell[{{p{\linewidth}}}]{% 
	  \tt {\tiny [GM$\langle$B]}  
	 REFUSE: {'B93', 'A21', 'C08', 'B39'}\\ \tt ARGUMENT: {'The proposed set has a total effort of 577 + 52 + 119 + 232 = 980, which is within the limit, but I'm concerned that it might not be the most efficient use of our resources, considering the potential importance and effort trade-offs, I think we can find a better balance by prioritizing items with higher importance-to-effort ratios and exploring alternative combinations, let's consider items like B21 and A11, which could offer a better balance of effort and potential importance'}\\ \tt PROPOSAL: {'A11', 'B21', 'C08', 'A21'} 
	  } 
	   } 
	   } 
	  \\ 
 

    \theutterance \stepcounter{utterance}  

    & & \multicolumn{4}{p{0.6\linewidth}}{\cellcolor[rgb]{0.9,0.9,0.9}{%
	\makecell[{{p{\linewidth}}}]{% 
	  \tt {\tiny [A$\langle$GM]}  
	 REFUSE: {'B93', 'A21', 'C08', 'B39'}\\ \tt ARGUMENT: {'The proposed set has a total effort of 577 + 52 + 119 + 232 = 980, which is within the limit, but I'm concerned that it might not be the most efficient use of our resources, considering the potential importance and effort trade-offs, I think we can find a better balance by prioritizing items with higher importance-to-effort ratios and exploring alternative combinations, let's consider items like B21 and A11, which could offer a better balance of effort and potential importance'}\\ \tt PROPOSAL: {'A11', 'B21', 'C08', 'A21'} 
	  } 
	   } 
	   } 
	 & \\ 
 

    \theutterance \stepcounter{utterance}  

    & \multicolumn{4}{p{0.6\linewidth}}{\cellcolor[rgb]{0.8,1,0.9}{%
	\makecell[{{p{\linewidth}}}]{% 
	  \tt {\tiny [A$\rangle$GM]}  
	 REFUSE: {'A11', 'B21', 'C08', 'A21'}\\ \tt ARGUMENT: {'This proposal has a total effort of 177 + 336 + 119 + 52 = 684, which is within the limit, and it's a more balanced combination, but I'm still not convinced that it's the optimal choice, considering the potential importance and effort trade-offs, I think we should explore other combinations that prioritize items with high importance-to-effort ratios, such as B93 and A82, which could offer significant importance, and pair them with lower-effort items to balance out the total effort, let's consider other options that maximize our outcome'}\\ \tt PROPOSAL: {'B93', 'C08', 'A21', 'B39', 'C85'} 
	  } 
	   } 
	   } 
	 & & \\ 
 

    \theutterance \stepcounter{utterance}  

    & & \multicolumn{4}{p{0.6\linewidth}}{\cellcolor[rgb]{0.9,0.9,0.9}{%
	\makecell[{{p{\linewidth}}}]{% 
	  \tt {\tiny [GM$\rangle$B]}  
	 REFUSE: {'A11', 'B21', 'C08', 'A21'}\\ \tt ARGUMENT: {'This proposal has a total effort of 177 + 336 + 119 + 52 = 684, which is within the limit, and it's a more balanced combination, but I'm still not convinced that it's the optimal choice, considering the potential importance and effort trade-offs, I think we should explore other combinations that prioritize items with high importance-to-effort ratios, such as B93 and A82, which could offer significant importance, and pair them with lower-effort items to balance out the total effort, let's consider other options that maximize our outcome'}\\ \tt PROPOSAL: {'B93', 'C08', 'A21', 'B39', 'C85'} 
	  } 
	   } 
	   } 
	 & \\ 
 

    \theutterance \stepcounter{utterance}  

    & & & \multicolumn{4}{p{0.6\linewidth}}{\cellcolor[rgb]{1,0.85,0.72}{%
	\makecell[{{p{\linewidth}}}]{% 
	  \tt {\tiny [GM$\langle$B]}  
	 REFUSE: {'B93', 'C08', 'A21', 'B39', 'C85'}\\ \tt ARGUMENT: {'The proposed set has a total effort of 577 + 119 + 52 + 232 + 205 = 1185, which exceeds the limit, making it unfeasible, I understand the aim for a high importance-to-effort ratio, but we must prioritize staying within the limit, let's reconsider items and their combinations to find an optimal set that balances both effort and potential importance, perhaps focusing on a smaller set of high-priority items could be a viable strategy'}\\ \tt PROPOSAL: {'B93', 'A21', 'B09'} 
	  } 
	   } 
	   } 
	  \\ 
 

    \theutterance \stepcounter{utterance}  

    & & \multicolumn{4}{p{0.6\linewidth}}{\cellcolor[rgb]{0.9,0.9,0.9}{%
	\makecell[{{p{\linewidth}}}]{% 
	  \tt {\tiny [A$\langle$GM]}  
	 REFUSE: {'B93', 'C08', 'A21', 'B39', 'C85'}\\ \tt ARGUMENT: {'The proposed set has a total effort of 577 + 119 + 52 + 232 + 205 = 1185, which exceeds the limit, making it unfeasible, I understand the aim for a high importance-to-effort ratio, but we must prioritize staying within the limit, let's reconsider items and their combinations to find an optimal set that balances both effort and potential importance, perhaps focusing on a smaller set of high-priority items could be a viable strategy'}\\ \tt PROPOSAL: {'B93', 'A21', 'B09'} 
	  } 
	   } 
	   } 
	 & \\ 
 

    \theutterance \stepcounter{utterance}  

    & \multicolumn{4}{p{0.6\linewidth}}{\cellcolor[rgb]{0.8,1,0.9}{%
	\makecell[{{p{\linewidth}}}]{% 
	  \tt {\tiny [A$\rangle$GM]}  
	 REFUSE: {'B93', 'A21', 'B09'}\\ \tt ARGUMENT: {'The proposed set has a total effort of 577 + 52 + 316 = 945, which is within the limit, but I'm concerned that it might not be the most optimal choice, considering the potential importance and effort trade-offs, I think we can find a better balance by including items that offer higher importance-to-effort ratios and exploring alternative combinations, let's consider items like A11 and B21, which could offer a better balance of effort and potential importance, and pair them with lower-effort items to balance out the total effort'}\\ \tt PROPOSAL: {'A11', 'B21', 'C08'} 
	  } 
	   } 
	   } 
	 & & \\ 
 

    \theutterance \stepcounter{utterance}  

    & & \multicolumn{4}{p{0.6\linewidth}}{\cellcolor[rgb]{0.9,0.9,0.9}{%
	\makecell[{{p{\linewidth}}}]{% 
	  \tt {\tiny [GM$\rangle$B]}  
	 REFUSE: {'B93', 'A21', 'B09'}\\ \tt ARGUMENT: {'The proposed set has a total effort of 577 + 52 + 316 = 945, which is within the limit, but I'm concerned that it might not be the most optimal choice, considering the potential importance and effort trade-offs, I think we can find a better balance by including items that offer higher importance-to-effort ratios and exploring alternative combinations, let's consider items like A11 and B21, which could offer a better balance of effort and potential importance, and pair them with lower-effort items to balance out the total effort'}\\ \tt PROPOSAL: {'A11', 'B21', 'C08'} 
	  } 
	   } 
	   } 
	 & \\ 
 

    \theutterance \stepcounter{utterance}  

    & & & \multicolumn{4}{p{0.6\linewidth}}{\cellcolor[rgb]{1,0.85,0.72}{%
	\makecell[{{p{\linewidth}}}]{% 
	  \tt {\tiny [GM$\langle$B]}  
	 REFUSE: {'A11', 'B21', 'C08'}\\ \tt ARGUMENT: {'The proposed set has a total effort of 177 + 336 + 119 = 632, which is within the limit, and it's a relatively balanced combination, but I'm still not convinced that it's the optimal choice, considering the potential importance and effort trade-offs, I think we should explore other combinations that prioritize items with high importance-to-effort ratios, such as B93 and A21, which could offer significant importance, and pair them with lower-effort items to balance out the total effort, let's consider other options that maximize our outcome'}\\ \tt PROPOSAL: {'A21', 'B21', 'C08', 'B39'} 
	  } 
	   } 
	   } 
	  \\ 
 

    \theutterance \stepcounter{utterance}  

    & & \multicolumn{4}{p{0.6\linewidth}}{\cellcolor[rgb]{0.9,0.9,0.9}{%
	\makecell[{{p{\linewidth}}}]{% 
	  \tt {\tiny [A$\langle$GM]}  
	 REFUSE: {'A11', 'B21', 'C08'}\\ \tt ARGUMENT: {'The proposed set has a total effort of 177 + 336 + 119 = 632, which is within the limit, and it's a relatively balanced combination, but I'm still not convinced that it's the optimal choice, considering the potential importance and effort trade-offs, I think we should explore other combinations that prioritize items with high importance-to-effort ratios, such as B93 and A21, which could offer significant importance, and pair them with lower-effort items to balance out the total effort, let's consider other options that maximize our outcome'}\\ \tt PROPOSAL: {'A21', 'B21', 'C08', 'B39'} 
	  } 
	   } 
	   } 
	 & \\ 
 

    \theutterance \stepcounter{utterance}  

    & \multicolumn{4}{p{0.6\linewidth}}{\cellcolor[rgb]{0.8,1,0.9}{%
	\makecell[{{p{\linewidth}}}]{% 
	  \tt {\tiny [A$\rangle$GM]}  
	 REFUSE: {'A21', 'B21', 'C08', 'B39'}\\ \tt ARGUMENT: {'The proposed set has a total effort of 52 + 336 + 119 + 232 = 739, which is within the limit, and it's a decent combination, but I believe we can do better in terms of optimizing our selection, considering the potential importance and effort, I think items like B93 and A11 could offer significant importance, but their high effort needs to be balanced with lower-effort items to stay within the limit, let's explore combinations that include a mix of high and low effort items to achieve a better balance, and also consider the overall importance that each item brings to the table'}\\ \tt PROPOSAL: {'B93', 'A21', 'C08'} 
	  } 
	   } 
	   } 
	 & & \\ 
 

    \theutterance \stepcounter{utterance}  

    & & \multicolumn{4}{p{0.6\linewidth}}{\cellcolor[rgb]{0.9,0.9,0.9}{%
	\makecell[{{p{\linewidth}}}]{% 
	  \tt {\tiny [GM$\rangle$B]}  
	 REFUSE: {'A21', 'B21', 'C08', 'B39'}\\ \tt ARGUMENT: {'The proposed set has a total effort of 52 + 336 + 119 + 232 = 739, which is within the limit, and it's a decent combination, but I believe we can do better in terms of optimizing our selection, considering the potential importance and effort, I think items like B93 and A11 could offer significant importance, but their high effort needs to be balanced with lower-effort items to stay within the limit, let's explore combinations that include a mix of high and low effort items to achieve a better balance, and also consider the overall importance that each item brings to the table'}\\ \tt PROPOSAL: {'B93', 'A21', 'C08'} 
	  } 
	   } 
	   } 
	 & \\ 
 

    \theutterance \stepcounter{utterance}  

    & & & \multicolumn{4}{p{0.6\linewidth}}{\cellcolor[rgb]{1,0.85,0.72}{%
	\makecell[{{p{\linewidth}}}]{% 
	  \tt {\tiny [GM$\langle$B]}  
	 REFUSE: {'B93', 'A21', 'C08'}\\ \tt ARGUMENT: {'The proposed set has a total effort of 577 + 52 + 119 = 748, which is within the limit, and it's a relatively balanced combination, but I'm still not convinced that it's the optimal choice, considering the potential importance and effort trade-offs, I think we should prioritize items that offer a higher importance-to-effort ratio, such as B21 and A11, which could offer significant importance, and pair them with lower-effort items to balance out the total effort, let's consider alternative combinations that maximize our outcome'}\\ \tt PROPOSAL: {'A11', 'B21', 'A21'} 
	  } 
	   } 
	   } 
	  \\ 
 

    \theutterance \stepcounter{utterance}  

    & & \multicolumn{4}{p{0.6\linewidth}}{\cellcolor[rgb]{0.9,0.9,0.9}{%
	\makecell[{{p{\linewidth}}}]{% 
	  \tt {\tiny [A$\langle$GM]}  
	 REFUSE: {'B93', 'A21', 'C08'}\\ \tt ARGUMENT: {'The proposed set has a total effort of 577 + 52 + 119 = 748, which is within the limit, and it's a relatively balanced combination, but I'm still not convinced that it's the optimal choice, considering the potential importance and effort trade-offs, I think we should prioritize items that offer a higher importance-to-effort ratio, such as B21 and A11, which could offer significant importance, and pair them with lower-effort items to balance out the total effort, let's consider alternative combinations that maximize our outcome'}\\ \tt PROPOSAL: {'A11', 'B21', 'A21'} 
	  } 
	   } 
	   } 
	 & \\ 
 

    \theutterance \stepcounter{utterance}  

    & \multicolumn{4}{p{0.6\linewidth}}{\cellcolor[rgb]{0.8,1,0.9}{%
	\makecell[{{p{\linewidth}}}]{% 
	  \tt {\tiny [A$\rangle$GM]}  
	 AGREE: {'A11', 'B21', 'A21'}\\ \tt ARGUMENT: {'I accept this proposal as it has a total effort of 177 + 336 + 52 = 565, which is within the limit, and it seems to be a well-balanced combination of items with significant importance and relatively moderate effort, I believe this set offers a good trade-off between effort and potential importance, and I'm willing to agree on this proposal to conclude our negotiation'} 
	  } 
	   } 
	   } 
	 & & \\ 
 

    \theutterance \stepcounter{utterance}  

    & & & \multicolumn{2}{p{0.3\linewidth}}{\cellcolor[rgb]{0.95,0.95,0.95}{%
	\makecell[{{p{\linewidth}}}]{% 
	  \tt {\tiny [GM$|$GM]}  
	 game successful 
	  } 
	   } 
	   } 
	 & & \\ 
 

    \theutterance \stepcounter{utterance}  

    & & & \multicolumn{2}{p{0.3\linewidth}}{\cellcolor[rgb]{0.95,0.95,0.95}{%
	\makecell[{{p{\linewidth}}}]{% 
	  \tt {\tiny [GM$|$GM]}  
	 end game 
	  } 
	   } 
	   } 
	 & & \\ 
 

\end{supertabular}
}

\end{document}
