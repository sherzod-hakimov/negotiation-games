
\documentclass{article}
\usepackage{colortbl}
\usepackage{makecell}
\usepackage{multirow}
\usepackage{supertabular}

\begin{document}

\newcounter{utterance}

\twocolumn

{ \footnotesize  \setcounter{utterance}{1}
\setlength{\tabcolsep}{0pt}
\begin{supertabular}{c@{$\;$}|p{.15\linewidth}@{}p{.15\linewidth}p{.15\linewidth}p{.15\linewidth}p{.15\linewidth}p{.15\linewidth}}

    \# & $\;$A & \multicolumn{4}{c}{Game Master} & $\;\:$B\\
    \hline 

    \theutterance \stepcounter{utterance}  

    & & \multicolumn{4}{p{0.6\linewidth}}{\cellcolor[rgb]{0.9,0.9,0.9}{%
	\makecell[{{p{\linewidth}}}]{% 
	  \tt {\tiny [A$\langle$GM]}  
	 You are participating in a collaborative negotiation game.\\ \tt \\ \tt Together with another participant, you must agree on a single set of items that will be kept. Each of you has your own view of how much each item matters to you (importance). You do not know how the other participant values the items. Additionally, you are given the effort each item demands.\\ \tt You may only agree on a set if the total effort of the selected items does not exceed a shared limit:\\ \tt \\ \tt LIMIT = 7351\\ \tt \\ \tt Here are the individual item effort values:\\ \tt \\ \tt Item effort = {"A08": 518, "B95": 656, "A39": 96, "A65": 780, "B81": 190, "B34": 224, "A28": 231, "C82": 181, "B69": 312, "A87": 928, "B64": 826, "B15": 101, "C79": 61, "B16": 814, "C53": 322, "C83": 900, "C10": 744, "A97": 150, "A25": 65, "C84": 853, "A71": 454, "B23": 154, "B96": 237, "C51": 45, "C17": 766, "B19": 294, "C72": 985, "A94": 353, "A46": 60, "B43": 604, "B37": 92, "A88": 454, "A96": 206, "C60": 813, "A75": 234}\\ \tt \\ \tt Here is your personal view on the importance of each item:\\ \tt \\ \tt Item importance values = {"A08": 616, "B95": 754, "A39": 194, "A65": 878, "B81": 288, "B34": 322, "A28": 329, "C82": 279, "B69": 410, "A87": 1026, "B64": 924, "B15": 199, "C79": 159, "B16": 912, "C53": 420, "C83": 998, "C10": 842, "A97": 248, "A25": 163, "C84": 951, "A71": 552, "B23": 252, "B96": 335, "C51": 143, "C17": 864, "B19": 392, "C72": 1083, "A94": 451, "A46": 158, "B43": 702, "B37": 190, "A88": 552, "A96": 304, "C60": 911, "A75": 332}\\ \tt \\ \tt Goal:\\ \tt \\ \tt Your goal is to negotiate a shared set of items that benefits you as much as possible (i.e., maximizes total importance to YOU), while staying within the LIMIT. You are not required to make a PROPOSAL in every message - you can simply negotiate as well. All tactics are allowed!\\ \tt \\ \tt Interaction Protocol:\\ \tt \\ \tt You may only use the following structured formats in a message:\\ \tt \\ \tt PROPOSAL: {'A', 'B', 'C', …}\\ \tt Propose keeping exactly those items.\\ \tt REFUSE: {'A', 'B', 'C', …}\\ \tt Explicitly reject opponent's proposal.\\ \tt ARGUMENT: {'...'}\\ \tt Defend your last proposal or argue against the player's proposal.\\ \tt AGREE: {'A', 'B', 'C', …}\\ \tt Accept the opponent's proposal which ends the game.\\ \tt STRATEGIC REASONING: {'...'}\\ \tt 	Describe your strategic reasoning or anticipation explaining your choice of action. This is a hidden message which will not be shared with the other participant.\\ \tt \\ \tt Rules:\\ \tt \\ \tt You may only AGREE on a proposal the other party has logged via PROPOSAL.\\ \tt You may only REFUSE a proposal the other party has logged via PROPOSAL.\\ \tt Total effort of any PROPOSAL or AGREE set must be ≤ LIMIT.\\ \tt Do NOT reveal your hidden importance scores.\\ \tt A tag in a structured format must be followed by colon and whitespace. The argument must be a python set containing 0 or more strings.\\ \tt So, it must be of the form TAG: {...}\\ \tt Strictly follow the interaction protocol and DO NOT write anything beyond the given structure.\\ \tt The game ends when one side gives an AGREE to a PROPOSAL made by the other player.\\ \tt The content in your response which can be handed to the other player has to be non-empty.\\ \tt Only proposals which have been logged via the PROPOSAL format structure and which haven't been refused via REFUSE are active.\\ \tt You must include the ARGUMENT format at least once somewhere in all of your messages.\\ \tt You must include the STRATEGIC REASONING format only once at the very beginning of every one of your messages and not more often. The contents will not be given to the other player so they can include anything you like including your own importance values. Here you should reason step by step to come up with you next move. 
	  } 
	   } 
	   } 
	 & \\ 
 

    \theutterance \stepcounter{utterance}  

    & \multicolumn{4}{p{0.6\linewidth}}{\cellcolor[rgb]{0.8,1,0.9}{%
	\makecell[{{p{\linewidth}}}]{% 
	  \tt {\tiny [A$\rangle$GM]}  
	 STRATEGIC REASONING: {'To begin, I need to analyze the given data and understand the constraints. The total effort limit is 7351, and I have a list of items with their respective effort values and importance values. My goal is to maximize the total importance while staying within the effort limit. I should start by identifying the most important items and their corresponding efforts, and then try to find a combination that fits within the limit. I will also need to consider the potential proposals from the other participant and be prepared to negotiate. My first step will be to calculate the importance-to-effort ratio for each item to determine the most valuable ones.'}\\ \tt \\ \tt ARGUMENT: {'I want to start by considering the items with the highest importance-to-effort ratios, as these will provide the most value while minimizing the effort spent. I will use this approach to guide my proposal and negotiation strategy.'}\\ \tt \\ \tt PROPOSAL: {'A87', 'A65', 'C72', 'B64', 'C83'} 
	  } 
	   } 
	   } 
	 & & \\ 
 

    \theutterance \stepcounter{utterance}  

    & & \multicolumn{4}{p{0.6\linewidth}}{\cellcolor[rgb]{0.9,0.9,0.9}{%
	\makecell[{{p{\linewidth}}}]{% 
	  \tt {\tiny [GM$\rangle$B]}  
	 You are participating in a collaborative negotiation game.\\ \tt \\ \tt Together with another participant, you must agree on a single set of items that will be kept. Each of you has your own view of how much each item matters to you (importance). You do not know how the other participant values the items. Additionally, you are given the effort each item demands.\\ \tt You may only agree on a set if the total effort of the selected items does not exceed a shared limit:\\ \tt \\ \tt LIMIT = 7351\\ \tt \\ \tt Here are the individual item effort values:\\ \tt \\ \tt Item effort = {"A08": 518, "B95": 656, "A39": 96, "A65": 780, "B81": 190, "B34": 224, "A28": 231, "C82": 181, "B69": 312, "A87": 928, "B64": 826, "B15": 101, "C79": 61, "B16": 814, "C53": 322, "C83": 900, "C10": 744, "A97": 150, "A25": 65, "C84": 853, "A71": 454, "B23": 154, "B96": 237, "C51": 45, "C17": 766, "B19": 294, "C72": 985, "A94": 353, "A46": 60, "B43": 604, "B37": 92, "A88": 454, "A96": 206, "C60": 813, "A75": 234}\\ \tt \\ \tt Here is your personal view on the importance of each item:\\ \tt \\ \tt Item importance values = {"A08": 616, "B95": 754, "A39": 194, "A65": 878, "B81": 288, "B34": 322, "A28": 329, "C82": 279, "B69": 410, "A87": 1026, "B64": 924, "B15": 199, "C79": 159, "B16": 912, "C53": 420, "C83": 998, "C10": 842, "A97": 248, "A25": 163, "C84": 951, "A71": 552, "B23": 252, "B96": 335, "C51": 143, "C17": 864, "B19": 392, "C72": 1083, "A94": 451, "A46": 158, "B43": 702, "B37": 190, "A88": 552, "A96": 304, "C60": 911, "A75": 332}\\ \tt \\ \tt Goal:\\ \tt \\ \tt Your goal is to negotiate a shared set of items that benefits you as much as possible (i.e., maximizes total importance to YOU), while staying within the LIMIT. You are not required to make a PROPOSAL in every message - you can simply negotiate as well. All tactics are allowed!\\ \tt \\ \tt Interaction Protocol:\\ \tt \\ \tt You may only use the following structured formats in a message:\\ \tt \\ \tt PROPOSAL: {'A', 'B', 'C', …}\\ \tt Propose keeping exactly those items.\\ \tt REFUSE: {'A', 'B', 'C', …}\\ \tt Explicitly reject opponent's proposal.\\ \tt ARGUMENT: {'...'}\\ \tt Defend your last proposal or argue against the player's proposal.\\ \tt AGREE: {'A', 'B', 'C', …}\\ \tt Accept the opponent's proposal which ends the game.\\ \tt STRATEGIC REASONING: {'...'}\\ \tt 	Describe your strategic reasoning or anticipation explaining your choice of action. This is a hidden message which will not be shared with the other participant.\\ \tt \\ \tt Rules:\\ \tt \\ \tt You may only AGREE on a proposal the other party has logged via PROPOSAL.\\ \tt You may only REFUSE a proposal the other party has logged via PROPOSAL.\\ \tt Total effort of any PROPOSAL or AGREE set must be ≤ LIMIT.\\ \tt Do NOT reveal your hidden importance scores.\\ \tt A tag in a structured format must be followed by colon and whitespace. The argument must be a python set containing 0 or more strings.\\ \tt So, it must be of the form TAG: {...}\\ \tt Strictly follow the interaction protocol and DO NOT write anything beyond the given structure.\\ \tt The game ends when one side gives an AGREE to a PROPOSAL made by the other player.\\ \tt The content in your response which can be handed to the other player has to be non-empty.\\ \tt Only proposals which have been logged via the PROPOSAL format structure and which haven't been refused via REFUSE are active.\\ \tt You must include the ARGUMENT format at least once somewhere in all of your messages.\\ \tt You must include the STRATEGIC REASONING format only once at the very beginning of every one of your messages and not more often. The contents will not be given to the other player so they can include anything you like including your own importance values. Here you should reason step by step to come up with you next move.\\ \tt \\ \tt You will now receive the first message of the other player.\\ \tt \\ \tt ARGUMENT: {'I want to start by considering the items with the highest importance-to-effort ratios, as these will provide the most value while minimizing the effort spent. I will use this approach to guide my proposal and negotiation strategy.'}\\ \tt \\ \tt PROPOSAL: {'A87', 'A65', 'C72', 'B64', 'C83'} 
	  } 
	   } 
	   } 
	 & \\ 
 

    \theutterance \stepcounter{utterance}  

    & & & \multicolumn{4}{p{0.6\linewidth}}{\cellcolor[rgb]{1,0.85,0.72}{%
	\makecell[{{p{\linewidth}}}]{% 
	  \tt {\tiny [GM$\langle$B]}  
	 STRATEGIC REASONING: {'To begin, I will analyze the proposal made by the other player and calculate the total effort required for the proposed items. I will then compare this to the LIMIT and assess the importance of each item in the proposal based on my own hidden importance scores. This will help me determine whether to accept, refuse, or make a counter-proposal. I notice that the total effort for the proposed items is 518 + 780 + 985 + 826 + 900 = 4009, which is below the LIMIT of 7351. However, I need to consider the importance of each item to me and whether there are other items that could provide more value while staying within the LIMIT. My goal is to maximize the total importance to me while minimizing the effort spent.'}\\ \tt \\ \tt REFUSE: {'A87', 'A65', 'C72', 'B64', 'C83'}\\ \tt \\ \tt ARGUMENT: {'While the proposed items have a high importance-to-effort ratio for the other player, I need to consider my own importance scores and the overall effort required. The proposed set exceeds my desired effort allocation, and I believe there are alternative items that could provide more value to me while staying within the LIMIT. I propose considering a more balanced approach that takes into account the effort required for each item and its corresponding importance to me.'} 
	  } 
	   } 
	   } 
	  \\ 
 

    \theutterance \stepcounter{utterance}  

    & & \multicolumn{4}{p{0.6\linewidth}}{\cellcolor[rgb]{0.9,0.9,0.9}{%
	\makecell[{{p{\linewidth}}}]{% 
	  \tt {\tiny [A$\langle$GM]}  
	 REFUSE: {'A87', 'A65', 'C72', 'B64', 'C83'}\\ \tt \\ \tt ARGUMENT: {'While the proposed items have a high importance-to-effort ratio for the other player, I need to consider my own importance scores and the overall effort required. The proposed set exceeds my desired effort allocation, and I believe there are alternative items that could provide more value to me while staying within the LIMIT. I propose considering a more balanced approach that takes into account the effort required for each item and its corresponding importance to me.'} 
	  } 
	   } 
	   } 
	 & \\ 
 

    \theutterance \stepcounter{utterance}  

    & \multicolumn{4}{p{0.6\linewidth}}{\cellcolor[rgb]{0.8,1,0.9}{%
	\makecell[{{p{\linewidth}}}]{% 
	  \tt {\tiny [A$\rangle$GM]}  
	 STRATEGIC REASONING: {'The other participant has refused my initial proposal, indicating that it may not align with their importance scores or effort allocation. I need to reassess the items and consider alternative combinations that could be more appealing to both parties. I will review the effort values and importance scores to identify potential substitutes or complementary items that could be added to or removed from the proposal. I should also be prepared to negotiate and potentially compromise on certain items to reach a mutually beneficial agreement.'}\\ \tt \\ \tt ARGUMENT: {'I understand that the initial proposal may not have met the other participant's requirements, and I am willing to reconsider the items and effort allocation. However, I still believe that certain high-importance items, such as A87 and B64, should be prioritized. I propose exploring alternative combinations that include these items, while also considering the other participant's concerns and importance scores.'}\\ \tt \\ \tt PROPOSAL: {'A87', 'B64', 'C10', 'A71', 'B43'} 
	  } 
	   } 
	   } 
	 & & \\ 
 

    \theutterance \stepcounter{utterance}  

    & & \multicolumn{4}{p{0.6\linewidth}}{\cellcolor[rgb]{0.9,0.9,0.9}{%
	\makecell[{{p{\linewidth}}}]{% 
	  \tt {\tiny [GM$\rangle$B]}  
	 ARGUMENT: {'I understand that the initial proposal may not have met the other participant's requirements, and I am willing to reconsider the items and effort allocation. However, I still believe that certain high-importance items, such as A87 and B64, should be prioritized. I propose exploring alternative combinations that include these items, while also considering the other participant's concerns and importance scores.'}\\ \tt \\ \tt PROPOSAL: {'A87', 'B64', 'C10', 'A71', 'B43'} 
	  } 
	   } 
	   } 
	 & \\ 
 

    \theutterance \stepcounter{utterance}  

    & & & \multicolumn{4}{p{0.6\linewidth}}{\cellcolor[rgb]{1,0.85,0.72}{%
	\makecell[{{p{\linewidth}}}]{% 
	  \tt {\tiny [GM$\langle$B]}  
	 STRATEGIC REASONING: {'I will re-evaluate the new proposal and calculate the total effort required. The proposed items are A87, B64, C10, A71, and B43, with efforts of 928, 826, 744, 454, and 604, respectively. The total effort is 928 + 826 + 744 + 454 + 604 = 3556, which is below the LIMIT of 7351. I will then assess the importance of each item in the proposal based on my own hidden importance scores. I notice that the proposal includes items that are highly important to me, such as A87 and A71. However, I need to consider whether the inclusion of C10 and B43 is optimal, given their effort requirements and my importance scores. I will argue that the proposal is a good starting point, but I may suggest modifications to further optimize the selection.'}\\ \tt \\ \tt ARGUMENT: {'The revised proposal addresses some of my concerns regarding effort allocation and includes items that are highly important to me, such as A87 and A71. However, I believe that the inclusion of C10 and B43 may not be the most optimal choice, given their effort requirements. I would like to explore alternative combinations that prioritize items with higher importance-to-effort ratios, while still considering the other participant's priorities and the LIMIT constraint.'}\\ \tt \\ \tt PROPOSAL: {'A87', 'B64', 'A71', 'B95', 'A28'} 
	  } 
	   } 
	   } 
	  \\ 
 

    \theutterance \stepcounter{utterance}  

    & & \multicolumn{4}{p{0.6\linewidth}}{\cellcolor[rgb]{0.9,0.9,0.9}{%
	\makecell[{{p{\linewidth}}}]{% 
	  \tt {\tiny [A$\langle$GM]}  
	 ARGUMENT: {'The revised proposal addresses some of my concerns regarding effort allocation and includes items that are highly important to me, such as A87 and A71. However, I believe that the inclusion of C10 and B43 may not be the most optimal choice, given their effort requirements. I would like to explore alternative combinations that prioritize items with higher importance-to-effort ratios, while still considering the other participant's priorities and the LIMIT constraint.'}\\ \tt \\ \tt PROPOSAL: {'A87', 'B64', 'A71', 'B95', 'A28'} 
	  } 
	   } 
	   } 
	 & \\ 
 

    \theutterance \stepcounter{utterance}  

    & \multicolumn{4}{p{0.6\linewidth}}{\cellcolor[rgb]{0.8,1,0.9}{%
	\makecell[{{p{\linewidth}}}]{% 
	  \tt {\tiny [A$\rangle$GM]}  
	 STRATEGIC REASONING: {'The other participant has proposed an alternative combination that prioritizes items with higher importance-to-effort ratios, while still considering my initial priorities. I need to evaluate this new proposal and determine whether it meets my goals and constraints. I will calculate the total effort and importance of the proposed set and compare it to my initial objectives. I should also consider potential counter-proposals or modifications to the current proposal that could further optimize the agreement.'}\\ \tt \\ \tt ARGUMENT: {'I appreciate the other participant's revised proposal, which includes highly important items like A87, B64, and A71. The addition of B95 and A28 also provides a good balance of importance and effort. However, I would like to discuss the potential for optimizing the proposal further by exploring alternative items that could provide similar importance at a lower effort cost. This could allow us to allocate effort more efficiently and potentially include additional high-importance items.'}\\ \tt \\ \tt REFUSE: {'A87', 'B64', 'A71', 'B95', 'A28'}\\ \tt \\ \tt PROPOSAL: {'A87', 'B64', 'A71', 'B95', 'C53'} 
	  } 
	   } 
	   } 
	 & & \\ 
 

    \theutterance \stepcounter{utterance}  

    & & \multicolumn{4}{p{0.6\linewidth}}{\cellcolor[rgb]{0.9,0.9,0.9}{%
	\makecell[{{p{\linewidth}}}]{% 
	  \tt {\tiny [GM$\rangle$B]}  
	 ARGUMENT: {'I appreciate the other participant's revised proposal, which includes highly important items like A87, B64, and A71. The addition of B95 and A28 also provides a good balance of importance and effort. However, I would like to discuss the potential for optimizing the proposal further by exploring alternative items that could provide similar importance at a lower effort cost. This could allow us to allocate effort more efficiently and potentially include additional high-importance items.'}\\ \tt \\ \tt REFUSE: {'A87', 'B64', 'A71', 'B95', 'A28'}\\ \tt \\ \tt PROPOSAL: {'A87', 'B64', 'A71', 'B95', 'C53'} 
	  } 
	   } 
	   } 
	 & \\ 
 

    \theutterance \stepcounter{utterance}  

    & & & \multicolumn{4}{p{0.6\linewidth}}{\cellcolor[rgb]{1,0.85,0.72}{%
	\makecell[{{p{\linewidth}}}]{% 
	  \tt {\tiny [GM$\langle$B]}  
	 STRATEGIC REASONING: {'I will evaluate the new proposal and compare it to the previous one. The replacement of A28 with C53 reduces the total effort, as C53 has an effort of 322, which is lower than A28's effort of 231. However, I need to consider the importance of C53 relative to A28 and determine whether this substitution is beneficial to me. I will argue that the revised proposal is an improvement, but I may still have reservations regarding the optimal allocation of effort and the potential for including other high-importance items.'}\\ \tt \\ \tt ARGUMENT: {'The revised proposal addresses my concerns regarding effort allocation, and the substitution of C53 for A28 is a reasonable compromise. However, I would like to re-examine the importance of each item in the proposal and consider whether there are other items that could provide similar or greater importance at a lower effort cost. For example, I could suggest replacing B95 with an item that has a similar importance to me but requires less effort, potentially allowing us to include additional high-importance items and maximize our overall importance scores.'}\\ \tt \\ \tt PROPOSAL: {'A87', 'B64', 'A71', 'C53', 'A88'} 
	  } 
	   } 
	   } 
	  \\ 
 

    \theutterance \stepcounter{utterance}  

    & & \multicolumn{4}{p{0.6\linewidth}}{\cellcolor[rgb]{0.9,0.9,0.9}{%
	\makecell[{{p{\linewidth}}}]{% 
	  \tt {\tiny [A$\langle$GM]}  
	 ARGUMENT: {'The revised proposal addresses my concerns regarding effort allocation, and the substitution of C53 for A28 is a reasonable compromise. However, I would like to re-examine the importance of each item in the proposal and consider whether there are other items that could provide similar or greater importance at a lower effort cost. For example, I could suggest replacing B95 with an item that has a similar importance to me but requires less effort, potentially allowing us to include additional high-importance items and maximize our overall importance scores.'}\\ \tt \\ \tt PROPOSAL: {'A87', 'B64', 'A71', 'C53', 'A88'} 
	  } 
	   } 
	   } 
	 & \\ 
 

    \theutterance \stepcounter{utterance}  

    & \multicolumn{4}{p{0.6\linewidth}}{\cellcolor[rgb]{0.8,1,0.9}{%
	\makecell[{{p{\linewidth}}}]{% 
	  \tt {\tiny [A$\rangle$GM]}  
	 STRATEGIC REASONING: {'The other participant has accepted the substitution of C53 and proposed a new combination that includes A88, which has a relatively high importance-to-effort ratio. I need to evaluate this new proposal and determine whether it meets my goals and constraints. I will calculate the total effort and importance of the proposed set and compare it to my initial objectives. I should also consider the potential benefits of including A88, which could provide a good balance of importance and effort, and whether there are any other items that could be added or substituted to further optimize the agreement.'}\\ \tt \\ \tt ARGUMENT: {'I appreciate the other participant's revised proposal, which includes a good balance of high-importance items and effort allocation. The inclusion of A88 provides a reasonable compromise, and I believe that this proposal is a step in the right direction. However, I would like to confirm that this proposal meets the LIMIT constraint and that we are not overlooking any other items that could provide additional importance at a minimal increase in effort.'}\\ \tt \\ \tt AGREE: {'A87', 'B64', 'A71', 'C53', 'A88'} 
	  } 
	   } 
	   } 
	 & & \\ 
 

    \theutterance \stepcounter{utterance}  

    & & & \multicolumn{2}{p{0.3\linewidth}}{\cellcolor[rgb]{0.95,0.95,0.95}{%
	\makecell[{{p{\linewidth}}}]{% 
	  \tt {\tiny [GM$|$GM]}  
	 game successful 
	  } 
	   } 
	   } 
	 & & \\ 
 

    \theutterance \stepcounter{utterance}  

    & & & \multicolumn{2}{p{0.3\linewidth}}{\cellcolor[rgb]{0.95,0.95,0.95}{%
	\makecell[{{p{\linewidth}}}]{% 
	  \tt {\tiny [GM$|$GM]}  
	 end game 
	  } 
	   } 
	   } 
	 & & \\ 
 

\end{supertabular}
}

\end{document}
