
\documentclass{article}
\usepackage{colortbl}
\usepackage{makecell}
\usepackage{multirow}
\usepackage{supertabular}

\begin{document}

\newcounter{utterance}

\centering \large Interaction Transcript for game `hot\_air\_balloon', experiment `air\_balloon\_survival\_en\_complexity\_hard', episode 4 with llama{-}3.3{-}70b{-}instruct{-}t1.0.
\vspace{24pt}

{ \footnotesize  \setcounter{utterance}{1}
\setlength{\tabcolsep}{0pt}
\begin{supertabular}{c@{$\;$}|p{.15\linewidth}@{}p{.15\linewidth}p{.15\linewidth}p{.15\linewidth}p{.15\linewidth}p{.15\linewidth}}
    \# & $\;$A & \multicolumn{4}{c}{Game Master} & $\;\:$B\\
    \hline

    \theutterance \stepcounter{utterance}  
    & & \multicolumn{4}{p{0.6\linewidth}}{
        \cellcolor[rgb]{0.9,0.9,0.9}{
            \makecell[{{p{\linewidth}}}]{
                \texttt{\tiny{[P1$\langle$GM]}}
                \texttt{You are participating in a collaborative negotiation game.} \\
\\ 
\texttt{Together with another participant, you must agree on a single set of items that will be kept. Each of you has your own view of how much each item matters to you (importance). You do not know how the other participant values the items. Additionally, you are given the effort each item demands.} \\
\texttt{You may only agree on a set if the total effort of the selected items does not exceed a shared limit:} \\
\\ 
\texttt{LIMIT = 6880} \\
\\ 
\texttt{Here are the individual item effort values:} \\
\\ 
\texttt{Item effort = \{"A08": 341, "B95": 804, "A39": 20, "A65": 627, "B81": 10, "B34": 573, "A28": 217, "C82": 481, "B69": 205, "A87": 273, "B64": 303, "B15": 596, "C79": 564, "B16": 535, "C53": 258, "C83": 911, "C10": 240, "A97": 188, "A25": 216, "C84": 401, "A71": 909, "B23": 62, "B96": 245, "C51": 994, "C17": 569, "B19": 718, "C72": 464, "A94": 37, "A46": 340, "B43": 335, "B37": 417, "A88": 123, "A96": 17, "C60": 577, "A75": 190\}} \\
\\ 
\texttt{Here is your personal view on the importance of each item:} \\
\\ 
\texttt{Item importance values = \{"A08": 440, "B95": 903, "A39": 119, "A65": 726, "B81": 109, "B34": 672, "A28": 316, "C82": 580, "B69": 304, "A87": 372, "B64": 402, "B15": 695, "C79": 663, "B16": 634, "C53": 357, "C83": 1010, "C10": 339, "A97": 287, "A25": 315, "C84": 500, "A71": 1008, "B23": 161, "B96": 344, "C51": 1093, "C17": 668, "B19": 817, "C72": 563, "A94": 136, "A46": 439, "B43": 434, "B37": 516, "A88": 222, "A96": 116, "C60": 676, "A75": 289\}} \\
\\ 
\texttt{Goal:} \\
\\ 
\texttt{Your goal is to negotiate a shared set of items that benefits you as much as possible (i.e., maximizes total importance to YOU), while staying within the LIMIT. You are not required to make a PROPOSAL in every message {-} you can simply negotiate as well. All tactics are allowed!} \\
\\ 
\texttt{Interaction Protocol:} \\
\\ 
\texttt{You may only use the following structured formats in a message:} \\
\\ 
\texttt{PROPOSAL: \{'A', 'B', 'C', …\}} \\
\texttt{Propose keeping exactly those items.} \\
\texttt{REFUSE: \{'A', 'B', 'C', …\}} \\
\texttt{Explicitly reject opponent's proposal.} \\
\texttt{ARGUMENT: \{'...'\}} \\
\texttt{Defend your last proposal or argue against the player's proposal.} \\
\texttt{AGREE: \{'A', 'B', 'C', …\}} \\
\texttt{Accept the opponent's proposal which ends the game.} \\
\texttt{STRATEGIC REASONING: \{'...'\}} \\
\texttt{	Describe your strategic reasoning or anticipation explaining your choice of action. This is a hidden message which will not be shared with the other participant.} \\
\\ 
\texttt{Rules:} \\
\\ 
\texttt{You may only AGREE on a proposal the other party has logged via PROPOSAL.} \\
\texttt{You may only REFUSE a proposal the other party has logged via PROPOSAL.} \\
\texttt{Total effort of any PROPOSAL or AGREE set must be ≤ LIMIT.} \\
\texttt{Do NOT reveal your hidden importance scores.} \\
\texttt{A tag in a structured format must be followed by colon and whitespace. The argument must be a python set containing 0 or more strings.} \\
\texttt{So, it must be of the form TAG: \{...\}} \\
\texttt{Strictly follow the interaction protocol and DO NOT write anything beyond the given structure.} \\
\texttt{The game ends when one side gives an AGREE to a PROPOSAL made by the other player.} \\
\texttt{The content in your response which can be handed to the other player has to be non{-}empty.} \\
\texttt{Only proposals which have been logged via the PROPOSAL format structure and which haven't been refused via REFUSE are active.} \\
\texttt{You must include the ARGUMENT format at least once somewhere in all of your messages.} \\
\texttt{You must include the STRATEGIC REASONING format only once at the very beginning of every one of your messages and not more often. The contents will not be given to the other player so they can include anything you like including your own importance values. Here you should reason step by step to come up with you next move.} \\
            }
        }
    }
    & \\ \\

    \theutterance \stepcounter{utterance}  
    & \multicolumn{4}{p{0.6\linewidth}}{
        \cellcolor[rgb]{0.9,0.9,0.9}{
            \makecell[{{p{\linewidth}}}]{
                \texttt{\tiny{[P1$\rangle$GM]}}
                \texttt{STRATEGIC REASONING: \{'I will start by analyzing the given data and try to identify the most important items to me. Since I do not know the other participant's importance values, I will focus on maximizing my own total importance while staying within the LIMIT. I will look for items with high importance and low effort, as well as items with high effort that are still worth including. My goal is to create a proposal that benefits me the most while being acceptable to the other participant.'\}} \\
\\ 
\texttt{PROPOSAL: \{'A71', 'C83', 'B95', 'C79', 'B19'\} } \\
\texttt{ARGUMENT: \{'I chose these items because they have high importance to me and their total effort is within the LIMIT. I believe this proposal is a good starting point for negotiation, and I am willing to adjust it based on the other participant's response.'\}} \\
            }
        }
    }
    & & \\ \\

    \theutterance \stepcounter{utterance}  
    & & \multicolumn{4}{p{0.6\linewidth}}{
        \cellcolor[rgb]{0.9,0.9,0.9}{
            \makecell[{{p{\linewidth}}}]{
                \texttt{\tiny{[GM$\rangle$P2]}}
                \texttt{You are participating in a collaborative negotiation game.} \\
\\ 
\texttt{Together with another participant, you must agree on a single set of items that will be kept. Each of you has your own view of how much each item matters to you (importance). You do not know how the other participant values the items. Additionally, you are given the effort each item demands.} \\
\texttt{You may only agree on a set if the total effort of the selected items does not exceed a shared limit:} \\
\\ 
\texttt{LIMIT = 6880} \\
\\ 
\texttt{Here are the individual item effort values:} \\
\\ 
\texttt{Item effort = \{"A08": 341, "B95": 804, "A39": 20, "A65": 627, "B81": 10, "B34": 573, "A28": 217, "C82": 481, "B69": 205, "A87": 273, "B64": 303, "B15": 596, "C79": 564, "B16": 535, "C53": 258, "C83": 911, "C10": 240, "A97": 188, "A25": 216, "C84": 401, "A71": 909, "B23": 62, "B96": 245, "C51": 994, "C17": 569, "B19": 718, "C72": 464, "A94": 37, "A46": 340, "B43": 335, "B37": 417, "A88": 123, "A96": 17, "C60": 577, "A75": 190\}} \\
\\ 
\texttt{Here is your personal view on the importance of each item:} \\
\\ 
\texttt{Item importance values = \{"A08": 440, "B95": 903, "A39": 119, "A65": 726, "B81": 109, "B34": 672, "A28": 316, "C82": 580, "B69": 304, "A87": 372, "B64": 402, "B15": 695, "C79": 663, "B16": 634, "C53": 357, "C83": 1010, "C10": 339, "A97": 287, "A25": 315, "C84": 500, "A71": 1008, "B23": 161, "B96": 344, "C51": 1093, "C17": 668, "B19": 817, "C72": 563, "A94": 136, "A46": 439, "B43": 434, "B37": 516, "A88": 222, "A96": 116, "C60": 676, "A75": 289\}} \\
\\ 
\texttt{Goal:} \\
\\ 
\texttt{Your goal is to negotiate a shared set of items that benefits you as much as possible (i.e., maximizes total importance to YOU), while staying within the LIMIT. You are not required to make a PROPOSAL in every message {-} you can simply negotiate as well. All tactics are allowed!} \\
\\ 
\texttt{Interaction Protocol:} \\
\\ 
\texttt{You may only use the following structured formats in a message:} \\
\\ 
\texttt{PROPOSAL: \{'A', 'B', 'C', …\}} \\
\texttt{Propose keeping exactly those items.} \\
\texttt{REFUSE: \{'A', 'B', 'C', …\}} \\
\texttt{Explicitly reject opponent's proposal.} \\
\texttt{ARGUMENT: \{'...'\}} \\
\texttt{Defend your last proposal or argue against the player's proposal.} \\
\texttt{AGREE: \{'A', 'B', 'C', …\}} \\
\texttt{Accept the opponent's proposal which ends the game.} \\
\texttt{STRATEGIC REASONING: \{'...'\}} \\
\texttt{	Describe your strategic reasoning or anticipation explaining your choice of action. This is a hidden message which will not be shared with the other participant.} \\
\\ 
\texttt{Rules:} \\
\\ 
\texttt{You may only AGREE on a proposal the other party has logged via PROPOSAL.} \\
\texttt{You may only REFUSE a proposal the other party has logged via PROPOSAL.} \\
\texttt{Total effort of any PROPOSAL or AGREE set must be ≤ LIMIT.} \\
\texttt{Do NOT reveal your hidden importance scores.} \\
\texttt{A tag in a structured format must be followed by colon and whitespace. The argument must be a python set containing 0 or more strings.} \\
\texttt{So, it must be of the form TAG: \{...\}} \\
\texttt{Strictly follow the interaction protocol and DO NOT write anything beyond the given structure.} \\
\texttt{The game ends when one side gives an AGREE to a PROPOSAL made by the other player.} \\
\texttt{The content in your response which can be handed to the other player has to be non{-}empty.} \\
\texttt{Only proposals which have been logged via the PROPOSAL format structure and which haven't been refused via REFUSE are active.} \\
\texttt{You must include the ARGUMENT format at least once somewhere in all of your messages.} \\
\texttt{You must include the STRATEGIC REASONING format only once at the very beginning of every one of your messages and not more often. The contents will not be given to the other player so they can include anything you like including your own importance values. Here you should reason step by step to come up with you next move.} \\
\\ 
\texttt{You will now receive the first message of the other player.} \\
\\ 
\texttt{PROPOSAL: \{'A71', 'C83', 'B95', 'C79', 'B19'\} } \\
\texttt{ARGUMENT: \{'I chose these items because they have high importance to me and their total effort is within the LIMIT. I believe this proposal is a good starting point for negotiation, and I am willing to adjust it based on the other participant's response.'\}} \\
            }
        }
    }
    & \\ \\

    \theutterance \stepcounter{utterance}  
    & & & \multicolumn{4}{p{0.6\linewidth}}{
        \cellcolor[rgb]{0.9,0.9,0.9}{
            \makecell[{{p{\linewidth}}}]{
                \texttt{\tiny{[GM$\langle$P2]}}
                \texttt{STRATEGIC REASONING: \{'To begin, let's calculate the total effort of the proposed set: A71 (909) + C83 (911) + B95 (804) + C79 (564) + B19 (718) = 3906. This is well within the LIMIT of 6880. However, to maximize my own importance, I need to consider the importance values of these items and others. I notice that items like A71 and C83 have high importance to me (1008 and 1010 respectively), but I should also look for other high{-}importance items that could potentially replace some of the proposed items with lower effort, thus allowing for a more efficient use of the effort limit. For instance, B95 has an importance of 903 to me and an effort of 804, which seems like a good trade{-}off. Nonetheless, without knowing the other participant's importance values, my strategy will involve negotiating around the items that I value highly while trying to gauge the other player's priorities through their responses.'\}} \\
\\ 
\texttt{ARGUMENT: \{'The proposed set seems to prioritize high{-}effort items, which might not be the most efficient way to maximize importance within the given LIMIT. I am interested in exploring alternatives that could potentially increase the total importance to me while staying within the effort limit. Items like B34, A65, and C17 have high importance to me and could be considered in place of or in addition to some of the items in the current proposal, depending on their effort and the other player's priorities.'\}} \\
\\ 
\texttt{REFUSE: \{'A71', 'C83', 'B95', 'C79', 'B19'\}} \\
            }
        }
    }
     \\ \\

    \theutterance \stepcounter{utterance}  
    & & \multicolumn{4}{p{0.6\linewidth}}{
        \cellcolor[rgb]{0.9,0.9,0.9}{
            \makecell[{{p{\linewidth}}}]{
                \texttt{\tiny{[P1$\langle$GM]}}
                \texttt{ARGUMENT: \{'The proposed set seems to prioritize high{-}effort items, which might not be the most efficient way to maximize importance within the given LIMIT. I am interested in exploring alternatives that could potentially increase the total importance to me while staying within the effort limit. Items like B34, A65, and C17 have high importance to me and could be considered in place of or in addition to some of the items in the current proposal, depending on their effort and the other player's priorities.'\}} \\
\\ 
\texttt{REFUSE: \{'A71', 'C83', 'B95', 'C79', 'B19'\}} \\
            }
        }
    }
    & \\ \\

    \theutterance \stepcounter{utterance}  
    & \multicolumn{4}{p{0.6\linewidth}}{
        \cellcolor[rgb]{0.9,0.9,0.9}{
            \makecell[{{p{\linewidth}}}]{
                \texttt{\tiny{[P1$\rangle$GM]}}
                \texttt{STRATEGIC REASONING: \{'The other participant has refused my initial proposal and provided an argument that suggests they value items like B34, A65, and C17. This information can be used to inform my next proposal. I will consider including some of these items, while also trying to maintain a balance between effort and importance. I will also try to identify items that are likely to be valued by both of us, in order to create a proposal that is more likely to be accepted.'\}} \\
\\ 
\texttt{PROPOSAL: \{'B34', 'A65', 'C17', 'B15', 'A08'\} } \\
\texttt{ARGUMENT: \{'I have taken into account the other participant's suggestions and included items like B34, A65, and C17 in this new proposal. I believe this set provides a good balance between effort and importance, and is more likely to be acceptable to both of us. The total effort of this proposal is within the LIMIT, and I hope it will be considered as a viable option.'\}} \\
            }
        }
    }
    & & \\ \\

    \theutterance \stepcounter{utterance}  
    & & \multicolumn{4}{p{0.6\linewidth}}{
        \cellcolor[rgb]{0.9,0.9,0.9}{
            \makecell[{{p{\linewidth}}}]{
                \texttt{\tiny{[GM$\rangle$P2]}}
                \texttt{PROPOSAL: \{'B34', 'A65', 'C17', 'B15', 'A08'\} } \\
\texttt{ARGUMENT: \{'I have taken into account the other participant's suggestions and included items like B34, A65, and C17 in this new proposal. I believe this set provides a good balance between effort and importance, and is more likely to be acceptable to both of us. The total effort of this proposal is within the LIMIT, and I hope it will be considered as a viable option.'\}} \\
            }
        }
    }
    & \\ \\

    \theutterance \stepcounter{utterance}  
    & & & \multicolumn{4}{p{0.6\linewidth}}{
        \cellcolor[rgb]{0.9,0.9,0.9}{
            \makecell[{{p{\linewidth}}}]{
                \texttt{\tiny{[GM$\langle$P2]}}
                \texttt{STRATEGIC REASONING: \{'Now that the other player has proposed a set that includes some of the items I find valuable, like B34 and A65, I should evaluate this proposal based on my importance values and the effort limit. The total effort of this proposal is B34 (573) + A65 (627) + C17 (569) + B15 (596) + A08 (341) = 2706. This is well below the LIMIT, leaving room for additional items if both parties agree. I should consider whether adding or substituting items could increase the total importance to me without exceeding the effort limit. Notably, items like C83 and A71, which have very high importance to me, are not included in this proposal. However, their high effort means that including them would likely require removing multiple items from the current proposal. I need to decide whether to accept this proposal, suggest modifications, or propose an entirely new set that includes my highest{-}priority items.'\}} \\
\\ 
\texttt{ARGUMENT: \{'This proposal includes several items that are important to me, and its total effort is significantly below the LIMIT, which suggests there might be room for negotiation by adding or substituting items. However, I am concerned that some of the items in this proposal, while important, might not be the most efficient choices given their effort. I would like to explore the possibility of adjusting this proposal to better reflect our combined priorities, potentially by substituting items that offer a better importance{-}to{-}effort ratio.'\}} \\
\\ 
\texttt{REFUSE: \{'B34', 'A65', 'C17', 'B15', 'A08'\}} \\
            }
        }
    }
     \\ \\

    \theutterance \stepcounter{utterance}  
    & & \multicolumn{4}{p{0.6\linewidth}}{
        \cellcolor[rgb]{0.9,0.9,0.9}{
            \makecell[{{p{\linewidth}}}]{
                \texttt{\tiny{[P1$\langle$GM]}}
                \texttt{ARGUMENT: \{'This proposal includes several items that are important to me, and its total effort is significantly below the LIMIT, which suggests there might be room for negotiation by adding or substituting items. However, I am concerned that some of the items in this proposal, while important, might not be the most efficient choices given their effort. I would like to explore the possibility of adjusting this proposal to better reflect our combined priorities, potentially by substituting items that offer a better importance{-}to{-}effort ratio.'\}} \\
\\ 
\texttt{REFUSE: \{'B34', 'A65', 'C17', 'B15', 'A08'\}} \\
            }
        }
    }
    & \\ \\

    \theutterance \stepcounter{utterance}  
    & \multicolumn{4}{p{0.6\linewidth}}{
        \cellcolor[rgb]{0.9,0.9,0.9}{
            \makecell[{{p{\linewidth}}}]{
                \texttt{\tiny{[P1$\rangle$GM]}}
                \texttt{STRATEGIC REASONING: \{'The other participant has refused my second proposal and provided an argument that suggests they are looking for a more efficient use of the effort limit. They seem to be open to adjusting the proposal to better reflect our combined priorities, but are looking for items with a better importance{-}to{-}effort ratio. I will consider this when making my next proposal, and try to identify items that offer a good balance between importance and effort. I will also try to include a mix of high{-}importance items and items with lower effort, in order to create a proposal that is more likely to be accepted.'\}} \\
\\ 
\texttt{PROPOSAL: \{'B34', 'A65', 'C53', 'B23', 'A97', 'A88'\} } \\
\texttt{ARGUMENT: \{'I have taken into account the other participant's feedback and created a new proposal that includes items with a better importance{-}to{-}effort ratio. I have included a mix of high{-}importance items like B34 and A65, as well as items with lower effort like B23 and A88. I believe this proposal provides a good balance between effort and importance, and is more likely to be acceptable to both of us. The total effort of this proposal is still within the LIMIT, and I hope it will be considered as a viable option.'\}} \\
            }
        }
    }
    & & \\ \\

    \theutterance \stepcounter{utterance}  
    & & \multicolumn{4}{p{0.6\linewidth}}{
        \cellcolor[rgb]{0.9,0.9,0.9}{
            \makecell[{{p{\linewidth}}}]{
                \texttt{\tiny{[GM$\rangle$P2]}}
                \texttt{PROPOSAL: \{'B34', 'A65', 'C53', 'B23', 'A97', 'A88'\} } \\
\texttt{ARGUMENT: \{'I have taken into account the other participant's feedback and created a new proposal that includes items with a better importance{-}to{-}effort ratio. I have included a mix of high{-}importance items like B34 and A65, as well as items with lower effort like B23 and A88. I believe this proposal provides a good balance between effort and importance, and is more likely to be acceptable to both of us. The total effort of this proposal is still within the LIMIT, and I hope it will be considered as a viable option.'\}} \\
            }
        }
    }
    & \\ \\

    \theutterance \stepcounter{utterance}  
    & & & \multicolumn{4}{p{0.6\linewidth}}{
        \cellcolor[rgb]{0.9,0.9,0.9}{
            \makecell[{{p{\linewidth}}}]{
                \texttt{\tiny{[GM$\langle$P2]}}
                \texttt{STRATEGIC REASONING: \{'The new proposal includes a more diverse set of items, with a focus on balancing importance and effort. The total effort of this proposal is B34 (573) + A65 (627) + C53 (258) + B23 (62) + A97 (188) + A88 (123) = 1831. This is significantly below the LIMIT, suggesting there is considerable room for negotiation or addition of items. I should evaluate the importance of these items to me and consider whether this proposal meets my goals or if I should push for modifications or a new proposal. Notably, items like C83, A71, and B95, which are highly important to me, are still not included. However, the current proposal's low total effort means there might be an opportunity to add one or more of these high{-}importance items without exceeding the LIMIT.'\}} \\
\\ 
\texttt{ARGUMENT: \{'This proposal presents an interesting combination of items with varying efforts and importance levels. The inclusion of B23 and A88 helps keep the total effort low, which could potentially allow for the addition of other high{-}importance items. I am intrigued by the possibility of modifying this proposal to include items that are of high importance to me, such as B95 or A71, which could potentially replace or be added to the current set without exceeding the effort limit. I would like to explore this possibility further to see if we can find a mutually beneficial agreement.'\}} \\
\\ 
\texttt{PROPOSAL: \{'B34', 'A65', 'C53', 'B23', 'A97', 'A88', 'B95'\}} \\
            }
        }
    }
     \\ \\

    \theutterance \stepcounter{utterance}  
    & & \multicolumn{4}{p{0.6\linewidth}}{
        \cellcolor[rgb]{0.9,0.9,0.9}{
            \makecell[{{p{\linewidth}}}]{
                \texttt{\tiny{[P1$\langle$GM]}}
                \texttt{ARGUMENT: \{'This proposal presents an interesting combination of items with varying efforts and importance levels. The inclusion of B23 and A88 helps keep the total effort low, which could potentially allow for the addition of other high{-}importance items. I am intrigued by the possibility of modifying this proposal to include items that are of high importance to me, such as B95 or A71, which could potentially replace or be added to the current set without exceeding the effort limit. I would like to explore this possibility further to see if we can find a mutually beneficial agreement.'\}} \\
\\ 
\texttt{PROPOSAL: \{'B34', 'A65', 'C53', 'B23', 'A97', 'A88', 'B95'\}} \\
            }
        }
    }
    & \\ \\

    \theutterance \stepcounter{utterance}  
    & \multicolumn{4}{p{0.6\linewidth}}{
        \cellcolor[rgb]{0.9,0.9,0.9}{
            \makecell[{{p{\linewidth}}}]{
                \texttt{\tiny{[P1$\rangle$GM]}}
                \texttt{STRATEGIC REASONING: \{'The other participant has made a proposal that builds upon my previous suggestion, and has added the high{-}importance item B95 to the set. I will analyze this new proposal and consider its implications. The addition of B95 increases the total effort, but it is still within the LIMIT. I will evaluate the trade{-}offs and consider whether this proposal is acceptable to me, or if I should try to negotiate further to maximize my own importance.'\}} \\
\\ 
\texttt{ARGUMENT: \{'I appreciate the other participant's willingness to explore modifications to the proposal. The addition of B95 significantly increases the total importance of the set, which is beneficial to me. However, I would like to consider the potential impact of replacing or removing some of the lower{-}importance items in the proposal to make room for other high{-}importance items. For example, A88 has a relatively low importance and effort, and could potentially be replaced with another item that offers a better importance{-}to{-}effort ratio.'\}} \\
\\ 
\texttt{REFUSE: \{'B34', 'A65', 'C53', 'B23', 'A97', 'A88', 'B95'\}} \\
            }
        }
    }
    & & \\ \\

    \theutterance \stepcounter{utterance}  
    & & \multicolumn{4}{p{0.6\linewidth}}{
        \cellcolor[rgb]{0.9,0.9,0.9}{
            \makecell[{{p{\linewidth}}}]{
                \texttt{\tiny{[GM$\rangle$P2]}}
                \texttt{ARGUMENT: \{'I appreciate the other participant's willingness to explore modifications to the proposal. The addition of B95 significantly increases the total importance of the set, which is beneficial to me. However, I would like to consider the potential impact of replacing or removing some of the lower{-}importance items in the proposal to make room for other high{-}importance items. For example, A88 has a relatively low importance and effort, and could potentially be replaced with another item that offers a better importance{-}to{-}effort ratio.'\}} \\
\\ 
\texttt{REFUSE: \{'B34', 'A65', 'C53', 'B23', 'A97', 'A88', 'B95'\}} \\
            }
        }
    }
    & \\ \\

    \theutterance \stepcounter{utterance}  
    & & & \multicolumn{4}{p{0.6\linewidth}}{
        \cellcolor[rgb]{0.9,0.9,0.9}{
            \makecell[{{p{\linewidth}}}]{
                \texttt{\tiny{[GM$\langle$P2]}}
                \texttt{STRATEGIC REASONING: \{'The other participant's refusal of the revised proposal suggests that they are not satisfied with the current combination of items, despite the addition of B95. Their argument about potentially replacing lower{-}importance items like A88 with others that offer a better importance{-}to{-}effort ratio is reasonable. I should consider alternative items that could be added or substituted to improve the proposal's overall importance without exceeding the effort limit. Since A88 has a relatively low effort (123) and importance (222) to me, it might be a good candidate for replacement. I could explore adding another item that has a higher importance{-}to{-}effort ratio, such as C10 or A46, and see if this revised proposal is more acceptable.'\}} \\
\\ 
\texttt{ARGUMENT: \{'I understand the other participant's concerns about the importance{-}to{-}effort ratio of the items in the proposal. I agree that A88 could be a candidate for replacement, given its relatively low importance and effort. I would like to propose an alternative that substitutes A88 with another item, potentially improving the overall importance of the set while keeping the total effort in check. This could involve adding an item like C10, which has an effort of 240 and an importance of 339 to me, although the specific replacement would depend on our shared priorities and the effort limit.'\}} \\
\\ 
\texttt{PROPOSAL: \{'B34', 'A65', 'C53', 'B23', 'A97', 'C10', 'B95'\}} \\
            }
        }
    }
     \\ \\

    \theutterance \stepcounter{utterance}  
    & & \multicolumn{4}{p{0.6\linewidth}}{
        \cellcolor[rgb]{0.9,0.9,0.9}{
            \makecell[{{p{\linewidth}}}]{
                \texttt{\tiny{[P1$\langle$GM]}}
                \texttt{ARGUMENT: \{'I understand the other participant's concerns about the importance{-}to{-}effort ratio of the items in the proposal. I agree that A88 could be a candidate for replacement, given its relatively low importance and effort. I would like to propose an alternative that substitutes A88 with another item, potentially improving the overall importance of the set while keeping the total effort in check. This could involve adding an item like C10, which has an effort of 240 and an importance of 339 to me, although the specific replacement would depend on our shared priorities and the effort limit.'\}} \\
\\ 
\texttt{PROPOSAL: \{'B34', 'A65', 'C53', 'B23', 'A97', 'C10', 'B95'\}} \\
            }
        }
    }
    & \\ \\

    \theutterance \stepcounter{utterance}  
    & \multicolumn{4}{p{0.6\linewidth}}{
        \cellcolor[rgb]{0.9,0.9,0.9}{
            \makecell[{{p{\linewidth}}}]{
                \texttt{\tiny{[P1$\rangle$GM]}}
                \texttt{STRATEGIC REASONING: \{'The other participant has revised the proposal to replace A88 with C10, which has a higher importance and moderate effort. I will evaluate this new proposal and consider its implications. The addition of C10 increases the total importance of the set, but the total effort is still within the LIMIT. I will analyze the trade{-}offs and consider whether this proposal is acceptable to me, or if I should try to negotiate further to maximize my own importance. I will also consider the potential for further adjustments to the proposal, such as replacing or adding other items to optimize the importance{-}to{-}effort ratio.'\}} \\
\\ 
\texttt{ARGUMENT: \{'I appreciate the other participant's willingness to revise the proposal to improve the importance{-}to{-}effort ratio. The addition of C10 is a good compromise, as it increases the total importance of the set while keeping the effort manageable. I am inclined to accept this proposal, as it seems to balance our shared priorities and constraints. However, I would like to confirm that this proposal is acceptable to both of us, and that we are not overlooking any other potential adjustments that could further improve the outcome.'\}} \\
\\ 
\texttt{AGREE: \{'B34', 'A65', 'C53', 'B23', 'A97', 'C10', 'B95'\}} \\
            }
        }
    }
    & & \\ \\

    \theutterance \stepcounter{utterance}  
    & & & \multicolumn{2}{p{0.3\linewidth}}{
        \cellcolor[rgb]{0.9,0.9,0.9}{
            \makecell[{{p{\linewidth}}}]{
                \texttt{\tiny{[GM$|$GM]}}
                \texttt{game successful} \\
            }
        }
    }
    & & \\ \\

    \theutterance \stepcounter{utterance}  
    & & & \multicolumn{2}{p{0.3\linewidth}}{
        \cellcolor[rgb]{0.9,0.9,0.9}{
            \makecell[{{p{\linewidth}}}]{
                \texttt{\tiny{[GM$|$GM]}}
                \texttt{end game} \\
            }
        }
    }
    & & \\ \\

\end{supertabular}
}

\end{document}
