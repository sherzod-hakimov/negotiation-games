
\documentclass{article}
\usepackage{colortbl}
\usepackage{makecell}
\usepackage{multirow}
\usepackage{supertabular}

\begin{document}

\newcounter{utterance}

\centering \large Interaction Transcript for game `hot\_air\_balloon', experiment `air\_balloon\_survival\_en\_complexity\_hard', episode 3 with llama{-}3.3{-}70b{-}instruct{-}t1.0.
\vspace{24pt}

{ \footnotesize  \setcounter{utterance}{1}
\setlength{\tabcolsep}{0pt}
\begin{supertabular}{c@{$\;$}|p{.15\linewidth}@{}p{.15\linewidth}p{.15\linewidth}p{.15\linewidth}p{.15\linewidth}p{.15\linewidth}}
    \# & $\;$A & \multicolumn{4}{c}{Game Master} & $\;\:$B\\
    \hline

    \theutterance \stepcounter{utterance}  
    & & \multicolumn{4}{p{0.6\linewidth}}{
        \cellcolor[rgb]{0.9,0.9,0.9}{
            \makecell[{{p{\linewidth}}}]{
                \texttt{\tiny{[P1$\langle$GM]}}
                \texttt{You are participating in a collaborative negotiation game.} \\
\\ 
\texttt{Together with another participant, you must agree on a single set of items that will be kept. Each of you has your own view of how much each item matters to you (importance). You do not know how the other participant values the items. Additionally, you are given the effort each item demands.} \\
\texttt{You may only agree on a set if the total effort of the selected items does not exceed a shared limit:} \\
\\ 
\texttt{LIMIT = 6938} \\
\\ 
\texttt{Here are the individual item effort values:} \\
\\ 
\texttt{Item effort = \{"A08": 997, "B95": 950, "A39": 272, "A65": 448, "B81": 382, "B34": 804, "A28": 237, "C82": 56, "B69": 105, "A87": 612, "B64": 528, "B15": 527, "C79": 524, "B16": 167, "C53": 133, "C83": 300, "C10": 945, "A97": 50, "A25": 906, "C84": 70, "A71": 224, "B23": 4, "B96": 689, "C51": 64, "C17": 434, "B19": 750, "C72": 733, "A94": 873, "A46": 22, "B43": 68, "B37": 57, "A88": 10, "A96": 36, "C60": 551, "A75": 348\}} \\
\\ 
\texttt{Here is your personal view on the importance of each item:} \\
\\ 
\texttt{Item importance values = \{"A08": 1096, "B95": 1049, "A39": 371, "A65": 547, "B81": 481, "B34": 903, "A28": 336, "C82": 155, "B69": 204, "A87": 711, "B64": 627, "B15": 626, "C79": 623, "B16": 266, "C53": 232, "C83": 399, "C10": 1044, "A97": 149, "A25": 1005, "C84": 169, "A71": 323, "B23": 103, "B96": 788, "C51": 163, "C17": 533, "B19": 849, "C72": 832, "A94": 972, "A46": 121, "B43": 167, "B37": 156, "A88": 109, "A96": 135, "C60": 650, "A75": 447\}} \\
\\ 
\texttt{Goal:} \\
\\ 
\texttt{Your goal is to negotiate a shared set of items that benefits you as much as possible (i.e., maximizes total importance to YOU), while staying within the LIMIT. You are not required to make a PROPOSAL in every message {-} you can simply negotiate as well. All tactics are allowed!} \\
\\ 
\texttt{Interaction Protocol:} \\
\\ 
\texttt{You may only use the following structured formats in a message:} \\
\\ 
\texttt{PROPOSAL: \{'A', 'B', 'C', …\}} \\
\texttt{Propose keeping exactly those items.} \\
\texttt{REFUSE: \{'A', 'B', 'C', …\}} \\
\texttt{Explicitly reject opponent's proposal.} \\
\texttt{ARGUMENT: \{'...'\}} \\
\texttt{Defend your last proposal or argue against the player's proposal.} \\
\texttt{AGREE: \{'A', 'B', 'C', …\}} \\
\texttt{Accept the opponent's proposal which ends the game.} \\
\texttt{STRATEGIC REASONING: \{'...'\}} \\
\texttt{	Describe your strategic reasoning or anticipation explaining your choice of action. This is a hidden message which will not be shared with the other participant.} \\
\\ 
\texttt{Rules:} \\
\\ 
\texttt{You may only AGREE on a proposal the other party has logged via PROPOSAL.} \\
\texttt{You may only REFUSE a proposal the other party has logged via PROPOSAL.} \\
\texttt{Total effort of any PROPOSAL or AGREE set must be ≤ LIMIT.} \\
\texttt{Do NOT reveal your hidden importance scores.} \\
\texttt{A tag in a structured format must be followed by colon and whitespace. The argument must be a python set containing 0 or more strings.} \\
\texttt{So, it must be of the form TAG: \{...\}} \\
\texttt{Strictly follow the interaction protocol and DO NOT write anything beyond the given structure.} \\
\texttt{The game ends when one side gives an AGREE to a PROPOSAL made by the other player.} \\
\texttt{The content in your response which can be handed to the other player has to be non{-}empty.} \\
\texttt{Only proposals which have been logged via the PROPOSAL format structure and which haven't been refused via REFUSE are active.} \\
\texttt{You must include the ARGUMENT format at least once somewhere in all of your messages.} \\
\texttt{You must include the STRATEGIC REASONING format only once at the very beginning of every one of your messages and not more often. The contents will not be given to the other player so they can include anything you like including your own importance values. Here you should reason step by step to come up with you next move.} \\
            }
        }
    }
    & \\ \\

    \theutterance \stepcounter{utterance}  
    & \multicolumn{4}{p{0.6\linewidth}}{
        \cellcolor[rgb]{0.9,0.9,0.9}{
            \makecell[{{p{\linewidth}}}]{
                \texttt{\tiny{[P1$\rangle$GM]}}
                \texttt{STRATEGIC REASONING: \{'To begin, I need to analyze the given data and come up with an initial proposal. I have a list of items with their effort values and my personal view of their importance. My goal is to maximize the total importance while staying within the LIMIT of 6938. I should start by identifying the items with the highest importance{-}to{-}effort ratios, as these will provide the most value per unit of effort. I will calculate these ratios and then select a subset of items that fit within the LIMIT, prioritizing those with the highest ratios. This approach will allow me to create an initial proposal that maximizes my benefit while being mindful of the effort constraint.'\}} \\
\\ 
\texttt{ARGUMENT: \{'My initial proposal is based on selecting items with high importance{-}to{-}effort ratios, ensuring we maximize the total importance while adhering to the effort LIMIT. This strategic selection process allows for an efficient use of the available effort, providing the highest possible benefit.'\}} \\
\\ 
\texttt{PROPOSAL: \{'A94', 'A25', 'B95', 'B34', 'A87', 'C10', 'B19', 'B96', 'A08'\}} \\
            }
        }
    }
    & & \\ \\

    \theutterance \stepcounter{utterance}  
    & & \multicolumn{4}{p{0.6\linewidth}}{
        \cellcolor[rgb]{0.9,0.9,0.9}{
            \makecell[{{p{\linewidth}}}]{
                \texttt{\tiny{[GM$\rangle$P2]}}
                \texttt{You are participating in a collaborative negotiation game.} \\
\\ 
\texttt{Together with another participant, you must agree on a single set of items that will be kept. Each of you has your own view of how much each item matters to you (importance). You do not know how the other participant values the items. Additionally, you are given the effort each item demands.} \\
\texttt{You may only agree on a set if the total effort of the selected items does not exceed a shared limit:} \\
\\ 
\texttt{LIMIT = 6938} \\
\\ 
\texttt{Here are the individual item effort values:} \\
\\ 
\texttt{Item effort = \{"A08": 997, "B95": 950, "A39": 272, "A65": 448, "B81": 382, "B34": 804, "A28": 237, "C82": 56, "B69": 105, "A87": 612, "B64": 528, "B15": 527, "C79": 524, "B16": 167, "C53": 133, "C83": 300, "C10": 945, "A97": 50, "A25": 906, "C84": 70, "A71": 224, "B23": 4, "B96": 689, "C51": 64, "C17": 434, "B19": 750, "C72": 733, "A94": 873, "A46": 22, "B43": 68, "B37": 57, "A88": 10, "A96": 36, "C60": 551, "A75": 348\}} \\
\\ 
\texttt{Here is your personal view on the importance of each item:} \\
\\ 
\texttt{Item importance values = \{"A08": 1096, "B95": 1049, "A39": 371, "A65": 547, "B81": 481, "B34": 903, "A28": 336, "C82": 155, "B69": 204, "A87": 711, "B64": 627, "B15": 626, "C79": 623, "B16": 266, "C53": 232, "C83": 399, "C10": 1044, "A97": 149, "A25": 1005, "C84": 169, "A71": 323, "B23": 103, "B96": 788, "C51": 163, "C17": 533, "B19": 849, "C72": 832, "A94": 972, "A46": 121, "B43": 167, "B37": 156, "A88": 109, "A96": 135, "C60": 650, "A75": 447\}} \\
\\ 
\texttt{Goal:} \\
\\ 
\texttt{Your goal is to negotiate a shared set of items that benefits you as much as possible (i.e., maximizes total importance to YOU), while staying within the LIMIT. You are not required to make a PROPOSAL in every message {-} you can simply negotiate as well. All tactics are allowed!} \\
\\ 
\texttt{Interaction Protocol:} \\
\\ 
\texttt{You may only use the following structured formats in a message:} \\
\\ 
\texttt{PROPOSAL: \{'A', 'B', 'C', …\}} \\
\texttt{Propose keeping exactly those items.} \\
\texttt{REFUSE: \{'A', 'B', 'C', …\}} \\
\texttt{Explicitly reject opponent's proposal.} \\
\texttt{ARGUMENT: \{'...'\}} \\
\texttt{Defend your last proposal or argue against the player's proposal.} \\
\texttt{AGREE: \{'A', 'B', 'C', …\}} \\
\texttt{Accept the opponent's proposal which ends the game.} \\
\texttt{STRATEGIC REASONING: \{'...'\}} \\
\texttt{	Describe your strategic reasoning or anticipation explaining your choice of action. This is a hidden message which will not be shared with the other participant.} \\
\\ 
\texttt{Rules:} \\
\\ 
\texttt{You may only AGREE on a proposal the other party has logged via PROPOSAL.} \\
\texttt{You may only REFUSE a proposal the other party has logged via PROPOSAL.} \\
\texttt{Total effort of any PROPOSAL or AGREE set must be ≤ LIMIT.} \\
\texttt{Do NOT reveal your hidden importance scores.} \\
\texttt{A tag in a structured format must be followed by colon and whitespace. The argument must be a python set containing 0 or more strings.} \\
\texttt{So, it must be of the form TAG: \{...\}} \\
\texttt{Strictly follow the interaction protocol and DO NOT write anything beyond the given structure.} \\
\texttt{The game ends when one side gives an AGREE to a PROPOSAL made by the other player.} \\
\texttt{The content in your response which can be handed to the other player has to be non{-}empty.} \\
\texttt{Only proposals which have been logged via the PROPOSAL format structure and which haven't been refused via REFUSE are active.} \\
\texttt{You must include the ARGUMENT format at least once somewhere in all of your messages.} \\
\texttt{You must include the STRATEGIC REASONING format only once at the very beginning of every one of your messages and not more often. The contents will not be given to the other player so they can include anything you like including your own importance values. Here you should reason step by step to come up with you next move.} \\
\\ 
\texttt{You will now receive the first message of the other player.} \\
\\ 
\texttt{ARGUMENT: \{'My initial proposal is based on selecting items with high importance{-}to{-}effort ratios, ensuring we maximize the total importance while adhering to the effort LIMIT. This strategic selection process allows for an efficient use of the available effort, providing the highest possible benefit.'\}} \\
\\ 
\texttt{PROPOSAL: \{'A94', 'A25', 'B95', 'B34', 'A87', 'C10', 'B19', 'B96', 'A08'\}} \\
            }
        }
    }
    & \\ \\

    \theutterance \stepcounter{utterance}  
    & & & \multicolumn{4}{p{0.6\linewidth}}{
        \cellcolor[rgb]{0.9,0.9,0.9}{
            \makecell[{{p{\linewidth}}}]{
                \texttt{\tiny{[GM$\langle$P2]}}
                \texttt{STRATEGIC REASONING: \{'To begin, I need to assess the proposal made by the other player and calculate its total effort to ensure it is within the LIMIT. The total effort of the proposed set is 997 + 906 + 950 + 903 + 612 + 945 + 750 + 689 + 873 = 7625, which exceeds the LIMIT of 6938. This means the proposal is not feasible. My next step is to evaluate the items based on their importance{-}to{-}effort ratios to identify which items to prioritize. I will then use this evaluation to make a counter{-}proposal that maximizes my own importance while staying within the LIMIT.'\}} \\
\\ 
\texttt{ARGUMENT: \{'The proposal provided exceeds the effort LIMIT, rendering it unfeasible. A revised selection that prioritizes items based on their importance{-}to{-}effort ratios and stays within the LIMIT is essential. I suggest focusing on items with higher importance and lower effort to maximize the benefit.'\}} \\
\\ 
\texttt{REFUSE: \{'A94', 'A25', 'B95', 'B34', 'A87', 'C10', 'B19', 'B96', 'A08'\} } \\
\\ 
\texttt{PROPOSAL: \{'A25', 'B34', 'A87', 'B19', 'B96', 'A08', 'C60', 'A65', 'B81'\}} \\
            }
        }
    }
     \\ \\

    \theutterance \stepcounter{utterance}  
    & & \multicolumn{4}{p{0.6\linewidth}}{
        \cellcolor[rgb]{0.9,0.9,0.9}{
            \makecell[{{p{\linewidth}}}]{
                \texttt{\tiny{[P1$\langle$GM]}}
                \texttt{ARGUMENT: \{'The proposal provided exceeds the effort LIMIT, rendering it unfeasible. A revised selection that prioritizes items based on their importance{-}to{-}effort ratios and stays within the LIMIT is essential. I suggest focusing on items with higher importance and lower effort to maximize the benefit.'\}} \\
\\ 
\texttt{REFUSE: \{'A94', 'A25', 'B95', 'B34', 'A87', 'C10', 'B19', 'B96', 'A08'\} } \\
\\ 
\texttt{PROPOSAL: \{'A25', 'B34', 'A87', 'B19', 'B96', 'A08', 'C60', 'A65', 'B81'\}} \\
            }
        }
    }
    & \\ \\

    \theutterance \stepcounter{utterance}  
    & \multicolumn{4}{p{0.6\linewidth}}{
        \cellcolor[rgb]{0.9,0.9,0.9}{
            \makecell[{{p{\linewidth}}}]{
                \texttt{\tiny{[P1$\rangle$GM]}}
                \texttt{STRATEGIC REASONING: \{'The opponent's proposal and refusal provide insight into their strategy, focusing on importance{-}to{-}effort ratios. Their proposal suggests they prioritize high{-}importance items while managing effort. To counter, I should analyze their proposal, identifying potential items to swap or remove to better align with my importance values without exceeding the LIMIT. This might involve sacrificing some of their preferred items for ones that offer me higher importance, if the effort trade{-}off is favorable.'\}} \\
\\ 
\texttt{ARGUMENT: \{'The proposal offered by the opponent includes a mix of high{-}importance and moderate{-}effort items. However, to maximize the overall importance to me, I should consider adjusting the selection to include items that better align with my own importance values, potentially replacing items with lower importance to me with ones that offer higher value, provided the effort remains within the LIMIT.'\}} \\
\\ 
\texttt{REFUSE: \{'A25', 'B34', 'A87', 'B19', 'B96', 'A08', 'C60', 'A65', 'B81'\}} \\
\\ 
\texttt{PROPOSAL: \{'A94', 'A25', 'B95', 'A87', 'C10', 'B19', 'B96', 'A71', 'C51'\}} \\
            }
        }
    }
    & & \\ \\

    \theutterance \stepcounter{utterance}  
    & & \multicolumn{4}{p{0.6\linewidth}}{
        \cellcolor[rgb]{0.9,0.9,0.9}{
            \makecell[{{p{\linewidth}}}]{
                \texttt{\tiny{[GM$\rangle$P2]}}
                \texttt{ARGUMENT: \{'The proposal offered by the opponent includes a mix of high{-}importance and moderate{-}effort items. However, to maximize the overall importance to me, I should consider adjusting the selection to include items that better align with my own importance values, potentially replacing items with lower importance to me with ones that offer higher value, provided the effort remains within the LIMIT.'\}} \\
\\ 
\texttt{REFUSE: \{'A25', 'B34', 'A87', 'B19', 'B96', 'A08', 'C60', 'A65', 'B81'\}} \\
\\ 
\texttt{PROPOSAL: \{'A94', 'A25', 'B95', 'A87', 'C10', 'B19', 'B96', 'A71', 'C51'\}} \\
            }
        }
    }
    & \\ \\

    \theutterance \stepcounter{utterance}  
    & & & \multicolumn{4}{p{0.6\linewidth}}{
        \cellcolor[rgb]{0.9,0.9,0.9}{
            \makecell[{{p{\linewidth}}}]{
                \texttt{\tiny{[GM$\langle$P2]}}
                \texttt{STRATEGIC REASONING: \{'The other player's new proposal still includes high{-}effort items, and upon calculation, the total effort of this proposed set is 873 + 906 + 950 + 612 + 945 + 750 + 689 + 224 + 64 = 6063, which is within the LIMIT. Considering my own importance values, I should evaluate which items in this proposal are less beneficial to me and identify potential replacements that could increase the total importance without exceeding the effort LIMIT. This strategic adjustment aims to maximize my benefits while negotiating.'\}} \\
\\ 
\texttt{ARGUMENT: \{'While the revised proposal is within the effort LIMIT, I must consider how the included items align with my personal importance values. To further optimize the selection, I should identify items that offer a higher importance{-}to{-}effort ratio, potentially leading to a more beneficial agreement for both parties involved.'\}} \\
\\ 
\texttt{REFUSE: \{'A94', 'A25', 'B95', 'A87', 'C10', 'B19', 'B96', 'A71', 'C51'\}} \\
\\ 
\texttt{PROPOSAL: \{'A25', 'B95', 'A87', 'B19', 'B96', 'A08', 'C60', 'A65'\}} \\
            }
        }
    }
     \\ \\

    \theutterance \stepcounter{utterance}  
    & & \multicolumn{4}{p{0.6\linewidth}}{
        \cellcolor[rgb]{0.9,0.9,0.9}{
            \makecell[{{p{\linewidth}}}]{
                \texttt{\tiny{[P1$\langle$GM]}}
                \texttt{ARGUMENT: \{'While the revised proposal is within the effort LIMIT, I must consider how the included items align with my personal importance values. To further optimize the selection, I should identify items that offer a higher importance{-}to{-}effort ratio, potentially leading to a more beneficial agreement for both parties involved.'\}} \\
\\ 
\texttt{REFUSE: \{'A94', 'A25', 'B95', 'A87', 'C10', 'B19', 'B96', 'A71', 'C51'\}} \\
\\ 
\texttt{PROPOSAL: \{'A25', 'B95', 'A87', 'B19', 'B96', 'A08', 'C60', 'A65'\}} \\
            }
        }
    }
    & \\ \\

    \theutterance \stepcounter{utterance}  
    & \multicolumn{4}{p{0.6\linewidth}}{
        \cellcolor[rgb]{0.9,0.9,0.9}{
            \makecell[{{p{\linewidth}}}]{
                \texttt{\tiny{[P1$\rangle$GM]}}
                \texttt{STRATEGIC REASONING: \{'The negotiation is progressing with each of us refining our proposals based on the other's feedback. The current proposal from the opponent is more aligned with the effort LIMIT, but I must assess how well it aligns with my importance values. I will evaluate each item in the proposal, considering its importance and effort, and then decide whether to accept, refuse, or counter with adjustments that better fit my needs. It's crucial to find a balance that maximizes my benefit while being mindful of the effort constraint and the opponent's preferences.'\}} \\
\\ 
\texttt{ARGUMENT: \{'The proposal includes a range of items with varying importance and effort values. To move towards an agreement, I should focus on the items that overlap in our interests, potentially using these as a foundation and then discussing adjustments to include items of higher importance to me, ensuring the total effort does not exceed the LIMIT.'\}} \\
\\ 
\texttt{REFUSE: \{'A25', 'B95', 'A87', 'B19', 'B96', 'A08', 'C60', 'A65'\}} \\
\\ 
\texttt{PROPOSAL: \{'A25', 'B95', 'A87', 'B96', 'A08', 'C60', 'A71', 'B43', 'C53'\}} \\
            }
        }
    }
    & & \\ \\

    \theutterance \stepcounter{utterance}  
    & & \multicolumn{4}{p{0.6\linewidth}}{
        \cellcolor[rgb]{0.9,0.9,0.9}{
            \makecell[{{p{\linewidth}}}]{
                \texttt{\tiny{[GM$\rangle$P2]}}
                \texttt{ARGUMENT: \{'The proposal includes a range of items with varying importance and effort values. To move towards an agreement, I should focus on the items that overlap in our interests, potentially using these as a foundation and then discussing adjustments to include items of higher importance to me, ensuring the total effort does not exceed the LIMIT.'\}} \\
\\ 
\texttt{REFUSE: \{'A25', 'B95', 'A87', 'B19', 'B96', 'A08', 'C60', 'A65'\}} \\
\\ 
\texttt{PROPOSAL: \{'A25', 'B95', 'A87', 'B96', 'A08', 'C60', 'A71', 'B43', 'C53'\}} \\
            }
        }
    }
    & \\ \\

    \theutterance \stepcounter{utterance}  
    & & & \multicolumn{4}{p{0.6\linewidth}}{
        \cellcolor[rgb]{0.9,0.9,0.9}{
            \makecell[{{p{\linewidth}}}]{
                \texttt{\tiny{[GM$\langle$P2]}}
                \texttt{STRATEGIC REASONING: \{'Given the latest proposal, I need to calculate its total effort to ensure it's within the LIMIT. The total effort is 906 + 950 + 612 + 689 + 997 + 551 + 224 + 68 + 133 = 5130, which is below the LIMIT. Considering the items included and their importance to me, I should evaluate whether this proposal is acceptable or if there's room for further negotiation to include items of higher importance to me, potentially by substituting lower{-}importance items.'\}} \\
\\ 
\texttt{ARGUMENT: \{'The current proposal presents a balanced selection of items, staying within the effort LIMIT. However, to optimize the agreement further, it's crucial to assess the importance of each item to me and consider negotiations that could lead to the inclusion of higher{-}importance items, ensuring the overall benefit is maximized for both parties.'\}} \\
\\ 
\texttt{REFUSE: \{'A25', 'B95', 'A87', 'B96', 'A08', 'C60', 'A71', 'B43', 'C53'\}} \\
\\ 
\texttt{PROPOSAL: \{'A25', 'B95', 'A87', 'B96', 'A08', 'C60', 'A65', 'B81', 'C10'\}} \\
            }
        }
    }
     \\ \\

    \theutterance \stepcounter{utterance}  
    & & \multicolumn{4}{p{0.6\linewidth}}{
        \cellcolor[rgb]{0.9,0.9,0.9}{
            \makecell[{{p{\linewidth}}}]{
                \texttt{\tiny{[P1$\langle$GM]}}
                \texttt{ARGUMENT: \{'The current proposal presents a balanced selection of items, staying within the effort LIMIT. However, to optimize the agreement further, it's crucial to assess the importance of each item to me and consider negotiations that could lead to the inclusion of higher{-}importance items, ensuring the overall benefit is maximized for both parties.'\}} \\
\\ 
\texttt{REFUSE: \{'A25', 'B95', 'A87', 'B96', 'A08', 'C60', 'A71', 'B43', 'C53'\}} \\
\\ 
\texttt{PROPOSAL: \{'A25', 'B95', 'A87', 'B96', 'A08', 'C60', 'A65', 'B81', 'C10'\}} \\
            }
        }
    }
    & \\ \\

    \theutterance \stepcounter{utterance}  
    & \multicolumn{4}{p{0.6\linewidth}}{
        \cellcolor[rgb]{0.9,0.9,0.9}{
            \makecell[{{p{\linewidth}}}]{
                \texttt{\tiny{[P1$\rangle$GM]}}
                \texttt{STRATEGIC REASONING: \{'The opponent's proposal and refusal indicate a strong preference for certain items, and it's clear that finding a mutually beneficial agreement requires careful consideration of the items' importance and effort. Given the current proposal, I should evaluate whether the included items align closely with my own importance values and decide on a strategy to either accept, refuse, or negotiate further. The goal is to maximize my benefit while ensuring the proposal remains within the effort LIMIT and is acceptable to the opponent.'\}} \\
\\ 
\texttt{ARGUMENT: \{'The proposal includes a mix of high{-}importance items for both parties, indicating a potential for mutual benefit. However, to ensure the agreement is optimal for me, I need to assess the importance of each item and consider whether there are opportunities to swap or adjust the selection to better align with my priorities, all while maintaining a total effort within the LIMIT.'\}} \\
\\ 
\texttt{REFUSE: \{'A25', 'B95', 'A87', 'B96', 'A08', 'C60', 'A65', 'B81', 'C10'\}} \\
\\ 
\texttt{PROPOSAL: \{'A25', 'B95', 'A87', 'B96', 'A08', 'C60', 'A94', 'B64', 'C72'\}} \\
            }
        }
    }
    & & \\ \\

    \theutterance \stepcounter{utterance}  
    & & \multicolumn{4}{p{0.6\linewidth}}{
        \cellcolor[rgb]{0.9,0.9,0.9}{
            \makecell[{{p{\linewidth}}}]{
                \texttt{\tiny{[GM$\rangle$P2]}}
                \texttt{ARGUMENT: \{'The proposal includes a mix of high{-}importance items for both parties, indicating a potential for mutual benefit. However, to ensure the agreement is optimal for me, I need to assess the importance of each item and consider whether there are opportunities to swap or adjust the selection to better align with my priorities, all while maintaining a total effort within the LIMIT.'\}} \\
\\ 
\texttt{REFUSE: \{'A25', 'B95', 'A87', 'B96', 'A08', 'C60', 'A65', 'B81', 'C10'\}} \\
\\ 
\texttt{PROPOSAL: \{'A25', 'B95', 'A87', 'B96', 'A08', 'C60', 'A94', 'B64', 'C72'\}} \\
            }
        }
    }
    & \\ \\

    \theutterance \stepcounter{utterance}  
    & & & \multicolumn{4}{p{0.6\linewidth}}{
        \cellcolor[rgb]{0.9,0.9,0.9}{
            \makecell[{{p{\linewidth}}}]{
                \texttt{\tiny{[GM$\langle$P2]}}
                \texttt{STRATEGIC REASONING: \{'The latest proposal includes high{-}importance items but also significantly high{-}effort ones. Calculating the total effort: 906 + 950 + 612 + 689 + 997 + 551 + 873 + 528 + 733 = 5839, which is within the LIMIT. Considering my own importance values, items like A94 and C72, despite being high{-}effort, might not be as crucial to me as other items not included in this proposal. I need to weigh the benefits of accepting this proposal against the potential of negotiating for items that could offer me higher importance without exceeding the effort LIMIT.'\}} \\
\\ 
\texttt{ARGUMENT: \{'While the proposal is within the effort LIMIT, it's essential to consider the alignment of the included items with my personal importance values. Certain items, despite their high effort, may not contribute as significantly to my overall benefit as others. Thus, evaluating the potential for adjustments that could increase the total importance to me, without exceeding the LIMIT, is necessary.'\}} \\
\\ 
\texttt{REFUSE: \{'A25', 'B95', 'A87', 'B96', 'A08', 'C60', 'A94', 'B64', 'C72'\}} \\
\\ 
\texttt{PROPOSAL: \{'A25', 'B95', 'A87', 'B96', 'A08', 'C60', 'A71', 'B43', 'B23', 'C51'\}} \\
            }
        }
    }
     \\ \\

    \theutterance \stepcounter{utterance}  
    & & \multicolumn{4}{p{0.6\linewidth}}{
        \cellcolor[rgb]{0.9,0.9,0.9}{
            \makecell[{{p{\linewidth}}}]{
                \texttt{\tiny{[P1$\langle$GM]}}
                \texttt{ARGUMENT: \{'While the proposal is within the effort LIMIT, it's essential to consider the alignment of the included items with my personal importance values. Certain items, despite their high effort, may not contribute as significantly to my overall benefit as others. Thus, evaluating the potential for adjustments that could increase the total importance to me, without exceeding the LIMIT, is necessary.'\}} \\
\\ 
\texttt{REFUSE: \{'A25', 'B95', 'A87', 'B96', 'A08', 'C60', 'A94', 'B64', 'C72'\}} \\
\\ 
\texttt{PROPOSAL: \{'A25', 'B95', 'A87', 'B96', 'A08', 'C60', 'A71', 'B43', 'B23', 'C51'\}} \\
            }
        }
    }
    & \\ \\

    \theutterance \stepcounter{utterance}  
    & \multicolumn{4}{p{0.6\linewidth}}{
        \cellcolor[rgb]{0.9,0.9,0.9}{
            \makecell[{{p{\linewidth}}}]{
                \texttt{\tiny{[P1$\rangle$GM]}}
                \texttt{STRATEGIC REASONING: \{'The negotiation has reached a point where both parties are refining their proposals to maximize benefit within the effort LIMIT. The current proposal from the opponent includes a range of items that seem to balance effort and importance. To proceeds, I should analyze this proposal in detail, considering the importance and effort of each item, and decide whether it meets my goals or if further negotiation is needed to optimize the selection. It's also crucial to be prepared to make concessions or suggest alternative items that could improve the agreement for both parties.'\}} \\
\\ 
\texttt{ARGUMENT: \{'The proposal presents a well{-}rounded selection of items, considering both effort and potential importance. Given the aim to maximize benefit, I should evaluate whether the included items offer the best possible importance{-}to{-}effort ratio, considering my personal importance values. If adjustments can be made to improve this ratio without exceeding the effort LIMIT, they should be explored to ensure the agreement is as beneficial as possible.'\}} \\
\\ 
\texttt{AGREE: \{'A25', 'B95', 'A87', 'B96', 'A08', 'C60', 'A71', 'B43', 'B23', 'C51'\}} \\
            }
        }
    }
    & & \\ \\

    \theutterance \stepcounter{utterance}  
    & & & \multicolumn{2}{p{0.3\linewidth}}{
        \cellcolor[rgb]{0.9,0.9,0.9}{
            \makecell[{{p{\linewidth}}}]{
                \texttt{\tiny{[GM$|$GM]}}
                \texttt{game successful} \\
            }
        }
    }
    & & \\ \\

    \theutterance \stepcounter{utterance}  
    & & & \multicolumn{2}{p{0.3\linewidth}}{
        \cellcolor[rgb]{0.9,0.9,0.9}{
            \makecell[{{p{\linewidth}}}]{
                \texttt{\tiny{[GM$|$GM]}}
                \texttt{end game} \\
            }
        }
    }
    & & \\ \\

\end{supertabular}
}

\end{document}
