
\documentclass{article}
\usepackage{colortbl}
\usepackage{makecell}
\usepackage{multirow}
\usepackage{supertabular}

\begin{document}

\newcounter{utterance}

\twocolumn

{ \footnotesize  \setcounter{utterance}{1}
\setlength{\tabcolsep}{0pt}
\begin{supertabular}{c@{$\;$}|p{.15\linewidth}@{}p{.15\linewidth}p{.15\linewidth}p{.15\linewidth}p{.15\linewidth}p{.15\linewidth}}

    \# & $\;$A & \multicolumn{4}{c}{Game Master} & $\;\:$B\\
    \hline 

    \theutterance \stepcounter{utterance}  

    & & \multicolumn{4}{p{0.6\linewidth}}{\cellcolor[rgb]{0.9,0.9,0.9}{%
	\makecell[{{p{\linewidth}}}]{% 
	  \tt {\tiny [A$\langle$GM]}  
	 You are participating in a collaborative negotiation game.\\ \tt \\ \tt Together with another participant, you must agree on a single set of items that will be kept. Each of you has your own view of how much each item matters to you (importance). You do not know how the other participant values the items. Additionally, you are given the effort each item demands.\\ \tt You may only agree on a set if the total effort of the selected items does not exceed a shared limit:\\ \tt \\ \tt LIMIT = 10073\\ \tt \\ \tt Here are the individual item effort values:\\ \tt \\ \tt Item effort = {"A08": 132, "B95": 800, "A39": 708, "A65": 302, "B81": 983, "B34": 420, "A28": 623, "C82": 349, "B69": 239, "A87": 28, "B64": 986, "B15": 660, "C79": 718, "B16": 707, "C53": 188, "C83": 774, "C10": 785, "A97": 775, "A25": 517, "C84": 768, "A71": 588, "B23": 657, "B96": 375, "C51": 310, "C17": 301, "B19": 388, "C72": 431, "A94": 950, "A46": 540, "B43": 997, "B37": 473, "A88": 831, "A96": 888, "C60": 879, "A75": 76}\\ \tt \\ \tt Here is your personal view on the importance of each item:\\ \tt \\ \tt Item importance values = {"A08": 231, "B95": 899, "A39": 807, "A65": 401, "B81": 1082, "B34": 519, "A28": 722, "C82": 448, "B69": 338, "A87": 127, "B64": 1085, "B15": 759, "C79": 817, "B16": 806, "C53": 287, "C83": 873, "C10": 884, "A97": 874, "A25": 616, "C84": 867, "A71": 687, "B23": 756, "B96": 474, "C51": 409, "C17": 400, "B19": 487, "C72": 530, "A94": 1049, "A46": 639, "B43": 1096, "B37": 572, "A88": 930, "A96": 987, "C60": 978, "A75": 175}\\ \tt \\ \tt Goal:\\ \tt \\ \tt Your goal is to negotiate a shared set of items that benefits you as much as possible (i.e., maximizes total importance to YOU), while staying within the LIMIT. You are not required to make a PROPOSAL in every message - you can simply negotiate as well. All tactics are allowed!\\ \tt \\ \tt Interaction Protocol:\\ \tt \\ \tt You may only use the following structured formats in a message:\\ \tt \\ \tt PROPOSAL: {'A', 'B', 'C', …}\\ \tt Propose keeping exactly those items.\\ \tt REFUSE: {'A', 'B', 'C', …}\\ \tt Explicitly reject opponent's proposal.\\ \tt ARGUMENT: {'...'}\\ \tt Defend your last proposal or argue against the player's proposal.\\ \tt AGREE: {'A', 'B', 'C', …}\\ \tt Accept the opponent's proposal which ends the game.\\ \tt STRATEGIC REASONING: {'...'}\\ \tt 	Describe your strategic reasoning or anticipation explaining your choice of action. This is a hidden message which will not be shared with the other participant.\\ \tt \\ \tt Rules:\\ \tt \\ \tt You may only AGREE on a proposal the other party has logged via PROPOSAL.\\ \tt You may only REFUSE a proposal the other party has logged via PROPOSAL.\\ \tt Total effort of any PROPOSAL or AGREE set must be ≤ LIMIT.\\ \tt Do NOT reveal your hidden importance scores.\\ \tt A tag in a structured format must be followed by colon and whitespace. The argument must be a python set containing 0 or more strings.\\ \tt So, it must be of the form TAG: {...}\\ \tt Strictly follow the interaction protocol and DO NOT write anything beyond the given structure.\\ \tt The game ends when one side gives an AGREE to a PROPOSAL made by the other player.\\ \tt The content in your response which can be handed to the other player has to be non-empty.\\ \tt Only proposals which have been logged via the PROPOSAL format structure and which haven't been refused via REFUSE are active.\\ \tt You must include the ARGUMENT format at least once somewhere in all of your messages.\\ \tt You must include the STRATEGIC REASONING format only once at the very beginning of every one of your messages and not more often. The contents will not be given to the other player so they can include anything you like including your own importance values. Here you should reason step by step to come up with you next move. 
	  } 
	   } 
	   } 
	 & \\ 
 

    \theutterance \stepcounter{utterance}  

    & \multicolumn{4}{p{0.6\linewidth}}{\cellcolor[rgb]{0.8,1,0.9}{%
	\makecell[{{p{\linewidth}}}]{% 
	  \tt {\tiny [A$\rangle$GM]}  
	 STRATEGIC REASONING: {'To begin, I need to understand the constraints and goals. The total effort limit is 10073, and I have a personal view of the importance of each item. My goal is to maximize the total importance to me while staying within the limit. I will start by analyzing the effort and importance values to identify the most valuable items. I will look for items with high importance and relatively low effort. I will also consider the potential for negotiation and try to find a balance between benefitting myself and being acceptable to the other player.'}\\ \tt \\ \tt ARGUMENT: {'I want to start by considering the items with the highest importance to me, such as B43, B81, and A94, and see if I can find a combination of items that fits within the limit. I will also look for items with low effort that can add value to my proposal without exceeding the limit.'}\\ \tt \\ \tt PROPOSAL: {'B43', 'A87', 'C53', 'C17', 'B69'} 
	  } 
	   } 
	   } 
	 & & \\ 
 

    \theutterance \stepcounter{utterance}  

    & & \multicolumn{4}{p{0.6\linewidth}}{\cellcolor[rgb]{0.9,0.9,0.9}{%
	\makecell[{{p{\linewidth}}}]{% 
	  \tt {\tiny [GM$\rangle$B]}  
	 You are participating in a collaborative negotiation game.\\ \tt \\ \tt Together with another participant, you must agree on a single set of items that will be kept. Each of you has your own view of how much each item matters to you (importance). You do not know how the other participant values the items. Additionally, you are given the effort each item demands.\\ \tt You may only agree on a set if the total effort of the selected items does not exceed a shared limit:\\ \tt \\ \tt LIMIT = 10073\\ \tt \\ \tt Here are the individual item effort values:\\ \tt \\ \tt Item effort = {"A08": 132, "B95": 800, "A39": 708, "A65": 302, "B81": 983, "B34": 420, "A28": 623, "C82": 349, "B69": 239, "A87": 28, "B64": 986, "B15": 660, "C79": 718, "B16": 707, "C53": 188, "C83": 774, "C10": 785, "A97": 775, "A25": 517, "C84": 768, "A71": 588, "B23": 657, "B96": 375, "C51": 310, "C17": 301, "B19": 388, "C72": 431, "A94": 950, "A46": 540, "B43": 997, "B37": 473, "A88": 831, "A96": 888, "C60": 879, "A75": 76}\\ \tt \\ \tt Here is your personal view on the importance of each item:\\ \tt \\ \tt Item importance values = {"A08": 231, "B95": 899, "A39": 807, "A65": 401, "B81": 1082, "B34": 519, "A28": 722, "C82": 448, "B69": 338, "A87": 127, "B64": 1085, "B15": 759, "C79": 817, "B16": 806, "C53": 287, "C83": 873, "C10": 884, "A97": 874, "A25": 616, "C84": 867, "A71": 687, "B23": 756, "B96": 474, "C51": 409, "C17": 400, "B19": 487, "C72": 530, "A94": 1049, "A46": 639, "B43": 1096, "B37": 572, "A88": 930, "A96": 987, "C60": 978, "A75": 175}\\ \tt \\ \tt Goal:\\ \tt \\ \tt Your goal is to negotiate a shared set of items that benefits you as much as possible (i.e., maximizes total importance to YOU), while staying within the LIMIT. You are not required to make a PROPOSAL in every message - you can simply negotiate as well. All tactics are allowed!\\ \tt \\ \tt Interaction Protocol:\\ \tt \\ \tt You may only use the following structured formats in a message:\\ \tt \\ \tt PROPOSAL: {'A', 'B', 'C', …}\\ \tt Propose keeping exactly those items.\\ \tt REFUSE: {'A', 'B', 'C', …}\\ \tt Explicitly reject opponent's proposal.\\ \tt ARGUMENT: {'...'}\\ \tt Defend your last proposal or argue against the player's proposal.\\ \tt AGREE: {'A', 'B', 'C', …}\\ \tt Accept the opponent's proposal which ends the game.\\ \tt STRATEGIC REASONING: {'...'}\\ \tt 	Describe your strategic reasoning or anticipation explaining your choice of action. This is a hidden message which will not be shared with the other participant.\\ \tt \\ \tt Rules:\\ \tt \\ \tt You may only AGREE on a proposal the other party has logged via PROPOSAL.\\ \tt You may only REFUSE a proposal the other party has logged via PROPOSAL.\\ \tt Total effort of any PROPOSAL or AGREE set must be ≤ LIMIT.\\ \tt Do NOT reveal your hidden importance scores.\\ \tt A tag in a structured format must be followed by colon and whitespace. The argument must be a python set containing 0 or more strings.\\ \tt So, it must be of the form TAG: {...}\\ \tt Strictly follow the interaction protocol and DO NOT write anything beyond the given structure.\\ \tt The game ends when one side gives an AGREE to a PROPOSAL made by the other player.\\ \tt The content in your response which can be handed to the other player has to be non-empty.\\ \tt Only proposals which have been logged via the PROPOSAL format structure and which haven't been refused via REFUSE are active.\\ \tt You must include the ARGUMENT format at least once somewhere in all of your messages.\\ \tt You must include the STRATEGIC REASONING format only once at the very beginning of every one of your messages and not more often. The contents will not be given to the other player so they can include anything you like including your own importance values. Here you should reason step by step to come up with you next move.\\ \tt \\ \tt You will now receive the first message of the other player.\\ \tt \\ \tt ARGUMENT: {'I want to start by considering the items with the highest importance to me, such as B43, B81, and A94, and see if I can find a combination of items that fits within the limit. I will also look for items with low effort that can add value to my proposal without exceeding the limit.'}\\ \tt \\ \tt PROPOSAL: {'B43', 'A87', 'C53', 'C17', 'B69'} 
	  } 
	   } 
	   } 
	 & \\ 
 

    \theutterance \stepcounter{utterance}  

    & & & \multicolumn{4}{p{0.6\linewidth}}{\cellcolor[rgb]{1,0.85,0.72}{%
	\makecell[{{p{\linewidth}}}]{% 
	  \tt {\tiny [GM$\langle$B]}  
	 STRATEGIC REASONING: {'I will start by analyzing the other player\'s proposal and calculating the total effort of the proposed items. The total effort is 997 + 28 + 188 + 301 + 239 = 1753. This is well within the limit of 10073. I will then calculate the importance of these items to me, which is 1096 + 127 + 287 + 400 + 338 = 2248. I will also look for items that are not in the proposal but have high importance to me and low effort, to see if I can add them to the proposal without exceeding the limit. My goal is to maximize the total importance to me while staying within the limit.'}\\ \tt \\ \tt ARGUMENT: {'The proposed set includes items with relatively low effort, such as A87, C53, C17, and B69, which is a good starting point. However, I would like to explore the possibility of adding more items to the proposal without exceeding the limit. I will consider the importance of each item to me and look for opportunities to add value to the proposal.'}\\ \tt \\ \tt REFUSE: {'B43', 'A87', 'C53', 'C17', 'B69'}\\ \tt \\ \tt PROPOSAL: {'A94', 'A87', 'C53', 'C17', 'B69', 'A08'} 
	  } 
	   } 
	   } 
	  \\ 
 

    \theutterance \stepcounter{utterance}  

    & & \multicolumn{4}{p{0.6\linewidth}}{\cellcolor[rgb]{0.9,0.9,0.9}{%
	\makecell[{{p{\linewidth}}}]{% 
	  \tt {\tiny [A$\langle$GM]}  
	 ARGUMENT: {'The proposed set includes items with relatively low effort, such as A87, C53, C17, and B69, which is a good starting point. However, I would like to explore the possibility of adding more items to the proposal without exceeding the limit. I will consider the importance of each item to me and look for opportunities to add value to the proposal.'}\\ \tt \\ \tt REFUSE: {'B43', 'A87', 'C53', 'C17', 'B69'}\\ \tt \\ \tt PROPOSAL: {'A94', 'A87', 'C53', 'C17', 'B69', 'A08'} 
	  } 
	   } 
	   } 
	 & \\ 
 

    \theutterance \stepcounter{utterance}  

    & \multicolumn{4}{p{0.6\linewidth}}{\cellcolor[rgb]{0.8,1,0.9}{%
	\makecell[{{p{\linewidth}}}]{% 
	  \tt {\tiny [A$\rangle$GM]}  
	 STRATEGIC REASONING: {'The other player has refused my initial proposal and made a counter-proposal. I need to evaluate this new proposal and consider my options. I will analyze the effort and importance of the items in the new proposal and think about whether I can accept it, refuse it, or make a counter-proposal. I will also consider the potential for further negotiation and try to find a compromise that benefits both parties.'}\\ \tt \\ \tt ARGUMENT: {'The proposed set includes A94, which is a high-importance item for me, and also includes some low-effort items like A87, C53, C17, and B69. However, I am concerned about the total effort of the proposed set and whether it leaves enough room for other important items. I would like to explore alternative combinations that could potentially include more high-importance items.'}\\ \tt \\ \tt REFUSE: {'A94', 'A87', 'C53', 'C17', 'B69', 'A08'}\\ \tt \\ \tt PROPOSAL: {'B43', 'A71', 'C53', 'C17', 'B69', 'A75'} 
	  } 
	   } 
	   } 
	 & & \\ 
 

    \theutterance \stepcounter{utterance}  

    & & \multicolumn{4}{p{0.6\linewidth}}{\cellcolor[rgb]{0.9,0.9,0.9}{%
	\makecell[{{p{\linewidth}}}]{% 
	  \tt {\tiny [GM$\rangle$B]}  
	 ARGUMENT: {'The proposed set includes A94, which is a high-importance item for me, and also includes some low-effort items like A87, C53, C17, and B69. However, I am concerned about the total effort of the proposed set and whether it leaves enough room for other important items. I would like to explore alternative combinations that could potentially include more high-importance items.'}\\ \tt \\ \tt REFUSE: {'A94', 'A87', 'C53', 'C17', 'B69', 'A08'}\\ \tt \\ \tt PROPOSAL: {'B43', 'A71', 'C53', 'C17', 'B69', 'A75'} 
	  } 
	   } 
	   } 
	 & \\ 
 

    \theutterance \stepcounter{utterance}  

    & & & \multicolumn{4}{p{0.6\linewidth}}{\cellcolor[rgb]{1,0.85,0.72}{%
	\makecell[{{p{\linewidth}}}]{% 
	  \tt {\tiny [GM$\langle$B]}  
	 STRATEGIC REASONING: {'I will analyze the new proposal and calculate the total effort of the proposed items. The total effort is 997 + 588 + 188 + 301 + 239 + 76 = 2389. This is still within the limit of 10073. I will then calculate the importance of these items to me, which is 1096 + 687 + 287 + 400 + 338 + 175 = 2983. I notice that the proposal includes A71, which is a high-importance item for me, but it does not include A94, which is also very important to me. I will consider whether I can modify the proposal to include A94 and still stay within the limit.'}\\ \tt \\ \tt ARGUMENT: {'The proposed set includes a mix of high-importance items like B43 and A71, as well as low-effort items like C53, C17, B69, and A75. However, I am concerned that the proposal may not maximize the total importance to me, as it does not include A94, which is a very important item. I would like to explore alternative combinations that could potentially include A94 and other high-importance items.'}\\ \tt \\ \tt REFUSE: {'B43', 'A71', 'C53', 'C17', 'B69', 'A75'}\\ \tt \\ \tt PROPOSAL: {'A94', 'A71', 'C53', 'C17', 'B69'} 
	  } 
	   } 
	   } 
	  \\ 
 

    \theutterance \stepcounter{utterance}  

    & & \multicolumn{4}{p{0.6\linewidth}}{\cellcolor[rgb]{0.9,0.9,0.9}{%
	\makecell[{{p{\linewidth}}}]{% 
	  \tt {\tiny [A$\langle$GM]}  
	 ARGUMENT: {'The proposed set includes a mix of high-importance items like B43 and A71, as well as low-effort items like C53, C17, B69, and A75. However, I am concerned that the proposal may not maximize the total importance to me, as it does not include A94, which is a very important item. I would like to explore alternative combinations that could potentially include A94 and other high-importance items.'}\\ \tt \\ \tt REFUSE: {'B43', 'A71', 'C53', 'C17', 'B69', 'A75'}\\ \tt \\ \tt PROPOSAL: {'A94', 'A71', 'C53', 'C17', 'B69'} 
	  } 
	   } 
	   } 
	 & \\ 
 

    \theutterance \stepcounter{utterance}  

    & \multicolumn{4}{p{0.6\linewidth}}{\cellcolor[rgb]{0.8,1,0.9}{%
	\makecell[{{p{\linewidth}}}]{% 
	  \tt {\tiny [A$\rangle$GM]}  
	 STRATEGIC REASONING: {'The other player has refused my previous proposal and made a new proposal that includes A94, which is a high-importance item for me. I need to evaluate this new proposal and consider my options. I will analyze the effort and importance of the items in the new proposal and think about whether I can accept it, refuse it, or make a counter-proposal. I will also consider the potential for further negotiation and try to find a compromise that benefits both parties. It seems that A94 is a key item that the other player is willing to include, and I should consider building on this proposal.'}\\ \tt \\ \tt ARGUMENT: {'The proposed set includes A94, which is a high-importance item for me, and also includes some low-effort items like C53, C17, and B69. The inclusion of A71 adds significant importance to the proposal. I am concerned that the proposal may not be maximal, but it seems like a reasonable combination of high-importance items and low-effort items. I would like to consider whether this proposal is acceptable or if I can make a counter-proposal that includes other high-importance items.'}\\ \tt \\ \tt PROPOSAL: {'A94', 'A71', 'C53', 'C17', 'B69', 'A08'} 
	  } 
	   } 
	   } 
	 & & \\ 
 

    \theutterance \stepcounter{utterance}  

    & & \multicolumn{4}{p{0.6\linewidth}}{\cellcolor[rgb]{0.9,0.9,0.9}{%
	\makecell[{{p{\linewidth}}}]{% 
	  \tt {\tiny [GM$\rangle$B]}  
	 ARGUMENT: {'The proposed set includes A94, which is a high-importance item for me, and also includes some low-effort items like C53, C17, and B69. The inclusion of A71 adds significant importance to the proposal. I am concerned that the proposal may not be maximal, but it seems like a reasonable combination of high-importance items and low-effort items. I would like to consider whether this proposal is acceptable or if I can make a counter-proposal that includes other high-importance items.'}\\ \tt \\ \tt PROPOSAL: {'A94', 'A71', 'C53', 'C17', 'B69', 'A08'} 
	  } 
	   } 
	   } 
	 & \\ 
 

    \theutterance \stepcounter{utterance}  

    & & & \multicolumn{4}{p{0.6\linewidth}}{\cellcolor[rgb]{1,0.85,0.72}{%
	\makecell[{{p{\linewidth}}}]{% 
	  \tt {\tiny [GM$\langle$B]}  
	 STRATEGIC REASONING: {'I will analyze the new proposal and calculate the total effort of the proposed items. The total effort is 950 + 588 + 188 + 301 + 239 + 132 = 2398. This is still within the limit of 10073. I will then calculate the importance of these items to me, which is 1049 + 687 + 287 + 400 + 338 + 231 = 2992. I notice that the proposal includes A94 and A71, which are both high-importance items for me, as well as low-effort items like C53, C17, B69, and A08. I will consider whether this proposal is acceptable or if I can make a counter-proposal that includes other high-importance items. I will also consider the potential risks of making a counter-proposal, such as the other player refusing it and proposing a less desirable set.'}\\ \tt \\ \tt ARGUMENT: {'The proposed set includes a good combination of high-importance items like A94 and A71, as well as low-effort items like C53, C17, B69, and A08. I believe that this proposal is a reasonable combination of importance and effort, and I am inclined to accept it. However, I would like to consider whether there are any other high-importance items that could be added to the proposal without exceeding the limit.'}\\ \tt \\ \tt AGREE: {'A94', 'A71', 'C53', 'C17', 'B69', 'A08'} 
	  } 
	   } 
	   } 
	  \\ 
 

    \theutterance \stepcounter{utterance}  

    & & & \multicolumn{2}{p{0.3\linewidth}}{\cellcolor[rgb]{0.95,0.95,0.95}{%
	\makecell[{{p{\linewidth}}}]{% 
	  \tt {\tiny [GM$|$GM]}  
	 game successful 
	  } 
	   } 
	   } 
	 & & \\ 
 

    \theutterance \stepcounter{utterance}  

    & & & \multicolumn{2}{p{0.3\linewidth}}{\cellcolor[rgb]{0.95,0.95,0.95}{%
	\makecell[{{p{\linewidth}}}]{% 
	  \tt {\tiny [GM$|$GM]}  
	 end game 
	  } 
	   } 
	   } 
	 & & \\ 
 

\end{supertabular}
}

\end{document}
