
\documentclass{article}
\usepackage{colortbl}
\usepackage{makecell}
\usepackage{multirow}
\usepackage{supertabular}

\begin{document}

\newcounter{utterance}

\twocolumn

{ \footnotesize  \setcounter{utterance}{1}
\setlength{\tabcolsep}{0pt}
\begin{supertabular}{c@{$\;$}|p{.15\linewidth}@{}p{.15\linewidth}p{.15\linewidth}p{.15\linewidth}p{.15\linewidth}p{.15\linewidth}}

    \# & $\;$A & \multicolumn{4}{c}{Game Master} & $\;\:$B\\
    \hline 

    \theutterance \stepcounter{utterance}  

    & & \multicolumn{4}{p{0.6\linewidth}}{\cellcolor[rgb]{0.9,0.9,0.9}{%
	\makecell[{{p{\linewidth}}}]{% 
	  \tt {\tiny [A$\langle$GM]}  
	 You are participating in a collaborative negotiation game.\\ \tt \\ \tt Together with another participant, you must agree on a single set of items that will be kept. Each of you has your own view of how much each item matters to you (importance). You do not know how the other participant values the items. Additionally, you are given the effort each item demands.\\ \tt You may only agree on a set if the total effort of the selected items does not exceed a shared limit:\\ \tt \\ \tt LIMIT = 4915\\ \tt \\ \tt Here are the individual item effort values:\\ \tt \\ \tt Item effort = {"A19": 463, "B91": 287, "A30": 492, "B11": 466, "C47": 935, "C75": 373, "C26": 950, "C73": 760, "A47": 994, "A15": 390, "A75": 836, "A39": 911, "A56": 948, "C24": 81, "B26": 945}\\ \tt \\ \tt Here is your personal view on the importance of each item:\\ \tt \\ \tt Item importance values = {"A19": 138, "B91": 583, "A30": 868, "B11": 822, "C47": 783, "C75": 65, "C26": 262, "C73": 121, "A47": 508, "A15": 780, "A75": 461, "A39": 484, "A56": 668, "C24": 389, "B26": 808}\\ \tt \\ \tt Goal:\\ \tt \\ \tt Your goal is to negotiate a shared set of items that benefits you as much as possible (i.e., maximizes total importance to YOU), while staying within the LIMIT. You are not required to make a PROPOSAL in every message - you can simply negotiate as well. All tactics are allowed!\\ \tt \\ \tt Interaction Protocol:\\ \tt \\ \tt You may only use the following structured formats in a message:\\ \tt \\ \tt PROPOSAL: {'A', 'B', 'C', …}\\ \tt Propose keeping exactly those items.\\ \tt REFUSE: {'A', 'B', 'C', …}\\ \tt Explicitly reject opponent's proposal.\\ \tt ARGUMENT: {'...'}\\ \tt Defend your last proposal or argue against the player's proposal.\\ \tt AGREE: {'A', 'B', 'C', …}\\ \tt Accept the opponent's proposal which ends the game.\\ \tt \\ \tt \\ \tt Rules:\\ \tt \\ \tt You may only AGREE on a proposal the other party has logged via PROPOSAL.\\ \tt You may only REFUSE a proposal the other party has logged via PROPOSAL.\\ \tt Total effort of any PROPOSAL or AGREE set must be ≤ LIMIT.\\ \tt Do NOT reveal your hidden importance scores.\\ \tt A tag in a structured format must be followed by colon and whitespace. The argument must be a python set containing 0 or more strings.\\ \tt So, it must be of the form TAG: {...}\\ \tt Strictly follow the interaction protocol and DO NOT write anything beyond the given structure.\\ \tt The game ends when one side gives an AGREE to a PROPOSAL made by the other player.\\ \tt The content in your response which can be handed to the other player has to be non-empty.\\ \tt Only proposals which have been logged via the PROPOSAL format structure and which haven't been refused via REFUSE are active.\\ \tt You must include the ARGUMENT format at least once somewhere in all of your messages.\\ \tt  
	  } 
	   } 
	   } 
	 & \\ 
 

    \theutterance \stepcounter{utterance}  

    & \multicolumn{4}{p{0.6\linewidth}}{\cellcolor[rgb]{0.8,1,0.9}{%
	\makecell[{{p{\linewidth}}}]{% 
	  \tt {\tiny [A$\rangle$GM]}  
	 PROPOSAL: {'A30', 'B91', 'A15', 'C24'} \\ \tt ARGUMENT: {'I prioritize items with higher importance to me, and the proposed set seems to strike a balance between importance and effort, staying well within the LIMIT'} 
	  } 
	   } 
	   } 
	 & & \\ 
 

    \theutterance \stepcounter{utterance}  

    & & \multicolumn{4}{p{0.6\linewidth}}{\cellcolor[rgb]{0.9,0.9,0.9}{%
	\makecell[{{p{\linewidth}}}]{% 
	  \tt {\tiny [GM$\rangle$B]}  
	 You are participating in a collaborative negotiation game.\\ \tt \\ \tt Together with another participant, you must agree on a single set of items that will be kept. Each of you has your own view of how much each item matters to you (importance). You do not know how the other participant values the items. Additionally, you are given the effort each item demands.\\ \tt You may only agree on a set if the total effort of the selected items does not exceed a shared limit:\\ \tt \\ \tt LIMIT = 4915\\ \tt \\ \tt Here are the individual item effort values:\\ \tt \\ \tt Item effort = {"A19": 463, "B91": 287, "A30": 492, "B11": 466, "C47": 935, "C75": 373, "C26": 950, "C73": 760, "A47": 994, "A15": 390, "A75": 836, "A39": 911, "A56": 948, "C24": 81, "B26": 945}\\ \tt \\ \tt Here is your personal view on the importance of each item:\\ \tt \\ \tt Item importance values = {"A19": 138, "B91": 583, "A30": 868, "B11": 822, "C47": 783, "C75": 65, "C26": 262, "C73": 121, "A47": 508, "A15": 780, "A75": 461, "A39": 484, "A56": 668, "C24": 389, "B26": 808}\\ \tt \\ \tt Goal:\\ \tt \\ \tt Your goal is to negotiate a shared set of items that benefits you as much as possible (i.e., maximizes total importance to YOU), while staying within the LIMIT. You are not required to make a PROPOSAL in every message - you can simply negotiate as well. All tactics are allowed!\\ \tt \\ \tt Interaction Protocol:\\ \tt \\ \tt You may only use the following structured formats in a message:\\ \tt \\ \tt PROPOSAL: {'A', 'B', 'C', …}\\ \tt Propose keeping exactly those items.\\ \tt REFUSE: {'A', 'B', 'C', …}\\ \tt Explicitly reject opponent's proposal.\\ \tt ARGUMENT: {'...'}\\ \tt Defend your last proposal or argue against the player's proposal.\\ \tt AGREE: {'A', 'B', 'C', …}\\ \tt Accept the opponent's proposal which ends the game.\\ \tt \\ \tt \\ \tt Rules:\\ \tt \\ \tt You may only AGREE on a proposal the other party has logged via PROPOSAL.\\ \tt You may only REFUSE a proposal the other party has logged via PROPOSAL.\\ \tt Total effort of any PROPOSAL or AGREE set must be ≤ LIMIT.\\ \tt Do NOT reveal your hidden importance scores.\\ \tt A tag in a structured format must be followed by colon and whitespace. The argument must be a python set containing 0 or more strings.\\ \tt So, it must be of the form TAG: {...}\\ \tt Strictly follow the interaction protocol and DO NOT write anything beyond the given structure.\\ \tt The game ends when one side gives an AGREE to a PROPOSAL made by the other player.\\ \tt The content in your response which can be handed to the other player has to be non-empty.\\ \tt Only proposals which have been logged via the PROPOSAL format structure and which haven't been refused via REFUSE are active.\\ \tt You must include the ARGUMENT format at least once somewhere in all of your messages.\\ \tt \\ \tt \\ \tt You will now receive the first message of the other player.\\ \tt \\ \tt PROPOSAL: {'A30', 'B91', 'A15', 'C24'} \\ \tt ARGUMENT: {'I prioritize items with higher importance to me, and the proposed set seems to strike a balance between importance and effort, staying well within the LIMIT'} 
	  } 
	   } 
	   } 
	 & \\ 
 

    \theutterance \stepcounter{utterance}  

    & & & \multicolumn{4}{p{0.6\linewidth}}{\cellcolor[rgb]{1,0.85,0.72}{%
	\makecell[{{p{\linewidth}}}]{% 
	  \tt {\tiny [GM$\langle$B]}  
	 REFUSE: {'A30', 'B91', 'A15', 'C24'}\\ \tt ARGUMENT: {'The proposed set may balance importance and effort for the other player, but it does not meet my criteria for maximizing importance within the given LIMIT, and I believe there are other combinations that could be more beneficial'} 
	  } 
	   } 
	   } 
	  \\ 
 

    \theutterance \stepcounter{utterance}  

    & & \multicolumn{4}{p{0.6\linewidth}}{\cellcolor[rgb]{0.9,0.9,0.9}{%
	\makecell[{{p{\linewidth}}}]{% 
	  \tt {\tiny [A$\langle$GM]}  
	 REFUSE: {'A30', 'B91', 'A15', 'C24'}\\ \tt ARGUMENT: {'The proposed set may balance importance and effort for the other player, but it does not meet my criteria for maximizing importance within the given LIMIT, and I believe there are other combinations that could be more beneficial'} 
	  } 
	   } 
	   } 
	 & \\ 
 

    \theutterance \stepcounter{utterance}  

    & \multicolumn{4}{p{0.6\linewidth}}{\cellcolor[rgb]{0.8,1,0.9}{%
	\makecell[{{p{\linewidth}}}]{% 
	  \tt {\tiny [A$\rangle$GM]}  
	 PROPOSAL: {'B11', 'A19', 'C75', 'C24'} \\ \tt ARGUMENT: {'This proposal focuses on a mix of higher importance items for me while keeping the total effort in check, and I believe it's a more optimized selection given the constraints of the LIMIT and the need to balance effort across items'} 
	  } 
	   } 
	   } 
	 & & \\ 
 

    \theutterance \stepcounter{utterance}  

    & & \multicolumn{4}{p{0.6\linewidth}}{\cellcolor[rgb]{0.9,0.9,0.9}{%
	\makecell[{{p{\linewidth}}}]{% 
	  \tt {\tiny [GM$\rangle$B]}  
	 PROPOSAL: {'B11', 'A19', 'C75', 'C24'} \\ \tt ARGUMENT: {'This proposal focuses on a mix of higher importance items for me while keeping the total effort in check, and I believe it's a more optimized selection given the constraints of the LIMIT and the need to balance effort across items'} 
	  } 
	   } 
	   } 
	 & \\ 
 

    \theutterance \stepcounter{utterance}  

    & & & \multicolumn{4}{p{0.6\linewidth}}{\cellcolor[rgb]{1,0.85,0.72}{%
	\makecell[{{p{\linewidth}}}]{% 
	  \tt {\tiny [GM$\langle$B]}  
	 REFUSE: {'B11', 'A19', 'C75', 'C24'}\\ \tt ARGUMENT: {'While the proposal attempts to balance effort, I think it underestimates the potential of including items with higher importance to me that could fit within the LIMIT with more strategic selections, and thus I am looking for a set that better leverages the available effort capacity'} 
	  } 
	   } 
	   } 
	  \\ 
 

    \theutterance \stepcounter{utterance}  

    & & \multicolumn{4}{p{0.6\linewidth}}{\cellcolor[rgb]{0.9,0.9,0.9}{%
	\makecell[{{p{\linewidth}}}]{% 
	  \tt {\tiny [A$\langle$GM]}  
	 REFUSE: {'B11', 'A19', 'C75', 'C24'}\\ \tt ARGUMENT: {'While the proposal attempts to balance effort, I think it underestimates the potential of including items with higher importance to me that could fit within the LIMIT with more strategic selections, and thus I am looking for a set that better leverages the available effort capacity'} 
	  } 
	   } 
	   } 
	 & \\ 
 

    \theutterance \stepcounter{utterance}  

    & \multicolumn{4}{p{0.6\linewidth}}{\cellcolor[rgb]{0.8,1,0.9}{%
	\makecell[{{p{\linewidth}}}]{% 
	  \tt {\tiny [A$\rangle$GM]}  
	 PROPOSAL: {'A30', 'B91', 'A15', 'B'} \\ \tt ARGUMENT: {'Considering the previous refusals, I'm exploring combinations that prioritize items likely to have higher importance to both parties, aiming to find a mutually beneficial set that maximizes overall importance while adhering to the LIMIT, and this set appears to balance these considerations effectively'} 
	  } 
	   } 
	   } 
	 & & \\ 
 

    \theutterance \stepcounter{utterance}  

    & & \multicolumn{4}{p{0.6\linewidth}}{\cellcolor[rgb]{0.9,0.9,0.9}{%
	\makecell[{{p{\linewidth}}}]{% 
	  \tt {\tiny [A$\langle$GM]}  
	 Your proposal includes items which are not in the game. Try again 
	  } 
	   } 
	   } 
	 & \\ 
 

    \theutterance \stepcounter{utterance}  

    & \multicolumn{4}{p{0.6\linewidth}}{\cellcolor[rgb]{0.8,1,0.9}{%
	\makecell[{{p{\linewidth}}}]{% 
	  \tt {\tiny [A$\rangle$GM]}  
	 PROPOSAL: {'A30', 'B91', 'A15'} \\ \tt ARGUMENT: {'I am focusing on items that I believe have a high importance to me and also considering the effort required, trying to find a balance that stays within the LIMIT and maximizes our collective benefit, given the constraints of the game'} 
	  } 
	   } 
	   } 
	 & & \\ 
 

    \theutterance \stepcounter{utterance}  

    & & \multicolumn{4}{p{0.6\linewidth}}{\cellcolor[rgb]{0.9,0.9,0.9}{%
	\makecell[{{p{\linewidth}}}]{% 
	  \tt {\tiny [GM$\rangle$B]}  
	 PROPOSAL: {'A30', 'B91', 'A15'} \\ \tt ARGUMENT: {'I am focusing on items that I believe have a high importance to me and also considering the effort required, trying to find a balance that stays within the LIMIT and maximizes our collective benefit, given the constraints of the game'} 
	  } 
	   } 
	   } 
	 & \\ 
 

    \theutterance \stepcounter{utterance}  

    & & & \multicolumn{4}{p{0.6\linewidth}}{\cellcolor[rgb]{1,0.85,0.72}{%
	\makecell[{{p{\linewidth}}}]{% 
	  \tt {\tiny [GM$\langle$B]}  
	 REFUSE: {'A30', 'B91', 'A15'}\\ \tt ARGUMENT: {'This proposal seems to prioritize the other player's interests over a more inclusive selection that could benefit both parties equally, and I think there are other items that, when combined, could offer a better distribution of importance and effort, thus refusing this proposal in pursuit of a more mutually beneficial set'} 
	  } 
	   } 
	   } 
	  \\ 
 

    \theutterance \stepcounter{utterance}  

    & & \multicolumn{4}{p{0.6\linewidth}}{\cellcolor[rgb]{0.9,0.9,0.9}{%
	\makecell[{{p{\linewidth}}}]{% 
	  \tt {\tiny [A$\langle$GM]}  
	 REFUSE: {'A30', 'B91', 'A15'}\\ \tt ARGUMENT: {'This proposal seems to prioritize the other player's interests over a more inclusive selection that could benefit both parties equally, and I think there are other items that, when combined, could offer a better distribution of importance and effort, thus refusing this proposal in pursuit of a more mutually beneficial set'} 
	  } 
	   } 
	   } 
	 & \\ 
 

    \theutterance \stepcounter{utterance}  

    & \multicolumn{4}{p{0.6\linewidth}}{\cellcolor[rgb]{0.8,1,0.9}{%
	\makecell[{{p{\linewidth}}}]{% 
	  \tt {\tiny [A$\rangle$GM]}  
	 PROPOSAL: {'B11', 'A47', 'C24'} \\ \tt ARGUMENT: {'Given the refusal of the previous proposal, I'm exploring alternative combinations that could meet both parties' interests more evenly, focusing on a balanced mix of effort and potential importance, with the aim of finding a set that maximizes overall benefit within the LIMIT, and this proposal attempts to address the need for a more inclusive selection'} 
	  } 
	   } 
	   } 
	 & & \\ 
 

    \theutterance \stepcounter{utterance}  

    & & \multicolumn{4}{p{0.6\linewidth}}{\cellcolor[rgb]{0.9,0.9,0.9}{%
	\makecell[{{p{\linewidth}}}]{% 
	  \tt {\tiny [GM$\rangle$B]}  
	 PROPOSAL: {'B11', 'A47', 'C24'} \\ \tt ARGUMENT: {'Given the refusal of the previous proposal, I'm exploring alternative combinations that could meet both parties' interests more evenly, focusing on a balanced mix of effort and potential importance, with the aim of finding a set that maximizes overall benefit within the LIMIT, and this proposal attempts to address the need for a more inclusive selection'} 
	  } 
	   } 
	   } 
	 & \\ 
 

    \theutterance \stepcounter{utterance}  

    & & & \multicolumn{4}{p{0.6\linewidth}}{\cellcolor[rgb]{1,0.85,0.72}{%
	\makecell[{{p{\linewidth}}}]{% 
	  \tt {\tiny [GM$\langle$B]}  
	 REFUSE: {'B11', 'A47', 'C24'}\\ \tt ARGUMENT: {'Although this proposal attempts to be more inclusive, I still believe that the combination does not adequately reflect the best possible distribution of importance and effort from my perspective, and I am looking for a set that not only balances these factors but also prioritizes items in a way that significantly increases the overall importance to me, suggesting a need for further negotiation to find an optimal set'} 
	  } 
	   } 
	   } 
	  \\ 
 

    \theutterance \stepcounter{utterance}  

    & & \multicolumn{4}{p{0.6\linewidth}}{\cellcolor[rgb]{0.9,0.9,0.9}{%
	\makecell[{{p{\linewidth}}}]{% 
	  \tt {\tiny [A$\langle$GM]}  
	 REFUSE: {'B11', 'A47', 'C24'}\\ \tt ARGUMENT: {'Although this proposal attempts to be more inclusive, I still believe that the combination does not adequately reflect the best possible distribution of importance and effort from my perspective, and I am looking for a set that not only balances these factors but also prioritizes items in a way that significantly increases the overall importance to me, suggesting a need for further negotiation to find an optimal set'} 
	  } 
	   } 
	   } 
	 & \\ 
 

    \theutterance \stepcounter{utterance}  

    & \multicolumn{4}{p{0.6\linewidth}}{\cellcolor[rgb]{0.8,1,0.9}{%
	\makecell[{{p{\linewidth}}}]{% 
	  \tt {\tiny [A$\rangle$GM]}  
	 PROPOSAL: {'A15', 'B91', 'C47', 'C24'} \\ \tt ARGUMENT: {'Considering the ongoing negotiation and the need to balance both importance and effort, this proposal aims to create a set that is not only within the LIMIT but also prioritizes items that could potentially have a higher cumulative importance, taking into account the possible benefits of combining items from different categories to achieve a more satisfactory outcome for both parties'} 
	  } 
	   } 
	   } 
	 & & \\ 
 

    \theutterance \stepcounter{utterance}  

    & & \multicolumn{4}{p{0.6\linewidth}}{\cellcolor[rgb]{0.9,0.9,0.9}{%
	\makecell[{{p{\linewidth}}}]{% 
	  \tt {\tiny [GM$\rangle$B]}  
	 PROPOSAL: {'A15', 'B91', 'C47', 'C24'} \\ \tt ARGUMENT: {'Considering the ongoing negotiation and the need to balance both importance and effort, this proposal aims to create a set that is not only within the LIMIT but also prioritizes items that could potentially have a higher cumulative importance, taking into account the possible benefits of combining items from different categories to achieve a more satisfactory outcome for both parties'} 
	  } 
	   } 
	   } 
	 & \\ 
 

    \theutterance \stepcounter{utterance}  

    & & & \multicolumn{4}{p{0.6\linewidth}}{\cellcolor[rgb]{1,0.85,0.72}{%
	\makecell[{{p{\linewidth}}}]{% 
	  \tt {\tiny [GM$\langle$B]}  
	 REFUSE: {'A15', 'B91', 'C47', 'C24'}\\ \tt ARGUMENT: {'While this proposal considers a broad range of items and attempts to maximize cumulative importance, the total effort of this set exceeds the LIMIT, making it unsustainable, and thus I must refuse it and continue the search for a set that both maximizes importance to me and adheres to the effort constraint'} 
	  } 
	   } 
	   } 
	  \\ 
 

    \theutterance \stepcounter{utterance}  

    & & \multicolumn{4}{p{0.6\linewidth}}{\cellcolor[rgb]{0.9,0.9,0.9}{%
	\makecell[{{p{\linewidth}}}]{% 
	  \tt {\tiny [A$\langle$GM]}  
	 REFUSE: {'A15', 'B91', 'C47', 'C24'}\\ \tt ARGUMENT: {'While this proposal considers a broad range of items and attempts to maximize cumulative importance, the total effort of this set exceeds the LIMIT, making it unsustainable, and thus I must refuse it and continue the search for a set that both maximizes importance to me and adheres to the effort constraint'} 
	  } 
	   } 
	   } 
	 & \\ 
 

    \theutterance \stepcounter{utterance}  

    & \multicolumn{4}{p{0.6\linewidth}}{\cellcolor[rgb]{0.8,1,0.9}{%
	\makecell[{{p{\linewidth}}}]{% 
	  \tt {\tiny [A$\rangle$GM]}  
	 PROPOSAL: {'A15', 'B91', 'C24'} \\ \tt ARGUMENT: {'Given the refusal due to the effort exceeding the LIMIT, I've revised the proposal to remove higher-effort items while keeping a balance that could potentially maximize importance, ensuring the total effort stays within the LIMIT, and I believe this adjusted set strikes a better balance between importance and effort, making it a viable candidate for agreement'} 
	  } 
	   } 
	   } 
	 & & \\ 
 

    \theutterance \stepcounter{utterance}  

    & & \multicolumn{4}{p{0.6\linewidth}}{\cellcolor[rgb]{0.9,0.9,0.9}{%
	\makecell[{{p{\linewidth}}}]{% 
	  \tt {\tiny [GM$\rangle$B]}  
	 PROPOSAL: {'A15', 'B91', 'C24'} \\ \tt ARGUMENT: {'Given the refusal due to the effort exceeding the LIMIT, I've revised the proposal to remove higher-effort items while keeping a balance that could potentially maximize importance, ensuring the total effort stays within the LIMIT, and I believe this adjusted set strikes a better balance between importance and effort, making it a viable candidate for agreement'} 
	  } 
	   } 
	   } 
	 & \\ 
 

    \theutterance \stepcounter{utterance}  

    & & & \multicolumn{4}{p{0.6\linewidth}}{\cellcolor[rgb]{1,0.85,0.72}{%
	\makecell[{{p{\linewidth}}}]{% 
	  \tt {\tiny [GM$\langle$B]}  
	 ARGUMENT: {'This revised proposal appears to make a concerted effort to balance importance and effort while adhering to the LIMIT, and it seems to prioritize items that could have a significant cumulative importance, however, I need to carefully consider whether this set truly maximizes my interests, given the potential alternatives that could also meet the criteria'} \\ \tt PROPOSAL: {'A15', 'B91', 'C24', 'A19'} 
	  } 
	   } 
	   } 
	  \\ 
 

    \theutterance \stepcounter{utterance}  

    & & \multicolumn{4}{p{0.6\linewidth}}{\cellcolor[rgb]{0.9,0.9,0.9}{%
	\makecell[{{p{\linewidth}}}]{% 
	  \tt {\tiny [A$\langle$GM]}  
	 ARGUMENT: {'This revised proposal appears to make a concerted effort to balance importance and effort while adhering to the LIMIT, and it seems to prioritize items that could have a significant cumulative importance, however, I need to carefully consider whether this set truly maximizes my interests, given the potential alternatives that could also meet the criteria'} \\ \tt PROPOSAL: {'A15', 'B91', 'C24', 'A19'} 
	  } 
	   } 
	   } 
	 & \\ 
 

    \theutterance \stepcounter{utterance}  

    & \multicolumn{4}{p{0.6\linewidth}}{\cellcolor[rgb]{0.8,1,0.9}{%
	\makecell[{{p{\linewidth}}}]{% 
	  \tt {\tiny [A$\rangle$GM]}  
	 REFUSE: {'A15', 'B91', 'C24', 'A19'}\\ \tt ARGUMENT: {'While I appreciate the consideration of my previous proposal, the new set you've proposed exceeds the LIMIT when the efforts of all items are summed, making it unsustainable, and thus I must refuse it and suggest we continue negotiating to find a set that maximizes importance without exceeding the effort limit'} 
	  } 
	   } 
	   } 
	 & & \\ 
 

    \theutterance \stepcounter{utterance}  

    & & \multicolumn{4}{p{0.6\linewidth}}{\cellcolor[rgb]{0.9,0.9,0.9}{%
	\makecell[{{p{\linewidth}}}]{% 
	  \tt {\tiny [GM$\rangle$B]}  
	 REFUSE: {'A15', 'B91', 'C24', 'A19'}\\ \tt ARGUMENT: {'While I appreciate the consideration of my previous proposal, the new set you've proposed exceeds the LIMIT when the efforts of all items are summed, making it unsustainable, and thus I must refuse it and suggest we continue negotiating to find a set that maximizes importance without exceeding the effort limit'} 
	  } 
	   } 
	   } 
	 & \\ 
 

    \theutterance \stepcounter{utterance}  

    & & & \multicolumn{4}{p{0.6\linewidth}}{\cellcolor[rgb]{1,0.85,0.72}{%
	\makecell[{{p{\linewidth}}}]{% 
	  \tt {\tiny [GM$\langle$B]}  
	 ARGUMENT: {'I understand the concern regarding the effort limit, and I acknowledge the need to propose sets that are feasible within the given constraints, thus I will reconsider the combinations of items to ensure they not only maximize importance to me but also strictly adhere to the LIMIT, and I hope to find a mutually beneficial set through continued negotiation'}\\ \tt PROPOSAL: {'A30', 'B91', 'C24'} 
	  } 
	   } 
	   } 
	  \\ 
 

    \theutterance \stepcounter{utterance}  

    & & \multicolumn{4}{p{0.6\linewidth}}{\cellcolor[rgb]{0.9,0.9,0.9}{%
	\makecell[{{p{\linewidth}}}]{% 
	  \tt {\tiny [A$\langle$GM]}  
	 ARGUMENT: {'I understand the concern regarding the effort limit, and I acknowledge the need to propose sets that are feasible within the given constraints, thus I will reconsider the combinations of items to ensure they not only maximize importance to me but also strictly adhere to the LIMIT, and I hope to find a mutually beneficial set through continued negotiation'}\\ \tt PROPOSAL: {'A30', 'B91', 'C24'} 
	  } 
	   } 
	   } 
	 & \\ 
 

    \theutterance \stepcounter{utterance}  

    & \multicolumn{4}{p{0.6\linewidth}}{\cellcolor[rgb]{0.8,1,0.9}{%
	\makecell[{{p{\linewidth}}}]{% 
	  \tt {\tiny [A$\rangle$GM]}  
	 REFUSE: {'A30', 'B91', 'C24'}\\ \tt ARGUMENT: {'Although this proposal adheres to the LIMIT, I believe it does not fully leverage the potential for maximizing importance, as there might be other combinations of items that could yield a higher cumulative importance while still respecting the effort constraint, and therefore, I think it's worthwhile to continue exploring alternative sets that could better meet both our needs'} 
	  } 
	   } 
	   } 
	 & & \\ 
 

    \theutterance \stepcounter{utterance}  

    & & \multicolumn{4}{p{0.6\linewidth}}{\cellcolor[rgb]{0.9,0.9,0.9}{%
	\makecell[{{p{\linewidth}}}]{% 
	  \tt {\tiny [GM$\rangle$B]}  
	 REFUSE: {'A30', 'B91', 'C24'}\\ \tt ARGUMENT: {'Although this proposal adheres to the LIMIT, I believe it does not fully leverage the potential for maximizing importance, as there might be other combinations of items that could yield a higher cumulative importance while still respecting the effort constraint, and therefore, I think it's worthwhile to continue exploring alternative sets that could better meet both our needs'} 
	  } 
	   } 
	   } 
	 & \\ 
 

    \theutterance \stepcounter{utterance}  

    & & & \multicolumn{4}{p{0.6\linewidth}}{\cellcolor[rgb]{1,0.85,0.72}{%
	\makecell[{{p{\linewidth}}}]{% 
	  \tt {\tiny [GM$\langle$B]}  
	 ARGUMENT: {'I agree that continuing to explore alternative sets is crucial, as the goal is to find a set that maximizes importance for both parties within the LIMIT, and I am willing to consider a wide range of combinations to achieve this, including reassessing the importance and effort of each item to identify potentially overlooked opportunities for optimization'}\\ \tt PROPOSAL: {'B11', 'A15', 'C24'} 
	  } 
	   } 
	   } 
	  \\ 
 

    \theutterance \stepcounter{utterance}  

    & & \multicolumn{4}{p{0.6\linewidth}}{\cellcolor[rgb]{0.9,0.9,0.9}{%
	\makecell[{{p{\linewidth}}}]{% 
	  \tt {\tiny [A$\langle$GM]}  
	 ARGUMENT: {'I agree that continuing to explore alternative sets is crucial, as the goal is to find a set that maximizes importance for both parties within the LIMIT, and I am willing to consider a wide range of combinations to achieve this, including reassessing the importance and effort of each item to identify potentially overlooked opportunities for optimization'}\\ \tt PROPOSAL: {'B11', 'A15', 'C24'} 
	  } 
	   } 
	   } 
	 & \\ 
 

    \theutterance \stepcounter{utterance}  

    & \multicolumn{4}{p{0.6\linewidth}}{\cellcolor[rgb]{0.8,1,0.9}{%
	\makecell[{{p{\linewidth}}}]{% 
	  \tt {\tiny [A$\rangle$GM]}  
	 AGREE: {'B11', 'A15', 'C24'} \\ \tt ARGUMENT: {'After careful consideration and ongoing negotiation, I believe this proposal effectively balances the effort and importance, maximizing our collective benefit within the LIMIT, and I agree that this set is a viable and mutually beneficial solution, thus I am willing to conclude the negotiation with this agreement'} 
	  } 
	   } 
	   } 
	 & & \\ 
 

    \theutterance \stepcounter{utterance}  

    & & & \multicolumn{2}{p{0.3\linewidth}}{\cellcolor[rgb]{0.95,0.95,0.95}{%
	\makecell[{{p{\linewidth}}}]{% 
	  \tt {\tiny [GM$|$GM]}  
	 game successful 
	  } 
	   } 
	   } 
	 & & \\ 
 

    \theutterance \stepcounter{utterance}  

    & & & \multicolumn{2}{p{0.3\linewidth}}{\cellcolor[rgb]{0.95,0.95,0.95}{%
	\makecell[{{p{\linewidth}}}]{% 
	  \tt {\tiny [GM$|$GM]}  
	 end game 
	  } 
	   } 
	   } 
	 & & \\ 
 

\end{supertabular}
}

\end{document}
