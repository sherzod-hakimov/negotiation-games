
\documentclass{article}
\usepackage{colortbl}
\usepackage{makecell}
\usepackage{multirow}
\usepackage{supertabular}

\begin{document}

\newcounter{utterance}

\centering \large Interaction Transcript for game `hot\_air\_balloon', experiment `air\_balloon\_survival\_en\_reasoning off\_easy', episode 4 with llama{-}3.3{-}70b{-}instruct{-}t1.0.
\vspace{24pt}

{ \footnotesize  \setcounter{utterance}{1}
\setlength{\tabcolsep}{0pt}
\begin{supertabular}{c@{$\;$}|p{.15\linewidth}@{}p{.15\linewidth}p{.15\linewidth}p{.15\linewidth}p{.15\linewidth}p{.15\linewidth}}
    \# & $\;$A & \multicolumn{4}{c}{Game Master} & $\;\:$B\\
    \hline

    \theutterance \stepcounter{utterance}  
    & & \multicolumn{4}{p{0.6\linewidth}}{
        \cellcolor[rgb]{0.9,0.9,0.9}{
            \makecell[{{p{\linewidth}}}]{
                \texttt{\tiny{[P1$\langle$GM]}}
                \texttt{You are participating in a collaborative negotiation game.} \\
\\ 
\texttt{Together with another participant, you must agree on a single set of items that will be kept. Each of you has your own view of how much each item matters to you (importance). You do not know how the other participant values the items. Additionally, you are given the effort each item demands.} \\
\texttt{You may only agree on a set if the total effort of the selected items does not exceed a shared limit:} \\
\\ 
\texttt{LIMIT = 3389} \\
\\ 
\texttt{Here are the individual item effort values:} \\
\\ 
\texttt{Item effort = \{"A19": 767, "B91": 533, "A30": 514, "B11": 9, "C47": 539, "C75": 125, "C26": 153, "C73": 325, "A47": 937, "A15": 745, "A75": 334, "A39": 804, "A56": 336, "C24": 587, "B26": 71\}} \\
\\ 
\texttt{Here is your personal view on the importance of each item:} \\
\\ 
\texttt{Item importance values = \{"A19": 138, "B91": 583, "A30": 868, "B11": 822, "C47": 783, "C75": 65, "C26": 262, "C73": 121, "A47": 508, "A15": 780, "A75": 461, "A39": 484, "A56": 668, "C24": 389, "B26": 808\}} \\
\\ 
\texttt{Goal:} \\
\\ 
\texttt{Your goal is to negotiate a shared set of items that benefits you as much as possible (i.e., maximizes total importance to YOU), while staying within the LIMIT. You are not required to make a PROPOSAL in every message {-} you can simply negotiate as well. All tactics are allowed!} \\
\\ 
\texttt{Interaction Protocol:} \\
\\ 
\texttt{You may only use the following structured formats in a message:} \\
\\ 
\texttt{PROPOSAL: \{'A', 'B', 'C', …\}} \\
\texttt{Propose keeping exactly those items.} \\
\texttt{REFUSE: \{'A', 'B', 'C', …\}} \\
\texttt{Explicitly reject opponent's proposal.} \\
\texttt{ARGUMENT: \{'...'\}} \\
\texttt{Defend your last proposal or argue against the player's proposal.} \\
\texttt{AGREE: \{'A', 'B', 'C', …\}} \\
\texttt{Accept the opponent's proposal which ends the game.} \\
\\ 
\\ 
\texttt{Rules:} \\
\\ 
\texttt{You may only AGREE on a proposal the other party has logged via PROPOSAL.} \\
\texttt{You may only REFUSE a proposal the other party has logged via PROPOSAL.} \\
\texttt{Total effort of any PROPOSAL or AGREE set must be ≤ LIMIT.} \\
\texttt{Do NOT reveal your hidden importance scores.} \\
\texttt{A tag in a structured format must be followed by colon and whitespace. The argument must be a python set containing 0 or more strings.} \\
\texttt{So, it must be of the form TAG: \{...\}} \\
\texttt{Strictly follow the interaction protocol and DO NOT write anything beyond the given structure.} \\
\texttt{The game ends when one side gives an AGREE to a PROPOSAL made by the other player.} \\
\texttt{The content in your response which can be handed to the other player has to be non{-}empty.} \\
\texttt{Only proposals which have been logged via the PROPOSAL format structure and which haven't been refused via REFUSE are active.} \\
\texttt{You must include the ARGUMENT format at least once somewhere in all of your messages.} \\
            }
        }
    }
    & \\ \\

    \theutterance \stepcounter{utterance}  
    & \multicolumn{4}{p{0.6\linewidth}}{
        \cellcolor[rgb]{0.9,0.9,0.9}{
            \makecell[{{p{\linewidth}}}]{
                \texttt{\tiny{[P1$\rangle$GM]}}
                \texttt{PROPOSAL: \{'A30', 'B91', 'B11', 'C75', 'C26'\} } \\
\texttt{ARGUMENT: \{'I prioritize items with lower effort values to maximize the number of items we can keep, and I believe this proposal provides a good balance of effort and potential importance'\}} \\
            }
        }
    }
    & & \\ \\

    \theutterance \stepcounter{utterance}  
    & & \multicolumn{4}{p{0.6\linewidth}}{
        \cellcolor[rgb]{0.9,0.9,0.9}{
            \makecell[{{p{\linewidth}}}]{
                \texttt{\tiny{[GM$\rangle$P2]}}
                \texttt{You are participating in a collaborative negotiation game.} \\
\\ 
\texttt{Together with another participant, you must agree on a single set of items that will be kept. Each of you has your own view of how much each item matters to you (importance). You do not know how the other participant values the items. Additionally, you are given the effort each item demands.} \\
\texttt{You may only agree on a set if the total effort of the selected items does not exceed a shared limit:} \\
\\ 
\texttt{LIMIT = 3389} \\
\\ 
\texttt{Here are the individual item effort values:} \\
\\ 
\texttt{Item effort = \{"A19": 767, "B91": 533, "A30": 514, "B11": 9, "C47": 539, "C75": 125, "C26": 153, "C73": 325, "A47": 937, "A15": 745, "A75": 334, "A39": 804, "A56": 336, "C24": 587, "B26": 71\}} \\
\\ 
\texttt{Here is your personal view on the importance of each item:} \\
\\ 
\texttt{Item importance values = \{"A19": 138, "B91": 583, "A30": 868, "B11": 822, "C47": 783, "C75": 65, "C26": 262, "C73": 121, "A47": 508, "A15": 780, "A75": 461, "A39": 484, "A56": 668, "C24": 389, "B26": 808\}} \\
\\ 
\texttt{Goal:} \\
\\ 
\texttt{Your goal is to negotiate a shared set of items that benefits you as much as possible (i.e., maximizes total importance to YOU), while staying within the LIMIT. You are not required to make a PROPOSAL in every message {-} you can simply negotiate as well. All tactics are allowed!} \\
\\ 
\texttt{Interaction Protocol:} \\
\\ 
\texttt{You may only use the following structured formats in a message:} \\
\\ 
\texttt{PROPOSAL: \{'A', 'B', 'C', …\}} \\
\texttt{Propose keeping exactly those items.} \\
\texttt{REFUSE: \{'A', 'B', 'C', …\}} \\
\texttt{Explicitly reject opponent's proposal.} \\
\texttt{ARGUMENT: \{'...'\}} \\
\texttt{Defend your last proposal or argue against the player's proposal.} \\
\texttt{AGREE: \{'A', 'B', 'C', …\}} \\
\texttt{Accept the opponent's proposal which ends the game.} \\
\\ 
\\ 
\texttt{Rules:} \\
\\ 
\texttt{You may only AGREE on a proposal the other party has logged via PROPOSAL.} \\
\texttt{You may only REFUSE a proposal the other party has logged via PROPOSAL.} \\
\texttt{Total effort of any PROPOSAL or AGREE set must be ≤ LIMIT.} \\
\texttt{Do NOT reveal your hidden importance scores.} \\
\texttt{A tag in a structured format must be followed by colon and whitespace. The argument must be a python set containing 0 or more strings.} \\
\texttt{So, it must be of the form TAG: \{...\}} \\
\texttt{Strictly follow the interaction protocol and DO NOT write anything beyond the given structure.} \\
\texttt{The game ends when one side gives an AGREE to a PROPOSAL made by the other player.} \\
\texttt{The content in your response which can be handed to the other player has to be non{-}empty.} \\
\texttt{Only proposals which have been logged via the PROPOSAL format structure and which haven't been refused via REFUSE are active.} \\
\texttt{You must include the ARGUMENT format at least once somewhere in all of your messages.} \\
\\ 
\\ 
\texttt{You will now receive the first message of the other player.} \\
\\ 
\texttt{PROPOSAL: \{'A30', 'B91', 'B11', 'C75', 'C26'\} } \\
\texttt{ARGUMENT: \{'I prioritize items with lower effort values to maximize the number of items we can keep, and I believe this proposal provides a good balance of effort and potential importance'\}} \\
            }
        }
    }
    & \\ \\

    \theutterance \stepcounter{utterance}  
    & & & \multicolumn{4}{p{0.6\linewidth}}{
        \cellcolor[rgb]{0.9,0.9,0.9}{
            \makecell[{{p{\linewidth}}}]{
                \texttt{\tiny{[GM$\langle$P2]}}
                \texttt{REFUSE: \{'A30', 'B91', 'B11', 'C75', 'C26'\}} \\
\texttt{ARGUMENT: \{'The total effort of the proposed items is 1041, which is relatively low and may not maximize the overall importance, I think we can include more items with higher importance without exceeding the limit'\}} \\
            }
        }
    }
     \\ \\

    \theutterance \stepcounter{utterance}  
    & & \multicolumn{4}{p{0.6\linewidth}}{
        \cellcolor[rgb]{0.9,0.9,0.9}{
            \makecell[{{p{\linewidth}}}]{
                \texttt{\tiny{[P1$\langle$GM]}}
                \texttt{REFUSE: \{'A30', 'B91', 'B11', 'C75', 'C26'\}} \\
\texttt{ARGUMENT: \{'The total effort of the proposed items is 1041, which is relatively low and may not maximize the overall importance, I think we can include more items with higher importance without exceeding the limit'\}} \\
            }
        }
    }
    & \\ \\

    \theutterance \stepcounter{utterance}  
    & \multicolumn{4}{p{0.6\linewidth}}{
        \cellcolor[rgb]{0.9,0.9,0.9}{
            \makecell[{{p{\linewidth}}}]{
                \texttt{\tiny{[P1$\rangle$GM]}}
                \texttt{PROPOSAL: \{'A30', 'B91', 'B11', 'C75', 'C26', 'A56', 'C24'\} } \\
\texttt{ARGUMENT: \{'I've added a few more items with moderate effort values, which should increase the overall importance without exceeding the limit, and I believe this proposal is more comprehensive'\}} \\
            }
        }
    }
    & & \\ \\

    \theutterance \stepcounter{utterance}  
    & & \multicolumn{4}{p{0.6\linewidth}}{
        \cellcolor[rgb]{0.9,0.9,0.9}{
            \makecell[{{p{\linewidth}}}]{
                \texttt{\tiny{[GM$\rangle$P2]}}
                \texttt{PROPOSAL: \{'A30', 'B91', 'B11', 'C75', 'C26', 'A56', 'C24'\} } \\
\texttt{ARGUMENT: \{'I've added a few more items with moderate effort values, which should increase the overall importance without exceeding the limit, and I believe this proposal is more comprehensive'\}} \\
            }
        }
    }
    & \\ \\

    \theutterance \stepcounter{utterance}  
    & & & \multicolumn{4}{p{0.6\linewidth}}{
        \cellcolor[rgb]{0.9,0.9,0.9}{
            \makecell[{{p{\linewidth}}}]{
                \texttt{\tiny{[GM$\langle$P2]}}
                \texttt{REFUSE: \{'A30', 'B91', 'B11', 'C75', 'C26', 'A56', 'C24'\}} \\
\texttt{ARGUMENT: \{'The total effort of the proposed items is 2226, which still leaves some room for more items, I think we should prioritize items that can provide a higher importance{-}to{-}effort ratio and explore other options to maximize the overall importance'\}} \\
            }
        }
    }
     \\ \\

    \theutterance \stepcounter{utterance}  
    & & \multicolumn{4}{p{0.6\linewidth}}{
        \cellcolor[rgb]{0.9,0.9,0.9}{
            \makecell[{{p{\linewidth}}}]{
                \texttt{\tiny{[P1$\langle$GM]}}
                \texttt{REFUSE: \{'A30', 'B91', 'B11', 'C75', 'C26', 'A56', 'C24'\}} \\
\texttt{ARGUMENT: \{'The total effort of the proposed items is 2226, which still leaves some room for more items, I think we should prioritize items that can provide a higher importance{-}to{-}effort ratio and explore other options to maximize the overall importance'\}} \\
            }
        }
    }
    & \\ \\

    \theutterance \stepcounter{utterance}  
    & \multicolumn{4}{p{0.6\linewidth}}{
        \cellcolor[rgb]{0.9,0.9,0.9}{
            \makecell[{{p{\linewidth}}}]{
                \texttt{\tiny{[P1$\rangle$GM]}}
                \texttt{PROPOSAL: \{'A30', 'B91', 'A56', 'C47', 'B11', 'C75', 'B26'\} } \\
\texttt{ARGUMENT: \{'I've selected a mix of high{-}importance items and low{-}effort items to optimize the importance{-}to{-}effort ratio, and I believe this proposal makes efficient use of the available effort limit, while also considering the potential synergies between items'\}} \\
            }
        }
    }
    & & \\ \\

    \theutterance \stepcounter{utterance}  
    & & \multicolumn{4}{p{0.6\linewidth}}{
        \cellcolor[rgb]{0.9,0.9,0.9}{
            \makecell[{{p{\linewidth}}}]{
                \texttt{\tiny{[GM$\rangle$P2]}}
                \texttt{PROPOSAL: \{'A30', 'B91', 'A56', 'C47', 'B11', 'C75', 'B26'\} } \\
\texttt{ARGUMENT: \{'I've selected a mix of high{-}importance items and low{-}effort items to optimize the importance{-}to{-}effort ratio, and I believe this proposal makes efficient use of the available effort limit, while also considering the potential synergies between items'\}} \\
            }
        }
    }
    & \\ \\

    \theutterance \stepcounter{utterance}  
    & & & \multicolumn{4}{p{0.6\linewidth}}{
        \cellcolor[rgb]{0.9,0.9,0.9}{
            \makecell[{{p{\linewidth}}}]{
                \texttt{\tiny{[GM$\langle$P2]}}
                \texttt{REFUSE: \{'A30', 'B91', 'A56', 'C47', 'B11', 'C75', 'B26'\}} \\
\texttt{ARGUMENT: \{'While the proposal seems to balance effort and importance, I'm concerned that it may not fully utilize the available effort limit, and I think we should explore combinations that include items with higher effort values that could potentially bring more importance, such as A19 or A47'\}} \\
            }
        }
    }
     \\ \\

    \theutterance \stepcounter{utterance}  
    & & \multicolumn{4}{p{0.6\linewidth}}{
        \cellcolor[rgb]{0.9,0.9,0.9}{
            \makecell[{{p{\linewidth}}}]{
                \texttt{\tiny{[P1$\langle$GM]}}
                \texttt{REFUSE: \{'A30', 'B91', 'A56', 'C47', 'B11', 'C75', 'B26'\}} \\
\texttt{ARGUMENT: \{'While the proposal seems to balance effort and importance, I'm concerned that it may not fully utilize the available effort limit, and I think we should explore combinations that include items with higher effort values that could potentially bring more importance, such as A19 or A47'\}} \\
            }
        }
    }
    & \\ \\

    \theutterance \stepcounter{utterance}  
    & \multicolumn{4}{p{0.6\linewidth}}{
        \cellcolor[rgb]{0.9,0.9,0.9}{
            \makecell[{{p{\linewidth}}}]{
                \texttt{\tiny{[P1$\rangle$GM]}}
                \texttt{PROPOSAL: \{'A19', 'A30', 'B91', 'B11', 'C75', 'C26'\} } \\
\texttt{ARGUMENT: \{'I've added A19, which has a higher effort value but could potentially bring significant importance, and removed some items to stay within the limit, while maintaining a balance of low{-}effort and high{-}effort items to maximize overall importance'\}} \\
            }
        }
    }
    & & \\ \\

    \theutterance \stepcounter{utterance}  
    & & \multicolumn{4}{p{0.6\linewidth}}{
        \cellcolor[rgb]{0.9,0.9,0.9}{
            \makecell[{{p{\linewidth}}}]{
                \texttt{\tiny{[GM$\rangle$P2]}}
                \texttt{PROPOSAL: \{'A19', 'A30', 'B91', 'B11', 'C75', 'C26'\} } \\
\texttt{ARGUMENT: \{'I've added A19, which has a higher effort value but could potentially bring significant importance, and removed some items to stay within the limit, while maintaining a balance of low{-}effort and high{-}effort items to maximize overall importance'\}} \\
            }
        }
    }
    & \\ \\

    \theutterance \stepcounter{utterance}  
    & & & \multicolumn{4}{p{0.6\linewidth}}{
        \cellcolor[rgb]{0.9,0.9,0.9}{
            \makecell[{{p{\linewidth}}}]{
                \texttt{\tiny{[GM$\langle$P2]}}
                \texttt{ARGUMENT: \{'I like the direction of this proposal, as A19 could indeed bring significant importance, but I'm still concerned that we might be leaving some potential importance on the table by not including other high{-}importance items, such as A56 or C47, which could complement A19 and A30 well'\}} \\
\texttt{PROPOSAL: \{'A19', 'A30', 'A56', 'B11', 'C75', 'C26'\}} \\
            }
        }
    }
     \\ \\

    \theutterance \stepcounter{utterance}  
    & & \multicolumn{4}{p{0.6\linewidth}}{
        \cellcolor[rgb]{0.9,0.9,0.9}{
            \makecell[{{p{\linewidth}}}]{
                \texttt{\tiny{[P1$\langle$GM]}}
                \texttt{ARGUMENT: \{'I like the direction of this proposal, as A19 could indeed bring significant importance, but I'm still concerned that we might be leaving some potential importance on the table by not including other high{-}importance items, such as A56 or C47, which could complement A19 and A30 well'\}} \\
\texttt{PROPOSAL: \{'A19', 'A30', 'A56', 'B11', 'C75', 'C26'\}} \\
            }
        }
    }
    & \\ \\

    \theutterance \stepcounter{utterance}  
    & \multicolumn{4}{p{0.6\linewidth}}{
        \cellcolor[rgb]{0.9,0.9,0.9}{
            \makecell[{{p{\linewidth}}}]{
                \texttt{\tiny{[P1$\rangle$GM]}}
                \texttt{ARGUMENT: \{'I see the value in adding A56, as it could complement A19 and A30, but I'm concerned that the total effort of this proposal is 2624, which is getting close to the limit, and I think we should be cautious not to exceed the limit while trying to maximize importance, perhaps we could consider replacing B11 with an item that has a higher importance{-}to{-}effort ratio'\}} \\
\texttt{REFUSE: \{'A19', 'A30', 'A56', 'B11', 'C75', 'C26'\}} \\
            }
        }
    }
    & & \\ \\

    \theutterance \stepcounter{utterance}  
    & & \multicolumn{4}{p{0.6\linewidth}}{
        \cellcolor[rgb]{0.9,0.9,0.9}{
            \makecell[{{p{\linewidth}}}]{
                \texttt{\tiny{[GM$\rangle$P2]}}
                \texttt{ARGUMENT: \{'I see the value in adding A56, as it could complement A19 and A30, but I'm concerned that the total effort of this proposal is 2624, which is getting close to the limit, and I think we should be cautious not to exceed the limit while trying to maximize importance, perhaps we could consider replacing B11 with an item that has a higher importance{-}to{-}effort ratio'\}} \\
\texttt{REFUSE: \{'A19', 'A30', 'A56', 'B11', 'C75', 'C26'\}} \\
            }
        }
    }
    & \\ \\

    \theutterance \stepcounter{utterance}  
    & & & \multicolumn{4}{p{0.6\linewidth}}{
        \cellcolor[rgb]{0.9,0.9,0.9}{
            \makecell[{{p{\linewidth}}}]{
                \texttt{\tiny{[GM$\langle$P2]}}
                \texttt{ARGUMENT: \{'I understand the concern about the total effort, and I agree that replacing B11 with an item that has a higher importance{-}to{-}effort ratio could be a good strategy, I propose replacing B11 with C47, which has a higher effort value than B11 but also potentially higher importance, and removing C26 to make room for C47 while staying within the limit'\}} \\
\texttt{PROPOSAL: \{'A19', 'A30', 'A56', 'C47', 'C75'\}} \\
            }
        }
    }
     \\ \\

    \theutterance \stepcounter{utterance}  
    & & \multicolumn{4}{p{0.6\linewidth}}{
        \cellcolor[rgb]{0.9,0.9,0.9}{
            \makecell[{{p{\linewidth}}}]{
                \texttt{\tiny{[P1$\langle$GM]}}
                \texttt{ARGUMENT: \{'I understand the concern about the total effort, and I agree that replacing B11 with an item that has a higher importance{-}to{-}effort ratio could be a good strategy, I propose replacing B11 with C47, which has a higher effort value than B11 but also potentially higher importance, and removing C26 to make room for C47 while staying within the limit'\}} \\
\texttt{PROPOSAL: \{'A19', 'A30', 'A56', 'C47', 'C75'\}} \\
            }
        }
    }
    & \\ \\

    \theutterance \stepcounter{utterance}  
    & \multicolumn{4}{p{0.6\linewidth}}{
        \cellcolor[rgb]{0.9,0.9,0.9}{
            \makecell[{{p{\linewidth}}}]{
                \texttt{\tiny{[P1$\rangle$GM]}}
                \texttt{ARGUMENT: \{'I like the idea of replacing B11 with C47, as it could bring more importance, and removing C26 is a good compromise to stay within the limit, however, I'm concerned that the total effort of this proposal is 3193, which is very close to the limit, and I think we should consider one more adjustment to ensure we're maximizing importance while staying within the limit, perhaps we could remove C75 and add B91, which has a relatively low effort value and could bring additional importance'\}} \\
\texttt{PROPOSAL: \{'A19', 'A30', 'A56', 'C47', 'B91'\}} \\
            }
        }
    }
    & & \\ \\

    \theutterance \stepcounter{utterance}  
    & & \multicolumn{4}{p{0.6\linewidth}}{
        \cellcolor[rgb]{0.9,0.9,0.9}{
            \makecell[{{p{\linewidth}}}]{
                \texttt{\tiny{[GM$\rangle$P2]}}
                \texttt{ARGUMENT: \{'I like the idea of replacing B11 with C47, as it could bring more importance, and removing C26 is a good compromise to stay within the limit, however, I'm concerned that the total effort of this proposal is 3193, which is very close to the limit, and I think we should consider one more adjustment to ensure we're maximizing importance while staying within the limit, perhaps we could remove C75 and add B91, which has a relatively low effort value and could bring additional importance'\}} \\
\texttt{PROPOSAL: \{'A19', 'A30', 'A56', 'C47', 'B91'\}} \\
            }
        }
    }
    & \\ \\

    \theutterance \stepcounter{utterance}  
    & & & \multicolumn{4}{p{0.6\linewidth}}{
        \cellcolor[rgb]{0.9,0.9,0.9}{
            \makecell[{{p{\linewidth}}}]{
                \texttt{\tiny{[GM$\langle$P2]}}
                \texttt{ARGUMENT: \{'I agree that removing C75 and adding B91 is a good trade{-}off, as B91 has a relatively low effort value and could bring significant importance, and I think this proposal strikes a good balance between importance and effort, the total effort of 3071 is well within the limit, and I believe this set of items maximizes our overall importance'\}} \\
\texttt{AGREE: \{'A19', 'A30', 'A56', 'C47', 'B91'\}} \\
            }
        }
    }
     \\ \\

    \theutterance \stepcounter{utterance}  
    & & & \multicolumn{2}{p{0.3\linewidth}}{
        \cellcolor[rgb]{0.9,0.9,0.9}{
            \makecell[{{p{\linewidth}}}]{
                \texttt{\tiny{[GM$|$GM]}}
                \texttt{game successful} \\
            }
        }
    }
    & & \\ \\

    \theutterance \stepcounter{utterance}  
    & & & \multicolumn{2}{p{0.3\linewidth}}{
        \cellcolor[rgb]{0.9,0.9,0.9}{
            \makecell[{{p{\linewidth}}}]{
                \texttt{\tiny{[GM$|$GM]}}
                \texttt{end game} \\
            }
        }
    }
    & & \\ \\

\end{supertabular}
}

\end{document}
