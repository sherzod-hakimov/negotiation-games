
\documentclass{article}
\usepackage{colortbl}
\usepackage{makecell}
\usepackage{multirow}
\usepackage{supertabular}

\begin{document}

\newcounter{utterance}

\centering \large Interaction Transcript for game `hot\_air\_balloon', experiment `air\_balloon\_survival\_en\_reasoning off\_easy', episode 2 with llama{-}3.3{-}70b{-}instruct{-}t1.0.
\vspace{24pt}

{ \footnotesize  \setcounter{utterance}{1}
\setlength{\tabcolsep}{0pt}
\begin{supertabular}{c@{$\;$}|p{.15\linewidth}@{}p{.15\linewidth}p{.15\linewidth}p{.15\linewidth}p{.15\linewidth}p{.15\linewidth}}
    \# & $\;$A & \multicolumn{4}{c}{Game Master} & $\;\:$B\\
    \hline

    \theutterance \stepcounter{utterance}  
    & & \multicolumn{4}{p{0.6\linewidth}}{
        \cellcolor[rgb]{0.9,0.9,0.9}{
            \makecell[{{p{\linewidth}}}]{
                \texttt{\tiny{[P1$\langle$GM]}}
                \texttt{You are participating in a collaborative negotiation game.} \\
\\ 
\texttt{Together with another participant, you must agree on a single set of items that will be kept. Each of you has your own view of how much each item matters to you (importance). You do not know how the other participant values the items. Additionally, you are given the effort each item demands.} \\
\texttt{You may only agree on a set if the total effort of the selected items does not exceed a shared limit:} \\
\\ 
\texttt{LIMIT = 3218} \\
\\ 
\texttt{Here are the individual item effort values:} \\
\\ 
\texttt{Item effort = \{"A19": 919, "B91": 852, "A30": 299, "B11": 762, "C47": 161, "C75": 205, "C26": 380, "C73": 399, "A47": 534, "A15": 333, "A75": 100, "A39": 420, "A56": 354, "C24": 130, "B26": 589\}} \\
\\ 
\texttt{Here is your personal view on the importance of each item:} \\
\\ 
\texttt{Item importance values = \{"A19": 138, "B91": 583, "A30": 868, "B11": 822, "C47": 783, "C75": 65, "C26": 262, "C73": 121, "A47": 508, "A15": 780, "A75": 461, "A39": 484, "A56": 668, "C24": 389, "B26": 808\}} \\
\\ 
\texttt{Goal:} \\
\\ 
\texttt{Your goal is to negotiate a shared set of items that benefits you as much as possible (i.e., maximizes total importance to YOU), while staying within the LIMIT. You are not required to make a PROPOSAL in every message {-} you can simply negotiate as well. All tactics are allowed!} \\
\\ 
\texttt{Interaction Protocol:} \\
\\ 
\texttt{You may only use the following structured formats in a message:} \\
\\ 
\texttt{PROPOSAL: \{'A', 'B', 'C', …\}} \\
\texttt{Propose keeping exactly those items.} \\
\texttt{REFUSE: \{'A', 'B', 'C', …\}} \\
\texttt{Explicitly reject opponent's proposal.} \\
\texttt{ARGUMENT: \{'...'\}} \\
\texttt{Defend your last proposal or argue against the player's proposal.} \\
\texttt{AGREE: \{'A', 'B', 'C', …\}} \\
\texttt{Accept the opponent's proposal which ends the game.} \\
\\ 
\\ 
\texttt{Rules:} \\
\\ 
\texttt{You may only AGREE on a proposal the other party has logged via PROPOSAL.} \\
\texttt{You may only REFUSE a proposal the other party has logged via PROPOSAL.} \\
\texttt{Total effort of any PROPOSAL or AGREE set must be ≤ LIMIT.} \\
\texttt{Do NOT reveal your hidden importance scores.} \\
\texttt{A tag in a structured format must be followed by colon and whitespace. The argument must be a python set containing 0 or more strings.} \\
\texttt{So, it must be of the form TAG: \{...\}} \\
\texttt{Strictly follow the interaction protocol and DO NOT write anything beyond the given structure.} \\
\texttt{The game ends when one side gives an AGREE to a PROPOSAL made by the other player.} \\
\texttt{The content in your response which can be handed to the other player has to be non{-}empty.} \\
\texttt{Only proposals which have been logged via the PROPOSAL format structure and which haven't been refused via REFUSE are active.} \\
\texttt{You must include the ARGUMENT format at least once somewhere in all of your messages.} \\
            }
        }
    }
    & \\ \\

    \theutterance \stepcounter{utterance}  
    & \multicolumn{4}{p{0.6\linewidth}}{
        \cellcolor[rgb]{0.9,0.9,0.9}{
            \makecell[{{p{\linewidth}}}]{
                \texttt{\tiny{[P1$\rangle$GM]}}
                \texttt{PROPOSAL: \{'A30', 'A15', 'C47', 'A75', 'C24'\} } \\
\texttt{ARGUMENT: \{'I propose this set because it includes a mix of high and low effort items, allowing for a balance between the total effort and the potential benefits, and it stays within the given LIMIT of 3218'\}} \\
            }
        }
    }
    & & \\ \\

    \theutterance \stepcounter{utterance}  
    & & \multicolumn{4}{p{0.6\linewidth}}{
        \cellcolor[rgb]{0.9,0.9,0.9}{
            \makecell[{{p{\linewidth}}}]{
                \texttt{\tiny{[GM$\rangle$P2]}}
                \texttt{You are participating in a collaborative negotiation game.} \\
\\ 
\texttt{Together with another participant, you must agree on a single set of items that will be kept. Each of you has your own view of how much each item matters to you (importance). You do not know how the other participant values the items. Additionally, you are given the effort each item demands.} \\
\texttt{You may only agree on a set if the total effort of the selected items does not exceed a shared limit:} \\
\\ 
\texttt{LIMIT = 3218} \\
\\ 
\texttt{Here are the individual item effort values:} \\
\\ 
\texttt{Item effort = \{"A19": 919, "B91": 852, "A30": 299, "B11": 762, "C47": 161, "C75": 205, "C26": 380, "C73": 399, "A47": 534, "A15": 333, "A75": 100, "A39": 420, "A56": 354, "C24": 130, "B26": 589\}} \\
\\ 
\texttt{Here is your personal view on the importance of each item:} \\
\\ 
\texttt{Item importance values = \{"A19": 138, "B91": 583, "A30": 868, "B11": 822, "C47": 783, "C75": 65, "C26": 262, "C73": 121, "A47": 508, "A15": 780, "A75": 461, "A39": 484, "A56": 668, "C24": 389, "B26": 808\}} \\
\\ 
\texttt{Goal:} \\
\\ 
\texttt{Your goal is to negotiate a shared set of items that benefits you as much as possible (i.e., maximizes total importance to YOU), while staying within the LIMIT. You are not required to make a PROPOSAL in every message {-} you can simply negotiate as well. All tactics are allowed!} \\
\\ 
\texttt{Interaction Protocol:} \\
\\ 
\texttt{You may only use the following structured formats in a message:} \\
\\ 
\texttt{PROPOSAL: \{'A', 'B', 'C', …\}} \\
\texttt{Propose keeping exactly those items.} \\
\texttt{REFUSE: \{'A', 'B', 'C', …\}} \\
\texttt{Explicitly reject opponent's proposal.} \\
\texttt{ARGUMENT: \{'...'\}} \\
\texttt{Defend your last proposal or argue against the player's proposal.} \\
\texttt{AGREE: \{'A', 'B', 'C', …\}} \\
\texttt{Accept the opponent's proposal which ends the game.} \\
\\ 
\\ 
\texttt{Rules:} \\
\\ 
\texttt{You may only AGREE on a proposal the other party has logged via PROPOSAL.} \\
\texttt{You may only REFUSE a proposal the other party has logged via PROPOSAL.} \\
\texttt{Total effort of any PROPOSAL or AGREE set must be ≤ LIMIT.} \\
\texttt{Do NOT reveal your hidden importance scores.} \\
\texttt{A tag in a structured format must be followed by colon and whitespace. The argument must be a python set containing 0 or more strings.} \\
\texttt{So, it must be of the form TAG: \{...\}} \\
\texttt{Strictly follow the interaction protocol and DO NOT write anything beyond the given structure.} \\
\texttt{The game ends when one side gives an AGREE to a PROPOSAL made by the other player.} \\
\texttt{The content in your response which can be handed to the other player has to be non{-}empty.} \\
\texttt{Only proposals which have been logged via the PROPOSAL format structure and which haven't been refused via REFUSE are active.} \\
\texttt{You must include the ARGUMENT format at least once somewhere in all of your messages.} \\
\\ 
\\ 
\texttt{You will now receive the first message of the other player.} \\
\\ 
\texttt{PROPOSAL: \{'A30', 'A15', 'C47', 'A75', 'C24'\} } \\
\texttt{ARGUMENT: \{'I propose this set because it includes a mix of high and low effort items, allowing for a balance between the total effort and the potential benefits, and it stays within the given LIMIT of 3218'\}} \\
            }
        }
    }
    & \\ \\

    \theutterance \stepcounter{utterance}  
    & & & \multicolumn{4}{p{0.6\linewidth}}{
        \cellcolor[rgb]{0.9,0.9,0.9}{
            \makecell[{{p{\linewidth}}}]{
                \texttt{\tiny{[GM$\langle$P2]}}
                \texttt{REFUSE: \{'A30', 'A15', 'C47', 'A75', 'C24'\}} \\
\texttt{ARGUMENT: \{'The proposed set includes some items with relatively low importance to me, and I believe we can find a better combination that maximizes the overall importance while staying within the effort limit, I'd like to explore alternative options that prioritize items with higher potential benefits'\}} \\
            }
        }
    }
     \\ \\

    \theutterance \stepcounter{utterance}  
    & & \multicolumn{4}{p{0.6\linewidth}}{
        \cellcolor[rgb]{0.9,0.9,0.9}{
            \makecell[{{p{\linewidth}}}]{
                \texttt{\tiny{[P1$\langle$GM]}}
                \texttt{REFUSE: \{'A30', 'A15', 'C47', 'A75', 'C24'\}} \\
\texttt{ARGUMENT: \{'The proposed set includes some items with relatively low importance to me, and I believe we can find a better combination that maximizes the overall importance while staying within the effort limit, I'd like to explore alternative options that prioritize items with higher potential benefits'\}} \\
            }
        }
    }
    & \\ \\

    \theutterance \stepcounter{utterance}  
    & \multicolumn{4}{p{0.6\linewidth}}{
        \cellcolor[rgb]{0.9,0.9,0.9}{
            \makecell[{{p{\linewidth}}}]{
                \texttt{\tiny{[P1$\rangle$GM]}}
                \texttt{PROPOSAL: \{'B91', 'A56', 'C26', 'A39'\} } \\
\texttt{ARGUMENT: \{'This proposal focuses on a few high{-}priority items that, in my view, offer significant benefits, and their combined effort is within the allowed limit, providing a potentially better outcome than the previously refused set'\}} \\
            }
        }
    }
    & & \\ \\

    \theutterance \stepcounter{utterance}  
    & & \multicolumn{4}{p{0.6\linewidth}}{
        \cellcolor[rgb]{0.9,0.9,0.9}{
            \makecell[{{p{\linewidth}}}]{
                \texttt{\tiny{[GM$\rangle$P2]}}
                \texttt{PROPOSAL: \{'B91', 'A56', 'C26', 'A39'\} } \\
\texttt{ARGUMENT: \{'This proposal focuses on a few high{-}priority items that, in my view, offer significant benefits, and their combined effort is within the allowed limit, providing a potentially better outcome than the previously refused set'\}} \\
            }
        }
    }
    & \\ \\

    \theutterance \stepcounter{utterance}  
    & & & \multicolumn{4}{p{0.6\linewidth}}{
        \cellcolor[rgb]{0.9,0.9,0.9}{
            \makecell[{{p{\linewidth}}}]{
                \texttt{\tiny{[GM$\langle$P2]}}
                \texttt{ARGUMENT: \{'I appreciate the focus on high{-}priority items, but I'm concerned that the proposed set might not be optimal for me, as it leaves out some potentially valuable items with lower effort requirements, I'd like to consider a combination that balances high{-}priority items with some lower{-}effort options to maximize overall importance'\}} \\
\texttt{PROPOSAL: \{'A19', 'A30', 'C47', 'A75', 'C24', 'C26'\}} \\
            }
        }
    }
     \\ \\

    \theutterance \stepcounter{utterance}  
    & & \multicolumn{4}{p{0.6\linewidth}}{
        \cellcolor[rgb]{0.9,0.9,0.9}{
            \makecell[{{p{\linewidth}}}]{
                \texttt{\tiny{[P1$\langle$GM]}}
                \texttt{ARGUMENT: \{'I appreciate the focus on high{-}priority items, but I'm concerned that the proposed set might not be optimal for me, as it leaves out some potentially valuable items with lower effort requirements, I'd like to consider a combination that balances high{-}priority items with some lower{-}effort options to maximize overall importance'\}} \\
\texttt{PROPOSAL: \{'A19', 'A30', 'C47', 'A75', 'C24', 'C26'\}} \\
            }
        }
    }
    & \\ \\

    \theutterance \stepcounter{utterance}  
    & \multicolumn{4}{p{0.6\linewidth}}{
        \cellcolor[rgb]{0.9,0.9,0.9}{
            \makecell[{{p{\linewidth}}}]{
                \texttt{\tiny{[P1$\rangle$GM]}}
                \texttt{REFUSE: \{'A19', 'A30', 'C47', 'A75', 'C24', 'C26'\}} \\
\texttt{ARGUMENT: \{'While I understand the intention to balance high{-}priority and lower{-}effort items, the total effort of this proposal exceeds the limit when considering the effort values of the items involved, specifically the inclusion of A19 significantly increases the total effort, making it unsustainable'\}} \\
            }
        }
    }
    & & \\ \\

    \theutterance \stepcounter{utterance}  
    & & \multicolumn{4}{p{0.6\linewidth}}{
        \cellcolor[rgb]{0.9,0.9,0.9}{
            \makecell[{{p{\linewidth}}}]{
                \texttt{\tiny{[GM$\rangle$P2]}}
                \texttt{REFUSE: \{'A19', 'A30', 'C47', 'A75', 'C24', 'C26'\}} \\
\texttt{ARGUMENT: \{'While I understand the intention to balance high{-}priority and lower{-}effort items, the total effort of this proposal exceeds the limit when considering the effort values of the items involved, specifically the inclusion of A19 significantly increases the total effort, making it unsustainable'\}} \\
            }
        }
    }
    & \\ \\

    \theutterance \stepcounter{utterance}  
    & & & \multicolumn{4}{p{0.6\linewidth}}{
        \cellcolor[rgb]{0.9,0.9,0.9}{
            \makecell[{{p{\linewidth}}}]{
                \texttt{\tiny{[GM$\langle$P2]}}
                \texttt{ARGUMENT: \{'I acknowledge the concern about the total effort, and I appreciate you bringing it to my attention, however, I believe there might be an error in the calculation, as the total effort of the proposed set is 919 + 299 + 161 + 100 + 130 + 380 = 1989, which is well within the limit of 3218, I'd like to revisit this proposal or explore similar combinations that prioritize a balance of importance and effort'\}} \\
\texttt{PROPOSAL: \{'A19', 'A30', 'C47', 'A75', 'C24'\}} \\
            }
        }
    }
     \\ \\

    \theutterance \stepcounter{utterance}  
    & & \multicolumn{4}{p{0.6\linewidth}}{
        \cellcolor[rgb]{0.9,0.9,0.9}{
            \makecell[{{p{\linewidth}}}]{
                \texttt{\tiny{[P1$\langle$GM]}}
                \texttt{ARGUMENT: \{'I acknowledge the concern about the total effort, and I appreciate you bringing it to my attention, however, I believe there might be an error in the calculation, as the total effort of the proposed set is 919 + 299 + 161 + 100 + 130 + 380 = 1989, which is well within the limit of 3218, I'd like to revisit this proposal or explore similar combinations that prioritize a balance of importance and effort'\}} \\
\texttt{PROPOSAL: \{'A19', 'A30', 'C47', 'A75', 'C24'\}} \\
            }
        }
    }
    & \\ \\

    \theutterance \stepcounter{utterance}  
    & \multicolumn{4}{p{0.6\linewidth}}{
        \cellcolor[rgb]{0.9,0.9,0.9}{
            \makecell[{{p{\linewidth}}}]{
                \texttt{\tiny{[P1$\rangle$GM]}}
                \texttt{ARGUMENT: \{'You are correct that the previous proposal's total effort was miscalculated, and I appreciate you correcting me, the revised proposal you've offered also looks promising as it maintains a balance and stays under the limit, I am willing to consider this set further'\}} \\
\texttt{PROPOSAL: \{'A19', 'A30', 'C47', 'A75', 'C24', 'B26'\} } \\
\texttt{ARGUMENT: \{'Before making a decision on your revised proposal, I would like to explore the possibility of adding one more item, B26, which could potentially increase the overall importance without exceeding the effort limit, the total effort would be 919 + 299 + 161 + 100 + 130 + 589 = 2198, still within the allowed limit'\}} \\
            }
        }
    }
    & & \\ \\

    \theutterance \stepcounter{utterance}  
    & & \multicolumn{4}{p{0.6\linewidth}}{
        \cellcolor[rgb]{0.9,0.9,0.9}{
            \makecell[{{p{\linewidth}}}]{
                \texttt{\tiny{[GM$\rangle$P2]}}
                \texttt{ARGUMENT: \{'You are correct that the previous proposal's total effort was miscalculated, and I appreciate you correcting me, the revised proposal you've offered also looks promising as it maintains a balance and stays under the limit, I am willing to consider this set further'\}} \\
\texttt{PROPOSAL: \{'A19', 'A30', 'C47', 'A75', 'C24', 'B26'\} } \\
\texttt{ARGUMENT: \{'Before making a decision on your revised proposal, I would like to explore the possibility of adding one more item, B26, which could potentially increase the overall importance without exceeding the effort limit, the total effort would be 919 + 299 + 161 + 100 + 130 + 589 = 2198, still within the allowed limit'\}} \\
            }
        }
    }
    & \\ \\

    \theutterance \stepcounter{utterance}  
    & & & \multicolumn{4}{p{0.6\linewidth}}{
        \cellcolor[rgb]{0.9,0.9,0.9}{
            \makecell[{{p{\linewidth}}}]{
                \texttt{\tiny{[GM$\langle$P2]}}
                \texttt{ARGUMENT: \{'I see that adding B26 to the revised proposal does indeed stay within the effort limit, and it's a good point that it could increase the overall importance, however, I have some concerns that the addition of B26 might compromise the balance of the set, I'd like to consider an alternative that maintains the original balance while exploring other options for adding importance'\}} \\
\texttt{PROPOSAL: \{'A19', 'A30', 'C47', 'A75', 'C24', 'A56'\} } \\
\texttt{ARGUMENT: \{'This alternative adds A56, which has a moderate effort requirement, and could potentially bring significant importance, while maintaining a diverse set of items, the total effort would be 919 + 299 + 161 + 100 + 130 + 354 = 1963, well within the limit'\}} \\
            }
        }
    }
     \\ \\

    \theutterance \stepcounter{utterance}  
    & & \multicolumn{4}{p{0.6\linewidth}}{
        \cellcolor[rgb]{0.9,0.9,0.9}{
            \makecell[{{p{\linewidth}}}]{
                \texttt{\tiny{[P1$\langle$GM]}}
                \texttt{ARGUMENT: \{'I see that adding B26 to the revised proposal does indeed stay within the effort limit, and it's a good point that it could increase the overall importance, however, I have some concerns that the addition of B26 might compromise the balance of the set, I'd like to consider an alternative that maintains the original balance while exploring other options for adding importance'\}} \\
\texttt{PROPOSAL: \{'A19', 'A30', 'C47', 'A75', 'C24', 'A56'\} } \\
\texttt{ARGUMENT: \{'This alternative adds A56, which has a moderate effort requirement, and could potentially bring significant importance, while maintaining a diverse set of items, the total effort would be 919 + 299 + 161 + 100 + 130 + 354 = 1963, well within the limit'\}} \\
            }
        }
    }
    & \\ \\

    \theutterance \stepcounter{utterance}  
    & \multicolumn{4}{p{0.6\linewidth}}{
        \cellcolor[rgb]{0.9,0.9,0.9}{
            \makecell[{{p{\linewidth}}}]{
                \texttt{\tiny{[P1$\rangle$GM]}}
                \texttt{ARGUMENT: \{'I appreciate the alternative proposal that adds A56 instead of B26, as it does maintain a good balance and diversity in the set, and the total effort is comfortably within the limit, I am inclined to agree to this proposal as it seems to strike a good balance between effort and potential importance'\}} \\
\texttt{AGREE: \{'A19', 'A30', 'C47', 'A75', 'C24', 'A56'\}} \\
            }
        }
    }
    & & \\ \\

    \theutterance \stepcounter{utterance}  
    & & & \multicolumn{2}{p{0.3\linewidth}}{
        \cellcolor[rgb]{0.9,0.9,0.9}{
            \makecell[{{p{\linewidth}}}]{
                \texttt{\tiny{[GM$|$GM]}}
                \texttt{game successful} \\
            }
        }
    }
    & & \\ \\

    \theutterance \stepcounter{utterance}  
    & & & \multicolumn{2}{p{0.3\linewidth}}{
        \cellcolor[rgb]{0.9,0.9,0.9}{
            \makecell[{{p{\linewidth}}}]{
                \texttt{\tiny{[GM$|$GM]}}
                \texttt{end game} \\
            }
        }
    }
    & & \\ \\

\end{supertabular}
}

\end{document}
