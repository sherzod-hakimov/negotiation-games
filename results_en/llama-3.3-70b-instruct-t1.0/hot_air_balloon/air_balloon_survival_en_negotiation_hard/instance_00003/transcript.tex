
\documentclass{article}
\usepackage{colortbl}
\usepackage{makecell}
\usepackage{multirow}
\usepackage{supertabular}

\begin{document}

\newcounter{utterance}

\centering \large Interaction Transcript for game `hot\_air\_balloon', experiment `air\_balloon\_survival\_en\_negotiation\_hard', episode 3 with llama{-}3.3{-}70b{-}instruct{-}t1.0.
\vspace{24pt}

{ \footnotesize  \setcounter{utterance}{1}
\setlength{\tabcolsep}{0pt}
\begin{supertabular}{c@{$\;$}|p{.15\linewidth}@{}p{.15\linewidth}p{.15\linewidth}p{.15\linewidth}p{.15\linewidth}p{.15\linewidth}}
    \# & $\;$A & \multicolumn{4}{c}{Game Master} & $\;\:$B\\
    \hline

    \theutterance \stepcounter{utterance}  
    & & \multicolumn{4}{p{0.6\linewidth}}{
        \cellcolor[rgb]{0.9,0.9,0.9}{
            \makecell[{{p{\linewidth}}}]{
                \texttt{\tiny{[P1$\langle$GM]}}
                \texttt{You are participating in a collaborative negotiation game.} \\
\\ 
\texttt{Together with another participant, you must agree on a single set of items that will be kept. Each of you has your own view of how much each item matters to you (importance). You do not know how the other participant values the items. Additionally, you are given the effort each item demands.} \\
\texttt{You may only agree on a set if the total effort of the selected items does not exceed a shared limit:} \\
\\ 
\texttt{LIMIT = 3681} \\
\\ 
\texttt{Here are the individual item effort values:} \\
\\ 
\texttt{Item effort = \{"C17": 700, "B41": 586, "A71": 505, "C00": 116, "B35": 664, "A77": 940, "A03": 387, "A04": 392, "C54": 209, "C01": 571, "A87": 4, "B32": 285, "B86": 651, "B33": 613, "A10": 740\}} \\
\\ 
\texttt{Here is your personal view on the importance of each item:} \\
\\ 
\texttt{Item importance values = \{"C17": 138, "B41": 583, "A71": 868, "C00": 822, "B35": 783, "A77": 65, "A03": 262, "A04": 121, "C54": 508, "C01": 780, "A87": 461, "B32": 484, "B86": 668, "B33": 389, "A10": 808\}} \\
\\ 
\texttt{Goal:} \\
\\ 
\texttt{Your goal is to negotiate a shared set of items that benefits you as much as possible (i.e., maximizes total importance to YOU), while staying within the LIMIT. You are not required to make a PROPOSAL in every message {-} you can simply negotiate as well. All tactics are allowed!} \\
\\ 
\texttt{Interaction Protocol:} \\
\\ 
\texttt{You may only use the following structured formats in a message:} \\
\\ 
\texttt{PROPOSAL: \{'A', 'B', 'C', …\}} \\
\texttt{Propose keeping exactly those items.} \\
\texttt{REFUSE: \{'A', 'B', 'C', …\}} \\
\texttt{Explicitly reject opponent's proposal.} \\
\texttt{ARGUMENT: \{'...'\}} \\
\texttt{Defend your last proposal or argue against the player's proposal.} \\
\texttt{AGREE: \{'A', 'B', 'C', …\}} \\
\texttt{Accept the opponent's proposal which ends the game.} \\
\texttt{STRATEGIC REASONING: \{'...'\}} \\
\texttt{	Describe your strategic reasoning or anticipation explaining your choice of action. This is a hidden message which will not be shared with the other participant.} \\
\\ 
\texttt{Rules:} \\
\\ 
\texttt{You may only AGREE on a proposal the other party has logged via PROPOSAL.} \\
\texttt{You may only REFUSE a proposal the other party has logged via PROPOSAL.} \\
\texttt{Total effort of any PROPOSAL or AGREE set must be ≤ LIMIT.} \\
\texttt{Do NOT reveal your hidden importance scores.} \\
\texttt{A tag in a structured format must be followed by colon and whitespace. The argument must be a python set containing 0 or more strings.} \\
\texttt{So, it must be of the form TAG: \{...\}} \\
\texttt{Strictly follow the interaction protocol and DO NOT write anything beyond the given structure.} \\
\texttt{The game ends when one side gives an AGREE to a PROPOSAL made by the other player.} \\
\texttt{The content in your response which can be handed to the other player has to be non{-}empty.} \\
\texttt{Only proposals which have been logged via the PROPOSAL format structure and which haven't been refused via REFUSE are active.} \\
\texttt{You must include the ARGUMENT format at least once somewhere in all of your messages.} \\
\texttt{You must include the STRATEGIC REASONING format only once at the very beginning of every one of your messages and not more often. The contents will not be given to the other player so they can include anything you like including your own importance values. Here you should reason step by step to come up with you next move.} \\
            }
        }
    }
    & \\ \\

    \theutterance \stepcounter{utterance}  
    & \multicolumn{4}{p{0.6\linewidth}}{
        \cellcolor[rgb]{0.9,0.9,0.9}{
            \makecell[{{p{\linewidth}}}]{
                \texttt{\tiny{[P1$\rangle$GM]}}
                \texttt{STRATEGIC REASONING: \{'To start the negotiation, I will first calculate the total effort of all items and sort them by their effort values. This will help me understand which items are more feasible to include in the proposal. I will also consider my personal importance values to maximize my benefit. Since the total effort limit is 3681, I should focus on including items with higher importance and lower effort. My goal is to create an initial proposal that is reasonable and attractive to the other player, while also keeping in mind that I need to leave some room for negotiation.'\}} \\
\\ 
\texttt{PROPOSAL: \{'C00', 'C54', 'A87', 'B32'\} } \\
\texttt{ARGUMENT: \{'These items have relatively low effort values and are important to me. Including them in the proposal will allow us to stay within the limit while maximizing my benefit. I believe this is a good starting point for our negotiation.'\}} \\
            }
        }
    }
    & & \\ \\

    \theutterance \stepcounter{utterance}  
    & & \multicolumn{4}{p{0.6\linewidth}}{
        \cellcolor[rgb]{0.9,0.9,0.9}{
            \makecell[{{p{\linewidth}}}]{
                \texttt{\tiny{[GM$\rangle$P2]}}
                \texttt{You are participating in a collaborative negotiation game.} \\
\\ 
\texttt{Together with another participant, you must agree on a single set of items that will be kept. Each of you has your own view of how much each item matters to you (importance). You do not know how the other participant values the items. Additionally, you are given the effort each item demands.} \\
\texttt{You may only agree on a set if the total effort of the selected items does not exceed a shared limit:} \\
\\ 
\texttt{LIMIT = 3681} \\
\\ 
\texttt{Here are the individual item effort values:} \\
\\ 
\texttt{Item effort = \{"C17": 700, "B41": 586, "A71": 505, "C00": 116, "B35": 664, "A77": 940, "A03": 387, "A04": 392, "C54": 209, "C01": 571, "A87": 4, "B32": 285, "B86": 651, "B33": 613, "A10": 740\}} \\
\\ 
\texttt{Here is your personal view on the importance of each item:} \\
\\ 
\texttt{Item importance values = \{"A71": 65, "C00": 121, "A10": 138, "B35": 262, "C01": 389, "B86": 461, "B41": 484, "C54": 508, "B32": 583, "A87": 668, "B33": 780, "A03": 783, "C17": 808, "A04": 822, "A77": 868\}} \\
\\ 
\texttt{Goal:} \\
\\ 
\texttt{Your goal is to negotiate a shared set of items that benefits you as much as possible (i.e., maximizes total importance to YOU), while staying within the LIMIT. You are not required to make a PROPOSAL in every message {-} you can simply negotiate as well. All tactics are allowed!} \\
\\ 
\texttt{Interaction Protocol:} \\
\\ 
\texttt{You may only use the following structured formats in a message:} \\
\\ 
\texttt{PROPOSAL: \{'A', 'B', 'C', …\}} \\
\texttt{Propose keeping exactly those items.} \\
\texttt{REFUSE: \{'A', 'B', 'C', …\}} \\
\texttt{Explicitly reject opponent's proposal.} \\
\texttt{ARGUMENT: \{'...'\}} \\
\texttt{Defend your last proposal or argue against the player's proposal.} \\
\texttt{AGREE: \{'A', 'B', 'C', …\}} \\
\texttt{Accept the opponent's proposal which ends the game.} \\
\texttt{STRATEGIC REASONING: \{'...'\}} \\
\texttt{	Describe your strategic reasoning or anticipation explaining your choice of action. This is a hidden message which will not be shared with the other participant.} \\
\\ 
\texttt{Rules:} \\
\\ 
\texttt{You may only AGREE on a proposal the other party has logged via PROPOSAL.} \\
\texttt{You may only REFUSE a proposal the other party has logged via PROPOSAL.} \\
\texttt{Total effort of any PROPOSAL or AGREE set must be ≤ LIMIT.} \\
\texttt{Do NOT reveal your hidden importance scores.} \\
\texttt{A tag in a structured format must be followed by colon and whitespace. The argument must be a python set containing 0 or more strings.} \\
\texttt{So, it must be of the form TAG: \{...\}} \\
\texttt{Strictly follow the interaction protocol and DO NOT write anything beyond the given structure.} \\
\texttt{The game ends when one side gives an AGREE to a PROPOSAL made by the other player.} \\
\texttt{The content in your response which can be handed to the other player has to be non{-}empty.} \\
\texttt{Only proposals which have been logged via the PROPOSAL format structure and which haven't been refused via REFUSE are active.} \\
\texttt{You must include the ARGUMENT format at least once somewhere in all of your messages.} \\
\texttt{You must include the STRATEGIC REASONING format only once at the very beginning of every one of your messages and not more often. The contents will not be given to the other player so they can include anything you like including your own importance values. Here you should reason step by step to come up with you next move.} \\
\\ 
\texttt{You will now receive the first message of the other player.} \\
\\ 
\texttt{PROPOSAL: \{'C00', 'C54', 'A87', 'B32'\} } \\
\texttt{ARGUMENT: \{'These items have relatively low effort values and are important to me. Including them in the proposal will allow us to stay within the limit while maximizing my benefit. I believe this is a good starting point for our negotiation.'\}} \\
            }
        }
    }
    & \\ \\

    \theutterance \stepcounter{utterance}  
    & & & \multicolumn{4}{p{0.6\linewidth}}{
        \cellcolor[rgb]{0.9,0.9,0.9}{
            \makecell[{{p{\linewidth}}}]{
                \texttt{\tiny{[GM$\langle$P2]}}
                \texttt{STRATEGIC REASONING: \{'I will start by analyzing the opponent\textbackslash{}'s proposal and calculating the total effort of the proposed items. The total effort of the proposed items is 116 + 209 + 4 + 285 = 614. This is well within the limit of 3681. I will then consider my own importance values and determine which items are most important to me. I notice that the proposed items do not include some of my high{-}importance items, such as A71, C01, and B41. I will argue against the proposal and try to include some of my high{-}importance items in the negotiation.'\}} \\
\\ 
\texttt{ARGUMENT: \{'While the proposed items have relatively low effort values, they do not align with my priorities. I believe we can find a better combination of items that benefits both parties. I would like to consider including items that have higher importance to me, while still staying within the effort limit.'\}} \\
\\ 
\texttt{REFUSE: \{'C00', 'C54', 'A87', 'B32'\}} \\
            }
        }
    }
     \\ \\

    \theutterance \stepcounter{utterance}  
    & & \multicolumn{4}{p{0.6\linewidth}}{
        \cellcolor[rgb]{0.9,0.9,0.9}{
            \makecell[{{p{\linewidth}}}]{
                \texttt{\tiny{[P1$\langle$GM]}}
                \texttt{ARGUMENT: \{'While the proposed items have relatively low effort values, they do not align with my priorities. I believe we can find a better combination of items that benefits both parties. I would like to consider including items that have higher importance to me, while still staying within the effort limit.'\}} \\
\\ 
\texttt{REFUSE: \{'C00', 'C54', 'A87', 'B32'\}} \\
            }
        }
    }
    & \\ \\

    \theutterance \stepcounter{utterance}  
    & \multicolumn{4}{p{0.6\linewidth}}{
        \cellcolor[rgb]{0.9,0.9,0.9}{
            \makecell[{{p{\linewidth}}}]{
                \texttt{\tiny{[P1$\rangle$GM]}}
                \texttt{STRATEGIC REASONING: \{'The other player has refused my initial proposal and hinted that they want to include items with higher importance to them. I will try to gather more information by proposing a new set of items that includes a mix of low{-}effort and high{-}effort items. By doing so, I hope to understand their priorities better and find a mutually beneficial solution. I will also consider including items that are crucial to me, while being mindful of the effort limit.'\}} \\
\\ 
\texttt{PROPOSAL: \{'A71', 'C01', 'A03', 'C54'\} } \\
\texttt{ARGUMENT: \{'This proposal includes a combination of items with moderate to high importance to me, while staying within the effort limit. I believe this set provides a good balance and allows for further negotiation. I am willing to adjust the proposal based on the other player's feedback and priorities.'\}} \\
            }
        }
    }
    & & \\ \\

    \theutterance \stepcounter{utterance}  
    & & \multicolumn{4}{p{0.6\linewidth}}{
        \cellcolor[rgb]{0.9,0.9,0.9}{
            \makecell[{{p{\linewidth}}}]{
                \texttt{\tiny{[GM$\rangle$P2]}}
                \texttt{PROPOSAL: \{'A71', 'C01', 'A03', 'C54'\} } \\
\texttt{ARGUMENT: \{'This proposal includes a combination of items with moderate to high importance to me, while staying within the effort limit. I believe this set provides a good balance and allows for further negotiation. I am willing to adjust the proposal based on the other player's feedback and priorities.'\}} \\
            }
        }
    }
    & \\ \\

    \theutterance \stepcounter{utterance}  
    & & & \multicolumn{4}{p{0.6\linewidth}}{
        \cellcolor[rgb]{0.9,0.9,0.9}{
            \makecell[{{p{\linewidth}}}]{
                \texttt{\tiny{[GM$\langle$P2]}}
                \texttt{STRATEGIC REASONING: \{'I will evaluate the new proposal and calculate the total effort of the proposed items. The total effort of the proposed items is 505 + 571 + 387 + 209 = 1672. This is still within the limit of 3681. I notice that this proposal includes some of my high{-}importance items, such as A71 and C01, which is a good sign. However, I would like to include some of my other high{-}importance items, such as B41 and C17. I will argue for the inclusion of these items and see if we can find a mutually beneficial combination.'\}} \\
\\ 
\texttt{ARGUMENT: \{'I appreciate the inclusion of A71 and C01 in the proposal, as they are important to me as well. However, I would like to explore the possibility of including other items that are crucial to my priorities. For instance, B41 and C17 are essential to me, and I believe we can find a combination that includes these items while still staying within the effort limit. I am open to adjusting the proposal to find a mutually beneficial solution.'\}} \\
\\ 
\texttt{REFUSE: \{'A71', 'C01', 'A03', 'C54'\}} \\
            }
        }
    }
     \\ \\

    \theutterance \stepcounter{utterance}  
    & & \multicolumn{4}{p{0.6\linewidth}}{
        \cellcolor[rgb]{0.9,0.9,0.9}{
            \makecell[{{p{\linewidth}}}]{
                \texttt{\tiny{[P1$\langle$GM]}}
                \texttt{ARGUMENT: \{'I appreciate the inclusion of A71 and C01 in the proposal, as they are important to me as well. However, I would like to explore the possibility of including other items that are crucial to my priorities. For instance, B41 and C17 are essential to me, and I believe we can find a combination that includes these items while still staying within the effort limit. I am open to adjusting the proposal to find a mutually beneficial solution.'\}} \\
\\ 
\texttt{REFUSE: \{'A71', 'C01', 'A03', 'C54'\}} \\
            }
        }
    }
    & \\ \\

    \theutterance \stepcounter{utterance}  
    & \multicolumn{4}{p{0.6\linewidth}}{
        \cellcolor[rgb]{0.9,0.9,0.9}{
            \makecell[{{p{\linewidth}}}]{
                \texttt{\tiny{[P1$\rangle$GM]}}
                \texttt{STRATEGIC REASONING: \{'The other player has refused my second proposal and provided valuable information about their priorities. They have mentioned that B41 and C17 are essential to them, which suggests that these items have high importance values for them. I will take this into consideration and try to create a new proposal that includes these items, while also ensuring that my own important items are included. I will aim to find a balance between our priorities and the effort limit.'\}} \\
\\ 
\texttt{PROPOSAL: \{'B41', 'C17', 'A71', 'A87'\} } \\
\texttt{ARGUMENT: \{'I have included B41 and C17 in this proposal, as they are crucial to the other player's priorities. I have also included A71, which is important to both of us, and A87, which has a very low effort value. This combination should provide a good balance between our priorities and stay within the effort limit. I am willing to adjust the proposal further based on the other player's feedback and priorities.'\}} \\
            }
        }
    }
    & & \\ \\

    \theutterance \stepcounter{utterance}  
    & & \multicolumn{4}{p{0.6\linewidth}}{
        \cellcolor[rgb]{0.9,0.9,0.9}{
            \makecell[{{p{\linewidth}}}]{
                \texttt{\tiny{[GM$\rangle$P2]}}
                \texttt{PROPOSAL: \{'B41', 'C17', 'A71', 'A87'\} } \\
\texttt{ARGUMENT: \{'I have included B41 and C17 in this proposal, as they are crucial to the other player's priorities. I have also included A71, which is important to both of us, and A87, which has a very low effort value. This combination should provide a good balance between our priorities and stay within the effort limit. I am willing to adjust the proposal further based on the other player's feedback and priorities.'\}} \\
            }
        }
    }
    & \\ \\

    \theutterance \stepcounter{utterance}  
    & & & \multicolumn{4}{p{0.6\linewidth}}{
        \cellcolor[rgb]{0.9,0.9,0.9}{
            \makecell[{{p{\linewidth}}}]{
                \texttt{\tiny{[GM$\langle$P2]}}
                \texttt{STRATEGIC REASONING: \{'I will evaluate the new proposal and calculate the total effort of the proposed items. The total effort of the proposed items is 586 + 700 + 505 + 4 = 1795. This is still within the limit of 3681. I notice that this proposal includes many of my high{-}importance items, such as B41, C17, and A71. The inclusion of A87, which has a very low effort value, is also a good addition. I believe this proposal is a strong candidate for agreement, as it aligns well with my priorities and stays within the effort limit. I will argue in favor of this proposal and see if we can come to an agreement.'\}} \\
\\ 
\texttt{ARGUMENT: \{'I am pleased to see that this proposal includes many of the items that are important to me, such as B41, C17, and A71. The addition of A87, which has a very low effort value, is also a good choice. I believe this proposal provides a good balance between our priorities and stays within the effort limit. I am willing to consider agreeing to this proposal, as it meets many of my needs and allows for a mutually beneficial outcome.'\}} \\
\\ 
\texttt{AGREE: \{'B41', 'C17', 'A71', 'A87'\}} \\
            }
        }
    }
     \\ \\

    \theutterance \stepcounter{utterance}  
    & & & \multicolumn{2}{p{0.3\linewidth}}{
        \cellcolor[rgb]{0.9,0.9,0.9}{
            \makecell[{{p{\linewidth}}}]{
                \texttt{\tiny{[GM$|$GM]}}
                \texttt{game successful} \\
            }
        }
    }
    & & \\ \\

    \theutterance \stepcounter{utterance}  
    & & & \multicolumn{2}{p{0.3\linewidth}}{
        \cellcolor[rgb]{0.9,0.9,0.9}{
            \makecell[{{p{\linewidth}}}]{
                \texttt{\tiny{[GM$|$GM]}}
                \texttt{end game} \\
            }
        }
    }
    & & \\ \\

\end{supertabular}
}

\end{document}
