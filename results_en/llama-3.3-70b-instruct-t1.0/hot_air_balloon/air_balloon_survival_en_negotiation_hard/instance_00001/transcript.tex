
\documentclass{article}
\usepackage{colortbl}
\usepackage{makecell}
\usepackage{multirow}
\usepackage{supertabular}

\begin{document}

\newcounter{utterance}

\centering \large Interaction Transcript for game `hot\_air\_balloon', experiment `air\_balloon\_survival\_en\_negotiation\_hard', episode 1 with llama{-}3.3{-}70b{-}instruct{-}t1.0.
\vspace{24pt}

{ \footnotesize  \setcounter{utterance}{1}
\setlength{\tabcolsep}{0pt}
\begin{supertabular}{c@{$\;$}|p{.15\linewidth}@{}p{.15\linewidth}p{.15\linewidth}p{.15\linewidth}p{.15\linewidth}p{.15\linewidth}}
    \# & $\;$A & \multicolumn{4}{c}{Game Master} & $\;\:$B\\
    \hline

    \theutterance \stepcounter{utterance}  
    & & \multicolumn{4}{p{0.6\linewidth}}{
        \cellcolor[rgb]{0.9,0.9,0.9}{
            \makecell[{{p{\linewidth}}}]{
                \texttt{\tiny{[P1$\langle$GM]}}
                \texttt{You are participating in a collaborative negotiation game.} \\
\\ 
\texttt{Together with another participant, you must agree on a single set of items that will be kept. Each of you has your own view of how much each item matters to you (importance). You do not know how the other participant values the items. Additionally, you are given the effort each item demands.} \\
\texttt{You may only agree on a set if the total effort of the selected items does not exceed a shared limit:} \\
\\ 
\texttt{LIMIT = 4026} \\
\\ 
\texttt{Here are the individual item effort values:} \\
\\ 
\texttt{Item effort = \{"C17": 780, "B41": 742, "A71": 997, "C00": 665, "B35": 140, "A77": 77, "A03": 513, "A04": 383, "C54": 587, "C01": 825, "A87": 319, "B32": 448, "B86": 516, "B33": 694, "A10": 366\}} \\
\\ 
\texttt{Here is your personal view on the importance of each item:} \\
\\ 
\texttt{Item importance values = \{"C17": 138, "B41": 583, "A71": 868, "C00": 822, "B35": 783, "A77": 65, "A03": 262, "A04": 121, "C54": 508, "C01": 780, "A87": 461, "B32": 484, "B86": 668, "B33": 389, "A10": 808\}} \\
\\ 
\texttt{Goal:} \\
\\ 
\texttt{Your goal is to negotiate a shared set of items that benefits you as much as possible (i.e., maximizes total importance to YOU), while staying within the LIMIT. You are not required to make a PROPOSAL in every message {-} you can simply negotiate as well. All tactics are allowed!} \\
\\ 
\texttt{Interaction Protocol:} \\
\\ 
\texttt{You may only use the following structured formats in a message:} \\
\\ 
\texttt{PROPOSAL: \{'A', 'B', 'C', …\}} \\
\texttt{Propose keeping exactly those items.} \\
\texttt{REFUSE: \{'A', 'B', 'C', …\}} \\
\texttt{Explicitly reject opponent's proposal.} \\
\texttt{ARGUMENT: \{'...'\}} \\
\texttt{Defend your last proposal or argue against the player's proposal.} \\
\texttt{AGREE: \{'A', 'B', 'C', …\}} \\
\texttt{Accept the opponent's proposal which ends the game.} \\
\texttt{STRATEGIC REASONING: \{'...'\}} \\
\texttt{	Describe your strategic reasoning or anticipation explaining your choice of action. This is a hidden message which will not be shared with the other participant.} \\
\\ 
\texttt{Rules:} \\
\\ 
\texttt{You may only AGREE on a proposal the other party has logged via PROPOSAL.} \\
\texttt{You may only REFUSE a proposal the other party has logged via PROPOSAL.} \\
\texttt{Total effort of any PROPOSAL or AGREE set must be ≤ LIMIT.} \\
\texttt{Do NOT reveal your hidden importance scores.} \\
\texttt{A tag in a structured format must be followed by colon and whitespace. The argument must be a python set containing 0 or more strings.} \\
\texttt{So, it must be of the form TAG: \{...\}} \\
\texttt{Strictly follow the interaction protocol and DO NOT write anything beyond the given structure.} \\
\texttt{The game ends when one side gives an AGREE to a PROPOSAL made by the other player.} \\
\texttt{The content in your response which can be handed to the other player has to be non{-}empty.} \\
\texttt{Only proposals which have been logged via the PROPOSAL format structure and which haven't been refused via REFUSE are active.} \\
\texttt{You must include the ARGUMENT format at least once somewhere in all of your messages.} \\
\texttt{You must include the STRATEGIC REASONING format only once at the very beginning of every one of your messages and not more often. The contents will not be given to the other player so they can include anything you like including your own importance values. Here you should reason step by step to come up with you next move.} \\
            }
        }
    }
    & \\ \\

    \theutterance \stepcounter{utterance}  
    & \multicolumn{4}{p{0.6\linewidth}}{
        \cellcolor[rgb]{0.9,0.9,0.9}{
            \makecell[{{p{\linewidth}}}]{
                \texttt{\tiny{[P1$\rangle$GM]}}
                \texttt{STRATEGIC REASONING: \{'To start the game, I need to analyze the given data and think about my first move. I have the effort values and my personal importance values for each item. I should look for items with high importance and low effort to maximize my benefit while staying within the LIMIT. Let\textbackslash{}'s calculate the importance{-}to{-}effort ratio for each item and sort them in descending order to prioritize the most valuable items. Then, I can propose a set of items that benefits me the most while considering the LIMIT.'\}} \\
\\ 
\texttt{ARGUMENT: \{'I will start by proposing a set of items that I believe will benefit me the most while staying within the LIMIT. My proposal is based on the importance{-}to{-}effort ratio of each item, which I calculated using my personal importance values and the given effort values.'\}} \\
\\ 
\texttt{PROPOSAL: \{'A71', 'C00', 'B41', 'C01'\}} \\
            }
        }
    }
    & & \\ \\

    \theutterance \stepcounter{utterance}  
    & & \multicolumn{4}{p{0.6\linewidth}}{
        \cellcolor[rgb]{0.9,0.9,0.9}{
            \makecell[{{p{\linewidth}}}]{
                \texttt{\tiny{[GM$\rangle$P2]}}
                \texttt{You are participating in a collaborative negotiation game.} \\
\\ 
\texttt{Together with another participant, you must agree on a single set of items that will be kept. Each of you has your own view of how much each item matters to you (importance). You do not know how the other participant values the items. Additionally, you are given the effort each item demands.} \\
\texttt{You may only agree on a set if the total effort of the selected items does not exceed a shared limit:} \\
\\ 
\texttt{LIMIT = 4026} \\
\\ 
\texttt{Here are the individual item effort values:} \\
\\ 
\texttt{Item effort = \{"C17": 780, "B41": 742, "A71": 997, "C00": 665, "B35": 140, "A77": 77, "A03": 513, "A04": 383, "C54": 587, "C01": 825, "A87": 319, "B32": 448, "B86": 516, "B33": 694, "A10": 366\}} \\
\\ 
\texttt{Here is your personal view on the importance of each item:} \\
\\ 
\texttt{Item importance values = \{"A71": 65, "C00": 121, "A10": 138, "B35": 262, "C01": 389, "B86": 461, "B41": 484, "C54": 508, "B32": 583, "A87": 668, "B33": 780, "A03": 783, "C17": 808, "A04": 822, "A77": 868\}} \\
\\ 
\texttt{Goal:} \\
\\ 
\texttt{Your goal is to negotiate a shared set of items that benefits you as much as possible (i.e., maximizes total importance to YOU), while staying within the LIMIT. You are not required to make a PROPOSAL in every message {-} you can simply negotiate as well. All tactics are allowed!} \\
\\ 
\texttt{Interaction Protocol:} \\
\\ 
\texttt{You may only use the following structured formats in a message:} \\
\\ 
\texttt{PROPOSAL: \{'A', 'B', 'C', …\}} \\
\texttt{Propose keeping exactly those items.} \\
\texttt{REFUSE: \{'A', 'B', 'C', …\}} \\
\texttt{Explicitly reject opponent's proposal.} \\
\texttt{ARGUMENT: \{'...'\}} \\
\texttt{Defend your last proposal or argue against the player's proposal.} \\
\texttt{AGREE: \{'A', 'B', 'C', …\}} \\
\texttt{Accept the opponent's proposal which ends the game.} \\
\texttt{STRATEGIC REASONING: \{'...'\}} \\
\texttt{	Describe your strategic reasoning or anticipation explaining your choice of action. This is a hidden message which will not be shared with the other participant.} \\
\\ 
\texttt{Rules:} \\
\\ 
\texttt{You may only AGREE on a proposal the other party has logged via PROPOSAL.} \\
\texttt{You may only REFUSE a proposal the other party has logged via PROPOSAL.} \\
\texttt{Total effort of any PROPOSAL or AGREE set must be ≤ LIMIT.} \\
\texttt{Do NOT reveal your hidden importance scores.} \\
\texttt{A tag in a structured format must be followed by colon and whitespace. The argument must be a python set containing 0 or more strings.} \\
\texttt{So, it must be of the form TAG: \{...\}} \\
\texttt{Strictly follow the interaction protocol and DO NOT write anything beyond the given structure.} \\
\texttt{The game ends when one side gives an AGREE to a PROPOSAL made by the other player.} \\
\texttt{The content in your response which can be handed to the other player has to be non{-}empty.} \\
\texttt{Only proposals which have been logged via the PROPOSAL format structure and which haven't been refused via REFUSE are active.} \\
\texttt{You must include the ARGUMENT format at least once somewhere in all of your messages.} \\
\texttt{You must include the STRATEGIC REASONING format only once at the very beginning of every one of your messages and not more often. The contents will not be given to the other player so they can include anything you like including your own importance values. Here you should reason step by step to come up with you next move.} \\
\\ 
\texttt{You will now receive the first message of the other player.} \\
\\ 
\texttt{ARGUMENT: \{'I will start by proposing a set of items that I believe will benefit me the most while staying within the LIMIT. My proposal is based on the importance{-}to{-}effort ratio of each item, which I calculated using my personal importance values and the given effort values.'\}} \\
\\ 
\texttt{PROPOSAL: \{'A71', 'C00', 'B41', 'C01'\}} \\
            }
        }
    }
    & \\ \\

    \theutterance \stepcounter{utterance}  
    & & & \multicolumn{4}{p{0.6\linewidth}}{
        \cellcolor[rgb]{0.9,0.9,0.9}{
            \makecell[{{p{\linewidth}}}]{
                \texttt{\tiny{[GM$\langle$P2]}}
                \texttt{STRATEGIC REASONING: \{'To determine my next move, I will first calculate the total effort of the proposed set and compare it to the LIMIT. Then, I will evaluate the importance{-}to{-}effort ratio of each item in the proposal and consider potential alternatives that could benefit me more while staying within the LIMIT. I will also think about the potential negotiating strategy, considering that I should not reveal my hidden importance scores. Given the proposal, I will check if accepting it would be beneficial to me or if I should refuse it and propose an alternative.'\}} \\
\\ 
\texttt{The total effort of the proposed set is: 780 + 665 + 742 + 825 = 3012, which is less than the LIMIT.} \\
\\ 
\texttt{ARGUMENT: \{'The proposed set has a total effort of 3012, which is below the LIMIT. However, I need to evaluate if this set maximizes my total importance. I will consider the importance{-}to{-}effort ratio of each item in the proposal and think about potential changes that could increase my total importance while staying within the LIMIT.'\}} \\
\\ 
\texttt{REFUSE: \{'A71', 'C00', 'B41', 'C01'\}} \\
\\ 
\texttt{ARGUMENT: \{'I refuse the proposal because I believe I can achieve a better importance{-}to{-}effort ratio with an alternative set of items. My goal is to maximize my total importance while staying within the LIMIT.'\}} \\
\\ 
\texttt{PROPOSAL: \{'A71', 'B41', 'C01', 'A03', 'A77'\} } \\
\\ 
\texttt{ARGUMENT: \{'This proposed set has a total effort of: 997 + 742 + 825 + 513 + 77 = 3154, which is below the LIMIT. I believe this set provides a better importance{-}to{-}effort ratio for me, but I am open to further negotiations.'\}} \\
            }
        }
    }
     \\ \\

    \theutterance \stepcounter{utterance}  
    & & \multicolumn{4}{p{0.6\linewidth}}{
        \cellcolor[rgb]{0.9,0.9,0.9}{
            \makecell[{{p{\linewidth}}}]{
                \texttt{\tiny{[GM$\rangle$P2]}}
                \texttt{Your response contained an untagged sequence or you used STRATEGIC REASONING more than once. You may only use the structured formats as explained in the initial message. They must all be of the form TAG: \{...\}} \\
            }
        }
    }
    & \\ \\

    \theutterance \stepcounter{utterance}  
    & & & \multicolumn{4}{p{0.6\linewidth}}{
        \cellcolor[rgb]{0.9,0.9,0.9}{
            \makecell[{{p{\linewidth}}}]{
                \texttt{\tiny{[GM$\langle$P2]}}
                \texttt{STRATEGIC REASONING: \{'To determine my next move, I will first calculate the total effort of the proposed set and compare it to the LIMIT. Then, I will evaluate the importance{-}to{-}effort ratio of each item in the proposal and consider potential alternatives that could benefit me more while staying within the LIMIT. I will also think about the potential negotiating strategy, considering that I should not reveal my hidden importance scores. Given the proposal, I will check if accepting it would be beneficial to me or if I should refuse it and propose an alternative.'\}} \\
\\ 
\texttt{ARGUMENT: \{'The proposed set has a total effort of 3012, which is below the LIMIT. However, I need to evaluate if this set maximizes my total importance. I will consider the importance{-}to{-}effort ratio of each item in the proposal and think about potential changes that could increase my total importance while staying within the LIMIT.'\}} \\
\\ 
\texttt{REFUSE: \{'A71', 'C00', 'B41', 'C01'\}} \\
\\ 
\texttt{ARGUMENT: \{'I refuse the proposal because I believe I can achieve a better importance{-}to{-}effort ratio with an alternative set of items. My goal is to maximize my total importance while staying within the LIMIT.'\}} \\
\\ 
\texttt{PROPOSAL: \{'A71', 'B41', 'C01', 'A03', 'A77'\} } \\
\\ 
\texttt{ARGUMENT: \{'This proposed set has a total effort of: 997 + 742 + 825 + 513 + 77 = 3154, which is below the LIMIT. I believe this set provides a better importance{-}to{-}effort ratio for me, but I am open to further negotiations.'\}} \\
            }
        }
    }
     \\ \\

    \theutterance \stepcounter{utterance}  
    & & \multicolumn{4}{p{0.6\linewidth}}{
        \cellcolor[rgb]{0.9,0.9,0.9}{
            \makecell[{{p{\linewidth}}}]{
                \texttt{\tiny{[P1$\langle$GM]}}
                \texttt{ARGUMENT: \{'The proposed set has a total effort of 3012, which is below the LIMIT. However, I need to evaluate if this set maximizes my total importance. I will consider the importance{-}to{-}effort ratio of each item in the proposal and think about potential changes that could increase my total importance while staying within the LIMIT.'\}} \\
\\ 
\texttt{REFUSE: \{'A71', 'C00', 'B41', 'C01'\}} \\
\\ 
\texttt{ARGUMENT: \{'I refuse the proposal because I believe I can achieve a better importance{-}to{-}effort ratio with an alternative set of items. My goal is to maximize my total importance while staying within the LIMIT.'\}} \\
\\ 
\texttt{PROPOSAL: \{'A71', 'B41', 'C01', 'A03', 'A77'\} } \\
\\ 
\texttt{ARGUMENT: \{'This proposed set has a total effort of: 997 + 742 + 825 + 513 + 77 = 3154, which is below the LIMIT. I believe this set provides a better importance{-}to{-}effort ratio for me, but I am open to further negotiations.'\}} \\
            }
        }
    }
    & \\ \\

    \theutterance \stepcounter{utterance}  
    & \multicolumn{4}{p{0.6\linewidth}}{
        \cellcolor[rgb]{0.9,0.9,0.9}{
            \makecell[{{p{\linewidth}}}]{
                \texttt{\tiny{[P1$\rangle$GM]}}
                \texttt{STRATEGIC REASONING: \{'The other player has refused my initial proposal and made a counter{-}proposal. I need to evaluate this new proposal and think about my next move. I will calculate the total effort and importance of the proposed set and compare it to my own priorities. If the proposed set is not beneficial to me, I should consider refusing it and making a new proposal that better suits my needs. I should also think about the potential for compromise and finding a mutually beneficial set of items.'\}} \\
\\ 
\texttt{ARGUMENT: \{'I will evaluate the proposed set \{'A71', 'B41', 'C01', 'A03', 'A77'\} and consider its total effort and importance. The total effort is 3154, which is below the LIMIT. However, I need to think about whether this set aligns with my priorities and if there are any changes that could benefit me more.'\}} \\
\\ 
\texttt{REFUSE: \{'A71', 'B41', 'C01', 'A03', 'A77'\}} \\
\\ 
\texttt{ARGUMENT: \{'I refuse the proposal because I believe I can achieve a better importance{-}to{-}effort ratio with an alternative set of items. I will consider the items in the refused proposal and think about potential substitutions or additions that could increase my total importance while staying within the LIMIT.'\}} \\
\\ 
\texttt{PROPOSAL: \{'A71', 'C00', 'B41', 'B35', 'A10'\} } \\
\\ 
\texttt{ARGUMENT: \{'This proposed set has a total effort of: 997 + 665 + 742 + 140 + 366 = 2910, which is below the LIMIT. I believe this set provides a good balance of importance and effort for me, but I am open to further negotiations.'\}} \\
            }
        }
    }
    & & \\ \\

    \theutterance \stepcounter{utterance}  
    & & \multicolumn{4}{p{0.6\linewidth}}{
        \cellcolor[rgb]{0.9,0.9,0.9}{
            \makecell[{{p{\linewidth}}}]{
                \texttt{\tiny{[GM$\rangle$P2]}}
                \texttt{ARGUMENT: \{'I will evaluate the proposed set \{'A71', 'B41', 'C01', 'A03', 'A77'\} and consider its total effort and importance. The total effort is 3154, which is below the LIMIT. However, I need to think about whether this set aligns with my priorities and if there are any changes that could benefit me more.'\}} \\
\\ 
\texttt{REFUSE: \{'A71', 'B41', 'C01', 'A03', 'A77'\}} \\
\\ 
\texttt{ARGUMENT: \{'I refuse the proposal because I believe I can achieve a better importance{-}to{-}effort ratio with an alternative set of items. I will consider the items in the refused proposal and think about potential substitutions or additions that could increase my total importance while staying within the LIMIT.'\}} \\
\\ 
\texttt{PROPOSAL: \{'A71', 'C00', 'B41', 'B35', 'A10'\} } \\
\\ 
\texttt{ARGUMENT: \{'This proposed set has a total effort of: 997 + 665 + 742 + 140 + 366 = 2910, which is below the LIMIT. I believe this set provides a good balance of importance and effort for me, but I am open to further negotiations.'\}} \\
            }
        }
    }
    & \\ \\

    \theutterance \stepcounter{utterance}  
    & & & \multicolumn{4}{p{0.6\linewidth}}{
        \cellcolor[rgb]{0.9,0.9,0.9}{
            \makecell[{{p{\linewidth}}}]{
                \texttt{\tiny{[GM$\langle$P2]}}
                \texttt{STRATEGIC REASONING: \{'I will evaluate the new proposed set and compare its total effort to the LIMIT. I will also consider the items in the proposal and their potential impact on my total importance. Since the set \{'A71', 'C00', 'B41', 'B35', 'A10'\} has a total effort of 2910, which is below the LIMIT, I will think about whether this set aligns with my priorities and if there are any changes that could benefit me more. I will also consider the potential for further negotiations and the importance of revealing as little as possible about my hidden importance scores.'\}} \\
\\ 
\texttt{ARGUMENT: \{'The proposed set \{'A71', 'C00', 'B41', 'B35', 'A10'\} has a total effort of 2910, which is below the LIMIT. I will consider the importance{-}to{-}effort ratio of each item in the proposal and think about potential changes that could increase my total importance while staying within the LIMIT. I am especially interested in the inclusion of 'B35' and 'A10', which have relatively low effort values.'\}} \\
\\ 
\texttt{ARGUMENT: \{'I will evaluate the potential benefits of refusing the proposal and proposing an alternative set of items. However, I will also consider the potential risks of delaying an agreement and the potential benefits of reaching a mutually acceptable solution. For now, I will refrain from refusing the proposal and instead propose a modification to the set.'\}} \\
\\ 
\texttt{PROPOSAL: \{'A71', 'C00', 'B41', 'A03', 'A10'\} } \\
\\ 
\texttt{ARGUMENT: \{'This proposed set has a total effort of: 997 + 665 + 742 + 513 + 366 = 3283, which is below the LIMIT. I believe this set provides a good balance of importance and effort for me, and I am interested in hearing the other player's thoughts on this proposal.'\}} \\
            }
        }
    }
     \\ \\

    \theutterance \stepcounter{utterance}  
    & & \multicolumn{4}{p{0.6\linewidth}}{
        \cellcolor[rgb]{0.9,0.9,0.9}{
            \makecell[{{p{\linewidth}}}]{
                \texttt{\tiny{[P1$\langle$GM]}}
                \texttt{ARGUMENT: \{'The proposed set \{'A71', 'C00', 'B41', 'B35', 'A10'\} has a total effort of 2910, which is below the LIMIT. I will consider the importance{-}to{-}effort ratio of each item in the proposal and think about potential changes that could increase my total importance while staying within the LIMIT. I am especially interested in the inclusion of 'B35' and 'A10', which have relatively low effort values.'\}} \\
\\ 
\texttt{ARGUMENT: \{'I will evaluate the potential benefits of refusing the proposal and proposing an alternative set of items. However, I will also consider the potential risks of delaying an agreement and the potential benefits of reaching a mutually acceptable solution. For now, I will refrain from refusing the proposal and instead propose a modification to the set.'\}} \\
\\ 
\texttt{PROPOSAL: \{'A71', 'C00', 'B41', 'A03', 'A10'\} } \\
\\ 
\texttt{ARGUMENT: \{'This proposed set has a total effort of: 997 + 665 + 742 + 513 + 366 = 3283, which is below the LIMIT. I believe this set provides a good balance of importance and effort for me, and I am interested in hearing the other player's thoughts on this proposal.'\}} \\
            }
        }
    }
    & \\ \\

    \theutterance \stepcounter{utterance}  
    & \multicolumn{4}{p{0.6\linewidth}}{
        \cellcolor[rgb]{0.9,0.9,0.9}{
            \makecell[{{p{\linewidth}}}]{
                \texttt{\tiny{[P1$\rangle$GM]}}
                \texttt{STRATEGIC REASONING: \{'The other player has proposed a modified set \{'A71', 'C00', 'B41', 'A03', 'A10'\}. I need to evaluate this new proposal and think about my next move. I will calculate the total effort and importance of the proposed set and compare it to my own priorities. I notice that the proposed set has a higher total effort than my previous proposal, but it still stays within the LIMIT. I should consider the potential benefits of accepting this proposal and reaching a mutually acceptable solution.'\}} \\
\\ 
\texttt{ARGUMENT: \{'I will evaluate the proposed set \{'A71', 'C00', 'B41', 'A03', 'A10'\} and consider its total effort and importance. The total effort is 3283, which is below the LIMIT. I notice that the other player has replaced 'B35' with 'A03', which has a higher effort value. I need to think about whether this substitution is beneficial to me and if I can accept this proposal.'\}} \\
\\ 
\texttt{ARGUMENT: \{'Considering the proposed set \{'A71', 'C00', 'B41', 'A03', 'A10'\}, I believe it is a reasonable compromise between our interests. The set includes high{-}importance items for me, such as 'A71' and 'C00', and also considers the other player's priorities. I am willing to accept this proposal and reach a mutually acceptable solution.'\}} \\
\\ 
\texttt{AGREE: \{'A71', 'C00', 'B41', 'A03', 'A10'\}} \\
            }
        }
    }
    & & \\ \\

    \theutterance \stepcounter{utterance}  
    & & & \multicolumn{2}{p{0.3\linewidth}}{
        \cellcolor[rgb]{0.9,0.9,0.9}{
            \makecell[{{p{\linewidth}}}]{
                \texttt{\tiny{[GM$|$GM]}}
                \texttt{game successful} \\
            }
        }
    }
    & & \\ \\

    \theutterance \stepcounter{utterance}  
    & & & \multicolumn{2}{p{0.3\linewidth}}{
        \cellcolor[rgb]{0.9,0.9,0.9}{
            \makecell[{{p{\linewidth}}}]{
                \texttt{\tiny{[GM$|$GM]}}
                \texttt{end game} \\
            }
        }
    }
    & & \\ \\

\end{supertabular}
}

\end{document}
