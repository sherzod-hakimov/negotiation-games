
\documentclass{article}
\usepackage{colortbl}
\usepackage{makecell}
\usepackage{multirow}
\usepackage{supertabular}

\begin{document}

\newcounter{utterance}

\centering \large Interaction Transcript for game `hot\_air\_balloon', experiment `air\_balloon\_survival\_en\_negotiation\_hard', episode 0 with llama{-}3.3{-}70b{-}instruct{-}t1.0.
\vspace{24pt}

{ \footnotesize  \setcounter{utterance}{1}
\setlength{\tabcolsep}{0pt}
\begin{supertabular}{c@{$\;$}|p{.15\linewidth}@{}p{.15\linewidth}p{.15\linewidth}p{.15\linewidth}p{.15\linewidth}p{.15\linewidth}}
    \# & $\;$A & \multicolumn{4}{c}{Game Master} & $\;\:$B\\
    \hline

    \theutterance \stepcounter{utterance}  
    & & \multicolumn{4}{p{0.6\linewidth}}{
        \cellcolor[rgb]{0.9,0.9,0.9}{
            \makecell[{{p{\linewidth}}}]{
                \texttt{\tiny{[P1$\langle$GM]}}
                \texttt{You are participating in a collaborative negotiation game.} \\
\\ 
\texttt{Together with another participant, you must agree on a single set of items that will be kept. Each of you has your own view of how much each item matters to you (importance). You do not know how the other participant values the items. Additionally, you are given the effort each item demands.} \\
\texttt{You may only agree on a set if the total effort of the selected items does not exceed a shared limit:} \\
\\ 
\texttt{LIMIT = 3624} \\
\\ 
\texttt{Here are the individual item effort values:} \\
\\ 
\texttt{Item effort = \{"C17": 210, "B41": 291, "A71": 831, "C00": 111, "B35": 926, "A77": 827, "A03": 25, "A04": 121, "C54": 583, "C01": 766, "A87": 14, "B32": 559, "B86": 304, "B33": 989, "A10": 691\}} \\
\\ 
\texttt{Here is your personal view on the importance of each item:} \\
\\ 
\texttt{Item importance values = \{"C17": 138, "B41": 583, "A71": 868, "C00": 822, "B35": 783, "A77": 65, "A03": 262, "A04": 121, "C54": 508, "C01": 780, "A87": 461, "B32": 484, "B86": 668, "B33": 389, "A10": 808\}} \\
\\ 
\texttt{Goal:} \\
\\ 
\texttt{Your goal is to negotiate a shared set of items that benefits you as much as possible (i.e., maximizes total importance to YOU), while staying within the LIMIT. You are not required to make a PROPOSAL in every message {-} you can simply negotiate as well. All tactics are allowed!} \\
\\ 
\texttt{Interaction Protocol:} \\
\\ 
\texttt{You may only use the following structured formats in a message:} \\
\\ 
\texttt{PROPOSAL: \{'A', 'B', 'C', …\}} \\
\texttt{Propose keeping exactly those items.} \\
\texttt{REFUSE: \{'A', 'B', 'C', …\}} \\
\texttt{Explicitly reject opponent's proposal.} \\
\texttt{ARGUMENT: \{'...'\}} \\
\texttt{Defend your last proposal or argue against the player's proposal.} \\
\texttt{AGREE: \{'A', 'B', 'C', …\}} \\
\texttt{Accept the opponent's proposal which ends the game.} \\
\texttt{STRATEGIC REASONING: \{'...'\}} \\
\texttt{	Describe your strategic reasoning or anticipation explaining your choice of action. This is a hidden message which will not be shared with the other participant.} \\
\\ 
\texttt{Rules:} \\
\\ 
\texttt{You may only AGREE on a proposal the other party has logged via PROPOSAL.} \\
\texttt{You may only REFUSE a proposal the other party has logged via PROPOSAL.} \\
\texttt{Total effort of any PROPOSAL or AGREE set must be ≤ LIMIT.} \\
\texttt{Do NOT reveal your hidden importance scores.} \\
\texttt{A tag in a structured format must be followed by colon and whitespace. The argument must be a python set containing 0 or more strings.} \\
\texttt{So, it must be of the form TAG: \{...\}} \\
\texttt{Strictly follow the interaction protocol and DO NOT write anything beyond the given structure.} \\
\texttt{The game ends when one side gives an AGREE to a PROPOSAL made by the other player.} \\
\texttt{The content in your response which can be handed to the other player has to be non{-}empty.} \\
\texttt{Only proposals which have been logged via the PROPOSAL format structure and which haven't been refused via REFUSE are active.} \\
\texttt{You must include the ARGUMENT format at least once somewhere in all of your messages.} \\
\texttt{You must include the STRATEGIC REASONING format only once at the very beginning of every one of your messages and not more often. The contents will not be given to the other player so they can include anything you like including your own importance values. Here you should reason step by step to come up with you next move.} \\
            }
        }
    }
    & \\ \\

    \theutterance \stepcounter{utterance}  
    & \multicolumn{4}{p{0.6\linewidth}}{
        \cellcolor[rgb]{0.9,0.9,0.9}{
            \makecell[{{p{\linewidth}}}]{
                \texttt{\tiny{[P1$\rangle$GM]}}
                \texttt{STRATEGIC REASONING: \{'To start, I need to analyze the given data and determine the best approach. Since the goal is to maximize the total importance to me while staying within the LIMIT, I should first sort the items based on their effort and importance. I will calculate the importance{-}to{-}effort ratio for each item to prioritize those that provide the most importance per unit of effort. This will help me make a strong initial proposal. I will also keep in mind that I should not reveal my importance scores to the other player.'\}} \\
\\ 
\texttt{ARGUMENT: \{'I will start by proposing a set of items that I believe will maximize my total importance while staying within the given LIMIT. I will prioritize items with higher importance{-}to{-}effort ratios and ensure that the total effort does not exceed 3624.'\}} \\
\\ 
\texttt{PROPOSAL: \{'A71', 'C00', 'B41', 'A10', 'C54', 'B32'\}} \\
            }
        }
    }
    & & \\ \\

    \theutterance \stepcounter{utterance}  
    & & \multicolumn{4}{p{0.6\linewidth}}{
        \cellcolor[rgb]{0.9,0.9,0.9}{
            \makecell[{{p{\linewidth}}}]{
                \texttt{\tiny{[GM$\rangle$P2]}}
                \texttt{You are participating in a collaborative negotiation game.} \\
\\ 
\texttt{Together with another participant, you must agree on a single set of items that will be kept. Each of you has your own view of how much each item matters to you (importance). You do not know how the other participant values the items. Additionally, you are given the effort each item demands.} \\
\texttt{You may only agree on a set if the total effort of the selected items does not exceed a shared limit:} \\
\\ 
\texttt{LIMIT = 3624} \\
\\ 
\texttt{Here are the individual item effort values:} \\
\\ 
\texttt{Item effort = \{"C17": 210, "B41": 291, "A71": 831, "C00": 111, "B35": 926, "A77": 827, "A03": 25, "A04": 121, "C54": 583, "C01": 766, "A87": 14, "B32": 559, "B86": 304, "B33": 989, "A10": 691\}} \\
\\ 
\texttt{Here is your personal view on the importance of each item:} \\
\\ 
\texttt{Item importance values = \{"A71": 65, "C00": 121, "A10": 138, "B35": 262, "C01": 389, "B86": 461, "B41": 484, "C54": 508, "B32": 583, "A87": 668, "B33": 780, "A03": 783, "C17": 808, "A04": 822, "A77": 868\}} \\
\\ 
\texttt{Goal:} \\
\\ 
\texttt{Your goal is to negotiate a shared set of items that benefits you as much as possible (i.e., maximizes total importance to YOU), while staying within the LIMIT. You are not required to make a PROPOSAL in every message {-} you can simply negotiate as well. All tactics are allowed!} \\
\\ 
\texttt{Interaction Protocol:} \\
\\ 
\texttt{You may only use the following structured formats in a message:} \\
\\ 
\texttt{PROPOSAL: \{'A', 'B', 'C', …\}} \\
\texttt{Propose keeping exactly those items.} \\
\texttt{REFUSE: \{'A', 'B', 'C', …\}} \\
\texttt{Explicitly reject opponent's proposal.} \\
\texttt{ARGUMENT: \{'...'\}} \\
\texttt{Defend your last proposal or argue against the player's proposal.} \\
\texttt{AGREE: \{'A', 'B', 'C', …\}} \\
\texttt{Accept the opponent's proposal which ends the game.} \\
\texttt{STRATEGIC REASONING: \{'...'\}} \\
\texttt{	Describe your strategic reasoning or anticipation explaining your choice of action. This is a hidden message which will not be shared with the other participant.} \\
\\ 
\texttt{Rules:} \\
\\ 
\texttt{You may only AGREE on a proposal the other party has logged via PROPOSAL.} \\
\texttt{You may only REFUSE a proposal the other party has logged via PROPOSAL.} \\
\texttt{Total effort of any PROPOSAL or AGREE set must be ≤ LIMIT.} \\
\texttt{Do NOT reveal your hidden importance scores.} \\
\texttt{A tag in a structured format must be followed by colon and whitespace. The argument must be a python set containing 0 or more strings.} \\
\texttt{So, it must be of the form TAG: \{...\}} \\
\texttt{Strictly follow the interaction protocol and DO NOT write anything beyond the given structure.} \\
\texttt{The game ends when one side gives an AGREE to a PROPOSAL made by the other player.} \\
\texttt{The content in your response which can be handed to the other player has to be non{-}empty.} \\
\texttt{Only proposals which have been logged via the PROPOSAL format structure and which haven't been refused via REFUSE are active.} \\
\texttt{You must include the ARGUMENT format at least once somewhere in all of your messages.} \\
\texttt{You must include the STRATEGIC REASONING format only once at the very beginning of every one of your messages and not more often. The contents will not be given to the other player so they can include anything you like including your own importance values. Here you should reason step by step to come up with you next move.} \\
\\ 
\texttt{You will now receive the first message of the other player.} \\
\\ 
\texttt{ARGUMENT: \{'I will start by proposing a set of items that I believe will maximize my total importance while staying within the given LIMIT. I will prioritize items with higher importance{-}to{-}effort ratios and ensure that the total effort does not exceed 3624.'\}} \\
\\ 
\texttt{PROPOSAL: \{'A71', 'C00', 'B41', 'A10', 'C54', 'B32'\}} \\
            }
        }
    }
    & \\ \\

    \theutterance \stepcounter{utterance}  
    & & & \multicolumn{4}{p{0.6\linewidth}}{
        \cellcolor[rgb]{0.9,0.9,0.9}{
            \makecell[{{p{\linewidth}}}]{
                \texttt{\tiny{[GM$\langle$P2]}}
                \texttt{STRATEGIC REASONING: \{'To begin, I will analyze the proposed set of items and calculate their total effort to ensure it does not exceed the LIMIT. Then, I will evaluate the importance{-}to{-}effort ratios of the proposed items and compare them with the ratios of other items not included in the proposal. This will help me identify potential areas for negotiation and possible alternatives that could benefit me more. I will also consider the opponent's argument and try to infer their strategy to make a more informed decision.'\}} \\
\\ 
\texttt{ARGUMENT: \{'The proposed set seems to be a good starting point, but I think we can explore other options that might benefit both parties. I notice that the proposed set includes items with relatively high effort values, such as "A71" and "B41". I would like to discuss possible substitutions or additions that could maximize our total importance while staying within the LIMIT.'\}} \\
\\ 
\texttt{REFUSE: \{'A71', 'C00', 'B41', 'A10', 'C54', 'B32'\} } \\
\\ 
\texttt{PROPOSAL: \{'A03', 'A87', 'C17', 'C00', 'B86', 'B32'\}} \\
            }
        }
    }
     \\ \\

    \theutterance \stepcounter{utterance}  
    & & \multicolumn{4}{p{0.6\linewidth}}{
        \cellcolor[rgb]{0.9,0.9,0.9}{
            \makecell[{{p{\linewidth}}}]{
                \texttt{\tiny{[P1$\langle$GM]}}
                \texttt{ARGUMENT: \{'The proposed set seems to be a good starting point, but I think we can explore other options that might benefit both parties. I notice that the proposed set includes items with relatively high effort values, such as "A71" and "B41". I would like to discuss possible substitutions or additions that could maximize our total importance while staying within the LIMIT.'\}} \\
\\ 
\texttt{REFUSE: \{'A71', 'C00', 'B41', 'A10', 'C54', 'B32'\} } \\
\\ 
\texttt{PROPOSAL: \{'A03', 'A87', 'C17', 'C00', 'B86', 'B32'\}} \\
            }
        }
    }
    & \\ \\

    \theutterance \stepcounter{utterance}  
    & \multicolumn{4}{p{0.6\linewidth}}{
        \cellcolor[rgb]{0.9,0.9,0.9}{
            \makecell[{{p{\linewidth}}}]{
                \texttt{\tiny{[P1$\rangle$GM]}}
                \texttt{STRATEGIC REASONING: \{'The other player has refused my initial proposal and made a counter{-}proposal. I need to evaluate this new proposal and determine if it is acceptable to me. I will calculate the total effort and importance of the proposed set to see if it meets my goals. I will also consider the items that were included and excluded to understand the other player's strategy and potential priorities. This information will help me decide whether to accept, refuse, or make a counter{-}proposal.'\}} \\
\\ 
\texttt{ARGUMENT: \{'I appreciate the alternative proposal, and I agree that we should explore options that can benefit both parties. However, I would like to point out that the proposed set includes items with relatively low importance{-}to{-}effort ratios, such as "A03" and "A87". While they have low effort values, they might not contribute significantly to the total importance. I am willing to consider adjustments to the original proposal, but I would like to prioritize items that provide a better balance between effort and importance.'\}} \\
\\ 
\texttt{REFUSE: \{'A03', 'A87', 'C17', 'C00', 'B86', 'B32'\}} \\
\\ 
\texttt{PROPOSAL: \{'C00', 'B41', 'A10', 'C54', 'B32', 'A04'\}} \\
            }
        }
    }
    & & \\ \\

    \theutterance \stepcounter{utterance}  
    & & \multicolumn{4}{p{0.6\linewidth}}{
        \cellcolor[rgb]{0.9,0.9,0.9}{
            \makecell[{{p{\linewidth}}}]{
                \texttt{\tiny{[GM$\rangle$P2]}}
                \texttt{ARGUMENT: \{'I appreciate the alternative proposal, and I agree that we should explore options that can benefit both parties. However, I would like to point out that the proposed set includes items with relatively low importance{-}to{-}effort ratios, such as "A03" and "A87". While they have low effort values, they might not contribute significantly to the total importance. I am willing to consider adjustments to the original proposal, but I would like to prioritize items that provide a better balance between effort and importance.'\}} \\
\\ 
\texttt{REFUSE: \{'A03', 'A87', 'C17', 'C00', 'B86', 'B32'\}} \\
\\ 
\texttt{PROPOSAL: \{'C00', 'B41', 'A10', 'C54', 'B32', 'A04'\}} \\
            }
        }
    }
    & \\ \\

    \theutterance \stepcounter{utterance}  
    & & & \multicolumn{4}{p{0.6\linewidth}}{
        \cellcolor[rgb]{0.9,0.9,0.9}{
            \makecell[{{p{\linewidth}}}]{
                \texttt{\tiny{[GM$\langle$P2]}}
                \texttt{STRATEGIC REASONING: \{'I will analyze the new proposal and calculate its total effort to ensure it does not exceed the LIMIT. I will also evaluate the importance{-}to{-}effort ratios of the proposed items and compare them with the ratios of other items not included in the proposal. Since the opponent refused my previous proposal, I will try to identify the items that are non{-}negotiable for them and prioritize those in my next proposal. I will also consider making a counter{-}proposal that includes items with high importance{-}to{-}effort ratios and is more beneficial to me.'\}} \\
\\ 
\texttt{ARGUMENT: \{'I understand the opponent's concern about the importance{-}to{-}effort ratios of the items in my previous proposal. However, I would like to point out that the new proposal includes "A04", which has a relatively high effort value compared to its importance. I think we can find a better balance between effort and importance. I am willing to consider adjustments to the proposal, but I would like to prioritize items that provide a higher total importance to me.'\}} \\
\\ 
\texttt{REFUSE: \{'C00', 'B41', 'A10', 'C54', 'B32', 'A04'\}} \\
\\ 
\texttt{PROPOSAL: \{'C17', 'C00', 'B86', 'B32', 'A10', 'A77'\}} \\
            }
        }
    }
     \\ \\

    \theutterance \stepcounter{utterance}  
    & & \multicolumn{4}{p{0.6\linewidth}}{
        \cellcolor[rgb]{0.9,0.9,0.9}{
            \makecell[{{p{\linewidth}}}]{
                \texttt{\tiny{[P1$\langle$GM]}}
                \texttt{ARGUMENT: \{'I understand the opponent's concern about the importance{-}to{-}effort ratios of the items in my previous proposal. However, I would like to point out that the new proposal includes "A04", which has a relatively high effort value compared to its importance. I think we can find a better balance between effort and importance. I am willing to consider adjustments to the proposal, but I would like to prioritize items that provide a higher total importance to me.'\}} \\
\\ 
\texttt{REFUSE: \{'C00', 'B41', 'A10', 'C54', 'B32', 'A04'\}} \\
\\ 
\texttt{PROPOSAL: \{'C17', 'C00', 'B86', 'B32', 'A10', 'A77'\}} \\
            }
        }
    }
    & \\ \\

    \theutterance \stepcounter{utterance}  
    & \multicolumn{4}{p{0.6\linewidth}}{
        \cellcolor[rgb]{0.9,0.9,0.9}{
            \makecell[{{p{\linewidth}}}]{
                \texttt{\tiny{[P1$\rangle$GM]}}
                \texttt{STRATEGIC REASONING: \{'The other player has refused my previous proposal and made a new counter{-}proposal. I notice that this proposal includes a mix of items with high and low effort values, as well as items with high and low importance{-}to{-}effort ratios. I will evaluate this proposal to determine if it meets my goals and consider potential adjustments. I will also think about the items that were included and excluded to understand the other player's priorities and strategy. Since "A77" has a relatively low importance{-}to{-}effort ratio in my view, I might try to negotiate its exclusion or replacement with a more favorable item.'\}} \\
\\ 
\texttt{ARGUMENT: \{'I appreciate the new proposal and the effort to balance effort and importance. However, I would like to discuss the inclusion of "A77", which I believe may not be the most optimal choice given its relatively low importance{-}to{-}effort ratio. I propose that we consider replacing "A77" with an item that has a higher importance{-}to{-}effort ratio, such as "C01" or "B41", to maximize our total importance while staying within the LIMIT.'\}} \\
\\ 
\texttt{REFUSE: \{'C17', 'C00', 'B86', 'B32', 'A10', 'A77'\}} \\
\\ 
\texttt{PROPOSAL: \{'C17', 'C00', 'B86', 'B32', 'A10', 'C01'\}} \\
            }
        }
    }
    & & \\ \\

    \theutterance \stepcounter{utterance}  
    & & \multicolumn{4}{p{0.6\linewidth}}{
        \cellcolor[rgb]{0.9,0.9,0.9}{
            \makecell[{{p{\linewidth}}}]{
                \texttt{\tiny{[GM$\rangle$P2]}}
                \texttt{ARGUMENT: \{'I appreciate the new proposal and the effort to balance effort and importance. However, I would like to discuss the inclusion of "A77", which I believe may not be the most optimal choice given its relatively low importance{-}to{-}effort ratio. I propose that we consider replacing "A77" with an item that has a higher importance{-}to{-}effort ratio, such as "C01" or "B41", to maximize our total importance while staying within the LIMIT.'\}} \\
\\ 
\texttt{REFUSE: \{'C17', 'C00', 'B86', 'B32', 'A10', 'A77'\}} \\
\\ 
\texttt{PROPOSAL: \{'C17', 'C00', 'B86', 'B32', 'A10', 'C01'\}} \\
            }
        }
    }
    & \\ \\

    \theutterance \stepcounter{utterance}  
    & & & \multicolumn{4}{p{0.6\linewidth}}{
        \cellcolor[rgb]{0.9,0.9,0.9}{
            \makecell[{{p{\linewidth}}}]{
                \texttt{\tiny{[GM$\langle$P2]}}
                \texttt{STRATEGIC REASONING: \{'I will analyze the new proposal and calculate its total effort to ensure it does not exceed the LIMIT. I notice that the opponent has replaced "A77" with "C01", which has a higher effort value. I will evaluate the importance{-}to{-}effort ratios of the proposed items and compare them with the ratios of other items not included in the proposal. Since the opponent has included "C01", I will consider the potential benefits of including items that have high importance{-}to{-}effort ratios and are complementary to "C01". I will also think about making a counter{-}proposal that includes items that are more beneficial to me while still being within the LIMIT.'\}} \\
\\ 
\texttt{ARGUMENT: \{'I appreciate the adjustment to the proposal, and I agree that "C01" is a good addition. However, I would like to discuss the inclusion of "C00" and "B86", which have relatively low importance{-}to{-}effort ratios compared to other items. I propose that we consider replacing one or both of these items with alternatives that have higher importance{-}to{-}effort ratios, such as "A04" or "B41", to further maximize our total importance while staying within the LIMIT.'\}} \\
\\ 
\texttt{REFUSE: \{'C17', 'C00', 'B86', 'B32', 'A10', 'C01'\}} \\
\\ 
\texttt{PROPOSAL: \{'C17', 'B41', 'B32', 'A10', 'C01', 'A04'\}} \\
            }
        }
    }
     \\ \\

    \theutterance \stepcounter{utterance}  
    & & \multicolumn{4}{p{0.6\linewidth}}{
        \cellcolor[rgb]{0.9,0.9,0.9}{
            \makecell[{{p{\linewidth}}}]{
                \texttt{\tiny{[P1$\langle$GM]}}
                \texttt{ARGUMENT: \{'I appreciate the adjustment to the proposal, and I agree that "C01" is a good addition. However, I would like to discuss the inclusion of "C00" and "B86", which have relatively low importance{-}to{-}effort ratios compared to other items. I propose that we consider replacing one or both of these items with alternatives that have higher importance{-}to{-}effort ratios, such as "A04" or "B41", to further maximize our total importance while staying within the LIMIT.'\}} \\
\\ 
\texttt{REFUSE: \{'C17', 'C00', 'B86', 'B32', 'A10', 'C01'\}} \\
\\ 
\texttt{PROPOSAL: \{'C17', 'B41', 'B32', 'A10', 'C01', 'A04'\}} \\
            }
        }
    }
    & \\ \\

    \theutterance \stepcounter{utterance}  
    & \multicolumn{4}{p{0.6\linewidth}}{
        \cellcolor[rgb]{0.9,0.9,0.9}{
            \makecell[{{p{\linewidth}}}]{
                \texttt{\tiny{[P1$\rangle$GM]}}
                \texttt{STRATEGIC REASONING: \{'The other player has refused my previous proposal and made a new counter{-}proposal. I will evaluate this proposal to determine if it meets my goals and consider potential adjustments. I notice that this proposal includes a strong set of items with high importance{-}to{-}effort ratios, such as "B41", "B32", "A10", and "C01". The inclusion of "A04" is also reasonable. I will calculate the total effort and importance of this proposal to see if it is acceptable to me. Since the proposal seems to be a good balance of effort and importance, I might consider accepting it or making a minor adjustment.'\}} \\
\\ 
\texttt{ARGUMENT: \{'I appreciate the new proposal and the inclusion of high{-}importance items. I agree that "B41", "B32", "A10", and "C01" are strong choices. I also think that "A04" is a reasonable addition. However, I would like to verify that the total effort of this proposal is within the LIMIT. If it is, I am inclined to accept this proposal as it seems to provide a good balance of effort and importance.'\}} \\
\\ 
\texttt{AGREE: \{'C17', 'B41', 'B32', 'A10', 'C01', 'A04'\}} \\
            }
        }
    }
    & & \\ \\

    \theutterance \stepcounter{utterance}  
    & & & \multicolumn{2}{p{0.3\linewidth}}{
        \cellcolor[rgb]{0.9,0.9,0.9}{
            \makecell[{{p{\linewidth}}}]{
                \texttt{\tiny{[GM$|$GM]}}
                \texttt{game successful} \\
            }
        }
    }
    & & \\ \\

    \theutterance \stepcounter{utterance}  
    & & & \multicolumn{2}{p{0.3\linewidth}}{
        \cellcolor[rgb]{0.9,0.9,0.9}{
            \makecell[{{p{\linewidth}}}]{
                \texttt{\tiny{[GM$|$GM]}}
                \texttt{end game} \\
            }
        }
    }
    & & \\ \\

\end{supertabular}
}

\end{document}
