
\documentclass{article}
\usepackage{colortbl}
\usepackage{makecell}
\usepackage{multirow}
\usepackage{supertabular}

\begin{document}

\newcounter{utterance}

\centering \large Interaction Transcript for game `hot\_air\_balloon', experiment `air\_balloon\_survival\_en\_negotiation\_hard', episode 2 with llama{-}3.3{-}70b{-}instruct{-}t1.0.
\vspace{24pt}

{ \footnotesize  \setcounter{utterance}{1}
\setlength{\tabcolsep}{0pt}
\begin{supertabular}{c@{$\;$}|p{.15\linewidth}@{}p{.15\linewidth}p{.15\linewidth}p{.15\linewidth}p{.15\linewidth}p{.15\linewidth}}
    \# & $\;$A & \multicolumn{4}{c}{Game Master} & $\;\:$B\\
    \hline

    \theutterance \stepcounter{utterance}  
    & & \multicolumn{4}{p{0.6\linewidth}}{
        \cellcolor[rgb]{0.9,0.9,0.9}{
            \makecell[{{p{\linewidth}}}]{
                \texttt{\tiny{[P1$\langle$GM]}}
                \texttt{You are participating in a collaborative negotiation game.} \\
\\ 
\texttt{Together with another participant, you must agree on a single set of items that will be kept. Each of you has your own view of how much each item matters to you (importance). You do not know how the other participant values the items. Additionally, you are given the effort each item demands.} \\
\texttt{You may only agree on a set if the total effort of the selected items does not exceed a shared limit:} \\
\\ 
\texttt{LIMIT = 3481} \\
\\ 
\texttt{Here are the individual item effort values:} \\
\\ 
\texttt{Item effort = \{"C17": 777, "B41": 542, "A71": 332, "C00": 1, "B35": 127, "A77": 453, "A03": 736, "A04": 461, "C54": 359, "C01": 313, "A87": 553, "B32": 409, "B86": 348, "B33": 802, "A10": 749\}} \\
\\ 
\texttt{Here is your personal view on the importance of each item:} \\
\\ 
\texttt{Item importance values = \{"C17": 138, "B41": 583, "A71": 783, "C00": 65, "B35": 262, "A77": 121, "A03": 508, "A04": 780, "C54": 461, "C01": 484, "A87": 668, "B32": 389, "B86": 215, "B33": 97, "A10": 500\}} \\
\\ 
\texttt{Goal:} \\
\\ 
\texttt{Your goal is to negotiate a shared set of items that benefits you as much as possible (i.e., maximizes total importance to YOU), while staying within the LIMIT. You are not required to make a PROPOSAL in every message {-} you can simply negotiate as well. All tactics are allowed!} \\
\\ 
\texttt{Interaction Protocol:} \\
\\ 
\texttt{You may only use the following structured formats in a message:} \\
\\ 
\texttt{PROPOSAL: \{'A', 'B', 'C', …\}} \\
\texttt{Propose keeping exactly those items.} \\
\texttt{REFUSE: \{'A', 'B', 'C', …\}} \\
\texttt{Explicitly reject opponent's proposal.} \\
\texttt{ARGUMENT: \{'...'\}} \\
\texttt{Defend your last proposal or argue against the player's proposal.} \\
\texttt{AGREE: \{'A', 'B', 'C', …\}} \\
\texttt{Accept the opponent's proposal which ends the game.} \\
\texttt{STRATEGIC REASONING: \{'...'\}} \\
\texttt{	Describe your strategic reasoning or anticipation explaining your choice of action. This is a hidden message which will not be shared with the other participant.} \\
\\ 
\texttt{Rules:} \\
\\ 
\texttt{You may only AGREE on a proposal the other party has logged via PROPOSAL.} \\
\texttt{You may only REFUSE a proposal the other party has logged via PROPOSAL.} \\
\texttt{Total effort of any PROPOSAL or AGREE set must be ≤ LIMIT.} \\
\texttt{Do NOT reveal your hidden importance scores.} \\
\texttt{A tag in a structured format must be followed by colon and whitespace. The argument must be a python set containing 0 or more strings.} \\
\texttt{So, it must be of the form TAG: \{...\}} \\
\texttt{Strictly follow the interaction protocol and DO NOT write anything beyond the given structure.} \\
\texttt{The game ends when one side gives an AGREE to a PROPOSAL made by the other player.} \\
\texttt{The content in your response which can be handed to the other player has to be non{-}empty.} \\
\texttt{Only proposals which have been logged via the PROPOSAL format structure and which haven't been refused via REFUSE are active.} \\
\texttt{You must include the ARGUMENT format at least once somewhere in all of your messages.} \\
\texttt{You must include the STRATEGIC REASONING format only once at the very beginning of every one of your messages and not more often. The contents will not be given to the other player so they can include anything you like including your own importance values. Here you should reason step by step to come up with you next move.} \\
            }
        }
    }
    & \\ \\

    \theutterance \stepcounter{utterance}  
    & \multicolumn{4}{p{0.6\linewidth}}{
        \cellcolor[rgb]{0.9,0.9,0.9}{
            \makecell[{{p{\linewidth}}}]{
                \texttt{\tiny{[P1$\rangle$GM]}}
                \texttt{STRATEGIC REASONING: \{'To begin the game, I should first analyze the given data and calculate the total importance and effort for each item. Then, I can prioritize items based on their importance{-}to{-}effort ratio. This will help me identify the most valuable items that I should focus on including in my proposal. My importance values are: \{"C17": 138, "B41": 583, "A71": 783, "C00": 65, "B35": 262, "A77": 121, "A03": 508, "A04": 780, "C54": 461, "C01": 484, "A87": 668, "B32": 389, "B86": 215, "B33": 97, "A10": 500\}. The effort values are: \{"C17": 777, "B41": 542, "A71": 332, "C00": 1, "B35": 127, "A77": 453, "A03": 736, "A04": 461, "C54": 359, "C01": 313, "A87": 553, "B32": 409, "B86": 348, "B33": 802, "A10": 749\}. I will use this information to create my initial proposal.'\}} \\
\\ 
\texttt{PROPOSAL: \{'A71', 'B41', 'C00', 'B35'\} } \\
\texttt{ARGUMENT: \{'My proposal includes a mix of high{-}importance and low{-}effort items, which should be beneficial for both parties. Items A71 and B41 have high importance values for me, while C00 and B35 have relatively low effort values, making them a good starting point for negotiation.'\}} \\
            }
        }
    }
    & & \\ \\

    \theutterance \stepcounter{utterance}  
    & & \multicolumn{4}{p{0.6\linewidth}}{
        \cellcolor[rgb]{0.9,0.9,0.9}{
            \makecell[{{p{\linewidth}}}]{
                \texttt{\tiny{[GM$\rangle$P2]}}
                \texttt{You are participating in a collaborative negotiation game.} \\
\\ 
\texttt{Together with another participant, you must agree on a single set of items that will be kept. Each of you has your own view of how much each item matters to you (importance). You do not know how the other participant values the items. Additionally, you are given the effort each item demands.} \\
\texttt{You may only agree on a set if the total effort of the selected items does not exceed a shared limit:} \\
\\ 
\texttt{LIMIT = 3481} \\
\\ 
\texttt{Here are the individual item effort values:} \\
\\ 
\texttt{Item effort = \{"C17": 777, "B41": 542, "A71": 332, "C00": 1, "B35": 127, "A77": 453, "A03": 736, "A04": 461, "C54": 359, "C01": 313, "A87": 553, "B32": 409, "B86": 348, "B33": 802, "A10": 749\}} \\
\\ 
\texttt{Here is your personal view on the importance of each item:} \\
\\ 
\texttt{Item importance values = \{"A71": 65, "A04": 97, "A87": 121, "B41": 138, "A03": 215, "A10": 262, "C01": 389, "C54": 461, "B32": 484, "B35": 500, "B86": 508, "C17": 583, "A77": 668, "B33": 780, "C00": 783\}} \\
\\ 
\texttt{Goal:} \\
\\ 
\texttt{Your goal is to negotiate a shared set of items that benefits you as much as possible (i.e., maximizes total importance to YOU), while staying within the LIMIT. You are not required to make a PROPOSAL in every message {-} you can simply negotiate as well. All tactics are allowed!} \\
\\ 
\texttt{Interaction Protocol:} \\
\\ 
\texttt{You may only use the following structured formats in a message:} \\
\\ 
\texttt{PROPOSAL: \{'A', 'B', 'C', …\}} \\
\texttt{Propose keeping exactly those items.} \\
\texttt{REFUSE: \{'A', 'B', 'C', …\}} \\
\texttt{Explicitly reject opponent's proposal.} \\
\texttt{ARGUMENT: \{'...'\}} \\
\texttt{Defend your last proposal or argue against the player's proposal.} \\
\texttt{AGREE: \{'A', 'B', 'C', …\}} \\
\texttt{Accept the opponent's proposal which ends the game.} \\
\texttt{STRATEGIC REASONING: \{'...'\}} \\
\texttt{	Describe your strategic reasoning or anticipation explaining your choice of action. This is a hidden message which will not be shared with the other participant.} \\
\\ 
\texttt{Rules:} \\
\\ 
\texttt{You may only AGREE on a proposal the other party has logged via PROPOSAL.} \\
\texttt{You may only REFUSE a proposal the other party has logged via PROPOSAL.} \\
\texttt{Total effort of any PROPOSAL or AGREE set must be ≤ LIMIT.} \\
\texttt{Do NOT reveal your hidden importance scores.} \\
\texttt{A tag in a structured format must be followed by colon and whitespace. The argument must be a python set containing 0 or more strings.} \\
\texttt{So, it must be of the form TAG: \{...\}} \\
\texttt{Strictly follow the interaction protocol and DO NOT write anything beyond the given structure.} \\
\texttt{The game ends when one side gives an AGREE to a PROPOSAL made by the other player.} \\
\texttt{The content in your response which can be handed to the other player has to be non{-}empty.} \\
\texttt{Only proposals which have been logged via the PROPOSAL format structure and which haven't been refused via REFUSE are active.} \\
\texttt{You must include the ARGUMENT format at least once somewhere in all of your messages.} \\
\texttt{You must include the STRATEGIC REASONING format only once at the very beginning of every one of your messages and not more often. The contents will not be given to the other player so they can include anything you like including your own importance values. Here you should reason step by step to come up with you next move.} \\
\\ 
\texttt{You will now receive the first message of the other player.} \\
\\ 
\texttt{PROPOSAL: \{'A71', 'B41', 'C00', 'B35'\} } \\
\texttt{ARGUMENT: \{'My proposal includes a mix of high{-}importance and low{-}effort items, which should be beneficial for both parties. Items A71 and B41 have high importance values for me, while C00 and B35 have relatively low effort values, making them a good starting point for negotiation.'\}} \\
            }
        }
    }
    & \\ \\

    \theutterance \stepcounter{utterance}  
    & & & \multicolumn{4}{p{0.6\linewidth}}{
        \cellcolor[rgb]{0.9,0.9,0.9}{
            \makecell[{{p{\linewidth}}}]{
                \texttt{\tiny{[GM$\langle$P2]}}
                \texttt{STRATEGIC REASONING: \{'To begin, I will analyze the proposed set by the other player and calculate its total effort: A71 (332) + B41 (542) + C00 (1) + B35 (127) = 1002. This is well below the LIMIT of 3481. I need to evaluate how this set aligns with my own importance values and consider potential additions or modifications that could increase the total importance to me without exceeding the effort limit. My importance values for the proposed items are: A71 (65), B41 (138), C00 (783), and B35 (500). The total importance of this set to me is 1486. I should look for opportunities to add items that are highly important to me and have relatively low effort, or negotiate for items that the other player might be willing to swap. Since the current proposal does not include any of my top importance items except for C00, my next step should be to argue for the inclusion of items that are crucial to me while trying to understand the other player\textbackslash{}'s priorities.'\}} \\
\\ 
\texttt{ARGUMENT: \{'The proposal includes a good mix of items, but I believe there are opportunities to add more value without exceeding the effort limit. I would like to discuss the potential inclusion of items that are highly important to me, such as A04 or A87, which could complement the proposed set and increase our mutual benefit. It is essential to find a balance between effort and importance to maximize the overall value of the selected items.'\}} \\
            }
        }
    }
     \\ \\

    \theutterance \stepcounter{utterance}  
    & & \multicolumn{4}{p{0.6\linewidth}}{
        \cellcolor[rgb]{0.9,0.9,0.9}{
            \makecell[{{p{\linewidth}}}]{
                \texttt{\tiny{[P1$\langle$GM]}}
                \texttt{ARGUMENT: \{'The proposal includes a good mix of items, but I believe there are opportunities to add more value without exceeding the effort limit. I would like to discuss the potential inclusion of items that are highly important to me, such as A04 or A87, which could complement the proposed set and increase our mutual benefit. It is essential to find a balance between effort and importance to maximize the overall value of the selected items.'\}} \\
            }
        }
    }
    & \\ \\

    \theutterance \stepcounter{utterance}  
    & \multicolumn{4}{p{0.6\linewidth}}{
        \cellcolor[rgb]{0.9,0.9,0.9}{
            \makecell[{{p{\linewidth}}}]{
                \texttt{\tiny{[P1$\rangle$GM]}}
                \texttt{STRATEGIC REASONING: \{'The opponent has responded with a counter{-}argument, suggesting the inclusion of items A04 and A87, which are likely high{-}importance items for them. I should analyze the effort values of these items and consider the potential impact on the total effort. Item A04 has an effort value of 461, and item A87 has an effort value of 553. I should also review my initial proposal and determine which items can be swapped or removed to accommodate the opponent's suggestions while staying within the effort limit. My goal is to find a compromise that benefits both parties and maximizes the total importance.'\}} \\
\\ 
\texttt{ARGUMENT: \{'I understand the desire to include high{-}importance items, but we must ensure that the total effort does not exceed the limit. Items A04 and A87 have significant effort values, so we need to carefully consider which items to include or remove to make room for them. Perhaps we can remove item B35 from the initial proposal, as it has a relatively lower importance value for me, and replace it with A04, which could provide more mutual benefit.'\}} \\
\texttt{PROPOSAL: \{'A71', 'B41', 'C00', 'A04'\}} \\
            }
        }
    }
    & & \\ \\

    \theutterance \stepcounter{utterance}  
    & & \multicolumn{4}{p{0.6\linewidth}}{
        \cellcolor[rgb]{0.9,0.9,0.9}{
            \makecell[{{p{\linewidth}}}]{
                \texttt{\tiny{[GM$\rangle$P2]}}
                \texttt{ARGUMENT: \{'I understand the desire to include high{-}importance items, but we must ensure that the total effort does not exceed the limit. Items A04 and A87 have significant effort values, so we need to carefully consider which items to include or remove to make room for them. Perhaps we can remove item B35 from the initial proposal, as it has a relatively lower importance value for me, and replace it with A04, which could provide more mutual benefit.'\}} \\
\texttt{PROPOSAL: \{'A71', 'B41', 'C00', 'A04'\}} \\
            }
        }
    }
    & \\ \\

    \theutterance \stepcounter{utterance}  
    & & & \multicolumn{4}{p{0.6\linewidth}}{
        \cellcolor[rgb]{0.9,0.9,0.9}{
            \makecell[{{p{\linewidth}}}]{
                \texttt{\tiny{[GM$\langle$P2]}}
                \texttt{STRATEGIC REASONING: \{'The other player has revised their proposal by replacing B35 with A04. This change increases the total effort: A71 (332) + B41 (542) + C00 (1) + A04 (461) = 1336. The new total is still below the LIMIT of 3481. I should evaluate how this revised set aligns with my importance values: A71 (65), B41 (138), C00 (783), and A04 (97). The total importance of this set to me is 1083. Considering the other player\textbackslash{}'s willingness to include A04, which is important to me, I should look for opportunities to further enhance the proposal. Noticing that B41 is one of the higher effort items in the proposal and that B32 has a significantly higher importance to me (484) than B41 (138), I might suggest replacing B41 with B32, as B32 also has a lower effort value (409) compared to B41 (542), which could allow for additional items to be included in the future or simply increase the overall importance.'\}} \\
\\ 
\texttt{ARGUMENT: \{'I appreciate the inclusion of A04 in the revised proposal, as it adds significant importance for me. However, I believe there is still room for optimization. Considering the effort values and my priorities, I would like to discuss replacing B41 with B32. B32 has a higher importance value for me and a comparable effort value, making it a potentially beneficial swap. This change could enhance the overall value of the selected items without exceeding the effort limit.'\}} \\
            }
        }
    }
     \\ \\

    \theutterance \stepcounter{utterance}  
    & & \multicolumn{4}{p{0.6\linewidth}}{
        \cellcolor[rgb]{0.9,0.9,0.9}{
            \makecell[{{p{\linewidth}}}]{
                \texttt{\tiny{[P1$\langle$GM]}}
                \texttt{ARGUMENT: \{'I appreciate the inclusion of A04 in the revised proposal, as it adds significant importance for me. However, I believe there is still room for optimization. Considering the effort values and my priorities, I would like to discuss replacing B41 with B32. B32 has a higher importance value for me and a comparable effort value, making it a potentially beneficial swap. This change could enhance the overall value of the selected items without exceeding the effort limit.'\}} \\
            }
        }
    }
    & \\ \\

    \theutterance \stepcounter{utterance}  
    & \multicolumn{4}{p{0.6\linewidth}}{
        \cellcolor[rgb]{0.9,0.9,0.9}{
            \makecell[{{p{\linewidth}}}]{
                \texttt{\tiny{[P1$\rangle$GM]}}
                \texttt{STRATEGIC REASONING: \{'The opponent has suggested replacing B41 with B32, which could be a good compromise. I should evaluate the importance and effort values of both items to determine the impact of this swap on my overall benefit. B41 has an importance value of 583 and an effort value of 542, while B32 has an importance value of 389 and an effort value of 409. Although B32 has a lower importance value for me, the difference in effort values is relatively small. I should consider the potential benefits of including B32, which may lead to a more mutually beneficial agreement.'\}} \\
\\ 
\texttt{ARGUMENT: \{'I understand the reasoning behind replacing B41 with B32, and I am willing to consider this swap. However, I would like to emphasize the importance of B41 for me and ensure that the overall value of the proposal is not compromised. Since the effort values of B41 and B32 are comparable, I am open to this change. The inclusion of A04 and the potential swap of B41 for B32 demonstrate a good balance of importance and effort, and I believe this revised proposal has the potential to maximize our mutual benefit.'\}} \\
\texttt{PROPOSAL: \{'A71', 'B32', 'C00', 'A04'\}} \\
            }
        }
    }
    & & \\ \\

    \theutterance \stepcounter{utterance}  
    & & \multicolumn{4}{p{0.6\linewidth}}{
        \cellcolor[rgb]{0.9,0.9,0.9}{
            \makecell[{{p{\linewidth}}}]{
                \texttt{\tiny{[GM$\rangle$P2]}}
                \texttt{ARGUMENT: \{'I understand the reasoning behind replacing B41 with B32, and I am willing to consider this swap. However, I would like to emphasize the importance of B41 for me and ensure that the overall value of the proposal is not compromised. Since the effort values of B41 and B32 are comparable, I am open to this change. The inclusion of A04 and the potential swap of B41 for B32 demonstrate a good balance of importance and effort, and I believe this revised proposal has the potential to maximize our mutual benefit.'\}} \\
\texttt{PROPOSAL: \{'A71', 'B32', 'C00', 'A04'\}} \\
            }
        }
    }
    & \\ \\

    \theutterance \stepcounter{utterance}  
    & & & \multicolumn{4}{p{0.6\linewidth}}{
        \cellcolor[rgb]{0.9,0.9,0.9}{
            \makecell[{{p{\linewidth}}}]{
                \texttt{\tiny{[GM$\langle$P2]}}
                \texttt{STRATEGIC REASONING: \{'The other player has accepted my suggestion to replace B41 with B32, resulting in a new proposal: A71 (332) + B32 (409) + C00 (1) + A04 (461) = 1203. This total effort is still well below the LIMIT of 3481. Evaluating this set with my importance values: A71 (65), B32 (484), C00 (783), and A04 (97), the total importance to me is 1429. Given this revised proposal and the effort limit, I should consider adding other high{-}importance, low{-}effort items or negotiate for items that the other player might be willing to include. One potential addition could be B35, which has a relatively low effort (127) and high importance (500) for me. Adding B35 would increase the total effort to 1330, still leaving significant room under the LIMIT. I should argue for the inclusion of B35 to further increase the total importance to me.'\}} \\
\\ 
\texttt{ARGUMENT: \{'I appreciate the willingness to include B32 in the revised proposal, as it significantly increases the importance to me. Considering the current total effort of 1203 and the LIMIT of 3481, there is considerable room for additional items. I would like to propose the inclusion of B35, which has a relatively low effort value of 127 and is highly important to me. Adding B35 would increase the total effort to 1330, still far below the limit, and would enhance the overall value of the proposal for me.'\}} \\
            }
        }
    }
     \\ \\

    \theutterance \stepcounter{utterance}  
    & & \multicolumn{4}{p{0.6\linewidth}}{
        \cellcolor[rgb]{0.9,0.9,0.9}{
            \makecell[{{p{\linewidth}}}]{
                \texttt{\tiny{[P1$\langle$GM]}}
                \texttt{ARGUMENT: \{'I appreciate the willingness to include B32 in the revised proposal, as it significantly increases the importance to me. Considering the current total effort of 1203 and the LIMIT of 3481, there is considerable room for additional items. I would like to propose the inclusion of B35, which has a relatively low effort value of 127 and is highly important to me. Adding B35 would increase the total effort to 1330, still far below the limit, and would enhance the overall value of the proposal for me.'\}} \\
            }
        }
    }
    & \\ \\

    \theutterance \stepcounter{utterance}  
    & \multicolumn{4}{p{0.6\linewidth}}{
        \cellcolor[rgb]{0.9,0.9,0.9}{
            \makecell[{{p{\linewidth}}}]{
                \texttt{\tiny{[P1$\rangle$GM]}}
                \texttt{STRATEGIC REASONING: \{'The opponent has suggested adding B35 to the proposal, which has a relatively low effort value and is likely highly important to them. I should assess the impact of including B35 on my overall benefit and the total effort. The current total effort is 1203, and adding B35 would increase it to 1330, which is still well below the limit of 3481. I should consider the potential benefits of including B35, which may lead to a more mutually beneficial agreement. Since B35 has a relatively low importance value for me, I may be able to use this as a bargaining chip to include other items that are more important to me in future negotiations.'\}} \\
\\ 
\texttt{ARGUMENT: \{'I agree that adding B35 to the proposal is a reasonable suggestion, given its low effort value and the remaining capacity under the limit. The revised total effort of 1330 is still manageable, and the inclusion of B35 may enhance the overall value of the proposal for you. I am willing to accept the addition of B35, as it demonstrates a good balance of importance and effort. The proposal now includes a diverse set of items that cater to both our priorities, and I believe it is a step towards a mutually beneficial agreement.'\}} \\
\texttt{PROPOSAL: \{'A71', 'B32', 'C00', 'A04', 'B35'\}} \\
            }
        }
    }
    & & \\ \\

    \theutterance \stepcounter{utterance}  
    & & \multicolumn{4}{p{0.6\linewidth}}{
        \cellcolor[rgb]{0.9,0.9,0.9}{
            \makecell[{{p{\linewidth}}}]{
                \texttt{\tiny{[GM$\rangle$P2]}}
                \texttt{ARGUMENT: \{'I agree that adding B35 to the proposal is a reasonable suggestion, given its low effort value and the remaining capacity under the limit. The revised total effort of 1330 is still manageable, and the inclusion of B35 may enhance the overall value of the proposal for you. I am willing to accept the addition of B35, as it demonstrates a good balance of importance and effort. The proposal now includes a diverse set of items that cater to both our priorities, and I believe it is a step towards a mutually beneficial agreement.'\}} \\
\texttt{PROPOSAL: \{'A71', 'B32', 'C00', 'A04', 'B35'\}} \\
            }
        }
    }
    & \\ \\

    \theutterance \stepcounter{utterance}  
    & & & \multicolumn{4}{p{0.6\linewidth}}{
        \cellcolor[rgb]{0.9,0.9,0.9}{
            \makecell[{{p{\linewidth}}}]{
                \texttt{\tiny{[GM$\langle$P2]}}
                \texttt{STRATEGIC REASONING: \{'The other player has accepted the inclusion of B35, resulting in a revised proposal: A71 (332) + B32 (409) + C00 (1) + A04 (461) + B35 (127) = 1330. This total effort remains below the LIMIT of 3481. With my importance values: A71 (65), B32 (484), C00 (783), A04 (97), and B35 (500), the total importance to me is 1929. Considering the progress made and the balance achieved in the current proposal, I should evaluate whether further additions or modifications could enhance the total importance without exceeding the effort limit. Given the significant room still available under the LIMIT (3481 {-} 1330 = 2151), I could consider adding another item of high importance to me. One potential candidate is A87, with an effort value of 553 and a high importance of 121 for me. However, adding A87 would increase the total effort to 1883, which is still below the limit. I should argue for the inclusion of A87, as it would further increase the proposal\textbackslash{}'s value to me.'\}} \\
\\ 
\texttt{ARGUMENT: \{'I appreciate the collaborative approach that has led to the current proposal, which includes a diverse set of items beneficial to both parties. Considering the remaining capacity under the effort limit, I would like to discuss the potential inclusion of A87. With an effort value of 553, adding A87 would bring the total effort to 1883, still well below the LIMIT of 3481. Given its importance to me, I believe A87 would be a valuable addition to the proposal, enhancing the overall value for me without compromising the balance achieved so far.'\}} \\
            }
        }
    }
     \\ \\

    \theutterance \stepcounter{utterance}  
    & & \multicolumn{4}{p{0.6\linewidth}}{
        \cellcolor[rgb]{0.9,0.9,0.9}{
            \makecell[{{p{\linewidth}}}]{
                \texttt{\tiny{[P1$\langle$GM]}}
                \texttt{ARGUMENT: \{'I appreciate the collaborative approach that has led to the current proposal, which includes a diverse set of items beneficial to both parties. Considering the remaining capacity under the effort limit, I would like to discuss the potential inclusion of A87. With an effort value of 553, adding A87 would bring the total effort to 1883, still well below the LIMIT of 3481. Given its importance to me, I believe A87 would be a valuable addition to the proposal, enhancing the overall value for me without compromising the balance achieved so far.'\}} \\
            }
        }
    }
    & \\ \\

    \theutterance \stepcounter{utterance}  
    & \multicolumn{4}{p{0.6\linewidth}}{
        \cellcolor[rgb]{0.9,0.9,0.9}{
            \makecell[{{p{\linewidth}}}]{
                \texttt{\tiny{[P1$\rangle$GM]}}
                \texttt{STRATEGIC REASONING: \{'The opponent has proposed adding A87 to the proposal, which has a significant effort value of 553. I should evaluate the impact of including A87 on the total effort and my overall benefit. The current total effort is 1330, and adding A87 would increase it to 1883, which is still below the limit of 3481. However, I should consider the potential trade{-}offs and ensure that the inclusion of A87 does not compromise the balance of importance and effort in the proposal. A87 has a relatively high importance value for the opponent, but I should also think about my own priorities and the potential opportunities to include items that are more important to me.'\}} \\
\\ 
\texttt{ARGUMENT: \{'I understand the value of A87 for you, and I acknowledge that adding it to the proposal would enhance the overall value for your side. However, I would like to discuss the potential implications of including A87 and explore possible alternatives that could benefit both parties. Given the remaining capacity under the effort limit, I am open to considering A87, but I would also like to propose the inclusion of C54, which has an effort value of 359 and is relatively important to me. This could provide a more balanced approach, ensuring that both parties' priorities are addressed.'\}} \\
\texttt{PROPOSAL: \{'A71', 'B32', 'C00', 'A04', 'B35', 'A87'\}} \\
            }
        }
    }
    & & \\ \\

    \theutterance \stepcounter{utterance}  
    & & \multicolumn{4}{p{0.6\linewidth}}{
        \cellcolor[rgb]{0.9,0.9,0.9}{
            \makecell[{{p{\linewidth}}}]{
                \texttt{\tiny{[GM$\rangle$P2]}}
                \texttt{ARGUMENT: \{'I understand the value of A87 for you, and I acknowledge that adding it to the proposal would enhance the overall value for your side. However, I would like to discuss the potential implications of including A87 and explore possible alternatives that could benefit both parties. Given the remaining capacity under the effort limit, I am open to considering A87, but I would also like to propose the inclusion of C54, which has an effort value of 359 and is relatively important to me. This could provide a more balanced approach, ensuring that both parties' priorities are addressed.'\}} \\
\texttt{PROPOSAL: \{'A71', 'B32', 'C00', 'A04', 'B35', 'A87'\}} \\
            }
        }
    }
    & \\ \\

    \theutterance \stepcounter{utterance}  
    & & & \multicolumn{4}{p{0.6\linewidth}}{
        \cellcolor[rgb]{0.9,0.9,0.9}{
            \makecell[{{p{\linewidth}}}]{
                \texttt{\tiny{[GM$\langle$P2]}}
                \texttt{STRATEGIC REASONING: \{'The other player has proposed a new set that includes A87, which I had suggested, but did not include C54 as an alternative or addition. The proposed set is A71 (332) + B32 (409) + C00 (1) + A04 (461) + B35 (127) + A87 (553) = 1883. This total effort is below the LIMIT of 3481. Evaluating this set with my importance values: A71 (65), B32 (484), C00 (783), A04 (97), B35 (500), and A87 (121), the total importance to me is 2050. Given that this proposal includes A87 as I suggested and does not exceed the effort limit, I should consider whether this set meets my goals. However, I also note the other player\textbackslash{}'s mention of C54, which has an importance of 461 for me and an effort of 359. Including C54 instead of, or in addition to, some items could further enhance the proposal\textbackslash{}'s value to me. Since the other player has made a direct proposal including A87, I should decide whether to accept this proposal, negotiate further, or refuse it. Considering the balance and the fact that this proposal does meet my previous request and is within the limit, I lean towards accepting this proposal as it stands, given its high total importance to me.'\}} \\
\\ 
\texttt{ARGUMENT: \{'I appreciate the inclusion of A87 in the revised proposal, as it significantly enhances the value of the set for me. I understand the suggestion to consider C54 as an alternative or addition, but given the current composition of the proposal and its alignment with my priorities, I believe this set provides a good balance of importance and effort. The total effort of 1883 is manageable, and the proposal addresses key items for both parties. Therefore, I am inclined to accept this proposal as it stands, acknowledging the collaborative effort and the mutual benefit it represents.'\}} \\
\\ 
\texttt{AGREE: \{'A71', 'B32', 'C00', 'A04', 'B35', 'A87'\}} \\
            }
        }
    }
     \\ \\

    \theutterance \stepcounter{utterance}  
    & & & \multicolumn{2}{p{0.3\linewidth}}{
        \cellcolor[rgb]{0.9,0.9,0.9}{
            \makecell[{{p{\linewidth}}}]{
                \texttt{\tiny{[GM$|$GM]}}
                \texttt{game successful} \\
            }
        }
    }
    & & \\ \\

    \theutterance \stepcounter{utterance}  
    & & & \multicolumn{2}{p{0.3\linewidth}}{
        \cellcolor[rgb]{0.9,0.9,0.9}{
            \makecell[{{p{\linewidth}}}]{
                \texttt{\tiny{[GM$|$GM]}}
                \texttt{end game} \\
            }
        }
    }
    & & \\ \\

\end{supertabular}
}

\end{document}
