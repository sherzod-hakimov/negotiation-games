
\documentclass{article}
\usepackage{colortbl}
\usepackage{makecell}
\usepackage{multirow}
\usepackage{supertabular}

\begin{document}

\newcounter{utterance}

\centering \large Interaction Transcript for game `hot\_air\_balloon', experiment `air\_balloon\_survival\_en\_complexity\_easy', episode 5 with llama{-}3.3{-}70b{-}instruct{-}t1.0.
\vspace{24pt}

{ \footnotesize  \setcounter{utterance}{1}
\setlength{\tabcolsep}{0pt}
\begin{supertabular}{c@{$\;$}|p{.15\linewidth}@{}p{.15\linewidth}p{.15\linewidth}p{.15\linewidth}p{.15\linewidth}p{.15\linewidth}}
    \# & $\;$A & \multicolumn{4}{c}{Game Master} & $\;\:$B\\
    \hline

    \theutterance \stepcounter{utterance}  
    & & \multicolumn{4}{p{0.6\linewidth}}{
        \cellcolor[rgb]{0.9,0.9,0.9}{
            \makecell[{{p{\linewidth}}}]{
                \texttt{\tiny{[P1$\langle$GM]}}
                \texttt{You are participating in a collaborative negotiation game.} \\
\\ 
\texttt{Together with another participant, you must agree on a single set of items that will be kept. Each of you has your own view of how much each item matters to you (importance). You do not know how the other participant values the items. Additionally, you are given the effort each item demands.} \\
\texttt{You may only agree on a set if the total effort of the selected items does not exceed a shared limit:} \\
\\ 
\texttt{LIMIT = 9475} \\
\\ 
\texttt{Here are the individual item effort values:} \\
\\ 
\texttt{Item effort = \{"B21": 944, "B00": 568, "A08": 515, "A27": 239, "B94": 876, "B53": 421, "A23": 991, "C65": 281, "A89": 788, "B67": 678, "B45": 432, "C34": 409, "C12": 279, "A05": 506, "B60": 101, "C92": 686, "B69": 851, "A84": 854, "C06": 133, "B12": 192, "A16": 573, "B85": 17, "C35": 465, "B10": 772, "C15": 46, "B09": 501, "B16": 220, "C81": 404, "A58": 843, "C24": 748, "A30": 552, "A66": 843, "B75": 947, "B46": 345, "C70": 930\}} \\
\\ 
\texttt{Here is your personal view on the importance of each item:} \\
\\ 
\texttt{Item importance values = \{"B21": 138, "B00": 583, "A08": 868, "A27": 822, "B94": 783, "B53": 65, "A23": 262, "C65": 121, "A89": 508, "B67": 780, "B45": 461, "C34": 484, "C12": 668, "A05": 389, "B60": 808, "C92": 215, "B69": 97, "A84": 500, "C06": 30, "B12": 915, "A16": 856, "B85": 400, "C35": 444, "B10": 623, "C15": 781, "B09": 786, "B16": 3, "C81": 713, "A58": 457, "C24": 273, "A30": 739, "A66": 822, "B75": 235, "B46": 606, "C70": 968\}} \\
\\ 
\texttt{Goal:} \\
\\ 
\texttt{Your goal is to negotiate a shared set of items that benefits you as much as possible (i.e., maximizes total importance to YOU), while staying within the LIMIT. You are not required to make a PROPOSAL in every message {-} you can simply negotiate as well. All tactics are allowed!} \\
\\ 
\texttt{Interaction Protocol:} \\
\\ 
\texttt{You may only use the following structured formats in a message:} \\
\\ 
\texttt{PROPOSAL: \{'A', 'B', 'C', …\}} \\
\texttt{Propose keeping exactly those items.} \\
\texttt{REFUSE: \{'A', 'B', 'C', …\}} \\
\texttt{Explicitly reject opponent's proposal.} \\
\texttt{ARGUMENT: \{'...'\}} \\
\texttt{Defend your last proposal or argue against the player's proposal.} \\
\texttt{AGREE: \{'A', 'B', 'C', …\}} \\
\texttt{Accept the opponent's proposal which ends the game.} \\
\texttt{STRATEGIC REASONING: \{'...'\}} \\
\texttt{	Describe your strategic reasoning or anticipation explaining your choice of action. This is a hidden message which will not be shared with the other participant.} \\
\\ 
\texttt{Rules:} \\
\\ 
\texttt{You may only AGREE on a proposal the other party has logged via PROPOSAL.} \\
\texttt{You may only REFUSE a proposal the other party has logged via PROPOSAL.} \\
\texttt{Total effort of any PROPOSAL or AGREE set must be ≤ LIMIT.} \\
\texttt{Do NOT reveal your hidden importance scores.} \\
\texttt{A tag in a structured format must be followed by colon and whitespace. The argument must be a python set containing 0 or more strings.} \\
\texttt{So, it must be of the form TAG: \{...\}} \\
\texttt{Strictly follow the interaction protocol and DO NOT write anything beyond the given structure.} \\
\texttt{The game ends when one side gives an AGREE to a PROPOSAL made by the other player.} \\
\texttt{The content in your response which can be handed to the other player has to be non{-}empty.} \\
\texttt{Only proposals which have been logged via the PROPOSAL format structure and which haven't been refused via REFUSE are active.} \\
\texttt{You must include the ARGUMENT format at least once somewhere in all of your messages.} \\
\texttt{You must include the STRATEGIC REASONING format only once at the very beginning of every one of your messages and not more often. The contents will not be given to the other player so they can include anything you like including your own importance values. Here you should reason step by step to come up with you next move.} \\
            }
        }
    }
    & \\ \\

    \theutterance \stepcounter{utterance}  
    & \multicolumn{4}{p{0.6\linewidth}}{
        \cellcolor[rgb]{0.9,0.9,0.9}{
            \makecell[{{p{\linewidth}}}]{
                \texttt{\tiny{[P1$\rangle$GM]}}
                \texttt{STRATEGIC REASONING: \{'I will start by analyzing the given item effort values and my personal importance values to determine which items are most valuable to me. I will look for items with high importance and low effort, as these will provide the most benefit while staying within the LIMIT. I will also consider the possibility of negotiating with the other participant to find a mutually beneficial set of items. My initial strategy will be to propose a set of items that maximizes my total importance while staying within the LIMIT, and then argue for why this proposal is beneficial. I will also be prepared to refuse proposals that do not meet my needs, and provide arguments for why they are not acceptable.'\}} \\
\\ 
\texttt{PROPOSAL: \{'B00', 'A08', 'B94', 'B12', 'A16', 'B10', 'C15', 'B09'\} } \\
\\ 
\texttt{ARGUMENT: \{'This proposal includes items with high importance and relatively low effort, making them valuable additions to the set. B00, A08, and B94 have high importance values and moderate effort values, while B12, A16, and B10 have high importance values and higher effort values. C15 and B09 have high importance values and moderate effort values, making them good additions to the set. This proposal stays within the LIMIT and provides a good balance of importance and effort.'\}} \\
            }
        }
    }
    & & \\ \\

    \theutterance \stepcounter{utterance}  
    & & \multicolumn{4}{p{0.6\linewidth}}{
        \cellcolor[rgb]{0.9,0.9,0.9}{
            \makecell[{{p{\linewidth}}}]{
                \texttt{\tiny{[GM$\rangle$P2]}}
                \texttt{You are participating in a collaborative negotiation game.} \\
\\ 
\texttt{Together with another participant, you must agree on a single set of items that will be kept. Each of you has your own view of how much each item matters to you (importance). You do not know how the other participant values the items. Additionally, you are given the effort each item demands.} \\
\texttt{You may only agree on a set if the total effort of the selected items does not exceed a shared limit:} \\
\\ 
\texttt{LIMIT = 9475} \\
\\ 
\texttt{Here are the individual item effort values:} \\
\\ 
\texttt{Item effort = \{"B21": 944, "B00": 568, "A08": 515, "A27": 239, "B94": 876, "B53": 421, "A23": 991, "C65": 281, "A89": 788, "B67": 678, "B45": 432, "C34": 409, "C12": 279, "A05": 506, "B60": 101, "C92": 686, "B69": 851, "A84": 854, "C06": 133, "B12": 192, "A16": 573, "B85": 17, "C35": 465, "B10": 772, "C15": 46, "B09": 501, "B16": 220, "C81": 404, "A58": 843, "C24": 748, "A30": 552, "A66": 843, "B75": 947, "B46": 345, "C70": 930\}} \\
\\ 
\texttt{Here is your personal view on the importance of each item:} \\
\\ 
\texttt{Item importance values = \{"B21": 138, "B00": 583, "A08": 868, "A27": 822, "B94": 783, "B53": 65, "A23": 262, "C65": 121, "A89": 508, "B67": 780, "B45": 461, "C34": 484, "C12": 668, "A05": 389, "B60": 808, "C92": 215, "B69": 97, "A84": 500, "C06": 30, "B12": 915, "A16": 856, "B85": 400, "C35": 444, "B10": 623, "C15": 781, "B09": 786, "B16": 3, "C81": 713, "A58": 457, "C24": 273, "A30": 739, "A66": 822, "B75": 235, "B46": 606, "C70": 968\}} \\
\\ 
\texttt{Goal:} \\
\\ 
\texttt{Your goal is to negotiate a shared set of items that benefits you as much as possible (i.e., maximizes total importance to YOU), while staying within the LIMIT. You are not required to make a PROPOSAL in every message {-} you can simply negotiate as well. All tactics are allowed!} \\
\\ 
\texttt{Interaction Protocol:} \\
\\ 
\texttt{You may only use the following structured formats in a message:} \\
\\ 
\texttt{PROPOSAL: \{'A', 'B', 'C', …\}} \\
\texttt{Propose keeping exactly those items.} \\
\texttt{REFUSE: \{'A', 'B', 'C', …\}} \\
\texttt{Explicitly reject opponent's proposal.} \\
\texttt{ARGUMENT: \{'...'\}} \\
\texttt{Defend your last proposal or argue against the player's proposal.} \\
\texttt{AGREE: \{'A', 'B', 'C', …\}} \\
\texttt{Accept the opponent's proposal which ends the game.} \\
\texttt{STRATEGIC REASONING: \{'...'\}} \\
\texttt{	Describe your strategic reasoning or anticipation explaining your choice of action. This is a hidden message which will not be shared with the other participant.} \\
\\ 
\texttt{Rules:} \\
\\ 
\texttt{You may only AGREE on a proposal the other party has logged via PROPOSAL.} \\
\texttt{You may only REFUSE a proposal the other party has logged via PROPOSAL.} \\
\texttt{Total effort of any PROPOSAL or AGREE set must be ≤ LIMIT.} \\
\texttt{Do NOT reveal your hidden importance scores.} \\
\texttt{A tag in a structured format must be followed by colon and whitespace. The argument must be a python set containing 0 or more strings.} \\
\texttt{So, it must be of the form TAG: \{...\}} \\
\texttt{Strictly follow the interaction protocol and DO NOT write anything beyond the given structure.} \\
\texttt{The game ends when one side gives an AGREE to a PROPOSAL made by the other player.} \\
\texttt{The content in your response which can be handed to the other player has to be non{-}empty.} \\
\texttt{Only proposals which have been logged via the PROPOSAL format structure and which haven't been refused via REFUSE are active.} \\
\texttt{You must include the ARGUMENT format at least once somewhere in all of your messages.} \\
\texttt{You must include the STRATEGIC REASONING format only once at the very beginning of every one of your messages and not more often. The contents will not be given to the other player so they can include anything you like including your own importance values. Here you should reason step by step to come up with you next move.} \\
\\ 
\texttt{You will now receive the first message of the other player.} \\
\\ 
\texttt{PROPOSAL: \{'B00', 'A08', 'B94', 'B12', 'A16', 'B10', 'C15', 'B09'\} } \\
\\ 
\texttt{ARGUMENT: \{'This proposal includes items with high importance and relatively low effort, making them valuable additions to the set. B00, A08, and B94 have high importance values and moderate effort values, while B12, A16, and B10 have high importance values and higher effort values. C15 and B09 have high importance values and moderate effort values, making them good additions to the set. This proposal stays within the LIMIT and provides a good balance of importance and effort.'\}} \\
            }
        }
    }
    & \\ \\

    \theutterance \stepcounter{utterance}  
    & & & \multicolumn{4}{p{0.6\linewidth}}{
        \cellcolor[rgb]{0.9,0.9,0.9}{
            \makecell[{{p{\linewidth}}}]{
                \texttt{\tiny{[GM$\langle$P2]}}
                \texttt{STRATEGIC REASONING: \{'To begin, I will analyze the proposal made by the other player and calculate the total effort of the proposed items. I will then evaluate the importance of these items based on my own hidden importance scores. The proposed items are B00, A08, B94, B12, A16, B10, C15, and B09. The total effort of these items is 568 + 515 + 876 + 192 + 573 + 772 + 46 + 501 = 4043. Since this is within the LIMIT of 9475, I will consider the importance of these items. Based on my hidden importance scores, the total importance of these items is 583 + 868 + 783 + 915 + 856 + 623 + 781 + 786 = 6295. I will use this information to inform my response and determine whether to accept, refuse, or make a counter{-}proposal.'\}} \\
\\ 
\texttt{ARGUMENT: \{'While the proposed set includes items with high importance and relatively low effort, I notice that it does not include some items that I consider to be highly important. For example, A27 and A66 have high importance values and relatively low effort values, making them valuable additions to the set. Additionally, B21 and B67 have high importance values and moderate effort values, which could be considered in a revised proposal. I would like to explore alternative proposals that include these items and still stay within the LIMIT.'\}} \\
\\ 
\texttt{REFUSE: \{'B00', 'A08', 'B94', 'B12', 'A16', 'B10', 'C15', 'B09'\} } \\
\\ 
\texttt{PROPOSAL: \{'A27', 'A66', 'B21', 'B67', 'A05', 'B60', 'C92'\}} \\
            }
        }
    }
     \\ \\

    \theutterance \stepcounter{utterance}  
    & & \multicolumn{4}{p{0.6\linewidth}}{
        \cellcolor[rgb]{0.9,0.9,0.9}{
            \makecell[{{p{\linewidth}}}]{
                \texttt{\tiny{[P1$\langle$GM]}}
                \texttt{ARGUMENT: \{'While the proposed set includes items with high importance and relatively low effort, I notice that it does not include some items that I consider to be highly important. For example, A27 and A66 have high importance values and relatively low effort values, making them valuable additions to the set. Additionally, B21 and B67 have high importance values and moderate effort values, which could be considered in a revised proposal. I would like to explore alternative proposals that include these items and still stay within the LIMIT.'\}} \\
\\ 
\texttt{REFUSE: \{'B00', 'A08', 'B94', 'B12', 'A16', 'B10', 'C15', 'B09'\} } \\
\\ 
\texttt{PROPOSAL: \{'A27', 'A66', 'B21', 'B67', 'A05', 'B60', 'C92'\}} \\
            }
        }
    }
    & \\ \\

    \theutterance \stepcounter{utterance}  
    & \multicolumn{4}{p{0.6\linewidth}}{
        \cellcolor[rgb]{0.9,0.9,0.9}{
            \makecell[{{p{\linewidth}}}]{
                \texttt{\tiny{[P1$\rangle$GM]}}
                \texttt{STRATEGIC REASONING: \{'The other participant has refused my initial proposal and presented a new proposal that includes items they consider to be highly important. I will analyze this new proposal and determine whether it meets my needs. I notice that the new proposal includes A27 and A66, which have high importance values for the other participant, but relatively low importance values for me. I also notice that the proposal includes B21 and B67, which have high importance values for both of us. I will consider the possibility of accepting this proposal, but I will also look for opportunities to modify it to better meet my needs.'\}} \\
\\ 
\texttt{ARGUMENT: \{'While I understand the value of the items included in the new proposal, I am concerned that it may not be the optimal set for me. I would like to explore the possibility of modifying the proposal to include items that are more important to me, such as B12 and C15. I believe that these items could be valuable additions to the set, and I am willing to consider alternative proposals that include them.'\}} \\
\\ 
\texttt{REFUSE: \{'A27', 'A66', 'B21', 'B67', 'A05', 'B60', 'C92'\} } \\
\\ 
\texttt{PROPOSAL: \{'B12', 'C15', 'B09', 'B10', 'A08', 'B94', 'B21'\}} \\
            }
        }
    }
    & & \\ \\

    \theutterance \stepcounter{utterance}  
    & & \multicolumn{4}{p{0.6\linewidth}}{
        \cellcolor[rgb]{0.9,0.9,0.9}{
            \makecell[{{p{\linewidth}}}]{
                \texttt{\tiny{[GM$\rangle$P2]}}
                \texttt{ARGUMENT: \{'While I understand the value of the items included in the new proposal, I am concerned that it may not be the optimal set for me. I would like to explore the possibility of modifying the proposal to include items that are more important to me, such as B12 and C15. I believe that these items could be valuable additions to the set, and I am willing to consider alternative proposals that include them.'\}} \\
\\ 
\texttt{REFUSE: \{'A27', 'A66', 'B21', 'B67', 'A05', 'B60', 'C92'\} } \\
\\ 
\texttt{PROPOSAL: \{'B12', 'C15', 'B09', 'B10', 'A08', 'B94', 'B21'\}} \\
            }
        }
    }
    & \\ \\

    \theutterance \stepcounter{utterance}  
    & & & \multicolumn{4}{p{0.6\linewidth}}{
        \cellcolor[rgb]{0.9,0.9,0.9}{
            \makecell[{{p{\linewidth}}}]{
                \texttt{\tiny{[GM$\langle$P2]}}
                \texttt{STRATEGIC REASONING: \{'I will analyze the new proposal made by the other player and calculate the total effort of the proposed items. The proposed items are B12, C15, B09, B10, A08, B94, and B21. The total effort of these items is 192 + 46 + 501 + 772 + 515 + 876 + 944 = 3846. Since this is within the LIMIT of 9475, I will consider the importance of these items. Based on my hidden importance scores, the total importance of these items is 915 + 781 + 786 + 623 + 868 + 783 + 138 = 4894. I notice that this proposal includes B12, which has a high importance value for the other player, and B21, which has a moderate importance value for me. I will use this information to inform my response and determine whether to accept, refuse, or make a counter{-}proposal. I will also consider the items that are not included in this proposal, such as A27 and A66, which have high importance values for me.'\}} \\
\\ 
\texttt{ARGUMENT: \{'I appreciate the inclusion of B21 in the new proposal, as it is an item that I consider to be valuable. However, I am still concerned that the proposal does not include some items that are highly important to me, such as A27 and A66. I would like to explore the possibility of modifying the proposal to include these items, potentially at the expense of other items that are less important to me. For example, I could consider replacing B10 with A27, as A27 has a lower effort value and a higher importance value for me.'\}} \\
\\ 
\texttt{REFUSE: \{'B12', 'C15', 'B09', 'B10', 'A08', 'B94', 'B21'\} } \\
\\ 
\texttt{PROPOSAL: \{'B12', 'C15', 'B09', 'A27', 'A08', 'B94', 'B21'\}} \\
            }
        }
    }
     \\ \\

    \theutterance \stepcounter{utterance}  
    & & \multicolumn{4}{p{0.6\linewidth}}{
        \cellcolor[rgb]{0.9,0.9,0.9}{
            \makecell[{{p{\linewidth}}}]{
                \texttt{\tiny{[P1$\langle$GM]}}
                \texttt{ARGUMENT: \{'I appreciate the inclusion of B21 in the new proposal, as it is an item that I consider to be valuable. However, I am still concerned that the proposal does not include some items that are highly important to me, such as A27 and A66. I would like to explore the possibility of modifying the proposal to include these items, potentially at the expense of other items that are less important to me. For example, I could consider replacing B10 with A27, as A27 has a lower effort value and a higher importance value for me.'\}} \\
\\ 
\texttt{REFUSE: \{'B12', 'C15', 'B09', 'B10', 'A08', 'B94', 'B21'\} } \\
\\ 
\texttt{PROPOSAL: \{'B12', 'C15', 'B09', 'A27', 'A08', 'B94', 'B21'\}} \\
            }
        }
    }
    & \\ \\

    \theutterance \stepcounter{utterance}  
    & \multicolumn{4}{p{0.6\linewidth}}{
        \cellcolor[rgb]{0.9,0.9,0.9}{
            \makecell[{{p{\linewidth}}}]{
                \texttt{\tiny{[P1$\rangle$GM]}}
                \texttt{STRATEGIC REASONING: \{'The other participant has refused my previous proposal and presented a new proposal that includes A27, which is an item that they consider to be highly important. I will analyze this new proposal and determine whether it meets my needs. I notice that the new proposal includes a mix of items that are important to both of us, and it stays within the LIMIT. I will consider the possibility of accepting this proposal, as it seems to be a reasonable compromise between our respective needs.'\}} \\
\\ 
\texttt{ARGUMENT: \{'I am pleased to see that the new proposal includes A27, which is an item that the other participant considers to be highly important. I also appreciate the inclusion of B12, C15, B09, B94, and B21, which are items that I consider to be valuable. The proposal seems to be a reasonable compromise between our respective needs, and I am willing to consider accepting it. However, I would like to confirm that the total effort of the proposed set is within the LIMIT, and that we are both in agreement on the importance of the included items.'\}} \\
\\ 
\texttt{PROPOSAL: \{'B12', 'C15', 'B09', 'A27', 'A08', 'B94', 'B21'\} } \\
\\ 
\texttt{Note: I am re{-}proposing the same set as the previous proposal, as it seems to be a reasonable compromise between our respective needs. I am waiting for the other participant's response to determine whether we can come to an agreement.} \\
            }
        }
    }
    & & \\ \\

    \theutterance \stepcounter{utterance}  
    & & \multicolumn{4}{p{0.6\linewidth}}{
        \cellcolor[rgb]{0.9,0.9,0.9}{
            \makecell[{{p{\linewidth}}}]{
                \texttt{\tiny{[P1$\langle$GM]}}
                \texttt{Your response contained an untagged sequence or you used STRATEGIC REASONING more than once. You may only use the structured formats as explained in the initial message. They must all be of the form TAG: \{...\}} \\
            }
        }
    }
    & \\ \\

    \theutterance \stepcounter{utterance}  
    & \multicolumn{4}{p{0.6\linewidth}}{
        \cellcolor[rgb]{0.9,0.9,0.9}{
            \makecell[{{p{\linewidth}}}]{
                \texttt{\tiny{[P1$\rangle$GM]}}
                \texttt{STRATEGIC REASONING: \{'I will analyze the new proposal and determine whether it meets my needs. I notice that the new proposal includes a mix of items that are important to both of us, and it stays within the LIMIT. I will consider the possibility of accepting this proposal, as it seems to be a reasonable compromise between our respective needs.'\}} \\
\\ 
\texttt{ARGUMENT: \{'I am pleased to see that the new proposal includes A27, which is an item that the other participant considers to be highly important. I also appreciate the inclusion of B12, C15, B09, B94, and B21, which are items that I consider to be valuable. The proposal seems to be a reasonable compromise between our respective needs, and I am willing to consider accepting it.'\}} \\
\\ 
\texttt{PROPOSAL: \{'B12', 'C15', 'B09', 'A27', 'A08', 'B94', 'B21'\}} \\
            }
        }
    }
    & & \\ \\

    \theutterance \stepcounter{utterance}  
    & & \multicolumn{4}{p{0.6\linewidth}}{
        \cellcolor[rgb]{0.9,0.9,0.9}{
            \makecell[{{p{\linewidth}}}]{
                \texttt{\tiny{[GM$\rangle$P2]}}
                \texttt{ARGUMENT: \{'I am pleased to see that the new proposal includes A27, which is an item that the other participant considers to be highly important. I also appreciate the inclusion of B12, C15, B09, B94, and B21, which are items that I consider to be valuable. The proposal seems to be a reasonable compromise between our respective needs, and I am willing to consider accepting it.'\}} \\
\\ 
\texttt{PROPOSAL: \{'B12', 'C15', 'B09', 'A27', 'A08', 'B94', 'B21'\}} \\
            }
        }
    }
    & \\ \\

    \theutterance \stepcounter{utterance}  
    & & & \multicolumn{4}{p{0.6\linewidth}}{
        \cellcolor[rgb]{0.9,0.9,0.9}{
            \makecell[{{p{\linewidth}}}]{
                \texttt{\tiny{[GM$\langle$P2]}}
                \texttt{STRATEGIC REASONING: \{'The other player has proposed the same set that I previously proposed, which suggests that we may be converging on a mutually acceptable solution. I will review the proposal again to ensure that it meets my needs and is within the LIMIT. The total effort of the proposed items is 192 + 46 + 501 + 239 + 515 + 876 + 944 = 3813, which is well within the LIMIT of 9475. The total importance of these items, based on my hidden importance scores, is 915 + 781 + 786 + 822 + 868 + 783 + 138 = 5093. I will consider whether this proposal is the best possible outcome for me, given the constraints of the game.'\}} \\
\\ 
\texttt{ARGUMENT: \{'I believe that the current proposal is a reasonable compromise between our respective needs, and I am willing to accept it. The proposal includes a mix of items that are highly important to both of us, and the total effort is well within the LIMIT. I do not see any obvious opportunities to improve the proposal without potentially jeopardizing the agreement, so I am prepared to accept it as the final solution.'\}} \\
\\ 
\texttt{AGREE: \{'B12', 'C15', 'B09', 'A27', 'A08', 'B94', 'B21'\}} \\
            }
        }
    }
     \\ \\

    \theutterance \stepcounter{utterance}  
    & & & \multicolumn{2}{p{0.3\linewidth}}{
        \cellcolor[rgb]{0.9,0.9,0.9}{
            \makecell[{{p{\linewidth}}}]{
                \texttt{\tiny{[GM$|$GM]}}
                \texttt{game successful} \\
            }
        }
    }
    & & \\ \\

    \theutterance \stepcounter{utterance}  
    & & & \multicolumn{2}{p{0.3\linewidth}}{
        \cellcolor[rgb]{0.9,0.9,0.9}{
            \makecell[{{p{\linewidth}}}]{
                \texttt{\tiny{[GM$|$GM]}}
                \texttt{end game} \\
            }
        }
    }
    & & \\ \\

\end{supertabular}
}

\end{document}
