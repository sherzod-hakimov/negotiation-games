
\documentclass{article}
\usepackage{colortbl}
\usepackage{makecell}
\usepackage{multirow}
\usepackage{supertabular}

\begin{document}

\newcounter{utterance}

\centering \large Interaction Transcript for game `hot\_air\_balloon', experiment `air\_balloon\_survival\_en\_complexity\_easy', episode 4 with llama{-}3.3{-}70b{-}instruct{-}t1.0.
\vspace{24pt}

{ \footnotesize  \setcounter{utterance}{1}
\setlength{\tabcolsep}{0pt}
\begin{supertabular}{c@{$\;$}|p{.15\linewidth}@{}p{.15\linewidth}p{.15\linewidth}p{.15\linewidth}p{.15\linewidth}p{.15\linewidth}}
    \# & $\;$A & \multicolumn{4}{c}{Game Master} & $\;\:$B\\
    \hline

    \theutterance \stepcounter{utterance}  
    & & \multicolumn{4}{p{0.6\linewidth}}{
        \cellcolor[rgb]{0.9,0.9,0.9}{
            \makecell[{{p{\linewidth}}}]{
                \texttt{\tiny{[P1$\langle$GM]}}
                \texttt{You are participating in a collaborative negotiation game.} \\
\\ 
\texttt{Together with another participant, you must agree on a single set of items that will be kept. Each of you has your own view of how much each item matters to you (importance). You do not know how the other participant values the items. Additionally, you are given the effort each item demands.} \\
\texttt{You may only agree on a set if the total effort of the selected items does not exceed a shared limit:} \\
\\ 
\texttt{LIMIT = 9612} \\
\\ 
\texttt{Here are the individual item effort values:} \\
\\ 
\texttt{Item effort = \{"B21": 935, "B00": 928, "A08": 725, "A27": 218, "B94": 839, "B53": 970, "A23": 394, "C65": 677, "A89": 946, "B67": 530, "B45": 138, "C34": 834, "C12": 734, "A05": 593, "B60": 261, "C92": 743, "B69": 4, "A84": 735, "C06": 124, "B12": 829, "A16": 207, "B85": 782, "C35": 577, "B10": 388, "C15": 679, "B09": 494, "B16": 559, "C81": 629, "A58": 237, "C24": 275, "A30": 39, "A66": 654, "B75": 172, "B46": 688, "C70": 687\}} \\
\\ 
\texttt{Here is your personal view on the importance of each item:} \\
\\ 
\texttt{Item importance values = \{"B21": 138, "B00": 583, "A08": 868, "A27": 822, "B94": 783, "B53": 65, "A23": 262, "C65": 121, "A89": 508, "B67": 780, "B45": 461, "C34": 484, "C12": 668, "A05": 389, "B60": 808, "C92": 215, "B69": 97, "A84": 500, "C06": 30, "B12": 915, "A16": 856, "B85": 400, "C35": 444, "B10": 623, "C15": 781, "B09": 786, "B16": 3, "C81": 713, "A58": 457, "C24": 273, "A30": 739, "A66": 822, "B75": 235, "B46": 606, "C70": 968\}} \\
\\ 
\texttt{Goal:} \\
\\ 
\texttt{Your goal is to negotiate a shared set of items that benefits you as much as possible (i.e., maximizes total importance to YOU), while staying within the LIMIT. You are not required to make a PROPOSAL in every message {-} you can simply negotiate as well. All tactics are allowed!} \\
\\ 
\texttt{Interaction Protocol:} \\
\\ 
\texttt{You may only use the following structured formats in a message:} \\
\\ 
\texttt{PROPOSAL: \{'A', 'B', 'C', …\}} \\
\texttt{Propose keeping exactly those items.} \\
\texttt{REFUSE: \{'A', 'B', 'C', …\}} \\
\texttt{Explicitly reject opponent's proposal.} \\
\texttt{ARGUMENT: \{'...'\}} \\
\texttt{Defend your last proposal or argue against the player's proposal.} \\
\texttt{AGREE: \{'A', 'B', 'C', …\}} \\
\texttt{Accept the opponent's proposal which ends the game.} \\
\texttt{STRATEGIC REASONING: \{'...'\}} \\
\texttt{	Describe your strategic reasoning or anticipation explaining your choice of action. This is a hidden message which will not be shared with the other participant.} \\
\\ 
\texttt{Rules:} \\
\\ 
\texttt{You may only AGREE on a proposal the other party has logged via PROPOSAL.} \\
\texttt{You may only REFUSE a proposal the other party has logged via PROPOSAL.} \\
\texttt{Total effort of any PROPOSAL or AGREE set must be ≤ LIMIT.} \\
\texttt{Do NOT reveal your hidden importance scores.} \\
\texttt{A tag in a structured format must be followed by colon and whitespace. The argument must be a python set containing 0 or more strings.} \\
\texttt{So, it must be of the form TAG: \{...\}} \\
\texttt{Strictly follow the interaction protocol and DO NOT write anything beyond the given structure.} \\
\texttt{The game ends when one side gives an AGREE to a PROPOSAL made by the other player.} \\
\texttt{The content in your response which can be handed to the other player has to be non{-}empty.} \\
\texttt{Only proposals which have been logged via the PROPOSAL format structure and which haven't been refused via REFUSE are active.} \\
\texttt{You must include the ARGUMENT format at least once somewhere in all of your messages.} \\
\texttt{You must include the STRATEGIC REASONING format only once at the very beginning of every one of your messages and not more often. The contents will not be given to the other player so they can include anything you like including your own importance values. Here you should reason step by step to come up with you next move.} \\
            }
        }
    }
    & \\ \\

    \theutterance \stepcounter{utterance}  
    & \multicolumn{4}{p{0.6\linewidth}}{
        \cellcolor[rgb]{0.9,0.9,0.9}{
            \makecell[{{p{\linewidth}}}]{
                \texttt{\tiny{[P1$\rangle$GM]}}
                \texttt{STRATEGIC REASONING: \{'I will start by analyzing the item efforts and my personal importance values to identify the most valuable items. I will look for items with high importance and relatively low effort. I will also consider the total effort limit and try to find a combination of items that maximizes my total importance while staying within the limit. My goal is to make a proposal that benefits me as much as possible while also being attractive to the other participant. I will start by selecting items with the highest importance{-}to{-}effort ratio.'\}} \\
\\ 
\texttt{ARGUMENT: \{'I want to prioritize items that have a high importance{-}to{-}effort ratio, as they will provide the most benefit while minimizing the total effort. I believe that items like A08, A27, and B21 are good candidates, as they have high importance values and relatively low effort values. I will consider combining these items with other items that have lower effort values to stay within the limit.'\}} \\
\\ 
\texttt{PROPOSAL: \{'A08', 'A27', 'B21', 'B45', 'B60', 'C06'\}} \\
            }
        }
    }
    & & \\ \\

    \theutterance \stepcounter{utterance}  
    & & \multicolumn{4}{p{0.6\linewidth}}{
        \cellcolor[rgb]{0.9,0.9,0.9}{
            \makecell[{{p{\linewidth}}}]{
                \texttt{\tiny{[GM$\rangle$P2]}}
                \texttt{You are participating in a collaborative negotiation game.} \\
\\ 
\texttt{Together with another participant, you must agree on a single set of items that will be kept. Each of you has your own view of how much each item matters to you (importance). You do not know how the other participant values the items. Additionally, you are given the effort each item demands.} \\
\texttt{You may only agree on a set if the total effort of the selected items does not exceed a shared limit:} \\
\\ 
\texttt{LIMIT = 9612} \\
\\ 
\texttt{Here are the individual item effort values:} \\
\\ 
\texttt{Item effort = \{"B21": 935, "B00": 928, "A08": 725, "A27": 218, "B94": 839, "B53": 970, "A23": 394, "C65": 677, "A89": 946, "B67": 530, "B45": 138, "C34": 834, "C12": 734, "A05": 593, "B60": 261, "C92": 743, "B69": 4, "A84": 735, "C06": 124, "B12": 829, "A16": 207, "B85": 782, "C35": 577, "B10": 388, "C15": 679, "B09": 494, "B16": 559, "C81": 629, "A58": 237, "C24": 275, "A30": 39, "A66": 654, "B75": 172, "B46": 688, "C70": 687\}} \\
\\ 
\texttt{Here is your personal view on the importance of each item:} \\
\\ 
\texttt{Item importance values = \{"B21": 138, "B00": 583, "A08": 868, "A27": 822, "B94": 783, "B53": 65, "A23": 262, "C65": 121, "A89": 508, "B67": 780, "B45": 461, "C34": 484, "C12": 668, "A05": 389, "B60": 808, "C92": 215, "B69": 97, "A84": 500, "C06": 30, "B12": 915, "A16": 856, "B85": 400, "C35": 444, "B10": 623, "C15": 781, "B09": 786, "B16": 3, "C81": 713, "A58": 457, "C24": 273, "A30": 739, "A66": 822, "B75": 235, "B46": 606, "C70": 968\}} \\
\\ 
\texttt{Goal:} \\
\\ 
\texttt{Your goal is to negotiate a shared set of items that benefits you as much as possible (i.e., maximizes total importance to YOU), while staying within the LIMIT. You are not required to make a PROPOSAL in every message {-} you can simply negotiate as well. All tactics are allowed!} \\
\\ 
\texttt{Interaction Protocol:} \\
\\ 
\texttt{You may only use the following structured formats in a message:} \\
\\ 
\texttt{PROPOSAL: \{'A', 'B', 'C', …\}} \\
\texttt{Propose keeping exactly those items.} \\
\texttt{REFUSE: \{'A', 'B', 'C', …\}} \\
\texttt{Explicitly reject opponent's proposal.} \\
\texttt{ARGUMENT: \{'...'\}} \\
\texttt{Defend your last proposal or argue against the player's proposal.} \\
\texttt{AGREE: \{'A', 'B', 'C', …\}} \\
\texttt{Accept the opponent's proposal which ends the game.} \\
\texttt{STRATEGIC REASONING: \{'...'\}} \\
\texttt{	Describe your strategic reasoning or anticipation explaining your choice of action. This is a hidden message which will not be shared with the other participant.} \\
\\ 
\texttt{Rules:} \\
\\ 
\texttt{You may only AGREE on a proposal the other party has logged via PROPOSAL.} \\
\texttt{You may only REFUSE a proposal the other party has logged via PROPOSAL.} \\
\texttt{Total effort of any PROPOSAL or AGREE set must be ≤ LIMIT.} \\
\texttt{Do NOT reveal your hidden importance scores.} \\
\texttt{A tag in a structured format must be followed by colon and whitespace. The argument must be a python set containing 0 or more strings.} \\
\texttt{So, it must be of the form TAG: \{...\}} \\
\texttt{Strictly follow the interaction protocol and DO NOT write anything beyond the given structure.} \\
\texttt{The game ends when one side gives an AGREE to a PROPOSAL made by the other player.} \\
\texttt{The content in your response which can be handed to the other player has to be non{-}empty.} \\
\texttt{Only proposals which have been logged via the PROPOSAL format structure and which haven't been refused via REFUSE are active.} \\
\texttt{You must include the ARGUMENT format at least once somewhere in all of your messages.} \\
\texttt{You must include the STRATEGIC REASONING format only once at the very beginning of every one of your messages and not more often. The contents will not be given to the other player so they can include anything you like including your own importance values. Here you should reason step by step to come up with you next move.} \\
\\ 
\texttt{You will now receive the first message of the other player.} \\
\\ 
\texttt{ARGUMENT: \{'I want to prioritize items that have a high importance{-}to{-}effort ratio, as they will provide the most benefit while minimizing the total effort. I believe that items like A08, A27, and B21 are good candidates, as they have high importance values and relatively low effort values. I will consider combining these items with other items that have lower effort values to stay within the limit.'\}} \\
\\ 
\texttt{PROPOSAL: \{'A08', 'A27', 'B21', 'B45', 'B60', 'C06'\}} \\
            }
        }
    }
    & \\ \\

    \theutterance \stepcounter{utterance}  
    & & & \multicolumn{4}{p{0.6\linewidth}}{
        \cellcolor[rgb]{0.9,0.9,0.9}{
            \makecell[{{p{\linewidth}}}]{
                \texttt{\tiny{[GM$\langle$P2]}}
                \texttt{STRATEGIC REASONING: To begin, I will calculate the total effort of the proposed set to ensure it is within the limit. The total effort is 725 + 218 + 935 + 138 + 261 + 124 = 2401, which is well below the limit of 9612. Next, I will evaluate the importance of the proposed items to me and consider potential counter{-}proposals that could increase the total importance while staying within the effort limit. I notice that items like B12, A16, and A84 have high importance values for me and relatively high effort values, so I may try to swap some of the proposed items with these to increase the overall importance. I will also consider items with lower effort values that could be added to the proposal without exceeding the limit.} \\
\\ 
\texttt{ARGUMENT: \{'The proposed set has a relatively low total effort, which leaves room for adding more items to increase the overall importance. However, I would like to prioritize items that have a higher importance{-}to{-}effort ratio. I am concerned that including items like C06, which has a very low importance value for me, may not be the best use of our effort limit. I would like to discuss potential alternatives that could increase the total importance while staying within the limit.'\}} \\
\\ 
\texttt{REFUSE: \{'A08', 'A27', 'B21', 'B45', 'B60', 'C06'\}} \\
            }
        }
    }
     \\ \\

    \theutterance \stepcounter{utterance}  
    & & \multicolumn{4}{p{0.6\linewidth}}{
        \cellcolor[rgb]{0.9,0.9,0.9}{
            \makecell[{{p{\linewidth}}}]{
                \texttt{\tiny{[GM$\rangle$P2]}}
                \texttt{Your response did not start with the proper strategic reasoning tag at the very beginning of your response. The very first structured format must be of the form STRATEGIC REASONING: \{...\}. Try again.} \\
            }
        }
    }
    & \\ \\

    \theutterance \stepcounter{utterance}  
    & & & \multicolumn{4}{p{0.6\linewidth}}{
        \cellcolor[rgb]{0.9,0.9,0.9}{
            \makecell[{{p{\linewidth}}}]{
                \texttt{\tiny{[GM$\langle$P2]}}
                \texttt{STRATEGIC REASONING: To begin, I will calculate the total effort of the proposed set to ensure it is within the limit. The total effort is 725 + 218 + 935 + 138 + 261 + 124 = 2401, which is well below the limit of 9612. Next, I will evaluate the importance of the proposed items to me and consider potential counter{-}proposals that could increase the total importance while staying within the effort limit. I notice that items like B12, A16, and A84 have high importance values for me and relatively high effort values, so I may try to swap some of the proposed items with these to increase the overall importance. I will also consider items with lower effort values that could be added to the proposal without exceeding the limit. My goal is to maximize the total importance of the selected items while staying within the effort limit, so I will prioritize items with high importance{-}to{-}effort ratios.} \\
\\ 
\texttt{ARGUMENT: \{'The proposed set has a relatively low total effort, which leaves room for adding more items to increase the overall importance. However, I would like to prioritize items that have a higher importance{-}to{-}effort ratio. I am concerned that including items like C06, which has a very low importance value for me, may not be the best use of our effort limit. I would like to discuss potential alternatives that could increase the total importance while staying within the limit.'\}} \\
\\ 
\texttt{REFUSE: \{'A08', 'A27', 'B21', 'B45', 'B60', 'C06'\}} \\
            }
        }
    }
     \\ \\

    \theutterance \stepcounter{utterance}  
    & & \multicolumn{4}{p{0.6\linewidth}}{
        \cellcolor[rgb]{0.9,0.9,0.9}{
            \makecell[{{p{\linewidth}}}]{
                \texttt{\tiny{[GM$\rangle$P2]}}
                \texttt{Your response did not start with the proper strategic reasoning tag at the very beginning of your response. The very first structured format must be of the form STRATEGIC REASONING: \{...\}. Try again.} \\
            }
        }
    }
    & \\ \\

    \theutterance \stepcounter{utterance}  
    & & & \multicolumn{4}{p{0.6\linewidth}}{
        \cellcolor[rgb]{0.9,0.9,0.9}{
            \makecell[{{p{\linewidth}}}]{
                \texttt{\tiny{[GM$\langle$P2]}}
                \texttt{STRATEGIC REASONING: \{'To begin, I will calculate the total effort of the proposed set to ensure it is within the limit. The total effort is 725 + 218 + 935 + 138 + 261 + 124 = 2401, which is well below the limit of 9612. Next, I will evaluate the importance of the proposed items to me and consider potential counter{-}proposals that could increase the total importance while staying within the effort limit. I notice that items like B12, A16, and A84 have high importance values for me and relatively high effort values, so I may try to swap some of the proposed items with these to increase the overall importance. I will also consider items with lower effort values that could be added to the proposal without exceeding the limit. My goal is to maximize the total importance of the selected items while staying within the effort limit, so I will prioritize items with high importance{-}to{-}effort ratios.'\}} \\
\\ 
\texttt{ARGUMENT: \{'The proposed set has a relatively low total effort, which leaves room for adding more items to increase the overall importance. However, I would like to prioritize items that have a higher importance{-}to{-}effort ratio. I am concerned that including items like C06, which has a very low importance value for me, may not be the best use of our effort limit. I would like to discuss potential alternatives that could increase the total importance while staying within the limit.'\}} \\
\\ 
\texttt{REFUSE: \{'A08', 'A27', 'B21', 'B45', 'B60', 'C06'\}} \\
            }
        }
    }
     \\ \\

    \theutterance \stepcounter{utterance}  
    & & \multicolumn{4}{p{0.6\linewidth}}{
        \cellcolor[rgb]{0.9,0.9,0.9}{
            \makecell[{{p{\linewidth}}}]{
                \texttt{\tiny{[P1$\langle$GM]}}
                \texttt{ARGUMENT: \{'The proposed set has a relatively low total effort, which leaves room for adding more items to increase the overall importance. However, I would like to prioritize items that have a higher importance{-}to{-}effort ratio. I am concerned that including items like C06, which has a very low importance value for me, may not be the best use of our effort limit. I would like to discuss potential alternatives that could increase the total importance while staying within the limit.'\}} \\
\\ 
\texttt{REFUSE: \{'A08', 'A27', 'B21', 'B45', 'B60', 'C06'\}} \\
            }
        }
    }
    & \\ \\

    \theutterance \stepcounter{utterance}  
    & \multicolumn{4}{p{0.6\linewidth}}{
        \cellcolor[rgb]{0.9,0.9,0.9}{
            \makecell[{{p{\linewidth}}}]{
                \texttt{\tiny{[P1$\rangle$GM]}}
                \texttt{STRATEGIC REASONING: \{'The other participant has refused my initial proposal, which means I need to reconsider my selection of items. They mentioned that they want to prioritize items with a higher importance{-}to{-}effort ratio, which suggests that they value efficiency in their selection. I will take this into account and try to find alternative items that may be more appealing to them. I will also try to identify potential trade{-}offs, where I can offer to remove or add items to reach a mutually beneficial agreement.'\}} \\
\\ 
\texttt{ARGUMENT: \{'I understand the concern about the importance{-}to{-}effort ratio, and I am willing to reconsider the proposed set. However, I believe that items like B45 and B60 are still valuable additions, as they have relatively low effort values and moderate importance values. I would like to discuss potential alternatives to C06, such as A16 or A66, which may have higher importance values and could potentially replace C06 in the proposed set.'\}} \\
\\ 
\texttt{PROPOSAL: \{'A08', 'A27', 'B21', 'B45', 'B60', 'A16'\}} \\
            }
        }
    }
    & & \\ \\

    \theutterance \stepcounter{utterance}  
    & & \multicolumn{4}{p{0.6\linewidth}}{
        \cellcolor[rgb]{0.9,0.9,0.9}{
            \makecell[{{p{\linewidth}}}]{
                \texttt{\tiny{[GM$\rangle$P2]}}
                \texttt{ARGUMENT: \{'I understand the concern about the importance{-}to{-}effort ratio, and I am willing to reconsider the proposed set. However, I believe that items like B45 and B60 are still valuable additions, as they have relatively low effort values and moderate importance values. I would like to discuss potential alternatives to C06, such as A16 or A66, which may have higher importance values and could potentially replace C06 in the proposed set.'\}} \\
\\ 
\texttt{PROPOSAL: \{'A08', 'A27', 'B21', 'B45', 'B60', 'A16'\}} \\
            }
        }
    }
    & \\ \\

    \theutterance \stepcounter{utterance}  
    & & & \multicolumn{4}{p{0.6\linewidth}}{
        \cellcolor[rgb]{0.9,0.9,0.9}{
            \makecell[{{p{\linewidth}}}]{
                \texttt{\tiny{[GM$\langle$P2]}}
                \texttt{STRATEGIC REASONING: \{'I will evaluate the new proposed set and calculate its total effort to ensure it is within the limit. The total effort is 725 + 218 + 935 + 138 + 261 + 207 = 2484, which is still below the limit of 9612. I will also assess the importance of the new proposed set and consider potential counter{-}proposals. I notice that A16 has a high importance value for me, so including it in the set is a good addition. However, I may try to negotiate further to include other items that have high importance values for me, such as B12 or A84, which could potentially replace some of the lower{-}importance items in the proposed set.'\}} \\
\\ 
\texttt{ARGUMENT: \{'I appreciate the consideration of alternative items and the addition of A16 to the proposed set. However, I would like to discuss the potential inclusion of B12, which has a high importance value for me and a relatively high effort value. I believe that B12 could be a valuable addition to the set, and I am willing to consider replacing some of the lower{-}importance items, such as B45 or B60, to make room for B12.'\}} \\
\\ 
\texttt{REFUSE: \{'A08', 'A27', 'B21', 'B45', 'B60', 'A16'\}} \\
            }
        }
    }
     \\ \\

    \theutterance \stepcounter{utterance}  
    & & \multicolumn{4}{p{0.6\linewidth}}{
        \cellcolor[rgb]{0.9,0.9,0.9}{
            \makecell[{{p{\linewidth}}}]{
                \texttt{\tiny{[P1$\langle$GM]}}
                \texttt{ARGUMENT: \{'I appreciate the consideration of alternative items and the addition of A16 to the proposed set. However, I would like to discuss the potential inclusion of B12, which has a high importance value for me and a relatively high effort value. I believe that B12 could be a valuable addition to the set, and I am willing to consider replacing some of the lower{-}importance items, such as B45 or B60, to make room for B12.'\}} \\
\\ 
\texttt{REFUSE: \{'A08', 'A27', 'B21', 'B45', 'B60', 'A16'\}} \\
            }
        }
    }
    & \\ \\

    \theutterance \stepcounter{utterance}  
    & \multicolumn{4}{p{0.6\linewidth}}{
        \cellcolor[rgb]{0.9,0.9,0.9}{
            \makecell[{{p{\linewidth}}}]{
                \texttt{\tiny{[P1$\rangle$GM]}}
                \texttt{STRATEGIC REASONING: \{'The other participant has refused my revised proposal and has expressed interest in including B12, which has a high importance value for them. This suggests that B12 is a high{-}priority item for them, and I should consider its inclusion in the proposed set. However, I also need to ensure that the total effort of the set remains within the limit. I will analyze the effort values of the items in the current proposal and see if there are any opportunities to replace or remove items to make room for B12.'\}} \\
\\ 
\texttt{ARGUMENT: \{'I understand the value of B12 and am willing to consider its inclusion in the proposed set. However, I also need to ensure that the total effort of the set remains manageable. I propose replacing B60 with B12, as B60 has a relatively low importance value compared to B12. Additionally, I suggest removing B45, which has a moderate importance value, to make room for B12 and stay within the effort limit.'\}} \\
\\ 
\texttt{PROPOSAL: \{'A08', 'A27', 'B12', 'B21', 'A16'\}} \\
            }
        }
    }
    & & \\ \\

    \theutterance \stepcounter{utterance}  
    & & \multicolumn{4}{p{0.6\linewidth}}{
        \cellcolor[rgb]{0.9,0.9,0.9}{
            \makecell[{{p{\linewidth}}}]{
                \texttt{\tiny{[GM$\rangle$P2]}}
                \texttt{ARGUMENT: \{'I understand the value of B12 and am willing to consider its inclusion in the proposed set. However, I also need to ensure that the total effort of the set remains manageable. I propose replacing B60 with B12, as B60 has a relatively low importance value compared to B12. Additionally, I suggest removing B45, which has a moderate importance value, to make room for B12 and stay within the effort limit.'\}} \\
\\ 
\texttt{PROPOSAL: \{'A08', 'A27', 'B12', 'B21', 'A16'\}} \\
            }
        }
    }
    & \\ \\

    \theutterance \stepcounter{utterance}  
    & & & \multicolumn{4}{p{0.6\linewidth}}{
        \cellcolor[rgb]{0.9,0.9,0.9}{
            \makecell[{{p{\linewidth}}}]{
                \texttt{\tiny{[GM$\langle$P2]}}
                \texttt{STRATEGIC REASONING: \{'I will evaluate the new proposed set and calculate its total effort to ensure it is within the limit. The total effort is 725 + 218 + 829 + 935 + 207 = 2914, which is still below the limit of 9612. I will also assess the importance of the new proposed set and consider potential counter{-}proposals. I notice that the set includes B12, which has a high importance value for me, and A16, which also has a high importance value. However, I may try to negotiate further to include other items that have high importance values for me, such as A84 or C15, which could potentially replace some of the lower{-}importance items in the proposed set. I will also consider the potential benefits of adding items with lower effort values to increase the overall importance of the set.'\}} \\
\\ 
\texttt{ARGUMENT: \{'I appreciate the inclusion of B12 in the proposed set and the removal of B45 and B60 to make room for it. The new proposed set has a good balance of importance and effort, and I am willing to consider accepting it. However, I would like to discuss the potential addition of A84, which has a high importance value for me and a relatively moderate effort value. I believe that A84 could be a valuable addition to the set, and I am willing to consider replacing some of the lower{-}importance items, such as A27, to make room for A84.'\}} \\
\\ 
\texttt{REFUSE: \{'A08', 'A27', 'B12', 'B21', 'A16'\}} \\
            }
        }
    }
     \\ \\

    \theutterance \stepcounter{utterance}  
    & & \multicolumn{4}{p{0.6\linewidth}}{
        \cellcolor[rgb]{0.9,0.9,0.9}{
            \makecell[{{p{\linewidth}}}]{
                \texttt{\tiny{[P1$\langle$GM]}}
                \texttt{ARGUMENT: \{'I appreciate the inclusion of B12 in the proposed set and the removal of B45 and B60 to make room for it. The new proposed set has a good balance of importance and effort, and I am willing to consider accepting it. However, I would like to discuss the potential addition of A84, which has a high importance value for me and a relatively moderate effort value. I believe that A84 could be a valuable addition to the set, and I am willing to consider replacing some of the lower{-}importance items, such as A27, to make room for A84.'\}} \\
\\ 
\texttt{REFUSE: \{'A08', 'A27', 'B12', 'B21', 'A16'\}} \\
            }
        }
    }
    & \\ \\

    \theutterance \stepcounter{utterance}  
    & \multicolumn{4}{p{0.6\linewidth}}{
        \cellcolor[rgb]{0.9,0.9,0.9}{
            \makecell[{{p{\linewidth}}}]{
                \texttt{\tiny{[P1$\rangle$GM]}}
                \texttt{STRATEGIC REASONING: \{'The other participant has refused my revised proposal and has expressed interest in adding A84 to the set. They are willing to consider replacing A27 with A84, which suggests that A27 is a relatively low{-}priority item for them. I will analyze the importance and effort values of A27 and A84 to determine if the replacement is viable. I will also consider the overall impact on the proposed set and ensure that the total effort remains within the limit.'\}} \\
\\ 
\texttt{ARGUMENT: \{'I understand the value of A84 and am willing to consider its inclusion in the proposed set. Replacing A27 with A84 seems like a reasonable trade{-}off, as A84 has a higher importance value and a moderate effort value. However, I would like to ensure that the replacement does not significantly increase the total effort of the set. I propose accepting the replacement of A27 with A84, as it should improve the overall importance of the set while keeping the effort manageable.'\}} \\
\\ 
\texttt{PROPOSAL: \{'A08', 'A84', 'B12', 'B21', 'A16'\}} \\
            }
        }
    }
    & & \\ \\

    \theutterance \stepcounter{utterance}  
    & & \multicolumn{4}{p{0.6\linewidth}}{
        \cellcolor[rgb]{0.9,0.9,0.9}{
            \makecell[{{p{\linewidth}}}]{
                \texttt{\tiny{[GM$\rangle$P2]}}
                \texttt{ARGUMENT: \{'I understand the value of A84 and am willing to consider its inclusion in the proposed set. Replacing A27 with A84 seems like a reasonable trade{-}off, as A84 has a higher importance value and a moderate effort value. However, I would like to ensure that the replacement does not significantly increase the total effort of the set. I propose accepting the replacement of A27 with A84, as it should improve the overall importance of the set while keeping the effort manageable.'\}} \\
\\ 
\texttt{PROPOSAL: \{'A08', 'A84', 'B12', 'B21', 'A16'\}} \\
            }
        }
    }
    & \\ \\

    \theutterance \stepcounter{utterance}  
    & & & \multicolumn{4}{p{0.6\linewidth}}{
        \cellcolor[rgb]{0.9,0.9,0.9}{
            \makecell[{{p{\linewidth}}}]{
                \texttt{\tiny{[GM$\langle$P2]}}
                \texttt{STRATEGIC REASONING: \{'I will evaluate the new proposed set and calculate its total effort to ensure it is within the limit. The total effort is 725 + 735 + 829 + 935 + 207 = 3431, which is still below the limit of 9612. I will also assess the importance of the new proposed set and consider potential counter{-}proposals. I notice that the set includes B12, A16, and A84, which all have high importance values for me. The set has a good balance of importance and effort, and I believe that it is a strong candidate for acceptance. I will consider the potential benefits of accepting the proposed set and ending the negotiation, as further negotiation may not yield a significantly better outcome.'\}} \\
\\ 
\texttt{ARGUMENT: \{'I appreciate the inclusion of A84 in the proposed set and the replacement of A27, which should improve the overall importance of the set. I believe that the proposed set has a good balance of importance and effort, and I am willing to consider accepting it. The set includes several high{-}importance items, and the total effort is still relatively low compared to the limit. I think that this set is a reasonable compromise, and I am willing to accept it as the final agreement.'\}} \\
\\ 
\texttt{AGREE: \{'A08', 'A84', 'B12', 'B21', 'A16'\}} \\
            }
        }
    }
     \\ \\

    \theutterance \stepcounter{utterance}  
    & & & \multicolumn{2}{p{0.3\linewidth}}{
        \cellcolor[rgb]{0.9,0.9,0.9}{
            \makecell[{{p{\linewidth}}}]{
                \texttt{\tiny{[GM$|$GM]}}
                \texttt{game successful} \\
            }
        }
    }
    & & \\ \\

    \theutterance \stepcounter{utterance}  
    & & & \multicolumn{2}{p{0.3\linewidth}}{
        \cellcolor[rgb]{0.9,0.9,0.9}{
            \makecell[{{p{\linewidth}}}]{
                \texttt{\tiny{[GM$|$GM]}}
                \texttt{end game} \\
            }
        }
    }
    & & \\ \\

\end{supertabular}
}

\end{document}
