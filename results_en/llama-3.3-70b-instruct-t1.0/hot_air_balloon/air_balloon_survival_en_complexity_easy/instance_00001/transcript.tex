
\documentclass{article}
\usepackage{colortbl}
\usepackage{makecell}
\usepackage{multirow}
\usepackage{supertabular}

\begin{document}

\newcounter{utterance}

\centering \large Interaction Transcript for game `hot\_air\_balloon', experiment `air\_balloon\_survival\_en\_complexity\_easy', episode 1 with llama{-}3.3{-}70b{-}instruct{-}t1.0.
\vspace{24pt}

{ \footnotesize  \setcounter{utterance}{1}
\setlength{\tabcolsep}{0pt}
\begin{supertabular}{c@{$\;$}|p{.15\linewidth}@{}p{.15\linewidth}p{.15\linewidth}p{.15\linewidth}p{.15\linewidth}p{.15\linewidth}}
    \# & $\;$A & \multicolumn{4}{c}{Game Master} & $\;\:$B\\
    \hline

    \theutterance \stepcounter{utterance}  
    & & \multicolumn{4}{p{0.6\linewidth}}{
        \cellcolor[rgb]{0.9,0.9,0.9}{
            \makecell[{{p{\linewidth}}}]{
                \texttt{\tiny{[P1$\langle$GM]}}
                \texttt{You are participating in a collaborative negotiation game.} \\
\\ 
\texttt{Together with another participant, you must agree on a single set of items that will be kept. Each of you has your own view of how much each item matters to you (importance). You do not know how the other participant values the items. Additionally, you are given the effort each item demands.} \\
\texttt{You may only agree on a set if the total effort of the selected items does not exceed a shared limit:} \\
\\ 
\texttt{LIMIT = 8170} \\
\\ 
\texttt{Here are the individual item effort values:} \\
\\ 
\texttt{Item effort = \{"B21": 594, "B00": 551, "A08": 634, "A27": 204, "B94": 557, "B53": 440, "A23": 734, "C65": 893, "A89": 246, "B67": 592, "B45": 143, "C34": 568, "C12": 472, "A05": 401, "B60": 729, "C92": 201, "B69": 85, "A84": 641, "C06": 958, "B12": 80, "A16": 157, "B85": 806, "C35": 684, "B10": 59, "C15": 31, "B09": 765, "B16": 416, "C81": 392, "A58": 428, "C24": 699, "A30": 141, "A66": 606, "B75": 612, "B46": 133, "C70": 689\}} \\
\\ 
\texttt{Here is your personal view on the importance of each item:} \\
\\ 
\texttt{Item importance values = \{"B21": 138, "B00": 583, "A08": 868, "A27": 822, "B94": 783, "B53": 65, "A23": 262, "C65": 121, "A89": 508, "B67": 780, "B45": 461, "C34": 484, "C12": 668, "A05": 389, "B60": 808, "C92": 215, "B69": 97, "A84": 500, "C06": 30, "B12": 915, "A16": 856, "B85": 400, "C35": 444, "B10": 623, "C15": 781, "B09": 786, "B16": 3, "C81": 713, "A58": 457, "C24": 273, "A30": 739, "A66": 822, "B75": 235, "B46": 606, "C70": 105\}} \\
\\ 
\texttt{Goal:} \\
\\ 
\texttt{Your goal is to negotiate a shared set of items that benefits you as much as possible (i.e., maximizes total importance to YOU), while staying within the LIMIT. You are not required to make a PROPOSAL in every message {-} you can simply negotiate as well. All tactics are allowed!} \\
\\ 
\texttt{Interaction Protocol:} \\
\\ 
\texttt{You may only use the following structured formats in a message:} \\
\\ 
\texttt{PROPOSAL: \{'A', 'B', 'C', …\}} \\
\texttt{Propose keeping exactly those items.} \\
\texttt{REFUSE: \{'A', 'B', 'C', …\}} \\
\texttt{Explicitly reject opponent's proposal.} \\
\texttt{ARGUMENT: \{'...'\}} \\
\texttt{Defend your last proposal or argue against the player's proposal.} \\
\texttt{AGREE: \{'A', 'B', 'C', …\}} \\
\texttt{Accept the opponent's proposal which ends the game.} \\
\texttt{STRATEGIC REASONING: \{'...'\}} \\
\texttt{	Describe your strategic reasoning or anticipation explaining your choice of action. This is a hidden message which will not be shared with the other participant.} \\
\\ 
\texttt{Rules:} \\
\\ 
\texttt{You may only AGREE on a proposal the other party has logged via PROPOSAL.} \\
\texttt{You may only REFUSE a proposal the other party has logged via PROPOSAL.} \\
\texttt{Total effort of any PROPOSAL or AGREE set must be ≤ LIMIT.} \\
\texttt{Do NOT reveal your hidden importance scores.} \\
\texttt{A tag in a structured format must be followed by colon and whitespace. The argument must be a python set containing 0 or more strings.} \\
\texttt{So, it must be of the form TAG: \{...\}} \\
\texttt{Strictly follow the interaction protocol and DO NOT write anything beyond the given structure.} \\
\texttt{The game ends when one side gives an AGREE to a PROPOSAL made by the other player.} \\
\texttt{The content in your response which can be handed to the other player has to be non{-}empty.} \\
\texttt{Only proposals which have been logged via the PROPOSAL format structure and which haven't been refused via REFUSE are active.} \\
\texttt{You must include the ARGUMENT format at least once somewhere in all of your messages.} \\
\texttt{You must include the STRATEGIC REASONING format only once at the very beginning of every one of your messages and not more often. The contents will not be given to the other player so they can include anything you like including your own importance values. Here you should reason step by step to come up with you next move.} \\
            }
        }
    }
    & \\ \\

    \theutterance \stepcounter{utterance}  
    & \multicolumn{4}{p{0.6\linewidth}}{
        \cellcolor[rgb]{0.9,0.9,0.9}{
            \makecell[{{p{\linewidth}}}]{
                \texttt{\tiny{[P1$\rangle$GM]}}
                \texttt{STRATEGIC REASONING: \{'To start the negotiation, I will first calculate the total effort and importance of each item to prioritize them. I will then select a subset of items with high importance and low effort to propose. My goal is to maximize the total importance while staying within the LIMIT. I will also keep in mind that the other player may have different importance values, so I will be prepared to adjust my proposal based on their responses.'\}} \\
\\ 
\texttt{ARGUMENT: \{'I will start by proposing a set of items that have high importance to me and relatively low effort, to maximize the total importance while minimizing the total effort. This will give us a good starting point for the negotiation.'\}} \\
\\ 
\texttt{PROPOSAL: \{'B00', 'A08', 'B94', 'B67', 'B60', 'C15', 'B09', 'A30', 'A66'\}} \\
            }
        }
    }
    & & \\ \\

    \theutterance \stepcounter{utterance}  
    & & \multicolumn{4}{p{0.6\linewidth}}{
        \cellcolor[rgb]{0.9,0.9,0.9}{
            \makecell[{{p{\linewidth}}}]{
                \texttt{\tiny{[GM$\rangle$P2]}}
                \texttt{You are participating in a collaborative negotiation game.} \\
\\ 
\texttt{Together with another participant, you must agree on a single set of items that will be kept. Each of you has your own view of how much each item matters to you (importance). You do not know how the other participant values the items. Additionally, you are given the effort each item demands.} \\
\texttt{You may only agree on a set if the total effort of the selected items does not exceed a shared limit:} \\
\\ 
\texttt{LIMIT = 8170} \\
\\ 
\texttt{Here are the individual item effort values:} \\
\\ 
\texttt{Item effort = \{"B21": 594, "B00": 551, "A08": 634, "A27": 204, "B94": 557, "B53": 440, "A23": 734, "C65": 893, "A89": 246, "B67": 592, "B45": 143, "C34": 568, "C12": 472, "A05": 401, "B60": 729, "C92": 201, "B69": 85, "A84": 641, "C06": 958, "B12": 80, "A16": 157, "B85": 806, "C35": 684, "B10": 59, "C15": 31, "B09": 765, "B16": 416, "C81": 392, "A58": 428, "C24": 699, "A30": 141, "A66": 606, "B75": 612, "B46": 133, "C70": 689\}} \\
\\ 
\texttt{Here is your personal view on the importance of each item:} \\
\\ 
\texttt{Item importance values = \{"B21": 138, "B00": 583, "A08": 868, "A27": 822, "B94": 783, "B53": 65, "A23": 262, "C65": 121, "A89": 508, "B67": 780, "B45": 461, "C34": 484, "C12": 668, "A05": 389, "B60": 808, "C92": 215, "B69": 97, "A84": 500, "C06": 30, "B12": 915, "A16": 856, "B85": 400, "C35": 444, "B10": 623, "C15": 781, "B09": 786, "B16": 3, "C81": 713, "A58": 457, "C24": 273, "A30": 739, "A66": 822, "B75": 235, "B46": 606, "C70": 105\}} \\
\\ 
\texttt{Goal:} \\
\\ 
\texttt{Your goal is to negotiate a shared set of items that benefits you as much as possible (i.e., maximizes total importance to YOU), while staying within the LIMIT. You are not required to make a PROPOSAL in every message {-} you can simply negotiate as well. All tactics are allowed!} \\
\\ 
\texttt{Interaction Protocol:} \\
\\ 
\texttt{You may only use the following structured formats in a message:} \\
\\ 
\texttt{PROPOSAL: \{'A', 'B', 'C', …\}} \\
\texttt{Propose keeping exactly those items.} \\
\texttt{REFUSE: \{'A', 'B', 'C', …\}} \\
\texttt{Explicitly reject opponent's proposal.} \\
\texttt{ARGUMENT: \{'...'\}} \\
\texttt{Defend your last proposal or argue against the player's proposal.} \\
\texttt{AGREE: \{'A', 'B', 'C', …\}} \\
\texttt{Accept the opponent's proposal which ends the game.} \\
\texttt{STRATEGIC REASONING: \{'...'\}} \\
\texttt{	Describe your strategic reasoning or anticipation explaining your choice of action. This is a hidden message which will not be shared with the other participant.} \\
\\ 
\texttt{Rules:} \\
\\ 
\texttt{You may only AGREE on a proposal the other party has logged via PROPOSAL.} \\
\texttt{You may only REFUSE a proposal the other party has logged via PROPOSAL.} \\
\texttt{Total effort of any PROPOSAL or AGREE set must be ≤ LIMIT.} \\
\texttt{Do NOT reveal your hidden importance scores.} \\
\texttt{A tag in a structured format must be followed by colon and whitespace. The argument must be a python set containing 0 or more strings.} \\
\texttt{So, it must be of the form TAG: \{...\}} \\
\texttt{Strictly follow the interaction protocol and DO NOT write anything beyond the given structure.} \\
\texttt{The game ends when one side gives an AGREE to a PROPOSAL made by the other player.} \\
\texttt{The content in your response which can be handed to the other player has to be non{-}empty.} \\
\texttt{Only proposals which have been logged via the PROPOSAL format structure and which haven't been refused via REFUSE are active.} \\
\texttt{You must include the ARGUMENT format at least once somewhere in all of your messages.} \\
\texttt{You must include the STRATEGIC REASONING format only once at the very beginning of every one of your messages and not more often. The contents will not be given to the other player so they can include anything you like including your own importance values. Here you should reason step by step to come up with you next move.} \\
\\ 
\texttt{You will now receive the first message of the other player.} \\
\\ 
\texttt{ARGUMENT: \{'I will start by proposing a set of items that have high importance to me and relatively low effort, to maximize the total importance while minimizing the total effort. This will give us a good starting point for the negotiation.'\}} \\
\\ 
\texttt{PROPOSAL: \{'B00', 'A08', 'B94', 'B67', 'B60', 'C15', 'B09', 'A30', 'A66'\}} \\
            }
        }
    }
    & \\ \\

    \theutterance \stepcounter{utterance}  
    & & & \multicolumn{4}{p{0.6\linewidth}}{
        \cellcolor[rgb]{0.9,0.9,0.9}{
            \makecell[{{p{\linewidth}}}]{
                \texttt{\tiny{[GM$\langle$P2]}}
                \texttt{STRATEGIC REASONING: \{'I need to evaluate the proposal made by the other player and determine the total effort and importance of the proposed items. I will calculate the total effort of the proposed items: 551 + 634 + 557 + 592 + 729 + 781 + 765 + 141 + 606 = 5556. This is below the limit of 8170. Next, I will evaluate the importance of these items to me: 583 + 868 + 783 + 780 + 808 + 781 + 786 + 739 + 822 = 6210. This is a significant amount of importance. However, I need to consider that this proposal may not be the best for me, and I should try to negotiate for a better set of items. I will refuse this proposal and make a counter{-}proposal that includes items with high importance to me and relatively low effort.'\}} \\
\\ 
\texttt{REFUSE: \{'B00', 'A08', 'B94', 'B67', 'B60', 'C15', 'B09', 'A30', 'A66'\}} \\
\\ 
\texttt{ARGUMENT: \{'The proposed set of items has a high total effort and may not be the most efficient use of our limit. I believe we can find a better set of items that maximizes the total importance while minimizing the total effort. I will make a counter{-}proposal that includes items with high importance to me and relatively low effort.'\}} \\
\\ 
\texttt{PROPOSAL: \{'A27', 'B12', 'A16', 'A30', 'A66', 'B75', 'B46', 'C12', 'A58'\}} \\
            }
        }
    }
     \\ \\

    \theutterance \stepcounter{utterance}  
    & & \multicolumn{4}{p{0.6\linewidth}}{
        \cellcolor[rgb]{0.9,0.9,0.9}{
            \makecell[{{p{\linewidth}}}]{
                \texttt{\tiny{[P1$\langle$GM]}}
                \texttt{REFUSE: \{'B00', 'A08', 'B94', 'B67', 'B60', 'C15', 'B09', 'A30', 'A66'\}} \\
\\ 
\texttt{ARGUMENT: \{'The proposed set of items has a high total effort and may not be the most efficient use of our limit. I believe we can find a better set of items that maximizes the total importance while minimizing the total effort. I will make a counter{-}proposal that includes items with high importance to me and relatively low effort.'\}} \\
\\ 
\texttt{PROPOSAL: \{'A27', 'B12', 'A16', 'A30', 'A66', 'B75', 'B46', 'C12', 'A58'\}} \\
            }
        }
    }
    & \\ \\

    \theutterance \stepcounter{utterance}  
    & \multicolumn{4}{p{0.6\linewidth}}{
        \cellcolor[rgb]{0.9,0.9,0.9}{
            \makecell[{{p{\linewidth}}}]{
                \texttt{\tiny{[P1$\rangle$GM]}}
                \texttt{STRATEGIC REASONING: \{'The other player has refused my initial proposal and made a counter{-}proposal. I will evaluate the counter{-}proposal to determine if it is acceptable to me. I will also consider making adjustments to the proposal to better meet my own needs while still being acceptable to the other player. I notice that the counter{-}proposal includes some items with high importance to me, such as A30 and A66, but also includes some items with lower importance. I will consider adding or removing items to optimize the proposal.'\}} \\
\\ 
\texttt{ARGUMENT: \{'The counter{-}proposal includes some items that are important to me, but also some that are not as important. I believe we can refine the proposal to better meet both of our needs. I will refuse the counter{-}proposal and make a new proposal that builds on the ideas presented.'\}} \\
\\ 
\texttt{REFUSE: \{'A27', 'B12', 'A16', 'A30', 'A66', 'B75', 'B46', 'C12', 'A58'\}} \\
\\ 
\texttt{PROPOSAL: \{'A08', 'B94', 'B67', 'B60', 'A30', 'A66', 'B10', 'C15'\}} \\
            }
        }
    }
    & & \\ \\

    \theutterance \stepcounter{utterance}  
    & & \multicolumn{4}{p{0.6\linewidth}}{
        \cellcolor[rgb]{0.9,0.9,0.9}{
            \makecell[{{p{\linewidth}}}]{
                \texttt{\tiny{[GM$\rangle$P2]}}
                \texttt{ARGUMENT: \{'The counter{-}proposal includes some items that are important to me, but also some that are not as important. I believe we can refine the proposal to better meet both of our needs. I will refuse the counter{-}proposal and make a new proposal that builds on the ideas presented.'\}} \\
\\ 
\texttt{REFUSE: \{'A27', 'B12', 'A16', 'A30', 'A66', 'B75', 'B46', 'C12', 'A58'\}} \\
\\ 
\texttt{PROPOSAL: \{'A08', 'B94', 'B67', 'B60', 'A30', 'A66', 'B10', 'C15'\}} \\
            }
        }
    }
    & \\ \\

    \theutterance \stepcounter{utterance}  
    & & & \multicolumn{4}{p{0.6\linewidth}}{
        \cellcolor[rgb]{0.9,0.9,0.9}{
            \makecell[{{p{\linewidth}}}]{
                \texttt{\tiny{[GM$\langle$P2]}}
                \texttt{STRATEGIC REASONING: \{'I need to evaluate the new proposal made by the other player and determine the total effort and importance of the proposed items. I will calculate the total effort of the proposed items: 634 + 557 + 592 + 729 + 141 + 606 + 623 + 781 = 4963. This is below the limit of 8170. Next, I will evaluate the importance of these items to me: 868 + 783 + 780 + 808 + 739 + 822 + 623 + 781 = 6096. This is a significant amount of importance. I notice that this proposal includes some of the most important items to me, and the total effort is relatively low. I will consider accepting this proposal, but I also want to see if I can negotiate for an even better set of items. I will make an argument to defend my interests and see if the other player is willing to make any further concessions.'\}} \\
\\ 
\texttt{ARGUMENT: \{'The new proposal includes many items that are important to me, and the total effort is relatively low. However, I believe we can still refine the proposal to better meet both of our needs. I am willing to consider accepting this proposal, but I would like to discuss possible alternatives that could benefit both parties. I would like to explore the possibility of adding or substituting items to increase the total importance while keeping the total effort below the limit.'\}} \\
\\ 
\texttt{PROPOSAL: \{'A08', 'B94', 'B67', 'B60', 'A30', 'A66', 'B10', 'C15', 'A05'\}} \\
            }
        }
    }
     \\ \\

    \theutterance \stepcounter{utterance}  
    & & \multicolumn{4}{p{0.6\linewidth}}{
        \cellcolor[rgb]{0.9,0.9,0.9}{
            \makecell[{{p{\linewidth}}}]{
                \texttt{\tiny{[P1$\langle$GM]}}
                \texttt{ARGUMENT: \{'The new proposal includes many items that are important to me, and the total effort is relatively low. However, I believe we can still refine the proposal to better meet both of our needs. I am willing to consider accepting this proposal, but I would like to discuss possible alternatives that could benefit both parties. I would like to explore the possibility of adding or substituting items to increase the total importance while keeping the total effort below the limit.'\}} \\
\\ 
\texttt{PROPOSAL: \{'A08', 'B94', 'B67', 'B60', 'A30', 'A66', 'B10', 'C15', 'A05'\}} \\
            }
        }
    }
    & \\ \\

    \theutterance \stepcounter{utterance}  
    & \multicolumn{4}{p{0.6\linewidth}}{
        \cellcolor[rgb]{0.9,0.9,0.9}{
            \makecell[{{p{\linewidth}}}]{
                \texttt{\tiny{[P1$\rangle$GM]}}
                \texttt{STRATEGIC REASONING: \{'The other player has made a new proposal that builds on my previous one, and has added item A05. I will evaluate the new proposal to determine if it is acceptable to me. I will consider the trade{-}offs between the added item and the potential benefits. I notice that item A05 has a moderate importance to me and a relatively low effort, so it may be a good addition to the proposal. However, I will also consider the potential impact on the total effort and the limit.'\}} \\
\\ 
\texttt{ARGUMENT: \{'The addition of item A05 to the proposal is an interesting suggestion. I believe it could be a valuable addition, but I would like to explore other possibilities as well. I am willing to consider accepting this proposal, but I would like to discuss possible alternatives that could benefit both parties. I will refuse the proposal for now and make a counter{-}proposal that includes item A05, but also considers other items that could be added or substituted to increase the total importance.'\}} \\
\\ 
\texttt{REFUSE: \{'A08', 'B94', 'B67', 'B60', 'A30', 'A66', 'B10', 'C15', 'A05'\}} \\
\\ 
\texttt{PROPOSAL: \{'A08', 'B94', 'B67', 'B60', 'A30', 'A66', 'B10', 'C15', 'A05', 'B12'\}} \\
            }
        }
    }
    & & \\ \\

    \theutterance \stepcounter{utterance}  
    & & \multicolumn{4}{p{0.6\linewidth}}{
        \cellcolor[rgb]{0.9,0.9,0.9}{
            \makecell[{{p{\linewidth}}}]{
                \texttt{\tiny{[GM$\rangle$P2]}}
                \texttt{ARGUMENT: \{'The addition of item A05 to the proposal is an interesting suggestion. I believe it could be a valuable addition, but I would like to explore other possibilities as well. I am willing to consider accepting this proposal, but I would like to discuss possible alternatives that could benefit both parties. I will refuse the proposal for now and make a counter{-}proposal that includes item A05, but also considers other items that could be added or substituted to increase the total importance.'\}} \\
\\ 
\texttt{REFUSE: \{'A08', 'B94', 'B67', 'B60', 'A30', 'A66', 'B10', 'C15', 'A05'\}} \\
\\ 
\texttt{PROPOSAL: \{'A08', 'B94', 'B67', 'B60', 'A30', 'A66', 'B10', 'C15', 'A05', 'B12'\}} \\
            }
        }
    }
    & \\ \\

    \theutterance \stepcounter{utterance}  
    & & & \multicolumn{4}{p{0.6\linewidth}}{
        \cellcolor[rgb]{0.9,0.9,0.9}{
            \makecell[{{p{\linewidth}}}]{
                \texttt{\tiny{[GM$\langle$P2]}}
                \texttt{STRATEGIC REASONING: \{'I need to evaluate the new proposal made by the other player and determine the total effort and importance of the proposed items. I will calculate the total effort of the proposed items: 634 + 557 + 592 + 729 + 141 + 606 + 623 + 781 + 401 + 80 = 5544. This is below the limit of 8170. Next, I will evaluate the importance of these items to me: 868 + 783 + 780 + 808 + 739 + 822 + 623 + 781 + 389 + 915 = 6598. This is a significant amount of importance. I notice that this proposal includes item B12, which has very high importance to me. I am inclined to accept this proposal, but I want to make sure that I am not conceding too much. I will make an argument to defend my interests and see if the other player is willing to make any further concessions, but I am also prepared to accept this proposal if necessary.'\}} \\
\\ 
\texttt{ARGUMENT: \{'The addition of item B12 to the proposal is a significant concession on the part of the other player, and I appreciate the willingness to consider my interests. I believe that this proposal is a good compromise, and I am willing to consider accepting it. However, I would like to confirm that this is the final proposal and that there are no further concessions to be made. If this is the case, I am prepared to accept the proposal and bring the negotiation to a close.'\}} \\
\\ 
\texttt{PROPOSAL: \{'A08', 'B94', 'B67', 'B60', 'A30', 'A66', 'B10', 'C15', 'A05', 'B12'\}} \\
            }
        }
    }
     \\ \\

    \theutterance \stepcounter{utterance}  
    & & \multicolumn{4}{p{0.6\linewidth}}{
        \cellcolor[rgb]{0.9,0.9,0.9}{
            \makecell[{{p{\linewidth}}}]{
                \texttt{\tiny{[P1$\langle$GM]}}
                \texttt{ARGUMENT: \{'The addition of item B12 to the proposal is a significant concession on the part of the other player, and I appreciate the willingness to consider my interests. I believe that this proposal is a good compromise, and I am willing to consider accepting it. However, I would like to confirm that this is the final proposal and that there are no further concessions to be made. If this is the case, I am prepared to accept the proposal and bring the negotiation to a close.'\}} \\
\\ 
\texttt{PROPOSAL: \{'A08', 'B94', 'B67', 'B60', 'A30', 'A66', 'B10', 'C15', 'A05', 'B12'\}} \\
            }
        }
    }
    & \\ \\

    \theutterance \stepcounter{utterance}  
    & \multicolumn{4}{p{0.6\linewidth}}{
        \cellcolor[rgb]{0.9,0.9,0.9}{
            \makecell[{{p{\linewidth}}}]{
                \texttt{\tiny{[P1$\rangle$GM]}}
                \texttt{STRATEGIC REASONING: \{'The other player has accepted the addition of item B12 to the proposal and has indicated a willingness to consider accepting the proposal. I believe that this is a good sign, and that we may be close to reaching an agreement. I will review the proposal one last time to ensure that it meets my needs, and then consider accepting it. I notice that the proposal includes a good balance of items with high importance to me, and that the total effort is still below the limit. I am willing to accept this proposal and bring the negotiation to a close.'\}} \\
\\ 
\texttt{ARGUMENT: \{'I believe that the current proposal is a good compromise that meets the needs of both parties. I am willing to accept this proposal and bring the negotiation to a close. I appreciate the other player's willingness to consider my interests and make concessions to reach an agreement.'\}} \\
\\ 
\texttt{AGREE: \{'A08', 'B94', 'B67', 'B60', 'A30', 'A66', 'B10', 'C15', 'A05', 'B12'\}} \\
            }
        }
    }
    & & \\ \\

    \theutterance \stepcounter{utterance}  
    & & & \multicolumn{2}{p{0.3\linewidth}}{
        \cellcolor[rgb]{0.9,0.9,0.9}{
            \makecell[{{p{\linewidth}}}]{
                \texttt{\tiny{[GM$|$GM]}}
                \texttt{game successful} \\
            }
        }
    }
    & & \\ \\

    \theutterance \stepcounter{utterance}  
    & & & \multicolumn{2}{p{0.3\linewidth}}{
        \cellcolor[rgb]{0.9,0.9,0.9}{
            \makecell[{{p{\linewidth}}}]{
                \texttt{\tiny{[GM$|$GM]}}
                \texttt{end game} \\
            }
        }
    }
    & & \\ \\

\end{supertabular}
}

\end{document}
