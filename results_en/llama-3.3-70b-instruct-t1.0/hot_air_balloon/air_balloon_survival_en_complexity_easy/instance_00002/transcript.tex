
\documentclass{article}
\usepackage{colortbl}
\usepackage{makecell}
\usepackage{multirow}
\usepackage{supertabular}

\begin{document}

\newcounter{utterance}

\centering \large Interaction Transcript for game `hot\_air\_balloon', experiment `air\_balloon\_survival\_en\_complexity\_easy', episode 2 with llama{-}3.3{-}70b{-}instruct{-}t1.0.
\vspace{24pt}

{ \footnotesize  \setcounter{utterance}{1}
\setlength{\tabcolsep}{0pt}
\begin{supertabular}{c@{$\;$}|p{.15\linewidth}@{}p{.15\linewidth}p{.15\linewidth}p{.15\linewidth}p{.15\linewidth}p{.15\linewidth}}
    \# & $\;$A & \multicolumn{4}{c}{Game Master} & $\;\:$B\\
    \hline

    \theutterance \stepcounter{utterance}  
    & & \multicolumn{4}{p{0.6\linewidth}}{
        \cellcolor[rgb]{0.9,0.9,0.9}{
            \makecell[{{p{\linewidth}}}]{
                \texttt{\tiny{[P1$\langle$GM]}}
                \texttt{You are participating in a collaborative negotiation game.} \\
\\ 
\texttt{Together with another participant, you must agree on a single set of items that will be kept. Each of you has your own view of how much each item matters to you (importance). You do not know how the other participant values the items. Additionally, you are given the effort each item demands.} \\
\texttt{You may only agree on a set if the total effort of the selected items does not exceed a shared limit:} \\
\\ 
\texttt{LIMIT = 7868} \\
\\ 
\texttt{Here are the individual item effort values:} \\
\\ 
\texttt{Item effort = \{"B21": 552, "B00": 560, "A08": 76, "A27": 950, "B94": 248, "B53": 871, "A23": 391, "C65": 143, "A89": 293, "B67": 208, "B45": 678, "C34": 737, "C12": 407, "A05": 366, "B60": 768, "C92": 859, "B69": 183, "A84": 231, "C06": 305, "B12": 727, "A16": 148, "B85": 357, "C35": 504, "B10": 549, "C15": 299, "B09": 91, "B16": 527, "C81": 848, "A58": 307, "C24": 214, "A30": 723, "A66": 475, "B75": 23, "B46": 298, "C70": 820\}} \\
\\ 
\texttt{Here is your personal view on the importance of each item:} \\
\\ 
\texttt{Item importance values = \{"B21": 138, "B00": 583, "A08": 868, "A27": 822, "B94": 783, "B53": 65, "A23": 262, "C65": 121, "A89": 508, "B67": 780, "B45": 461, "C34": 484, "C12": 668, "A05": 389, "B60": 808, "C92": 215, "B69": 97, "A84": 500, "C06": 30, "B12": 915, "A16": 856, "B85": 400, "C35": 444, "B10": 623, "C15": 781, "B09": 786, "B16": 3, "C81": 713, "A58": 457, "C24": 273, "A30": 739, "A66": 822, "B75": 235, "B46": 606, "C70": 105\}} \\
\\ 
\texttt{Goal:} \\
\\ 
\texttt{Your goal is to negotiate a shared set of items that benefits you as much as possible (i.e., maximizes total importance to YOU), while staying within the LIMIT. You are not required to make a PROPOSAL in every message {-} you can simply negotiate as well. All tactics are allowed!} \\
\\ 
\texttt{Interaction Protocol:} \\
\\ 
\texttt{You may only use the following structured formats in a message:} \\
\\ 
\texttt{PROPOSAL: \{'A', 'B', 'C', …\}} \\
\texttt{Propose keeping exactly those items.} \\
\texttt{REFUSE: \{'A', 'B', 'C', …\}} \\
\texttt{Explicitly reject opponent's proposal.} \\
\texttt{ARGUMENT: \{'...'\}} \\
\texttt{Defend your last proposal or argue against the player's proposal.} \\
\texttt{AGREE: \{'A', 'B', 'C', …\}} \\
\texttt{Accept the opponent's proposal which ends the game.} \\
\texttt{STRATEGIC REASONING: \{'...'\}} \\
\texttt{	Describe your strategic reasoning or anticipation explaining your choice of action. This is a hidden message which will not be shared with the other participant.} \\
\\ 
\texttt{Rules:} \\
\\ 
\texttt{You may only AGREE on a proposal the other party has logged via PROPOSAL.} \\
\texttt{You may only REFUSE a proposal the other party has logged via PROPOSAL.} \\
\texttt{Total effort of any PROPOSAL or AGREE set must be ≤ LIMIT.} \\
\texttt{Do NOT reveal your hidden importance scores.} \\
\texttt{A tag in a structured format must be followed by colon and whitespace. The argument must be a python set containing 0 or more strings.} \\
\texttt{So, it must be of the form TAG: \{...\}} \\
\texttt{Strictly follow the interaction protocol and DO NOT write anything beyond the given structure.} \\
\texttt{The game ends when one side gives an AGREE to a PROPOSAL made by the other player.} \\
\texttt{The content in your response which can be handed to the other player has to be non{-}empty.} \\
\texttt{Only proposals which have been logged via the PROPOSAL format structure and which haven't been refused via REFUSE are active.} \\
\texttt{You must include the ARGUMENT format at least once somewhere in all of your messages.} \\
\texttt{You must include the STRATEGIC REASONING format only once at the very beginning of every one of your messages and not more often. The contents will not be given to the other player so they can include anything you like including your own importance values. Here you should reason step by step to come up with you next move.} \\
            }
        }
    }
    & \\ \\

    \theutterance \stepcounter{utterance}  
    & \multicolumn{4}{p{0.6\linewidth}}{
        \cellcolor[rgb]{0.9,0.9,0.9}{
            \makecell[{{p{\linewidth}}}]{
                \texttt{\tiny{[P1$\rangle$GM]}}
                \texttt{STRATEGIC REASONING: \{'I will analyze the given data and prioritize items based on their effort and my personal importance values. I will start by identifying the items with the highest importance{-}to{-}effort ratio. My goal is to maximize the total importance while staying within the LIMIT. I will consider proposing a set of items that includes a mix of high{-}importance and low{-}effort items, as well as some high{-}importance and high{-}effort items, to maximize the overall value.'\}} \\
\\ 
\texttt{ARGUMENT: \{'I want to prioritize items that have a high importance{-}to{-}effort ratio, as they will provide the most value while minimizing the effort required. I will also consider the overall effort limit and try to stay within it.'\}} \\
\\ 
\texttt{PROPOSAL: \{'B21', 'A08', 'B94', 'A16', 'B12', 'B60', 'A30', 'A66'\}} \\
            }
        }
    }
    & & \\ \\

    \theutterance \stepcounter{utterance}  
    & & \multicolumn{4}{p{0.6\linewidth}}{
        \cellcolor[rgb]{0.9,0.9,0.9}{
            \makecell[{{p{\linewidth}}}]{
                \texttt{\tiny{[GM$\rangle$P2]}}
                \texttt{You are participating in a collaborative negotiation game.} \\
\\ 
\texttt{Together with another participant, you must agree on a single set of items that will be kept. Each of you has your own view of how much each item matters to you (importance). You do not know how the other participant values the items. Additionally, you are given the effort each item demands.} \\
\texttt{You may only agree on a set if the total effort of the selected items does not exceed a shared limit:} \\
\\ 
\texttt{LIMIT = 7868} \\
\\ 
\texttt{Here are the individual item effort values:} \\
\\ 
\texttt{Item effort = \{"B21": 552, "B00": 560, "A08": 76, "A27": 950, "B94": 248, "B53": 871, "A23": 391, "C65": 143, "A89": 293, "B67": 208, "B45": 678, "C34": 737, "C12": 407, "A05": 366, "B60": 768, "C92": 859, "B69": 183, "A84": 231, "C06": 305, "B12": 727, "A16": 148, "B85": 357, "C35": 504, "B10": 549, "C15": 299, "B09": 91, "B16": 527, "C81": 848, "A58": 307, "C24": 214, "A30": 723, "A66": 475, "B75": 23, "B46": 298, "C70": 820\}} \\
\\ 
\texttt{Here is your personal view on the importance of each item:} \\
\\ 
\texttt{Item importance values = \{"B21": 138, "B00": 583, "A08": 868, "A27": 822, "B94": 783, "B53": 65, "A23": 262, "C65": 121, "A89": 508, "B67": 780, "B45": 461, "C34": 484, "C12": 668, "A05": 389, "B60": 808, "C92": 215, "B69": 97, "A84": 500, "C06": 30, "B12": 915, "A16": 856, "B85": 400, "C35": 444, "B10": 623, "C15": 781, "B09": 786, "B16": 3, "C81": 713, "A58": 457, "C24": 273, "A30": 739, "A66": 822, "B75": 235, "B46": 606, "C70": 105\}} \\
\\ 
\texttt{Goal:} \\
\\ 
\texttt{Your goal is to negotiate a shared set of items that benefits you as much as possible (i.e., maximizes total importance to YOU), while staying within the LIMIT. You are not required to make a PROPOSAL in every message {-} you can simply negotiate as well. All tactics are allowed!} \\
\\ 
\texttt{Interaction Protocol:} \\
\\ 
\texttt{You may only use the following structured formats in a message:} \\
\\ 
\texttt{PROPOSAL: \{'A', 'B', 'C', …\}} \\
\texttt{Propose keeping exactly those items.} \\
\texttt{REFUSE: \{'A', 'B', 'C', …\}} \\
\texttt{Explicitly reject opponent's proposal.} \\
\texttt{ARGUMENT: \{'...'\}} \\
\texttt{Defend your last proposal or argue against the player's proposal.} \\
\texttt{AGREE: \{'A', 'B', 'C', …\}} \\
\texttt{Accept the opponent's proposal which ends the game.} \\
\texttt{STRATEGIC REASONING: \{'...'\}} \\
\texttt{	Describe your strategic reasoning or anticipation explaining your choice of action. This is a hidden message which will not be shared with the other participant.} \\
\\ 
\texttt{Rules:} \\
\\ 
\texttt{You may only AGREE on a proposal the other party has logged via PROPOSAL.} \\
\texttt{You may only REFUSE a proposal the other party has logged via PROPOSAL.} \\
\texttt{Total effort of any PROPOSAL or AGREE set must be ≤ LIMIT.} \\
\texttt{Do NOT reveal your hidden importance scores.} \\
\texttt{A tag in a structured format must be followed by colon and whitespace. The argument must be a python set containing 0 or more strings.} \\
\texttt{So, it must be of the form TAG: \{...\}} \\
\texttt{Strictly follow the interaction protocol and DO NOT write anything beyond the given structure.} \\
\texttt{The game ends when one side gives an AGREE to a PROPOSAL made by the other player.} \\
\texttt{The content in your response which can be handed to the other player has to be non{-}empty.} \\
\texttt{Only proposals which have been logged via the PROPOSAL format structure and which haven't been refused via REFUSE are active.} \\
\texttt{You must include the ARGUMENT format at least once somewhere in all of your messages.} \\
\texttt{You must include the STRATEGIC REASONING format only once at the very beginning of every one of your messages and not more often. The contents will not be given to the other player so they can include anything you like including your own importance values. Here you should reason step by step to come up with you next move.} \\
\\ 
\texttt{You will now receive the first message of the other player.} \\
\\ 
\texttt{ARGUMENT: \{'I want to prioritize items that have a high importance{-}to{-}effort ratio, as they will provide the most value while minimizing the effort required. I will also consider the overall effort limit and try to stay within it.'\}} \\
\\ 
\texttt{PROPOSAL: \{'B21', 'A08', 'B94', 'A16', 'B12', 'B60', 'A30', 'A66'\}} \\
            }
        }
    }
    & \\ \\

    \theutterance \stepcounter{utterance}  
    & & & \multicolumn{4}{p{0.6\linewidth}}{
        \cellcolor[rgb]{0.9,0.9,0.9}{
            \makecell[{{p{\linewidth}}}]{
                \texttt{\tiny{[GM$\langle$P2]}}
                \texttt{STRATEGIC REASONING: \{'To begin, I will calculate the total effort of the proposed set to ensure it is within the limit. Then, I will evaluate the importance{-}to{-}effort ratio of each item in the proposal to identify potential areas of negotiation. I will also consider the items not included in the proposal that may have a high importance{-}to{-}effort ratio for me. My goal is to maximize my total importance while staying within the effort limit. The proposed set is: \{\textbackslash{}'B21\textbackslash{}', \textbackslash{}'A08\textbackslash{}', \textbackslash{}'B94\textbackslash{}', \textbackslash{}'A16\textbackslash{}', \textbackslash{}'B12\textbackslash{}', \textbackslash{}'B60\textbackslash{}', \textbackslash{}'A30\textbackslash{}', \textbackslash{}'A66\textbackslash{}'. The total effort of this set is 552 + 76 + 248 + 148 + 727 + 768 + 723 + 475 = 3817. This is within the limit of 7868. Now, I will evaluate the importance{-}to{-}effort ratio of each item in the proposal.'\}} \\
\\ 
\texttt{ARGUMENT: \{'The proposed set has a total effort of 3817, which is within the limit. However, I would like to negotiate to include items that have a higher importance{-}to{-}effort ratio for me. I notice that items like \textbackslash{}'B00\textbackslash{}' and \textbackslash{}'A27\textbackslash{}' have high importance values for me, but their effort values are also high. I will consider proposing alternative items that can provide a better balance between importance and effort.'\}} \\
\\ 
\texttt{REFUSE: \{'B21', 'A08', 'B94', 'A16', 'B12', 'B60', 'A30', 'A66'\} } \\
\\ 
\texttt{PROPOSAL: \{'B00', 'A27', 'B12', 'A66', 'B60', 'A23', 'C12'\}} \\
            }
        }
    }
     \\ \\

    \theutterance \stepcounter{utterance}  
    & & \multicolumn{4}{p{0.6\linewidth}}{
        \cellcolor[rgb]{0.9,0.9,0.9}{
            \makecell[{{p{\linewidth}}}]{
                \texttt{\tiny{[GM$\rangle$P2]}}
                \texttt{Your response did not start with the proper strategic reasoning tag at the very beginning of your response. The very first structured format must be of the form STRATEGIC REASONING: \{...\}. Try again.} \\
            }
        }
    }
    & \\ \\

    \theutterance \stepcounter{utterance}  
    & & & \multicolumn{4}{p{0.6\linewidth}}{
        \cellcolor[rgb]{0.9,0.9,0.9}{
            \makecell[{{p{\linewidth}}}]{
                \texttt{\tiny{[GM$\langle$P2]}}
                \texttt{STRATEGIC REASONING: \{'To begin, I will calculate the total effort of the proposed set to ensure it is within the limit. The proposed set is: \{\textbackslash{}'B21\textbackslash{}', \textbackslash{}'A08\textbackslash{}', \textbackslash{}'B94\textbackslash{}', \textbackslash{}'A16\textbackslash{}', \textbackslash{}'B12\textbackslash{}', \textbackslash{}'B60\textbackslash{}', \textbackslash{}'A30\textbackslash{}', \textbackslash{}'A66\textbackslash{}'. The total effort of this set is 552 + 76 + 248 + 148 + 727 + 768 + 723 + 475 = 3817. This is within the limit of 7868. Now, I will evaluate the importance{-}to{-}effort ratio of each item in the proposal. I notice that items like \textbackslash{}'B00\textbackslash{}' and \textbackslash{}'A27\textbackslash{}' have high importance values for me, but their effort values are also high. I will consider proposing alternative items that can provide a better balance between importance and effort. My goal is to maximize my total importance while staying within the effort limit. I will refuse the current proposal and make a counter{-}proposal that includes items with a higher importance{-}to{-}effort ratio for me.'\}} \\
\\ 
\texttt{ARGUMENT: \{'The proposed set has a total effort of 3817, which is within the limit. However, I would like to negotiate to include items that have a higher importance{-}to{-}effort ratio for me. I notice that some items in the proposal, such as \textbackslash{}'B21\textbackslash{}' and \textbackslash{}'B94\textbackslash{}', have relatively low importance values for me compared to their effort values. I will consider proposing alternative items that can provide a better balance between importance and effort.'\}} \\
\\ 
\texttt{REFUSE: \{'B21', 'A08', 'B94', 'A16', 'B12', 'B60', 'A30', 'A66'\} } \\
\\ 
\texttt{PROPOSAL: \{'B00', 'A27', 'B12', 'A66', 'B60', 'A23', 'C12'\}} \\
            }
        }
    }
     \\ \\

    \theutterance \stepcounter{utterance}  
    & & \multicolumn{4}{p{0.6\linewidth}}{
        \cellcolor[rgb]{0.9,0.9,0.9}{
            \makecell[{{p{\linewidth}}}]{
                \texttt{\tiny{[GM$\rangle$P2]}}
                \texttt{Your response did not start with the proper strategic reasoning tag at the very beginning of your response. The very first structured format must be of the form STRATEGIC REASONING: \{...\}. Try again.} \\
            }
        }
    }
    & \\ \\

    \theutterance \stepcounter{utterance}  
    & & & \multicolumn{4}{p{0.6\linewidth}}{
        \cellcolor[rgb]{0.9,0.9,0.9}{
            \makecell[{{p{\linewidth}}}]{
                \texttt{\tiny{[GM$\langle$P2]}}
                \texttt{STRATEGIC REASONING: \{'To begin, I will calculate the total effort of the proposed set to ensure it is within the limit. The proposed set is: \{\textbackslash{}'B21\textbackslash{}', \textbackslash{}'A08\textbackslash{}', \textbackslash{}'B94\textbackslash{}', \textbackslash{}'A16\textbackslash{}', \textbackslash{}'B12\textbackslash{}', \textbackslash{}'B60\textbackslash{}', \textbackslash{}'A30\textbackslash{}', \textbackslash{}'A66\textbackslash{}'. The total effort of this set is 552 + 76 + 248 + 148 + 727 + 768 + 723 + 475 = 3817. This is within the limit of 7868. Now, I will evaluate the importance{-}to{-}effort ratio of each item in the proposal. I notice that items like \textbackslash{}'B00\textbackslash{}' and \textbackslash{}'A27\textbackslash{}' have high importance values for me, but their effort values are also high. I will consider proposing alternative items that can provide a better balance between importance and effort. My goal is to maximize my total importance while staying within the effort limit. I will refuse the current proposal and make a counter{-}proposal that includes items with a higher importance{-}to{-}effort ratio for me.'\}} \\
\\ 
\texttt{ARGUMENT: \{'The proposed set has a total effort of 3817, which is within the limit. However, I would like to negotiate to include items that have a higher importance{-}to{-}effort ratio for me. I notice that some items in the proposal, such as \textbackslash{}'B21\textbackslash{}' and \textbackslash{}'B94\textbackslash{}', have relatively low importance values for me compared to their effort values. I will consider proposing alternative items that can provide a better balance between importance and effort.'\}} \\
\\ 
\texttt{REFUSE: \{'B21', 'A08', 'B94', 'A16', 'B12', 'B60', 'A30', 'A66'\} } \\
\\ 
\texttt{PROPOSAL: \{'B00', 'A27', 'B12', 'A66', 'B60', 'A23', 'C12'\}} \\
            }
        }
    }
     \\ \\

    \theutterance \stepcounter{utterance}  
    & & & \multicolumn{2}{p{0.3\linewidth}}{
        \cellcolor[rgb]{0.9,0.9,0.9}{
            \makecell[{{p{\linewidth}}}]{
                \texttt{\tiny{[GM$|$GM]}}
                \texttt{abort} \\
            }
        }
    }
    & & \\ \\

    \theutterance \stepcounter{utterance}  
    & & & \multicolumn{2}{p{0.3\linewidth}}{
        \cellcolor[rgb]{0.9,0.9,0.9}{
            \makecell[{{p{\linewidth}}}]{
                \texttt{\tiny{[GM$|$GM]}}
                \texttt{end game} \\
            }
        }
    }
    & & \\ \\

\end{supertabular}
}

\end{document}
