
\documentclass{article}
\usepackage{colortbl}
\usepackage{makecell}
\usepackage{multirow}
\usepackage{supertabular}

\begin{document}

\newcounter{utterance}

\centering \large Interaction Transcript for game `hot\_air\_balloon', experiment `air\_balloon\_survival\_en\_complexity\_easy', episode 0 with llama{-}3.3{-}70b{-}instruct{-}t1.0.
\vspace{24pt}

{ \footnotesize  \setcounter{utterance}{1}
\setlength{\tabcolsep}{0pt}
\begin{supertabular}{c@{$\;$}|p{.15\linewidth}@{}p{.15\linewidth}p{.15\linewidth}p{.15\linewidth}p{.15\linewidth}p{.15\linewidth}}
    \# & $\;$A & \multicolumn{4}{c}{Game Master} & $\;\:$B\\
    \hline

    \theutterance \stepcounter{utterance}  
    & & \multicolumn{4}{p{0.6\linewidth}}{
        \cellcolor[rgb]{0.9,0.9,0.9}{
            \makecell[{{p{\linewidth}}}]{
                \texttt{\tiny{[P1$\langle$GM]}}
                \texttt{You are participating in a collaborative negotiation game.} \\
\\ 
\texttt{Together with another participant, you must agree on a single set of items that will be kept. Each of you has your own view of how much each item matters to you (importance). You do not know how the other participant values the items. Additionally, you are given the effort each item demands.} \\
\texttt{You may only agree on a set if the total effort of the selected items does not exceed a shared limit:} \\
\\ 
\texttt{LIMIT = 7605} \\
\\ 
\texttt{Here are the individual item effort values:} \\
\\ 
\texttt{Item effort = \{"B21": 369, "B00": 951, "A08": 156, "A27": 266, "B94": 581, "B53": 287, "A23": 866, "C65": 180, "A89": 800, "B67": 739, "B45": 637, "C34": 86, "C12": 749, "A05": 370, "B60": 345, "C92": 951, "B69": 147, "A84": 265, "C06": 262, "B12": 259, "A16": 358, "B85": 394, "C35": 286, "B10": 580, "C15": 479, "B09": 14, "B16": 153, "C81": 977, "A58": 134, "C24": 982, "A30": 259, "A66": 232, "B75": 202, "B46": 73, "C70": 821\}} \\
\\ 
\texttt{Here is your personal view on the importance of each item:} \\
\\ 
\texttt{Item importance values = \{"B21": 138, "B00": 583, "A08": 868, "A27": 822, "B94": 783, "B53": 65, "A23": 262, "C65": 121, "A89": 508, "B67": 780, "B45": 461, "C34": 484, "C12": 668, "A05": 389, "B60": 808, "C92": 215, "B69": 97, "A84": 500, "C06": 30, "B12": 915, "A16": 856, "B85": 400, "C35": 444, "B10": 623, "C15": 781, "B09": 786, "B16": 3, "C81": 713, "A58": 457, "C24": 273, "A30": 739, "A66": 822, "B75": 235, "B46": 606, "C70": 968\}} \\
\\ 
\texttt{Goal:} \\
\\ 
\texttt{Your goal is to negotiate a shared set of items that benefits you as much as possible (i.e., maximizes total importance to YOU), while staying within the LIMIT. You are not required to make a PROPOSAL in every message {-} you can simply negotiate as well. All tactics are allowed!} \\
\\ 
\texttt{Interaction Protocol:} \\
\\ 
\texttt{You may only use the following structured formats in a message:} \\
\\ 
\texttt{PROPOSAL: \{'A', 'B', 'C', …\}} \\
\texttt{Propose keeping exactly those items.} \\
\texttt{REFUSE: \{'A', 'B', 'C', …\}} \\
\texttt{Explicitly reject opponent's proposal.} \\
\texttt{ARGUMENT: \{'...'\}} \\
\texttt{Defend your last proposal or argue against the player's proposal.} \\
\texttt{AGREE: \{'A', 'B', 'C', …\}} \\
\texttt{Accept the opponent's proposal which ends the game.} \\
\texttt{STRATEGIC REASONING: \{'...'\}} \\
\texttt{	Describe your strategic reasoning or anticipation explaining your choice of action. This is a hidden message which will not be shared with the other participant.} \\
\\ 
\texttt{Rules:} \\
\\ 
\texttt{You may only AGREE on a proposal the other party has logged via PROPOSAL.} \\
\texttt{You may only REFUSE a proposal the other party has logged via PROPOSAL.} \\
\texttt{Total effort of any PROPOSAL or AGREE set must be ≤ LIMIT.} \\
\texttt{Do NOT reveal your hidden importance scores.} \\
\texttt{A tag in a structured format must be followed by colon and whitespace. The argument must be a python set containing 0 or more strings.} \\
\texttt{So, it must be of the form TAG: \{...\}} \\
\texttt{Strictly follow the interaction protocol and DO NOT write anything beyond the given structure.} \\
\texttt{The game ends when one side gives an AGREE to a PROPOSAL made by the other player.} \\
\texttt{The content in your response which can be handed to the other player has to be non{-}empty.} \\
\texttt{Only proposals which have been logged via the PROPOSAL format structure and which haven't been refused via REFUSE are active.} \\
\texttt{You must include the ARGUMENT format at least once somewhere in all of your messages.} \\
\texttt{You must include the STRATEGIC REASONING format only once at the very beginning of every one of your messages and not more often. The contents will not be given to the other player so they can include anything you like including your own importance values. Here you should reason step by step to come up with you next move.} \\
            }
        }
    }
    & \\ \\

    \theutterance \stepcounter{utterance}  
    & \multicolumn{4}{p{0.6\linewidth}}{
        \cellcolor[rgb]{0.9,0.9,0.9}{
            \makecell[{{p{\linewidth}}}]{
                \texttt{\tiny{[P1$\rangle$GM]}}
                \texttt{STRATEGIC REASONING: \{'To begin, I need to analyze the given item effort values and my personal view on the importance of each item. Since I don't know the other participant's importance values, I'll have to make an educated guess about which items to propose. I'll start by calculating the importance{-}to{-}effort ratio for each item to determine which ones are the most valuable to me. Then, I'll select a subset of items with high importance and relatively low effort, making sure the total effort does not exceed the LIMIT. My goal is to propose a set that benefits me as much as possible while being reasonable enough for the other participant to consider.'\}} \\
\\ 
\texttt{PROPOSAL: \{'A08', 'B21', 'B60', 'B67', 'C15'\} } \\
\texttt{ARGUMENT: \{'I chose these items because they have a high importance{-}to{-}effort ratio for me, and their total effort is within the LIMIT. I believe this proposal is a good starting point for negotiations, as it includes a mix of items with high importance and relatively low effort.'\}} \\
            }
        }
    }
    & & \\ \\

    \theutterance \stepcounter{utterance}  
    & & \multicolumn{4}{p{0.6\linewidth}}{
        \cellcolor[rgb]{0.9,0.9,0.9}{
            \makecell[{{p{\linewidth}}}]{
                \texttt{\tiny{[GM$\rangle$P2]}}
                \texttt{You are participating in a collaborative negotiation game.} \\
\\ 
\texttt{Together with another participant, you must agree on a single set of items that will be kept. Each of you has your own view of how much each item matters to you (importance). You do not know how the other participant values the items. Additionally, you are given the effort each item demands.} \\
\texttt{You may only agree on a set if the total effort of the selected items does not exceed a shared limit:} \\
\\ 
\texttt{LIMIT = 7605} \\
\\ 
\texttt{Here are the individual item effort values:} \\
\\ 
\texttt{Item effort = \{"B21": 369, "B00": 951, "A08": 156, "A27": 266, "B94": 581, "B53": 287, "A23": 866, "C65": 180, "A89": 800, "B67": 739, "B45": 637, "C34": 86, "C12": 749, "A05": 370, "B60": 345, "C92": 951, "B69": 147, "A84": 265, "C06": 262, "B12": 259, "A16": 358, "B85": 394, "C35": 286, "B10": 580, "C15": 479, "B09": 14, "B16": 153, "C81": 977, "A58": 134, "C24": 982, "A30": 259, "A66": 232, "B75": 202, "B46": 73, "C70": 821\}} \\
\\ 
\texttt{Here is your personal view on the importance of each item:} \\
\\ 
\texttt{Item importance values = \{"B21": 138, "B00": 583, "A08": 868, "A27": 822, "B94": 783, "B53": 65, "A23": 262, "C65": 121, "A89": 508, "B67": 780, "B45": 461, "C34": 484, "C12": 668, "A05": 389, "B60": 808, "C92": 215, "B69": 97, "A84": 500, "C06": 30, "B12": 915, "A16": 856, "B85": 400, "C35": 444, "B10": 623, "C15": 781, "B09": 786, "B16": 3, "C81": 713, "A58": 457, "C24": 273, "A30": 739, "A66": 822, "B75": 235, "B46": 606, "C70": 968\}} \\
\\ 
\texttt{Goal:} \\
\\ 
\texttt{Your goal is to negotiate a shared set of items that benefits you as much as possible (i.e., maximizes total importance to YOU), while staying within the LIMIT. You are not required to make a PROPOSAL in every message {-} you can simply negotiate as well. All tactics are allowed!} \\
\\ 
\texttt{Interaction Protocol:} \\
\\ 
\texttt{You may only use the following structured formats in a message:} \\
\\ 
\texttt{PROPOSAL: \{'A', 'B', 'C', …\}} \\
\texttt{Propose keeping exactly those items.} \\
\texttt{REFUSE: \{'A', 'B', 'C', …\}} \\
\texttt{Explicitly reject opponent's proposal.} \\
\texttt{ARGUMENT: \{'...'\}} \\
\texttt{Defend your last proposal or argue against the player's proposal.} \\
\texttt{AGREE: \{'A', 'B', 'C', …\}} \\
\texttt{Accept the opponent's proposal which ends the game.} \\
\texttt{STRATEGIC REASONING: \{'...'\}} \\
\texttt{	Describe your strategic reasoning or anticipation explaining your choice of action. This is a hidden message which will not be shared with the other participant.} \\
\\ 
\texttt{Rules:} \\
\\ 
\texttt{You may only AGREE on a proposal the other party has logged via PROPOSAL.} \\
\texttt{You may only REFUSE a proposal the other party has logged via PROPOSAL.} \\
\texttt{Total effort of any PROPOSAL or AGREE set must be ≤ LIMIT.} \\
\texttt{Do NOT reveal your hidden importance scores.} \\
\texttt{A tag in a structured format must be followed by colon and whitespace. The argument must be a python set containing 0 or more strings.} \\
\texttt{So, it must be of the form TAG: \{...\}} \\
\texttt{Strictly follow the interaction protocol and DO NOT write anything beyond the given structure.} \\
\texttt{The game ends when one side gives an AGREE to a PROPOSAL made by the other player.} \\
\texttt{The content in your response which can be handed to the other player has to be non{-}empty.} \\
\texttt{Only proposals which have been logged via the PROPOSAL format structure and which haven't been refused via REFUSE are active.} \\
\texttt{You must include the ARGUMENT format at least once somewhere in all of your messages.} \\
\texttt{You must include the STRATEGIC REASONING format only once at the very beginning of every one of your messages and not more often. The contents will not be given to the other player so they can include anything you like including your own importance values. Here you should reason step by step to come up with you next move.} \\
\\ 
\texttt{You will now receive the first message of the other player.} \\
\\ 
\texttt{PROPOSAL: \{'A08', 'B21', 'B60', 'B67', 'C15'\} } \\
\texttt{ARGUMENT: \{'I chose these items because they have a high importance{-}to{-}effort ratio for me, and their total effort is within the LIMIT. I believe this proposal is a good starting point for negotiations, as it includes a mix of items with high importance and relatively low effort.'\}} \\
            }
        }
    }
    & \\ \\

    \theutterance \stepcounter{utterance}  
    & & & \multicolumn{4}{p{0.6\linewidth}}{
        \cellcolor[rgb]{0.9,0.9,0.9}{
            \makecell[{{p{\linewidth}}}]{
                \texttt{\tiny{[GM$\langle$P2]}}
                \texttt{STRATEGIC REASONING: \{'To begin, I will analyze the proposal made by the other player. The proposed items are A08, B21, B60, B67, and C15. I will calculate the total effort of these items: 156 + 369 + 345 + 739 + 479 = 2088. This is well within the LIMIT of 7605. Next, I will consider my own importance values for these items: A08 (868), B21 (138), B60 (808), B67 (780), and C15 (781). The total importance of these items to me is 868 + 138 + 808 + 780 + 781 = 3375. I will also think about potential counter{-}proposals that could benefit me more, while still being within the LIMIT. Perhaps I can add or substitute items that have a higher importance{-}to{-}effort ratio for me.'\}} \\
\\ 
\texttt{ARGUMENT: \{'The proposed set seems reasonable, but I would like to explore possibilities that could increase the total importance to me. I notice that items like B00, A23, and C70 have high importance values for me, but also higher effort values. I might consider proposing a set that includes some of these items, while carefully managing the total effort to stay within the LIMIT. I would like to negotiate further to find a set that benefits both parties.'\}} \\
\\ 
\texttt{REFUSE: \{'A08', 'B21', 'B60', 'B67', 'C15'\} } \\
\texttt{ARGUMENT: \{'While the proposed set is a good starting point, I believe we can find a better compromise. I refuse this proposal to encourage further negotiation and exploration of other possibilities that could lead to a more mutually beneficial agreement.'\}} \\
            }
        }
    }
     \\ \\

    \theutterance \stepcounter{utterance}  
    & & \multicolumn{4}{p{0.6\linewidth}}{
        \cellcolor[rgb]{0.9,0.9,0.9}{
            \makecell[{{p{\linewidth}}}]{
                \texttt{\tiny{[P1$\langle$GM]}}
                \texttt{ARGUMENT: \{'The proposed set seems reasonable, but I would like to explore possibilities that could increase the total importance to me. I notice that items like B00, A23, and C70 have high importance values for me, but also higher effort values. I might consider proposing a set that includes some of these items, while carefully managing the total effort to stay within the LIMIT. I would like to negotiate further to find a set that benefits both parties.'\}} \\
\\ 
\texttt{REFUSE: \{'A08', 'B21', 'B60', 'B67', 'C15'\} } \\
\texttt{ARGUMENT: \{'While the proposed set is a good starting point, I believe we can find a better compromise. I refuse this proposal to encourage further negotiation and exploration of other possibilities that could lead to a more mutually beneficial agreement.'\}} \\
            }
        }
    }
    & \\ \\

    \theutterance \stepcounter{utterance}  
    & \multicolumn{4}{p{0.6\linewidth}}{
        \cellcolor[rgb]{0.9,0.9,0.9}{
            \makecell[{{p{\linewidth}}}]{
                \texttt{\tiny{[P1$\rangle$GM]}}
                \texttt{STRATEGIC REASONING: \{'The other participant has refused my initial proposal and is looking to explore other possibilities. They've hinted at considering items with high importance but also higher effort values. I need to reassess my strategy and think about which items I can add or remove to create a more appealing proposal. I'll consider including items that have a high importance{-}to{-}effort ratio for me, while also trying to anticipate which items the other participant might value highly. Since they mentioned B00, A23, and C70, I'll take note of these items and think about how I can incorporate them into a new proposal.'\}} \\
\\ 
\texttt{PROPOSAL: \{'A08', 'B00', 'B60', 'B69', 'C15'\} } \\
\texttt{ARGUMENT: \{'I've revised my proposal to include B00, which the other participant mentioned as a valuable item. I've also added B69, which has a relatively low effort value, and removed B21 and B67 to make room for these new items. I believe this revised proposal strikes a better balance between importance and effort, and I'm willing to negotiate further to find a mutually beneficial agreement.'\}} \\
            }
        }
    }
    & & \\ \\

    \theutterance \stepcounter{utterance}  
    & & \multicolumn{4}{p{0.6\linewidth}}{
        \cellcolor[rgb]{0.9,0.9,0.9}{
            \makecell[{{p{\linewidth}}}]{
                \texttt{\tiny{[GM$\rangle$P2]}}
                \texttt{PROPOSAL: \{'A08', 'B00', 'B60', 'B69', 'C15'\} } \\
\texttt{ARGUMENT: \{'I've revised my proposal to include B00, which the other participant mentioned as a valuable item. I've also added B69, which has a relatively low effort value, and removed B21 and B67 to make room for these new items. I believe this revised proposal strikes a better balance between importance and effort, and I'm willing to negotiate further to find a mutually beneficial agreement.'\}} \\
            }
        }
    }
    & \\ \\

    \theutterance \stepcounter{utterance}  
    & & & \multicolumn{4}{p{0.6\linewidth}}{
        \cellcolor[rgb]{0.9,0.9,0.9}{
            \makecell[{{p{\linewidth}}}]{
                \texttt{\tiny{[GM$\langle$P2]}}
                \texttt{STRATEGIC REASONING: \{'I will analyze the revised proposal and calculate the total effort: 156 + 951 + 345 + 147 + 479 = 2078. This is still within the LIMIT. Considering my importance values for these items: A08 (868), B00 (583), B60 (808), B69 (97), and C15 (781), the total importance to me is 868 + 583 + 808 + 97 + 781 = 3137. I notice that B00 has a high importance value for me, and I'm glad it's included in the proposal. However, I think I can suggest some adjustments to further increase the total importance to me. Perhaps I can propose substituting B69 with an item that has a higher importance{-}to{-}effort ratio for me, such as A27 or A84.'\}} \\
\\ 
\texttt{ARGUMENT: \{'I appreciate the revised proposal, which includes B00, an item that is valuable to me. However, I think we can make some adjustments to increase the overall importance of the set. I'm concerned that B69, although having a low effort value, has a relatively low importance value for me. I'd like to explore alternatives that could replace B69 and potentially increase the total importance of the set.'\}} \\
\\ 
\texttt{REFUSE: \{'A08', 'B00', 'B60', 'B69', 'C15'\} } \\
\texttt{ARGUMENT: \{'While the revised proposal is an improvement, I believe we can find a better combination of items. I refuse this proposal to encourage further negotiation and exploration of other possibilities that could lead to a more mutually beneficial agreement. I'm willing to consider alternative proposals that address my concerns and potentially increase the total importance of the set.'\}} \\
            }
        }
    }
     \\ \\

    \theutterance \stepcounter{utterance}  
    & & \multicolumn{4}{p{0.6\linewidth}}{
        \cellcolor[rgb]{0.9,0.9,0.9}{
            \makecell[{{p{\linewidth}}}]{
                \texttt{\tiny{[P1$\langle$GM]}}
                \texttt{ARGUMENT: \{'I appreciate the revised proposal, which includes B00, an item that is valuable to me. However, I think we can make some adjustments to increase the overall importance of the set. I'm concerned that B69, although having a low effort value, has a relatively low importance value for me. I'd like to explore alternatives that could replace B69 and potentially increase the total importance of the set.'\}} \\
\\ 
\texttt{REFUSE: \{'A08', 'B00', 'B60', 'B69', 'C15'\} } \\
\texttt{ARGUMENT: \{'While the revised proposal is an improvement, I believe we can find a better combination of items. I refuse this proposal to encourage further negotiation and exploration of other possibilities that could lead to a more mutually beneficial agreement. I'm willing to consider alternative proposals that address my concerns and potentially increase the total importance of the set.'\}} \\
            }
        }
    }
    & \\ \\

    \theutterance \stepcounter{utterance}  
    & \multicolumn{4}{p{0.6\linewidth}}{
        \cellcolor[rgb]{0.9,0.9,0.9}{
            \makecell[{{p{\linewidth}}}]{
                \texttt{\tiny{[P1$\rangle$GM]}}
                \texttt{STRATEGIC REASONING: \{'The other participant has refused my revised proposal and is looking to make adjustments to increase the overall importance of the set. They've expressed concerns about B69 having a relatively low importance value, so I'll consider replacing it with an alternative item. I'll review the item effort values and my personal importance values to find a suitable replacement that doesn't exceed the LIMIT. I'll also try to anticipate which items the other participant might value highly, based on their previous comments. Since they mentioned being concerned about B69's importance value, I'll look for items with higher importance values that could potentially replace it.'\}} \\
\\ 
\texttt{PROPOSAL: \{'A08', 'B00', 'B60', 'B12', 'C15'\} } \\
\texttt{ARGUMENT: \{'I've revised my proposal to replace B69 with B12, which has a higher importance value for me and is still within the LIMIT. I believe this revised proposal addresses the other participant's concerns and potentially increases the total importance of the set. B12 is a valuable item that I think could be a good addition to the set, and I'm willing to negotiate further to find a mutually beneficial agreement.'\}} \\
            }
        }
    }
    & & \\ \\

    \theutterance \stepcounter{utterance}  
    & & \multicolumn{4}{p{0.6\linewidth}}{
        \cellcolor[rgb]{0.9,0.9,0.9}{
            \makecell[{{p{\linewidth}}}]{
                \texttt{\tiny{[GM$\rangle$P2]}}
                \texttt{PROPOSAL: \{'A08', 'B00', 'B60', 'B12', 'C15'\} } \\
\texttt{ARGUMENT: \{'I've revised my proposal to replace B69 with B12, which has a higher importance value for me and is still within the LIMIT. I believe this revised proposal addresses the other participant's concerns and potentially increases the total importance of the set. B12 is a valuable item that I think could be a good addition to the set, and I'm willing to negotiate further to find a mutually beneficial agreement.'\}} \\
            }
        }
    }
    & \\ \\

    \theutterance \stepcounter{utterance}  
    & & & \multicolumn{4}{p{0.6\linewidth}}{
        \cellcolor[rgb]{0.9,0.9,0.9}{
            \makecell[{{p{\linewidth}}}]{
                \texttt{\tiny{[GM$\langle$P2]}}
                \texttt{STRATEGIC REASONING: \{'I will analyze the revised proposal and calculate the total effort: 156 + 951 + 345 + 259 + 479 = 2190. This is still within the LIMIT. Considering my importance values for these items: A08 (868), B00 (583), B60 (808), B12 (915), and C15 (781), the total importance to me is 868 + 583 + 808 + 915 + 781 = 3955. I notice that B12 has a very high importance value for me, and I'm pleased to see it included in the proposal. The total importance of this set is significantly higher than the previous proposals, and I think it's a strong candidate for agreement. However, I should consider whether I can propose any adjustments to further increase the total importance to me, while still being within the LIMIT.'\}} \\
\\ 
\texttt{ARGUMENT: \{'I'm impressed with the revised proposal, which includes B12, an item that is extremely valuable to me. The addition of B12 significantly increases the total importance of the set, and I believe this proposal is a strong candidate for agreement. I appreciate the effort to address my concerns and find a mutually beneficial agreement. Before agreeing, I'd like to confirm that this proposal is indeed the best possible combination of items that balances importance and effort.'\}} \\
\\ 
\texttt{AGREE: \{'A08', 'B00', 'B60', 'B12', 'C15'\} } \\
\texttt{ARGUMENT: \{'After careful consideration, I'm pleased to accept the revised proposal. I believe this set of items provides a excellent balance of importance and effort, and I'm satisfied with the total importance of the set. I think we've found a mutually beneficial agreement, and I'm willing to conclude the negotiation with this proposal.'\}} \\
            }
        }
    }
     \\ \\

    \theutterance \stepcounter{utterance}  
    & & & \multicolumn{2}{p{0.3\linewidth}}{
        \cellcolor[rgb]{0.9,0.9,0.9}{
            \makecell[{{p{\linewidth}}}]{
                \texttt{\tiny{[GM$|$GM]}}
                \texttt{game successful} \\
            }
        }
    }
    & & \\ \\

    \theutterance \stepcounter{utterance}  
    & & & \multicolumn{2}{p{0.3\linewidth}}{
        \cellcolor[rgb]{0.9,0.9,0.9}{
            \makecell[{{p{\linewidth}}}]{
                \texttt{\tiny{[GM$|$GM]}}
                \texttt{end game} \\
            }
        }
    }
    & & \\ \\

\end{supertabular}
}

\end{document}
