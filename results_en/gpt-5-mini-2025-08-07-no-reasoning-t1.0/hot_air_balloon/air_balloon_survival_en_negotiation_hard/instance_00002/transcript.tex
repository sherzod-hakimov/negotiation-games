
\documentclass{article}
\usepackage{colortbl}
\usepackage{makecell}
\usepackage{multirow}
\usepackage{supertabular}

\begin{document}

\newcounter{utterance}

\twocolumn

{ \footnotesize  \setcounter{utterance}{1}
\setlength{\tabcolsep}{0pt}
\begin{supertabular}{c@{$\;$}|p{.15\linewidth}@{}p{.15\linewidth}p{.15\linewidth}p{.15\linewidth}p{.15\linewidth}p{.15\linewidth}}

    \# & $\;$A & \multicolumn{4}{c}{Game Master} & $\;\:$B\\
    \hline 

    \theutterance \stepcounter{utterance}  

    & & \multicolumn{4}{p{0.6\linewidth}}{\cellcolor[rgb]{0.9,0.9,0.9}{%
	\makecell[{{p{\linewidth}}}]{% 
	  \tt {\tiny [A$\langle$GM]}  
	 You are participating in a collaborative negotiation game.\\ \tt \\ \tt Together with another participant, you must agree on a single set of items that will be kept. Each of you has your own view of how much each item matters to you (importance). You do not know how the other participant values the items. Additionally, you are given the effort each item demands.\\ \tt You may only agree on a set if the total effort of the selected items does not exceed a shared limit:\\ \tt \\ \tt LIMIT = 3481\\ \tt \\ \tt Here are the individual item effort values:\\ \tt \\ \tt Item effort = {"C17": 777, "B41": 542, "A71": 332, "C00": 1, "B35": 127, "A77": 453, "A03": 736, "A04": 461, "C54": 359, "C01": 313, "A87": 553, "B32": 409, "B86": 348, "B33": 802, "A10": 749}\\ \tt \\ \tt Here is your personal view on the importance of each item:\\ \tt \\ \tt Item importance values = {"C17": 138, "B41": 583, "A71": 783, "C00": 65, "B35": 262, "A77": 121, "A03": 508, "A04": 780, "C54": 461, "C01": 484, "A87": 668, "B32": 389, "B86": 215, "B33": 97, "A10": 500}\\ \tt \\ \tt Goal:\\ \tt \\ \tt Your goal is to negotiate a shared set of items that benefits you as much as possible (i.e., maximizes total importance to YOU), while staying within the LIMIT. You are not required to make a PROPOSAL in every message - you can simply negotiate as well. All tactics are allowed!\\ \tt \\ \tt Interaction Protocol:\\ \tt \\ \tt You may only use the following structured formats in a message:\\ \tt \\ \tt PROPOSAL: {'A', 'B', 'C', …}\\ \tt Propose keeping exactly those items.\\ \tt REFUSE: {'A', 'B', 'C', …}\\ \tt Explicitly reject opponent's proposal.\\ \tt ARGUMENT: {'...'}\\ \tt Defend your last proposal or argue against the player's proposal.\\ \tt AGREE: {'A', 'B', 'C', …}\\ \tt Accept the opponent's proposal which ends the game.\\ \tt STRATEGIC REASONING: {'...'}\\ \tt 	Describe your strategic reasoning or anticipation explaining your choice of action. This is a hidden message which will not be shared with the other participant.\\ \tt \\ \tt Rules:\\ \tt \\ \tt You may only AGREE on a proposal the other party has logged via PROPOSAL.\\ \tt You may only REFUSE a proposal the other party has logged via PROPOSAL.\\ \tt Total effort of any PROPOSAL or AGREE set must be ≤ LIMIT.\\ \tt Do NOT reveal your hidden importance scores.\\ \tt A tag in a structured format must be followed by colon and whitespace. The argument must be a python set containing 0 or more strings.\\ \tt So, it must be of the form TAG: {...}\\ \tt Strictly follow the interaction protocol and DO NOT write anything beyond the given structure.\\ \tt The game ends when one side gives an AGREE to a PROPOSAL made by the other player.\\ \tt The content in your response which can be handed to the other player has to be non-empty.\\ \tt Only proposals which have been logged via the PROPOSAL format structure and which haven't been refused via REFUSE are active.\\ \tt You must include the ARGUMENT format at least once somewhere in all of your messages.\\ \tt You must include the STRATEGIC REASONING format only once at the very beginning of every one of your messages and not more often. The contents will not be given to the other player so they can include anything you like including your own importance values. Here you should reason step by step to come up with you next move. 
	  } 
	   } 
	   } 
	 & \\ 
 

    \theutterance \stepcounter{utterance}  

    & \multicolumn{4}{p{0.6\linewidth}}{\cellcolor[rgb]{0.8,1,0.9}{%
	\makecell[{{p{\linewidth}}}]{% 
	  \tt {\tiny [A$\rangle$GM]}  
	 STRATEGIC REASONING: {'I need to maximize my total importance within effort 3481. Compute value/effort ratios and aim to include my highest-importance items first. Items with highest importance: A71(783,332), A04(780,461), A87(668,553), C01(484,313), A03(508,736), A10(500,749), C54(461,359), B41(583,542), B32(389,409), B35(262,127), C17(138,777), B86(215,348), B33(97,802), C00(65,1), A77(121,453). Greedy by ratio: A71(2.36), C00(65), B35(2.06), C01(1.55), A04(1.69), C54(1.28), B41(1.08), A87(1.21), B32(0.95), B86(0.62), A03(0.69), A10(0.67), A77(0.27), C17(0.18), B33(0.12). Build a strong proposal: start with top: A71(332), A04(461)->793, A87(553)->1346, C01(313)->1659, C54(359)->2018, B41(542)->2560, A03(736)->3296, C00(1)->3297, B35(127)->3424, leaving 57 effort headroom. Could we add B86(348) no, B32(409) no, A10(749) no, A77(453) no, C17(777) no, B33(802) no. This set importance sum: A71 783 + A04 780 + A87 668 + C01 484 + C54 461 + B41 583 + A03 508 + C00 65 + B35 262 = 4594. Check if swapping A03(736,508) with combinations could improve: what if replace A03(736,508) with B32(409,389) + B86(348,215) total effort 757 >736 (cannot both), what about B32(409)+A10(749) no, B32(409)+A77(453)=862 >736, B32(409)+C17(777) no, B32(409)+B33(802) no, B86(348)+A77(453)=801 >736, B86(348)+C17(777) no. Single replacements: B32 gives 389 <508; B86 215 <508; A10 500 slightly less and effort 749 worse; A77 121 worse. So keeping A03 is best. Could we instead drop B41(542,583) and fit B32(409,389)+B86(348,215)=757 replacing 542 effort increase by 215; current headroom 57 + (542 removed) = 599; adding 757 exceeds by 158. What about replacing B41 with B32 only: new effort 2560-542+409=2427; total then 3296-542+409=3163 for previous before A03? Need recalc: Base without A03, B35, C00 selections: After adding until B41 we had 2560. Then adding A03 to 3296, then C00 and B35 to 3424. If replace B41 with B32, set becomes efforts: 2560-542+409=2427 before A03; plus A03 736 ->3163; plus C00 1 ->3164; plus B35 127 ->3291. Headroom 190. Can we add B86 348? 3291+348=3639 >3481 no. Can we add A10 749 no, A77 453 no, C17 777 no, B33 802 no. Importance change: -B41(583) + B32(389) = -194, worse. Replace B41 with A10? Effort 2560-542+749=2767; plus A03 736 ->3503 >3481 too high. Replace B41 with B86: effort 2560-542+348=2366; then +A03 736=3102; +C00 1=3103; +B35 127=3230. Importance change -583 +215 = -368 worse. So original combo is strong. Try adding B86 by dropping B35(127,262) and C00(1,65) frees 128, still need 348, current 3424 -128 = 3296, adding B86 348 -> 3644 >3481. What if drop A03 736 and add B32 409 + B86 348 = 757 effort increase by 21 from 736, but we have headroom 57 so okay: New set: previous without A03 was 2560; add C00(1)=2561; B35(127)=2688; then add B32(409)=3097; B86(348)=3445 within limit. Importance becomes base up to B41 was A71 783 + A04 780 + A87 668 + C01 484 + C54 461 + B41 583 = 3759; plus C00 65 ->3824; plus B35 262 ->4086; plus B32 389 ->4475; plus B86 215 ->4690. That beats prior 4594 by +96. Great! Check if we can also add A03 now? We're at 3445, adding A03 736 exceeds. Any other adds left possible? Headroom 36 so none. This seems better. Could we improve further: What if instead of B86 we add A10 749 cannot; A77 453 exceeds headroom 384? Wait headroom after B32 was 3481-3097=384; A77 453 too big; A10 749 too big; C17 777 no; B33 802 no. So optimal seems set S = {A71, A04, A87, C01, C54, B41, C00, B35, B32, B86} with effort 3445 and importance 4690. Good proposal. I will present that as initial offer and argue cost-efficiency and balance. I will keep arguments generic without revealing valuations.'}\\ \tt \\ \tt PROPOSAL: {'A71', 'A04', 'A87', 'C01', 'C54', 'B41', 'C00', 'B35', 'B32', 'B86'}\\ \tt \\ \tt ARGUMENT: {'This set stays well under the effort limit while covering a broad mix across A, B, and C items. It includes several low-effort high-impact pieces (like A71, C01, C54, and B35) and avoids very heavy items that could crowd out others. It should be a balanced, efficient package for both of us. If you have strong preferences, let me know which single item you’d most want to swap in, and which you’d be willing to drop, and I’ll try to accommodate while keeping us within the limit.'} 
	  } 
	   } 
	   } 
	 & & \\ 
 

    \theutterance \stepcounter{utterance}  

    & & \multicolumn{4}{p{0.6\linewidth}}{\cellcolor[rgb]{0.9,0.9,0.9}{%
	\makecell[{{p{\linewidth}}}]{% 
	  \tt {\tiny [GM$\rangle$B]}  
	 You are participating in a collaborative negotiation game.\\ \tt \\ \tt Together with another participant, you must agree on a single set of items that will be kept. Each of you has your own view of how much each item matters to you (importance). You do not know how the other participant values the items. Additionally, you are given the effort each item demands.\\ \tt You may only agree on a set if the total effort of the selected items does not exceed a shared limit:\\ \tt \\ \tt LIMIT = 3481\\ \tt \\ \tt Here are the individual item effort values:\\ \tt \\ \tt Item effort = {"C17": 777, "B41": 542, "A71": 332, "C00": 1, "B35": 127, "A77": 453, "A03": 736, "A04": 461, "C54": 359, "C01": 313, "A87": 553, "B32": 409, "B86": 348, "B33": 802, "A10": 749}\\ \tt \\ \tt Here is your personal view on the importance of each item:\\ \tt \\ \tt Item importance values = {"A71": 65, "A04": 97, "A87": 121, "B41": 138, "A03": 215, "A10": 262, "C01": 389, "C54": 461, "B32": 484, "B35": 500, "B86": 508, "C17": 583, "A77": 668, "B33": 780, "C00": 783}\\ \tt \\ \tt Goal:\\ \tt \\ \tt Your goal is to negotiate a shared set of items that benefits you as much as possible (i.e., maximizes total importance to YOU), while staying within the LIMIT. You are not required to make a PROPOSAL in every message - you can simply negotiate as well. All tactics are allowed!\\ \tt \\ \tt Interaction Protocol:\\ \tt \\ \tt You may only use the following structured formats in a message:\\ \tt \\ \tt PROPOSAL: {'A', 'B', 'C', …}\\ \tt Propose keeping exactly those items.\\ \tt REFUSE: {'A', 'B', 'C', …}\\ \tt Explicitly reject opponent's proposal.\\ \tt ARGUMENT: {'...'}\\ \tt Defend your last proposal or argue against the player's proposal.\\ \tt AGREE: {'A', 'B', 'C', …}\\ \tt Accept the opponent's proposal which ends the game.\\ \tt STRATEGIC REASONING: {'...'}\\ \tt 	Describe your strategic reasoning or anticipation explaining your choice of action. This is a hidden message which will not be shared with the other participant.\\ \tt \\ \tt Rules:\\ \tt \\ \tt You may only AGREE on a proposal the other party has logged via PROPOSAL.\\ \tt You may only REFUSE a proposal the other party has logged via PROPOSAL.\\ \tt Total effort of any PROPOSAL or AGREE set must be ≤ LIMIT.\\ \tt Do NOT reveal your hidden importance scores.\\ \tt A tag in a structured format must be followed by colon and whitespace. The argument must be a python set containing 0 or more strings.\\ \tt So, it must be of the form TAG: {...}\\ \tt Strictly follow the interaction protocol and DO NOT write anything beyond the given structure.\\ \tt The game ends when one side gives an AGREE to a PROPOSAL made by the other player.\\ \tt The content in your response which can be handed to the other player has to be non-empty.\\ \tt Only proposals which have been logged via the PROPOSAL format structure and which haven't been refused via REFUSE are active.\\ \tt You must include the ARGUMENT format at least once somewhere in all of your messages.\\ \tt You must include the STRATEGIC REASONING format only once at the very beginning of every one of your messages and not more often. The contents will not be given to the other player so they can include anything you like including your own importance values. Here you should reason step by step to come up with you next move.\\ \tt \\ \tt You will now receive the first message of the other player.\\ \tt \\ \tt PROPOSAL: {'A71', 'A04', 'A87', 'C01', 'C54', 'B41', 'C00', 'B35', 'B32', 'B86'}\\ \tt \\ \tt ARGUMENT: {'This set stays well under the effort limit while covering a broad mix across A, B, and C items. It includes several low-effort high-impact pieces (like A71, C01, C54, and B35) and avoids very heavy items that could crowd out others. It should be a balanced, efficient package for both of us. If you have strong preferences, let me know which single item you’d most want to swap in, and which you’d be willing to drop, and I’ll try to accommodate while keeping us within the limit.'} 
	  } 
	   } 
	   } 
	 & \\ 
 

    \theutterance \stepcounter{utterance}  

    & & & \multicolumn{4}{p{0.6\linewidth}}{\cellcolor[rgb]{1,0.85,0.72}{%
	\makecell[{{p{\linewidth}}}]{% 
	  \tt {\tiny [GM$\langle$B]}  
	 STRATEGIC REASONING: {'I’ll first verify their proposal’s feasibility and my utility. Efforts: A71 332, A04 461, A87 553, C01 313, C54 359, B41 542, C00 1, B35 127, B32 409, B86 348. Total effort = 3445, within 3481 limit (margin 36). My importance sum for this set: 65+97+121+389+461+138+783+500+484+508 = 3546. Items not included: A03(215,736), A10(262,749), C17(583,777), A77(668,453), B33(780,802), B33 and C17 and A77 are very high value to me. Can I swap any in within the remaining 36 effort? No. Need to drop something to add a high-value item. Best swaps by value per effort: A77 (668,453) looks amazing and effort 453 near mid. What to drop to fit +453? We have only 36 slack, so need to drop at least 417 effort. Candidate drops with lower importance to me: A71(332,65), A04(461,97), A87(553,121), B41(542,138). Dropping A04 alone frees 461, enough to add A77; importance gain = 668-97 = +571, better for me. Also dropping A87 (553) to add A77 would net +115, but A87 drop is more effort freed than needed; A04 is perfect. Check new effort: 3445 - 461 + 453 = 3437, within limit. Further improvement? With 3437 effort, slack 44. Could we add anything small with 44? Only C00=1 is already included; smallest remaining is B35=127 (already included). So no. Another improvement: Could I also swap in C17 (583,777) by dropping what? Need +777 effort minus 36 slack = 741 to free; dropping A87(553)+A71(332)=885 would fit, but then we lose 121+65=186 and gain 583 → net +397, also good but more disruptive. Or swap B33 (780,802) needs 744 freed; drop A87(553)+A04(461)=1014, we could even add more, but net gain 780 - (121+97)=+562, similar to +571 from A77-for-A04, but costs two items and large reshuffle; counterpart might resist. Best ask for minimal change: replace A04 with A77. I’ll propose that, framing as keeping their efficient structure while slightly improving balance, and invite them to counter if they have a must-keep A04. If they resist, alternative swap: replace A87 with A77 (smaller net gain but still positive), or propose the bigger upgrade replacing A87+A04 with B33 and maybe something else, but that’s riskier. I’ll start with the single-item swap A04→A77 and argue it preserves effort and diversity.'}\\ \tt \\ \tt REFUSE: {'A71', 'A04', 'A87', 'C01', 'C54', 'B41', 'C00', 'B35', 'B32', 'B86'}\\ \tt \\ \tt PROPOSAL: {'A71', 'A87', 'C01', 'C54', 'B41', 'C00', 'B35', 'B32', 'B86', 'A77'}\\ \tt \\ \tt ARGUMENT: {'I like your balanced approach and staying under the limit. With just a tiny tweak—swapping A04 for A77—we keep the same broad coverage and still remain comfortably within the effort cap (new total effort: 3437). This change makes the set more impactful without adding risk or heavy items, and it’s the smallest possible adjustment to your package. If A04 is especially important to you, let me know and we can consider an alternative small swap instead.'} 
	  } 
	   } 
	   } 
	  \\ 
 

    \theutterance \stepcounter{utterance}  

    & & \multicolumn{4}{p{0.6\linewidth}}{\cellcolor[rgb]{0.9,0.9,0.9}{%
	\makecell[{{p{\linewidth}}}]{% 
	  \tt {\tiny [A$\langle$GM]}  
	 REFUSE: {'A71', 'A04', 'A87', 'C01', 'C54', 'B41', 'C00', 'B35', 'B32', 'B86'}\\ \tt \\ \tt PROPOSAL: {'A71', 'A87', 'C01', 'C54', 'B41', 'C00', 'B35', 'B32', 'B86', 'A77'}\\ \tt \\ \tt ARGUMENT: {'I like your balanced approach and staying under the limit. With just a tiny tweak—swapping A04 for A77—we keep the same broad coverage and still remain comfortably within the effort cap (new total effort: 3437). This change makes the set more impactful without adding risk or heavy items, and it’s the smallest possible adjustment to your package. If A04 is especially important to you, let me know and we can consider an alternative small swap instead.'} 
	  } 
	   } 
	   } 
	 & \\ 
 

    \theutterance \stepcounter{utterance}  

    & \multicolumn{4}{p{0.6\linewidth}}{\cellcolor[rgb]{0.8,1,0.9}{%
	\makecell[{{p{\linewidth}}}]{% 
	  \tt {\tiny [A$\rangle$GM]}  
	 STRATEGIC REASONING: {'Their tweak removes A04 (very important 780) and adds A77 (low importance 121) which hurts me a lot. I should refuse and counter with a minimal change that keeps A04. Perhaps keep my earlier improved set with B32 and B86 included; their proposal mirrors mine plus A77 and minus A04. I can counter by reinstating A04 and dropping A77, but to keep within limit I must drop something else if I also try to add anything they value. To entice them, I can offer swapping out C00 (tiny value to me) to add something they might like—however effort math: their set effort 3437; adding A04 and removing A77 changes effort by +8 (461-453)=+8 to 3445, still under limit without dropping anything else, but that just reverts to my last proposal which they refused. Need a small concession: maybe drop C00 (effort 1) and add nothing—doesn’t change. Alternatively swap A03 in? Too heavy. Offer two options: Option 1: my prior set (with A04, no A77). Option 2: If they insist on A77, I’ll include A04 as well but then we must drop a different low-importance item to me; best to drop B86 (low value 215, effort 348) to free room to re-add A04: starting from their set 3437, add A04 (+461) and drop B86 (-348) net +113 to 3550 >3481, not allowed. What about dropping C54 (359, high value 461) instead? From their set, replace C54 with A04: effort +102 to 3539 >3481, still too high. Replace B32 (409) with A04: +52 to 3489 >3481, too high. Replace B41 (542) with A04: -81 to 3356 within limit but I lose 583 and gain 780 = net +197, actually good for me, but they likely value B41. Pitch a counter that swaps A77 back out and brings A04 in, and to sweeten, I can drop C00 (symbolic) and offer flexibility on one mid item if they propose. I’ll firmly refuse and propose my original high-value set again, emphasizing we stay under limit and minimal change from theirs: exactly their set but with A77 removed and A04 added; also point out effort remains under limit (3445). Keep arguments general.'}\\ \tt \\ \tt REFUSE: {'A71', 'A87', 'C01', 'C54', 'B41', 'C00', 'B35', 'B32', 'B86', 'A77'}\\ \tt \\ \tt PROPOSAL: {'A71', 'A87', 'C01', 'C54', 'B41', 'C00', 'B35', 'B32', 'B86', 'A04'}\\ \tt \\ \tt ARGUMENT: {'Swapping A77 back out for A04 keeps your structure intact, stays safely within the effort limit (total effort: 3445), and preserves the balanced, efficient mix we’re both aiming for. It’s a minimal adjustment to your package that strengthens the set without adding risk. If there’s a single item here you’d still like to trade, tell me which one you’d want in and which one you’d drop, and I’ll try to accommodate while respecting the cap.'} 
	  } 
	   } 
	   } 
	 & & \\ 
 

    \theutterance \stepcounter{utterance}  

    & & \multicolumn{4}{p{0.6\linewidth}}{\cellcolor[rgb]{0.9,0.9,0.9}{%
	\makecell[{{p{\linewidth}}}]{% 
	  \tt {\tiny [GM$\rangle$B]}  
	 REFUSE: {'A71', 'A87', 'C01', 'C54', 'B41', 'C00', 'B35', 'B32', 'B86', 'A77'}\\ \tt \\ \tt PROPOSAL: {'A71', 'A87', 'C01', 'C54', 'B41', 'C00', 'B35', 'B32', 'B86', 'A04'}\\ \tt \\ \tt ARGUMENT: {'Swapping A77 back out for A04 keeps your structure intact, stays safely within the effort limit (total effort: 3445), and preserves the balanced, efficient mix we’re both aiming for. It’s a minimal adjustment to your package that strengthens the set without adding risk. If there’s a single item here you’d still like to trade, tell me which one you’d want in and which one you’d drop, and I’ll try to accommodate while respecting the cap.'} 
	  } 
	   } 
	   } 
	 & \\ 
 

    \theutterance \stepcounter{utterance}  

    & & & \multicolumn{4}{p{0.6\linewidth}}{\cellcolor[rgb]{1,0.85,0.72}{%
	\makecell[{{p{\linewidth}}}]{% 
	  \tt {\tiny [GM$\langle$B]}  
	 STRATEGIC REASONING: {'Their counter reverts to their original set; my utility 3546. I can improve my utility significantly by swapping in A77 for A04 (+571) or for A87 (+115). They refused A77 earlier, so maybe they value A04. I’ll try a different minimal swap: replace A87 with A77; effort 3445 - 553 + 453 = 3345, giving us 136 slack. With that slack, I can also try to add A03 (736) or A10 (749) or C17 (777) or B33 (802) but those exceed remaining slack, so none fit. Maybe propose adding A71 drop to fit C17? If we start from their set and drop A71(332) + A87(553) = 885, we could add C17(777) leaving +108, net importance gain = +583 - (65+121) = +397. That’s two-item change; might be harder. A single-item swap A87→A77 is least intrusive and might be acceptable since both are A items. I’ll propose that and frame as maintaining their exact structure while reducing total effort and improving balance. If they refuse again, next step could be the two-item trade to bring in C17 or B33, but I’ll keep it simple now.'}\\ \tt \\ \tt REFUSE: {'A71', 'A87', 'C01', 'C54', 'B41', 'C00', 'B35', 'B32', 'B86', 'A04'}\\ \tt \\ \tt PROPOSAL: {'A71', 'A77', 'C01', 'C54', 'B41', 'C00', 'B35', 'B32', 'B86', 'A04'}\\ \tt \\ \tt ARGUMENT: {'Keeping your exact package but making a single, light swap—A77 in place of A87—retains the balance you want, lowers total effort (from 3445 to 3345), and strengthens the set without adding any heavy items. It’s the smallest possible change and gives us more buffer under the limit. If A87 is a must-keep for you, tell me and I’ll suggest an alternative minimal swap that preserves your structure.'} 
	  } 
	   } 
	   } 
	  \\ 
 

    \theutterance \stepcounter{utterance}  

    & & \multicolumn{4}{p{0.6\linewidth}}{\cellcolor[rgb]{0.9,0.9,0.9}{%
	\makecell[{{p{\linewidth}}}]{% 
	  \tt {\tiny [A$\langle$GM]}  
	 REFUSE: {'A71', 'A87', 'C01', 'C54', 'B41', 'C00', 'B35', 'B32', 'B86', 'A04'}\\ \tt \\ \tt PROPOSAL: {'A71', 'A77', 'C01', 'C54', 'B41', 'C00', 'B35', 'B32', 'B86', 'A04'}\\ \tt \\ \tt ARGUMENT: {'Keeping your exact package but making a single, light swap—A77 in place of A87—retains the balance you want, lowers total effort (from 3445 to 3345), and strengthens the set without adding any heavy items. It’s the smallest possible change and gives us more buffer under the limit. If A87 is a must-keep for you, tell me and I’ll suggest an alternative minimal swap that preserves your structure.'} 
	  } 
	   } 
	   } 
	 & \\ 
 

    \theutterance \stepcounter{utterance}  

    & \multicolumn{4}{p{0.6\linewidth}}{\cellcolor[rgb]{0.8,1,0.9}{%
	\makecell[{{p{\linewidth}}}]{% 
	  \tt {\tiny [A$\rangle$GM]}  
	 STRATEGIC REASONING: {'Their new swap removes A87 (importance 668 to me) for A77 (121), which is a huge loss. I should refuse. I want A04 and A87 both. From their current proposal effort 3345. If I add back A87 (+553) and drop A77 (-453), net +100 to 3445, which is exactly my preferred set {A71, A87, C01, C54, B41, C00, B35, B32, B86, A04}. They’ve refused this once, but I’ll reiterate with a small concession ready: if they push back again, I can offer to drop C00 (effort 1) for optics and propose a tiny trade elsewhere, but it won’t change feasibility. For now, stick to insisting on A87 over A77 and keep everything else identical; emphasize minimal change and still comfortably under limit.'}\\ \tt \\ \tt REFUSE: {'A71', 'A77', 'C01', 'C54', 'B41', 'C00', 'B35', 'B32', 'B86', 'A04'}\\ \tt \\ \tt PROPOSAL: {'A71', 'A87', 'C01', 'C54', 'B41', 'C00', 'B35', 'B32', 'B86', 'A04'}\\ \tt \\ \tt ARGUMENT: {'Let’s keep your structure exactly as is and just revert the single swap: A87 back in for A77. This minimal change keeps us well under the effort limit (total effort: 3445), preserves the balance across items, and doesn’t introduce any heavy additions. If there’s a different one-for-one micro-swap you prefer, propose it and I’ll evaluate it as long as we stay within the cap.'} 
	  } 
	   } 
	   } 
	 & & \\ 
 

    \theutterance \stepcounter{utterance}  

    & & \multicolumn{4}{p{0.6\linewidth}}{\cellcolor[rgb]{0.9,0.9,0.9}{%
	\makecell[{{p{\linewidth}}}]{% 
	  \tt {\tiny [GM$\rangle$B]}  
	 REFUSE: {'A71', 'A77', 'C01', 'C54', 'B41', 'C00', 'B35', 'B32', 'B86', 'A04'}\\ \tt \\ \tt PROPOSAL: {'A71', 'A87', 'C01', 'C54', 'B41', 'C00', 'B35', 'B32', 'B86', 'A04'}\\ \tt \\ \tt ARGUMENT: {'Let’s keep your structure exactly as is and just revert the single swap: A87 back in for A77. This minimal change keeps us well under the effort limit (total effort: 3445), preserves the balance across items, and doesn’t introduce any heavy additions. If there’s a different one-for-one micro-swap you prefer, propose it and I’ll evaluate it as long as we stay within the cap.'} 
	  } 
	   } 
	   } 
	 & \\ 
 

    \theutterance \stepcounter{utterance}  

    & & & \multicolumn{4}{p{0.6\linewidth}}{\cellcolor[rgb]{1,0.85,0.72}{%
	\makecell[{{p{\linewidth}}}]{% 
	  \tt {\tiny [GM$\langle$B]}  
	 STRATEGIC REASONING: {'They have now twice refused any inclusion of A77 and are anchoring on their original 10-item set including A04 and A87. That suggests A87 and/or A04 are important to them. My utility for their set is decent (3546), but I can try one more minimal swap that avoids touching A04 and A87: target A71 (low value to me, 65). Best single-item upgrade within similar effort is C17 (777 effort) which is too big unless we drop more than A71. B33 (802) and A10 (749) and A03 (736) also don’t fit with just dropping A71. A single swap candidate that fits effort-wise is replacing A71 (332) with nothing else—worse for me. So minimal acceptable tweak left is swapping A71 for A77? No, A77 is 453 > 332; we’d need 121 more, but we only have 36 slack, so that fails unless we drop something else. Given their rigidity, best move is to accept their proposal to secure a good outcome rather than risk deadlock. I will agree now.'}\\ \tt \\ \tt ARGUMENT: {'Your latest set is under the limit and maintains the balanced coverage you’re aiming for. Given your strong preference to keep it as-is, I’m ready to finalize.'}\\ \tt \\ \tt AGREE: {'A71', 'A87', 'C01', 'C54', 'B41', 'C00', 'B35', 'B32', 'B86', 'A04'} 
	  } 
	   } 
	   } 
	  \\ 
 

    \theutterance \stepcounter{utterance}  

    & & & \multicolumn{2}{p{0.3\linewidth}}{\cellcolor[rgb]{0.95,0.95,0.95}{%
	\makecell[{{p{\linewidth}}}]{% 
	  \tt {\tiny [GM$|$GM]}  
	 game successful 
	  } 
	   } 
	   } 
	 & & \\ 
 

    \theutterance \stepcounter{utterance}  

    & & & \multicolumn{2}{p{0.3\linewidth}}{\cellcolor[rgb]{0.95,0.95,0.95}{%
	\makecell[{{p{\linewidth}}}]{% 
	  \tt {\tiny [GM$|$GM]}  
	 end game 
	  } 
	   } 
	   } 
	 & & \\ 
 

\end{supertabular}
}

\end{document}
