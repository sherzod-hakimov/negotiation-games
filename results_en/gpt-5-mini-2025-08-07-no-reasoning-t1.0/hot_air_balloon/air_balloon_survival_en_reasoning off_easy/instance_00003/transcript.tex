
\documentclass{article}
\usepackage{colortbl}
\usepackage{makecell}
\usepackage{multirow}
\usepackage{supertabular}

\begin{document}

\newcounter{utterance}

\twocolumn

{ \footnotesize  \setcounter{utterance}{1}
\setlength{\tabcolsep}{0pt}
\begin{supertabular}{c@{$\;$}|p{.15\linewidth}@{}p{.15\linewidth}p{.15\linewidth}p{.15\linewidth}p{.15\linewidth}p{.15\linewidth}}

    \# & $\;$A & \multicolumn{4}{c}{Game Master} & $\;\:$B\\
    \hline 

    \theutterance \stepcounter{utterance}  

    & & \multicolumn{4}{p{0.6\linewidth}}{\cellcolor[rgb]{0.9,0.9,0.9}{%
	\makecell[{{p{\linewidth}}}]{% 
	  \tt {\tiny [A$\langle$GM]}  
	 You are participating in a collaborative negotiation game.\\ \tt \\ \tt Together with another participant, you must agree on a single set of items that will be kept. Each of you has your own view of how much each item matters to you (importance). You do not know how the other participant values the items. Additionally, you are given the effort each item demands.\\ \tt You may only agree on a set if the total effort of the selected items does not exceed a shared limit:\\ \tt \\ \tt LIMIT = 3207\\ \tt \\ \tt Here are the individual item effort values:\\ \tt \\ \tt Item effort = {"A19": 67, "B91": 45, "A30": 308, "B11": 835, "C47": 819, "C75": 667, "C26": 547, "C73": 322, "A47": 428, "A15": 306, "A75": 327, "A39": 362, "A56": 280, "C24": 334, "B26": 767}\\ \tt \\ \tt Here is your personal view on the importance of each item:\\ \tt \\ \tt Item importance values = {"A19": 138, "B91": 583, "A30": 822, "B11": 783, "C47": 65, "C75": 262, "C26": 121, "C73": 508, "A47": 780, "A15": 461, "A75": 484, "A39": 668, "A56": 389, "C24": 808, "B26": 215}\\ \tt \\ \tt Goal:\\ \tt \\ \tt Your goal is to negotiate a shared set of items that benefits you as much as possible (i.e., maximizes total importance to YOU), while staying within the LIMIT. You are not required to make a PROPOSAL in every message - you can simply negotiate as well. All tactics are allowed!\\ \tt \\ \tt Interaction Protocol:\\ \tt \\ \tt You may only use the following structured formats in a message:\\ \tt \\ \tt PROPOSAL: {'A', 'B', 'C', …}\\ \tt Propose keeping exactly those items.\\ \tt REFUSE: {'A', 'B', 'C', …}\\ \tt Explicitly reject opponent's proposal.\\ \tt ARGUMENT: {'...'}\\ \tt Defend your last proposal or argue against the player's proposal.\\ \tt AGREE: {'A', 'B', 'C', …}\\ \tt Accept the opponent's proposal which ends the game.\\ \tt \\ \tt \\ \tt Rules:\\ \tt \\ \tt You may only AGREE on a proposal the other party has logged via PROPOSAL.\\ \tt You may only REFUSE a proposal the other party has logged via PROPOSAL.\\ \tt Total effort of any PROPOSAL or AGREE set must be ≤ LIMIT.\\ \tt Do NOT reveal your hidden importance scores.\\ \tt A tag in a structured format must be followed by colon and whitespace. The argument must be a python set containing 0 or more strings.\\ \tt So, it must be of the form TAG: {...}\\ \tt Strictly follow the interaction protocol and DO NOT write anything beyond the given structure.\\ \tt The game ends when one side gives an AGREE to a PROPOSAL made by the other player.\\ \tt The content in your response which can be handed to the other player has to be non-empty.\\ \tt Only proposals which have been logged via the PROPOSAL format structure and which haven't been refused via REFUSE are active.\\ \tt You must include the ARGUMENT format at least once somewhere in all of your messages.\\ \tt  
	  } 
	   } 
	   } 
	 & \\ 
 

    \theutterance \stepcounter{utterance}  

    & \multicolumn{4}{p{0.6\linewidth}}{\cellcolor[rgb]{0.8,1,0.9}{%
	\makecell[{{p{\linewidth}}}]{% 
	  \tt {\tiny [A$\rangle$GM]}  
	 PROPOSAL: {'A30', 'B11', 'C73', 'A47', 'A15', 'A75', 'A39', 'A56', 'C24'}\\ \tt ARGUMENT: {'I aimed for a balanced, high-impact set within the limit that emphasizes multiple A-items with strong combined value and includes selective B/C items where the effort-to-benefit tradeoff is reasonable. This set stays under the shared effort limit while capturing a broad spread, which should make it easier to find mutual gains. If you have strong preferences for specific items, I can adjust—let me know which ones matter most to you.'} 
	  } 
	   } 
	   } 
	 & & \\ 
 

    \theutterance \stepcounter{utterance}  

    & & \multicolumn{4}{p{0.6\linewidth}}{\cellcolor[rgb]{0.9,0.9,0.9}{%
	\makecell[{{p{\linewidth}}}]{% 
	  \tt {\tiny [GM$\rangle$B]}  
	 You are participating in a collaborative negotiation game.\\ \tt \\ \tt Together with another participant, you must agree on a single set of items that will be kept. Each of you has your own view of how much each item matters to you (importance). You do not know how the other participant values the items. Additionally, you are given the effort each item demands.\\ \tt You may only agree on a set if the total effort of the selected items does not exceed a shared limit:\\ \tt \\ \tt LIMIT = 3207\\ \tt \\ \tt Here are the individual item effort values:\\ \tt \\ \tt Item effort = {"A19": 67, "B91": 45, "A30": 308, "B11": 835, "C47": 819, "C75": 667, "C26": 547, "C73": 322, "A47": 428, "A15": 306, "A75": 327, "A39": 362, "A56": 280, "C24": 334, "B26": 767}\\ \tt \\ \tt Here is your personal view on the importance of each item:\\ \tt \\ \tt Item importance values = {"A19": 138, "B91": 583, "A30": 822, "B11": 783, "C47": 65, "C75": 262, "C26": 121, "C73": 508, "A47": 780, "A15": 461, "A75": 484, "A39": 668, "A56": 389, "C24": 808, "B26": 215}\\ \tt \\ \tt Goal:\\ \tt \\ \tt Your goal is to negotiate a shared set of items that benefits you as much as possible (i.e., maximizes total importance to YOU), while staying within the LIMIT. You are not required to make a PROPOSAL in every message - you can simply negotiate as well. All tactics are allowed!\\ \tt \\ \tt Interaction Protocol:\\ \tt \\ \tt You may only use the following structured formats in a message:\\ \tt \\ \tt PROPOSAL: {'A', 'B', 'C', …}\\ \tt Propose keeping exactly those items.\\ \tt REFUSE: {'A', 'B', 'C', …}\\ \tt Explicitly reject opponent's proposal.\\ \tt ARGUMENT: {'...'}\\ \tt Defend your last proposal or argue against the player's proposal.\\ \tt AGREE: {'A', 'B', 'C', …}\\ \tt Accept the opponent's proposal which ends the game.\\ \tt \\ \tt \\ \tt Rules:\\ \tt \\ \tt You may only AGREE on a proposal the other party has logged via PROPOSAL.\\ \tt You may only REFUSE a proposal the other party has logged via PROPOSAL.\\ \tt Total effort of any PROPOSAL or AGREE set must be ≤ LIMIT.\\ \tt Do NOT reveal your hidden importance scores.\\ \tt A tag in a structured format must be followed by colon and whitespace. The argument must be a python set containing 0 or more strings.\\ \tt So, it must be of the form TAG: {...}\\ \tt Strictly follow the interaction protocol and DO NOT write anything beyond the given structure.\\ \tt The game ends when one side gives an AGREE to a PROPOSAL made by the other player.\\ \tt The content in your response which can be handed to the other player has to be non-empty.\\ \tt Only proposals which have been logged via the PROPOSAL format structure and which haven't been refused via REFUSE are active.\\ \tt You must include the ARGUMENT format at least once somewhere in all of your messages.\\ \tt \\ \tt \\ \tt You will now receive the first message of the other player.\\ \tt \\ \tt PROPOSAL: {'A30', 'B11', 'C73', 'A47', 'A15', 'A75', 'A39', 'A56', 'C24'}\\ \tt ARGUMENT: {'I aimed for a balanced, high-impact set within the limit that emphasizes multiple A-items with strong combined value and includes selective B/C items where the effort-to-benefit tradeoff is reasonable. This set stays under the shared effort limit while capturing a broad spread, which should make it easier to find mutual gains. If you have strong preferences for specific items, I can adjust—let me know which ones matter most to you.'} 
	  } 
	   } 
	   } 
	 & \\ 
 

    \theutterance \stepcounter{utterance}  

    & & & \multicolumn{4}{p{0.6\linewidth}}{\cellcolor[rgb]{1,0.85,0.72}{%
	\makecell[{{p{\linewidth}}}]{% 
	  \tt {\tiny [GM$\langle$B]}  
	 REFUSE: {'A30', 'B11', 'C73', 'A47', 'A15', 'A75', 'A39', 'A56', 'C24'}\\ \tt ARGUMENT: {'Your set exceeds our shared effort limit once I factor in the high-effort B11 (835) alongside multiple mid-to-high A/C items. I’m proposing a near-max-value mix for me that stays safely under the limit while keeping a broad spread of A’s and key C’s. I’m open to swaps if you have must-haves.'}\\ \tt PROPOSAL: {'A30', 'C73', 'A47', 'A15', 'A75', 'A39', 'A56', 'C24', 'B91', 'A19'} 
	  } 
	   } 
	   } 
	  \\ 
 

    \theutterance \stepcounter{utterance}  

    & & \multicolumn{4}{p{0.6\linewidth}}{\cellcolor[rgb]{0.9,0.9,0.9}{%
	\makecell[{{p{\linewidth}}}]{% 
	  \tt {\tiny [A$\langle$GM]}  
	 REFUSE: {'A30', 'B11', 'C73', 'A47', 'A15', 'A75', 'A39', 'A56', 'C24'}\\ \tt ARGUMENT: {'Your set exceeds our shared effort limit once I factor in the high-effort B11 (835) alongside multiple mid-to-high A/C items. I’m proposing a near-max-value mix for me that stays safely under the limit while keeping a broad spread of A’s and key C’s. I’m open to swaps if you have must-haves.'}\\ \tt PROPOSAL: {'A30', 'C73', 'A47', 'A15', 'A75', 'A39', 'A56', 'C24', 'B91', 'A19'} 
	  } 
	   } 
	   } 
	 & \\ 
 

    \theutterance \stepcounter{utterance}  

    & \multicolumn{4}{p{0.6\linewidth}}{\cellcolor[rgb]{0.8,1,0.9}{%
	\makecell[{{p{\linewidth}}}]{% 
	  \tt {\tiny [A$\rangle$GM]}  
	 REFUSE: {'A30', 'C73', 'A47', 'A15', 'A75', 'A39', 'A56', 'C24', 'B91', 'A19'}\\ \tt ARGUMENT: {'Thanks for tightening the set. I like much of this, but I’d prefer trading out lower-impact items for me to upgrade overall value without breaking the limit. In particular, B11 is heavy but very impactful for me; we can accommodate it by dropping some mid-value pieces while keeping the strong core.'}\\ \tt PROPOSAL: {'A30', 'B11', 'C73', 'A47', 'A15', 'A39', 'C24', 'B91', 'A19'} 
	  } 
	   } 
	   } 
	 & & \\ 
 

    \theutterance \stepcounter{utterance}  

    & & \multicolumn{4}{p{0.6\linewidth}}{\cellcolor[rgb]{0.9,0.9,0.9}{%
	\makecell[{{p{\linewidth}}}]{% 
	  \tt {\tiny [GM$\rangle$B]}  
	 REFUSE: {'A30', 'C73', 'A47', 'A15', 'A75', 'A39', 'A56', 'C24', 'B91', 'A19'}\\ \tt ARGUMENT: {'Thanks for tightening the set. I like much of this, but I’d prefer trading out lower-impact items for me to upgrade overall value without breaking the limit. In particular, B11 is heavy but very impactful for me; we can accommodate it by dropping some mid-value pieces while keeping the strong core.'}\\ \tt PROPOSAL: {'A30', 'B11', 'C73', 'A47', 'A15', 'A39', 'C24', 'B91', 'A19'} 
	  } 
	   } 
	   } 
	 & \\ 
 

    \theutterance \stepcounter{utterance}  

    & & & \multicolumn{4}{p{0.6\linewidth}}{\cellcolor[rgb]{1,0.85,0.72}{%
	\makecell[{{p{\linewidth}}}]{% 
	  \tt {\tiny [GM$\langle$B]}  
	 REFUSE: {'A30', 'B11', 'C73', 'A47', 'A15', 'A39', 'C24', 'B91', 'A19'}\\ \tt ARGUMENT: {'I can’t accept including B11—its effort (835) forces us to cut multiple high-yield items for me. We can reach the limit cleanly while preserving a much stronger package on my side by excluding B11 and keeping a balanced mix of A/C plus B91.'}\\ \tt PROPOSAL: {'A30', 'C73', 'A47', 'A15', 'A39', 'A56', 'C24', 'B91', 'A19'} 
	  } 
	   } 
	   } 
	  \\ 
 

    \theutterance \stepcounter{utterance}  

    & & \multicolumn{4}{p{0.6\linewidth}}{\cellcolor[rgb]{0.9,0.9,0.9}{%
	\makecell[{{p{\linewidth}}}]{% 
	  \tt {\tiny [A$\langle$GM]}  
	 REFUSE: {'A30', 'B11', 'C73', 'A47', 'A15', 'A39', 'C24', 'B91', 'A19'}\\ \tt ARGUMENT: {'I can’t accept including B11—its effort (835) forces us to cut multiple high-yield items for me. We can reach the limit cleanly while preserving a much stronger package on my side by excluding B11 and keeping a balanced mix of A/C plus B91.'}\\ \tt PROPOSAL: {'A30', 'C73', 'A47', 'A15', 'A39', 'A56', 'C24', 'B91', 'A19'} 
	  } 
	   } 
	   } 
	 & \\ 
 

    \theutterance \stepcounter{utterance}  

    & \multicolumn{4}{p{0.6\linewidth}}{\cellcolor[rgb]{0.8,1,0.9}{%
	\makecell[{{p{\linewidth}}}]{% 
	  \tt {\tiny [A$\rangle$GM]}  
	 REFUSE: {'A30', 'C73', 'A47', 'A15', 'A39', 'A56', 'C24', 'B91', 'A19'}\\ \tt ARGUMENT: {'I appreciate the balance and agree on most of your core. To make this work for me, I need one upgrade swap that keeps us within the limit: replacing A56 with A75. This maintains your broad A/C spread, retains B91, and stays under the cap while improving my side’s value without asking for B11.'}\\ \tt PROPOSAL: {'A30', 'C73', 'A47', 'A15', 'A75', 'A39', 'C24', 'B91', 'A19'} 
	  } 
	   } 
	   } 
	 & & \\ 
 

    \theutterance \stepcounter{utterance}  

    & & \multicolumn{4}{p{0.6\linewidth}}{\cellcolor[rgb]{0.9,0.9,0.9}{%
	\makecell[{{p{\linewidth}}}]{% 
	  \tt {\tiny [GM$\rangle$B]}  
	 REFUSE: {'A30', 'C73', 'A47', 'A15', 'A39', 'A56', 'C24', 'B91', 'A19'}\\ \tt ARGUMENT: {'I appreciate the balance and agree on most of your core. To make this work for me, I need one upgrade swap that keeps us within the limit: replacing A56 with A75. This maintains your broad A/C spread, retains B91, and stays under the cap while improving my side’s value without asking for B11.'}\\ \tt PROPOSAL: {'A30', 'C73', 'A47', 'A15', 'A75', 'A39', 'C24', 'B91', 'A19'} 
	  } 
	   } 
	   } 
	 & \\ 
 

    \theutterance \stepcounter{utterance}  

    & & & \multicolumn{4}{p{0.6\linewidth}}{\cellcolor[rgb]{1,0.85,0.72}{%
	\makecell[{{p{\linewidth}}}]{% 
	  \tt {\tiny [GM$\langle$B]}  
	 REFUSE: {'A30', 'C73', 'A47', 'A15', 'A75', 'A39', 'C24', 'B91', 'A19'}\\ \tt ARGUMENT: {'I’m on board with most of this, but swapping A56 (280) for A75 (327) pushes us over the limit when combined with the rest of your set. If we keep your upgrade to A75, we can stay within 3207 by dropping the lightest, least critical item for me: A19 (67). That preserves your requested swap and the strong core without exceeding the cap.'}\\ \tt PROPOSAL: {'A30', 'C73', 'A47', 'A15', 'A75', 'A39', 'C24', 'B91'} 
	  } 
	   } 
	   } 
	  \\ 
 

    \theutterance \stepcounter{utterance}  

    & & \multicolumn{4}{p{0.6\linewidth}}{\cellcolor[rgb]{0.9,0.9,0.9}{%
	\makecell[{{p{\linewidth}}}]{% 
	  \tt {\tiny [A$\langle$GM]}  
	 REFUSE: {'A30', 'C73', 'A47', 'A15', 'A75', 'A39', 'C24', 'B91', 'A19'}\\ \tt ARGUMENT: {'I’m on board with most of this, but swapping A56 (280) for A75 (327) pushes us over the limit when combined with the rest of your set. If we keep your upgrade to A75, we can stay within 3207 by dropping the lightest, least critical item for me: A19 (67). That preserves your requested swap and the strong core without exceeding the cap.'}\\ \tt PROPOSAL: {'A30', 'C73', 'A47', 'A15', 'A75', 'A39', 'C24', 'B91'} 
	  } 
	   } 
	   } 
	 & \\ 
 

    \theutterance \stepcounter{utterance}  

    & \multicolumn{4}{p{0.6\linewidth}}{\cellcolor[rgb]{0.8,1,0.9}{%
	\makecell[{{p{\linewidth}}}]{% 
	  \tt {\tiny [A$\rangle$GM]}  
	 ARGUMENT: {'Dropping A19 while keeping the A75 swap works for me and respects the cap. This keeps the strong core we both favored and avoids B11, aligning with your constraints while still giving me solid coverage across A and C with B91 included.'}\\ \tt AGREE: {'A30', 'C73', 'A47', 'A15', 'A75', 'A39', 'C24', 'B91'} 
	  } 
	   } 
	   } 
	 & & \\ 
 

    \theutterance \stepcounter{utterance}  

    & & & \multicolumn{2}{p{0.3\linewidth}}{\cellcolor[rgb]{0.95,0.95,0.95}{%
	\makecell[{{p{\linewidth}}}]{% 
	  \tt {\tiny [GM$|$GM]}  
	 game successful 
	  } 
	   } 
	   } 
	 & & \\ 
 

    \theutterance \stepcounter{utterance}  

    & & & \multicolumn{2}{p{0.3\linewidth}}{\cellcolor[rgb]{0.95,0.95,0.95}{%
	\makecell[{{p{\linewidth}}}]{% 
	  \tt {\tiny [GM$|$GM]}  
	 end game 
	  } 
	   } 
	   } 
	 & & \\ 
 

\end{supertabular}
}

\end{document}
