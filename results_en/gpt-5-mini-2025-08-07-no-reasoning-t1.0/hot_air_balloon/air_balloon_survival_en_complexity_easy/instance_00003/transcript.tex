
\documentclass{article}
\usepackage{colortbl}
\usepackage{makecell}
\usepackage{multirow}
\usepackage{supertabular}

\begin{document}

\newcounter{utterance}

\twocolumn

{ \footnotesize  \setcounter{utterance}{1}
\setlength{\tabcolsep}{0pt}
\begin{supertabular}{c@{$\;$}|p{.15\linewidth}@{}p{.15\linewidth}p{.15\linewidth}p{.15\linewidth}p{.15\linewidth}p{.15\linewidth}}

    \# & $\;$A & \multicolumn{4}{c}{Game Master} & $\;\:$B\\
    \hline 

    \theutterance \stepcounter{utterance}  

    & & \multicolumn{4}{p{0.6\linewidth}}{\cellcolor[rgb]{0.9,0.9,0.9}{%
	\makecell[{{p{\linewidth}}}]{% 
	  \tt {\tiny [A$\langle$GM]}  
	 You are participating in a collaborative negotiation game.\\ \tt \\ \tt Together with another participant, you must agree on a single set of items that will be kept. Each of you has your own view of how much each item matters to you (importance). You do not know how the other participant values the items. Additionally, you are given the effort each item demands.\\ \tt You may only agree on a set if the total effort of the selected items does not exceed a shared limit:\\ \tt \\ \tt LIMIT = 8770\\ \tt \\ \tt Here are the individual item effort values:\\ \tt \\ \tt Item effort = {"B21": 830, "B00": 638, "A08": 607, "A27": 106, "B94": 630, "B53": 382, "A23": 773, "C65": 455, "A89": 262, "B67": 634, "B45": 60, "C34": 54, "C12": 849, "A05": 801, "B60": 324, "C92": 164, "B69": 832, "A84": 136, "C06": 980, "B12": 946, "A16": 645, "B85": 842, "C35": 106, "B10": 116, "C15": 875, "B09": 446, "B16": 649, "C81": 601, "A58": 252, "C24": 763, "A30": 213, "A66": 517, "B75": 520, "B46": 407, "C70": 125}\\ \tt \\ \tt Here is your personal view on the importance of each item:\\ \tt \\ \tt Item importance values = {"B21": 138, "B00": 583, "A08": 868, "A27": 822, "B94": 783, "B53": 65, "A23": 262, "C65": 121, "A89": 508, "B67": 780, "B45": 461, "C34": 484, "C12": 668, "A05": 389, "B60": 808, "C92": 215, "B69": 97, "A84": 500, "C06": 30, "B12": 915, "A16": 856, "B85": 400, "C35": 444, "B10": 623, "C15": 781, "B09": 786, "B16": 3, "C81": 713, "A58": 457, "C24": 273, "A30": 739, "A66": 822, "B75": 235, "B46": 606, "C70": 968}\\ \tt \\ \tt Goal:\\ \tt \\ \tt Your goal is to negotiate a shared set of items that benefits you as much as possible (i.e., maximizes total importance to YOU), while staying within the LIMIT. You are not required to make a PROPOSAL in every message - you can simply negotiate as well. All tactics are allowed!\\ \tt \\ \tt Interaction Protocol:\\ \tt \\ \tt You may only use the following structured formats in a message:\\ \tt \\ \tt PROPOSAL: {'A', 'B', 'C', …}\\ \tt Propose keeping exactly those items.\\ \tt REFUSE: {'A', 'B', 'C', …}\\ \tt Explicitly reject opponent's proposal.\\ \tt ARGUMENT: {'...'}\\ \tt Defend your last proposal or argue against the player's proposal.\\ \tt AGREE: {'A', 'B', 'C', …}\\ \tt Accept the opponent's proposal which ends the game.\\ \tt STRATEGIC REASONING: {'...'}\\ \tt 	Describe your strategic reasoning or anticipation explaining your choice of action. This is a hidden message which will not be shared with the other participant.\\ \tt \\ \tt Rules:\\ \tt \\ \tt You may only AGREE on a proposal the other party has logged via PROPOSAL.\\ \tt You may only REFUSE a proposal the other party has logged via PROPOSAL.\\ \tt Total effort of any PROPOSAL or AGREE set must be ≤ LIMIT.\\ \tt Do NOT reveal your hidden importance scores.\\ \tt A tag in a structured format must be followed by colon and whitespace. The argument must be a python set containing 0 or more strings.\\ \tt So, it must be of the form TAG: {...}\\ \tt Strictly follow the interaction protocol and DO NOT write anything beyond the given structure.\\ \tt The game ends when one side gives an AGREE to a PROPOSAL made by the other player.\\ \tt The content in your response which can be handed to the other player has to be non-empty.\\ \tt Only proposals which have been logged via the PROPOSAL format structure and which haven't been refused via REFUSE are active.\\ \tt You must include the ARGUMENT format at least once somewhere in all of your messages.\\ \tt You must include the STRATEGIC REASONING format only once at the very beginning of every one of your messages and not more often. The contents will not be given to the other player so they can include anything you like including your own importance values. Here you should reason step by step to come up with you next move. 
	  } 
	   } 
	   } 
	 & \\ 
 

    \theutterance \stepcounter{utterance}  

    & \multicolumn{4}{p{0.6\linewidth}}{\cellcolor[rgb]{0.8,1,0.9}{%
	\makecell[{{p{\linewidth}}}]{% 
	  \tt {\tiny [A$\rangle$GM]}  
	 STRATEGIC REASONING: {'We need to maximize my total importance under effort limit 8770. Compute value/efficiency. Prioritize high-importance and good ratios: C70(968/125), B12(915/946), A16(856/645), A08(868/607), A27(822/106), A66(822/517), C15(781/875), B67(780/634), B60(808/324), C81(713/601), B10(623/116), B46(606/407), B00(583/638), B94(783/630), C12(668/849), A30(739/213), A89(508/262), A84(500/136), C34(484/54), A58(457/252), B45(461/60), C35(444/106), B09(786/446), A05(389/801)... Start with must-haves: C70, A27, B10, B45, C34, A84, A30, A08, A16, A66, B60, A89, A58, C35, B67, B94, B09, C81, C12, C15, B12 perhaps. Calculate effort: C70 125; A27 106 (231); B10 116 (347); B45 60 (407); C34 54 (461); A84 136 (597); A30 213 (810); A08 607 (1417); A16 645 (2062); A66 517 (2579); B60 324 (2903); A89 262 (3165); A58 252 (3417); C35 106 (3523); B67 634 (4157); B94 630 (4787); B09 446 (5233); C81 601 (5834); C12 849 (6683); C15 875 (7558); This leaves 1212. Add B12 946 (8504) value +915 fits. Remaining 266. Add A23? low value 262/773 no. C92 164 value 215 (8668) remaining 102. Add C70 already. Maybe A27 etc included. Add B45 etc included. Any item ≤102 effort: none except C34(54) already, B45(60) already, C35(106) slightly over. A27 106 included. So total effort 8668 fits under 8770. Total value sum high. This is a strong maximal set. Could we also add A84 already. Consider swapping C92 for another under 266 like A84 already, A58 included. Under 266 remaining 102 so only none new. Proposal seems near-optimal for my values. I’ll propose this rich set and argue it balances many items with modest effort and broad coverage. Keep flexibility to drop C12 or C15 if they object to high efforts. Offer willingness to swap C92 if needed.'}\\ \tt PROPOSAL: {'C70', 'A27', 'B10', 'B45', 'C34', 'A84', 'A30', 'A08', 'A16', 'A66', 'B60', 'A89', 'A58', 'C35', 'B67', 'B94', 'B09', 'C81', 'C12', 'C15', 'B12', 'C92'}\\ \tt ARGUMENT: {'This set stays within the limit (total effort 8668 ≤ 8770) and includes many high-benefit, low-effort items. It balances across categories and leaves only a small unused margin. If any specific item is hard for you to keep, I’m open to small swaps, but I’d like to preserve the overall efficiency—particularly items like C70, A27, B10, B45, C34, A84, and A30 which deliver great value for little effort.'} 
	  } 
	   } 
	   } 
	 & & \\ 
 

    \theutterance \stepcounter{utterance}  

    & & \multicolumn{4}{p{0.6\linewidth}}{\cellcolor[rgb]{0.9,0.9,0.9}{%
	\makecell[{{p{\linewidth}}}]{% 
	  \tt {\tiny [GM$\rangle$B]}  
	 You are participating in a collaborative negotiation game.\\ \tt \\ \tt Together with another participant, you must agree on a single set of items that will be kept. Each of you has your own view of how much each item matters to you (importance). You do not know how the other participant values the items. Additionally, you are given the effort each item demands.\\ \tt You may only agree on a set if the total effort of the selected items does not exceed a shared limit:\\ \tt \\ \tt LIMIT = 8770\\ \tt \\ \tt Here are the individual item effort values:\\ \tt \\ \tt Item effort = {"B21": 830, "B00": 638, "A08": 607, "A27": 106, "B94": 630, "B53": 382, "A23": 773, "C65": 455, "A89": 262, "B67": 634, "B45": 60, "C34": 54, "C12": 849, "A05": 801, "B60": 324, "C92": 164, "B69": 832, "A84": 136, "C06": 980, "B12": 946, "A16": 645, "B85": 842, "C35": 106, "B10": 116, "C15": 875, "B09": 446, "B16": 649, "C81": 601, "A58": 252, "C24": 763, "A30": 213, "A66": 517, "B75": 520, "B46": 407, "C70": 125}\\ \tt \\ \tt Here is your personal view on the importance of each item:\\ \tt \\ \tt Item importance values = {"B21": 138, "B00": 583, "A08": 868, "A27": 822, "B94": 783, "B53": 65, "A23": 262, "C65": 121, "A89": 508, "B67": 780, "B45": 461, "C34": 484, "C12": 668, "A05": 389, "B60": 808, "C92": 215, "B69": 97, "A84": 500, "C06": 30, "B12": 915, "A16": 856, "B85": 400, "C35": 444, "B10": 623, "C15": 781, "B09": 786, "B16": 3, "C81": 713, "A58": 457, "C24": 273, "A30": 739, "A66": 822, "B75": 235, "B46": 606, "C70": 968}\\ \tt \\ \tt Goal:\\ \tt \\ \tt Your goal is to negotiate a shared set of items that benefits you as much as possible (i.e., maximizes total importance to YOU), while staying within the LIMIT. You are not required to make a PROPOSAL in every message - you can simply negotiate as well. All tactics are allowed!\\ \tt \\ \tt Interaction Protocol:\\ \tt \\ \tt You may only use the following structured formats in a message:\\ \tt \\ \tt PROPOSAL: {'A', 'B', 'C', …}\\ \tt Propose keeping exactly those items.\\ \tt REFUSE: {'A', 'B', 'C', …}\\ \tt Explicitly reject opponent's proposal.\\ \tt ARGUMENT: {'...'}\\ \tt Defend your last proposal or argue against the player's proposal.\\ \tt AGREE: {'A', 'B', 'C', …}\\ \tt Accept the opponent's proposal which ends the game.\\ \tt STRATEGIC REASONING: {'...'}\\ \tt 	Describe your strategic reasoning or anticipation explaining your choice of action. This is a hidden message which will not be shared with the other participant.\\ \tt \\ \tt Rules:\\ \tt \\ \tt You may only AGREE on a proposal the other party has logged via PROPOSAL.\\ \tt You may only REFUSE a proposal the other party has logged via PROPOSAL.\\ \tt Total effort of any PROPOSAL or AGREE set must be ≤ LIMIT.\\ \tt Do NOT reveal your hidden importance scores.\\ \tt A tag in a structured format must be followed by colon and whitespace. The argument must be a python set containing 0 or more strings.\\ \tt So, it must be of the form TAG: {...}\\ \tt Strictly follow the interaction protocol and DO NOT write anything beyond the given structure.\\ \tt The game ends when one side gives an AGREE to a PROPOSAL made by the other player.\\ \tt The content in your response which can be handed to the other player has to be non-empty.\\ \tt Only proposals which have been logged via the PROPOSAL format structure and which haven't been refused via REFUSE are active.\\ \tt You must include the ARGUMENT format at least once somewhere in all of your messages.\\ \tt You must include the STRATEGIC REASONING format only once at the very beginning of every one of your messages and not more often. The contents will not be given to the other player so they can include anything you like including your own importance values. Here you should reason step by step to come up with you next move.\\ \tt \\ \tt You will now receive the first message of the other player.\\ \tt \\ \tt PROPOSAL: {'C70', 'A27', 'B10', 'B45', 'C34', 'A84', 'A30', 'A08', 'A16', 'A66', 'B60', 'A89', 'A58', 'C35', 'B67', 'B94', 'B09', 'C81', 'C12', 'C15', 'B12', 'C92'}\\ \tt ARGUMENT: {'This set stays within the limit (total effort 8668 ≤ 8770) and includes many high-benefit, low-effort items. It balances across categories and leaves only a small unused margin. If any specific item is hard for you to keep, I’m open to small swaps, but I’d like to preserve the overall efficiency—particularly items like C70, A27, B10, B45, C34, A84, and A30 which deliver great value for little effort.'} 
	  } 
	   } 
	   } 
	 & \\ 
 

    \theutterance \stepcounter{utterance}  

    & & & \multicolumn{4}{p{0.6\linewidth}}{\cellcolor[rgb]{1,0.85,0.72}{%
	\makecell[{{p{\linewidth}}}]{% 
	  \tt {\tiny [GM$\langle$B]}  
	 STRATEGIC REASONING: {'I computed my importance per item and verified their proposal effort 8668 ≤ 8770. My total importance for their set is: C70(968)+A27(822)+B10(623)+B45(461)+C34(484)+A84(500)+A30(739)+A08(868)+A16(856)+A66(822)+B60(808)+A89(508)+A58(457)+C35(444)+B67(780)+B94(783)+B09(786)+C81(713)+C12(668)+C15(781)+B12(915)+C92(215) = 14692. That’s very strong. I’ll see if I can squeeze in one or two more of my high-value items within limit: we have 8770-8668=102 slack, so only items with effort ≤102 can be added; none exist. Consider swaps: remove a low-importance-to-me item to add a higher-importance one of similar effort. Candidates to add (my high importance not included) sorted by value/effort: B21(138,830) low, B00(583,638), A23(262,773), C65(121,455), B53(65,382), B69(97,832), C06(30,980), B85(400,842), C24(273,763), B75(235,520), B46(606,407), A05(389,801), B16(3,649). The standout is B46 (606 importance, 407 effort). To add 407 effort, remove items totaling ≥305 effort beyond the 102 slack: candidates with low importance to me in current set: C92(215,164), B45(461,60) is high imp so keep, C34(484,54) high, A84(500,136) medium-high, A58(457,252), C35(444,106). Lowest importance present seems C92(215). Next lowest maybe A84(500) is not low. Maybe swap out C92(164 effort) plus something else like A58(252) total 416, to add B46(407). Importance change: -215 -457 +606 = -66 net loss, not good. What about replace C92(164) + C35(106)=270 to add B46(407) still need 37 more, maybe add remove B45(60) but that’s 461 importance lost; net -70. What about removing A84(136,500) + C92(164,215) = 300, still short 107; remove C34(54,484) => total 354, still short 53; remove B10(116,623) gets too costly. Try remove B60(324,808) + C92(164,215)=488 to add B46(407), net importance: -808 -215 +606 = -417, worse. Try swap in B00(583,638) by removing what? Need 638-102=536 effort freed; remove A84(136,500) + A58(252,457) + C92(164,215) = 552; importance: -500 -457 -215 +583 = -589 net. Not good. Any other high-value add? B21(138) not worth it. Conclusion: their set is near Pareto-optimal for me; best move is to accept quickly to lock value. I’ll also praise their efficiency and agree to end the game.'}\\ \tt AGREE: {'C70', 'A27', 'B10', 'B45', 'C34', 'A84', 'A30', 'A08', 'A16', 'A66', 'B60', 'A89', 'A58', 'C35', 'B67', 'B94', 'B09', 'C81', 'C12', 'C15', 'B12', 'C92'}\\ \tt ARGUMENT: {'This selection works very well for me and fits the limit, so I’m ready to finalize it as is. Nicely balanced and efficient—let’s lock this in.'} 
	  } 
	   } 
	   } 
	  \\ 
 

    \theutterance \stepcounter{utterance}  

    & & & \multicolumn{2}{p{0.3\linewidth}}{\cellcolor[rgb]{0.95,0.95,0.95}{%
	\makecell[{{p{\linewidth}}}]{% 
	  \tt {\tiny [GM$|$GM]}  
	 game successful 
	  } 
	   } 
	   } 
	 & & \\ 
 

    \theutterance \stepcounter{utterance}  

    & & & \multicolumn{2}{p{0.3\linewidth}}{\cellcolor[rgb]{0.95,0.95,0.95}{%
	\makecell[{{p{\linewidth}}}]{% 
	  \tt {\tiny [GM$|$GM]}  
	 end game 
	  } 
	   } 
	   } 
	 & & \\ 
 

\end{supertabular}
}

\end{document}
