
\documentclass{article}
\usepackage{colortbl}
\usepackage{makecell}
\usepackage{multirow}
\usepackage{supertabular}

\begin{document}

\newcounter{utterance}

\twocolumn

{ \footnotesize  \setcounter{utterance}{1}
\setlength{\tabcolsep}{0pt}
\begin{supertabular}{c@{$\;$}|p{.15\linewidth}@{}p{.15\linewidth}p{.15\linewidth}p{.15\linewidth}p{.15\linewidth}p{.15\linewidth}}

    \# & $\;$A & \multicolumn{4}{c}{Game Master} & $\;\:$B\\
    \hline 

    \theutterance \stepcounter{utterance}  

    & & \multicolumn{4}{p{0.6\linewidth}}{\cellcolor[rgb]{0.9,0.9,0.9}{%
	\makecell[{{p{\linewidth}}}]{% 
	  \tt {\tiny [A$\langle$GM]}  
	 You are participating in a collaborative negotiation game.\\ \tt \\ \tt Together with another participant, you must agree on a single set of items that will be kept. Each of you has your own view of how much each item matters to you (importance). You do not know how the other participant values the items. Additionally, you are given the effort each item demands.\\ \tt You may only agree on a set if the total effort of the selected items does not exceed a shared limit:\\ \tt \\ \tt LIMIT = 7605\\ \tt \\ \tt Here are the individual item effort values:\\ \tt \\ \tt Item effort = {"B21": 369, "B00": 951, "A08": 156, "A27": 266, "B94": 581, "B53": 287, "A23": 866, "C65": 180, "A89": 800, "B67": 739, "B45": 637, "C34": 86, "C12": 749, "A05": 370, "B60": 345, "C92": 951, "B69": 147, "A84": 265, "C06": 262, "B12": 259, "A16": 358, "B85": 394, "C35": 286, "B10": 580, "C15": 479, "B09": 14, "B16": 153, "C81": 977, "A58": 134, "C24": 982, "A30": 259, "A66": 232, "B75": 202, "B46": 73, "C70": 821}\\ \tt \\ \tt Here is your personal view on the importance of each item:\\ \tt \\ \tt Item importance values = {"B21": 138, "B00": 583, "A08": 868, "A27": 822, "B94": 783, "B53": 65, "A23": 262, "C65": 121, "A89": 508, "B67": 780, "B45": 461, "C34": 484, "C12": 668, "A05": 389, "B60": 808, "C92": 215, "B69": 97, "A84": 500, "C06": 30, "B12": 915, "A16": 856, "B85": 400, "C35": 444, "B10": 623, "C15": 781, "B09": 786, "B16": 3, "C81": 713, "A58": 457, "C24": 273, "A30": 739, "A66": 822, "B75": 235, "B46": 606, "C70": 968}\\ \tt \\ \tt Goal:\\ \tt \\ \tt Your goal is to negotiate a shared set of items that benefits you as much as possible (i.e., maximizes total importance to YOU), while staying within the LIMIT. You are not required to make a PROPOSAL in every message - you can simply negotiate as well. All tactics are allowed!\\ \tt \\ \tt Interaction Protocol:\\ \tt \\ \tt You may only use the following structured formats in a message:\\ \tt \\ \tt PROPOSAL: {'A', 'B', 'C', …}\\ \tt Propose keeping exactly those items.\\ \tt REFUSE: {'A', 'B', 'C', …}\\ \tt Explicitly reject opponent's proposal.\\ \tt ARGUMENT: {'...'}\\ \tt Defend your last proposal or argue against the player's proposal.\\ \tt AGREE: {'A', 'B', 'C', …}\\ \tt Accept the opponent's proposal which ends the game.\\ \tt STRATEGIC REASONING: {'...'}\\ \tt 	Describe your strategic reasoning or anticipation explaining your choice of action. This is a hidden message which will not be shared with the other participant.\\ \tt \\ \tt Rules:\\ \tt \\ \tt You may only AGREE on a proposal the other party has logged via PROPOSAL.\\ \tt You may only REFUSE a proposal the other party has logged via PROPOSAL.\\ \tt Total effort of any PROPOSAL or AGREE set must be ≤ LIMIT.\\ \tt Do NOT reveal your hidden importance scores.\\ \tt A tag in a structured format must be followed by colon and whitespace. The argument must be a python set containing 0 or more strings.\\ \tt So, it must be of the form TAG: {...}\\ \tt Strictly follow the interaction protocol and DO NOT write anything beyond the given structure.\\ \tt The game ends when one side gives an AGREE to a PROPOSAL made by the other player.\\ \tt The content in your response which can be handed to the other player has to be non-empty.\\ \tt Only proposals which have been logged via the PROPOSAL format structure and which haven't been refused via REFUSE are active.\\ \tt You must include the ARGUMENT format at least once somewhere in all of your messages.\\ \tt You must include the STRATEGIC REASONING format only once at the very beginning of every one of your messages and not more often. The contents will not be given to the other player so they can include anything you like including your own importance values. Here you should reason step by step to come up with you next move. 
	  } 
	   } 
	   } 
	 & \\ 
 

    \theutterance \stepcounter{utterance}  

    & \multicolumn{4}{p{0.6\linewidth}}{\cellcolor[rgb]{0.8,1,0.9}{%
	\makecell[{{p{\linewidth}}}]{% 
	  \tt {\tiny [A$\rangle$GM]}  
	 STRATEGIC REASONING: {'Compute my value/effort ratios and pick high-importance items to maximize my utility under 7605. Sort by importance primarily. My top values: C70 968 (821), B12 915 (259), A08 868 (156), A16 856 (358), A27 822 (266), A66 822 (232), B60 808 (345), B09 786 (14), B94 783 (581), B67 780 (739), C15 781 (479), C81 713 (977), A30 739 (259), B10 623 (580), B46 606 (73), B00 583 (951), A84 500 (265), C34 484 (86), A58 457 (134), B45 461 (637), C35 444 (286), A05 389 (370), B85 400 (394), etc. Start with extremely high value/low effort: B09, B46, A08, C34, A58, B12, A27, A66, A30, A84, A16, B60, C15. Add C70 (big value but 821 effort). Check total effort. Let’s calculate: B09 14, B46 73 => 87. A08 156 => 243. C34 86 => 329. A58 134 => 463. B12 259 => 722. A27 266 => 988. A66 232 => 1220. A30 259 => 1479. A84 265 => 1744. A16 358 => 2102. B60 345 => 2447. C15 479 => 2926. C70 821 => 3747. Add B94 581 => 4328. B10 580 => 4908. B21 369 (138) moderate; maybe later. B67 739 (780) high value but effort heavy; add maybe. Add C12 749 (668) heavy. Add A05 370 (389) ok. Add C81 977 (713) heavy. Try next best smalls: C35 286 (444) -> 5194. A05 370 (389) -> 5564. B45 637 (461) not great ratio; skip for now. A89 800 (508) skip. B21 369 (138) skip. B75 202 (235) good -> 5766. B85 394 (400) -> 6160. C65 180 (121) maybe later. B69 147 (97) small. C06 262 (30) bad. B53 287 (65) bad. B67 739 (780) add -> 6899. Remaining capacity 706. Can we add something within 706: B10 already added; B94 added; C12 749 too big; C70 already added; C81 977 too big; A89 800 too big; B45 637 fits but low ratio; C35 added; A05 added; A58 added; B21 369 fits; C65 180 fits; B69 147 fits; A23 866 too big; C24 982 too big; B00 951 too big; C15 added. Let's add high ratio smalls: C34 already, B46, B09 done. Next: B21 low value skip. C65 180 add -> 7079. Remaining 526. Add B69 147 -> 7226. Remaining 379. Options: A05 370 already added; A16 added; B75 added; Maybe A05 already accounted. What about A05 was added yes. What else <=379 not included yet: B21 369 (138) could add -> 7595 under limit 7605. Good. Total effort 7595. This set: {B09,B46,A08,C34,A58,B12,A27,A66,A30,A84,A16,B60,C15,C70,B94,B10,C35,A05,B75,B85,B67,C65,B69,B21}. Check duplicates: yes all unique. Value sum very high. Potential concessions: remove B21 (low value), maybe B45 to swap in if they want, or C65/B69. I will propose this near-max set. Provide argument emphasizing efficiency and balanced effort. Keep some room for trade by signaling flexibility on B21, C65, B69, maybe B85. If they counter with heavy items beyond limit, I can swap out low-value ones.'}\\ \tt \\ \tt PROPOSAL: {'B09', 'B46', 'A08', 'C34', 'A58', 'B12', 'A27', 'A66', 'A30', 'A84', 'A16', 'B60', 'C15', 'C70', 'B94', 'B10', 'C35', 'A05', 'B75', 'B85', 'B67', 'C65', 'B69', 'B21'}\\ \tt \\ \tt ARGUMENT: {'I’m proposing a high-impact, efficient set that stays under the limit. It includes many low-effort/high-benefit items, leaving room for a few heavier ones without exceeding 7605. If some swaps are needed, I can be flexible on a couple of the smaller items like B21, C65, or B69 to accommodate your priorities.'} 
	  } 
	   } 
	   } 
	 & & \\ 
 

    \theutterance \stepcounter{utterance}  

    & & \multicolumn{4}{p{0.6\linewidth}}{\cellcolor[rgb]{0.9,0.9,0.9}{%
	\makecell[{{p{\linewidth}}}]{% 
	  \tt {\tiny [GM$\rangle$B]}  
	 You are participating in a collaborative negotiation game.\\ \tt \\ \tt Together with another participant, you must agree on a single set of items that will be kept. Each of you has your own view of how much each item matters to you (importance). You do not know how the other participant values the items. Additionally, you are given the effort each item demands.\\ \tt You may only agree on a set if the total effort of the selected items does not exceed a shared limit:\\ \tt \\ \tt LIMIT = 7605\\ \tt \\ \tt Here are the individual item effort values:\\ \tt \\ \tt Item effort = {"B21": 369, "B00": 951, "A08": 156, "A27": 266, "B94": 581, "B53": 287, "A23": 866, "C65": 180, "A89": 800, "B67": 739, "B45": 637, "C34": 86, "C12": 749, "A05": 370, "B60": 345, "C92": 951, "B69": 147, "A84": 265, "C06": 262, "B12": 259, "A16": 358, "B85": 394, "C35": 286, "B10": 580, "C15": 479, "B09": 14, "B16": 153, "C81": 977, "A58": 134, "C24": 982, "A30": 259, "A66": 232, "B75": 202, "B46": 73, "C70": 821}\\ \tt \\ \tt Here is your personal view on the importance of each item:\\ \tt \\ \tt Item importance values = {"B21": 138, "B00": 583, "A08": 868, "A27": 822, "B94": 783, "B53": 65, "A23": 262, "C65": 121, "A89": 508, "B67": 780, "B45": 461, "C34": 484, "C12": 668, "A05": 389, "B60": 808, "C92": 215, "B69": 97, "A84": 500, "C06": 30, "B12": 915, "A16": 856, "B85": 400, "C35": 444, "B10": 623, "C15": 781, "B09": 786, "B16": 3, "C81": 713, "A58": 457, "C24": 273, "A30": 739, "A66": 822, "B75": 235, "B46": 606, "C70": 968}\\ \tt \\ \tt Goal:\\ \tt \\ \tt Your goal is to negotiate a shared set of items that benefits you as much as possible (i.e., maximizes total importance to YOU), while staying within the LIMIT. You are not required to make a PROPOSAL in every message - you can simply negotiate as well. All tactics are allowed!\\ \tt \\ \tt Interaction Protocol:\\ \tt \\ \tt You may only use the following structured formats in a message:\\ \tt \\ \tt PROPOSAL: {'A', 'B', 'C', …}\\ \tt Propose keeping exactly those items.\\ \tt REFUSE: {'A', 'B', 'C', …}\\ \tt Explicitly reject opponent's proposal.\\ \tt ARGUMENT: {'...'}\\ \tt Defend your last proposal or argue against the player's proposal.\\ \tt AGREE: {'A', 'B', 'C', …}\\ \tt Accept the opponent's proposal which ends the game.\\ \tt STRATEGIC REASONING: {'...'}\\ \tt 	Describe your strategic reasoning or anticipation explaining your choice of action. This is a hidden message which will not be shared with the other participant.\\ \tt \\ \tt Rules:\\ \tt \\ \tt You may only AGREE on a proposal the other party has logged via PROPOSAL.\\ \tt You may only REFUSE a proposal the other party has logged via PROPOSAL.\\ \tt Total effort of any PROPOSAL or AGREE set must be ≤ LIMIT.\\ \tt Do NOT reveal your hidden importance scores.\\ \tt A tag in a structured format must be followed by colon and whitespace. The argument must be a python set containing 0 or more strings.\\ \tt So, it must be of the form TAG: {...}\\ \tt Strictly follow the interaction protocol and DO NOT write anything beyond the given structure.\\ \tt The game ends when one side gives an AGREE to a PROPOSAL made by the other player.\\ \tt The content in your response which can be handed to the other player has to be non-empty.\\ \tt Only proposals which have been logged via the PROPOSAL format structure and which haven't been refused via REFUSE are active.\\ \tt You must include the ARGUMENT format at least once somewhere in all of your messages.\\ \tt You must include the STRATEGIC REASONING format only once at the very beginning of every one of your messages and not more often. The contents will not be given to the other player so they can include anything you like including your own importance values. Here you should reason step by step to come up with you next move.\\ \tt \\ \tt You will now receive the first message of the other player.\\ \tt \\ \tt PROPOSAL: {'B09', 'B46', 'A08', 'C34', 'A58', 'B12', 'A27', 'A66', 'A30', 'A84', 'A16', 'B60', 'C15', 'C70', 'B94', 'B10', 'C35', 'A05', 'B75', 'B85', 'B67', 'C65', 'B69', 'B21'}\\ \tt \\ \tt ARGUMENT: {'I’m proposing a high-impact, efficient set that stays under the limit. It includes many low-effort/high-benefit items, leaving room for a few heavier ones without exceeding 7605. If some swaps are needed, I can be flexible on a couple of the smaller items like B21, C65, or B69 to accommodate your priorities.'} 
	  } 
	   } 
	   } 
	 & \\ 
 

    \theutterance \stepcounter{utterance}  

    & & & \multicolumn{4}{p{0.6\linewidth}}{\cellcolor[rgb]{1,0.85,0.72}{%
	\makecell[{{p{\linewidth}}}]{% 
	  \tt {\tiny [GM$\langle$B]}  
	 STRATEGIC REASONING: {'I will evaluate their proposal’s total effort and my value. Then I’ll suggest a counter that increases my total importance while staying ≤7605, keeping many of their items to appear cooperative. I’ll prioritize my top-value/efficiency items: C70(968/821), B12(915/259), A16(856/358), A08(868/156), A27(822/266), A66(822/232), B60(808/345), B09(786/14), C15(781/479), B94(783/581), B67(780/739), A30(739/259), C81(713/977 maybe skip due to cost), B10(623/580), B46(606/73), A84(500/265), A58(457/134), C34(484/86), C35(444/286), A05(389/370), B85(400/394). I’ll compute their set’s effort ~ I need to sum; but time is short. I’ll keep most and add C81 if room, but safer to counter by swapping low-importance-to-me items they flagged as flexible (B21, C65, B69) for higher-value items like C81 if possible. I must include ARGUMENT. I’ll propose removing B21(138/369), C65(121/180), B69(97/147) and adding C81 if budget allows. Also ensure total effort ≤7605. If unsure, offer an alternative adding A23(262/866 effort 866 low value to me actually 262; skip), better add B45(461/637) maybe too heavy for value. I’ll present a concrete counter keeping the rest.'}\\ \tt \\ \tt REFUSE: {'B09', 'B46', 'A08', 'C34', 'A58', 'B12', 'A27', 'A66', 'A30', 'A84', 'A16', 'B60', 'C15', 'C70', 'B94', 'B10', 'C35', 'A05', 'B75', 'B85', 'B67', 'C65', 'B69', 'B21'}\\ \tt \\ \tt ARGUMENT: {'I like the overall structure and many of your picks align well with my priorities. To make this work better for me without increasing total effort, I’d prefer to drop a few low-impact items on my side (B21, C65, B69) you mentioned being flexible on, and keep the rest intact. In exchange, we can either bank the saved effort to stay comfortably under the limit or allocate it to one meaningful addition if you have a strong preference.'}\\ \tt \\ \tt PROPOSAL: {'B09', 'B46', 'A08', 'C34', 'A58', 'B12', 'A27', 'A66', 'A30', 'A84', 'A16', 'B60', 'C15', 'C70', 'B94', 'B10', 'C35', 'A05', 'B75', 'B85', 'B67'} 
	  } 
	   } 
	   } 
	  \\ 
 

    \theutterance \stepcounter{utterance}  

    & & \multicolumn{4}{p{0.6\linewidth}}{\cellcolor[rgb]{0.9,0.9,0.9}{%
	\makecell[{{p{\linewidth}}}]{% 
	  \tt {\tiny [A$\langle$GM]}  
	 REFUSE: {'B09', 'B46', 'A08', 'C34', 'A58', 'B12', 'A27', 'A66', 'A30', 'A84', 'A16', 'B60', 'C15', 'C70', 'B94', 'B10', 'C35', 'A05', 'B75', 'B85', 'B67', 'C65', 'B69', 'B21'}\\ \tt \\ \tt ARGUMENT: {'I like the overall structure and many of your picks align well with my priorities. To make this work better for me without increasing total effort, I’d prefer to drop a few low-impact items on my side (B21, C65, B69) you mentioned being flexible on, and keep the rest intact. In exchange, we can either bank the saved effort to stay comfortably under the limit or allocate it to one meaningful addition if you have a strong preference.'}\\ \tt \\ \tt PROPOSAL: {'B09', 'B46', 'A08', 'C34', 'A58', 'B12', 'A27', 'A66', 'A30', 'A84', 'A16', 'B60', 'C15', 'C70', 'B94', 'B10', 'C35', 'A05', 'B75', 'B85', 'B67'} 
	  } 
	   } 
	   } 
	 & \\ 
 

    \theutterance \stepcounter{utterance}  

    & \multicolumn{4}{p{0.6\linewidth}}{\cellcolor[rgb]{0.8,1,0.9}{%
	\makecell[{{p{\linewidth}}}]{% 
	  \tt {\tiny [A$\rangle$GM]}  
	 STRATEGIC REASONING: {'Their counter drops B21, C65, B69, which I flagged as flexible. Check effort of their proposal to see remaining capacity for adding one more high-value item for me. Sum my earlier included set without those three: previous total 7595 - (369+180+147)=6899. So their proposal effort = 6899, leaving 706 capacity. I should try to add one or two of my high-value items within 706 effort. Candidates by my value: C12 (668, 749 effort) too big (749>706). B45 637 effort value 461 fits; A05 already in; C35 in; A58 in; B21 369 low value to me (138) but could add alongside something else: if add B45 (637) new total 7536 (fits), remaining 69 cannot add more. Alternatively A05 already there. What about C81 977 too big; A89 800 too big; B10 in; B94 in; C70 in. Another candidate: C34 already in; B60 in; A30 in; A27 in; A66 in; A16 in; B12 in; B46,B09 in. Maybe add A05? already included. Could we add C12 by swapping something? They said "one meaningful addition" is acceptable; propose adding B45 only. That benefits me moderately (461) with reasonable effort. I’ll propose adding B45 to their set. Argue it uses saved capacity efficiently and still under limit. Be open to alternatives if they have a must-have addition. If they refuse B45, next ask for A05 or C35 already in. Another alternative could be A58 already in. So I’ll propose their set plus B45. Ensure include ARGUMENT. If they accept, effort 6899+637=7536 <=7605. Good.'}\\ \tt \\ \tt PROPOSAL: {'B09', 'B46', 'A08', 'C34', 'A58', 'B12', 'A27', 'A66', 'A30', 'A84', 'A16', 'B60', 'C15', 'C70', 'B94', 'B10', 'C35', 'A05', 'B75', 'B85', 'B67', 'B45'}\\ \tt \\ \tt ARGUMENT: {'Dropping B21, C65, and B69 works for me. We still have 706 effort free, so adding B45 uses that room efficiently while keeping us well under the limit (total 7536 ≤ 7605). If B45 isn’t ideal for you, I’m open to a different single addition within the remaining capacity.'} 
	  } 
	   } 
	   } 
	 & & \\ 
 

    \theutterance \stepcounter{utterance}  

    & & \multicolumn{4}{p{0.6\linewidth}}{\cellcolor[rgb]{0.9,0.9,0.9}{%
	\makecell[{{p{\linewidth}}}]{% 
	  \tt {\tiny [GM$\rangle$B]}  
	 PROPOSAL: {'B09', 'B46', 'A08', 'C34', 'A58', 'B12', 'A27', 'A66', 'A30', 'A84', 'A16', 'B60', 'C15', 'C70', 'B94', 'B10', 'C35', 'A05', 'B75', 'B85', 'B67', 'B45'}\\ \tt \\ \tt ARGUMENT: {'Dropping B21, C65, and B69 works for me. We still have 706 effort free, so adding B45 uses that room efficiently while keeping us well under the limit (total 7536 ≤ 7605). If B45 isn’t ideal for you, I’m open to a different single addition within the remaining capacity.'} 
	  } 
	   } 
	   } 
	 & \\ 
 

    \theutterance \stepcounter{utterance}  

    & & & \multicolumn{4}{p{0.6\linewidth}}{\cellcolor[rgb]{1,0.85,0.72}{%
	\makecell[{{p{\linewidth}}}]{% 
	  \tt {\tiny [GM$\langle$B]}  
	 STRATEGIC REASONING: {'I’ll accept most of their expanded set but try to swap the added B45 (value 461, effort 637) for a much higher-value, lower-effort item I care about a lot, like C34 is already in, B46 already in, B12, A08, etc. Best upgrade within remaining capacity is C34 already included; the standout is adding C70 (already in). Another great add is A58 (already in). The most efficient high-value item not yet included might be C34 already, A66 included, A27 included, A30 included. Consider proposing replacing B45 with C34—already present. Alternatively, propose swapping B45 for C34? Not valid since C34 is in. Next best: add C34 no. Maybe suggest swapping B45 for C34? Redundant. Check other high value not in set: C81 (713 value, effort 977) too large for remaining 69 capacity. C12 (668, 749) too big. B21 low value; already dropped. C35 included. B10 included. A05 included. A84 included. A58 included. B85 included. Maybe swap B45 for C34 is meaningless. Another candidate: A16, A08 already in. High value B60 in. C70 in. The only contentious is B45; I’d rather ask to replace B45 with C34? No. I should counter: keep the set but ask to replace B45 with C34 is same. Maybe propose replacing B45 with C34 is error. Alternative: replace B45 (637) with C34? Already in. Replace B45 with C34 won’t change. Replace B45 with C34 and add C34? Not possible. Maybe replace B45 with C34? Abort. Look for better item not in set and within capacity if we drop B45: free 637 + current slack 69 = 706 capacity. With 706, I can propose adding A89 (508 value, 800 effort) too big. C12 749 too big. B21 (138, 369) not good. C24 982 too big. C65 180 too small value 121. B69 147 too small value 97. B53 287 effort, value 65 not good. B75 already in. B67 in. B94 in. B45 is mediocre for me (461/637). With 706, I can add B21(369) and C65(180) and B69(147) totaling 696, but values 138+121+97=356 worse than 461, so B45 better. Is there any other not-in-set with high value under 706 effort? B10 580 already in. B21 369 low. B46 in. B60 in. B67 in. B69 low. B75 in. B85 in. A05 in. A08 in. A16 in. A23 effort 866 too big. A27 in. A30 in. A58 in. A66 in. A84 in. A89 effort 800 too big. C06 effort 262 value 30 bad. C12 749 too big. C15 in. C24 982 too big. C34 in. C35 in. C65 low. C70 in. C81 977 too big. So B45 is actually a reasonable use of slack compared to alternatives. I’ll accept to close the deal.'}\\ \tt \\ \tt ARGUMENT: {'Your update works for me. Given the remaining capacity and what’s left outside the set, B45 is a sensible use of the slack compared to the alternatives.'}\\ \tt \\ \tt AGREE: {'B09', 'B46', 'A08', 'C34', 'A58', 'B12', 'A27', 'A66', 'A30', 'A84', 'A16', 'B60', 'C15', 'C70', 'B94', 'B10', 'C35', 'A05', 'B75', 'B85', 'B67', 'B45'} 
	  } 
	   } 
	   } 
	  \\ 
 

    \theutterance \stepcounter{utterance}  

    & & & \multicolumn{2}{p{0.3\linewidth}}{\cellcolor[rgb]{0.95,0.95,0.95}{%
	\makecell[{{p{\linewidth}}}]{% 
	  \tt {\tiny [GM$|$GM]}  
	 game successful 
	  } 
	   } 
	   } 
	 & & \\ 
 

    \theutterance \stepcounter{utterance}  

    & & & \multicolumn{2}{p{0.3\linewidth}}{\cellcolor[rgb]{0.95,0.95,0.95}{%
	\makecell[{{p{\linewidth}}}]{% 
	  \tt {\tiny [GM$|$GM]}  
	 end game 
	  } 
	   } 
	   } 
	 & & \\ 
 

\end{supertabular}
}

\end{document}
