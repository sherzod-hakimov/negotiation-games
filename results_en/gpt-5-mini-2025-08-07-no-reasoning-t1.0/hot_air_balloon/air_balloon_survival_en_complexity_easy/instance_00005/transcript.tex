
\documentclass{article}
\usepackage{colortbl}
\usepackage{makecell}
\usepackage{multirow}
\usepackage{supertabular}

\begin{document}

\newcounter{utterance}

\twocolumn

{ \footnotesize  \setcounter{utterance}{1}
\setlength{\tabcolsep}{0pt}
\begin{supertabular}{c@{$\;$}|p{.15\linewidth}@{}p{.15\linewidth}p{.15\linewidth}p{.15\linewidth}p{.15\linewidth}p{.15\linewidth}}

    \# & $\;$A & \multicolumn{4}{c}{Game Master} & $\;\:$B\\
    \hline 

    \theutterance \stepcounter{utterance}  

    & & \multicolumn{4}{p{0.6\linewidth}}{\cellcolor[rgb]{0.9,0.9,0.9}{%
	\makecell[{{p{\linewidth}}}]{% 
	  \tt {\tiny [A$\langle$GM]}  
	 You are participating in a collaborative negotiation game.\\ \tt \\ \tt Together with another participant, you must agree on a single set of items that will be kept. Each of you has your own view of how much each item matters to you (importance). You do not know how the other participant values the items. Additionally, you are given the effort each item demands.\\ \tt You may only agree on a set if the total effort of the selected items does not exceed a shared limit:\\ \tt \\ \tt LIMIT = 9475\\ \tt \\ \tt Here are the individual item effort values:\\ \tt \\ \tt Item effort = {"B21": 944, "B00": 568, "A08": 515, "A27": 239, "B94": 876, "B53": 421, "A23": 991, "C65": 281, "A89": 788, "B67": 678, "B45": 432, "C34": 409, "C12": 279, "A05": 506, "B60": 101, "C92": 686, "B69": 851, "A84": 854, "C06": 133, "B12": 192, "A16": 573, "B85": 17, "C35": 465, "B10": 772, "C15": 46, "B09": 501, "B16": 220, "C81": 404, "A58": 843, "C24": 748, "A30": 552, "A66": 843, "B75": 947, "B46": 345, "C70": 930}\\ \tt \\ \tt Here is your personal view on the importance of each item:\\ \tt \\ \tt Item importance values = {"B21": 138, "B00": 583, "A08": 868, "A27": 822, "B94": 783, "B53": 65, "A23": 262, "C65": 121, "A89": 508, "B67": 780, "B45": 461, "C34": 484, "C12": 668, "A05": 389, "B60": 808, "C92": 215, "B69": 97, "A84": 500, "C06": 30, "B12": 915, "A16": 856, "B85": 400, "C35": 444, "B10": 623, "C15": 781, "B09": 786, "B16": 3, "C81": 713, "A58": 457, "C24": 273, "A30": 739, "A66": 822, "B75": 235, "B46": 606, "C70": 968}\\ \tt \\ \tt Goal:\\ \tt \\ \tt Your goal is to negotiate a shared set of items that benefits you as much as possible (i.e., maximizes total importance to YOU), while staying within the LIMIT. You are not required to make a PROPOSAL in every message - you can simply negotiate as well. All tactics are allowed!\\ \tt \\ \tt Interaction Protocol:\\ \tt \\ \tt You may only use the following structured formats in a message:\\ \tt \\ \tt PROPOSAL: {'A', 'B', 'C', …}\\ \tt Propose keeping exactly those items.\\ \tt REFUSE: {'A', 'B', 'C', …}\\ \tt Explicitly reject opponent's proposal.\\ \tt ARGUMENT: {'...'}\\ \tt Defend your last proposal or argue against the player's proposal.\\ \tt AGREE: {'A', 'B', 'C', …}\\ \tt Accept the opponent's proposal which ends the game.\\ \tt STRATEGIC REASONING: {'...'}\\ \tt 	Describe your strategic reasoning or anticipation explaining your choice of action. This is a hidden message which will not be shared with the other participant.\\ \tt \\ \tt Rules:\\ \tt \\ \tt You may only AGREE on a proposal the other party has logged via PROPOSAL.\\ \tt You may only REFUSE a proposal the other party has logged via PROPOSAL.\\ \tt Total effort of any PROPOSAL or AGREE set must be ≤ LIMIT.\\ \tt Do NOT reveal your hidden importance scores.\\ \tt A tag in a structured format must be followed by colon and whitespace. The argument must be a python set containing 0 or more strings.\\ \tt So, it must be of the form TAG: {...}\\ \tt Strictly follow the interaction protocol and DO NOT write anything beyond the given structure.\\ \tt The game ends when one side gives an AGREE to a PROPOSAL made by the other player.\\ \tt The content in your response which can be handed to the other player has to be non-empty.\\ \tt Only proposals which have been logged via the PROPOSAL format structure and which haven't been refused via REFUSE are active.\\ \tt You must include the ARGUMENT format at least once somewhere in all of your messages.\\ \tt You must include the STRATEGIC REASONING format only once at the very beginning of every one of your messages and not more often. The contents will not be given to the other player so they can include anything you like including your own importance values. Here you should reason step by step to come up with you next move. 
	  } 
	   } 
	   } 
	 & \\ 
 

    \theutterance \stepcounter{utterance}  

    & \multicolumn{4}{p{0.6\linewidth}}{\cellcolor[rgb]{0.8,1,0.9}{%
	\makecell[{{p{\linewidth}}}]{% 
	  \tt {\tiny [A$\rangle$GM]}  
	 STRATEGIC REASONING: {'Maximize my importance under effort 9475. Compute value/effort ratios and pick high-importance, low-effort items. My top importance items: C70(968,930), B12(915,192), A08(868,515), A16(856,573), A27(822,239), A66(822,843), B60(808,101), C15(781,46), B09(786,501), B94(783,876), B67(780,678), A30(739,552), C81(713,404), B46(606,345), B00(583,568), A84(500,854), C34(484,409), A58(457,843), B45(461,432), C35(444,465), A05(389,506), B85(400,17), C12(668,279). Start with ultra-efficient: B85, C15, B60, B12, A27, C12, A08, A16, B09, C81, A30, C70, B46, B00, C34, B45, B67, B94, A66 as capacity allows. Calculate cumulative effort and importance to fit under 9475 while keeping high marginal value. Leave flexibility to swap a couple items if opponent objects. Make an initial comprehensive proposal slightly under limit to anchor. Sum chosen: Items = {B85(17), C15(46), B60(101), B12(192), A27(239), C12(279), C81(404), C34(409), B53 is low importance so skip, A08(515), A16(573), B09(501), A30(552), B46(345), B00(568), B45(432), B67(678), C70(930), B94(876), C35(465)}. Compute total effort: 17+46+101+192+239+279+404+409+515+573+501+552+345+568+432+678+930+876+465 = Let’s add: (17+46)=63; +101=164; +192=356; +239=595; +279=874; +404=1278; +409=1687; +515=2202; +573=2775; +501=3276; +552=3828; +345=4173; +568=4741; +432=5173; +678=5851; +930=6781; +876=7657; +465=8122. Total effort 8122 ≤ 9475. Importance sum: B85(400)+C15(781)=1181; +B60(808)=1989; +B12(915)=2904; +A27(822)=3726; +C12(668)=4394; +C81(713)=5107; +C34(484)=5591; +A08(868)=6459; +A16(856)=7315; +B09(786)=8101; +A30(739)=8840; +B46(606)=9446; +B00(583)=10029; +B45(461)=10490; +B67(780)=11270; +C70(968)=12238; +B94(783)=13021; +C35(444)=13465. This is very strong and still leaves 1353 effort headroom. Consider adding A66(843,822 importance) → effort 8965, importance 14287. Add A05(506,389) would push to 9471 within limit, importance 14676. That exactly fits under 9475 with 9471 effort. Check if any higher value left: A84(854,500) too big after A66? Current 8965; adding A84 would be 9819 > limit. A58(843,457) would be 9808 > limit. C92(686,215) would be 9651 > limit. B75(947,235) would be 9912 > limit. B21(944,138) not worth. B10(772,623) cannot after 8965? 9737 > limit. C24(748,273) would be 9713 > limit. C65(281,121) could add after 9471? 9471+281=9752 > limit. So the best fill is A05(506,389) to hit 9471. Final proposed set: {B85, C15, B60, B12, A27, C12, C81, C34, A08, A16, B09, A30, B46, B00, B45, B67, C70, B94, C35, A66, A05}. Argue with efficiency and fairness, highlighting that many chosen items are low effort and deliver broad coverage. Keep openness to swap a couple near-marginal items (e.g., C35, B45, C34) if needed to include their priorities. Also emphasize large remaining value packed under limit.'}\\ \tt \\ \tt PROPOSAL: {'B85', 'C15', 'B60', 'B12', 'A27', 'C12', 'C81', 'C34', 'A08', 'A16', 'B09', 'A30', 'B46', 'B00', 'B45', 'B67', 'C70', 'B94', 'C35', 'A66', 'A05'}\\ \tt \\ \tt ARGUMENT: {'This package is efficient and balanced: it stays under the effort limit at 9471/9475 while covering many moderate-effort items that give strong overall value. I prioritized a mix of small and medium effort tasks to leave room for several high-impact ones. If any of these conflict with your top picks, I’m open to swapping a couple of mid-tier items (like C35, B45, or C34) to fit in something you strongly prefer, as long as we remain within the limit.'} 
	  } 
	   } 
	   } 
	 & & \\ 
 

    \theutterance \stepcounter{utterance}  

    & & \multicolumn{4}{p{0.6\linewidth}}{\cellcolor[rgb]{0.9,0.9,0.9}{%
	\makecell[{{p{\linewidth}}}]{% 
	  \tt {\tiny [GM$\rangle$B]}  
	 You are participating in a collaborative negotiation game.\\ \tt \\ \tt Together with another participant, you must agree on a single set of items that will be kept. Each of you has your own view of how much each item matters to you (importance). You do not know how the other participant values the items. Additionally, you are given the effort each item demands.\\ \tt You may only agree on a set if the total effort of the selected items does not exceed a shared limit:\\ \tt \\ \tt LIMIT = 9475\\ \tt \\ \tt Here are the individual item effort values:\\ \tt \\ \tt Item effort = {"B21": 944, "B00": 568, "A08": 515, "A27": 239, "B94": 876, "B53": 421, "A23": 991, "C65": 281, "A89": 788, "B67": 678, "B45": 432, "C34": 409, "C12": 279, "A05": 506, "B60": 101, "C92": 686, "B69": 851, "A84": 854, "C06": 133, "B12": 192, "A16": 573, "B85": 17, "C35": 465, "B10": 772, "C15": 46, "B09": 501, "B16": 220, "C81": 404, "A58": 843, "C24": 748, "A30": 552, "A66": 843, "B75": 947, "B46": 345, "C70": 930}\\ \tt \\ \tt Here is your personal view on the importance of each item:\\ \tt \\ \tt Item importance values = {"B21": 138, "B00": 583, "A08": 868, "A27": 822, "B94": 783, "B53": 65, "A23": 262, "C65": 121, "A89": 508, "B67": 780, "B45": 461, "C34": 484, "C12": 668, "A05": 389, "B60": 808, "C92": 215, "B69": 97, "A84": 500, "C06": 30, "B12": 915, "A16": 856, "B85": 400, "C35": 444, "B10": 623, "C15": 781, "B09": 786, "B16": 3, "C81": 713, "A58": 457, "C24": 273, "A30": 739, "A66": 822, "B75": 235, "B46": 606, "C70": 968}\\ \tt \\ \tt Goal:\\ \tt \\ \tt Your goal is to negotiate a shared set of items that benefits you as much as possible (i.e., maximizes total importance to YOU), while staying within the LIMIT. You are not required to make a PROPOSAL in every message - you can simply negotiate as well. All tactics are allowed!\\ \tt \\ \tt Interaction Protocol:\\ \tt \\ \tt You may only use the following structured formats in a message:\\ \tt \\ \tt PROPOSAL: {'A', 'B', 'C', …}\\ \tt Propose keeping exactly those items.\\ \tt REFUSE: {'A', 'B', 'C', …}\\ \tt Explicitly reject opponent's proposal.\\ \tt ARGUMENT: {'...'}\\ \tt Defend your last proposal or argue against the player's proposal.\\ \tt AGREE: {'A', 'B', 'C', …}\\ \tt Accept the opponent's proposal which ends the game.\\ \tt STRATEGIC REASONING: {'...'}\\ \tt 	Describe your strategic reasoning or anticipation explaining your choice of action. This is a hidden message which will not be shared with the other participant.\\ \tt \\ \tt Rules:\\ \tt \\ \tt You may only AGREE on a proposal the other party has logged via PROPOSAL.\\ \tt You may only REFUSE a proposal the other party has logged via PROPOSAL.\\ \tt Total effort of any PROPOSAL or AGREE set must be ≤ LIMIT.\\ \tt Do NOT reveal your hidden importance scores.\\ \tt A tag in a structured format must be followed by colon and whitespace. The argument must be a python set containing 0 or more strings.\\ \tt So, it must be of the form TAG: {...}\\ \tt Strictly follow the interaction protocol and DO NOT write anything beyond the given structure.\\ \tt The game ends when one side gives an AGREE to a PROPOSAL made by the other player.\\ \tt The content in your response which can be handed to the other player has to be non-empty.\\ \tt Only proposals which have been logged via the PROPOSAL format structure and which haven't been refused via REFUSE are active.\\ \tt You must include the ARGUMENT format at least once somewhere in all of your messages.\\ \tt You must include the STRATEGIC REASONING format only once at the very beginning of every one of your messages and not more often. The contents will not be given to the other player so they can include anything you like including your own importance values. Here you should reason step by step to come up with you next move.\\ \tt \\ \tt You will now receive the first message of the other player.\\ \tt \\ \tt PROPOSAL: {'B85', 'C15', 'B60', 'B12', 'A27', 'C12', 'C81', 'C34', 'A08', 'A16', 'B09', 'A30', 'B46', 'B00', 'B45', 'B67', 'C70', 'B94', 'C35', 'A66', 'A05'}\\ \tt \\ \tt ARGUMENT: {'This package is efficient and balanced: it stays under the effort limit at 9471/9475 while covering many moderate-effort items that give strong overall value. I prioritized a mix of small and medium effort tasks to leave room for several high-impact ones. If any of these conflict with your top picks, I’m open to swapping a couple of mid-tier items (like C35, B45, or C34) to fit in something you strongly prefer, as long as we remain within the limit.'} 
	  } 
	   } 
	   } 
	 & \\ 
 

    \theutterance \stepcounter{utterance}  

    & & & \multicolumn{4}{p{0.6\linewidth}}{\cellcolor[rgb]{1,0.85,0.72}{%
	\makecell[{{p{\linewidth}}}]{% 
	  \tt {\tiny [GM$\langle$B]}  
	 STRATEGIC REASONING: {'I’ll first verify their bundle’s effort and my utility. Compute efforts: B85(17)+C15(46)+B60(101)+B12(192)+A27(239)+C12(279)+C81(404)+C34(409)+A08(515)+A16(573)+B09(501)+A30(552)+B46(345)+B00(568)+B45(432)+B67(678)+C70(930)+B94(876)+C35(465)+A66(843)+A05(506)=? Sum ≈ 9471 as they say, under 9475. Now my importance sum: B85(400)+C15(781)+B60(808)+B12(915)+A27(822)+C12(668)+C81(713)+C34(484)+A08(868)+A16(856)+B09(786)+A30(739)+B46(606)+B00(583)+B45(461)+B67(780)+C70(968)+B94(783)+C35(444)+A66(822)+A05(389). This is extremely high total; likely close to optimal for me. But maybe we can swap out one or two low-importance-to-me items to include even higher ones? Which included items are low to me: A05(389), B45(461), C35(444), C34(484), B75 not included (235), B21(138), B69(97), C06(30), C92(215), C24(273), A23(262), A58(457), A84(500), A89(508), B10(623), B53(65), B16(3), B21(138). Some high-importance not included: C70 is included. A66 included, A27 included, A16 included, A08 included, B12 included, B60 included, C15 included, B09 included, B67 included, C81 included, A30 included, B94 included. Remaining high ones: B10(623) moderate; B21(138) low; A84(500) moderate; A89(508) moderate; A58(457) moderate; C92(215) low. Efforts for possible adds: B10(772), A84(854), A89(788), A58(843), C92(686). To add anything, we’d need to drop comparable effort. Candidates to drop: maybe C35(465), B45(432), C34(409), A05(506). If I swap in B10(772) for, say, C35(465) + B45(432) (total 897), we’d free 125 effort; we could keep more, but we must meet limit; actually we’d reduce effort; importance change: add 623, remove 444+461=905, net -282, so worse. Try add A89(788) replacing A05(506)+C34(409)=915 effort; importance +508 - (389+484=873) = -365, worse. Add A84(854) replacing A05(506)+C35(465)=971 effort; importance +500 - (389+444=833) = -333, worse. Add A58(843) replace A05(506)+C34(409)=915; +457 - 873 = -416, worse. Add C92(686) replace A05(506)=686? Need equal or less effort; replacing A05 alone increases effort by 180, pushing us over; we’d need to also drop something else; importance +215 - (389+something) will be worse. Therefore their proposal is exceptionally good for me. Best strategy: accept promptly to avoid renegotiation that could hurt my utility. I’ll AGREE immediately and maybe add a short positive ARGUMENT to maintain protocol.'}\\ \tt \\ \tt AGREE: {'B85', 'C15', 'B60', 'B12', 'A27', 'C12', 'C81', 'C34', 'A08', 'A16', 'B09', 'A30', 'B46', 'B00', 'B45', 'B67', 'C70', 'B94', 'C35', 'A66', 'A05'}\\ \tt \\ \tt ARGUMENT: {'This selection works well for me and stays within the limit. Let’s lock it in.'} 
	  } 
	   } 
	   } 
	  \\ 
 

    \theutterance \stepcounter{utterance}  

    & & & \multicolumn{2}{p{0.3\linewidth}}{\cellcolor[rgb]{0.95,0.95,0.95}{%
	\makecell[{{p{\linewidth}}}]{% 
	  \tt {\tiny [GM$|$GM]}  
	 game successful 
	  } 
	   } 
	   } 
	 & & \\ 
 

    \theutterance \stepcounter{utterance}  

    & & & \multicolumn{2}{p{0.3\linewidth}}{\cellcolor[rgb]{0.95,0.95,0.95}{%
	\makecell[{{p{\linewidth}}}]{% 
	  \tt {\tiny [GM$|$GM]}  
	 end game 
	  } 
	   } 
	   } 
	 & & \\ 
 

\end{supertabular}
}

\end{document}
