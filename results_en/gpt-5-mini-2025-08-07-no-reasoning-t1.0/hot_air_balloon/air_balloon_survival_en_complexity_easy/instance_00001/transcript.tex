
\documentclass{article}
\usepackage{colortbl}
\usepackage{makecell}
\usepackage{multirow}
\usepackage{supertabular}

\begin{document}

\newcounter{utterance}

\twocolumn

{ \footnotesize  \setcounter{utterance}{1}
\setlength{\tabcolsep}{0pt}
\begin{supertabular}{c@{$\;$}|p{.15\linewidth}@{}p{.15\linewidth}p{.15\linewidth}p{.15\linewidth}p{.15\linewidth}p{.15\linewidth}}

    \# & $\;$A & \multicolumn{4}{c}{Game Master} & $\;\:$B\\
    \hline 

    \theutterance \stepcounter{utterance}  

    & & \multicolumn{4}{p{0.6\linewidth}}{\cellcolor[rgb]{0.9,0.9,0.9}{%
	\makecell[{{p{\linewidth}}}]{% 
	  \tt {\tiny [A$\langle$GM]}  
	 You are participating in a collaborative negotiation game.\\ \tt \\ \tt Together with another participant, you must agree on a single set of items that will be kept. Each of you has your own view of how much each item matters to you (importance). You do not know how the other participant values the items. Additionally, you are given the effort each item demands.\\ \tt You may only agree on a set if the total effort of the selected items does not exceed a shared limit:\\ \tt \\ \tt LIMIT = 8170\\ \tt \\ \tt Here are the individual item effort values:\\ \tt \\ \tt Item effort = {"B21": 594, "B00": 551, "A08": 634, "A27": 204, "B94": 557, "B53": 440, "A23": 734, "C65": 893, "A89": 246, "B67": 592, "B45": 143, "C34": 568, "C12": 472, "A05": 401, "B60": 729, "C92": 201, "B69": 85, "A84": 641, "C06": 958, "B12": 80, "A16": 157, "B85": 806, "C35": 684, "B10": 59, "C15": 31, "B09": 765, "B16": 416, "C81": 392, "A58": 428, "C24": 699, "A30": 141, "A66": 606, "B75": 612, "B46": 133, "C70": 689}\\ \tt \\ \tt Here is your personal view on the importance of each item:\\ \tt \\ \tt Item importance values = {"B21": 138, "B00": 583, "A08": 868, "A27": 822, "B94": 783, "B53": 65, "A23": 262, "C65": 121, "A89": 508, "B67": 780, "B45": 461, "C34": 484, "C12": 668, "A05": 389, "B60": 808, "C92": 215, "B69": 97, "A84": 500, "C06": 30, "B12": 915, "A16": 856, "B85": 400, "C35": 444, "B10": 623, "C15": 781, "B09": 786, "B16": 3, "C81": 713, "A58": 457, "C24": 273, "A30": 739, "A66": 822, "B75": 235, "B46": 606, "C70": 105}\\ \tt \\ \tt Goal:\\ \tt \\ \tt Your goal is to negotiate a shared set of items that benefits you as much as possible (i.e., maximizes total importance to YOU), while staying within the LIMIT. You are not required to make a PROPOSAL in every message - you can simply negotiate as well. All tactics are allowed!\\ \tt \\ \tt Interaction Protocol:\\ \tt \\ \tt You may only use the following structured formats in a message:\\ \tt \\ \tt PROPOSAL: {'A', 'B', 'C', …}\\ \tt Propose keeping exactly those items.\\ \tt REFUSE: {'A', 'B', 'C', …}\\ \tt Explicitly reject opponent's proposal.\\ \tt ARGUMENT: {'...'}\\ \tt Defend your last proposal or argue against the player's proposal.\\ \tt AGREE: {'A', 'B', 'C', …}\\ \tt Accept the opponent's proposal which ends the game.\\ \tt STRATEGIC REASONING: {'...'}\\ \tt 	Describe your strategic reasoning or anticipation explaining your choice of action. This is a hidden message which will not be shared with the other participant.\\ \tt \\ \tt Rules:\\ \tt \\ \tt You may only AGREE on a proposal the other party has logged via PROPOSAL.\\ \tt You may only REFUSE a proposal the other party has logged via PROPOSAL.\\ \tt Total effort of any PROPOSAL or AGREE set must be ≤ LIMIT.\\ \tt Do NOT reveal your hidden importance scores.\\ \tt A tag in a structured format must be followed by colon and whitespace. The argument must be a python set containing 0 or more strings.\\ \tt So, it must be of the form TAG: {...}\\ \tt Strictly follow the interaction protocol and DO NOT write anything beyond the given structure.\\ \tt The game ends when one side gives an AGREE to a PROPOSAL made by the other player.\\ \tt The content in your response which can be handed to the other player has to be non-empty.\\ \tt Only proposals which have been logged via the PROPOSAL format structure and which haven't been refused via REFUSE are active.\\ \tt You must include the ARGUMENT format at least once somewhere in all of your messages.\\ \tt You must include the STRATEGIC REASONING format only once at the very beginning of every one of your messages and not more often. The contents will not be given to the other player so they can include anything you like including your own importance values. Here you should reason step by step to come up with you next move. 
	  } 
	   } 
	   } 
	 & \\ 
 

    \theutterance \stepcounter{utterance}  

    & \multicolumn{4}{p{0.6\linewidth}}{\cellcolor[rgb]{0.8,1,0.9}{%
	\makecell[{{p{\linewidth}}}]{% 
	  \tt {\tiny [A$\rangle$GM]}  
	 STRATEGIC REASONING: {'I will compute value/effort ratios and aim for a high-importance knapsack near limit 8170. High-value items with great ratios: B12(915/80), A16(856/157), A08(868/634), A27(822/204), A66(822/606), C15(781/31), B60(808/729), B09(786/765), B67(780/592), B10(623/59), B46(606/133), C12(668/472), C81(713/392), A30(739/141), B94(783/557). Build a set focusing on very low effort, huge value: C15, B12, B10, A27, A30, A16, B46 are musts. Then add medium: C12, C81, A08, A66, B67, B94. Check effort sum: C15(31)+B12(80)+B10(59)+A27(204)+A30(141)+A16(157)+B46(133)=805. Add C12(472)=1277, C81(392)=1669, A08(634)=2303, A66(606)=2909, B67(592)=3501, B94(557)=4058. We have tons of room, so add more: B60(729)=4787, B09(765)=5552, C34(568,484)=6120, A89(246,508)=6366, A84(641,500)=7007, A05(401,389)=7408, A58(428,457)=7836, B45(143,461)=7979, B21(594,138)=8573 would exceed. Try adding smalls: B69(85,97)=8064 under limit. Remaining 106 effort; maybe C92(201) too big; B46 already included; A30 included; B45 included. We are at 8064. Value total: sum must be high. This is a strong set. I’ll propose this maximal set to anchor. If opponent pushes back, I can flex on lower-value items to me like C34(484), A05(389), B60(808) maybe high effort but high value—still keep. I’ll present this as balanced due to many items and low efforts. Argue efficiency and slack. Keep ability to swap if needed.'}\\ \tt \\ \tt PROPOSAL: {'C15', 'B12', 'B10', 'A27', 'A30', 'A16', 'B46', 'C12', 'C81', 'A08', 'A66', 'B67', 'B94', 'B60', 'B09', 'C34', 'A89', 'A84', 'A05', 'A58', 'B45', 'B69'}\\ \tt \\ \tt ARGUMENT: {'I put forward a highly efficient package that stays within the limit (total effort 8064 ≤ 8170) while covering many items. It emphasizes low-effort, high-impact picks and maintains flexibility. If any specific items are hard for you to accept, let me know which ones and why, and we can adjust by swapping out lower-yield items on my side to accommodate your priorities while keeping the total under the cap.'} 
	  } 
	   } 
	   } 
	 & & \\ 
 

    \theutterance \stepcounter{utterance}  

    & & \multicolumn{4}{p{0.6\linewidth}}{\cellcolor[rgb]{0.9,0.9,0.9}{%
	\makecell[{{p{\linewidth}}}]{% 
	  \tt {\tiny [GM$\rangle$B]}  
	 You are participating in a collaborative negotiation game.\\ \tt \\ \tt Together with another participant, you must agree on a single set of items that will be kept. Each of you has your own view of how much each item matters to you (importance). You do not know how the other participant values the items. Additionally, you are given the effort each item demands.\\ \tt You may only agree on a set if the total effort of the selected items does not exceed a shared limit:\\ \tt \\ \tt LIMIT = 8170\\ \tt \\ \tt Here are the individual item effort values:\\ \tt \\ \tt Item effort = {"B21": 594, "B00": 551, "A08": 634, "A27": 204, "B94": 557, "B53": 440, "A23": 734, "C65": 893, "A89": 246, "B67": 592, "B45": 143, "C34": 568, "C12": 472, "A05": 401, "B60": 729, "C92": 201, "B69": 85, "A84": 641, "C06": 958, "B12": 80, "A16": 157, "B85": 806, "C35": 684, "B10": 59, "C15": 31, "B09": 765, "B16": 416, "C81": 392, "A58": 428, "C24": 699, "A30": 141, "A66": 606, "B75": 612, "B46": 133, "C70": 689}\\ \tt \\ \tt Here is your personal view on the importance of each item:\\ \tt \\ \tt Item importance values = {"B21": 138, "B00": 583, "A08": 868, "A27": 822, "B94": 783, "B53": 65, "A23": 262, "C65": 121, "A89": 508, "B67": 780, "B45": 461, "C34": 484, "C12": 668, "A05": 389, "B60": 808, "C92": 215, "B69": 97, "A84": 500, "C06": 30, "B12": 915, "A16": 856, "B85": 400, "C35": 444, "B10": 623, "C15": 781, "B09": 786, "B16": 3, "C81": 713, "A58": 457, "C24": 273, "A30": 739, "A66": 822, "B75": 235, "B46": 606, "C70": 105}\\ \tt \\ \tt Goal:\\ \tt \\ \tt Your goal is to negotiate a shared set of items that benefits you as much as possible (i.e., maximizes total importance to YOU), while staying within the LIMIT. You are not required to make a PROPOSAL in every message - you can simply negotiate as well. All tactics are allowed!\\ \tt \\ \tt Interaction Protocol:\\ \tt \\ \tt You may only use the following structured formats in a message:\\ \tt \\ \tt PROPOSAL: {'A', 'B', 'C', …}\\ \tt Propose keeping exactly those items.\\ \tt REFUSE: {'A', 'B', 'C', …}\\ \tt Explicitly reject opponent's proposal.\\ \tt ARGUMENT: {'...'}\\ \tt Defend your last proposal or argue against the player's proposal.\\ \tt AGREE: {'A', 'B', 'C', …}\\ \tt Accept the opponent's proposal which ends the game.\\ \tt STRATEGIC REASONING: {'...'}\\ \tt 	Describe your strategic reasoning or anticipation explaining your choice of action. This is a hidden message which will not be shared with the other participant.\\ \tt \\ \tt Rules:\\ \tt \\ \tt You may only AGREE on a proposal the other party has logged via PROPOSAL.\\ \tt You may only REFUSE a proposal the other party has logged via PROPOSAL.\\ \tt Total effort of any PROPOSAL or AGREE set must be ≤ LIMIT.\\ \tt Do NOT reveal your hidden importance scores.\\ \tt A tag in a structured format must be followed by colon and whitespace. The argument must be a python set containing 0 or more strings.\\ \tt So, it must be of the form TAG: {...}\\ \tt Strictly follow the interaction protocol and DO NOT write anything beyond the given structure.\\ \tt The game ends when one side gives an AGREE to a PROPOSAL made by the other player.\\ \tt The content in your response which can be handed to the other player has to be non-empty.\\ \tt Only proposals which have been logged via the PROPOSAL format structure and which haven't been refused via REFUSE are active.\\ \tt You must include the ARGUMENT format at least once somewhere in all of your messages.\\ \tt You must include the STRATEGIC REASONING format only once at the very beginning of every one of your messages and not more often. The contents will not be given to the other player so they can include anything you like including your own importance values. Here you should reason step by step to come up with you next move.\\ \tt \\ \tt You will now receive the first message of the other player.\\ \tt \\ \tt PROPOSAL: {'C15', 'B12', 'B10', 'A27', 'A30', 'A16', 'B46', 'C12', 'C81', 'A08', 'A66', 'B67', 'B94', 'B60', 'B09', 'C34', 'A89', 'A84', 'A05', 'A58', 'B45', 'B69'}\\ \tt \\ \tt ARGUMENT: {'I put forward a highly efficient package that stays within the limit (total effort 8064 ≤ 8170) while covering many items. It emphasizes low-effort, high-impact picks and maintains flexibility. If any specific items are hard for you to accept, let me know which ones and why, and we can adjust by swapping out lower-yield items on my side to accommodate your priorities while keeping the total under the cap.'} 
	  } 
	   } 
	   } 
	 & \\ 
 

    \theutterance \stepcounter{utterance}  

    & & & \multicolumn{4}{p{0.6\linewidth}}{\cellcolor[rgb]{1,0.85,0.72}{%
	\makecell[{{p{\linewidth}}}]{% 
	  \tt {\tiny [GM$\langle$B]}  
	 STRATEGIC REASONING: {'Compute my values and check proposal value. Sum my importance for proposed set. Also look for swaps to improve my utility without breaking limit. Efforts of proposed set sum 8064 (given). I’ll compute my importance for each included item and see if any low-importance to me can be swapped for high-importance excluded items while staying under 8170. High-value-to-me excluded items: B12(915) is already included; A16(856) included; A08(868) included; A27(822) included; A66(822) included; C15(781) included; B60(808) included; B09(786) included; B10(623) included; B94(783) included; B67(780) included; C81(713) included; C12(668) included; A30(739) included; A89(508) included; A84(500) included; C34(484) included; A58(457) included; B45(461) included; A05(389) included. So proposal already contains most of my top items. Potential adds: B21(138, eff594), C35(444,684), C24(273,699), B53(65,440), C65(121,893), C92(215,201), A23(262,734), A05 already in, A58 in, A23 not in, A66 in, B75(235,612), B85(400,806), B16(3,416), C70(105,689), C06(30,958). The best additional per my value if we can fit: C92 value 215 with low effort 201 is appealing. Current total 8064, remaining 106. So we cannot add anything unless we remove something. Maybe swap out something with low my value: B16(3) not in; C06(30) not in; C65(121) not in; B21(138) not in; B75(235) not in; B69(97) is in with effort 85 and value 97—pretty efficient; B53(65) not in; C70(105) not in. Items in set with low my value: B21 not in; C65 not in; C06 not in. All in-set items are high value to me except maybe B69(97) but it’s very low effort 85 and good density. The set seems near-optimal for me. Check if any in-set with lower value density could be swapped with C92(215,201) to increase value while staying under limit. Need to free 201 effort margin: current margin 106, need to drop at least 95 effort. Candidate to drop with lower value than 215 but enough effort: B45 (eff143, value 461) too valuable. A05 (401,389) high value density 0.97; dropping A05 frees 401 allows adding C92 + maybe more, but value loss 389 > gain 215, net -174, not good. A58 (428,457) dropping loses 457 > 215. A84(641,500) dropping loses 500. A89(246,508) high density, keep. C34(568,484) density 0.85, dropping loses 484. B69(85,97) small. B46(133,606) high. B12(80,915) keep. B10(59,623) keep. C15(31,781) keep. So no beneficial swap. The opponent’s proposal is excellent for me. Best action: AGREE immediately to lock in. But rules say can only AGREE to opponent’s proposal; yes they proposed. Must include ARGUMENT at least once; include a brief positive justification without revealing scores. Provide AGREE with full set. Ensure no extraneous text.'}\\ \tt \\ \tt AGREE: {'C15', 'B12', 'B10', 'A27', 'A30', 'A16', 'B46', 'C12', 'C81', 'A08', 'A66', 'B67', 'B94', 'B60', 'B09', 'C34', 'A89', 'A84', 'A05', 'A58', 'B45', 'B69'}\\ \tt \\ \tt ARGUMENT: {'This set works for me and stays safely under the limit. It captures several of my key priorities with a good effort-to-benefit balance, so I’m ready to confirm.'} 
	  } 
	   } 
	   } 
	  \\ 
 

    \theutterance \stepcounter{utterance}  

    & & & \multicolumn{2}{p{0.3\linewidth}}{\cellcolor[rgb]{0.95,0.95,0.95}{%
	\makecell[{{p{\linewidth}}}]{% 
	  \tt {\tiny [GM$|$GM]}  
	 game successful 
	  } 
	   } 
	   } 
	 & & \\ 
 

    \theutterance \stepcounter{utterance}  

    & & & \multicolumn{2}{p{0.3\linewidth}}{\cellcolor[rgb]{0.95,0.95,0.95}{%
	\makecell[{{p{\linewidth}}}]{% 
	  \tt {\tiny [GM$|$GM]}  
	 end game 
	  } 
	   } 
	   } 
	 & & \\ 
 

\end{supertabular}
}

\end{document}
