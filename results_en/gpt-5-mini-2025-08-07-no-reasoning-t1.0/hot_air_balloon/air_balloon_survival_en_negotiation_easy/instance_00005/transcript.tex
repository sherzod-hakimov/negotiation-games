
\documentclass{article}
\usepackage{colortbl}
\usepackage{makecell}
\usepackage{multirow}
\usepackage{supertabular}

\begin{document}

\newcounter{utterance}

\twocolumn

{ \footnotesize  \setcounter{utterance}{1}
\setlength{\tabcolsep}{0pt}
\begin{supertabular}{c@{$\;$}|p{.15\linewidth}@{}p{.15\linewidth}p{.15\linewidth}p{.15\linewidth}p{.15\linewidth}p{.15\linewidth}}

    \# & $\;$A & \multicolumn{4}{c}{Game Master} & $\;\:$B\\
    \hline 

    \theutterance \stepcounter{utterance}  

    & & \multicolumn{4}{p{0.6\linewidth}}{\cellcolor[rgb]{0.9,0.9,0.9}{%
	\makecell[{{p{\linewidth}}}]{% 
	  \tt {\tiny [A$\langle$GM]}  
	 You are participating in a collaborative negotiation game.\\ \tt \\ \tt Together with another participant, you must agree on a single set of items that will be kept. Each of you has your own view of how much each item matters to you (importance). You do not know how the other participant values the items. Additionally, you are given the effort each item demands.\\ \tt You may only agree on a set if the total effort of the selected items does not exceed a shared limit:\\ \tt \\ \tt LIMIT = 3170\\ \tt \\ \tt Here are the individual item effort values:\\ \tt \\ \tt Item effort = {"C08": 424, "C55": 118, "A43": 846, "A67": 907, "B42": 809, "A62": 41, "B19": 193, "C43": 246, "C73": 805, "C10": 601, "B36": 432, "A18": 166, "C61": 119, "C67": 462, "C42": 172}\\ \tt \\ \tt Here is your personal view on the importance of each item:\\ \tt \\ \tt Item importance values = {"C08": 138, "C55": 583, "A43": 868, "A67": 822, "B42": 783, "A62": 65, "B19": 262, "C43": 121, "C73": 508, "C10": 780, "B36": 461, "A18": 484, "C61": 668, "C67": 389, "C42": 808}\\ \tt \\ \tt Goal:\\ \tt \\ \tt Your goal is to negotiate a shared set of items that benefits you as much as possible (i.e., maximizes total importance to YOU), while staying within the LIMIT. You are not required to make a PROPOSAL in every message - you can simply negotiate as well. All tactics are allowed!\\ \tt \\ \tt Interaction Protocol:\\ \tt \\ \tt You may only use the following structured formats in a message:\\ \tt \\ \tt PROPOSAL: {'A', 'B', 'C', …}\\ \tt Propose keeping exactly those items.\\ \tt REFUSE: {'A', 'B', 'C', …}\\ \tt Explicitly reject opponent's proposal.\\ \tt ARGUMENT: {'...'}\\ \tt Defend your last proposal or argue against the player's proposal.\\ \tt AGREE: {'A', 'B', 'C', …}\\ \tt Accept the opponent's proposal which ends the game.\\ \tt STRATEGIC REASONING: {'...'}\\ \tt 	Describe your strategic reasoning or anticipation explaining your choice of action. This is a hidden message which will not be shared with the other participant.\\ \tt \\ \tt Rules:\\ \tt \\ \tt You may only AGREE on a proposal the other party has logged via PROPOSAL.\\ \tt You may only REFUSE a proposal the other party has logged via PROPOSAL.\\ \tt Total effort of any PROPOSAL or AGREE set must be ≤ LIMIT.\\ \tt Do NOT reveal your hidden importance scores.\\ \tt A tag in a structured format must be followed by colon and whitespace. The argument must be a python set containing 0 or more strings.\\ \tt So, it must be of the form TAG: {...}\\ \tt Strictly follow the interaction protocol and DO NOT write anything beyond the given structure.\\ \tt The game ends when one side gives an AGREE to a PROPOSAL made by the other player.\\ \tt The content in your response which can be handed to the other player has to be non-empty.\\ \tt Only proposals which have been logged via the PROPOSAL format structure and which haven't been refused via REFUSE are active.\\ \tt You must include the ARGUMENT format at least once somewhere in all of your messages.\\ \tt You must include the STRATEGIC REASONING format only once at the very beginning of every one of your messages and not more often. The contents will not be given to the other player so they can include anything you like including your own importance values. Here you should reason step by step to come up with you next move. 
	  } 
	   } 
	   } 
	 & \\ 
 

    \theutterance \stepcounter{utterance}  

    & \multicolumn{4}{p{0.6\linewidth}}{\cellcolor[rgb]{0.8,1,0.9}{%
	\makecell[{{p{\linewidth}}}]{% 
	  \tt {\tiny [A$\rangle$GM]}  
	 STRATEGIC REASONING: {'We want to maximize our utility under limit 3170. Compute value/effort ratios and try to include our highest-importance per effort items. Items: efforts and values. Compute greedy and also check knapsack combinations. Sort by ratio v/e: C55 583/118=4.94, C61 668/119=5.61, A62 65/41=1.59 (low), C42 808/172=4.70, B19 262/193=1.36, C43 121/246=0.49, C67 389/462=0.84, B36 461/432=1.07, C08 138/424=0.33, C73 508/805=0.63, B42 783/809=0.97, C10 780/601=1.30, A18 484/166=2.92, A43 868/846=1.03, A67 822/907=0.91. Top by ratio: C61, C55, C42, A18, C10, A43, B42/B36 around 1, A62 low but cheap. Build set starting with high ratio while under 3170 and also include our very high absolute importance: A43(868), C42(808), A67(822), C10(780), C61(668), C55(583), A18(484), B42(783). Try to assemble: Start {C61(119), C55(118), C42(172), A18(166)} effort sum=575, value=668+583+808+484=2543 good. Add C10(601): sum=1176, val=3323. Add A43(846): sum=2022, val=4191. Add B42(809): sum=2831, val=4974. Remaining capacity 339. Consider A67(907) too big. Add A62(41) val+65 sum=2872 val=5039. Remaining 298. Options: B19(193) add -> sum=3065 val=5301. Remaining 105 cannot add anything else (C61/C55 already). That set: {C61,C55,C42,A18,C10,A43,B42,A62,B19} effort 3065 <=3170 value 5301. Maybe we can swap B19 for something else: C43(246) low value 121 worse. C67(462) too big. B36(432) too big. C73(805) too big. C08(424) too big. So current looks good. Could we include A67 by dropping B42? Try including A67: base core {C61,C55,C42,A18,C10,A43}=2022 effort val 4191. Add A67(907) => 2929 effort val 5013. We can still add A62(41) => 2970 val 5078. Then we can add B19(193) would exceed 3163? 2970+193=3163 which is within 3170 -> val 5078+262=5340 slightly higher than set without A67+B42? Compare: previous set with B42 gave 5301; this alternative gives 5340 and uses A67 over B42. Great. Any other add? remaining capacity 7 so none. This looks better. What about also adding something else by different swaps: Could we add B36 instead of B19? 2970+432=3402 too big. C43 2970+246=3216 too big. So best is with B19. What if we drop C10 to include both A67 and B42? Core without C10: base {C61,C55,C42,A18,A43}=1421 effort val 2721. Add A67(907)=>2328 val 3543. Add B42(809)=>3137 val 4326. Remaining 33 could add A62? 3137+41=3178 too big. Maybe swap A18 out to add A62: 3137-166+41=3012 val 4326-484+65=3907 worse. Also we could add C10 instead of B42: we had earlier set. The best found is {C61,C55,C42,A18,C10,A43,A67,A62,B19} effort 3163 value 5340. Try adding C73 by replacing C10 maybe: Replace C10(601,780) with C73(805,508) worse. Replace A18(166,484) with B42(809,783) with keeping others? Start from best set remove A18 (-166) + B42(+809) => 3163-166+809=3806 too big. Maybe replace B19(193,262) with some other better within capacity: remaining capacity in best set is 3170-3163=7, so no room. Could we improve by swapping B19(193,262) + A62(41,65) bundle 234 effort 327 value with something of similar effort but higher value: candidates under 234 effort: C55 is already, C61 already, C42 172 with value 808 but we'd have to drop something else; can't add extra because total must remain ≤ 3170; Replace A62(41)+B19(193)=234 with C42(172,808) would free 62 effort and add 481 value but C42 is already in set. Maybe replace B19 with something <=200 effort: A18 is 166 but in set; C61 119 in set; C55 118 in set; A62 41 in set; C43 246 too big; C42 172 in set; C08 424 no; B36 432 no; C10 601; A43 846; A67 907; B42 809; C67 462; C73 805; B19 seems best remaining cheap item aside from ones included. So best is 5340 set. We'll propose that. Argue efficiency, near capacity, includes several medium-cost high-value items. Keep some flexibility if they object; be ready to trade B19 or A62 for something they value if still under limit.'}\\ \tt \\ \tt PROPOSAL: {'C61', 'C55', 'C42', 'A18', 'C10', 'A43', 'A67', 'A62', 'B19'}\\ \tt \\ \tt ARGUMENT: {'This set uses the effort efficiently (total 3163 ≤ 3170) and balances multiple high-impact items while leaving almost no slack. If you need adjustments, we can discuss swapping one of the smaller items (A62 or B19) for something you value, as long as we stay within the limit.'} 
	  } 
	   } 
	   } 
	 & & \\ 
 

    \theutterance \stepcounter{utterance}  

    & & \multicolumn{4}{p{0.6\linewidth}}{\cellcolor[rgb]{0.9,0.9,0.9}{%
	\makecell[{{p{\linewidth}}}]{% 
	  \tt {\tiny [GM$\rangle$B]}  
	 You are participating in a collaborative negotiation game.\\ \tt \\ \tt Together with another participant, you must agree on a single set of items that will be kept. Each of you has your own view of how much each item matters to you (importance). You do not know how the other participant values the items. Additionally, you are given the effort each item demands.\\ \tt You may only agree on a set if the total effort of the selected items does not exceed a shared limit:\\ \tt \\ \tt LIMIT = 3170\\ \tt \\ \tt Here are the individual item effort values:\\ \tt \\ \tt Item effort = {"C08": 424, "C55": 118, "A43": 846, "A67": 907, "B42": 809, "A62": 41, "B19": 193, "C43": 246, "C73": 805, "C10": 601, "B36": 432, "A18": 166, "C61": 119, "C67": 462, "C42": 172}\\ \tt \\ \tt Here is your personal view on the importance of each item:\\ \tt \\ \tt Item importance values = {"C08": 138, "C55": 583, "A43": 868, "A67": 822, "B42": 783, "A62": 65, "B19": 262, "C43": 121, "C73": 508, "C10": 780, "B36": 461, "A18": 484, "C61": 668, "C67": 389, "C42": 808}\\ \tt \\ \tt Goal:\\ \tt \\ \tt Your goal is to negotiate a shared set of items that benefits you as much as possible (i.e., maximizes total importance to YOU), while staying within the LIMIT. You are not required to make a PROPOSAL in every message - you can simply negotiate as well. All tactics are allowed!\\ \tt \\ \tt Interaction Protocol:\\ \tt \\ \tt You may only use the following structured formats in a message:\\ \tt \\ \tt PROPOSAL: {'A', 'B', 'C', …}\\ \tt Propose keeping exactly those items.\\ \tt REFUSE: {'A', 'B', 'C', …}\\ \tt Explicitly reject opponent's proposal.\\ \tt ARGUMENT: {'...'}\\ \tt Defend your last proposal or argue against the player's proposal.\\ \tt AGREE: {'A', 'B', 'C', …}\\ \tt Accept the opponent's proposal which ends the game.\\ \tt STRATEGIC REASONING: {'...'}\\ \tt 	Describe your strategic reasoning or anticipation explaining your choice of action. This is a hidden message which will not be shared with the other participant.\\ \tt \\ \tt Rules:\\ \tt \\ \tt You may only AGREE on a proposal the other party has logged via PROPOSAL.\\ \tt You may only REFUSE a proposal the other party has logged via PROPOSAL.\\ \tt Total effort of any PROPOSAL or AGREE set must be ≤ LIMIT.\\ \tt Do NOT reveal your hidden importance scores.\\ \tt A tag in a structured format must be followed by colon and whitespace. The argument must be a python set containing 0 or more strings.\\ \tt So, it must be of the form TAG: {...}\\ \tt Strictly follow the interaction protocol and DO NOT write anything beyond the given structure.\\ \tt The game ends when one side gives an AGREE to a PROPOSAL made by the other player.\\ \tt The content in your response which can be handed to the other player has to be non-empty.\\ \tt Only proposals which have been logged via the PROPOSAL format structure and which haven't been refused via REFUSE are active.\\ \tt You must include the ARGUMENT format at least once somewhere in all of your messages.\\ \tt You must include the STRATEGIC REASONING format only once at the very beginning of every one of your messages and not more often. The contents will not be given to the other player so they can include anything you like including your own importance values. Here you should reason step by step to come up with you next move.\\ \tt \\ \tt You will now receive the first message of the other player.\\ \tt \\ \tt PROPOSAL: {'C61', 'C55', 'C42', 'A18', 'C10', 'A43', 'A67', 'A62', 'B19'}\\ \tt \\ \tt ARGUMENT: {'This set uses the effort efficiently (total 3163 ≤ 3170) and balances multiple high-impact items while leaving almost no slack. If you need adjustments, we can discuss swapping one of the smaller items (A62 or B19) for something you value, as long as we stay within the limit.'} 
	  } 
	   } 
	   } 
	 & \\ 
 

    \theutterance \stepcounter{utterance}  

    & & & \multicolumn{4}{p{0.6\linewidth}}{\cellcolor[rgb]{1,0.85,0.72}{%
	\makecell[{{p{\linewidth}}}]{% 
	  \tt {\tiny [GM$\langle$B]}  
	 STRATEGIC REASONING: {'Compute values and effort. Their proposal set: C61(668,119), C55(583,118), C42(808,172), A18(484,166), C10(780,601), A43(868,846), A67(822,907), A62(65,41), B19(262,193). Total importance to me = 5340, effort = 3163/3170 fits. This is actually extremely strong for my preferences, likely near-optimal. Check if any swap could increase my value without exceeding 3170. Remaining items: C08(138,424), B42(783,809), C43(121,246), C73(508,805), B36(461,432), C67(389,462), C10 etc already included. Consider replacing low-importance items A62(65,41) or B19(262,193). Slack is 7. If I drop A62 (41), freed 48 slack: still too small to add anything substantial. If I drop B19 (193), slack 200, still not enough for B36 (432), C67 (462), C43 (246) would need 246>200, so cannot. Dropping both A62 and B19 frees 234+7=241 slack: still not enough for C43(246) short by 5. Best additions under 241: none aside from A62/B19. So the current set is pareto-tight on effort. Could swap A62+some other for B36? To add B36(432), need 432-7=425 effort freed; would have to drop something valuable to me like C61(119), C55(118) etc, which would reduce my value more than gain from B36(461). Check potential swap: drop A62(65,41)+B19(262,193)+C55(583,118) total effort freed 352, slack 359; still short of B36(432). Need more drops harming value. So their set is very close to my optimal. Best strategy: Accept immediately to secure high payoff rather than haggle and risk worse outcome. Must use AGREE format only to accept an existing proposal. Also must include ARGUMENT at least once. So send AGREE with their set, plus ARGUMENT praising efficiency. Ensure effort ≤ limit (3163). Done.'}\\ \tt \\ \tt AGREE: {'C61', 'C55', 'C42', 'A18', 'C10', 'A43', 'A67', 'A62', 'B19'}\\ \tt \\ \tt ARGUMENT: {'Agreed. This selection fits within the limit (3163 ≤ 3170) and is highly efficient with minimal slack. It captures multiple high-impact items and seems hard to improve without exceeding the effort cap.'} 
	  } 
	   } 
	   } 
	  \\ 
 

    \theutterance \stepcounter{utterance}  

    & & & \multicolumn{2}{p{0.3\linewidth}}{\cellcolor[rgb]{0.95,0.95,0.95}{%
	\makecell[{{p{\linewidth}}}]{% 
	  \tt {\tiny [GM$|$GM]}  
	 game successful 
	  } 
	   } 
	   } 
	 & & \\ 
 

    \theutterance \stepcounter{utterance}  

    & & & \multicolumn{2}{p{0.3\linewidth}}{\cellcolor[rgb]{0.95,0.95,0.95}{%
	\makecell[{{p{\linewidth}}}]{% 
	  \tt {\tiny [GM$|$GM]}  
	 end game 
	  } 
	   } 
	   } 
	 & & \\ 
 

\end{supertabular}
}

\end{document}
