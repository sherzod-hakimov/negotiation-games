
\documentclass{article}
\usepackage{colortbl}
\usepackage{makecell}
\usepackage{multirow}
\usepackage{supertabular}

\begin{document}

\newcounter{utterance}

\centering \large Interaction Transcript for game `hot\_air\_balloon', experiment `air\_balloon\_survival\_en\_negotiation\_easy', episode 4 with gpt{-}5{-}mini{-}2025{-}08{-}07{-}no{-}reasoning{-}t1.0.
\vspace{24pt}

{ \footnotesize  \setcounter{utterance}{1}
\setlength{\tabcolsep}{0pt}
\begin{supertabular}{c@{$\;$}|p{.15\linewidth}@{}p{.15\linewidth}p{.15\linewidth}p{.15\linewidth}p{.15\linewidth}p{.15\linewidth}}
    \# & $\;$A & \multicolumn{4}{c}{Game Master} & $\;\:$B\\
    \hline

    \theutterance \stepcounter{utterance}  
    & & \multicolumn{4}{p{0.6\linewidth}}{
        \cellcolor[rgb]{0.9,0.9,0.9}{
            \makecell[{{p{\linewidth}}}]{
                \texttt{\tiny{[P1$\langle$GM]}}
                \texttt{You are participating in a collaborative negotiation game.} \\
\\ 
\texttt{Together with another participant, you must agree on a single set of items that will be kept. Each of you has your own view of how much each item matters to you (importance). You do not know how the other participant values the items. Additionally, you are given the effort each item demands.} \\
\texttt{You may only agree on a set if the total effort of the selected items does not exceed a shared limit:} \\
\\ 
\texttt{LIMIT = 3145} \\
\\ 
\texttt{Here are the individual item effort values:} \\
\\ 
\texttt{Item effort = \{"C08": 948, "C55": 118, "A43": 469, "A67": 919, "B42": 284, "A62": 111, "B19": 806, "C43": 47, "C73": 848, "C10": 303, "B36": 13, "A18": 629, "C61": 687, "C67": 15, "C42": 94\}} \\
\\ 
\texttt{Here is your personal view on the importance of each item:} \\
\\ 
\texttt{Item importance values = \{"C08": 138, "C55": 583, "A43": 868, "A67": 822, "B42": 783, "A62": 65, "B19": 262, "C43": 121, "C73": 508, "C10": 780, "B36": 461, "A18": 484, "C61": 668, "C67": 389, "C42": 808\}} \\
\\ 
\texttt{Goal:} \\
\\ 
\texttt{Your goal is to negotiate a shared set of items that benefits you as much as possible (i.e., maximizes total importance to YOU), while staying within the LIMIT. You are not required to make a PROPOSAL in every message {-} you can simply negotiate as well. All tactics are allowed!} \\
\\ 
\texttt{Interaction Protocol:} \\
\\ 
\texttt{You may only use the following structured formats in a message:} \\
\\ 
\texttt{PROPOSAL: \{'A', 'B', 'C', …\}} \\
\texttt{Propose keeping exactly those items.} \\
\texttt{REFUSE: \{'A', 'B', 'C', …\}} \\
\texttt{Explicitly reject opponent's proposal.} \\
\texttt{ARGUMENT: \{'...'\}} \\
\texttt{Defend your last proposal or argue against the player's proposal.} \\
\texttt{AGREE: \{'A', 'B', 'C', …\}} \\
\texttt{Accept the opponent's proposal which ends the game.} \\
\texttt{STRATEGIC REASONING: \{'...'\}} \\
\texttt{	Describe your strategic reasoning or anticipation explaining your choice of action. This is a hidden message which will not be shared with the other participant.} \\
\\ 
\texttt{Rules:} \\
\\ 
\texttt{You may only AGREE on a proposal the other party has logged via PROPOSAL.} \\
\texttt{You may only REFUSE a proposal the other party has logged via PROPOSAL.} \\
\texttt{Total effort of any PROPOSAL or AGREE set must be ≤ LIMIT.} \\
\texttt{Do NOT reveal your hidden importance scores.} \\
\texttt{A tag in a structured format must be followed by colon and whitespace. The argument must be a python set containing 0 or more strings.} \\
\texttt{So, it must be of the form TAG: \{...\}} \\
\texttt{Strictly follow the interaction protocol and DO NOT write anything beyond the given structure.} \\
\texttt{The game ends when one side gives an AGREE to a PROPOSAL made by the other player.} \\
\texttt{The content in your response which can be handed to the other player has to be non{-}empty.} \\
\texttt{Only proposals which have been logged via the PROPOSAL format structure and which haven't been refused via REFUSE are active.} \\
\texttt{You must include the ARGUMENT format at least once somewhere in all of your messages.} \\
\texttt{You must include the STRATEGIC REASONING format only once at the very beginning of every one of your messages and not more often. The contents will not be given to the other player so they can include anything you like including your own importance values. Here you should reason step by step to come up with you next move.} \\
            }
        }
    }
    & \\ \\

    \theutterance \stepcounter{utterance}  
    & \multicolumn{4}{p{0.6\linewidth}}{
        \cellcolor[rgb]{0.9,0.9,0.9}{
            \makecell[{{p{\linewidth}}}]{
                \texttt{\tiny{[P1$\rangle$GM]}}
                \texttt{STRATEGIC REASONING: \{'Compute value density (importance/effort) and pick high{-}impact items under 3145. Densities: C55 4.94, C67 25.9, B36 35.5, C43 2.57, C42 8.6, C10 2.58, B42 2.76, C61 0.97, A43 1.85, A67 0.895, C73 0.599, A18 0.77, C08 0.145, B19 0.325, A62 0.586. Start with must{-}haves for me: A43(868), C42(808), C10(780), C55(583), C61(668), B36(461), C67(389). Check effort: A43 469, C42 94, C10 303, C55 118, C61 687, B36 13, C67 15 => total 1699, value 4556. Add A67(822,919) {-}> total 2618, value 5378. Room 527. Best additions within 527: B42(284,783) great, brings to 2902 value 6161 room 243. With 243 remaining, options: C43 47(121) {-}> 2949, C08 948 no, C73 848 no, C10 already, A18 629 no, A62 111(65) {-}> 3060, B19 806 no, C61 already, C67 already, C42 already. Try add C43 and A62: new total 3060, value +121+65=6347 room 85 left. Could also add something else ≤85 none. Alternative: instead of A62, pick nothing else, or consider B42 if not already included—yes included. This set seems very strong. Maybe also consider swapping A62(65/111) for slightly better density—none available under 243 except C43 we already took. So proposed set: \{A43, C42, C10, C55, C61, B36, C67, A67, B42, C43, A62\}. Effort 3060 ≤ 3145, value 6347. Leave 85 slack for potential concession if they want to add a tiny item; but only remaining ≤85 is none; so to allow counterpart preferences, we can be ready to swap out A62 (low value to me) or C43 (moderate) to free budget for one of their items up to \textasciitilde{}196 or \textasciitilde{}158 respectively. Start with confident proposal and signal flexibility on small swaps.'\}} \\
\\ 
\texttt{PROPOSAL: \{'A43', 'C42', 'C10', 'C55', 'C61', 'B36', 'C67', 'A67', 'B42', 'C43', 'A62'\}} \\
\\ 
\texttt{ARGUMENT: \{'This set delivers a strong balance of high impact items while staying safely under the limit. It prioritizes broadly useful, lower{-}effort items and keeps room for minor adjustments. If you need to include a small item of yours, I can consider swapping out A62 or C43 to make space.'\}} \\
            }
        }
    }
    & & \\ \\

    \theutterance \stepcounter{utterance}  
    & & \multicolumn{4}{p{0.6\linewidth}}{
        \cellcolor[rgb]{0.9,0.9,0.9}{
            \makecell[{{p{\linewidth}}}]{
                \texttt{\tiny{[GM$\rangle$P2]}}
                \texttt{You are participating in a collaborative negotiation game.} \\
\\ 
\texttt{Together with another participant, you must agree on a single set of items that will be kept. Each of you has your own view of how much each item matters to you (importance). You do not know how the other participant values the items. Additionally, you are given the effort each item demands.} \\
\texttt{You may only agree on a set if the total effort of the selected items does not exceed a shared limit:} \\
\\ 
\texttt{LIMIT = 3145} \\
\\ 
\texttt{Here are the individual item effort values:} \\
\\ 
\texttt{Item effort = \{"C08": 948, "C55": 118, "A43": 469, "A67": 919, "B42": 284, "A62": 111, "B19": 806, "C43": 47, "C73": 848, "C10": 303, "B36": 13, "A18": 629, "C61": 687, "C67": 15, "C42": 94\}} \\
\\ 
\texttt{Here is your personal view on the importance of each item:} \\
\\ 
\texttt{Item importance values = \{"C08": 138, "C55": 583, "A43": 868, "A67": 822, "B42": 783, "A62": 65, "B19": 262, "C43": 121, "C73": 508, "C10": 780, "B36": 461, "A18": 484, "C61": 668, "C67": 389, "C42": 808\}} \\
\\ 
\texttt{Goal:} \\
\\ 
\texttt{Your goal is to negotiate a shared set of items that benefits you as much as possible (i.e., maximizes total importance to YOU), while staying within the LIMIT. You are not required to make a PROPOSAL in every message {-} you can simply negotiate as well. All tactics are allowed!} \\
\\ 
\texttt{Interaction Protocol:} \\
\\ 
\texttt{You may only use the following structured formats in a message:} \\
\\ 
\texttt{PROPOSAL: \{'A', 'B', 'C', …\}} \\
\texttt{Propose keeping exactly those items.} \\
\texttt{REFUSE: \{'A', 'B', 'C', …\}} \\
\texttt{Explicitly reject opponent's proposal.} \\
\texttt{ARGUMENT: \{'...'\}} \\
\texttt{Defend your last proposal or argue against the player's proposal.} \\
\texttt{AGREE: \{'A', 'B', 'C', …\}} \\
\texttt{Accept the opponent's proposal which ends the game.} \\
\texttt{STRATEGIC REASONING: \{'...'\}} \\
\texttt{	Describe your strategic reasoning or anticipation explaining your choice of action. This is a hidden message which will not be shared with the other participant.} \\
\\ 
\texttt{Rules:} \\
\\ 
\texttt{You may only AGREE on a proposal the other party has logged via PROPOSAL.} \\
\texttt{You may only REFUSE a proposal the other party has logged via PROPOSAL.} \\
\texttt{Total effort of any PROPOSAL or AGREE set must be ≤ LIMIT.} \\
\texttt{Do NOT reveal your hidden importance scores.} \\
\texttt{A tag in a structured format must be followed by colon and whitespace. The argument must be a python set containing 0 or more strings.} \\
\texttt{So, it must be of the form TAG: \{...\}} \\
\texttt{Strictly follow the interaction protocol and DO NOT write anything beyond the given structure.} \\
\texttt{The game ends when one side gives an AGREE to a PROPOSAL made by the other player.} \\
\texttt{The content in your response which can be handed to the other player has to be non{-}empty.} \\
\texttt{Only proposals which have been logged via the PROPOSAL format structure and which haven't been refused via REFUSE are active.} \\
\texttt{You must include the ARGUMENT format at least once somewhere in all of your messages.} \\
\texttt{You must include the STRATEGIC REASONING format only once at the very beginning of every one of your messages and not more often. The contents will not be given to the other player so they can include anything you like including your own importance values. Here you should reason step by step to come up with you next move.} \\
\\ 
\texttt{You will now receive the first message of the other player.} \\
\\ 
\texttt{PROPOSAL: \{'A43', 'C42', 'C10', 'C55', 'C61', 'B36', 'C67', 'A67', 'B42', 'C43', 'A62'\}} \\
\\ 
\texttt{ARGUMENT: \{'This set delivers a strong balance of high impact items while staying safely under the limit. It prioritizes broadly useful, lower{-}effort items and keeps room for minor adjustments. If you need to include a small item of yours, I can consider swapping out A62 or C43 to make space.'\}} \\
            }
        }
    }
    & \\ \\

    \theutterance \stepcounter{utterance}  
    & & & \multicolumn{4}{p{0.6\linewidth}}{
        \cellcolor[rgb]{0.9,0.9,0.9}{
            \makecell[{{p{\linewidth}}}]{
                \texttt{\tiny{[GM$\langle$P2]}}
                \texttt{STRATEGIC REASONING: \{'Compute efforts and values. Opponent proposal items efforts: A43 469, C42 94, C10 303, C55 118, C61 687, B36 13, C67 15, A67 919, B42 284, C43 47, A62 111. Total effort = 3060, under 3145. My values sum: A43 868 + C42 808 + C10 780 + C55 583 + C61 668 + B36 461 + C67 389 + A67 822 + B42 783 + C43 121 + A62 65 = 6438. This is very good for me, missing only a few items with value: C08 138 (effort 948), B19 262 (806), C73 508 (848), A18 484 (629), C10 etc already included. We have 85 effort slack. I could try to swap in A18 (629, value 484) by removing something low value per my prefs: A62 (65, 111) and C43 (121, 47) and maybe C67 (389, 15) or C55 (583,118) etc to make room. Need 629 effort: remove A62 111 + C43 47 + maybe C55 118 + C42 94 + B36 13 + C67 15 + use 85 slack. Sum removable smalls: 111+47+118+94+13+15+85=483, still short; need remove another like C10 303 or C61 687 or A67 919 or B42 284 or A43 469. Best trade{-}off for me would be to keep high{-}value{-}per{-}effort; compute V/E: A43 1.85, C42 8.6, C10 2.57, C55 4.94, C61 0.97, B36 35.46, C67 25.93, A67 0.895, B42 2.76, C43 2.57, A62 0.59. Low efficiency are A62 and A67 and C61 and A43 moderately good, B42 okay. Replacing A67 (919 effort, 822 value) with A18 (629, 484) would free 290 effort but lose 338 value, not good. Replacing C61 (687,668) with A18 (629,484) frees 58 effort but loses 184 value, bad. Replace A62 (111,65) + use slack 85 + C43 (47,121) + maybe small more to reach 629{-} (85+111+47)=386 left; could drop C10 (303,780) but that loses big value. Not worth it. Adding any remaining small item? Slack 85 so none fit: smallest remaining not included: C67 already included; B36 included; C43 included; A62 included; next smallest C42 included; C55 included; C10 included; B42 included; so no addition possible. Conclusion: Opponent proposal is near{-}optimal for me; I should accept without revealing. To appear collaborative, I can praise balance and agree outright. Must include ARGUMENT at least once. Also must only AGREE to a proposal made; yes they proposed. Do AGREE with same set.'\}} \\
\\ 
\texttt{ARGUMENT: \{'This is a well{-}balanced, efficient mix that stays comfortably under the limit and captures strong value across categories. I’m satisfied with this selection as{-}is.'\}} \\
\\ 
\texttt{AGREE: \{'A43', 'C42', 'C10', 'C55', 'C61', 'B36', 'C67', 'A67', 'B42', 'C43', 'A62'\}} \\
            }
        }
    }
     \\ \\

    \theutterance \stepcounter{utterance}  
    & & & \multicolumn{2}{p{0.3\linewidth}}{
        \cellcolor[rgb]{0.9,0.9,0.9}{
            \makecell[{{p{\linewidth}}}]{
                \texttt{\tiny{[GM$|$GM]}}
                \texttt{game successful} \\
            }
        }
    }
    & & \\ \\

    \theutterance \stepcounter{utterance}  
    & & & \multicolumn{2}{p{0.3\linewidth}}{
        \cellcolor[rgb]{0.9,0.9,0.9}{
            \makecell[{{p{\linewidth}}}]{
                \texttt{\tiny{[GM$|$GM]}}
                \texttt{end game} \\
            }
        }
    }
    & & \\ \\

\end{supertabular}
}

\end{document}
