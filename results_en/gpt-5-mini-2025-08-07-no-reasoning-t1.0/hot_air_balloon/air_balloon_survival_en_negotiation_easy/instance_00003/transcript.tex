
\documentclass{article}
\usepackage{colortbl}
\usepackage{makecell}
\usepackage{multirow}
\usepackage{supertabular}

\begin{document}

\newcounter{utterance}

\twocolumn

{ \footnotesize  \setcounter{utterance}{1}
\setlength{\tabcolsep}{0pt}
\begin{supertabular}{c@{$\;$}|p{.15\linewidth}@{}p{.15\linewidth}p{.15\linewidth}p{.15\linewidth}p{.15\linewidth}p{.15\linewidth}}

    \# & $\;$A & \multicolumn{4}{c}{Game Master} & $\;\:$B\\
    \hline 

    \theutterance \stepcounter{utterance}  

    & & \multicolumn{4}{p{0.6\linewidth}}{\cellcolor[rgb]{0.9,0.9,0.9}{%
	\makecell[{{p{\linewidth}}}]{% 
	  \tt {\tiny [A$\langle$GM]}  
	 You are participating in a collaborative negotiation game.\\ \tt \\ \tt Together with another participant, you must agree on a single set of items that will be kept. Each of you has your own view of how much each item matters to you (importance). You do not know how the other participant values the items. Additionally, you are given the effort each item demands.\\ \tt You may only agree on a set if the total effort of the selected items does not exceed a shared limit:\\ \tt \\ \tt LIMIT = 3712\\ \tt \\ \tt Here are the individual item effort values:\\ \tt \\ \tt Item effort = {"C08": 949, "C55": 388, "A43": 79, "A67": 585, "B42": 564, "A62": 230, "B19": 580, "C43": 84, "C73": 976, "C10": 274, "B36": 374, "A18": 913, "C61": 303, "C67": 578, "C42": 548}\\ \tt \\ \tt Here is your personal view on the importance of each item:\\ \tt \\ \tt Item importance values = {"C08": 138, "C55": 583, "A43": 868, "A67": 822, "B42": 783, "A62": 65, "B19": 262, "C43": 121, "C73": 508, "C10": 780, "B36": 461, "A18": 484, "C61": 668, "C67": 389, "C42": 808}\\ \tt \\ \tt Goal:\\ \tt \\ \tt Your goal is to negotiate a shared set of items that benefits you as much as possible (i.e., maximizes total importance to YOU), while staying within the LIMIT. You are not required to make a PROPOSAL in every message - you can simply negotiate as well. All tactics are allowed!\\ \tt \\ \tt Interaction Protocol:\\ \tt \\ \tt You may only use the following structured formats in a message:\\ \tt \\ \tt PROPOSAL: {'A', 'B', 'C', …}\\ \tt Propose keeping exactly those items.\\ \tt REFUSE: {'A', 'B', 'C', …}\\ \tt Explicitly reject opponent's proposal.\\ \tt ARGUMENT: {'...'}\\ \tt Defend your last proposal or argue against the player's proposal.\\ \tt AGREE: {'A', 'B', 'C', …}\\ \tt Accept the opponent's proposal which ends the game.\\ \tt STRATEGIC REASONING: {'...'}\\ \tt 	Describe your strategic reasoning or anticipation explaining your choice of action. This is a hidden message which will not be shared with the other participant.\\ \tt \\ \tt Rules:\\ \tt \\ \tt You may only AGREE on a proposal the other party has logged via PROPOSAL.\\ \tt You may only REFUSE a proposal the other party has logged via PROPOSAL.\\ \tt Total effort of any PROPOSAL or AGREE set must be ≤ LIMIT.\\ \tt Do NOT reveal your hidden importance scores.\\ \tt A tag in a structured format must be followed by colon and whitespace. The argument must be a python set containing 0 or more strings.\\ \tt So, it must be of the form TAG: {...}\\ \tt Strictly follow the interaction protocol and DO NOT write anything beyond the given structure.\\ \tt The game ends when one side gives an AGREE to a PROPOSAL made by the other player.\\ \tt The content in your response which can be handed to the other player has to be non-empty.\\ \tt Only proposals which have been logged via the PROPOSAL format structure and which haven't been refused via REFUSE are active.\\ \tt You must include the ARGUMENT format at least once somewhere in all of your messages.\\ \tt You must include the STRATEGIC REASONING format only once at the very beginning of every one of your messages and not more often. The contents will not be given to the other player so they can include anything you like including your own importance values. Here you should reason step by step to come up with you next move. 
	  } 
	   } 
	   } 
	 & \\ 
 

    \theutterance \stepcounter{utterance}  

    & \multicolumn{4}{p{0.6\linewidth}}{\cellcolor[rgb]{0.8,1,0.9}{%
	\makecell[{{p{\linewidth}}}]{% 
	  \tt {\tiny [A$\rangle$GM]}  
	 STRATEGIC REASONING: {'Compute my values/efforts. Aim to maximize my importance under 3712. Sort by importance/effort. Items (value, effort): A43(868,79), C42(808,548), A67(822,585), C10(780,274), B42(783,564), C61(668,303), C55(583,388), A18(484,913), B36(461,374), C73(508,976), C67(389,578), B19(262,580), C08(138,949), C43(121,84), A62(65,230). Greedy pick high value/low effort first: A43, C10, C61, C55, B42, A67, C42, B36, then see capacity. Sum efforts: 79+274+303+388+564+585+548+374=3115. Value sum: 868+780+668+583+783+822+808+461=5773. Remaining capacity 597. Consider adding C43(84,121)->effort 3199 value 5894. Remaining 513. Candidates under 513: A62(230,65), C10 already, C61 already, C55 already, B36 already, C42 548 no, A67 585 no, B42 564 no, C67 578 no, B19 580 no, A18 913, C73 976, C08 949. Try add A62 -> effort 3429 value 5959 remaining 283. Nothing else fits except nothing. Also check maybe swap to include C67(389) replacing something? But we’re near cap; C67 effort 578 > remaining. Could we include C73 big? No. This set seems strong. Try see if adding A18 by removing something low-ratio: A62 has low value; A18 adds 484 for 913 effort, would need to drop at least 913 of effort: maybe drop B36(374,461) and C55(388,583) total 762 effort; still need 151 more like C43(84,121) tot 846 still short; plus A62(230) would be too much drop and lose value. Our current is efficient. So propose set: {A43,C10,C61,C55,B42,A67,C42,B36,C43,A62} effort 3429 ≤ 3712. Keep some wiggle room to concede later by dropping B36 or A62 or C43 depending on their asks. Start with confident argument stressing low effort/high impact items A43, C10. Ready to negotiate.'}\\ \tt PROPOSAL: {'A43', 'C10', 'C61', 'C55', 'B42', 'A67', 'C42', 'B36', 'C43', 'A62'}\\ \tt ARGUMENT: {'This set stays well under the effort limit while focusing on highly impactful, efficient items. In particular, A43 and C10 deliver strong benefits for minimal effort, and the core group {A67, C42, B42, C61, C55} provides substantial value. I included B36 and small-effort items C43 and A62 to round out the capacity without overloading. If you need adjustments, I’m open to minor swaps on the fringe items like A62, C43, or B36 while preserving the core.'} 
	  } 
	   } 
	   } 
	 & & \\ 
 

    \theutterance \stepcounter{utterance}  

    & & \multicolumn{4}{p{0.6\linewidth}}{\cellcolor[rgb]{0.9,0.9,0.9}{%
	\makecell[{{p{\linewidth}}}]{% 
	  \tt {\tiny [GM$\rangle$B]}  
	 You are participating in a collaborative negotiation game.\\ \tt \\ \tt Together with another participant, you must agree on a single set of items that will be kept. Each of you has your own view of how much each item matters to you (importance). You do not know how the other participant values the items. Additionally, you are given the effort each item demands.\\ \tt You may only agree on a set if the total effort of the selected items does not exceed a shared limit:\\ \tt \\ \tt LIMIT = 3712\\ \tt \\ \tt Here are the individual item effort values:\\ \tt \\ \tt Item effort = {"C08": 949, "C55": 388, "A43": 79, "A67": 585, "B42": 564, "A62": 230, "B19": 580, "C43": 84, "C73": 976, "C10": 274, "B36": 374, "A18": 913, "C61": 303, "C67": 578, "C42": 548}\\ \tt \\ \tt Here is your personal view on the importance of each item:\\ \tt \\ \tt Item importance values = {"C08": 138, "C55": 583, "A43": 868, "A67": 822, "B42": 783, "A62": 65, "B19": 262, "C43": 121, "C73": 508, "C10": 780, "B36": 461, "A18": 484, "C61": 668, "C67": 389, "C42": 808}\\ \tt \\ \tt Goal:\\ \tt \\ \tt Your goal is to negotiate a shared set of items that benefits you as much as possible (i.e., maximizes total importance to YOU), while staying within the LIMIT. You are not required to make a PROPOSAL in every message - you can simply negotiate as well. All tactics are allowed!\\ \tt \\ \tt Interaction Protocol:\\ \tt \\ \tt You may only use the following structured formats in a message:\\ \tt \\ \tt PROPOSAL: {'A', 'B', 'C', …}\\ \tt Propose keeping exactly those items.\\ \tt REFUSE: {'A', 'B', 'C', …}\\ \tt Explicitly reject opponent's proposal.\\ \tt ARGUMENT: {'...'}\\ \tt Defend your last proposal or argue against the player's proposal.\\ \tt AGREE: {'A', 'B', 'C', …}\\ \tt Accept the opponent's proposal which ends the game.\\ \tt STRATEGIC REASONING: {'...'}\\ \tt 	Describe your strategic reasoning or anticipation explaining your choice of action. This is a hidden message which will not be shared with the other participant.\\ \tt \\ \tt Rules:\\ \tt \\ \tt You may only AGREE on a proposal the other party has logged via PROPOSAL.\\ \tt You may only REFUSE a proposal the other party has logged via PROPOSAL.\\ \tt Total effort of any PROPOSAL or AGREE set must be ≤ LIMIT.\\ \tt Do NOT reveal your hidden importance scores.\\ \tt A tag in a structured format must be followed by colon and whitespace. The argument must be a python set containing 0 or more strings.\\ \tt So, it must be of the form TAG: {...}\\ \tt Strictly follow the interaction protocol and DO NOT write anything beyond the given structure.\\ \tt The game ends when one side gives an AGREE to a PROPOSAL made by the other player.\\ \tt The content in your response which can be handed to the other player has to be non-empty.\\ \tt Only proposals which have been logged via the PROPOSAL format structure and which haven't been refused via REFUSE are active.\\ \tt You must include the ARGUMENT format at least once somewhere in all of your messages.\\ \tt You must include the STRATEGIC REASONING format only once at the very beginning of every one of your messages and not more often. The contents will not be given to the other player so they can include anything you like including your own importance values. Here you should reason step by step to come up with you next move.\\ \tt \\ \tt You will now receive the first message of the other player.\\ \tt \\ \tt PROPOSAL: {'A43', 'C10', 'C61', 'C55', 'B42', 'A67', 'C42', 'B36', 'C43', 'A62'}\\ \tt ARGUMENT: {'This set stays well under the effort limit while focusing on highly impactful, efficient items. In particular, A43 and C10 deliver strong benefits for minimal effort, and the core group {A67, C42, B42, C61, C55} provides substantial value. I included B36 and small-effort items C43 and A62 to round out the capacity without overloading. If you need adjustments, I’m open to minor swaps on the fringe items like A62, C43, or B36 while preserving the core.'} 
	  } 
	   } 
	   } 
	 & \\ 
 

    \theutterance \stepcounter{utterance}  

    & & & \multicolumn{4}{p{0.6\linewidth}}{\cellcolor[rgb]{1,0.85,0.72}{%
	\makecell[{{p{\linewidth}}}]{% 
	  \tt {\tiny [GM$\langle$B]}  
	 STRATEGIC REASONING: {'I’ll evaluate their proposal’s effort and my value, then counter with a higher-value set within 3712. I want to keep my top items: A43(868,79), C42(808,548), C10(780,274), B42(783,564), A67(822,585), C61(668,303), C55(583,388). That core matches much of theirs. I should add other high-value-per-effort pieces like B36(461,374), C67(389,578) only if room, but C67 is mediocre efficiency. A18(484,913) is heavy; C73(508,976) heavy; C08(138,949) bad. B19(262,580) mediocre. A62(65,230) and C43(121,84) are cheap fillers. First, compute their proposal: items = {A43(79), C10(274), C61(303), C55(388), B42(564), A67(585), C42(548), B36(374), C43(84), A62(230)} effort sum = 79+274+303+388+564+585+548+374+84+230 = 3429, under 3712. My value of their set = 868+780+668+583+783+822+808+461+121+65 = 5959. I can try to add one more high-value item within remaining 3712-3429=283. Options under 283 effort: A43 already in; C10 in; C61 in; C55 in; A62 in; C43 in; C10 is 274 already in; C61 303 too big; C55 388 too big; B36 374; C42 548; B42 564; A67 585; C67 578; B19 580; C08 949; C73 976; A18 913. Only remaining not-in and <=283: none. So their set is near capacity. Could we swap to increase my value? Consider replacing low-value A62(65,230) with something around 230-230+remaining 283 = 513 extra capacity if we free A62, new cap=3429-230=3199, remaining to 3712 = 513. What can fit ≤513? C61 303 (already in), C10 274 (in), C55 388 (in), B36 374 (in), C42 548 (too big), B42 564 (too big), A67 585 too big, C67 578 too big, B19 580 too big, A18 913, C73 976, C08 949. The only high-value not in and ≤513 is none besides those already included. What about replacing C43(121,84) plus A62(65,230) combined free 314, then remaining becomes 3712- (3429-314)=597. With 597 we could add C67(389,578)? Effort 578 fits within 597, yes. Value gain: add C67 value 389, remove 121+65=186 net +203. New effort = 3429 - 314 + 578 = 3693 ≤ 3712. That improves my total. What about instead add B19(262,580) value 262 effort 580 fits within 597; net +262-186=+76, worse than C67. Any other items under 597: C42 548 already in; B42 564 in; A67 585 in; A18 913 no; C73 976; C08 949. So best swap is replace A62 and C43 with C67. Alternatively replace B36(461,374) and A62(65,230) total freed 604; could add C67(389,578) and keep C43 84 remainder 604-578=26, net value change: +389 - (461+65) = -137, worse. Replace B36 and C43 (374+84=458), add C67 578 exceeds extra 283+? Actually base remaining 283; removing 458 increases remaining to 741; 741-578=163; net value +389 - (461+121) = -193 worse. Replace B36 with C67 alone: effort +204; current remaining 283, so fits; net value -72 worse. Replace nothing, add nothing yields value 5959. So propose optimized set: swap out A62 and C43 for C67 to reach value 6162. I’ll present as a minor swap aligned with their openness to adjust fringe. I will include ARGUMENT. Then make PROPOSAL with that set.'}\\ \tt PROPOSAL: {'A43', 'C10', 'C61', 'C55', 'B42', 'A67', 'C42', 'B36', 'C67'}\\ \tt ARGUMENT: {'Great proposal and core. To maximize impact while staying within the same capacity, let’s do a small tweak you invited on the fringe: replace the two low-impact fillers A62 and C43 with a single stronger item C67. Effort stays under the limit (new total 3693 ≤ 3712), and the overall value improves materially. Everything else remains exactly as you suggested.'} 
	  } 
	   } 
	   } 
	  \\ 
 

    \theutterance \stepcounter{utterance}  

    & & \multicolumn{4}{p{0.6\linewidth}}{\cellcolor[rgb]{0.9,0.9,0.9}{%
	\makecell[{{p{\linewidth}}}]{% 
	  \tt {\tiny [A$\langle$GM]}  
	 PROPOSAL: {'A43', 'C10', 'C61', 'C55', 'B42', 'A67', 'C42', 'B36', 'C67'}\\ \tt ARGUMENT: {'Great proposal and core. To maximize impact while staying within the same capacity, let’s do a small tweak you invited on the fringe: replace the two low-impact fillers A62 and C43 with a single stronger item C67. Effort stays under the limit (new total 3693 ≤ 3712), and the overall value improves materially. Everything else remains exactly as you suggested.'} 
	  } 
	   } 
	   } 
	 & \\ 
 

    \theutterance \stepcounter{utterance}  

    & \multicolumn{4}{p{0.6\linewidth}}{\cellcolor[rgb]{0.8,1,0.9}{%
	\makecell[{{p{\linewidth}}}]{% 
	  \tt {\tiny [A$\rangle$GM]}  
	 STRATEGIC REASONING: {'Opponent proposes swapping out A62(65,230) and C43(121,84) for C67(389,578). My current set value sum was 5959 with A62 and C43 included. Removing them loses 186 value and 314 effort; adding C67 adds 389 value and 578 effort. Net value +203, net effort +264. My prior effort was 3429; plus 264 = 3693, within 3712 as they stated. From my perspective, this is a strict improvement in my total importance and still under the cap. No reason to refuse; accept to end the game.'}\\ \tt AGREE: {'A43', 'C10', 'C61', 'C55', 'B42', 'A67', 'C42', 'B36', 'C67'}\\ \tt ARGUMENT: {'Agreed. Your tweak increases overall impact while keeping us within the limit. This set preserves the strong core and upgrades the fringe efficiently.'} 
	  } 
	   } 
	   } 
	 & & \\ 
 

    \theutterance \stepcounter{utterance}  

    & & & \multicolumn{2}{p{0.3\linewidth}}{\cellcolor[rgb]{0.95,0.95,0.95}{%
	\makecell[{{p{\linewidth}}}]{% 
	  \tt {\tiny [GM$|$GM]}  
	 game successful 
	  } 
	   } 
	   } 
	 & & \\ 
 

    \theutterance \stepcounter{utterance}  

    & & & \multicolumn{2}{p{0.3\linewidth}}{\cellcolor[rgb]{0.95,0.95,0.95}{%
	\makecell[{{p{\linewidth}}}]{% 
	  \tt {\tiny [GM$|$GM]}  
	 end game 
	  } 
	   } 
	   } 
	 & & \\ 
 

\end{supertabular}
}

\end{document}
