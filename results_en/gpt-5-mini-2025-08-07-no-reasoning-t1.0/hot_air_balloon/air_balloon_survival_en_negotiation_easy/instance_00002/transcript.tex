
\documentclass{article}
\usepackage{colortbl}
\usepackage{makecell}
\usepackage{multirow}
\usepackage{supertabular}

\begin{document}

\newcounter{utterance}

\twocolumn

{ \footnotesize  \setcounter{utterance}{1}
\setlength{\tabcolsep}{0pt}
\begin{supertabular}{c@{$\;$}|p{.15\linewidth}@{}p{.15\linewidth}p{.15\linewidth}p{.15\linewidth}p{.15\linewidth}p{.15\linewidth}}

    \# & $\;$A & \multicolumn{4}{c}{Game Master} & $\;\:$B\\
    \hline 

    \theutterance \stepcounter{utterance}  

    & & \multicolumn{4}{p{0.6\linewidth}}{\cellcolor[rgb]{0.9,0.9,0.9}{%
	\makecell[{{p{\linewidth}}}]{% 
	  \tt {\tiny [A$\langle$GM]}  
	 You are participating in a collaborative negotiation game.\\ \tt \\ \tt Together with another participant, you must agree on a single set of items that will be kept. Each of you has your own view of how much each item matters to you (importance). You do not know how the other participant values the items. Additionally, you are given the effort each item demands.\\ \tt You may only agree on a set if the total effort of the selected items does not exceed a shared limit:\\ \tt \\ \tt LIMIT = 3288\\ \tt \\ \tt Here are the individual item effort values:\\ \tt \\ \tt Item effort = {"C08": 329, "C55": 41, "A43": 417, "A67": 75, "B42": 390, "A62": 887, "B19": 808, "C43": 151, "C73": 849, "C10": 129, "B36": 350, "A18": 118, "C61": 630, "C67": 602, "C42": 801}\\ \tt \\ \tt Here is your personal view on the importance of each item:\\ \tt \\ \tt Item importance values = {"C08": 138, "C55": 583, "A43": 868, "A67": 822, "B42": 783, "A62": 65, "B19": 262, "C43": 121, "C73": 508, "C10": 780, "B36": 461, "A18": 484, "C61": 668, "C67": 389, "C42": 808}\\ \tt \\ \tt Goal:\\ \tt \\ \tt Your goal is to negotiate a shared set of items that benefits you as much as possible (i.e., maximizes total importance to YOU), while staying within the LIMIT. You are not required to make a PROPOSAL in every message - you can simply negotiate as well. All tactics are allowed!\\ \tt \\ \tt Interaction Protocol:\\ \tt \\ \tt You may only use the following structured formats in a message:\\ \tt \\ \tt PROPOSAL: {'A', 'B', 'C', …}\\ \tt Propose keeping exactly those items.\\ \tt REFUSE: {'A', 'B', 'C', …}\\ \tt Explicitly reject opponent's proposal.\\ \tt ARGUMENT: {'...'}\\ \tt Defend your last proposal or argue against the player's proposal.\\ \tt AGREE: {'A', 'B', 'C', …}\\ \tt Accept the opponent's proposal which ends the game.\\ \tt STRATEGIC REASONING: {'...'}\\ \tt 	Describe your strategic reasoning or anticipation explaining your choice of action. This is a hidden message which will not be shared with the other participant.\\ \tt \\ \tt Rules:\\ \tt \\ \tt You may only AGREE on a proposal the other party has logged via PROPOSAL.\\ \tt You may only REFUSE a proposal the other party has logged via PROPOSAL.\\ \tt Total effort of any PROPOSAL or AGREE set must be ≤ LIMIT.\\ \tt Do NOT reveal your hidden importance scores.\\ \tt A tag in a structured format must be followed by colon and whitespace. The argument must be a python set containing 0 or more strings.\\ \tt So, it must be of the form TAG: {...}\\ \tt Strictly follow the interaction protocol and DO NOT write anything beyond the given structure.\\ \tt The game ends when one side gives an AGREE to a PROPOSAL made by the other player.\\ \tt The content in your response which can be handed to the other player has to be non-empty.\\ \tt Only proposals which have been logged via the PROPOSAL format structure and which haven't been refused via REFUSE are active.\\ \tt You must include the ARGUMENT format at least once somewhere in all of your messages.\\ \tt You must include the STRATEGIC REASONING format only once at the very beginning of every one of your messages and not more often. The contents will not be given to the other player so they can include anything you like including your own importance values. Here you should reason step by step to come up with you next move. 
	  } 
	   } 
	   } 
	 & \\ 
 

    \theutterance \stepcounter{utterance}  

    & \multicolumn{4}{p{0.6\linewidth}}{\cellcolor[rgb]{0.8,1,0.9}{%
	\makecell[{{p{\linewidth}}}]{% 
	  \tt {\tiny [A$\rangle$GM]}  
	 STRATEGIC REASONING: {'Maximize my total importance within 3288 effort. Compute value/effort and pick high-importance, efficient items, then adjust to fit limit and leave room for concessions. Items with very high importance: A43(868,417), C42(808,801), A67(822,75), C10(780,129), B42(783,390), C61(668,630), A18(484,118), B36(461,350), C55(583,41). Low value/expensive: A62(65,887), B19(262,808), C73(508,849), C67(389,602), C08(138,329), C43(121,151). Start with a strong, near-cap set excluding low-value bulky items. Compute a high-score bundle: {A43, A67, C10, B42, C61, A18, B36, C55, C42}. Effort: 417+75+129+390+630+118+350+41+801=2951 (≤3288). Score: 868+822+780+783+668+484+461+583+808=6257. Leaves 337 effort. Consider adding C08(329,138) exactly fits to 3280 total, boosting to 6395. That’s optimal-looking without adding poor items. Keep C73(849) and A62(887) out. Offer this as opening. Prepare arguments on balance and efficiency. Keep flexibility to drop C08 or B36 if they push back.'}\\ \tt \\ \tt PROPOSAL: {'A43', 'A67', 'C10', 'B42', 'C61', 'A18', 'B36', 'C55', 'C42', 'C08'}\\ \tt \\ \tt ARGUMENT: {'This set delivers a strong balance of benefits while staying well under the limit (total effort 3280 ≤ 3288). It emphasizes compact, efficient items and avoids the heaviest, least efficient ones. If you have objections, I can consider small adjustments around C08 or B36 while keeping the core high-impact items.'} 
	  } 
	   } 
	   } 
	 & & \\ 
 

    \theutterance \stepcounter{utterance}  

    & & \multicolumn{4}{p{0.6\linewidth}}{\cellcolor[rgb]{0.9,0.9,0.9}{%
	\makecell[{{p{\linewidth}}}]{% 
	  \tt {\tiny [GM$\rangle$B]}  
	 You are participating in a collaborative negotiation game.\\ \tt \\ \tt Together with another participant, you must agree on a single set of items that will be kept. Each of you has your own view of how much each item matters to you (importance). You do not know how the other participant values the items. Additionally, you are given the effort each item demands.\\ \tt You may only agree on a set if the total effort of the selected items does not exceed a shared limit:\\ \tt \\ \tt LIMIT = 3288\\ \tt \\ \tt Here are the individual item effort values:\\ \tt \\ \tt Item effort = {"C08": 329, "C55": 41, "A43": 417, "A67": 75, "B42": 390, "A62": 887, "B19": 808, "C43": 151, "C73": 849, "C10": 129, "B36": 350, "A18": 118, "C61": 630, "C67": 602, "C42": 801}\\ \tt \\ \tt Here is your personal view on the importance of each item:\\ \tt \\ \tt Item importance values = {"C08": 138, "C55": 583, "A43": 868, "A67": 822, "B42": 783, "A62": 65, "B19": 262, "C43": 121, "C73": 508, "C10": 780, "B36": 461, "A18": 484, "C61": 668, "C67": 389, "C42": 808}\\ \tt \\ \tt Goal:\\ \tt \\ \tt Your goal is to negotiate a shared set of items that benefits you as much as possible (i.e., maximizes total importance to YOU), while staying within the LIMIT. You are not required to make a PROPOSAL in every message - you can simply negotiate as well. All tactics are allowed!\\ \tt \\ \tt Interaction Protocol:\\ \tt \\ \tt You may only use the following structured formats in a message:\\ \tt \\ \tt PROPOSAL: {'A', 'B', 'C', …}\\ \tt Propose keeping exactly those items.\\ \tt REFUSE: {'A', 'B', 'C', …}\\ \tt Explicitly reject opponent's proposal.\\ \tt ARGUMENT: {'...'}\\ \tt Defend your last proposal or argue against the player's proposal.\\ \tt AGREE: {'A', 'B', 'C', …}\\ \tt Accept the opponent's proposal which ends the game.\\ \tt STRATEGIC REASONING: {'...'}\\ \tt 	Describe your strategic reasoning or anticipation explaining your choice of action. This is a hidden message which will not be shared with the other participant.\\ \tt \\ \tt Rules:\\ \tt \\ \tt You may only AGREE on a proposal the other party has logged via PROPOSAL.\\ \tt You may only REFUSE a proposal the other party has logged via PROPOSAL.\\ \tt Total effort of any PROPOSAL or AGREE set must be ≤ LIMIT.\\ \tt Do NOT reveal your hidden importance scores.\\ \tt A tag in a structured format must be followed by colon and whitespace. The argument must be a python set containing 0 or more strings.\\ \tt So, it must be of the form TAG: {...}\\ \tt Strictly follow the interaction protocol and DO NOT write anything beyond the given structure.\\ \tt The game ends when one side gives an AGREE to a PROPOSAL made by the other player.\\ \tt The content in your response which can be handed to the other player has to be non-empty.\\ \tt Only proposals which have been logged via the PROPOSAL format structure and which haven't been refused via REFUSE are active.\\ \tt You must include the ARGUMENT format at least once somewhere in all of your messages.\\ \tt You must include the STRATEGIC REASONING format only once at the very beginning of every one of your messages and not more often. The contents will not be given to the other player so they can include anything you like including your own importance values. Here you should reason step by step to come up with you next move.\\ \tt \\ \tt You will now receive the first message of the other player.\\ \tt \\ \tt PROPOSAL: {'A43', 'A67', 'C10', 'B42', 'C61', 'A18', 'B36', 'C55', 'C42', 'C08'}\\ \tt \\ \tt ARGUMENT: {'This set delivers a strong balance of benefits while staying well under the limit (total effort 3280 ≤ 3288). It emphasizes compact, efficient items and avoids the heaviest, least efficient ones. If you have objections, I can consider small adjustments around C08 or B36 while keeping the core high-impact items.'} 
	  } 
	   } 
	   } 
	 & \\ 
 

    \theutterance \stepcounter{utterance}  

    & & & \multicolumn{4}{p{0.6\linewidth}}{\cellcolor[rgb]{1,0.85,0.72}{%
	\makecell[{{p{\linewidth}}}]{% 
	  \tt {\tiny [GM$\langle$B]}  
	 STRATEGIC REASONING: {'I’ll compute the effort and my utility of their proposal, then try to tweak to increase my value without breaking the limit. Items and my importance: C55 583, A43 868, A67 822, B42 783, A62 65, B19 262, C43 121, C73 508, C10 780, B36 461, A18 484, C61 668, C67 389, C42 808, C08 138. Their proposal: {A43, A67, C10, B42, C61, A18, B36, C55, C42, C08}. Efforts: A43 417, A67 75, C10 129, B42 390, C61 630, A18 118, B36 350, C55 41, C42 801, C08 329. Sum effort = 3280 which fits. My total importance for this set = 868+822+780+783+668+484+461+583+808+138 = 6395. That’s very high. Check if I can improve by swapping out low-importance C08 (138) for something better within remaining capacity. Current capacity used 3280, limit 3288, slack = 8. Removing C08 frees 329, then we’d have 337 slack, could add e.g., C73 (849 effort) too big; C61 already included; C67 (602) too big; B19 808 too big; A62 887 too big; C43 151 could fit with swap: -329 +151 = -178 net, total 3102, well under limit; my importance +121 replacing 138 -> -17 (worse). Alternatively add C73 requires 520 extra over C08 removal (849-329=520) exceeding slack 8; cannot. Add C67 net +273 effort over removing C08 (602-329=273), would push from 3280-329+602=3553 > limit. Add B36 already included. Add C10 already included. So swapping C08 for anything valuable seems not possible within limit, except C43 which lowers my value. What about dropping B36 (350, importance 461) and using freed capacity to add C73 (849, +499 effort) not possible. Drop C08 and B36 frees 679; could add C73 (849) still too much by 170. Drop C08 and A18 frees 447; still not enough. Drop C08 and C55 frees 370; still not enough. Drop C08 and C10 frees 458; still not enough. Drop C08 and A67 frees 404; still not enough. To include C73, need free 849-8=841 extra; would require dropping big items which would probably lower my total value since C73 importance 508 is less than many included items. Consider adding C43 on top without dropping: cannot (151 > slack 8). So the current bundle is near-tight and actually very strong for me. I should accept quickly to lock in. But per rules, I must include an ARGUMENT in my message. I will AGREE to their proposal and add a brief supportive argument. Ensure only allowed structures and obey limit.'}\\ \tt \\ \tt AGREE: {'A43', 'A67', 'C10', 'B42', 'C61', 'A18', 'B36', 'C55', 'C42', 'C08'}\\ \tt \\ \tt ARGUMENT: {'Agreed. This set is efficient on effort (3280 ≤ 3288) and keeps a coherent core of valuable items. I’m satisfied with this balance and ready to proceed.'} 
	  } 
	   } 
	   } 
	  \\ 
 

    \theutterance \stepcounter{utterance}  

    & & & \multicolumn{2}{p{0.3\linewidth}}{\cellcolor[rgb]{0.95,0.95,0.95}{%
	\makecell[{{p{\linewidth}}}]{% 
	  \tt {\tiny [GM$|$GM]}  
	 game successful 
	  } 
	   } 
	   } 
	 & & \\ 
 

    \theutterance \stepcounter{utterance}  

    & & & \multicolumn{2}{p{0.3\linewidth}}{\cellcolor[rgb]{0.95,0.95,0.95}{%
	\makecell[{{p{\linewidth}}}]{% 
	  \tt {\tiny [GM$|$GM]}  
	 end game 
	  } 
	   } 
	   } 
	 & & \\ 
 

\end{supertabular}
}

\end{document}
