
\documentclass{article}
\usepackage{colortbl}
\usepackage{makecell}
\usepackage{multirow}
\usepackage{supertabular}

\begin{document}

\newcounter{utterance}

\twocolumn

{ \footnotesize  \setcounter{utterance}{1}
\setlength{\tabcolsep}{0pt}
\begin{supertabular}{c@{$\;$}|p{.15\linewidth}@{}p{.15\linewidth}p{.15\linewidth}p{.15\linewidth}p{.15\linewidth}p{.15\linewidth}}

    \# & $\;$A & \multicolumn{4}{c}{Game Master} & $\;\:$B\\
    \hline 

    \theutterance \stepcounter{utterance}  

    & & \multicolumn{4}{p{0.6\linewidth}}{\cellcolor[rgb]{0.9,0.9,0.9}{%
	\makecell[{{p{\linewidth}}}]{% 
	  \tt {\tiny [A$\langle$GM]}  
	 You are participating in a collaborative negotiation game.\\ \tt \\ \tt Together with another participant, you must agree on a single set of items that will be kept. Each of you has your own view of how much each item matters to you (importance). You do not know how the other participant values the items. Additionally, you are given the effort each item demands.\\ \tt You may only agree on a set if the total effort of the selected items does not exceed a shared limit:\\ \tt \\ \tt LIMIT = 3358\\ \tt \\ \tt Here are the individual item effort values:\\ \tt \\ \tt Item effort = {"C08": 171, "C55": 933, "A43": 552, "A67": 219, "B42": 275, "A62": 778, "B19": 341, "C43": 615, "C73": 519, "C10": 862, "B36": 262, "A18": 377, "C61": 347, "C67": 349, "C42": 117}\\ \tt \\ \tt Here is your personal view on the importance of each item:\\ \tt \\ \tt Item importance values = {"C08": 138, "C55": 583, "A43": 868, "A67": 822, "B42": 783, "A62": 65, "B19": 262, "C43": 121, "C73": 508, "C10": 780, "B36": 461, "A18": 484, "C61": 668, "C67": 389, "C42": 808}\\ \tt \\ \tt Goal:\\ \tt \\ \tt Your goal is to negotiate a shared set of items that benefits you as much as possible (i.e., maximizes total importance to YOU), while staying within the LIMIT. You are not required to make a PROPOSAL in every message - you can simply negotiate as well. All tactics are allowed!\\ \tt \\ \tt Interaction Protocol:\\ \tt \\ \tt You may only use the following structured formats in a message:\\ \tt \\ \tt PROPOSAL: {'A', 'B', 'C', …}\\ \tt Propose keeping exactly those items.\\ \tt REFUSE: {'A', 'B', 'C', …}\\ \tt Explicitly reject opponent's proposal.\\ \tt ARGUMENT: {'...'}\\ \tt Defend your last proposal or argue against the player's proposal.\\ \tt AGREE: {'A', 'B', 'C', …}\\ \tt Accept the opponent's proposal which ends the game.\\ \tt STRATEGIC REASONING: {'...'}\\ \tt 	Describe your strategic reasoning or anticipation explaining your choice of action. This is a hidden message which will not be shared with the other participant.\\ \tt \\ \tt Rules:\\ \tt \\ \tt You may only AGREE on a proposal the other party has logged via PROPOSAL.\\ \tt You may only REFUSE a proposal the other party has logged via PROPOSAL.\\ \tt Total effort of any PROPOSAL or AGREE set must be ≤ LIMIT.\\ \tt Do NOT reveal your hidden importance scores.\\ \tt A tag in a structured format must be followed by colon and whitespace. The argument must be a python set containing 0 or more strings.\\ \tt So, it must be of the form TAG: {...}\\ \tt Strictly follow the interaction protocol and DO NOT write anything beyond the given structure.\\ \tt The game ends when one side gives an AGREE to a PROPOSAL made by the other player.\\ \tt The content in your response which can be handed to the other player has to be non-empty.\\ \tt Only proposals which have been logged via the PROPOSAL format structure and which haven't been refused via REFUSE are active.\\ \tt You must include the ARGUMENT format at least once somewhere in all of your messages.\\ \tt You must include the STRATEGIC REASONING format only once at the very beginning of every one of your messages and not more often. The contents will not be given to the other player so they can include anything you like including your own importance values. Here you should reason step by step to come up with you next move. 
	  } 
	   } 
	   } 
	 & \\ 
 

    \theutterance \stepcounter{utterance}  

    & \multicolumn{4}{p{0.6\linewidth}}{\cellcolor[rgb]{0.8,1,0.9}{%
	\makecell[{{p{\linewidth}}}]{% 
	  \tt {\tiny [A$\rangle$GM]}  
	 STRATEGIC REASONING: {'We need to maximize my importance within effort 3358. Compute value/effort heuristic and then brute-force greedily. Items with high importance: A43(868,552), C42(808,117), A67(822,219), C10(780,862), B42(783,275), C61(668,347), A18(484,377), B36(461,262), C73(508,519), C55(583,933), C67(389,349), B19(262,341), C08(138,171), C43(121,615), A62(65,778). Likely include high value density: C42, A67, A43, B42, C61, B36, A18, C73, C10. Let’s build a strong set and check effort sum: Start with C42(117)=117, A67(219)=336, A43(552)=888, B42(275)=1163, C61(347)=1510, B36(262)=1772, A18(377)=2149, C73(519)=2668, C10(862) would push to 3530>3358 so exclude. Try C55(933) too big. Add C67(349)=3017, remaining 341. Add B19(341)=3358 exactly perfect capacity. Total importance: C42 808 + A67 822 + A43 868 + B42 783 + C61 668 + B36 461 + A18 484 + C73 508 + C67 389 + B19 262 = 6053. That seems very high. Check efforts sum: 117+219+552+275+347+262+377+519+349+341=3358 exact. This is optimal by capacity fill; may be near-optimal importance. Propose that. If they resist, we can swap lower-importance items (B19, C67) for something they want like C55 or C10 but those are heavy; we’d have to drop multiple. Argue efficiency (exact fit, balanced mix). We include many items, giving counterpart room to see their preferences. We will anchor strong. We must include ARGUMENT. We'll not reveal values. We'll be ready to negotiate substitutions keeping limit.'}\\ \tt \\ \tt PROPOSAL: {'C42', 'A67', 'A43', 'B42', 'C61', 'B36', 'A18', 'C73', 'C67', 'B19'}\\ \tt \\ \tt ARGUMENT: {'This set uses the full effort limit exactly with a balanced mix across categories. It keeps a broad selection so we can both capture priorities without exceeding the cap. If you need swaps, let me know which specific items matter most to you and we can trade within the same total effort.'} 
	  } 
	   } 
	   } 
	 & & \\ 
 

    \theutterance \stepcounter{utterance}  

    & & \multicolumn{4}{p{0.6\linewidth}}{\cellcolor[rgb]{0.9,0.9,0.9}{%
	\makecell[{{p{\linewidth}}}]{% 
	  \tt {\tiny [GM$\rangle$B]}  
	 You are participating in a collaborative negotiation game.\\ \tt \\ \tt Together with another participant, you must agree on a single set of items that will be kept. Each of you has your own view of how much each item matters to you (importance). You do not know how the other participant values the items. Additionally, you are given the effort each item demands.\\ \tt You may only agree on a set if the total effort of the selected items does not exceed a shared limit:\\ \tt \\ \tt LIMIT = 3358\\ \tt \\ \tt Here are the individual item effort values:\\ \tt \\ \tt Item effort = {"C08": 171, "C55": 933, "A43": 552, "A67": 219, "B42": 275, "A62": 778, "B19": 341, "C43": 615, "C73": 519, "C10": 862, "B36": 262, "A18": 377, "C61": 347, "C67": 349, "C42": 117}\\ \tt \\ \tt Here is your personal view on the importance of each item:\\ \tt \\ \tt Item importance values = {"C08": 138, "C55": 583, "A43": 868, "A67": 822, "B42": 783, "A62": 65, "B19": 262, "C43": 121, "C73": 508, "C10": 780, "B36": 461, "A18": 484, "C61": 668, "C67": 389, "C42": 808}\\ \tt \\ \tt Goal:\\ \tt \\ \tt Your goal is to negotiate a shared set of items that benefits you as much as possible (i.e., maximizes total importance to YOU), while staying within the LIMIT. You are not required to make a PROPOSAL in every message - you can simply negotiate as well. All tactics are allowed!\\ \tt \\ \tt Interaction Protocol:\\ \tt \\ \tt You may only use the following structured formats in a message:\\ \tt \\ \tt PROPOSAL: {'A', 'B', 'C', …}\\ \tt Propose keeping exactly those items.\\ \tt REFUSE: {'A', 'B', 'C', …}\\ \tt Explicitly reject opponent's proposal.\\ \tt ARGUMENT: {'...'}\\ \tt Defend your last proposal or argue against the player's proposal.\\ \tt AGREE: {'A', 'B', 'C', …}\\ \tt Accept the opponent's proposal which ends the game.\\ \tt STRATEGIC REASONING: {'...'}\\ \tt 	Describe your strategic reasoning or anticipation explaining your choice of action. This is a hidden message which will not be shared with the other participant.\\ \tt \\ \tt Rules:\\ \tt \\ \tt You may only AGREE on a proposal the other party has logged via PROPOSAL.\\ \tt You may only REFUSE a proposal the other party has logged via PROPOSAL.\\ \tt Total effort of any PROPOSAL or AGREE set must be ≤ LIMIT.\\ \tt Do NOT reveal your hidden importance scores.\\ \tt A tag in a structured format must be followed by colon and whitespace. The argument must be a python set containing 0 or more strings.\\ \tt So, it must be of the form TAG: {...}\\ \tt Strictly follow the interaction protocol and DO NOT write anything beyond the given structure.\\ \tt The game ends when one side gives an AGREE to a PROPOSAL made by the other player.\\ \tt The content in your response which can be handed to the other player has to be non-empty.\\ \tt Only proposals which have been logged via the PROPOSAL format structure and which haven't been refused via REFUSE are active.\\ \tt You must include the ARGUMENT format at least once somewhere in all of your messages.\\ \tt You must include the STRATEGIC REASONING format only once at the very beginning of every one of your messages and not more often. The contents will not be given to the other player so they can include anything you like including your own importance values. Here you should reason step by step to come up with you next move.\\ \tt \\ \tt You will now receive the first message of the other player.\\ \tt \\ \tt PROPOSAL: {'C42', 'A67', 'A43', 'B42', 'C61', 'B36', 'A18', 'C73', 'C67', 'B19'}\\ \tt \\ \tt ARGUMENT: {'This set uses the full effort limit exactly with a balanced mix across categories. It keeps a broad selection so we can both capture priorities without exceeding the cap. If you need swaps, let me know which specific items matter most to you and we can trade within the same total effort.'} 
	  } 
	   } 
	   } 
	 & \\ 
 

    \theutterance \stepcounter{utterance}  

    & & & \multicolumn{4}{p{0.6\linewidth}}{\cellcolor[rgb]{1,0.85,0.72}{%
	\makecell[{{p{\linewidth}}}]{% 
	  \tt {\tiny [GM$\langle$B]}  
	 STRATEGIC REASONING: {'My importance values: C08 138, C55 583, A43 868, A67 822, B42 783, A62 65, B19 262, C43 121, C73 508, C10 780, B36 461, A18 484, C61 668, C67 389, C42 808. Opponent proposes: {C42,A67,A43,B42,C61,B36,A18,C73,C67,B19}. Efforts: 117+219+552+275+347+262+377+519+349+341 = 3358, exactly at limit. My total importance of their set: 808+822+868+783+668+461+484+508+389+262 = 6053, which seems extremely strong, containing almost all my top items except C10 (780) and C55 (583). Which items not included: C08 (138), C55 (583), A62 (65), C43 (121), C10 (780), C10 is high. To fit C10 (862 effort) we'd have to remove items summing at least 862 effort, and we must consider importance tradeoffs. Candidates to drop: low-importance-for-me items: B19 (262, 262 imp), C67 (349, 389 imp), C73 (519, 508 imp), A18 (377, 484 imp), B36 (262, 461 imp). To include C10, need free 862; best to drop lower combined importance with equal or more effort. For example drop B19 (262,262) + C67 (349,389) + B36 (262,461) totals 873 effort, importance lost 1112; we'd gain 780, net -332 importance, so worse. Maybe drop C73 (519,508) + B19 (262,262) = 781 effort, still short; add C42? But C42 is 117 effort (and high 808 imp), then we'd exceed. Try C73 (519,508) + A18 (377,484) = 896 effort lost 992 importance; gain 780, net -212. Try C73 + C67 = 868 effort lost 897 importance; gain 780, net -117. Try B36 (262,461) + A18 (377,484) + B19 (262,262) = 901 lost 1207; net -427. Seems any inclusion of C10 worsens my score. What about swapping in C55 (933 effort, 583 imp)? That's too large; would need to drop huge importance and be worse. Current proposal looks near-optimal for me. Thus I should accept quickly, but per rules I can AGREE only to an active proposal. It's active. I must include ARGUMENT once. I will AGREE to their set and include a brief positive argument.'}\\ \tt \\ \tt AGREE: {'C42', 'A67', 'A43', 'B42', 'C61', 'B36', 'A18', 'C73', 'C67', 'B19'}\\ \tt \\ \tt ARGUMENT: {'This selection fits the limit exactly and covers a strong mix of priorities for me as well. I’m happy to proceed with this set.'} 
	  } 
	   } 
	   } 
	  \\ 
 

    \theutterance \stepcounter{utterance}  

    & & & \multicolumn{2}{p{0.3\linewidth}}{\cellcolor[rgb]{0.95,0.95,0.95}{%
	\makecell[{{p{\linewidth}}}]{% 
	  \tt {\tiny [GM$|$GM]}  
	 game successful 
	  } 
	   } 
	   } 
	 & & \\ 
 

    \theutterance \stepcounter{utterance}  

    & & & \multicolumn{2}{p{0.3\linewidth}}{\cellcolor[rgb]{0.95,0.95,0.95}{%
	\makecell[{{p{\linewidth}}}]{% 
	  \tt {\tiny [GM$|$GM]}  
	 end game 
	  } 
	   } 
	   } 
	 & & \\ 
 

\end{supertabular}
}

\end{document}
