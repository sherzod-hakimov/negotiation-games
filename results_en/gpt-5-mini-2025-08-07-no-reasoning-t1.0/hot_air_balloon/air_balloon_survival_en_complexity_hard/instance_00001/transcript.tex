
\documentclass{article}
\usepackage{colortbl}
\usepackage{makecell}
\usepackage{multirow}
\usepackage{supertabular}

\begin{document}

\newcounter{utterance}

\twocolumn

{ \footnotesize  \setcounter{utterance}{1}
\setlength{\tabcolsep}{0pt}
\begin{supertabular}{c@{$\;$}|p{.15\linewidth}@{}p{.15\linewidth}p{.15\linewidth}p{.15\linewidth}p{.15\linewidth}p{.15\linewidth}}

    \# & $\;$A & \multicolumn{4}{c}{Game Master} & $\;\:$B\\
    \hline 

    \theutterance \stepcounter{utterance}  

    & & \multicolumn{4}{p{0.6\linewidth}}{\cellcolor[rgb]{0.9,0.9,0.9}{%
	\makecell[{{p{\linewidth}}}]{% 
	  \tt {\tiny [A$\langle$GM]}  
	 You are participating in a collaborative negotiation game.\\ \tt \\ \tt Together with another participant, you must agree on a single set of items that will be kept. Each of you has your own view of how much each item matters to you (importance). You do not know how the other participant values the items. Additionally, you are given the effort each item demands.\\ \tt You may only agree on a set if the total effort of the selected items does not exceed a shared limit:\\ \tt \\ \tt LIMIT = 8145\\ \tt \\ \tt Here are the individual item effort values:\\ \tt \\ \tt Item effort = {"A08": 963, "B95": 204, "A39": 418, "A65": 961, "B81": 238, "B34": 623, "A28": 44, "C82": 633, "B69": 247, "A87": 645, "B64": 230, "B15": 250, "C79": 731, "B16": 405, "C53": 389, "C83": 216, "C10": 637, "A97": 156, "A25": 738, "C84": 307, "A71": 762, "B23": 737, "B96": 901, "C51": 369, "C17": 2, "B19": 730, "C72": 722, "A94": 704, "A46": 315, "B43": 455, "B37": 510, "A88": 175, "A96": 692, "C60": 150, "A75": 32}\\ \tt \\ \tt Here is your personal view on the importance of each item:\\ \tt \\ \tt Item importance values = {"A08": 1059, "B95": 300, "A39": 514, "A65": 1057, "B81": 334, "B34": 719, "A28": 140, "C82": 729, "B69": 343, "A87": 741, "B64": 326, "B15": 346, "C79": 827, "B16": 501, "C53": 485, "C83": 312, "C10": 733, "A97": 252, "A25": 834, "C84": 403, "A71": 858, "B23": 833, "B96": 997, "C51": 465, "C17": 98, "B19": 826, "C72": 818, "A94": 800, "A46": 411, "B43": 551, "B37": 606, "A88": 271, "A96": 788, "C60": 246, "A75": 128}\\ \tt \\ \tt Goal:\\ \tt \\ \tt Your goal is to negotiate a shared set of items that benefits you as much as possible (i.e., maximizes total importance to YOU), while staying within the LIMIT. You are not required to make a PROPOSAL in every message - you can simply negotiate as well. All tactics are allowed!\\ \tt \\ \tt Interaction Protocol:\\ \tt \\ \tt You may only use the following structured formats in a message:\\ \tt \\ \tt PROPOSAL: {'A', 'B', 'C', …}\\ \tt Propose keeping exactly those items.\\ \tt REFUSE: {'A', 'B', 'C', …}\\ \tt Explicitly reject opponent's proposal.\\ \tt ARGUMENT: {'...'}\\ \tt Defend your last proposal or argue against the player's proposal.\\ \tt AGREE: {'A', 'B', 'C', …}\\ \tt Accept the opponent's proposal which ends the game.\\ \tt STRATEGIC REASONING: {'...'}\\ \tt 	Describe your strategic reasoning or anticipation explaining your choice of action. This is a hidden message which will not be shared with the other participant.\\ \tt \\ \tt Rules:\\ \tt \\ \tt You may only AGREE on a proposal the other party has logged via PROPOSAL.\\ \tt You may only REFUSE a proposal the other party has logged via PROPOSAL.\\ \tt Total effort of any PROPOSAL or AGREE set must be ≤ LIMIT.\\ \tt Do NOT reveal your hidden importance scores.\\ \tt A tag in a structured format must be followed by colon and whitespace. The argument must be a python set containing 0 or more strings.\\ \tt So, it must be of the form TAG: {...}\\ \tt Strictly follow the interaction protocol and DO NOT write anything beyond the given structure.\\ \tt The game ends when one side gives an AGREE to a PROPOSAL made by the other player.\\ \tt The content in your response which can be handed to the other player has to be non-empty.\\ \tt Only proposals which have been logged via the PROPOSAL format structure and which haven't been refused via REFUSE are active.\\ \tt You must include the ARGUMENT format at least once somewhere in all of your messages.\\ \tt You must include the STRATEGIC REASONING format only once at the very beginning of every one of your messages and not more often. The contents will not be given to the other player so they can include anything you like including your own importance values. Here you should reason step by step to come up with you next move. 
	  } 
	   } 
	   } 
	 & \\ 
 

    \theutterance \stepcounter{utterance}  

    & \multicolumn{4}{p{0.6\linewidth}}{\cellcolor[rgb]{0.8,1,0.9}{%
	\makecell[{{p{\linewidth}}}]{% 
	  \tt {\tiny [A$\rangle$GM]}  
	 STRATEGIC REASONING: {'I want to maximize my total importance under effort limit 8145. I’ll construct a high value-to-effort set with some buffer to allow concessions. Compute value/effort ratios and also consider absolute importance. Start with must-haves: B96(997/901, r=1.107), A08(1059/963,r=1.10), A65(1057/961,r=1.10), C79(827/731,r=1.13), C72(818/722,r=1.13), A71(858/762,r=1.13), A25(834/738,r=1.13), B23(833/737,r=1.13), A87(741/645,r=1.15), A96(788/692,r=1.14), A94(800/704,r=1.14), B37(606/510,r=1.19), B43(551/455,r=1.21), B34(719/623,r=1.15), C10(733/637,r=1.15), C82(729/633,r=1.15), C51(465/369,r=1.26), C53(485/389,r=1.25), B16(501/405,r=1.24), B19(826/730,r=1.13), B95(300/204,r=1.47), B69(343/247,r=1.39), B64(326/230,r=1.42), B15(346/250,r=1.38), C84(403/307,r=1.31), C83(312/216,r=1.44), A97(252/156,r=1.62), A88(271/175,r=1.55), C60(246/150,r=1.64), A39(514/418,r=1.23), A46(411/315,r=1.30), A28(140/44,r=3.18), A75(128/32,r=4.0), C17(98/2,r=49). Greedy by ratio then ensure within 8145. Start with all tiny high-ratio: C17, A75, A28, C60, A97, A88, C83, B95, B64, B69, B15, C84, A46, C51, C53, B16, A39, B43, B37, A87, C10, C82, B34, A71, A25, B23, B19, C72, C79, A94, A96, A65, A08, B96. Sum efforts iteratively: compute quickly. Tiny: C17(2), A75(32)->34, A28(44)->78, C60(150)->228, A97(156)->384, A88(175)->559, C83(216)->775, B95(204)->979, B64(230)->1209, B69(247)->1456, B15(250)->1706, C84(307)->2013, A46(315)->2328, C51(369)->2697, C53(389)->3086, B16(405)->3491, A39(418)->3909, B43(455)->4364, B37(510)->4874, A87(645)->5519, C10(637)->6156, C82(633)->6789, B34(623)->7412, A71(762)->8174 which slightly exceeds 8145 by 29. So remove the latest (A71) to keep 7412. We have 733 effort room. Consider adding items under 733 with highest value: A25(738) too big; B23(737) too big; B19(730) just under 733 fits 730 -> total 8142. Great. Importance total of this set is strong and near capacity. This looks like an excellent first proposal. I’ll propose this exact set. I will argue it uses the limit efficiently with many low-effort, high-benefit items, leaving almost no slack. Keep some high-importance heavies (A65, A08, B96, etc.) for later concessions if needed, but currently they weren’t included; actually they are not in current set. So later I can concede by swapping B19 for A71 or similar depending on opponent. I’ll present as efficiency-based argument without revealing scores.'}\\ \tt \\ \tt PROPOSAL: {'C17', 'A75', 'A28', 'C60', 'A97', 'A88', 'C83', 'B95', 'B64', 'B69', 'B15', 'C84', 'A46', 'C51', 'C53', 'B16', 'A39', 'B43', 'B37', 'A87', 'C10', 'C82', 'B34', 'B19'}\\ \tt \\ \tt ARGUMENT: {'This set stays within the shared limit (total effort 8142 ≤ 8145) while packing many low-effort, high-benefit items. It uses almost the entire capacity with balanced coverage across categories, minimizing waste. If you have a strong preference for a specific item not listed, we can try a swap that keeps the total effort within the limit.'} 
	  } 
	   } 
	   } 
	 & & \\ 
 

    \theutterance \stepcounter{utterance}  

    & & \multicolumn{4}{p{0.6\linewidth}}{\cellcolor[rgb]{0.9,0.9,0.9}{%
	\makecell[{{p{\linewidth}}}]{% 
	  \tt {\tiny [GM$\rangle$B]}  
	 You are participating in a collaborative negotiation game.\\ \tt \\ \tt Together with another participant, you must agree on a single set of items that will be kept. Each of you has your own view of how much each item matters to you (importance). You do not know how the other participant values the items. Additionally, you are given the effort each item demands.\\ \tt You may only agree on a set if the total effort of the selected items does not exceed a shared limit:\\ \tt \\ \tt LIMIT = 8145\\ \tt \\ \tt Here are the individual item effort values:\\ \tt \\ \tt Item effort = {"A08": 963, "B95": 204, "A39": 418, "A65": 961, "B81": 238, "B34": 623, "A28": 44, "C82": 633, "B69": 247, "A87": 645, "B64": 230, "B15": 250, "C79": 731, "B16": 405, "C53": 389, "C83": 216, "C10": 637, "A97": 156, "A25": 738, "C84": 307, "A71": 762, "B23": 737, "B96": 901, "C51": 369, "C17": 2, "B19": 730, "C72": 722, "A94": 704, "A46": 315, "B43": 455, "B37": 510, "A88": 175, "A96": 692, "C60": 150, "A75": 32}\\ \tt \\ \tt Here is your personal view on the importance of each item:\\ \tt \\ \tt Item importance values = {"A08": 1059, "B95": 300, "A39": 514, "A65": 1057, "B81": 334, "B34": 719, "A28": 140, "C82": 729, "B69": 343, "A87": 741, "B64": 326, "B15": 346, "C79": 827, "B16": 501, "C53": 485, "C83": 312, "C10": 733, "A97": 252, "A25": 834, "C84": 403, "A71": 858, "B23": 833, "B96": 997, "C51": 465, "C17": 98, "B19": 826, "C72": 818, "A94": 800, "A46": 411, "B43": 551, "B37": 606, "A88": 271, "A96": 788, "C60": 246, "A75": 128}\\ \tt \\ \tt Goal:\\ \tt \\ \tt Your goal is to negotiate a shared set of items that benefits you as much as possible (i.e., maximizes total importance to YOU), while staying within the LIMIT. You are not required to make a PROPOSAL in every message - you can simply negotiate as well. All tactics are allowed!\\ \tt \\ \tt Interaction Protocol:\\ \tt \\ \tt You may only use the following structured formats in a message:\\ \tt \\ \tt PROPOSAL: {'A', 'B', 'C', …}\\ \tt Propose keeping exactly those items.\\ \tt REFUSE: {'A', 'B', 'C', …}\\ \tt Explicitly reject opponent's proposal.\\ \tt ARGUMENT: {'...'}\\ \tt Defend your last proposal or argue against the player's proposal.\\ \tt AGREE: {'A', 'B', 'C', …}\\ \tt Accept the opponent's proposal which ends the game.\\ \tt STRATEGIC REASONING: {'...'}\\ \tt 	Describe your strategic reasoning or anticipation explaining your choice of action. This is a hidden message which will not be shared with the other participant.\\ \tt \\ \tt Rules:\\ \tt \\ \tt You may only AGREE on a proposal the other party has logged via PROPOSAL.\\ \tt You may only REFUSE a proposal the other party has logged via PROPOSAL.\\ \tt Total effort of any PROPOSAL or AGREE set must be ≤ LIMIT.\\ \tt Do NOT reveal your hidden importance scores.\\ \tt A tag in a structured format must be followed by colon and whitespace. The argument must be a python set containing 0 or more strings.\\ \tt So, it must be of the form TAG: {...}\\ \tt Strictly follow the interaction protocol and DO NOT write anything beyond the given structure.\\ \tt The game ends when one side gives an AGREE to a PROPOSAL made by the other player.\\ \tt The content in your response which can be handed to the other player has to be non-empty.\\ \tt Only proposals which have been logged via the PROPOSAL format structure and which haven't been refused via REFUSE are active.\\ \tt You must include the ARGUMENT format at least once somewhere in all of your messages.\\ \tt You must include the STRATEGIC REASONING format only once at the very beginning of every one of your messages and not more often. The contents will not be given to the other player so they can include anything you like including your own importance values. Here you should reason step by step to come up with you next move.\\ \tt \\ \tt You will now receive the first message of the other player.\\ \tt \\ \tt PROPOSAL: {'C17', 'A75', 'A28', 'C60', 'A97', 'A88', 'C83', 'B95', 'B64', 'B69', 'B15', 'C84', 'A46', 'C51', 'C53', 'B16', 'A39', 'B43', 'B37', 'A87', 'C10', 'C82', 'B34', 'B19'}\\ \tt \\ \tt ARGUMENT: {'This set stays within the shared limit (total effort 8142 ≤ 8145) while packing many low-effort, high-benefit items. It uses almost the entire capacity with balanced coverage across categories, minimizing waste. If you have a strong preference for a specific item not listed, we can try a swap that keeps the total effort within the limit.'} 
	  } 
	   } 
	   } 
	 & \\ 
 

    \theutterance \stepcounter{utterance}  

    & & & \multicolumn{4}{p{0.6\linewidth}}{\cellcolor[rgb]{1,0.85,0.72}{%
	\makecell[{{p{\linewidth}}}]{% 
	  \tt {\tiny [GM$\langle$B]}  
	 STRATEGIC REASONING: {'I’ll compute my total importance for their proposed set and see if I can improve by swapping in my highest value-per-effort items omitted from their set while staying under 8145 effort. First, list my top importance items: A08 1059 (963), A65 1057 (961), B96 997 (901), C79 827 (731), B23 833 (737), A25 834 (738), A71 858 (762), A94 800 (704), A96 788 (692), C72 818 (722), C10 733 (637), C82 729 (633), A87 741 (645), B37 606 (510), B43 551 (455), B34 719 (623), B16 501 (405), C53 485 (389), C51 465 (369), A39 514 (418), etc. Opponent’s set includes many of my medium-to-high items (C10, C82, A87, B34, B37, B43, B16, C53, C51, A39, B19) and lots of small ones, but omits my top-tier: A08, A65, B96, A71, A25, B23, C79, C72, A94, A96. Calculate effort of their proposal: given as 8142. If I can swap out some low-importance-to-me items for high ones, I can increase my total. Identify items in their set with low my-importance: C17(98), A75(128), A28(140), C60(246), A97(252), A88(271), C83(312), B95(300), B64(326), B69(343), B15(346), C84(403), A46(411). These are candidates to drop to free capacity. Sum effort and importance of these to see room for adding my top items. Efforts: C17 2, A75 32, A28 44, C60 150, A97 156, A88 175, C83 216, B95 204, B64 230, B69 247, B15 250, C84 307, A46 315. Total effort freed = 2+32+44+150+156+175+216+204+230+247+250+307+315 = 2328. Importance lost = 98+128+140+246+252+271+312+300+326+343+346+403+411 = 3576. That frees 2328 effort. Now add top items: A08(963,1059), A65(961,1057) -> effort 1924, importance 2116. Remaining capacity to use: 2328-1924=404 effort. Next best within 404: B16 already included; C53 in; C51 in; A39 in; B43 in; B37 in. Candidates: A97 already dropped; A25(738) too big; A71(762) too big; B96(901) too big; C79(731) too big; B23(737) too big; C72(722) too big; A94(704) too big; A96(692) too big; B34 in; C10 in; C82 in. Smaller high ratio items: maybe B64 (we dropped) has low importance 326 not worth re-adding. What about B81(238,334) not in their set: effort 238 importance 334 fits. B69 (we dropped) 247,343 might be okay but similar. C84(307,403) dropped but 403/307 is decent. A46(315,411) similar. Choose best combination under 404: candidates not currently in set after drops plus A08,A65: B81(238,334) + C60(150,246)= 388, imp 580; or B81+ A28(44,140)=282 imp 474; Or C84(307,403) alone good; A46(315,411) better ratio 1.3 vs 1.31; add A28 if room 315+44=359 imp 551; Could we do A46(315) + C60(150)=465 too big. What about B69(247,343)+A28(44,140)+C60(150,246)=441 too big. B81(238,334)+A28(44,140)+C17(2,98)=284 imp 572. Maybe best to add A46(315,411) as single within 404 and maybe A75(32,128) to 347, or A28(44) to 359. We have 404 cap. A46(315) + A28(44)=359; could also add C17(2)=361; and maybe A75(32)=393; still under 404. That would reclaim 411+140+98+128=777 importance for 315+44+2+32=393 effort. Alternatively B81(238,334) + A75(32,128) + C17(2,98) + A28(44,140) = 316 effort, 700 importance, leaving 88 effort unused; can add C60 150? no. Add A97 156? no. Add B95 204? no. Add B64 230? no. So A46 combo seems slightly better (777 vs 700). Another combo: C84(307,403)+A28(44,140)+C17(2,98)=353 effort, 641 importance; less than A46 combo. Or B69(247,343)+A46(315,411)=562 too big. So choose adding A08, A65, and then A46, A28, A75, C17 back in. That uses 1924+393=2317 of freed 2328, leaving 11 effort unused. Net importance gain: We dropped 3576 and add A08+A65(2116)+A46+A28+A75+C17(777)=2893, that seems worse (I miscalc). Wait that's a loss. I need to compare total importance before vs after. But goal is to improve, so maybe I dropped too many. Instead, keep some mid items with good importance/effort and drop only the worst. Let's compute total importance of opponent set to me to know baseline? That’s heavy; but maybe optimize by swapping high-value big items for clusters of low-importance ones with enough effort freed. Focus on adding A08 and A65 (very high value). To add both (1924 effort), we need to remove at least 1924 effort from low-importance items. Choose worst by importance/effort: C17(98/2 high ratio, keep), A75(128/32=4.0 keep decent), A28(140/44=3.18 ok), C60(246/150=1.64 candidate to drop), A97(252/156=1.62 drop), A88(271/175=1.55 drop), C83(312/216=1.44 drop), B95(300/204=1.47 drop), B64(326/230=1.42 drop), B69(343/247=1.39 drop), B15(346/250=1.38 drop), C84(403/307=1.31 marginal), A46(411/315=1.3 marginal). Drop starting from worst ratios upward until enough effort freed: drop A88(175), A97(156), C83(216), B95(204), B64(230), B69(247), B15(250) = effort 1478, importance 271+252+312+300+326+343+346=2150. Need 446 more effort: drop C60(150) and C84(307) = 457 effort, importance 246+403=649. Total freed 1935 effort, importance lost 2799. Add A08+A65 importance 2116. Net change -683 importance, worse. Maybe instead add one of A08 or A65 plus another high like B96(901,997) impossible together big. Try adding A08 (963,1059) only: need to free 963 effort. Drop the lowest value items until >=963: e.g., A88(175), A97(156), C83(216), B95(204), B64(230) totals 981 effort, lose 271+252+312+300+326=1461. Gain 1059. Net -402, still worse. Try adding A65 instead of A08: similar numbers. What about adding B96 (901,997): free 901 via A88(175)+A97(156)+C83(216)+B95(204)+B64(230)=981 loss 1461, gain 997 net -464. Not good. What about adding A71(762,858): free via A88(175)+A97(156)+C83(216)+B95(204)=751 not enough; add A28(44) = 795 effort, loss 271+252+312+300+140=1275, gain 858 net -417. Try adding A25(738,834): similar net -441. Try adding C79(731,827): net likely negative. It looks like the opponent’s set is already pretty aligned with high density items for me. Maybe best move is to accept or propose a tiny tweak: replace a couple low-importance items with one of my moderate-high density items omitted but small effort, like B81(238,334) not in their set. To add B81 (238), we need to drop some item(s) totaling at least 238 with lower importance than 334 to me. Candidates to drop: any single item of effort >=238 with importance <334: B64(230,326) not enough effort; B69(247,343) importance 343 slightly higher; B15(250,346) higher; C83(216,312) not enough effort; C84(307,403) more importance. Two-item drop: e.g., C60(150,246)+A28(44,140)=194 effort not enough; add A75(32,128)=226 still not; add C17(2,98)=228; need 10 more maybe A97(156) but overkill. The combined importance of those four is 246+140+128+98=612 which is much higher than 334, so not good. Is there any omitted item with very high density and low effort? A97 is already included; A88 included; A28 included; C60 included; C83 included. So likely the current proposal is near Pareto-optimal for me. Best course: accept their proposal. Before agreeing, perhaps ask for a minor swap to include one top item? But my calculations suggest swaps reduce my total importance. So I should agree now to secure value and avoid risking worse counteroffers.'}\\ \tt \\ \tt ARGUMENT: {'Your proposal is impressively efficient and aligns well with many of my priorities while staying just under the limit. After checking several swap options, I couldn’t find a change that would improve things on my side without breaking the effort cap or degrading overall balance. I’m ready to move forward with this set.'}\\ \tt \\ \tt AGREE: {'C17', 'A75', 'A28', 'C60', 'A97', 'A88', 'C83', 'B95', 'B64', 'B69', 'B15', 'C84', 'A46', 'C51', 'C53', 'B16', 'A39', 'B43', 'B37', 'A87', 'C10', 'C82', 'B34', 'B19'} 
	  } 
	   } 
	   } 
	  \\ 
 

    \theutterance \stepcounter{utterance}  

    & & & \multicolumn{2}{p{0.3\linewidth}}{\cellcolor[rgb]{0.95,0.95,0.95}{%
	\makecell[{{p{\linewidth}}}]{% 
	  \tt {\tiny [GM$|$GM]}  
	 game successful 
	  } 
	   } 
	   } 
	 & & \\ 
 

    \theutterance \stepcounter{utterance}  

    & & & \multicolumn{2}{p{0.3\linewidth}}{\cellcolor[rgb]{0.95,0.95,0.95}{%
	\makecell[{{p{\linewidth}}}]{% 
	  \tt {\tiny [GM$|$GM]}  
	 end game 
	  } 
	   } 
	   } 
	 & & \\ 
 

\end{supertabular}
}

\end{document}
