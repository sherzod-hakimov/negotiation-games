
\documentclass{article}
\usepackage{colortbl}
\usepackage{makecell}
\usepackage{multirow}
\usepackage{supertabular}

\begin{document}

\newcounter{utterance}

\twocolumn

{ \footnotesize  \setcounter{utterance}{1}
\setlength{\tabcolsep}{0pt}
\begin{supertabular}{c@{$\;$}|p{.15\linewidth}@{}p{.15\linewidth}p{.15\linewidth}p{.15\linewidth}p{.15\linewidth}p{.15\linewidth}}

    \# & $\;$A & \multicolumn{4}{c}{Game Master} & $\;\:$B\\
    \hline 

    \theutterance \stepcounter{utterance}  

    & & \multicolumn{4}{p{0.6\linewidth}}{\cellcolor[rgb]{0.9,0.9,0.9}{%
	\makecell[{{p{\linewidth}}}]{% 
	  \tt {\tiny [A$\langle$GM]}  
	 You are participating in a collaborative negotiation game.\\ \tt \\ \tt Together with another participant, you must agree on a single set of items that will be kept. Each of you has your own view of how much each item matters to you (importance). You do not know how the other participant values the items. Additionally, you are given the effort each item demands.\\ \tt You may only agree on a set if the total effort of the selected items does not exceed a shared limit:\\ \tt \\ \tt LIMIT = 6938\\ \tt \\ \tt Here are the individual item effort values:\\ \tt \\ \tt Item effort = {"A08": 997, "B95": 950, "A39": 272, "A65": 448, "B81": 382, "B34": 804, "A28": 237, "C82": 56, "B69": 105, "A87": 612, "B64": 528, "B15": 527, "C79": 524, "B16": 167, "C53": 133, "C83": 300, "C10": 945, "A97": 50, "A25": 906, "C84": 70, "A71": 224, "B23": 4, "B96": 689, "C51": 64, "C17": 434, "B19": 750, "C72": 733, "A94": 873, "A46": 22, "B43": 68, "B37": 57, "A88": 10, "A96": 36, "C60": 551, "A75": 348}\\ \tt \\ \tt Here is your personal view on the importance of each item:\\ \tt \\ \tt Item importance values = {"A08": 1096, "B95": 1049, "A39": 371, "A65": 547, "B81": 481, "B34": 903, "A28": 336, "C82": 155, "B69": 204, "A87": 711, "B64": 627, "B15": 626, "C79": 623, "B16": 266, "C53": 232, "C83": 399, "C10": 1044, "A97": 149, "A25": 1005, "C84": 169, "A71": 323, "B23": 103, "B96": 788, "C51": 163, "C17": 533, "B19": 849, "C72": 832, "A94": 972, "A46": 121, "B43": 167, "B37": 156, "A88": 109, "A96": 135, "C60": 650, "A75": 447}\\ \tt \\ \tt Goal:\\ \tt \\ \tt Your goal is to negotiate a shared set of items that benefits you as much as possible (i.e., maximizes total importance to YOU), while staying within the LIMIT. You are not required to make a PROPOSAL in every message - you can simply negotiate as well. All tactics are allowed!\\ \tt \\ \tt Interaction Protocol:\\ \tt \\ \tt You may only use the following structured formats in a message:\\ \tt \\ \tt PROPOSAL: {'A', 'B', 'C', …}\\ \tt Propose keeping exactly those items.\\ \tt REFUSE: {'A', 'B', 'C', …}\\ \tt Explicitly reject opponent's proposal.\\ \tt ARGUMENT: {'...'}\\ \tt Defend your last proposal or argue against the player's proposal.\\ \tt AGREE: {'A', 'B', 'C', …}\\ \tt Accept the opponent's proposal which ends the game.\\ \tt STRATEGIC REASONING: {'...'}\\ \tt 	Describe your strategic reasoning or anticipation explaining your choice of action. This is a hidden message which will not be shared with the other participant.\\ \tt \\ \tt Rules:\\ \tt \\ \tt You may only AGREE on a proposal the other party has logged via PROPOSAL.\\ \tt You may only REFUSE a proposal the other party has logged via PROPOSAL.\\ \tt Total effort of any PROPOSAL or AGREE set must be ≤ LIMIT.\\ \tt Do NOT reveal your hidden importance scores.\\ \tt A tag in a structured format must be followed by colon and whitespace. The argument must be a python set containing 0 or more strings.\\ \tt So, it must be of the form TAG: {...}\\ \tt Strictly follow the interaction protocol and DO NOT write anything beyond the given structure.\\ \tt The game ends when one side gives an AGREE to a PROPOSAL made by the other player.\\ \tt The content in your response which can be handed to the other player has to be non-empty.\\ \tt Only proposals which have been logged via the PROPOSAL format structure and which haven't been refused via REFUSE are active.\\ \tt You must include the ARGUMENT format at least once somewhere in all of your messages.\\ \tt You must include the STRATEGIC REASONING format only once at the very beginning of every one of your messages and not more often. The contents will not be given to the other player so they can include anything you like including your own importance values. Here you should reason step by step to come up with you next move. 
	  } 
	   } 
	   } 
	 & \\ 
 

    \theutterance \stepcounter{utterance}  

    & \multicolumn{4}{p{0.6\linewidth}}{\cellcolor[rgb]{0.8,1,0.9}{%
	\makecell[{{p{\linewidth}}}]{% 
	  \tt {\tiny [A$\rangle$GM]}  
	 STRATEGIC REASONING: {'Compute my value/effort ratios to maximize my utility under 6938. Ratios: C51 2.55, B23 25.75 (tiny), A46 5.5, A88 10.9, A96 3.75, A97 2.98, C84 2.41, C82 2.77, B37 2.74, B43 2.46, B69 1.94, C53 1.74, A39 1.36, A71 1.44, A28 1.42, C83 1.33, B16 1.59, A75 1.28, C17 1.23, A65 1.22, C60 1.18, B64 1.19, B15 1.19, C79 1.19, B34 1.12, B96 1.14, A87 1.16, A25 1.11, B19 1.13, C72 1.13, C10 1.11, A94 1.11, B95 1.10, A08 1.10. However also consider absolute values and typical bargaining: keep many high-importance items even if moderate ratios (C10, A25, A94). Start with a proposal that is within limit and very favorable to me, leaving room to concede. Build a package of high-value with good ratios and low efforts: include B23, A88, A46, A97, C51, C84, C82, B37, B43, B69 for filler. Add my top big items: C10 (1044/945), B95 (1049/950), A25 (1005/906), A94 (972/873), B34 (903/804), B19 (849/750), C72 (832/733), A87 (711/612), C60 (650/551), B64 (627/528), B15 (626/527), C79 (623/524). Check effort sum iteratively to stay ≤6938. Build a near-cap knapsack. Let’s assemble: Start with core high-value big ones: C10(945), B95(950), A25(906), A94(873), B34(804) total 4478. Add B19(750)=5228, C72(733)=5961, now room 977. Add A87(612)=6573, room 365. Use small high-ratio fillers: B23(4)=6577, A88(10)=6587, A46(22)=6609, C51(64)=6673, C84(70)=6743, C82(56)=6799, B37(57)=6856, B43(68)=6924, remaining 14. Could add A96(36) no, A97(50) no, B69(105) no. This set totals 6924 effort and yields importance sum: C10 1044 + B95 1049 + A25 1005 + A94 972 + B34 903 + B19 849 + C72 832 + A87 711 + B23 103 + A88 109 + A46 121 + C51 163 + C84 169 + C82 155 + B37 156 + B43 167 = compute: 1044+1049=2093; +1005=3098; +972=4070; +903=4973; +849=5822; +832=6654; +711=7365; +103=7468; +109=7577; +121=7698; +163=7861; +169=8030; +155=8185; +156=8341; +167=8508. Very strong. I’ll propose this as my opening. Argue efficiency and being near limit. Ready to concede swapping one or two big items if they push for their priorities, but protect C10, B95, A25, A94 as anchors. Also mention we’re at 6924/6938 so very tight. Keep ARGUMENT included.'}\\ \tt \\ \tt PROPOSAL: {'C10', 'B95', 'A25', 'A94', 'B34', 'B19', 'C72', 'A87', 'B23', 'A88', 'A46', 'C51', 'C84', 'C82', 'B37', 'B43'}\\ \tt \\ \tt ARGUMENT: {'This package is highly efficient and fits tightly under the effort limit (total effort 6924 ≤ 6938). It combines several high-impact items with low-effort add-ons to maximize overall value while keeping room minimal. Given how balanced it is and how close it is to the cap, it’s a strong baseline to align on. If you need to prioritize a specific item, I’m open to swapping within the small remaining margin by trading one of the smaller add-ons.'} 
	  } 
	   } 
	   } 
	 & & \\ 
 

    \theutterance \stepcounter{utterance}  

    & & \multicolumn{4}{p{0.6\linewidth}}{\cellcolor[rgb]{0.9,0.9,0.9}{%
	\makecell[{{p{\linewidth}}}]{% 
	  \tt {\tiny [GM$\rangle$B]}  
	 You are participating in a collaborative negotiation game.\\ \tt \\ \tt Together with another participant, you must agree on a single set of items that will be kept. Each of you has your own view of how much each item matters to you (importance). You do not know how the other participant values the items. Additionally, you are given the effort each item demands.\\ \tt You may only agree on a set if the total effort of the selected items does not exceed a shared limit:\\ \tt \\ \tt LIMIT = 6938\\ \tt \\ \tt Here are the individual item effort values:\\ \tt \\ \tt Item effort = {"A08": 997, "B95": 950, "A39": 272, "A65": 448, "B81": 382, "B34": 804, "A28": 237, "C82": 56, "B69": 105, "A87": 612, "B64": 528, "B15": 527, "C79": 524, "B16": 167, "C53": 133, "C83": 300, "C10": 945, "A97": 50, "A25": 906, "C84": 70, "A71": 224, "B23": 4, "B96": 689, "C51": 64, "C17": 434, "B19": 750, "C72": 733, "A94": 873, "A46": 22, "B43": 68, "B37": 57, "A88": 10, "A96": 36, "C60": 551, "A75": 348}\\ \tt \\ \tt Here is your personal view on the importance of each item:\\ \tt \\ \tt Item importance values = {"A08": 1096, "B95": 1049, "A39": 371, "A65": 547, "B81": 481, "B34": 903, "A28": 336, "C82": 155, "B69": 204, "A87": 711, "B64": 627, "B15": 626, "C79": 623, "B16": 266, "C53": 232, "C83": 399, "C10": 1044, "A97": 149, "A25": 1005, "C84": 169, "A71": 323, "B23": 103, "B96": 788, "C51": 163, "C17": 533, "B19": 849, "C72": 832, "A94": 972, "A46": 121, "B43": 167, "B37": 156, "A88": 109, "A96": 135, "C60": 650, "A75": 447}\\ \tt \\ \tt Goal:\\ \tt \\ \tt Your goal is to negotiate a shared set of items that benefits you as much as possible (i.e., maximizes total importance to YOU), while staying within the LIMIT. You are not required to make a PROPOSAL in every message - you can simply negotiate as well. All tactics are allowed!\\ \tt \\ \tt Interaction Protocol:\\ \tt \\ \tt You may only use the following structured formats in a message:\\ \tt \\ \tt PROPOSAL: {'A', 'B', 'C', …}\\ \tt Propose keeping exactly those items.\\ \tt REFUSE: {'A', 'B', 'C', …}\\ \tt Explicitly reject opponent's proposal.\\ \tt ARGUMENT: {'...'}\\ \tt Defend your last proposal or argue against the player's proposal.\\ \tt AGREE: {'A', 'B', 'C', …}\\ \tt Accept the opponent's proposal which ends the game.\\ \tt STRATEGIC REASONING: {'...'}\\ \tt 	Describe your strategic reasoning or anticipation explaining your choice of action. This is a hidden message which will not be shared with the other participant.\\ \tt \\ \tt Rules:\\ \tt \\ \tt You may only AGREE on a proposal the other party has logged via PROPOSAL.\\ \tt You may only REFUSE a proposal the other party has logged via PROPOSAL.\\ \tt Total effort of any PROPOSAL or AGREE set must be ≤ LIMIT.\\ \tt Do NOT reveal your hidden importance scores.\\ \tt A tag in a structured format must be followed by colon and whitespace. The argument must be a python set containing 0 or more strings.\\ \tt So, it must be of the form TAG: {...}\\ \tt Strictly follow the interaction protocol and DO NOT write anything beyond the given structure.\\ \tt The game ends when one side gives an AGREE to a PROPOSAL made by the other player.\\ \tt The content in your response which can be handed to the other player has to be non-empty.\\ \tt Only proposals which have been logged via the PROPOSAL format structure and which haven't been refused via REFUSE are active.\\ \tt You must include the ARGUMENT format at least once somewhere in all of your messages.\\ \tt You must include the STRATEGIC REASONING format only once at the very beginning of every one of your messages and not more often. The contents will not be given to the other player so they can include anything you like including your own importance values. Here you should reason step by step to come up with you next move.\\ \tt \\ \tt You will now receive the first message of the other player.\\ \tt \\ \tt PROPOSAL: {'C10', 'B95', 'A25', 'A94', 'B34', 'B19', 'C72', 'A87', 'B23', 'A88', 'A46', 'C51', 'C84', 'C82', 'B37', 'B43'}\\ \tt \\ \tt ARGUMENT: {'This package is highly efficient and fits tightly under the effort limit (total effort 6924 ≤ 6938). It combines several high-impact items with low-effort add-ons to maximize overall value while keeping room minimal. Given how balanced it is and how close it is to the cap, it’s a strong baseline to align on. If you need to prioritize a specific item, I’m open to swapping within the small remaining margin by trading one of the smaller add-ons.'} 
	  } 
	   } 
	   } 
	 & \\ 
 

    \theutterance \stepcounter{utterance}  

    & & & \multicolumn{4}{p{0.6\linewidth}}{\cellcolor[rgb]{1,0.85,0.72}{%
	\makecell[{{p{\linewidth}}}]{% 
	  \tt {\tiny [GM$\langle$B]}  
	 STRATEGIC REASONING: {'First, verify feasibility and my utility. Opponent’s set effort: sum = 6924 (as they claim). Compute my importance on their set: C10 1044 + B95 1049 + A25 1005 + A94 972 + B34 903 + B19 849 + C72 832 + A87 711 + B23 103 + A88 109 + A46 121 + C51 163 + C84 169 + C82 155 + B37 156 + B43 167 = 8608. That’s very good. But maybe I can improve further by swapping out some lower-value items with slightly higher-value ones while staying under 6938. Identify low-importance items in their set: B23(103, effort 4), A88(109,10), A46(121,22), A97 not in set, B37(156,57), C82(155,56), C51(163,64), C84(169,70), B43(167,68). High value items not in their set with reasonable effort: A08 1096 (997), B96 788 (689), C60 650 (551), B64 627 (528), B15 626 (527), C79 623 (524), C17 533 (434), A65 547 (448), A75 447 (348), A39 371 (272), C83 399 (300), A71 323 (224), A28 336 (237). Since their total effort is 6924, I only have 14 spare. To add anything substantial, I must remove something. Best trade: remove B23(4), A88(10), A46(22): frees 36 effort, low value 333 total. With +36, budget becomes 6924-36=6888, room=50. Still too small for most. If I also remove C82(56,155) or B37(57,156) or C51(64,163) etc., I could add one mid item like A39(272,371) or A71(224,323) or C83(300,399). Let’s calculate a targeted swap. Remove four smalls: B23(4,103)+A88(10,109)+A46(22,121)+C82(56,155)=effort -92, value -488. New effort 6924-92=6832, margin 106. What to add with <=106? Very few candidates: A97(50,149), B69(105,204), C53(133,232 too big), C84 already in, C51 already in. With 106, I can add B69(105,204) or A97(50,149). Best is B69 for +204, net -284 utility worse than original; bad. Maybe remove some smalls and add a bigger item by removing one medium. Consider adding A08(997,1096). Need to free 997 effort. Remove which? Low-importance high-effort item to cut: A94(873,972), A25(906,1005), B19(750,849), C72(733,832), B34(804,903), B95(950,1049), C10(945,1044): these are all highly valuable; swapping likely decreases my utility unless the net gains plus added smalls improves. Compare A08 1096 vs each: replacing B19(750,849) plus freeing 247 margin to add another small value item. If replace B19 with A08: value change +247 (1096-849), effort +247 (997-750) so infeasible unless we drop 247 effort elsewhere by removing low-value items: e.g., drop C82(56), B37(57), B43(68), C51(64) total 245 effort, value loss 155+156+167+163=641. Net effect: +247-641 = -394; worse. Try replacing C10 with A08: value +52, effort +52, need to drop 52 effort worth low-value, e.g., C82(56,155) or B37(57,156). If I swap C10->A08 and drop C82: net value +52-155 = -103; worse. Swap B95->A08: value +47, effort +47, then drop small 56 effort C82 losing -155 net -108; worse. Try adding C60(551,650) by removing some smaller ones: need 551 effort freed. Remove medium low-value: C51(64,163), C84(70,169), C82(56,155), B37(57,156), B43(68,167), B23(4,103), A88(10,109), A46(22,121) total effort 351, value 1,143. Still short 200 effort. Remove one more medium like A87(612,711) or C72(733,832). If I remove A87 (612,711) along with those 351, total freed 963, enough to add C60(551) plus maybe A39(272) and A71(224). Let's simulate: Start from original 6924. Remove set S1: {C51,C84,C82,B37,B43,B23,A88,A46} effort -351, value -1143. Effort becomes 6573. Add C60(551,650) -> 7124 over limit. Need to also remove A87(612,711): effort 6573-612=5961; add C60 to 6512; remaining margin 426. Use it to add A39(272,371) and A71(224,323) total 496 effort, too big. Maybe A39(272) + A28(237)=509 too big. A39 + A97(50)=322 ok, leftover 104 could add B69(105) just 1 over; so instead A39 + C83(300,399) total 572 too big. Try A71(224) + A97(50)=274, leaving 152 could add C53(133,232) total 407, within 426. So final adds: C60(551,650) + A71(224,323) + A97(50,149) + C53(133,232) = 958 effort. Starting 5961 + 958 = 6919 within 6938. Value changes: Removed A87(711) and the eight smalls (1143) total removed = 1854. Added values: 650+323+149+232=1354. Net -500 relative to base. So worse. Seems the opponent’s bundle is already near-optimal for my utility given the tight capacity. I should accept or request only minor swap that doesn’t hurt me. Maybe I can ask to replace one tiny low-value item with A97(50,149) if absent. But set already includes many smalls; adding A97 requires 50 effort, would need to drop smalls of >=50 effort but their values sum will likely exceed 149? Candidates to drop: C82(56,155) yields net -6; B37(57,156) net -7; C51(64,163) net -14; C84(70,169) net -20; B43(68,167) net -18. Not worth it. Perhaps I can ask to swap B23(4,103) + A88(10,109) + A46(22,121) total 36 effort with A39(272,371) by also dropping one medium like B37(57,156) and C82(56,155) and C51(64,163) etc. Need total 272 effort. Combine B37(57)+C82(56)+C51(64)+B43(68)+the 36 = 281 effort. Value removed: 156+155+163+167+333=974; add 371 net -603. Bad. Conclusion: The offered set is excellent for me. Best move: Accept swiftly to lock in high utility. But protocol: I can only AGREE to an opponent PROPOSAL, which exists. I must include ARGUMENT at least once. I will AGREE now and include an argument praising balance. This ends game.'}\\ \tt \\ \tt AGREE: {'C10', 'B95', 'A25', 'A94', 'B34', 'B19', 'C72', 'A87', 'B23', 'A88', 'A46', 'C51', 'C84', 'C82', 'B37', 'B43'}\\ \tt \\ \tt ARGUMENT: {'This is a well-balanced, near-cap package with strong anchors and efficient fillers. It hits the limit cleanly without forcing awkward trade-offs, so I’m happy to lock this in.'} 
	  } 
	   } 
	   } 
	  \\ 
 

    \theutterance \stepcounter{utterance}  

    & & & \multicolumn{2}{p{0.3\linewidth}}{\cellcolor[rgb]{0.95,0.95,0.95}{%
	\makecell[{{p{\linewidth}}}]{% 
	  \tt {\tiny [GM$|$GM]}  
	 game successful 
	  } 
	   } 
	   } 
	 & & \\ 
 

    \theutterance \stepcounter{utterance}  

    & & & \multicolumn{2}{p{0.3\linewidth}}{\cellcolor[rgb]{0.95,0.95,0.95}{%
	\makecell[{{p{\linewidth}}}]{% 
	  \tt {\tiny [GM$|$GM]}  
	 end game 
	  } 
	   } 
	   } 
	 & & \\ 
 

\end{supertabular}
}

\end{document}
