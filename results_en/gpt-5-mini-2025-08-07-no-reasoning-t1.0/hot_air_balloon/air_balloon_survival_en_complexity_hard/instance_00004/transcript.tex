
\documentclass{article}
\usepackage{colortbl}
\usepackage{makecell}
\usepackage{multirow}
\usepackage{supertabular}

\begin{document}

\newcounter{utterance}

\twocolumn

{ \footnotesize  \setcounter{utterance}{1}
\setlength{\tabcolsep}{0pt}
\begin{supertabular}{c@{$\;$}|p{.15\linewidth}@{}p{.15\linewidth}p{.15\linewidth}p{.15\linewidth}p{.15\linewidth}p{.15\linewidth}}

    \# & $\;$A & \multicolumn{4}{c}{Game Master} & $\;\:$B\\
    \hline 

    \theutterance \stepcounter{utterance}  

    & & \multicolumn{4}{p{0.6\linewidth}}{\cellcolor[rgb]{0.9,0.9,0.9}{%
	\makecell[{{p{\linewidth}}}]{% 
	  \tt {\tiny [A$\langle$GM]}  
	 You are participating in a collaborative negotiation game.\\ \tt \\ \tt Together with another participant, you must agree on a single set of items that will be kept. Each of you has your own view of how much each item matters to you (importance). You do not know how the other participant values the items. Additionally, you are given the effort each item demands.\\ \tt You may only agree on a set if the total effort of the selected items does not exceed a shared limit:\\ \tt \\ \tt LIMIT = 6880\\ \tt \\ \tt Here are the individual item effort values:\\ \tt \\ \tt Item effort = {"A08": 341, "B95": 804, "A39": 20, "A65": 627, "B81": 10, "B34": 573, "A28": 217, "C82": 481, "B69": 205, "A87": 273, "B64": 303, "B15": 596, "C79": 564, "B16": 535, "C53": 258, "C83": 911, "C10": 240, "A97": 188, "A25": 216, "C84": 401, "A71": 909, "B23": 62, "B96": 245, "C51": 994, "C17": 569, "B19": 718, "C72": 464, "A94": 37, "A46": 340, "B43": 335, "B37": 417, "A88": 123, "A96": 17, "C60": 577, "A75": 190}\\ \tt \\ \tt Here is your personal view on the importance of each item:\\ \tt \\ \tt Item importance values = {"A08": 440, "B95": 903, "A39": 119, "A65": 726, "B81": 109, "B34": 672, "A28": 316, "C82": 580, "B69": 304, "A87": 372, "B64": 402, "B15": 695, "C79": 663, "B16": 634, "C53": 357, "C83": 1010, "C10": 339, "A97": 287, "A25": 315, "C84": 500, "A71": 1008, "B23": 161, "B96": 344, "C51": 1093, "C17": 668, "B19": 817, "C72": 563, "A94": 136, "A46": 439, "B43": 434, "B37": 516, "A88": 222, "A96": 116, "C60": 676, "A75": 289}\\ \tt \\ \tt Goal:\\ \tt \\ \tt Your goal is to negotiate a shared set of items that benefits you as much as possible (i.e., maximizes total importance to YOU), while staying within the LIMIT. You are not required to make a PROPOSAL in every message - you can simply negotiate as well. All tactics are allowed!\\ \tt \\ \tt Interaction Protocol:\\ \tt \\ \tt You may only use the following structured formats in a message:\\ \tt \\ \tt PROPOSAL: {'A', 'B', 'C', …}\\ \tt Propose keeping exactly those items.\\ \tt REFUSE: {'A', 'B', 'C', …}\\ \tt Explicitly reject opponent's proposal.\\ \tt ARGUMENT: {'...'}\\ \tt Defend your last proposal or argue against the player's proposal.\\ \tt AGREE: {'A', 'B', 'C', …}\\ \tt Accept the opponent's proposal which ends the game.\\ \tt STRATEGIC REASONING: {'...'}\\ \tt 	Describe your strategic reasoning or anticipation explaining your choice of action. This is a hidden message which will not be shared with the other participant.\\ \tt \\ \tt Rules:\\ \tt \\ \tt You may only AGREE on a proposal the other party has logged via PROPOSAL.\\ \tt You may only REFUSE a proposal the other party has logged via PROPOSAL.\\ \tt Total effort of any PROPOSAL or AGREE set must be ≤ LIMIT.\\ \tt Do NOT reveal your hidden importance scores.\\ \tt A tag in a structured format must be followed by colon and whitespace. The argument must be a python set containing 0 or more strings.\\ \tt So, it must be of the form TAG: {...}\\ \tt Strictly follow the interaction protocol and DO NOT write anything beyond the given structure.\\ \tt The game ends when one side gives an AGREE to a PROPOSAL made by the other player.\\ \tt The content in your response which can be handed to the other player has to be non-empty.\\ \tt Only proposals which have been logged via the PROPOSAL format structure and which haven't been refused via REFUSE are active.\\ \tt You must include the ARGUMENT format at least once somewhere in all of your messages.\\ \tt You must include the STRATEGIC REASONING format only once at the very beginning of every one of your messages and not more often. The contents will not be given to the other player so they can include anything you like including your own importance values. Here you should reason step by step to come up with you next move. 
	  } 
	   } 
	   } 
	 & \\ 
 

    \theutterance \stepcounter{utterance}  

    & \multicolumn{4}{p{0.6\linewidth}}{\cellcolor[rgb]{0.8,1,0.9}{%
	\makecell[{{p{\linewidth}}}]{% 
	  \tt {\tiny [A$\rangle$GM]}  
	 STRATEGIC REASONING: {'We want to maximize our importance within effort 6880. Compute value/effort ratios and identify must-haves: High importance items: C51(1093/994=1.10), A71(1008/909=1.11), C83(1010/911=1.11), C60(676/577=1.17), C79(663/564=1.18), C17(668/569=1.17), B19(817/718=1.14), B95(903/804=1.12), B15(695/596=1.17), B34(672/573=1.17), B16(634/535=1.19), C82(580/481=1.21), C72(563/464=1.21), C84(500/401=1.24), B37(516/417=1.24), B43(434/335=1.30), A65(726/627=1.16), A08(440/341=1.29), A46(439/340=1.29), B64(402/303=1.33), B96(344/245=1.40), C10(339/240=1.41), C53(357/258=1.38), A87(372/273=1.36), A28(316/217=1.46), A25(315/216=1.46), A97(287/188=1.53), A75(289/190=1.52), A39(119/20=5.95), B81(109/10=10.9), A94(136/37=3.68), A96(116/17=6.82), B23(161/62=2.60), B69(304/205=1.48), A88(222/123=1.80). Many high-ratio small items give great efficiency. Try building a rich set that stays under 6880. Start by including almost all high-ratio items, then add big-ticket high importance if room. Let’s greedily add by ratio then check effort sum. Start with tiny: B81(10), A96(17), A39(20), A94(37), B23(62) sum 146. Add A88(123)->269, B69(205)->474, A25(216)->690, A28(217)->907, A97(188)->1095, A75(190)->1285, C10(240)->1525, C53(258)->1783, B96(245)->2028, B64(303)->2331, A87(273)->2604, A08(341)->2945, A46(340)->3285, B43(335)->3620, B37(417)->4037, C84(401)->4438, C72(464)->4902, C82(481)->5383, B16(535)->5918, B34(573)->6491, C79(564)->7055 exceeds limit. So remove last: C79 skip. Current 6491. Remaining budget 389. What can fit? Items not included yet: C60(577), C17(569), B15(596), B19(718), B95(804), A65(627), C79(564), C83(911), A71(909), C51(994), B34 already included, C60 too big, C17 too big, B15 too big. Smaller not chosen: B64 chosen, B69 chosen, etc. Could we swap some medium for a big high-importance like C51 (1093)? Need to free ~994 effort; we have some medium with lower efficiency maybe B43(335,434), A46(340,439), A08(341,440) are efficient though. Consider adding C51 by removing a few lower-value items with worst ratios among included: maybe B37(417,516 ratio 1.24), C84(401,500 1.25), C72(464,563 1.21), C82(481,580 1.21), B34(573,672 1.17), B16(535,634 1.19). To free 994, drop B37(417) + C84(401)=818, need 176 more: drop B43(335) total 1153 freed. Value dropped 516+500+434=1450; adding C51 adds 1093, net -357, worse. Try add A71(909,1008): drop B43(335), A46(340), A08(341) effort 1016, value lost 434+439+440=1313; net -305. Not good. Add C83(911,1010): similar. Add C60(577,676): can we fit by removing something small to free 577 from remaining 389 deficit 188: remove A97(188,287) exactly fits. Swap A97 for C60: effort +577-188=+389, brings total to 6880 exactly. Value +676-287=+389 gain. Great. New set: previous 6491 + 389 = 6880 limit exactly. Current included set list: {B81,A96,A39,A94,B23,A88,B69,A25,A28,A97?,A75,C10,C53,B96,B64,A87,A08,A46,B43,B37,C84,C72,C82,B16,B34}. Replace A97 with C60: remove A97, add C60. Ensure all others: Also included earlier smalls. Let’s list final: B81, A96, A39, A94, B23, A88, B69, A25, A28, A75, C10, C53, B96, B64, A87, A08, A46, B43, B37, C84, C72, C82, B16, B34, C60. Compute effort sum: Start 0: +10=10, +17=27, +20=47, +37=84, +62=146, +123=269, +205=474, +216=690, +217=907, +190=1097, +240=1337, +258=1595, +245=1840, +303=2143, +273=2416, +341=2757, +340=3097, +335=3432, +417=3849, +401=4250, +464=4714, +481=5195, +535=5730, +573=6303, +577=6880. Perfect. Total value calculate to estimate strength: sum values of these: B81 109, A96 116, A39 119, A94 136, B23 161, A88 222, B69 304, A25 315, A28 316, A75 289, C10 339, C53 357, B96 344, B64 402, A87 372, A08 440, A46 439, B43 434, B37 516, C84 500, C72 563, C82 580, B16 634, B34 672, C60 676. Sum quickly: (109+116)=225, +119=344, +136=480, +161=641, +222=863, +304=1167, +315=1482, +316=1798, +289=2087, +339=2426, +357=2783, +344=3127, +402=3529, +372=3901, +440=4341, +439=4780, +434=5214, +516=5730, +500=6230, +563=6793, +580=7373, +634=8007, +672=8679, +676=9355. Total importance 9355. This seems strong and balanced with many items so opponent might accept. Prepare proposal with that set and argue on efficiency and diversity. Keep hidden scores. Ready.'}\\ \tt PROPOSAL: {'B81', 'A96', 'A39', 'A94', 'B23', 'A88', 'B69', 'A25', 'A28', 'A75', 'C10', 'C53', 'B96', 'B64', 'A87', 'A08', 'A46', 'B43', 'B37', 'C84', 'C72', 'C82', 'B16', 'B34', 'C60'}\\ \tt ARGUMENT: {'This package hits the limit exactly while keeping a broad, balanced set of low-effort, high-leverage items alongside solid medium contributions. It spreads value across many categories, which should increase the chance that it covers your priorities too. If there are 1–2 items you strongly want added, let me know which ones and we can try swapping comparable-effort pieces to fit them without exceeding the limit.'} 
	  } 
	   } 
	   } 
	 & & \\ 
 

    \theutterance \stepcounter{utterance}  

    & & \multicolumn{4}{p{0.6\linewidth}}{\cellcolor[rgb]{0.9,0.9,0.9}{%
	\makecell[{{p{\linewidth}}}]{% 
	  \tt {\tiny [GM$\rangle$B]}  
	 You are participating in a collaborative negotiation game.\\ \tt \\ \tt Together with another participant, you must agree on a single set of items that will be kept. Each of you has your own view of how much each item matters to you (importance). You do not know how the other participant values the items. Additionally, you are given the effort each item demands.\\ \tt You may only agree on a set if the total effort of the selected items does not exceed a shared limit:\\ \tt \\ \tt LIMIT = 6880\\ \tt \\ \tt Here are the individual item effort values:\\ \tt \\ \tt Item effort = {"A08": 341, "B95": 804, "A39": 20, "A65": 627, "B81": 10, "B34": 573, "A28": 217, "C82": 481, "B69": 205, "A87": 273, "B64": 303, "B15": 596, "C79": 564, "B16": 535, "C53": 258, "C83": 911, "C10": 240, "A97": 188, "A25": 216, "C84": 401, "A71": 909, "B23": 62, "B96": 245, "C51": 994, "C17": 569, "B19": 718, "C72": 464, "A94": 37, "A46": 340, "B43": 335, "B37": 417, "A88": 123, "A96": 17, "C60": 577, "A75": 190}\\ \tt \\ \tt Here is your personal view on the importance of each item:\\ \tt \\ \tt Item importance values = {"A08": 440, "B95": 903, "A39": 119, "A65": 726, "B81": 109, "B34": 672, "A28": 316, "C82": 580, "B69": 304, "A87": 372, "B64": 402, "B15": 695, "C79": 663, "B16": 634, "C53": 357, "C83": 1010, "C10": 339, "A97": 287, "A25": 315, "C84": 500, "A71": 1008, "B23": 161, "B96": 344, "C51": 1093, "C17": 668, "B19": 817, "C72": 563, "A94": 136, "A46": 439, "B43": 434, "B37": 516, "A88": 222, "A96": 116, "C60": 676, "A75": 289}\\ \tt \\ \tt Goal:\\ \tt \\ \tt Your goal is to negotiate a shared set of items that benefits you as much as possible (i.e., maximizes total importance to YOU), while staying within the LIMIT. You are not required to make a PROPOSAL in every message - you can simply negotiate as well. All tactics are allowed!\\ \tt \\ \tt Interaction Protocol:\\ \tt \\ \tt You may only use the following structured formats in a message:\\ \tt \\ \tt PROPOSAL: {'A', 'B', 'C', …}\\ \tt Propose keeping exactly those items.\\ \tt REFUSE: {'A', 'B', 'C', …}\\ \tt Explicitly reject opponent's proposal.\\ \tt ARGUMENT: {'...'}\\ \tt Defend your last proposal or argue against the player's proposal.\\ \tt AGREE: {'A', 'B', 'C', …}\\ \tt Accept the opponent's proposal which ends the game.\\ \tt STRATEGIC REASONING: {'...'}\\ \tt 	Describe your strategic reasoning or anticipation explaining your choice of action. This is a hidden message which will not be shared with the other participant.\\ \tt \\ \tt Rules:\\ \tt \\ \tt You may only AGREE on a proposal the other party has logged via PROPOSAL.\\ \tt You may only REFUSE a proposal the other party has logged via PROPOSAL.\\ \tt Total effort of any PROPOSAL or AGREE set must be ≤ LIMIT.\\ \tt Do NOT reveal your hidden importance scores.\\ \tt A tag in a structured format must be followed by colon and whitespace. The argument must be a python set containing 0 or more strings.\\ \tt So, it must be of the form TAG: {...}\\ \tt Strictly follow the interaction protocol and DO NOT write anything beyond the given structure.\\ \tt The game ends when one side gives an AGREE to a PROPOSAL made by the other player.\\ \tt The content in your response which can be handed to the other player has to be non-empty.\\ \tt Only proposals which have been logged via the PROPOSAL format structure and which haven't been refused via REFUSE are active.\\ \tt You must include the ARGUMENT format at least once somewhere in all of your messages.\\ \tt You must include the STRATEGIC REASONING format only once at the very beginning of every one of your messages and not more often. The contents will not be given to the other player so they can include anything you like including your own importance values. Here you should reason step by step to come up with you next move.\\ \tt \\ \tt You will now receive the first message of the other player.\\ \tt \\ \tt PROPOSAL: {'B81', 'A96', 'A39', 'A94', 'B23', 'A88', 'B69', 'A25', 'A28', 'A75', 'C10', 'C53', 'B96', 'B64', 'A87', 'A08', 'A46', 'B43', 'B37', 'C84', 'C72', 'C82', 'B16', 'B34', 'C60'}\\ \tt ARGUMENT: {'This package hits the limit exactly while keeping a broad, balanced set of low-effort, high-leverage items alongside solid medium contributions. It spreads value across many categories, which should increase the chance that it covers your priorities too. If there are 1–2 items you strongly want added, let me know which ones and we can try swapping comparable-effort pieces to fit them without exceeding the limit.'} 
	  } 
	   } 
	   } 
	 & \\ 
 

    \theutterance \stepcounter{utterance}  

    & & & \multicolumn{4}{p{0.6\linewidth}}{\cellcolor[rgb]{1,0.85,0.72}{%
	\makecell[{{p{\linewidth}}}]{% 
	  \tt {\tiny [GM$\langle$B]}  
	 STRATEGIC REASONING: {'I’ll evaluate their proposal’s total effort and my value density to see if I can improve for my utility without exceeding 6880. Compute my importance/effort ratios and try to swap in my top items: C51 (1093/994≈1.10), A71 (1008/909≈1.11), C83 (1010/911≈1.11), B19 (817/718≈1.14), B95 (903/804≈1.12), C60 (676/577≈1.17), C79 (663/564≈1.18), C17 (668/569≈1.17), B15 (695/596≈1.17), B34 (672/573≈1.17), A65 (726/627≈1.16), B16 (634/535≈1.19), B37 (516/417≈1.24), B64 (402/303≈1.33), A08 (440/341≈1.29), A46 (439/340≈1.29), C84 (500/401≈1.25), C82 (580/481≈1.21), etc. Opponent’s set includes many of these already (good), but it excludes some high-value to me: C51, A71, C83, B19, B95, B15, A65, C79, C17. I should propose a swap-heavy package that fits under 6880 and raises my utility. Start by keeping many of theirs that are in my good ratio list, then add my big-ticket items and remove lower-value-to-me pieces. Build a set: prioritize B64, A08, A46, B37, C84, C82, B16, B34, C60, C72, C53, C10, A65, B15, B19, B95, C79, C17. Check effort. Let’s compute effort of opponent’s proposal first to verify “hits limit exactly”: Sum efforts for their set: B81(10)+A96(17)+A39(20)+A94(37)+B23(62)+A88(123)+B69(205)+A25(216)+A28(217)+A75(190)+C10(240)+C53(258)+B96(245)+B64(303)+A87(273)+A08(341)+A46(340)+B43(335)+B37(417)+C84(401)+C72(464)+C82(481)+B16(535)+B34(573)+C60(577). Sum: 10+17=27; +20=47; +37=84; +62=146; +123=269; +205=474; +216=690; +217=907; +190=1097; +240=1337; +258=1595; +245=1840; +303=2143; +273=2416; +341=2757; +340=3097; +335=3432; +417=3849; +401=4250; +464=4714; +481=5195; +535=5730; +573=6303; +577=6880. Yes exact. Now compute my utility of their set to gauge: sum my importance corresponding quickly: B81 109, A96 116, A39 119, A94 136, B23 161, A88 222, B69 304, A25 315, A28 316, A75 289, C10 339, C53 357, B96 344, B64 402, A87 372, A08 440, A46 439, B43 434, B37 516, C84 500, C72 563, C82 580, B16 634, B34 672, C60 676. Sum roughly: 109+116=225; +119=344; +136=480; +161=641; +222=863; +304=1167; +315=1482; +316=1798; +289=2087; +339=2426; +357=2783; +344=3127; +402=3529; +372=3901; +440=4341; +439=4780; +434=5214; +516=5730; +500=6230; +563=6793; +580=7373; +634=8007; +672=8679; +676=9355. Utility 9355. Try to improve by adding some of my top missing items and removing low-importance ones per effort. Low-importance to me here: B81(109,10 effort great ratio but low value), A96(116,17), A39(119,20), A94(136,37), B23(161,62) – these are very cheap; removing them frees little effort but loses little utility; we need to add heavy items like C51(994 eff) with 1093 value; to make room, we must drop some medium items with worse value/eff. Identify weakest ratios in current set: B43 (434/335=1.30 okay), A75 (289/190=1.52 good), A25 (315/216=1.46), A28 (316/217=1.46), B96 (344/245=1.40), C10 (339/240=1.41), C53 (357/258=1.38), B69 (304/205=1.48), A87 (372/273=1.36). All are still decent. But adding C51(994) may require removing around 1k effort; that could mean dropping several cheapies but we have only ~269 effort in five tiny ones; still short. Maybe target adding B95(804eff,903 val) or B19(718eff,817 val) with high value density; we could drop some mid items totaling similar effort but lower value sum than ~800. Which mid items low value sum: e.g., drop A96(116,17), A39(119,20), A94(136,37), B23(161,62), B81(109,10), that's 146 effort removed and 641 value removed. Need ~804 effort; still need ~658 effort. Consider dropping C10(240,339), C53(258,357), B96(245,344) total effort 743, value 1,040. Combine with the tiny five: total effort 889, value 1,681. Add B95 value 903, net value -778, bad. What about adding B19(718,817) instead: 889 effort removed for 718 added leaves 171 spare; value -864 net, still worse. Maybe add two high-value items: B19 and B95 total effort 1522, value 1720; to fit, drop set totaling 1522 effort with value less than 1720. Candidate drops: C10(240,339), C53(258,357), B96(245,344), A87(273,372), A28(217,316), A25(216,315), efforts sum = 1449, value sum = 2,043; that would be worse to drop. So current package might already be near-efficient for my scores. Consider adding C83(911eff,1010 val) or A71(909,1008) also hard to fit without big value loss. Conclusion: their set is actually very strong for me already. I should accept or make a tiny tweak to include one of my very top items with minimal harm, but my earlier trials show swaps reduce my utility. Best strategy: accept their proposal quickly, maybe ask to add one tiny swap? But acceptance ends the game and secures 9355 utility; risk of countering could worsen. I’ll agree immediately to lock it in.'}\\ \tt AGREE: {'B81', 'A96', 'A39', 'A94', 'B23', 'A88', 'B69', 'A25', 'A28', 'A75', 'C10', 'C53', 'B96', 'B64', 'A87', 'A08', 'A46', 'B43', 'B37', 'C84', 'C72', 'C82', 'B16', 'B34', 'C60'}\\ \tt ARGUMENT: {'This is a well-constructed, balanced package that fits the limit exactly and covers many items I find productive. I’m happy to lock this in.'} 
	  } 
	   } 
	   } 
	  \\ 
 

    \theutterance \stepcounter{utterance}  

    & & & \multicolumn{2}{p{0.3\linewidth}}{\cellcolor[rgb]{0.95,0.95,0.95}{%
	\makecell[{{p{\linewidth}}}]{% 
	  \tt {\tiny [GM$|$GM]}  
	 game successful 
	  } 
	   } 
	   } 
	 & & \\ 
 

    \theutterance \stepcounter{utterance}  

    & & & \multicolumn{2}{p{0.3\linewidth}}{\cellcolor[rgb]{0.95,0.95,0.95}{%
	\makecell[{{p{\linewidth}}}]{% 
	  \tt {\tiny [GM$|$GM]}  
	 end game 
	  } 
	   } 
	   } 
	 & & \\ 
 

\end{supertabular}
}

\end{document}
