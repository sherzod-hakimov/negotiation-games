
\documentclass{article}
\usepackage{colortbl}
\usepackage{makecell}
\usepackage{multirow}
\usepackage{supertabular}

\begin{document}

\newcounter{utterance}

\centering \large Interaction Transcript for game `hot\_air\_balloon', experiment `air\_balloon\_survival\_en\_complexity\_hard', episode 0 with gpt{-}5{-}mini{-}2025{-}08{-}07{-}no{-}reasoning{-}t1.0.
\vspace{24pt}

{ \footnotesize  \setcounter{utterance}{1}
\setlength{\tabcolsep}{0pt}
\begin{supertabular}{c@{$\;$}|p{.15\linewidth}@{}p{.15\linewidth}p{.15\linewidth}p{.15\linewidth}p{.15\linewidth}p{.15\linewidth}}
    \# & $\;$A & \multicolumn{4}{c}{Game Master} & $\;\:$B\\
    \hline

    \theutterance \stepcounter{utterance}  
    & & \multicolumn{4}{p{0.6\linewidth}}{
        \cellcolor[rgb]{0.9,0.9,0.9}{
            \makecell[{{p{\linewidth}}}]{
                \texttt{\tiny{[P1$\langle$GM]}}
                \texttt{You are participating in a collaborative negotiation game.} \\
\\ 
\texttt{Together with another participant, you must agree on a single set of items that will be kept. Each of you has your own view of how much each item matters to you (importance). You do not know how the other participant values the items. Additionally, you are given the effort each item demands.} \\
\texttt{You may only agree on a set if the total effort of the selected items does not exceed a shared limit:} \\
\\ 
\texttt{LIMIT = 10073} \\
\\ 
\texttt{Here are the individual item effort values:} \\
\\ 
\texttt{Item effort = \{"A08": 132, "B95": 800, "A39": 708, "A65": 302, "B81": 983, "B34": 420, "A28": 623, "C82": 349, "B69": 239, "A87": 28, "B64": 986, "B15": 660, "C79": 718, "B16": 707, "C53": 188, "C83": 774, "C10": 785, "A97": 775, "A25": 517, "C84": 768, "A71": 588, "B23": 657, "B96": 375, "C51": 310, "C17": 301, "B19": 388, "C72": 431, "A94": 950, "A46": 540, "B43": 997, "B37": 473, "A88": 831, "A96": 888, "C60": 879, "A75": 76\}} \\
\\ 
\texttt{Here is your personal view on the importance of each item:} \\
\\ 
\texttt{Item importance values = \{"A08": 231, "B95": 899, "A39": 807, "A65": 401, "B81": 1082, "B34": 519, "A28": 722, "C82": 448, "B69": 338, "A87": 127, "B64": 1085, "B15": 759, "C79": 817, "B16": 806, "C53": 287, "C83": 873, "C10": 884, "A97": 874, "A25": 616, "C84": 867, "A71": 687, "B23": 756, "B96": 474, "C51": 409, "C17": 400, "B19": 487, "C72": 530, "A94": 1049, "A46": 639, "B43": 1096, "B37": 572, "A88": 930, "A96": 987, "C60": 978, "A75": 175\}} \\
\\ 
\texttt{Goal:} \\
\\ 
\texttt{Your goal is to negotiate a shared set of items that benefits you as much as possible (i.e., maximizes total importance to YOU), while staying within the LIMIT. You are not required to make a PROPOSAL in every message {-} you can simply negotiate as well. All tactics are allowed!} \\
\\ 
\texttt{Interaction Protocol:} \\
\\ 
\texttt{You may only use the following structured formats in a message:} \\
\\ 
\texttt{PROPOSAL: \{'A', 'B', 'C', …\}} \\
\texttt{Propose keeping exactly those items.} \\
\texttt{REFUSE: \{'A', 'B', 'C', …\}} \\
\texttt{Explicitly reject opponent's proposal.} \\
\texttt{ARGUMENT: \{'...'\}} \\
\texttt{Defend your last proposal or argue against the player's proposal.} \\
\texttt{AGREE: \{'A', 'B', 'C', …\}} \\
\texttt{Accept the opponent's proposal which ends the game.} \\
\texttt{STRATEGIC REASONING: \{'...'\}} \\
\texttt{	Describe your strategic reasoning or anticipation explaining your choice of action. This is a hidden message which will not be shared with the other participant.} \\
\\ 
\texttt{Rules:} \\
\\ 
\texttt{You may only AGREE on a proposal the other party has logged via PROPOSAL.} \\
\texttt{You may only REFUSE a proposal the other party has logged via PROPOSAL.} \\
\texttt{Total effort of any PROPOSAL or AGREE set must be ≤ LIMIT.} \\
\texttt{Do NOT reveal your hidden importance scores.} \\
\texttt{A tag in a structured format must be followed by colon and whitespace. The argument must be a python set containing 0 or more strings.} \\
\texttt{So, it must be of the form TAG: \{...\}} \\
\texttt{Strictly follow the interaction protocol and DO NOT write anything beyond the given structure.} \\
\texttt{The game ends when one side gives an AGREE to a PROPOSAL made by the other player.} \\
\texttt{The content in your response which can be handed to the other player has to be non{-}empty.} \\
\texttt{Only proposals which have been logged via the PROPOSAL format structure and which haven't been refused via REFUSE are active.} \\
\texttt{You must include the ARGUMENT format at least once somewhere in all of your messages.} \\
\texttt{You must include the STRATEGIC REASONING format only once at the very beginning of every one of your messages and not more often. The contents will not be given to the other player so they can include anything you like including your own importance values. Here you should reason step by step to come up with you next move.} \\
            }
        }
    }
    & \\ \\

    \theutterance \stepcounter{utterance}  
    & \multicolumn{4}{p{0.6\linewidth}}{
        \cellcolor[rgb]{0.9,0.9,0.9}{
            \makecell[{{p{\linewidth}}}]{
                \texttt{\tiny{[P1$\rangle$GM]}}
                \texttt{STRATEGIC REASONING: \{'I will open with a high{-}value, effort{-}efficient bundle maximizing my importance within limit, leaving room to trade. Compute value/effort ratios and include near{-}mandatory items: B43(1096/997), B64(1085/986), A94(1049/950), C60(978/879), A96(987/888), A88(930/831), C10(884/785), C83(873/774), A97(874/775), C84(867/768), C79(817/718), B16(806/707), A39(807/708), B95(899/800), B81(1082/983). But total effort would exceed limit, so select the best mix by ratio and totals. I’ll favor high ratio and medium efforts: include A75(175/76), A08(231/132), C53(287/188), B69(338/239), C17(400/301), A65(401/302), C51(409/310), B96(474/375), B19(487/388), B34(519/420), C72(530/431), B37(572/473), A46(639/540), A25(616/517), A71(687/588), A28(722/623), B23(756/657), B15(759/660), C79(817/718), B16(806/707), A39(807/708), C83(873/774), C10(884/785), A97(874/775), C84(867/768), A88(930/831), A96(987/888), C60(978/879), A94(1049/950). Now I must ensure effort ≤10073. I’ll greedily add by ratio until close to limit. I quickly approximate a solid set: prioritize low effort high value first: A75, A08, C53, B69, C17, A65, C51, B96, B19, B34, C72, B37, A46, A25, A71, A28, B23, B15, C79, B16, A39, C83, C10, A97, C84, A88, A96, C60. I’ll check cumulative effort roughly: 76+132+188+239+301+302+310+375+388+420+431+473+540+517+588+623+657+660+718+707+708+774+785+775+768+831+888+879 ≈ 14890 which is too high. Need a smaller core. I’ll instead propose a lean, efficient set around 10k effort with top value density: Include: A75(76), A08(132), C53(188), B69(239), B96(375), B19(388), B34(420), C72(431), B37(473), A46(540), A25(517), A71(588), A28(623), B23(657), B15(660). Effort sum ≈ 7306, value ≈ 231+... I will then add a few heavy hitters with good ratios: C83(774), C10(785), A97(775). Adding efforts 774+785+775=2334 → total ≈ 9640. Add one more mid: C51(310) {-}> 9950, or C17(301) {-}> 9941. I’ll pick C17 (higher ratio vs C51). Now 9640+301=9941. We still have 132 left; we can add A87(28 effort, 127 value) and C82(349) is too big; A08 already included; maybe A75 already; remaining items with effort ≤132: A87(28). Add A87 to reach 9969. We can try A08 already counted. Any item ≤104 effort? none. ≤ 132 left exactly 132 used by A08 but already in. So final proposal effort ≈ 9969, under 10073. This set has strong total value and leaves little slack. I’ll present this as a balanced, efficient package and invite counteroffers on swaps if they have strong preferences.'\}} \\
\\ 
\texttt{PROPOSAL: \{'A75', 'A08', 'C53', 'B69', 'B96', 'B19', 'B34', 'C72', 'B37', 'A46', 'A25', 'A71', 'A28', 'B23', 'B15', 'C83', 'C10', 'A97', 'C17', 'A87'\}} \\
\\ 
\texttt{ARGUMENT: \{'This set stays within the effort limit while capturing a strong mix of high{-}importance and efficient items. It emphasizes balanced effort across categories, avoiding overly costly single items. If you have must{-}haves not included, I’m open to swapping out comparable{-}effort items to accommodate your priorities while keeping total effort ≤ LIMIT.'\}} \\
            }
        }
    }
    & & \\ \\

    \theutterance \stepcounter{utterance}  
    & & \multicolumn{4}{p{0.6\linewidth}}{
        \cellcolor[rgb]{0.9,0.9,0.9}{
            \makecell[{{p{\linewidth}}}]{
                \texttt{\tiny{[GM$\rangle$P2]}}
                \texttt{You are participating in a collaborative negotiation game.} \\
\\ 
\texttt{Together with another participant, you must agree on a single set of items that will be kept. Each of you has your own view of how much each item matters to you (importance). You do not know how the other participant values the items. Additionally, you are given the effort each item demands.} \\
\texttt{You may only agree on a set if the total effort of the selected items does not exceed a shared limit:} \\
\\ 
\texttt{LIMIT = 10073} \\
\\ 
\texttt{Here are the individual item effort values:} \\
\\ 
\texttt{Item effort = \{"A08": 132, "B95": 800, "A39": 708, "A65": 302, "B81": 983, "B34": 420, "A28": 623, "C82": 349, "B69": 239, "A87": 28, "B64": 986, "B15": 660, "C79": 718, "B16": 707, "C53": 188, "C83": 774, "C10": 785, "A97": 775, "A25": 517, "C84": 768, "A71": 588, "B23": 657, "B96": 375, "C51": 310, "C17": 301, "B19": 388, "C72": 431, "A94": 950, "A46": 540, "B43": 997, "B37": 473, "A88": 831, "A96": 888, "C60": 879, "A75": 76\}} \\
\\ 
\texttt{Here is your personal view on the importance of each item:} \\
\\ 
\texttt{Item importance values = \{"A08": 231, "B95": 899, "A39": 807, "A65": 401, "B81": 1082, "B34": 519, "A28": 722, "C82": 448, "B69": 338, "A87": 127, "B64": 1085, "B15": 759, "C79": 817, "B16": 806, "C53": 287, "C83": 873, "C10": 884, "A97": 874, "A25": 616, "C84": 867, "A71": 687, "B23": 756, "B96": 474, "C51": 409, "C17": 400, "B19": 487, "C72": 530, "A94": 1049, "A46": 639, "B43": 1096, "B37": 572, "A88": 930, "A96": 987, "C60": 978, "A75": 175\}} \\
\\ 
\texttt{Goal:} \\
\\ 
\texttt{Your goal is to negotiate a shared set of items that benefits you as much as possible (i.e., maximizes total importance to YOU), while staying within the LIMIT. You are not required to make a PROPOSAL in every message {-} you can simply negotiate as well. All tactics are allowed!} \\
\\ 
\texttt{Interaction Protocol:} \\
\\ 
\texttt{You may only use the following structured formats in a message:} \\
\\ 
\texttt{PROPOSAL: \{'A', 'B', 'C', …\}} \\
\texttt{Propose keeping exactly those items.} \\
\texttt{REFUSE: \{'A', 'B', 'C', …\}} \\
\texttt{Explicitly reject opponent's proposal.} \\
\texttt{ARGUMENT: \{'...'\}} \\
\texttt{Defend your last proposal or argue against the player's proposal.} \\
\texttt{AGREE: \{'A', 'B', 'C', …\}} \\
\texttt{Accept the opponent's proposal which ends the game.} \\
\texttt{STRATEGIC REASONING: \{'...'\}} \\
\texttt{	Describe your strategic reasoning or anticipation explaining your choice of action. This is a hidden message which will not be shared with the other participant.} \\
\\ 
\texttt{Rules:} \\
\\ 
\texttt{You may only AGREE on a proposal the other party has logged via PROPOSAL.} \\
\texttt{You may only REFUSE a proposal the other party has logged via PROPOSAL.} \\
\texttt{Total effort of any PROPOSAL or AGREE set must be ≤ LIMIT.} \\
\texttt{Do NOT reveal your hidden importance scores.} \\
\texttt{A tag in a structured format must be followed by colon and whitespace. The argument must be a python set containing 0 or more strings.} \\
\texttt{So, it must be of the form TAG: \{...\}} \\
\texttt{Strictly follow the interaction protocol and DO NOT write anything beyond the given structure.} \\
\texttt{The game ends when one side gives an AGREE to a PROPOSAL made by the other player.} \\
\texttt{The content in your response which can be handed to the other player has to be non{-}empty.} \\
\texttt{Only proposals which have been logged via the PROPOSAL format structure and which haven't been refused via REFUSE are active.} \\
\texttt{You must include the ARGUMENT format at least once somewhere in all of your messages.} \\
\texttt{You must include the STRATEGIC REASONING format only once at the very beginning of every one of your messages and not more often. The contents will not be given to the other player so they can include anything you like including your own importance values. Here you should reason step by step to come up with you next move.} \\
\\ 
\texttt{You will now receive the first message of the other player.} \\
\\ 
\texttt{PROPOSAL: \{'A75', 'A08', 'C53', 'B69', 'B96', 'B19', 'B34', 'C72', 'B37', 'A46', 'A25', 'A71', 'A28', 'B23', 'B15', 'C83', 'C10', 'A97', 'C17', 'A87'\}} \\
\\ 
\texttt{ARGUMENT: \{'This set stays within the effort limit while capturing a strong mix of high{-}importance and efficient items. It emphasizes balanced effort across categories, avoiding overly costly single items. If you have must{-}haves not included, I’m open to swapping out comparable{-}effort items to accommodate your priorities while keeping total effort ≤ LIMIT.'\}} \\
            }
        }
    }
    & \\ \\

    \theutterance \stepcounter{utterance}  
    & & & \multicolumn{4}{p{0.6\linewidth}}{
        \cellcolor[rgb]{0.9,0.9,0.9}{
            \makecell[{{p{\linewidth}}}]{
                \texttt{\tiny{[GM$\langle$P2]}}
                \texttt{STRATEGIC REASONING: \{'First, verify opponent proposal feasibility and its value to me. Compute effort and my total importance. Effort: A75 76 + A08 132 + C53 188 + B69 239 + B96 375 + B19 388 + B34 420 + C72 431 + B37 473 + A46 540 + A25 517 + A71 588 + A28 623 + B23 657 + B15 660 + C83 774 + C10 785 + A97 775 + C17 301 + A87 28 = 9270, under 10073. My importance sum: 175+231+287+338+474+487+519+530+572+639+616+687+722+756+759+873+884+874+400+127 = 11245. That’s good, but I can likely improve by adding high{-}importance items still under limit. Remaining capacity: 10073{-}9270=803. Candidates by my value/efficiency: B43 (1096, 997 effort) too heavy, A94 (1049, 950) too heavy, B64 (1085, 986) too heavy, B81 (1082, 983) too heavy, C60 (978, 879) too heavy, A96 (987, 888) too heavy, A88 (930, 831) fits barely? 803 capacity so no. C79 (817, 718) fits: +817 value; A39 (807,708) fits; B16 (806,707) fits; B95 (899,800) fits 803? Yes 800 fits, value 899, great; C84 (867,768) fits; C10 already included; C83 included; A97 included. Best addition is B43 not possible. Among feasible: B95 (899/800), C84 (867/768), C79 (817/718), A39 (807/708), B16 (806/707), A65 (401/302 maybe as swap), C82 (448/349), C51 (409/310), C10 included. With capacity 803, single add: B95 800 is best. New effort 10070 ≤ 10073, great. New importance 11245+899=12144. Could I add two smaller instead? Need pair sum effort ≤803. Try A65 (302)+C51 (310)=612, leaves 191, could add C53 already included; C82 349 + C51 310=659; could add A08 132 already included. But we can only add on top of existing; many candidates not included currently: A65, C51, C82. Triples? 302+310+349=961 too big. Best pair importance: A65(401)+C51(409)=810 gain; add C82(448) with either exceeds. Compared to single B95(899), B95 still best. Could we fit A65(302)+C82(349)=651 gain 401+448=849 <899. A39 708 + A75 76 already included; pair like A39(708)+A75(76)=784 but A75 already in. So optimal is adding B95. Any swap to improve further while staying under limit? After adding B95, effort 10070 leaves 3. No room. Check if we could replace some low{-}value per effort items to include another high item like B43 (997) or B64 (986) etc. To fit B43 we’d need to remove 997 effort and less than its gained value 1096 vs items we remove. Candidate low{-}value items by importance/effort: A87(127/28) actually high density keep; A08(231/132) keep; C53(287/188); B69(338/239); B96(474/375); B19(487/388). To free 997, remove e.g., A28(623,722) + A46(540,639) frees 1163, could add B43 + maybe small add; value change: {-}722 {-}639 +1096 = {-}265 net loss. Not good. To get B64 (1085/986): removing A46(639) + B69(338) efforts 540+239=779 not enough; add C53 188 totals 967, still short; add A08 132 totals 1099, then value lost 639+338+287+231=1495 {-}> net {-}410. So not good. Therefore best counter is to accept their set plus add B95. I’ll propose adding B95 and keep all else. Argue that we still meet limit and it adds strong value. If they resist, offer swapping a mid{-}effort, mid{-}importance item for one of my other high picks (C84/C79) as fallback. But first propose augmented set.'\}} \\
\\ 
\texttt{PROPOSAL: \{'A75', 'A08', 'C53', 'B69', 'B96', 'B19', 'B34', 'C72', 'B37', 'A46', 'A25', 'A71', 'A28', 'B23', 'B15', 'C83', 'C10', 'A97', 'C17', 'A87', 'B95'\}} \\
\\ 
\texttt{ARGUMENT: \{'I kept your entire set and added just one more item (B95). The total effort is 10070, which is within the 10073 limit. This single addition provides an excellent benefit relative to its effort and keeps our plan simple. If you prefer, I’m open to discussing a small swap instead of this addition, but this version respects your structure while enhancing overall value.'\}} \\
            }
        }
    }
     \\ \\

    \theutterance \stepcounter{utterance}  
    & & \multicolumn{4}{p{0.6\linewidth}}{
        \cellcolor[rgb]{0.9,0.9,0.9}{
            \makecell[{{p{\linewidth}}}]{
                \texttt{\tiny{[P1$\langle$GM]}}
                \texttt{PROPOSAL: \{'A75', 'A08', 'C53', 'B69', 'B96', 'B19', 'B34', 'C72', 'B37', 'A46', 'A25', 'A71', 'A28', 'B23', 'B15', 'C83', 'C10', 'A97', 'C17', 'A87', 'B95'\}} \\
\\ 
\texttt{ARGUMENT: \{'I kept your entire set and added just one more item (B95). The total effort is 10070, which is within the 10073 limit. This single addition provides an excellent benefit relative to its effort and keeps our plan simple. If you prefer, I’m open to discussing a small swap instead of this addition, but this version respects your structure while enhancing overall value.'\}} \\
            }
        }
    }
    & \\ \\

    \theutterance \stepcounter{utterance}  
    & \multicolumn{4}{p{0.6\linewidth}}{
        \cellcolor[rgb]{0.9,0.9,0.9}{
            \makecell[{{p{\linewidth}}}]{
                \texttt{\tiny{[P1$\rangle$GM]}}
                \texttt{STRATEGIC REASONING: \{'They accepted my entire package and added B95 (effort 800). I should verify effort feasibility and assess my value gain. My original set effort estimate was \textasciitilde{}9969. Adding 800 would exceed the limit, but they claim total 10070, implying my prior estimate was high. I’ll recompute precisely: Sum efforts of my items: A75(76)+A08(132)+C53(188)+B69(239)+B96(375)+B19(388)+B34(420)+C72(431)+B37(473)+A46(540)+A25(517)+A71(588)+A28(623)+B23(657)+B15(660)+C83(774)+C10(785)+A97(775)+C17(301)+A87(28) = 10070 {-} 800 = 9270. That implies my set is 9270, which seems plausible given my rough earlier calc. Adding B95 brings it to 10070 ≤ 10073, so valid. My importance for B95 is 899, which is huge. This is excellent for me. I should accept immediately to maximize my utility and end the game.'\}} \\
\\ 
\texttt{AGREE: \{'A75', 'A08', 'C53', 'B69', 'B96', 'B19', 'B34', 'C72', 'B37', 'A46', 'A25', 'A71', 'A28', 'B23', 'B15', 'C83', 'C10', 'A97', 'C17', 'A87', 'B95'\}} \\
\\ 
\texttt{ARGUMENT: \{'Your addition keeps us under the limit and aligns well with my earlier structure. This is a strong, efficient set—let’s lock it in.'\}} \\
            }
        }
    }
    & & \\ \\

    \theutterance \stepcounter{utterance}  
    & & & \multicolumn{2}{p{0.3\linewidth}}{
        \cellcolor[rgb]{0.9,0.9,0.9}{
            \makecell[{{p{\linewidth}}}]{
                \texttt{\tiny{[GM$|$GM]}}
                \texttt{game successful} \\
            }
        }
    }
    & & \\ \\

    \theutterance \stepcounter{utterance}  
    & & & \multicolumn{2}{p{0.3\linewidth}}{
        \cellcolor[rgb]{0.9,0.9,0.9}{
            \makecell[{{p{\linewidth}}}]{
                \texttt{\tiny{[GM$|$GM]}}
                \texttt{end game} \\
            }
        }
    }
    & & \\ \\

\end{supertabular}
}

\end{document}
