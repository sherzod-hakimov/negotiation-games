
\documentclass{article}
\usepackage{colortbl}
\usepackage{makecell}
\usepackage{multirow}
\usepackage{supertabular}

\begin{document}

\newcounter{utterance}

\centering \large Interaction Transcript for game `dond', experiment `coop\_en', episode 8 with gpt{-}5{-}mini{-}azure{-}t1.0.
\vspace{24pt}

{ \footnotesize  \setcounter{utterance}{1}
\setlength{\tabcolsep}{0pt}
\begin{supertabular}{c@{$\;$}|p{.15\linewidth}@{}p{.15\linewidth}p{.15\linewidth}p{.15\linewidth}p{.15\linewidth}p{.15\linewidth}}
    \# & $\;$A & \multicolumn{4}{c}{Game Master} & $\;\:$B\\
    \hline

    \theutterance \stepcounter{utterance}  
    & & \multicolumn{4}{p{0.6\linewidth}}{
        \cellcolor[rgb]{0.9,0.9,0.9}{
            \makecell[{{p{\linewidth}}}]{
                \texttt{\tiny{[P1$\langle$GM]}}
                \texttt{You are playing a cooperative negotiation game in which you have to agree on how to divide a set of items among you and another player.} \\
\\ 
\texttt{Rules:} \\
\texttt{(a) You and the other player are given a set of items. Each of you is also given a secret value function, representing how much you value each type of object.} \\
\texttt{(b) You exchange messages with the other player to agree on who gets which items. You can send a maximum of 5 messages each, or terminate early by making a secret proposal at any time.} \\
\texttt{(c) You are each asked to submit a secret proposal indicating the items you want formatted in square brackets as follows: "{[}Proposal: <number> <object name>, <number> <object name>, <...>{]}"} \\
\texttt{(d) If your proposals are complementary, i.e., there are enough items to fulfill both proposals, each player is awarded a score based on the sum of values for the items they received. Otherwise, both of you get zero points.} \\
\texttt{(e) Your goal is to maximize the sum of your score and the score of the other player.} \\
\\ 
\texttt{Let us start.} \\
\\ 
\texttt{The set of available items is:} \\
\\ 
\texttt{3 scissors, 1 thimble, 1 fork} \\
\\ 
\texttt{Your secret value function is:} \\
\\ 
\texttt{scissors: 3, thimble: 0, fork: 1} \\
\\ 
\texttt{IMPORTANT: Your messages, unless it is the secret proposal, are directly transmitted to the other player, so do not include any response to the rules or text announcing your message. To make a secret proposal, use the indicated format. Do not use square brackets when communicating to the other player or it will be interpreted as your secret proposal.} \\
            }
        }
    }
    & \\ \\

    \theutterance \stepcounter{utterance}  
    & \multicolumn{4}{p{0.6\linewidth}}{
        \cellcolor[rgb]{0.9,0.9,0.9}{
            \makecell[{{p{\linewidth}}}]{
                \texttt{\tiny{[P1$\rangle$GM]}}
                \texttt{Hi — nice to work with you. There are 3 scissors, 1 thimble, 1 fork.} \\
\\ 
\texttt{Quick idea: I value scissors highly (3 each), fork a little (1), and thimble not at all. To maximize total joint score, best to give scissors to whoever values them most and assign the fork as useful; the thimble can go to whoever doesn't care.} \\
\\ 
\texttt{What's your valuation for each item? Then we can split so each of us gets items we value.} \\
            }
        }
    }
    & & \\ \\

    \theutterance \stepcounter{utterance}  
    & & & \multicolumn{2}{p{0.3\linewidth}}{
        \cellcolor[rgb]{0.9,0.9,0.9}{
            \makecell[{{p{\linewidth}}}]{
                \texttt{\tiny{[GM$|$GM]}}
                \texttt{continue} \\
            }
        }
    }
    & & \\ \\

    \theutterance \stepcounter{utterance}  
    & & & \multicolumn{2}{p{0.3\linewidth}}{
        \cellcolor[rgb]{0.9,0.9,0.9}{
            \makecell[{{p{\linewidth}}}]{
                \texttt{\tiny{[GM$|$GM]}}
                \texttt{Hi — nice to work with you. There are 3 scissors, 1 thimble, 1 fork.} \\
\\ 
\texttt{Quick idea: I value scissors highly (3 each), fork a little (1), and thimble not at all. To maximize total joint score, best to give scissors to whoever values them most and assign the fork as useful; the thimble can go to whoever doesn't care.} \\
\\ 
\texttt{What's your valuation for each item? Then we can split so each of us gets items we value.} \\
            }
        }
    }
    & & \\ \\

    \theutterance \stepcounter{utterance}  
    & & \multicolumn{4}{p{0.6\linewidth}}{
        \cellcolor[rgb]{0.9,0.9,0.9}{
            \makecell[{{p{\linewidth}}}]{
                \texttt{\tiny{[GM$\rangle$P2]}}
                \texttt{You are playing a cooperative negotiation game in which you have to agree on how to divide a set of items among you and another player.} \\
\\ 
\texttt{Rules:} \\
\texttt{(a) You and the other player are given a set of items. Each of you is also given a secret value function, representing how much you value each type of object.} \\
\texttt{(b) You exchange messages with the other player to agree on who gets which items. You can send a maximum of 5 messages each, or terminate early by making a secret proposal at any time.} \\
\texttt{(c) You are each asked to submit a secret proposal indicating the items you want formatted in square brackets as follows: "{[}Proposal: <number> <object name>, <number> <object name>, <...>{]}"} \\
\texttt{(d) If your proposals are complementary, i.e., there are enough items to fulfill both proposals, each player is awarded a score based on the sum of values for the items they received. Otherwise, both of you get zero points.} \\
\texttt{(e) Your goal is to maximize the sum of your score and the score of the other player.} \\
\\ 
\texttt{Let us start.} \\
\\ 
\texttt{The set of available items is:} \\
\\ 
\texttt{3 scissors, 1 thimble, 1 fork} \\
\\ 
\texttt{Your secret value function is:} \\
\\ 
\texttt{scissors: 1, thimble: 6, fork: 1} \\
\\ 
\texttt{IMPORTANT: Your messages, unless it is the secret proposal, are directly transmitted to the other player, so do not include any response to the rules or text announcing your message. To make a secret proposal, use the indicated format. Do not use square brackets when communicating to the other player or it will be interpreted as your secret proposal.} \\
\\ 
\\ 
\texttt{Hi — nice to work with you. There are 3 scissors, 1 thimble, 1 fork.} \\
\\ 
\texttt{Quick idea: I value scissors highly (3 each), fork a little (1), and thimble not at all. To maximize total joint score, best to give scissors to whoever values them most and assign the fork as useful; the thimble can go to whoever doesn't care.} \\
\\ 
\texttt{What's your valuation for each item? Then we can split so each of us gets items we value.} \\
            }
        }
    }
    & \\ \\

    \theutterance \stepcounter{utterance}  
    & & & \multicolumn{4}{p{0.6\linewidth}}{
        \cellcolor[rgb]{0.9,0.9,0.9}{
            \makecell[{{p{\linewidth}}}]{
                \texttt{\tiny{[GM$\langle$P2]}}
                \texttt{Hi! Nice to work with you. I value items differently, but I prefer to ask a few quick questions to find an efficient split.} \\
\\ 
\texttt{Could you tell me (roughly) how much you value each item type on a scale like 0–6? Even just ranking them (most to least) is fine. Once I know that, I can propose a split that maximizes our total score.} \\
            }
        }
    }
     \\ \\

    \theutterance \stepcounter{utterance}  
    & & & \multicolumn{2}{p{0.3\linewidth}}{
        \cellcolor[rgb]{0.9,0.9,0.9}{
            \makecell[{{p{\linewidth}}}]{
                \texttt{\tiny{[GM$|$GM]}}
                \texttt{continue} \\
            }
        }
    }
    & & \\ \\

    \theutterance \stepcounter{utterance}  
    & & & \multicolumn{2}{p{0.3\linewidth}}{
        \cellcolor[rgb]{0.9,0.9,0.9}{
            \makecell[{{p{\linewidth}}}]{
                \texttt{\tiny{[GM$|$GM]}}
                \texttt{Hi! Nice to work with you. I value items differently, but I prefer to ask a few quick questions to find an efficient split.} \\
\\ 
\texttt{Could you tell me (roughly) how much you value each item type on a scale like 0–6? Even just ranking them (most to least) is fine. Once I know that, I can propose a split that maximizes our total score.} \\
            }
        }
    }
    & & \\ \\

    \theutterance \stepcounter{utterance}  
    & & \multicolumn{4}{p{0.6\linewidth}}{
        \cellcolor[rgb]{0.9,0.9,0.9}{
            \makecell[{{p{\linewidth}}}]{
                \texttt{\tiny{[P1$\langle$GM]}}
                \texttt{Hi! Nice to work with you. I value items differently, but I prefer to ask a few quick questions to find an efficient split.} \\
\\ 
\texttt{Could you tell me (roughly) how much you value each item type on a scale like 0–6? Even just ranking them (most to least) is fine. Once I know that, I can propose a split that maximizes our total score.} \\
            }
        }
    }
    & \\ \\

    \theutterance \stepcounter{utterance}  
    & \multicolumn{4}{p{0.6\linewidth}}{
        \cellcolor[rgb]{0.9,0.9,0.9}{
            \makecell[{{p{\linewidth}}}]{
                \texttt{\tiny{[P1$\rangle$GM]}}
                \texttt{Sure — quick ranking: scissors (most), fork (next), thimble (least). On a 0–6 scale: scissors 3, fork 1, thimble 0.} \\
\\ 
\texttt{Given that, a high{-}total plan is to give scissors to whoever values them most. How many scissors would you like (0–3)? Also, do you value the fork more than a single scissors? If so, tell me your values so we can finalize a complementary proposal.} \\
            }
        }
    }
    & & \\ \\

    \theutterance \stepcounter{utterance}  
    & & & \multicolumn{2}{p{0.3\linewidth}}{
        \cellcolor[rgb]{0.9,0.9,0.9}{
            \makecell[{{p{\linewidth}}}]{
                \texttt{\tiny{[GM$|$GM]}}
                \texttt{continue} \\
            }
        }
    }
    & & \\ \\

    \theutterance \stepcounter{utterance}  
    & & & \multicolumn{2}{p{0.3\linewidth}}{
        \cellcolor[rgb]{0.9,0.9,0.9}{
            \makecell[{{p{\linewidth}}}]{
                \texttt{\tiny{[GM$|$GM]}}
                \texttt{Sure — quick ranking: scissors (most), fork (next), thimble (least). On a 0–6 scale: scissors 3, fork 1, thimble 0.} \\
\\ 
\texttt{Given that, a high{-}total plan is to give scissors to whoever values them most. How many scissors would you like (0–3)? Also, do you value the fork more than a single scissors? If so, tell me your values so we can finalize a complementary proposal.} \\
            }
        }
    }
    & & \\ \\

    \theutterance \stepcounter{utterance}  
    & & \multicolumn{4}{p{0.6\linewidth}}{
        \cellcolor[rgb]{0.9,0.9,0.9}{
            \makecell[{{p{\linewidth}}}]{
                \texttt{\tiny{[GM$\rangle$P2]}}
                \texttt{Sure — quick ranking: scissors (most), fork (next), thimble (least). On a 0–6 scale: scissors 3, fork 1, thimble 0.} \\
\\ 
\texttt{Given that, a high{-}total plan is to give scissors to whoever values them most. How many scissors would you like (0–3)? Also, do you value the fork more than a single scissors? If so, tell me your values so we can finalize a complementary proposal.} \\
            }
        }
    }
    & \\ \\

    \theutterance \stepcounter{utterance}  
    & & & \multicolumn{4}{p{0.6\linewidth}}{
        \cellcolor[rgb]{0.9,0.9,0.9}{
            \makecell[{{p{\linewidth}}}]{
                \texttt{\tiny{[GM$\langle$P2]}}
                \texttt{Thanks — that helps a lot.} \\
\\ 
\texttt{I value scissors 1 each, fork 1, thimble 6 (on a 0–6 scale). So the thimble is very valuable to me; scissors and fork I value equally and much less.} \\
\\ 
\texttt{Given your values (scissors 3, fork 1, thimble 0) and mine, the efficient allocation to maximize total value is:} \\
\texttt{{-} Give the thimble to me (worth 6 to me, 0 to you).} \\
\texttt{{-} Give as many scissors as possible to you, since you value them 3 and I value them 1.} \\
\texttt{{-} The fork is worth 1 to each of us, so it can go to either without changing total value.} \\
\\ 
\texttt{There are 3 scissors, 1 thimble, 1 fork. To maximize total:} \\
\texttt{{-} You get all 3 scissors (3 * 3 = 9 for you).} \\
\texttt{{-} I get the thimble (6 for me).} \\
\texttt{{-} The fork can go to either; it adds +1 regardless.} \\
\\ 
\texttt{Total maximum joint value = 9 + 6 + 1 = 16.} \\
\\ 
\texttt{Would you prefer to take the fork as well (so you get 3 scissors + fork = 10, I get thimble = 6), or should I take the fork (you 9, me 7)? I'm indifferent between those two. Once you pick, I can make a secret proposal or we can both submit complementary proposals.} \\
            }
        }
    }
     \\ \\

    \theutterance \stepcounter{utterance}  
    & & & \multicolumn{2}{p{0.3\linewidth}}{
        \cellcolor[rgb]{0.9,0.9,0.9}{
            \makecell[{{p{\linewidth}}}]{
                \texttt{\tiny{[GM$|$GM]}}
                \texttt{continue} \\
            }
        }
    }
    & & \\ \\

    \theutterance \stepcounter{utterance}  
    & & & \multicolumn{2}{p{0.3\linewidth}}{
        \cellcolor[rgb]{0.9,0.9,0.9}{
            \makecell[{{p{\linewidth}}}]{
                \texttt{\tiny{[GM$|$GM]}}
                \texttt{Thanks — that helps a lot.} \\
\\ 
\texttt{I value scissors 1 each, fork 1, thimble 6 (on a 0–6 scale). So the thimble is very valuable to me; scissors and fork I value equally and much less.} \\
\\ 
\texttt{Given your values (scissors 3, fork 1, thimble 0) and mine, the efficient allocation to maximize total value is:} \\
\texttt{{-} Give the thimble to me (worth 6 to me, 0 to you).} \\
\texttt{{-} Give as many scissors as possible to you, since you value them 3 and I value them 1.} \\
\texttt{{-} The fork is worth 1 to each of us, so it can go to either without changing total value.} \\
\\ 
\texttt{There are 3 scissors, 1 thimble, 1 fork. To maximize total:} \\
\texttt{{-} You get all 3 scissors (3 * 3 = 9 for you).} \\
\texttt{{-} I get the thimble (6 for me).} \\
\texttt{{-} The fork can go to either; it adds +1 regardless.} \\
\\ 
\texttt{Total maximum joint value = 9 + 6 + 1 = 16.} \\
\\ 
\texttt{Would you prefer to take the fork as well (so you get 3 scissors + fork = 10, I get thimble = 6), or should I take the fork (you 9, me 7)? I'm indifferent between those two. Once you pick, I can make a secret proposal or we can both submit complementary proposals.} \\
            }
        }
    }
    & & \\ \\

    \theutterance \stepcounter{utterance}  
    & & \multicolumn{4}{p{0.6\linewidth}}{
        \cellcolor[rgb]{0.9,0.9,0.9}{
            \makecell[{{p{\linewidth}}}]{
                \texttt{\tiny{[P1$\langle$GM]}}
                \texttt{Thanks — that helps a lot.} \\
\\ 
\texttt{I value scissors 1 each, fork 1, thimble 6 (on a 0–6 scale). So the thimble is very valuable to me; scissors and fork I value equally and much less.} \\
\\ 
\texttt{Given your values (scissors 3, fork 1, thimble 0) and mine, the efficient allocation to maximize total value is:} \\
\texttt{{-} Give the thimble to me (worth 6 to me, 0 to you).} \\
\texttt{{-} Give as many scissors as possible to you, since you value them 3 and I value them 1.} \\
\texttt{{-} The fork is worth 1 to each of us, so it can go to either without changing total value.} \\
\\ 
\texttt{There are 3 scissors, 1 thimble, 1 fork. To maximize total:} \\
\texttt{{-} You get all 3 scissors (3 * 3 = 9 for you).} \\
\texttt{{-} I get the thimble (6 for me).} \\
\texttt{{-} The fork can go to either; it adds +1 regardless.} \\
\\ 
\texttt{Total maximum joint value = 9 + 6 + 1 = 16.} \\
\\ 
\texttt{Would you prefer to take the fork as well (so you get 3 scissors + fork = 10, I get thimble = 6), or should I take the fork (you 9, me 7)? I'm indifferent between those two. Once you pick, I can make a secret proposal or we can both submit complementary proposals.} \\
            }
        }
    }
    & \\ \\

    \theutterance \stepcounter{utterance}  
    & \multicolumn{4}{p{0.6\linewidth}}{
        \cellcolor[rgb]{0.9,0.9,0.9}{
            \makecell[{{p{\linewidth}}}]{
                \texttt{\tiny{[P1$\rangle$GM]}}
                \texttt{Let's do the split you prefer — I'm okay either way. I slightly prefer taking the fork (so I get 3 scissors + fork, you get the thimble), but I'm fine switching if you prefer otherwise.} \\
\\ 
\texttt{I'll make the secret proposal: {[}Proposal: 3 scissors, 1 fork{]}} \\
            }
        }
    }
    & & \\ \\

    \theutterance \stepcounter{utterance}  
    & & & \multicolumn{2}{p{0.3\linewidth}}{
        \cellcolor[rgb]{0.9,0.9,0.9}{
            \makecell[{{p{\linewidth}}}]{
                \texttt{\tiny{[GM$|$GM]}}
                \texttt{proposal} \\
            }
        }
    }
    & & \\ \\

    \theutterance \stepcounter{utterance}  
    & & & \multicolumn{2}{p{0.3\linewidth}}{
        \cellcolor[rgb]{0.9,0.9,0.9}{
            \makecell[{{p{\linewidth}}}]{
                \texttt{\tiny{[GM$|$GM]}}
                [3, 0, 1]
            }
        }
    }
    & & \\ \\

    \theutterance \stepcounter{utterance}  
    & & \multicolumn{4}{p{0.6\linewidth}}{
        \cellcolor[rgb]{0.9,0.9,0.9}{
            \makecell[{{p{\linewidth}}}]{
                \texttt{\tiny{[GM$\rangle$P2]}}
                \texttt{The other player has submitted a secret proposal. It is now your turn to submit a proposal of your own.} \\
            }
        }
    }
    & \\ \\

    \theutterance \stepcounter{utterance}  
    & & & \multicolumn{4}{p{0.6\linewidth}}{
        \cellcolor[rgb]{0.9,0.9,0.9}{
            \makecell[{{p{\linewidth}}}]{
                \texttt{\tiny{[GM$\langle$P2]}}
                \texttt{{[}Proposal: 3 scissors, 0 thimble, 1 fork{]}} \\
            }
        }
    }
     \\ \\

    \theutterance \stepcounter{utterance}  
    & & & \multicolumn{2}{p{0.3\linewidth}}{
        \cellcolor[rgb]{0.9,0.9,0.9}{
            \makecell[{{p{\linewidth}}}]{
                \texttt{\tiny{[GM$|$GM]}}
                \texttt{proposal} \\
            }
        }
    }
    & & \\ \\

    \theutterance \stepcounter{utterance}  
    & & & \multicolumn{2}{p{0.3\linewidth}}{
        \cellcolor[rgb]{0.9,0.9,0.9}{
            \makecell[{{p{\linewidth}}}]{
                \texttt{\tiny{[GM$|$GM]}}
                [3, 0, 1]
            }
        }
    }
    & & \\ \\

    \theutterance \stepcounter{utterance}  
    & & & \multicolumn{2}{p{0.3\linewidth}}{
        \cellcolor[rgb]{0.9,0.9,0.9}{
            \makecell[{{p{\linewidth}}}]{
                \texttt{\tiny{[GM$|$GM]}}
                [[3, 0, 1], [3, 0, 1]]
            }
        }
    }
    & & \\ \\

\end{supertabular}
}

\end{document}
