
\documentclass{article}
\usepackage{colortbl}
\usepackage{makecell}
\usepackage{multirow}
\usepackage{supertabular}

\begin{document}

\newcounter{utterance}

\twocolumn

{ \footnotesize  \setcounter{utterance}{1}
\setlength{\tabcolsep}{0pt}
\begin{supertabular}{c@{$\;$}|p{.15\linewidth}@{}p{.15\linewidth}p{.15\linewidth}p{.15\linewidth}p{.15\linewidth}p{.15\linewidth}}

    \# & $\;$A & \multicolumn{4}{c}{Game Master} & $\;\:$B\\
    \hline 

    \theutterance \stepcounter{utterance}  

    & & \multicolumn{4}{p{0.6\linewidth}}{\cellcolor[rgb]{0.9,0.9,0.9}{%
	\makecell[{{p{\linewidth}}}]{% 
	  \tt {\tiny [A$\langle$GM]}  
	 You are playing a cooperative negotiation game in which you have to agree on how to divide a set of items among you and another player.\\ \tt \\ \tt Rules:\\ \tt (a) You and the other player are given a set of items. Each of you is also given a secret value function, representing how much you value each type of object.\\ \tt (b) You exchange messages with the other player to agree on who gets which items. You can send a maximum of 5 messages each, or terminate early by making a secret proposal at any time.\\ \tt (c) You are each asked to submit a secret proposal indicating the items you want formatted in square brackets as follows: "[Proposal: <number> <object name>, <number> <object name>, <...>]"\\ \tt (d) If your proposals are complementary, i.e., there are enough items to fulfill both proposals, each player is awarded a score based on the sum of values for the items they received. Otherwise, both of you get zero points.\\ \tt (e) Your goal is to maximize the sum of your score and the score of the other player.\\ \tt \\ \tt Let us start.\\ \tt \\ \tt The set of available items is:\\ \tt \\ \tt 1 pen, 1 flower, 3 frames\\ \tt \\ \tt Your secret value function is:\\ \tt \\ \tt pen: 4, flower: 0, frame: 2\\ \tt \\ \tt IMPORTANT: Your messages, unless it is the secret proposal, are directly transmitted to the other player, so do not include any response to the rules or text announcing your message. To make a secret proposal, use the indicated format. Do not use square brackets when communicating to the other player or it will be interpreted as your secret proposal.\\ \tt  
	  } 
	   } 
	   } 
	 & \\ 
 

    \theutterance \stepcounter{utterance}  

    & \multicolumn{4}{p{0.6\linewidth}}{\cellcolor[rgb]{0.8,1,0.9}{%
	\makecell[{{p{\linewidth}}}]{% 
	  \tt {\tiny [A$\rangle$GM]}  
	 Hi — nice to play together! Quick check: we each get some subset of the items (1 pen, 1 flower, 3 frames). I care about the pen (value 4) and frames (value 2 each); flower is worthless to me. What's most important to you? If you tell me your preferences, we can split to maximize our combined score. 
	  } 
	   } 
	   } 
	 & & \\ 
 

    \theutterance \stepcounter{utterance}  

    & & & \multicolumn{2}{p{0.3\linewidth}}{\cellcolor[rgb]{0.95,0.95,0.95}{%
	\makecell[{{p{\linewidth}}}]{% 
	  \tt {\tiny [GM$|$GM]}  
	 continue 
	  } 
	   } 
	   } 
	 & & \\ 
 

    \theutterance \stepcounter{utterance}  

    & & & \multicolumn{2}{p{0.3\linewidth}}{\cellcolor[rgb]{0.95,0.95,0.95}{%
	\makecell[{{p{\linewidth}}}]{% 
	  \tt {\tiny [GM$|$GM]}  
	 Hi — nice to play together! Quick check: we each get some subset of the items (1 pen, 1 flower, 3 frames). I care about the pen (value 4) and frames (value 2 each); flower is worthless to me. What's most important to you? If you tell me your preferences, we can split to maximize our combined score. 
	  } 
	   } 
	   } 
	 & & \\ 
 

    \theutterance \stepcounter{utterance}  

    & & \multicolumn{4}{p{0.6\linewidth}}{\cellcolor[rgb]{0.9,0.9,0.9}{%
	\makecell[{{p{\linewidth}}}]{% 
	  \tt {\tiny [GM$\rangle$B]}  
	 You are playing a cooperative negotiation game in which you have to agree on how to divide a set of items among you and another player.\\ \tt \\ \tt Rules:\\ \tt (a) You and the other player are given a set of items. Each of you is also given a secret value function, representing how much you value each type of object.\\ \tt (b) You exchange messages with the other player to agree on who gets which items. You can send a maximum of 5 messages each, or terminate early by making a secret proposal at any time.\\ \tt (c) You are each asked to submit a secret proposal indicating the items you want formatted in square brackets as follows: "[Proposal: <number> <object name>, <number> <object name>, <...>]"\\ \tt (d) If your proposals are complementary, i.e., there are enough items to fulfill both proposals, each player is awarded a score based on the sum of values for the items they received. Otherwise, both of you get zero points.\\ \tt (e) Your goal is to maximize the sum of your score and the score of the other player.\\ \tt \\ \tt Let us start.\\ \tt \\ \tt The set of available items is:\\ \tt \\ \tt 1 pen, 1 flower, 3 frames\\ \tt \\ \tt Your secret value function is:\\ \tt \\ \tt pen: 2, flower: 5, frame: 1\\ \tt \\ \tt IMPORTANT: Your messages, unless it is the secret proposal, are directly transmitted to the other player, so do not include any response to the rules or text announcing your message. To make a secret proposal, use the indicated format. Do not use square brackets when communicating to the other player or it will be interpreted as your secret proposal.\\ \tt \\ \tt \\ \tt Hi — nice to play together! Quick check: we each get some subset of the items (1 pen, 1 flower, 3 frames). I care about the pen (value 4) and frames (value 2 each); flower is worthless to me. What's most important to you? If you tell me your preferences, we can split to maximize our combined score. 
	  } 
	   } 
	   } 
	 & \\ 
 

    \theutterance \stepcounter{utterance}  

    & & & \multicolumn{4}{p{0.6\linewidth}}{\cellcolor[rgb]{1,0.85,0.72}{%
	\makecell[{{p{\linewidth}}}]{% 
	  \tt {\tiny [GM$\langle$B]}  
	 Nice to meet you! I care most about the flower and then the pen; frames are least important to me. If we want to maximize combined value, a simple efficient split would be for you to take the item I value most (the flower) and I take items you value most (the pen and possibly frames) depending on how we score.\\ \tt \\ \tt One proposal that seems likely to be Pareto-optimal: you take the flower; I take the pen and one or two frames depending on your needs. What would you prefer? 
	  } 
	   } 
	   } 
	  \\ 
 

    \theutterance \stepcounter{utterance}  

    & & & \multicolumn{2}{p{0.3\linewidth}}{\cellcolor[rgb]{0.95,0.95,0.95}{%
	\makecell[{{p{\linewidth}}}]{% 
	  \tt {\tiny [GM$|$GM]}  
	 continue 
	  } 
	   } 
	   } 
	 & & \\ 
 

    \theutterance \stepcounter{utterance}  

    & & & \multicolumn{2}{p{0.3\linewidth}}{\cellcolor[rgb]{0.95,0.95,0.95}{%
	\makecell[{{p{\linewidth}}}]{% 
	  \tt {\tiny [GM$|$GM]}  
	 Nice to meet you! I care most about the flower and then the pen; frames are least important to me. If we want to maximize combined value, a simple efficient split would be for you to take the item I value most (the flower) and I take items you value most (the pen and possibly frames) depending on how we score.\\ \tt \\ \tt One proposal that seems likely to be Pareto-optimal: you take the flower; I take the pen and one or two frames depending on your needs. What would you prefer? 
	  } 
	   } 
	   } 
	 & & \\ 
 

    \theutterance \stepcounter{utterance}  

    & & \multicolumn{4}{p{0.6\linewidth}}{\cellcolor[rgb]{0.9,0.9,0.9}{%
	\makecell[{{p{\linewidth}}}]{% 
	  \tt {\tiny [A$\langle$GM]}  
	 Nice to meet you! I care most about the flower and then the pen; frames are least important to me. If we want to maximize combined value, a simple efficient split would be for you to take the item I value most (the flower) and I take items you value most (the pen and possibly frames) depending on how we score.\\ \tt \\ \tt One proposal that seems likely to be Pareto-optimal: you take the flower; I take the pen and one or two frames depending on your needs. What would you prefer? 
	  } 
	   } 
	   } 
	 & \\ 
 

    \theutterance \stepcounter{utterance}  

    & \multicolumn{4}{p{0.6\linewidth}}{\cellcolor[rgb]{0.8,1,0.9}{%
	\makecell[{{p{\linewidth}}}]{% 
	  \tt {\tiny [A$\rangle$GM]}  
	 Thanks — that sounds cooperative. I value the pen (4) and frames (2 each); I don't care about the flower. So the efficient allocation is: you get the flower, I get the pen, and the frames can be split to maximize total value.\\ \tt \\ \tt Total item values from my side: pen 4 + 3 frames *2 = 6 -> total 10 (flower 0). From your description you value the flower most and pen next; frames least. To maximize combined score, all items should be allocated (no waste). I prefer taking the pen and one frame (pen + 1 frame = 6 for me), leaving you the flower and two frames. But if you value frames very little, giving me extra frames increases total only if you value them less than I do — which you implied.\\ \tt \\ \tt Are you happy with: you get the flower + 1 frame + 1 frame (2 frames), and I get the pen + 1 frame? If yes, shall we lock that in? If you prefer fewer frames for me, tell me and we can adjust. 
	  } 
	   } 
	   } 
	 & & \\ 
 

    \theutterance \stepcounter{utterance}  

    & & & \multicolumn{2}{p{0.3\linewidth}}{\cellcolor[rgb]{0.95,0.95,0.95}{%
	\makecell[{{p{\linewidth}}}]{% 
	  \tt {\tiny [GM$|$GM]}  
	 continue 
	  } 
	   } 
	   } 
	 & & \\ 
 

    \theutterance \stepcounter{utterance}  

    & & & \multicolumn{2}{p{0.3\linewidth}}{\cellcolor[rgb]{0.95,0.95,0.95}{%
	\makecell[{{p{\linewidth}}}]{% 
	  \tt {\tiny [GM$|$GM]}  
	 Thanks — that sounds cooperative. I value the pen (4) and frames (2 each); I don't care about the flower. So the efficient allocation is: you get the flower, I get the pen, and the frames can be split to maximize total value.\\ \tt \\ \tt Total item values from my side: pen 4 + 3 frames *2 = 6 -> total 10 (flower 0). From your description you value the flower most and pen next; frames least. To maximize combined score, all items should be allocated (no waste). I prefer taking the pen and one frame (pen + 1 frame = 6 for me), leaving you the flower and two frames. But if you value frames very little, giving me extra frames increases total only if you value them less than I do — which you implied.\\ \tt \\ \tt Are you happy with: you get the flower + 1 frame + 1 frame (2 frames), and I get the pen + 1 frame? If yes, shall we lock that in? If you prefer fewer frames for me, tell me and we can adjust. 
	  } 
	   } 
	   } 
	 & & \\ 
 

    \theutterance \stepcounter{utterance}  

    & & \multicolumn{4}{p{0.6\linewidth}}{\cellcolor[rgb]{0.9,0.9,0.9}{%
	\makecell[{{p{\linewidth}}}]{% 
	  \tt {\tiny [GM$\rangle$B]}  
	 Thanks — that sounds cooperative. I value the pen (4) and frames (2 each); I don't care about the flower. So the efficient allocation is: you get the flower, I get the pen, and the frames can be split to maximize total value.\\ \tt \\ \tt Total item values from my side: pen 4 + 3 frames *2 = 6 -> total 10 (flower 0). From your description you value the flower most and pen next; frames least. To maximize combined score, all items should be allocated (no waste). I prefer taking the pen and one frame (pen + 1 frame = 6 for me), leaving you the flower and two frames. But if you value frames very little, giving me extra frames increases total only if you value them less than I do — which you implied.\\ \tt \\ \tt Are you happy with: you get the flower + 1 frame + 1 frame (2 frames), and I get the pen + 1 frame? If yes, shall we lock that in? If you prefer fewer frames for me, tell me and we can adjust. 
	  } 
	   } 
	   } 
	 & \\ 
 

    \theutterance \stepcounter{utterance}  

    & & & \multicolumn{4}{p{0.6\linewidth}}{\cellcolor[rgb]{1,0.85,0.72}{%
	\makecell[{{p{\linewidth}}}]{% 
	  \tt {\tiny [GM$\langle$B]}  
	 That split (you: pen + 1 frame; me: flower + 2 frames) sounds good and efficient. Let's lock it in.\\ \tt \\ \tt [Proposal: 1 pen, 1 frame] 
	  } 
	   } 
	   } 
	  \\ 
 

    \theutterance \stepcounter{utterance}  

    & & & \multicolumn{2}{p{0.3\linewidth}}{\cellcolor[rgb]{0.95,0.95,0.95}{%
	\makecell[{{p{\linewidth}}}]{% 
	  \tt {\tiny [GM$|$GM]}  
	 proposal 
	  } 
	   } 
	   } 
	 & & \\ 
 

    \theutterance \stepcounter{utterance}  

    & & & \multicolumn{2}{p{0.3\linewidth}}{\cellcolor[rgb]{0.95,0.95,0.95}{%
	\makecell[{{p{\linewidth}}}]{% 
	  \tt {\tiny [GM$|$GM]}  
	 [1, 0, 1] 
	  } 
	   } 
	   } 
	 & & \\ 
 

    \theutterance \stepcounter{utterance}  

    & & \multicolumn{4}{p{0.6\linewidth}}{\cellcolor[rgb]{0.9,0.9,0.9}{%
	\makecell[{{p{\linewidth}}}]{% 
	  \tt {\tiny [A$\langle$GM]}  
	 The other player has submitted a secret proposal. It is now your turn to submit a proposal of your own.\\ \tt  
	  } 
	   } 
	   } 
	 & \\ 
 

    \theutterance \stepcounter{utterance}  

    & \multicolumn{4}{p{0.6\linewidth}}{\cellcolor[rgb]{0.8,1,0.9}{%
	\makecell[{{p{\linewidth}}}]{% 
	  \tt {\tiny [A$\rangle$GM]}  
	 [Proposal: 1 pen, 1 frame] 
	  } 
	   } 
	   } 
	 & & \\ 
 

    \theutterance \stepcounter{utterance}  

    & & & \multicolumn{2}{p{0.3\linewidth}}{\cellcolor[rgb]{0.95,0.95,0.95}{%
	\makecell[{{p{\linewidth}}}]{% 
	  \tt {\tiny [GM$|$GM]}  
	 proposal 
	  } 
	   } 
	   } 
	 & & \\ 
 

    \theutterance \stepcounter{utterance}  

    & & & \multicolumn{2}{p{0.3\linewidth}}{\cellcolor[rgb]{0.95,0.95,0.95}{%
	\makecell[{{p{\linewidth}}}]{% 
	  \tt {\tiny [GM$|$GM]}  
	 [1, 0, 1] 
	  } 
	   } 
	   } 
	 & & \\ 
 

    \theutterance \stepcounter{utterance}  

    & & & \multicolumn{2}{p{0.3\linewidth}}{\cellcolor[rgb]{0.95,0.95,0.95}{%
	\makecell[{{p{\linewidth}}}]{% 
	  \tt {\tiny [GM$|$GM]}  
	 [[1, 0, 1], [1, 0, 1]] 
	  } 
	   } 
	   } 
	 & & \\ 
 

\end{supertabular}
}

\end{document}
