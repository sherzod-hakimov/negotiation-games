
\documentclass{article}
\usepackage{colortbl}
\usepackage{makecell}
\usepackage{multirow}
\usepackage{supertabular}

\begin{document}

\newcounter{utterance}

\centering \large Interaction Transcript for game `dond', experiment `coop\_en', episode 18 with gpt{-}5{-}mini{-}azure{-}t1.0.
\vspace{24pt}

{ \footnotesize  \setcounter{utterance}{1}
\setlength{\tabcolsep}{0pt}
\begin{supertabular}{c@{$\;$}|p{.15\linewidth}@{}p{.15\linewidth}p{.15\linewidth}p{.15\linewidth}p{.15\linewidth}p{.15\linewidth}}
    \# & $\;$A & \multicolumn{4}{c}{Game Master} & $\;\:$B\\
    \hline

    \theutterance \stepcounter{utterance}  
    & & \multicolumn{4}{p{0.6\linewidth}}{
        \cellcolor[rgb]{0.9,0.9,0.9}{
            \makecell[{{p{\linewidth}}}]{
                \texttt{\tiny{[P1$\langle$GM]}}
                \texttt{You are playing a cooperative negotiation game in which you have to agree on how to divide a set of items among you and another player.} \\
\\ 
\texttt{Rules:} \\
\texttt{(a) You and the other player are given a set of items. Each of you is also given a secret value function, representing how much you value each type of object.} \\
\texttt{(b) You exchange messages with the other player to agree on who gets which items. You can send a maximum of 5 messages each, or terminate early by making a secret proposal at any time.} \\
\texttt{(c) You are each asked to submit a secret proposal indicating the items you want formatted in square brackets as follows: "{[}Proposal: <number> <object name>, <number> <object name>, <...>{]}"} \\
\texttt{(d) If your proposals are complementary, i.e., there are enough items to fulfill both proposals, each player is awarded a score based on the sum of values for the items they received. Otherwise, both of you get zero points.} \\
\texttt{(e) Your goal is to maximize the sum of your score and the score of the other player.} \\
\\ 
\texttt{Let us start.} \\
\\ 
\texttt{The set of available items is:} \\
\\ 
\texttt{1 coin, 2 envelopes, 2 potatoes} \\
\\ 
\texttt{Your secret value function is:} \\
\\ 
\texttt{coin: 2, envelope: 2, potato: 2} \\
\\ 
\texttt{IMPORTANT: Your messages, unless it is the secret proposal, are directly transmitted to the other player, so do not include any response to the rules or text announcing your message. To make a secret proposal, use the indicated format. Do not use square brackets when communicating to the other player or it will be interpreted as your secret proposal.} \\
            }
        }
    }
    & \\ \\

    \theutterance \stepcounter{utterance}  
    & \multicolumn{4}{p{0.6\linewidth}}{
        \cellcolor[rgb]{0.9,0.9,0.9}{
            \makecell[{{p{\linewidth}}}]{
                \texttt{\tiny{[P1$\rangle$GM]}}
                \texttt{Hello — let's split everything so both of us get full value. There are 1 coin, 2 envelopes, 2 potatoes. I value each item equally. I propose we split items so each gets 1 coin? (there's only one) Maybe one of us takes the coin plus one potato and the other takes 2 envelopes and one potato, or similar. I prefer an even{-}value split. What do you think?} \\
            }
        }
    }
    & & \\ \\

    \theutterance \stepcounter{utterance}  
    & & & \multicolumn{2}{p{0.3\linewidth}}{
        \cellcolor[rgb]{0.9,0.9,0.9}{
            \makecell[{{p{\linewidth}}}]{
                \texttt{\tiny{[GM$|$GM]}}
                \texttt{continue} \\
            }
        }
    }
    & & \\ \\

    \theutterance \stepcounter{utterance}  
    & & & \multicolumn{2}{p{0.3\linewidth}}{
        \cellcolor[rgb]{0.9,0.9,0.9}{
            \makecell[{{p{\linewidth}}}]{
                \texttt{\tiny{[GM$|$GM]}}
                \texttt{Hello — let's split everything so both of us get full value. There are 1 coin, 2 envelopes, 2 potatoes. I value each item equally. I propose we split items so each gets 1 coin? (there's only one) Maybe one of us takes the coin plus one potato and the other takes 2 envelopes and one potato, or similar. I prefer an even{-}value split. What do you think?} \\
            }
        }
    }
    & & \\ \\

    \theutterance \stepcounter{utterance}  
    & & \multicolumn{4}{p{0.6\linewidth}}{
        \cellcolor[rgb]{0.9,0.9,0.9}{
            \makecell[{{p{\linewidth}}}]{
                \texttt{\tiny{[GM$\rangle$P2]}}
                \texttt{You are playing a cooperative negotiation game in which you have to agree on how to divide a set of items among you and another player.} \\
\\ 
\texttt{Rules:} \\
\texttt{(a) You and the other player are given a set of items. Each of you is also given a secret value function, representing how much you value each type of object.} \\
\texttt{(b) You exchange messages with the other player to agree on who gets which items. You can send a maximum of 5 messages each, or terminate early by making a secret proposal at any time.} \\
\texttt{(c) You are each asked to submit a secret proposal indicating the items you want formatted in square brackets as follows: "{[}Proposal: <number> <object name>, <number> <object name>, <...>{]}"} \\
\texttt{(d) If your proposals are complementary, i.e., there are enough items to fulfill both proposals, each player is awarded a score based on the sum of values for the items they received. Otherwise, both of you get zero points.} \\
\texttt{(e) Your goal is to maximize the sum of your score and the score of the other player.} \\
\\ 
\texttt{Let us start.} \\
\\ 
\texttt{The set of available items is:} \\
\\ 
\texttt{1 coin, 2 envelopes, 2 potatoes} \\
\\ 
\texttt{Your secret value function is:} \\
\\ 
\texttt{coin: 2, envelope: 1, potato: 3} \\
\\ 
\texttt{IMPORTANT: Your messages, unless it is the secret proposal, are directly transmitted to the other player, so do not include any response to the rules or text announcing your message. To make a secret proposal, use the indicated format. Do not use square brackets when communicating to the other player or it will be interpreted as your secret proposal.} \\
\\ 
\\ 
\texttt{Hello — let's split everything so both of us get full value. There are 1 coin, 2 envelopes, 2 potatoes. I value each item equally. I propose we split items so each gets 1 coin? (there's only one) Maybe one of us takes the coin plus one potato and the other takes 2 envelopes and one potato, or similar. I prefer an even{-}value split. What do you think?} \\
            }
        }
    }
    & \\ \\

    \theutterance \stepcounter{utterance}  
    & & & \multicolumn{4}{p{0.6\linewidth}}{
        \cellcolor[rgb]{0.9,0.9,0.9}{
            \makecell[{{p{\linewidth}}}]{
                \texttt{\tiny{[GM$\langle$P2]}}
                \texttt{I agree — let's make sure all items are allocated so both of us get high total value. Given there are 1 coin, 2 envelopes, 2 potatoes, a natural full split is one player takes the coin + 1 potato, the other takes 2 envelopes + 1 potato. That uses everything.} \\
\\ 
\texttt{From my values (coin 2, envelope 1, potato 3), that split gives:} \\
\texttt{{-} coin + 1 potato = 2 + 3 = 5} \\
\texttt{{-} 2 envelopes + 1 potato = 1 + 1 + 3 = 5} \\
\texttt{Total 10, perfectly even for me.} \\
\\ 
\texttt{Would you be okay with that allocation (coin+potato vs. 2 envelopes+potato)? If you prefer a different assignment of which of us gets which bundle, or a swap (e.g., I take the envelopes bundle and you take the coin bundle), tell me and we can finalize.} \\
            }
        }
    }
     \\ \\

    \theutterance \stepcounter{utterance}  
    & & & \multicolumn{2}{p{0.3\linewidth}}{
        \cellcolor[rgb]{0.9,0.9,0.9}{
            \makecell[{{p{\linewidth}}}]{
                \texttt{\tiny{[GM$|$GM]}}
                \texttt{continue} \\
            }
        }
    }
    & & \\ \\

    \theutterance \stepcounter{utterance}  
    & & & \multicolumn{2}{p{0.3\linewidth}}{
        \cellcolor[rgb]{0.9,0.9,0.9}{
            \makecell[{{p{\linewidth}}}]{
                \texttt{\tiny{[GM$|$GM]}}
                \texttt{I agree — let's make sure all items are allocated so both of us get high total value. Given there are 1 coin, 2 envelopes, 2 potatoes, a natural full split is one player takes the coin + 1 potato, the other takes 2 envelopes + 1 potato. That uses everything.} \\
\\ 
\texttt{From my values (coin 2, envelope 1, potato 3), that split gives:} \\
\texttt{{-} coin + 1 potato = 2 + 3 = 5} \\
\texttt{{-} 2 envelopes + 1 potato = 1 + 1 + 3 = 5} \\
\texttt{Total 10, perfectly even for me.} \\
\\ 
\texttt{Would you be okay with that allocation (coin+potato vs. 2 envelopes+potato)? If you prefer a different assignment of which of us gets which bundle, or a swap (e.g., I take the envelopes bundle and you take the coin bundle), tell me and we can finalize.} \\
            }
        }
    }
    & & \\ \\

    \theutterance \stepcounter{utterance}  
    & & \multicolumn{4}{p{0.6\linewidth}}{
        \cellcolor[rgb]{0.9,0.9,0.9}{
            \makecell[{{p{\linewidth}}}]{
                \texttt{\tiny{[P1$\langle$GM]}}
                \texttt{I agree — let's make sure all items are allocated so both of us get high total value. Given there are 1 coin, 2 envelopes, 2 potatoes, a natural full split is one player takes the coin + 1 potato, the other takes 2 envelopes + 1 potato. That uses everything.} \\
\\ 
\texttt{From my values (coin 2, envelope 1, potato 3), that split gives:} \\
\texttt{{-} coin + 1 potato = 2 + 3 = 5} \\
\texttt{{-} 2 envelopes + 1 potato = 1 + 1 + 3 = 5} \\
\texttt{Total 10, perfectly even for me.} \\
\\ 
\texttt{Would you be okay with that allocation (coin+potato vs. 2 envelopes+potato)? If you prefer a different assignment of which of us gets which bundle, or a swap (e.g., I take the envelopes bundle and you take the coin bundle), tell me and we can finalize.} \\
            }
        }
    }
    & \\ \\

    \theutterance \stepcounter{utterance}  
    & \multicolumn{4}{p{0.6\linewidth}}{
        \cellcolor[rgb]{0.9,0.9,0.9}{
            \makecell[{{p{\linewidth}}}]{
                \texttt{\tiny{[P1$\rangle$GM]}}
                \texttt{That split works perfectly for me. With my values (2 each), coin+potato = 4 and 2 envelopes+potato = 6, total 10 — all items allocated.} \\
\\ 
\texttt{I don't have a preference over who gets which bundle, so let's assign as you suggested: you take 2 envelopes + 1 potato, and I'll take the coin + 1 potato.} \\
\\ 
\texttt{I'll finalize with my secret proposal now.} \\
\\ 
\texttt{{[}Proposal: 1 coin, 1 potato{]}} \\
            }
        }
    }
    & & \\ \\

    \theutterance \stepcounter{utterance}  
    & & & \multicolumn{2}{p{0.3\linewidth}}{
        \cellcolor[rgb]{0.9,0.9,0.9}{
            \makecell[{{p{\linewidth}}}]{
                \texttt{\tiny{[GM$|$GM]}}
                \texttt{proposal} \\
            }
        }
    }
    & & \\ \\

    \theutterance \stepcounter{utterance}  
    & & & \multicolumn{2}{p{0.3\linewidth}}{
        \cellcolor[rgb]{0.9,0.9,0.9}{
            \makecell[{{p{\linewidth}}}]{
                \texttt{\tiny{[GM$|$GM]}}
                [1, 0, 1]
            }
        }
    }
    & & \\ \\

    \theutterance \stepcounter{utterance}  
    & & \multicolumn{4}{p{0.6\linewidth}}{
        \cellcolor[rgb]{0.9,0.9,0.9}{
            \makecell[{{p{\linewidth}}}]{
                \texttt{\tiny{[GM$\rangle$P2]}}
                \texttt{The other player has submitted a secret proposal. It is now your turn to submit a proposal of your own.} \\
            }
        }
    }
    & \\ \\

    \theutterance \stepcounter{utterance}  
    & & & \multicolumn{4}{p{0.6\linewidth}}{
        \cellcolor[rgb]{0.9,0.9,0.9}{
            \makecell[{{p{\linewidth}}}]{
                \texttt{\tiny{[GM$\langle$P2]}}
                \texttt{{[}Proposal: 1 coin, 1 potato{]}} \\
            }
        }
    }
     \\ \\

    \theutterance \stepcounter{utterance}  
    & & & \multicolumn{2}{p{0.3\linewidth}}{
        \cellcolor[rgb]{0.9,0.9,0.9}{
            \makecell[{{p{\linewidth}}}]{
                \texttt{\tiny{[GM$|$GM]}}
                \texttt{proposal} \\
            }
        }
    }
    & & \\ \\

    \theutterance \stepcounter{utterance}  
    & & & \multicolumn{2}{p{0.3\linewidth}}{
        \cellcolor[rgb]{0.9,0.9,0.9}{
            \makecell[{{p{\linewidth}}}]{
                \texttt{\tiny{[GM$|$GM]}}
                [1, 0, 1]
            }
        }
    }
    & & \\ \\

    \theutterance \stepcounter{utterance}  
    & & & \multicolumn{2}{p{0.3\linewidth}}{
        \cellcolor[rgb]{0.9,0.9,0.9}{
            \makecell[{{p{\linewidth}}}]{
                \texttt{\tiny{[GM$|$GM]}}
                [[1, 0, 1], [1, 0, 1]]
            }
        }
    }
    & & \\ \\

\end{supertabular}
}

\end{document}
