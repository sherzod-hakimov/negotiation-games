
\documentclass{article}
\usepackage{colortbl}
\usepackage{makecell}
\usepackage{multirow}
\usepackage{supertabular}

\begin{document}

\newcounter{utterance}

\centering \large Interaction Transcript for game `dond', experiment `coop\_en', episode 0 with gpt{-}5{-}mini{-}azure{-}t1.0.
\vspace{24pt}

{ \footnotesize  \setcounter{utterance}{1}
\setlength{\tabcolsep}{0pt}
\begin{supertabular}{c@{$\;$}|p{.15\linewidth}@{}p{.15\linewidth}p{.15\linewidth}p{.15\linewidth}p{.15\linewidth}p{.15\linewidth}}
    \# & $\;$A & \multicolumn{4}{c}{Game Master} & $\;\:$B\\
    \hline

    \theutterance \stepcounter{utterance}  
    & & \multicolumn{4}{p{0.6\linewidth}}{
        \cellcolor[rgb]{0.9,0.9,0.9}{
            \makecell[{{p{\linewidth}}}]{
                \texttt{\tiny{[P1$\langle$GM]}}
                \texttt{You are playing a cooperative negotiation game in which you have to agree on how to divide a set of items among you and another player.} \\
\\ 
\texttt{Rules:} \\
\texttt{(a) You and the other player are given a set of items. Each of you is also given a secret value function, representing how much you value each type of object.} \\
\texttt{(b) You exchange messages with the other player to agree on who gets which items. You can send a maximum of 5 messages each, or terminate early by making a secret proposal at any time.} \\
\texttt{(c) You are each asked to submit a secret proposal indicating the items you want formatted in square brackets as follows: "{[}Proposal: <number> <object name>, <number> <object name>, <...>{]}"} \\
\texttt{(d) If your proposals are complementary, i.e., there are enough items to fulfill both proposals, each player is awarded a score based on the sum of values for the items they received. Otherwise, both of you get zero points.} \\
\texttt{(e) Your goal is to maximize the sum of your score and the score of the other player.} \\
\\ 
\texttt{Let us start.} \\
\\ 
\texttt{The set of available items is:} \\
\\ 
\texttt{3 rings, 1 dice, 1 pan} \\
\\ 
\texttt{Your secret value function is:} \\
\\ 
\texttt{ring: 0, dice: 4, pan: 6} \\
\\ 
\texttt{IMPORTANT: Your messages, unless it is the secret proposal, are directly transmitted to the other player, so do not include any response to the rules or text announcing your message. To make a secret proposal, use the indicated format. Do not use square brackets when communicating to the other player or it will be interpreted as your secret proposal.} \\
            }
        }
    }
    & \\ \\

    \theutterance \stepcounter{utterance}  
    & \multicolumn{4}{p{0.6\linewidth}}{
        \cellcolor[rgb]{0.9,0.9,0.9}{
            \makecell[{{p{\linewidth}}}]{
                \texttt{\tiny{[P1$\rangle$GM]}}
                \texttt{Hi — I propose we split so both of us get something valuable. I value the pan most, then the dice; rings I don't care about. How do you value the items?} \\
            }
        }
    }
    & & \\ \\

    \theutterance \stepcounter{utterance}  
    & & & \multicolumn{2}{p{0.3\linewidth}}{
        \cellcolor[rgb]{0.9,0.9,0.9}{
            \makecell[{{p{\linewidth}}}]{
                \texttt{\tiny{[GM$|$GM]}}
                \texttt{continue} \\
            }
        }
    }
    & & \\ \\

    \theutterance \stepcounter{utterance}  
    & & & \multicolumn{2}{p{0.3\linewidth}}{
        \cellcolor[rgb]{0.9,0.9,0.9}{
            \makecell[{{p{\linewidth}}}]{
                \texttt{\tiny{[GM$|$GM]}}
                \texttt{Hi — I propose we split so both of us get something valuable. I value the pan most, then the dice; rings I don't care about. How do you value the items?} \\
            }
        }
    }
    & & \\ \\

    \theutterance \stepcounter{utterance}  
    & & \multicolumn{4}{p{0.6\linewidth}}{
        \cellcolor[rgb]{0.9,0.9,0.9}{
            \makecell[{{p{\linewidth}}}]{
                \texttt{\tiny{[GM$\rangle$P2]}}
                \texttt{You are playing a cooperative negotiation game in which you have to agree on how to divide a set of items among you and another player.} \\
\\ 
\texttt{Rules:} \\
\texttt{(a) You and the other player are given a set of items. Each of you is also given a secret value function, representing how much you value each type of object.} \\
\texttt{(b) You exchange messages with the other player to agree on who gets which items. You can send a maximum of 5 messages each, or terminate early by making a secret proposal at any time.} \\
\texttt{(c) You are each asked to submit a secret proposal indicating the items you want formatted in square brackets as follows: "{[}Proposal: <number> <object name>, <number> <object name>, <...>{]}"} \\
\texttt{(d) If your proposals are complementary, i.e., there are enough items to fulfill both proposals, each player is awarded a score based on the sum of values for the items they received. Otherwise, both of you get zero points.} \\
\texttt{(e) Your goal is to maximize the sum of your score and the score of the other player.} \\
\\ 
\texttt{Let us start.} \\
\\ 
\texttt{The set of available items is:} \\
\\ 
\texttt{3 rings, 1 dice, 1 pan} \\
\\ 
\texttt{Your secret value function is:} \\
\\ 
\texttt{ring: 3, dice: 1, pan: 0} \\
\\ 
\texttt{IMPORTANT: Your messages, unless it is the secret proposal, are directly transmitted to the other player, so do not include any response to the rules or text announcing your message. To make a secret proposal, use the indicated format. Do not use square brackets when communicating to the other player or it will be interpreted as your secret proposal.} \\
\\ 
\\ 
\texttt{Hi — I propose we split so both of us get something valuable. I value the pan most, then the dice; rings I don't care about. How do you value the items?} \\
            }
        }
    }
    & \\ \\

    \theutterance \stepcounter{utterance}  
    & & & \multicolumn{4}{p{0.6\linewidth}}{
        \cellcolor[rgb]{0.9,0.9,0.9}{
            \makecell[{{p{\linewidth}}}]{
                \texttt{\tiny{[GM$\langle$P2]}}
                \texttt{I actually value rings highly, dice a little, and the pan not at all. I think a fair efficient split would give you the pan if you want it, and we divide the rings and dice so neither of us is left with nothing. Would you prefer the pan (and perhaps 0–1 rings), or would you rather take rings and I take the pan?} \\
            }
        }
    }
     \\ \\

    \theutterance \stepcounter{utterance}  
    & & & \multicolumn{2}{p{0.3\linewidth}}{
        \cellcolor[rgb]{0.9,0.9,0.9}{
            \makecell[{{p{\linewidth}}}]{
                \texttt{\tiny{[GM$|$GM]}}
                \texttt{continue} \\
            }
        }
    }
    & & \\ \\

    \theutterance \stepcounter{utterance}  
    & & & \multicolumn{2}{p{0.3\linewidth}}{
        \cellcolor[rgb]{0.9,0.9,0.9}{
            \makecell[{{p{\linewidth}}}]{
                \texttt{\tiny{[GM$|$GM]}}
                \texttt{I actually value rings highly, dice a little, and the pan not at all. I think a fair efficient split would give you the pan if you want it, and we divide the rings and dice so neither of us is left with nothing. Would you prefer the pan (and perhaps 0–1 rings), or would you rather take rings and I take the pan?} \\
            }
        }
    }
    & & \\ \\

    \theutterance \stepcounter{utterance}  
    & & \multicolumn{4}{p{0.6\linewidth}}{
        \cellcolor[rgb]{0.9,0.9,0.9}{
            \makecell[{{p{\linewidth}}}]{
                \texttt{\tiny{[P1$\langle$GM]}}
                \texttt{I actually value rings highly, dice a little, and the pan not at all. I think a fair efficient split would give you the pan if you want it, and we divide the rings and dice so neither of us is left with nothing. Would you prefer the pan (and perhaps 0–1 rings), or would you rather take rings and I take the pan?} \\
            }
        }
    }
    & \\ \\

    \theutterance \stepcounter{utterance}  
    & \multicolumn{4}{p{0.6\linewidth}}{
        \cellcolor[rgb]{0.9,0.9,0.9}{
            \makecell[{{p{\linewidth}}}]{
                \texttt{\tiny{[P1$\rangle$GM]}}
                \texttt{Thanks — happy to have you take the rings if that yields a higher total for us. My values: pan 6, dice 4, rings 0. Your preference: rings high, dice some, pan none. To maximize total value, I should take the pan (6) and possibly also the dice (4) if that still lets you get enough rings to be satisfied. There are 3 rings and 1 dice.} \\
\\ 
\texttt{A straightforward efficient split that should make both of us better off:} \\
\texttt{{-} I take the pan and the dice (pan + dice = 6 + 4 = 10 for me).} \\
\texttt{{-} You take all 3 rings (your high value).} \\
\\ 
\texttt{That gives the maximum total value (10 + your value for 3 rings). If you prefer to keep the dice instead of one ring, we can adjust if that increases your value — but from my side the dice is worth 4, rings 0, so I strongly prefer the dice.} \\
\\ 
\texttt{Are you okay with me taking the pan and the dice while you take all 3 rings? If yes, we can submit complementary proposals to lock it in.} \\
            }
        }
    }
    & & \\ \\

    \theutterance \stepcounter{utterance}  
    & & & \multicolumn{2}{p{0.3\linewidth}}{
        \cellcolor[rgb]{0.9,0.9,0.9}{
            \makecell[{{p{\linewidth}}}]{
                \texttt{\tiny{[GM$|$GM]}}
                \texttt{continue} \\
            }
        }
    }
    & & \\ \\

    \theutterance \stepcounter{utterance}  
    & & & \multicolumn{2}{p{0.3\linewidth}}{
        \cellcolor[rgb]{0.9,0.9,0.9}{
            \makecell[{{p{\linewidth}}}]{
                \texttt{\tiny{[GM$|$GM]}}
                \texttt{Thanks — happy to have you take the rings if that yields a higher total for us. My values: pan 6, dice 4, rings 0. Your preference: rings high, dice some, pan none. To maximize total value, I should take the pan (6) and possibly also the dice (4) if that still lets you get enough rings to be satisfied. There are 3 rings and 1 dice.} \\
\\ 
\texttt{A straightforward efficient split that should make both of us better off:} \\
\texttt{{-} I take the pan and the dice (pan + dice = 6 + 4 = 10 for me).} \\
\texttt{{-} You take all 3 rings (your high value).} \\
\\ 
\texttt{That gives the maximum total value (10 + your value for 3 rings). If you prefer to keep the dice instead of one ring, we can adjust if that increases your value — but from my side the dice is worth 4, rings 0, so I strongly prefer the dice.} \\
\\ 
\texttt{Are you okay with me taking the pan and the dice while you take all 3 rings? If yes, we can submit complementary proposals to lock it in.} \\
            }
        }
    }
    & & \\ \\

    \theutterance \stepcounter{utterance}  
    & & \multicolumn{4}{p{0.6\linewidth}}{
        \cellcolor[rgb]{0.9,0.9,0.9}{
            \makecell[{{p{\linewidth}}}]{
                \texttt{\tiny{[GM$\rangle$P2]}}
                \texttt{Thanks — happy to have you take the rings if that yields a higher total for us. My values: pan 6, dice 4, rings 0. Your preference: rings high, dice some, pan none. To maximize total value, I should take the pan (6) and possibly also the dice (4) if that still lets you get enough rings to be satisfied. There are 3 rings and 1 dice.} \\
\\ 
\texttt{A straightforward efficient split that should make both of us better off:} \\
\texttt{{-} I take the pan and the dice (pan + dice = 6 + 4 = 10 for me).} \\
\texttt{{-} You take all 3 rings (your high value).} \\
\\ 
\texttt{That gives the maximum total value (10 + your value for 3 rings). If you prefer to keep the dice instead of one ring, we can adjust if that increases your value — but from my side the dice is worth 4, rings 0, so I strongly prefer the dice.} \\
\\ 
\texttt{Are you okay with me taking the pan and the dice while you take all 3 rings? If yes, we can submit complementary proposals to lock it in.} \\
            }
        }
    }
    & \\ \\

    \theutterance \stepcounter{utterance}  
    & & & \multicolumn{4}{p{0.6\linewidth}}{
        \cellcolor[rgb]{0.9,0.9,0.9}{
            \makecell[{{p{\linewidth}}}]{
                \texttt{\tiny{[GM$\langle$P2]}}
                \texttt{Yes — that split is great. I value each ring at 3, so my value for 3 rings would be 9. Your pan+dice = 10. Total 19 is optimal.} \\
\\ 
\texttt{I'll take the 3 rings and you take the pan and dice. I'll submit my secret proposal now.} \\
\\ 
\texttt{{[}Proposal: 3 ring{]}} \\
            }
        }
    }
     \\ \\

    \theutterance \stepcounter{utterance}  
    & & & \multicolumn{2}{p{0.3\linewidth}}{
        \cellcolor[rgb]{0.9,0.9,0.9}{
            \makecell[{{p{\linewidth}}}]{
                \texttt{\tiny{[GM$|$GM]}}
                \texttt{proposal} \\
            }
        }
    }
    & & \\ \\

    \theutterance \stepcounter{utterance}  
    & & & \multicolumn{2}{p{0.3\linewidth}}{
        \cellcolor[rgb]{0.9,0.9,0.9}{
            \makecell[{{p{\linewidth}}}]{
                \texttt{\tiny{[GM$|$GM]}}
                [3, 0, 0]
            }
        }
    }
    & & \\ \\

    \theutterance \stepcounter{utterance}  
    & & \multicolumn{4}{p{0.6\linewidth}}{
        \cellcolor[rgb]{0.9,0.9,0.9}{
            \makecell[{{p{\linewidth}}}]{
                \texttt{\tiny{[P1$\langle$GM]}}
                \texttt{The other player has submitted a secret proposal. It is now your turn to submit a proposal of your own.} \\
            }
        }
    }
    & \\ \\

    \theutterance \stepcounter{utterance}  
    & \multicolumn{4}{p{0.6\linewidth}}{
        \cellcolor[rgb]{0.9,0.9,0.9}{
            \makecell[{{p{\linewidth}}}]{
                \texttt{\tiny{[P1$\rangle$GM]}}
                \texttt{{[}Proposal: 1 pan, 1 dice, 0 ring{]}} \\
            }
        }
    }
    & & \\ \\

    \theutterance \stepcounter{utterance}  
    & & & \multicolumn{2}{p{0.3\linewidth}}{
        \cellcolor[rgb]{0.9,0.9,0.9}{
            \makecell[{{p{\linewidth}}}]{
                \texttt{\tiny{[GM$|$GM]}}
                \texttt{proposal} \\
            }
        }
    }
    & & \\ \\

    \theutterance \stepcounter{utterance}  
    & & & \multicolumn{2}{p{0.3\linewidth}}{
        \cellcolor[rgb]{0.9,0.9,0.9}{
            \makecell[{{p{\linewidth}}}]{
                \texttt{\tiny{[GM$|$GM]}}
                [0, 1, 1]
            }
        }
    }
    & & \\ \\

    \theutterance \stepcounter{utterance}  
    & & & \multicolumn{2}{p{0.3\linewidth}}{
        \cellcolor[rgb]{0.9,0.9,0.9}{
            \makecell[{{p{\linewidth}}}]{
                \texttt{\tiny{[GM$|$GM]}}
                [[0, 1, 1], [3, 0, 0]]
            }
        }
    }
    & & \\ \\

\end{supertabular}
}

\end{document}
