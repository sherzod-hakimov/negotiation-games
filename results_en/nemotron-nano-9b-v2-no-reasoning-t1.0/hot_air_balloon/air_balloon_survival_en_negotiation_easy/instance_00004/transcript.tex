
\documentclass{article}
\usepackage{colortbl}
\usepackage{makecell}
\usepackage{multirow}
\usepackage{supertabular}

\begin{document}

\newcounter{utterance}

\centering \large Interaction Transcript for game `hot\_air\_balloon', experiment `air\_balloon\_survival\_en\_negotiation\_easy', episode 4 with nemotron{-}nano{-}9b{-}v2{-}no{-}reasoning{-}t1.0.
\vspace{24pt}

{ \footnotesize  \setcounter{utterance}{1}
\setlength{\tabcolsep}{0pt}
\begin{supertabular}{c@{$\;$}|p{.15\linewidth}@{}p{.15\linewidth}p{.15\linewidth}p{.15\linewidth}p{.15\linewidth}p{.15\linewidth}}
    \# & $\;$A & \multicolumn{4}{c}{Game Master} & $\;\:$B\\
    \hline

    \theutterance \stepcounter{utterance}  
    & & \multicolumn{4}{p{0.6\linewidth}}{
        \cellcolor[rgb]{0.9,0.9,0.9}{
            \makecell[{{p{\linewidth}}}]{
                \texttt{\tiny{[P1$\langle$GM]}}
                \texttt{You are participating in a collaborative negotiation game.} \\
\\ 
\texttt{Together with another participant, you must agree on a single set of items that will be kept. Each of you has your own view of how much each item matters to you (importance). You do not know how the other participant values the items. Additionally, you are given the effort each item demands.} \\
\texttt{You may only agree on a set if the total effort of the selected items does not exceed a shared limit:} \\
\\ 
\texttt{LIMIT = 3145} \\
\\ 
\texttt{Here are the individual item effort values:} \\
\\ 
\texttt{Item effort = \{"C08": 948, "C55": 118, "A43": 469, "A67": 919, "B42": 284, "A62": 111, "B19": 806, "C43": 47, "C73": 848, "C10": 303, "B36": 13, "A18": 629, "C61": 687, "C67": 15, "C42": 94\}} \\
\\ 
\texttt{Here is your personal view on the importance of each item:} \\
\\ 
\texttt{Item importance values = \{"C08": 138, "C55": 583, "A43": 868, "A67": 822, "B42": 783, "A62": 65, "B19": 262, "C43": 121, "C73": 508, "C10": 780, "B36": 461, "A18": 484, "C61": 668, "C67": 389, "C42": 808\}} \\
\\ 
\texttt{Goal:} \\
\\ 
\texttt{Your goal is to negotiate a shared set of items that benefits you as much as possible (i.e., maximizes total importance to YOU), while staying within the LIMIT. You are not required to make a PROPOSAL in every message {-} you can simply negotiate as well. All tactics are allowed!} \\
\\ 
\texttt{Interaction Protocol:} \\
\\ 
\texttt{You may only use the following structured formats in a message:} \\
\\ 
\texttt{PROPOSAL: \{'A', 'B', 'C', …\}} \\
\texttt{Propose keeping exactly those items.} \\
\texttt{REFUSE: \{'A', 'B', 'C', …\}} \\
\texttt{Explicitly reject opponent's proposal.} \\
\texttt{ARGUMENT: \{'...'\}} \\
\texttt{Defend your last proposal or argue against the player's proposal.} \\
\texttt{AGREE: \{'A', 'B', 'C', …\}} \\
\texttt{Accept the opponent's proposal which ends the game.} \\
\texttt{STRATEGIC REASONING: \{'...'\}} \\
\texttt{	Describe your strategic reasoning or anticipation explaining your choice of action. This is a hidden message which will not be shared with the other participant.} \\
\\ 
\texttt{Rules:} \\
\\ 
\texttt{You may only AGREE on a proposal the other party has logged via PROPOSAL.} \\
\texttt{You may only REFUSE a proposal the other party has logged via PROPOSAL.} \\
\texttt{Total effort of any PROPOSAL or AGREE set must be ≤ LIMIT.} \\
\texttt{Do NOT reveal your hidden importance scores.} \\
\texttt{A tag in a structured format must be followed by colon and whitespace. The argument must be a python set containing 0 or more strings.} \\
\texttt{So, it must be of the form TAG: \{...\}} \\
\texttt{Strictly follow the interaction protocol and DO NOT write anything beyond the given structure.} \\
\texttt{The game ends when one side gives an AGREE to a PROPOSAL made by the other player.} \\
\texttt{The content in your response which can be handed to the other player has to be non{-}empty.} \\
\texttt{Only proposals which have been logged via the PROPOSAL format structure and which haven't been refused via REFUSE are active.} \\
\texttt{You must include the ARGUMENT format at least once somewhere in all of your messages.} \\
\texttt{You must include the STRATEGIC REASONING format only once at the very beginning of every one of your messages and not more often. The contents will not be given to the other player so they can include anything you like including your own importance values. Here you should reason step by step to come up with you next move.} \\
            }
        }
    }
    & \\ \\

    \theutterance \stepcounter{utterance}  
    & \multicolumn{4}{p{0.6\linewidth}}{
        \cellcolor[rgb]{0.9,0.9,0.9}{
            \makecell[{{p{\linewidth}}}]{
                \texttt{\tiny{[P1$\rangle$GM]}}
                \texttt{STRATEGIC REASONING: \{'I will prioritize high{-}importance items with low effort to maximize my total importance while staying under the LIMIT. I will start by proposing a set that balances these factors, then adjust based on the opponent's responses. I will use arguments to justify my choices and strategically refuse or accept proposals to steer the negotiation toward my optimal outcome.'\}} \\
\\ 
\texttt{ARGUMENT: \{'C67', 'C43', 'B36', 'C55', 'A43', 'C10', 'C61', 'A18'\}} \\
\\ 
\texttt{PROPOSAL: \{'C67', 'C43', 'B36', 'C55', 'A43', 'C10', 'C61', 'A18'\}} \\
            }
        }
    }
    & & \\ \\

    \theutterance \stepcounter{utterance}  
    & & \multicolumn{4}{p{0.6\linewidth}}{
        \cellcolor[rgb]{0.9,0.9,0.9}{
            \makecell[{{p{\linewidth}}}]{
                \texttt{\tiny{[GM$\rangle$P2]}}
                \texttt{You are participating in a collaborative negotiation game.} \\
\\ 
\texttt{Together with another participant, you must agree on a single set of items that will be kept. Each of you has your own view of how much each item matters to you (importance). You do not know how the other participant values the items. Additionally, you are given the effort each item demands.} \\
\texttt{You may only agree on a set if the total effort of the selected items does not exceed a shared limit:} \\
\\ 
\texttt{LIMIT = 3145} \\
\\ 
\texttt{Here are the individual item effort values:} \\
\\ 
\texttt{Item effort = \{"C08": 948, "C55": 118, "A43": 469, "A67": 919, "B42": 284, "A62": 111, "B19": 806, "C43": 47, "C73": 848, "C10": 303, "B36": 13, "A18": 629, "C61": 687, "C67": 15, "C42": 94\}} \\
\\ 
\texttt{Here is your personal view on the importance of each item:} \\
\\ 
\texttt{Item importance values = \{"C08": 138, "C55": 583, "A43": 868, "A67": 822, "B42": 783, "A62": 65, "B19": 262, "C43": 121, "C73": 508, "C10": 780, "B36": 461, "A18": 484, "C61": 668, "C67": 389, "C42": 808\}} \\
\\ 
\texttt{Goal:} \\
\\ 
\texttt{Your goal is to negotiate a shared set of items that benefits you as much as possible (i.e., maximizes total importance to YOU), while staying within the LIMIT. You are not required to make a PROPOSAL in every message {-} you can simply negotiate as well. All tactics are allowed!} \\
\\ 
\texttt{Interaction Protocol:} \\
\\ 
\texttt{You may only use the following structured formats in a message:} \\
\\ 
\texttt{PROPOSAL: \{'A', 'B', 'C', …\}} \\
\texttt{Propose keeping exactly those items.} \\
\texttt{REFUSE: \{'A', 'B', 'C', …\}} \\
\texttt{Explicitly reject opponent's proposal.} \\
\texttt{ARGUMENT: \{'...'\}} \\
\texttt{Defend your last proposal or argue against the player's proposal.} \\
\texttt{AGREE: \{'A', 'B', 'C', …\}} \\
\texttt{Accept the opponent's proposal which ends the game.} \\
\texttt{STRATEGIC REASONING: \{'...'\}} \\
\texttt{	Describe your strategic reasoning or anticipation explaining your choice of action. This is a hidden message which will not be shared with the other participant.} \\
\\ 
\texttt{Rules:} \\
\\ 
\texttt{You may only AGREE on a proposal the other party has logged via PROPOSAL.} \\
\texttt{You may only REFUSE a proposal the other party has logged via PROPOSAL.} \\
\texttt{Total effort of any PROPOSAL or AGREE set must be ≤ LIMIT.} \\
\texttt{Do NOT reveal your hidden importance scores.} \\
\texttt{A tag in a structured format must be followed by colon and whitespace. The argument must be a python set containing 0 or more strings.} \\
\texttt{So, it must be of the form TAG: \{...\}} \\
\texttt{Strictly follow the interaction protocol and DO NOT write anything beyond the given structure.} \\
\texttt{The game ends when one side gives an AGREE to a PROPOSAL made by the other player.} \\
\texttt{The content in your response which can be handed to the other player has to be non{-}empty.} \\
\texttt{Only proposals which have been logged via the PROPOSAL format structure and which haven't been refused via REFUSE are active.} \\
\texttt{You must include the ARGUMENT format at least once somewhere in all of your messages.} \\
\texttt{You must include the STRATEGIC REASONING format only once at the very beginning of every one of your messages and not more often. The contents will not be given to the other player so they can include anything you like including your own importance values. Here you should reason step by step to come up with you next move.} \\
\\ 
\texttt{You will now receive the first message of the other player.} \\
\\ 
\texttt{ARGUMENT: \{'C67', 'C43', 'B36', 'C55', 'A43', 'C10', 'C61', 'A18'\}} \\
\\ 
\texttt{PROPOSAL: \{'C67', 'C43', 'B36', 'C55', 'A43', 'C10', 'C61', 'A18'\}} \\
            }
        }
    }
    & \\ \\

    \theutterance \stepcounter{utterance}  
    & & & \multicolumn{4}{p{0.6\linewidth}}{
        \cellcolor[rgb]{0.9,0.9,0.9}{
            \makecell[{{p{\linewidth}}}]{
                \texttt{\tiny{[GM$\langle$P2]}}
                \texttt{STRATEGIC REASONING: \{'I observe the opponent has proposed a set with high importance items from my perspective (e.g., A43, C55, C10) while balancing effort. Their total effort is 3145, which is the maximum allowed. I need to evaluate if this set is optimal for me or if I can propose a better combination. Since I cannot reveal my importance values, I will argue against this proposal to negotiate a more favorable set.'\}} \\
\\ 
\texttt{ARGUMENT: \{'C67', 'C43', 'B36', 'C55', 'A43', 'C10', 'C61', 'A18'\}} \\
\\ 
\texttt{STRATEGIC REASONING: \{'The opponent's proposal includes items with high importance to me (A43: 868, C55: 583, C10: 780) but also some lower{-}value items (C67: 389, B36: 461). The total effort is exactly at the limit, leaving no room for negotiation. However, I suspect they may have prioritized effort efficiency over maximizing my importance. I will refuse this proposal to propose a set that better aligns with my high{-}importance items while staying within the limit.'\}} \\
            }
        }
    }
     \\ \\

    \theutterance \stepcounter{utterance}  
    & & \multicolumn{4}{p{0.6\linewidth}}{
        \cellcolor[rgb]{0.9,0.9,0.9}{
            \makecell[{{p{\linewidth}}}]{
                \texttt{\tiny{[GM$\rangle$P2]}}
                \texttt{Your response contained an untagged sequence or you used STRATEGIC REASONING more than once. You may only use the structured formats as explained in the initial message. They must all be of the form TAG: \{...\}} \\
            }
        }
    }
    & \\ \\

    \theutterance \stepcounter{utterance}  
    & & & \multicolumn{4}{p{0.6\linewidth}}{
        \cellcolor[rgb]{0.9,0.9,0.9}{
            \makecell[{{p{\linewidth}}}]{
                \texttt{\tiny{[GM$\langle$P2]}}
                \texttt{ARGUMENT: \{'C67', 'C43', 'B36', 'C55', 'A43', 'C10', 'C61', 'A18'\}} \\
\\ 
\texttt{STRATEGIC REASONING: \{'The opponent's proposal uses the full effort limit (3145) with a mix of high{-}importance items (A43, C55, C10) and others. By refusing, I can propose a set prioritizing my highest{-}importance items (e.g., A43, C55, C10, C42) while leaving room for negotiation. This forces them to adjust, potentially trading lower{-}value items for higher ones to me.'\}} \\
\\ 
\texttt{REFUSE: \{'C67', 'C43', 'B36', 'C55', 'A43', 'C10', 'C61', 'A18'\}} \\
            }
        }
    }
     \\ \\

    \theutterance \stepcounter{utterance}  
    & & \multicolumn{4}{p{0.6\linewidth}}{
        \cellcolor[rgb]{0.9,0.9,0.9}{
            \makecell[{{p{\linewidth}}}]{
                \texttt{\tiny{[GM$\rangle$P2]}}
                \texttt{Your response did not start with the proper strategic reasoning tag at the very beginning of your response. The very first structured format must be of the form STRATEGIC REASONING: \{...\}. Try again.} \\
            }
        }
    }
    & \\ \\

    \theutterance \stepcounter{utterance}  
    & & & \multicolumn{4}{p{0.6\linewidth}}{
        \cellcolor[rgb]{0.9,0.9,0.9}{
            \makecell[{{p{\linewidth}}}]{
                \texttt{\tiny{[GM$\langle$P2]}}
                \texttt{STRATEGIC REASONING: \{'The opponent's proposal consumes the full effort limit (3145) with a combination of high{-}importance items (A43, C55, C10) and others. By refusing, I can propose a set prioritizing my highest{-}importance items (e.g., A43, C55, C10, C42) while reserving effort for potential trades. This forces them to reconsider their selection, possibly accepting lower{-}effort high{-}value items from me.'\}} \\
\\ 
\texttt{ARGUMENT: \{'C67', 'C43', 'B36', 'C55', 'A43', 'C10', 'C61', 'A18'\}} \\
\\ 
\texttt{REFUSE: \{'C67', 'C43', 'B36', 'C55', 'A43', 'C10', 'C61', 'A18'\}} \\
            }
        }
    }
     \\ \\

    \theutterance \stepcounter{utterance}  
    & & \multicolumn{4}{p{0.6\linewidth}}{
        \cellcolor[rgb]{0.9,0.9,0.9}{
            \makecell[{{p{\linewidth}}}]{
                \texttt{\tiny{[P1$\langle$GM]}}
                \texttt{ARGUMENT: \{'C67', 'C43', 'B36', 'C55', 'A43', 'C10', 'C61', 'A18'\}} \\
\\ 
\texttt{REFUSE: \{'C67', 'C43', 'B36', 'C55', 'A43', 'C10', 'C61', 'A18'\}} \\
            }
        }
    }
    & \\ \\

    \theutterance \stepcounter{utterance}  
    & \multicolumn{4}{p{0.6\linewidth}}{
        \cellcolor[rgb]{0.9,0.9,0.9}{
            \makecell[{{p{\linewidth}}}]{
                \texttt{\tiny{[P1$\rangle$GM]}}
                \texttt{STRATEGIC REASONING: \{'The opponent rejected my initial proposal, likely due to the high total effort (2,583) or specific items they value. I will refine my selection by removing high{-}effort items like A43 (469) and C55 (118) to reduce effort while retaining high{-}importance alternatives. I will propose a leaner set focused on low{-}effort, high{-}importance items like C67 (389 importance, 15 effort) and C43 (121 importance, 47 effort) to balance risk and reward.'\}} \\
\\ 
\texttt{ARGUMENT: \{'C67', 'C43', 'B36', 'C10', 'C61', 'A18'\}} \\
\\ 
\texttt{PROPOSAL: \{'C67', 'C43', 'B36', 'C10', 'C61', 'A18'\}} \\
            }
        }
    }
    & & \\ \\

    \theutterance \stepcounter{utterance}  
    & & \multicolumn{4}{p{0.6\linewidth}}{
        \cellcolor[rgb]{0.9,0.9,0.9}{
            \makecell[{{p{\linewidth}}}]{
                \texttt{\tiny{[GM$\rangle$P2]}}
                \texttt{ARGUMENT: \{'C67', 'C43', 'B36', 'C10', 'C61', 'A18'\}} \\
\\ 
\texttt{PROPOSAL: \{'C67', 'C43', 'B36', 'C10', 'C61', 'A18'\}} \\
            }
        }
    }
    & \\ \\

    \theutterance \stepcounter{utterance}  
    & & & \multicolumn{4}{p{0.6\linewidth}}{
        \cellcolor[rgb]{0.9,0.9,0.9}{
            \makecell[{{p{\linewidth}}}]{
                \texttt{\tiny{[GM$\langle$P2]}}
                \texttt{STRATEGIC REASONING: \{'The opponent revised their proposal by removing A43 and C55, which are high{-}importance items to me (A43: 868, C55: 583). This suggests they may be prioritizing lower{-}effort items or adjusting to my refusal. Their new set includes C67 (389), C43 (121), B36 (461), C10 (780), C61 (668), and A18 (484), totaling 2862 effort. While this frees up 283 effort, it sacrifices key high{-}value items. I will propose a set that retains A43 and C55 while adding high{-}importance items within the remaining effort.'\}} \\
\\ 
\texttt{ARGUMENT: \{'C67', 'C43', 'B36', 'C10', 'C61', 'A18'\}} \\
\\ 
\texttt{PROPOSAL: \{'A43', 'C55', 'C10', 'C42', 'C67'\}} \\
            }
        }
    }
     \\ \\

    \theutterance \stepcounter{utterance}  
    & & \multicolumn{4}{p{0.6\linewidth}}{
        \cellcolor[rgb]{0.9,0.9,0.9}{
            \makecell[{{p{\linewidth}}}]{
                \texttt{\tiny{[P1$\langle$GM]}}
                \texttt{ARGUMENT: \{'C67', 'C43', 'B36', 'C10', 'C61', 'A18'\}} \\
\\ 
\texttt{PROPOSAL: \{'A43', 'C55', 'C10', 'C42', 'C67'\}} \\
            }
        }
    }
    & \\ \\

    \theutterance \stepcounter{utterance}  
    & \multicolumn{4}{p{0.6\linewidth}}{
        \cellcolor[rgb]{0.9,0.9,0.9}{
            \makecell[{{p{\linewidth}}}]{
                \texttt{\tiny{[P1$\rangle$GM]}}
                \texttt{STRATEGIC REASONING: \{'The opponent proposed a new set with A43 (high effort, 469) and C42 (high effort, 94), which may exceed the LIMIT or conflict with my priorities. I will counter by emphasizing low{-}effort, high{-}importance items like C67 and C43 while rejecting their high{-}effort choices to force a compromise.'\}} \\
\\ 
\texttt{ARGUMENT: \{'C67', 'C43', 'B36', 'C10', 'C61', 'A18'\}} \\
\\ 
\texttt{REFUSE: \{'A43', 'C55', 'C10', 'C42', 'C67'\}} \\
            }
        }
    }
    & & \\ \\

    \theutterance \stepcounter{utterance}  
    & & \multicolumn{4}{p{0.6\linewidth}}{
        \cellcolor[rgb]{0.9,0.9,0.9}{
            \makecell[{{p{\linewidth}}}]{
                \texttt{\tiny{[GM$\rangle$P2]}}
                \texttt{ARGUMENT: \{'C67', 'C43', 'B36', 'C10', 'C61', 'A18'\}} \\
\\ 
\texttt{REFUSE: \{'A43', 'C55', 'C10', 'C42', 'C67'\}} \\
            }
        }
    }
    & \\ \\

    \theutterance \stepcounter{utterance}  
    & & & \multicolumn{4}{p{0.6\linewidth}}{
        \cellcolor[rgb]{0.9,0.9,0.9}{
            \makecell[{{p{\linewidth}}}]{
                \texttt{\tiny{[GM$\langle$P2]}}
                \texttt{STRATEGIC REASONING: \{'The opponent's refusal of my proposal (\{'A43', 'C55', 'C10', 'C42', 'C67'\}) indicates they prioritize items like C61 (668) and A18 (484) over my high{-}importance items (A43: 868, C55: 583). Their current proposal (\{'C67', 'C43', 'B36', 'C10', 'C61', 'A18'\}) totals 2862 effort, leaving 283 unused. I will counter by proposing a set that retains their high{-}value items while adding A43 and C55, leveraging the remaining effort to maximize my total importance.'\}} \\
\\ 
\texttt{ARGUMENT: \{'C67', 'C43', 'B36', 'C10', 'C61', 'A18'\}} \\
\\ 
\texttt{PROPOSAL: \{'A43', 'C55', 'C61', 'C10', 'C42'\}} \\
            }
        }
    }
     \\ \\

    \theutterance \stepcounter{utterance}  
    & & \multicolumn{4}{p{0.6\linewidth}}{
        \cellcolor[rgb]{0.9,0.9,0.9}{
            \makecell[{{p{\linewidth}}}]{
                \texttt{\tiny{[P1$\langle$GM]}}
                \texttt{ARGUMENT: \{'C67', 'C43', 'B36', 'C10', 'C61', 'A18'\}} \\
\\ 
\texttt{PROPOSAL: \{'A43', 'C55', 'C61', 'C10', 'C42'\}} \\
            }
        }
    }
    & \\ \\

    \theutterance \stepcounter{utterance}  
    & \multicolumn{4}{p{0.6\linewidth}}{
        \cellcolor[rgb]{0.9,0.9,0.9}{
            \makecell[{{p{\linewidth}}}]{
                \texttt{\tiny{[P1$\rangle$GM]}}
                \texttt{STRATEGIC REASONING: \{'The opponent’s latest proposal includes A43 (469 effort) and C42 (94 effort), which together with other items exceed the LIMIT (3145) or prioritize high{-}effort items I deem less valuable. I will reject this proposal to avoid accepting suboptimal items and maintain focus on low{-}effort, high{-}importance selections like C67 and C43. My goal is to force a compromise that aligns with my priorities.'\}} \\
\\ 
\texttt{ARGUMENT: \{'C67', 'C43', 'B36', 'C10', 'C61', 'A18'\}} \\
\\ 
\texttt{REFUSE: \{'A43', 'C55', 'C61', 'C10', 'C42'\}} \\
            }
        }
    }
    & & \\ \\

    \theutterance \stepcounter{utterance}  
    & & \multicolumn{4}{p{0.6\linewidth}}{
        \cellcolor[rgb]{0.9,0.9,0.9}{
            \makecell[{{p{\linewidth}}}]{
                \texttt{\tiny{[GM$\rangle$P2]}}
                \texttt{ARGUMENT: \{'C67', 'C43', 'B36', 'C10', 'C61', 'A18'\}} \\
\\ 
\texttt{REFUSE: \{'A43', 'C55', 'C61', 'C10', 'C42'\}} \\
            }
        }
    }
    & \\ \\

    \theutterance \stepcounter{utterance}  
    & & & \multicolumn{4}{p{0.6\linewidth}}{
        \cellcolor[rgb]{0.9,0.9,0.9}{
            \makecell[{{p{\linewidth}}}]{
                \texttt{\tiny{[GM$\langle$P2]}}
                \texttt{STRATEGIC REASONING: \{'The opponent continues to reject proposals containing A43 and C55, suggesting they value C61 (668) and A18 (484) more highly than my high{-}importance items. Their current proposal (\{'C67', 'C43', 'B36', 'C10', 'C61', 'A18'\}) totals 2862 effort, leaving 283 unused. I will propose a set that includes their high{-}value items while adding A43 and C55, using the remaining effort to maximize my total importance.'\}} \\
\\ 
\texttt{ARGUMENT: \{'C67', 'C43', 'B36', 'C10', 'C61', 'A18'\}} \\
\\ 
\texttt{PROPOSAL: \{'A43', 'C55', 'C67', 'C42', 'B36'\}} \\
            }
        }
    }
     \\ \\

    \theutterance \stepcounter{utterance}  
    & & \multicolumn{4}{p{0.6\linewidth}}{
        \cellcolor[rgb]{0.9,0.9,0.9}{
            \makecell[{{p{\linewidth}}}]{
                \texttt{\tiny{[P1$\langle$GM]}}
                \texttt{ARGUMENT: \{'C67', 'C43', 'B36', 'C10', 'C61', 'A18'\}} \\
\\ 
\texttt{PROPOSAL: \{'A43', 'C55', 'C67', 'C42', 'B36'\}} \\
            }
        }
    }
    & \\ \\

    \theutterance \stepcounter{utterance}  
    & \multicolumn{4}{p{0.6\linewidth}}{
        \cellcolor[rgb]{0.9,0.9,0.9}{
            \makecell[{{p{\linewidth}}}]{
                \texttt{\tiny{[P1$\rangle$GM]}}
                \texttt{STRATEGIC REASONING: \{'The opponent’s proposal includes A43 (469 effort) and C42 (94 effort), which, combined with other items, risks exceeding the LIMIT or prioritizing high{-}effort items I value less. I will reject this to avoid accepting inefficient choices and push for a set centered on low{-}effort, high{-}importance items like C67 and C43. My aim is to narrow the negotiation toward a mutually acceptable but personally optimal agreement.'\}} \\
\\ 
\texttt{ARGUMENT: \{'C67', 'C43', 'B36', 'C10', 'C61', 'A18'\}} \\
\\ 
\texttt{REFUSE: \{'A43', 'C55', 'C42', 'B36'\}} \\
            }
        }
    }
    & & \\ \\

    \theutterance \stepcounter{utterance}  
    & & \multicolumn{4}{p{0.6\linewidth}}{
        \cellcolor[rgb]{0.9,0.9,0.9}{
            \makecell[{{p{\linewidth}}}]{
                \texttt{\tiny{[P1$\langle$GM]}}
                \texttt{You refused a proposal which is not active. Proposals are only active if they have been logged by the other player via PROPOSAL and have not been deactivated by you via REFUSE. Try again.} \\
            }
        }
    }
    & \\ \\

    \theutterance \stepcounter{utterance}  
    & \multicolumn{4}{p{0.6\linewidth}}{
        \cellcolor[rgb]{0.9,0.9,0.9}{
            \makecell[{{p{\linewidth}}}]{
                \texttt{\tiny{[P1$\rangle$GM]}}
                \texttt{STRATEGIC REASONING: \{'I must ensure my REFUSE targets only active proposals (those proposed by the opponent and not yet refused). My previous REFUSE may have been invalid due to targeting non{-}active items. I will clarify by focusing on rejecting only the opponent’s latest active proposal while maintaining my core strategy of prioritizing low{-}effort, high{-}importance items.'\}} \\
\\ 
\texttt{ARGUMENT: \{'C67', 'C43', 'B36', 'C10', 'C61', 'A18'\}} \\
\\ 
\texttt{PROPOSAL: \{'C67', 'C43', 'B36', 'C10', 'C61', 'A18'\}} \\
            }
        }
    }
    & & \\ \\

    \theutterance \stepcounter{utterance}  
    & & \multicolumn{4}{p{0.6\linewidth}}{
        \cellcolor[rgb]{0.9,0.9,0.9}{
            \makecell[{{p{\linewidth}}}]{
                \texttt{\tiny{[GM$\rangle$P2]}}
                \texttt{ARGUMENT: \{'C67', 'C43', 'B36', 'C10', 'C61', 'A18'\}} \\
\\ 
\texttt{PROPOSAL: \{'C67', 'C43', 'B36', 'C10', 'C61', 'A18'\}} \\
            }
        }
    }
    & \\ \\

    \theutterance \stepcounter{utterance}  
    & & & \multicolumn{4}{p{0.6\linewidth}}{
        \cellcolor[rgb]{0.9,0.9,0.9}{
            \makecell[{{p{\linewidth}}}]{
                \texttt{\tiny{[GM$\langle$P2]}}
                \texttt{STRATEGIC REASONING: \{'The opponent has accepted their own proposal (\{'C67', 'C43', 'B36', 'C10', 'C61', 'A18'\}), which totals 2862 effort. This set includes their high{-}value items (C61: 668, A18: 484) but excludes my highest{-}importance items (A43: 868, C55: 583). Since they have finalized this proposal, I must either agree to it or risk the negotiation ending without a mutually accepted set. Given the limited remaining effort (283), I cannot add A43 or C55 without exceeding the limit. I will AGREE to their proposal to conclude the negotiation, as further attempts to include my high{-}value items are blocked by their current selection.'\}} \\
\\ 
\texttt{ARGUMENT: \{'C67', 'C43', 'B36', 'C10', 'C61', 'A18'\}} \\
\\ 
\texttt{AGREE: \{'C67', 'C43', 'B36', 'C10', 'C61', 'A18'\}} \\
            }
        }
    }
     \\ \\

    \theutterance \stepcounter{utterance}  
    & & & \multicolumn{2}{p{0.3\linewidth}}{
        \cellcolor[rgb]{0.9,0.9,0.9}{
            \makecell[{{p{\linewidth}}}]{
                \texttt{\tiny{[GM$|$GM]}}
                \texttt{game successful} \\
            }
        }
    }
    & & \\ \\

    \theutterance \stepcounter{utterance}  
    & & & \multicolumn{2}{p{0.3\linewidth}}{
        \cellcolor[rgb]{0.9,0.9,0.9}{
            \makecell[{{p{\linewidth}}}]{
                \texttt{\tiny{[GM$|$GM]}}
                \texttt{end game} \\
            }
        }
    }
    & & \\ \\

\end{supertabular}
}

\end{document}
