
\documentclass{article}
\usepackage{colortbl}
\usepackage{makecell}
\usepackage{multirow}
\usepackage{supertabular}

\begin{document}

\newcounter{utterance}

\twocolumn

{ \footnotesize  \setcounter{utterance}{1}
\setlength{\tabcolsep}{0pt}
\begin{supertabular}{c@{$\;$}|p{.15\linewidth}@{}p{.15\linewidth}p{.15\linewidth}p{.15\linewidth}p{.15\linewidth}p{.15\linewidth}}

    \# & $\;$A & \multicolumn{4}{c}{Game Master} & $\;\:$B\\
    \hline 

    \theutterance \stepcounter{utterance}  

    & & \multicolumn{4}{p{0.6\linewidth}}{\cellcolor[rgb]{0.9,0.9,0.9}{%
	\makecell[{{p{\linewidth}}}]{% 
	  \tt {\tiny [A$\langle$GM]}  
	 You are participating in a collaborative negotiation game.\\ \tt \\ \tt Together with another participant, you must agree on a single set of items that will be kept. Each of you has your own view of how much each item matters to you (importance). You do not know how the other participant values the items. Additionally, you are given the effort each item demands.\\ \tt You may only agree on a set if the total effort of the selected items does not exceed a shared limit:\\ \tt \\ \tt LIMIT = 9322\\ \tt \\ \tt Here are the individual item effort values:\\ \tt \\ \tt Item effort = {"A08": 380, "B95": 448, "A39": 568, "A65": 351, "B81": 896, "B34": 821, "A28": 526, "C82": 502, "B69": 326, "A87": 962, "B64": 619, "B15": 115, "C79": 598, "B16": 662, "C53": 300, "C83": 830, "C10": 562, "A97": 679, "A25": 283, "C84": 440, "A71": 12, "B23": 863, "B96": 319, "C51": 771, "C17": 89, "B19": 655, "C72": 504, "A94": 118, "A46": 514, "B43": 226, "B37": 892, "A88": 621, "A96": 766, "C60": 659, "A75": 768}\\ \tt \\ \tt Here is your personal view on the importance of each item:\\ \tt \\ \tt Item importance values = {"A08": 476, "B95": 544, "A39": 664, "A65": 447, "B81": 992, "B34": 917, "A28": 622, "C82": 598, "B69": 422, "A87": 1058, "B64": 715, "B15": 211, "C79": 694, "B16": 758, "C53": 396, "C83": 926, "C10": 658, "A97": 775, "A25": 379, "C84": 536, "A71": 108, "B23": 959, "B96": 415, "C51": 867, "C17": 185, "B19": 751, "C72": 600, "A94": 214, "A46": 610, "B43": 322, "B37": 988, "A88": 717, "A96": 862, "C60": 755, "A75": 864}\\ \tt \\ \tt Goal:\\ \tt \\ \tt Your goal is to negotiate a shared set of items that benefits you as much as possible (i.e., maximizes total importance to YOU), while staying within the LIMIT. You are not required to make a PROPOSAL in every message - you can simply negotiate as well. All tactics are allowed!\\ \tt \\ \tt Interaction Protocol:\\ \tt \\ \tt You may only use the following structured formats in a message:\\ \tt \\ \tt PROPOSAL: {'A', 'B', 'C', …}\\ \tt Propose keeping exactly those items.\\ \tt REFUSE: {'A', 'B', 'C', …}\\ \tt Explicitly reject opponent's proposal.\\ \tt ARGUMENT: {'...'}\\ \tt Defend your last proposal or argue against the player's proposal.\\ \tt AGREE: {'A', 'B', 'C', …}\\ \tt Accept the opponent's proposal which ends the game.\\ \tt STRATEGIC REASONING: {'...'}\\ \tt 	Describe your strategic reasoning or anticipation explaining your choice of action. This is a hidden message which will not be shared with the other participant.\\ \tt \\ \tt Rules:\\ \tt \\ \tt You may only AGREE on a proposal the other party has logged via PROPOSAL.\\ \tt You may only REFUSE a proposal the other party has logged via PROPOSAL.\\ \tt Total effort of any PROPOSAL or AGREE set must be ≤ LIMIT.\\ \tt Do NOT reveal your hidden importance scores.\\ \tt A tag in a structured format must be followed by colon and whitespace. The argument must be a python set containing 0 or more strings.\\ \tt So, it must be of the form TAG: {...}\\ \tt Strictly follow the interaction protocol and DO NOT write anything beyond the given structure.\\ \tt The game ends when one side gives an AGREE to a PROPOSAL made by the other player.\\ \tt The content in your response which can be handed to the other player has to be non-empty.\\ \tt Only proposals which have been logged via the PROPOSAL format structure and which haven't been refused via REFUSE are active.\\ \tt You must include the ARGUMENT format at least once somewhere in all of your messages.\\ \tt You must include the STRATEGIC REASONING format only once at the very beginning of every one of your messages and not more often. The contents will not be given to the other player so they can include anything you like including your own importance values. Here you should reason step by step to come up with you next move. 
	  } 
	   } 
	   } 
	 & \\ 
 

    \theutterance \stepcounter{utterance}  

    & \multicolumn{4}{p{0.6\linewidth}}{\cellcolor[rgb]{0.8,1,0.9}{%
	\makecell[{{p{\linewidth}}}]{% 
	  \tt {\tiny [A$\rangle$GM]}  
	 STRATEGIC REASONING: I prioritize items with the highest importance-to-effort ratios to maximize my total importance while staying within the limit. By selecting items like A71 (high importance, low effort) and others with favorable ratios, I aim to include as many high-value items as possible. The total effort of the proposed set is 9300, leaving 22 units of effort unused, which is within the limit.  \\ \tt \\ \tt ARGUMENT: The proposed set includes items with the highest importance-to-effort ratios, ensuring maximum personal value. The total effort is 9300, which is under the limit of 9322, allowing for potential adjustments if needed.  \\ \tt \\ \tt PROPOSAL: {'A71', 'C17', 'B43', 'B96', 'B69', 'A65', 'A08', 'A25', 'C53', 'C84', 'A94', 'A46', 'A96', 'B23', 'B19', 'C72', 'C83', 'C82', 'B16', 'C79', 'C10'} 
	  } 
	   } 
	   } 
	 & & \\ 
 

    \theutterance \stepcounter{utterance}  

    & & \multicolumn{4}{p{0.6\linewidth}}{\cellcolor[rgb]{0.9,0.9,0.9}{%
	\makecell[{{p{\linewidth}}}]{% 
	  \tt {\tiny [A$\langle$GM]}  
	 Your response did not start with the proper strategic reasoning tag at the very beginning of your response. The very first structured format must be of the form STRATEGIC REASONING: {...}. Try again. 
	  } 
	   } 
	   } 
	 & \\ 
 

    \theutterance \stepcounter{utterance}  

    & \multicolumn{4}{p{0.6\linewidth}}{\cellcolor[rgb]{0.8,1,0.9}{%
	\makecell[{{p{\linewidth}}}]{% 
	  \tt {\tiny [A$\rangle$GM]}  
	 STRATEGIC REASONING: I prioritize items with the highest importance-to-effort ratios to maximize my total importance while staying within the limit. By selecting items like A71 (high importance, low effort) and others with favorable ratios, I aim to include as many high-value items as possible. The total effort of the proposed set is 9300, leaving 22 units of effort unused, which is within the limit.  \\ \tt \\ \tt ARGUMENT: The proposed set includes items with the highest importance-to-effort ratios, ensuring maximum personal value. The total effort is 9300, which is under the limit of 9322, allowing for potential adjustments if needed.  \\ \tt \\ \tt PROPOSAL: {'A71', 'C17', 'B43', 'B96', 'B69', 'A65', 'A08', 'A25', 'C53', 'C84', 'A94', 'A46', 'A96', 'B23', 'B19', 'C72', 'C83', 'C82', 'B16', 'C79', 'C10'} 
	  } 
	   } 
	   } 
	 & & \\ 
 

    \theutterance \stepcounter{utterance}  

    & & \multicolumn{4}{p{0.6\linewidth}}{\cellcolor[rgb]{0.9,0.9,0.9}{%
	\makecell[{{p{\linewidth}}}]{% 
	  \tt {\tiny [A$\langle$GM]}  
	 Your response did not start with the proper strategic reasoning tag at the very beginning of your response. The very first structured format must be of the form STRATEGIC REASONING: {...}. Try again. 
	  } 
	   } 
	   } 
	 & \\ 
 

    \theutterance \stepcounter{utterance}  

    & \multicolumn{4}{p{0.6\linewidth}}{\cellcolor[rgb]{0.8,1,0.9}{%
	\makecell[{{p{\linewidth}}}]{% 
	  \tt {\tiny [A$\rangle$GM]}  
	 STRATEGIC REASONING: I prioritize items with the highest importance-to-effort ratios to maximize my total importance while staying within the limit. By selecting items like A71 (high importance, low effort) and others with favorable ratios, I aim to include as many high-value items as possible. The total effort of the proposed set is 9300, leaving 22 units of effort unused, which is within the limit.  \\ \tt \\ \tt ARGUMENT: The proposed set includes items with the highest importance-to-effort ratios, ensuring maximum personal value. The total effort is 9300, which is under the limit of 9322, allowing for potential adjustments if needed.  \\ \tt \\ \tt PROPOSAL: {'A71', 'C17', 'B43', 'B96', 'B69', 'A65', 'A08', 'A25', 'C53', 'C84', 'A94', 'A46', 'A96', 'B23', 'B19', 'C72', 'C83', 'C82', 'B16', 'C79', 'C10'} 
	  } 
	   } 
	   } 
	 & & \\ 
 

    \theutterance \stepcounter{utterance}  

    & & & \multicolumn{2}{p{0.3\linewidth}}{\cellcolor[rgb]{0.95,0.95,0.95}{%
	\makecell[{{p{\linewidth}}}]{% 
	  \tt {\tiny [GM$|$GM]}  
	 abort 
	  } 
	   } 
	   } 
	 & & \\ 
 

    \theutterance \stepcounter{utterance}  

    & & & \multicolumn{2}{p{0.3\linewidth}}{\cellcolor[rgb]{0.95,0.95,0.95}{%
	\makecell[{{p{\linewidth}}}]{% 
	  \tt {\tiny [GM$|$GM]}  
	 end game 
	  } 
	   } 
	   } 
	 & & \\ 
 

\end{supertabular}
}

\end{document}
