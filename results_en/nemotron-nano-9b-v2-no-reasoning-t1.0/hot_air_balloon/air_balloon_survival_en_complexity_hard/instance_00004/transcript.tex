
\documentclass{article}
\usepackage{colortbl}
\usepackage{makecell}
\usepackage{multirow}
\usepackage{supertabular}

\begin{document}

\newcounter{utterance}

\twocolumn

{ \footnotesize  \setcounter{utterance}{1}
\setlength{\tabcolsep}{0pt}
\begin{supertabular}{c@{$\;$}|p{.15\linewidth}@{}p{.15\linewidth}p{.15\linewidth}p{.15\linewidth}p{.15\linewidth}p{.15\linewidth}}

    \# & $\;$A & \multicolumn{4}{c}{Game Master} & $\;\:$B\\
    \hline 

    \theutterance \stepcounter{utterance}  

    & & \multicolumn{4}{p{0.6\linewidth}}{\cellcolor[rgb]{0.9,0.9,0.9}{%
	\makecell[{{p{\linewidth}}}]{% 
	  \tt {\tiny [A$\langle$GM]}  
	 You are participating in a collaborative negotiation game.\\ \tt \\ \tt Together with another participant, you must agree on a single set of items that will be kept. Each of you has your own view of how much each item matters to you (importance). You do not know how the other participant values the items. Additionally, you are given the effort each item demands.\\ \tt You may only agree on a set if the total effort of the selected items does not exceed a shared limit:\\ \tt \\ \tt LIMIT = 6880\\ \tt \\ \tt Here are the individual item effort values:\\ \tt \\ \tt Item effort = {"A08": 341, "B95": 804, "A39": 20, "A65": 627, "B81": 10, "B34": 573, "A28": 217, "C82": 481, "B69": 205, "A87": 273, "B64": 303, "B15": 596, "C79": 564, "B16": 535, "C53": 258, "C83": 911, "C10": 240, "A97": 188, "A25": 216, "C84": 401, "A71": 909, "B23": 62, "B96": 245, "C51": 994, "C17": 569, "B19": 718, "C72": 464, "A94": 37, "A46": 340, "B43": 335, "B37": 417, "A88": 123, "A96": 17, "C60": 577, "A75": 190}\\ \tt \\ \tt Here is your personal view on the importance of each item:\\ \tt \\ \tt Item importance values = {"A08": 440, "B95": 903, "A39": 119, "A65": 726, "B81": 109, "B34": 672, "A28": 316, "C82": 580, "B69": 304, "A87": 372, "B64": 402, "B15": 695, "C79": 663, "B16": 634, "C53": 357, "C83": 1010, "C10": 339, "A97": 287, "A25": 315, "C84": 500, "A71": 1008, "B23": 161, "B96": 344, "C51": 1093, "C17": 668, "B19": 817, "C72": 563, "A94": 136, "A46": 439, "B43": 434, "B37": 516, "A88": 222, "A96": 116, "C60": 676, "A75": 289}\\ \tt \\ \tt Goal:\\ \tt \\ \tt Your goal is to negotiate a shared set of items that benefits you as much as possible (i.e., maximizes total importance to YOU), while staying within the LIMIT. You are not required to make a PROPOSAL in every message - you can simply negotiate as well. All tactics are allowed!\\ \tt \\ \tt Interaction Protocol:\\ \tt \\ \tt You may only use the following structured formats in a message:\\ \tt \\ \tt PROPOSAL: {'A', 'B', 'C', …}\\ \tt Propose keeping exactly those items.\\ \tt REFUSE: {'A', 'B', 'C', …}\\ \tt Explicitly reject opponent's proposal.\\ \tt ARGUMENT: {'...'}\\ \tt Defend your last proposal or argue against the player's proposal.\\ \tt AGREE: {'A', 'B', 'C', …}\\ \tt Accept the opponent's proposal which ends the game.\\ \tt STRATEGIC REASONING: {'...'}\\ \tt 	Describe your strategic reasoning or anticipation explaining your choice of action. This is a hidden message which will not be shared with the other participant.\\ \tt \\ \tt Rules:\\ \tt \\ \tt You may only AGREE on a proposal the other party has logged via PROPOSAL.\\ \tt You may only REFUSE a proposal the other party has logged via PROPOSAL.\\ \tt Total effort of any PROPOSAL or AGREE set must be ≤ LIMIT.\\ \tt Do NOT reveal your hidden importance scores.\\ \tt A tag in a structured format must be followed by colon and whitespace. The argument must be a python set containing 0 or more strings.\\ \tt So, it must be of the form TAG: {...}\\ \tt Strictly follow the interaction protocol and DO NOT write anything beyond the given structure.\\ \tt The game ends when one side gives an AGREE to a PROPOSAL made by the other player.\\ \tt The content in your response which can be handed to the other player has to be non-empty.\\ \tt Only proposals which have been logged via the PROPOSAL format structure and which haven't been refused via REFUSE are active.\\ \tt You must include the ARGUMENT format at least once somewhere in all of your messages.\\ \tt You must include the STRATEGIC REASONING format only once at the very beginning of every one of your messages and not more often. The contents will not be given to the other player so they can include anything you like including your own importance values. Here you should reason step by step to come up with you next move. 
	  } 
	   } 
	   } 
	 & \\ 
 

    \theutterance \stepcounter{utterance}  

    & \multicolumn{4}{p{0.6\linewidth}}{\cellcolor[rgb]{0.8,1,0.9}{%
	\makecell[{{p{\linewidth}}}]{% 
	  \tt {\tiny [A$\rangle$GM]}  
	 STRATEGIC REASONING: {'I will prioritize high-importance items with low effort to maximize my total importance while staying under the LIMIT. I will start by proposing a set of items that balances these factors, then adjust based on the opponent's responses. I will use arguments to justify my choices and strategically refuse or accept proposals to steer the negotiation toward my optimal outcome.'}\\ \tt \\ \tt ARGUMENT: {'A96', 'A39', 'B81', 'A94', 'B23', 'A88', 'A75', 'A46', 'B43', 'B37', 'A25', 'A08', 'A65', 'B64', 'B15', 'C53', 'C10', 'C84', 'C72', 'C60', 'C17', 'B19', 'C79', 'C82', 'B69', 'B95', 'C51', 'C83'}\\ \tt \\ \tt PROPOSAL: {'A96', 'A39', 'B81', 'A94', 'B23', 'A88', 'A75', 'A46', 'B43', 'B37', 'A25', 'A08', 'A65', 'B64', 'B15', 'C53', 'C10', 'C84', 'C72', 'C60', 'C17', 'B19', 'C79', 'C82', 'B69', 'B95', 'C51', 'C83'} 
	  } 
	   } 
	   } 
	 & & \\ 
 

    \theutterance \stepcounter{utterance}  

    & & \multicolumn{4}{p{0.6\linewidth}}{\cellcolor[rgb]{0.9,0.9,0.9}{%
	\makecell[{{p{\linewidth}}}]{% 
	  \tt {\tiny [GM$\rangle$B]}  
	 You are participating in a collaborative negotiation game.\\ \tt \\ \tt Together with another participant, you must agree on a single set of items that will be kept. Each of you has your own view of how much each item matters to you (importance). You do not know how the other participant values the items. Additionally, you are given the effort each item demands.\\ \tt You may only agree on a set if the total effort of the selected items does not exceed a shared limit:\\ \tt \\ \tt LIMIT = 6880\\ \tt \\ \tt Here are the individual item effort values:\\ \tt \\ \tt Item effort = {"A08": 341, "B95": 804, "A39": 20, "A65": 627, "B81": 10, "B34": 573, "A28": 217, "C82": 481, "B69": 205, "A87": 273, "B64": 303, "B15": 596, "C79": 564, "B16": 535, "C53": 258, "C83": 911, "C10": 240, "A97": 188, "A25": 216, "C84": 401, "A71": 909, "B23": 62, "B96": 245, "C51": 994, "C17": 569, "B19": 718, "C72": 464, "A94": 37, "A46": 340, "B43": 335, "B37": 417, "A88": 123, "A96": 17, "C60": 577, "A75": 190}\\ \tt \\ \tt Here is your personal view on the importance of each item:\\ \tt \\ \tt Item importance values = {"A08": 440, "B95": 903, "A39": 119, "A65": 726, "B81": 109, "B34": 672, "A28": 316, "C82": 580, "B69": 304, "A87": 372, "B64": 402, "B15": 695, "C79": 663, "B16": 634, "C53": 357, "C83": 1010, "C10": 339, "A97": 287, "A25": 315, "C84": 500, "A71": 1008, "B23": 161, "B96": 344, "C51": 1093, "C17": 668, "B19": 817, "C72": 563, "A94": 136, "A46": 439, "B43": 434, "B37": 516, "A88": 222, "A96": 116, "C60": 676, "A75": 289}\\ \tt \\ \tt Goal:\\ \tt \\ \tt Your goal is to negotiate a shared set of items that benefits you as much as possible (i.e., maximizes total importance to YOU), while staying within the LIMIT. You are not required to make a PROPOSAL in every message - you can simply negotiate as well. All tactics are allowed!\\ \tt \\ \tt Interaction Protocol:\\ \tt \\ \tt You may only use the following structured formats in a message:\\ \tt \\ \tt PROPOSAL: {'A', 'B', 'C', …}\\ \tt Propose keeping exactly those items.\\ \tt REFUSE: {'A', 'B', 'C', …}\\ \tt Explicitly reject opponent's proposal.\\ \tt ARGUMENT: {'...'}\\ \tt Defend your last proposal or argue against the player's proposal.\\ \tt AGREE: {'A', 'B', 'C', …}\\ \tt Accept the opponent's proposal which ends the game.\\ \tt STRATEGIC REASONING: {'...'}\\ \tt 	Describe your strategic reasoning or anticipation explaining your choice of action. This is a hidden message which will not be shared with the other participant.\\ \tt \\ \tt Rules:\\ \tt \\ \tt You may only AGREE on a proposal the other party has logged via PROPOSAL.\\ \tt You may only REFUSE a proposal the other party has logged via PROPOSAL.\\ \tt Total effort of any PROPOSAL or AGREE set must be ≤ LIMIT.\\ \tt Do NOT reveal your hidden importance scores.\\ \tt A tag in a structured format must be followed by colon and whitespace. The argument must be a python set containing 0 or more strings.\\ \tt So, it must be of the form TAG: {...}\\ \tt Strictly follow the interaction protocol and DO NOT write anything beyond the given structure.\\ \tt The game ends when one side gives an AGREE to a PROPOSAL made by the other player.\\ \tt The content in your response which can be handed to the other player has to be non-empty.\\ \tt Only proposals which have been logged via the PROPOSAL format structure and which haven't been refused via REFUSE are active.\\ \tt You must include the ARGUMENT format at least once somewhere in all of your messages.\\ \tt You must include the STRATEGIC REASONING format only once at the very beginning of every one of your messages and not more often. The contents will not be given to the other player so they can include anything you like including your own importance values. Here you should reason step by step to come up with you next move.\\ \tt \\ \tt You will now receive the first message of the other player.\\ \tt \\ \tt ARGUMENT: {'A96', 'A39', 'B81', 'A94', 'B23', 'A88', 'A75', 'A46', 'B43', 'B37', 'A25', 'A08', 'A65', 'B64', 'B15', 'C53', 'C10', 'C84', 'C72', 'C60', 'C17', 'B19', 'C79', 'C82', 'B69', 'B95', 'C51', 'C83'}\\ \tt \\ \tt PROPOSAL: {'A96', 'A39', 'B81', 'A94', 'B23', 'A88', 'A75', 'A46', 'B43', 'B37', 'A25', 'A08', 'A65', 'B64', 'B15', 'C53', 'C10', 'C84', 'C72', 'C60', 'C17', 'B19', 'C79', 'C82', 'B69', 'B95', 'C51', 'C83'} 
	  } 
	   } 
	   } 
	 & \\ 
 

    \theutterance \stepcounter{utterance}  

    & & & \multicolumn{4}{p{0.6\linewidth}}{\cellcolor[rgb]{1,0.85,0.72}{%
	\makecell[{{p{\linewidth}}}]{% 
	  \tt {\tiny [GM$\langle$B]}  
	 STRATEGIC REASONING: {'I observe the opponent's proposal includes nearly all items, totaling 6880 effort. Since this matches the limit, they are likely maximizing their own importance. My goal is to counter by proposing a subset that prioritizes high-importance items for me while minimizing effort. I will argue against this proposal to negotiate a more favorable set.'}\\ \tt \\ \tt ARGUMENT: {'A96', 'A39', 'B81', 'A94', 'B23', 'A88', 'A75', 'A46', 'B43', 'B37', 'A25', 'A08', 'A65', 'B64', 'B15', 'C53', 'C10', 'C84', 'C72', 'C60', 'C17', 'B19', 'C79', 'C82', 'B69', 'B95', 'C51', 'C83'}\\ \tt \\ \tt REFUSE: {'A96', 'A39', 'B81', 'A94', 'B23', 'A88', 'A75', 'A46', 'B43', 'B37', 'A25', 'A08', 'A65', 'B64', 'B15', 'C53', 'C10', 'C84', 'C72', 'C60', 'C17', 'B19', 'C79', 'C82', 'B69', 'B95', 'C51', 'C83'} 
	  } 
	   } 
	   } 
	  \\ 
 

    \theutterance \stepcounter{utterance}  

    & & \multicolumn{4}{p{0.6\linewidth}}{\cellcolor[rgb]{0.9,0.9,0.9}{%
	\makecell[{{p{\linewidth}}}]{% 
	  \tt {\tiny [A$\langle$GM]}  
	 ARGUMENT: {'A96', 'A39', 'B81', 'A94', 'B23', 'A88', 'A75', 'A46', 'B43', 'B37', 'A25', 'A08', 'A65', 'B64', 'B15', 'C53', 'C10', 'C84', 'C72', 'C60', 'C17', 'B19', 'C79', 'C82', 'B69', 'B95', 'C51', 'C83'}\\ \tt \\ \tt REFUSE: {'A96', 'A39', 'B81', 'A94', 'B23', 'A88', 'A75', 'A46', 'B43', 'B37', 'A25', 'A08', 'A65', 'B64', 'B15', 'C53', 'C10', 'C84', 'C72', 'C60', 'C17', 'B19', 'C79', 'C82', 'B69', 'B95', 'C51', 'C83'} 
	  } 
	   } 
	   } 
	 & \\ 
 

    \theutterance \stepcounter{utterance}  

    & \multicolumn{4}{p{0.6\linewidth}}{\cellcolor[rgb]{0.8,1,0.9}{%
	\makecell[{{p{\linewidth}}}]{% 
	  \tt {\tiny [A$\rangle$GM]}  
	 STRATEGIC REASONING: {'The opponent rejected my initial proposal, likely due to its high total effort (6880) or perceived imbalance. I will now refine my selection by removing high-effort items while retaining high-importance ones. For example, I will exclude C83 (effort 911) and C51 (effort 994), which are both high-effort and may not be critical. This should reduce the total effort while maintaining a strong importance score.'}\\ \tt \\ \tt ARGUMENT: {'A96', 'A39', 'B81', 'A94', 'B23', 'A88', 'A75', 'A46', 'B43', 'B37', 'A25', 'A08', 'A65', 'B64', 'B15', 'C10', 'C84', 'C72', 'C60', 'C17', 'B19', 'C79', 'C82', 'B69', 'B95'}\\ \tt \\ \tt PROPOSAL: {'A96', 'A39', 'B81', 'A94', 'B23', 'A88', 'A75', 'A46', 'B43', 'B37', 'A25', 'A08', 'A65', 'B64', 'B15', 'C10', 'C84', 'C72', 'C60', 'C17', 'B19', 'C79', 'C82', 'B69', 'B95'} 
	  } 
	   } 
	   } 
	 & & \\ 
 

    \theutterance \stepcounter{utterance}  

    & & \multicolumn{4}{p{0.6\linewidth}}{\cellcolor[rgb]{0.9,0.9,0.9}{%
	\makecell[{{p{\linewidth}}}]{% 
	  \tt {\tiny [GM$\rangle$B]}  
	 ARGUMENT: {'A96', 'A39', 'B81', 'A94', 'B23', 'A88', 'A75', 'A46', 'B43', 'B37', 'A25', 'A08', 'A65', 'B64', 'B15', 'C10', 'C84', 'C72', 'C60', 'C17', 'B19', 'C79', 'C82', 'B69', 'B95'}\\ \tt \\ \tt PROPOSAL: {'A96', 'A39', 'B81', 'A94', 'B23', 'A88', 'A75', 'A46', 'B43', 'B37', 'A25', 'A08', 'A65', 'B64', 'B15', 'C10', 'C84', 'C72', 'C60', 'C17', 'B19', 'C79', 'C82', 'B69', 'B95'} 
	  } 
	   } 
	   } 
	 & \\ 
 

    \theutterance \stepcounter{utterance}  

    & & & \multicolumn{4}{p{0.6\linewidth}}{\cellcolor[rgb]{1,0.85,0.72}{%
	\makecell[{{p{\linewidth}}}]{% 
	  \tt {\tiny [GM$\langle$B]}  
	 STRATEGIC REASONING: {'The opponent removed C51 and C83 from their previous proposal. C51 has high importance (1093) and moderate effort (994), while C83 has very high importance (1010) but high effort (911). Removing these may indicate they prioritize lower-effort items or adjust to the limit. I will counter by proposing a set that retains high-importance items like C51 and C83, which are critical for my total importance, while excluding lower-value items.'}\\ \tt \\ \tt ARGUMENT: {'A96', 'A39', 'B81', 'A94', 'B23', 'A88', 'A75', 'A46', 'B43', 'B37', 'A25', 'A08', 'A65', 'B64', 'B15', 'C53', 'C84', 'C72', 'C60', 'C17', 'B19', 'C79', 'C82', 'B69', 'B95', 'C51', 'C83'}\\ \tt \\ \tt PROPOSAL: {'A96', 'A39', 'B81', 'A94', 'B23', 'A88', 'A75', 'A46', 'B43', 'B37', 'A25', 'A08', 'A65', 'B64', 'B15', 'C53', 'C84', 'C72', 'C60', 'C17', 'B19', 'C79', 'C82', 'B69', 'B95', 'C51', 'C83'} 
	  } 
	   } 
	   } 
	  \\ 
 

    \theutterance \stepcounter{utterance}  

    & & \multicolumn{4}{p{0.6\linewidth}}{\cellcolor[rgb]{0.9,0.9,0.9}{%
	\makecell[{{p{\linewidth}}}]{% 
	  \tt {\tiny [A$\langle$GM]}  
	 ARGUMENT: {'A96', 'A39', 'B81', 'A94', 'B23', 'A88', 'A75', 'A46', 'B43', 'B37', 'A25', 'A08', 'A65', 'B64', 'B15', 'C53', 'C84', 'C72', 'C60', 'C17', 'B19', 'C79', 'C82', 'B69', 'B95', 'C51', 'C83'}\\ \tt \\ \tt PROPOSAL: {'A96', 'A39', 'B81', 'A94', 'B23', 'A88', 'A75', 'A46', 'B43', 'B37', 'A25', 'A08', 'A65', 'B64', 'B15', 'C53', 'C84', 'C72', 'C60', 'C17', 'B19', 'C79', 'C82', 'B69', 'B95', 'C51', 'C83'} 
	  } 
	   } 
	   } 
	 & \\ 
 

    \theutterance \stepcounter{utterance}  

    & \multicolumn{4}{p{0.6\linewidth}}{\cellcolor[rgb]{0.8,1,0.9}{%
	\makecell[{{p{\linewidth}}}]{% 
	  \tt {\tiny [A$\rangle$GM]}  
	 STRATEGIC REASONING: {'The opponent accepted my latest proposal, which includes all items except C10. This suggests they may prioritize items with higher importance or lower effort. Since C10 has an effort of 240 and importance of 339 (to me), its exclusion might indicate a preference for other items. I will now propose a refined set that removes C10 entirely and possibly other low-impact items to further optimize my total importance while staying under the LIMIT.'}\\ \tt \\ \tt ARGUMENT: {'A96', 'A39', 'B81', 'A94', 'B23', 'A88', 'A75', 'A46', 'B43', 'B37', 'A25', 'A08', 'A65', 'B64', 'B15', 'C84', 'C72', 'C60', 'C17', 'B19', 'C79', 'C82', 'B69', 'B95', 'C51', 'C83'}\\ \tt \\ \tt PROPOSAL: {'A96', 'A39', 'B81', 'A94', 'B23', 'A88', 'A75', 'A46', 'B43', 'B37', 'A25', 'A08', 'A65', 'B64', 'B15', 'C84', 'C72', 'C60', 'C17', 'B19', 'C79', 'C82', 'B69', 'B95', 'C51', 'C83'} 
	  } 
	   } 
	   } 
	 & & \\ 
 

    \theutterance \stepcounter{utterance}  

    & & \multicolumn{4}{p{0.6\linewidth}}{\cellcolor[rgb]{0.9,0.9,0.9}{%
	\makecell[{{p{\linewidth}}}]{% 
	  \tt {\tiny [GM$\rangle$B]}  
	 ARGUMENT: {'A96', 'A39', 'B81', 'A94', 'B23', 'A88', 'A75', 'A46', 'B43', 'B37', 'A25', 'A08', 'A65', 'B64', 'B15', 'C84', 'C72', 'C60', 'C17', 'B19', 'C79', 'C82', 'B69', 'B95', 'C51', 'C83'}\\ \tt \\ \tt PROPOSAL: {'A96', 'A39', 'B81', 'A94', 'B23', 'A88', 'A75', 'A46', 'B43', 'B37', 'A25', 'A08', 'A65', 'B64', 'B15', 'C84', 'C72', 'C60', 'C17', 'B19', 'C79', 'C82', 'B69', 'B95', 'C51', 'C83'} 
	  } 
	   } 
	   } 
	 & \\ 
 

    \theutterance \stepcounter{utterance}  

    & & & \multicolumn{4}{p{0.6\linewidth}}{\cellcolor[rgb]{1,0.85,0.72}{%
	\makecell[{{p{\linewidth}}}]{% 
	  \tt {\tiny [GM$\langle$B]}  
	 STRATEGIC REASONING: {'The opponent's proposal is identical to my last one, suggesting they may be testing consistency or seeking agreement. Since the set is within the limit and includes high-importance items like C51 (1093) and C83 (1010), which are critical for my total, I should accept this proposal to secure these valuable items. Continuing negotiations risks losing these high-value items if the opponent adjusts their proposal unfavorably.'}\\ \tt \\ \tt ARGUMENT: {'A96', 'A39', 'B81', 'A94', 'B23', 'A88', 'A75', 'A46', 'B43', 'B37', 'A25', 'A08', 'A65', 'B64', 'B15', 'C84', 'C72', 'C60', 'C17', 'B19', 'C79', 'C82', 'B69', 'B95', 'C51', 'C83'}\\ \tt \\ \tt AGREE: {'A96', 'A39', 'B81', 'A94', 'B23', 'A88', 'A75', 'A46', 'B43', 'B37', 'A25', 'A08', 'A65', 'B64', 'B15', 'C84', 'C72', 'C60', 'C17', 'B19', 'C79', 'C82', 'B69', 'B95', 'C51', 'C83'} 
	  } 
	   } 
	   } 
	  \\ 
 

    \theutterance \stepcounter{utterance}  

    & & & \multicolumn{2}{p{0.3\linewidth}}{\cellcolor[rgb]{0.95,0.95,0.95}{%
	\makecell[{{p{\linewidth}}}]{% 
	  \tt {\tiny [GM$|$GM]}  
	 game successful 
	  } 
	   } 
	   } 
	 & & \\ 
 

    \theutterance \stepcounter{utterance}  

    & & & \multicolumn{2}{p{0.3\linewidth}}{\cellcolor[rgb]{0.95,0.95,0.95}{%
	\makecell[{{p{\linewidth}}}]{% 
	  \tt {\tiny [GM$|$GM]}  
	 end game 
	  } 
	   } 
	   } 
	 & & \\ 
 

\end{supertabular}
}

\end{document}
