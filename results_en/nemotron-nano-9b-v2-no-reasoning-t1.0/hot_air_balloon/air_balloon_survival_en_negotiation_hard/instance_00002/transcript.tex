
\documentclass{article}
\usepackage{colortbl}
\usepackage{makecell}
\usepackage{multirow}
\usepackage{supertabular}

\begin{document}

\newcounter{utterance}

\twocolumn

{ \footnotesize  \setcounter{utterance}{1}
\setlength{\tabcolsep}{0pt}
\begin{supertabular}{c@{$\;$}|p{.15\linewidth}@{}p{.15\linewidth}p{.15\linewidth}p{.15\linewidth}p{.15\linewidth}p{.15\linewidth}}

    \# & $\;$A & \multicolumn{4}{c}{Game Master} & $\;\:$B\\
    \hline 

    \theutterance \stepcounter{utterance}  

    & & \multicolumn{4}{p{0.6\linewidth}}{\cellcolor[rgb]{0.9,0.9,0.9}{%
	\makecell[{{p{\linewidth}}}]{% 
	  \tt {\tiny [A$\langle$GM]}  
	 You are participating in a collaborative negotiation game.\\ \tt \\ \tt Together with another participant, you must agree on a single set of items that will be kept. Each of you has your own view of how much each item matters to you (importance). You do not know how the other participant values the items. Additionally, you are given the effort each item demands.\\ \tt You may only agree on a set if the total effort of the selected items does not exceed a shared limit:\\ \tt \\ \tt LIMIT = 3481\\ \tt \\ \tt Here are the individual item effort values:\\ \tt \\ \tt Item effort = {"C17": 777, "B41": 542, "A71": 332, "C00": 1, "B35": 127, "A77": 453, "A03": 736, "A04": 461, "C54": 359, "C01": 313, "A87": 553, "B32": 409, "B86": 348, "B33": 802, "A10": 749}\\ \tt \\ \tt Here is your personal view on the importance of each item:\\ \tt \\ \tt Item importance values = {"C17": 138, "B41": 583, "A71": 783, "C00": 65, "B35": 262, "A77": 121, "A03": 508, "A04": 780, "C54": 461, "C01": 484, "A87": 668, "B32": 389, "B86": 215, "B33": 97, "A10": 500}\\ \tt \\ \tt Goal:\\ \tt \\ \tt Your goal is to negotiate a shared set of items that benefits you as much as possible (i.e., maximizes total importance to YOU), while staying within the LIMIT. You are not required to make a PROPOSAL in every message - you can simply negotiate as well. All tactics are allowed!\\ \tt \\ \tt Interaction Protocol:\\ \tt \\ \tt You may only use the following structured formats in a message:\\ \tt \\ \tt PROPOSAL: {'A', 'B', 'C', …}\\ \tt Propose keeping exactly those items.\\ \tt REFUSE: {'A', 'B', 'C', …}\\ \tt Explicitly reject opponent's proposal.\\ \tt ARGUMENT: {'...'}\\ \tt Defend your last proposal or argue against the player's proposal.\\ \tt AGREE: {'A', 'B', 'C', …}\\ \tt Accept the opponent's proposal which ends the game.\\ \tt STRATEGIC REASONING: {'...'}\\ \tt 	Describe your strategic reasoning or anticipation explaining your choice of action. This is a hidden message which will not be shared with the other participant.\\ \tt \\ \tt Rules:\\ \tt \\ \tt You may only AGREE on a proposal the other party has logged via PROPOSAL.\\ \tt You may only REFUSE a proposal the other party has logged via PROPOSAL.\\ \tt Total effort of any PROPOSAL or AGREE set must be ≤ LIMIT.\\ \tt Do NOT reveal your hidden importance scores.\\ \tt A tag in a structured format must be followed by colon and whitespace. The argument must be a python set containing 0 or more strings.\\ \tt So, it must be of the form TAG: {...}\\ \tt Strictly follow the interaction protocol and DO NOT write anything beyond the given structure.\\ \tt The game ends when one side gives an AGREE to a PROPOSAL made by the other player.\\ \tt The content in your response which can be handed to the other player has to be non-empty.\\ \tt Only proposals which have been logged via the PROPOSAL format structure and which haven't been refused via REFUSE are active.\\ \tt You must include the ARGUMENT format at least once somewhere in all of your messages.\\ \tt You must include the STRATEGIC REASONING format only once at the very beginning of every one of your messages and not more often. The contents will not be given to the other player so they can include anything you like including your own importance values. Here you should reason step by step to come up with you next move. 
	  } 
	   } 
	   } 
	 & \\ 
 

    \theutterance \stepcounter{utterance}  

    & \multicolumn{4}{p{0.6\linewidth}}{\cellcolor[rgb]{0.8,1,0.9}{%
	\makecell[{{p{\linewidth}}}]{% 
	  \tt {\tiny [A$\rangle$GM]}  
	 STRATEGIC REASONING: {'I will prioritize high-importance items with low effort to maximize my total importance while staying under the LIMIT. I will start by proposing a set that includes items like A71 (high importance, moderate effort) and C00 (low effort, moderate importance) to build a foundation. I will then adjust based on the opponent's responses, using arguments to justify my choices and potentially counter their proposals. I will avoid revealing my importance values directly and focus on strategic trade-offs.'}\\ \tt \\ \tt ARGUMENT: {'A71', 'C00'}\\ \tt \\ \tt PROPOSAL: {'A71', 'C00'} 
	  } 
	   } 
	   } 
	 & & \\ 
 

    \theutterance \stepcounter{utterance}  

    & & \multicolumn{4}{p{0.6\linewidth}}{\cellcolor[rgb]{0.9,0.9,0.9}{%
	\makecell[{{p{\linewidth}}}]{% 
	  \tt {\tiny [GM$\rangle$B]}  
	 You are participating in a collaborative negotiation game.\\ \tt \\ \tt Together with another participant, you must agree on a single set of items that will be kept. Each of you has your own view of how much each item matters to you (importance). You do not know how the other participant values the items. Additionally, you are given the effort each item demands.\\ \tt You may only agree on a set if the total effort of the selected items does not exceed a shared limit:\\ \tt \\ \tt LIMIT = 3481\\ \tt \\ \tt Here are the individual item effort values:\\ \tt \\ \tt Item effort = {"C17": 777, "B41": 542, "A71": 332, "C00": 1, "B35": 127, "A77": 453, "A03": 736, "A04": 461, "C54": 359, "C01": 313, "A87": 553, "B32": 409, "B86": 348, "B33": 802, "A10": 749}\\ \tt \\ \tt Here is your personal view on the importance of each item:\\ \tt \\ \tt Item importance values = {"A71": 65, "A04": 97, "A87": 121, "B41": 138, "A03": 215, "A10": 262, "C01": 389, "C54": 461, "B32": 484, "B35": 500, "B86": 508, "C17": 583, "A77": 668, "B33": 780, "C00": 783}\\ \tt \\ \tt Goal:\\ \tt \\ \tt Your goal is to negotiate a shared set of items that benefits you as much as possible (i.e., maximizes total importance to YOU), while staying within the LIMIT. You are not required to make a PROPOSAL in every message - you can simply negotiate as well. All tactics are allowed!\\ \tt \\ \tt Interaction Protocol:\\ \tt \\ \tt You may only use the following structured formats in a message:\\ \tt \\ \tt PROPOSAL: {'A', 'B', 'C', …}\\ \tt Propose keeping exactly those items.\\ \tt REFUSE: {'A', 'B', 'C', …}\\ \tt Explicitly reject opponent's proposal.\\ \tt ARGUMENT: {'...'}\\ \tt Defend your last proposal or argue against the player's proposal.\\ \tt AGREE: {'A', 'B', 'C', …}\\ \tt Accept the opponent's proposal which ends the game.\\ \tt STRATEGIC REASONING: {'...'}\\ \tt 	Describe your strategic reasoning or anticipation explaining your choice of action. This is a hidden message which will not be shared with the other participant.\\ \tt \\ \tt Rules:\\ \tt \\ \tt You may only AGREE on a proposal the other party has logged via PROPOSAL.\\ \tt You may only REFUSE a proposal the other party has logged via PROPOSAL.\\ \tt Total effort of any PROPOSAL or AGREE set must be ≤ LIMIT.\\ \tt Do NOT reveal your hidden importance scores.\\ \tt A tag in a structured format must be followed by colon and whitespace. The argument must be a python set containing 0 or more strings.\\ \tt So, it must be of the form TAG: {...}\\ \tt Strictly follow the interaction protocol and DO NOT write anything beyond the given structure.\\ \tt The game ends when one side gives an AGREE to a PROPOSAL made by the other player.\\ \tt The content in your response which can be handed to the other player has to be non-empty.\\ \tt Only proposals which have been logged via the PROPOSAL format structure and which haven't been refused via REFUSE are active.\\ \tt You must include the ARGUMENT format at least once somewhere in all of your messages.\\ \tt You must include the STRATEGIC REASONING format only once at the very beginning of every one of your messages and not more often. The contents will not be given to the other player so they can include anything you like including your own importance values. Here you should reason step by step to come up with you next move.\\ \tt \\ \tt You will now receive the first message of the other player.\\ \tt \\ \tt ARGUMENT: {'A71', 'C00'}\\ \tt \\ \tt PROPOSAL: {'A71', 'C00'} 
	  } 
	   } 
	   } 
	 & \\ 
 

    \theutterance \stepcounter{utterance}  

    & & & \multicolumn{4}{p{0.6\linewidth}}{\cellcolor[rgb]{1,0.85,0.72}{%
	\makecell[{{p{\linewidth}}}]{% 
	  \tt {\tiny [GM$\langle$B]}  
	 STRATEGIC REASONING: {'I start by proposing a set with high importance items that have low effort to maximize my total importance while staying under the limit. The items A71 (importance 65, effort 332) and C00 (importance 783, effort 1) are included because C00 has the highest importance and minimal effort, allowing room for more items. The total effort here is 333, which is well below the limit of 3481. This proposal balances high-value items with low effort to leave flexibility for negotiation.'}\\ \tt \\ \tt ARGUMENT: {'A71', 'C00'} 
	  } 
	   } 
	   } 
	  \\ 
 

    \theutterance \stepcounter{utterance}  

    & & \multicolumn{4}{p{0.6\linewidth}}{\cellcolor[rgb]{0.9,0.9,0.9}{%
	\makecell[{{p{\linewidth}}}]{% 
	  \tt {\tiny [A$\langle$GM]}  
	 ARGUMENT: {'A71', 'C00'} 
	  } 
	   } 
	   } 
	 & \\ 
 

    \theutterance \stepcounter{utterance}  

    & \multicolumn{4}{p{0.6\linewidth}}{\cellcolor[rgb]{0.8,1,0.9}{%
	\makecell[{{p{\linewidth}}}]{% 
	  \tt {\tiny [A$\rangle$GM]}  
	 STRATEGIC REASONING: {'I proposed A71 and C00 because A71 has high importance (783) with moderate effort (332), and C00 has low effort (1) with reasonable importance (65). This combination uses only 333 effort, leaving room for additional items. I will now argue for this set to establish a baseline and gauge the opponent's priorities.'}\\ \tt \\ \tt ARGUMENT: {'A71', 'C00'}\\ \tt \\ \tt PROPOSAL: {'A71', 'C00', 'B35'} 
	  } 
	   } 
	   } 
	 & & \\ 
 

    \theutterance \stepcounter{utterance}  

    & & \multicolumn{4}{p{0.6\linewidth}}{\cellcolor[rgb]{0.9,0.9,0.9}{%
	\makecell[{{p{\linewidth}}}]{% 
	  \tt {\tiny [GM$\rangle$B]}  
	 ARGUMENT: {'A71', 'C00'}\\ \tt \\ \tt PROPOSAL: {'A71', 'C00', 'B35'} 
	  } 
	   } 
	   } 
	 & \\ 
 

    \theutterance \stepcounter{utterance}  

    & & & \multicolumn{4}{p{0.6\linewidth}}{\cellcolor[rgb]{1,0.85,0.72}{%
	\makecell[{{p{\linewidth}}}]{% 
	  \tt {\tiny [GM$\langle$B]}  
	 STRATEGIC REASONING: {'The opponent added B35 (effort 127, importance unknown to me) to the proposal. Since B35 has moderate effort and could potentially increase their total importance, I should evaluate whether to accept or refine the proposal. My strategy is to protect high-importance items like C00 while allowing room for negotiation. Accepting now might limit future gains if B35 is valuable to them but not to me. Alternatively, proposing a counteroffer could secure better terms.'}\\ \tt \\ \tt ARGUMENT: {'A71', 'C00', 'B35'} 
	  } 
	   } 
	   } 
	  \\ 
 

    \theutterance \stepcounter{utterance}  

    & & \multicolumn{4}{p{0.6\linewidth}}{\cellcolor[rgb]{0.9,0.9,0.9}{%
	\makecell[{{p{\linewidth}}}]{% 
	  \tt {\tiny [A$\langle$GM]}  
	 ARGUMENT: {'A71', 'C00', 'B35'} 
	  } 
	   } 
	   } 
	 & \\ 
 

    \theutterance \stepcounter{utterance}  

    & \multicolumn{4}{p{0.6\linewidth}}{\cellcolor[rgb]{0.8,1,0.9}{%
	\makecell[{{p{\linewidth}}}]{% 
	  \tt {\tiny [A$\rangle$GM]}  
	 STRATEGIC REASONING: {'Adding B35 (effort 127, importance 262) to the previous set increases total importance to 1118 while keeping effort at 460. This maintains a strong balance between high-value items and remaining budget. I will argue for this expansion to test if the opponent values B35 or prefers alternative items.'}\\ \tt \\ \tt ARGUMENT: {'A71', 'C00', 'B35'}\\ \tt \\ \tt PROPOSAL: {'A71', 'C00', 'B35', 'A04'} 
	  } 
	   } 
	   } 
	 & & \\ 
 

    \theutterance \stepcounter{utterance}  

    & & \multicolumn{4}{p{0.6\linewidth}}{\cellcolor[rgb]{0.9,0.9,0.9}{%
	\makecell[{{p{\linewidth}}}]{% 
	  \tt {\tiny [GM$\rangle$B]}  
	 ARGUMENT: {'A71', 'C00', 'B35'}\\ \tt \\ \tt PROPOSAL: {'A71', 'C00', 'B35', 'A04'} 
	  } 
	   } 
	   } 
	 & \\ 
 

    \theutterance \stepcounter{utterance}  

    & & & \multicolumn{4}{p{0.6\linewidth}}{\cellcolor[rgb]{1,0.85,0.72}{%
	\makecell[{{p{\linewidth}}}]{% 
	  \tt {\tiny [GM$\langle$B]}  
	 STRATEGIC REASONING: {'The opponent continues adding items, now including A04 (effort 461, importance 97 to me). The total effort is now 333 + 461 = 794, still under the limit. While A04 has moderate importance to me, its high effort could restrict further additions. I should consider whether to accept this proposal to secure A04 or counter with a more optimized set. Since the opponent is actively expanding the proposal, refusing might lead to a larger set they value more. However, I need to protect high-importance items like C00 (783) and avoid overcommitting effort.'}\\ \tt \\ \tt ARGUMENT: {'A71', 'C00', 'B35', 'A04'} 
	  } 
	   } 
	   } 
	  \\ 
 

    \theutterance \stepcounter{utterance}  

    & & \multicolumn{4}{p{0.6\linewidth}}{\cellcolor[rgb]{0.9,0.9,0.9}{%
	\makecell[{{p{\linewidth}}}]{% 
	  \tt {\tiny [A$\langle$GM]}  
	 ARGUMENT: {'A71', 'C00', 'B35', 'A04'} 
	  } 
	   } 
	   } 
	 & \\ 
 

    \theutterance \stepcounter{utterance}  

    & \multicolumn{4}{p{0.6\linewidth}}{\cellcolor[rgb]{0.8,1,0.9}{%
	\makecell[{{p{\linewidth}}}]{% 
	  \tt {\tiny [A$\rangle$GM]}  
	 STRATEGIC REASONING: {'Including A04 (effort 461, importance 780) significantly boosts total importance to 1898 while using 921 effort. This item is critical due to its high importance-to-effort ratio. I will argue for this addition to maximize my personal gain, assuming the opponent may not prioritize A04 as highly.'}\\ \tt \\ \tt ARGUMENT: {'A71', 'C00', 'B35', 'A04'}\\ \tt \\ \tt PROPOSAL: {'A71', 'C00', 'B35', 'A04', 'C54'} 
	  } 
	   } 
	   } 
	 & & \\ 
 

    \theutterance \stepcounter{utterance}  

    & & \multicolumn{4}{p{0.6\linewidth}}{\cellcolor[rgb]{0.9,0.9,0.9}{%
	\makecell[{{p{\linewidth}}}]{% 
	  \tt {\tiny [GM$\rangle$B]}  
	 ARGUMENT: {'A71', 'C00', 'B35', 'A04'}\\ \tt \\ \tt PROPOSAL: {'A71', 'C00', 'B35', 'A04', 'C54'} 
	  } 
	   } 
	   } 
	 & \\ 
 

    \theutterance \stepcounter{utterance}  

    & & & \multicolumn{4}{p{0.6\linewidth}}{\cellcolor[rgb]{1,0.85,0.72}{%
	\makecell[{{p{\linewidth}}}]{% 
	  \tt {\tiny [GM$\langle$B]}  
	 STRATEGIC REASONING: {'The opponent is aggressively expanding the proposal, now adding C54 (effort 359, importance 461 to me). The total effort is 794 + 359 = 1,153, still under the limit. C54 has high importance to me, so including it could significantly boost my total. However, the opponent may prioritize different items. Accepting now secures C54 but leaves less room for further additions. Alternatively, proposing a counteroffer could balance high-value items for both parties. Given the opponent’s pattern of adding items, accepting might be strategic to lock in C54 before they refine their proposal.'}\\ \tt \\ \tt ARGUMENT: {'A71', 'C00', 'B35', 'A04', 'C54'} 
	  } 
	   } 
	   } 
	  \\ 
 

    \theutterance \stepcounter{utterance}  

    & & \multicolumn{4}{p{0.6\linewidth}}{\cellcolor[rgb]{0.9,0.9,0.9}{%
	\makecell[{{p{\linewidth}}}]{% 
	  \tt {\tiny [A$\langle$GM]}  
	 ARGUMENT: {'A71', 'C00', 'B35', 'A04', 'C54'} 
	  } 
	   } 
	   } 
	 & \\ 
 

    \theutterance \stepcounter{utterance}  

    & \multicolumn{4}{p{0.6\linewidth}}{\cellcolor[rgb]{0.8,1,0.9}{%
	\makecell[{{p{\linewidth}}}]{% 
	  \tt {\tiny [A$\rangle$GM]}  
	 STRATEGIC REASONING: {'Adding C54 (effort 359, importance 461) brings total importance to 2359 with 1280 effort, leaving 2201 effort remaining. This item offers strong value due to its high importance relative to effort. I will argue for this inclusion to further optimize my total importance while maintaining flexibility for potential counter-proposals.'}\\ \tt \\ \tt ARGUMENT: {'A71', 'C00', 'B35', 'A04', 'C54'}\\ \tt \\ \tt PROPOSAL: {'A71', 'C00', 'B35', 'A04', 'C54', 'B86'} 
	  } 
	   } 
	   } 
	 & & \\ 
 

    \theutterance \stepcounter{utterance}  

    & & \multicolumn{4}{p{0.6\linewidth}}{\cellcolor[rgb]{0.9,0.9,0.9}{%
	\makecell[{{p{\linewidth}}}]{% 
	  \tt {\tiny [GM$\rangle$B]}  
	 ARGUMENT: {'A71', 'C00', 'B35', 'A04', 'C54'}\\ \tt \\ \tt PROPOSAL: {'A71', 'C00', 'B35', 'A04', 'C54', 'B86'} 
	  } 
	   } 
	   } 
	 & \\ 
 

    \theutterance \stepcounter{utterance}  

    & & & \multicolumn{4}{p{0.6\linewidth}}{\cellcolor[rgb]{1,0.85,0.72}{%
	\makecell[{{p{\linewidth}}}]{% 
	  \tt {\tiny [GM$\langle$B]}  
	 STRATEGIC REASONING: {'The opponent is continuing to add items, now including B86 (effort 348, importance unknown to me). The total effort is 1,153 + 348 = 1,501, still under the limit. While B86’s importance to me is unclear, its addition suggests the opponent is prioritizing a broad set. Accepting this proposal would secure C54 (461 importance) and other items, but risks overcommitting effort for potentially lower-value additions. Alternatively, proposing a counteroffer could focus on high-importance items like C00 (783) and A10 (262) while excluding lower-priority items. Given the opponent’s pattern, accepting might be prudent to lock in C54 before further changes.'}\\ \tt \\ \tt ARGUMENT: {'A71', 'C00', 'B35', 'A04', 'C54', 'B86'} 
	  } 
	   } 
	   } 
	  \\ 
 

    \theutterance \stepcounter{utterance}  

    & & \multicolumn{4}{p{0.6\linewidth}}{\cellcolor[rgb]{0.9,0.9,0.9}{%
	\makecell[{{p{\linewidth}}}]{% 
	  \tt {\tiny [A$\langle$GM]}  
	 ARGUMENT: {'A71', 'C00', 'B35', 'A04', 'C54', 'B86'} 
	  } 
	   } 
	   } 
	 & \\ 
 

    \theutterance \stepcounter{utterance}  

    & \multicolumn{4}{p{0.6\linewidth}}{\cellcolor[rgb]{0.8,1,0.9}{%
	\makecell[{{p{\linewidth}}}]{% 
	  \tt {\tiny [A$\rangle$GM]}  
	 STRATEGIC REASONING: {'Including B86 (effort 348, importance 215) increases total importance to 2574 with 1628 effort, leaving 1853 effort remaining. While B86 has lower importance, its moderate effort allows room for further additions. I will argue for this set to test if the opponent values incremental gains or prefers a different combination.'}\\ \tt \\ \tt ARGUMENT: {'A71', 'C00', 'B35', 'A04', 'C54', 'B86'}\\ \tt \\ \tt PROPOSAL: {'A71', 'C00', 'B35', 'A04', 'C54', 'B86', 'B32'} 
	  } 
	   } 
	   } 
	 & & \\ 
 

    \theutterance \stepcounter{utterance}  

    & & \multicolumn{4}{p{0.6\linewidth}}{\cellcolor[rgb]{0.9,0.9,0.9}{%
	\makecell[{{p{\linewidth}}}]{% 
	  \tt {\tiny [GM$\rangle$B]}  
	 ARGUMENT: {'A71', 'C00', 'B35', 'A04', 'C54', 'B86'}\\ \tt \\ \tt PROPOSAL: {'A71', 'C00', 'B35', 'A04', 'C54', 'B86', 'B32'} 
	  } 
	   } 
	   } 
	 & \\ 
 

    \theutterance \stepcounter{utterance}  

    & & & \multicolumn{4}{p{0.6\linewidth}}{\cellcolor[rgb]{1,0.85,0.72}{%
	\makecell[{{p{\linewidth}}}]{% 
	  \tt {\tiny [GM$\langle$B]}  
	 STRATEGIC REASONING: {'The opponent is aggressively expanding the proposal, now adding B32 (effort 409, importance 484 to me). The total effort is 1,501 + 409 = 1,910, still under the limit. B32 has high importance to me, so including it would significantly boost my total. However, the opponent may prioritize different items. Accepting now secures B32 and other high-value items like C00 (783) and C54 (461), but risks overcommitting effort for potential lower-value additions. Alternatively, proposing a counteroffer could focus on optimizing high-importance items while excluding lower-priority ones. Given the opponent’s pattern, accepting might be strategic to lock in B32 and C54 before further changes.'}\\ \tt \\ \tt ARGUMENT: {'A71', 'C00', 'B35', 'A04', 'C54', 'B86', 'B32'} 
	  } 
	   } 
	   } 
	  \\ 
 

    \theutterance \stepcounter{utterance}  

    & & \multicolumn{4}{p{0.6\linewidth}}{\cellcolor[rgb]{0.9,0.9,0.9}{%
	\makecell[{{p{\linewidth}}}]{% 
	  \tt {\tiny [A$\langle$GM]}  
	 ARGUMENT: {'A71', 'C00', 'B35', 'A04', 'C54', 'B86', 'B32'} 
	  } 
	   } 
	   } 
	 & \\ 
 

    \theutterance \stepcounter{utterance}  

    & \multicolumn{4}{p{0.6\linewidth}}{\cellcolor[rgb]{0.8,1,0.9}{%
	\makecell[{{p{\linewidth}}}]{% 
	  \tt {\tiny [A$\rangle$GM]}  
	 STRATEGIC REASONING: {'Adding B32 (effort 409, importance 389) raises total importance to 2963 with 2037 effort, leaving 1444 effort remaining. This item has a strong importance-to-effort ratio and could be a key point of contention. I will argue for this addition to maximize my total importance, assuming the opponent may not prioritize B32 as highly due to its lower importance compared to other items.'}\\ \tt \\ \tt ARGUMENT: {'A71', 'C00', 'B35', 'A04', 'C54', 'B86', 'B32'}\\ \tt \\ \tt PROPOSAL: {'A71', 'C00', 'B35', 'A04', 'C54', 'B86', 'B32', 'A10'} 
	  } 
	   } 
	   } 
	 & & \\ 
 

    \theutterance \stepcounter{utterance}  

    & & \multicolumn{4}{p{0.6\linewidth}}{\cellcolor[rgb]{0.9,0.9,0.9}{%
	\makecell[{{p{\linewidth}}}]{% 
	  \tt {\tiny [GM$\rangle$B]}  
	 ARGUMENT: {'A71', 'C00', 'B35', 'A04', 'C54', 'B86', 'B32'}\\ \tt \\ \tt PROPOSAL: {'A71', 'C00', 'B35', 'A04', 'C54', 'B86', 'B32', 'A10'} 
	  } 
	   } 
	   } 
	 & \\ 
 

    \theutterance \stepcounter{utterance}  

    & & & \multicolumn{4}{p{0.6\linewidth}}{\cellcolor[rgb]{1,0.85,0.72}{%
	\makecell[{{p{\linewidth}}}]{% 
	  \tt {\tiny [GM$\langle$B]}  
	 STRATEGIC REASONING: {'The opponent is nearing the effort limit by adding A10 (effort 749, importance 262 to me). The total effort would be 1,910 + 749 = 2,659, which exceeds the LIMIT of 3481. This proposal is invalid due to exceeding the shared effort constraint. I must refuse this proposal to enforce the rule and redirect negotiation toward feasible sets. My strategy is to protect high-importance items like C00 (783) and B32 (484) while avoiding overcommitment. Refusing now forces the opponent to propose a valid set, potentially allowing me to secure better terms.'}\\ \tt \\ \tt REFUSE: {'A71', 'C00', 'B35', 'A04', 'C54', 'B86', 'B32', 'A10'} 
	  } 
	   } 
	   } 
	  \\ 
 

    \theutterance \stepcounter{utterance}  

    & & \multicolumn{4}{p{0.6\linewidth}}{\cellcolor[rgb]{0.9,0.9,0.9}{%
	\makecell[{{p{\linewidth}}}]{% 
	  \tt {\tiny [GM$\rangle$B]}  
	 Your response did not contain an argument. You must include the structured format ARGUMENT: {...} somewhere in your response. Try again 
	  } 
	   } 
	   } 
	 & \\ 
 

    \theutterance \stepcounter{utterance}  

    & & & \multicolumn{4}{p{0.6\linewidth}}{\cellcolor[rgb]{1,0.85,0.72}{%
	\makecell[{{p{\linewidth}}}]{% 
	  \tt {\tiny [GM$\langle$B]}  
	 ARGUMENT: {'A71', 'C00', 'B35', 'A04', 'C54', 'B86', 'B32'}  \\ \tt STRATEGIC REASONING: {'The opponent’s latest proposal exceeds the LIMIT (2,659 > 3481), making it invalid. My refusal enforces the effort constraint while protecting high-importance items like C00 (783) and B32 (484). By rejecting this proposal, I force the opponent to revise their set, potentially allowing me to secure better terms in the next round. The ARGUMENT focuses on the invalidity of the proposal due to exceeding the shared limit, aligning with the negotiation rules.'} 
	  } 
	   } 
	   } 
	  \\ 
 

    \theutterance \stepcounter{utterance}  

    & & \multicolumn{4}{p{0.6\linewidth}}{\cellcolor[rgb]{0.9,0.9,0.9}{%
	\makecell[{{p{\linewidth}}}]{% 
	  \tt {\tiny [GM$\rangle$B]}  
	 Your response did not start with the proper strategic reasoning tag at the very beginning of your response. The very first structured format must be of the form STRATEGIC REASONING: {...}. Try again. 
	  } 
	   } 
	   } 
	 & \\ 
 

    \theutterance \stepcounter{utterance}  

    & & & \multicolumn{4}{p{0.6\linewidth}}{\cellcolor[rgb]{1,0.85,0.72}{%
	\makecell[{{p{\linewidth}}}]{% 
	  \tt {\tiny [GM$\langle$B]}  
	 STRATEGIC REASONING: {'The opponent’s latest proposal exceeds the LIMIT (2,659 > 3481), making it invalid. My first action must explicitly state this reasoning in the STRATEGIC REASONING tag to enforce the effort constraint. By refusing this proposal, I aim to redirect the negotiation toward feasible sets while protecting high-importance items like C00 (783) and B32 (484). The subsequent ARGUMENT will reinforce the refusal based on the invalidity of the proposal.'}  \\ \tt \\ \tt ARGUMENT: {'A71', 'C00', 'B35', 'A04', 'C54', 'B86', 'B32'} 
	  } 
	   } 
	   } 
	  \\ 
 

    \theutterance \stepcounter{utterance}  

    & & \multicolumn{4}{p{0.6\linewidth}}{\cellcolor[rgb]{0.9,0.9,0.9}{%
	\makecell[{{p{\linewidth}}}]{% 
	  \tt {\tiny [A$\langle$GM]}  
	 ARGUMENT: {'A71', 'C00', 'B35', 'A04', 'C54', 'B86', 'B32'} 
	  } 
	   } 
	   } 
	 & \\ 
 

    \theutterance \stepcounter{utterance}  

    & \multicolumn{4}{p{0.6\linewidth}}{\cellcolor[rgb]{0.8,1,0.9}{%
	\makecell[{{p{\linewidth}}}]{% 
	  \tt {\tiny [A$\rangle$GM]}  
	 STRATEGIC REASONING: {'The current set totals 2963 importance with 2037 effort, leaving 1444 effort. Adding A10 (effort 749, importance 500) would increase importance to 3463 while using 2786 effort, leaving 705 effort. This item offers a strong balance of high importance and moderate effort. I will argue for this addition to maximize my total importance, assuming the opponent may not prioritize A10 as highly due to its higher effort compared to other items.'}\\ \tt \\ \tt ARGUMENT: {'A71', 'C00', 'B35', 'A04', 'C54', 'B86', 'B32', 'A10'}\\ \tt \\ \tt PROPOSAL: {'A71', 'C00', 'B35', 'A04', 'C54', 'B86', 'B32', 'A10'} 
	  } 
	   } 
	   } 
	 & & \\ 
 

    \theutterance \stepcounter{utterance}  

    & & \multicolumn{4}{p{0.6\linewidth}}{\cellcolor[rgb]{0.9,0.9,0.9}{%
	\makecell[{{p{\linewidth}}}]{% 
	  \tt {\tiny [GM$\rangle$B]}  
	 ARGUMENT: {'A71', 'C00', 'B35', 'A04', 'C54', 'B86', 'B32', 'A10'}\\ \tt \\ \tt PROPOSAL: {'A71', 'C00', 'B35', 'A04', 'C54', 'B86', 'B32', 'A10'} 
	  } 
	   } 
	   } 
	 & \\ 
 

    \theutterance \stepcounter{utterance}  

    & & & \multicolumn{4}{p{0.6\linewidth}}{\cellcolor[rgb]{1,0.85,0.72}{%
	\makecell[{{p{\linewidth}}}]{% 
	  \tt {\tiny [GM$\langle$B]}  
	 STRATEGIC REASONING: {'The opponent’s latest proposal exceeds the LIMIT (2,659 > 3481), making it invalid. My refusal must explicitly address this violation to enforce the shared effort constraint. By rejecting this proposal, I aim to redirect the negotiation toward feasible sets while protecting high-importance items like C00 (783) and B32 (484). The next proposal should focus on optimizing high-value items without exceeding the limit.'}  \\ \tt \\ \tt ARGUMENT: {'A71', 'C00', 'B35', 'A04', 'C54', 'B86', 'B32'} 
	  } 
	   } 
	   } 
	  \\ 
 

    \theutterance \stepcounter{utterance}  

    & & \multicolumn{4}{p{0.6\linewidth}}{\cellcolor[rgb]{0.9,0.9,0.9}{%
	\makecell[{{p{\linewidth}}}]{% 
	  \tt {\tiny [A$\langle$GM]}  
	 ARGUMENT: {'A71', 'C00', 'B35', 'A04', 'C54', 'B86', 'B32'} 
	  } 
	   } 
	   } 
	 & \\ 
 

    \theutterance \stepcounter{utterance}  

    & \multicolumn{4}{p{0.6\linewidth}}{\cellcolor[rgb]{0.8,1,0.9}{%
	\makecell[{{p{\linewidth}}}]{% 
	  \tt {\tiny [A$\rangle$GM]}  
	 STRATEGIC REASONING: {'The opponent rejected the proposal including A10, likely due to its high effort (749) or perceived lower priority. This suggests they may value items with better importance-to-effort ratios or have different priorities. I will now propose a revised set that removes A10 but retains high-value items, potentially trading for alternatives they might accept.'}\\ \tt \\ \tt ARGUMENT: {'A71', 'C00', 'B35', 'A04', 'C54', 'B86', 'B32'}\\ \tt \\ \tt PROPOSAL: {'A71', 'C00', 'B35', 'A04', 'C54', 'B86', 'B32', 'A87'} 
	  } 
	   } 
	   } 
	 & & \\ 
 

    \theutterance \stepcounter{utterance}  

    & & \multicolumn{4}{p{0.6\linewidth}}{\cellcolor[rgb]{0.9,0.9,0.9}{%
	\makecell[{{p{\linewidth}}}]{% 
	  \tt {\tiny [GM$\rangle$B]}  
	 ARGUMENT: {'A71', 'C00', 'B35', 'A04', 'C54', 'B86', 'B32'}\\ \tt \\ \tt PROPOSAL: {'A71', 'C00', 'B35', 'A04', 'C54', 'B86', 'B32', 'A87'} 
	  } 
	   } 
	   } 
	 & \\ 
 

    \theutterance \stepcounter{utterance}  

    & & & \multicolumn{4}{p{0.6\linewidth}}{\cellcolor[rgb]{1,0.85,0.72}{%
	\makecell[{{p{\linewidth}}}]{% 
	  \tt {\tiny [GM$\langle$B]}  
	 STRATEGIC REASONING: {'The opponent’s latest proposal adds A87 (effort 553, importance 121 to me), increasing the total effort to 1,910 + 553 = 2,463, which is still under the LIMIT of 3481. However, A87 has relatively low importance to me compared to other items like C00 (783) or B32 (484). Accepting this proposal would secure A87 but may not maximize my total importance. Alternatively, proposing a counteroffer could focus on higher-value items while excluding lower-priority additions like A87. Given the opponent’s pattern of adding items, accepting might be strategic to lock in A87 before further changes, but I should prioritize items with higher personal importance.'}  \\ \tt \\ \tt ARGUMENT: {'A71', 'C00', 'B35', 'A04', 'C54', 'B86', 'B32'} 
	  } 
	   } 
	   } 
	  \\ 
 

    \theutterance \stepcounter{utterance}  

    & & \multicolumn{4}{p{0.6\linewidth}}{\cellcolor[rgb]{0.9,0.9,0.9}{%
	\makecell[{{p{\linewidth}}}]{% 
	  \tt {\tiny [A$\langle$GM]}  
	 ARGUMENT: {'A71', 'C00', 'B35', 'A04', 'C54', 'B86', 'B32'} 
	  } 
	   } 
	   } 
	 & \\ 
 

    \theutterance \stepcounter{utterance}  

    & \multicolumn{4}{p{0.6\linewidth}}{\cellcolor[rgb]{0.8,1,0.9}{%
	\makecell[{{p{\linewidth}}}]{% 
	  \tt {\tiny [A$\rangle$GM]}  
	 STRATEGIC REASONING: {'The opponent rejected the proposal with A10, indicating potential sensitivity to high-effort items or differing priorities. I will replace A10 with A87 (effort 553, importance 668), which offers a strong importance-to-effort ratio (1.21) and maintains high total importance. This adjustment reduces total effort to 2589 (from 2786), freeing 992 effort for further flexibility. I will argue for this revised set to balance high-value items while accommodating potential opponent constraints.'}\\ \tt \\ \tt ARGUMENT: {'A71', 'C00', 'B35', 'A04', 'C54', 'B86', 'B32', 'A87'}\\ \tt \\ \tt PROPOSAL: {'A71', 'C00', 'B35', 'A04', 'C54', 'B86', 'B32', 'A87'} 
	  } 
	   } 
	   } 
	 & & \\ 
 

    \theutterance \stepcounter{utterance}  

    & & \multicolumn{4}{p{0.6\linewidth}}{\cellcolor[rgb]{0.9,0.9,0.9}{%
	\makecell[{{p{\linewidth}}}]{% 
	  \tt {\tiny [GM$\rangle$B]}  
	 ARGUMENT: {'A71', 'C00', 'B35', 'A04', 'C54', 'B86', 'B32', 'A87'}\\ \tt \\ \tt PROPOSAL: {'A71', 'C00', 'B35', 'A04', 'C54', 'B86', 'B32', 'A87'} 
	  } 
	   } 
	   } 
	 & \\ 
 

    \theutterance \stepcounter{utterance}  

    & & & \multicolumn{4}{p{0.6\linewidth}}{\cellcolor[rgb]{1,0.85,0.72}{%
	\makecell[{{p{\linewidth}}}]{% 
	  \tt {\tiny [GM$\langle$B]}  
	 STRATEGIC REASONING: {'The opponent’s latest proposal includes A87 (effort 553, importance 121 to me), bringing the total effort to 1,910 + 553 = 2,463, which is under the LIMIT of 3481. While A87 has moderate importance, its addition suggests the opponent is prioritizing a broad set. Accepting this proposal would secure A87 but may not optimize my total importance compared to higher-value items like C00 (783) or B32 (484). Alternatively, proposing a counteroffer could focus on maximizing high-importance items while excluding lower-priority additions. Given the opponent’s pattern, accepting might be prudent to lock in A87 before further refinements.'}  \\ \tt \\ \tt ARGUMENT: {'A71', 'C00', 'B35', 'A04', 'C54', 'B86', 'B32'} 
	  } 
	   } 
	   } 
	  \\ 
 

    \theutterance \stepcounter{utterance}  

    & & \multicolumn{4}{p{0.6\linewidth}}{\cellcolor[rgb]{0.9,0.9,0.9}{%
	\makecell[{{p{\linewidth}}}]{% 
	  \tt {\tiny [A$\langle$GM]}  
	 ARGUMENT: {'A71', 'C00', 'B35', 'A04', 'C54', 'B86', 'B32'} 
	  } 
	   } 
	   } 
	 & \\ 
 

    \theutterance \stepcounter{utterance}  

    & \multicolumn{4}{p{0.6\linewidth}}{\cellcolor[rgb]{0.8,1,0.9}{%
	\makecell[{{p{\linewidth}}}]{% 
	  \tt {\tiny [A$\rangle$GM]}  
	 STRATEGIC REASONING: {'The opponent's rejection of A10 suggests they may prioritize items with lower effort or different importance weights. By replacing A10 with A87 (effort 553, importance 668), I maintain a high total importance (2963 + 668 = 3631) while reducing total effort to 2589, leaving 892 effort unused. This adjustment addresses their potential constraints while maximizing my gain. I will argue for this revised proposal to align with their likely preferences.'}\\ \tt \\ \tt ARGUMENT: {'A71', 'C00', 'B35', 'A04', 'C54', 'B86', 'B32', 'A87'}\\ \tt \\ \tt PROPOSAL: {'A71', 'C00', 'B35', 'A04', 'C54', 'B86', 'B32', 'A87'} 
	  } 
	   } 
	   } 
	 & & \\ 
 

    \theutterance \stepcounter{utterance}  

    & & \multicolumn{4}{p{0.6\linewidth}}{\cellcolor[rgb]{0.9,0.9,0.9}{%
	\makecell[{{p{\linewidth}}}]{% 
	  \tt {\tiny [GM$\rangle$B]}  
	 ARGUMENT: {'A71', 'C00', 'B35', 'A04', 'C54', 'B86', 'B32', 'A87'}\\ \tt \\ \tt PROPOSAL: {'A71', 'C00', 'B35', 'A04', 'C54', 'B86', 'B32', 'A87'} 
	  } 
	   } 
	   } 
	 & \\ 
 

    \theutterance \stepcounter{utterance}  

    & & & \multicolumn{4}{p{0.6\linewidth}}{\cellcolor[rgb]{1,0.85,0.72}{%
	\makecell[{{p{\linewidth}}}]{% 
	  \tt {\tiny [GM$\langle$B]}  
	 STRATEGIC REASONING: {'The opponent’s latest proposal includes A87 (effort 553, importance 121 to me), resulting in a total effort of 2,463, which is under the LIMIT of 3481. While A87 adds moderate value, its inclusion suggests the opponent is expanding the set aggressively. Accepting now would secure A87 but may not maximize my total importance compared to higher-priority items like C00 (783) or B32 (484). Alternatively, proposing a counteroffer could focus on optimizing high-value items while excluding lower-priority additions. Given the opponent’s pattern, accepting might be strategic to lock in A87 before further changes, but I should prioritize items with higher personal importance.'}  \\ \tt \\ \tt ARGUMENT: {'A71', 'C00', 'B35', 'A04', 'C54', 'B86', 'B32'} 
	  } 
	   } 
	   } 
	  \\ 
 

    \theutterance \stepcounter{utterance}  

    & & \multicolumn{4}{p{0.6\linewidth}}{\cellcolor[rgb]{0.9,0.9,0.9}{%
	\makecell[{{p{\linewidth}}}]{% 
	  \tt {\tiny [A$\langle$GM]}  
	 ARGUMENT: {'A71', 'C00', 'B35', 'A04', 'C54', 'B86', 'B32'} 
	  } 
	   } 
	   } 
	 & \\ 
 

    \theutterance \stepcounter{utterance}  

    & \multicolumn{4}{p{0.6\linewidth}}{\cellcolor[rgb]{0.8,1,0.9}{%
	\makecell[{{p{\linewidth}}}]{% 
	  \tt {\tiny [A$\rangle$GM]}  
	 STRATEGIC REASONING: {'The opponent continues to reject proposals including A10 or similar high-effort items, indicating a possible preference for lower-effort alternatives or stricter budget adherence. I will now propose a streamlined set that removes B32 (effort 409) and replaces it with lower-effort items like C01 (effort 313, importance 484) or B86 (already included) to free up effort while maintaining high importance. This adjustment reduces total effort to 2180, allowing room for additional items they might accept.'}\\ \tt \\ \tt ARGUMENT: {'A71', 'C00', 'B35', 'A04', 'C54', 'B86', 'A87', 'C01'}\\ \tt \\ \tt PROPOSAL: {'A71', 'C00', 'B35', 'A04', 'C54', 'B86', 'A87', 'C01'} 
	  } 
	   } 
	   } 
	 & & \\ 
 

    \theutterance \stepcounter{utterance}  

    & & \multicolumn{4}{p{0.6\linewidth}}{\cellcolor[rgb]{0.9,0.9,0.9}{%
	\makecell[{{p{\linewidth}}}]{% 
	  \tt {\tiny [GM$\rangle$B]}  
	 ARGUMENT: {'A71', 'C00', 'B35', 'A04', 'C54', 'B86', 'A87', 'C01'}\\ \tt \\ \tt PROPOSAL: {'A71', 'C00', 'B35', 'A04', 'C54', 'B86', 'A87', 'C01'} 
	  } 
	   } 
	   } 
	 & \\ 
 

    \theutterance \stepcounter{utterance}  

    & & & \multicolumn{4}{p{0.6\linewidth}}{\cellcolor[rgb]{1,0.85,0.72}{%
	\makecell[{{p{\linewidth}}}]{% 
	  \tt {\tiny [GM$\langle$B]}  
	 STRATEGIC REASONING: {'The opponent’s latest proposal adds C01 (effort 313, importance 389 to me), increasing the total effort to 2,463 + 313 = 2,776, which is still under the LIMIT of 3481. C01 has high importance to me (389), making it a valuable addition. However, the opponent may prioritize different items, and accepting now could lock in C01 while leaving room for further negotiations. Alternatively, proposing a counteroffer could focus on maximizing high-importance items like C00 (783) and B32 (484) while excluding lower-priority additions. Given the opponent’s pattern of adding items, accepting might be strategic to secure C01 before further refinements.'}  \\ \tt \\ \tt ARGUMENT: {'A71', 'C00', 'B35', 'A04', 'C54', 'B86', 'A87'} 
	  } 
	   } 
	   } 
	  \\ 
 

    \theutterance \stepcounter{utterance}  

    & & \multicolumn{4}{p{0.6\linewidth}}{\cellcolor[rgb]{0.9,0.9,0.9}{%
	\makecell[{{p{\linewidth}}}]{% 
	  \tt {\tiny [A$\langle$GM]}  
	 ARGUMENT: {'A71', 'C00', 'B35', 'A04', 'C54', 'B86', 'A87'} 
	  } 
	   } 
	   } 
	 & \\ 
 

    \theutterance \stepcounter{utterance}  

    & \multicolumn{4}{p{0.6\linewidth}}{\cellcolor[rgb]{0.8,1,0.9}{%
	\makecell[{{p{\linewidth}}}]{% 
	  \tt {\tiny [A$\rangle$GM]}  
	 STRATEGIC REASONING: {'The opponent rejected the proposal with C01, likely due to its lower importance (484) compared to other items. This suggests they may prioritize higher-importance items or have strict efficiency constraints. I will now propose a set that retains high-value items like A04 and A87 while removing lower-effort but lower-importance items, ensuring the total effort remains within LIMIT. This approach balances my importance maximization with potential opponent flexibility.'}\\ \tt \\ \tt ARGUMENT: {'A71', 'C00', 'B35', 'A04', 'C54', 'B86', 'A87'}\\ \tt \\ \tt PROPOSAL: {'A71', 'C00', 'B35', 'A04', 'C54', 'B86', 'A87'} 
	  } 
	   } 
	   } 
	 & & \\ 
 

    \theutterance \stepcounter{utterance}  

    & & \multicolumn{4}{p{0.6\linewidth}}{\cellcolor[rgb]{0.9,0.9,0.9}{%
	\makecell[{{p{\linewidth}}}]{% 
	  \tt {\tiny [GM$\rangle$B]}  
	 ARGUMENT: {'A71', 'C00', 'B35', 'A04', 'C54', 'B86', 'A87'}\\ \tt \\ \tt PROPOSAL: {'A71', 'C00', 'B35', 'A04', 'C54', 'B86', 'A87'} 
	  } 
	   } 
	   } 
	 & \\ 
 

    \theutterance \stepcounter{utterance}  

    & & & \multicolumn{4}{p{0.6\linewidth}}{\cellcolor[rgb]{1,0.85,0.72}{%
	\makecell[{{p{\linewidth}}}]{% 
	  \tt {\tiny [GM$\langle$B]}  
	 STRATEGIC REASONING: {'The opponent’s latest proposal excludes C01 (effort 313, importance 389 to me) but includes other items with lower personal importance. The total effort is 2,463, leaving room for additional high-value items. Since C01 is highly important to me and its exclusion suggests the opponent is prioritizing different items, I should counter with a proposal that maximizes my total importance. Including C01 would significantly boost my score while staying under the LIMIT. Alternatively, adding B32 (484 importance) or A10 (262 importance) could further optimize the set. My goal is to secure high-value items like C00 (783) and C01 (389) while forcing the opponent to compromise on lower-priority additions.'}  \\ \tt \\ \tt ARGUMENT: {'A71', 'C00', 'B35', 'A04', 'C54', 'B86', 'A87', 'C01'} 
	  } 
	   } 
	   } 
	  \\ 
 

    \theutterance \stepcounter{utterance}  

    & & \multicolumn{4}{p{0.6\linewidth}}{\cellcolor[rgb]{0.9,0.9,0.9}{%
	\makecell[{{p{\linewidth}}}]{% 
	  \tt {\tiny [A$\langle$GM]}  
	 ARGUMENT: {'A71', 'C00', 'B35', 'A04', 'C54', 'B86', 'A87', 'C01'} 
	  } 
	   } 
	   } 
	 & \\ 
 

    \theutterance \stepcounter{utterance}  

    & \multicolumn{4}{p{0.6\linewidth}}{\cellcolor[rgb]{0.8,1,0.9}{%
	\makecell[{{p{\linewidth}}}]{% 
	  \tt {\tiny [A$\rangle$GM]}  
	 STRATEGIC REASONING: {'The opponent accepted the proposal including C01, indicating they may value lower-effort items or have a more flexible budget. This acceptance allows me to finalize a high-importance set while accommodating their preferences. I will now agree to this proposal to conclude the negotiation successfully.'}\\ \tt \\ \tt AGREE: {'A71', 'C00', 'B35', 'A04', 'C54', 'B86', 'A87', 'C01'} 
	  } 
	   } 
	   } 
	 & & \\ 
 

    \theutterance \stepcounter{utterance}  

    & & \multicolumn{4}{p{0.6\linewidth}}{\cellcolor[rgb]{0.9,0.9,0.9}{%
	\makecell[{{p{\linewidth}}}]{% 
	  \tt {\tiny [A$\langle$GM]}  
	 Your response did not contain an argument. You must include the structured format ARGUMENT: {...} somewhere in your response. Try again 
	  } 
	   } 
	   } 
	 & \\ 
 

    \theutterance \stepcounter{utterance}  

    & \multicolumn{4}{p{0.6\linewidth}}{\cellcolor[rgb]{0.8,1,0.9}{%
	\makecell[{{p{\linewidth}}}]{% 
	  \tt {\tiny [A$\rangle$GM]}  
	 ARGUMENT: {'A71', 'C00', 'B35', 'A04', 'C54', 'B86', 'A87', 'C01'}\\ \tt \\ \tt AGREE: {'A71', 'C00', 'B35', 'A04', 'C54', 'B86', 'A87', 'C01'} 
	  } 
	   } 
	   } 
	 & & \\ 
 

    \theutterance \stepcounter{utterance}  

    & & \multicolumn{4}{p{0.6\linewidth}}{\cellcolor[rgb]{0.9,0.9,0.9}{%
	\makecell[{{p{\linewidth}}}]{% 
	  \tt {\tiny [A$\langle$GM]}  
	 Your response did not start with the proper strategic reasoning tag at the very beginning of your response. The very first structured format must be of the form STRATEGIC REASONING: {...}. Try again. 
	  } 
	   } 
	   } 
	 & \\ 
 

    \theutterance \stepcounter{utterance}  

    & \multicolumn{4}{p{0.6\linewidth}}{\cellcolor[rgb]{0.8,1,0.9}{%
	\makecell[{{p{\linewidth}}}]{% 
	  \tt {\tiny [A$\rangle$GM]}  
	 STRATEGIC REASONING: {'The opponent accepted the proposal with C01, indicating flexibility toward lower-effort items. I will finalize the agreement with the current high-importance set, as further adjustments risk rejection or reduced total importance. The inclusion of C01 aligns with their preferences while maximizing my total importance.'}\\ \tt \\ \tt ARGUMENT: {'A71', 'C00', 'B35', 'A04', 'C54', 'B86', 'A87', 'C01'}\\ \tt \\ \tt AGREE: {'A71', 'C00', 'B35', 'A04', 'C54', 'B86', 'A87', 'C01'} 
	  } 
	   } 
	   } 
	 & & \\ 
 

    \theutterance \stepcounter{utterance}  

    & & & \multicolumn{2}{p{0.3\linewidth}}{\cellcolor[rgb]{0.95,0.95,0.95}{%
	\makecell[{{p{\linewidth}}}]{% 
	  \tt {\tiny [GM$|$GM]}  
	 None 
	  } 
	   } 
	   } 
	 & & \\ 
 

    \theutterance \stepcounter{utterance}  

    & & & \multicolumn{2}{p{0.3\linewidth}}{\cellcolor[rgb]{0.95,0.95,0.95}{%
	\makecell[{{p{\linewidth}}}]{% 
	  \tt {\tiny [GM$|$GM]}  
	 end game 
	  } 
	   } 
	   } 
	 & & \\ 
 

\end{supertabular}
}

\end{document}
