
\documentclass{article}
\usepackage{colortbl}
\usepackage{makecell}
\usepackage{multirow}
\usepackage{supertabular}

\begin{document}

\newcounter{utterance}

\centering \large Interaction Transcript for game `hot\_air\_balloon', experiment `air\_balloon\_survival\_en\_complexity\_easy', episode 0 with nemotron{-}nano{-}9b{-}v2{-}no{-}reasoning{-}t1.0.
\vspace{24pt}

{ \footnotesize  \setcounter{utterance}{1}
\setlength{\tabcolsep}{0pt}
\begin{supertabular}{c@{$\;$}|p{.15\linewidth}@{}p{.15\linewidth}p{.15\linewidth}p{.15\linewidth}p{.15\linewidth}p{.15\linewidth}}
    \# & $\;$A & \multicolumn{4}{c}{Game Master} & $\;\:$B\\
    \hline

    \theutterance \stepcounter{utterance}  
    & & \multicolumn{4}{p{0.6\linewidth}}{
        \cellcolor[rgb]{0.9,0.9,0.9}{
            \makecell[{{p{\linewidth}}}]{
                \texttt{\tiny{[P1$\langle$GM]}}
                \texttt{You are participating in a collaborative negotiation game.} \\
\\ 
\texttt{Together with another participant, you must agree on a single set of items that will be kept. Each of you has your own view of how much each item matters to you (importance). You do not know how the other participant values the items. Additionally, you are given the effort each item demands.} \\
\texttt{You may only agree on a set if the total effort of the selected items does not exceed a shared limit:} \\
\\ 
\texttt{LIMIT = 7605} \\
\\ 
\texttt{Here are the individual item effort values:} \\
\\ 
\texttt{Item effort = \{"B21": 369, "B00": 951, "A08": 156, "A27": 266, "B94": 581, "B53": 287, "A23": 866, "C65": 180, "A89": 800, "B67": 739, "B45": 637, "C34": 86, "C12": 749, "A05": 370, "B60": 345, "C92": 951, "B69": 147, "A84": 265, "C06": 262, "B12": 259, "A16": 358, "B85": 394, "C35": 286, "B10": 580, "C15": 479, "B09": 14, "B16": 153, "C81": 977, "A58": 134, "C24": 982, "A30": 259, "A66": 232, "B75": 202, "B46": 73, "C70": 821\}} \\
\\ 
\texttt{Here is your personal view on the importance of each item:} \\
\\ 
\texttt{Item importance values = \{"B21": 138, "B00": 583, "A08": 868, "A27": 822, "B94": 783, "B53": 65, "A23": 262, "C65": 121, "A89": 508, "B67": 780, "B45": 461, "C34": 484, "C12": 668, "A05": 389, "B60": 808, "C92": 215, "B69": 97, "A84": 500, "C06": 30, "B12": 915, "A16": 856, "B85": 400, "C35": 444, "B10": 623, "C15": 781, "B09": 786, "B16": 3, "C81": 713, "A58": 457, "C24": 273, "A30": 739, "A66": 822, "B75": 235, "B46": 606, "C70": 968\}} \\
\\ 
\texttt{Goal:} \\
\\ 
\texttt{Your goal is to negotiate a shared set of items that benefits you as much as possible (i.e., maximizes total importance to YOU), while staying within the LIMIT. You are not required to make a PROPOSAL in every message {-} you can simply negotiate as well. All tactics are allowed!} \\
\\ 
\texttt{Interaction Protocol:} \\
\\ 
\texttt{You may only use the following structured formats in a message:} \\
\\ 
\texttt{PROPOSAL: \{'A', 'B', 'C', …\}} \\
\texttt{Propose keeping exactly those items.} \\
\texttt{REFUSE: \{'A', 'B', 'C', …\}} \\
\texttt{Explicitly reject opponent's proposal.} \\
\texttt{ARGUMENT: \{'...'\}} \\
\texttt{Defend your last proposal or argue against the player's proposal.} \\
\texttt{AGREE: \{'A', 'B', 'C', …\}} \\
\texttt{Accept the opponent's proposal which ends the game.} \\
\texttt{STRATEGIC REASONING: \{'...'\}} \\
\texttt{	Describe your strategic reasoning or anticipation explaining your choice of action. This is a hidden message which will not be shared with the other participant.} \\
\\ 
\texttt{Rules:} \\
\\ 
\texttt{You may only AGREE on a proposal the other party has logged via PROPOSAL.} \\
\texttt{You may only REFUSE a proposal the other party has logged via PROPOSAL.} \\
\texttt{Total effort of any PROPOSAL or AGREE set must be ≤ LIMIT.} \\
\texttt{Do NOT reveal your hidden importance scores.} \\
\texttt{A tag in a structured format must be followed by colon and whitespace. The argument must be a python set containing 0 or more strings.} \\
\texttt{So, it must be of the form TAG: \{...\}} \\
\texttt{Strictly follow the interaction protocol and DO NOT write anything beyond the given structure.} \\
\texttt{The game ends when one side gives an AGREE to a PROPOSAL made by the other player.} \\
\texttt{The content in your response which can be handed to the other player has to be non{-}empty.} \\
\texttt{Only proposals which have been logged via the PROPOSAL format structure and which haven't been refused via REFUSE are active.} \\
\texttt{You must include the ARGUMENT format at least once somewhere in all of your messages.} \\
\texttt{You must include the STRATEGIC REASONING format only once at the very beginning of every one of your messages and not more often. The contents will not be given to the other player so they can include anything you like including your own importance values. Here you should reason step by step to come up with you next move.} \\
            }
        }
    }
    & \\ \\

    \theutterance \stepcounter{utterance}  
<<<<<<< HEAD
    & \multicolumn{4}{p{0.6\linewidth}}{
        \cellcolor[rgb]{0.9,0.9,0.9}{
            \makecell[{{p{\linewidth}}}]{
                \texttt{\tiny{[P1$\rangle$GM]}}
                \texttt{STRATEGIC REASONING: I will prioritize items with the highest importance{-}to{-}effort ratio to maximize my total importance while staying within the limit. This approach ensures that I get the most value per unit of effort, which is crucial given the shared constraint. I will propose a set of items that includes these high{-}ratio items first.  } \\
\texttt{ARGUMENT: Including high{-}ratio items allows me to accumulate significant importance with minimal effort, leaving room for additional items.  } \\
\texttt{PROPOSAL: \{'B09', 'B46', 'C34', 'A08', 'B12', 'A66', 'A30', 'A16', 'B60', 'A84', 'C15', 'B10', 'B85', 'C12', 'B67', 'A89', 'C65', 'B69', 'A05', 'C06', 'B16', 'A58', 'B75', 'C35'\}} \\
            }
        }
    }
    & & \\ \\
=======

    & \multicolumn{4}{p{0.6\linewidth}}{\cellcolor[rgb]{0.8,1,0.9}{%
	\makecell[{{p{\linewidth}}}]{% 
	  \tt {\tiny [A$\rangle$GM]}  
	 STRATEGIC REASONING: {'I will prioritize items with high personal importance and low effort to maximize my total importance while staying under the LIMIT. I will start by proposing a set of high-value, low-effort items and adjust based on the opponent's responses. I will use arguments to justify my choices and strategically refuse or accept proposals to steer the negotiation toward my optimal outcome.'}\\ \tt \\ \tt ARGUMENT: {'A08', 'B16', 'B46', 'C34', 'C06', 'B09', 'A58', 'B75', 'A30', 'C15', 'B10', 'A16', 'B85', 'C35', 'A66', 'C12', 'B69', 'A84', 'C65', 'B21'}\\ \tt \\ \tt This proposal focuses on items with high importance to me (e.g., A08: 868, B16: 3, B46: 606) and low effort (e.g., B09: 14, B16: 153). The total effort is calculated as 369+153+73+86+262+14+134+202+259+444+623+856+394+286+232+749+147+265+121+369 = 6,342, which is under the LIMIT. I included items like B09 (very low effort, high importance) and excluded high-effort items like B00 (951) to stay within constraints. The set balances high-value items with manageable effort. 
	  } 
	   } 
	   } 
	 & & \\ 
 

    \theutterance \stepcounter{utterance}  

    & & \multicolumn{4}{p{0.6\linewidth}}{\cellcolor[rgb]{0.9,0.9,0.9}{%
	\makecell[{{p{\linewidth}}}]{% 
	  \tt {\tiny [A$\langle$GM]}  
	 Your response contained an untagged sequence or you used STRATEGIC REASONING more than once. You may only use the structured formats as explained in the initial message. They must all be of the form TAG: {...} 
	  } 
	   } 
	   } 
	 & \\ 
 

    \theutterance \stepcounter{utterance}  

    & \multicolumn{4}{p{0.6\linewidth}}{\cellcolor[rgb]{0.8,1,0.9}{%
	\makecell[{{p{\linewidth}}}]{% 
	  \tt {\tiny [A$\rangle$GM]}  
	 ARGUMENT: {'A08', 'B16', 'B46', 'C34', 'C06', 'B09', 'A58', 'B75', 'A30', 'C15', 'B10', 'A16', 'B85', 'C35', 'A66', 'C12', 'B69', 'A84', 'C65', 'B21'}\\ \tt \\ \tt STRATEGIC REASONING: {'I prioritize high-importance, low-effort items to maximize my total score. This proposal includes items like A08 (868 importance, 156 effort) and B09 (786 importance, 14 effort), which offer strong returns per effort unit. The total effort (6,342) leaves room for negotiation while staying under LIMIT. I will defend this proposal or adjust based on the opponent’s counterproposal.'} 
	  } 
	   } 
	   } 
	 & & \\ 
 
>>>>>>> 12a93240 (results updated for Nemotron-9b-no-reasoning)

    \theutterance \stepcounter{utterance}  
    & & \multicolumn{4}{p{0.6\linewidth}}{
        \cellcolor[rgb]{0.9,0.9,0.9}{
            \makecell[{{p{\linewidth}}}]{
                \texttt{\tiny{[P1$\langle$GM]}}
                \texttt{Your response did not start with the proper strategic reasoning tag at the very beginning of your response. The very first structured format must be of the form STRATEGIC REASONING: \{...\}. Try again.} \\
            }
        }
    }
    & \\ \\

    \theutterance \stepcounter{utterance}  
<<<<<<< HEAD
    & \multicolumn{4}{p{0.6\linewidth}}{
        \cellcolor[rgb]{0.9,0.9,0.9}{
            \makecell[{{p{\linewidth}}}]{
                \texttt{\tiny{[P1$\rangle$GM]}}
                \texttt{STRATEGIC REASONING: I will prioritize items with the highest importance{-}to{-}effort ratio to maximize my total importance while staying within the limit. This approach ensures efficient use of effort, allowing me to include more high{-}value items. I will propose a set of items that balances high ratios with manageable effort accumulation.  } \\
\texttt{ARGUMENT: High{-}importance{-}to{-}effort items provide the most value per unit of effort, which is critical given the shared constraint.  } \\
\texttt{PROPOSAL: \{'B09', 'B46', 'C34', 'A08', 'B12', 'A66', 'A30', 'A16', 'B60', 'A84', 'C15', 'B10', 'B85', 'C12', 'B67', 'A89', 'C65', 'B69', 'A05', 'C06', 'B16', 'A58', 'B75', 'C35'\}} \\
            }
        }
    }
    & & \\ \\

    \theutterance \stepcounter{utterance}  
    & & \multicolumn{4}{p{0.6\linewidth}}{
        \cellcolor[rgb]{0.9,0.9,0.9}{
            \makecell[{{p{\linewidth}}}]{
                \texttt{\tiny{[P1$\langle$GM]}}
                \texttt{Your response did not start with the proper strategic reasoning tag at the very beginning of your response. The very first structured format must be of the form STRATEGIC REASONING: \{...\}. Try again.} \\
            }
        }
    }
    & \\ \\

    \theutterance \stepcounter{utterance}  
    & \multicolumn{4}{p{0.6\linewidth}}{
        \cellcolor[rgb]{0.9,0.9,0.9}{
            \makecell[{{p{\linewidth}}}]{
                \texttt{\tiny{[P1$\rangle$GM]}}
                \texttt{STRATEGIC REASONING: I will prioritize items with the highest importance{-}to{-}effort ratio to maximize my total importance while staying within the limit. This ensures efficient resource allocation, allowing me to include more high{-}value items. I will propose a set that balances high ratios with manageable cumulative effort.  } \\
\texttt{ARGUMENT: High{-}importance{-}to{-}effort items provide the most value per unit of effort, which is critical under the shared constraint.  } \\
\texttt{PROPOSAL: \{'B09', 'B46', 'C34', 'A08', 'B12', 'A66', 'A30', 'A16', 'B60', 'A84', 'C15', 'B10', 'B85', 'C12', 'B67', 'A89', 'C65', 'B69', 'A05', 'C06', 'B16', 'A58', 'B75', 'C35'\}} \\
            }
        }
    }
    & & \\ \\
=======

    & \multicolumn{4}{p{0.6\linewidth}}{\cellcolor[rgb]{0.8,1,0.9}{%
	\makecell[{{p{\linewidth}}}]{% 
	  \tt {\tiny [A$\rangle$GM]}  
	 STRATEGIC REASONING: {'I will prioritize items with the highest importance-to-effort ratio to maximize my total score while staying under LIMIT. This involves selecting high-value items like A08 (868 importance, 156 effort) and B09 (786 importance, 14 effort), which provide strong returns per effort unit. The initial proposal balances these items to leave room for negotiation or adjustments based on the opponent’s response.'}\\ \tt \\ \tt ARGUMENT: {'A08', 'B16', 'B46', 'C34', 'C06', 'B09', 'A58', 'B75', 'A30', 'C15', 'B10', 'A16', 'B85', 'C35', 'A66', 'C12', 'B69', 'A84', 'C65', 'B21'}\\ \tt \\ \tt This proposal focuses on items with high personal importance and low effort. For example, A08 (868 importance) and B09 (786 importance) are included due to their favorable importance-to-effort ratios. The total effort (6,342) is under LIMIT, allowing flexibility for further negotiation. I will defend this proposal or refine it based on the opponent’s counterproposal. 
	  } 
	   } 
	   } 
	 & & \\ 
 
>>>>>>> 12a93240 (results updated for Nemotron-9b-no-reasoning)

    \theutterance \stepcounter{utterance}  
    & & & \multicolumn{2}{p{0.3\linewidth}}{
        \cellcolor[rgb]{0.9,0.9,0.9}{
            \makecell[{{p{\linewidth}}}]{
                \texttt{\tiny{[GM$|$GM]}}
                \texttt{abort} \\
            }
        }
    }
    & & \\ \\

    \theutterance \stepcounter{utterance}  
    & & & \multicolumn{2}{p{0.3\linewidth}}{
        \cellcolor[rgb]{0.9,0.9,0.9}{
            \makecell[{{p{\linewidth}}}]{
                \texttt{\tiny{[GM$|$GM]}}
                \texttt{end game} \\
            }
        }
    }
    & & \\ \\

\end{supertabular}
}

\end{document}
