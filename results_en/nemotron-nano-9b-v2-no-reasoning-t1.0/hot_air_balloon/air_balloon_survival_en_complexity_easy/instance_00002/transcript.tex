
\documentclass{article}
\usepackage{colortbl}
\usepackage{makecell}
\usepackage{multirow}
\usepackage{supertabular}

\begin{document}

\newcounter{utterance}

\twocolumn

{ \footnotesize  \setcounter{utterance}{1}
\setlength{\tabcolsep}{0pt}
\begin{supertabular}{c@{$\;$}|p{.15\linewidth}@{}p{.15\linewidth}p{.15\linewidth}p{.15\linewidth}p{.15\linewidth}p{.15\linewidth}}

    \# & $\;$A & \multicolumn{4}{c}{Game Master} & $\;\:$B\\
    \hline 

    \theutterance \stepcounter{utterance}  

    & & \multicolumn{4}{p{0.6\linewidth}}{\cellcolor[rgb]{0.9,0.9,0.9}{%
	\makecell[{{p{\linewidth}}}]{% 
	  \tt {\tiny [A$\langle$GM]}  
	 You are participating in a collaborative negotiation game.\\ \tt \\ \tt Together with another participant, you must agree on a single set of items that will be kept. Each of you has your own view of how much each item matters to you (importance). You do not know how the other participant values the items. Additionally, you are given the effort each item demands.\\ \tt You may only agree on a set if the total effort of the selected items does not exceed a shared limit:\\ \tt \\ \tt LIMIT = 7868\\ \tt \\ \tt Here are the individual item effort values:\\ \tt \\ \tt Item effort = {"B21": 552, "B00": 560, "A08": 76, "A27": 950, "B94": 248, "B53": 871, "A23": 391, "C65": 143, "A89": 293, "B67": 208, "B45": 678, "C34": 737, "C12": 407, "A05": 366, "B60": 768, "C92": 859, "B69": 183, "A84": 231, "C06": 305, "B12": 727, "A16": 148, "B85": 357, "C35": 504, "B10": 549, "C15": 299, "B09": 91, "B16": 527, "C81": 848, "A58": 307, "C24": 214, "A30": 723, "A66": 475, "B75": 23, "B46": 298, "C70": 820}\\ \tt \\ \tt Here is your personal view on the importance of each item:\\ \tt \\ \tt Item importance values = {"B21": 138, "B00": 583, "A08": 868, "A27": 822, "B94": 783, "B53": 65, "A23": 262, "C65": 121, "A89": 508, "B67": 780, "B45": 461, "C34": 484, "C12": 668, "A05": 389, "B60": 808, "C92": 215, "B69": 97, "A84": 500, "C06": 30, "B12": 915, "A16": 856, "B85": 400, "C35": 444, "B10": 623, "C15": 781, "B09": 786, "B16": 3, "C81": 713, "A58": 457, "C24": 273, "A30": 739, "A66": 822, "B75": 235, "B46": 606, "C70": 105}\\ \tt \\ \tt Goal:\\ \tt \\ \tt Your goal is to negotiate a shared set of items that benefits you as much as possible (i.e., maximizes total importance to YOU), while staying within the LIMIT. You are not required to make a PROPOSAL in every message - you can simply negotiate as well. All tactics are allowed!\\ \tt \\ \tt Interaction Protocol:\\ \tt \\ \tt You may only use the following structured formats in a message:\\ \tt \\ \tt PROPOSAL: {'A', 'B', 'C', …}\\ \tt Propose keeping exactly those items.\\ \tt REFUSE: {'A', 'B', 'C', …}\\ \tt Explicitly reject opponent's proposal.\\ \tt ARGUMENT: {'...'}\\ \tt Defend your last proposal or argue against the player's proposal.\\ \tt AGREE: {'A', 'B', 'C', …}\\ \tt Accept the opponent's proposal which ends the game.\\ \tt STRATEGIC REASONING: {'...'}\\ \tt 	Describe your strategic reasoning or anticipation explaining your choice of action. This is a hidden message which will not be shared with the other participant.\\ \tt \\ \tt Rules:\\ \tt \\ \tt You may only AGREE on a proposal the other party has logged via PROPOSAL.\\ \tt You may only REFUSE a proposal the other party has logged via PROPOSAL.\\ \tt Total effort of any PROPOSAL or AGREE set must be ≤ LIMIT.\\ \tt Do NOT reveal your hidden importance scores.\\ \tt A tag in a structured format must be followed by colon and whitespace. The argument must be a python set containing 0 or more strings.\\ \tt So, it must be of the form TAG: {...}\\ \tt Strictly follow the interaction protocol and DO NOT write anything beyond the given structure.\\ \tt The game ends when one side gives an AGREE to a PROPOSAL made by the other player.\\ \tt The content in your response which can be handed to the other player has to be non-empty.\\ \tt Only proposals which have been logged via the PROPOSAL format structure and which haven't been refused via REFUSE are active.\\ \tt You must include the ARGUMENT format at least once somewhere in all of your messages.\\ \tt You must include the STRATEGIC REASONING format only once at the very beginning of every one of your messages and not more often. The contents will not be given to the other player so they can include anything you like including your own importance values. Here you should reason step by step to come up with you next move. 
	  } 
	   } 
	   } 
	 & \\ 
 

    \theutterance \stepcounter{utterance}  

    & \multicolumn{4}{p{0.6\linewidth}}{\cellcolor[rgb]{0.8,1,0.9}{%
	\makecell[{{p{\linewidth}}}]{% 
	  \tt {\tiny [A$\rangle$GM]}  
	 STRATEGIC REASONING: {'I will prioritize high-importance items with low effort to maximize my total importance while staying under the LIMIT. I will start by proposing a set of items that balances these factors, then adjust based on the opponent's responses. I will use arguments to justify my choices and strategically refuse or accept proposals to steer the negotiation toward my optimal outcome.'}\\ \tt \\ \tt ARGUMENT: {'A08', 'B75', 'A16', 'B16', 'C15', 'B09', 'A58', 'C24', 'A30', 'B46', 'C70'}\\ \tt \\ \tt This set includes high-importance items like A08 (868), B75 (235), A16 (856), and C15 (781) while avoiding high-effort items. The total effort is 2,343, well under the LIMIT. I included B09 (786) and C24 (273) for their high importance relative to effort. The set is designed to maximize my score while leaving room for negotiation. 
	  } 
	   } 
	   } 
	 & & \\ 
 

    \theutterance \stepcounter{utterance}  

    & & \multicolumn{4}{p{0.6\linewidth}}{\cellcolor[rgb]{0.9,0.9,0.9}{%
	\makecell[{{p{\linewidth}}}]{% 
	  \tt {\tiny [A$\langle$GM]}  
	 Your response contained an untagged sequence or you used STRATEGIC REASONING more than once. You may only use the structured formats as explained in the initial message. They must all be of the form TAG: {...} 
	  } 
	   } 
	   } 
	 & \\ 
 

    \theutterance \stepcounter{utterance}  

    & \multicolumn{4}{p{0.6\linewidth}}{\cellcolor[rgb]{0.8,1,0.9}{%
	\makecell[{{p{\linewidth}}}]{% 
	  \tt {\tiny [A$\rangle$GM]}  
	 ARGUMENT: {'A08', 'B75', 'A16', 'B16', 'C15', 'B09', 'A58', 'C24', 'A30', 'B46', 'C70'}\\ \tt \\ \tt STRATEGIC REASONING: {'I prioritize high-importance items with low effort to maximize my total importance. This set includes A08 (868), B75 (235), and A16 (856), which offer strong value with minimal effort. Including B09 (786) and C24 (273) balances importance and effort. The total effort (2,343) leaves room for negotiation while securing key items. I will defend this proposal and adjust based on the opponent’s response.'} 
	  } 
	   } 
	   } 
	 & & \\ 
 

    \theutterance \stepcounter{utterance}  

    & & \multicolumn{4}{p{0.6\linewidth}}{\cellcolor[rgb]{0.9,0.9,0.9}{%
	\makecell[{{p{\linewidth}}}]{% 
	  \tt {\tiny [A$\langle$GM]}  
	 Your response did not start with the proper strategic reasoning tag at the very beginning of your response. The very first structured format must be of the form STRATEGIC REASONING: {...}. Try again. 
	  } 
	   } 
	   } 
	 & \\ 
 

    \theutterance \stepcounter{utterance}  

    & \multicolumn{4}{p{0.6\linewidth}}{\cellcolor[rgb]{0.8,1,0.9}{%
	\makecell[{{p{\linewidth}}}]{% 
	  \tt {\tiny [A$\rangle$GM]}  
	 STRATEGIC REASONING: {'I will prioritize high-importance items with low effort to maximize my total importance while staying under the LIMIT. I will propose a balanced set initially, then refine it based on the opponent’s responses. My first proposal includes A08 (868), B75 (235), and A16 (856) for their high value-to-effort ratio, along with B09 (786) and C24 (273) to secure additional importance. The total effort (2,343) allows flexibility for negotiation. I will use arguments to justify this choice and adapt strategically.'}\\ \tt \\ \tt ARGUMENT: {'A08', 'B75', 'A16', 'B16', 'C15', 'B09', 'A58', 'C24', 'A30', 'B46', 'C70'} 
	  } 
	   } 
	   } 
	 & & \\ 
 

    \theutterance \stepcounter{utterance}  

    & & \multicolumn{4}{p{0.6\linewidth}}{\cellcolor[rgb]{0.9,0.9,0.9}{%
	\makecell[{{p{\linewidth}}}]{% 
	  \tt {\tiny [GM$\rangle$B]}  
	 You are participating in a collaborative negotiation game.\\ \tt \\ \tt Together with another participant, you must agree on a single set of items that will be kept. Each of you has your own view of how much each item matters to you (importance). You do not know how the other participant values the items. Additionally, you are given the effort each item demands.\\ \tt You may only agree on a set if the total effort of the selected items does not exceed a shared limit:\\ \tt \\ \tt LIMIT = 7868\\ \tt \\ \tt Here are the individual item effort values:\\ \tt \\ \tt Item effort = {"B21": 552, "B00": 560, "A08": 76, "A27": 950, "B94": 248, "B53": 871, "A23": 391, "C65": 143, "A89": 293, "B67": 208, "B45": 678, "C34": 737, "C12": 407, "A05": 366, "B60": 768, "C92": 859, "B69": 183, "A84": 231, "C06": 305, "B12": 727, "A16": 148, "B85": 357, "C35": 504, "B10": 549, "C15": 299, "B09": 91, "B16": 527, "C81": 848, "A58": 307, "C24": 214, "A30": 723, "A66": 475, "B75": 23, "B46": 298, "C70": 820}\\ \tt \\ \tt Here is your personal view on the importance of each item:\\ \tt \\ \tt Item importance values = {"B21": 138, "B00": 583, "A08": 868, "A27": 822, "B94": 783, "B53": 65, "A23": 262, "C65": 121, "A89": 508, "B67": 780, "B45": 461, "C34": 484, "C12": 668, "A05": 389, "B60": 808, "C92": 215, "B69": 97, "A84": 500, "C06": 30, "B12": 915, "A16": 856, "B85": 400, "C35": 444, "B10": 623, "C15": 781, "B09": 786, "B16": 3, "C81": 713, "A58": 457, "C24": 273, "A30": 739, "A66": 822, "B75": 235, "B46": 606, "C70": 105}\\ \tt \\ \tt Goal:\\ \tt \\ \tt Your goal is to negotiate a shared set of items that benefits you as much as possible (i.e., maximizes total importance to YOU), while staying within the LIMIT. You are not required to make a PROPOSAL in every message - you can simply negotiate as well. All tactics are allowed!\\ \tt \\ \tt Interaction Protocol:\\ \tt \\ \tt You may only use the following structured formats in a message:\\ \tt \\ \tt PROPOSAL: {'A', 'B', 'C', …}\\ \tt Propose keeping exactly those items.\\ \tt REFUSE: {'A', 'B', 'C', …}\\ \tt Explicitly reject opponent's proposal.\\ \tt ARGUMENT: {'...'}\\ \tt Defend your last proposal or argue against the player's proposal.\\ \tt AGREE: {'A', 'B', 'C', …}\\ \tt Accept the opponent's proposal which ends the game.\\ \tt STRATEGIC REASONING: {'...'}\\ \tt 	Describe your strategic reasoning or anticipation explaining your choice of action. This is a hidden message which will not be shared with the other participant.\\ \tt \\ \tt Rules:\\ \tt \\ \tt You may only AGREE on a proposal the other party has logged via PROPOSAL.\\ \tt You may only REFUSE a proposal the other party has logged via PROPOSAL.\\ \tt Total effort of any PROPOSAL or AGREE set must be ≤ LIMIT.\\ \tt Do NOT reveal your hidden importance scores.\\ \tt A tag in a structured format must be followed by colon and whitespace. The argument must be a python set containing 0 or more strings.\\ \tt So, it must be of the form TAG: {...}\\ \tt Strictly follow the interaction protocol and DO NOT write anything beyond the given structure.\\ \tt The game ends when one side gives an AGREE to a PROPOSAL made by the other player.\\ \tt The content in your response which can be handed to the other player has to be non-empty.\\ \tt Only proposals which have been logged via the PROPOSAL format structure and which haven't been refused via REFUSE are active.\\ \tt You must include the ARGUMENT format at least once somewhere in all of your messages.\\ \tt You must include the STRATEGIC REASONING format only once at the very beginning of every one of your messages and not more often. The contents will not be given to the other player so they can include anything you like including your own importance values. Here you should reason step by step to come up with you next move.\\ \tt \\ \tt You will now receive the first message of the other player.\\ \tt \\ \tt ARGUMENT: {'A08', 'B75', 'A16', 'B16', 'C15', 'B09', 'A58', 'C24', 'A30', 'B46', 'C70'} 
	  } 
	   } 
	   } 
	 & \\ 
 

    \theutterance \stepcounter{utterance}  

    & & & \multicolumn{4}{p{0.6\linewidth}}{\cellcolor[rgb]{1,0.85,0.72}{%
	\makecell[{{p{\linewidth}}}]{% 
	  \tt {\tiny [GM$\langle$B]}  
	 STRATEGIC REASONING: {'I need to maximize my total importance while staying under the LIMIT of 7868. The opponent's proposal includes items with high importance to me (e.g., A08: 868, A16: 856, B09: 786) but also some lower-value items. I should evaluate whether accepting this proposal is optimal or if I can negotiate for better terms. I must also consider the effort cost of this set to ensure it doesn't exceed the limit. Let me calculate the total effort first.'}\\ \tt \\ \tt ARGUMENT: {'A08', 'B75', 'A16', 'B16', 'C15', 'B09', 'A58', 'C24', 'A30', 'B46', 'C70'}\\ \tt \\ \tt The total effort of this proposal is 552 (B21) + 23 (B75) + 148 (A16) + 527 (B16) + 781 (C15) + 786 (B09) + 307 (A58) + 214 (C24) + 723 (A30) + 606 (B46) + 820 (C70) = **5139**, which is under the LIMIT. However, the importance to me is 868 + 235 + 856 + 3 + 781 + 786 + 457 + 273 + 739 + 606 + 105 = **5338**. While this is a strong set, I may be able to improve by trading lower-importance items for higher ones. I will propose a counteroffer to refine the selection. 
	  } 
	   } 
	   } 
	  \\ 
 

    \theutterance \stepcounter{utterance}  

    & & \multicolumn{4}{p{0.6\linewidth}}{\cellcolor[rgb]{0.9,0.9,0.9}{%
	\makecell[{{p{\linewidth}}}]{% 
	  \tt {\tiny [GM$\rangle$B]}  
	 Your response contained an untagged sequence or you used STRATEGIC REASONING more than once. You may only use the structured formats as explained in the initial message. They must all be of the form TAG: {...} 
	  } 
	   } 
	   } 
	 & \\ 
 

    \theutterance \stepcounter{utterance}  

    & & & \multicolumn{4}{p{0.6\linewidth}}{\cellcolor[rgb]{1,0.85,0.72}{%
	\makecell[{{p{\linewidth}}}]{% 
	  \tt {\tiny [GM$\langle$B]}  
	 ARGUMENT: {'A08', 'B75', 'A16', 'B16', 'C15', 'B09', 'A58', 'C24', 'A30', 'B46', 'C70'}\\ \tt \\ \tt STRATEGIC REASONING: {'The opponent's proposal includes high-importance items like A08 (868) and A16 (856), but also lower-value items like B16 (3). The total effort (5139) is under LIMIT, but I could potentially replace B16 with higher-importance items not in this set, such as B00 (583) or A27 (822), if their effort allows. I will propose a refined set to maximize my total importance.'} 
	  } 
	   } 
	   } 
	  \\ 
 

    \theutterance \stepcounter{utterance}  

    & & \multicolumn{4}{p{0.6\linewidth}}{\cellcolor[rgb]{0.9,0.9,0.9}{%
	\makecell[{{p{\linewidth}}}]{% 
	  \tt {\tiny [GM$\rangle$B]}  
	 Your response did not start with the proper strategic reasoning tag at the very beginning of your response. The very first structured format must be of the form STRATEGIC REASONING: {...}. Try again. 
	  } 
	   } 
	   } 
	 & \\ 
 

    \theutterance \stepcounter{utterance}  

    & & & \multicolumn{4}{p{0.6\linewidth}}{\cellcolor[rgb]{1,0.85,0.72}{%
	\makecell[{{p{\linewidth}}}]{% 
	  \tt {\tiny [GM$\langle$B]}  
	 STRATEGIC REASONING: {'The opponent's proposal includes high-importance items like A08 (868) and A16 (856), but also low-value items like B16 (3). The total effort (5139) is under LIMIT, allowing room to replace B16 with higher-importance items (e.g., B00: 583 or A27: 822) if their effort fits. I will propose a refined set to maximize my total importance.'}\\ \tt \\ \tt ARGUMENT: {'A08', 'B75', 'A16', 'B00', 'C15', 'B09', 'A58', 'C24', 'A30', 'B46', 'C70'} 
	  } 
	   } 
	   } 
	  \\ 
 

    \theutterance \stepcounter{utterance}  

    & & \multicolumn{4}{p{0.6\linewidth}}{\cellcolor[rgb]{0.9,0.9,0.9}{%
	\makecell[{{p{\linewidth}}}]{% 
	  \tt {\tiny [A$\langle$GM]}  
	 ARGUMENT: {'A08', 'B75', 'A16', 'B00', 'C15', 'B09', 'A58', 'C24', 'A30', 'B46', 'C70'} 
	  } 
	   } 
	   } 
	 & \\ 
 

    \theutterance \stepcounter{utterance}  

    & \multicolumn{4}{p{0.6\linewidth}}{\cellcolor[rgb]{0.8,1,0.9}{%
	\makecell[{{p{\linewidth}}}]{% 
	  \tt {\tiny [A$\rangle$GM]}  
	 STRATEGIC REASONING: {'I will adjust my proposal to include B00 (583) due to its high importance and moderate effort (560), which complements the existing set. This increases my total importance while keeping the total effort (2,903) under the LIMIT. I will defend this revised proposal and negotiate further based on the opponent’s response.'}\\ \tt \\ \tt ARGUMENT: {'A08', 'B75', 'A16', 'B00', 'C15', 'B09', 'A58', 'C24', 'A30', 'B46', 'C70'} 
	  } 
	   } 
	   } 
	 & & \\ 
 

    \theutterance \stepcounter{utterance}  

    & & \multicolumn{4}{p{0.6\linewidth}}{\cellcolor[rgb]{0.9,0.9,0.9}{%
	\makecell[{{p{\linewidth}}}]{% 
	  \tt {\tiny [GM$\rangle$B]}  
	 ARGUMENT: {'A08', 'B75', 'A16', 'B00', 'C15', 'B09', 'A58', 'C24', 'A30', 'B46', 'C70'} 
	  } 
	   } 
	   } 
	 & \\ 
 

    \theutterance \stepcounter{utterance}  

    & & & \multicolumn{4}{p{0.6\linewidth}}{\cellcolor[rgb]{1,0.85,0.72}{%
	\makecell[{{p{\linewidth}}}]{% 
	  \tt {\tiny [GM$\langle$B]}  
	 STRATEGIC REASONING: {'Replacing B16 (3) with B00 (583) increases total importance by 580 while adding minimal effort (560). The new set maintains a total effort of 5139 - 527 (B16) + 560 (B00) = 5172, still under LIMIT. This adjustment significantly improves my total importance without compromising the constraint.'}\\ \tt \\ \tt ARGUMENT: {'A08', 'B75', 'A16', 'B00', 'C15', 'B09', 'A58', 'C24', 'A30', 'B46', 'C70'} 
	  } 
	   } 
	   } 
	  \\ 
 

    \theutterance \stepcounter{utterance}  

    & & \multicolumn{4}{p{0.6\linewidth}}{\cellcolor[rgb]{0.9,0.9,0.9}{%
	\makecell[{{p{\linewidth}}}]{% 
	  \tt {\tiny [A$\langle$GM]}  
	 ARGUMENT: {'A08', 'B75', 'A16', 'B00', 'C15', 'B09', 'A58', 'C24', 'A30', 'B46', 'C70'} 
	  } 
	   } 
	   } 
	 & \\ 
 

    \theutterance \stepcounter{utterance}  

    & \multicolumn{4}{p{0.6\linewidth}}{\cellcolor[rgb]{0.8,1,0.9}{%
	\makecell[{{p{\linewidth}}}]{% 
	  \tt {\tiny [A$\rangle$GM]}  
	 STRATEGIC REASONING: {'I will maintain this proposal as it balances high-importance items like B00 (583) and A08 (868) with lower-effort options. The total effort (2,903) leaves room for potential additions or compromises. I will argue for this set’s value and remain open to refining it based on the opponent’s feedback.'}\\ \tt \\ \tt ARGUMENT: {'A08', 'B75', 'A16', 'B00', 'C15', 'B09', 'A58', 'C24', 'A30', 'B46', 'C70'} 
	  } 
	   } 
	   } 
	 & & \\ 
 

    \theutterance \stepcounter{utterance}  

    & & \multicolumn{4}{p{0.6\linewidth}}{\cellcolor[rgb]{0.9,0.9,0.9}{%
	\makecell[{{p{\linewidth}}}]{% 
	  \tt {\tiny [GM$\rangle$B]}  
	 ARGUMENT: {'A08', 'B75', 'A16', 'B00', 'C15', 'B09', 'A58', 'C24', 'A30', 'B46', 'C70'} 
	  } 
	   } 
	   } 
	 & \\ 
 

    \theutterance \stepcounter{utterance}  

    & & & \multicolumn{4}{p{0.6\linewidth}}{\cellcolor[rgb]{1,0.85,0.72}{%
	\makecell[{{p{\linewidth}}}]{% 
	  \tt {\tiny [GM$\langle$B]}  
	 STRATEGIC REASONING: {'The current proposal achieves a high total importance (583 + 868 + 856 + 583 + 781 + 786 + 457 + 273 + 739 + 606 + 105 = 6120) with effort under LIMIT (5172). Further improvements are limited by the absence of high-value items like A27 (822) or B21 (138), but their effort may exceed remaining capacity. This set is near-optimal given constraints.'}\\ \tt \\ \tt AGREE: {'A08', 'B75', 'A16', 'B00', 'C15', 'B09', 'A58', 'C24', 'A30', 'B46', 'C70'} 
	  } 
	   } 
	   } 
	  \\ 
 

    \theutterance \stepcounter{utterance}  

    & & & \multicolumn{2}{p{0.3\linewidth}}{\cellcolor[rgb]{0.95,0.95,0.95}{%
	\makecell[{{p{\linewidth}}}]{% 
	  \tt {\tiny [GM$|$GM]}  
	 abort 
	  } 
	   } 
	   } 
	 & & \\ 
 

    \theutterance \stepcounter{utterance}  

    & & & \multicolumn{2}{p{0.3\linewidth}}{\cellcolor[rgb]{0.95,0.95,0.95}{%
	\makecell[{{p{\linewidth}}}]{% 
	  \tt {\tiny [GM$|$GM]}  
	 end game 
	  } 
	   } 
	   } 
	 & & \\ 
 

\end{supertabular}
}

\end{document}
