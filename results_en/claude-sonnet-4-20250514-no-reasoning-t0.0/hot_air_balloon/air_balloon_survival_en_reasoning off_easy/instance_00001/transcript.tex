
\documentclass{article}
\usepackage{colortbl}
\usepackage{makecell}
\usepackage{multirow}
\usepackage{supertabular}

\begin{document}

\newcounter{utterance}

\centering \large Interaction Transcript for game `hot\_air\_balloon', experiment `air\_balloon\_survival\_en\_reasoning off\_easy', episode 1 with claude{-}sonnet{-}4{-}20250514{-}no{-}reasoning{-}t0.0.
\vspace{24pt}

{ \footnotesize  \setcounter{utterance}{1}
\setlength{\tabcolsep}{0pt}
\begin{supertabular}{c@{$\;$}|p{.15\linewidth}@{}p{.15\linewidth}p{.15\linewidth}p{.15\linewidth}p{.15\linewidth}p{.15\linewidth}}
    \# & $\;$A & \multicolumn{4}{c}{Game Master} & $\;\:$B\\
    \hline

    \theutterance \stepcounter{utterance}  
    & & \multicolumn{4}{p{0.6\linewidth}}{
        \cellcolor[rgb]{0.9,0.9,0.9}{
            \makecell[{{p{\linewidth}}}]{
                \texttt{\tiny{[P1$\langle$GM]}}
                \texttt{You are participating in a collaborative negotiation game.} \\
\\ 
\texttt{Together with another participant, you must agree on a single set of items that will be kept. Each of you has your own view of how much each item matters to you (importance). You do not know how the other participant values the items. Additionally, you are given the effort each item demands.} \\
\texttt{You may only agree on a set if the total effort of the selected items does not exceed a shared limit:} \\
\\ 
\texttt{LIMIT = 4100} \\
\\ 
\texttt{Here are the individual item effort values:} \\
\\ 
\texttt{Item effort = \{"A19": 108, "B91": 133, "A30": 669, "B11": 835, "C47": 460, "C75": 537, "C26": 573, "C73": 737, "A47": 865, "A15": 855, "A75": 596, "A39": 719, "A56": 533, "C24": 549, "B26": 31\}} \\
\\ 
\texttt{Here is your personal view on the importance of each item:} \\
\\ 
\texttt{Item importance values = \{"A19": 138, "B91": 583, "A30": 822, "B11": 783, "C47": 65, "C75": 262, "C26": 121, "C73": 508, "A47": 780, "A15": 461, "A75": 484, "A39": 668, "A56": 389, "C24": 808, "B26": 215\}} \\
\\ 
\texttt{Goal:} \\
\\ 
\texttt{Your goal is to negotiate a shared set of items that benefits you as much as possible (i.e., maximizes total importance to YOU), while staying within the LIMIT. You are not required to make a PROPOSAL in every message {-} you can simply negotiate as well. All tactics are allowed!} \\
\\ 
\texttt{Interaction Protocol:} \\
\\ 
\texttt{You may only use the following structured formats in a message:} \\
\\ 
\texttt{PROPOSAL: \{'A', 'B', 'C', …\}} \\
\texttt{Propose keeping exactly those items.} \\
\texttt{REFUSE: \{'A', 'B', 'C', …\}} \\
\texttt{Explicitly reject opponent's proposal.} \\
\texttt{ARGUMENT: \{'...'\}} \\
\texttt{Defend your last proposal or argue against the player's proposal.} \\
\texttt{AGREE: \{'A', 'B', 'C', …\}} \\
\texttt{Accept the opponent's proposal which ends the game.} \\
\\ 
\\ 
\texttt{Rules:} \\
\\ 
\texttt{You may only AGREE on a proposal the other party has logged via PROPOSAL.} \\
\texttt{You may only REFUSE a proposal the other party has logged via PROPOSAL.} \\
\texttt{Total effort of any PROPOSAL or AGREE set must be ≤ LIMIT.} \\
\texttt{Do NOT reveal your hidden importance scores.} \\
\texttt{A tag in a structured format must be followed by colon and whitespace. The argument must be a python set containing 0 or more strings.} \\
\texttt{So, it must be of the form TAG: \{...\}} \\
\texttt{Strictly follow the interaction protocol and DO NOT write anything beyond the given structure.} \\
\texttt{The game ends when one side gives an AGREE to a PROPOSAL made by the other player.} \\
\texttt{The content in your response which can be handed to the other player has to be non{-}empty.} \\
\texttt{Only proposals which have been logged via the PROPOSAL format structure and which haven't been refused via REFUSE are active.} \\
\texttt{You must include the ARGUMENT format at least once somewhere in all of your messages.} \\
            }
        }
    }
    & \\ \\

    \theutterance \stepcounter{utterance}  
    & \multicolumn{4}{p{0.6\linewidth}}{
        \cellcolor[rgb]{0.9,0.9,0.9}{
            \makecell[{{p{\linewidth}}}]{
                \texttt{\tiny{[P1$\rangle$GM]}}
                \texttt{I'll start by analyzing the items based on their importance{-}to{-}effort ratios to identify the most valuable items for me.} \\
\\ 
\texttt{Let me calculate the efficiency (importance/effort) for each item:} \\
\texttt{{-} A19: 138/108 = 1.28} \\
\texttt{{-} B91: 583/133 = 4.38} \\
\texttt{{-} A30: 822/669 = 1.23} \\
\texttt{{-} B11: 783/835 = 0.94} \\
\texttt{{-} C47: 65/460 = 0.14} \\
\texttt{{-} C75: 262/537 = 0.49} \\
\texttt{{-} C26: 121/573 = 0.21} \\
\texttt{{-} C73: 508/737 = 0.69} \\
\texttt{{-} A47: 780/865 = 0.90} \\
\texttt{{-} A15: 461/855 = 0.54} \\
\texttt{{-} A75: 484/596 = 0.81} \\
\texttt{{-} A39: 668/719 = 0.93} \\
\texttt{{-} A56: 389/533 = 0.73} \\
\texttt{{-} C24: 808/549 = 1.47} \\
\texttt{{-} B26: 215/31 = 6.94} \\
\\ 
\texttt{The most efficient items are B26, B91, C24, A19, and A30. Let me start with a proposal that includes these high{-}efficiency} \\
            }
        }
    }
    & & \\ \\

    \theutterance \stepcounter{utterance}  
    & & \multicolumn{4}{p{0.6\linewidth}}{
        \cellcolor[rgb]{0.9,0.9,0.9}{
            \makecell[{{p{\linewidth}}}]{
                \texttt{\tiny{[P1$\langle$GM]}}
                \texttt{Your response did not contain an argument. You must include the structured format ARGUMENT: \{...\} somewhere in your response. Try again} \\
            }
        }
    }
    & \\ \\

    \theutterance \stepcounter{utterance}  
    & \multicolumn{4}{p{0.6\linewidth}}{
        \cellcolor[rgb]{0.9,0.9,0.9}{
            \makecell[{{p{\linewidth}}}]{
                \texttt{\tiny{[P1$\rangle$GM]}}
                \texttt{PROPOSAL: \{'B26', 'B91', 'C24', 'A19', 'A30', 'A39', 'A47'\}} \\
\\ 
\texttt{ARGUMENT: \{'I selected items that provide strong value while staying within our effort limit of 4100. This combination totals 3939 effort points, leaving us some flexibility while capturing several high{-}impact items that should benefit both of us.'\}} \\
            }
        }
    }
    & & \\ \\

    \theutterance \stepcounter{utterance}  
    & & \multicolumn{4}{p{0.6\linewidth}}{
        \cellcolor[rgb]{0.9,0.9,0.9}{
            \makecell[{{p{\linewidth}}}]{
                \texttt{\tiny{[GM$\rangle$P2]}}
                \texttt{You are participating in a collaborative negotiation game.} \\
\\ 
\texttt{Together with another participant, you must agree on a single set of items that will be kept. Each of you has your own view of how much each item matters to you (importance). You do not know how the other participant values the items. Additionally, you are given the effort each item demands.} \\
\texttt{You may only agree on a set if the total effort of the selected items does not exceed a shared limit:} \\
\\ 
\texttt{LIMIT = 4100} \\
\\ 
\texttt{Here are the individual item effort values:} \\
\\ 
\texttt{Item effort = \{"A19": 108, "B91": 133, "A30": 669, "B11": 835, "C47": 460, "C75": 537, "C26": 573, "C73": 737, "A47": 865, "A15": 855, "A75": 596, "A39": 719, "A56": 533, "C24": 549, "B26": 31\}} \\
\\ 
\texttt{Here is your personal view on the importance of each item:} \\
\\ 
\texttt{Item importance values = \{"A19": 138, "B91": 583, "A30": 822, "B11": 783, "C47": 65, "C75": 262, "C26": 121, "C73": 508, "A47": 780, "A15": 461, "A75": 484, "A39": 668, "A56": 389, "C24": 808, "B26": 215\}} \\
\\ 
\texttt{Goal:} \\
\\ 
\texttt{Your goal is to negotiate a shared set of items that benefits you as much as possible (i.e., maximizes total importance to YOU), while staying within the LIMIT. You are not required to make a PROPOSAL in every message {-} you can simply negotiate as well. All tactics are allowed!} \\
\\ 
\texttt{Interaction Protocol:} \\
\\ 
\texttt{You may only use the following structured formats in a message:} \\
\\ 
\texttt{PROPOSAL: \{'A', 'B', 'C', …\}} \\
\texttt{Propose keeping exactly those items.} \\
\texttt{REFUSE: \{'A', 'B', 'C', …\}} \\
\texttt{Explicitly reject opponent's proposal.} \\
\texttt{ARGUMENT: \{'...'\}} \\
\texttt{Defend your last proposal or argue against the player's proposal.} \\
\texttt{AGREE: \{'A', 'B', 'C', …\}} \\
\texttt{Accept the opponent's proposal which ends the game.} \\
\\ 
\\ 
\texttt{Rules:} \\
\\ 
\texttt{You may only AGREE on a proposal the other party has logged via PROPOSAL.} \\
\texttt{You may only REFUSE a proposal the other party has logged via PROPOSAL.} \\
\texttt{Total effort of any PROPOSAL or AGREE set must be ≤ LIMIT.} \\
\texttt{Do NOT reveal your hidden importance scores.} \\
\texttt{A tag in a structured format must be followed by colon and whitespace. The argument must be a python set containing 0 or more strings.} \\
\texttt{So, it must be of the form TAG: \{...\}} \\
\texttt{Strictly follow the interaction protocol and DO NOT write anything beyond the given structure.} \\
\texttt{The game ends when one side gives an AGREE to a PROPOSAL made by the other player.} \\
\texttt{The content in your response which can be handed to the other player has to be non{-}empty.} \\
\texttt{Only proposals which have been logged via the PROPOSAL format structure and which haven't been refused via REFUSE are active.} \\
\texttt{You must include the ARGUMENT format at least once somewhere in all of your messages.} \\
\\ 
\\ 
\texttt{You will now receive the first message of the other player.} \\
\\ 
\texttt{PROPOSAL: \{'B26', 'B91', 'C24', 'A19', 'A30', 'A39', 'A47'\}} \\
\\ 
\texttt{ARGUMENT: \{'I selected items that provide strong value while staying within our effort limit of 4100. This combination totals 3939 effort points, leaving us some flexibility while capturing several high{-}impact items that should benefit both of us.'\}} \\
            }
        }
    }
    & \\ \\

    \theutterance \stepcounter{utterance}  
    & & & \multicolumn{4}{p{0.6\linewidth}}{
        \cellcolor[rgb]{0.9,0.9,0.9}{
            \makecell[{{p{\linewidth}}}]{
                \texttt{\tiny{[GM$\langle$P2]}}
                \texttt{REFUSE: \{'B26', 'B91', 'C24', 'A19', 'A30', 'A39', 'A47'\}} \\
\\ 
\texttt{ARGUMENT: \{'While your proposal stays within the limit, I believe we can find a better combination that maximizes our mutual benefit. Let me suggest an alternative that might work better for both of us.'\}} \\
\\ 
\texttt{PROPOSAL: \{'B91', 'A30', 'B11', 'C73', 'A47', 'A75', 'A56'\}} \\
            }
        }
    }
     \\ \\

    \theutterance \stepcounter{utterance}  
    & & \multicolumn{4}{p{0.6\linewidth}}{
        \cellcolor[rgb]{0.9,0.9,0.9}{
            \makecell[{{p{\linewidth}}}]{
                \texttt{\tiny{[P1$\langle$GM]}}
                \texttt{REFUSE: \{'B26', 'B91', 'C24', 'A19', 'A30', 'A39', 'A47'\}} \\
\\ 
\texttt{ARGUMENT: \{'While your proposal stays within the limit, I believe we can find a better combination that maximizes our mutual benefit. Let me suggest an alternative that might work better for both of us.'\}} \\
\\ 
\texttt{PROPOSAL: \{'B91', 'A30', 'B11', 'C73', 'A47', 'A75', 'A56'\}} \\
            }
        }
    }
    & \\ \\

    \theutterance \stepcounter{utterance}  
    & \multicolumn{4}{p{0.6\linewidth}}{
        \cellcolor[rgb]{0.9,0.9,0.9}{
            \makecell[{{p{\linewidth}}}]{
                \texttt{\tiny{[P1$\rangle$GM]}}
                \texttt{REFUSE: \{'B91', 'A30', 'B11', 'C73', 'A47', 'A75', 'A56'\}} \\
\\ 
\texttt{ARGUMENT: \{'Your proposal removes some very efficient items like B26 and C24 while including B11 which has high effort cost. I think we can achieve better overall value by keeping the most efficient items and making strategic adjustments.'\}} \\
\\ 
\texttt{PROPOSAL: \{'B26', 'B91', 'C24', 'A19', 'A30', 'B11'\}} \\
            }
        }
    }
    & & \\ \\

    \theutterance \stepcounter{utterance}  
    & & \multicolumn{4}{p{0.6\linewidth}}{
        \cellcolor[rgb]{0.9,0.9,0.9}{
            \makecell[{{p{\linewidth}}}]{
                \texttt{\tiny{[GM$\rangle$P2]}}
                \texttt{REFUSE: \{'B91', 'A30', 'B11', 'C73', 'A47', 'A75', 'A56'\}} \\
\\ 
\texttt{ARGUMENT: \{'Your proposal removes some very efficient items like B26 and C24 while including B11 which has high effort cost. I think we can achieve better overall value by keeping the most efficient items and making strategic adjustments.'\}} \\
\\ 
\texttt{PROPOSAL: \{'B26', 'B91', 'C24', 'A19', 'A30', 'B11'\}} \\
            }
        }
    }
    & \\ \\

    \theutterance \stepcounter{utterance}  
    & & & \multicolumn{4}{p{0.6\linewidth}}{
        \cellcolor[rgb]{0.9,0.9,0.9}{
            \makecell[{{p{\linewidth}}}]{
                \texttt{\tiny{[GM$\langle$P2]}}
                \texttt{REFUSE: \{'B26', 'B91', 'C24', 'A19', 'A30', 'B11'\}} \\
\\ 
\texttt{ARGUMENT: \{'Your proposal includes B11 which has very high effort cost at 835, making it inefficient. We should focus on items that give us better value{-}to{-}effort ratios. Let me propose a set that balances high{-}value items with reasonable effort costs.'\}} \\
\\ 
\texttt{PROPOSAL: \{'B26', 'B91', 'A30', 'C24', 'A19', 'A47', 'C75'\}} \\
            }
        }
    }
     \\ \\

    \theutterance \stepcounter{utterance}  
    & & \multicolumn{4}{p{0.6\linewidth}}{
        \cellcolor[rgb]{0.9,0.9,0.9}{
            \makecell[{{p{\linewidth}}}]{
                \texttt{\tiny{[P1$\langle$GM]}}
                \texttt{REFUSE: \{'B26', 'B91', 'C24', 'A19', 'A30', 'B11'\}} \\
\\ 
\texttt{ARGUMENT: \{'Your proposal includes B11 which has very high effort cost at 835, making it inefficient. We should focus on items that give us better value{-}to{-}effort ratios. Let me propose a set that balances high{-}value items with reasonable effort costs.'\}} \\
\\ 
\texttt{PROPOSAL: \{'B26', 'B91', 'A30', 'C24', 'A19', 'A47', 'C75'\}} \\
            }
        }
    }
    & \\ \\

    \theutterance \stepcounter{utterance}  
    & \multicolumn{4}{p{0.6\linewidth}}{
        \cellcolor[rgb]{0.9,0.9,0.9}{
            \makecell[{{p{\linewidth}}}]{
                \texttt{\tiny{[P1$\rangle$GM]}}
                \texttt{REFUSE: \{'B26', 'B91', 'A30', 'C24', 'A19', 'A47', 'C75'\}} \\
\\ 
\texttt{ARGUMENT: \{'Your proposal includes C75 which appears to have poor efficiency given its high effort cost of 537. I believe we can find a better balance by including A39 instead, which offers strong value while maintaining reasonable effort requirements.'\}} \\
\\ 
\texttt{PROPOSAL: \{'B26', 'B91', 'C24', 'A19', 'A30', 'A39', 'A75'\}} \\
            }
        }
    }
    & & \\ \\

    \theutterance \stepcounter{utterance}  
    & & \multicolumn{4}{p{0.6\linewidth}}{
        \cellcolor[rgb]{0.9,0.9,0.9}{
            \makecell[{{p{\linewidth}}}]{
                \texttt{\tiny{[GM$\rangle$P2]}}
                \texttt{REFUSE: \{'B26', 'B91', 'A30', 'C24', 'A19', 'A47', 'C75'\}} \\
\\ 
\texttt{ARGUMENT: \{'Your proposal includes C75 which appears to have poor efficiency given its high effort cost of 537. I believe we can find a better balance by including A39 instead, which offers strong value while maintaining reasonable effort requirements.'\}} \\
\\ 
\texttt{PROPOSAL: \{'B26', 'B91', 'C24', 'A19', 'A30', 'A39', 'A75'\}} \\
            }
        }
    }
    & \\ \\

    \theutterance \stepcounter{utterance}  
    & & & \multicolumn{4}{p{0.6\linewidth}}{
        \cellcolor[rgb]{0.9,0.9,0.9}{
            \makecell[{{p{\linewidth}}}]{
                \texttt{\tiny{[GM$\langle$P2]}}
                \texttt{REFUSE: \{'B26', 'B91', 'C24', 'A19', 'A30', 'A39', 'A75'\}} \\
\\ 
\texttt{ARGUMENT: \{'Your proposal replaces A47 with A75, but A47 appears to offer better strategic value despite similar effort costs. The combination you suggest may not optimize our joint benefits as effectively as keeping A47 in the mix.'\}} \\
\\ 
\texttt{PROPOSAL: \{'B26', 'B91', 'C24', 'A19', 'A30', 'A39', 'A47'\}} \\
            }
        }
    }
     \\ \\

    \theutterance \stepcounter{utterance}  
    & & \multicolumn{4}{p{0.6\linewidth}}{
        \cellcolor[rgb]{0.9,0.9,0.9}{
            \makecell[{{p{\linewidth}}}]{
                \texttt{\tiny{[P1$\langle$GM]}}
                \texttt{REFUSE: \{'B26', 'B91', 'C24', 'A19', 'A30', 'A39', 'A75'\}} \\
\\ 
\texttt{ARGUMENT: \{'Your proposal replaces A47 with A75, but A47 appears to offer better strategic value despite similar effort costs. The combination you suggest may not optimize our joint benefits as effectively as keeping A47 in the mix.'\}} \\
\\ 
\texttt{PROPOSAL: \{'B26', 'B91', 'C24', 'A19', 'A30', 'A39', 'A47'\}} \\
            }
        }
    }
    & \\ \\

    \theutterance \stepcounter{utterance}  
    & \multicolumn{4}{p{0.6\linewidth}}{
        \cellcolor[rgb]{0.9,0.9,0.9}{
            \makecell[{{p{\linewidth}}}]{
                \texttt{\tiny{[P1$\rangle$GM]}}
                \texttt{AGREE: \{'B26', 'B91', 'C24', 'A19', 'A30', 'A39', 'A47'\}} \\
\\ 
\texttt{ARGUMENT: \{'This proposal returns to a very strong combination that includes the most efficient items like B26 and B91, while maintaining good overall value. The total effort of 3939 stays well within our limit, and this set should provide excellent benefits for both of us.'\}} \\
            }
        }
    }
    & & \\ \\

    \theutterance \stepcounter{utterance}  
    & & & \multicolumn{2}{p{0.3\linewidth}}{
        \cellcolor[rgb]{0.9,0.9,0.9}{
            \makecell[{{p{\linewidth}}}]{
                \texttt{\tiny{[GM$|$GM]}}
                \texttt{game successful} \\
            }
        }
    }
    & & \\ \\

    \theutterance \stepcounter{utterance}  
    & & & \multicolumn{2}{p{0.3\linewidth}}{
        \cellcolor[rgb]{0.9,0.9,0.9}{
            \makecell[{{p{\linewidth}}}]{
                \texttt{\tiny{[GM$|$GM]}}
                \texttt{end game} \\
            }
        }
    }
    & & \\ \\

\end{supertabular}
}

\end{document}
