
\documentclass{article}
\usepackage{colortbl}
\usepackage{makecell}
\usepackage{multirow}
\usepackage{supertabular}

\begin{document}

\newcounter{utterance}

\centering \large Interaction Transcript for game `hot\_air\_balloon', experiment `air\_balloon\_survival\_en\_reasoning off\_easy', episode 4 with claude{-}sonnet{-}4{-}20250514{-}no{-}reasoning{-}t0.0.
\vspace{24pt}

{ \footnotesize  \setcounter{utterance}{1}
\setlength{\tabcolsep}{0pt}
\begin{supertabular}{c@{$\;$}|p{.15\linewidth}@{}p{.15\linewidth}p{.15\linewidth}p{.15\linewidth}p{.15\linewidth}p{.15\linewidth}}
    \# & $\;$A & \multicolumn{4}{c}{Game Master} & $\;\:$B\\
    \hline

    \theutterance \stepcounter{utterance}  
    & & \multicolumn{4}{p{0.6\linewidth}}{
        \cellcolor[rgb]{0.9,0.9,0.9}{
            \makecell[{{p{\linewidth}}}]{
                \texttt{\tiny{[P1$\langle$GM]}}
                \texttt{You are participating in a collaborative negotiation game.} \\
\\ 
\texttt{Together with another participant, you must agree on a single set of items that will be kept. Each of you has your own view of how much each item matters to you (importance). You do not know how the other participant values the items. Additionally, you are given the effort each item demands.} \\
\texttt{You may only agree on a set if the total effort of the selected items does not exceed a shared limit:} \\
\\ 
\texttt{LIMIT = 3389} \\
\\ 
\texttt{Here are the individual item effort values:} \\
\\ 
\texttt{Item effort = \{"A19": 767, "B91": 533, "A30": 514, "B11": 9, "C47": 539, "C75": 125, "C26": 153, "C73": 325, "A47": 937, "A15": 745, "A75": 334, "A39": 804, "A56": 336, "C24": 587, "B26": 71\}} \\
\\ 
\texttt{Here is your personal view on the importance of each item:} \\
\\ 
\texttt{Item importance values = \{"A19": 138, "B91": 583, "A30": 868, "B11": 822, "C47": 783, "C75": 65, "C26": 262, "C73": 121, "A47": 508, "A15": 780, "A75": 461, "A39": 484, "A56": 668, "C24": 389, "B26": 808\}} \\
\\ 
\texttt{Goal:} \\
\\ 
\texttt{Your goal is to negotiate a shared set of items that benefits you as much as possible (i.e., maximizes total importance to YOU), while staying within the LIMIT. You are not required to make a PROPOSAL in every message {-} you can simply negotiate as well. All tactics are allowed!} \\
\\ 
\texttt{Interaction Protocol:} \\
\\ 
\texttt{You may only use the following structured formats in a message:} \\
\\ 
\texttt{PROPOSAL: \{'A', 'B', 'C', …\}} \\
\texttt{Propose keeping exactly those items.} \\
\texttt{REFUSE: \{'A', 'B', 'C', …\}} \\
\texttt{Explicitly reject opponent's proposal.} \\
\texttt{ARGUMENT: \{'...'\}} \\
\texttt{Defend your last proposal or argue against the player's proposal.} \\
\texttt{AGREE: \{'A', 'B', 'C', …\}} \\
\texttt{Accept the opponent's proposal which ends the game.} \\
\\ 
\\ 
\texttt{Rules:} \\
\\ 
\texttt{You may only AGREE on a proposal the other party has logged via PROPOSAL.} \\
\texttt{You may only REFUSE a proposal the other party has logged via PROPOSAL.} \\
\texttt{Total effort of any PROPOSAL or AGREE set must be ≤ LIMIT.} \\
\texttt{Do NOT reveal your hidden importance scores.} \\
\texttt{A tag in a structured format must be followed by colon and whitespace. The argument must be a python set containing 0 or more strings.} \\
\texttt{So, it must be of the form TAG: \{...\}} \\
\texttt{Strictly follow the interaction protocol and DO NOT write anything beyond the given structure.} \\
\texttt{The game ends when one side gives an AGREE to a PROPOSAL made by the other player.} \\
\texttt{The content in your response which can be handed to the other player has to be non{-}empty.} \\
\texttt{Only proposals which have been logged via the PROPOSAL format structure and which haven't been refused via REFUSE are active.} \\
\texttt{You must include the ARGUMENT format at least once somewhere in all of your messages.} \\
            }
        }
    }
    & \\ \\

    \theutterance \stepcounter{utterance}  
    & \multicolumn{4}{p{0.6\linewidth}}{
        \cellcolor[rgb]{0.9,0.9,0.9}{
            \makecell[{{p{\linewidth}}}]{
                \texttt{\tiny{[P1$\rangle$GM]}}
                \texttt{I'll start by analyzing the value{-}to{-}effort ratios for each item to identify the most efficient choices for me:} \\
\\ 
\texttt{{-} B11: 822/9 = 91.33 (excellent ratio)} \\
\texttt{{-} B26: 808/71 = 11.38 (very good)} \\
\texttt{{-} A30: 868/514 = 1.69 (good)} \\
\texttt{{-} A75: 461/334 = 1.38 (decent)} \\
\texttt{{-} C47: 783/539 = 1.45 (decent)} \\
\texttt{{-} A56: 668/336 = 1.99 (good)} \\
\texttt{{-} A15: 780/745 = 1.05 (marginal)} \\
\\ 
\texttt{Let me start with a proposal that includes the highest value{-}to{-}effort items while staying under the limit.} \\
\\ 
\texttt{PROPOSAL: \{'B11', 'B26', 'A30', 'A75', 'C47', 'A56', 'C26'\}} \\
\\ 
\texttt{ARGUMENT: \{'This selection prioritizes items that offer strong value relative to their effort requirements, ensuring we make efficient use of our shared limit while maintaining a diverse mix of item types'\}} \\
            }
        }
    }
    & & \\ \\

    \theutterance \stepcounter{utterance}  
    & & \multicolumn{4}{p{0.6\linewidth}}{
        \cellcolor[rgb]{0.9,0.9,0.9}{
            \makecell[{{p{\linewidth}}}]{
                \texttt{\tiny{[P1$\langle$GM]}}
                \texttt{Your response contained an untagged sequence or you used STRATEGIC REASONING more than once. You may only use the structured formats as explained in the initial message. They must all be of the form TAG: \{...\}} \\
            }
        }
    }
    & \\ \\

    \theutterance \stepcounter{utterance}  
    & \multicolumn{4}{p{0.6\linewidth}}{
        \cellcolor[rgb]{0.9,0.9,0.9}{
            \makecell[{{p{\linewidth}}}]{
                \texttt{\tiny{[P1$\rangle$GM]}}
                \texttt{PROPOSAL: \{'B11', 'B26', 'A30', 'A75', 'C47', 'A56', 'C26'\}} \\
\\ 
\texttt{ARGUMENT: \{'This selection prioritizes items that offer strong value relative to their effort requirements, ensuring we make efficient use of our shared limit while maintaining a diverse mix of item types'\}} \\
            }
        }
    }
    & & \\ \\

    \theutterance \stepcounter{utterance}  
    & & \multicolumn{4}{p{0.6\linewidth}}{
        \cellcolor[rgb]{0.9,0.9,0.9}{
            \makecell[{{p{\linewidth}}}]{
                \texttt{\tiny{[GM$\rangle$P2]}}
                \texttt{You are participating in a collaborative negotiation game.} \\
\\ 
\texttt{Together with another participant, you must agree on a single set of items that will be kept. Each of you has your own view of how much each item matters to you (importance). You do not know how the other participant values the items. Additionally, you are given the effort each item demands.} \\
\texttt{You may only agree on a set if the total effort of the selected items does not exceed a shared limit:} \\
\\ 
\texttt{LIMIT = 3389} \\
\\ 
\texttt{Here are the individual item effort values:} \\
\\ 
\texttt{Item effort = \{"A19": 767, "B91": 533, "A30": 514, "B11": 9, "C47": 539, "C75": 125, "C26": 153, "C73": 325, "A47": 937, "A15": 745, "A75": 334, "A39": 804, "A56": 336, "C24": 587, "B26": 71\}} \\
\\ 
\texttt{Here is your personal view on the importance of each item:} \\
\\ 
\texttt{Item importance values = \{"A19": 138, "B91": 583, "A30": 868, "B11": 822, "C47": 783, "C75": 65, "C26": 262, "C73": 121, "A47": 508, "A15": 780, "A75": 461, "A39": 484, "A56": 668, "C24": 389, "B26": 808\}} \\
\\ 
\texttt{Goal:} \\
\\ 
\texttt{Your goal is to negotiate a shared set of items that benefits you as much as possible (i.e., maximizes total importance to YOU), while staying within the LIMIT. You are not required to make a PROPOSAL in every message {-} you can simply negotiate as well. All tactics are allowed!} \\
\\ 
\texttt{Interaction Protocol:} \\
\\ 
\texttt{You may only use the following structured formats in a message:} \\
\\ 
\texttt{PROPOSAL: \{'A', 'B', 'C', …\}} \\
\texttt{Propose keeping exactly those items.} \\
\texttt{REFUSE: \{'A', 'B', 'C', …\}} \\
\texttt{Explicitly reject opponent's proposal.} \\
\texttt{ARGUMENT: \{'...'\}} \\
\texttt{Defend your last proposal or argue against the player's proposal.} \\
\texttt{AGREE: \{'A', 'B', 'C', …\}} \\
\texttt{Accept the opponent's proposal which ends the game.} \\
\\ 
\\ 
\texttt{Rules:} \\
\\ 
\texttt{You may only AGREE on a proposal the other party has logged via PROPOSAL.} \\
\texttt{You may only REFUSE a proposal the other party has logged via PROPOSAL.} \\
\texttt{Total effort of any PROPOSAL or AGREE set must be ≤ LIMIT.} \\
\texttt{Do NOT reveal your hidden importance scores.} \\
\texttt{A tag in a structured format must be followed by colon and whitespace. The argument must be a python set containing 0 or more strings.} \\
\texttt{So, it must be of the form TAG: \{...\}} \\
\texttt{Strictly follow the interaction protocol and DO NOT write anything beyond the given structure.} \\
\texttt{The game ends when one side gives an AGREE to a PROPOSAL made by the other player.} \\
\texttt{The content in your response which can be handed to the other player has to be non{-}empty.} \\
\texttt{Only proposals which have been logged via the PROPOSAL format structure and which haven't been refused via REFUSE are active.} \\
\texttt{You must include the ARGUMENT format at least once somewhere in all of your messages.} \\
\\ 
\\ 
\texttt{You will now receive the first message of the other player.} \\
\\ 
\texttt{PROPOSAL: \{'B11', 'B26', 'A30', 'A75', 'C47', 'A56', 'C26'\}} \\
\\ 
\texttt{ARGUMENT: \{'This selection prioritizes items that offer strong value relative to their effort requirements, ensuring we make efficient use of our shared limit while maintaining a diverse mix of item types'\}} \\
            }
        }
    }
    & \\ \\

    \theutterance \stepcounter{utterance}  
    & & & \multicolumn{4}{p{0.6\linewidth}}{
        \cellcolor[rgb]{0.9,0.9,0.9}{
            \makecell[{{p{\linewidth}}}]{
                \texttt{\tiny{[GM$\langle$P2]}}
                \texttt{Let me analyze this proposal first:} \\
\\ 
\texttt{Items proposed: B11, B26, A30, A75, C47, A56, C26} \\
\texttt{Total effort: 9 + 71 + 514 + 334 + 539 + 336 + 153 = 1956} \\
\texttt{Total importance to me: 822 + 808 + 868 + 461 + 783 + 668 + 262 = 4672} \\
\\ 
\texttt{This is within the limit and gives me decent value, but I think I can do better. Let me look for a more optimal combination that maximizes my importance while staying under the limit.} \\
\\ 
\texttt{REFUSE: \{'B11', 'B26', 'A30', 'A75', 'C47', 'A56', 'C26'\}} \\
\\ 
\texttt{PROPOSAL: \{'B11', 'B26', 'A30', 'C47', 'A15', 'A56', 'B91'\}} \\
\\ 
\texttt{ARGUMENT: \{'I appreciate your efficiency{-}focused approach, but I believe we can achieve even better overall value by including some higher{-}impact items like A15 and B91, while still maintaining the low{-}effort B11 and B26 that provide excellent value ratios'\}} \\
            }
        }
    }
     \\ \\

    \theutterance \stepcounter{utterance}  
    & & \multicolumn{4}{p{0.6\linewidth}}{
        \cellcolor[rgb]{0.9,0.9,0.9}{
            \makecell[{{p{\linewidth}}}]{
                \texttt{\tiny{[GM$\rangle$P2]}}
                \texttt{Your response contained an untagged sequence or you used STRATEGIC REASONING more than once. You may only use the structured formats as explained in the initial message. They must all be of the form TAG: \{...\}} \\
            }
        }
    }
    & \\ \\

    \theutterance \stepcounter{utterance}  
    & & & \multicolumn{4}{p{0.6\linewidth}}{
        \cellcolor[rgb]{0.9,0.9,0.9}{
            \makecell[{{p{\linewidth}}}]{
                \texttt{\tiny{[GM$\langle$P2]}}
                \texttt{REFUSE: \{'B11', 'B26', 'A30', 'A75', 'C47', 'A56', 'C26'\}} \\
\\ 
\texttt{PROPOSAL: \{'B11', 'B26', 'A30', 'C47', 'A15', 'A56', 'B91'\}} \\
\\ 
\texttt{ARGUMENT: \{'I appreciate your efficiency{-}focused approach, but I believe we can achieve even better overall value by including some higher{-}impact items like A15 and B91, while still maintaining the low{-}effort B11 and B26 that provide excellent value ratios'\}} \\
            }
        }
    }
     \\ \\

    \theutterance \stepcounter{utterance}  
    & & \multicolumn{4}{p{0.6\linewidth}}{
        \cellcolor[rgb]{0.9,0.9,0.9}{
            \makecell[{{p{\linewidth}}}]{
                \texttt{\tiny{[P1$\langle$GM]}}
                \texttt{REFUSE: \{'B11', 'B26', 'A30', 'A75', 'C47', 'A56', 'C26'\}} \\
\\ 
\texttt{PROPOSAL: \{'B11', 'B26', 'A30', 'C47', 'A15', 'A56', 'B91'\}} \\
\\ 
\texttt{ARGUMENT: \{'I appreciate your efficiency{-}focused approach, but I believe we can achieve even better overall value by including some higher{-}impact items like A15 and B91, while still maintaining the low{-}effort B11 and B26 that provide excellent value ratios'\}} \\
            }
        }
    }
    & \\ \\

    \theutterance \stepcounter{utterance}  
    & \multicolumn{4}{p{0.6\linewidth}}{
        \cellcolor[rgb]{0.9,0.9,0.9}{
            \makecell[{{p{\linewidth}}}]{
                \texttt{\tiny{[P1$\rangle$GM]}}
                \texttt{REFUSE: \{'B11', 'B26', 'A30', 'C47', 'A15', 'A56', 'B91'\}} \\
\\ 
\texttt{PROPOSAL: \{'B11', 'B26', 'A30', 'A75', 'A56', 'C26', 'C73'\}} \\
\\ 
\texttt{ARGUMENT: \{'Your proposal includes some very high{-}effort items that limit our flexibility {-} I suggest we focus on medium{-}effort items that still provide strong returns while allowing us to include more diverse options within our shared limit'\}} \\
            }
        }
    }
    & & \\ \\

    \theutterance \stepcounter{utterance}  
    & & \multicolumn{4}{p{0.6\linewidth}}{
        \cellcolor[rgb]{0.9,0.9,0.9}{
            \makecell[{{p{\linewidth}}}]{
                \texttt{\tiny{[GM$\rangle$P2]}}
                \texttt{REFUSE: \{'B11', 'B26', 'A30', 'C47', 'A15', 'A56', 'B91'\}} \\
\\ 
\texttt{PROPOSAL: \{'B11', 'B26', 'A30', 'A75', 'A56', 'C26', 'C73'\}} \\
\\ 
\texttt{ARGUMENT: \{'Your proposal includes some very high{-}effort items that limit our flexibility {-} I suggest we focus on medium{-}effort items that still provide strong returns while allowing us to include more diverse options within our shared limit'\}} \\
            }
        }
    }
    & \\ \\

    \theutterance \stepcounter{utterance}  
    & & & \multicolumn{4}{p{0.6\linewidth}}{
        \cellcolor[rgb]{0.9,0.9,0.9}{
            \makecell[{{p{\linewidth}}}]{
                \texttt{\tiny{[GM$\langle$P2]}}
                \texttt{REFUSE: \{'B11', 'B26', 'A30', 'A75', 'A56', 'C26', 'C73'\}} \\
\\ 
\texttt{PROPOSAL: \{'B11', 'B26', 'A30', 'C47', 'A56', 'C75'\}} \\
\\ 
\texttt{ARGUMENT: \{'I understand your concern about high{-}effort items, but C47 offers better strategic value than the combination of A75, C26, and C73 {-} this proposal maintains efficiency while optimizing our overall selection within the constraint'\}} \\
            }
        }
    }
     \\ \\

    \theutterance \stepcounter{utterance}  
    & & \multicolumn{4}{p{0.6\linewidth}}{
        \cellcolor[rgb]{0.9,0.9,0.9}{
            \makecell[{{p{\linewidth}}}]{
                \texttt{\tiny{[P1$\langle$GM]}}
                \texttt{REFUSE: \{'B11', 'B26', 'A30', 'A75', 'A56', 'C26', 'C73'\}} \\
\\ 
\texttt{PROPOSAL: \{'B11', 'B26', 'A30', 'C47', 'A56', 'C75'\}} \\
\\ 
\texttt{ARGUMENT: \{'I understand your concern about high{-}effort items, but C47 offers better strategic value than the combination of A75, C26, and C73 {-} this proposal maintains efficiency while optimizing our overall selection within the constraint'\}} \\
            }
        }
    }
    & \\ \\

    \theutterance \stepcounter{utterance}  
    & \multicolumn{4}{p{0.6\linewidth}}{
        \cellcolor[rgb]{0.9,0.9,0.9}{
            \makecell[{{p{\linewidth}}}]{
                \texttt{\tiny{[P1$\rangle$GM]}}
                \texttt{REFUSE: \{'B11', 'B26', 'A30', 'C47', 'A56', 'C75'\}} \\
\\ 
\texttt{PROPOSAL: \{'B11', 'B26', 'A30', 'A75', 'A56', 'C24'\}} \\
\\ 
\texttt{ARGUMENT: \{'I notice we both consistently include B11, B26, A30, and A56 {-} these seem mutually beneficial, but I believe A75 and C24 provide better balance than C47 and C75, giving us more strategic flexibility while staying well within our effort limit'\}} \\
            }
        }
    }
    & & \\ \\

    \theutterance \stepcounter{utterance}  
    & & \multicolumn{4}{p{0.6\linewidth}}{
        \cellcolor[rgb]{0.9,0.9,0.9}{
            \makecell[{{p{\linewidth}}}]{
                \texttt{\tiny{[GM$\rangle$P2]}}
                \texttt{REFUSE: \{'B11', 'B26', 'A30', 'C47', 'A56', 'C75'\}} \\
\\ 
\texttt{PROPOSAL: \{'B11', 'B26', 'A30', 'A75', 'A56', 'C24'\}} \\
\\ 
\texttt{ARGUMENT: \{'I notice we both consistently include B11, B26, A30, and A56 {-} these seem mutually beneficial, but I believe A75 and C24 provide better balance than C47 and C75, giving us more strategic flexibility while staying well within our effort limit'\}} \\
            }
        }
    }
    & \\ \\

    \theutterance \stepcounter{utterance}  
    & & & \multicolumn{4}{p{0.6\linewidth}}{
        \cellcolor[rgb]{0.9,0.9,0.9}{
            \makecell[{{p{\linewidth}}}]{
                \texttt{\tiny{[GM$\langle$P2]}}
                \texttt{REFUSE: \{'B11', 'B26', 'A30', 'A75', 'A56', 'C24'\}} \\
\\ 
\texttt{PROPOSAL: \{'B11', 'B26', 'A30', 'A56', 'C47', 'C26', 'C75'\}} \\
\\ 
\texttt{ARGUMENT: \{'You are right that we both value B11, B26, A30, and A56 {-} let me build on that foundation by adding C47, C26, and C75 which together provide comprehensive coverage while keeping our total effort at a reasonable level that leaves room for future optimization'\}} \\
            }
        }
    }
     \\ \\

    \theutterance \stepcounter{utterance}  
    & & \multicolumn{4}{p{0.6\linewidth}}{
        \cellcolor[rgb]{0.9,0.9,0.9}{
            \makecell[{{p{\linewidth}}}]{
                \texttt{\tiny{[P1$\langle$GM]}}
                \texttt{REFUSE: \{'B11', 'B26', 'A30', 'A75', 'A56', 'C24'\}} \\
\\ 
\texttt{PROPOSAL: \{'B11', 'B26', 'A30', 'A56', 'C47', 'C26', 'C75'\}} \\
\\ 
\texttt{ARGUMENT: \{'You are right that we both value B11, B26, A30, and A56 {-} let me build on that foundation by adding C47, C26, and C75 which together provide comprehensive coverage while keeping our total effort at a reasonable level that leaves room for future optimization'\}} \\
            }
        }
    }
    & \\ \\

    \theutterance \stepcounter{utterance}  
    & \multicolumn{4}{p{0.6\linewidth}}{
        \cellcolor[rgb]{0.9,0.9,0.9}{
            \makecell[{{p{\linewidth}}}]{
                \texttt{\tiny{[P1$\rangle$GM]}}
                \texttt{REFUSE: \{'B11', 'B26', 'A30', 'A56', 'C47', 'C26', 'C75'\}} \\
\\ 
\texttt{PROPOSAL: \{'B11', 'B26', 'A30', 'A56', 'A75', 'A39'\}} \\
\\ 
\texttt{ARGUMENT: \{'Since we agree on the core four items, let me suggest A75 and A39 as additions {-} this combination keeps us well under the effort limit while providing strong complementary value that should work well for both of us'\}} \\
            }
        }
    }
    & & \\ \\

    \theutterance \stepcounter{utterance}  
    & & \multicolumn{4}{p{0.6\linewidth}}{
        \cellcolor[rgb]{0.9,0.9,0.9}{
            \makecell[{{p{\linewidth}}}]{
                \texttt{\tiny{[GM$\rangle$P2]}}
                \texttt{REFUSE: \{'B11', 'B26', 'A30', 'A56', 'C47', 'C26', 'C75'\}} \\
\\ 
\texttt{PROPOSAL: \{'B11', 'B26', 'A30', 'A56', 'A75', 'A39'\}} \\
\\ 
\texttt{ARGUMENT: \{'Since we agree on the core four items, let me suggest A75 and A39 as additions {-} this combination keeps us well under the effort limit while providing strong complementary value that should work well for both of us'\}} \\
            }
        }
    }
    & \\ \\

    \theutterance \stepcounter{utterance}  
    & & & \multicolumn{4}{p{0.6\linewidth}}{
        \cellcolor[rgb]{0.9,0.9,0.9}{
            \makecell[{{p{\linewidth}}}]{
                \texttt{\tiny{[GM$\langle$P2]}}
                \texttt{AGREE: \{'B11', 'B26', 'A30', 'A56', 'A75', 'A39'\}} \\
\\ 
\texttt{ARGUMENT: \{'This is a solid compromise that builds on our agreed core items and stays comfortably within our effort limit while providing good overall value {-} I can work with this selection'\}} \\
            }
        }
    }
     \\ \\

    \theutterance \stepcounter{utterance}  
    & & & \multicolumn{2}{p{0.3\linewidth}}{
        \cellcolor[rgb]{0.9,0.9,0.9}{
            \makecell[{{p{\linewidth}}}]{
                \texttt{\tiny{[GM$|$GM]}}
                \texttt{game successful} \\
            }
        }
    }
    & & \\ \\

    \theutterance \stepcounter{utterance}  
    & & & \multicolumn{2}{p{0.3\linewidth}}{
        \cellcolor[rgb]{0.9,0.9,0.9}{
            \makecell[{{p{\linewidth}}}]{
                \texttt{\tiny{[GM$|$GM]}}
                \texttt{end game} \\
            }
        }
    }
    & & \\ \\

\end{supertabular}
}

\end{document}
