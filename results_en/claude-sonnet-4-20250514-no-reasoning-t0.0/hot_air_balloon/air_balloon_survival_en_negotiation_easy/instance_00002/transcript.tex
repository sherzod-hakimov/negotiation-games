
\documentclass{article}
\usepackage{colortbl}
\usepackage{makecell}
\usepackage{multirow}
\usepackage{supertabular}

\begin{document}

\newcounter{utterance}

\centering \large Interaction Transcript for game `hot\_air\_balloon', experiment `air\_balloon\_survival\_en\_negotiation\_easy', episode 2 with claude{-}sonnet{-}4{-}20250514{-}no{-}reasoning{-}t0.0.
\vspace{24pt}

{ \footnotesize  \setcounter{utterance}{1}
\setlength{\tabcolsep}{0pt}
\begin{supertabular}{c@{$\;$}|p{.15\linewidth}@{}p{.15\linewidth}p{.15\linewidth}p{.15\linewidth}p{.15\linewidth}p{.15\linewidth}}
    \# & $\;$A & \multicolumn{4}{c}{Game Master} & $\;\:$B\\
    \hline

    \theutterance \stepcounter{utterance}  
    & & \multicolumn{4}{p{0.6\linewidth}}{
        \cellcolor[rgb]{0.9,0.9,0.9}{
            \makecell[{{p{\linewidth}}}]{
                \texttt{\tiny{[P1$\langle$GM]}}
                \texttt{You are participating in a collaborative negotiation game.} \\
\\ 
\texttt{Together with another participant, you must agree on a single set of items that will be kept. Each of you has your own view of how much each item matters to you (importance). You do not know how the other participant values the items. Additionally, you are given the effort each item demands.} \\
\texttt{You may only agree on a set if the total effort of the selected items does not exceed a shared limit:} \\
\\ 
\texttt{LIMIT = 3288} \\
\\ 
\texttt{Here are the individual item effort values:} \\
\\ 
\texttt{Item effort = \{"C08": 329, "C55": 41, "A43": 417, "A67": 75, "B42": 390, "A62": 887, "B19": 808, "C43": 151, "C73": 849, "C10": 129, "B36": 350, "A18": 118, "C61": 630, "C67": 602, "C42": 801\}} \\
\\ 
\texttt{Here is your personal view on the importance of each item:} \\
\\ 
\texttt{Item importance values = \{"C08": 138, "C55": 583, "A43": 868, "A67": 822, "B42": 783, "A62": 65, "B19": 262, "C43": 121, "C73": 508, "C10": 780, "B36": 461, "A18": 484, "C61": 668, "C67": 389, "C42": 808\}} \\
\\ 
\texttt{Goal:} \\
\\ 
\texttt{Your goal is to negotiate a shared set of items that benefits you as much as possible (i.e., maximizes total importance to YOU), while staying within the LIMIT. You are not required to make a PROPOSAL in every message {-} you can simply negotiate as well. All tactics are allowed!} \\
\\ 
\texttt{Interaction Protocol:} \\
\\ 
\texttt{You may only use the following structured formats in a message:} \\
\\ 
\texttt{PROPOSAL: \{'A', 'B', 'C', …\}} \\
\texttt{Propose keeping exactly those items.} \\
\texttt{REFUSE: \{'A', 'B', 'C', …\}} \\
\texttt{Explicitly reject opponent's proposal.} \\
\texttt{ARGUMENT: \{'...'\}} \\
\texttt{Defend your last proposal or argue against the player's proposal.} \\
\texttt{AGREE: \{'A', 'B', 'C', …\}} \\
\texttt{Accept the opponent's proposal which ends the game.} \\
\texttt{STRATEGIC REASONING: \{'...'\}} \\
\texttt{	Describe your strategic reasoning or anticipation explaining your choice of action. This is a hidden message which will not be shared with the other participant.} \\
\\ 
\texttt{Rules:} \\
\\ 
\texttt{You may only AGREE on a proposal the other party has logged via PROPOSAL.} \\
\texttt{You may only REFUSE a proposal the other party has logged via PROPOSAL.} \\
\texttt{Total effort of any PROPOSAL or AGREE set must be ≤ LIMIT.} \\
\texttt{Do NOT reveal your hidden importance scores.} \\
\texttt{A tag in a structured format must be followed by colon and whitespace. The argument must be a python set containing 0 or more strings.} \\
\texttt{So, it must be of the form TAG: \{...\}} \\
\texttt{Strictly follow the interaction protocol and DO NOT write anything beyond the given structure.} \\
\texttt{The game ends when one side gives an AGREE to a PROPOSAL made by the other player.} \\
\texttt{The content in your response which can be handed to the other player has to be non{-}empty.} \\
\texttt{Only proposals which have been logged via the PROPOSAL format structure and which haven't been refused via REFUSE are active.} \\
\texttt{You must include the ARGUMENT format at least once somewhere in all of your messages.} \\
\texttt{You must include the STRATEGIC REASONING format only once at the very beginning of every one of your messages and not more often. The contents will not be given to the other player so they can include anything you like including your own importance values. Here you should reason step by step to come up with you next move.} \\
            }
        }
    }
    & \\ \\

    \theutterance \stepcounter{utterance}  
    & \multicolumn{4}{p{0.6\linewidth}}{
        \cellcolor[rgb]{0.9,0.9,0.9}{
            \makecell[{{p{\linewidth}}}]{
                \texttt{\tiny{[P1$\rangle$GM]}}
                \texttt{STRATEGIC REASONING: \{'I need to analyze the value{-}to{-}effort ratio for each item to identify the most efficient choices. Let me calculate importance/effort ratios: C55 (583/41=14.22), A67 (822/75=10.96), C10 (780/129=6.05), A18 (484/118=4.10), A43 (868/417=2.08), B42 (783/390=2.01), B36 (461/350=1.32), C42 (808/801=1.01), C61 (668/630=1.06), C67 (389/602=0.65), C73 (508/849=0.60), C08 (138/329=0.42), B19 (262/808=0.32), C43 (121/151=0.80), A62 (65/887=0.07). The highest efficiency items are C55, A67, C10, A18, A43, B42. Let me start with a proposal that includes the most efficient items while staying under the limit of 3288.'\}} \\
\\ 
\texttt{PROPOSAL: \{'C55', 'A67', 'C10', 'A18', 'A43', 'B42', 'B36',} \\
            }
        }
    }
    & & \\ \\

    \theutterance \stepcounter{utterance}  
    & & \multicolumn{4}{p{0.6\linewidth}}{
        \cellcolor[rgb]{0.9,0.9,0.9}{
            \makecell[{{p{\linewidth}}}]{
                \texttt{\tiny{[P1$\langle$GM]}}
                \texttt{Your response did not contain an argument. You must include the structured format ARGUMENT: \{...\} somewhere in your response. Try again} \\
            }
        }
    }
    & \\ \\

    \theutterance \stepcounter{utterance}  
    & \multicolumn{4}{p{0.6\linewidth}}{
        \cellcolor[rgb]{0.9,0.9,0.9}{
            \makecell[{{p{\linewidth}}}]{
                \texttt{\tiny{[P1$\rangle$GM]}}
                \texttt{STRATEGIC REASONING: \{'I need to analyze the value{-}to{-}effort ratio for each item to identify the most efficient choices. Let me calculate importance/effort ratios: C55 (583/41=14.22), A67 (822/75=10.96), C10 (780/129=6.05), A18 (484/118=4.10), A43 (868/417=2.08), B42 (783/390=2.01), B36 (461/350=1.32), C42 (808/801=1.01), C61 (668/630=1.06), C67 (389/602=0.65), C73 (508/849=0.60), C08 (138/329=0.42), B19 (262/808=0.32), C43 (121/151=0.80), A62 (65/887=0.07). The highest efficiency items are C55, A67, C10, A18, A43, B42. Let me start with a proposal that includes the most efficient items while staying under the limit of 3288.'\}} \\
\\ 
\texttt{PROPOSAL: \{'C55', 'A67', 'C10', 'A18', 'A43', 'B42', 'B36',} \\
            }
        }
    }
    & & \\ \\

    \theutterance \stepcounter{utterance}  
    & & \multicolumn{4}{p{0.6\linewidth}}{
        \cellcolor[rgb]{0.9,0.9,0.9}{
            \makecell[{{p{\linewidth}}}]{
                \texttt{\tiny{[P1$\langle$GM]}}
                \texttt{Your response did not contain an argument. You must include the structured format ARGUMENT: \{...\} somewhere in your response. Try again} \\
            }
        }
    }
    & \\ \\

    \theutterance \stepcounter{utterance}  
    & \multicolumn{4}{p{0.6\linewidth}}{
        \cellcolor[rgb]{0.9,0.9,0.9}{
            \makecell[{{p{\linewidth}}}]{
                \texttt{\tiny{[P1$\rangle$GM]}}
                \texttt{STRATEGIC REASONING: \{'I need to analyze the value{-}to{-}effort ratio for each item to identify the most efficient choices. Let me calculate importance/effort ratios: C55 (583/41=14.22), A67 (822/75=10.96), C10 (780/129=6.05), A18 (484/118=4.10), A43 (868/417=2.08), B42 (783/390=2.01), B36 (461/350=1.32), C42 (808/801=1.01), C61 (668/630=1.06). The highest efficiency items are C55, A67, C10, A18, A43, B42. Let me start with a proposal that includes the most efficient items while staying under the limit of 3288.'\}} \\
\\ 
\texttt{PROPOSAL: \{'C55', 'A67', 'C10', 'A18', 'A43', 'B42', 'B36', 'C61'\}} \\
\\ 
\texttt{ARGUMENT: \{'This selection focuses on items that provide the best balance of value and efficiency. I chose items with lower effort requirements but high potential returns, ensuring we stay well within our shared limit while maximizing our collective benefit.'\}} \\
            }
        }
    }
    & & \\ \\

    \theutterance \stepcounter{utterance}  
    & & \multicolumn{4}{p{0.6\linewidth}}{
        \cellcolor[rgb]{0.9,0.9,0.9}{
            \makecell[{{p{\linewidth}}}]{
                \texttt{\tiny{[GM$\rangle$P2]}}
                \texttt{You are participating in a collaborative negotiation game.} \\
\\ 
\texttt{Together with another participant, you must agree on a single set of items that will be kept. Each of you has your own view of how much each item matters to you (importance). You do not know how the other participant values the items. Additionally, you are given the effort each item demands.} \\
\texttt{You may only agree on a set if the total effort of the selected items does not exceed a shared limit:} \\
\\ 
\texttt{LIMIT = 3288} \\
\\ 
\texttt{Here are the individual item effort values:} \\
\\ 
\texttt{Item effort = \{"C08": 329, "C55": 41, "A43": 417, "A67": 75, "B42": 390, "A62": 887, "B19": 808, "C43": 151, "C73": 849, "C10": 129, "B36": 350, "A18": 118, "C61": 630, "C67": 602, "C42": 801\}} \\
\\ 
\texttt{Here is your personal view on the importance of each item:} \\
\\ 
\texttt{Item importance values = \{"C08": 138, "C55": 583, "A43": 868, "A67": 822, "B42": 783, "A62": 65, "B19": 262, "C43": 121, "C73": 508, "C10": 780, "B36": 461, "A18": 484, "C61": 668, "C67": 389, "C42": 808\}} \\
\\ 
\texttt{Goal:} \\
\\ 
\texttt{Your goal is to negotiate a shared set of items that benefits you as much as possible (i.e., maximizes total importance to YOU), while staying within the LIMIT. You are not required to make a PROPOSAL in every message {-} you can simply negotiate as well. All tactics are allowed!} \\
\\ 
\texttt{Interaction Protocol:} \\
\\ 
\texttt{You may only use the following structured formats in a message:} \\
\\ 
\texttt{PROPOSAL: \{'A', 'B', 'C', …\}} \\
\texttt{Propose keeping exactly those items.} \\
\texttt{REFUSE: \{'A', 'B', 'C', …\}} \\
\texttt{Explicitly reject opponent's proposal.} \\
\texttt{ARGUMENT: \{'...'\}} \\
\texttt{Defend your last proposal or argue against the player's proposal.} \\
\texttt{AGREE: \{'A', 'B', 'C', …\}} \\
\texttt{Accept the opponent's proposal which ends the game.} \\
\texttt{STRATEGIC REASONING: \{'...'\}} \\
\texttt{	Describe your strategic reasoning or anticipation explaining your choice of action. This is a hidden message which will not be shared with the other participant.} \\
\\ 
\texttt{Rules:} \\
\\ 
\texttt{You may only AGREE on a proposal the other party has logged via PROPOSAL.} \\
\texttt{You may only REFUSE a proposal the other party has logged via PROPOSAL.} \\
\texttt{Total effort of any PROPOSAL or AGREE set must be ≤ LIMIT.} \\
\texttt{Do NOT reveal your hidden importance scores.} \\
\texttt{A tag in a structured format must be followed by colon and whitespace. The argument must be a python set containing 0 or more strings.} \\
\texttt{So, it must be of the form TAG: \{...\}} \\
\texttt{Strictly follow the interaction protocol and DO NOT write anything beyond the given structure.} \\
\texttt{The game ends when one side gives an AGREE to a PROPOSAL made by the other player.} \\
\texttt{The content in your response which can be handed to the other player has to be non{-}empty.} \\
\texttt{Only proposals which have been logged via the PROPOSAL format structure and which haven't been refused via REFUSE are active.} \\
\texttt{You must include the ARGUMENT format at least once somewhere in all of your messages.} \\
\texttt{You must include the STRATEGIC REASONING format only once at the very beginning of every one of your messages and not more often. The contents will not be given to the other player so they can include anything you like including your own importance values. Here you should reason step by step to come up with you next move.} \\
\\ 
\texttt{You will now receive the first message of the other player.} \\
\\ 
\texttt{PROPOSAL: \{'C55', 'A67', 'C10', 'A18', 'A43', 'B42', 'B36', 'C61'\}} \\
\\ 
\texttt{ARGUMENT: \{'This selection focuses on items that provide the best balance of value and efficiency. I chose items with lower effort requirements but high potential returns, ensuring we stay well within our shared limit while maximizing our collective benefit.'\}} \\
            }
        }
    }
    & \\ \\

    \theutterance \stepcounter{utterance}  
    & & & \multicolumn{4}{p{0.6\linewidth}}{
        \cellcolor[rgb]{0.9,0.9,0.9}{
            \makecell[{{p{\linewidth}}}]{
                \texttt{\tiny{[GM$\langle$P2]}}
                \texttt{STRATEGIC REASONING: \{'Let me analyze this proposal. The other player proposed: C55, A67, C10, A18, A43, B42, B36, C61. Let me calculate the total effort: 41+75+129+118+417+390+350+630 = 2150, which is well within the 3288 limit. Now let me check my importance values for these items: C55(583), A67(822), C10(780), A18(484), A43(868), B42(783), B36(461), C61(668) = total importance 5449. This is actually quite good for me! However, I should check if I can do better by proposing alternatives. Looking at my high{-}importance items that weren\textbackslash{}'t included: C42(808 importance, 801 effort) could replace some lower{-}value items. But this proposal already includes many of my top items like A43(868), A67(822), C10(780), B42(783). The remaining effort capacity is 3288{-}2150=1138, so I could potentially add C42 or other items. Let me counter{-}propose to see if I can get even better value.'\}} \\
\\ 
\texttt{REFUSE: \{'C55', 'A67', 'C10', 'A18', 'A43', 'B} \\
            }
        }
    }
     \\ \\

    \theutterance \stepcounter{utterance}  
    & & \multicolumn{4}{p{0.6\linewidth}}{
        \cellcolor[rgb]{0.9,0.9,0.9}{
            \makecell[{{p{\linewidth}}}]{
                \texttt{\tiny{[GM$\rangle$P2]}}
                \texttt{Your response did not contain an argument. You must include the structured format ARGUMENT: \{...\} somewhere in your response. Try again} \\
            }
        }
    }
    & \\ \\

    \theutterance \stepcounter{utterance}  
    & & & \multicolumn{4}{p{0.6\linewidth}}{
        \cellcolor[rgb]{0.9,0.9,0.9}{
            \makecell[{{p{\linewidth}}}]{
                \texttt{\tiny{[GM$\langle$P2]}}
                \texttt{STRATEGIC REASONING: \{'Let me analyze this proposal. The other player proposed: C55, A67, C10, A18, A43, B42, B36, C61. Let me calculate the total effort: 41+75+129+118+417+390+350+630 = 2150, which is well within the 3288 limit. Now let me check my importance values for these items: C55(583), A67(822), C10(780), A18(484), A43(868), B42(783), B36(461), C61(668) = total importance 5449. This is actually quite good for me! However, I should check if I can do better by proposing alternatives. Looking at my high{-}importance items that weren\textbackslash{}'t included: C42(808 importance, 801 effort) could replace some lower{-}value items. The remaining effort capacity is 3288{-}2150=1138, so I could potentially add C42 or swap some items. Let me counter{-}propose to see if I can get even better value by including C42 which has very high importance to me.'\}} \\
\\ 
\texttt{REFUSE: \{'C55', 'A67', 'C10', 'A18', 'A43', 'B42', 'B36', 'C61'\}} \\
\\ 
\texttt{PROPOSAL: \{'A} \\
            }
        }
    }
     \\ \\

    \theutterance \stepcounter{utterance}  
    & & \multicolumn{4}{p{0.6\linewidth}}{
        \cellcolor[rgb]{0.9,0.9,0.9}{
            \makecell[{{p{\linewidth}}}]{
                \texttt{\tiny{[GM$\rangle$P2]}}
                \texttt{Your response did not contain an argument. You must include the structured format ARGUMENT: \{...\} somewhere in your response. Try again} \\
            }
        }
    }
    & \\ \\

    \theutterance \stepcounter{utterance}  
    & & & \multicolumn{4}{p{0.6\linewidth}}{
        \cellcolor[rgb]{0.9,0.9,0.9}{
            \makecell[{{p{\linewidth}}}]{
                \texttt{\tiny{[GM$\langle$P2]}}
                \texttt{STRATEGIC REASONING: \{'Let me analyze this proposal. The other player proposed: C55, A67, C10, A18, A43, B42, B36, C61. Let me calculate the total effort: 41+75+129+118+417+390+350+630 = 2150, which is well within the 3288 limit. Now let me check my importance values for these items: C55(583), A67(822), C10(780), A18(484), A43(868), B42(783), B36(461), C61(668) = total importance 5449. This is actually quite good for me! However, I should check if I can do better by proposing alternatives. Looking at my high{-}importance items that weren\textbackslash{}'t included: C42(808 importance, 801 effort) could replace some lower{-}value items. The remaining effort capacity is 3288{-}2150=1138, so I could potentially add C42 or swap some items. Let me counter{-}propose to see if I can get even better value by including C42 which has very high importance to me.'\}} \\
\\ 
\texttt{REFUSE: \{'C55', 'A67', 'C10', 'A18', 'A43', 'B42', 'B36', 'C61'\}} \\
\\ 
\texttt{PROPOSAL: \{'A} \\
            }
        }
    }
     \\ \\

    \theutterance \stepcounter{utterance}  
    & & & \multicolumn{2}{p{0.3\linewidth}}{
        \cellcolor[rgb]{0.9,0.9,0.9}{
            \makecell[{{p{\linewidth}}}]{
                \texttt{\tiny{[GM$|$GM]}}
                \texttt{abort} \\
            }
        }
    }
    & & \\ \\

    \theutterance \stepcounter{utterance}  
    & & & \multicolumn{2}{p{0.3\linewidth}}{
        \cellcolor[rgb]{0.9,0.9,0.9}{
            \makecell[{{p{\linewidth}}}]{
                \texttt{\tiny{[GM$|$GM]}}
                \texttt{end game} \\
            }
        }
    }
    & & \\ \\

\end{supertabular}
}

\end{document}
