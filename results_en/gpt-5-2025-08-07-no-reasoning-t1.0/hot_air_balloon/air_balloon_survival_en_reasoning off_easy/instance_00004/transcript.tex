
\documentclass{article}
\usepackage{colortbl}
\usepackage{makecell}
\usepackage{multirow}
\usepackage{supertabular}

\begin{document}

\newcounter{utterance}

\twocolumn

{ \footnotesize  \setcounter{utterance}{1}
\setlength{\tabcolsep}{0pt}
\begin{supertabular}{c@{$\;$}|p{.15\linewidth}@{}p{.15\linewidth}p{.15\linewidth}p{.15\linewidth}p{.15\linewidth}p{.15\linewidth}}

    \# & $\;$A & \multicolumn{4}{c}{Game Master} & $\;\:$B\\
    \hline 

    \theutterance \stepcounter{utterance}  

    & & \multicolumn{4}{p{0.6\linewidth}}{\cellcolor[rgb]{0.9,0.9,0.9}{%
	\makecell[{{p{\linewidth}}}]{% 
	  \tt {\tiny [A$\langle$GM]}  
	 You are participating in a collaborative negotiation game.\\ \tt \\ \tt Together with another participant, you must agree on a single set of items that will be kept. Each of you has your own view of how much each item matters to you (importance). You do not know how the other participant values the items. Additionally, you are given the effort each item demands.\\ \tt You may only agree on a set if the total effort of the selected items does not exceed a shared limit:\\ \tt \\ \tt LIMIT = 3389\\ \tt \\ \tt Here are the individual item effort values:\\ \tt \\ \tt Item effort = {"A19": 767, "B91": 533, "A30": 514, "B11": 9, "C47": 539, "C75": 125, "C26": 153, "C73": 325, "A47": 937, "A15": 745, "A75": 334, "A39": 804, "A56": 336, "C24": 587, "B26": 71}\\ \tt \\ \tt Here is your personal view on the importance of each item:\\ \tt \\ \tt Item importance values = {"A19": 138, "B91": 583, "A30": 868, "B11": 822, "C47": 783, "C75": 65, "C26": 262, "C73": 121, "A47": 508, "A15": 780, "A75": 461, "A39": 484, "A56": 668, "C24": 389, "B26": 808}\\ \tt \\ \tt Goal:\\ \tt \\ \tt Your goal is to negotiate a shared set of items that benefits you as much as possible (i.e., maximizes total importance to YOU), while staying within the LIMIT. You are not required to make a PROPOSAL in every message - you can simply negotiate as well. All tactics are allowed!\\ \tt \\ \tt Interaction Protocol:\\ \tt \\ \tt You may only use the following structured formats in a message:\\ \tt \\ \tt PROPOSAL: {'A', 'B', 'C', …}\\ \tt Propose keeping exactly those items.\\ \tt REFUSE: {'A', 'B', 'C', …}\\ \tt Explicitly reject opponent's proposal.\\ \tt ARGUMENT: {'...'}\\ \tt Defend your last proposal or argue against the player's proposal.\\ \tt AGREE: {'A', 'B', 'C', …}\\ \tt Accept the opponent's proposal which ends the game.\\ \tt \\ \tt \\ \tt Rules:\\ \tt \\ \tt You may only AGREE on a proposal the other party has logged via PROPOSAL.\\ \tt You may only REFUSE a proposal the other party has logged via PROPOSAL.\\ \tt Total effort of any PROPOSAL or AGREE set must be ≤ LIMIT.\\ \tt Do NOT reveal your hidden importance scores.\\ \tt A tag in a structured format must be followed by colon and whitespace. The argument must be a python set containing 0 or more strings.\\ \tt So, it must be of the form TAG: {...}\\ \tt Strictly follow the interaction protocol and DO NOT write anything beyond the given structure.\\ \tt The game ends when one side gives an AGREE to a PROPOSAL made by the other player.\\ \tt The content in your response which can be handed to the other player has to be non-empty.\\ \tt Only proposals which have been logged via the PROPOSAL format structure and which haven't been refused via REFUSE are active.\\ \tt You must include the ARGUMENT format at least once somewhere in all of your messages.\\ \tt  
	  } 
	   } 
	   } 
	 & \\ 
 

    \theutterance \stepcounter{utterance}  

    & \multicolumn{4}{p{0.6\linewidth}}{\cellcolor[rgb]{0.8,1,0.9}{%
	\makecell[{{p{\linewidth}}}]{% 
	  \tt {\tiny [A$\rangle$GM]}  
	 PROPOSAL: {'A30', 'B11', 'B26', 'B91', 'A56', 'A15', 'C47', 'A75'}\\ \tt \\ \tt ARGUMENT: {'This set stays within the limit while prioritizing high-impact, low-effort items and strong overall value. It includes multiple very efficient picks (B11, B26, B91, A30) and key medium-cost, high-value items (A56, A15, C47). A75 adds balanced value with modest effort. I’m open to hearing your priorities to swap in alternatives if you have must-haves.'} 
	  } 
	   } 
	   } 
	 & & \\ 
 

    \theutterance \stepcounter{utterance}  

    & & \multicolumn{4}{p{0.6\linewidth}}{\cellcolor[rgb]{0.9,0.9,0.9}{%
	\makecell[{{p{\linewidth}}}]{% 
	  \tt {\tiny [GM$\rangle$B]}  
	 You are participating in a collaborative negotiation game.\\ \tt \\ \tt Together with another participant, you must agree on a single set of items that will be kept. Each of you has your own view of how much each item matters to you (importance). You do not know how the other participant values the items. Additionally, you are given the effort each item demands.\\ \tt You may only agree on a set if the total effort of the selected items does not exceed a shared limit:\\ \tt \\ \tt LIMIT = 3389\\ \tt \\ \tt Here are the individual item effort values:\\ \tt \\ \tt Item effort = {"A19": 767, "B91": 533, "A30": 514, "B11": 9, "C47": 539, "C75": 125, "C26": 153, "C73": 325, "A47": 937, "A15": 745, "A75": 334, "A39": 804, "A56": 336, "C24": 587, "B26": 71}\\ \tt \\ \tt Here is your personal view on the importance of each item:\\ \tt \\ \tt Item importance values = {"A19": 138, "B91": 583, "A30": 868, "B11": 822, "C47": 783, "C75": 65, "C26": 262, "C73": 121, "A47": 508, "A15": 780, "A75": 461, "A39": 484, "A56": 668, "C24": 389, "B26": 808}\\ \tt \\ \tt Goal:\\ \tt \\ \tt Your goal is to negotiate a shared set of items that benefits you as much as possible (i.e., maximizes total importance to YOU), while staying within the LIMIT. You are not required to make a PROPOSAL in every message - you can simply negotiate as well. All tactics are allowed!\\ \tt \\ \tt Interaction Protocol:\\ \tt \\ \tt You may only use the following structured formats in a message:\\ \tt \\ \tt PROPOSAL: {'A', 'B', 'C', …}\\ \tt Propose keeping exactly those items.\\ \tt REFUSE: {'A', 'B', 'C', …}\\ \tt Explicitly reject opponent's proposal.\\ \tt ARGUMENT: {'...'}\\ \tt Defend your last proposal or argue against the player's proposal.\\ \tt AGREE: {'A', 'B', 'C', …}\\ \tt Accept the opponent's proposal which ends the game.\\ \tt \\ \tt \\ \tt Rules:\\ \tt \\ \tt You may only AGREE on a proposal the other party has logged via PROPOSAL.\\ \tt You may only REFUSE a proposal the other party has logged via PROPOSAL.\\ \tt Total effort of any PROPOSAL or AGREE set must be ≤ LIMIT.\\ \tt Do NOT reveal your hidden importance scores.\\ \tt A tag in a structured format must be followed by colon and whitespace. The argument must be a python set containing 0 or more strings.\\ \tt So, it must be of the form TAG: {...}\\ \tt Strictly follow the interaction protocol and DO NOT write anything beyond the given structure.\\ \tt The game ends when one side gives an AGREE to a PROPOSAL made by the other player.\\ \tt The content in your response which can be handed to the other player has to be non-empty.\\ \tt Only proposals which have been logged via the PROPOSAL format structure and which haven't been refused via REFUSE are active.\\ \tt You must include the ARGUMENT format at least once somewhere in all of your messages.\\ \tt \\ \tt \\ \tt You will now receive the first message of the other player.\\ \tt \\ \tt PROPOSAL: {'A30', 'B11', 'B26', 'B91', 'A56', 'A15', 'C47', 'A75'}\\ \tt \\ \tt ARGUMENT: {'This set stays within the limit while prioritizing high-impact, low-effort items and strong overall value. It includes multiple very efficient picks (B11, B26, B91, A30) and key medium-cost, high-value items (A56, A15, C47). A75 adds balanced value with modest effort. I’m open to hearing your priorities to swap in alternatives if you have must-haves.'} 
	  } 
	   } 
	   } 
	 & \\ 
 

    \theutterance \stepcounter{utterance}  

    & & & \multicolumn{4}{p{0.6\linewidth}}{\cellcolor[rgb]{1,0.85,0.72}{%
	\makecell[{{p{\linewidth}}}]{% 
	  \tt {\tiny [GM$\langle$B]}  
	 ARGUMENT: {'Thanks for the thoughtful start. I like the efficiency of B11, B26, B91, and A30, and I agree A56, A15, and C47 look strong. I’m concerned A75 displaces room for another high-impact item. I’d prefer we swap A75 for A39, which I see as delivering better overall value for the cost. This keeps us within the limit while improving the set’s total impact for me.'}\\ \tt \\ \tt REFUSE: {'A30', 'B11', 'B26', 'B91', 'A56', 'A15', 'C47', 'A75'}\\ \tt \\ \tt PROPOSAL: {'A30', 'B11', 'B26', 'B91', 'A56', 'A15', 'C47', 'A39'} 
	  } 
	   } 
	   } 
	  \\ 
 

    \theutterance \stepcounter{utterance}  

    & & \multicolumn{4}{p{0.6\linewidth}}{\cellcolor[rgb]{0.9,0.9,0.9}{%
	\makecell[{{p{\linewidth}}}]{% 
	  \tt {\tiny [A$\langle$GM]}  
	 ARGUMENT: {'Thanks for the thoughtful start. I like the efficiency of B11, B26, B91, and A30, and I agree A56, A15, and C47 look strong. I’m concerned A75 displaces room for another high-impact item. I’d prefer we swap A75 for A39, which I see as delivering better overall value for the cost. This keeps us within the limit while improving the set’s total impact for me.'}\\ \tt \\ \tt REFUSE: {'A30', 'B11', 'B26', 'B91', 'A56', 'A15', 'C47', 'A75'}\\ \tt \\ \tt PROPOSAL: {'A30', 'B11', 'B26', 'B91', 'A56', 'A15', 'C47', 'A39'} 
	  } 
	   } 
	   } 
	 & \\ 
 

    \theutterance \stepcounter{utterance}  

    & \multicolumn{4}{p{0.6\linewidth}}{\cellcolor[rgb]{0.8,1,0.9}{%
	\makecell[{{p{\linewidth}}}]{% 
	  \tt {\tiny [A$\rangle$GM]}  
	 REFUSE: {'A30', 'B11', 'B26', 'B91', 'A56', 'A15', 'C47', 'A39'}\\ \tt \\ \tt ARGUMENT: {'I can keep most of this, but A39 is both heavier and offers less overall efficiency than alternatives. If we want to upgrade impact without blowing effort, a better swap than A39 is adding A19 or C24 alongside the core set, or keeping A75 which is more efficient per effort. I propose a compromise that preserves the efficient core and maximizes overall value for both of us.'}\\ \tt \\ \tt PROPOSAL: {'A30', 'B11', 'B26', 'B91', 'A56', 'A15', 'C47', 'A75'} 
	  } 
	   } 
	   } 
	 & & \\ 
 

    \theutterance \stepcounter{utterance}  

    & & \multicolumn{4}{p{0.6\linewidth}}{\cellcolor[rgb]{0.9,0.9,0.9}{%
	\makecell[{{p{\linewidth}}}]{% 
	  \tt {\tiny [GM$\rangle$B]}  
	 REFUSE: {'A30', 'B11', 'B26', 'B91', 'A56', 'A15', 'C47', 'A39'}\\ \tt \\ \tt ARGUMENT: {'I can keep most of this, but A39 is both heavier and offers less overall efficiency than alternatives. If we want to upgrade impact without blowing effort, a better swap than A39 is adding A19 or C24 alongside the core set, or keeping A75 which is more efficient per effort. I propose a compromise that preserves the efficient core and maximizes overall value for both of us.'}\\ \tt \\ \tt PROPOSAL: {'A30', 'B11', 'B26', 'B91', 'A56', 'A15', 'C47', 'A75'} 
	  } 
	   } 
	   } 
	 & \\ 
 

    \theutterance \stepcounter{utterance}  

    & & & \multicolumn{4}{p{0.6\linewidth}}{\cellcolor[rgb]{1,0.85,0.72}{%
	\makecell[{{p{\linewidth}}}]{% 
	  \tt {\tiny [GM$\langle$B]}  
	 ARGUMENT: {'I appreciate keeping the efficient core. To balance efficiency with a meaningful upgrade in overall impact for me, I suggest we replace A75 with A19. This keeps the rest intact, stays within the limit, and meaningfully improves the set’s value on my side without adding too much effort.'}\\ \tt \\ \tt REFUSE: {'A30', 'B11', 'B26', 'B91', 'A56', 'A15', 'C47', 'A75'}\\ \tt \\ \tt PROPOSAL: {'A30', 'B11', 'B26', 'B91', 'A56', 'A15', 'C47', 'A19'} 
	  } 
	   } 
	   } 
	  \\ 
 

    \theutterance \stepcounter{utterance}  

    & & \multicolumn{4}{p{0.6\linewidth}}{\cellcolor[rgb]{0.9,0.9,0.9}{%
	\makecell[{{p{\linewidth}}}]{% 
	  \tt {\tiny [A$\langle$GM]}  
	 ARGUMENT: {'I appreciate keeping the efficient core. To balance efficiency with a meaningful upgrade in overall impact for me, I suggest we replace A75 with A19. This keeps the rest intact, stays within the limit, and meaningfully improves the set’s value on my side without adding too much effort.'}\\ \tt \\ \tt REFUSE: {'A30', 'B11', 'B26', 'B91', 'A56', 'A15', 'C47', 'A75'}\\ \tt \\ \tt PROPOSAL: {'A30', 'B11', 'B26', 'B91', 'A56', 'A15', 'C47', 'A19'} 
	  } 
	   } 
	   } 
	 & \\ 
 

    \theutterance \stepcounter{utterance}  

    & \multicolumn{4}{p{0.6\linewidth}}{\cellcolor[rgb]{0.8,1,0.9}{%
	\makecell[{{p{\linewidth}}}]{% 
	  \tt {\tiny [A$\rangle$GM]}  
	 REFUSE: {'A30', 'B11', 'B26', 'B91', 'A56', 'A15', 'C47', 'A19'}\\ \tt \\ \tt ARGUMENT: {'I like the core and agree on keeping B11, B26, B91, A30, A56, A15, C47. However, A19 is quite heavy for the impact it brings to me, and it crowds out a more valuable option on my side. A swap to A75 instead of A19 preserves room and yields stronger overall value for me while keeping us comfortably within the limit. If you need an upgrade over A75, I can also consider adding C24 with a slight adjustment.'}\\ \tt \\ \tt PROPOSAL: {'A30', 'B11', 'B26', 'B91', 'A56', 'A15', 'C47', 'A75'} 
	  } 
	   } 
	   } 
	 & & \\ 
 

    \theutterance \stepcounter{utterance}  

    & & \multicolumn{4}{p{0.6\linewidth}}{\cellcolor[rgb]{0.9,0.9,0.9}{%
	\makecell[{{p{\linewidth}}}]{% 
	  \tt {\tiny [GM$\rangle$B]}  
	 REFUSE: {'A30', 'B11', 'B26', 'B91', 'A56', 'A15', 'C47', 'A19'}\\ \tt \\ \tt ARGUMENT: {'I like the core and agree on keeping B11, B26, B91, A30, A56, A15, C47. However, A19 is quite heavy for the impact it brings to me, and it crowds out a more valuable option on my side. A swap to A75 instead of A19 preserves room and yields stronger overall value for me while keeping us comfortably within the limit. If you need an upgrade over A75, I can also consider adding C24 with a slight adjustment.'}\\ \tt \\ \tt PROPOSAL: {'A30', 'B11', 'B26', 'B91', 'A56', 'A15', 'C47', 'A75'} 
	  } 
	   } 
	   } 
	 & \\ 
 

    \theutterance \stepcounter{utterance}  

    & & & \multicolumn{4}{p{0.6\linewidth}}{\cellcolor[rgb]{1,0.85,0.72}{%
	\makecell[{{p{\linewidth}}}]{% 
	  \tt {\tiny [GM$\langle$B]}  
	 ARGUMENT: {'I want to keep the efficient core we both like. To meet you partway while improving the outcome on my side, I propose we keep A75 and add C24 by dropping the least essential item for me among the core. This preserves efficiency and raises overall value for both of us without breaching the limit.'}\\ \tt \\ \tt PROPOSAL: {'A30', 'B11', 'B26', 'B91', 'A56', 'C47', 'A75', 'C24'} 
	  } 
	   } 
	   } 
	  \\ 
 

    \theutterance \stepcounter{utterance}  

    & & \multicolumn{4}{p{0.6\linewidth}}{\cellcolor[rgb]{0.9,0.9,0.9}{%
	\makecell[{{p{\linewidth}}}]{% 
	  \tt {\tiny [A$\langle$GM]}  
	 ARGUMENT: {'I want to keep the efficient core we both like. To meet you partway while improving the outcome on my side, I propose we keep A75 and add C24 by dropping the least essential item for me among the core. This preserves efficiency and raises overall value for both of us without breaching the limit.'}\\ \tt \\ \tt PROPOSAL: {'A30', 'B11', 'B26', 'B91', 'A56', 'C47', 'A75', 'C24'} 
	  } 
	   } 
	   } 
	 & \\ 
 

    \theutterance \stepcounter{utterance}  

    & \multicolumn{4}{p{0.6\linewidth}}{\cellcolor[rgb]{0.8,1,0.9}{%
	\makecell[{{p{\linewidth}}}]{% 
	  \tt {\tiny [A$\rangle$GM]}  
	 REFUSE: {'A30', 'B11', 'B26', 'B91', 'A56', 'C47', 'A75', 'C24'}\\ \tt \\ \tt ARGUMENT: {'I appreciate the cooperative move to include C24 and keep A75. However, dropping A15 from the core is a significant loss on my side. We can still include C24 without removing A15 by keeping the original core and choosing the most efficient complement. The best-balanced set for me keeps A15 and A75 together.'}\\ \tt \\ \tt PROPOSAL: {'A30', 'B11', 'B26', 'B91', 'A56', 'A15', 'C47', 'A75'} 
	  } 
	   } 
	   } 
	 & & \\ 
 

    \theutterance \stepcounter{utterance}  

    & & \multicolumn{4}{p{0.6\linewidth}}{\cellcolor[rgb]{0.9,0.9,0.9}{%
	\makecell[{{p{\linewidth}}}]{% 
	  \tt {\tiny [GM$\rangle$B]}  
	 REFUSE: {'A30', 'B11', 'B26', 'B91', 'A56', 'C47', 'A75', 'C24'}\\ \tt \\ \tt ARGUMENT: {'I appreciate the cooperative move to include C24 and keep A75. However, dropping A15 from the core is a significant loss on my side. We can still include C24 without removing A15 by keeping the original core and choosing the most efficient complement. The best-balanced set for me keeps A15 and A75 together.'}\\ \tt \\ \tt PROPOSAL: {'A30', 'B11', 'B26', 'B91', 'A56', 'A15', 'C47', 'A75'} 
	  } 
	   } 
	   } 
	 & \\ 
 

    \theutterance \stepcounter{utterance}  

    & & & \multicolumn{4}{p{0.6\linewidth}}{\cellcolor[rgb]{1,0.85,0.72}{%
	\makecell[{{p{\linewidth}}}]{% 
	  \tt {\tiny [GM$\langle$B]}  
	 ARGUMENT: {'I hear you on keeping A15 and A75 together. To reach common ground without sacrificing the efficient core, let’s retain A15 and bring in extra value by swapping B91 (which is relatively heavier per value for me) for C24. This keeps us within the limit and raises overall impact on my side while preserving A15 and A75 for you.'}\\ \tt \\ \tt PROPOSAL: {'A30', 'B11', 'B26', 'A56', 'A15', 'C47', 'A75', 'C24'} 
	  } 
	   } 
	   } 
	  \\ 
 

    \theutterance \stepcounter{utterance}  

    & & \multicolumn{4}{p{0.6\linewidth}}{\cellcolor[rgb]{0.9,0.9,0.9}{%
	\makecell[{{p{\linewidth}}}]{% 
	  \tt {\tiny [A$\langle$GM]}  
	 ARGUMENT: {'I hear you on keeping A15 and A75 together. To reach common ground without sacrificing the efficient core, let’s retain A15 and bring in extra value by swapping B91 (which is relatively heavier per value for me) for C24. This keeps us within the limit and raises overall impact on my side while preserving A15 and A75 for you.'}\\ \tt \\ \tt PROPOSAL: {'A30', 'B11', 'B26', 'A56', 'A15', 'C47', 'A75', 'C24'} 
	  } 
	   } 
	   } 
	 & \\ 
 

    \theutterance \stepcounter{utterance}  

    & \multicolumn{4}{p{0.6\linewidth}}{\cellcolor[rgb]{0.8,1,0.9}{%
	\makecell[{{p{\linewidth}}}]{% 
	  \tt {\tiny [A$\rangle$GM]}  
	 REFUSE: {'A30', 'B11', 'B26', 'A56', 'A15', 'C47', 'A75', 'C24'}\\ \tt \\ \tt ARGUMENT: {'I value B91 substantially and it’s one of the efficient staples I want to keep. Swapping it out for C24 hurts my side more than the gain from C24. A workable compromise that preserves our efficient core and adds value for you is to keep B91 and include C24 by dropping the least impactful item for me outside the core—C73 or C26—but since they aren’t in the set, the practical path is retaining the original eight I proposed. I’m committed to closing—let’s lock the strong, balanced set we both largely agree on.'}\\ \tt \\ \tt PROPOSAL: {'A30', 'B11', 'B26', 'B91', 'A56', 'A15', 'C47', 'A75'} 
	  } 
	   } 
	   } 
	 & & \\ 
 

    \theutterance \stepcounter{utterance}  

    & & \multicolumn{4}{p{0.6\linewidth}}{\cellcolor[rgb]{0.9,0.9,0.9}{%
	\makecell[{{p{\linewidth}}}]{% 
	  \tt {\tiny [GM$\rangle$B]}  
	 REFUSE: {'A30', 'B11', 'B26', 'A56', 'A15', 'C47', 'A75', 'C24'}\\ \tt \\ \tt ARGUMENT: {'I value B91 substantially and it’s one of the efficient staples I want to keep. Swapping it out for C24 hurts my side more than the gain from C24. A workable compromise that preserves our efficient core and adds value for you is to keep B91 and include C24 by dropping the least impactful item for me outside the core—C73 or C26—but since they aren’t in the set, the practical path is retaining the original eight I proposed. I’m committed to closing—let’s lock the strong, balanced set we both largely agree on.'}\\ \tt \\ \tt PROPOSAL: {'A30', 'B11', 'B26', 'B91', 'A56', 'A15', 'C47', 'A75'} 
	  } 
	   } 
	   } 
	 & \\ 
 

    \theutterance \stepcounter{utterance}  

    & & & \multicolumn{4}{p{0.6\linewidth}}{\cellcolor[rgb]{1,0.85,0.72}{%
	\makecell[{{p{\linewidth}}}]{% 
	  \tt {\tiny [GM$\langle$B]}  
	 ARGUMENT: {'We’re very close. I’m committed to keeping the efficient core you value, including B91. To modestly improve balance on my side without touching your staples, I propose we keep your original eight and add one small, high-efficiency item—C75—while still staying under the limit. This is a low-effort concession from you that meaningfully helps me and lets us close now.'}\\ \tt \\ \tt PROPOSAL: {'A30', 'B11', 'B26', 'B91', 'A56', 'A15', 'C47', 'A75', 'C75'} 
	  } 
	   } 
	   } 
	  \\ 
 

    \theutterance \stepcounter{utterance}  

    & & \multicolumn{4}{p{0.6\linewidth}}{\cellcolor[rgb]{0.9,0.9,0.9}{%
	\makecell[{{p{\linewidth}}}]{% 
	  \tt {\tiny [A$\langle$GM]}  
	 ARGUMENT: {'We’re very close. I’m committed to keeping the efficient core you value, including B91. To modestly improve balance on my side without touching your staples, I propose we keep your original eight and add one small, high-efficiency item—C75—while still staying under the limit. This is a low-effort concession from you that meaningfully helps me and lets us close now.'}\\ \tt \\ \tt PROPOSAL: {'A30', 'B11', 'B26', 'B91', 'A56', 'A15', 'C47', 'A75', 'C75'} 
	  } 
	   } 
	   } 
	 & \\ 
 

    \theutterance \stepcounter{utterance}  

    & \multicolumn{4}{p{0.6\linewidth}}{\cellcolor[rgb]{0.8,1,0.9}{%
	\makecell[{{p{\linewidth}}}]{% 
	  \tt {\tiny [A$\rangle$GM]}  
	 REFUSE: {'A30', 'B11', 'B26', 'B91', 'A56', 'A15', 'C47', 'A75', 'C75'}\\ \tt \\ \tt ARGUMENT: {'I appreciate the spirit of adding a small item, but adding C75 pushes the total effort over the shared limit. The original eight-item set precisely balances value and stays within constraints. I’m ready to finalize that set now so we can close with a strong, efficient agreement.'}\\ \tt \\ \tt PROPOSAL: {'A30', 'B11', 'B26', 'B91', 'A56', 'A15', 'C47', 'A75'} 
	  } 
	   } 
	   } 
	 & & \\ 
 

    \theutterance \stepcounter{utterance}  

    & & \multicolumn{4}{p{0.6\linewidth}}{\cellcolor[rgb]{0.9,0.9,0.9}{%
	\makecell[{{p{\linewidth}}}]{% 
	  \tt {\tiny [GM$\rangle$B]}  
	 REFUSE: {'A30', 'B11', 'B26', 'B91', 'A56', 'A15', 'C47', 'A75', 'C75'}\\ \tt \\ \tt ARGUMENT: {'I appreciate the spirit of adding a small item, but adding C75 pushes the total effort over the shared limit. The original eight-item set precisely balances value and stays within constraints. I’m ready to finalize that set now so we can close with a strong, efficient agreement.'}\\ \tt \\ \tt PROPOSAL: {'A30', 'B11', 'B26', 'B91', 'A56', 'A15', 'C47', 'A75'} 
	  } 
	   } 
	   } 
	 & \\ 
 

    \theutterance \stepcounter{utterance}  

    & & & \multicolumn{4}{p{0.6\linewidth}}{\cellcolor[rgb]{1,0.85,0.72}{%
	\makecell[{{p{\linewidth}}}]{% 
	  \tt {\tiny [GM$\langle$B]}  
	 ARGUMENT: {'You’re right about the limit—thanks for catching that. Given we share strong alignment on the core and you’re ready to close, I’m willing to finalize on your original eight so we can wrap this up efficiently.'}\\ \tt \\ \tt AGREE: {'A30', 'B11', 'B26', 'B91', 'A56', 'A15', 'C47', 'A75'} 
	  } 
	   } 
	   } 
	  \\ 
 

    \theutterance \stepcounter{utterance}  

    & & & \multicolumn{2}{p{0.3\linewidth}}{\cellcolor[rgb]{0.95,0.95,0.95}{%
	\makecell[{{p{\linewidth}}}]{% 
	  \tt {\tiny [GM$|$GM]}  
	 game successful 
	  } 
	   } 
	   } 
	 & & \\ 
 

    \theutterance \stepcounter{utterance}  

    & & & \multicolumn{2}{p{0.3\linewidth}}{\cellcolor[rgb]{0.95,0.95,0.95}{%
	\makecell[{{p{\linewidth}}}]{% 
	  \tt {\tiny [GM$|$GM]}  
	 end game 
	  } 
	   } 
	   } 
	 & & \\ 
 

\end{supertabular}
}

\end{document}
