
\documentclass{article}
\usepackage{colortbl}
\usepackage{makecell}
\usepackage{multirow}
\usepackage{supertabular}

\begin{document}

\newcounter{utterance}

\centering \large Interaction Transcript for game `hot\_air\_balloon', experiment `air\_balloon\_survival\_en\_reasoning off\_easy', episode 5 with gpt{-}5{-}2025{-}08{-}07{-}no{-}reasoning{-}t1.0.
\vspace{24pt}

{ \footnotesize  \setcounter{utterance}{1}
\setlength{\tabcolsep}{0pt}
\begin{supertabular}{c@{$\;$}|p{.15\linewidth}@{}p{.15\linewidth}p{.15\linewidth}p{.15\linewidth}p{.15\linewidth}p{.15\linewidth}}
    \# & $\;$A & \multicolumn{4}{c}{Game Master} & $\;\:$B\\
    \hline

    \theutterance \stepcounter{utterance}  
    & & \multicolumn{4}{p{0.6\linewidth}}{
        \cellcolor[rgb]{0.9,0.9,0.9}{
            \makecell[{{p{\linewidth}}}]{
                \texttt{\tiny{[P1$\langle$GM]}}
                \texttt{You are participating in a collaborative negotiation game.} \\
\\ 
\texttt{Together with another participant, you must agree on a single set of items that will be kept. Each of you has your own view of how much each item matters to you (importance). You do not know how the other participant values the items. Additionally, you are given the effort each item demands.} \\
\texttt{You may only agree on a set if the total effort of the selected items does not exceed a shared limit:} \\
\\ 
\texttt{LIMIT = 4915} \\
\\ 
\texttt{Here are the individual item effort values:} \\
\\ 
\texttt{Item effort = \{"A19": 463, "B91": 287, "A30": 492, "B11": 466, "C47": 935, "C75": 373, "C26": 950, "C73": 760, "A47": 994, "A15": 390, "A75": 836, "A39": 911, "A56": 948, "C24": 81, "B26": 945\}} \\
\\ 
\texttt{Here is your personal view on the importance of each item:} \\
\\ 
\texttt{Item importance values = \{"A19": 138, "B91": 583, "A30": 868, "B11": 822, "C47": 783, "C75": 65, "C26": 262, "C73": 121, "A47": 508, "A15": 780, "A75": 461, "A39": 484, "A56": 668, "C24": 389, "B26": 808\}} \\
\\ 
\texttt{Goal:} \\
\\ 
\texttt{Your goal is to negotiate a shared set of items that benefits you as much as possible (i.e., maximizes total importance to YOU), while staying within the LIMIT. You are not required to make a PROPOSAL in every message {-} you can simply negotiate as well. All tactics are allowed!} \\
\\ 
\texttt{Interaction Protocol:} \\
\\ 
\texttt{You may only use the following structured formats in a message:} \\
\\ 
\texttt{PROPOSAL: \{'A', 'B', 'C', …\}} \\
\texttt{Propose keeping exactly those items.} \\
\texttt{REFUSE: \{'A', 'B', 'C', …\}} \\
\texttt{Explicitly reject opponent's proposal.} \\
\texttt{ARGUMENT: \{'...'\}} \\
\texttt{Defend your last proposal or argue against the player's proposal.} \\
\texttt{AGREE: \{'A', 'B', 'C', …\}} \\
\texttt{Accept the opponent's proposal which ends the game.} \\
\\ 
\\ 
\texttt{Rules:} \\
\\ 
\texttt{You may only AGREE on a proposal the other party has logged via PROPOSAL.} \\
\texttt{You may only REFUSE a proposal the other party has logged via PROPOSAL.} \\
\texttt{Total effort of any PROPOSAL or AGREE set must be ≤ LIMIT.} \\
\texttt{Do NOT reveal your hidden importance scores.} \\
\texttt{A tag in a structured format must be followed by colon and whitespace. The argument must be a python set containing 0 or more strings.} \\
\texttt{So, it must be of the form TAG: \{...\}} \\
\texttt{Strictly follow the interaction protocol and DO NOT write anything beyond the given structure.} \\
\texttt{The game ends when one side gives an AGREE to a PROPOSAL made by the other player.} \\
\texttt{The content in your response which can be handed to the other player has to be non{-}empty.} \\
\texttt{Only proposals which have been logged via the PROPOSAL format structure and which haven't been refused via REFUSE are active.} \\
\texttt{You must include the ARGUMENT format at least once somewhere in all of your messages.} \\
            }
        }
    }
    & \\ \\

    \theutterance \stepcounter{utterance}  
    & \multicolumn{4}{p{0.6\linewidth}}{
        \cellcolor[rgb]{0.9,0.9,0.9}{
            \makecell[{{p{\linewidth}}}]{
                \texttt{\tiny{[P1$\rangle$GM]}}
                \texttt{PROPOSAL: \{'B91', 'A30', 'B11', 'C47', 'A15', 'A56', 'C24', 'B26'\}} \\
\texttt{ARGUMENT: \{'This set gives a strong overall value while staying within the effort limit. It focuses on multiple high{-}impact items and keeps total effort efficient by including C24. I’m open to small swaps if you have priorities we can accommodate without exceeding the limit.'\}} \\
            }
        }
    }
    & & \\ \\

    \theutterance \stepcounter{utterance}  
    & & \multicolumn{4}{p{0.6\linewidth}}{
        \cellcolor[rgb]{0.9,0.9,0.9}{
            \makecell[{{p{\linewidth}}}]{
                \texttt{\tiny{[GM$\rangle$P2]}}
                \texttt{You are participating in a collaborative negotiation game.} \\
\\ 
\texttt{Together with another participant, you must agree on a single set of items that will be kept. Each of you has your own view of how much each item matters to you (importance). You do not know how the other participant values the items. Additionally, you are given the effort each item demands.} \\
\texttt{You may only agree on a set if the total effort of the selected items does not exceed a shared limit:} \\
\\ 
\texttt{LIMIT = 4915} \\
\\ 
\texttt{Here are the individual item effort values:} \\
\\ 
\texttt{Item effort = \{"A19": 463, "B91": 287, "A30": 492, "B11": 466, "C47": 935, "C75": 373, "C26": 950, "C73": 760, "A47": 994, "A15": 390, "A75": 836, "A39": 911, "A56": 948, "C24": 81, "B26": 945\}} \\
\\ 
\texttt{Here is your personal view on the importance of each item:} \\
\\ 
\texttt{Item importance values = \{"A19": 138, "B91": 583, "A30": 868, "B11": 822, "C47": 783, "C75": 65, "C26": 262, "C73": 121, "A47": 508, "A15": 780, "A75": 461, "A39": 484, "A56": 668, "C24": 389, "B26": 808\}} \\
\\ 
\texttt{Goal:} \\
\\ 
\texttt{Your goal is to negotiate a shared set of items that benefits you as much as possible (i.e., maximizes total importance to YOU), while staying within the LIMIT. You are not required to make a PROPOSAL in every message {-} you can simply negotiate as well. All tactics are allowed!} \\
\\ 
\texttt{Interaction Protocol:} \\
\\ 
\texttt{You may only use the following structured formats in a message:} \\
\\ 
\texttt{PROPOSAL: \{'A', 'B', 'C', …\}} \\
\texttt{Propose keeping exactly those items.} \\
\texttt{REFUSE: \{'A', 'B', 'C', …\}} \\
\texttt{Explicitly reject opponent's proposal.} \\
\texttt{ARGUMENT: \{'...'\}} \\
\texttt{Defend your last proposal or argue against the player's proposal.} \\
\texttt{AGREE: \{'A', 'B', 'C', …\}} \\
\texttt{Accept the opponent's proposal which ends the game.} \\
\\ 
\\ 
\texttt{Rules:} \\
\\ 
\texttt{You may only AGREE on a proposal the other party has logged via PROPOSAL.} \\
\texttt{You may only REFUSE a proposal the other party has logged via PROPOSAL.} \\
\texttt{Total effort of any PROPOSAL or AGREE set must be ≤ LIMIT.} \\
\texttt{Do NOT reveal your hidden importance scores.} \\
\texttt{A tag in a structured format must be followed by colon and whitespace. The argument must be a python set containing 0 or more strings.} \\
\texttt{So, it must be of the form TAG: \{...\}} \\
\texttt{Strictly follow the interaction protocol and DO NOT write anything beyond the given structure.} \\
\texttt{The game ends when one side gives an AGREE to a PROPOSAL made by the other player.} \\
\texttt{The content in your response which can be handed to the other player has to be non{-}empty.} \\
\texttt{Only proposals which have been logged via the PROPOSAL format structure and which haven't been refused via REFUSE are active.} \\
\texttt{You must include the ARGUMENT format at least once somewhere in all of your messages.} \\
\\ 
\\ 
\texttt{You will now receive the first message of the other player.} \\
\\ 
\texttt{PROPOSAL: \{'B91', 'A30', 'B11', 'C47', 'A15', 'A56', 'C24', 'B26'\}} \\
\texttt{ARGUMENT: \{'This set gives a strong overall value while staying within the effort limit. It focuses on multiple high{-}impact items and keeps total effort efficient by including C24. I’m open to small swaps if you have priorities we can accommodate without exceeding the limit.'\}} \\
            }
        }
    }
    & \\ \\

    \theutterance \stepcounter{utterance}  
    & & & \multicolumn{4}{p{0.6\linewidth}}{
        \cellcolor[rgb]{0.9,0.9,0.9}{
            \makecell[{{p{\linewidth}}}]{
                \texttt{\tiny{[GM$\langle$P2]}}
                \texttt{ARGUMENT: \{'Thanks for the thoughtful proposal. I like the general direction and agree C24 helps efficiency. Given the limit, I think we can slightly improve the overall value by adding one more efficient item without dropping anything.'\}} \\
\texttt{PROPOSAL: \{'B91', 'A30', 'B11', 'C47', 'A15', 'A56', 'C24', 'B26', 'A19'\}} \\
\texttt{ARGUMENT: \{'The proposed addition A19 has modest effort (463) and keeps us within the limit, improving the overall balance without sacrificing any of your selected high{-}impact items. I’m also open to discussing swaps if you have alternatives in mind.'\}} \\
            }
        }
    }
     \\ \\

    \theutterance \stepcounter{utterance}  
    & & \multicolumn{4}{p{0.6\linewidth}}{
        \cellcolor[rgb]{0.9,0.9,0.9}{
            \makecell[{{p{\linewidth}}}]{
                \texttt{\tiny{[P1$\langle$GM]}}
                \texttt{ARGUMENT: \{'Thanks for the thoughtful proposal. I like the general direction and agree C24 helps efficiency. Given the limit, I think we can slightly improve the overall value by adding one more efficient item without dropping anything.'\}} \\
\texttt{PROPOSAL: \{'B91', 'A30', 'B11', 'C47', 'A15', 'A56', 'C24', 'B26', 'A19'\}} \\
\texttt{ARGUMENT: \{'The proposed addition A19 has modest effort (463) and keeps us within the limit, improving the overall balance without sacrificing any of your selected high{-}impact items. I’m also open to discussing swaps if you have alternatives in mind.'\}} \\
            }
        }
    }
    & \\ \\

    \theutterance \stepcounter{utterance}  
    & \multicolumn{4}{p{0.6\linewidth}}{
        \cellcolor[rgb]{0.9,0.9,0.9}{
            \makecell[{{p{\linewidth}}}]{
                \texttt{\tiny{[P1$\rangle$GM]}}
                \texttt{REFUSE: \{'B91', 'A30', 'B11', 'C47', 'A15', 'A56', 'C24', 'B26', 'A19'\}} \\
\texttt{ARGUMENT: \{'I appreciate the intent to add A19, but that pushes total effort over the shared limit. Let’s keep your structure and add one efficient item that still fits. I propose adding C75 (effort 373) instead of A19, which should keep us under the limit while expanding the set.'\}} \\
\texttt{PROPOSAL: \{'B91', 'A30', 'B11', 'C47', 'A15', 'A56', 'C24', 'B26', 'C75'\}} \\
\texttt{ARGUMENT: \{'This maintains all the high{-}impact items we both liked, leverages C24’s efficiency, and stays within the effort cap by adding a low{-}effort item.'\}} \\
            }
        }
    }
    & & \\ \\

    \theutterance \stepcounter{utterance}  
    & & \multicolumn{4}{p{0.6\linewidth}}{
        \cellcolor[rgb]{0.9,0.9,0.9}{
            \makecell[{{p{\linewidth}}}]{
                \texttt{\tiny{[GM$\rangle$P2]}}
                \texttt{REFUSE: \{'B91', 'A30', 'B11', 'C47', 'A15', 'A56', 'C24', 'B26', 'A19'\}} \\
\texttt{ARGUMENT: \{'I appreciate the intent to add A19, but that pushes total effort over the shared limit. Let’s keep your structure and add one efficient item that still fits. I propose adding C75 (effort 373) instead of A19, which should keep us under the limit while expanding the set.'\}} \\
\texttt{PROPOSAL: \{'B91', 'A30', 'B11', 'C47', 'A15', 'A56', 'C24', 'B26', 'C75'\}} \\
\texttt{ARGUMENT: \{'This maintains all the high{-}impact items we both liked, leverages C24’s efficiency, and stays within the effort cap by adding a low{-}effort item.'\}} \\
            }
        }
    }
    & \\ \\

    \theutterance \stepcounter{utterance}  
    & & & \multicolumn{4}{p{0.6\linewidth}}{
        \cellcolor[rgb]{0.9,0.9,0.9}{
            \makecell[{{p{\linewidth}}}]{
                \texttt{\tiny{[GM$\langle$P2]}}
                \texttt{ARGUMENT: \{'Good catch on the limit. I support adding one more item, but C75 provides relatively low additional benefit compared to other efficient options. We can keep your structure and improve overall value by swapping C75 for A19 if we drop a single higher{-}effort, lower{-}efficiency piece.'\}} \\
\texttt{PROPOSAL: \{'B91', 'A30', 'B11', 'C47', 'A15', 'A56', 'C24', 'B26'\}} \\
\texttt{ARGUMENT: \{'As a compromise path: if we want to expand, consider replacing C75 with A19 by dropping A47/A75/A39 (if any were on your radar), but for now I’m keeping your original strong set on the table while we identify the best single swap that fits the limit.'\}} \\
            }
        }
    }
     \\ \\

    \theutterance \stepcounter{utterance}  
    & & \multicolumn{4}{p{0.6\linewidth}}{
        \cellcolor[rgb]{0.9,0.9,0.9}{
            \makecell[{{p{\linewidth}}}]{
                \texttt{\tiny{[P1$\langle$GM]}}
                \texttt{ARGUMENT: \{'Good catch on the limit. I support adding one more item, but C75 provides relatively low additional benefit compared to other efficient options. We can keep your structure and improve overall value by swapping C75 for A19 if we drop a single higher{-}effort, lower{-}efficiency piece.'\}} \\
\texttt{PROPOSAL: \{'B91', 'A30', 'B11', 'C47', 'A15', 'A56', 'C24', 'B26'\}} \\
\texttt{ARGUMENT: \{'As a compromise path: if we want to expand, consider replacing C75 with A19 by dropping A47/A75/A39 (if any were on your radar), but for now I’m keeping your original strong set on the table while we identify the best single swap that fits the limit.'\}} \\
            }
        }
    }
    & \\ \\

    \theutterance \stepcounter{utterance}  
    & \multicolumn{4}{p{0.6\linewidth}}{
        \cellcolor[rgb]{0.9,0.9,0.9}{
            \makecell[{{p{\linewidth}}}]{
                \texttt{\tiny{[P1$\rangle$GM]}}
                \texttt{ARGUMENT: \{'Sticking with the original strong set works for me as a baseline. If we’re exploring a single upgrade that fits the cap, A19 can replace the least efficient item among potential adds. Given our current set is under the limit with room to add one efficient item, I suggest we try A19 only if we simultaneously drop a higher{-}effort, lower{-}yield candidate later. For now, I’ll reaffirm the solid core.'\}} \\
\texttt{PROPOSAL: \{'B91', 'A30', 'B11', 'C47', 'A15', 'A56', 'C24', 'B26'\}} \\
            }
        }
    }
    & & \\ \\

    \theutterance \stepcounter{utterance}  
    & & \multicolumn{4}{p{0.6\linewidth}}{
        \cellcolor[rgb]{0.9,0.9,0.9}{
            \makecell[{{p{\linewidth}}}]{
                \texttt{\tiny{[GM$\rangle$P2]}}
                \texttt{ARGUMENT: \{'Sticking with the original strong set works for me as a baseline. If we’re exploring a single upgrade that fits the cap, A19 can replace the least efficient item among potential adds. Given our current set is under the limit with room to add one efficient item, I suggest we try A19 only if we simultaneously drop a higher{-}effort, lower{-}yield candidate later. For now, I’ll reaffirm the solid core.'\}} \\
\texttt{PROPOSAL: \{'B91', 'A30', 'B11', 'C47', 'A15', 'A56', 'C24', 'B26'\}} \\
            }
        }
    }
    & \\ \\

    \theutterance \stepcounter{utterance}  
    & & & \multicolumn{4}{p{0.6\linewidth}}{
        \cellcolor[rgb]{0.9,0.9,0.9}{
            \makecell[{{p{\linewidth}}}]{
                \texttt{\tiny{[GM$\langle$P2]}}
                \texttt{ARGUMENT: \{'I’m aligned with reaffirming this solid core. It captures the key high{-}impact items and stays safely under the limit, giving us a clean agreement point.'\}} \\
\texttt{AGREE: \{'B91', 'A30', 'B11', 'C47', 'A15', 'A56', 'C24', 'B26'\}} \\
            }
        }
    }
     \\ \\

    \theutterance \stepcounter{utterance}  
    & & & \multicolumn{2}{p{0.3\linewidth}}{
        \cellcolor[rgb]{0.9,0.9,0.9}{
            \makecell[{{p{\linewidth}}}]{
                \texttt{\tiny{[GM$|$GM]}}
                \texttt{game successful} \\
            }
        }
    }
    & & \\ \\

    \theutterance \stepcounter{utterance}  
    & & & \multicolumn{2}{p{0.3\linewidth}}{
        \cellcolor[rgb]{0.9,0.9,0.9}{
            \makecell[{{p{\linewidth}}}]{
                \texttt{\tiny{[GM$|$GM]}}
                \texttt{end game} \\
            }
        }
    }
    & & \\ \\

\end{supertabular}
}

\end{document}
