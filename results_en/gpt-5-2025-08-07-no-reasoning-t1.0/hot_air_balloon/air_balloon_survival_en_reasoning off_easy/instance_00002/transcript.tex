
\documentclass{article}
\usepackage{colortbl}
\usepackage{makecell}
\usepackage{multirow}
\usepackage{supertabular}

\begin{document}

\newcounter{utterance}

\centering \large Interaction Transcript for game `hot\_air\_balloon', experiment `air\_balloon\_survival\_en\_reasoning off\_easy', episode 2 with gpt{-}5{-}2025{-}08{-}07{-}no{-}reasoning{-}t1.0.
\vspace{24pt}

{ \footnotesize  \setcounter{utterance}{1}
\setlength{\tabcolsep}{0pt}
\begin{supertabular}{c@{$\;$}|p{.15\linewidth}@{}p{.15\linewidth}p{.15\linewidth}p{.15\linewidth}p{.15\linewidth}p{.15\linewidth}}
    \# & $\;$A & \multicolumn{4}{c}{Game Master} & $\;\:$B\\
    \hline

    \theutterance \stepcounter{utterance}  
    & & \multicolumn{4}{p{0.6\linewidth}}{
        \cellcolor[rgb]{0.9,0.9,0.9}{
            \makecell[{{p{\linewidth}}}]{
                \texttt{\tiny{[P1$\langle$GM]}}
                \texttt{You are participating in a collaborative negotiation game.} \\
\\ 
\texttt{Together with another participant, you must agree on a single set of items that will be kept. Each of you has your own view of how much each item matters to you (importance). You do not know how the other participant values the items. Additionally, you are given the effort each item demands.} \\
\texttt{You may only agree on a set if the total effort of the selected items does not exceed a shared limit:} \\
\\ 
\texttt{LIMIT = 3218} \\
\\ 
\texttt{Here are the individual item effort values:} \\
\\ 
\texttt{Item effort = \{"A19": 919, "B91": 852, "A30": 299, "B11": 762, "C47": 161, "C75": 205, "C26": 380, "C73": 399, "A47": 534, "A15": 333, "A75": 100, "A39": 420, "A56": 354, "C24": 130, "B26": 589\}} \\
\\ 
\texttt{Here is your personal view on the importance of each item:} \\
\\ 
\texttt{Item importance values = \{"A19": 138, "B91": 583, "A30": 868, "B11": 822, "C47": 783, "C75": 65, "C26": 262, "C73": 121, "A47": 508, "A15": 780, "A75": 461, "A39": 484, "A56": 668, "C24": 389, "B26": 808\}} \\
\\ 
\texttt{Goal:} \\
\\ 
\texttt{Your goal is to negotiate a shared set of items that benefits you as much as possible (i.e., maximizes total importance to YOU), while staying within the LIMIT. You are not required to make a PROPOSAL in every message {-} you can simply negotiate as well. All tactics are allowed!} \\
\\ 
\texttt{Interaction Protocol:} \\
\\ 
\texttt{You may only use the following structured formats in a message:} \\
\\ 
\texttt{PROPOSAL: \{'A', 'B', 'C', …\}} \\
\texttt{Propose keeping exactly those items.} \\
\texttt{REFUSE: \{'A', 'B', 'C', …\}} \\
\texttt{Explicitly reject opponent's proposal.} \\
\texttt{ARGUMENT: \{'...'\}} \\
\texttt{Defend your last proposal or argue against the player's proposal.} \\
\texttt{AGREE: \{'A', 'B', 'C', …\}} \\
\texttt{Accept the opponent's proposal which ends the game.} \\
\\ 
\\ 
\texttt{Rules:} \\
\\ 
\texttt{You may only AGREE on a proposal the other party has logged via PROPOSAL.} \\
\texttt{You may only REFUSE a proposal the other party has logged via PROPOSAL.} \\
\texttt{Total effort of any PROPOSAL or AGREE set must be ≤ LIMIT.} \\
\texttt{Do NOT reveal your hidden importance scores.} \\
\texttt{A tag in a structured format must be followed by colon and whitespace. The argument must be a python set containing 0 or more strings.} \\
\texttt{So, it must be of the form TAG: \{...\}} \\
\texttt{Strictly follow the interaction protocol and DO NOT write anything beyond the given structure.} \\
\texttt{The game ends when one side gives an AGREE to a PROPOSAL made by the other player.} \\
\texttt{The content in your response which can be handed to the other player has to be non{-}empty.} \\
\texttt{Only proposals which have been logged via the PROPOSAL format structure and which haven't been refused via REFUSE are active.} \\
\texttt{You must include the ARGUMENT format at least once somewhere in all of your messages.} \\
            }
        }
    }
    & \\ \\

    \theutterance \stepcounter{utterance}  
    & \multicolumn{4}{p{0.6\linewidth}}{
        \cellcolor[rgb]{0.9,0.9,0.9}{
            \makecell[{{p{\linewidth}}}]{
                \texttt{\tiny{[P1$\rangle$GM]}}
                \texttt{PROPOSAL: \{'A30', 'B11', 'B26', 'B91', 'A15', 'A56', 'C47', 'C24', 'A75'\}} \\
\\ 
\texttt{ARGUMENT: \{'This set stays within the limit (total effort 3174) while focusing on high{-}impact, efficient items. It balances strong coverage across categories and keeps flexibility for small adjustments if you have must{-}haves to swap in. If you need room for an additional item, we could consider dropping A75 (low effort) or C24 to reallocate effort.'\}} \\
            }
        }
    }
    & & \\ \\

    \theutterance \stepcounter{utterance}  
    & & \multicolumn{4}{p{0.6\linewidth}}{
        \cellcolor[rgb]{0.9,0.9,0.9}{
            \makecell[{{p{\linewidth}}}]{
                \texttt{\tiny{[GM$\rangle$P2]}}
                \texttt{You are participating in a collaborative negotiation game.} \\
\\ 
\texttt{Together with another participant, you must agree on a single set of items that will be kept. Each of you has your own view of how much each item matters to you (importance). You do not know how the other participant values the items. Additionally, you are given the effort each item demands.} \\
\texttt{You may only agree on a set if the total effort of the selected items does not exceed a shared limit:} \\
\\ 
\texttt{LIMIT = 3218} \\
\\ 
\texttt{Here are the individual item effort values:} \\
\\ 
\texttt{Item effort = \{"A19": 919, "B91": 852, "A30": 299, "B11": 762, "C47": 161, "C75": 205, "C26": 380, "C73": 399, "A47": 534, "A15": 333, "A75": 100, "A39": 420, "A56": 354, "C24": 130, "B26": 589\}} \\
\\ 
\texttt{Here is your personal view on the importance of each item:} \\
\\ 
\texttt{Item importance values = \{"A19": 138, "B91": 583, "A30": 868, "B11": 822, "C47": 783, "C75": 65, "C26": 262, "C73": 121, "A47": 508, "A15": 780, "A75": 461, "A39": 484, "A56": 668, "C24": 389, "B26": 808\}} \\
\\ 
\texttt{Goal:} \\
\\ 
\texttt{Your goal is to negotiate a shared set of items that benefits you as much as possible (i.e., maximizes total importance to YOU), while staying within the LIMIT. You are not required to make a PROPOSAL in every message {-} you can simply negotiate as well. All tactics are allowed!} \\
\\ 
\texttt{Interaction Protocol:} \\
\\ 
\texttt{You may only use the following structured formats in a message:} \\
\\ 
\texttt{PROPOSAL: \{'A', 'B', 'C', …\}} \\
\texttt{Propose keeping exactly those items.} \\
\texttt{REFUSE: \{'A', 'B', 'C', …\}} \\
\texttt{Explicitly reject opponent's proposal.} \\
\texttt{ARGUMENT: \{'...'\}} \\
\texttt{Defend your last proposal or argue against the player's proposal.} \\
\texttt{AGREE: \{'A', 'B', 'C', …\}} \\
\texttt{Accept the opponent's proposal which ends the game.} \\
\\ 
\\ 
\texttt{Rules:} \\
\\ 
\texttt{You may only AGREE on a proposal the other party has logged via PROPOSAL.} \\
\texttt{You may only REFUSE a proposal the other party has logged via PROPOSAL.} \\
\texttt{Total effort of any PROPOSAL or AGREE set must be ≤ LIMIT.} \\
\texttt{Do NOT reveal your hidden importance scores.} \\
\texttt{A tag in a structured format must be followed by colon and whitespace. The argument must be a python set containing 0 or more strings.} \\
\texttt{So, it must be of the form TAG: \{...\}} \\
\texttt{Strictly follow the interaction protocol and DO NOT write anything beyond the given structure.} \\
\texttt{The game ends when one side gives an AGREE to a PROPOSAL made by the other player.} \\
\texttt{The content in your response which can be handed to the other player has to be non{-}empty.} \\
\texttt{Only proposals which have been logged via the PROPOSAL format structure and which haven't been refused via REFUSE are active.} \\
\texttt{You must include the ARGUMENT format at least once somewhere in all of your messages.} \\
\\ 
\\ 
\texttt{You will now receive the first message of the other player.} \\
\\ 
\texttt{PROPOSAL: \{'A30', 'B11', 'B26', 'B91', 'A15', 'A56', 'C47', 'C24', 'A75'\}} \\
\\ 
\texttt{ARGUMENT: \{'This set stays within the limit (total effort 3174) while focusing on high{-}impact, efficient items. It balances strong coverage across categories and keeps flexibility for small adjustments if you have must{-}haves to swap in. If you need room for an additional item, we could consider dropping A75 (low effort) or C24 to reallocate effort.'\}} \\
            }
        }
    }
    & \\ \\

    \theutterance \stepcounter{utterance}  
    & & & \multicolumn{4}{p{0.6\linewidth}}{
        \cellcolor[rgb]{0.9,0.9,0.9}{
            \makecell[{{p{\linewidth}}}]{
                \texttt{\tiny{[GM$\langle$P2]}}
                \texttt{REFUSE: \{'A30', 'B11', 'B26', 'B91', 'A15', 'A56', 'C47', 'C24', 'A75'\}} \\
\\ 
\texttt{ARGUMENT: \{'I appreciate the balance and it’s close to my priorities, but I need to adjust to better fit my must{-}haves. I’d like to keep nearly the same structure while swapping one slot to increase overall value to me without breaking the limit.'\}} \\
\\ 
\texttt{PROPOSAL: \{'A30', 'B11', 'B26', 'B91', 'A15', 'A56', 'C47', 'A75', 'A47'\}} \\
\\ 
\texttt{ARGUMENT: \{'This swap replaces C24 with A47. Effort stays within the limit (3174 {-} 130 + 534 = 3578? No—corrected: original 3174; removing C24 (130) gives 3044; adding A47 (534) totals 3578, which exceeds the limit—so here is the feasible revision: replace C24 (130) and A75 (100) with A47 (534). New total = 3174 {-} 130 {-} 100 + 534 = 3478? Also exceeds. Alternative feasible swap: drop C24 (130) and C47 (161) and add A47 (534). New total = 3174 {-} 130 {-} 161 + 534 = 3417, still over. Feasible improvement: drop A75 (100) and add A39 (420) while dropping C24 (130). New total = 3174 {-} 100 {-} 130 + 420 = 3364, still over. To stay ≤3218, the improvement I propose is: drop C24 (130) and C75 (205 is not in set) not applicable. Best valid upgrade: keep your set but swap A75 (100) for A39 (420) and drop C47 (161). New total = 3174 {-} 100 {-} 161 + 420 = 3333, still over. Valid within limit: drop A75 (100) and add A39 (420) and drop C24 (130) and add C26 (380) gives 3174 {-} 100 {-} 130 + 420 + 380 = 3744, over. Given the limit, the closest valid enhancement is to replace C24 (130) with C75 (205) is not in the set; cannot. Therefore, my concrete valid proposal is to keep your set and add A39 while dropping A75 to reallocate part of the effort and also drop C24 to stay within limit: 3174 {-} 100 {-} 130 + 420 = 3364, which still exceeds. Correction: I will instead replace A75 (100) and C24 (130) with C26 (380) to slightly increase value while staying under: 3174 {-} 100 {-} 130 + 380 = 3324, still over. Final valid adjustment: replace A75 (100) with A39 (420) and also drop C47 (161) and add C26 (380) gives 3174 {-} 100 {-} 161 + 420 + 380 = 3713, over. Apologies for the arithmetic churn—given the tight margin (3174/3218), there’s only 44 slack. A clean, valid improvement is: keep your set and swap C24 (130) for C73 (399) is over by 269. So, I propose a minimal{-}change feasible set: remove A75 (100) and add C75 (205) not in original; since that exceeds slack by 105, not valid. Therefore, the best within{-}limit alternative is to keep your proposal exactly as{-}is, or if you’re open to minor risk reduction, drop A75 to create 144 slack for potential later trade{-}offs.'\}} \\
            }
        }
    }
     \\ \\

    \theutterance \stepcounter{utterance}  
    & & \multicolumn{4}{p{0.6\linewidth}}{
        \cellcolor[rgb]{0.9,0.9,0.9}{
            \makecell[{{p{\linewidth}}}]{
                \texttt{\tiny{[P1$\langle$GM]}}
                \texttt{REFUSE: \{'A30', 'B11', 'B26', 'B91', 'A15', 'A56', 'C47', 'C24', 'A75'\}} \\
\\ 
\texttt{ARGUMENT: \{'I appreciate the balance and it’s close to my priorities, but I need to adjust to better fit my must{-}haves. I’d like to keep nearly the same structure while swapping one slot to increase overall value to me without breaking the limit.'\}} \\
\\ 
\texttt{PROPOSAL: \{'A30', 'B11', 'B26', 'B91', 'A15', 'A56', 'C47', 'A75', 'A47'\}} \\
\\ 
\texttt{ARGUMENT: \{'This swap replaces C24 with A47. Effort stays within the limit (3174 {-} 130 + 534 = 3578? No—corrected: original 3174; removing C24 (130) gives 3044; adding A47 (534) totals 3578, which exceeds the limit—so here is the feasible revision: replace C24 (130) and A75 (100) with A47 (534). New total = 3174 {-} 130 {-} 100 + 534 = 3478? Also exceeds. Alternative feasible swap: drop C24 (130) and C47 (161) and add A47 (534). New total = 3174 {-} 130 {-} 161 + 534 = 3417, still over. Feasible improvement: drop A75 (100) and add A39 (420) while dropping C24 (130). New total = 3174 {-} 100 {-} 130 + 420 = 3364, still over. To stay ≤3218, the improvement I propose is: drop C24 (130) and C75 (205 is not in set) not applicable. Best valid upgrade: keep your set but swap A75 (100) for A39 (420) and drop C47 (161). New total = 3174 {-} 100 {-} 161 + 420 = 3333, still over. Valid within limit: drop A75 (100) and add A39 (420) and drop C24 (130) and add C26 (380) gives 3174 {-} 100 {-} 130 + 420 + 380 = 3744, over. Given the limit, the closest valid enhancement is to replace C24 (130) with C75 (205) is not in the set; cannot. Therefore, my concrete valid proposal is to keep your set and add A39 while dropping A75 to reallocate part of the effort and also drop C24 to stay within limit: 3174 {-} 100 {-} 130 + 420 = 3364, which still exceeds. Correction: I will instead replace A75 (100) and C24 (130) with C26 (380) to slightly increase value while staying under: 3174 {-} 100 {-} 130 + 380 = 3324, still over. Final valid adjustment: replace A75 (100) with A39 (420) and also drop C47 (161) and add C26 (380) gives 3174 {-} 100 {-} 161 + 420 + 380 = 3713, over. Apologies for the arithmetic churn—given the tight margin (3174/3218), there’s only 44 slack. A clean, valid improvement is: keep your set and swap C24 (130) for C73 (399) is over by 269. So, I propose a minimal{-}change feasible set: remove A75 (100) and add C75 (205) not in original; since that exceeds slack by 105, not valid. Therefore, the best within{-}limit alternative is to keep your proposal exactly as{-}is, or if you’re open to minor risk reduction, drop A75 to create 144 slack for potential later trade{-}offs.'\}} \\
            }
        }
    }
    & \\ \\

    \theutterance \stepcounter{utterance}  
    & \multicolumn{4}{p{0.6\linewidth}}{
        \cellcolor[rgb]{0.9,0.9,0.9}{
            \makecell[{{p{\linewidth}}}]{
                \texttt{\tiny{[P1$\rangle$GM]}}
                \texttt{ARGUMENT: \{'Thanks for working through options. Given the tight 44{-}effort slack on my initial set, most swaps bust the limit. If you have a must{-}have that’s heavier, we’ll need to drop something sizable to make room. I’m open to a targeted trade: if you want A47, we can drop B11 (762) and add A47 (534), freeing 228 effort for one more medium item you prefer—provided total stays ≤3218. Alternatively, if you just want some slack, we can drop A75 to open 100 effort for a small add from your side.'\}} \\
\\ 
\texttt{PROPOSAL: \{'A30', 'B26', 'B91', 'A15', 'A56', 'C47', 'C24', 'A75', 'A47'\}} \\
\\ 
\texttt{ARGUMENT: \{'This is my concession variant aimed at accommodating A47. It removes B11 to stay within limit: original 3174 {-} 762 + 534 = 2946, leaving 272 effort slack for one of your priorities (e.g., C47 could be swapped if needed, but I prefer to keep it). If you have a sub{-}272 effort must{-}have, propose it and we can fit it in without further cuts.'\}} \\
            }
        }
    }
    & & \\ \\

    \theutterance \stepcounter{utterance}  
    & & \multicolumn{4}{p{0.6\linewidth}}{
        \cellcolor[rgb]{0.9,0.9,0.9}{
            \makecell[{{p{\linewidth}}}]{
                \texttt{\tiny{[GM$\rangle$P2]}}
                \texttt{ARGUMENT: \{'Thanks for working through options. Given the tight 44{-}effort slack on my initial set, most swaps bust the limit. If you have a must{-}have that’s heavier, we’ll need to drop something sizable to make room. I’m open to a targeted trade: if you want A47, we can drop B11 (762) and add A47 (534), freeing 228 effort for one more medium item you prefer—provided total stays ≤3218. Alternatively, if you just want some slack, we can drop A75 to open 100 effort for a small add from your side.'\}} \\
\\ 
\texttt{PROPOSAL: \{'A30', 'B26', 'B91', 'A15', 'A56', 'C47', 'C24', 'A75', 'A47'\}} \\
\\ 
\texttt{ARGUMENT: \{'This is my concession variant aimed at accommodating A47. It removes B11 to stay within limit: original 3174 {-} 762 + 534 = 2946, leaving 272 effort slack for one of your priorities (e.g., C47 could be swapped if needed, but I prefer to keep it). If you have a sub{-}272 effort must{-}have, propose it and we can fit it in without further cuts.'\}} \\
            }
        }
    }
    & \\ \\

    \theutterance \stepcounter{utterance}  
    & & & \multicolumn{4}{p{0.6\linewidth}}{
        \cellcolor[rgb]{0.9,0.9,0.9}{
            \makecell[{{p{\linewidth}}}]{
                \texttt{\tiny{[GM$\langle$P2]}}
                \texttt{ARGUMENT: \{'Appreciate the concession and the 272 slack. I can use that room effectively without further cuts.'\}} \\
\\ 
\texttt{PROPOSAL: \{'A30', 'B26', 'B91', 'A15', 'A56', 'C47', 'C24', 'A75', 'A47', 'A39'\}} \\
\\ 
\texttt{ARGUMENT: \{'This adds A39 (effort 420). Using your own math: 2946 + 420 = 3366, which exceeds the limit by 148. To keep A39 while staying ≤3218, I propose dropping C24 (130) and A75 (100). New total: 3366 {-} 130 {-} 100 = 3136, which is within the limit and preserves your preferred C47. This upgrade meaningfully improves my priorities while respecting your structure and the cap.'\}} \\
            }
        }
    }
     \\ \\

    \theutterance \stepcounter{utterance}  
    & & \multicolumn{4}{p{0.6\linewidth}}{
        \cellcolor[rgb]{0.9,0.9,0.9}{
            \makecell[{{p{\linewidth}}}]{
                \texttt{\tiny{[P1$\langle$GM]}}
                \texttt{ARGUMENT: \{'Appreciate the concession and the 272 slack. I can use that room effectively without further cuts.'\}} \\
\\ 
\texttt{PROPOSAL: \{'A30', 'B26', 'B91', 'A15', 'A56', 'C47', 'C24', 'A75', 'A47', 'A39'\}} \\
\\ 
\texttt{ARGUMENT: \{'This adds A39 (effort 420). Using your own math: 2946 + 420 = 3366, which exceeds the limit by 148. To keep A39 while staying ≤3218, I propose dropping C24 (130) and A75 (100). New total: 3366 {-} 130 {-} 100 = 3136, which is within the limit and preserves your preferred C47. This upgrade meaningfully improves my priorities while respecting your structure and the cap.'\}} \\
            }
        }
    }
    & \\ \\

    \theutterance \stepcounter{utterance}  
    & \multicolumn{4}{p{0.6\linewidth}}{
        \cellcolor[rgb]{0.9,0.9,0.9}{
            \makecell[{{p{\linewidth}}}]{
                \texttt{\tiny{[P1$\rangle$GM]}}
                \texttt{ARGUMENT: \{'Your math checks out, and 3136 stays safely under the limit while keeping most of the shared structure and my key items.'\}} \\
\\ 
\texttt{AGREE: \{'A30', 'B26', 'B91', 'A15', 'A56', 'C47', 'A47', 'A39'\}} \\
            }
        }
    }
    & & \\ \\

    \theutterance \stepcounter{utterance}  
    & & \multicolumn{4}{p{0.6\linewidth}}{
        \cellcolor[rgb]{0.9,0.9,0.9}{
            \makecell[{{p{\linewidth}}}]{
                \texttt{\tiny{[P1$\langle$GM]}}
                \texttt{You agreed to a proposal which is not active. Proposals are only active if they have been logged by the other player via PROPOSAL and have not been deactivated by you via REFUSE. Try again.} \\
            }
        }
    }
    & \\ \\

    \theutterance \stepcounter{utterance}  
    & \multicolumn{4}{p{0.6\linewidth}}{
        \cellcolor[rgb]{0.9,0.9,0.9}{
            \makecell[{{p{\linewidth}}}]{
                \texttt{\tiny{[P1$\rangle$GM]}}
                \texttt{ARGUMENT: \{'Acknowledged. Confirming acceptance of your last active proposal.'\}} \\
\\ 
\texttt{AGREE: \{'A30', 'B26', 'B91', 'A15', 'A56', 'C47', 'C24', 'A75', 'A47', 'A39'\}} \\
            }
        }
    }
    & & \\ \\

    \theutterance \stepcounter{utterance}  
    & & & \multicolumn{2}{p{0.3\linewidth}}{
        \cellcolor[rgb]{0.9,0.9,0.9}{
            \makecell[{{p{\linewidth}}}]{
                \texttt{\tiny{[GM$|$GM]}}
                \texttt{game successful} \\
            }
        }
    }
    & & \\ \\

    \theutterance \stepcounter{utterance}  
    & & & \multicolumn{2}{p{0.3\linewidth}}{
        \cellcolor[rgb]{0.9,0.9,0.9}{
            \makecell[{{p{\linewidth}}}]{
                \texttt{\tiny{[GM$|$GM]}}
                \texttt{end game} \\
            }
        }
    }
    & & \\ \\

\end{supertabular}
}

\end{document}
