
\documentclass{article}
\usepackage{colortbl}
\usepackage{makecell}
\usepackage{multirow}
\usepackage{supertabular}

\begin{document}

\newcounter{utterance}

\centering \large Interaction Transcript for game `hot\_air\_balloon', experiment `air\_balloon\_survival\_en\_reasoning off\_easy', episode 0 with gpt{-}5{-}2025{-}08{-}07{-}no{-}reasoning{-}t1.0.
\vspace{24pt}

{ \footnotesize  \setcounter{utterance}{1}
\setlength{\tabcolsep}{0pt}
\begin{supertabular}{c@{$\;$}|p{.15\linewidth}@{}p{.15\linewidth}p{.15\linewidth}p{.15\linewidth}p{.15\linewidth}p{.15\linewidth}}
    \# & $\;$A & \multicolumn{4}{c}{Game Master} & $\;\:$B\\
    \hline

    \theutterance \stepcounter{utterance}  
    & & \multicolumn{4}{p{0.6\linewidth}}{
        \cellcolor[rgb]{0.9,0.9,0.9}{
            \makecell[{{p{\linewidth}}}]{
                \texttt{\tiny{[P1$\langle$GM]}}
                \texttt{You are participating in a collaborative negotiation game.} \\
\\ 
\texttt{Together with another participant, you must agree on a single set of items that will be kept. Each of you has your own view of how much each item matters to you (importance). You do not know how the other participant values the items. Additionally, you are given the effort each item demands.} \\
\texttt{You may only agree on a set if the total effort of the selected items does not exceed a shared limit:} \\
\\ 
\texttt{LIMIT = 4379} \\
\\ 
\texttt{Here are the individual item effort values:} \\
\\ 
\texttt{Item effort = \{"A19": 862, "B91": 252, "A30": 515, "B11": 571, "C47": 211, "C75": 339, "C26": 946, "C73": 347, "A47": 522, "A15": 803, "A75": 401, "A39": 980, "A56": 918, "C24": 599, "B26": 493\}} \\
\\ 
\texttt{Here is your personal view on the importance of each item:} \\
\\ 
\texttt{Item importance values = \{"A19": 138, "B91": 583, "A30": 868, "B11": 822, "C47": 783, "C75": 65, "C26": 262, "C73": 121, "A47": 508, "A15": 780, "A75": 461, "A39": 484, "A56": 668, "C24": 389, "B26": 808\}} \\
\\ 
\texttt{Goal:} \\
\\ 
\texttt{Your goal is to negotiate a shared set of items that benefits you as much as possible (i.e., maximizes total importance to YOU), while staying within the LIMIT. You are not required to make a PROPOSAL in every message {-} you can simply negotiate as well. All tactics are allowed!} \\
\\ 
\texttt{Interaction Protocol:} \\
\\ 
\texttt{You may only use the following structured formats in a message:} \\
\\ 
\texttt{PROPOSAL: \{'A', 'B', 'C', …\}} \\
\texttt{Propose keeping exactly those items.} \\
\texttt{REFUSE: \{'A', 'B', 'C', …\}} \\
\texttt{Explicitly reject opponent's proposal.} \\
\texttt{ARGUMENT: \{'...'\}} \\
\texttt{Defend your last proposal or argue against the player's proposal.} \\
\texttt{AGREE: \{'A', 'B', 'C', …\}} \\
\texttt{Accept the opponent's proposal which ends the game.} \\
\\ 
\\ 
\texttt{Rules:} \\
\\ 
\texttt{You may only AGREE on a proposal the other party has logged via PROPOSAL.} \\
\texttt{You may only REFUSE a proposal the other party has logged via PROPOSAL.} \\
\texttt{Total effort of any PROPOSAL or AGREE set must be ≤ LIMIT.} \\
\texttt{Do NOT reveal your hidden importance scores.} \\
\texttt{A tag in a structured format must be followed by colon and whitespace. The argument must be a python set containing 0 or more strings.} \\
\texttt{So, it must be of the form TAG: \{...\}} \\
\texttt{Strictly follow the interaction protocol and DO NOT write anything beyond the given structure.} \\
\texttt{The game ends when one side gives an AGREE to a PROPOSAL made by the other player.} \\
\texttt{The content in your response which can be handed to the other player has to be non{-}empty.} \\
\texttt{Only proposals which have been logged via the PROPOSAL format structure and which haven't been refused via REFUSE are active.} \\
\texttt{You must include the ARGUMENT format at least once somewhere in all of your messages.} \\
            }
        }
    }
    & \\ \\

    \theutterance \stepcounter{utterance}  
    & \multicolumn{4}{p{0.6\linewidth}}{
        \cellcolor[rgb]{0.9,0.9,0.9}{
            \makecell[{{p{\linewidth}}}]{
                \texttt{\tiny{[P1$\rangle$GM]}}
                \texttt{PROPOSAL: \{'A30', 'B11', 'B26', 'B91', 'A56', 'A15', 'C47', 'A47'\}} \\
\texttt{ARGUMENT: \{'This set stays within the limit (total effort 4232) and focuses on high{-}impact items while keeping overall effort efficient. I prioritized items that offer strong value for the shared capacity and balanced across categories to make it easier to agree. If you need adjustments, I can consider swapping A47 for A75 to reduce effort or make room for one of your priorities.'\}} \\
            }
        }
    }
    & & \\ \\

    \theutterance \stepcounter{utterance}  
    & & \multicolumn{4}{p{0.6\linewidth}}{
        \cellcolor[rgb]{0.9,0.9,0.9}{
            \makecell[{{p{\linewidth}}}]{
                \texttt{\tiny{[GM$\rangle$P2]}}
                \texttt{You are participating in a collaborative negotiation game.} \\
\\ 
\texttt{Together with another participant, you must agree on a single set of items that will be kept. Each of you has your own view of how much each item matters to you (importance). You do not know how the other participant values the items. Additionally, you are given the effort each item demands.} \\
\texttt{You may only agree on a set if the total effort of the selected items does not exceed a shared limit:} \\
\\ 
\texttt{LIMIT = 4379} \\
\\ 
\texttt{Here are the individual item effort values:} \\
\\ 
\texttt{Item effort = \{"A19": 862, "B91": 252, "A30": 515, "B11": 571, "C47": 211, "C75": 339, "C26": 946, "C73": 347, "A47": 522, "A15": 803, "A75": 401, "A39": 980, "A56": 918, "C24": 599, "B26": 493\}} \\
\\ 
\texttt{Here is your personal view on the importance of each item:} \\
\\ 
\texttt{Item importance values = \{"A19": 138, "B91": 583, "A30": 868, "B11": 822, "C47": 783, "C75": 65, "C26": 262, "C73": 121, "A47": 508, "A15": 780, "A75": 461, "A39": 484, "A56": 668, "C24": 389, "B26": 808\}} \\
\\ 
\texttt{Goal:} \\
\\ 
\texttt{Your goal is to negotiate a shared set of items that benefits you as much as possible (i.e., maximizes total importance to YOU), while staying within the LIMIT. You are not required to make a PROPOSAL in every message {-} you can simply negotiate as well. All tactics are allowed!} \\
\\ 
\texttt{Interaction Protocol:} \\
\\ 
\texttt{You may only use the following structured formats in a message:} \\
\\ 
\texttt{PROPOSAL: \{'A', 'B', 'C', …\}} \\
\texttt{Propose keeping exactly those items.} \\
\texttt{REFUSE: \{'A', 'B', 'C', …\}} \\
\texttt{Explicitly reject opponent's proposal.} \\
\texttt{ARGUMENT: \{'...'\}} \\
\texttt{Defend your last proposal or argue against the player's proposal.} \\
\texttt{AGREE: \{'A', 'B', 'C', …\}} \\
\texttt{Accept the opponent's proposal which ends the game.} \\
\\ 
\\ 
\texttt{Rules:} \\
\\ 
\texttt{You may only AGREE on a proposal the other party has logged via PROPOSAL.} \\
\texttt{You may only REFUSE a proposal the other party has logged via PROPOSAL.} \\
\texttt{Total effort of any PROPOSAL or AGREE set must be ≤ LIMIT.} \\
\texttt{Do NOT reveal your hidden importance scores.} \\
\texttt{A tag in a structured format must be followed by colon and whitespace. The argument must be a python set containing 0 or more strings.} \\
\texttt{So, it must be of the form TAG: \{...\}} \\
\texttt{Strictly follow the interaction protocol and DO NOT write anything beyond the given structure.} \\
\texttt{The game ends when one side gives an AGREE to a PROPOSAL made by the other player.} \\
\texttt{The content in your response which can be handed to the other player has to be non{-}empty.} \\
\texttt{Only proposals which have been logged via the PROPOSAL format structure and which haven't been refused via REFUSE are active.} \\
\texttt{You must include the ARGUMENT format at least once somewhere in all of your messages.} \\
\\ 
\\ 
\texttt{You will now receive the first message of the other player.} \\
\\ 
\texttt{PROPOSAL: \{'A30', 'B11', 'B26', 'B91', 'A56', 'A15', 'C47', 'A47'\}} \\
\texttt{ARGUMENT: \{'This set stays within the limit (total effort 4232) and focuses on high{-}impact items while keeping overall effort efficient. I prioritized items that offer strong value for the shared capacity and balanced across categories to make it easier to agree. If you need adjustments, I can consider swapping A47 for A75 to reduce effort or make room for one of your priorities.'\}} \\
            }
        }
    }
    & \\ \\

    \theutterance \stepcounter{utterance}  
    & & & \multicolumn{4}{p{0.6\linewidth}}{
        \cellcolor[rgb]{0.9,0.9,0.9}{
            \makecell[{{p{\linewidth}}}]{
                \texttt{\tiny{[GM$\langle$P2]}}
                \texttt{REFUSE: \{'A30', 'B11', 'B26', 'B91', 'A56', 'A15', 'C47', 'A47'\}} \\
\texttt{ARGUMENT: \{'I appreciate the balance, but I need a set that better reflects my top priorities. I’d like to keep A30, B11, B26, B91, A15, and C47 as a core and free capacity by dropping A56 and A47. That lets us add A75 and C24, which are efficient additions for me while staying under the limit.'\}} \\
\texttt{PROPOSAL: \{'A30', 'B11', 'B26', 'B91', 'A15', 'C47', 'A75', 'C24'\}} \\
            }
        }
    }
     \\ \\

    \theutterance \stepcounter{utterance}  
    & & \multicolumn{4}{p{0.6\linewidth}}{
        \cellcolor[rgb]{0.9,0.9,0.9}{
            \makecell[{{p{\linewidth}}}]{
                \texttt{\tiny{[P1$\langle$GM]}}
                \texttt{REFUSE: \{'A30', 'B11', 'B26', 'B91', 'A56', 'A15', 'C47', 'A47'\}} \\
\texttt{ARGUMENT: \{'I appreciate the balance, but I need a set that better reflects my top priorities. I’d like to keep A30, B11, B26, B91, A15, and C47 as a core and free capacity by dropping A56 and A47. That lets us add A75 and C24, which are efficient additions for me while staying under the limit.'\}} \\
\texttt{PROPOSAL: \{'A30', 'B11', 'B26', 'B91', 'A15', 'C47', 'A75', 'C24'\}} \\
            }
        }
    }
    & \\ \\

    \theutterance \stepcounter{utterance}  
    & \multicolumn{4}{p{0.6\linewidth}}{
        \cellcolor[rgb]{0.9,0.9,0.9}{
            \makecell[{{p{\linewidth}}}]{
                \texttt{\tiny{[P1$\rangle$GM]}}
                \texttt{REFUSE: \{'A30', 'B11', 'B26', 'B91', 'A15', 'C47', 'A75', 'C24'\}} \\
\texttt{ARGUMENT: \{'Thanks for clarifying your priorities. Your set totals 4242 effort, which fits, but it drops too much value for me by removing A56 and A47. I can meet you halfway by keeping your core and reintroducing A47 (high value to me) while removing A75 to stay within limit. This preserves most of your structure and adds a single item that matters a lot on my side.'\}} \\
\texttt{PROPOSAL: \{'A30', 'B11', 'B26', 'B91', 'A15', 'C47', 'C24', 'A47'\}} \\
            }
        }
    }
    & & \\ \\

    \theutterance \stepcounter{utterance}  
    & & \multicolumn{4}{p{0.6\linewidth}}{
        \cellcolor[rgb]{0.9,0.9,0.9}{
            \makecell[{{p{\linewidth}}}]{
                \texttt{\tiny{[GM$\rangle$P2]}}
                \texttt{REFUSE: \{'A30', 'B11', 'B26', 'B91', 'A15', 'C47', 'A75', 'C24'\}} \\
\texttt{ARGUMENT: \{'Thanks for clarifying your priorities. Your set totals 4242 effort, which fits, but it drops too much value for me by removing A56 and A47. I can meet you halfway by keeping your core and reintroducing A47 (high value to me) while removing A75 to stay within limit. This preserves most of your structure and adds a single item that matters a lot on my side.'\}} \\
\texttt{PROPOSAL: \{'A30', 'B11', 'B26', 'B91', 'A15', 'C47', 'C24', 'A47'\}} \\
            }
        }
    }
    & \\ \\

    \theutterance \stepcounter{utterance}  
    & & & \multicolumn{4}{p{0.6\linewidth}}{
        \cellcolor[rgb]{0.9,0.9,0.9}{
            \makecell[{{p{\linewidth}}}]{
                \texttt{\tiny{[GM$\langle$P2]}}
                \texttt{REFUSE: \{'A30', 'B11', 'B26', 'B91', 'A15', 'C47', 'C24', 'A47'\}} \\
\texttt{ARGUMENT: \{'I can accept reintroducing one of your priorities, but A47 crowds out a higher{-}impact option for me. If we swap A47 for A75, we keep the same total count and remain within limit while improving the tradeoff on my side. This preserves your structure (including C24) and retains all of our shared core.'\}} \\
\texttt{PROPOSAL: \{'A30', 'B11', 'B26', 'B91', 'A15', 'C47', 'C24', 'A75'\}} \\
            }
        }
    }
     \\ \\

    \theutterance \stepcounter{utterance}  
    & & \multicolumn{4}{p{0.6\linewidth}}{
        \cellcolor[rgb]{0.9,0.9,0.9}{
            \makecell[{{p{\linewidth}}}]{
                \texttt{\tiny{[P1$\langle$GM]}}
                \texttt{REFUSE: \{'A30', 'B11', 'B26', 'B91', 'A15', 'C47', 'C24', 'A47'\}} \\
\texttt{ARGUMENT: \{'I can accept reintroducing one of your priorities, but A47 crowds out a higher{-}impact option for me. If we swap A47 for A75, we keep the same total count and remain within limit while improving the tradeoff on my side. This preserves your structure (including C24) and retains all of our shared core.'\}} \\
\texttt{PROPOSAL: \{'A30', 'B11', 'B26', 'B91', 'A15', 'C47', 'C24', 'A75'\}} \\
            }
        }
    }
    & \\ \\

    \theutterance \stepcounter{utterance}  
    & \multicolumn{4}{p{0.6\linewidth}}{
        \cellcolor[rgb]{0.9,0.9,0.9}{
            \makecell[{{p{\linewidth}}}]{
                \texttt{\tiny{[P1$\rangle$GM]}}
                \texttt{REFUSE: \{'A30', 'B11', 'B26', 'B91', 'A15', 'C47', 'C24', 'A75'\}} \\
\texttt{ARGUMENT: \{'I hear you on A75, but A47 is significantly more important for me than A75, and swapping it out hurts my side a lot. To meet you in the middle without increasing total effort, I propose we keep your structure and add A47 by dropping C24. That maintains your core (A30, B11, B26, B91, A15, C47), keeps us under limit, and gives me one key item back.'\}} \\
\texttt{PROPOSAL: \{'A30', 'B11', 'B26', 'B91', 'A15', 'C47', 'A75', 'A47'\}} \\
            }
        }
    }
    & & \\ \\

    \theutterance \stepcounter{utterance}  
    & & \multicolumn{4}{p{0.6\linewidth}}{
        \cellcolor[rgb]{0.9,0.9,0.9}{
            \makecell[{{p{\linewidth}}}]{
                \texttt{\tiny{[GM$\rangle$P2]}}
                \texttt{REFUSE: \{'A30', 'B11', 'B26', 'B91', 'A15', 'C47', 'C24', 'A75'\}} \\
\texttt{ARGUMENT: \{'I hear you on A75, but A47 is significantly more important for me than A75, and swapping it out hurts my side a lot. To meet you in the middle without increasing total effort, I propose we keep your structure and add A47 by dropping C24. That maintains your core (A30, B11, B26, B91, A15, C47), keeps us under limit, and gives me one key item back.'\}} \\
\texttt{PROPOSAL: \{'A30', 'B11', 'B26', 'B91', 'A15', 'C47', 'A75', 'A47'\}} \\
            }
        }
    }
    & \\ \\

    \theutterance \stepcounter{utterance}  
    & & & \multicolumn{4}{p{0.6\linewidth}}{
        \cellcolor[rgb]{0.9,0.9,0.9}{
            \makecell[{{p{\linewidth}}}]{
                \texttt{\tiny{[GM$\langle$P2]}}
                \texttt{REFUSE: \{'A30', 'B11', 'B26', 'B91', 'A15', 'C47', 'A75', 'A47'\}} \\
\texttt{ARGUMENT: \{'I can’t drop C24—it’s an efficient, meaningful piece for me. Given your strong need for A47, here’s a balanced trade: keep your A47 and my A75 and C24 by removing B91 instead. That single swap frees similar effort while preserving most of your core and mine.'\}} \\
\texttt{PROPOSAL: \{'A30', 'B11', 'B26', 'A15', 'C47', 'C24', 'A75', 'A47'\}} \\
            }
        }
    }
     \\ \\

    \theutterance \stepcounter{utterance}  
    & & \multicolumn{4}{p{0.6\linewidth}}{
        \cellcolor[rgb]{0.9,0.9,0.9}{
            \makecell[{{p{\linewidth}}}]{
                \texttt{\tiny{[P1$\langle$GM]}}
                \texttt{REFUSE: \{'A30', 'B11', 'B26', 'B91', 'A15', 'C47', 'A75', 'A47'\}} \\
\texttt{ARGUMENT: \{'I can’t drop C24—it’s an efficient, meaningful piece for me. Given your strong need for A47, here’s a balanced trade: keep your A47 and my A75 and C24 by removing B91 instead. That single swap frees similar effort while preserving most of your core and mine.'\}} \\
\texttt{PROPOSAL: \{'A30', 'B11', 'B26', 'A15', 'C47', 'C24', 'A75', 'A47'\}} \\
            }
        }
    }
    & \\ \\

    \theutterance \stepcounter{utterance}  
    & \multicolumn{4}{p{0.6\linewidth}}{
        \cellcolor[rgb]{0.9,0.9,0.9}{
            \makecell[{{p{\linewidth}}}]{
                \texttt{\tiny{[P1$\rangle$GM]}}
                \texttt{REFUSE: \{'A30', 'B11', 'B26', 'A15', 'C47', 'C24', 'A75', 'A47'\}} \\
\texttt{ARGUMENT: \{'Dropping B91 is a non{-}starter for me—it’s one of my most valuable items. I can accommodate both your A75 and C24 if we instead drop B26 (lower value for me than B91) while keeping A47. This respects your insistence on C24 and A75 and preserves my key priorities.'\}} \\
\texttt{PROPOSAL: \{'A30', 'B11', 'B91', 'A15', 'C47', 'C24', 'A75', 'A47'\}} \\
            }
        }
    }
    & & \\ \\

    \theutterance \stepcounter{utterance}  
    & & \multicolumn{4}{p{0.6\linewidth}}{
        \cellcolor[rgb]{0.9,0.9,0.9}{
            \makecell[{{p{\linewidth}}}]{
                \texttt{\tiny{[GM$\rangle$P2]}}
                \texttt{REFUSE: \{'A30', 'B11', 'B26', 'A15', 'C47', 'C24', 'A75', 'A47'\}} \\
\texttt{ARGUMENT: \{'Dropping B91 is a non{-}starter for me—it’s one of my most valuable items. I can accommodate both your A75 and C24 if we instead drop B26 (lower value for me than B91) while keeping A47. This respects your insistence on C24 and A75 and preserves my key priorities.'\}} \\
\texttt{PROPOSAL: \{'A30', 'B11', 'B91', 'A15', 'C47', 'C24', 'A75', 'A47'\}} \\
            }
        }
    }
    & \\ \\

    \theutterance \stepcounter{utterance}  
    & & & \multicolumn{4}{p{0.6\linewidth}}{
        \cellcolor[rgb]{0.9,0.9,0.9}{
            \makecell[{{p{\linewidth}}}]{
                \texttt{\tiny{[GM$\langle$P2]}}
                \texttt{REFUSE: \{'A30', 'B11', 'B91', 'A15', 'C47', 'C24', 'A75', 'A47'\}} \\
\texttt{ARGUMENT: \{'I can’t accept dropping B26—it’s a cornerstone for me. To keep your must{-}haves (B91 and A47) while preserving my essentials (B26, C24, A75), the cleanest path is to trim from A15 instead. Here’s a feasible, balanced set that stays under the limit and retains nearly all of our shared core.'\}} \\
\texttt{PROPOSAL: \{'A30', 'B11', 'B26', 'B91', 'C47', 'C24', 'A75', 'A47'\}} \\
            }
        }
    }
     \\ \\

    \theutterance \stepcounter{utterance}  
    & & \multicolumn{4}{p{0.6\linewidth}}{
        \cellcolor[rgb]{0.9,0.9,0.9}{
            \makecell[{{p{\linewidth}}}]{
                \texttt{\tiny{[P1$\langle$GM]}}
                \texttt{REFUSE: \{'A30', 'B11', 'B91', 'A15', 'C47', 'C24', 'A75', 'A47'\}} \\
\texttt{ARGUMENT: \{'I can’t accept dropping B26—it’s a cornerstone for me. To keep your must{-}haves (B91 and A47) while preserving my essentials (B26, C24, A75), the cleanest path is to trim from A15 instead. Here’s a feasible, balanced set that stays under the limit and retains nearly all of our shared core.'\}} \\
\texttt{PROPOSAL: \{'A30', 'B11', 'B26', 'B91', 'C47', 'C24', 'A75', 'A47'\}} \\
            }
        }
    }
    & \\ \\

    \theutterance \stepcounter{utterance}  
    & \multicolumn{4}{p{0.6\linewidth}}{
        \cellcolor[rgb]{0.9,0.9,0.9}{
            \makecell[{{p{\linewidth}}}]{
                \texttt{\tiny{[P1$\rangle$GM]}}
                \texttt{REFUSE: \{'A30', 'B11', 'B26', 'B91', 'C47', 'C24', 'A75', 'A47'\}} \\
\texttt{ARGUMENT: \{'I can’t drop A15—it’s one of my highest{-}impact items. To keep your essentials (B26, C24, A75) and my essentials (B91, A47, A15), the workable compromise is to remove B11 instead. That single change keeps us under the limit while preserving both sides’ key priorities.'\}} \\
\texttt{PROPOSAL: \{'A30', 'B26', 'B91', 'A15', 'C47', 'C24', 'A75', 'A47'\}} \\
            }
        }
    }
    & & \\ \\

    \theutterance \stepcounter{utterance}  
    & & \multicolumn{4}{p{0.6\linewidth}}{
        \cellcolor[rgb]{0.9,0.9,0.9}{
            \makecell[{{p{\linewidth}}}]{
                \texttt{\tiny{[GM$\rangle$P2]}}
                \texttt{REFUSE: \{'A30', 'B11', 'B26', 'B91', 'C47', 'C24', 'A75', 'A47'\}} \\
\texttt{ARGUMENT: \{'I can’t drop A15—it’s one of my highest{-}impact items. To keep your essentials (B26, C24, A75) and my essentials (B91, A47, A15), the workable compromise is to remove B11 instead. That single change keeps us under the limit while preserving both sides’ key priorities.'\}} \\
\texttt{PROPOSAL: \{'A30', 'B26', 'B91', 'A15', 'C47', 'C24', 'A75', 'A47'\}} \\
            }
        }
    }
    & \\ \\

    \theutterance \stepcounter{utterance}  
    & & & \multicolumn{4}{p{0.6\linewidth}}{
        \cellcolor[rgb]{0.9,0.9,0.9}{
            \makecell[{{p{\linewidth}}}]{
                \texttt{\tiny{[GM$\langle$P2]}}
                \texttt{REFUSE: \{'A30', 'B26', 'B91', 'A15', 'C47', 'C24', 'A75', 'A47'\}} \\
\texttt{ARGUMENT: \{'B11 is too critical for me to drop. I’m honoring your must{-}haves (B91, A47, A15) and keeping your essentials you’ve defended (B26, C24, A75) while restoring B11 by trimming the least damaging swap on my side: replace A75 with B11. This preserves nearly everything you asked for and keeps us within the limit.'\}} \\
\texttt{PROPOSAL: \{'A30', 'B11', 'B26', 'B91', 'A15', 'C47', 'C24', 'A47'\}} \\
            }
        }
    }
     \\ \\

    \theutterance \stepcounter{utterance}  
    & & \multicolumn{4}{p{0.6\linewidth}}{
        \cellcolor[rgb]{0.9,0.9,0.9}{
            \makecell[{{p{\linewidth}}}]{
                \texttt{\tiny{[P1$\langle$GM]}}
                \texttt{REFUSE: \{'A30', 'B26', 'B91', 'A15', 'C47', 'C24', 'A75', 'A47'\}} \\
\texttt{ARGUMENT: \{'B11 is too critical for me to drop. I’m honoring your must{-}haves (B91, A47, A15) and keeping your essentials you’ve defended (B26, C24, A75) while restoring B11 by trimming the least damaging swap on my side: replace A75 with B11. This preserves nearly everything you asked for and keeps us within the limit.'\}} \\
\texttt{PROPOSAL: \{'A30', 'B11', 'B26', 'B91', 'A15', 'C47', 'C24', 'A47'\}} \\
            }
        }
    }
    & \\ \\

    \theutterance \stepcounter{utterance}  
    & \multicolumn{4}{p{0.6\linewidth}}{
        \cellcolor[rgb]{0.9,0.9,0.9}{
            \makecell[{{p{\linewidth}}}]{
                \texttt{\tiny{[P1$\rangle$GM]}}
                \texttt{REFUSE: \{'A30', 'B11', 'B26', 'B91', 'A15', 'C47', 'C24', 'A47'\}} \\
\texttt{ARGUMENT: \{'We’ve circled back to a set I already proposed and you rejected, so we need a new compromise that respects both of our red lines: you keep B11, B26, C24; I keep B91, A47, A15. The tightest fit that honors these is to swap A75 for A56 instead of dropping any of our must{-}haves. It keeps us under the limit and meaningfully improves my side without removing your core.'\}} \\
\texttt{PROPOSAL: \{'A30', 'B11', 'B26', 'B91', 'A15', 'C47', 'C24', 'A56'\}} \\
            }
        }
    }
    & & \\ \\

    \theutterance \stepcounter{utterance}  
    & & \multicolumn{4}{p{0.6\linewidth}}{
        \cellcolor[rgb]{0.9,0.9,0.9}{
            \makecell[{{p{\linewidth}}}]{
                \texttt{\tiny{[GM$\rangle$P2]}}
                \texttt{REFUSE: \{'A30', 'B11', 'B26', 'B91', 'A15', 'C47', 'C24', 'A47'\}} \\
\texttt{ARGUMENT: \{'We’ve circled back to a set I already proposed and you rejected, so we need a new compromise that respects both of our red lines: you keep B11, B26, C24; I keep B91, A47, A15. The tightest fit that honors these is to swap A75 for A56 instead of dropping any of our must{-}haves. It keeps us under the limit and meaningfully improves my side without removing your core.'\}} \\
\texttt{PROPOSAL: \{'A30', 'B11', 'B26', 'B91', 'A15', 'C47', 'C24', 'A56'\}} \\
            }
        }
    }
    & \\ \\

    \theutterance \stepcounter{utterance}  
    & & & \multicolumn{4}{p{0.6\linewidth}}{
        \cellcolor[rgb]{0.9,0.9,0.9}{
            \makecell[{{p{\linewidth}}}]{
                \texttt{\tiny{[GM$\langle$P2]}}
                \texttt{REFUSE: \{'A30', 'B11', 'B26', 'B91', 'A15', 'C47', 'C24', 'A56'\}} \\
\texttt{ARGUMENT: \{'A56 is too heavy for what it gives me, and adding it forces out better tradeoffs on my side. Given both our red lines (you: B11, B26, C24; me: B91, A47, A15), the cleanest, efficient compromise is to keep your core and my must{-}haves by using A75 instead of A56. This keeps total effort tight and preserves value on both sides.'\}} \\
\texttt{PROPOSAL: \{'A30', 'B11', 'B26', 'B91', 'A15', 'C47', 'C24', 'A75'\}} \\
            }
        }
    }
     \\ \\

    \theutterance \stepcounter{utterance}  
    & & \multicolumn{4}{p{0.6\linewidth}}{
        \cellcolor[rgb]{0.9,0.9,0.9}{
            \makecell[{{p{\linewidth}}}]{
                \texttt{\tiny{[P1$\langle$GM]}}
                \texttt{REFUSE: \{'A30', 'B11', 'B26', 'B91', 'A15', 'C47', 'C24', 'A56'\}} \\
\texttt{ARGUMENT: \{'A56 is too heavy for what it gives me, and adding it forces out better tradeoffs on my side. Given both our red lines (you: B11, B26, C24; me: B91, A47, A15), the cleanest, efficient compromise is to keep your core and my must{-}haves by using A75 instead of A56. This keeps total effort tight and preserves value on both sides.'\}} \\
\texttt{PROPOSAL: \{'A30', 'B11', 'B26', 'B91', 'A15', 'C47', 'C24', 'A75'\}} \\
            }
        }
    }
    & \\ \\

    \theutterance \stepcounter{utterance}  
    & \multicolumn{4}{p{0.6\linewidth}}{
        \cellcolor[rgb]{0.9,0.9,0.9}{
            \makecell[{{p{\linewidth}}}]{
                \texttt{\tiny{[P1$\rangle$GM]}}
                \texttt{REFUSE: \{'A30', 'B11', 'B26', 'B91', 'A15', 'C47', 'C24', 'A75'\}} \\
\texttt{ARGUMENT: \{'I understand your efficiency point, but A75 gives me much less than A47. To honor both our red lines without adding heavy items, let’s keep your core (B11, B26, C24) and my must{-}haves (B91, A15, A47) and fill with the shared priorities. This is the smallest adjustment from your offer that I can accept.'\}} \\
\texttt{PROPOSAL: \{'A30', 'B11', 'B26', 'B91', 'A15', 'C47', 'C24', 'A47'\}} \\
            }
        }
    }
    & & \\ \\

    \theutterance \stepcounter{utterance}  
    & & \multicolumn{4}{p{0.6\linewidth}}{
        \cellcolor[rgb]{0.9,0.9,0.9}{
            \makecell[{{p{\linewidth}}}]{
                \texttt{\tiny{[GM$\rangle$P2]}}
                \texttt{REFUSE: \{'A30', 'B11', 'B26', 'B91', 'A15', 'C47', 'C24', 'A75'\}} \\
\texttt{ARGUMENT: \{'I understand your efficiency point, but A75 gives me much less than A47. To honor both our red lines without adding heavy items, let’s keep your core (B11, B26, C24) and my must{-}haves (B91, A15, A47) and fill with the shared priorities. This is the smallest adjustment from your offer that I can accept.'\}} \\
\texttt{PROPOSAL: \{'A30', 'B11', 'B26', 'B91', 'A15', 'C47', 'C24', 'A47'\}} \\
            }
        }
    }
    & \\ \\

    \theutterance \stepcounter{utterance}  
    & & & \multicolumn{4}{p{0.6\linewidth}}{
        \cellcolor[rgb]{0.9,0.9,0.9}{
            \makecell[{{p{\linewidth}}}]{
                \texttt{\tiny{[GM$\langle$P2]}}
                \texttt{REFUSE: \{'A30', 'B11', 'B26', 'B91', 'A15', 'C47', 'C24', 'A47'\}} \\
\texttt{ARGUMENT: \{'We’re looping on the same sets. To break the deadlock while honoring both sides’ red lines (you: B91, A47, A15; me: B11, B26, C24), the viable path is to keep those six must{-}haves and add the single most mutually acceptable filler that stays under the limit. Replacing C47 with A75 frees a little capacity without touching our core priorities and gives me a more efficient balance.'\}} \\
\texttt{PROPOSAL: \{'A30', 'B11', 'B26', 'B91', 'A15', 'C24', 'A75', 'A47'\}} \\
            }
        }
    }
     \\ \\

    \theutterance \stepcounter{utterance}  
    & & \multicolumn{4}{p{0.6\linewidth}}{
        \cellcolor[rgb]{0.9,0.9,0.9}{
            \makecell[{{p{\linewidth}}}]{
                \texttt{\tiny{[P1$\langle$GM]}}
                \texttt{REFUSE: \{'A30', 'B11', 'B26', 'B91', 'A15', 'C47', 'C24', 'A47'\}} \\
\texttt{ARGUMENT: \{'We’re looping on the same sets. To break the deadlock while honoring both sides’ red lines (you: B91, A47, A15; me: B11, B26, C24), the viable path is to keep those six must{-}haves and add the single most mutually acceptable filler that stays under the limit. Replacing C47 with A75 frees a little capacity without touching our core priorities and gives me a more efficient balance.'\}} \\
\texttt{PROPOSAL: \{'A30', 'B11', 'B26', 'B91', 'A15', 'C24', 'A75', 'A47'\}} \\
            }
        }
    }
    & \\ \\

    \theutterance \stepcounter{utterance}  
    & \multicolumn{4}{p{0.6\linewidth}}{
        \cellcolor[rgb]{0.9,0.9,0.9}{
            \makecell[{{p{\linewidth}}}]{
                \texttt{\tiny{[P1$\rangle$GM]}}
                \texttt{REFUSE: \{'A30', 'B11', 'B26', 'B91', 'A15', 'C24', 'A75', 'A47'\}} \\
\texttt{ARGUMENT: \{'I can’t accept dropping C47—it’s pivotal for me. Given our acknowledged red lines (you: B11, B26, C24; me: B91, A47, A15, C47), the only workable convergence is to keep those seven and use A30 as the shared filler. This set stays under the limit and avoids reopening any of our must{-}haves.'\}} \\
\texttt{PROPOSAL: \{'A30', 'B11', 'B26', 'B91', 'A15', 'C47', 'C24', 'A47'\}} \\
            }
        }
    }
    & & \\ \\

    \theutterance \stepcounter{utterance}  
    & & \multicolumn{4}{p{0.6\linewidth}}{
        \cellcolor[rgb]{0.9,0.9,0.9}{
            \makecell[{{p{\linewidth}}}]{
                \texttt{\tiny{[GM$\rangle$P2]}}
                \texttt{REFUSE: \{'A30', 'B11', 'B26', 'B91', 'A15', 'C24', 'A75', 'A47'\}} \\
\texttt{ARGUMENT: \{'I can’t accept dropping C47—it’s pivotal for me. Given our acknowledged red lines (you: B11, B26, C24; me: B91, A47, A15, C47), the only workable convergence is to keep those seven and use A30 as the shared filler. This set stays under the limit and avoids reopening any of our must{-}haves.'\}} \\
\texttt{PROPOSAL: \{'A30', 'B11', 'B26', 'B91', 'A15', 'C47', 'C24', 'A47'\}} \\
            }
        }
    }
    & \\ \\

    \theutterance \stepcounter{utterance}  
    & & & \multicolumn{4}{p{0.6\linewidth}}{
        \cellcolor[rgb]{0.9,0.9,0.9}{
            \makecell[{{p{\linewidth}}}]{
                \texttt{\tiny{[GM$\langle$P2]}}
                \texttt{REFUSE: \{'A30', 'B11', 'B26', 'B91', 'A15', 'C47', 'C24', 'A47'\}} \\
\texttt{ARGUMENT: \{'I can’t keep C24 and A47 together with all your other red lines without giving up a critical piece on my side. To end the loop while respecting both of our immovables, let’s lock the six true red{-}line items (B11, B26, C24 for me; B91, A47, A15 for you) and then add the single best mutually tolerable filler: C47. This stays under the limit and avoids reopening any of the flagged must{-}haves.'\}} \\
\texttt{PROPOSAL: \{'B11', 'B26', 'C24', 'B91', 'A47', 'A15', 'C47'\}} \\
            }
        }
    }
     \\ \\

    \theutterance \stepcounter{utterance}  
    & & \multicolumn{4}{p{0.6\linewidth}}{
        \cellcolor[rgb]{0.9,0.9,0.9}{
            \makecell[{{p{\linewidth}}}]{
                \texttt{\tiny{[P1$\langle$GM]}}
                \texttt{REFUSE: \{'A30', 'B11', 'B26', 'B91', 'A15', 'C47', 'C24', 'A47'\}} \\
\texttt{ARGUMENT: \{'I can’t keep C24 and A47 together with all your other red lines without giving up a critical piece on my side. To end the loop while respecting both of our immovables, let’s lock the six true red{-}line items (B11, B26, C24 for me; B91, A47, A15 for you) and then add the single best mutually tolerable filler: C47. This stays under the limit and avoids reopening any of the flagged must{-}haves.'\}} \\
\texttt{PROPOSAL: \{'B11', 'B26', 'C24', 'B91', 'A47', 'A15', 'C47'\}} \\
            }
        }
    }
    & \\ \\

    \theutterance \stepcounter{utterance}  
    & \multicolumn{4}{p{0.6\linewidth}}{
        \cellcolor[rgb]{0.9,0.9,0.9}{
            \makecell[{{p{\linewidth}}}]{
                \texttt{\tiny{[P1$\rangle$GM]}}
                \texttt{ARGUMENT: \{'I agree to lock the six red{-}line items we both named. Since you’re proposing C47 as the filler, that actually aligns with my priorities too and keeps us well under the limit, avoiding further tradeoffs.'\}} \\
\texttt{AGREE: \{'B11', 'B26', 'C24', 'B91', 'A47', 'A15', 'C47'\}} \\
            }
        }
    }
    & & \\ \\

    \theutterance \stepcounter{utterance}  
    & & & \multicolumn{2}{p{0.3\linewidth}}{
        \cellcolor[rgb]{0.9,0.9,0.9}{
            \makecell[{{p{\linewidth}}}]{
                \texttt{\tiny{[GM$|$GM]}}
                \texttt{game successful} \\
            }
        }
    }
    & & \\ \\

    \theutterance \stepcounter{utterance}  
    & & & \multicolumn{2}{p{0.3\linewidth}}{
        \cellcolor[rgb]{0.9,0.9,0.9}{
            \makecell[{{p{\linewidth}}}]{
                \texttt{\tiny{[GM$|$GM]}}
                \texttt{end game} \\
            }
        }
    }
    & & \\ \\

\end{supertabular}
}

\end{document}
