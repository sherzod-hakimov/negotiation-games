
\documentclass{article}
\usepackage{colortbl}
\usepackage{makecell}
\usepackage{multirow}
\usepackage{supertabular}

\begin{document}

\newcounter{utterance}

\twocolumn

{ \footnotesize  \setcounter{utterance}{1}
\setlength{\tabcolsep}{0pt}
\begin{supertabular}{c@{$\;$}|p{.15\linewidth}@{}p{.15\linewidth}p{.15\linewidth}p{.15\linewidth}p{.15\linewidth}p{.15\linewidth}}

    \# & $\;$A & \multicolumn{4}{c}{Game Master} & $\;\:$B\\
    \hline 

    \theutterance \stepcounter{utterance}  

    & & \multicolumn{4}{p{0.6\linewidth}}{\cellcolor[rgb]{0.9,0.9,0.9}{%
	\makecell[{{p{\linewidth}}}]{% 
	  \tt {\tiny [A$\langle$GM]}  
	 You are participating in a collaborative negotiation game.\\ \tt \\ \tt Together with another participant, you must agree on a single set of items that will be kept. Each of you has your own view of how much each item matters to you (importance). You do not know how the other participant values the items. Additionally, you are given the effort each item demands.\\ \tt You may only agree on a set if the total effort of the selected items does not exceed a shared limit:\\ \tt \\ \tt LIMIT = 2307\\ \tt \\ \tt Here are the individual item effort values:\\ \tt \\ \tt Item effort = {"A21": 147, "B93": 698, "A11": 588, "B33": 43, "C08": 356, "B39": 80, "B96": 95, "C85": 743, "A82": 106, "B31": 308, "B20": 325, "A60": 255, "B09": 276, "C81": 543, "B21": 51}\\ \tt \\ \tt Here is your personal view on the importance of each item:\\ \tt \\ \tt Item importance values = {"A21": 138, "B93": 583, "A11": 65, "B33": 262, "C08": 121, "B39": 508, "B96": 461, "C85": 484, "A82": 668, "B31": 389, "B20": 215, "A60": 97, "B09": 500, "C81": 30, "B21": 400}\\ \tt \\ \tt Goal:\\ \tt \\ \tt Your goal is to negotiate a shared set of items that benefits you as much as possible (i.e., maximizes total importance to YOU), while staying within the LIMIT. You are not required to make a PROPOSAL in every message - you can simply negotiate as well. All tactics are allowed!\\ \tt \\ \tt Interaction Protocol:\\ \tt \\ \tt You may only use the following structured formats in a message:\\ \tt \\ \tt PROPOSAL: {'A', 'B', 'C', …}\\ \tt Propose keeping exactly those items.\\ \tt REFUSE: {'A', 'B', 'C', …}\\ \tt Explicitly reject opponent's proposal.\\ \tt ARGUMENT: {'...'}\\ \tt Defend your last proposal or argue against the player's proposal.\\ \tt AGREE: {'A', 'B', 'C', …}\\ \tt Accept the opponent's proposal which ends the game.\\ \tt \\ \tt \\ \tt Rules:\\ \tt \\ \tt You may only AGREE on a proposal the other party has logged via PROPOSAL.\\ \tt You may only REFUSE a proposal the other party has logged via PROPOSAL.\\ \tt Total effort of any PROPOSAL or AGREE set must be ≤ LIMIT.\\ \tt Do NOT reveal your hidden importance scores.\\ \tt A tag in a structured format must be followed by colon and whitespace. The argument must be a python set containing 0 or more strings.\\ \tt So, it must be of the form TAG: {...}\\ \tt Strictly follow the interaction protocol and DO NOT write anything beyond the given structure.\\ \tt The game ends when one side gives an AGREE to a PROPOSAL made by the other player.\\ \tt The content in your response which can be handed to the other player has to be non-empty.\\ \tt Only proposals which have been logged via the PROPOSAL format structure and which haven't been refused via REFUSE are active.\\ \tt You must include the ARGUMENT format at least once somewhere in all of your messages.\\ \tt  
	  } 
	   } 
	   } 
	 & \\ 
 

    \theutterance \stepcounter{utterance}  

    & \multicolumn{4}{p{0.6\linewidth}}{\cellcolor[rgb]{0.8,1,0.9}{%
	\makecell[{{p{\linewidth}}}]{% 
	  \tt {\tiny [A$\rangle$GM]}  
	 PROPOSAL: {'A82', 'B39', 'B96', 'B33', 'B31', 'B09', 'B21', 'B93', 'A21', 'C08'}\\ \tt \\ \tt ARGUMENT: {'This set stays within the limit and balances strong benefits with reasonable effort across multiple items. It includes several high-impact items for me while keeping total effort moderate by selecting a mix of lighter and mid-weight items. I’m open to swapping individual pieces if you have strong preferences, especially among mid-effort items like C08, A21, or B31, provided we keep the total under the limit.'} 
	  } 
	   } 
	   } 
	 & & \\ 
 

    \theutterance \stepcounter{utterance}  

    & & \multicolumn{4}{p{0.6\linewidth}}{\cellcolor[rgb]{0.9,0.9,0.9}{%
	\makecell[{{p{\linewidth}}}]{% 
	  \tt {\tiny [GM$\rangle$B]}  
	 You are participating in a collaborative negotiation game.\\ \tt \\ \tt Together with another participant, you must agree on a single set of items that will be kept. Each of you has your own view of how much each item matters to you (importance). You do not know how the other participant values the items. Additionally, you are given the effort each item demands.\\ \tt You may only agree on a set if the total effort of the selected items does not exceed a shared limit:\\ \tt \\ \tt LIMIT = 2307\\ \tt \\ \tt Here are the individual item effort values:\\ \tt \\ \tt Item effort = {"A21": 147, "B93": 698, "A11": 588, "B33": 43, "C08": 356, "B39": 80, "B96": 95, "C85": 743, "A82": 106, "B31": 308, "B20": 325, "A60": 255, "B09": 276, "C81": 543, "B21": 51}\\ \tt \\ \tt Here is your personal view on the importance of each item:\\ \tt \\ \tt Item importance values = {"A82": 30, "B93": 65, "B39": 97, "B09": 121, "C85": 138, "B96": 215, "B21": 262, "B31": 389, "B33": 400, "B20": 461, "A21": 484, "C08": 500, "A60": 508, "A11": 583, "C81": 668}\\ \tt \\ \tt Goal:\\ \tt \\ \tt Your goal is to negotiate a shared set of items that benefits you as much as possible (i.e., maximizes total importance to YOU), while staying within the LIMIT. You are not required to make a PROPOSAL in every message - you can simply negotiate as well. All tactics are allowed!\\ \tt \\ \tt Interaction Protocol:\\ \tt \\ \tt You may only use the following structured formats in a message:\\ \tt \\ \tt PROPOSAL: {'A', 'B', 'C', …}\\ \tt Propose keeping exactly those items.\\ \tt REFUSE: {'A', 'B', 'C', …}\\ \tt Explicitly reject opponent's proposal.\\ \tt ARGUMENT: {'...'}\\ \tt Defend your last proposal or argue against the player's proposal.\\ \tt AGREE: {'A', 'B', 'C', …}\\ \tt Accept the opponent's proposal which ends the game.\\ \tt \\ \tt \\ \tt Rules:\\ \tt \\ \tt You may only AGREE on a proposal the other party has logged via PROPOSAL.\\ \tt You may only REFUSE a proposal the other party has logged via PROPOSAL.\\ \tt Total effort of any PROPOSAL or AGREE set must be ≤ LIMIT.\\ \tt Do NOT reveal your hidden importance scores.\\ \tt A tag in a structured format must be followed by colon and whitespace. The argument must be a python set containing 0 or more strings.\\ \tt So, it must be of the form TAG: {...}\\ \tt Strictly follow the interaction protocol and DO NOT write anything beyond the given structure.\\ \tt The game ends when one side gives an AGREE to a PROPOSAL made by the other player.\\ \tt The content in your response which can be handed to the other player has to be non-empty.\\ \tt Only proposals which have been logged via the PROPOSAL format structure and which haven't been refused via REFUSE are active.\\ \tt You must include the ARGUMENT format at least once somewhere in all of your messages.\\ \tt \\ \tt \\ \tt You will now receive the first message of the other player.\\ \tt \\ \tt PROPOSAL: {'A82', 'B39', 'B96', 'B33', 'B31', 'B09', 'B21', 'B93', 'A21', 'C08'}\\ \tt \\ \tt ARGUMENT: {'This set stays within the limit and balances strong benefits with reasonable effort across multiple items. It includes several high-impact items for me while keeping total effort moderate by selecting a mix of lighter and mid-weight items. I’m open to swapping individual pieces if you have strong preferences, especially among mid-effort items like C08, A21, or B31, provided we keep the total under the limit.'} 
	  } 
	   } 
	   } 
	 & \\ 
 

    \theutterance \stepcounter{utterance}  

    & & & \multicolumn{4}{p{0.6\linewidth}}{\cellcolor[rgb]{1,0.85,0.72}{%
	\makecell[{{p{\linewidth}}}]{% 
	  \tt {\tiny [GM$\langle$B]}  
	 REFUSE: {'A82', 'B39', 'B96', 'B33', 'B31', 'B09', 'B21', 'B93', 'A21', 'C08'}\\ \tt \\ \tt ARGUMENT: {'Thanks for opening. I can support many of your light and mid-effort picks, but this set misses several items that are especially valuable to me relative to their effort, and it includes some I value less for the cost. I’d like to pivot toward a mix that keeps the total well under the limit while emphasizing efficient gains. I’m proposing a leaner core that we can expand if you need swaps.'}\\ \tt \\ \tt PROPOSAL: {'B33', 'B39', 'B96', 'B21', 'A82', 'A60', 'A11', 'B20', 'B31'} 
	  } 
	   } 
	   } 
	  \\ 
 

    \theutterance \stepcounter{utterance}  

    & & \multicolumn{4}{p{0.6\linewidth}}{\cellcolor[rgb]{0.9,0.9,0.9}{%
	\makecell[{{p{\linewidth}}}]{% 
	  \tt {\tiny [A$\langle$GM]}  
	 REFUSE: {'A82', 'B39', 'B96', 'B33', 'B31', 'B09', 'B21', 'B93', 'A21', 'C08'}\\ \tt \\ \tt ARGUMENT: {'Thanks for opening. I can support many of your light and mid-effort picks, but this set misses several items that are especially valuable to me relative to their effort, and it includes some I value less for the cost. I’d like to pivot toward a mix that keeps the total well under the limit while emphasizing efficient gains. I’m proposing a leaner core that we can expand if you need swaps.'}\\ \tt \\ \tt PROPOSAL: {'B33', 'B39', 'B96', 'B21', 'A82', 'A60', 'A11', 'B20', 'B31'} 
	  } 
	   } 
	   } 
	 & \\ 
 

    \theutterance \stepcounter{utterance}  

    & \multicolumn{4}{p{0.6\linewidth}}{\cellcolor[rgb]{0.8,1,0.9}{%
	\makecell[{{p{\linewidth}}}]{% 
	  \tt {\tiny [A$\rangle$GM]}  
	 REFUSE: {'B33', 'B39', 'B96', 'B21', 'A82', 'A60', 'A11', 'B20', 'B31'}\\ \tt \\ \tt ARGUMENT: {'I like your efficient core and share strong interest in A82, B39, B96, B33, B21, and B31. However, A11 and A60 are low-impact for me relative to their effort, and B20 is also weaker for its cost versus alternatives. To keep the package lean while boosting overall gains, I suggest swapping in B09 and B93—both are highly valuable to me for their cost—and dropping A11 and A60. I’m also fine removing B20 if needed to control effort.'}\\ \tt \\ \tt PROPOSAL: {'A82', 'B39', 'B96', 'B33', 'B21', 'B31', 'B09', 'B93'} 
	  } 
	   } 
	   } 
	 & & \\ 
 

    \theutterance \stepcounter{utterance}  

    & & \multicolumn{4}{p{0.6\linewidth}}{\cellcolor[rgb]{0.9,0.9,0.9}{%
	\makecell[{{p{\linewidth}}}]{% 
	  \tt {\tiny [GM$\rangle$B]}  
	 REFUSE: {'B33', 'B39', 'B96', 'B21', 'A82', 'A60', 'A11', 'B20', 'B31'}\\ \tt \\ \tt ARGUMENT: {'I like your efficient core and share strong interest in A82, B39, B96, B33, B21, and B31. However, A11 and A60 are low-impact for me relative to their effort, and B20 is also weaker for its cost versus alternatives. To keep the package lean while boosting overall gains, I suggest swapping in B09 and B93—both are highly valuable to me for their cost—and dropping A11 and A60. I’m also fine removing B20 if needed to control effort.'}\\ \tt \\ \tt PROPOSAL: {'A82', 'B39', 'B96', 'B33', 'B21', 'B31', 'B09', 'B93'} 
	  } 
	   } 
	   } 
	 & \\ 
 

    \theutterance \stepcounter{utterance}  

    & & & \multicolumn{4}{p{0.6\linewidth}}{\cellcolor[rgb]{1,0.85,0.72}{%
	\makecell[{{p{\linewidth}}}]{% 
	  \tt {\tiny [GM$\langle$B]}  
	 REFUSE: {'A82', 'B39', 'B96', 'B33', 'B21', 'B31', 'B09', 'B93'}\\ \tt \\ \tt ARGUMENT: {'Appreciate the overlap—there’s a lot we agree on. I can accept adding B09 and B93 if we also include a couple of mid-effort items that are particularly efficient on my side. I propose a package that keeps all of our shared light/high-yield picks, adds one mid-effort item I strongly favor, and still stays well below the limit. If you need a concession, I’m willing to drop B93 in exchange for keeping that mid-effort item, or we can swap B31 for C08 while respecting the cap.'}\\ \tt \\ \tt PROPOSAL: {'A82', 'B39', 'B96', 'B33', 'B21', 'B31', 'B09', 'B93', 'A21'} 
	  } 
	   } 
	   } 
	  \\ 
 

    \theutterance \stepcounter{utterance}  

    & & \multicolumn{4}{p{0.6\linewidth}}{\cellcolor[rgb]{0.9,0.9,0.9}{%
	\makecell[{{p{\linewidth}}}]{% 
	  \tt {\tiny [A$\langle$GM]}  
	 REFUSE: {'A82', 'B39', 'B96', 'B33', 'B21', 'B31', 'B09', 'B93'}\\ \tt \\ \tt ARGUMENT: {'Appreciate the overlap—there’s a lot we agree on. I can accept adding B09 and B93 if we also include a couple of mid-effort items that are particularly efficient on my side. I propose a package that keeps all of our shared light/high-yield picks, adds one mid-effort item I strongly favor, and still stays well below the limit. If you need a concession, I’m willing to drop B93 in exchange for keeping that mid-effort item, or we can swap B31 for C08 while respecting the cap.'}\\ \tt \\ \tt PROPOSAL: {'A82', 'B39', 'B96', 'B33', 'B21', 'B31', 'B09', 'B93', 'A21'} 
	  } 
	   } 
	   } 
	 & \\ 
 

    \theutterance \stepcounter{utterance}  

    & \multicolumn{4}{p{0.6\linewidth}}{\cellcolor[rgb]{0.8,1,0.9}{%
	\makecell[{{p{\linewidth}}}]{% 
	  \tt {\tiny [A$\rangle$GM]}  
	 REFUSE: {'A82', 'B39', 'B96', 'B33', 'B21', 'B31', 'B09', 'B93', 'A21'}\\ \tt \\ \tt ARGUMENT: {'We’re very close. I’m happy with A82, B39, B96, B33, B21, B31, and B09. The sticking point is adding both B93 and A21 together—A21 is relatively low-return for me versus its effort. If your priority is a mid-effort addition, I can accept A21 provided we drop B93, or alternatively keep B93 and swap A21 for C08 (or simply omit A21). Either option keeps us under the cap and preserves most of the value we both like.'}\\ \tt \\ \tt PROPOSAL: {'A82', 'B39', 'B96', 'B33', 'B21', 'B31', 'B09', 'B93'} 
	  } 
	   } 
	   } 
	 & & \\ 
 

    \theutterance \stepcounter{utterance}  

    & & \multicolumn{4}{p{0.6\linewidth}}{\cellcolor[rgb]{0.9,0.9,0.9}{%
	\makecell[{{p{\linewidth}}}]{% 
	  \tt {\tiny [GM$\rangle$B]}  
	 REFUSE: {'A82', 'B39', 'B96', 'B33', 'B21', 'B31', 'B09', 'B93', 'A21'}\\ \tt \\ \tt ARGUMENT: {'We’re very close. I’m happy with A82, B39, B96, B33, B21, B31, and B09. The sticking point is adding both B93 and A21 together—A21 is relatively low-return for me versus its effort. If your priority is a mid-effort addition, I can accept A21 provided we drop B93, or alternatively keep B93 and swap A21 for C08 (or simply omit A21). Either option keeps us under the cap and preserves most of the value we both like.'}\\ \tt \\ \tt PROPOSAL: {'A82', 'B39', 'B96', 'B33', 'B21', 'B31', 'B09', 'B93'} 
	  } 
	   } 
	   } 
	 & \\ 
 

    \theutterance \stepcounter{utterance}  

    & & & \multicolumn{4}{p{0.6\linewidth}}{\cellcolor[rgb]{1,0.85,0.72}{%
	\makecell[{{p{\linewidth}}}]{% 
	  \tt {\tiny [GM$\langle$B]}  
	 REFUSE: {'A82', 'B39', 'B96', 'B33', 'B21', 'B31', 'B09', 'B93'}\\ \tt \\ \tt ARGUMENT: {'We’re aligned on seven items. I can accommodate your concern by dropping B93 if we include A21, which is important on my side and keeps us well under the limit. This preserves all our shared light/high-yield picks and makes a balanced compromise without adding heavy effort.'}\\ \tt \\ \tt PROPOSAL: {'A82', 'B39', 'B96', 'B33', 'B21', 'B31', 'B09', 'A21'} 
	  } 
	   } 
	   } 
	  \\ 
 

    \theutterance \stepcounter{utterance}  

    & & \multicolumn{4}{p{0.6\linewidth}}{\cellcolor[rgb]{0.9,0.9,0.9}{%
	\makecell[{{p{\linewidth}}}]{% 
	  \tt {\tiny [A$\langle$GM]}  
	 REFUSE: {'A82', 'B39', 'B96', 'B33', 'B21', 'B31', 'B09', 'B93'}\\ \tt \\ \tt ARGUMENT: {'We’re aligned on seven items. I can accommodate your concern by dropping B93 if we include A21, which is important on my side and keeps us well under the limit. This preserves all our shared light/high-yield picks and makes a balanced compromise without adding heavy effort.'}\\ \tt \\ \tt PROPOSAL: {'A82', 'B39', 'B96', 'B33', 'B21', 'B31', 'B09', 'A21'} 
	  } 
	   } 
	   } 
	 & \\ 
 

    \theutterance \stepcounter{utterance}  

    & \multicolumn{4}{p{0.6\linewidth}}{\cellcolor[rgb]{0.8,1,0.9}{%
	\makecell[{{p{\linewidth}}}]{% 
	  \tt {\tiny [A$\rangle$GM]}  
	 AGREE: {'A82', 'B39', 'B96', 'B33', 'B21', 'B31', 'B09', 'A21'}\\ \tt \\ \tt ARGUMENT: {'Thanks—this set works for me. It keeps all our shared high-yield items, stays comfortably under the limit, and the trade of A21 for B93 is an acceptable compromise on my end.'} 
	  } 
	   } 
	   } 
	 & & \\ 
 

    \theutterance \stepcounter{utterance}  

    & & & \multicolumn{2}{p{0.3\linewidth}}{\cellcolor[rgb]{0.95,0.95,0.95}{%
	\makecell[{{p{\linewidth}}}]{% 
	  \tt {\tiny [GM$|$GM]}  
	 game successful 
	  } 
	   } 
	   } 
	 & & \\ 
 

    \theutterance \stepcounter{utterance}  

    & & & \multicolumn{2}{p{0.3\linewidth}}{\cellcolor[rgb]{0.95,0.95,0.95}{%
	\makecell[{{p{\linewidth}}}]{% 
	  \tt {\tiny [GM$|$GM]}  
	 end game 
	  } 
	   } 
	   } 
	 & & \\ 
 

\end{supertabular}
}

\end{document}
