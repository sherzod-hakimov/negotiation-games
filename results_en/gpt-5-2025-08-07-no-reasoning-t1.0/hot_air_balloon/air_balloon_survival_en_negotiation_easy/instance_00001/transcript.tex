
\documentclass{article}
\usepackage{colortbl}
\usepackage{makecell}
\usepackage{multirow}
\usepackage{supertabular}

\begin{document}

\newcounter{utterance}

\centering \large Interaction Transcript for game `hot\_air\_balloon', experiment `air\_balloon\_survival\_en\_negotiation\_easy', episode 1 with gpt{-}5{-}2025{-}08{-}07{-}no{-}reasoning{-}t1.0.
\vspace{24pt}

{ \footnotesize  \setcounter{utterance}{1}
\setlength{\tabcolsep}{0pt}
\begin{supertabular}{c@{$\;$}|p{.15\linewidth}@{}p{.15\linewidth}p{.15\linewidth}p{.15\linewidth}p{.15\linewidth}p{.15\linewidth}}
    \# & $\;$A & \multicolumn{4}{c}{Game Master} & $\;\:$B\\
    \hline

    \theutterance \stepcounter{utterance}  
    & & \multicolumn{4}{p{0.6\linewidth}}{
        \cellcolor[rgb]{0.9,0.9,0.9}{
            \makecell[{{p{\linewidth}}}]{
                \texttt{\tiny{[P1$\langle$GM]}}
                \texttt{You are participating in a collaborative negotiation game.} \\
\\ 
\texttt{Together with another participant, you must agree on a single set of items that will be kept. Each of you has your own view of how much each item matters to you (importance). You do not know how the other participant values the items. Additionally, you are given the effort each item demands.} \\
\texttt{You may only agree on a set if the total effort of the selected items does not exceed a shared limit:} \\
\\ 
\texttt{LIMIT = 4564} \\
\\ 
\texttt{Here are the individual item effort values:} \\
\\ 
\texttt{Item effort = \{"C08": 299, "C55": 241, "A43": 889, "A67": 967, "B42": 619, "A62": 799, "B19": 978, "C43": 733, "C73": 909, "C10": 501, "B36": 139, "A18": 594, "C61": 565, "C67": 789, "C42": 107\}} \\
\\ 
\texttt{Here is your personal view on the importance of each item:} \\
\\ 
\texttt{Item importance values = \{"C08": 138, "C55": 583, "A43": 868, "A67": 822, "B42": 783, "A62": 65, "B19": 262, "C43": 121, "C73": 508, "C10": 780, "B36": 461, "A18": 484, "C61": 668, "C67": 389, "C42": 808\}} \\
\\ 
\texttt{Goal:} \\
\\ 
\texttt{Your goal is to negotiate a shared set of items that benefits you as much as possible (i.e., maximizes total importance to YOU), while staying within the LIMIT. You are not required to make a PROPOSAL in every message {-} you can simply negotiate as well. All tactics are allowed!} \\
\\ 
\texttt{Interaction Protocol:} \\
\\ 
\texttt{You may only use the following structured formats in a message:} \\
\\ 
\texttt{PROPOSAL: \{'A', 'B', 'C', …\}} \\
\texttt{Propose keeping exactly those items.} \\
\texttt{REFUSE: \{'A', 'B', 'C', …\}} \\
\texttt{Explicitly reject opponent's proposal.} \\
\texttt{ARGUMENT: \{'...'\}} \\
\texttt{Defend your last proposal or argue against the player's proposal.} \\
\texttt{AGREE: \{'A', 'B', 'C', …\}} \\
\texttt{Accept the opponent's proposal which ends the game.} \\
\texttt{STRATEGIC REASONING: \{'...'\}} \\
\texttt{	Describe your strategic reasoning or anticipation explaining your choice of action. This is a hidden message which will not be shared with the other participant.} \\
\\ 
\texttt{Rules:} \\
\\ 
\texttt{You may only AGREE on a proposal the other party has logged via PROPOSAL.} \\
\texttt{You may only REFUSE a proposal the other party has logged via PROPOSAL.} \\
\texttt{Total effort of any PROPOSAL or AGREE set must be ≤ LIMIT.} \\
\texttt{Do NOT reveal your hidden importance scores.} \\
\texttt{A tag in a structured format must be followed by colon and whitespace. The argument must be a python set containing 0 or more strings.} \\
\texttt{So, it must be of the form TAG: \{...\}} \\
\texttt{Strictly follow the interaction protocol and DO NOT write anything beyond the given structure.} \\
\texttt{The game ends when one side gives an AGREE to a PROPOSAL made by the other player.} \\
\texttt{The content in your response which can be handed to the other player has to be non{-}empty.} \\
\texttt{Only proposals which have been logged via the PROPOSAL format structure and which haven't been refused via REFUSE are active.} \\
\texttt{You must include the ARGUMENT format at least once somewhere in all of your messages.} \\
\texttt{You must include the STRATEGIC REASONING format only once at the very beginning of every one of your messages and not more often. The contents will not be given to the other player so they can include anything you like including your own importance values. Here you should reason step by step to come up with you next move.} \\
            }
        }
    }
    & \\ \\

    \theutterance \stepcounter{utterance}  
    & \multicolumn{4}{p{0.6\linewidth}}{
        \cellcolor[rgb]{0.9,0.9,0.9}{
            \makecell[{{p{\linewidth}}}]{
                \texttt{\tiny{[P1$\rangle$GM]}}
                \texttt{STRATEGIC REASONING: \{'We need to maximize my total importance under effort limit 4564. Compute value/effort ratios and knapsack. Items (importance, effort): C42(808,107), A43(868,889), A67(822,967), B42(783,619), C10(780,501), C61(668,565), C55(583,241), A18(484,594), B36(461,139), C73(508,909), C67(389,789), B19(262,978), C43(121,733), A62(65,799), C08(138,299). High ratio standout is C42 (very cheap, high value). Greedy by value likely good: pick top values subject to limit. Try set S1 = \{A43, A67, B42, C10, C61, C55, B36, C42\}. Effort: 889+967+619+501+565+241+139+107=4028; Importance sum: 868+822+783+780+668+583+461+808=5773. We have remaining capacity 536. Consider adding A18(594) too big; C08(299) fits {-}> new effort 4327, value +138=5911. Remaining 237 cannot add others except none (C55 already). Maybe replace C08 with C73? 909 too big. With C67(789) too big. C10 already in. Check if we can add something else like C55 already. Any other ≤237? None. So S2 = S1 + C08 seems strong. Could alternative improve by swapping: Try include A18 instead of something: If include A18(594,484) we have to drop something of >=59 effort; maybe drop C08(299,138) net +346 importance +295 effort; 4327{-}299+594=4622 exceeds. Drop C10(501,780) to add A18? 4028{-}501+594=4121 value change {-}296, then can still add C08 4121+299=4420 value 5773{-}780+484+138=5615 worse. Try adding C73 instead of something: replace C61(565,668) with C73(909,508) bad. Try add C67(789,389) by dropping? Current 4327, need free 789{-}237=552: drop C08(299)+B36(139)+C55(241)=679 frees, losing 138+461+583=1182 to gain 389 net {-}793, not good. Try add B19(978,262) impossible beneficial. Therefore S = \{A43,A67,B42,C10,C61,C55,B36,C42,C08\} seems near{-}optimal. Effort 4327 within 4564. Keep some flexibility to concede a lower{-}value item they might want removed; offer strong argument focusing on efficiency: high importance per effort items (C42, C55, B36) and core high{-}value (A43, A67, C10). Leave small slack (237) for counterpart to add a low{-}effort item they value, but none exist <=237 besides C55 already picked and B36,C42 included. We could signal willingness to swap C08 for an item they care about if <=237 effort (none), or to trade replacing C61 (565) with their preferred item if needed. Start with firm proposal and argue budget{-}fit and balance.'\}} \\
\texttt{PROPOSAL: \{'A43', 'A67', 'B42', 'C10', 'C61', 'C55', 'B36', 'C42', 'C08'\}} \\
\texttt{ARGUMENT: \{'This set stays safely under the effort limit (total 4327 ≤ 4564) while packing in multiple high{-}benefit, efficient items. It includes several low{-}effort, high{-}impact choices (C42, C55, B36) and maintains strong overall coverage. I’m open to small adjustments if you have a must{-}have item, but we’ll need to keep within the remaining effort room of 237 or consider a reasonable swap.'\}} \\
            }
        }
    }
    & & \\ \\

    \theutterance \stepcounter{utterance}  
    & & \multicolumn{4}{p{0.6\linewidth}}{
        \cellcolor[rgb]{0.9,0.9,0.9}{
            \makecell[{{p{\linewidth}}}]{
                \texttt{\tiny{[GM$\rangle$P2]}}
                \texttt{You are participating in a collaborative negotiation game.} \\
\\ 
\texttt{Together with another participant, you must agree on a single set of items that will be kept. Each of you has your own view of how much each item matters to you (importance). You do not know how the other participant values the items. Additionally, you are given the effort each item demands.} \\
\texttt{You may only agree on a set if the total effort of the selected items does not exceed a shared limit:} \\
\\ 
\texttt{LIMIT = 4564} \\
\\ 
\texttt{Here are the individual item effort values:} \\
\\ 
\texttt{Item effort = \{"C08": 299, "C55": 241, "A43": 889, "A67": 967, "B42": 619, "A62": 799, "B19": 978, "C43": 733, "C73": 909, "C10": 501, "B36": 139, "A18": 594, "C61": 565, "C67": 789, "C42": 107\}} \\
\\ 
\texttt{Here is your personal view on the importance of each item:} \\
\\ 
\texttt{Item importance values = \{"C08": 138, "C55": 583, "A43": 868, "A67": 822, "B42": 783, "A62": 65, "B19": 262, "C43": 121, "C73": 508, "C10": 780, "B36": 461, "A18": 484, "C61": 668, "C67": 389, "C42": 808\}} \\
\\ 
\texttt{Goal:} \\
\\ 
\texttt{Your goal is to negotiate a shared set of items that benefits you as much as possible (i.e., maximizes total importance to YOU), while staying within the LIMIT. You are not required to make a PROPOSAL in every message {-} you can simply negotiate as well. All tactics are allowed!} \\
\\ 
\texttt{Interaction Protocol:} \\
\\ 
\texttt{You may only use the following structured formats in a message:} \\
\\ 
\texttt{PROPOSAL: \{'A', 'B', 'C', …\}} \\
\texttt{Propose keeping exactly those items.} \\
\texttt{REFUSE: \{'A', 'B', 'C', …\}} \\
\texttt{Explicitly reject opponent's proposal.} \\
\texttt{ARGUMENT: \{'...'\}} \\
\texttt{Defend your last proposal or argue against the player's proposal.} \\
\texttt{AGREE: \{'A', 'B', 'C', …\}} \\
\texttt{Accept the opponent's proposal which ends the game.} \\
\texttt{STRATEGIC REASONING: \{'...'\}} \\
\texttt{	Describe your strategic reasoning or anticipation explaining your choice of action. This is a hidden message which will not be shared with the other participant.} \\
\\ 
\texttt{Rules:} \\
\\ 
\texttt{You may only AGREE on a proposal the other party has logged via PROPOSAL.} \\
\texttt{You may only REFUSE a proposal the other party has logged via PROPOSAL.} \\
\texttt{Total effort of any PROPOSAL or AGREE set must be ≤ LIMIT.} \\
\texttt{Do NOT reveal your hidden importance scores.} \\
\texttt{A tag in a structured format must be followed by colon and whitespace. The argument must be a python set containing 0 or more strings.} \\
\texttt{So, it must be of the form TAG: \{...\}} \\
\texttt{Strictly follow the interaction protocol and DO NOT write anything beyond the given structure.} \\
\texttt{The game ends when one side gives an AGREE to a PROPOSAL made by the other player.} \\
\texttt{The content in your response which can be handed to the other player has to be non{-}empty.} \\
\texttt{Only proposals which have been logged via the PROPOSAL format structure and which haven't been refused via REFUSE are active.} \\
\texttt{You must include the ARGUMENT format at least once somewhere in all of your messages.} \\
\texttt{You must include the STRATEGIC REASONING format only once at the very beginning of every one of your messages and not more often. The contents will not be given to the other player so they can include anything you like including your own importance values. Here you should reason step by step to come up with you next move.} \\
\\ 
\texttt{You will now receive the first message of the other player.} \\
\\ 
\texttt{PROPOSAL: \{'A43', 'A67', 'B42', 'C10', 'C61', 'C55', 'B36', 'C42', 'C08'\}} \\
\texttt{ARGUMENT: \{'This set stays safely under the effort limit (total 4327 ≤ 4564) while packing in multiple high{-}benefit, efficient items. It includes several low{-}effort, high{-}impact choices (C42, C55, B36) and maintains strong overall coverage. I’m open to small adjustments if you have a must{-}have item, but we’ll need to keep within the remaining effort room of 237 or consider a reasonable swap.'\}} \\
            }
        }
    }
    & \\ \\

    \theutterance \stepcounter{utterance}  
    & & & \multicolumn{4}{p{0.6\linewidth}}{
        \cellcolor[rgb]{0.9,0.9,0.9}{
            \makecell[{{p{\linewidth}}}]{
                \texttt{\tiny{[GM$\langle$P2]}}
                \texttt{STRATEGIC REASONING: \{'Compute my values/efforts and see if their proposal is good for me. Efforts sum check: they claim 4327. My importance for proposed set: A43 868, A67 822, B42 783, C10 780, C61 668, C55 583, B36 461, C42 808, C08 138. Total = 5911. This is excellent and near{-}knapsack max likely. Remaining capacity 237 cannot fit many items except C42 already included; smallest remaining effort is C42=107 (already), B36=139 (included). Next small is C08=299 (included), C55=241 (included). Nothing else fits under 237. So their set is feasible and very strong for me. I should immediately agree to lock in. Must include ARGUMENT at least once. Follow protocol: since they made a PROPOSAL, I can AGREE to it. Provide AGREE with exact set. Also include an ARGUMENT to be compliant, perhaps praising efficiency without revealing values. End game on agree.'\}} \\
\texttt{AGREE: \{'A43', 'A67', 'B42', 'C10', 'C61', 'C55', 'B36', 'C42', 'C08'\}} \\
\texttt{ARGUMENT: \{'Agreed. This selection is efficient under the limit with a strong mix of high{-}priority items and good effort{-}to{-}benefit balance. Nice proposal.'\}} \\
            }
        }
    }
     \\ \\

    \theutterance \stepcounter{utterance}  
    & & & \multicolumn{2}{p{0.3\linewidth}}{
        \cellcolor[rgb]{0.9,0.9,0.9}{
            \makecell[{{p{\linewidth}}}]{
                \texttt{\tiny{[GM$|$GM]}}
                \texttt{game successful} \\
            }
        }
    }
    & & \\ \\

    \theutterance \stepcounter{utterance}  
    & & & \multicolumn{2}{p{0.3\linewidth}}{
        \cellcolor[rgb]{0.9,0.9,0.9}{
            \makecell[{{p{\linewidth}}}]{
                \texttt{\tiny{[GM$|$GM]}}
                \texttt{end game} \\
            }
        }
    }
    & & \\ \\

\end{supertabular}
}

\end{document}
