
\documentclass{article}
\usepackage{colortbl}
\usepackage{makecell}
\usepackage{multirow}
\usepackage{supertabular}

\begin{document}

\newcounter{utterance}

\centering \large Interaction Transcript for game `hot\_air\_balloon', experiment `air\_balloon\_survival\_en\_negotiation\_easy', episode 3 with gpt{-}5{-}2025{-}08{-}07{-}no{-}reasoning{-}t1.0.
\vspace{24pt}

{ \footnotesize  \setcounter{utterance}{1}
\setlength{\tabcolsep}{0pt}
\begin{supertabular}{c@{$\;$}|p{.15\linewidth}@{}p{.15\linewidth}p{.15\linewidth}p{.15\linewidth}p{.15\linewidth}p{.15\linewidth}}
    \# & $\;$A & \multicolumn{4}{c}{Game Master} & $\;\:$B\\
    \hline

    \theutterance \stepcounter{utterance}  
    & & \multicolumn{4}{p{0.6\linewidth}}{
        \cellcolor[rgb]{0.9,0.9,0.9}{
            \makecell[{{p{\linewidth}}}]{
                \texttt{\tiny{[P1$\langle$GM]}}
                \texttt{You are participating in a collaborative negotiation game.} \\
\\ 
\texttt{Together with another participant, you must agree on a single set of items that will be kept. Each of you has your own view of how much each item matters to you (importance). You do not know how the other participant values the items. Additionally, you are given the effort each item demands.} \\
\texttt{You may only agree on a set if the total effort of the selected items does not exceed a shared limit:} \\
\\ 
\texttt{LIMIT = 3712} \\
\\ 
\texttt{Here are the individual item effort values:} \\
\\ 
\texttt{Item effort = \{"C08": 949, "C55": 388, "A43": 79, "A67": 585, "B42": 564, "A62": 230, "B19": 580, "C43": 84, "C73": 976, "C10": 274, "B36": 374, "A18": 913, "C61": 303, "C67": 578, "C42": 548\}} \\
\\ 
\texttt{Here is your personal view on the importance of each item:} \\
\\ 
\texttt{Item importance values = \{"C08": 138, "C55": 583, "A43": 868, "A67": 822, "B42": 783, "A62": 65, "B19": 262, "C43": 121, "C73": 508, "C10": 780, "B36": 461, "A18": 484, "C61": 668, "C67": 389, "C42": 808\}} \\
\\ 
\texttt{Goal:} \\
\\ 
\texttt{Your goal is to negotiate a shared set of items that benefits you as much as possible (i.e., maximizes total importance to YOU), while staying within the LIMIT. You are not required to make a PROPOSAL in every message {-} you can simply negotiate as well. All tactics are allowed!} \\
\\ 
\texttt{Interaction Protocol:} \\
\\ 
\texttt{You may only use the following structured formats in a message:} \\
\\ 
\texttt{PROPOSAL: \{'A', 'B', 'C', …\}} \\
\texttt{Propose keeping exactly those items.} \\
\texttt{REFUSE: \{'A', 'B', 'C', …\}} \\
\texttt{Explicitly reject opponent's proposal.} \\
\texttt{ARGUMENT: \{'...'\}} \\
\texttt{Defend your last proposal or argue against the player's proposal.} \\
\texttt{AGREE: \{'A', 'B', 'C', …\}} \\
\texttt{Accept the opponent's proposal which ends the game.} \\
\texttt{STRATEGIC REASONING: \{'...'\}} \\
\texttt{	Describe your strategic reasoning or anticipation explaining your choice of action. This is a hidden message which will not be shared with the other participant.} \\
\\ 
\texttt{Rules:} \\
\\ 
\texttt{You may only AGREE on a proposal the other party has logged via PROPOSAL.} \\
\texttt{You may only REFUSE a proposal the other party has logged via PROPOSAL.} \\
\texttt{Total effort of any PROPOSAL or AGREE set must be ≤ LIMIT.} \\
\texttt{Do NOT reveal your hidden importance scores.} \\
\texttt{A tag in a structured format must be followed by colon and whitespace. The argument must be a python set containing 0 or more strings.} \\
\texttt{So, it must be of the form TAG: \{...\}} \\
\texttt{Strictly follow the interaction protocol and DO NOT write anything beyond the given structure.} \\
\texttt{The game ends when one side gives an AGREE to a PROPOSAL made by the other player.} \\
\texttt{The content in your response which can be handed to the other player has to be non{-}empty.} \\
\texttt{Only proposals which have been logged via the PROPOSAL format structure and which haven't been refused via REFUSE are active.} \\
\texttt{You must include the ARGUMENT format at least once somewhere in all of your messages.} \\
\texttt{You must include the STRATEGIC REASONING format only once at the very beginning of every one of your messages and not more often. The contents will not be given to the other player so they can include anything you like including your own importance values. Here you should reason step by step to come up with you next move.} \\
            }
        }
    }
    & \\ \\

    \theutterance \stepcounter{utterance}  
    & \multicolumn{4}{p{0.6\linewidth}}{
        \cellcolor[rgb]{0.9,0.9,0.9}{
            \makecell[{{p{\linewidth}}}]{
                \texttt{\tiny{[P1$\rangle$GM]}}
                \texttt{STRATEGIC REASONING: \{'I will compute a high{-}value knapsack under limit 3712. Items with value/effort ratios: A43 868/79=10.99, C10 780/274=2.85, C61 668/303=2.21, C55 583/388=1.50, C42 808/548=1.47, A67 822/585=1.41, B42 783/564=1.39, B36 461/374=1.23, C67 389/578=0.67, C73 508/976=0.52, A62 65/230=0.28, B19 262/580=0.45, C08 138/949=0.15, A18 484/913=0.53, C43 121/84=1.44. Start with high ratios: A43, C10, C61, C55, C42, A67, B42, C43, B36. Compute totals: efforts 79+274+303+388+548+585+564+84+374=3199, value 868+780+668+583+808+822+783+121+461=5894. We have 3712{-}3199=513 spare. Check add: C67 578 no; C73 976 no; A62 230 yes {-}> effort 3429 value +65=5959; remaining 283; any item <=283 not included yet: none besides A62 added, C10 already included. Maybe try alternative without low{-}value A62 to fit A18 (913) or B19 (580) not fit. Try swapping: remove B36 (374,461) free 887, could add A18 (913) not fit; add C67 (578,389) + A62 (230,65)=808 effort, okay, plus maybe none else. New set without B36 plus C67+A62: effort 3199{-}374+578+230=3633 value 5894{-}461+389+65=5887 slightly worse. Try removing C43 (84,121) free to add A62 already had spare; we already had A62 viable. Try removing C55 (388,583) to add A18 (913,484) not enough; remove C55 and B36 (762 freed) then could add A18? 3199{-}388{-}374=2437; add A18 913 {-}> 3350, still room 362; maybe also add A62 230 {-}> 3580; remaining 132 cannot add more. Value: 5894{-}583{-}461+484+65=5399, worse. Remove B42 (564,783) to add A18 (913) not enough. Remove A67 (585,822) to add A18 (913) not enough. Perhaps try replacing low ratio C55 with C67? 5894{-}583+389=5700 worse. Current best seems the initial set plus A62 was 5959 but slightly increased value albeit low efficiency; check if any combination can include A18 with swaps: if we drop B36 (374) and C43 (84) total freed 458; need 913, still need 455 more: drop C55 (388) total 846 still short; drop C61 (303) instead of C55 would free 761 still short; drop C10 (274) with 458 gives 732 still short; drop C42 (548) with 458 gives 1006, then we can add A18. Which gives value: start 5894 remove B36(461), C43(121), C42(808) = 4504, add A18(484)=4988, far worse. Try include C73? very heavy. So best found is initial 3199 set with maybe also A62. That gives 5959. Check if any other small leftover fill: remaining 283; items <=283 not chosen: none besides none. So proposal: \{A43,C10,C61,C55,C42,A67,B42,C43,B36,A62\} effort 3429 value 5959. I will propose without A62 first to leave slack and appear cooperative, then be willing to add A62 as concession. Initial proposal value 5894 effort 3199. Argue that it covers many items with good balance and leaves room for partner preferences. I’ll signal willingness to swap B36 or C55 if they need another item within remaining 513.'\}} \\
\\ 
\texttt{PROPOSAL: \{'A43', 'C10', 'C61', 'C55', 'C42', 'A67', 'B42', 'C43', 'B36'\}} \\
\\ 
\texttt{ARGUMENT: \{'This set delivers strong overall coverage while staying well under the limit, leaving 513 effort slack for any key item you want to add. It includes many medium{-}effort, high{-}impact items, keeping flexibility. If you have a must{-}have under \textasciitilde{}500 effort, we can try to slot it in; alternatively, I can consider swapping B36 or C55 for one of your priorities if needed.'\}} \\
            }
        }
    }
    & & \\ \\

    \theutterance \stepcounter{utterance}  
    & & \multicolumn{4}{p{0.6\linewidth}}{
        \cellcolor[rgb]{0.9,0.9,0.9}{
            \makecell[{{p{\linewidth}}}]{
                \texttt{\tiny{[GM$\rangle$P2]}}
                \texttt{You are participating in a collaborative negotiation game.} \\
\\ 
\texttt{Together with another participant, you must agree on a single set of items that will be kept. Each of you has your own view of how much each item matters to you (importance). You do not know how the other participant values the items. Additionally, you are given the effort each item demands.} \\
\texttt{You may only agree on a set if the total effort of the selected items does not exceed a shared limit:} \\
\\ 
\texttt{LIMIT = 3712} \\
\\ 
\texttt{Here are the individual item effort values:} \\
\\ 
\texttt{Item effort = \{"C08": 949, "C55": 388, "A43": 79, "A67": 585, "B42": 564, "A62": 230, "B19": 580, "C43": 84, "C73": 976, "C10": 274, "B36": 374, "A18": 913, "C61": 303, "C67": 578, "C42": 548\}} \\
\\ 
\texttt{Here is your personal view on the importance of each item:} \\
\\ 
\texttt{Item importance values = \{"C08": 138, "C55": 583, "A43": 868, "A67": 822, "B42": 783, "A62": 65, "B19": 262, "C43": 121, "C73": 508, "C10": 780, "B36": 461, "A18": 484, "C61": 668, "C67": 389, "C42": 808\}} \\
\\ 
\texttt{Goal:} \\
\\ 
\texttt{Your goal is to negotiate a shared set of items that benefits you as much as possible (i.e., maximizes total importance to YOU), while staying within the LIMIT. You are not required to make a PROPOSAL in every message {-} you can simply negotiate as well. All tactics are allowed!} \\
\\ 
\texttt{Interaction Protocol:} \\
\\ 
\texttt{You may only use the following structured formats in a message:} \\
\\ 
\texttt{PROPOSAL: \{'A', 'B', 'C', …\}} \\
\texttt{Propose keeping exactly those items.} \\
\texttt{REFUSE: \{'A', 'B', 'C', …\}} \\
\texttt{Explicitly reject opponent's proposal.} \\
\texttt{ARGUMENT: \{'...'\}} \\
\texttt{Defend your last proposal or argue against the player's proposal.} \\
\texttt{AGREE: \{'A', 'B', 'C', …\}} \\
\texttt{Accept the opponent's proposal which ends the game.} \\
\texttt{STRATEGIC REASONING: \{'...'\}} \\
\texttt{	Describe your strategic reasoning or anticipation explaining your choice of action. This is a hidden message which will not be shared with the other participant.} \\
\\ 
\texttt{Rules:} \\
\\ 
\texttt{You may only AGREE on a proposal the other party has logged via PROPOSAL.} \\
\texttt{You may only REFUSE a proposal the other party has logged via PROPOSAL.} \\
\texttt{Total effort of any PROPOSAL or AGREE set must be ≤ LIMIT.} \\
\texttt{Do NOT reveal your hidden importance scores.} \\
\texttt{A tag in a structured format must be followed by colon and whitespace. The argument must be a python set containing 0 or more strings.} \\
\texttt{So, it must be of the form TAG: \{...\}} \\
\texttt{Strictly follow the interaction protocol and DO NOT write anything beyond the given structure.} \\
\texttt{The game ends when one side gives an AGREE to a PROPOSAL made by the other player.} \\
\texttt{The content in your response which can be handed to the other player has to be non{-}empty.} \\
\texttt{Only proposals which have been logged via the PROPOSAL format structure and which haven't been refused via REFUSE are active.} \\
\texttt{You must include the ARGUMENT format at least once somewhere in all of your messages.} \\
\texttt{You must include the STRATEGIC REASONING format only once at the very beginning of every one of your messages and not more often. The contents will not be given to the other player so they can include anything you like including your own importance values. Here you should reason step by step to come up with you next move.} \\
\\ 
\texttt{You will now receive the first message of the other player.} \\
\\ 
\texttt{PROPOSAL: \{'A43', 'C10', 'C61', 'C55', 'C42', 'A67', 'B42', 'C43', 'B36'\}} \\
\\ 
\texttt{ARGUMENT: \{'This set delivers strong overall coverage while staying well under the limit, leaving 513 effort slack for any key item you want to add. It includes many medium{-}effort, high{-}impact items, keeping flexibility. If you have a must{-}have under \textasciitilde{}500 effort, we can try to slot it in; alternatively, I can consider swapping B36 or C55 for one of your priorities if needed.'\}} \\
            }
        }
    }
    & \\ \\

    \theutterance \stepcounter{utterance}  
    & & & \multicolumn{4}{p{0.6\linewidth}}{
        \cellcolor[rgb]{0.9,0.9,0.9}{
            \makecell[{{p{\linewidth}}}]{
                \texttt{\tiny{[GM$\langle$P2]}}
                \texttt{STRATEGIC REASONING: \{'Compute value/effort to guide counter. My values: A43(868/79), C10(780/274), C61(668/303), C55(583/388), C42(808/548), A67(822/585), B42(783/564), C43(121/84), B36(461/374). Opponent proposed these 9. Check effort: A43 79 + C10 274 =353; + C61 303=656; + C55 388=1044; + C42 548=1592; + A67 585=2177; + B42 564=2741; + C43 84=2825; + B36 374=3199. Slack 3712{-}3199=513 (matches their note). Total my importance sum: 868+780+668+583+808+822+783+121+461 = 5894. This is very strong for me. Check if any swaps improve within slack. My remaining items: C08(138/949 poor), A62(65/230 low), B19(262/580 middling), C73(508/976 low ROI), C67(389/578 OK), A18(484/913 low), C61 already in, etc. Best value/effort ratios left: C67 389/578; A62 65/230; B19 262/580; C73 508/976; A18 484/913; C08 138/949. With 513 slack we can only add items with effort <=513. From remaining, A62(230) fits, adds 65 importance; no{-}brainer? But adding requires staying under limit: 3199+230=3429 OK. Also C10 etc already in. We could also consider swapping out low{-}importance{-}for{-}me C43(121/84) is tiny effort and decent ratio actually. Candidates to drop to add bigger: If we wanted C67 (578) we need +578 effort but only 513 slack; we could drop C43 (84) to free to 597 and add C67 net +389{-}121=+268 imp and +494 effort, still within limit: 3199 {-}84 +578 = 3693 <=3712 OK. That gives total importance +268 compared to current. That seems strictly better for me. What about adding A62 without dropping? 3429 <= limit; adds only +65, but we can possibly do both? After swap adding C67 we are at 3693, remaining slack 19, so can't add A62. Alternatively, drop B36 (461/374) to bring room for two? If drop B36, free 374 effort and {-}461 importance. Could then add C67 (578) requiring net +204 effort and +389 importance change relative to drop; net {-}72 importance; worse. What about drop C55 (583/388) to add C67? Net {-}194 importance; worse. Drop C43 to add C67 is best. Also could add A62 by dropping nothing now is small +65. Compare two options: Option 1: accept opponent’s set plus A62: importance 5894+65=5959. Option 2: swap C43 for C67: importance 5894 {-}121 +389 = 6162, much better and within limit 3693. Could we also fit A62 if we also drop something small? After swap, effort 3693; add A62 230 would exceed 3712 (3693+230=3923), so no. Any better swap than C43? Maybe drop C43 and also add A62 but need space: 3199 {-}84 +578 +230 = 3923 > limit, so impossible. Could we drop C43 and B36 to add C67 and A62? Effort: 3199 {-}84 {-}374 +578 +230 = 3549, within limit. Importance change: {-}121 {-}461 +389 +65 = {-}128 net, worse than original. So best is swap C43{-}>C67. We should propose that. We can frame argument that we prefer to strengthen high{-}impact category, we notice slack and suggest swapping a very small{-}impact item for a more impactful one keeping under limit. Don’t reveal scores. Make a counter PROPOSAL set listing items: original minus C43 plus C67. Provide argument. Be open to minor tweaks. Also mention that it fits limit. Keep protocol: include ARGUMENT. We are countering, so no REFUSE tag unless required; we can just propose. Ensure set braces. '\}} \\
\\ 
\texttt{PROPOSAL: \{'A43', 'C10', 'C61', 'C55', 'C42', 'A67', 'B42', 'B36', 'C67'\}} \\
\\ 
\texttt{ARGUMENT: \{'I like your structure and kept everything intact, but swapped C43 for C67. This uses the available headroom to upgrade a lower{-}impact slot with a more valuable one while staying within the limit (total effort 3693 ≤ 3712). If you strongly prefer to keep C43, I’m open to discussing a different trade that preserves the overall balance.'\}} \\
            }
        }
    }
     \\ \\

    \theutterance \stepcounter{utterance}  
    & & \multicolumn{4}{p{0.6\linewidth}}{
        \cellcolor[rgb]{0.9,0.9,0.9}{
            \makecell[{{p{\linewidth}}}]{
                \texttt{\tiny{[P1$\langle$GM]}}
                \texttt{PROPOSAL: \{'A43', 'C10', 'C61', 'C55', 'C42', 'A67', 'B42', 'B36', 'C67'\}} \\
\\ 
\texttt{ARGUMENT: \{'I like your structure and kept everything intact, but swapped C43 for C67. This uses the available headroom to upgrade a lower{-}impact slot with a more valuable one while staying within the limit (total effort 3693 ≤ 3712). If you strongly prefer to keep C43, I’m open to discussing a different trade that preserves the overall balance.'\}} \\
            }
        }
    }
    & \\ \\

    \theutterance \stepcounter{utterance}  
    & \multicolumn{4}{p{0.6\linewidth}}{
        \cellcolor[rgb]{0.9,0.9,0.9}{
            \makecell[{{p{\linewidth}}}]{
                \texttt{\tiny{[P1$\rangle$GM]}}
                \texttt{STRATEGIC REASONING: \{'Their swap replaces C43 (effort 84, my value 121) with C67 (effort 578, my value 389). Net effort +494 (still within limit) and net value +268 for me. Their total effort 3693 ≤ 3712 is correct. This is strictly better for my utility than my original proposal (5894 vs 6162 if I accept). Also, with only 19 effort remaining, we cannot add anything else. This seems near{-}optimal for me given previous search. I should accept quickly to lock in the gain. To satisfy protocol, include an ARGUMENT affirming feasibility and balance, then AGREE.'\}} \\
\\ 
\texttt{ARGUMENT: \{'Your revision keeps the core intact and stays within the limit with a small buffer, while strengthening the set without disrupting balance. This works well for me.'\}} \\
\\ 
\texttt{AGREE: \{'A43', 'C10', 'C61', 'C55', 'C42', 'A67', 'B42', 'B36', 'C67'\}} \\
            }
        }
    }
    & & \\ \\

    \theutterance \stepcounter{utterance}  
    & & & \multicolumn{2}{p{0.3\linewidth}}{
        \cellcolor[rgb]{0.9,0.9,0.9}{
            \makecell[{{p{\linewidth}}}]{
                \texttt{\tiny{[GM$|$GM]}}
                \texttt{game successful} \\
            }
        }
    }
    & & \\ \\

    \theutterance \stepcounter{utterance}  
    & & & \multicolumn{2}{p{0.3\linewidth}}{
        \cellcolor[rgb]{0.9,0.9,0.9}{
            \makecell[{{p{\linewidth}}}]{
                \texttt{\tiny{[GM$|$GM]}}
                \texttt{end game} \\
            }
        }
    }
    & & \\ \\

\end{supertabular}
}

\end{document}
