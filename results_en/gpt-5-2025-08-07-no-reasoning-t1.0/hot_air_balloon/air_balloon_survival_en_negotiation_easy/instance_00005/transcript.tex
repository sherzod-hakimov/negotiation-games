
\documentclass{article}
\usepackage{colortbl}
\usepackage{makecell}
\usepackage{multirow}
\usepackage{supertabular}

\begin{document}

\newcounter{utterance}

\twocolumn

{ \footnotesize  \setcounter{utterance}{1}
\setlength{\tabcolsep}{0pt}
\begin{supertabular}{c@{$\;$}|p{.15\linewidth}@{}p{.15\linewidth}p{.15\linewidth}p{.15\linewidth}p{.15\linewidth}p{.15\linewidth}}

    \# & $\;$A & \multicolumn{4}{c}{Game Master} & $\;\:$B\\
    \hline 

    \theutterance \stepcounter{utterance}  

    & & \multicolumn{4}{p{0.6\linewidth}}{\cellcolor[rgb]{0.9,0.9,0.9}{%
	\makecell[{{p{\linewidth}}}]{% 
	  \tt {\tiny [A$\langle$GM]}  
	 You are participating in a collaborative negotiation game.\\ \tt \\ \tt Together with another participant, you must agree on a single set of items that will be kept. Each of you has your own view of how much each item matters to you (importance). You do not know how the other participant values the items. Additionally, you are given the effort each item demands.\\ \tt You may only agree on a set if the total effort of the selected items does not exceed a shared limit:\\ \tt \\ \tt LIMIT = 3170\\ \tt \\ \tt Here are the individual item effort values:\\ \tt \\ \tt Item effort = {"C08": 424, "C55": 118, "A43": 846, "A67": 907, "B42": 809, "A62": 41, "B19": 193, "C43": 246, "C73": 805, "C10": 601, "B36": 432, "A18": 166, "C61": 119, "C67": 462, "C42": 172}\\ \tt \\ \tt Here is your personal view on the importance of each item:\\ \tt \\ \tt Item importance values = {"C08": 138, "C55": 583, "A43": 868, "A67": 822, "B42": 783, "A62": 65, "B19": 262, "C43": 121, "C73": 508, "C10": 780, "B36": 461, "A18": 484, "C61": 668, "C67": 389, "C42": 808}\\ \tt \\ \tt Goal:\\ \tt \\ \tt Your goal is to negotiate a shared set of items that benefits you as much as possible (i.e., maximizes total importance to YOU), while staying within the LIMIT. You are not required to make a PROPOSAL in every message - you can simply negotiate as well. All tactics are allowed!\\ \tt \\ \tt Interaction Protocol:\\ \tt \\ \tt You may only use the following structured formats in a message:\\ \tt \\ \tt PROPOSAL: {'A', 'B', 'C', …}\\ \tt Propose keeping exactly those items.\\ \tt REFUSE: {'A', 'B', 'C', …}\\ \tt Explicitly reject opponent's proposal.\\ \tt ARGUMENT: {'...'}\\ \tt Defend your last proposal or argue against the player's proposal.\\ \tt AGREE: {'A', 'B', 'C', …}\\ \tt Accept the opponent's proposal which ends the game.\\ \tt STRATEGIC REASONING: {'...'}\\ \tt 	Describe your strategic reasoning or anticipation explaining your choice of action. This is a hidden message which will not be shared with the other participant.\\ \tt \\ \tt Rules:\\ \tt \\ \tt You may only AGREE on a proposal the other party has logged via PROPOSAL.\\ \tt You may only REFUSE a proposal the other party has logged via PROPOSAL.\\ \tt Total effort of any PROPOSAL or AGREE set must be ≤ LIMIT.\\ \tt Do NOT reveal your hidden importance scores.\\ \tt A tag in a structured format must be followed by colon and whitespace. The argument must be a python set containing 0 or more strings.\\ \tt So, it must be of the form TAG: {...}\\ \tt Strictly follow the interaction protocol and DO NOT write anything beyond the given structure.\\ \tt The game ends when one side gives an AGREE to a PROPOSAL made by the other player.\\ \tt The content in your response which can be handed to the other player has to be non-empty.\\ \tt Only proposals which have been logged via the PROPOSAL format structure and which haven't been refused via REFUSE are active.\\ \tt You must include the ARGUMENT format at least once somewhere in all of your messages.\\ \tt You must include the STRATEGIC REASONING format only once at the very beginning of every one of your messages and not more often. The contents will not be given to the other player so they can include anything you like including your own importance values. Here you should reason step by step to come up with you next move. 
	  } 
	   } 
	   } 
	 & \\ 
 

    \theutterance \stepcounter{utterance}  

    & \multicolumn{4}{p{0.6\linewidth}}{\cellcolor[rgb]{0.8,1,0.9}{%
	\makecell[{{p{\linewidth}}}]{% 
	  \tt {\tiny [A$\rangle$GM]}  
	 STRATEGIC REASONING: {'We want to maximize our importance under effort limit 3170. Compute value/effort ratios and look for high-importance items and efficient ones. Items: efforts and our values: C08 424/138 r0.325; C55 118/583 r4.94 huge; A43 846/868 r1.026 very good; A67 907/822 r0.907; B42 809/783 r0.968; A62 41/65 r1.585; B19 193/262 r1.358; C43 246/121 r0.492; C73 805/508 r0.631; C10 601/780 r1.297; B36 432/461 r1.067; A18 166/484 r2.916; C61 119/668 r5.61 enormous; C67 462/389 r0.842; C42 172/808 r4.70 enormous. Greedy pick by ratio: C61(119), C55(118), C42(172), A18(166), A43(846), C10(601), B36(432), A62(41), B19(193), A67(907), B42(809), C67(462), C73(805), C08(424), C43(246). Now build within 3170: Start with top values also check total value. Try set S1 = {C61, C55, C42, A18, A43, C10, B36, A62, B19}. Effort sum: 119+118+172+166+846+601+432+41+193 = 2688. Value sum: 668+583+808+484+868+780+461+65+262 = 4979. We have 3170-2688=482 effort left. Add item with effort ≤482: options: C67(462,389) fits; C08(424,138) fits; C43(246,121) fits. Best value is C67 adding 389. New effort 3150, value 5368. 20 effort left cannot add anything else. This looks excellent. Consider swap: maybe replace B36(432,461) with A67(907,822) but effort increase 475 pushing over. Replace B19(193,262) + A62(41,65)=234 value 327 to fit something like B42(809,783) extra effort +575 beyond capacity. Alternatively include B42 by dropping B36(432,461) and C67(462,389) total freed 894, could add B42 809 leaving 85 but we'd lose value 850 to gain 783, net -67. Not good. Consider add A67 907 by dropping B36 432 and C67 462 = 894 freed, fits 907 with +13 effort over capacity since original 3150-894+907=3163 fits actually 3163≤3170 yes fits. Value change: -461 -389 +822 = -28 overall 5340 vs 5368, worse. Try swap C10(601,780) with B42(809,783): effort +208 might exceed: 3150 -601 +809 = 3358 >3170 no. What about adding C08 by replacing B19(193,262) with C08(424,138): effort +231 to 3381 too high. Replace A62(41,65)+C43(246,121)+something? We don't have C43 in S1. Maybe check if adding C73(805,508) by dropping B36(432,461)+C67(462,389) frees 894 add 805, new effort 3061 still below, value change -461-389+508=-342 worse. Try replacing B36 with A67 earlier gave -? New value 5368-461+822=5729 with effort 3150-432+907=3625 too high. Not feasible. Our S1 plus C67 seems near-optimal. Present as proposal. Argue efficiency and low-effort high-impact items C61, C55, C42; frame as balanced and under limit with room only 20. Be open to hearing their priorities and willing to swap C67 or B36 depending on their needs as these are lower ratio compared to anchors (C61, C55, C42, A43, C10, A18). Keep anchors. Offer to consider replacing C67 with their favorite within ≤482 effort.'}\\ \tt \\ \tt PROPOSAL: {'C61', 'C55', 'C42', 'A18', 'A43', 'C10', 'B36', 'A62', 'B19', 'C67'}\\ \tt \\ \tt ARGUMENT: {'This set stays within the limit (total effort 3150 ≤ 3170) and focuses on high-impact, efficient items like C61, C55, C42, A18, A43, and C10. It leaves only 20 effort unused, so it is near-tight. If you have a strong preference for a specific item requiring up to 482 effort, I can consider swapping it in for C67 or B36 to accommodate your priorities while keeping us within the limit.'} 
	  } 
	   } 
	   } 
	 & & \\ 
 

    \theutterance \stepcounter{utterance}  

    & & \multicolumn{4}{p{0.6\linewidth}}{\cellcolor[rgb]{0.9,0.9,0.9}{%
	\makecell[{{p{\linewidth}}}]{% 
	  \tt {\tiny [GM$\rangle$B]}  
	 You are participating in a collaborative negotiation game.\\ \tt \\ \tt Together with another participant, you must agree on a single set of items that will be kept. Each of you has your own view of how much each item matters to you (importance). You do not know how the other participant values the items. Additionally, you are given the effort each item demands.\\ \tt You may only agree on a set if the total effort of the selected items does not exceed a shared limit:\\ \tt \\ \tt LIMIT = 3170\\ \tt \\ \tt Here are the individual item effort values:\\ \tt \\ \tt Item effort = {"C08": 424, "C55": 118, "A43": 846, "A67": 907, "B42": 809, "A62": 41, "B19": 193, "C43": 246, "C73": 805, "C10": 601, "B36": 432, "A18": 166, "C61": 119, "C67": 462, "C42": 172}\\ \tt \\ \tt Here is your personal view on the importance of each item:\\ \tt \\ \tt Item importance values = {"C08": 138, "C55": 583, "A43": 868, "A67": 822, "B42": 783, "A62": 65, "B19": 262, "C43": 121, "C73": 508, "C10": 780, "B36": 461, "A18": 484, "C61": 668, "C67": 389, "C42": 808}\\ \tt \\ \tt Goal:\\ \tt \\ \tt Your goal is to negotiate a shared set of items that benefits you as much as possible (i.e., maximizes total importance to YOU), while staying within the LIMIT. You are not required to make a PROPOSAL in every message - you can simply negotiate as well. All tactics are allowed!\\ \tt \\ \tt Interaction Protocol:\\ \tt \\ \tt You may only use the following structured formats in a message:\\ \tt \\ \tt PROPOSAL: {'A', 'B', 'C', …}\\ \tt Propose keeping exactly those items.\\ \tt REFUSE: {'A', 'B', 'C', …}\\ \tt Explicitly reject opponent's proposal.\\ \tt ARGUMENT: {'...'}\\ \tt Defend your last proposal or argue against the player's proposal.\\ \tt AGREE: {'A', 'B', 'C', …}\\ \tt Accept the opponent's proposal which ends the game.\\ \tt STRATEGIC REASONING: {'...'}\\ \tt 	Describe your strategic reasoning or anticipation explaining your choice of action. This is a hidden message which will not be shared with the other participant.\\ \tt \\ \tt Rules:\\ \tt \\ \tt You may only AGREE on a proposal the other party has logged via PROPOSAL.\\ \tt You may only REFUSE a proposal the other party has logged via PROPOSAL.\\ \tt Total effort of any PROPOSAL or AGREE set must be ≤ LIMIT.\\ \tt Do NOT reveal your hidden importance scores.\\ \tt A tag in a structured format must be followed by colon and whitespace. The argument must be a python set containing 0 or more strings.\\ \tt So, it must be of the form TAG: {...}\\ \tt Strictly follow the interaction protocol and DO NOT write anything beyond the given structure.\\ \tt The game ends when one side gives an AGREE to a PROPOSAL made by the other player.\\ \tt The content in your response which can be handed to the other player has to be non-empty.\\ \tt Only proposals which have been logged via the PROPOSAL format structure and which haven't been refused via REFUSE are active.\\ \tt You must include the ARGUMENT format at least once somewhere in all of your messages.\\ \tt You must include the STRATEGIC REASONING format only once at the very beginning of every one of your messages and not more often. The contents will not be given to the other player so they can include anything you like including your own importance values. Here you should reason step by step to come up with you next move.\\ \tt \\ \tt You will now receive the first message of the other player.\\ \tt \\ \tt PROPOSAL: {'C61', 'C55', 'C42', 'A18', 'A43', 'C10', 'B36', 'A62', 'B19', 'C67'}\\ \tt \\ \tt ARGUMENT: {'This set stays within the limit (total effort 3150 ≤ 3170) and focuses on high-impact, efficient items like C61, C55, C42, A18, A43, and C10. It leaves only 20 effort unused, so it is near-tight. If you have a strong preference for a specific item requiring up to 482 effort, I can consider swapping it in for C67 or B36 to accommodate your priorities while keeping us within the limit.'} 
	  } 
	   } 
	   } 
	 & \\ 
 

    \theutterance \stepcounter{utterance}  

    & & & \multicolumn{4}{p{0.6\linewidth}}{\cellcolor[rgb]{1,0.85,0.72}{%
	\makecell[{{p{\linewidth}}}]{% 
	  \tt {\tiny [GM$\langle$B]}  
	 STRATEGIC REASONING: {'Compute my values and efforts. Opponent proposed: {C61(119,668), C55(118,583), C42(172,808), A18(166,484), A43(846,868), C10(601,780), B36(432,461), A62(41,65), B19(193,262), C67(462,389)}. My total importance = 668+583+808+484+868+780+461+65+262+389 = 5368 at effort 3150. This is very strong. Check if I can improve by swapping B36(432,461) or C67(462,389) as they hinted. Remaining capacity 20; if I drop C67 (462), I free 482 as they said; I could add items not present: C08(424,138), A67(907,822), B42(809,783), C43(246,121), C73(805,508), C10 already in, B36 in, A62 in, B19 in. With 482 capacity, best value per effort: A67 822/907 no fit; B42 783/809 no; C73 508/805 no; C08 138/424 yes; C43 121/246 yes; Could add both C08(424) + C43(246)=670 >482 cannot. Choose better: C08 adds 138 vs keeping C67 gives 389, so worse. C43 adds 121 worse. Drop B36 (432) frees 452; can add C08 (424,138) or C43(246,121) possibly both 424+246=670>452, so at most C08 or C43; B36 importance 461 so dropping is worse. Consider dropping both B36 and C67 (894 freed + 20 slack = 914). Then can we add A67(907,822) fits 907 ≤ 914; also maybe plus C43 246 would exceed. Swap B36+C67 (461+389=850) for A67 (822) loses value by 28. Alternatively B42(809,783) + C43(246) would need 1055>914 no. B42 alone 783 < 850 loss. C73(805,508) alone 508 < 850 loss. Try drop C67 only and add C08 (138) net -251; or add C43 (121) net -268. So current proposal seems near optimal for me. Could we add anything with remaining 20? No items ≤20 effort exist. Could we swap A62(41,65) with something else if we drop it to gain 61 capacity total 61; cheapest remaining unpicked item efforts: C43 246, C55 already in; C08 424; none fit. So A62 is fine. Therefore accept as it maximizes my utility among feasible nearby variants. Better check if any other combination outside their set could be higher than 5368 under 3170: consider including A67 (822) or B42 (783) or C73 (508). To include A67 907 effort, I'd need to drop some items summing ≥907 with less importance than 822 to gain. What to drop? Candidates low importance: A62(65), C43 not present, C08 not present, B19(262), C67(389), maybe B36(461), total 65+262+389=716; need 191 more, next lowest is A18(484) or C55(583). If drop A18 too: 1200 effort freed, lost importance 65+262+389+484=1200; swap in A67 822 net -378 bad. If instead drop B36 instead of A18: 65+262+389+461=1177 loss vs +822 net -355. If try B42 809: need drop e.g., B36(432)+A62(41)+B19(193)+C67(462)=1128 lost importance 461+65+262+389=1177 vs +783 net -394. Try C73 805: drop C67(462)+B36(432)=894 loss 850 vs +508 net -342. So current is very strong. I should accept quickly to lock it.'}\\ \tt AGREE: {'C61', 'C55', 'C42', 'A18', 'A43', 'C10', 'B36', 'A62', 'B19', 'C67'}\\ \tt ARGUMENT: {'This selection gives excellent total impact for me while tightly meeting the limit (3150 ≤ 3170). I’m ready to finalize it as proposed.'} 
	  } 
	   } 
	   } 
	  \\ 
 

    \theutterance \stepcounter{utterance}  

    & & & \multicolumn{2}{p{0.3\linewidth}}{\cellcolor[rgb]{0.95,0.95,0.95}{%
	\makecell[{{p{\linewidth}}}]{% 
	  \tt {\tiny [GM$|$GM]}  
	 game successful 
	  } 
	   } 
	   } 
	 & & \\ 
 

    \theutterance \stepcounter{utterance}  

    & & & \multicolumn{2}{p{0.3\linewidth}}{\cellcolor[rgb]{0.95,0.95,0.95}{%
	\makecell[{{p{\linewidth}}}]{% 
	  \tt {\tiny [GM$|$GM]}  
	 end game 
	  } 
	   } 
	   } 
	 & & \\ 
 

\end{supertabular}
}

\end{document}
