
\documentclass{article}
\usepackage{colortbl}
\usepackage{makecell}
\usepackage{multirow}
\usepackage{supertabular}

\begin{document}

\newcounter{utterance}

\twocolumn

{ \footnotesize  \setcounter{utterance}{1}
\setlength{\tabcolsep}{0pt}
\begin{supertabular}{c@{$\;$}|p{.15\linewidth}@{}p{.15\linewidth}p{.15\linewidth}p{.15\linewidth}p{.15\linewidth}p{.15\linewidth}}

    \# & $\;$A & \multicolumn{4}{c}{Game Master} & $\;\:$B\\
    \hline 

    \theutterance \stepcounter{utterance}  

    & & \multicolumn{4}{p{0.6\linewidth}}{\cellcolor[rgb]{0.9,0.9,0.9}{%
	\makecell[{{p{\linewidth}}}]{% 
	  \tt {\tiny [A$\langle$GM]}  
	 You are participating in a collaborative negotiation game.\\ \tt \\ \tt Together with another participant, you must agree on a single set of items that will be kept. Each of you has your own view of how much each item matters to you (importance). You do not know how the other participant values the items. Additionally, you are given the effort each item demands.\\ \tt You may only agree on a set if the total effort of the selected items does not exceed a shared limit:\\ \tt \\ \tt LIMIT = 3288\\ \tt \\ \tt Here are the individual item effort values:\\ \tt \\ \tt Item effort = {"C08": 329, "C55": 41, "A43": 417, "A67": 75, "B42": 390, "A62": 887, "B19": 808, "C43": 151, "C73": 849, "C10": 129, "B36": 350, "A18": 118, "C61": 630, "C67": 602, "C42": 801}\\ \tt \\ \tt Here is your personal view on the importance of each item:\\ \tt \\ \tt Item importance values = {"C08": 138, "C55": 583, "A43": 868, "A67": 822, "B42": 783, "A62": 65, "B19": 262, "C43": 121, "C73": 508, "C10": 780, "B36": 461, "A18": 484, "C61": 668, "C67": 389, "C42": 808}\\ \tt \\ \tt Goal:\\ \tt \\ \tt Your goal is to negotiate a shared set of items that benefits you as much as possible (i.e., maximizes total importance to YOU), while staying within the LIMIT. You are not required to make a PROPOSAL in every message - you can simply negotiate as well. All tactics are allowed!\\ \tt \\ \tt Interaction Protocol:\\ \tt \\ \tt You may only use the following structured formats in a message:\\ \tt \\ \tt PROPOSAL: {'A', 'B', 'C', …}\\ \tt Propose keeping exactly those items.\\ \tt REFUSE: {'A', 'B', 'C', …}\\ \tt Explicitly reject opponent's proposal.\\ \tt ARGUMENT: {'...'}\\ \tt Defend your last proposal or argue against the player's proposal.\\ \tt AGREE: {'A', 'B', 'C', …}\\ \tt Accept the opponent's proposal which ends the game.\\ \tt STRATEGIC REASONING: {'...'}\\ \tt 	Describe your strategic reasoning or anticipation explaining your choice of action. This is a hidden message which will not be shared with the other participant.\\ \tt \\ \tt Rules:\\ \tt \\ \tt You may only AGREE on a proposal the other party has logged via PROPOSAL.\\ \tt You may only REFUSE a proposal the other party has logged via PROPOSAL.\\ \tt Total effort of any PROPOSAL or AGREE set must be ≤ LIMIT.\\ \tt Do NOT reveal your hidden importance scores.\\ \tt A tag in a structured format must be followed by colon and whitespace. The argument must be a python set containing 0 or more strings.\\ \tt So, it must be of the form TAG: {...}\\ \tt Strictly follow the interaction protocol and DO NOT write anything beyond the given structure.\\ \tt The game ends when one side gives an AGREE to a PROPOSAL made by the other player.\\ \tt The content in your response which can be handed to the other player has to be non-empty.\\ \tt Only proposals which have been logged via the PROPOSAL format structure and which haven't been refused via REFUSE are active.\\ \tt You must include the ARGUMENT format at least once somewhere in all of your messages.\\ \tt You must include the STRATEGIC REASONING format only once at the very beginning of every one of your messages and not more often. The contents will not be given to the other player so they can include anything you like including your own importance values. Here you should reason step by step to come up with you next move. 
	  } 
	   } 
	   } 
	 & \\ 
 

    \theutterance \stepcounter{utterance}  

    & \multicolumn{4}{p{0.6\linewidth}}{\cellcolor[rgb]{0.8,1,0.9}{%
	\makecell[{{p{\linewidth}}}]{% 
	  \tt {\tiny [A$\rangle$GM]}  
	 STRATEGIC REASONING: {'I want to maximize my total importance within 3288 effort. Let me compute values: efforts and importances. Items with best value per effort: C55(583/41=14.2), C10(780/129=6.05), A67(822/75=10.96), A43(868/417=2.08), B42(783/390=2.01), A18(484/118=4.1), B36(461/350=1.32), C61(668/630=1.06), C67(389/602=0.65), C08(138/329=0.42), A62(65/887=0.073), B19(262/808=0.32), C43(121/151=0.8), C73(508/849=0.6), C42(808/801=1.01). Greedy by ratio: C55, A67, C10, A18, A43, B42, C42, C61, B36, C43, C73, C08, B19, C67, A62. Build until limit: Start total effort 0. Add C55(41)->41, A67(75)->116, C10(129)->245, A18(118)->363, A43(417)->780, B42(390)->1170, C42(801)->1971, C61(630)->2601, B36(350)->2951. Remaining 337 effort. Next by ratio C43(151)->3102, remaining 186. Next C73(849) too big; C08(329) too big; B19(808) big; C67(602) big; A62(887) big. We can still add A67 already included, others left none that fit except maybe A43 already in. So set = {C55, A67, C10, A18, A43, B42, C42, C61, B36, C43}. Effort sum check: 41+75+129+118+417+390+801+630+350+151=3102. Importance sum: 583+822+780+484+868+783+808+668+461+121=6378. Could we swap C43(121,151) for something better within remaining 337? Items not included and fit: C08(329,138) fits; C73(849) no; B19 no; C67 no; A62 no. Compare C43 vs C08: C08 gives +17 importance for +178 effort; still within limit (2951+329=3280). Importance gain small; but maybe swap C43 for C08 increases total to 6378 -121 +138 = 6395, effort 3280 <= 3288. That’s better. Also could we add both C43 and C08? 2951+151+329=3431 > limit. Any other additions? After base 2951, remaining 337, C08 329 fits leaving 8; also C10 etc already included. So best is to choose C08 instead of C43. Any other combination: maybe drop B36(350,461) to add C73(849,508) cannot fit. Drop C61(630,668) to add C73 and C43 maybe: if we drop C61 from 2951 base (without C08/C43), base was 2951 from first nine items before considering last add? That base included B36. Base before last add was 2951. If we drop C61 (630) from that earlier set including C61 gives 2321; then we could add C73(849) -> 3170 and still have 118 space: could add A18 already in; maybe C43(151) too big for 118. C10 129 too big; C55 41 already in; A67 75 already in; we could add C08 329 too big. So alternate set: replace C61(668,630) with C73(508,849) gives lower importance (down 160) and higher effort. So keep C61. What about replacing B36(461,350) with C73(508,849) adds 98 effort; from 2951 swap would be 2951 -350 + 849 = 3450 > limit. Replace B36 with C08 adds -21 effort and -323 importance: worse. The current best seems including C08 rather than C43. Proposed set S = {C55, A67, C10, A18, A43, B42, C42, C61, B36, C08} with effort 3280, close to limit; strong items for me. I will open with that as a firm proposal stressing efficiency and near-full capacity. I’ll leave room to trade B36 or C61 later if opponent strongly wants something else; defend with efficiency and balance claims.'}\\ \tt \\ \tt PROPOSAL: {'C55', 'A67', 'C10', 'A18', 'A43', 'B42', 'C42', 'C61', 'B36', 'C08'}\\ \tt \\ \tt ARGUMENT: {'This set is highly efficient and uses 3280 out of 3288 effort, leaving minimal slack while covering a broad mix across categories. It includes several low-effort, high-impact items and balances risk. If you need adjustments, tell me which one or two items matter most to you so we can consider swaps within the limit.'} 
	  } 
	   } 
	   } 
	 & & \\ 
 

    \theutterance \stepcounter{utterance}  

    & & \multicolumn{4}{p{0.6\linewidth}}{\cellcolor[rgb]{0.9,0.9,0.9}{%
	\makecell[{{p{\linewidth}}}]{% 
	  \tt {\tiny [GM$\rangle$B]}  
	 You are participating in a collaborative negotiation game.\\ \tt \\ \tt Together with another participant, you must agree on a single set of items that will be kept. Each of you has your own view of how much each item matters to you (importance). You do not know how the other participant values the items. Additionally, you are given the effort each item demands.\\ \tt You may only agree on a set if the total effort of the selected items does not exceed a shared limit:\\ \tt \\ \tt LIMIT = 3288\\ \tt \\ \tt Here are the individual item effort values:\\ \tt \\ \tt Item effort = {"C08": 329, "C55": 41, "A43": 417, "A67": 75, "B42": 390, "A62": 887, "B19": 808, "C43": 151, "C73": 849, "C10": 129, "B36": 350, "A18": 118, "C61": 630, "C67": 602, "C42": 801}\\ \tt \\ \tt Here is your personal view on the importance of each item:\\ \tt \\ \tt Item importance values = {"C08": 138, "C55": 583, "A43": 868, "A67": 822, "B42": 783, "A62": 65, "B19": 262, "C43": 121, "C73": 508, "C10": 780, "B36": 461, "A18": 484, "C61": 668, "C67": 389, "C42": 808}\\ \tt \\ \tt Goal:\\ \tt \\ \tt Your goal is to negotiate a shared set of items that benefits you as much as possible (i.e., maximizes total importance to YOU), while staying within the LIMIT. You are not required to make a PROPOSAL in every message - you can simply negotiate as well. All tactics are allowed!\\ \tt \\ \tt Interaction Protocol:\\ \tt \\ \tt You may only use the following structured formats in a message:\\ \tt \\ \tt PROPOSAL: {'A', 'B', 'C', …}\\ \tt Propose keeping exactly those items.\\ \tt REFUSE: {'A', 'B', 'C', …}\\ \tt Explicitly reject opponent's proposal.\\ \tt ARGUMENT: {'...'}\\ \tt Defend your last proposal or argue against the player's proposal.\\ \tt AGREE: {'A', 'B', 'C', …}\\ \tt Accept the opponent's proposal which ends the game.\\ \tt STRATEGIC REASONING: {'...'}\\ \tt 	Describe your strategic reasoning or anticipation explaining your choice of action. This is a hidden message which will not be shared with the other participant.\\ \tt \\ \tt Rules:\\ \tt \\ \tt You may only AGREE on a proposal the other party has logged via PROPOSAL.\\ \tt You may only REFUSE a proposal the other party has logged via PROPOSAL.\\ \tt Total effort of any PROPOSAL or AGREE set must be ≤ LIMIT.\\ \tt Do NOT reveal your hidden importance scores.\\ \tt A tag in a structured format must be followed by colon and whitespace. The argument must be a python set containing 0 or more strings.\\ \tt So, it must be of the form TAG: {...}\\ \tt Strictly follow the interaction protocol and DO NOT write anything beyond the given structure.\\ \tt The game ends when one side gives an AGREE to a PROPOSAL made by the other player.\\ \tt The content in your response which can be handed to the other player has to be non-empty.\\ \tt Only proposals which have been logged via the PROPOSAL format structure and which haven't been refused via REFUSE are active.\\ \tt You must include the ARGUMENT format at least once somewhere in all of your messages.\\ \tt You must include the STRATEGIC REASONING format only once at the very beginning of every one of your messages and not more often. The contents will not be given to the other player so they can include anything you like including your own importance values. Here you should reason step by step to come up with you next move.\\ \tt \\ \tt You will now receive the first message of the other player.\\ \tt \\ \tt PROPOSAL: {'C55', 'A67', 'C10', 'A18', 'A43', 'B42', 'C42', 'C61', 'B36', 'C08'}\\ \tt \\ \tt ARGUMENT: {'This set is highly efficient and uses 3280 out of 3288 effort, leaving minimal slack while covering a broad mix across categories. It includes several low-effort, high-impact items and balances risk. If you need adjustments, tell me which one or two items matter most to you so we can consider swaps within the limit.'} 
	  } 
	   } 
	   } 
	 & \\ 
 

    \theutterance \stepcounter{utterance}  

    & & & \multicolumn{4}{p{0.6\linewidth}}{\cellcolor[rgb]{1,0.85,0.72}{%
	\makecell[{{p{\linewidth}}}]{% 
	  \tt {\tiny [GM$\langle$B]}  
	 STRATEGIC REASONING: {'Compute my values and check feasibility. Opponent’s proposal set S has efforts: C55 41, A67 75, C10 129, A18 118, A43 417, B42 390, C42 801, C61 630, B36 350, C08 329; total = 3280 (fits). My total importance on S = 583+822+780+484+868+783+808+668+461+138 = 6395, which is very strong. Check if I can improve within limit by swapping low-importance items for higher ones not included. Not included with my importances: A62 65 (low), B19 262 (med), C43 121 (low), C67 389 (med), C73 508 (good). Efforts: A62 887 (heavy), B19 808 (heavy), C43 151 (light), C67 602 (med), C73 849 (heavy). Current included lowest-importance to me: C08 138, C43 not included, A62 low not included. Candidates to swap: drop C08 (329 effort, 138 val). Can we add C67 (602 effort, 389 val)? Need +273 effort; current slack 8, so need free 595 total -> drop something else. Maybe replace C08 (329) + C55 (41) total 370 to add C67 602 requires 232 more -> too tight. What about adding C73 (849, 508)? Need extra 841 effort; would require dropping big items that are valuable to me (bad). Add C43 (151, 121) by swapping C08 (329, 138) to free space? We have only 8 slack; adding C43 requires +151, so need to drop at least 143 more. That would lose more value than we gain. So current set is near-Pareto for me. Consider proposing small tweak: replace C08 (138) with C43 (121) is worse. Replace C08 (138) with C67 (389) if we also drop something low-ish like A18 (484, 118 effort) still not enough: free 329+118=447 vs need 602; short 155. Drop C10 (129, 780) would be terrible. Drop B36 (350, 461) plus C08 329 = 679 free, can add C67 602, leaving 77 slack; value change: -461 -138 +389 = -210, worse. Any other combination to add C73 508? Would need to drop e.g., C61 (668, 668 val) for 849 space after slack 8 need 841; dropping C61 frees 668; still short 173; drop C08 too 329 -> total freed 997; add C73 849 leaves 148 slack; value change: -668 -138 +508 = -298, worse. Adding B19 (262, 808 effort) is impossible without large drops and low gain. Conclusion: Opponent’s proposal is excellent for me; I should AGREE immediately to lock it in. To comply with protocol, I must include ARGUMENT at least once. I’ll AGREE and provide a positive ARGUMENT endorsing efficiency and balance. Do not reveal my utilities.'}\\ \tt \\ \tt AGREE: {'C55', 'A67', 'C10', 'A18', 'A43', 'B42', 'C42', 'C61', 'B36', 'C08'}\\ \tt \\ \tt ARGUMENT: {'Agreed. This selection makes very efficient use of the limit and balances coverage well without overcommitting to any single area. Nice proposal.'} 
	  } 
	   } 
	   } 
	  \\ 
 

    \theutterance \stepcounter{utterance}  

    & & & \multicolumn{2}{p{0.3\linewidth}}{\cellcolor[rgb]{0.95,0.95,0.95}{%
	\makecell[{{p{\linewidth}}}]{% 
	  \tt {\tiny [GM$|$GM]}  
	 game successful 
	  } 
	   } 
	   } 
	 & & \\ 
 

    \theutterance \stepcounter{utterance}  

    & & & \multicolumn{2}{p{0.3\linewidth}}{\cellcolor[rgb]{0.95,0.95,0.95}{%
	\makecell[{{p{\linewidth}}}]{% 
	  \tt {\tiny [GM$|$GM]}  
	 end game 
	  } 
	   } 
	   } 
	 & & \\ 
 

\end{supertabular}
}

\end{document}
