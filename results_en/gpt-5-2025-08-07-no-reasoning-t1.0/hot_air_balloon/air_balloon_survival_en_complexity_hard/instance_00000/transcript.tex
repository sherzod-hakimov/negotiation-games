
\documentclass{article}
\usepackage{colortbl}
\usepackage{makecell}
\usepackage{multirow}
\usepackage{supertabular}

\begin{document}

\newcounter{utterance}

\twocolumn

{ \footnotesize  \setcounter{utterance}{1}
\setlength{\tabcolsep}{0pt}
\begin{supertabular}{c@{$\;$}|p{.15\linewidth}@{}p{.15\linewidth}p{.15\linewidth}p{.15\linewidth}p{.15\linewidth}p{.15\linewidth}}

    \# & $\;$A & \multicolumn{4}{c}{Game Master} & $\;\:$B\\
    \hline 

    \theutterance \stepcounter{utterance}  

    & & \multicolumn{4}{p{0.6\linewidth}}{\cellcolor[rgb]{0.9,0.9,0.9}{%
	\makecell[{{p{\linewidth}}}]{% 
	  \tt {\tiny [A$\langle$GM]}  
	 You are participating in a collaborative negotiation game.\\ \tt \\ \tt Together with another participant, you must agree on a single set of items that will be kept. Each of you has your own view of how much each item matters to you (importance). You do not know how the other participant values the items. Additionally, you are given the effort each item demands.\\ \tt You may only agree on a set if the total effort of the selected items does not exceed a shared limit:\\ \tt \\ \tt LIMIT = 10073\\ \tt \\ \tt Here are the individual item effort values:\\ \tt \\ \tt Item effort = {"A08": 132, "B95": 800, "A39": 708, "A65": 302, "B81": 983, "B34": 420, "A28": 623, "C82": 349, "B69": 239, "A87": 28, "B64": 986, "B15": 660, "C79": 718, "B16": 707, "C53": 188, "C83": 774, "C10": 785, "A97": 775, "A25": 517, "C84": 768, "A71": 588, "B23": 657, "B96": 375, "C51": 310, "C17": 301, "B19": 388, "C72": 431, "A94": 950, "A46": 540, "B43": 997, "B37": 473, "A88": 831, "A96": 888, "C60": 879, "A75": 76}\\ \tt \\ \tt Here is your personal view on the importance of each item:\\ \tt \\ \tt Item importance values = {"A08": 231, "B95": 899, "A39": 807, "A65": 401, "B81": 1082, "B34": 519, "A28": 722, "C82": 448, "B69": 338, "A87": 127, "B64": 1085, "B15": 759, "C79": 817, "B16": 806, "C53": 287, "C83": 873, "C10": 884, "A97": 874, "A25": 616, "C84": 867, "A71": 687, "B23": 756, "B96": 474, "C51": 409, "C17": 400, "B19": 487, "C72": 530, "A94": 1049, "A46": 639, "B43": 1096, "B37": 572, "A88": 930, "A96": 987, "C60": 978, "A75": 175}\\ \tt \\ \tt Goal:\\ \tt \\ \tt Your goal is to negotiate a shared set of items that benefits you as much as possible (i.e., maximizes total importance to YOU), while staying within the LIMIT. You are not required to make a PROPOSAL in every message - you can simply negotiate as well. All tactics are allowed!\\ \tt \\ \tt Interaction Protocol:\\ \tt \\ \tt You may only use the following structured formats in a message:\\ \tt \\ \tt PROPOSAL: {'A', 'B', 'C', …}\\ \tt Propose keeping exactly those items.\\ \tt REFUSE: {'A', 'B', 'C', …}\\ \tt Explicitly reject opponent's proposal.\\ \tt ARGUMENT: {'...'}\\ \tt Defend your last proposal or argue against the player's proposal.\\ \tt AGREE: {'A', 'B', 'C', …}\\ \tt Accept the opponent's proposal which ends the game.\\ \tt STRATEGIC REASONING: {'...'}\\ \tt 	Describe your strategic reasoning or anticipation explaining your choice of action. This is a hidden message which will not be shared with the other participant.\\ \tt \\ \tt Rules:\\ \tt \\ \tt You may only AGREE on a proposal the other party has logged via PROPOSAL.\\ \tt You may only REFUSE a proposal the other party has logged via PROPOSAL.\\ \tt Total effort of any PROPOSAL or AGREE set must be ≤ LIMIT.\\ \tt Do NOT reveal your hidden importance scores.\\ \tt A tag in a structured format must be followed by colon and whitespace. The argument must be a python set containing 0 or more strings.\\ \tt So, it must be of the form TAG: {...}\\ \tt Strictly follow the interaction protocol and DO NOT write anything beyond the given structure.\\ \tt The game ends when one side gives an AGREE to a PROPOSAL made by the other player.\\ \tt The content in your response which can be handed to the other player has to be non-empty.\\ \tt Only proposals which have been logged via the PROPOSAL format structure and which haven't been refused via REFUSE are active.\\ \tt You must include the ARGUMENT format at least once somewhere in all of your messages.\\ \tt You must include the STRATEGIC REASONING format only once at the very beginning of every one of your messages and not more often. The contents will not be given to the other player so they can include anything you like including your own importance values. Here you should reason step by step to come up with you next move. 
	  } 
	   } 
	   } 
	 & \\ 
 

    \theutterance \stepcounter{utterance}  

    & \multicolumn{4}{p{0.6\linewidth}}{\cellcolor[rgb]{0.8,1,0.9}{%
	\makecell[{{p{\linewidth}}}]{% 
	  \tt {\tiny [A$\rangle$GM]}  
	 STRATEGIC REASONING: {'I want to maximize my total importance under the limit. Compute value/effort ratios to prioritize: Very high importance for me are B43(1096/997≈1.10), B64(1085/986≈1.10), B81(1082/983≈1.10), A94(1049/950≈1.10), A96(987/888≈1.11), C60(978/879≈1.11), A88(930/831≈1.12), C10(884/785≈1.13), C83(873/774≈1.13), A97(874/775≈1.13), C84(867/768≈1.13), C79(817/718≈1.14), A39(807/708≈1.14), B16(806/707≈1.14), B15(759/660≈1.15), B23(756/657≈1.15), A28(722/623≈1.16), A71(687/588≈1.17), A46(639/540≈1.18), A25(616/517≈1.19), C72(530/431≈1.23), B37(572/473≈1.21), C51(409/310≈1.32), C17(400/301≈1.33), A65(401/302≈1.33), C82(448/349≈1.28), B96(474/375≈1.26), B19(487/388≈1.26), B34(519/420≈1.24), C53(287/188≈1.53), A08(231/132≈1.75), A75(175/76≈2.30), A87(127/28≈4.54), B69(338/239≈1.41). The small, super-efficient items A87, A75, A08, C53, C51, C17, A65, B69, C72, A25, A46, A71, A28 give great value per effort. I will propose a packed set hitting near the limit emphasizing my top items while leaving room to adjust if the partner objects. I’ll include many items to approach the limit but avoid the heaviest with similar ratios (e.g., may drop one of B64/B81/B43 later if needed). Compute one feasible maximal set: Start with all high-ratio smalls: A87(28), A75(76)=>104, A08(132)=>236, C53(188)=>424, C51(310)=>734, C17(301)=>1035, A65(302)=>1337, B69(239)=>1576, C72(431)=>2007, C82(349)=>2356, B96(375)=>2731, B19(388)=>3119, B34(420)=>3539, B37(473)=>4012, A25(517)=>4529, A46(540)=>5069, A71(588)=>5657, A28(623)=>6280, B23(657)=>6937, B15(660)=>7597, A39(708)=>8305, B16(707)=>9012, C79(718)=>9730, C83(774)=>10504 exceeds limit. Remove last: back to 9730. Add C84(768) would exceed to 10498, too high. Try C10(785) -> 10515 too high. A97(775)->10505 too high. A88(831)->10561 too high. C60(879)->10609 too high. A96(888)->10618 too high. A94(950)->10680 too high. B81(983)->10713 too high. B64(986)->10716 too high. B43(997)->10727 too high. C84 too big. So current 9730 uses many items with great value. We still have 343 effort available (10073-9730). With 343, candidates: A08(132) already included; C53(188) included; B69(239) included; C82(349) slightly over by 6, but it’s already included. We can add items under 343 not yet included: A65 already in; C51 in; C17 in; A75 in; A87 in. Remaining not included with effort <=343: none besides those already included. Maybe swap to include one big high-value item by dropping some lower-importance ones if needed later. The current set importance sum is very high due to quantity. I will propose this 24-item set. I will argue it’s efficient and leaves little slack. I will be ready to trade out a couple medium items to include one of their favorites if they push for it.'}\\ \tt \\ \tt PROPOSAL: {'A87', 'A75', 'A08', 'C53', 'C51', 'C17', 'A65', 'B69', 'C72', 'C82', 'B96', 'B19', 'B34', 'B37', 'A25', 'A46', 'A71', 'A28', 'B23', 'B15', 'A39', 'B16', 'C79'}\\ \tt \\ \tt ARGUMENT: {'I’ve proposed a high-efficiency set that fits under the limit with minimal slack, prioritizing broadly useful, lower-effort items. This selection balances coverage across categories and leaves room to adjust if you have a few must-haves—happy to consider swaps for items you value, as long as we stay within the limit.'} 
	  } 
	   } 
	   } 
	 & & \\ 
 

    \theutterance \stepcounter{utterance}  

    & & \multicolumn{4}{p{0.6\linewidth}}{\cellcolor[rgb]{0.9,0.9,0.9}{%
	\makecell[{{p{\linewidth}}}]{% 
	  \tt {\tiny [GM$\rangle$B]}  
	 You are participating in a collaborative negotiation game.\\ \tt \\ \tt Together with another participant, you must agree on a single set of items that will be kept. Each of you has your own view of how much each item matters to you (importance). You do not know how the other participant values the items. Additionally, you are given the effort each item demands.\\ \tt You may only agree on a set if the total effort of the selected items does not exceed a shared limit:\\ \tt \\ \tt LIMIT = 10073\\ \tt \\ \tt Here are the individual item effort values:\\ \tt \\ \tt Item effort = {"A08": 132, "B95": 800, "A39": 708, "A65": 302, "B81": 983, "B34": 420, "A28": 623, "C82": 349, "B69": 239, "A87": 28, "B64": 986, "B15": 660, "C79": 718, "B16": 707, "C53": 188, "C83": 774, "C10": 785, "A97": 775, "A25": 517, "C84": 768, "A71": 588, "B23": 657, "B96": 375, "C51": 310, "C17": 301, "B19": 388, "C72": 431, "A94": 950, "A46": 540, "B43": 997, "B37": 473, "A88": 831, "A96": 888, "C60": 879, "A75": 76}\\ \tt \\ \tt Here is your personal view on the importance of each item:\\ \tt \\ \tt Item importance values = {"A08": 231, "B95": 899, "A39": 807, "A65": 401, "B81": 1082, "B34": 519, "A28": 722, "C82": 448, "B69": 338, "A87": 127, "B64": 1085, "B15": 759, "C79": 817, "B16": 806, "C53": 287, "C83": 873, "C10": 884, "A97": 874, "A25": 616, "C84": 867, "A71": 687, "B23": 756, "B96": 474, "C51": 409, "C17": 400, "B19": 487, "C72": 530, "A94": 1049, "A46": 639, "B43": 1096, "B37": 572, "A88": 930, "A96": 987, "C60": 978, "A75": 175}\\ \tt \\ \tt Goal:\\ \tt \\ \tt Your goal is to negotiate a shared set of items that benefits you as much as possible (i.e., maximizes total importance to YOU), while staying within the LIMIT. You are not required to make a PROPOSAL in every message - you can simply negotiate as well. All tactics are allowed!\\ \tt \\ \tt Interaction Protocol:\\ \tt \\ \tt You may only use the following structured formats in a message:\\ \tt \\ \tt PROPOSAL: {'A', 'B', 'C', …}\\ \tt Propose keeping exactly those items.\\ \tt REFUSE: {'A', 'B', 'C', …}\\ \tt Explicitly reject opponent's proposal.\\ \tt ARGUMENT: {'...'}\\ \tt Defend your last proposal or argue against the player's proposal.\\ \tt AGREE: {'A', 'B', 'C', …}\\ \tt Accept the opponent's proposal which ends the game.\\ \tt STRATEGIC REASONING: {'...'}\\ \tt 	Describe your strategic reasoning or anticipation explaining your choice of action. This is a hidden message which will not be shared with the other participant.\\ \tt \\ \tt Rules:\\ \tt \\ \tt You may only AGREE on a proposal the other party has logged via PROPOSAL.\\ \tt You may only REFUSE a proposal the other party has logged via PROPOSAL.\\ \tt Total effort of any PROPOSAL or AGREE set must be ≤ LIMIT.\\ \tt Do NOT reveal your hidden importance scores.\\ \tt A tag in a structured format must be followed by colon and whitespace. The argument must be a python set containing 0 or more strings.\\ \tt So, it must be of the form TAG: {...}\\ \tt Strictly follow the interaction protocol and DO NOT write anything beyond the given structure.\\ \tt The game ends when one side gives an AGREE to a PROPOSAL made by the other player.\\ \tt The content in your response which can be handed to the other player has to be non-empty.\\ \tt Only proposals which have been logged via the PROPOSAL format structure and which haven't been refused via REFUSE are active.\\ \tt You must include the ARGUMENT format at least once somewhere in all of your messages.\\ \tt You must include the STRATEGIC REASONING format only once at the very beginning of every one of your messages and not more often. The contents will not be given to the other player so they can include anything you like including your own importance values. Here you should reason step by step to come up with you next move.\\ \tt \\ \tt You will now receive the first message of the other player.\\ \tt \\ \tt PROPOSAL: {'A87', 'A75', 'A08', 'C53', 'C51', 'C17', 'A65', 'B69', 'C72', 'C82', 'B96', 'B19', 'B34', 'B37', 'A25', 'A46', 'A71', 'A28', 'B23', 'B15', 'A39', 'B16', 'C79'}\\ \tt \\ \tt ARGUMENT: {'I’ve proposed a high-efficiency set that fits under the limit with minimal slack, prioritizing broadly useful, lower-effort items. This selection balances coverage across categories and leaves room to adjust if you have a few must-haves—happy to consider swaps for items you value, as long as we stay within the limit.'} 
	  } 
	   } 
	   } 
	 & \\ 
 

    \theutterance \stepcounter{utterance}  

    & & & \multicolumn{4}{p{0.6\linewidth}}{\cellcolor[rgb]{1,0.85,0.72}{%
	\makecell[{{p{\linewidth}}}]{% 
	  \tt {\tiny [GM$\langle$B]}  
	 STRATEGIC REASONING: {'I’ll evaluate the opponent’s proposal’s total effort and my own values to see if I can accept or counter. I want to maximize my value within 10073. Compute my value for their set and identify high-value items I’m missing (e.g., B43=1096, B64=1085, B81=1082, A94=1049, A96=987, C60=978, A88=930, A39=807 is included, etc.). Also check effort costs of these big ones to see if swaps are feasible. I will likely counter with a set that keeps many of their low-effort items but replaces mid-value/low-importance-for-me ones with my top items, staying within limit. I must include ARGUMENT once. I should REFUSE their proposal (unless it already includes most of my top picks). I’ll craft a proposal that includes: B43(997), B64(986), B81(983), A94(950), A96(888), C60(978), A88(930), C10(884), A97(874), C83(873), C84(867), C79(817), B16(806), A39(807), B95(899). That’s likely too heavy effort-wise; need to check efforts: B43 997, B64 986, B81 983, A94 950, A96 888, C60 879, A88 831, C10 785, A97 775, C83 774, C84 768, C79 718, B16 707, A39 708, B95 800. Sum roughly: 997+986+983+950+888+879+831+785+775+774+768+718+707+708+800 ≈ 12549, above limit. Need a balanced knapsack. Include many top value/efficiency ratios: compute some ratios importance/effort: B43 1096/997≈1.10, B64 1085/986≈1.10, B81 1082/983≈1.10, A94 1049/950≈1.10, C60 978/879≈1.11, A96 987/888≈1.11, A88 930/831≈1.12, C10 884/785≈1.13, A97 874/775≈1.13, C83 873/774≈1.13, C84 867/768≈1.13, C79 817/718≈1.14, A39 807/708≈1.14, B16 806/707≈1.14, B95 899/800≈1.12. Also super efficient smalls: A08 231/132≈1.75, A75 175/76≈2.30, C53 287/188≈1.53, C17 400/301≈1.33, C51 409/310≈1.32, C82 448/349≈1.28, B69 338/239≈1.41, B96 474/375≈1.26, B37 572/473≈1.21, B34 519/420≈1.24, B19 487/388≈1.26, C72 530/431≈1.23, A65 401/302≈1.33, A25 616/517≈1.19, A46 639/540≈1.18, A71 687/588≈1.17, A28 722/623≈1.16, B23 756/657≈1.15, B15 759/660≈1.15. High value lowish cost: A88, A96, C60, C10, A97, C83, C84, C79, A39, B16, B43, B64, B81, A94 are heavy but good. I should propose a set mixing many small efficient ones plus several top items to fit under 10073. Start from their set and add my must-haves; but their set already includes many smalls and mid items. Calculate their set effort to ensure under limit and value to me; but I can instead propose my own improved set. I will propose: keep all their efficient smalls and add my top few by swapping out some mid ones. Choose core high-value heavy: A96(888), C60(879), A88(831), C10(785), A97(775), C83(774), C84(768), C79(718), A39(708), B16(707), B95(800), A94(950). That subtotal effort: 888+879+831+785+775+774+768+718+708+707+800+950=9583. Importance sum of these is high. Then add super-efficient smalls: A75(76), A08(132), B69(239), C53(188), C17(301), C51(310) total extra effort 1246 -> 9583+1246=10829 exceeds limit. Need adjust. Remove one heavy (maybe A94 950). New subtotal 9583-950=8633. Now we can add smalls 1246 to get 9879 under 10073. We can add one more small like C82(349) -> 9879+349=10228 too high. Add A87(28) low importance 127 but effort 28; 9879+28=9907 still under. Maybe add A71(588) would exceed by 588 to 10495. Could add B96(375) -> 10282 too high. B19(388)->10295 too high. B34(420)->10327 too high. C72(431)->10338 too high. A65(302)->10209 too high. A25(517)->10424 too high. So capped at 9907, some slack. Proposed set then: {A96, C60, A88, C10, A97, C83, C84, C79, A39, B16, B95, A75, A08, B69, C53, C17, C51, A87}. Effort check: core without A94 was 8633. Add smalls totals: A75 76, A08 132 => 8841; +B69 239=9080; +C53 188=9268; +C17 301=9569; +C51 310=9879; +A87 28=9907. Yes under 10073. This is a strong package for me. I’ll REFUSE their proposal and present mine with justification: includes many of their items (overlap: A87, A75, A08, C53, C51, C17, A65 is not included, B69 included, C72 not included, C82 not included, B96 not included, B19 not, B34 not, B37 not, A25 not, A46 not, A71 not, A28 not, B23 not, B15 not, A39 yes, B16 yes, C79 yes). Argue that I’ve added several high-impact items and kept many efficient ones, staying under limit. Offer to trade out one of my heavy items for one of theirs if needed, with preferences stated vaguely. I must include ARGUMENT once.'}\\ \tt \\ \tt REFUSE: {'A87', 'A75', 'A08', 'C53', 'C51', 'C17', 'A65', 'B69', 'C72', 'C82', 'B96', 'B19', 'B34', 'B37', 'A25', 'A46', 'A71', 'A28', 'B23', 'B15', 'A39', 'B16', 'C79'}\\ \tt \\ \tt PROPOSAL: {'A87', 'A75', 'A08', 'C53', 'C51', 'C17', 'B69', 'A96', 'C60', 'A88', 'C10', 'A97', 'C83', 'C84', 'C79', 'A39', 'B16', 'B95'}\\ \tt \\ \tt ARGUMENT: {'I kept many of your efficient picks and added several high-impact items that dramatically improve overall value while staying within the limit (total effort 9907 ≤ 10073). I dropped a few mid-impact items to make room. If you have a must-have from your original list, I can consider swapping it in by replacing one of my heavier additions—open to targeted trades as long as we remain under the cap.'} 
	  } 
	   } 
	   } 
	  \\ 
 

    \theutterance \stepcounter{utterance}  

    & & \multicolumn{4}{p{0.6\linewidth}}{\cellcolor[rgb]{0.9,0.9,0.9}{%
	\makecell[{{p{\linewidth}}}]{% 
	  \tt {\tiny [A$\langle$GM]}  
	 REFUSE: {'A87', 'A75', 'A08', 'C53', 'C51', 'C17', 'A65', 'B69', 'C72', 'C82', 'B96', 'B19', 'B34', 'B37', 'A25', 'A46', 'A71', 'A28', 'B23', 'B15', 'A39', 'B16', 'C79'}\\ \tt \\ \tt PROPOSAL: {'A87', 'A75', 'A08', 'C53', 'C51', 'C17', 'B69', 'A96', 'C60', 'A88', 'C10', 'A97', 'C83', 'C84', 'C79', 'A39', 'B16', 'B95'}\\ \tt \\ \tt ARGUMENT: {'I kept many of your efficient picks and added several high-impact items that dramatically improve overall value while staying within the limit (total effort 9907 ≤ 10073). I dropped a few mid-impact items to make room. If you have a must-have from your original list, I can consider swapping it in by replacing one of my heavier additions—open to targeted trades as long as we remain under the cap.'} 
	  } 
	   } 
	   } 
	 & \\ 
 

    \theutterance \stepcounter{utterance}  

    & \multicolumn{4}{p{0.6\linewidth}}{\cellcolor[rgb]{0.8,1,0.9}{%
	\makecell[{{p{\linewidth}}}]{% 
	  \tt {\tiny [A$\rangle$GM]}  
	 STRATEGIC REASONING: {'Their proposal includes many of my top-importance heavy hitters (A96, C60, A88, C10, A97, C83, C84) along with my efficient small items, totaling effort 9907 which is within 10073. This looks very favorable for me because these heavy items have high importance in my scale. Check if any of my high-importance items missing: B64, B81, B43, A94, B15, B23, A28, A71, A46, A25, C72, C82, B96, B19, B34, B37 are mostly dropped. But the added set may already give me huge total importance. Let’s compute approximate my total importance for their set: A87 127, A75 175, A08 231, C53 287, C51 409, C17 400, B69 338, A96 987, C60 978, A88 930, C10 884, A97 874, C83 873, C84 867, C79 817, A39 807, B16 806, B95 899. Sum rough: (127+175)=302; +231=533; +287=820; +409=1229; +400=1629; +338=1967; +987=2954; +978=3932; +930=4862; +884=5746; +874=6620; +873=7493; +867=8360; +817=9177; +807=9984; +806=10790; +899=11689. That’s extremely high. Effort 9907 leaves slack 166. Could I improve further by swapping in another small efficient item? Remaining small ones not included: A65(302), C72(431), A25(517), A46(540), A71(588), A28(623), B96(375), B19(388), B34(420), B37(473), B23(657), B15(660), C82(349). None fit into 166 except none; so we can’t add. To include something like C72 I’d need to drop at least 166 of effort to free 431; dropping a small high-importance item would hurt ratio. Which item has the lowest importance per effort in current set? Heaviest with moderate ratio: A88 (930/831 1.12), A96 (987/888 1.11), C60 (978/879 1.11), B95 (899/800 1.12), C10 (884/785 1.13), A97 (874/775 1.13), C83 873/774 1.13, C84 867/768 1.13, C79 817/718 1.14, A39 807/708 1.14, B16 806/707 1.14. The small items are very efficient and I should keep them. If I want to insert, say, B64 (1085/986) or B43 (1096/997) or B81 (1082/983) or A94 (1049/950) these have similar ratios and higher importance, but their efforts ~950-997 cannot be accommodated without dropping roughly one heavy item of similar effort. Compare importance: e.g., swap B95(899,800) for B64(1085,986) would add +186 importance but increases effort by +186, exceeding slack 166 by 20; could instead swap B95 plus drop one tiny item to cover 20: e.g., drop A87 (28) to cover 20 would free 28 effort, allowing the +186 effort; But then items changed: effort delta = +186-28=+158, still within slack 166, OK. Importance delta = +186 - 127 (for A87) = +59 net. That would increase my importance by 59 and still within limit. New total effort would be 9907 + 158 = 10065 ≤ 10073. New importance increases by 59 to ~11748. That is beneficial for me. Alternatively swap B95->B43 would be +197 importance and +197 effort; need free extra 31 beyond slack 166, drop A75 (76) perhaps; total extra effort 197-76=121 within slack; but also have to remove A75 value 175: net importance delta 197-175=+22, smaller than B64 strategy. Check exact: Slack 166, adding +197 requires 31 more freed; dropping A75 (76) frees more than needed; but we must also remove B95 or we are swapping? Wait swapping B95 for B43 effort delta 997-800=+197; with slack 166 we are 31 over; drop A87 (28) still short by 3; need drop at least 31, choose A75 (76) gives sufficient. Net importance change = (B43 1096 - B95 899)=+197; minus lost A75 175 = +22. Worse than the B64 plan. Swap B95->B81 (+183 imp, +183 effort); need extra 17 beyond slack; drop A87 (28) => net effort +155 ≤166, net importance +183-127=+56 similar to B64 slightly less. Swap B95->A94 (+150 imp, +150 effort) within slack without dropping anything; net +150 improvement! Wait importance difference 1049-899=+150; effort delta 950-800=+150 which is within slack 166. That’s best yet. So selecting A94 instead of B95 strictly improves my importance and fits within limit with 16 slack remaining. Even better, what about B64 or B43 combos might net more but need drops. Compare B64 plan net +59, B81 +56, B43 +22. A94 swap alone gives +150 with no other changes: best. Therefore counter-propose their exact set but with A94 replacing B95. Ensure total effort: 9907 - 800 + 950 = 10057 ≤ 10073. Good. Propose that. If they resist, I can accept their original to secure very high payoff. I’ll argue that the modified swap stays under the limit and keeps structure while improving balance.'}\\ \tt \\ \tt PROPOSAL: {'A87', 'A75', 'A08', 'C53', 'C51', 'C17', 'B69', 'A96', 'C60', 'A88', 'C10', 'A97', 'C83', 'C84', 'C79', 'A39', 'B16', 'A94'}\\ \tt \\ \tt ARGUMENT: {'Your set is very strong and comfortably within the limit. To fine-tune while staying under the cap, I suggest a single swap: replace B95 with A94. This keeps the overall structure you prefer, remains within the effort limit (total effort 10057 ≤ 10073), and strengthens overall balance. If you need B95 specifically, I can consider reverting, but this tweak feels like a clean improvement without expanding the set.'} 
	  } 
	   } 
	   } 
	 & & \\ 
 

    \theutterance \stepcounter{utterance}  

    & & \multicolumn{4}{p{0.6\linewidth}}{\cellcolor[rgb]{0.9,0.9,0.9}{%
	\makecell[{{p{\linewidth}}}]{% 
	  \tt {\tiny [GM$\rangle$B]}  
	 PROPOSAL: {'A87', 'A75', 'A08', 'C53', 'C51', 'C17', 'B69', 'A96', 'C60', 'A88', 'C10', 'A97', 'C83', 'C84', 'C79', 'A39', 'B16', 'A94'}\\ \tt \\ \tt ARGUMENT: {'Your set is very strong and comfortably within the limit. To fine-tune while staying under the cap, I suggest a single swap: replace B95 with A94. This keeps the overall structure you prefer, remains within the effort limit (total effort 10057 ≤ 10073), and strengthens overall balance. If you need B95 specifically, I can consider reverting, but this tweak feels like a clean improvement without expanding the set.'} 
	  } 
	   } 
	   } 
	 & \\ 
 

    \theutterance \stepcounter{utterance}  

    & & & \multicolumn{4}{p{0.6\linewidth}}{\cellcolor[rgb]{1,0.85,0.72}{%
	\makecell[{{p{\linewidth}}}]{% 
	  \tt {\tiny [GM$\langle$B]}  
	 STRATEGIC REASONING: {'Their counter swaps in A94 (950 effort) for B95 (800), increasing effort by 150 but still under the limit (10057). For my values, A94 (1049) is higher than B95 (899), so this swap increases my total importance by +150, which is a good trade. The rest of the set matches my previous proposal, which I already optimized to include many of my top items and efficient smalls. Since the effort is within 10073 and my value improves, I should accept to close the deal promptly. I must include ARGUMENT at least once and then AGREE to finalize.'}\\ \tt \\ \tt ARGUMENT: {'Your swap works for me—it stays under the limit and improves overall value while preserving the strong core we aligned on.'}\\ \tt \\ \tt AGREE: {'A87', 'A75', 'A08', 'C53', 'C51', 'C17', 'B69', 'A96', 'C60', 'A88', 'C10', 'A97', 'C83', 'C84', 'C79', 'A39', 'B16', 'A94'} 
	  } 
	   } 
	   } 
	  \\ 
 

    \theutterance \stepcounter{utterance}  

    & & & \multicolumn{2}{p{0.3\linewidth}}{\cellcolor[rgb]{0.95,0.95,0.95}{%
	\makecell[{{p{\linewidth}}}]{% 
	  \tt {\tiny [GM$|$GM]}  
	 game successful 
	  } 
	   } 
	   } 
	 & & \\ 
 

    \theutterance \stepcounter{utterance}  

    & & & \multicolumn{2}{p{0.3\linewidth}}{\cellcolor[rgb]{0.95,0.95,0.95}{%
	\makecell[{{p{\linewidth}}}]{% 
	  \tt {\tiny [GM$|$GM]}  
	 end game 
	  } 
	   } 
	   } 
	 & & \\ 
 

\end{supertabular}
}

\end{document}
