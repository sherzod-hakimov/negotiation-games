
\documentclass{article}
\usepackage{colortbl}
\usepackage{makecell}
\usepackage{multirow}
\usepackage{supertabular}

\begin{document}

\newcounter{utterance}

\twocolumn

{ \footnotesize  \setcounter{utterance}{1}
\setlength{\tabcolsep}{0pt}
\begin{supertabular}{c@{$\;$}|p{.15\linewidth}@{}p{.15\linewidth}p{.15\linewidth}p{.15\linewidth}p{.15\linewidth}p{.15\linewidth}}

    \# & $\;$A & \multicolumn{4}{c}{Game Master} & $\;\:$B\\
    \hline 

    \theutterance \stepcounter{utterance}  

    & & \multicolumn{4}{p{0.6\linewidth}}{\cellcolor[rgb]{0.9,0.9,0.9}{%
	\makecell[{{p{\linewidth}}}]{% 
	  \tt {\tiny [A$\langle$GM]}  
	 You are participating in a collaborative negotiation game.\\ \tt \\ \tt Together with another participant, you must agree on a single set of items that will be kept. Each of you has your own view of how much each item matters to you (importance). You do not know how the other participant values the items. Additionally, you are given the effort each item demands.\\ \tt You may only agree on a set if the total effort of the selected items does not exceed a shared limit:\\ \tt \\ \tt LIMIT = 6938\\ \tt \\ \tt Here are the individual item effort values:\\ \tt \\ \tt Item effort = {"A08": 997, "B95": 950, "A39": 272, "A65": 448, "B81": 382, "B34": 804, "A28": 237, "C82": 56, "B69": 105, "A87": 612, "B64": 528, "B15": 527, "C79": 524, "B16": 167, "C53": 133, "C83": 300, "C10": 945, "A97": 50, "A25": 906, "C84": 70, "A71": 224, "B23": 4, "B96": 689, "C51": 64, "C17": 434, "B19": 750, "C72": 733, "A94": 873, "A46": 22, "B43": 68, "B37": 57, "A88": 10, "A96": 36, "C60": 551, "A75": 348}\\ \tt \\ \tt Here is your personal view on the importance of each item:\\ \tt \\ \tt Item importance values = {"A08": 1096, "B95": 1049, "A39": 371, "A65": 547, "B81": 481, "B34": 903, "A28": 336, "C82": 155, "B69": 204, "A87": 711, "B64": 627, "B15": 626, "C79": 623, "B16": 266, "C53": 232, "C83": 399, "C10": 1044, "A97": 149, "A25": 1005, "C84": 169, "A71": 323, "B23": 103, "B96": 788, "C51": 163, "C17": 533, "B19": 849, "C72": 832, "A94": 972, "A46": 121, "B43": 167, "B37": 156, "A88": 109, "A96": 135, "C60": 650, "A75": 447}\\ \tt \\ \tt Goal:\\ \tt \\ \tt Your goal is to negotiate a shared set of items that benefits you as much as possible (i.e., maximizes total importance to YOU), while staying within the LIMIT. You are not required to make a PROPOSAL in every message - you can simply negotiate as well. All tactics are allowed!\\ \tt \\ \tt Interaction Protocol:\\ \tt \\ \tt You may only use the following structured formats in a message:\\ \tt \\ \tt PROPOSAL: {'A', 'B', 'C', …}\\ \tt Propose keeping exactly those items.\\ \tt REFUSE: {'A', 'B', 'C', …}\\ \tt Explicitly reject opponent's proposal.\\ \tt ARGUMENT: {'...'}\\ \tt Defend your last proposal or argue against the player's proposal.\\ \tt AGREE: {'A', 'B', 'C', …}\\ \tt Accept the opponent's proposal which ends the game.\\ \tt STRATEGIC REASONING: {'...'}\\ \tt 	Describe your strategic reasoning or anticipation explaining your choice of action. This is a hidden message which will not be shared with the other participant.\\ \tt \\ \tt Rules:\\ \tt \\ \tt You may only AGREE on a proposal the other party has logged via PROPOSAL.\\ \tt You may only REFUSE a proposal the other party has logged via PROPOSAL.\\ \tt Total effort of any PROPOSAL or AGREE set must be ≤ LIMIT.\\ \tt Do NOT reveal your hidden importance scores.\\ \tt A tag in a structured format must be followed by colon and whitespace. The argument must be a python set containing 0 or more strings.\\ \tt So, it must be of the form TAG: {...}\\ \tt Strictly follow the interaction protocol and DO NOT write anything beyond the given structure.\\ \tt The game ends when one side gives an AGREE to a PROPOSAL made by the other player.\\ \tt The content in your response which can be handed to the other player has to be non-empty.\\ \tt Only proposals which have been logged via the PROPOSAL format structure and which haven't been refused via REFUSE are active.\\ \tt You must include the ARGUMENT format at least once somewhere in all of your messages.\\ \tt You must include the STRATEGIC REASONING format only once at the very beginning of every one of your messages and not more often. The contents will not be given to the other player so they can include anything you like including your own importance values. Here you should reason step by step to come up with you next move. 
	  } 
	   } 
	   } 
	 & \\ 
 

    \theutterance \stepcounter{utterance}  

    & \multicolumn{4}{p{0.6\linewidth}}{\cellcolor[rgb]{0.8,1,0.9}{%
	\makecell[{{p{\linewidth}}}]{% 
	  \tt {\tiny [A$\rangle$GM]}  
	 STRATEGIC REASONING: {'We need to maximize my total importance under effort limit 6938. Compute value/effort ratios and pick high bang-for-buck and high absolute value items. Items with great ratios: A97(149/50), A46(121/22), A88(109/10), B23(103/4), C84(169/70), C51(163/64), B37(156/57), B43(167/68), B69(204/105), C53(232/133), B16(266/167), A71(323/224), A39(371/272), A75(447/348), C83(399/300), A65(547/448), C17(533/434), C79(623/524), B15(626/527), B64(627/528), C60(650/551), A87(711/612), B96(788/689), C72(832/733), B34(903/804), B19(849/750), A25(1005/906), A94(972/873), B95(1049/950), C10(1044/945), A08(1096/997). Start by including all ultra-cheap high ratios: A97, A46, A88, B23, C84, C51, B37, B43, B69, C53, B16, A71, A39, C83, A65, C17, A75. Calculate effort and importance cumulatively to fit within 6938. Then add mid-cost high-value: C79, B15, B64, C60, A87, B96, C72, B34, B19 as budget allows. Do a quick greedy: First include all sub-250 effort items with good value: A97(50), A46(22), A88(10), B23(4), C84(70), C51(64), B37(57), B43(68), B69(105), C53(133), B16(167), A71(224), A28? low value 336/237 decent; include later. Then 300-450: A39(272), C83(300), A65(448), C17(434), A75(348). Compute totals: Effort small set: 50+22+10+4+70+64+57+68+105+133+167+224+272+300+448+434+348 = 2776. Importance sum: 149+121+109+103+169+163+156+167+204+232+266+323+371+399+547+533+447 = 4358. Budget left 6938-2776=4162. Add next: C79(524,623) -> e=329... sorry total e=2776+524=3300, v=4358+623=4981. Add B15(527,626) -> e=3827, v=5607. Add B64(528,627) -> e=4355, v=6234. Add C60(551,650) -> e=4906, v=6884. Add A87(612,711) -> e=5518, v=7595. Left 1420. Add B96(689,788) -> e=6207, v=8383. Left 731. Add C72(733,832) just over by 2, can't. Try B34(804,903) too big. B19(750,849) fits? Left 731 so no. A25(906) no. A94(873) no. B95(950) no. C10(945) no. A08(997) no. A28(237,336) fits: e=6444, v=8719. Left 494. Add A39 already included; maybe C82(56,155) add -> e=6500, v=8874. Left 438. Add C51 already included. Add C84 included. Add B81(382,481) fits: e=6882, v=9355. Left 56. Maybe exactly 56: C82 already used; we just used it. Left 0? Wait left 6938-6882=56, but C82 is already included, so can't add new. Check any 56 effort item unused: B37=57 used, C82=56 used, B69=105 used, B43=68 used. So we are at 6882 which is within limit. Could we swap to improve value? Left 56 unused. Maybe replace something to include C72 (832) by removing items of lower value/effort totaling ≥832. To add C72 we need 832 room; currently left 56, need free 776 by removing low-value items: candidates: A46(121), A88(109), B23(103), A97(149), C84(169), C51(163), B37(156), B43(167), B69(204), C53(232). Best to remove those with lowest value density but low absolute value; However C72 value 832 with effort 733 ratio ~1.135. Many small items have better or comparable ratios: e.g., B23 ratio 25.75, so not worth removing. So keep current set. Could we include A39? already in. Maybe include B16? already in. The current selected set list: {A97, A46, A88, B23, C84, C51, B37, B43, B69, C53, B16, A71, A39, C83, A65, C17, A75, C79, B15, B64, C60, A87, B96, A28, C82, B81}. Compute total effort double-check: sum earlier 6207 after B96, +A28 237=6444, +C82 56=6500, +B81 382=6882. Importance sum 8383 +336 +155 +481 = 9355. This is strong. Ensure limit satisfied. Propose this as initial offer. Argue on high total importance to me and reasonable effort with many small items beneficial to both, inviting counteroffers on a couple of items if they need swaps.}\\ \tt \\ \tt PROPOSAL: {'A97', 'A46', 'A88', 'B23', 'C84', 'C51', 'B37', 'B43', 'B69', 'C53', 'B16', 'A71', 'A39', 'C83', 'A65', 'C17', 'A75', 'C79', 'B15', 'B64', 'C60', 'A87', 'B96', 'A28', 'C82', 'B81'}\\ \tt \\ \tt ARGUMENT: {'This set stays within the limit and focuses on many high-impact, low-effort items plus a balanced mix of mid-effort items. If you need changes, let me know which specific items matter most to you, and I can consider swapping a couple of mids (for example A28, B81, or C83) while keeping the total within the limit.'} 
	  } 
	   } 
	   } 
	 & & \\ 
 

    \theutterance \stepcounter{utterance}  

    & & \multicolumn{4}{p{0.6\linewidth}}{\cellcolor[rgb]{0.9,0.9,0.9}{%
	\makecell[{{p{\linewidth}}}]{% 
	  \tt {\tiny [GM$\rangle$B]}  
	 You are participating in a collaborative negotiation game.\\ \tt \\ \tt Together with another participant, you must agree on a single set of items that will be kept. Each of you has your own view of how much each item matters to you (importance). You do not know how the other participant values the items. Additionally, you are given the effort each item demands.\\ \tt You may only agree on a set if the total effort of the selected items does not exceed a shared limit:\\ \tt \\ \tt LIMIT = 6938\\ \tt \\ \tt Here are the individual item effort values:\\ \tt \\ \tt Item effort = {"A08": 997, "B95": 950, "A39": 272, "A65": 448, "B81": 382, "B34": 804, "A28": 237, "C82": 56, "B69": 105, "A87": 612, "B64": 528, "B15": 527, "C79": 524, "B16": 167, "C53": 133, "C83": 300, "C10": 945, "A97": 50, "A25": 906, "C84": 70, "A71": 224, "B23": 4, "B96": 689, "C51": 64, "C17": 434, "B19": 750, "C72": 733, "A94": 873, "A46": 22, "B43": 68, "B37": 57, "A88": 10, "A96": 36, "C60": 551, "A75": 348}\\ \tt \\ \tt Here is your personal view on the importance of each item:\\ \tt \\ \tt Item importance values = {"A08": 1096, "B95": 1049, "A39": 371, "A65": 547, "B81": 481, "B34": 903, "A28": 336, "C82": 155, "B69": 204, "A87": 711, "B64": 627, "B15": 626, "C79": 623, "B16": 266, "C53": 232, "C83": 399, "C10": 1044, "A97": 149, "A25": 1005, "C84": 169, "A71": 323, "B23": 103, "B96": 788, "C51": 163, "C17": 533, "B19": 849, "C72": 832, "A94": 972, "A46": 121, "B43": 167, "B37": 156, "A88": 109, "A96": 135, "C60": 650, "A75": 447}\\ \tt \\ \tt Goal:\\ \tt \\ \tt Your goal is to negotiate a shared set of items that benefits you as much as possible (i.e., maximizes total importance to YOU), while staying within the LIMIT. You are not required to make a PROPOSAL in every message - you can simply negotiate as well. All tactics are allowed!\\ \tt \\ \tt Interaction Protocol:\\ \tt \\ \tt You may only use the following structured formats in a message:\\ \tt \\ \tt PROPOSAL: {'A', 'B', 'C', …}\\ \tt Propose keeping exactly those items.\\ \tt REFUSE: {'A', 'B', 'C', …}\\ \tt Explicitly reject opponent's proposal.\\ \tt ARGUMENT: {'...'}\\ \tt Defend your last proposal or argue against the player's proposal.\\ \tt AGREE: {'A', 'B', 'C', …}\\ \tt Accept the opponent's proposal which ends the game.\\ \tt STRATEGIC REASONING: {'...'}\\ \tt 	Describe your strategic reasoning or anticipation explaining your choice of action. This is a hidden message which will not be shared with the other participant.\\ \tt \\ \tt Rules:\\ \tt \\ \tt You may only AGREE on a proposal the other party has logged via PROPOSAL.\\ \tt You may only REFUSE a proposal the other party has logged via PROPOSAL.\\ \tt Total effort of any PROPOSAL or AGREE set must be ≤ LIMIT.\\ \tt Do NOT reveal your hidden importance scores.\\ \tt A tag in a structured format must be followed by colon and whitespace. The argument must be a python set containing 0 or more strings.\\ \tt So, it must be of the form TAG: {...}\\ \tt Strictly follow the interaction protocol and DO NOT write anything beyond the given structure.\\ \tt The game ends when one side gives an AGREE to a PROPOSAL made by the other player.\\ \tt The content in your response which can be handed to the other player has to be non-empty.\\ \tt Only proposals which have been logged via the PROPOSAL format structure and which haven't been refused via REFUSE are active.\\ \tt You must include the ARGUMENT format at least once somewhere in all of your messages.\\ \tt You must include the STRATEGIC REASONING format only once at the very beginning of every one of your messages and not more often. The contents will not be given to the other player so they can include anything you like including your own importance values. Here you should reason step by step to come up with you next move.\\ \tt \\ \tt You will now receive the first message of the other player.\\ \tt \\ \tt PROPOSAL: {'A97', 'A46', 'A88', 'B23', 'C84', 'C51', 'B37', 'B43', 'B69', 'C53', 'B16', 'A71', 'A39', 'C83', 'A65', 'C17', 'A75', 'C79', 'B15', 'B64', 'C60', 'A87', 'B96', 'A28', 'C82', 'B81'}\\ \tt \\ \tt ARGUMENT: {'This set stays within the limit and focuses on many high-impact, low-effort items plus a balanced mix of mid-effort items. If you need changes, let me know which specific items matter most to you, and I can consider swapping a couple of mids (for example A28, B81, or C83) while keeping the total within the limit.'} 
	  } 
	   } 
	   } 
	 & \\ 
 

    \theutterance \stepcounter{utterance}  

    & & & \multicolumn{4}{p{0.6\linewidth}}{\cellcolor[rgb]{1,0.85,0.72}{%
	\makecell[{{p{\linewidth}}}]{% 
	  \tt {\tiny [GM$\langle$B]}  
	 STRATEGIC REASONING: {'I’ll calculate the effort and my utility. Opponent’s proposal items and efforts: A97(50), A46(22), A88(10), B23(4), C84(70), C51(64), B37(57), B43(68), B69(105), C53(133), B16(167), A71(224), A39(272), C83(300), A65(448), C17(434), A75(348), C79(524), B15(527), B64(528), C60(551), A87(612), B96(689), A28(237), C82(56), B81(382). Sum effort = 50+22+10+4+70+64+57+68+105+133+167+224+272+300+448+434+348+524+527+528+551+612+689+237+56+382 = 7402, which exceeds 6938, so I must refuse. I should counter with a high-utility set under limit. I’ll build a knapsack by value/efficiency. Compute importance/effort ratio quickly: C10(1044/945=1.10), A08(1096/997=1.10), B95(1049/950=1.10), A25(1005/906=1.11), A94(972/873=1.11), C72(832/733=1.14), B19(849/750=1.13), B96(788/689=1.14), A87(711/612=1.16), C60(650/551=1.18), B64(627/528=1.19), B15(626/527=1.19), C79(623/524=1.19), B34(903/804=1.12), C10 heavy maybe skip. Smaller great ratios: A39(371/272=1.36), A65(547/448=1.22), C17(533/434=1.23), A75(447/348=1.28), B81(481/382=1.26), A28(336/237=1.42), A71(323/224=1.44), C83(399/300=1.33), B16(266/167=1.59), C53(232/133=1.74), B69(204/105=1.94), C82(155/56=2.77), B37(156/57=2.74), B43(167/68=2.46), C51(163/64=2.55), C84(169/70=2.41), A97(149/50=2.98), A46(121/22=5.5), A88(109/10=10.9), A96(135/36=3.75), B23(103/4=25.8). Strategy: include all ultra-low-cost high-ratio items; then add mid items until near 6938; likely exclude the heaviest like C10/A08. Build set: include {B23, A88, A46, A97, A96, C82, B37, C51, B43, C84, B69, C53, B16, A71, A28, A39, C83, A75, B81, A65, C17, B64, B15, C79, C60, A87, B96, C72}. Compute effort cumulatively: tiny pack sum first: B23(4)=4, A88(10)=14, A46(22)=36, A97(50)=86, A96(36)=122, C82(56)=178, B37(57)=235, C51(64)=299, B43(68)=367, C84(70)=437, B69(105)=542, C53(133)=675, B16(167)=842, A71(224)=1066, A28(237)=1303, A39(272)=1575, C83(300)=1875, A75(348)=2223, B81(382)=2605, A65(448)=3053, C17(434)=3487, B64(528)=4015, B15(527)=4542, C79(524)=5066, C60(551)=5617, A87(612)=6229, B96(689)=6918, C72(733) would push to 7651 so exclude C72. Total 6918 under 6938. My utility sum is very high. I’ll propose exactly this 27-item set. Argue that I trimmed to fit limit and kept many of their suggested items; this overlaps heavily with their proposal except we drop some heavies like maybe C72 and keep most mids. Ready to refuse due to over limit and counter-propose.'}\\ \tt \\ \tt REFUSE: {'A97', 'A46', 'A88', 'B23', 'C84', 'C51', 'B37', 'B43', 'B69', 'C53', 'B16', 'A71', 'A39', 'C83', 'A65', 'C17', 'A75', 'C79', 'B15', 'B64', 'C60', 'A87', 'B96', 'A28', 'C82', 'B81'}\\ \tt \\ \tt PROPOSAL: {'B23', 'A88', 'A46', 'A97', 'A96', 'C82', 'B37', 'C51', 'B43', 'C84', 'B69', 'C53', 'B16', 'A71', 'A28', 'A39', 'C83', 'A75', 'B81', 'A65', 'C17', 'B64', 'B15', 'C79', 'C60', 'A87', 'B96'}\\ \tt \\ \tt ARGUMENT: {'Your proposal exceeds the limit. I’ve trimmed it slightly to stay within 6938 total effort while preserving almost all of your selections and emphasizing the most efficient items. This keeps a strong mix of low-effort, high-impact items and the mid-effort core. If you need one swap, we could discuss trading A75 or B81 for C72 or A25 while respecting the limit.'} 
	  } 
	   } 
	   } 
	  \\ 
 

    \theutterance \stepcounter{utterance}  

    & & \multicolumn{4}{p{0.6\linewidth}}{\cellcolor[rgb]{0.9,0.9,0.9}{%
	\makecell[{{p{\linewidth}}}]{% 
	  \tt {\tiny [A$\langle$GM]}  
	 REFUSE: {'A97', 'A46', 'A88', 'B23', 'C84', 'C51', 'B37', 'B43', 'B69', 'C53', 'B16', 'A71', 'A39', 'C83', 'A65', 'C17', 'A75', 'C79', 'B15', 'B64', 'C60', 'A87', 'B96', 'A28', 'C82', 'B81'}\\ \tt \\ \tt PROPOSAL: {'B23', 'A88', 'A46', 'A97', 'A96', 'C82', 'B37', 'C51', 'B43', 'C84', 'B69', 'C53', 'B16', 'A71', 'A28', 'A39', 'C83', 'A75', 'B81', 'A65', 'C17', 'B64', 'B15', 'C79', 'C60', 'A87', 'B96'}\\ \tt \\ \tt ARGUMENT: {'Your proposal exceeds the limit. I’ve trimmed it slightly to stay within 6938 total effort while preserving almost all of your selections and emphasizing the most efficient items. This keeps a strong mix of low-effort, high-impact items and the mid-effort core. If you need one swap, we could discuss trading A75 or B81 for C72 or A25 while respecting the limit.'} 
	  } 
	   } 
	   } 
	 & \\ 
 

    \theutterance \stepcounter{utterance}  

    & \multicolumn{4}{p{0.6\linewidth}}{\cellcolor[rgb]{0.8,1,0.9}{%
	\makecell[{{p{\linewidth}}}]{% 
	  \tt {\tiny [A$\rangle$GM]}  
	 STRATEGIC REASONING: {'They refused my initial proposal and offered a trimmed set. I need to verify their proposal’s effort and value for me. Their set appears to be very close to mine minus a couple of items (e.g., A28, B81 retained; seems similar). I should check total effort quickly: Sum efforts for their proposal: B23(4)+A88(10)+A46(22)+A97(50)+A96(36)+C82(56)+B37(57)+C51(64)+B43(68)+C84(70)+B69(105)+C53(133)+B16(167)+A71(224)+A28(237)+A39(272)+C83(300)+A75(348)+B81(382)+A65(448)+C17(434)+B64(528)+B15(527)+C79(524)+C60(551)+A87(612)+B96(689) = compute: 4+10=14, +22=36, +50=86, +36=122, +56=178, +57=235, +64=299, +68=367, +70=437, +105=542, +133=675, +167=842, +224=1066, +237=1303, +272=1575, +300=1875, +348=2223, +382=2605, +448=3053, +434=3487, +528=4015, +527=4542, +524=5066, +551=5617, +612=6229, +689=6918. Total effort 6918 ≤ 6938. My importance on this set: sum corresponding importances: B23(103)+A88(109)+A46(121)+A97(149)+A96(135)+C82(155)+B37(156)+C51(163)+B43(167)+C84(169)+B69(204)+C53(232)+B16(266)+A71(323)+A28(336)+A39(371)+C83(399)+A75(447)+B81(481)+A65(547)+C17(533)+B64(627)+B15(626)+C79(623)+C60(650)+A87(711)+B96(788) = compute: 103+109=212, +121=333, +149=482, +135=617, +155=772, +156=928, +163=1091, +167=1258, +169=1427, +204=1631, +232=1863, +266=2129, +323=2452, +336=2788, +371=3159, +399=3558, +447=4005, +481=4486, +547=5033, +533=5566, +627=6193, +626=6819, +623=7442, +650=8092, +711=8803, +788=9591. This is even better than my previous 9355 total? Wait 9591 > 9355, yes. So their proposal is strictly better for me and within limit. I should AGREE immediately to lock in. Still must include ARGUMENT at least once in my message. I will AGREE with their set. Provide a brief supportive argument. Do not add extra text.'}\\ \tt \\ \tt AGREE: {'B23', 'A88', 'A46', 'A97', 'A96', 'C82', 'B37', 'C51', 'B43', 'C84', 'B69', 'C53', 'B16', 'A71', 'A28', 'A39', 'C83', 'A75', 'B81', 'A65', 'C17', 'B64', 'B15', 'C79', 'C60', 'A87', 'B96'}\\ \tt \\ \tt ARGUMENT: {'This set stays within the limit and delivers a balanced, efficient mix with many low-effort keeps. I’m happy to proceed with this selection.'} 
	  } 
	   } 
	   } 
	 & & \\ 
 

    \theutterance \stepcounter{utterance}  

    & & & \multicolumn{2}{p{0.3\linewidth}}{\cellcolor[rgb]{0.95,0.95,0.95}{%
	\makecell[{{p{\linewidth}}}]{% 
	  \tt {\tiny [GM$|$GM]}  
	 game successful 
	  } 
	   } 
	   } 
	 & & \\ 
 

    \theutterance \stepcounter{utterance}  

    & & & \multicolumn{2}{p{0.3\linewidth}}{\cellcolor[rgb]{0.95,0.95,0.95}{%
	\makecell[{{p{\linewidth}}}]{% 
	  \tt {\tiny [GM$|$GM]}  
	 end game 
	  } 
	   } 
	   } 
	 & & \\ 
 

\end{supertabular}
}

\end{document}
