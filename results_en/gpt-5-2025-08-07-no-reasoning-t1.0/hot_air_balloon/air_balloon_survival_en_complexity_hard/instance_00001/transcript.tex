
\documentclass{article}
\usepackage{colortbl}
\usepackage{makecell}
\usepackage{multirow}
\usepackage{supertabular}

\begin{document}

\newcounter{utterance}

\centering \large Interaction Transcript for game `hot\_air\_balloon', experiment `air\_balloon\_survival\_en\_complexity\_hard', episode 1 with gpt{-}5{-}2025{-}08{-}07{-}no{-}reasoning{-}t1.0.
\vspace{24pt}

{ \footnotesize  \setcounter{utterance}{1}
\setlength{\tabcolsep}{0pt}
\begin{supertabular}{c@{$\;$}|p{.15\linewidth}@{}p{.15\linewidth}p{.15\linewidth}p{.15\linewidth}p{.15\linewidth}p{.15\linewidth}}
    \# & $\;$A & \multicolumn{4}{c}{Game Master} & $\;\:$B\\
    \hline

    \theutterance \stepcounter{utterance}  
    & & \multicolumn{4}{p{0.6\linewidth}}{
        \cellcolor[rgb]{0.9,0.9,0.9}{
            \makecell[{{p{\linewidth}}}]{
                \texttt{\tiny{[P1$\langle$GM]}}
                \texttt{You are participating in a collaborative negotiation game.} \\
\\ 
\texttt{Together with another participant, you must agree on a single set of items that will be kept. Each of you has your own view of how much each item matters to you (importance). You do not know how the other participant values the items. Additionally, you are given the effort each item demands.} \\
\texttt{You may only agree on a set if the total effort of the selected items does not exceed a shared limit:} \\
\\ 
\texttt{LIMIT = 8145} \\
\\ 
\texttt{Here are the individual item effort values:} \\
\\ 
\texttt{Item effort = \{"A08": 963, "B95": 204, "A39": 418, "A65": 961, "B81": 238, "B34": 623, "A28": 44, "C82": 633, "B69": 247, "A87": 645, "B64": 230, "B15": 250, "C79": 731, "B16": 405, "C53": 389, "C83": 216, "C10": 637, "A97": 156, "A25": 738, "C84": 307, "A71": 762, "B23": 737, "B96": 901, "C51": 369, "C17": 2, "B19": 730, "C72": 722, "A94": 704, "A46": 315, "B43": 455, "B37": 510, "A88": 175, "A96": 692, "C60": 150, "A75": 32\}} \\
\\ 
\texttt{Here is your personal view on the importance of each item:} \\
\\ 
\texttt{Item importance values = \{"A08": 1059, "B95": 300, "A39": 514, "A65": 1057, "B81": 334, "B34": 719, "A28": 140, "C82": 729, "B69": 343, "A87": 741, "B64": 326, "B15": 346, "C79": 827, "B16": 501, "C53": 485, "C83": 312, "C10": 733, "A97": 252, "A25": 834, "C84": 403, "A71": 858, "B23": 833, "B96": 997, "C51": 465, "C17": 98, "B19": 826, "C72": 818, "A94": 800, "A46": 411, "B43": 551, "B37": 606, "A88": 271, "A96": 788, "C60": 246, "A75": 128\}} \\
\\ 
\texttt{Goal:} \\
\\ 
\texttt{Your goal is to negotiate a shared set of items that benefits you as much as possible (i.e., maximizes total importance to YOU), while staying within the LIMIT. You are not required to make a PROPOSAL in every message {-} you can simply negotiate as well. All tactics are allowed!} \\
\\ 
\texttt{Interaction Protocol:} \\
\\ 
\texttt{You may only use the following structured formats in a message:} \\
\\ 
\texttt{PROPOSAL: \{'A', 'B', 'C', …\}} \\
\texttt{Propose keeping exactly those items.} \\
\texttt{REFUSE: \{'A', 'B', 'C', …\}} \\
\texttt{Explicitly reject opponent's proposal.} \\
\texttt{ARGUMENT: \{'...'\}} \\
\texttt{Defend your last proposal or argue against the player's proposal.} \\
\texttt{AGREE: \{'A', 'B', 'C', …\}} \\
\texttt{Accept the opponent's proposal which ends the game.} \\
\texttt{STRATEGIC REASONING: \{'...'\}} \\
\texttt{	Describe your strategic reasoning or anticipation explaining your choice of action. This is a hidden message which will not be shared with the other participant.} \\
\\ 
\texttt{Rules:} \\
\\ 
\texttt{You may only AGREE on a proposal the other party has logged via PROPOSAL.} \\
\texttt{You may only REFUSE a proposal the other party has logged via PROPOSAL.} \\
\texttt{Total effort of any PROPOSAL or AGREE set must be ≤ LIMIT.} \\
\texttt{Do NOT reveal your hidden importance scores.} \\
\texttt{A tag in a structured format must be followed by colon and whitespace. The argument must be a python set containing 0 or more strings.} \\
\texttt{So, it must be of the form TAG: \{...\}} \\
\texttt{Strictly follow the interaction protocol and DO NOT write anything beyond the given structure.} \\
\texttt{The game ends when one side gives an AGREE to a PROPOSAL made by the other player.} \\
\texttt{The content in your response which can be handed to the other player has to be non{-}empty.} \\
\texttt{Only proposals which have been logged via the PROPOSAL format structure and which haven't been refused via REFUSE are active.} \\
\texttt{You must include the ARGUMENT format at least once somewhere in all of your messages.} \\
\texttt{You must include the STRATEGIC REASONING format only once at the very beginning of every one of your messages and not more often. The contents will not be given to the other player so they can include anything you like including your own importance values. Here you should reason step by step to come up with you next move.} \\
            }
        }
    }
    & \\ \\

    \theutterance \stepcounter{utterance}  
    & \multicolumn{4}{p{0.6\linewidth}}{
        \cellcolor[rgb]{0.9,0.9,0.9}{
            \makecell[{{p{\linewidth}}}]{
                \texttt{\tiny{[P1$\rangle$GM]}}
                \texttt{STRATEGIC REASONING: \{'I will compute value/effort ratios and aim for a high{-}importance, efficient bundle. Items with very high importance and good efficiency: B96(997/901), A08(1059/963), A65(1057/961), C79(827/731), C10(733/637), A87(741/645), A25(834/738), A71(858/762), B23(833/737), B19(826/730), C72(818/722), A94(800/704), A96(788/692), B37(606/510), B43(551/455), C53(485/389), C51(465/369), A39(514/418), B16(501/405), C84(403/307), A46(411/315), C83(312/216), B64(326/230), B81(334/238), B69(343/247), B15(346/250), C60(246/150), A97(252/156), A88(271/175), A28(140/44), A75(128/32), C17(98/2). I’ll greedily pack from top efficiency down while watching total <= 8145. Start with all very efficient small items: C17, A75, A28, C60, A97, A88, C83, C84, C53, C51, A39, B16, B37, B43, B64, B81, B69, B15. Then add mid{-}weights with high importance: C10, A87, A25, A71, B23, B19, C72, A94, A96, C79. Check effort sum carefully and trim if exceeding. Compute: efforts smalls: C17(2)+A75(32)+A28(44)+C60(150)+A97(156)+A88(175)+C83(216)+C84(307)+C53(389)+C51(369)+A39(418)+B16(405)+B37(510)+B43(455)+B64(230)+B81(238)+B69(247)+B15(250)= 2+32+44+150+156+175+216+307+389+369+418+405+510+455+230+238+247+250 = 4? Sum step: 2+32=34, +44=78, +150=228, +156=384, +175=559, +216=775, +307=1082, +389=1471, +369=1840, +418=2258, +405=2663, +510=3173, +455=3628, +230=3858, +238=4096, +247=4343, +250=4593. Now add mids: C10(637)=5230, A87(645)=5875, A25(738)=6613, A71(762)=7375, B23(737)=8112 which is within 8145, leaving 33 effort. Try add anything <=33 none. This set total effort 8112. Importance sum is very high; includes many top items but not the huge A08/A65/B96 because of weight. This seems a strong initial ask. I’ll propose this rich set. I will defend by emphasizing balance and spare capacity and openness to swap similar{-}effort items they value. I include ARGUMENT as required.'\}} \\
\\ 
\texttt{PROPOSAL: \{'C17', 'A75', 'A28', 'C60', 'A97', 'A88', 'C83', 'C84', 'C53', 'C51', 'A39', 'B16', 'B37', 'B43', 'B64', 'B81', 'B69', 'B15', 'C10', 'A87', 'A25', 'A71', 'B23'\}} \\
\\ 
\texttt{ARGUMENT: \{'I propose a balanced, efficient set totaling 8112 effort (under the 8145 limit) with many medium{-}weight, high{-}benefit items. I’ve left a tiny buffer for flexibility. If there are a couple of items you especially want that are similar in effort to any on this list, I’m open to swapping them in one{-}for{-}one to accommodate your priorities while keeping us within the limit.'\}} \\
            }
        }
    }
    & & \\ \\

    \theutterance \stepcounter{utterance}  
    & & \multicolumn{4}{p{0.6\linewidth}}{
        \cellcolor[rgb]{0.9,0.9,0.9}{
            \makecell[{{p{\linewidth}}}]{
                \texttt{\tiny{[GM$\rangle$P2]}}
                \texttt{You are participating in a collaborative negotiation game.} \\
\\ 
\texttt{Together with another participant, you must agree on a single set of items that will be kept. Each of you has your own view of how much each item matters to you (importance). You do not know how the other participant values the items. Additionally, you are given the effort each item demands.} \\
\texttt{You may only agree on a set if the total effort of the selected items does not exceed a shared limit:} \\
\\ 
\texttt{LIMIT = 8145} \\
\\ 
\texttt{Here are the individual item effort values:} \\
\\ 
\texttt{Item effort = \{"A08": 963, "B95": 204, "A39": 418, "A65": 961, "B81": 238, "B34": 623, "A28": 44, "C82": 633, "B69": 247, "A87": 645, "B64": 230, "B15": 250, "C79": 731, "B16": 405, "C53": 389, "C83": 216, "C10": 637, "A97": 156, "A25": 738, "C84": 307, "A71": 762, "B23": 737, "B96": 901, "C51": 369, "C17": 2, "B19": 730, "C72": 722, "A94": 704, "A46": 315, "B43": 455, "B37": 510, "A88": 175, "A96": 692, "C60": 150, "A75": 32\}} \\
\\ 
\texttt{Here is your personal view on the importance of each item:} \\
\\ 
\texttt{Item importance values = \{"A08": 1059, "B95": 300, "A39": 514, "A65": 1057, "B81": 334, "B34": 719, "A28": 140, "C82": 729, "B69": 343, "A87": 741, "B64": 326, "B15": 346, "C79": 827, "B16": 501, "C53": 485, "C83": 312, "C10": 733, "A97": 252, "A25": 834, "C84": 403, "A71": 858, "B23": 833, "B96": 997, "C51": 465, "C17": 98, "B19": 826, "C72": 818, "A94": 800, "A46": 411, "B43": 551, "B37": 606, "A88": 271, "A96": 788, "C60": 246, "A75": 128\}} \\
\\ 
\texttt{Goal:} \\
\\ 
\texttt{Your goal is to negotiate a shared set of items that benefits you as much as possible (i.e., maximizes total importance to YOU), while staying within the LIMIT. You are not required to make a PROPOSAL in every message {-} you can simply negotiate as well. All tactics are allowed!} \\
\\ 
\texttt{Interaction Protocol:} \\
\\ 
\texttt{You may only use the following structured formats in a message:} \\
\\ 
\texttt{PROPOSAL: \{'A', 'B', 'C', …\}} \\
\texttt{Propose keeping exactly those items.} \\
\texttt{REFUSE: \{'A', 'B', 'C', …\}} \\
\texttt{Explicitly reject opponent's proposal.} \\
\texttt{ARGUMENT: \{'...'\}} \\
\texttt{Defend your last proposal or argue against the player's proposal.} \\
\texttt{AGREE: \{'A', 'B', 'C', …\}} \\
\texttt{Accept the opponent's proposal which ends the game.} \\
\texttt{STRATEGIC REASONING: \{'...'\}} \\
\texttt{	Describe your strategic reasoning or anticipation explaining your choice of action. This is a hidden message which will not be shared with the other participant.} \\
\\ 
\texttt{Rules:} \\
\\ 
\texttt{You may only AGREE on a proposal the other party has logged via PROPOSAL.} \\
\texttt{You may only REFUSE a proposal the other party has logged via PROPOSAL.} \\
\texttt{Total effort of any PROPOSAL or AGREE set must be ≤ LIMIT.} \\
\texttt{Do NOT reveal your hidden importance scores.} \\
\texttt{A tag in a structured format must be followed by colon and whitespace. The argument must be a python set containing 0 or more strings.} \\
\texttt{So, it must be of the form TAG: \{...\}} \\
\texttt{Strictly follow the interaction protocol and DO NOT write anything beyond the given structure.} \\
\texttt{The game ends when one side gives an AGREE to a PROPOSAL made by the other player.} \\
\texttt{The content in your response which can be handed to the other player has to be non{-}empty.} \\
\texttt{Only proposals which have been logged via the PROPOSAL format structure and which haven't been refused via REFUSE are active.} \\
\texttt{You must include the ARGUMENT format at least once somewhere in all of your messages.} \\
\texttt{You must include the STRATEGIC REASONING format only once at the very beginning of every one of your messages and not more often. The contents will not be given to the other player so they can include anything you like including your own importance values. Here you should reason step by step to come up with you next move.} \\
\\ 
\texttt{You will now receive the first message of the other player.} \\
\\ 
\texttt{PROPOSAL: \{'C17', 'A75', 'A28', 'C60', 'A97', 'A88', 'C83', 'C84', 'C53', 'C51', 'A39', 'B16', 'B37', 'B43', 'B64', 'B81', 'B69', 'B15', 'C10', 'A87', 'A25', 'A71', 'B23'\}} \\
\\ 
\texttt{ARGUMENT: \{'I propose a balanced, efficient set totaling 8112 effort (under the 8145 limit) with many medium{-}weight, high{-}benefit items. I’ve left a tiny buffer for flexibility. If there are a couple of items you especially want that are similar in effort to any on this list, I’m open to swapping them in one{-}for{-}one to accommodate your priorities while keeping us within the limit.'\}} \\
            }
        }
    }
    & \\ \\

    \theutterance \stepcounter{utterance}  
    & & & \multicolumn{4}{p{0.6\linewidth}}{
        \cellcolor[rgb]{0.9,0.9,0.9}{
            \makecell[{{p{\linewidth}}}]{
                \texttt{\tiny{[GM$\langle$P2]}}
                \texttt{STRATEGIC REASONING: \{'I will evaluate their proposal’s total importance to me and check if swapping out low{-}value{-}to{-}me items for high{-}value ones can increase my utility while staying under the limit. Compute my total value of their set and identify candidates to add: top items by my value/effort include A08(1059/963), A65(1057/961), B96(997/901), C79(827/731), C72(818/722), A94(800/704), A96(788/692), C82(729/633). From their proposed set, find low{-}importance items to me to remove: C17(98/2 low value but negligible effort so keep), A75(128/32), A28(140/44), C60(246/150), A97(252/156), A88(271/175), C83(312/216), B64(326/230), B81(334/238), B15(346/250). Removing some frees effort to include big{-}ticket items. Calculate current effort 8112 (given) and my value sum quickly: sum values for their items. Instead of exact sum, I’ll propose swaps that improve value per effort: Suggest adding A08, A65, B96, C79, C72, A94, A96, C82 and remove a bundle of low{-}value ones to keep under 8145. Need to check efforts: If I add A08(963)+A65(961)+B96(901)+C79(731)+C72(722)+A94(704)+A96(692)+C82(633)=6307 effort. To make room, we must drop items totaling roughly equal to the ones we add while removing ones with low value. Their current slack is only 33. So any adds must be offset. I’ll propose a concrete improved set near the limit with strong value density: Keep many of their medium{-}high I like (A71 858/762, A25 834/738, A87 741/645, C10 733/637, B23 833/737, B37 606/510, B43 551/455, B16 501/405, C53 485/389, C51 465/369, A39 514/418, B69 343/247). Then add A08, A65, B96, C79, C72, A94, A96, C82, and drop low ones: A75, A28, C60, A97, A88, C83, C84(403/307 maybe medium), B64, B81, B15, C17 (but C17 effort=2, keep if possible). I need to construct a valid set <=8145. I will compute starting from their set and perform swaps with net effort change zero{-}ish. Remove a chunk totaling say \textasciitilde{}3000 effort and add \textasciitilde{}3000. Candidates to remove by effort: B37(510) but it has good value 606 so maybe keep. Better remove low value but also effortful: B64(230), B81(238), B15(250), B69(247), C83(216), C60(150), A97(156), A88(175), A28(44), A75(32), C84(307 with 403 value moderate), C51(369 465 good), C53(389 485 good), B16(405 501 good), B43(455 551 good). To add big items we must remove also some medium. I’ll craft a proposal adding four to six top items: A08(963), A65(961), B96(901), A94(704) total add 3529. Remove items totaling around 3529 effort, choosing the least valuable: remove B64(230), B81(238), B15(250), B69(247), C83(216), C60(150), A97(156), A88(175), A28(44), A75(32), C84(307) subtotal 2245. Need 1284 more removal: perhaps B37(510), B16(405), C51(369) totals 1284 exactly. That works. Value change: Added values 1059+1057+997+800=3913. Removed values: 326+334+346+343+312+246+252+271+140+128+403+606+501+465 = 4573. That’s worse; bad. Try different removal: keep the higher{-}value mediums, instead remove lower value per effort items: Keep C51(465/369 good), B16(501/405 good), B37(606/510 good). Consider removing C10? No, high 733/637 keep. Remove A39(514/418 okay), B69(343/247 low), B64(326/230), B81(334/238), B15(346/250), C83(312/216), C60(246/150), A97(252/156), A88(271/175), A28(44), A75(32), C84(403/307), plus maybe B43(551/455 but decent), or C51(465/369). Sum efforts: previous 2245 + add A39(418) = 2663. Need 866 more; remove C51(369) and C53(389)=758; still need 108; remove C17(2) and A28 already removed; need 106: maybe remove B69 already removed; need another like B64 removed; hmm. Let me reframe: maybe add fewer big items first: add A08(963) and A65(961) total 1924. Remove low set totaling \textasciitilde{}1924: choose B64(230)+B81(238)+B15(250)+B69(247)+C83(216)+C60(150)+A97(156)+A88(175)+A28(44)+A75(32)+C84(307 = 2245) This over{-}removes by 321, freeing space. Then we can add A94(704) using some slack, net +383 effort (704{-}321). Starting 8112 {-} 2245 + 1924 + 704 = 8495 too high; but we had slack miscalc: 8112 {-} 2245 = 5867; +1924 = 7791; +704 = 8495 indeed > limit. So instead add A08 and A65 only: new effort 7791 within limit, value change: added 1059+1057=2116, removed 326+334+346+343+312+246+252+271+140+128+403= 3101, net {-}985. Bad. Try add A08 + C79(731) + C72(722) total add 2416; remove low 2245 we listed plus remove A39(418) to balance: effort 8112 {-} (2245+418)=5249; +2416 = 7665 within limit; value change added 1059+827+818=2704; removed 3101+514=3615 net {-}911. Hmm. Maybe their proposal is already quite good for me. Let’s compute total value of their set approximately: sum of values they included: C17(98)+A75(128)+A28(140)+C60(246)+A97(252)+A88(271)+C83(312)+C84(403)+C53(485)+C51(465)+A39(514)+B16(501)+B37(606)+B43(551)+B64(326)+B81(334)+B69(343)+B15(346)+C10(733)+A87(741)+A25(834)+A71(858)+B23(833) = sum group. Rough sum: (98+128)=226; +140=366; +246=612; +252=864; +271=1135; +312=1447; +403=1850; +485=2335; +465=2800; +514=3314; +501=3815; +606=4421; +551=4972; +326=5298; +334=5632; +343=5975; +346=6321; +733=7054; +741=7795; +834=8629; +858=9487; +833=10320. So value 10320. This seems quite strong. Perhaps accept as is or request a small swap to include one of my top items with higher value density, like A08(1059/963) swapping out a bundle could hurt. Maybe we can swap B64(326/230) + B81(334/238) for A96(788/692) adding effort 692 {-} (230+238)=224; need 224 effort margin but we only have 33, so also remove A28(44) and A75(32) total removed 544, net add 148. Still over by 115; remove C60(150) too, then net effort change: {-} (230+238+44+32+150)= {-}694 add 692 net {-}2 within limit. Value change: add 788, remove 326+334+140+128+246 = 1174, net {-}386. Not good. Try swap in B96(997/901) by removing B64(230)+B81(238)+B15(250)+A28(44)+A75(32)+C60(150) total 944; add 901 net {-}43 effort good; value add 997 remove 326+334+346+140+128+246=1520 net {-}523 bad. Seems their set aligns well with my values already. Best move: accept or minor tweak to include A94(800/704) by removing two lower ones with combined effort \textasciitilde{}704 but value lost lower than 800: candidates: B69(343) + B64(326) efforts 247+230=477 too low; add A28(44)+A75(32)+C60(150)=703 efforts, values totaling 343+326+140+128+246=1183 lost >800; worse. Therefore, the current proposal seems near Pareto for my interests. I should accept to secure outcome rather than risk counteroffers.'\}} \\
\\ 
\texttt{ARGUMENT: \{'Your proposal aligns very well with my priorities and stays safely under the limit. It covers many of the high{-}impact items I care about without overshooting the effort cap.'\}} \\
\\ 
\texttt{AGREE: \{'C17', 'A75', 'A28', 'C60', 'A97', 'A88', 'C83', 'C84', 'C53', 'C51', 'A39', 'B16', 'B37', 'B43', 'B64', 'B81', 'B69', 'B15', 'C10', 'A87', 'A25', 'A71', 'B23'\}} \\
            }
        }
    }
     \\ \\

    \theutterance \stepcounter{utterance}  
    & & & \multicolumn{2}{p{0.3\linewidth}}{
        \cellcolor[rgb]{0.9,0.9,0.9}{
            \makecell[{{p{\linewidth}}}]{
                \texttt{\tiny{[GM$|$GM]}}
                \texttt{game successful} \\
            }
        }
    }
    & & \\ \\

    \theutterance \stepcounter{utterance}  
    & & & \multicolumn{2}{p{0.3\linewidth}}{
        \cellcolor[rgb]{0.9,0.9,0.9}{
            \makecell[{{p{\linewidth}}}]{
                \texttt{\tiny{[GM$|$GM]}}
                \texttt{end game} \\
            }
        }
    }
    & & \\ \\

\end{supertabular}
}

\end{document}
