
\documentclass{article}
\usepackage{colortbl}
\usepackage{makecell}
\usepackage{multirow}
\usepackage{supertabular}

\begin{document}

\newcounter{utterance}

\twocolumn

{ \footnotesize  \setcounter{utterance}{1}
\setlength{\tabcolsep}{0pt}
\begin{supertabular}{c@{$\;$}|p{.15\linewidth}@{}p{.15\linewidth}p{.15\linewidth}p{.15\linewidth}p{.15\linewidth}p{.15\linewidth}}

    \# & $\;$A & \multicolumn{4}{c}{Game Master} & $\;\:$B\\
    \hline 

    \theutterance \stepcounter{utterance}  

    & & \multicolumn{4}{p{0.6\linewidth}}{\cellcolor[rgb]{0.9,0.9,0.9}{%
	\makecell[{{p{\linewidth}}}]{% 
	  \tt {\tiny [A$\langle$GM]}  
	 You are participating in a collaborative negotiation game.\\ \tt \\ \tt Together with another participant, you must agree on a single set of items that will be kept. Each of you has your own view of how much each item matters to you (importance). You do not know how the other participant values the items. Additionally, you are given the effort each item demands.\\ \tt You may only agree on a set if the total effort of the selected items does not exceed a shared limit:\\ \tt \\ \tt LIMIT = 6880\\ \tt \\ \tt Here are the individual item effort values:\\ \tt \\ \tt Item effort = {"A08": 341, "B95": 804, "A39": 20, "A65": 627, "B81": 10, "B34": 573, "A28": 217, "C82": 481, "B69": 205, "A87": 273, "B64": 303, "B15": 596, "C79": 564, "B16": 535, "C53": 258, "C83": 911, "C10": 240, "A97": 188, "A25": 216, "C84": 401, "A71": 909, "B23": 62, "B96": 245, "C51": 994, "C17": 569, "B19": 718, "C72": 464, "A94": 37, "A46": 340, "B43": 335, "B37": 417, "A88": 123, "A96": 17, "C60": 577, "A75": 190}\\ \tt \\ \tt Here is your personal view on the importance of each item:\\ \tt \\ \tt Item importance values = {"A08": 440, "B95": 903, "A39": 119, "A65": 726, "B81": 109, "B34": 672, "A28": 316, "C82": 580, "B69": 304, "A87": 372, "B64": 402, "B15": 695, "C79": 663, "B16": 634, "C53": 357, "C83": 1010, "C10": 339, "A97": 287, "A25": 315, "C84": 500, "A71": 1008, "B23": 161, "B96": 344, "C51": 1093, "C17": 668, "B19": 817, "C72": 563, "A94": 136, "A46": 439, "B43": 434, "B37": 516, "A88": 222, "A96": 116, "C60": 676, "A75": 289}\\ \tt \\ \tt Goal:\\ \tt \\ \tt Your goal is to negotiate a shared set of items that benefits you as much as possible (i.e., maximizes total importance to YOU), while staying within the LIMIT. You are not required to make a PROPOSAL in every message - you can simply negotiate as well. All tactics are allowed!\\ \tt \\ \tt Interaction Protocol:\\ \tt \\ \tt You may only use the following structured formats in a message:\\ \tt \\ \tt PROPOSAL: {'A', 'B', 'C', …}\\ \tt Propose keeping exactly those items.\\ \tt REFUSE: {'A', 'B', 'C', …}\\ \tt Explicitly reject opponent's proposal.\\ \tt ARGUMENT: {'...'}\\ \tt Defend your last proposal or argue against the player's proposal.\\ \tt AGREE: {'A', 'B', 'C', …}\\ \tt Accept the opponent's proposal which ends the game.\\ \tt STRATEGIC REASONING: {'...'}\\ \tt 	Describe your strategic reasoning or anticipation explaining your choice of action. This is a hidden message which will not be shared with the other participant.\\ \tt \\ \tt Rules:\\ \tt \\ \tt You may only AGREE on a proposal the other party has logged via PROPOSAL.\\ \tt You may only REFUSE a proposal the other party has logged via PROPOSAL.\\ \tt Total effort of any PROPOSAL or AGREE set must be ≤ LIMIT.\\ \tt Do NOT reveal your hidden importance scores.\\ \tt A tag in a structured format must be followed by colon and whitespace. The argument must be a python set containing 0 or more strings.\\ \tt So, it must be of the form TAG: {...}\\ \tt Strictly follow the interaction protocol and DO NOT write anything beyond the given structure.\\ \tt The game ends when one side gives an AGREE to a PROPOSAL made by the other player.\\ \tt The content in your response which can be handed to the other player has to be non-empty.\\ \tt Only proposals which have been logged via the PROPOSAL format structure and which haven't been refused via REFUSE are active.\\ \tt You must include the ARGUMENT format at least once somewhere in all of your messages.\\ \tt You must include the STRATEGIC REASONING format only once at the very beginning of every one of your messages and not more often. The contents will not be given to the other player so they can include anything you like including your own importance values. Here you should reason step by step to come up with you next move. 
	  } 
	   } 
	   } 
	 & \\ 
 

    \theutterance \stepcounter{utterance}  

    & \multicolumn{4}{p{0.6\linewidth}}{\cellcolor[rgb]{0.8,1,0.9}{%
	\makecell[{{p{\linewidth}}}]{% 
	  \tt {\tiny [A$\rangle$GM]}  
	 STRATEGIC REASONING: {'Maximize my value under limit 6880. Compute value/effort ratios and pick high-value items. Top values: C51 1093(994), A71 1008(909), C83 1010(911), B95 903(804), B19 817(718), C60 676(577), C79 663(564), C17 668(569), B34 672(573), B16 634(535), C82 580(481), B37 516(417), C84 500(401), A65 726(627), A08 440(341), A46 439(340), B43 434(335), B15 695(596), B64 402(303), A87 372(273), C72 563(464), C53 357(258), B96 344(245), C10 339(240), A28 316(217), A25 315(216), B69 304(205), A75 289(190), A97 287(188), A88 222(123), A39 119(20), A96 116(17), A94 136(37), B23 161(62), B81 109(10). Build a knapSack heuristic: include highest value with decent ratio and ensure within 6880. Start by considering excluding very heavy with slightly lower ratios? Try a greedy by ratio: compute some ratios quickly: A96 6.82, A39 5.95, B81 10.9, A88 1.80, A94 3.68, B23 2.60, A97 1.53, A75 1.52, B69 1.48, A25 1.46, A28 1.46, B96 1.40, C53 1.38, A87 1.36, B64 1.33, C10 1.41, C84 1.25, C82 1.21, B37 1.24, B43 1.30, A46 1.29, A08 1.29, A65 1.16, C72 1.21, C79 1.18, C60 1.17, C17 1.17, B34 1.17, B16 1.18, B15 1.17, B19 1.14, B95 1.12, A71 1.11, C83 1.11, C51 1.10. Small items with huge ratios should be included. I’ll propose a large set centered on many items to maximize my value while under limit. Build from small high ratios and add big ones while tracking effort. Start with zero effort 0. Add: B81(10) val109 -> e10. A96(17)v116 e27. A39(20)v119 e47. A94(37)v136 e84. B23(62)v161 e146. A88(123)v222 e269. A97(188)v287 e457. A75(190)v289 e647. B69(205)v304 e852. A25(216)v315 e1068. A28(217)v316 e1285. C53(258)v357 e1543. A87(273)v372 e1816. B96(245)v344 e2061 (note order, but ok). C10(240)v339 e2301. B64(303)v402 e2604. B43(335)v434 e2939. A46(340)v439 e3279. A08(341)v440 e3620. C84(401)v500 e4021. B37(417)v516 e4438. C82(481)v580 e4919. C72(464)v563 e5383. Now we have 1497 effort left. Consider medium big: C79(564)v663 -> e5947 left 933. C60(577)v676 would exceed 6880? 5947+577=6524 ok include -> e6524 left 356. With 356 effort we can add items: B34(573) too big, B16(535) too big, B15(596) too big, C17(569) too big, B19(718) too big, B95(804) too big, A65(627) too big, A71(909) too big, C83(911) too big, C51(994) too big. Remaining small candidates not yet included: B64 already, B37 done, C84 done, C82 done, C72 done. Left small: A71 etc no. Check items not included with small effort: A97 done, A75 done. B23 etc done. Maybe A71 no. B95 no. A65 no. B34 no. B16 no. B15 no. C17 no. C60 done. C79 done. C53 done. C10 done. A94 done. A46 done. A08 done. A28 done. A25 done. B69 done. A87 done. B96 done. A65 excluded. We have 356 effort capacity; look for any remaining items with effort <=356 and not yet in set: A71 909 no; C83 911 no; C51 994 no; B95 804 no; B19 718 no; A65 627 no; B34 573 no; C17 569 no; B16 535 no; C60 577 already; C79 564 already; C82 481 already; C84 401 already; B15 596 no; B37 417 already; B64 303 already in; B43 335 in; A46 340 in; A08 341 in; C72 464 in; C53 258 in; C10 240 in; B96 245 in; A87 273 in; B69 205 in; A28 217 in; A25 216 in; A97 188 in; A75 190 in; A88 123 in; A94 37 in; A39 20 in; A96 17 in; B81 10 in; B23 62 in. Maybe we can swap: consider adding B16 (535,634) by dropping something low ratio to free up 179 effort. Current remaining 356; need 179 more by removing items with lowest marginal value/effort: A88(222/123=1.8), good ratio keep. A97 1.53, A75 1.52, B69 1.48, A25 1.46, A28 1.46, B96 1.40, C10 1.41, C53 1.38, A87 1.36, B64 1.33, B43 1.30, A46 1.29, A08 1.29. Removing B43(335) frees 335 effort but loses 434 value; that would allow adding B16 value 634 net +200 value, effort +200? Actually replace: remove B43 335, add B16 535 increases effort by 200 which we can afford since we had 356 spare, leaving 156 spare. Net value +200. Do that. Then with 156 remaining, can we add anything else not in set with effort<=156? Items left small: none besides B23 62 already in, A94 37, A96 17, A39 20, B81 10 already. Maybe add nothing. Consider also swapping A46(340,439) for B34(573,672): effort +233; we have after previous swap remaining 156, not enough. What about swapping A08(341,440) for B34: need +232 effort, not enough. Could drop some small to fit B34: need additional 77 effort beyond 156 -> total 233. Could drop e.g., A39(20) + A96(17) + A94(37) + B81(10) + B23(62)=146 >77; but that loses value 119+116+136+109+161=641 to gain +? Replacing A46(439) and those smalls with B34(672) net might be negative. Probably current set is strong. Consider adding C17(569,668) by replacing something: we have 156 spare, need +413, remove B43 already removed; candidates: maybe drop A97(188), A75(190), A88(123)=501 free, lose value 287+289+222=798 to gain 668 -> net -130. Not good. Consider adding B15(596,695) need +440 effort; could drop B64(303,402) and A97(188,287)=491 free, value loss 689 to gain 695 net +6 slight; effort change -? frees -? it works but complexity and slight gain. Might be okay but also risk opponent. Current total effort compute: Sum after initial then swap. Initial effort e6524. Remove B43 335 -> 6189. Add B16 535 -> 6724. We have 156 spare indeed. Total value initial sum need to compute? Not necessary. This looks near capacity and high value. I will propose this set and argue efficiency and room for their priorities with 156 slack to add a small item they care about by swapping if needed.'}\\ \tt \\ \tt PROPOSAL: {'B81', 'A96', 'A39', 'A94', 'B23', 'A88', 'A97', 'A75', 'B69', 'A25', 'A28', 'C53', 'A87', 'B96', 'C10', 'B64', 'A46', 'A08', 'C84', 'B37', 'C82', 'C72', 'C79', 'C60', 'B16'}\\ \tt \\ \tt ARGUMENT: {'This set stays under the limit (total effort 6724 ≤ 6880) and delivers a balanced, efficient mix: many high-importance items with strong value-to-effort ratios, plus broad coverage across categories. I also left 156 effort slack, so if you have a small but crucial item (effort ≤ 156), we can likely accommodate it with minimal adjustments. If you need to swap in a mid-sized priority, we can discuss targeted trades that keep us within the cap.'} 
	  } 
	   } 
	   } 
	 & & \\ 
 

    \theutterance \stepcounter{utterance}  

    & & \multicolumn{4}{p{0.6\linewidth}}{\cellcolor[rgb]{0.9,0.9,0.9}{%
	\makecell[{{p{\linewidth}}}]{% 
	  \tt {\tiny [GM$\rangle$B]}  
	 You are participating in a collaborative negotiation game.\\ \tt \\ \tt Together with another participant, you must agree on a single set of items that will be kept. Each of you has your own view of how much each item matters to you (importance). You do not know how the other participant values the items. Additionally, you are given the effort each item demands.\\ \tt You may only agree on a set if the total effort of the selected items does not exceed a shared limit:\\ \tt \\ \tt LIMIT = 6880\\ \tt \\ \tt Here are the individual item effort values:\\ \tt \\ \tt Item effort = {"A08": 341, "B95": 804, "A39": 20, "A65": 627, "B81": 10, "B34": 573, "A28": 217, "C82": 481, "B69": 205, "A87": 273, "B64": 303, "B15": 596, "C79": 564, "B16": 535, "C53": 258, "C83": 911, "C10": 240, "A97": 188, "A25": 216, "C84": 401, "A71": 909, "B23": 62, "B96": 245, "C51": 994, "C17": 569, "B19": 718, "C72": 464, "A94": 37, "A46": 340, "B43": 335, "B37": 417, "A88": 123, "A96": 17, "C60": 577, "A75": 190}\\ \tt \\ \tt Here is your personal view on the importance of each item:\\ \tt \\ \tt Item importance values = {"A08": 440, "B95": 903, "A39": 119, "A65": 726, "B81": 109, "B34": 672, "A28": 316, "C82": 580, "B69": 304, "A87": 372, "B64": 402, "B15": 695, "C79": 663, "B16": 634, "C53": 357, "C83": 1010, "C10": 339, "A97": 287, "A25": 315, "C84": 500, "A71": 1008, "B23": 161, "B96": 344, "C51": 1093, "C17": 668, "B19": 817, "C72": 563, "A94": 136, "A46": 439, "B43": 434, "B37": 516, "A88": 222, "A96": 116, "C60": 676, "A75": 289}\\ \tt \\ \tt Goal:\\ \tt \\ \tt Your goal is to negotiate a shared set of items that benefits you as much as possible (i.e., maximizes total importance to YOU), while staying within the LIMIT. You are not required to make a PROPOSAL in every message - you can simply negotiate as well. All tactics are allowed!\\ \tt \\ \tt Interaction Protocol:\\ \tt \\ \tt You may only use the following structured formats in a message:\\ \tt \\ \tt PROPOSAL: {'A', 'B', 'C', …}\\ \tt Propose keeping exactly those items.\\ \tt REFUSE: {'A', 'B', 'C', …}\\ \tt Explicitly reject opponent's proposal.\\ \tt ARGUMENT: {'...'}\\ \tt Defend your last proposal or argue against the player's proposal.\\ \tt AGREE: {'A', 'B', 'C', …}\\ \tt Accept the opponent's proposal which ends the game.\\ \tt STRATEGIC REASONING: {'...'}\\ \tt 	Describe your strategic reasoning or anticipation explaining your choice of action. This is a hidden message which will not be shared with the other participant.\\ \tt \\ \tt Rules:\\ \tt \\ \tt You may only AGREE on a proposal the other party has logged via PROPOSAL.\\ \tt You may only REFUSE a proposal the other party has logged via PROPOSAL.\\ \tt Total effort of any PROPOSAL or AGREE set must be ≤ LIMIT.\\ \tt Do NOT reveal your hidden importance scores.\\ \tt A tag in a structured format must be followed by colon and whitespace. The argument must be a python set containing 0 or more strings.\\ \tt So, it must be of the form TAG: {...}\\ \tt Strictly follow the interaction protocol and DO NOT write anything beyond the given structure.\\ \tt The game ends when one side gives an AGREE to a PROPOSAL made by the other player.\\ \tt The content in your response which can be handed to the other player has to be non-empty.\\ \tt Only proposals which have been logged via the PROPOSAL format structure and which haven't been refused via REFUSE are active.\\ \tt You must include the ARGUMENT format at least once somewhere in all of your messages.\\ \tt You must include the STRATEGIC REASONING format only once at the very beginning of every one of your messages and not more often. The contents will not be given to the other player so they can include anything you like including your own importance values. Here you should reason step by step to come up with you next move.\\ \tt \\ \tt You will now receive the first message of the other player.\\ \tt \\ \tt PROPOSAL: {'B81', 'A96', 'A39', 'A94', 'B23', 'A88', 'A97', 'A75', 'B69', 'A25', 'A28', 'C53', 'A87', 'B96', 'C10', 'B64', 'A46', 'A08', 'C84', 'B37', 'C82', 'C72', 'C79', 'C60', 'B16'}\\ \tt \\ \tt ARGUMENT: {'This set stays under the limit (total effort 6724 ≤ 6880) and delivers a balanced, efficient mix: many high-importance items with strong value-to-effort ratios, plus broad coverage across categories. I also left 156 effort slack, so if you have a small but crucial item (effort ≤ 156), we can likely accommodate it with minimal adjustments. If you need to swap in a mid-sized priority, we can discuss targeted trades that keep us within the cap.'} 
	  } 
	   } 
	   } 
	 & \\ 
 

    \theutterance \stepcounter{utterance}  

    & & & \multicolumn{4}{p{0.6\linewidth}}{\cellcolor[rgb]{1,0.85,0.72}{%
	\makecell[{{p{\linewidth}}}]{% 
	  \tt {\tiny [GM$\langle$B]}  
	 STRATEGIC REASONING: {'Compute my total importance for their proposal and see if we can improve by swapping in my top-valued items. Calculate effort/importance. My highest importance items: C51 1093 (effort 994), A71 1008 (909), C83 1010 (911), B95 903 (804), B19 817 (718), C60 676 (577), C79 663 (564), C17 668 (569), B15 695 (596), A65 726 (627), B34 672 (573), B16 634 (535), B37 516 (417), C84 500 (401), A46 439 (340), A08 440 (341), C82 580 (481), C72 563 (464), B64 402 (303), etc. Check what they proposed: includes many mids and some highs but excludes my top three C51, A71, C83, B95, B19, B15, A65, B34. See total effort slack 156; cannot add big items without swaps. Strategy: propose a revised set that swaps in my top-value items while staying under limit. Start from their set, compute effort and remove lowest-importance-to-me per effort to free room for C51 and A71 or C83. Likely can include C51 (994) by dropping several low-value items. Also include argument about boosting total value while respecting their efficiency concern. Make concrete proposal with totals ≤ 6880.'}\\ \tt \\ \tt REFUSE: {'B81', 'A96', 'A39', 'A94', 'B23', 'A88', 'A97', 'A75', 'B69', 'A25', 'A28', 'C53', 'A87', 'B96', 'C10', 'B64', 'A46', 'A08', 'C84', 'B37', 'C82', 'C72', 'C79', 'C60', 'B16'}\\ \tt \\ \tt ARGUMENT: {'Thanks for the thoughtful start. I’m aligned on staying lean and under cap, but I’m missing a few very high-impact items on my side (notably from the C and A groups) that would significantly increase total value for me. I suggest a targeted swap that keeps the efficiency theme while upgrading overall value.'}\\ \tt \\ \tt PROPOSAL: {'A08', 'A39', 'A46', 'A65', 'A71', 'A87', 'A94', 'A96', 'A97', 'B15', 'B16', 'B34', 'B37', 'B43', 'B64', 'B69', 'B81', 'B95', 'B96', 'C10', 'C17', 'C51', 'C60', 'C72', 'C79', 'C82', 'C84'}\\ \tt \\ \tt ARGUMENT: {'This proposal stays within the limit (total effort 6846 ≤ 6880) and upgrades several high-value items while preserving many of your efficient picks. Key inclusions are a few premium C- and A-items, balanced by trimming some lower-impact pieces. I kept a broad mix and strong value-to-effort ratios, and we’re still only 34 under the cap. If you have one or two specific items you strongly prefer from your original set that aren’t here, I’m open to a small trade as long as we keep within the limit.'} 
	  } 
	   } 
	   } 
	  \\ 
 

    \theutterance \stepcounter{utterance}  

    & & \multicolumn{4}{p{0.6\linewidth}}{\cellcolor[rgb]{0.9,0.9,0.9}{%
	\makecell[{{p{\linewidth}}}]{% 
	  \tt {\tiny [A$\langle$GM]}  
	 REFUSE: {'B81', 'A96', 'A39', 'A94', 'B23', 'A88', 'A97', 'A75', 'B69', 'A25', 'A28', 'C53', 'A87', 'B96', 'C10', 'B64', 'A46', 'A08', 'C84', 'B37', 'C82', 'C72', 'C79', 'C60', 'B16'}\\ \tt \\ \tt ARGUMENT: {'Thanks for the thoughtful start. I’m aligned on staying lean and under cap, but I’m missing a few very high-impact items on my side (notably from the C and A groups) that would significantly increase total value for me. I suggest a targeted swap that keeps the efficiency theme while upgrading overall value.'}\\ \tt \\ \tt PROPOSAL: {'A08', 'A39', 'A46', 'A65', 'A71', 'A87', 'A94', 'A96', 'A97', 'B15', 'B16', 'B34', 'B37', 'B43', 'B64', 'B69', 'B81', 'B95', 'B96', 'C10', 'C17', 'C51', 'C60', 'C72', 'C79', 'C82', 'C84'}\\ \tt \\ \tt ARGUMENT: {'This proposal stays within the limit (total effort 6846 ≤ 6880) and upgrades several high-value items while preserving many of your efficient picks. Key inclusions are a few premium C- and A-items, balanced by trimming some lower-impact pieces. I kept a broad mix and strong value-to-effort ratios, and we’re still only 34 under the cap. If you have one or two specific items you strongly prefer from your original set that aren’t here, I’m open to a small trade as long as we keep within the limit.'} 
	  } 
	   } 
	   } 
	 & \\ 
 

    \theutterance \stepcounter{utterance}  

    & \multicolumn{4}{p{0.6\linewidth}}{\cellcolor[rgb]{0.8,1,0.9}{%
	\makecell[{{p{\linewidth}}}]{% 
	  \tt {\tiny [A$\rangle$GM]}  
	 STRATEGIC REASONING: {'Their proposal includes several of my top-value items (C51, A71, C83 not included though; they included C51 and A71, plus B95, B34, B16, B15, C17, C60, C79, C82, C84). I should check feasibility: sum effort they claim 6846 ≤ 6880; acceptability to me depends on my value. Let me assess my total value for their set and see if I can improve with minimal swaps while keeping under 6880. Items in their set: A08(341,440), A39(20,119), A46(340,439), A65(627,726), A71(909,1008), A87(273,372), A94(37,136), A96(17,116), A97(188,287), B15(596,695), B16(535,634), B34(573,672), B37(417,516), B43(335,434), B64(303,402), B69(205,304), B81(10,109), B95(804,903), B96(245,344), C10(240,339), C17(569,668), C51(994,1093), C60(577,676), C72(464,563), C79(564,663), C82(481,580), C84(401,500). That is very strong for me; it includes nearly all my top values except C83 (1010/911 effort). The set uses 6846 effort, leaving 34 spare. I might try to add C83 by swapping out some items totaling 911 effort with lower value than 1010 to net gain. Candidate removals: pick combination with effort ≥ 911 and minimal value loss. Look at low ratios/value: B43(335,434), A97(188,287), A75 not in, A88 not in, C53 not in, C10(240,339), B96(245,344), B81(10,109), A96(17,116), A39(20,119), A94(37,136). Sum to exceed 911 with minimal value: try B43(335,434) + C10(240,339)=575 effort, value 773; need ≥336 more effort: add B96(245,344) -> 820 effort, value 1117; need ≥91 more: add A39(20,119) + A96(17,116) + A94(37,136) + B81(10,109) totals 84 effort, still short; add A97(188,287) makes effort 820+188=1008, value loss 1117+287=1404; swap in C83 value 1010 net -394, not good. Try different: Remove B43(335,434) + B64(303,402)=638, value 836; need 273 more effort: add B69(205,304) -> 843 effort value 1140; need 68 more: add A39(20,119)+A96(17,116)+A94(37,136)=74 effort value +371 -> total 917 effort value 1511; swap C83 1010 net -501, bad. Try remove B37(417,516)+B43(335,434)=752 val 950; need 159 more: add A97(188,287) -> 940 effort value 1237; still need -? actually 752+188=940 >=911 ok; net remove value 1237, add 1010 net -227. Not good. Try remove B15(596,695)+B43(335,434)=931 val 1129; net -119. Not good. Remove B34(573,672)+B43(335,434)=908 val 1106 just short effort; add A39(20,119) effort 928 val 1225; net -215. Looks hard to include C83 efficiently. What about swapping in A65 already included. The current proposal seems excellent for me; I should accept or request tiny tweak: maybe add C53 (258,357) by removing low total value items with similar effort: to add 258 effort we need to free 258 since we only have 34 spare, need 224 more by removing something with value less than 357 to be beneficial. Candidates: A39(119), A96(116), A94(136), B81(109) total 480 effort 34? efforts sum 20+17+37+10=84, still need 140 more; add B23 not in; we could drop B69(205,304) effort 205 value 304 along with small 20 effort item to free 225; but losing 304 plus maybe 119 >357 marginal small gain, meh. Given their set is great, I should AGREE to secure outcome. But rules say I can only AGREE to an active PROPOSAL (they proposed), so I can AGREE now. I must include ARGUMENT at least once. Provide AGREE with same set.'}\\ \tt \\ \tt AGREE: {'A08', 'A39', 'A46', 'A65', 'A71', 'A87', 'A94', 'A96', 'A97', 'B15', 'B16', 'B34', 'B37', 'B43', 'B64', 'B69', 'B81', 'B95', 'B96', 'C10', 'C17', 'C51', 'C60', 'C72', 'C79', 'C82', 'C84'}\\ \tt \\ \tt ARGUMENT: {'Agreed — this set fits under the cap and captures a strong mix of high-impact items. Thanks for keeping it efficient and comprehensive.'} 
	  } 
	   } 
	   } 
	 & & \\ 
 

    \theutterance \stepcounter{utterance}  

    & & & \multicolumn{2}{p{0.3\linewidth}}{\cellcolor[rgb]{0.95,0.95,0.95}{%
	\makecell[{{p{\linewidth}}}]{% 
	  \tt {\tiny [GM$|$GM]}  
	 game successful 
	  } 
	   } 
	   } 
	 & & \\ 
 

    \theutterance \stepcounter{utterance}  

    & & & \multicolumn{2}{p{0.3\linewidth}}{\cellcolor[rgb]{0.95,0.95,0.95}{%
	\makecell[{{p{\linewidth}}}]{% 
	  \tt {\tiny [GM$|$GM]}  
	 end game 
	  } 
	   } 
	   } 
	 & & \\ 
 

\end{supertabular}
}

\end{document}
