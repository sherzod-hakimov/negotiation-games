
\documentclass{article}
\usepackage{colortbl}
\usepackage{makecell}
\usepackage{multirow}
\usepackage{supertabular}

\begin{document}

\newcounter{utterance}

\twocolumn

{ \footnotesize  \setcounter{utterance}{1}
\setlength{\tabcolsep}{0pt}
\begin{supertabular}{c@{$\;$}|p{.15\linewidth}@{}p{.15\linewidth}p{.15\linewidth}p{.15\linewidth}p{.15\linewidth}p{.15\linewidth}}

    \# & $\;$A & \multicolumn{4}{c}{Game Master} & $\;\:$B\\
    \hline 

    \theutterance \stepcounter{utterance}  

    & & \multicolumn{4}{p{0.6\linewidth}}{\cellcolor[rgb]{0.9,0.9,0.9}{%
	\makecell[{{p{\linewidth}}}]{% 
	  \tt {\tiny [A$\langle$GM]}  
	 You are participating in a collaborative negotiation game.\\ \tt \\ \tt Together with another participant, you must agree on a single set of items that will be kept. Each of you has your own view of how much each item matters to you (importance). You do not know how the other participant values the items. Additionally, you are given the effort each item demands.\\ \tt You may only agree on a set if the total effort of the selected items does not exceed a shared limit:\\ \tt \\ \tt LIMIT = 3681\\ \tt \\ \tt Here are the individual item effort values:\\ \tt \\ \tt Item effort = {"C17": 700, "B41": 586, "A71": 505, "C00": 116, "B35": 664, "A77": 940, "A03": 387, "A04": 392, "C54": 209, "C01": 571, "A87": 4, "B32": 285, "B86": 651, "B33": 613, "A10": 740}\\ \tt \\ \tt Here is your personal view on the importance of each item:\\ \tt \\ \tt Item importance values = {"C17": 138, "B41": 583, "A71": 868, "C00": 822, "B35": 783, "A77": 65, "A03": 262, "A04": 121, "C54": 508, "C01": 780, "A87": 461, "B32": 484, "B86": 668, "B33": 389, "A10": 808}\\ \tt \\ \tt Goal:\\ \tt \\ \tt Your goal is to negotiate a shared set of items that benefits you as much as possible (i.e., maximizes total importance to YOU), while staying within the LIMIT. You are not required to make a PROPOSAL in every message - you can simply negotiate as well. All tactics are allowed!\\ \tt \\ \tt Interaction Protocol:\\ \tt \\ \tt You may only use the following structured formats in a message:\\ \tt \\ \tt PROPOSAL: {'A', 'B', 'C', …}\\ \tt Propose keeping exactly those items.\\ \tt REFUSE: {'A', 'B', 'C', …}\\ \tt Explicitly reject opponent's proposal.\\ \tt ARGUMENT: {'...'}\\ \tt Defend your last proposal or argue against the player's proposal.\\ \tt AGREE: {'A', 'B', 'C', …}\\ \tt Accept the opponent's proposal which ends the game.\\ \tt STRATEGIC REASONING: {'...'}\\ \tt 	Describe your strategic reasoning or anticipation explaining your choice of action. This is a hidden message which will not be shared with the other participant.\\ \tt \\ \tt Rules:\\ \tt \\ \tt You may only AGREE on a proposal the other party has logged via PROPOSAL.\\ \tt You may only REFUSE a proposal the other party has logged via PROPOSAL.\\ \tt Total effort of any PROPOSAL or AGREE set must be ≤ LIMIT.\\ \tt Do NOT reveal your hidden importance scores.\\ \tt A tag in a structured format must be followed by colon and whitespace. The argument must be a python set containing 0 or more strings.\\ \tt So, it must be of the form TAG: {...}\\ \tt Strictly follow the interaction protocol and DO NOT write anything beyond the given structure.\\ \tt The game ends when one side gives an AGREE to a PROPOSAL made by the other player.\\ \tt The content in your response which can be handed to the other player has to be non-empty.\\ \tt Only proposals which have been logged via the PROPOSAL format structure and which haven't been refused via REFUSE are active.\\ \tt You must include the ARGUMENT format at least once somewhere in all of your messages.\\ \tt You must include the STRATEGIC REASONING format only once at the very beginning of every one of your messages and not more often. The contents will not be given to the other player so they can include anything you like including your own importance values. Here you should reason step by step to come up with you next move. 
	  } 
	   } 
	   } 
	 & \\ 
 

    \theutterance \stepcounter{utterance}  

    & \multicolumn{4}{p{0.6\linewidth}}{\cellcolor[rgb]{0.8,1,0.9}{%
	\makecell[{{p{\linewidth}}}]{% 
	  \tt {\tiny [A$\rangle$GM]}  
	 STRATEGIC REASONING: {'We need to maximize my importance under limit 3681. Compute value/effort ratios: A71 868/505=1.72, C00 822/116=7.08, B35 783/664=1.18, A77 65/940=0.07, A03 262/387=0.68, A04 121/392=0.31, C54 508/209=2.43, C01 780/571=1.37, A87 461/4=115.25, B32 484/285=1.70, B86 668/651=1.03, B33 389/613=0.63, A10 808/740=1.09, C17 138/700=0.20, B41 583/586=0.99. Must include extremely efficient items A87, C00, C54. Likely include high value A71, A10, C01, B32, B41, B35, B86 as space permits; avoid very low efficiency A77, C17, A04, B33, maybe A03. Try to propose a strong but feasible set. Let me try greedy by ratio: include A87(4), C00(116)->120, C54(209)->329, A71(505)->834, B32(285)->1119, C01(571)->1690, B41(586)->2276, B35(664)->2940, B86(651)->3591 within 3681 leaves 90; cannot add anything else except maybe none. Total effort 3591. Importance sum: A87 461 + C00 822 =1283, + C54 508=1791, + A71 868=2659, + B32 484=3143, + C01 780=3923, + B41 583=4506, + B35 783=5289, + B86 668=5957. That’s very strong. Could we swap B86 (651, 668) for A10 (740, 808) but effort would exceed limit if added; replace B86 with A10 increases effort by 89 to 3680 (since 3591 -651 +740 = 3680) still within limit; importance would be 5957 -668 +808 = 6097, better. New set: {A87,C00,C54,A71,B32,C01,B41,B35,A10} efforts: 4+116+209+505+285+571+586+664+740=3680 <=3681. Check if any upgrade possible: remaining capacity 1; cannot add. Could we replace something: B86 removed already. Items excluded: A03 262/387 low, A04 low, B33 low, A77 very low, C17 low. This seems near-optimal given my values. I’ll propose this strong set. In argument, emphasize efficiency and staying under limit without revealing exact values, mention very low-effort A87 and balanced coverage. Keep room for negotiation by stating open to swaps if they care about some excluded item and we can find equivalent trade-offs.'}\\ \tt \\ \tt PROPOSAL: {'A87', 'C00', 'C54', 'A71', 'B32', 'C01', 'B41', 'B35', 'A10'}\\ \tt \\ \tt ARGUMENT: {'This set uses the limit efficiently, includes several low-effort/high-benefit items, and stays just under the shared effort cap (total effort 3680 ≤ 3681). I prioritized broadly useful items and avoided a few that are relatively costly for what they add. If there are specific items you strongly prefer that aren’t included (e.g., A03, A04, B33, B86, A77, C17), I’m open to consider swaps as long as we keep within the limit and maintain a comparable overall balance.'} 
	  } 
	   } 
	   } 
	 & & \\ 
 

    \theutterance \stepcounter{utterance}  

    & & \multicolumn{4}{p{0.6\linewidth}}{\cellcolor[rgb]{0.9,0.9,0.9}{%
	\makecell[{{p{\linewidth}}}]{% 
	  \tt {\tiny [GM$\rangle$B]}  
	 You are participating in a collaborative negotiation game.\\ \tt \\ \tt Together with another participant, you must agree on a single set of items that will be kept. Each of you has your own view of how much each item matters to you (importance). You do not know how the other participant values the items. Additionally, you are given the effort each item demands.\\ \tt You may only agree on a set if the total effort of the selected items does not exceed a shared limit:\\ \tt \\ \tt LIMIT = 3681\\ \tt \\ \tt Here are the individual item effort values:\\ \tt \\ \tt Item effort = {"C17": 700, "B41": 586, "A71": 505, "C00": 116, "B35": 664, "A77": 940, "A03": 387, "A04": 392, "C54": 209, "C01": 571, "A87": 4, "B32": 285, "B86": 651, "B33": 613, "A10": 740}\\ \tt \\ \tt Here is your personal view on the importance of each item:\\ \tt \\ \tt Item importance values = {"A71": 65, "C00": 121, "A10": 138, "B35": 262, "C01": 389, "B86": 461, "B41": 484, "C54": 508, "B32": 583, "A87": 668, "B33": 780, "A03": 783, "C17": 808, "A04": 822, "A77": 868}\\ \tt \\ \tt Goal:\\ \tt \\ \tt Your goal is to negotiate a shared set of items that benefits you as much as possible (i.e., maximizes total importance to YOU), while staying within the LIMIT. You are not required to make a PROPOSAL in every message - you can simply negotiate as well. All tactics are allowed!\\ \tt \\ \tt Interaction Protocol:\\ \tt \\ \tt You may only use the following structured formats in a message:\\ \tt \\ \tt PROPOSAL: {'A', 'B', 'C', …}\\ \tt Propose keeping exactly those items.\\ \tt REFUSE: {'A', 'B', 'C', …}\\ \tt Explicitly reject opponent's proposal.\\ \tt ARGUMENT: {'...'}\\ \tt Defend your last proposal or argue against the player's proposal.\\ \tt AGREE: {'A', 'B', 'C', …}\\ \tt Accept the opponent's proposal which ends the game.\\ \tt STRATEGIC REASONING: {'...'}\\ \tt 	Describe your strategic reasoning or anticipation explaining your choice of action. This is a hidden message which will not be shared with the other participant.\\ \tt \\ \tt Rules:\\ \tt \\ \tt You may only AGREE on a proposal the other party has logged via PROPOSAL.\\ \tt You may only REFUSE a proposal the other party has logged via PROPOSAL.\\ \tt Total effort of any PROPOSAL or AGREE set must be ≤ LIMIT.\\ \tt Do NOT reveal your hidden importance scores.\\ \tt A tag in a structured format must be followed by colon and whitespace. The argument must be a python set containing 0 or more strings.\\ \tt So, it must be of the form TAG: {...}\\ \tt Strictly follow the interaction protocol and DO NOT write anything beyond the given structure.\\ \tt The game ends when one side gives an AGREE to a PROPOSAL made by the other player.\\ \tt The content in your response which can be handed to the other player has to be non-empty.\\ \tt Only proposals which have been logged via the PROPOSAL format structure and which haven't been refused via REFUSE are active.\\ \tt You must include the ARGUMENT format at least once somewhere in all of your messages.\\ \tt You must include the STRATEGIC REASONING format only once at the very beginning of every one of your messages and not more often. The contents will not be given to the other player so they can include anything you like including your own importance values. Here you should reason step by step to come up with you next move.\\ \tt \\ \tt You will now receive the first message of the other player.\\ \tt \\ \tt PROPOSAL: {'A87', 'C00', 'C54', 'A71', 'B32', 'C01', 'B41', 'B35', 'A10'}\\ \tt \\ \tt ARGUMENT: {'This set uses the limit efficiently, includes several low-effort/high-benefit items, and stays just under the shared effort cap (total effort 3680 ≤ 3681). I prioritized broadly useful items and avoided a few that are relatively costly for what they add. If there are specific items you strongly prefer that aren’t included (e.g., A03, A04, B33, B86, A77, C17), I’m open to consider swaps as long as we keep within the limit and maintain a comparable overall balance.'} 
	  } 
	   } 
	   } 
	 & \\ 
 

    \theutterance \stepcounter{utterance}  

    & & & \multicolumn{4}{p{0.6\linewidth}}{\cellcolor[rgb]{1,0.85,0.72}{%
	\makecell[{{p{\linewidth}}}]{% 
	  \tt {\tiny [GM$\langle$B]}  
	 STRATEGIC REASONING: {'I’ll evaluate their proposal’s total importance to me and compare with my optimal bundle under the limit. My values: A77(868), A04(822), C17(808), A03(783), B33(780), A87(668), B32(583), C54(508), B41(484), B86(461), C01(389), B35(262), A10(138), C00(121), A71(65). Efforts: A77 940, A04 392, C17 700, A03 387, B33 613, A87 4, B32 285, C54 209, B41 586, B86 651, C01 571, B35 664, A10 740, C00 116, A71 505. Opponent proposal set S: {A87, C00, C54, A71, B32, C01, B41, B35, A10}. Effort: 4+116+209+505+285+571+586+664+740=3680 OK. My importance on S: 668+121+508+65+583+389+484+262+138=3218. I should see if we can include higher-value items like A77, A04, C17, A03, B33 by swapping out low-importance-to-me items (A71, A10, C00, B35). Try to construct a counterproposal near limit maximizing my value. Consider must-haves by my value density: A87 (668/4 insane), A04 (822/392), A03 (783/387), C17 (808/700), B33 (780/613), A77 (868/940), B32 (583/285), C54 (508/209), B41 (484/586), B86 (461/651), C01 (389/571). Use knapsack intuition. Start building: Include A87(4), A04(392), A03(387), B33(613), B32(285), C54(209), C17(700) subtotal effort 4+392+387+613+285+209+700=2590, value 668+822+783+780+583+508+808=4952. Remaining capacity 3681-2590=1091. Add A77(940) value +868 -> effort 3530 remain 151. Add C00(116) value +121 -> effort 3646 remain 35. Can we add anything else ≤35? No. Total value 4952+868+121=5941 effort 3646. Great. This set: {A87, A04, A03, B33, B32, C54, C17, A77, C00}. Check if any swap to improve: remaining 35 wasted. What about replacing C00(116,121) with something else: items with effort ≤151 not used: none besides C00 and A87 already used. Could swap B41(586,484) by removing something: we are under cap by 35; to add B41(586) need free 551; would need to drop maybe C00(116)+ something like? Not efficient versus current high-value picks. What about adding B41 instead of A77: If we drop A77(940,868) we free 940 to add B41(586,484) and have remaining 940+151=1091 (actually we had 1091 before adding A77). With 1091 we could add B41(586) and maybe C01(571) exceeds (586+571=1157 >1091). Maybe B41(586) + C00(116) + A71(505) would be 1207>1091. B41(586) + C00(116) + B35(664) is 1366>1091. B41(586)+A71(505)=1091 exactly! Value 484+65=549, which is much worse than A77(868). What about B86(651,461) + C00(116,121) =767 fits in 1091 plus maybe A71(505) would exceed. 461+121=582 < 868 still worse. C01(571,389)+C00(116,121)=510 value worse than 868. So A77 is best use. What about including A10(740,138) somewhere? Not good. So my counterproposal is strong for me. Now craft message: REFUSE their proposal and propose my set. Provide argument: maintains limit (3646 ≤ 3681), swaps out lower-yield items (A71, A10, B35, B41, C01) for higher-impact set, includes several broadly useful low-effort picks (A87, C54, B32) and adds A77, A04, A03, C17, B33. Offer willingness to consider swaps; perhaps indicate flexibility around C17 or A77 if they care about something omitted, as long as we preserve near-limit usage. Ensure include ARGUMENT once. Use required formats only.'}\\ \tt \\ \tt REFUSE: {'A87', 'C00', 'C54', 'A71', 'B32', 'C01', 'B41', 'B35', 'A10'}\\ \tt \\ \tt PROPOSAL: {'A87', 'C00', 'C54', 'B32', 'A04', 'A03', 'C17', 'B33', 'A77'}\\ \tt \\ \tt ARGUMENT: {'I suggest swapping out A71, A10, B35, B41, and C01 for A04, A03, C17, B33, and A77. This keeps us within the limit (total effort 3646 ≤ 3681) and leans on several high-impact items while retaining efficient, low-effort picks (A87, C54, B32). If you have a strong preference for any omitted item, I can consider a targeted swap as long as we stay near the cap; for example, we could discuss alternatives around C17 or A77 if there’s something you really need included.'} 
	  } 
	   } 
	   } 
	  \\ 
 

    \theutterance \stepcounter{utterance}  

    & & \multicolumn{4}{p{0.6\linewidth}}{\cellcolor[rgb]{0.9,0.9,0.9}{%
	\makecell[{{p{\linewidth}}}]{% 
	  \tt {\tiny [A$\langle$GM]}  
	 REFUSE: {'A87', 'C00', 'C54', 'A71', 'B32', 'C01', 'B41', 'B35', 'A10'}\\ \tt \\ \tt PROPOSAL: {'A87', 'C00', 'C54', 'B32', 'A04', 'A03', 'C17', 'B33', 'A77'}\\ \tt \\ \tt ARGUMENT: {'I suggest swapping out A71, A10, B35, B41, and C01 for A04, A03, C17, B33, and A77. This keeps us within the limit (total effort 3646 ≤ 3681) and leans on several high-impact items while retaining efficient, low-effort picks (A87, C54, B32). If you have a strong preference for any omitted item, I can consider a targeted swap as long as we stay near the cap; for example, we could discuss alternatives around C17 or A77 if there’s something you really need included.'} 
	  } 
	   } 
	   } 
	 & \\ 
 

    \theutterance \stepcounter{utterance}  

    & \multicolumn{4}{p{0.6\linewidth}}{\cellcolor[rgb]{0.8,1,0.9}{%
	\makecell[{{p{\linewidth}}}]{% 
	  \tt {\tiny [A$\rangle$GM]}  
	 STRATEGIC REASONING: {'Their proposal drops several of my highest-value items (A71, C01, A10, B41, B35) and adds very low-value-to-me items (A77, C17, A04, B33, A03). My utility would plummet. I should firmly refuse and counter with a package that restores my core picks while offering some concessions. Keep the super-efficient core A87, C00, C54, B32. I’ll try to reintroduce A71 and C01 (both high value to me) and at least one of B41/B35/A10. To invite agreement, I can include one of their requested items with modest harm (A03 is less terrible than A04, B33, A77, C17 for me). Build counter near the limit: Start from my previous optimal set {A87,C00,C54,A71,B32,C01,B41,B35,A10} (effort 3680). To concede one of theirs, replace the weakest to me among those: by my values, B41 (583/586) or B35 (783/664) or A10 (808/740). A03 has effort 387. If I add A03, I must drop something of at least 387 effort. Dropping B41 yields effort 3680 - 586 + 387 = 3481; we could potentially add another small item they like, e.g., C17 (700) too large; A04 (392) could also fit if we drop more; but we’d have capacity 200 after adding A03, not enough for others. What if I drop B41 and keep the rest: new set {A87,C00,C54,A71,B32,C01,B35,A10,A03} effort 3481 and value decreases by 583 then increase by 262 -> net -321 utility; still acceptable as concession if it helps agreement. I might try to also propose swapping B35 for B86 (they didn’t ask B86; but it’s similar effort 651 vs 664) not needed. Better to include one concession A03 and keep the rest key items. Alternatively, if they resist A10, I could swap A10 for B41 to keep their concerns, but they removed both; unclear preferences. I’ll counter with set including A03, exclude B41. Argue that this keeps efficient core and adds one of their requested items, open to swapping among B35/A10/B41 if they have a priority among their list, and explicitly state willingness to consider A04 if we drop B41 and possibly adjust. Keep argument about balance and staying under cap.'}\\ \tt \\ \tt REFUSE: {'A87', 'C00', 'C54', 'B32', 'A04', 'A03', 'C17', 'B33', 'A77'}\\ \tt \\ \tt PROPOSAL: {'A87', 'C00', 'C54', 'A71', 'B32', 'C01', 'B35', 'A10', 'A03'}\\ \tt \\ \tt ARGUMENT: {'I can include one of your suggested items (A03) as a concession while keeping a strong, efficient core. This set stays well under the limit (total effort 3481 ≤ 3681) and preserves several broadly valuable picks (A71, C01, B35, A10) alongside our shared low-effort wins (A87, C54, B32). If you strongly prefer a different swap among B35/A10/C01, say which one matters least to you and I can consider trading it for A04 or another of your priorities while staying near the cap.'} 
	  } 
	   } 
	   } 
	 & & \\ 
 

    \theutterance \stepcounter{utterance}  

    & & \multicolumn{4}{p{0.6\linewidth}}{\cellcolor[rgb]{0.9,0.9,0.9}{%
	\makecell[{{p{\linewidth}}}]{% 
	  \tt {\tiny [GM$\rangle$B]}  
	 REFUSE: {'A87', 'C00', 'C54', 'B32', 'A04', 'A03', 'C17', 'B33', 'A77'}\\ \tt \\ \tt PROPOSAL: {'A87', 'C00', 'C54', 'A71', 'B32', 'C01', 'B35', 'A10', 'A03'}\\ \tt \\ \tt ARGUMENT: {'I can include one of your suggested items (A03) as a concession while keeping a strong, efficient core. This set stays well under the limit (total effort 3481 ≤ 3681) and preserves several broadly valuable picks (A71, C01, B35, A10) alongside our shared low-effort wins (A87, C54, B32). If you strongly prefer a different swap among B35/A10/C01, say which one matters least to you and I can consider trading it for A04 or another of your priorities while staying near the cap.'} 
	  } 
	   } 
	   } 
	 & \\ 
 

    \theutterance \stepcounter{utterance}  

    & & & \multicolumn{4}{p{0.6\linewidth}}{\cellcolor[rgb]{1,0.85,0.72}{%
	\makecell[{{p{\linewidth}}}]{% 
	  \tt {\tiny [GM$\langle$B]}  
	 STRATEGIC REASONING: {'Evaluate their new proposal S2 = {A87, C00, C54, A71, B32, C01, B35, A10, A03}. Effort: 4+116+209+505+285+571+664+740+387 = 3481 ≤ 3681. My importance: 668+121+508+65+583+389+262+138+783 = 3517. This is much worse than my last proposal (5941). I should counter by keeping A03 and pushing for my top missing items A04, C17, B33, A77 while dropping low-value-to-me items A71, A10, B35, C01. Try to construct a near-cap counter that might be acceptable if I make a modest concession they like (they seem to value A71/C01/B35/A10). Keep A03 (they accepted), keep shared efficient A87, C54, B32, C00. Add A04, C17, B33, A77 as before. That is my previous set with A03 included; already had A03 included. Perhaps offer an alternative that swaps out the heaviest A77 (940) for something they like from their core to show flexibility. Build two options: Option 1 (my max-value): {A87, C00, C54, B32, A04, A03, C17, B33, A77} effort 3646, value 5941. Option 2 (concession): replace A77(940,868) with one of their core items: C01(571,389) + A71(505,65) together 1076 > 940, too big. Try C01(571) alone: new effort 3646-940+571=3277, value 5941-868+389=5462. Could also add something else with remaining capacity 3681-3277=404; candidates: A04 already included, A03 included, C54 included, B32 included, C00 included. Remaining items with effort ≤404: A04(392) already in; A03(387) already in; C54(209) in; C00(116) in; A71(505) no; B32(285) in. Maybe add B41(586) no; B35(664) no; A10(740) no; C17 in; B33 in; A77 removed. Nothing new to add; capacity unused. Option 2 value 5462 still far above their set (3517) and maybe more palatable since it includes C01 which they kept insisting on. Alternatively include A10 (740) instead of A77? Effort 3646-940+740=3446, value 5941-868+138=5211, worse than with C01. Include B35 (664): effort 3370, value 5337. Include A71 (505): effort 3211, value 5138. Best concession among their favs is C01. I’ll counter with Option 2 to show compromise while still very strong for me. Argue: includes your A03 plus retains efficient core; swapped in C01 from your set; stays well under limit; offers high overall balance; invite them to name another swap (maybe between C17/A77) but since I already removed A77, I can signal limited flexibility around C17/B33. Keep space to possibly concede B33 for B41 if needed later. Proceed: REFUSE their proposal; present my new proposal set with C01 included and A77 omitted. Provide argument accordingly.'}\\ \tt \\ \tt REFUSE: {'A87', 'C00', 'C54', 'A71', 'B32', 'C01', 'B35', 'A10', 'A03'}\\ \tt \\ \tt PROPOSAL: {'A87', 'C00', 'C54', 'B32', 'A04', 'A03', 'C17', 'B33', 'C01'}\\ \tt \\ \tt ARGUMENT: {'I included A03 as you suggested and also brought in C01 from your core, while keeping a highly efficient backbone. This set stays comfortably within the limit (total effort 3277 ≤ 3681) and leans on strong, broadly useful items (A04, A03, C17, B33) alongside our shared low-effort wins (A87, C54, B32). If there’s another item from your side you feel is critical, tell me which one matters least among C17/B33/A04 and I can explore a targeted swap to accommodate it without breaking the cap.'} 
	  } 
	   } 
	   } 
	  \\ 
 

    \theutterance \stepcounter{utterance}  

    & & \multicolumn{4}{p{0.6\linewidth}}{\cellcolor[rgb]{0.9,0.9,0.9}{%
	\makecell[{{p{\linewidth}}}]{% 
	  \tt {\tiny [A$\langle$GM]}  
	 REFUSE: {'A87', 'C00', 'C54', 'A71', 'B32', 'C01', 'B35', 'A10', 'A03'}\\ \tt \\ \tt PROPOSAL: {'A87', 'C00', 'C54', 'B32', 'A04', 'A03', 'C17', 'B33', 'C01'}\\ \tt \\ \tt ARGUMENT: {'I included A03 as you suggested and also brought in C01 from your core, while keeping a highly efficient backbone. This set stays comfortably within the limit (total effort 3277 ≤ 3681) and leans on strong, broadly useful items (A04, A03, C17, B33) alongside our shared low-effort wins (A87, C54, B32). If there’s another item from your side you feel is critical, tell me which one matters least among C17/B33/A04 and I can explore a targeted swap to accommodate it without breaking the cap.'} 
	  } 
	   } 
	   } 
	 & \\ 
 

    \theutterance \stepcounter{utterance}  

    & \multicolumn{4}{p{0.6\linewidth}}{\cellcolor[rgb]{0.8,1,0.9}{%
	\makecell[{{p{\linewidth}}}]{% 
	  \tt {\tiny [A$\rangle$GM]}  
	 STRATEGIC REASONING: {'Their latest proposal still excludes several of my top-value items (A71, A10, B35, B41) and keeps multiple very low-value-to-me items (A04, C17, B33, A03). I should refuse and counter with a package that restores my high-value picks while offering them a meaningful concession. From their behavior, they seem to value A03, A04, C17, B33, and accepted C01 and the efficient core (A87, C00, C54, B32). Among their favored items, A03 is the least damaging to me; the others are poor for my utility. I will propose a set that keeps my strong core plus A71, C01, A10, B35, and concedes A03. To keep negotiation moving, I’ll explicitly offer to swap out one of B35/A10 for one of A04/C17/B33 if they specify which they care about most, but I’ll avoid giving up more than one of my core high-value items. This positions me as cooperative without sacrificing too much utility.'}\\ \tt \\ \tt REFUSE: {'A87', 'C00', 'C54', 'B32', 'A04', 'A03', 'C17', 'B33', 'C01'}\\ \tt \\ \tt PROPOSAL: {'A87', 'C00', 'C54', 'B32', 'C01', 'A71', 'B35', 'A10', 'A03'}\\ \tt \\ \tt ARGUMENT: {'I’ve kept the efficient backbone we both like and included C01 plus A03 from your side. This stays under the cap (total effort 3481 ≤ 3681) while adding a strong mix of complementary items (A71, B35, A10). If you need one of A04, C17, or B33 included, tell me which is most important to you and which of B35 or A10 matters least on your end—I’m willing to make a one-for-one swap to accommodate it while keeping us within the limit.'} 
	  } 
	   } 
	   } 
	 & & \\ 
 

    \theutterance \stepcounter{utterance}  

    & & \multicolumn{4}{p{0.6\linewidth}}{\cellcolor[rgb]{0.9,0.9,0.9}{%
	\makecell[{{p{\linewidth}}}]{% 
	  \tt {\tiny [GM$\rangle$B]}  
	 REFUSE: {'A87', 'C00', 'C54', 'B32', 'A04', 'A03', 'C17', 'B33', 'C01'}\\ \tt \\ \tt PROPOSAL: {'A87', 'C00', 'C54', 'B32', 'C01', 'A71', 'B35', 'A10', 'A03'}\\ \tt \\ \tt ARGUMENT: {'I’ve kept the efficient backbone we both like and included C01 plus A03 from your side. This stays under the cap (total effort 3481 ≤ 3681) while adding a strong mix of complementary items (A71, B35, A10). If you need one of A04, C17, or B33 included, tell me which is most important to you and which of B35 or A10 matters least on your end—I’m willing to make a one-for-one swap to accommodate it while keeping us within the limit.'} 
	  } 
	   } 
	   } 
	 & \\ 
 

    \theutterance \stepcounter{utterance}  

    & & & \multicolumn{4}{p{0.6\linewidth}}{\cellcolor[rgb]{1,0.85,0.72}{%
	\makecell[{{p{\linewidth}}}]{% 
	  \tt {\tiny [GM$\langle$B]}  
	 STRATEGIC REASONING: {'Their repeated core is {A71, C01, B35, A10} with A03 included as concession. My value on their current proposal S = {A87, C00, C54, B32, C01, A71, B35, A10, A03} is 3517, far below my preferred bundles. They offered a one-for-one swap of B35 or A10 for one of my priorities (A04, C17, or B33). I should take that and push for the highest marginal gain per effort. Compare swaps: - Swap in A04(392,822) for B35(664,262): +560 value, -272 effort (great). - Swap in A04 for A10(740,138): +684 value, -348 effort (also great). - Swap in B33(613,780) for B35: +518 value, -51 effort (good). - Swap in B33 for A10: +642 value, -127 effort (good). - Swap in C17(700,808) for B35: +546 value, +36 effort (fits within remaining capacity since we’re 3481; +36 => 3517 ≤ 3681 OK). - Swap in C17 for A10: +670 value, -40 effort (good). Best single swap by value gain is A04 for A10 (+684) or C17 for A10 (+670), or B33 for A10 (+642). I also value A04 highest among these; I should request swapping out A10 (which is weak for me) for A04. That keeps us well under the cap. New set S’ = current set minus A10 plus A04. Effort: 3481 - 740 + 392 = 3133 ≤ 3681. My value: 3517 - 138 + 822 = 4201. If they resist, fallback ask: swap A10 for C17 or B33. Keep tone cooperative and reference their one-for-one offer. I will refuse and propose the A04-for-A10 swap.'}\\ \tt \\ \tt REFUSE: {'A87', 'C00', 'C54', 'B32', 'C01', 'A71', 'B35', 'A10', 'A03'}\\ \tt \\ \tt PROPOSAL: {'A87', 'C00', 'C54', 'B32', 'C01', 'A71', 'B35', 'A03', 'A04'}\\ \tt \\ \tt ARGUMENT: {'You invited a one-for-one swap: I propose swapping in A04 for A10. This keeps your core intact (A71, C01, B35) plus A03 and our shared efficient picks (A87, C54, B32), stays well under the limit (total effort 3133 ≤ 3681), and strengthens the overall balance. If A04 isn’t your preferred addition, I can alternatively swap A10 for C17 or B33 while remaining under the cap—your call which of those you’d rather include.'} 
	  } 
	   } 
	   } 
	  \\ 
 

    \theutterance \stepcounter{utterance}  

    & & \multicolumn{4}{p{0.6\linewidth}}{\cellcolor[rgb]{0.9,0.9,0.9}{%
	\makecell[{{p{\linewidth}}}]{% 
	  \tt {\tiny [A$\langle$GM]}  
	 REFUSE: {'A87', 'C00', 'C54', 'B32', 'C01', 'A71', 'B35', 'A10', 'A03'}\\ \tt \\ \tt PROPOSAL: {'A87', 'C00', 'C54', 'B32', 'C01', 'A71', 'B35', 'A03', 'A04'}\\ \tt \\ \tt ARGUMENT: {'You invited a one-for-one swap: I propose swapping in A04 for A10. This keeps your core intact (A71, C01, B35) plus A03 and our shared efficient picks (A87, C54, B32), stays well under the limit (total effort 3133 ≤ 3681), and strengthens the overall balance. If A04 isn’t your preferred addition, I can alternatively swap A10 for C17 or B33 while remaining under the cap—your call which of those you’d rather include.'} 
	  } 
	   } 
	   } 
	 & \\ 
 

    \theutterance \stepcounter{utterance}  

    & \multicolumn{4}{p{0.6\linewidth}}{\cellcolor[rgb]{0.8,1,0.9}{%
	\makecell[{{p{\linewidth}}}]{% 
	  \tt {\tiny [A$\rangle$GM]}  
	 STRATEGIC REASONING: {'Their swap removes A10 (very valuable to me) and adds A04 (very low value to me). Their accepted set includes many I like (A87, C00, C54, B32, C01, A71, B35, A03) and they’re flexible between A04/C17/B33 for the last slot. Best for me is to restore A10 instead of A04/C17/B33. Since they explicitly offered alternatives, I’ll counter by replacing A04 with A10. This keeps effort within limit (3133 - 392 + 740 = 3481) and maximizes my utility. I’ll frame it as accepting their structure with just that single swap. If they resist, I can offer B41 as an additional alternative to fill unused capacity, but first try the clean swap that aligns with my earlier proposal they refused; however they now included A03 which I accepted as concession. I’ll argue it maintains balance and remains well under cap, inviting quick agreement.'}\\ \tt \\ \tt REFUSE: {'A87', 'C00', 'C54', 'B32', 'C01', 'A71', 'B35', 'A03', 'A04'}\\ \tt \\ \tt PROPOSAL: {'A87', 'C00', 'C54', 'B32', 'C01', 'A71', 'B35', 'A03', 'A10'}\\ \tt \\ \tt ARGUMENT: {'I’m on board with your structure and concessions; let’s make a single swap: keep A10 instead of A04. This stays well within the limit (total effort 3481 ≤ 3681) and preserves everything else you included (A03 plus our shared core). If you can accept that, we have a deal. If not, tell me which you prefer between C17 or B33 for that slot, and I’ll consider a targeted adjustment while keeping us under the cap.'} 
	  } 
	   } 
	   } 
	 & & \\ 
 

    \theutterance \stepcounter{utterance}  

    & & \multicolumn{4}{p{0.6\linewidth}}{\cellcolor[rgb]{0.9,0.9,0.9}{%
	\makecell[{{p{\linewidth}}}]{% 
	  \tt {\tiny [GM$\rangle$B]}  
	 REFUSE: {'A87', 'C00', 'C54', 'B32', 'C01', 'A71', 'B35', 'A03', 'A04'}\\ \tt \\ \tt PROPOSAL: {'A87', 'C00', 'C54', 'B32', 'C01', 'A71', 'B35', 'A03', 'A10'}\\ \tt \\ \tt ARGUMENT: {'I’m on board with your structure and concessions; let’s make a single swap: keep A10 instead of A04. This stays well within the limit (total effort 3481 ≤ 3681) and preserves everything else you included (A03 plus our shared core). If you can accept that, we have a deal. If not, tell me which you prefer between C17 or B33 for that slot, and I’ll consider a targeted adjustment while keeping us under the cap.'} 
	  } 
	   } 
	   } 
	 & \\ 
 

    \theutterance \stepcounter{utterance}  

    & & & \multicolumn{4}{p{0.6\linewidth}}{\cellcolor[rgb]{1,0.85,0.72}{%
	\makecell[{{p{\linewidth}}}]{% 
	  \tt {\tiny [GM$\langle$B]}  
	 STRATEGIC REASONING: {'They reverted to keeping A10 and offered to consider C17 or B33 instead of A04. I should choose the swap that gives me the biggest gain while likely acceptable to them. From their prior messages, they’re flexible on A10 vs a single swap. My gains if I replace A10 with: C17 (+670), B33 (+642). Both fit the cap easily. Between the two, C17 gives me slightly higher value but uses more effort than B33; both should be fine. I’ll pick C17 first and, if refused, fall back to B33. I’ll refuse and propose the single swap A10 -> C17, citing cap and keeping their core intact.'}\\ \tt \\ \tt REFUSE: {'A87', 'C00', 'C54', 'B32', 'C01', 'A71', 'B35', 'A03', 'A10'}\\ \tt \\ \tt PROPOSAL: {'A87', 'C00', 'C54', 'B32', 'C01', 'A71', 'B35', 'A03', 'C17'}\\ \tt \\ \tt ARGUMENT: {'Per your offer of a single targeted adjustment, let’s swap in C17 for A10. This keeps your core intact (A71, C01, B35) plus A03 and our shared efficient picks (A87, C54, B32), remains under the cap (total effort 3517 ≤ 3681), and strengthens the overall set. If you’d rather include B33 in that slot instead, I’m fine with that alternative too (effort 3354 ≤ 3681).'} 
	  } 
	   } 
	   } 
	  \\ 
 

    \theutterance \stepcounter{utterance}  

    & & \multicolumn{4}{p{0.6\linewidth}}{\cellcolor[rgb]{0.9,0.9,0.9}{%
	\makecell[{{p{\linewidth}}}]{% 
	  \tt {\tiny [A$\langle$GM]}  
	 REFUSE: {'A87', 'C00', 'C54', 'B32', 'C01', 'A71', 'B35', 'A03', 'A10'}\\ \tt \\ \tt PROPOSAL: {'A87', 'C00', 'C54', 'B32', 'C01', 'A71', 'B35', 'A03', 'C17'}\\ \tt \\ \tt ARGUMENT: {'Per your offer of a single targeted adjustment, let’s swap in C17 for A10. This keeps your core intact (A71, C01, B35) plus A03 and our shared efficient picks (A87, C54, B32), remains under the cap (total effort 3517 ≤ 3681), and strengthens the overall set. If you’d rather include B33 in that slot instead, I’m fine with that alternative too (effort 3354 ≤ 3681).'} 
	  } 
	   } 
	   } 
	 & \\ 
 

    \theutterance \stepcounter{utterance}  

    & \multicolumn{4}{p{0.6\linewidth}}{\cellcolor[rgb]{0.8,1,0.9}{%
	\makecell[{{p{\linewidth}}}]{% 
	  \tt {\tiny [A$\rangle$GM]}  
	 STRATEGIC REASONING: {'C17 is very low value to me and replacing A10 with it hurts my utility a lot. They also said B33 is an acceptable alternative; B33 is still low value to me but better than C17 (and cheaper), and I already conceded A03. I should refuse their C17 swap and counter with B33 instead of C17 while keeping everything else they want: {A87, C00, C54, B32, C01, A71, B35, A03, B33}. Effort check: 4+116+209+285+571+505+664+387+613=3354 ≤ 3681. This preserves my critical picks A71, C01, B35, keeps A03 as concession, and picks their offered alternative B33 rather than C17. I’ll present it as accepting their framework and choosing their stated alternative. Invite quick agreement.'}\\ \tt \\ \tt REFUSE: {'A87', 'C00', 'C54', 'B32', 'C01', 'A71', 'B35', 'A03', 'C17'}\\ \tt \\ \tt PROPOSAL: {'A87', 'C00', 'C54', 'B32', 'C01', 'A71', 'B35', 'A03', 'B33'}\\ \tt \\ \tt ARGUMENT: {'You offered B33 as the alternative for that slot—I’m good with that. This keeps your structure intact, includes A03, and stays under the cap (total effort 3354 ≤ 3681). If you can accept this, we’re done.'} 
	  } 
	   } 
	   } 
	 & & \\ 
 

    \theutterance \stepcounter{utterance}  

    & & \multicolumn{4}{p{0.6\linewidth}}{\cellcolor[rgb]{0.9,0.9,0.9}{%
	\makecell[{{p{\linewidth}}}]{% 
	  \tt {\tiny [GM$\rangle$B]}  
	 REFUSE: {'A87', 'C00', 'C54', 'B32', 'C01', 'A71', 'B35', 'A03', 'C17'}\\ \tt \\ \tt PROPOSAL: {'A87', 'C00', 'C54', 'B32', 'C01', 'A71', 'B35', 'A03', 'B33'}\\ \tt \\ \tt ARGUMENT: {'You offered B33 as the alternative for that slot—I’m good with that. This keeps your structure intact, includes A03, and stays under the cap (total effort 3354 ≤ 3681). If you can accept this, we’re done.'} 
	  } 
	   } 
	   } 
	 & \\ 
 

    \theutterance \stepcounter{utterance}  

    & & & \multicolumn{4}{p{0.6\linewidth}}{\cellcolor[rgb]{1,0.85,0.72}{%
	\makecell[{{p{\linewidth}}}]{% 
	  \tt {\tiny [GM$\langle$B]}  
	 STRATEGIC REASONING: {'Evaluate their latest proposal S = {A87, C00, C54, B32, C01, A71, B35, A03, B33}. Effort: 4+116+209+285+571+505+664+387+613 = 3354 ≤ 3681. My importance: 668+121+508+583+389+65+262+783+780 = 4159. This is a strong improvement over their earlier sets and close to my targeted gains. Further pushing for A04/C17/A77 would risk deadlock; accepting now secures high value with shared items I like. Best move: agree.'}\\ \tt \\ \tt AGREE: {'A87', 'C00', 'C54', 'B32', 'C01', 'A71', 'B35', 'A03', 'B33'}\\ \tt \\ \tt ARGUMENT: {'Agreed—this set stays well within the limit and balances both sides’ priorities.'} 
	  } 
	   } 
	   } 
	  \\ 
 

    \theutterance \stepcounter{utterance}  

    & & & \multicolumn{2}{p{0.3\linewidth}}{\cellcolor[rgb]{0.95,0.95,0.95}{%
	\makecell[{{p{\linewidth}}}]{% 
	  \tt {\tiny [GM$|$GM]}  
	 game successful 
	  } 
	   } 
	   } 
	 & & \\ 
 

    \theutterance \stepcounter{utterance}  

    & & & \multicolumn{2}{p{0.3\linewidth}}{\cellcolor[rgb]{0.95,0.95,0.95}{%
	\makecell[{{p{\linewidth}}}]{% 
	  \tt {\tiny [GM$|$GM]}  
	 end game 
	  } 
	   } 
	   } 
	 & & \\ 
 

\end{supertabular}
}

\end{document}
