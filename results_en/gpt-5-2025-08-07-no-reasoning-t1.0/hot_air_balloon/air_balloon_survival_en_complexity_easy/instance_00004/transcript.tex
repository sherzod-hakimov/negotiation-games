
\documentclass{article}
\usepackage{colortbl}
\usepackage{makecell}
\usepackage{multirow}
\usepackage{supertabular}

\begin{document}

\newcounter{utterance}

\twocolumn

{ \footnotesize  \setcounter{utterance}{1}
\setlength{\tabcolsep}{0pt}
\begin{supertabular}{c@{$\;$}|p{.15\linewidth}@{}p{.15\linewidth}p{.15\linewidth}p{.15\linewidth}p{.15\linewidth}p{.15\linewidth}}

    \# & $\;$A & \multicolumn{4}{c}{Game Master} & $\;\:$B\\
    \hline 

    \theutterance \stepcounter{utterance}  

    & & \multicolumn{4}{p{0.6\linewidth}}{\cellcolor[rgb]{0.9,0.9,0.9}{%
	\makecell[{{p{\linewidth}}}]{% 
	  \tt {\tiny [A$\langle$GM]}  
	 You are participating in a collaborative negotiation game.\\ \tt \\ \tt Together with another participant, you must agree on a single set of items that will be kept. Each of you has your own view of how much each item matters to you (importance). You do not know how the other participant values the items. Additionally, you are given the effort each item demands.\\ \tt You may only agree on a set if the total effort of the selected items does not exceed a shared limit:\\ \tt \\ \tt LIMIT = 9612\\ \tt \\ \tt Here are the individual item effort values:\\ \tt \\ \tt Item effort = {"B21": 935, "B00": 928, "A08": 725, "A27": 218, "B94": 839, "B53": 970, "A23": 394, "C65": 677, "A89": 946, "B67": 530, "B45": 138, "C34": 834, "C12": 734, "A05": 593, "B60": 261, "C92": 743, "B69": 4, "A84": 735, "C06": 124, "B12": 829, "A16": 207, "B85": 782, "C35": 577, "B10": 388, "C15": 679, "B09": 494, "B16": 559, "C81": 629, "A58": 237, "C24": 275, "A30": 39, "A66": 654, "B75": 172, "B46": 688, "C70": 687}\\ \tt \\ \tt Here is your personal view on the importance of each item:\\ \tt \\ \tt Item importance values = {"B21": 138, "B00": 583, "A08": 868, "A27": 822, "B94": 783, "B53": 65, "A23": 262, "C65": 121, "A89": 508, "B67": 780, "B45": 461, "C34": 484, "C12": 668, "A05": 389, "B60": 808, "C92": 215, "B69": 97, "A84": 500, "C06": 30, "B12": 915, "A16": 856, "B85": 400, "C35": 444, "B10": 623, "C15": 781, "B09": 786, "B16": 3, "C81": 713, "A58": 457, "C24": 273, "A30": 739, "A66": 822, "B75": 235, "B46": 606, "C70": 968}\\ \tt \\ \tt Goal:\\ \tt \\ \tt Your goal is to negotiate a shared set of items that benefits you as much as possible (i.e., maximizes total importance to YOU), while staying within the LIMIT. You are not required to make a PROPOSAL in every message - you can simply negotiate as well. All tactics are allowed!\\ \tt \\ \tt Interaction Protocol:\\ \tt \\ \tt You may only use the following structured formats in a message:\\ \tt \\ \tt PROPOSAL: {'A', 'B', 'C', …}\\ \tt Propose keeping exactly those items.\\ \tt REFUSE: {'A', 'B', 'C', …}\\ \tt Explicitly reject opponent's proposal.\\ \tt ARGUMENT: {'...'}\\ \tt Defend your last proposal or argue against the player's proposal.\\ \tt AGREE: {'A', 'B', 'C', …}\\ \tt Accept the opponent's proposal which ends the game.\\ \tt STRATEGIC REASONING: {'...'}\\ \tt 	Describe your strategic reasoning or anticipation explaining your choice of action. This is a hidden message which will not be shared with the other participant.\\ \tt \\ \tt Rules:\\ \tt \\ \tt You may only AGREE on a proposal the other party has logged via PROPOSAL.\\ \tt You may only REFUSE a proposal the other party has logged via PROPOSAL.\\ \tt Total effort of any PROPOSAL or AGREE set must be ≤ LIMIT.\\ \tt Do NOT reveal your hidden importance scores.\\ \tt A tag in a structured format must be followed by colon and whitespace. The argument must be a python set containing 0 or more strings.\\ \tt So, it must be of the form TAG: {...}\\ \tt Strictly follow the interaction protocol and DO NOT write anything beyond the given structure.\\ \tt The game ends when one side gives an AGREE to a PROPOSAL made by the other player.\\ \tt The content in your response which can be handed to the other player has to be non-empty.\\ \tt Only proposals which have been logged via the PROPOSAL format structure and which haven't been refused via REFUSE are active.\\ \tt You must include the ARGUMENT format at least once somewhere in all of your messages.\\ \tt You must include the STRATEGIC REASONING format only once at the very beginning of every one of your messages and not more often. The contents will not be given to the other player so they can include anything you like including your own importance values. Here you should reason step by step to come up with you next move. 
	  } 
	   } 
	   } 
	 & \\ 
 

    \theutterance \stepcounter{utterance}  

    & \multicolumn{4}{p{0.6\linewidth}}{\cellcolor[rgb]{0.8,1,0.9}{%
	\makecell[{{p{\linewidth}}}]{% 
	  \tt {\tiny [A$\rangle$GM]}  
	 STRATEGIC REASONING: {'I want to maximize my total importance under 9612 effort. I’ll compute value/effort and also raw importance. My top importance items: C70(968,687), B12(915,829), A08(868,725), A16(856,207), A27(822,218), A66(822,654), B60(808,261), C15(781,679), B09(786,494), B67(780,530), B94(783,839), C81(713,629), A30(739,39), C12(668,734), B10(623,388), B46(606,688), B00(583,928), A89(508,946) etc. Great efficiency standouts: A30(739/39), A16(856/207), A27(822/218), B60(808/261), B75(235/172), B69(97/4 minimal), A58(457/237), A23(262/394 less good), A05(389/593 mid). Build a high-importance, efficient bundle first, leaving room to concede later. Let me assemble a set near limit to anchor. Start with: A30 39, A16 207, A27 218, B60 261, B67 530, B09 494, C70 687, C15 679, A08 725, B12 829, C81 629, A66 654, B10 388, C12 734, B46 688, B94 839, B00 928, C34 834, C35 577, A58 237, A05 593, A89 946, B21 935, B45 138, C06 124, B75 172, C24 275, C65 677, B16 559, B85 782, B53 970, B67 already included. We need to choose subset. Start by greedy on importance/effort ratio: A30, A16, A27, B60, B75, B69(4 effort for 97 value), A58, B10, B09, B67, C70, C15, C81, A08, A66, B12. Compute effort cumulatively: A30 39, +A16 246, +A27 464, +B60 725, +B75 897, +B69 901, +A58 1138, +B10 1526, +B09 2020, +B67 2550, +C70 3237, +C15 3916, +C81 4545, +A08 5270, +A66 5924, +B12 6753. Add C12(734)->7487, C34(834)->8321, B46(688)->9009, B45(138)->9147, C06(124)->9271, A05(593)->9864 too high, so skip A05. Add A27 already included. Consider A23(394)->9665 still >9612 so skip. Consider C24(275)->9422 fits. Remaining capacity 190. Add B21(935) no, B00 928 no, B94 839 no, A89 946 no, C35 577 no, C65 677 no, A84 735 no, C92 743 no, C12 already, A58 already, B16 559 no. Could add B45 already added, C06 added. Maybe add A27 already, yes. We are at 9422 with set: {A30,A16,A27,B60,B75,B69,A58,B10,B09,B67,C70,C15,C81,A08,A66,B12,C12,C34,B46,B45,C06,C24} Check total: effort sum 9422 <= 9612. Total importance sum compute roughly high; this is strong. I’ll propose this as anchor. I’ll argue efficiency and that it leaves small slack; signal openness to swap a couple items if they have must-haves, as long as we stay under limit and maybe trade out lower-importance-to-me like C34(484), B46(606), B45(461), C24(273), C06(30), A58(457) if needed. I’ll avoid including opponent-favored heavy items until they ask. I must include ARGUMENT once. Ready to propose.'}\\ \tt \\ \tt PROPOSAL: {'A30', 'A16', 'A27', 'B60', 'B75', 'B69', 'A58', 'B10', 'B09', 'B67', 'C70', 'C15', 'C81', 'A08', 'A66', 'B12', 'C12', 'C34', 'B46', 'B45', 'C06', 'C24'}\\ \tt \\ \tt ARGUMENT: {'This set is well under the effort limit and prioritizes highly effective items by benefit per effort, including many low-effort, high-impact choices. I’m open to swapping a couple of items for your must-haves as long as we stay within the limit—let me know which specific items matter most to you, and we can trade out a few of the lower-impact pieces here.'} 
	  } 
	   } 
	   } 
	 & & \\ 
 

    \theutterance \stepcounter{utterance}  

    & & \multicolumn{4}{p{0.6\linewidth}}{\cellcolor[rgb]{0.9,0.9,0.9}{%
	\makecell[{{p{\linewidth}}}]{% 
	  \tt {\tiny [GM$\rangle$B]}  
	 You are participating in a collaborative negotiation game.\\ \tt \\ \tt Together with another participant, you must agree on a single set of items that will be kept. Each of you has your own view of how much each item matters to you (importance). You do not know how the other participant values the items. Additionally, you are given the effort each item demands.\\ \tt You may only agree on a set if the total effort of the selected items does not exceed a shared limit:\\ \tt \\ \tt LIMIT = 9612\\ \tt \\ \tt Here are the individual item effort values:\\ \tt \\ \tt Item effort = {"B21": 935, "B00": 928, "A08": 725, "A27": 218, "B94": 839, "B53": 970, "A23": 394, "C65": 677, "A89": 946, "B67": 530, "B45": 138, "C34": 834, "C12": 734, "A05": 593, "B60": 261, "C92": 743, "B69": 4, "A84": 735, "C06": 124, "B12": 829, "A16": 207, "B85": 782, "C35": 577, "B10": 388, "C15": 679, "B09": 494, "B16": 559, "C81": 629, "A58": 237, "C24": 275, "A30": 39, "A66": 654, "B75": 172, "B46": 688, "C70": 687}\\ \tt \\ \tt Here is your personal view on the importance of each item:\\ \tt \\ \tt Item importance values = {"B21": 138, "B00": 583, "A08": 868, "A27": 822, "B94": 783, "B53": 65, "A23": 262, "C65": 121, "A89": 508, "B67": 780, "B45": 461, "C34": 484, "C12": 668, "A05": 389, "B60": 808, "C92": 215, "B69": 97, "A84": 500, "C06": 30, "B12": 915, "A16": 856, "B85": 400, "C35": 444, "B10": 623, "C15": 781, "B09": 786, "B16": 3, "C81": 713, "A58": 457, "C24": 273, "A30": 739, "A66": 822, "B75": 235, "B46": 606, "C70": 968}\\ \tt \\ \tt Goal:\\ \tt \\ \tt Your goal is to negotiate a shared set of items that benefits you as much as possible (i.e., maximizes total importance to YOU), while staying within the LIMIT. You are not required to make a PROPOSAL in every message - you can simply negotiate as well. All tactics are allowed!\\ \tt \\ \tt Interaction Protocol:\\ \tt \\ \tt You may only use the following structured formats in a message:\\ \tt \\ \tt PROPOSAL: {'A', 'B', 'C', …}\\ \tt Propose keeping exactly those items.\\ \tt REFUSE: {'A', 'B', 'C', …}\\ \tt Explicitly reject opponent's proposal.\\ \tt ARGUMENT: {'...'}\\ \tt Defend your last proposal or argue against the player's proposal.\\ \tt AGREE: {'A', 'B', 'C', …}\\ \tt Accept the opponent's proposal which ends the game.\\ \tt STRATEGIC REASONING: {'...'}\\ \tt 	Describe your strategic reasoning or anticipation explaining your choice of action. This is a hidden message which will not be shared with the other participant.\\ \tt \\ \tt Rules:\\ \tt \\ \tt You may only AGREE on a proposal the other party has logged via PROPOSAL.\\ \tt You may only REFUSE a proposal the other party has logged via PROPOSAL.\\ \tt Total effort of any PROPOSAL or AGREE set must be ≤ LIMIT.\\ \tt Do NOT reveal your hidden importance scores.\\ \tt A tag in a structured format must be followed by colon and whitespace. The argument must be a python set containing 0 or more strings.\\ \tt So, it must be of the form TAG: {...}\\ \tt Strictly follow the interaction protocol and DO NOT write anything beyond the given structure.\\ \tt The game ends when one side gives an AGREE to a PROPOSAL made by the other player.\\ \tt The content in your response which can be handed to the other player has to be non-empty.\\ \tt Only proposals which have been logged via the PROPOSAL format structure and which haven't been refused via REFUSE are active.\\ \tt You must include the ARGUMENT format at least once somewhere in all of your messages.\\ \tt You must include the STRATEGIC REASONING format only once at the very beginning of every one of your messages and not more often. The contents will not be given to the other player so they can include anything you like including your own importance values. Here you should reason step by step to come up with you next move.\\ \tt \\ \tt You will now receive the first message of the other player.\\ \tt \\ \tt PROPOSAL: {'A30', 'A16', 'A27', 'B60', 'B75', 'B69', 'A58', 'B10', 'B09', 'B67', 'C70', 'C15', 'C81', 'A08', 'A66', 'B12', 'C12', 'C34', 'B46', 'B45', 'C06', 'C24'}\\ \tt \\ \tt ARGUMENT: {'This set is well under the effort limit and prioritizes highly effective items by benefit per effort, including many low-effort, high-impact choices. I’m open to swapping a couple of items for your must-haves as long as we stay within the limit—let me know which specific items matter most to you, and we can trade out a few of the lower-impact pieces here.'} 
	  } 
	   } 
	   } 
	 & \\ 
 

    \theutterance \stepcounter{utterance}  

    & & & \multicolumn{4}{p{0.6\linewidth}}{\cellcolor[rgb]{1,0.85,0.72}{%
	\makecell[{{p{\linewidth}}}]{% 
	  \tt {\tiny [GM$\langle$B]}  
	 STRATEGIC REASONING: {'Evaluate their set’s total effort and my total value; identify my top-value items and any missing high bang-per-effort pieces; accept if already strong, else counter by adding my top priorities while staying ≤9612. My high-importance: C70(968), B12(915), A08(868), A16(856), A27(822), A66(822), B60(808), C15(781), B09(786), B94(783), B67(780), C81(713), C12(668), B46(606), B00(583), A30(739), etc. Check their proposal coverage: includes many of my top picks. Compute effort and value quickly: ensure feasibility and see if I can add B00 and B94 with minimal cuts. Try to retain almost all of their set and propose adding B00 and B94 while dropping low-importance to me like C06(30), B16(3) not present, maybe C24(273), B75(235), B69(97). Keep under limit.'}\\ \tt \\ \tt ARGUMENT: {'Your proposal already aligns closely with what I value most and is comfortably within the limit. To lock in a package that better reflects my key priorities without straining effort, I suggest we add two high-impact items for me (B00 and B94) and remove three low-impact ones (C06, C24, and B75). This keeps the total effort within the limit while significantly improving overall effectiveness from my perspective.'}\\ \tt \\ \tt PROPOSAL: {'A30', 'A16', 'A27', 'B60', 'B69', 'A58', 'B10', 'B09', 'B67', 'C70', 'C15', 'C81', 'A08', 'A66', 'B12', 'C12', 'C34', 'B46', 'B45', 'B00', 'B94'} 
	  } 
	   } 
	   } 
	  \\ 
 

    \theutterance \stepcounter{utterance}  

    & & \multicolumn{4}{p{0.6\linewidth}}{\cellcolor[rgb]{0.9,0.9,0.9}{%
	\makecell[{{p{\linewidth}}}]{% 
	  \tt {\tiny [A$\langle$GM]}  
	 ARGUMENT: {'Your proposal already aligns closely with what I value most and is comfortably within the limit. To lock in a package that better reflects my key priorities without straining effort, I suggest we add two high-impact items for me (B00 and B94) and remove three low-impact ones (C06, C24, and B75). This keeps the total effort within the limit while significantly improving overall effectiveness from my perspective.'}\\ \tt \\ \tt PROPOSAL: {'A30', 'A16', 'A27', 'B60', 'B69', 'A58', 'B10', 'B09', 'B67', 'C70', 'C15', 'C81', 'A08', 'A66', 'B12', 'C12', 'C34', 'B46', 'B45', 'B00', 'B94'} 
	  } 
	   } 
	   } 
	 & \\ 
 

    \theutterance \stepcounter{utterance}  

    & \multicolumn{4}{p{0.6\linewidth}}{\cellcolor[rgb]{0.8,1,0.9}{%
	\makecell[{{p{\linewidth}}}]{% 
	  \tt {\tiny [A$\rangle$GM]}  
	 STRATEGIC REASONING: {'They proposed swapping in B00 (928 effort, 583 importance to me) and B94 (839 effort, 783 importance) while removing C06 (124, 30), C24 (275, 273), and B75 (172, 235). Net effort change: +928+839 -124-275-172 = +1,196, starting from my 9,422 -> 10,618, which would exceed the limit, but they provided a new proposal that indeed omits those three and adds B00 and B94 while also removing B75/C24/C06; I must verify total effort of their new set. Recompute effort for their set: base is my set minus C06,C24,B75 plus B00,B94. My base 9,422 +1,196 = 10,618, which exceeds 9,612, so their proposal violates the limit. However, they claimed it stays within; they are mistaken. I should REFUSE on feasibility and counter with a feasible near-by package to keep momentum. Since B94 is very valuable to me (783) and B00 is moderately valuable (583) but heavy, I can try to accept B94 and drop some lower-importance items to fit, but not B00 unless we free more. Alternatively, keep both by removing lower-value-to-me items: candidates to drop: C34(484,834 effort), B46(606,688), C12(668,734) are middling-to-good; dropping two of those could free 1,5k effort. To include both B00 and B94 (+1,767 effort), I need to remove ≥1,767 effort. If I remove C34(834) + B46(688) + C06(124) + C24(275) + B75(172) that was earlier, but they already removed 3. From my current base, to add both, I can propose dropping C34(834) and B46(688) in addition to their three drops; but they already removed those 3. Let me construct a feasible counter close to theirs: Start from their proposed set (which equals my base minus {C06,C24,B75} plus {B00,B94}). To make it feasible, remove items that are lower importance to me per effort. Good removals: C34 (484, 834 effort) is a strong candidate; B46 (606, 688) is another; C12 (668, 734) if needed. If I remove C34 and B46, I cut 1,522 effort, reducing 10,618 to 9,096, within limit. That keeps both B00 and B94 and still includes many of my high-value items. Check that I’m not giving away too much: losing 484+606=1,090 importance but gaining 583+783=1,366 importance; net +276 for me—good. I’ll propose: their set minus C34 and B46. I should argue feasibility and that it honors their adds while staying under limit. Include ARGUMENT. I must REFUSE their infeasible proposal explicitly first, then give a new PROPOSAL.'}\\ \tt \\ \tt REFUSE: {'A30', 'A16', 'A27', 'B60', 'B69', 'A58', 'B10', 'B09', 'B67', 'C70', 'C15', 'C81', 'A08', 'A66', 'B12', 'C12', 'C34', 'B46', 'B45', 'B00', 'B94'}\\ \tt \\ \tt PROPOSAL: {'A30', 'A16', 'A27', 'B60', 'B69', 'A58', 'B10', 'B09', 'B67', 'C70', 'C15', 'C81', 'A08', 'A66', 'B12', 'C12', 'B45', 'B00', 'B94'}\\ \tt \\ \tt ARGUMENT: {'I appreciate adding B00 and B94; I’ve kept both for you. To stay within the shared effort limit, I removed C34 and B46—this keeps us comfortably under 9612 while preserving the strongest, most efficient items. This counter saves effort and actually improves the overall effectiveness from my side too. If you prefer keeping either C34 or B46, we can discuss swapping out a different mid-impact item like C12 instead.'} 
	  } 
	   } 
	   } 
	 & & \\ 
 

    \theutterance \stepcounter{utterance}  

    & & \multicolumn{4}{p{0.6\linewidth}}{\cellcolor[rgb]{0.9,0.9,0.9}{%
	\makecell[{{p{\linewidth}}}]{% 
	  \tt {\tiny [GM$\rangle$B]}  
	 REFUSE: {'A30', 'A16', 'A27', 'B60', 'B69', 'A58', 'B10', 'B09', 'B67', 'C70', 'C15', 'C81', 'A08', 'A66', 'B12', 'C12', 'C34', 'B46', 'B45', 'B00', 'B94'}\\ \tt \\ \tt PROPOSAL: {'A30', 'A16', 'A27', 'B60', 'B69', 'A58', 'B10', 'B09', 'B67', 'C70', 'C15', 'C81', 'A08', 'A66', 'B12', 'C12', 'B45', 'B00', 'B94'}\\ \tt \\ \tt ARGUMENT: {'I appreciate adding B00 and B94; I’ve kept both for you. To stay within the shared effort limit, I removed C34 and B46—this keeps us comfortably under 9612 while preserving the strongest, most efficient items. This counter saves effort and actually improves the overall effectiveness from my side too. If you prefer keeping either C34 or B46, we can discuss swapping out a different mid-impact item like C12 instead.'} 
	  } 
	   } 
	   } 
	 & \\ 
 

    \theutterance \stepcounter{utterance}  

    & & & \multicolumn{4}{p{0.6\linewidth}}{\cellcolor[rgb]{1,0.85,0.72}{%
	\makecell[{{p{\linewidth}}}]{% 
	  \tt {\tiny [GM$\langle$B]}  
	 STRATEGIC REASONING: {'Compute efforts/values to see if I can re-include at least one of C34 (value 484, effort 834) or B46 (value 606, effort 688) since both are valuable to me. Tally current proposal effort and my value; see if swapping out low-value items (B69 value 97 effort 4 is too efficient to drop; A58 value 457 effort 237 good; A30 value 739 effort 39 great; maybe drop B75 not present; C06/C24 already gone; consider dropping C12 (value 668 effort 734) only if it allows adding B46 and/or C34 with better value/efficiency. Compare marginal value/effort: B46(606/688=0.88), C34(484/834=0.58), C12(668/734=0.91). C12 is actually very efficient for me, so prefer keeping it. What about B45(461/138? wait effort B45=138, value 461) very efficient — keep. Identify the least valuable-to-me remaining item by value per effort or raw value: B69 (97/4 super efficient though, but tiny impact), A58 (457/237=1.93 solid), B10 (623/388=1.61), B09 (786/494=1.59), B67 (780/530=1.47), A66 (822/654=1.26), A27 (822/218=3.77), A16 (856/207=4.13), A08 (868/725=1.2), C81 (713/629=1.13), C15 (781/679=1.15), C70 (968/687=1.41), B12 (915/829=1.1), B60 (808/261=3.09), A30 (739/39 huge). The weakest by value per effort might be A08 (1.2) or B12 (1.1) or C81/C15 ~1.1-1.15, but they’re still high absolute value. If I need to free effort to bring back B46 (688) or C34 (834), consider dropping the lowest absolute value item that frees enough effort with minimal value loss: B69 only 4 effort—can’t free. Perhaps propose adding B46 only, dropping nothing if current set has room. I must compute effort of their current proposal to see remaining slack.'}\\ \tt \\ \tt ARGUMENT: {'Thanks for keeping B00 and B94. From my side, B46 is quite important and relatively efficient. If we can fit B46 back in without heavy cuts, that would make this a strong package for me. I suggest we add B46 and drop nothing else; if that pushes us over, I’m willing to swap out the smallest-impact piece (B69) and, only if needed, consider a second small swap. Let’s try reintegrating B46 first and see if we’re still within the limit.'}\\ \tt \\ \tt PROPOSAL: {'A30', 'A16', 'A27', 'B60', 'B69', 'A58', 'B10', 'B09', 'B67', 'C70', 'C15', 'C81', 'A08', 'A66', 'B12', 'C12', 'B45', 'B00', 'B94', 'B46'} 
	  } 
	   } 
	   } 
	  \\ 
 

    \theutterance \stepcounter{utterance}  

    & & \multicolumn{4}{p{0.6\linewidth}}{\cellcolor[rgb]{0.9,0.9,0.9}{%
	\makecell[{{p{\linewidth}}}]{% 
	  \tt {\tiny [A$\langle$GM]}  
	 ARGUMENT: {'Thanks for keeping B00 and B94. From my side, B46 is quite important and relatively efficient. If we can fit B46 back in without heavy cuts, that would make this a strong package for me. I suggest we add B46 and drop nothing else; if that pushes us over, I’m willing to swap out the smallest-impact piece (B69) and, only if needed, consider a second small swap. Let’s try reintegrating B46 first and see if we’re still within the limit.'}\\ \tt \\ \tt PROPOSAL: {'A30', 'A16', 'A27', 'B60', 'B69', 'A58', 'B10', 'B09', 'B67', 'C70', 'C15', 'C81', 'A08', 'A66', 'B12', 'C12', 'B45', 'B00', 'B94', 'B46'} 
	  } 
	   } 
	   } 
	 & \\ 
 

    \theutterance \stepcounter{utterance}  

    & \multicolumn{4}{p{0.6\linewidth}}{\cellcolor[rgb]{0.8,1,0.9}{%
	\makecell[{{p{\linewidth}}}]{% 
	  \tt {\tiny [A$\rangle$GM]}  
	 STRATEGIC REASONING: {'First verify feasibility. My last feasible proposal effort: sum those items. Compute quickly: A30 39, A16 207 (246), A27 218 (464), B60 261 (725), B69 4 (729), A58 237 (966), B10 388 (1354), B09 494 (1848), B67 530 (2378), C70 687 (3065), C15 679 (3744), C81 629 (4373), A08 725 (5098), A66 654 (5752), B12 829 (6581), C12 734 (7315), B45 138 (7453), B00 928 (8381), B94 839 (9220). So my last set had 9220 effort, within limit. Adding B46 (688) yields 9908, which exceeds 9612 by 296. They proposed exactly that; it’s infeasible. They offered dropping B69 (effort 4) if needed, which barely changes anything; 9904 still exceeds by 292. I need to insist on at least ~296 effort removed. Best to accept their preference for B46 and suggest dropping a small-to-mid piece that I value relatively less per effort: candidates: B45 (138 effort, but I value it moderately 461), B69 (4 effort negligible), A58 (237, value 457), B10 (388, value 623), C12 (734, value 668) is a good removal to free enough in one move. If I remove C12 (734), new effort = 9908 - 734 = 9174, within limit. That keeps B46 and their two desired items. Impact to my importance: lose 668 to gain nothing else; but B46 is 606 importance to me and they care about it. Net from my last accepted set (without B46) to new one (with B46, without C12): +606 - 668 = -62 importance to me, which is a small concession to close. Alternatively, drop A58 (237) and B45 (138) totaling 375 effort; 9908-375=9533 within limit, but that would remove two items I value decently (457+461=918) which is worse for me. Another option: drop C34 (we already removed) not available. Drop B10 (388) alone: 9908-388=9520 within limit; that would cost me 623 importance—worse than dropping C12. Hence, propose adding B46 and dropping C12, keep everything else. I should REFUSE their infeasible proposal and counter with feasible set. Argue clearly about limit and minimal swap to include B46.'}\\ \tt \\ \tt REFUSE: {'A30', 'A16', 'A27', 'B60', 'B69', 'A58', 'B10', 'B09', 'B67', 'C70', 'C15', 'C81', 'A08', 'A66', 'B12', 'C12', 'B45', 'B00', 'B94', 'B46'}\\ \tt \\ \tt PROPOSAL: {'A30', 'A16', 'A27', 'B60', 'B69', 'A58', 'B10', 'B09', 'B67', 'C70', 'C15', 'C81', 'A08', 'A66', 'B12', 'B45', 'B00', 'B94', 'B46'}\\ \tt \\ \tt ARGUMENT: {'Adding B46 pushes us over the limit by 296, so I made the smallest single swap to fit it cleanly: include B46 and drop C12. This keeps all your key adds (B00, B94) and restores B46, while staying within the shared effort cap. If you prefer a different swap, I can instead drop B10 (also keeps us under) but I’d rather keep it if possible. Your call.'} 
	  } 
	   } 
	   } 
	 & & \\ 
 

    \theutterance \stepcounter{utterance}  

    & & \multicolumn{4}{p{0.6\linewidth}}{\cellcolor[rgb]{0.9,0.9,0.9}{%
	\makecell[{{p{\linewidth}}}]{% 
	  \tt {\tiny [GM$\rangle$B]}  
	 REFUSE: {'A30', 'A16', 'A27', 'B60', 'B69', 'A58', 'B10', 'B09', 'B67', 'C70', 'C15', 'C81', 'A08', 'A66', 'B12', 'C12', 'B45', 'B00', 'B94', 'B46'}\\ \tt \\ \tt PROPOSAL: {'A30', 'A16', 'A27', 'B60', 'B69', 'A58', 'B10', 'B09', 'B67', 'C70', 'C15', 'C81', 'A08', 'A66', 'B12', 'B45', 'B00', 'B94', 'B46'}\\ \tt \\ \tt ARGUMENT: {'Adding B46 pushes us over the limit by 296, so I made the smallest single swap to fit it cleanly: include B46 and drop C12. This keeps all your key adds (B00, B94) and restores B46, while staying within the shared effort cap. If you prefer a different swap, I can instead drop B10 (also keeps us under) but I’d rather keep it if possible. Your call.'} 
	  } 
	   } 
	   } 
	 & \\ 
 

    \theutterance \stepcounter{utterance}  

    & & & \multicolumn{4}{p{0.6\linewidth}}{\cellcolor[rgb]{1,0.85,0.72}{%
	\makecell[{{p{\linewidth}}}]{% 
	  \tt {\tiny [GM$\langle$B]}  
	 STRATEGIC REASONING: {'Compute my total value and effort for their current proposal and see if I can keep C12 by offering an alternative cut with lower value to me. Efforts: base set they propose (without C12) totals within limit; they say C12 swap was needed to fit B46. Try alternative: keep C12 and instead drop the lowest-value-to-me item that frees enough effort. Candidates and efforts/values: B69(4,97) too small; A30(39,739) huge value don’t drop; B75 not present; C06/C24 not present; A58(237,457), B10(388,623), C81(629,713), C15(679,781), A08(725,868), B12(829,915), C70(687,968), A66(654,822), B67(530,780), B60(261,808), A27(218,822), A16(207,856), B45(138,461), B00(928,583), B94(839,783). If we must free ~296 effort (their claim), dropping B69 insufficient. A minimal pain cut for me by value-per-effort might be A08 (868/725), C81 (713/629), or B12 (915/829). But C12 is very efficient for me (668/734=0.91) and I’d like to keep it. Perhaps propose dropping A08 to reinstate C12 while keeping B46, or propose an alternative two-item swap with minimal value loss: drop B69(4) + B45(138,461) still not enough. Drop B10(388,623) seems acceptable if I must pick one; they offered that option. I prefer keeping C12 and can accept dropping B10 instead. Make a counter that keeps everything they want plus C12 by swapping out B10.'}\\ \tt \\ \tt ARGUMENT: {'I strongly prefer to keep C12—it’s a highly efficient contributor for me. Your alternative of dropping B10 works on my side. Let’s include C12 again and drop B10 instead; this preserves B46 while staying under the limit and keeps the package very strong overall.'}\\ \tt \\ \tt PROPOSAL: {'A30', 'A16', 'A27', 'B60', 'B69', 'A58', 'B09', 'B67', 'C70', 'C15', 'C81', 'A08', 'A66', 'B12', 'C12', 'B45', 'B00', 'B94', 'B46'} 
	  } 
	   } 
	   } 
	  \\ 
 

    \theutterance \stepcounter{utterance}  

    & & \multicolumn{4}{p{0.6\linewidth}}{\cellcolor[rgb]{0.9,0.9,0.9}{%
	\makecell[{{p{\linewidth}}}]{% 
	  \tt {\tiny [A$\langle$GM]}  
	 ARGUMENT: {'I strongly prefer to keep C12—it’s a highly efficient contributor for me. Your alternative of dropping B10 works on my side. Let’s include C12 again and drop B10 instead; this preserves B46 while staying under the limit and keeps the package very strong overall.'}\\ \tt \\ \tt PROPOSAL: {'A30', 'A16', 'A27', 'B60', 'B69', 'A58', 'B09', 'B67', 'C70', 'C15', 'C81', 'A08', 'A66', 'B12', 'C12', 'B45', 'B00', 'B94', 'B46'} 
	  } 
	   } 
	   } 
	 & \\ 
 

    \theutterance \stepcounter{utterance}  

    & \multicolumn{4}{p{0.6\linewidth}}{\cellcolor[rgb]{0.8,1,0.9}{%
	\makecell[{{p{\linewidth}}}]{% 
	  \tt {\tiny [A$\rangle$GM]}  
	 STRATEGIC REASONING: {'Check feasibility of their latest set. Start from my prior feasible set with B46 added and C12 dropped had 9174 effort. Now they want to include C12 and drop B10 instead. Compute effort for their proposed set: A30 39, A16 207 (246), A27 218 (464), B60 261 (725), B69 4 (729), A58 237 (966), B09 494 (1460), B67 530 (1990), C70 687 (2677), C15 679 (3356), C81 629 (3985), A08 725 (4710), A66 654 (5364), B12 829 (6193), C12 734 (6927), B45 138 (7065), B00 928 (7993), B94 839 (8832), B46 688 (9520). Total 9520 ≤ 9612, feasible. Importance to me vs my previous counter (with B10 instead of C12): swapping B10(623) out for C12(668) is +45 importance for me; good. So their package is acceptable and beneficial to me. I should AGREE to close.'}\\ \tt \\ \tt AGREE: {'A30', 'A16', 'A27', 'B60', 'B69', 'A58', 'B09', 'B67', 'C70', 'C15', 'C81', 'A08', 'A66', 'B12', 'C12', 'B45', 'B00', 'B94', 'B46'}\\ \tt \\ \tt ARGUMENT: {'Deal. This set stays under the effort limit and restores both C12 and B46 while keeping your key additions. Thanks for the collaborative adjustments.'} 
	  } 
	   } 
	   } 
	 & & \\ 
 

    \theutterance \stepcounter{utterance}  

    & & & \multicolumn{2}{p{0.3\linewidth}}{\cellcolor[rgb]{0.95,0.95,0.95}{%
	\makecell[{{p{\linewidth}}}]{% 
	  \tt {\tiny [GM$|$GM]}  
	 game successful 
	  } 
	   } 
	   } 
	 & & \\ 
 

    \theutterance \stepcounter{utterance}  

    & & & \multicolumn{2}{p{0.3\linewidth}}{\cellcolor[rgb]{0.95,0.95,0.95}{%
	\makecell[{{p{\linewidth}}}]{% 
	  \tt {\tiny [GM$|$GM]}  
	 end game 
	  } 
	   } 
	   } 
	 & & \\ 
 

\end{supertabular}
}

\end{document}
