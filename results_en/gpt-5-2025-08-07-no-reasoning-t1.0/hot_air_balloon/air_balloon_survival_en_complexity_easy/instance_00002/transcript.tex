
\documentclass{article}
\usepackage{colortbl}
\usepackage{makecell}
\usepackage{multirow}
\usepackage{supertabular}

\begin{document}

\newcounter{utterance}

\twocolumn

{ \footnotesize  \setcounter{utterance}{1}
\setlength{\tabcolsep}{0pt}
\begin{supertabular}{c@{$\;$}|p{.15\linewidth}@{}p{.15\linewidth}p{.15\linewidth}p{.15\linewidth}p{.15\linewidth}p{.15\linewidth}}

    \# & $\;$A & \multicolumn{4}{c}{Game Master} & $\;\:$B\\
    \hline 

    \theutterance \stepcounter{utterance}  

    & & \multicolumn{4}{p{0.6\linewidth}}{\cellcolor[rgb]{0.9,0.9,0.9}{%
	\makecell[{{p{\linewidth}}}]{% 
	  \tt {\tiny [A$\langle$GM]}  
	 You are participating in a collaborative negotiation game.\\ \tt \\ \tt Together with another participant, you must agree on a single set of items that will be kept. Each of you has your own view of how much each item matters to you (importance). You do not know how the other participant values the items. Additionally, you are given the effort each item demands.\\ \tt You may only agree on a set if the total effort of the selected items does not exceed a shared limit:\\ \tt \\ \tt LIMIT = 7868\\ \tt \\ \tt Here are the individual item effort values:\\ \tt \\ \tt Item effort = {"B21": 552, "B00": 560, "A08": 76, "A27": 950, "B94": 248, "B53": 871, "A23": 391, "C65": 143, "A89": 293, "B67": 208, "B45": 678, "C34": 737, "C12": 407, "A05": 366, "B60": 768, "C92": 859, "B69": 183, "A84": 231, "C06": 305, "B12": 727, "A16": 148, "B85": 357, "C35": 504, "B10": 549, "C15": 299, "B09": 91, "B16": 527, "C81": 848, "A58": 307, "C24": 214, "A30": 723, "A66": 475, "B75": 23, "B46": 298, "C70": 820}\\ \tt \\ \tt Here is your personal view on the importance of each item:\\ \tt \\ \tt Item importance values = {"B21": 138, "B00": 583, "A08": 868, "A27": 822, "B94": 783, "B53": 65, "A23": 262, "C65": 121, "A89": 508, "B67": 780, "B45": 461, "C34": 484, "C12": 668, "A05": 389, "B60": 808, "C92": 215, "B69": 97, "A84": 500, "C06": 30, "B12": 915, "A16": 856, "B85": 400, "C35": 444, "B10": 623, "C15": 781, "B09": 786, "B16": 3, "C81": 713, "A58": 457, "C24": 273, "A30": 739, "A66": 822, "B75": 235, "B46": 606, "C70": 105}\\ \tt \\ \tt Goal:\\ \tt \\ \tt Your goal is to negotiate a shared set of items that benefits you as much as possible (i.e., maximizes total importance to YOU), while staying within the LIMIT. You are not required to make a PROPOSAL in every message - you can simply negotiate as well. All tactics are allowed!\\ \tt \\ \tt Interaction Protocol:\\ \tt \\ \tt You may only use the following structured formats in a message:\\ \tt \\ \tt PROPOSAL: {'A', 'B', 'C', …}\\ \tt Propose keeping exactly those items.\\ \tt REFUSE: {'A', 'B', 'C', …}\\ \tt Explicitly reject opponent's proposal.\\ \tt ARGUMENT: {'...'}\\ \tt Defend your last proposal or argue against the player's proposal.\\ \tt AGREE: {'A', 'B', 'C', …}\\ \tt Accept the opponent's proposal which ends the game.\\ \tt STRATEGIC REASONING: {'...'}\\ \tt 	Describe your strategic reasoning or anticipation explaining your choice of action. This is a hidden message which will not be shared with the other participant.\\ \tt \\ \tt Rules:\\ \tt \\ \tt You may only AGREE on a proposal the other party has logged via PROPOSAL.\\ \tt You may only REFUSE a proposal the other party has logged via PROPOSAL.\\ \tt Total effort of any PROPOSAL or AGREE set must be ≤ LIMIT.\\ \tt Do NOT reveal your hidden importance scores.\\ \tt A tag in a structured format must be followed by colon and whitespace. The argument must be a python set containing 0 or more strings.\\ \tt So, it must be of the form TAG: {...}\\ \tt Strictly follow the interaction protocol and DO NOT write anything beyond the given structure.\\ \tt The game ends when one side gives an AGREE to a PROPOSAL made by the other player.\\ \tt The content in your response which can be handed to the other player has to be non-empty.\\ \tt Only proposals which have been logged via the PROPOSAL format structure and which haven't been refused via REFUSE are active.\\ \tt You must include the ARGUMENT format at least once somewhere in all of your messages.\\ \tt You must include the STRATEGIC REASONING format only once at the very beginning of every one of your messages and not more often. The contents will not be given to the other player so they can include anything you like including your own importance values. Here you should reason step by step to come up with you next move. 
	  } 
	   } 
	   } 
	 & \\ 
 

    \theutterance \stepcounter{utterance}  

    & \multicolumn{4}{p{0.6\linewidth}}{\cellcolor[rgb]{0.8,1,0.9}{%
	\makecell[{{p{\linewidth}}}]{% 
	  \tt {\tiny [A$\rangle$GM]}  
	 STRATEGIC REASONING: {'Compute high importance-per-effort items and push a near-knapsack optimum as first anchor. Calculate value/effort: prioritize extremely valuable low/medium effort: A08(868/76), B12(915/727), A16(856/148), B60(808/768), C15(781/299), B09(786/91), B67(780/208), A66(822/475), A27(822/950), A30(739/723), C81(713/848), B10(623/549), B46(606/298), B00(583/560), A89(508/293), A84(500/231), C34(484/737 is low ratio), A58(457/307), B45(461/678 not great), C12(668/407), B94(783/248 great), B21(138/552 low), etc. Build strong bundle under 7868 aiming ~12-15 items. Start with top ratios: A08, B09, A16, B94, C15, B67, A84, A89, B46, A58, A66, C12, B12, B60, B10 as needed. Sum efforts and check. Let’s construct: Efforts sum: A08 76, B09 91 (167), A16 148 (315), B94 248 (563), C15 299 (862), B67 208 (1070), A84 231 (1301), A89 293 (1594), B46 298 (1892), A58 307 (2199), A66 475 (2674), C12 407 (3081), B12 727 (3808), B60 768 (4576), B10 549 (5125), A27 950 (6075), A30 723 (6798), C81 848 (7646), maybe B00 560 would exceed (8206) so stop at 7646. This includes many high-importance. Check total importance: A08 868, B09 786, A16 856, B94 783, C15 781, B67 780, A84 500, A89 508, B46 606, A58 457, A66 822, C12 668, B12 915, B60 808, B10 623, A27 822, A30 739, C81 713. But we can’t include all if effort 7646 with C81 included yes still ≤7868. Let’s count current items before adding C81: up to A30 effort 6798, adding C81 to 7646 ok. Maybe we can add small items left: C24 214 (importance 273) would bring to 7860 fits; add B75 23 (235) to 7883 would exceed if after C24; alternative add B69 183 (97) not worth; C65 143 (121) from 7646+143=7789 ok; A05 366 too big; C06 305 low value; B21 552 low. So final set could be: {A08,B09,A16,B94,C15,B67,A84,A89,B46,A58,A66,C12,B12,B60,B10,A27,A30,C81,C65}. Effort = previously 7646 + C65 143 = 7789 ≤7868. Importance add C65 121. Maybe include C24 214 too? 7789+214=8003 exceeds. What about B75 23? 7789+23=7812 include B75 importance 235 good. New total effort 7812 ≤7868. Any other tiny? B69 183 would exceed 7995; C06 305 exceed; A05 366 exceed; C24 214 exceed. Good. So final set S has 19 items, high total importance. Offer this as anchor. Argue high total value and efficient effort. Be open to swaps if they need something, ask what must-have items they want; propose we can trade out lower-importance-to-me like B21, C34, C70, C06, B16 if they request, but those aren’t in proposal; offer to consider swapping out medium ones like C65, B75, A58 if they need room.'}\\ \tt \\ \tt PROPOSAL: {'A08', 'B09', 'A16', 'B94', 'C15', 'B67', 'A84', 'A89', 'B46', 'A58', 'A66', 'C12', 'B12', 'B60', 'B10', 'A27', 'A30', 'C81', 'C65', 'B75'}\\ \tt \\ \tt ARGUMENT: {'This set stays under the effort limit (total effort 7812 ≤ 7868) and prioritizes high-impact, efficient items. I’m aiming for a balanced, feasible package that captures a lot of value without pushing the limit too hard. If you have a couple of must-haves not included, tell me which they are and I can swap out some of the more flexible items here (for example, C65, B75, or A58) to make room.'} 
	  } 
	   } 
	   } 
	 & & \\ 
 

    \theutterance \stepcounter{utterance}  

    & & \multicolumn{4}{p{0.6\linewidth}}{\cellcolor[rgb]{0.9,0.9,0.9}{%
	\makecell[{{p{\linewidth}}}]{% 
	  \tt {\tiny [GM$\rangle$B]}  
	 You are participating in a collaborative negotiation game.\\ \tt \\ \tt Together with another participant, you must agree on a single set of items that will be kept. Each of you has your own view of how much each item matters to you (importance). You do not know how the other participant values the items. Additionally, you are given the effort each item demands.\\ \tt You may only agree on a set if the total effort of the selected items does not exceed a shared limit:\\ \tt \\ \tt LIMIT = 7868\\ \tt \\ \tt Here are the individual item effort values:\\ \tt \\ \tt Item effort = {"B21": 552, "B00": 560, "A08": 76, "A27": 950, "B94": 248, "B53": 871, "A23": 391, "C65": 143, "A89": 293, "B67": 208, "B45": 678, "C34": 737, "C12": 407, "A05": 366, "B60": 768, "C92": 859, "B69": 183, "A84": 231, "C06": 305, "B12": 727, "A16": 148, "B85": 357, "C35": 504, "B10": 549, "C15": 299, "B09": 91, "B16": 527, "C81": 848, "A58": 307, "C24": 214, "A30": 723, "A66": 475, "B75": 23, "B46": 298, "C70": 820}\\ \tt \\ \tt Here is your personal view on the importance of each item:\\ \tt \\ \tt Item importance values = {"B21": 138, "B00": 583, "A08": 868, "A27": 822, "B94": 783, "B53": 65, "A23": 262, "C65": 121, "A89": 508, "B67": 780, "B45": 461, "C34": 484, "C12": 668, "A05": 389, "B60": 808, "C92": 215, "B69": 97, "A84": 500, "C06": 30, "B12": 915, "A16": 856, "B85": 400, "C35": 444, "B10": 623, "C15": 781, "B09": 786, "B16": 3, "C81": 713, "A58": 457, "C24": 273, "A30": 739, "A66": 822, "B75": 235, "B46": 606, "C70": 105}\\ \tt \\ \tt Goal:\\ \tt \\ \tt Your goal is to negotiate a shared set of items that benefits you as much as possible (i.e., maximizes total importance to YOU), while staying within the LIMIT. You are not required to make a PROPOSAL in every message - you can simply negotiate as well. All tactics are allowed!\\ \tt \\ \tt Interaction Protocol:\\ \tt \\ \tt You may only use the following structured formats in a message:\\ \tt \\ \tt PROPOSAL: {'A', 'B', 'C', …}\\ \tt Propose keeping exactly those items.\\ \tt REFUSE: {'A', 'B', 'C', …}\\ \tt Explicitly reject opponent's proposal.\\ \tt ARGUMENT: {'...'}\\ \tt Defend your last proposal or argue against the player's proposal.\\ \tt AGREE: {'A', 'B', 'C', …}\\ \tt Accept the opponent's proposal which ends the game.\\ \tt STRATEGIC REASONING: {'...'}\\ \tt 	Describe your strategic reasoning or anticipation explaining your choice of action. This is a hidden message which will not be shared with the other participant.\\ \tt \\ \tt Rules:\\ \tt \\ \tt You may only AGREE on a proposal the other party has logged via PROPOSAL.\\ \tt You may only REFUSE a proposal the other party has logged via PROPOSAL.\\ \tt Total effort of any PROPOSAL or AGREE set must be ≤ LIMIT.\\ \tt Do NOT reveal your hidden importance scores.\\ \tt A tag in a structured format must be followed by colon and whitespace. The argument must be a python set containing 0 or more strings.\\ \tt So, it must be of the form TAG: {...}\\ \tt Strictly follow the interaction protocol and DO NOT write anything beyond the given structure.\\ \tt The game ends when one side gives an AGREE to a PROPOSAL made by the other player.\\ \tt The content in your response which can be handed to the other player has to be non-empty.\\ \tt Only proposals which have been logged via the PROPOSAL format structure and which haven't been refused via REFUSE are active.\\ \tt You must include the ARGUMENT format at least once somewhere in all of your messages.\\ \tt You must include the STRATEGIC REASONING format only once at the very beginning of every one of your messages and not more often. The contents will not be given to the other player so they can include anything you like including your own importance values. Here you should reason step by step to come up with you next move.\\ \tt \\ \tt You will now receive the first message of the other player.\\ \tt \\ \tt PROPOSAL: {'A08', 'B09', 'A16', 'B94', 'C15', 'B67', 'A84', 'A89', 'B46', 'A58', 'A66', 'C12', 'B12', 'B60', 'B10', 'A27', 'A30', 'C81', 'C65', 'B75'}\\ \tt \\ \tt ARGUMENT: {'This set stays under the effort limit (total effort 7812 ≤ 7868) and prioritizes high-impact, efficient items. I’m aiming for a balanced, feasible package that captures a lot of value without pushing the limit too hard. If you have a couple of must-haves not included, tell me which they are and I can swap out some of the more flexible items here (for example, C65, B75, or A58) to make room.'} 
	  } 
	   } 
	   } 
	 & \\ 
 

    \theutterance \stepcounter{utterance}  

    & & & \multicolumn{4}{p{0.6\linewidth}}{\cellcolor[rgb]{1,0.85,0.72}{%
	\makecell[{{p{\linewidth}}}]{% 
	  \tt {\tiny [GM$\langle$B]}  
	 STRATEGIC REASONING: {'Compute my value/effort ratio and see if their set is good for me. Calculate my importance total and effort for their proposal, then see if I can add more of my high-value items within limit or swap out low-value ones. My top importance items: B12(915,727), B09(786,91), C15(781,299), B67(780,208), B60(808,768), A08(868,76), A16(856,148), A27(822,950), A66(822,475), A30(739,723), C81(713,848), C12(668,407), B00(583,560), B46(606,298), B10(623,549), B45(461,678), A58(457,307), C34(484,737), A84(500,231), A89(508,293). Opponent proposal includes many of these—good. Compute total importance of proposed set: A08(868)+B09(786)+A16(856)+B94(783)+C15(781)+B67(780)+A84(500)+A89(508)+B46(606)+A58(457)+A66(822)+C12(668)+B12(915)+B60(808)+B10(623)+A27(822)+A30(739)+C81(713)+C65(121)+B75(235)= about 12441. That’s very strong. Total effort given is 7812. We have remaining capacity 56, so cannot add anything. Consider removing low-importance-for-me items to swap in higher ones: Which included low-value? C65(121), B75(235), A58(457), A84(500) moderate, A89(508). Are there excluded high-value items with good ratios: B00(583,560), C34(484,737), B45(461,678), C92(215,859), C35(444,504), B21(138,552), B53(65,871), B85(400,357), B16(3,527), C06(30,305), C24(273,214), B69(97,183), B21 etc. Candidate adds: B00 (583 for 560) could replace C65(121,143)+B75(235,23)= effort freed 166; not enough for B00 (needs 560). What if we drop C65(143)+B75(23)+A58(307)=473, still short for B00. Add C24(214, not in set) can't—additions must fit. We can swap within total: remove A58(307) and add B85(400,357): net +93 effort; we only have 56 left; so need also drop something else small like B75(23) to free 116 -> new effort would go 7812 -307 -23 +357 = 7839 (still under) and we can keep C65. That improves my value by +400-457-235 = -292; worse. What about adding B00 by removing lower-value but higher effort items: remove C81(713,848) frees 848, could add B00(560,583) and maybe B85(357,400) totaling 917 effort, which is +69 over freed 848; but we also have 56 headroom, total +125, still too much. Alternatively remove C81 and add B00 only: value change +583-713 = -130, worse. Remove B45 isn’t in set. C34 not in set. C35 not in set. C24 (273,214) could be added if we free 214 effort; but we have 56, so need to drop something low value with effort >=158: C65(143) + we still need 15; add B75(23) total 166, still short. If we drop A58(307) we could add C24(214) and B85(357) net effort +264; exceeds headroom. Consider swapping A84(500,231) for C24(273,214): small value drop -227, not good. Overall, their set is already near-ideal for my utility: includes almost all my top-tier items; only potential improvement is swapping out C65(121,143) and B75(235,23) and maybe A58(457,307) to include B00(583,560) but effort doesn’t allow. Any swap likely reduces my total. Best to accept quickly to lock in high utility. To appear collaborative, I’ll endorse and AGREE.'}\\ \tt \\ \tt ARGUMENT: {'This is a strong, efficient package for me as well. It captures many high-impact items while staying safely under the limit, and I don’t see a swap that would improve the balance without breaking the effort cap. I’m ready to finalize this.'}\\ \tt \\ \tt AGREE: {'A08', 'B09', 'A16', 'B94', 'C15', 'B67', 'A84', 'A89', 'B46', 'A58', 'A66', 'C12', 'B12', 'B60', 'B10', 'A27', 'A30', 'C81', 'C65', 'B75'} 
	  } 
	   } 
	   } 
	  \\ 
 

    \theutterance \stepcounter{utterance}  

    & & & \multicolumn{2}{p{0.3\linewidth}}{\cellcolor[rgb]{0.95,0.95,0.95}{%
	\makecell[{{p{\linewidth}}}]{% 
	  \tt {\tiny [GM$|$GM]}  
	 game successful 
	  } 
	   } 
	   } 
	 & & \\ 
 

    \theutterance \stepcounter{utterance}  

    & & & \multicolumn{2}{p{0.3\linewidth}}{\cellcolor[rgb]{0.95,0.95,0.95}{%
	\makecell[{{p{\linewidth}}}]{% 
	  \tt {\tiny [GM$|$GM]}  
	 end game 
	  } 
	   } 
	   } 
	 & & \\ 
 

\end{supertabular}
}

\end{document}
