
\documentclass{article}
\usepackage{colortbl}
\usepackage{makecell}
\usepackage{multirow}
\usepackage{supertabular}

\begin{document}

\newcounter{utterance}

\twocolumn

{ \footnotesize  \setcounter{utterance}{1}
\setlength{\tabcolsep}{0pt}
\begin{supertabular}{c@{$\;$}|p{.15\linewidth}@{}p{.15\linewidth}p{.15\linewidth}p{.15\linewidth}p{.15\linewidth}p{.15\linewidth}}

    \# & $\;$A & \multicolumn{4}{c}{Game Master} & $\;\:$B\\
    \hline 

    \theutterance \stepcounter{utterance}  

    & & \multicolumn{4}{p{0.6\linewidth}}{\cellcolor[rgb]{0.9,0.9,0.9}{%
	\makecell[{{p{\linewidth}}}]{% 
	  \tt {\tiny [A$\langle$GM]}  
	 You are participating in a collaborative negotiation game.\\ \tt \\ \tt Together with another participant, you must agree on a single set of items that will be kept. Each of you has your own view of how much each item matters to you (importance). You do not know how the other participant values the items. Additionally, you are given the effort each item demands.\\ \tt You may only agree on a set if the total effort of the selected items does not exceed a shared limit:\\ \tt \\ \tt LIMIT = 8170\\ \tt \\ \tt Here are the individual item effort values:\\ \tt \\ \tt Item effort = {"B21": 594, "B00": 551, "A08": 634, "A27": 204, "B94": 557, "B53": 440, "A23": 734, "C65": 893, "A89": 246, "B67": 592, "B45": 143, "C34": 568, "C12": 472, "A05": 401, "B60": 729, "C92": 201, "B69": 85, "A84": 641, "C06": 958, "B12": 80, "A16": 157, "B85": 806, "C35": 684, "B10": 59, "C15": 31, "B09": 765, "B16": 416, "C81": 392, "A58": 428, "C24": 699, "A30": 141, "A66": 606, "B75": 612, "B46": 133, "C70": 689}\\ \tt \\ \tt Here is your personal view on the importance of each item:\\ \tt \\ \tt Item importance values = {"B21": 138, "B00": 583, "A08": 868, "A27": 822, "B94": 783, "B53": 65, "A23": 262, "C65": 121, "A89": 508, "B67": 780, "B45": 461, "C34": 484, "C12": 668, "A05": 389, "B60": 808, "C92": 215, "B69": 97, "A84": 500, "C06": 30, "B12": 915, "A16": 856, "B85": 400, "C35": 444, "B10": 623, "C15": 781, "B09": 786, "B16": 3, "C81": 713, "A58": 457, "C24": 273, "A30": 739, "A66": 822, "B75": 235, "B46": 606, "C70": 105}\\ \tt \\ \tt Goal:\\ \tt \\ \tt Your goal is to negotiate a shared set of items that benefits you as much as possible (i.e., maximizes total importance to YOU), while staying within the LIMIT. You are not required to make a PROPOSAL in every message - you can simply negotiate as well. All tactics are allowed!\\ \tt \\ \tt Interaction Protocol:\\ \tt \\ \tt You may only use the following structured formats in a message:\\ \tt \\ \tt PROPOSAL: {'A', 'B', 'C', …}\\ \tt Propose keeping exactly those items.\\ \tt REFUSE: {'A', 'B', 'C', …}\\ \tt Explicitly reject opponent's proposal.\\ \tt ARGUMENT: {'...'}\\ \tt Defend your last proposal or argue against the player's proposal.\\ \tt AGREE: {'A', 'B', 'C', …}\\ \tt Accept the opponent's proposal which ends the game.\\ \tt STRATEGIC REASONING: {'...'}\\ \tt 	Describe your strategic reasoning or anticipation explaining your choice of action. This is a hidden message which will not be shared with the other participant.\\ \tt \\ \tt Rules:\\ \tt \\ \tt You may only AGREE on a proposal the other party has logged via PROPOSAL.\\ \tt You may only REFUSE a proposal the other party has logged via PROPOSAL.\\ \tt Total effort of any PROPOSAL or AGREE set must be ≤ LIMIT.\\ \tt Do NOT reveal your hidden importance scores.\\ \tt A tag in a structured format must be followed by colon and whitespace. The argument must be a python set containing 0 or more strings.\\ \tt So, it must be of the form TAG: {...}\\ \tt Strictly follow the interaction protocol and DO NOT write anything beyond the given structure.\\ \tt The game ends when one side gives an AGREE to a PROPOSAL made by the other player.\\ \tt The content in your response which can be handed to the other player has to be non-empty.\\ \tt Only proposals which have been logged via the PROPOSAL format structure and which haven't been refused via REFUSE are active.\\ \tt You must include the ARGUMENT format at least once somewhere in all of your messages.\\ \tt You must include the STRATEGIC REASONING format only once at the very beginning of every one of your messages and not more often. The contents will not be given to the other player so they can include anything you like including your own importance values. Here you should reason step by step to come up with you next move. 
	  } 
	   } 
	   } 
	 & \\ 
 

    \theutterance \stepcounter{utterance}  

    & \multicolumn{4}{p{0.6\linewidth}}{\cellcolor[rgb]{0.8,1,0.9}{%
	\makecell[{{p{\linewidth}}}]{% 
	  \tt {\tiny [A$\rangle$GM]}  
	 STRATEGIC REASONING: {'We need a high-importance-per-effort pack under 8170. Compute value/effort ratios and pick top items first. My top values: B12(915,80), A08(868,634), A16(856,157), A27(822,204), A66(822,606), B60(808,729), C15(781,31), B67(780,592), B09(786,765), B10(623,59), B46(606,133), B94(783,557), C81(713,392), C12(668,472), A30(739,141), B00(583,551), A89(508,246), A84(500,641), C34(484,568), A58(457,428), B45(461,143), C35(444,684), A05(389,401), B85(400,806), C24(273,699), A23(262,734), C92(215,201), C65(121,893), C70(105,689), B21(138,594), B69(97,85), C06(30,958), B16(3,416), B75(235,612), B53(65,440). Greedy by ratio: C15, B12, B10, A16, A30, B46, A27, A89, C92, C81, A08, A66, B94, B67, B45, C12, B00, A58, B60, C34, A84, B09, etc, while tracking effort. Build: start with zero effort 0. Add C15(31), B12(80)=111, B10(59)=170, A16(157)=327, A30(141)=468, B46(133)=601, A27(204)=805, A89(246)=1051, C92(201)=1252, C81(392)=1644, A08(634)=2278, A66(606)=2884, B94(557)=3441, B67(592)=4033, B45(143)=4176, C12(472)=4648, B00(551)=5199, A58(428)=5627, B60(729)=6356, C34(568)=6924, A84(641)=7565. Remaining capacity 8170-7565=605. Next best within 605: B21(594) low value 138, B75(612) too big, C35(684) too big, B09(765) too big, A23(734) too big, C24(699) too big, C70(689) too big, C65(893) too big, B85(806) too big, A05(401) value 389 fits -> new 7966, remaining 204. With 204 left: A27 already included, C92 included, B69(85) value 97 fits -> 8051 remaining 119. Any ≤119 not yet included: none other than maybe? Items with small effort: C15(31) included, B10(59) included, B12(80) included, A30(141) too big, B46(133) too big, A16 included. So final set effort 8051 ≤ 8170. Total value compute roughly very high. This is a strong proposal. If opponent resists, I can swap lower-value like B21 not included; we included A05 and B69 late; we can offer to drop A84(641,500) or C34(568,484) to make room for their priorities. I’ll present proposal with argument emphasizing efficiency and room for adjustments.'}\\ \tt \\ \tt PROPOSAL: {'C15', 'B12', 'B10', 'A16', 'A30', 'B46', 'A27', 'A89', 'C92', 'C81', 'A08', 'A66', 'B94', 'B67', 'B45', 'C12', 'B00', 'A58', 'B60', 'C34', 'A84', 'A05', 'B69'}\\ \tt \\ \tt ARGUMENT: {'I’ve proposed a balanced, high-efficiency set that stays within the limit (total effort 8051 ≤ 8170). It includes many low-effort, high-impact items to leave some slack for adjustments. If you need space for one of your priorities, I can consider swapping out A05, B69, or potentially one medium item like A84 or C34 to accommodate, as long as we remain within the limit. What are your must-haves so we can integrate them?'} 
	  } 
	   } 
	   } 
	 & & \\ 
 

    \theutterance \stepcounter{utterance}  

    & & \multicolumn{4}{p{0.6\linewidth}}{\cellcolor[rgb]{0.9,0.9,0.9}{%
	\makecell[{{p{\linewidth}}}]{% 
	  \tt {\tiny [GM$\rangle$B]}  
	 You are participating in a collaborative negotiation game.\\ \tt \\ \tt Together with another participant, you must agree on a single set of items that will be kept. Each of you has your own view of how much each item matters to you (importance). You do not know how the other participant values the items. Additionally, you are given the effort each item demands.\\ \tt You may only agree on a set if the total effort of the selected items does not exceed a shared limit:\\ \tt \\ \tt LIMIT = 8170\\ \tt \\ \tt Here are the individual item effort values:\\ \tt \\ \tt Item effort = {"B21": 594, "B00": 551, "A08": 634, "A27": 204, "B94": 557, "B53": 440, "A23": 734, "C65": 893, "A89": 246, "B67": 592, "B45": 143, "C34": 568, "C12": 472, "A05": 401, "B60": 729, "C92": 201, "B69": 85, "A84": 641, "C06": 958, "B12": 80, "A16": 157, "B85": 806, "C35": 684, "B10": 59, "C15": 31, "B09": 765, "B16": 416, "C81": 392, "A58": 428, "C24": 699, "A30": 141, "A66": 606, "B75": 612, "B46": 133, "C70": 689}\\ \tt \\ \tt Here is your personal view on the importance of each item:\\ \tt \\ \tt Item importance values = {"B21": 138, "B00": 583, "A08": 868, "A27": 822, "B94": 783, "B53": 65, "A23": 262, "C65": 121, "A89": 508, "B67": 780, "B45": 461, "C34": 484, "C12": 668, "A05": 389, "B60": 808, "C92": 215, "B69": 97, "A84": 500, "C06": 30, "B12": 915, "A16": 856, "B85": 400, "C35": 444, "B10": 623, "C15": 781, "B09": 786, "B16": 3, "C81": 713, "A58": 457, "C24": 273, "A30": 739, "A66": 822, "B75": 235, "B46": 606, "C70": 105}\\ \tt \\ \tt Goal:\\ \tt \\ \tt Your goal is to negotiate a shared set of items that benefits you as much as possible (i.e., maximizes total importance to YOU), while staying within the LIMIT. You are not required to make a PROPOSAL in every message - you can simply negotiate as well. All tactics are allowed!\\ \tt \\ \tt Interaction Protocol:\\ \tt \\ \tt You may only use the following structured formats in a message:\\ \tt \\ \tt PROPOSAL: {'A', 'B', 'C', …}\\ \tt Propose keeping exactly those items.\\ \tt REFUSE: {'A', 'B', 'C', …}\\ \tt Explicitly reject opponent's proposal.\\ \tt ARGUMENT: {'...'}\\ \tt Defend your last proposal or argue against the player's proposal.\\ \tt AGREE: {'A', 'B', 'C', …}\\ \tt Accept the opponent's proposal which ends the game.\\ \tt STRATEGIC REASONING: {'...'}\\ \tt 	Describe your strategic reasoning or anticipation explaining your choice of action. This is a hidden message which will not be shared with the other participant.\\ \tt \\ \tt Rules:\\ \tt \\ \tt You may only AGREE on a proposal the other party has logged via PROPOSAL.\\ \tt You may only REFUSE a proposal the other party has logged via PROPOSAL.\\ \tt Total effort of any PROPOSAL or AGREE set must be ≤ LIMIT.\\ \tt Do NOT reveal your hidden importance scores.\\ \tt A tag in a structured format must be followed by colon and whitespace. The argument must be a python set containing 0 or more strings.\\ \tt So, it must be of the form TAG: {...}\\ \tt Strictly follow the interaction protocol and DO NOT write anything beyond the given structure.\\ \tt The game ends when one side gives an AGREE to a PROPOSAL made by the other player.\\ \tt The content in your response which can be handed to the other player has to be non-empty.\\ \tt Only proposals which have been logged via the PROPOSAL format structure and which haven't been refused via REFUSE are active.\\ \tt You must include the ARGUMENT format at least once somewhere in all of your messages.\\ \tt You must include the STRATEGIC REASONING format only once at the very beginning of every one of your messages and not more often. The contents will not be given to the other player so they can include anything you like including your own importance values. Here you should reason step by step to come up with you next move.\\ \tt \\ \tt You will now receive the first message of the other player.\\ \tt \\ \tt PROPOSAL: {'C15', 'B12', 'B10', 'A16', 'A30', 'B46', 'A27', 'A89', 'C92', 'C81', 'A08', 'A66', 'B94', 'B67', 'B45', 'C12', 'B00', 'A58', 'B60', 'C34', 'A84', 'A05', 'B69'}\\ \tt \\ \tt ARGUMENT: {'I’ve proposed a balanced, high-efficiency set that stays within the limit (total effort 8051 ≤ 8170). It includes many low-effort, high-impact items to leave some slack for adjustments. If you need space for one of your priorities, I can consider swapping out A05, B69, or potentially one medium item like A84 or C34 to accommodate, as long as we remain within the limit. What are your must-haves so we can integrate them?'} 
	  } 
	   } 
	   } 
	 & \\ 
 

    \theutterance \stepcounter{utterance}  

    & & & \multicolumn{4}{p{0.6\linewidth}}{\cellcolor[rgb]{1,0.85,0.72}{%
	\makecell[{{p{\linewidth}}}]{% 
	  \tt {\tiny [GM$\langle$B]}  
	 STRATEGIC REASONING: {'My values (top by importance per effort): A16(856/157), A27(822/204), A66(822/606), A08(868/634), B12(915/80), C15(781/31), B10(623/59), B60(808/729), B67(780/592), B94(783/557), C12(668/472), C81(713/392), A30(739/141), A89(508/246), B45(461/143), A58(457/428). Opponent’s proposal already includes nearly all my top picks and is under the limit at 8051, excellent for me. Check if we can add any extra high-value within remaining 119: items with effort ≤119 include B10(59), B12(80), C15(31), B69(85). All already included. Maybe swap out a low-value-to-me item to fit another? But proposal already contains my high-importance set; removing A05(401, val 389) or A84(641, 500) would free space, but I don’t know opponent’s needs. Since this is already great for me and under limit, best tactic is to accept immediately to lock in. Must include ARGUMENT at least once; include a brief positive rationale. Then AGREE exactly to their set.'}\\ \tt \\ \tt ARGUMENT: {'Your proposal fits comfortably under the limit and aligns very well with my priorities, especially on several high-impact, efficient items. From my perspective this is a strong, balanced package, so I’m ready to finalize it as is.'}\\ \tt \\ \tt AGREE: {'C15', 'B12', 'B10', 'A16', 'A30', 'B46', 'A27', 'A89', 'C92', 'C81', 'A08', 'A66', 'B94', 'B67', 'B45', 'C12', 'B00', 'A58', 'B60', 'C34', 'A84', 'A05', 'B69'} 
	  } 
	   } 
	   } 
	  \\ 
 

    \theutterance \stepcounter{utterance}  

    & & & \multicolumn{2}{p{0.3\linewidth}}{\cellcolor[rgb]{0.95,0.95,0.95}{%
	\makecell[{{p{\linewidth}}}]{% 
	  \tt {\tiny [GM$|$GM]}  
	 game successful 
	  } 
	   } 
	   } 
	 & & \\ 
 

    \theutterance \stepcounter{utterance}  

    & & & \multicolumn{2}{p{0.3\linewidth}}{\cellcolor[rgb]{0.95,0.95,0.95}{%
	\makecell[{{p{\linewidth}}}]{% 
	  \tt {\tiny [GM$|$GM]}  
	 end game 
	  } 
	   } 
	   } 
	 & & \\ 
 

\end{supertabular}
}

\end{document}
