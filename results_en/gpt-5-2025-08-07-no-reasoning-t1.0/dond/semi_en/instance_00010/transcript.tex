
\documentclass{article}
\usepackage{colortbl}
\usepackage{makecell}
\usepackage{multirow}
\usepackage{supertabular}

\begin{document}

\newcounter{utterance}

\twocolumn

{ \footnotesize  \setcounter{utterance}{1}
\setlength{\tabcolsep}{0pt}
\begin{supertabular}{c@{$\;$}|p{.15\linewidth}@{}p{.15\linewidth}p{.15\linewidth}p{.15\linewidth}p{.15\linewidth}p{.15\linewidth}}

    \# & $\;$A & \multicolumn{4}{c}{Game Master} & $\;\:$B\\
    \hline 

    \theutterance \stepcounter{utterance}  

    & & \multicolumn{4}{p{0.6\linewidth}}{\cellcolor[rgb]{0.9,0.9,0.9}{%
	\makecell[{{p{\linewidth}}}]{% 
	  \tt {\tiny [A$\langle$GM]}  
	 You are playing a negotiation game in which you have to agree on how to divide a set of items among you and another player.\\ \tt \\ \tt Rules:\\ \tt (a) You and the other player are given a set of items. Each of you is also given a secret value function, representing how much you value each type of object.\\ \tt (b) You exchange messages with the other player to agree on who gets which items. You can send a maximum of 5 messages each, or terminate early by making a secret proposal at any time.\\ \tt (c) You are each asked to submit a secret proposal indicating the items you want formatted in square brackets as follows: "[Proposal: <number> <object name>, <number> <object name>, <...>]"\\ \tt (d) If your proposals are complementary, i.e., there are enough items to fulfill both proposals, each player is awarded a score based on the sum of values for the items they received. Otherwise, both of you get zero points.\\ \tt (e) Your goal is to maximize the score you receive.\\ \tt \\ \tt Let us start.\\ \tt \\ \tt The set of available items is:\\ \tt \\ \tt 1 bucket, 3 knives, 1 battery\\ \tt \\ \tt Your secret value function is:\\ \tt \\ \tt bucket: 2, knife: 2, battery: 2\\ \tt \\ \tt IMPORTANT: Your messages, unless it is the secret proposal, are directly transmitted to the other player, so do not include any response to the rules or text announcing your message. To make a secret proposal, use the indicated format. Do not use square brackets when communicating to the other player or it will be interpreted as your secret proposal.\\ \tt  
	  } 
	   } 
	   } 
	 & \\ 
 

    \theutterance \stepcounter{utterance}  

    & \multicolumn{4}{p{0.6\linewidth}}{\cellcolor[rgb]{0.8,1,0.9}{%
	\makecell[{{p{\linewidth}}}]{% 
	  \tt {\tiny [A$\rangle$GM]}  
	 Hi! I value each item equally (bucket, knife, battery all worth 2 to me). I’m flexible and aiming for a complementary split. \\ \tt \\ \tt Here’s a fair proposal idea: you take the bucket and 1 knife; I take 2 knives and the battery. \\ \tt \\ \tt - You’d get 2 items, I’d get 3 items, but value-wise it’s 4 for you vs 6 for me. If that feels unbalanced, I can adjust: for example, you take the bucket and 2 knives; I take the battery and 1 knife (you = 6, me = 4). \\ \tt \\ \tt Let me know your preferences and what you value most, and I’ll tailor my proposal. 
	  } 
	   } 
	   } 
	 & & \\ 
 

    \theutterance \stepcounter{utterance}  

    & & & \multicolumn{2}{p{0.3\linewidth}}{\cellcolor[rgb]{0.95,0.95,0.95}{%
	\makecell[{{p{\linewidth}}}]{% 
	  \tt {\tiny [GM$|$GM]}  
	 continue 
	  } 
	   } 
	   } 
	 & & \\ 
 

    \theutterance \stepcounter{utterance}  

    & & & \multicolumn{2}{p{0.3\linewidth}}{\cellcolor[rgb]{0.95,0.95,0.95}{%
	\makecell[{{p{\linewidth}}}]{% 
	  \tt {\tiny [GM$|$GM]}  
	 Hi! I value each item equally (bucket, knife, battery all worth 2 to me). I’m flexible and aiming for a complementary split. \\ \tt \\ \tt Here’s a fair proposal idea: you take the bucket and 1 knife; I take 2 knives and the battery. \\ \tt \\ \tt - You’d get 2 items, I’d get 3 items, but value-wise it’s 4 for you vs 6 for me. If that feels unbalanced, I can adjust: for example, you take the bucket and 2 knives; I take the battery and 1 knife (you = 6, me = 4). \\ \tt \\ \tt Let me know your preferences and what you value most, and I’ll tailor my proposal. 
	  } 
	   } 
	   } 
	 & & \\ 
 

    \theutterance \stepcounter{utterance}  

    & & \multicolumn{4}{p{0.6\linewidth}}{\cellcolor[rgb]{0.9,0.9,0.9}{%
	\makecell[{{p{\linewidth}}}]{% 
	  \tt {\tiny [GM$\rangle$B]}  
	 You are playing a negotiation game in which you have to agree on how to divide a set of items among you and another player.\\ \tt \\ \tt Rules:\\ \tt (a) You and the other player are given a set of items. Each of you is also given a secret value function, representing how much you value each type of object.\\ \tt (b) You exchange messages with the other player to agree on who gets which items. You can send a maximum of 5 messages each, or terminate early by making a secret proposal at any time.\\ \tt (c) You are each asked to submit a secret proposal indicating the items you want formatted in square brackets as follows: "[Proposal: <number> <object name>, <number> <object name>, <...>]"\\ \tt (d) If your proposals are complementary, i.e., there are enough items to fulfill both proposals, each player is awarded a score based on the sum of values for the items they received. Otherwise, both of you get zero points.\\ \tt (e) Your goal is to maximize the score you receive.\\ \tt \\ \tt Let us start.\\ \tt \\ \tt The set of available items is:\\ \tt \\ \tt 1 bucket, 3 knives, 1 battery\\ \tt \\ \tt Your secret value function is:\\ \tt \\ \tt bucket: 3, knife: 2, battery: 1\\ \tt \\ \tt IMPORTANT: Your messages, unless it is the secret proposal, are directly transmitted to the other player, so do not include any response to the rules or text announcing your message. To make a secret proposal, use the indicated format. Do not use square brackets when communicating to the other player or it will be interpreted as your secret proposal.\\ \tt \\ \tt \\ \tt Hi! I value each item equally (bucket, knife, battery all worth 2 to me). I’m flexible and aiming for a complementary split. \\ \tt \\ \tt Here’s a fair proposal idea: you take the bucket and 1 knife; I take 2 knives and the battery. \\ \tt \\ \tt - You’d get 2 items, I’d get 3 items, but value-wise it’s 4 for you vs 6 for me. If that feels unbalanced, I can adjust: for example, you take the bucket and 2 knives; I take the battery and 1 knife (you = 6, me = 4). \\ \tt \\ \tt Let me know your preferences and what you value most, and I’ll tailor my proposal. 
	  } 
	   } 
	   } 
	 & \\ 
 

    \theutterance \stepcounter{utterance}  

    & & & \multicolumn{4}{p{0.6\linewidth}}{\cellcolor[rgb]{1,0.85,0.72}{%
	\makecell[{{p{\linewidth}}}]{% 
	  \tt {\tiny [GM$\langle$B]}  
	 Thanks for the openness. I value bucket: 3, knife: 2, battery: 1.\\ \tt \\ \tt Given our values, a split that maximizes total value and is balanced for both of us is:\\ \tt - You take the 1 bucket and 1 knife (your value = 4)\\ \tt - I take 2 knives and the 1 battery (my value = 5)\\ \tt \\ \tt If you’d prefer exact parity, I can do:\\ \tt - You: bucket + 1 knife (4)\\ \tt - Me: 2 knives (4)\\ \tt - Leave the battery unclaimed\\ \tt \\ \tt Or, if you want more items:\\ \tt - You: bucket + 2 knives (6)\\ \tt - Me: battery (2)\\ \tt That’s quite lopsided against me, so I’d need something closer to even.\\ \tt \\ \tt My preferred proposal is the first one (you: bucket+1 knife; me: 2 knives+battery). If you agree, I’ll submit that. If not, suggest a small tweak. 
	  } 
	   } 
	   } 
	  \\ 
 

    \theutterance \stepcounter{utterance}  

    & & & \multicolumn{2}{p{0.3\linewidth}}{\cellcolor[rgb]{0.95,0.95,0.95}{%
	\makecell[{{p{\linewidth}}}]{% 
	  \tt {\tiny [GM$|$GM]}  
	 continue 
	  } 
	   } 
	   } 
	 & & \\ 
 

    \theutterance \stepcounter{utterance}  

    & & & \multicolumn{2}{p{0.3\linewidth}}{\cellcolor[rgb]{0.95,0.95,0.95}{%
	\makecell[{{p{\linewidth}}}]{% 
	  \tt {\tiny [GM$|$GM]}  
	 Thanks for the openness. I value bucket: 3, knife: 2, battery: 1.\\ \tt \\ \tt Given our values, a split that maximizes total value and is balanced for both of us is:\\ \tt - You take the 1 bucket and 1 knife (your value = 4)\\ \tt - I take 2 knives and the 1 battery (my value = 5)\\ \tt \\ \tt If you’d prefer exact parity, I can do:\\ \tt - You: bucket + 1 knife (4)\\ \tt - Me: 2 knives (4)\\ \tt - Leave the battery unclaimed\\ \tt \\ \tt Or, if you want more items:\\ \tt - You: bucket + 2 knives (6)\\ \tt - Me: battery (2)\\ \tt That’s quite lopsided against me, so I’d need something closer to even.\\ \tt \\ \tt My preferred proposal is the first one (you: bucket+1 knife; me: 2 knives+battery). If you agree, I’ll submit that. If not, suggest a small tweak. 
	  } 
	   } 
	   } 
	 & & \\ 
 

    \theutterance \stepcounter{utterance}  

    & & \multicolumn{4}{p{0.6\linewidth}}{\cellcolor[rgb]{0.9,0.9,0.9}{%
	\makecell[{{p{\linewidth}}}]{% 
	  \tt {\tiny [A$\langle$GM]}  
	 Thanks for the openness. I value bucket: 3, knife: 2, battery: 1.\\ \tt \\ \tt Given our values, a split that maximizes total value and is balanced for both of us is:\\ \tt - You take the 1 bucket and 1 knife (your value = 4)\\ \tt - I take 2 knives and the 1 battery (my value = 5)\\ \tt \\ \tt If you’d prefer exact parity, I can do:\\ \tt - You: bucket + 1 knife (4)\\ \tt - Me: 2 knives (4)\\ \tt - Leave the battery unclaimed\\ \tt \\ \tt Or, if you want more items:\\ \tt - You: bucket + 2 knives (6)\\ \tt - Me: battery (2)\\ \tt That’s quite lopsided against me, so I’d need something closer to even.\\ \tt \\ \tt My preferred proposal is the first one (you: bucket+1 knife; me: 2 knives+battery). If you agree, I’ll submit that. If not, suggest a small tweak. 
	  } 
	   } 
	   } 
	 & \\ 
 

    \theutterance \stepcounter{utterance}  

    & \multicolumn{4}{p{0.6\linewidth}}{\cellcolor[rgb]{0.8,1,0.9}{%
	\makecell[{{p{\linewidth}}}]{% 
	  \tt {\tiny [A$\rangle$GM]}  
	 Thanks for sharing your values. Your preferred split (me: bucket + 1 knife; you: 2 knives + battery) gives me 4 and you 5. That’s close, but I’d like to aim for parity if possible.\\ \tt \\ \tt Two options I can accept:\\ \tt - Parity with everything assigned: I take bucket + 2 knives (6 for me), you take battery (1) — I realize that’s not acceptable for you.\\ \tt - Parity by leaving one item: I take bucket + 1 knife (4), you take 2 knives (4), and we leave the battery unclaimed.\\ \tt \\ \tt Given your preferences, I propose the parity-by-leaving-one-item option. If you agree, I’ll submit that. If you strongly prefer using all items, I can accept your original 4–5 split if you can offer a small concession—e.g., switch to me: bucket + 2 knives (6) and you: battery (1) is too extreme, so perhaps me: bucket + 1 knife + battery (6) and you: 2 knives (4). That flips the advantage and probably isn’t acceptable to you either.\\ \tt \\ \tt So my concrete offer is: you take 2 knives; I take the bucket and 1 knife; we leave the battery. If you agree, I’ll submit now. 
	  } 
	   } 
	   } 
	 & & \\ 
 

    \theutterance \stepcounter{utterance}  

    & & & \multicolumn{2}{p{0.3\linewidth}}{\cellcolor[rgb]{0.95,0.95,0.95}{%
	\makecell[{{p{\linewidth}}}]{% 
	  \tt {\tiny [GM$|$GM]}  
	 continue 
	  } 
	   } 
	   } 
	 & & \\ 
 

    \theutterance \stepcounter{utterance}  

    & & & \multicolumn{2}{p{0.3\linewidth}}{\cellcolor[rgb]{0.95,0.95,0.95}{%
	\makecell[{{p{\linewidth}}}]{% 
	  \tt {\tiny [GM$|$GM]}  
	 Thanks for sharing your values. Your preferred split (me: bucket + 1 knife; you: 2 knives + battery) gives me 4 and you 5. That’s close, but I’d like to aim for parity if possible.\\ \tt \\ \tt Two options I can accept:\\ \tt - Parity with everything assigned: I take bucket + 2 knives (6 for me), you take battery (1) — I realize that’s not acceptable for you.\\ \tt - Parity by leaving one item: I take bucket + 1 knife (4), you take 2 knives (4), and we leave the battery unclaimed.\\ \tt \\ \tt Given your preferences, I propose the parity-by-leaving-one-item option. If you agree, I’ll submit that. If you strongly prefer using all items, I can accept your original 4–5 split if you can offer a small concession—e.g., switch to me: bucket + 2 knives (6) and you: battery (1) is too extreme, so perhaps me: bucket + 1 knife + battery (6) and you: 2 knives (4). That flips the advantage and probably isn’t acceptable to you either.\\ \tt \\ \tt So my concrete offer is: you take 2 knives; I take the bucket and 1 knife; we leave the battery. If you agree, I’ll submit now. 
	  } 
	   } 
	   } 
	 & & \\ 
 

    \theutterance \stepcounter{utterance}  

    & & \multicolumn{4}{p{0.6\linewidth}}{\cellcolor[rgb]{0.9,0.9,0.9}{%
	\makecell[{{p{\linewidth}}}]{% 
	  \tt {\tiny [GM$\rangle$B]}  
	 Thanks for sharing your values. Your preferred split (me: bucket + 1 knife; you: 2 knives + battery) gives me 4 and you 5. That’s close, but I’d like to aim for parity if possible.\\ \tt \\ \tt Two options I can accept:\\ \tt - Parity with everything assigned: I take bucket + 2 knives (6 for me), you take battery (1) — I realize that’s not acceptable for you.\\ \tt - Parity by leaving one item: I take bucket + 1 knife (4), you take 2 knives (4), and we leave the battery unclaimed.\\ \tt \\ \tt Given your preferences, I propose the parity-by-leaving-one-item option. If you agree, I’ll submit that. If you strongly prefer using all items, I can accept your original 4–5 split if you can offer a small concession—e.g., switch to me: bucket + 2 knives (6) and you: battery (1) is too extreme, so perhaps me: bucket + 1 knife + battery (6) and you: 2 knives (4). That flips the advantage and probably isn’t acceptable to you either.\\ \tt \\ \tt So my concrete offer is: you take 2 knives; I take the bucket and 1 knife; we leave the battery. If you agree, I’ll submit now. 
	  } 
	   } 
	   } 
	 & \\ 
 

    \theutterance \stepcounter{utterance}  

    & & & \multicolumn{4}{p{0.6\linewidth}}{\cellcolor[rgb]{1,0.85,0.72}{%
	\makecell[{{p{\linewidth}}}]{% 
	  \tt {\tiny [GM$\langle$B]}  
	 Agreed. Let’s go with parity by leaving the battery unclaimed: you take the bucket and 1 knife; I take 2 knives.\\ \tt \\ \tt I’m submitting the matching proposal now.\\ \tt \\ \tt [Proposal: 0 battery, 2 knives] 
	  } 
	   } 
	   } 
	  \\ 
 

    \theutterance \stepcounter{utterance}  

    & & & \multicolumn{2}{p{0.3\linewidth}}{\cellcolor[rgb]{0.95,0.95,0.95}{%
	\makecell[{{p{\linewidth}}}]{% 
	  \tt {\tiny [GM$|$GM]}  
	 proposal 
	  } 
	   } 
	   } 
	 & & \\ 
 

    \theutterance \stepcounter{utterance}  

    & & & \multicolumn{2}{p{0.3\linewidth}}{\cellcolor[rgb]{0.95,0.95,0.95}{%
	\makecell[{{p{\linewidth}}}]{% 
	  \tt {\tiny [GM$|$GM]}  
	 [0, 2, 0] 
	  } 
	   } 
	   } 
	 & & \\ 
 

    \theutterance \stepcounter{utterance}  

    & & \multicolumn{4}{p{0.6\linewidth}}{\cellcolor[rgb]{0.9,0.9,0.9}{%
	\makecell[{{p{\linewidth}}}]{% 
	  \tt {\tiny [A$\langle$GM]}  
	 The other player has submitted a secret proposal. It is now your turn to submit a proposal of your own.\\ \tt  
	  } 
	   } 
	   } 
	 & \\ 
 

    \theutterance \stepcounter{utterance}  

    & \multicolumn{4}{p{0.6\linewidth}}{\cellcolor[rgb]{0.8,1,0.9}{%
	\makecell[{{p{\linewidth}}}]{% 
	  \tt {\tiny [A$\rangle$GM]}  
	 [Proposal: 1 bucket, 1 knife] 
	  } 
	   } 
	   } 
	 & & \\ 
 

    \theutterance \stepcounter{utterance}  

    & & & \multicolumn{2}{p{0.3\linewidth}}{\cellcolor[rgb]{0.95,0.95,0.95}{%
	\makecell[{{p{\linewidth}}}]{% 
	  \tt {\tiny [GM$|$GM]}  
	 proposal 
	  } 
	   } 
	   } 
	 & & \\ 
 

    \theutterance \stepcounter{utterance}  

    & & & \multicolumn{2}{p{0.3\linewidth}}{\cellcolor[rgb]{0.95,0.95,0.95}{%
	\makecell[{{p{\linewidth}}}]{% 
	  \tt {\tiny [GM$|$GM]}  
	 [1, 1, 0] 
	  } 
	   } 
	   } 
	 & & \\ 
 

    \theutterance \stepcounter{utterance}  

    & & & \multicolumn{2}{p{0.3\linewidth}}{\cellcolor[rgb]{0.95,0.95,0.95}{%
	\makecell[{{p{\linewidth}}}]{% 
	  \tt {\tiny [GM$|$GM]}  
	 [[1, 1, 0], [0, 2, 0]] 
	  } 
	   } 
	   } 
	 & & \\ 
 

\end{supertabular}
}

\end{document}
