
\documentclass{article}
\usepackage{colortbl}
\usepackage{makecell}
\usepackage{multirow}
\usepackage{supertabular}

\begin{document}

\newcounter{utterance}

\centering \large Interaction Transcript for game `dond', experiment `coop\_en', episode 7 with gpt{-}5{-}azure{-}t1.0.
\vspace{24pt}

{ \footnotesize  \setcounter{utterance}{1}
\setlength{\tabcolsep}{0pt}
\begin{supertabular}{c@{$\;$}|p{.15\linewidth}@{}p{.15\linewidth}p{.15\linewidth}p{.15\linewidth}p{.15\linewidth}p{.15\linewidth}}
    \# & $\;$A & \multicolumn{4}{c}{Game Master} & $\;\:$B\\
    \hline

    \theutterance \stepcounter{utterance}  
    & & \multicolumn{4}{p{0.6\linewidth}}{
        \cellcolor[rgb]{0.9,0.9,0.9}{
            \makecell[{{p{\linewidth}}}]{
                \texttt{\tiny{[P1$\langle$GM]}}
                \texttt{You are playing a cooperative negotiation game in which you have to agree on how to divide a set of items among you and another player.} \\
\\ 
\texttt{Rules:} \\
\texttt{(a) You and the other player are given a set of items. Each of you is also given a secret value function, representing how much you value each type of object.} \\
\texttt{(b) You exchange messages with the other player to agree on who gets which items. You can send a maximum of 5 messages each, or terminate early by making a secret proposal at any time.} \\
\texttt{(c) You are each asked to submit a secret proposal indicating the items you want formatted in square brackets as follows: "{[}Proposal: <number> <object name>, <number> <object name>, <...>{]}"} \\
\texttt{(d) If your proposals are complementary, i.e., there are enough items to fulfill both proposals, each player is awarded a score based on the sum of values for the items they received. Otherwise, both of you get zero points.} \\
\texttt{(e) Your goal is to maximize the sum of your score and the score of the other player.} \\
\\ 
\texttt{Let us start.} \\
\\ 
\texttt{The set of available items is:} \\
\\ 
\texttt{2 hooks, 2 plates, 1 bowl} \\
\\ 
\texttt{Your secret value function is:} \\
\\ 
\texttt{hook: 3, plate: 1, bowl: 2} \\
\\ 
\texttt{IMPORTANT: Your messages, unless it is the secret proposal, are directly transmitted to the other player, so do not include any response to the rules or text announcing your message. To make a secret proposal, use the indicated format. Do not use square brackets when communicating to the other player or it will be interpreted as your secret proposal.} \\
            }
        }
    }
    & \\ \\

    \theutterance \stepcounter{utterance}  
    & \multicolumn{4}{p{0.6\linewidth}}{
        \cellcolor[rgb]{0.9,0.9,0.9}{
            \makecell[{{p{\linewidth}}}]{
                \texttt{\tiny{[P1$\rangle$GM]}}
                \texttt{Hi! I’d like to maximize our total points together. Here are my values: hook=3, bowl=2, plate=1. Given that, hooks are most valuable to me.} \\
\\ 
\texttt{If your values are different, could you share any strong preferences? For a starting point that should be near{-}efficient, how about:} \\
\texttt{{-} I take the 2 hooks (since I value them highly),} \\
\texttt{{-} You take the 1 bowl,} \\
\texttt{{-} And we split the 2 plates 1–1 (or you can take both plates if you value them more).} \\
\\ 
\texttt{If you prefer, propose your values or a counter{-}allocation that maximizes total value for both of us.} \\
            }
        }
    }
    & & \\ \\

    \theutterance \stepcounter{utterance}  
    & & & \multicolumn{2}{p{0.3\linewidth}}{
        \cellcolor[rgb]{0.9,0.9,0.9}{
            \makecell[{{p{\linewidth}}}]{
                \texttt{\tiny{[GM$|$GM]}}
                \texttt{continue} \\
            }
        }
    }
    & & \\ \\

    \theutterance \stepcounter{utterance}  
    & & & \multicolumn{2}{p{0.3\linewidth}}{
        \cellcolor[rgb]{0.9,0.9,0.9}{
            \makecell[{{p{\linewidth}}}]{
                \texttt{\tiny{[GM$|$GM]}}
                \texttt{Hi! I’d like to maximize our total points together. Here are my values: hook=3, bowl=2, plate=1. Given that, hooks are most valuable to me.} \\
\\ 
\texttt{If your values are different, could you share any strong preferences? For a starting point that should be near{-}efficient, how about:} \\
\texttt{{-} I take the 2 hooks (since I value them highly),} \\
\texttt{{-} You take the 1 bowl,} \\
\texttt{{-} And we split the 2 plates 1–1 (or you can take both plates if you value them more).} \\
\\ 
\texttt{If you prefer, propose your values or a counter{-}allocation that maximizes total value for both of us.} \\
            }
        }
    }
    & & \\ \\

    \theutterance \stepcounter{utterance}  
    & & \multicolumn{4}{p{0.6\linewidth}}{
        \cellcolor[rgb]{0.9,0.9,0.9}{
            \makecell[{{p{\linewidth}}}]{
                \texttt{\tiny{[GM$\rangle$P2]}}
                \texttt{You are playing a cooperative negotiation game in which you have to agree on how to divide a set of items among you and another player.} \\
\\ 
\texttt{Rules:} \\
\texttt{(a) You and the other player are given a set of items. Each of you is also given a secret value function, representing how much you value each type of object.} \\
\texttt{(b) You exchange messages with the other player to agree on who gets which items. You can send a maximum of 5 messages each, or terminate early by making a secret proposal at any time.} \\
\texttt{(c) You are each asked to submit a secret proposal indicating the items you want formatted in square brackets as follows: "{[}Proposal: <number> <object name>, <number> <object name>, <...>{]}"} \\
\texttt{(d) If your proposals are complementary, i.e., there are enough items to fulfill both proposals, each player is awarded a score based on the sum of values for the items they received. Otherwise, both of you get zero points.} \\
\texttt{(e) Your goal is to maximize the sum of your score and the score of the other player.} \\
\\ 
\texttt{Let us start.} \\
\\ 
\texttt{The set of available items is:} \\
\\ 
\texttt{2 hooks, 2 plates, 1 bowl} \\
\\ 
\texttt{Your secret value function is:} \\
\\ 
\texttt{hook: 1, plate: 0, bowl: 8} \\
\\ 
\texttt{IMPORTANT: Your messages, unless it is the secret proposal, are directly transmitted to the other player, so do not include any response to the rules or text announcing your message. To make a secret proposal, use the indicated format. Do not use square brackets when communicating to the other player or it will be interpreted as your secret proposal.} \\
\\ 
\\ 
\texttt{Hi! I’d like to maximize our total points together. Here are my values: hook=3, bowl=2, plate=1. Given that, hooks are most valuable to me.} \\
\\ 
\texttt{If your values are different, could you share any strong preferences? For a starting point that should be near{-}efficient, how about:} \\
\texttt{{-} I take the 2 hooks (since I value them highly),} \\
\texttt{{-} You take the 1 bowl,} \\
\texttt{{-} And we split the 2 plates 1–1 (or you can take both plates if you value them more).} \\
\\ 
\texttt{If you prefer, propose your values or a counter{-}allocation that maximizes total value for both of us.} \\
            }
        }
    }
    & \\ \\

    \theutterance \stepcounter{utterance}  
    & & & \multicolumn{4}{p{0.6\linewidth}}{
        \cellcolor[rgb]{0.9,0.9,0.9}{
            \makecell[{{p{\linewidth}}}]{
                \texttt{\tiny{[GM$\langle$P2]}}
                \texttt{Thanks for sharing your values. Mine are: hook=1, plate=0, bowl=8. So the bowl is extremely valuable to me, plates are worthless to me, and hooks are modest.} \\
\\ 
\texttt{Given our values, an efficient split is:} \\
\texttt{{-} You take both hooks (high value to you, low to me).} \\
\texttt{{-} I take the bowl (very high value to me, moderate to you).} \\
\texttt{{-} Plates go to you if you want them (they’re 0 to me, 1 to you).} \\
\\ 
\texttt{That yields total value: you get 3+3 from hooks plus up to 2 from plates; I get 8 from the bowl. This seems Pareto{-}efficient. If you agree, I’ll submit that as my proposal.} \\
            }
        }
    }
     \\ \\

    \theutterance \stepcounter{utterance}  
    & & & \multicolumn{2}{p{0.3\linewidth}}{
        \cellcolor[rgb]{0.9,0.9,0.9}{
            \makecell[{{p{\linewidth}}}]{
                \texttt{\tiny{[GM$|$GM]}}
                \texttt{continue} \\
            }
        }
    }
    & & \\ \\

    \theutterance \stepcounter{utterance}  
    & & & \multicolumn{2}{p{0.3\linewidth}}{
        \cellcolor[rgb]{0.9,0.9,0.9}{
            \makecell[{{p{\linewidth}}}]{
                \texttt{\tiny{[GM$|$GM]}}
                \texttt{Thanks for sharing your values. Mine are: hook=1, plate=0, bowl=8. So the bowl is extremely valuable to me, plates are worthless to me, and hooks are modest.} \\
\\ 
\texttt{Given our values, an efficient split is:} \\
\texttt{{-} You take both hooks (high value to you, low to me).} \\
\texttt{{-} I take the bowl (very high value to me, moderate to you).} \\
\texttt{{-} Plates go to you if you want them (they’re 0 to me, 1 to you).} \\
\\ 
\texttt{That yields total value: you get 3+3 from hooks plus up to 2 from plates; I get 8 from the bowl. This seems Pareto{-}efficient. If you agree, I’ll submit that as my proposal.} \\
            }
        }
    }
    & & \\ \\

    \theutterance \stepcounter{utterance}  
    & & \multicolumn{4}{p{0.6\linewidth}}{
        \cellcolor[rgb]{0.9,0.9,0.9}{
            \makecell[{{p{\linewidth}}}]{
                \texttt{\tiny{[P1$\langle$GM]}}
                \texttt{Thanks for sharing your values. Mine are: hook=1, plate=0, bowl=8. So the bowl is extremely valuable to me, plates are worthless to me, and hooks are modest.} \\
\\ 
\texttt{Given our values, an efficient split is:} \\
\texttt{{-} You take both hooks (high value to you, low to me).} \\
\texttt{{-} I take the bowl (very high value to me, moderate to you).} \\
\texttt{{-} Plates go to you if you want them (they’re 0 to me, 1 to you).} \\
\\ 
\texttt{That yields total value: you get 3+3 from hooks plus up to 2 from plates; I get 8 from the bowl. This seems Pareto{-}efficient. If you agree, I’ll submit that as my proposal.} \\
            }
        }
    }
    & \\ \\

    \theutterance \stepcounter{utterance}  
    & \multicolumn{4}{p{0.6\linewidth}}{
        \cellcolor[rgb]{0.9,0.9,0.9}{
            \makecell[{{p{\linewidth}}}]{
                \texttt{\tiny{[P1$\rangle$GM]}}
                \texttt{Agreed. That split is efficient given our values.} \\
\\ 
\texttt{I’ll plan to take: 2 hooks and 2 plates.} \\
\texttt{You take: 1 bowl.} \\
\\ 
\texttt{Submitting my secret proposal now.} \\
\\ 
\texttt{{[}Proposal: 2 hooks, 2 plates{]}} \\
            }
        }
    }
    & & \\ \\

    \theutterance \stepcounter{utterance}  
    & & & \multicolumn{2}{p{0.3\linewidth}}{
        \cellcolor[rgb]{0.9,0.9,0.9}{
            \makecell[{{p{\linewidth}}}]{
                \texttt{\tiny{[GM$|$GM]}}
                \texttt{proposal} \\
            }
        }
    }
    & & \\ \\

    \theutterance \stepcounter{utterance}  
    & & & \multicolumn{2}{p{0.3\linewidth}}{
        \cellcolor[rgb]{0.9,0.9,0.9}{
            \makecell[{{p{\linewidth}}}]{
                \texttt{\tiny{[GM$|$GM]}}
                [2, 2, 0]
            }
        }
    }
    & & \\ \\

    \theutterance \stepcounter{utterance}  
    & & \multicolumn{4}{p{0.6\linewidth}}{
        \cellcolor[rgb]{0.9,0.9,0.9}{
            \makecell[{{p{\linewidth}}}]{
                \texttt{\tiny{[GM$\rangle$P2]}}
                \texttt{The other player has submitted a secret proposal. It is now your turn to submit a proposal of your own.} \\
            }
        }
    }
    & \\ \\

    \theutterance \stepcounter{utterance}  
    & & & \multicolumn{4}{p{0.6\linewidth}}{
        \cellcolor[rgb]{0.9,0.9,0.9}{
            \makecell[{{p{\linewidth}}}]{
                \texttt{\tiny{[GM$\langle$P2]}}
                \texttt{{[}Proposal: 2 hooks, 2 plates, 1 bowl{]}} \\
            }
        }
    }
     \\ \\

    \theutterance \stepcounter{utterance}  
    & & & \multicolumn{2}{p{0.3\linewidth}}{
        \cellcolor[rgb]{0.9,0.9,0.9}{
            \makecell[{{p{\linewidth}}}]{
                \texttt{\tiny{[GM$|$GM]}}
                \texttt{proposal} \\
            }
        }
    }
    & & \\ \\

    \theutterance \stepcounter{utterance}  
    & & & \multicolumn{2}{p{0.3\linewidth}}{
        \cellcolor[rgb]{0.9,0.9,0.9}{
            \makecell[{{p{\linewidth}}}]{
                \texttt{\tiny{[GM$|$GM]}}
                [2, 2, 1]
            }
        }
    }
    & & \\ \\

    \theutterance \stepcounter{utterance}  
    & & & \multicolumn{2}{p{0.3\linewidth}}{
        \cellcolor[rgb]{0.9,0.9,0.9}{
            \makecell[{{p{\linewidth}}}]{
                \texttt{\tiny{[GM$|$GM]}}
                [[2, 2, 0], [2, 2, 1]]
            }
        }
    }
    & & \\ \\

\end{supertabular}
}

\end{document}
