
\documentclass{article}
\usepackage{colortbl}
\usepackage{makecell}
\usepackage{multirow}
\usepackage{supertabular}

\begin{document}

\newcounter{utterance}

\centering \large Interaction Transcript for game `dond', experiment `coop\_en', episode 2 with gpt{-}5{-}azure{-}t1.0.
\vspace{24pt}

{ \footnotesize  \setcounter{utterance}{1}
\setlength{\tabcolsep}{0pt}
\begin{supertabular}{c@{$\;$}|p{.15\linewidth}@{}p{.15\linewidth}p{.15\linewidth}p{.15\linewidth}p{.15\linewidth}p{.15\linewidth}}
    \# & $\;$A & \multicolumn{4}{c}{Game Master} & $\;\:$B\\
    \hline

    \theutterance \stepcounter{utterance}  
    & & \multicolumn{4}{p{0.6\linewidth}}{
        \cellcolor[rgb]{0.9,0.9,0.9}{
            \makecell[{{p{\linewidth}}}]{
                \texttt{\tiny{[P1$\langle$GM]}}
                \texttt{You are playing a cooperative negotiation game in which you have to agree on how to divide a set of items among you and another player.} \\
\\ 
\texttt{Rules:} \\
\texttt{(a) You and the other player are given a set of items. Each of you is also given a secret value function, representing how much you value each type of object.} \\
\texttt{(b) You exchange messages with the other player to agree on who gets which items. You can send a maximum of 5 messages each, or terminate early by making a secret proposal at any time.} \\
\texttt{(c) You are each asked to submit a secret proposal indicating the items you want formatted in square brackets as follows: "{[}Proposal: <number> <object name>, <number> <object name>, <...>{]}"} \\
\texttt{(d) If your proposals are complementary, i.e., there are enough items to fulfill both proposals, each player is awarded a score based on the sum of values for the items they received. Otherwise, both of you get zero points.} \\
\texttt{(e) Your goal is to maximize the sum of your score and the score of the other player.} \\
\\ 
\texttt{Let us start.} \\
\\ 
\texttt{The set of available items is:} \\
\\ 
\texttt{1 phone, 2 mirrors, 2 fruits} \\
\\ 
\texttt{Your secret value function is:} \\
\\ 
\texttt{phone: 4, mirror: 2, fruit: 1} \\
\\ 
\texttt{IMPORTANT: Your messages, unless it is the secret proposal, are directly transmitted to the other player, so do not include any response to the rules or text announcing your message. To make a secret proposal, use the indicated format. Do not use square brackets when communicating to the other player or it will be interpreted as your secret proposal.} \\
            }
        }
    }
    & \\ \\

    \theutterance \stepcounter{utterance}  
    & \multicolumn{4}{p{0.6\linewidth}}{
        \cellcolor[rgb]{0.9,0.9,0.9}{
            \makecell[{{p{\linewidth}}}]{
                \texttt{\tiny{[P1$\rangle$GM]}}
                \texttt{I value items as phone = 4, mirror = 2 each, fruit = 1 each. Total pool value to me is 4 + (2*2) + (2*1) = 10. I’m aiming to maximize our combined score, so I’d like to understand your preferences.} \\
\\ 
\texttt{Could you share how you value phone, mirror, and fruit? If you prefer, you can state rankings (e.g., phone > mirror > fruit) and any must{-}haves.} \\
\\ 
\texttt{Here are two fair{-}sum options to consider depending on your preferences:} \\
\texttt{{-} If you value the phone highly: You take the phone; I take 2 mirrors and 2 fruits.} \\
\texttt{{-} If you don’t care much about the phone: I take the phone and 1 mirror; you take 1 mirror and 2 fruits.} \\
\\ 
\texttt{Let me know which direction suits you, or propose a split that gives you high value without reducing our total.} \\
            }
        }
    }
    & & \\ \\

    \theutterance \stepcounter{utterance}  
    & & & \multicolumn{2}{p{0.3\linewidth}}{
        \cellcolor[rgb]{0.9,0.9,0.9}{
            \makecell[{{p{\linewidth}}}]{
                \texttt{\tiny{[GM$|$GM]}}
                \texttt{continue} \\
            }
        }
    }
    & & \\ \\

    \theutterance \stepcounter{utterance}  
    & & & \multicolumn{2}{p{0.3\linewidth}}{
        \cellcolor[rgb]{0.9,0.9,0.9}{
            \makecell[{{p{\linewidth}}}]{
                \texttt{\tiny{[GM$|$GM]}}
                \texttt{I value items as phone = 4, mirror = 2 each, fruit = 1 each. Total pool value to me is 4 + (2*2) + (2*1) = 10. I’m aiming to maximize our combined score, so I’d like to understand your preferences.} \\
\\ 
\texttt{Could you share how you value phone, mirror, and fruit? If you prefer, you can state rankings (e.g., phone > mirror > fruit) and any must{-}haves.} \\
\\ 
\texttt{Here are two fair{-}sum options to consider depending on your preferences:} \\
\texttt{{-} If you value the phone highly: You take the phone; I take 2 mirrors and 2 fruits.} \\
\texttt{{-} If you don’t care much about the phone: I take the phone and 1 mirror; you take 1 mirror and 2 fruits.} \\
\\ 
\texttt{Let me know which direction suits you, or propose a split that gives you high value without reducing our total.} \\
            }
        }
    }
    & & \\ \\

    \theutterance \stepcounter{utterance}  
    & & \multicolumn{4}{p{0.6\linewidth}}{
        \cellcolor[rgb]{0.9,0.9,0.9}{
            \makecell[{{p{\linewidth}}}]{
                \texttt{\tiny{[GM$\rangle$P2]}}
                \texttt{You are playing a cooperative negotiation game in which you have to agree on how to divide a set of items among you and another player.} \\
\\ 
\texttt{Rules:} \\
\texttt{(a) You and the other player are given a set of items. Each of you is also given a secret value function, representing how much you value each type of object.} \\
\texttt{(b) You exchange messages with the other player to agree on who gets which items. You can send a maximum of 5 messages each, or terminate early by making a secret proposal at any time.} \\
\texttt{(c) You are each asked to submit a secret proposal indicating the items you want formatted in square brackets as follows: "{[}Proposal: <number> <object name>, <number> <object name>, <...>{]}"} \\
\texttt{(d) If your proposals are complementary, i.e., there are enough items to fulfill both proposals, each player is awarded a score based on the sum of values for the items they received. Otherwise, both of you get zero points.} \\
\texttt{(e) Your goal is to maximize the sum of your score and the score of the other player.} \\
\\ 
\texttt{Let us start.} \\
\\ 
\texttt{The set of available items is:} \\
\\ 
\texttt{1 phone, 2 mirrors, 2 fruits} \\
\\ 
\texttt{Your secret value function is:} \\
\\ 
\texttt{phone: 6, mirror: 0, fruit: 2} \\
\\ 
\texttt{IMPORTANT: Your messages, unless it is the secret proposal, are directly transmitted to the other player, so do not include any response to the rules or text announcing your message. To make a secret proposal, use the indicated format. Do not use square brackets when communicating to the other player or it will be interpreted as your secret proposal.} \\
\\ 
\\ 
\texttt{I value items as phone = 4, mirror = 2 each, fruit = 1 each. Total pool value to me is 4 + (2*2) + (2*1) = 10. I’m aiming to maximize our combined score, so I’d like to understand your preferences.} \\
\\ 
\texttt{Could you share how you value phone, mirror, and fruit? If you prefer, you can state rankings (e.g., phone > mirror > fruit) and any must{-}haves.} \\
\\ 
\texttt{Here are two fair{-}sum options to consider depending on your preferences:} \\
\texttt{{-} If you value the phone highly: You take the phone; I take 2 mirrors and 2 fruits.} \\
\texttt{{-} If you don’t care much about the phone: I take the phone and 1 mirror; you take 1 mirror and 2 fruits.} \\
\\ 
\texttt{Let me know which direction suits you, or propose a split that gives you high value without reducing our total.} \\
            }
        }
    }
    & \\ \\

    \theutterance \stepcounter{utterance}  
    & & & \multicolumn{4}{p{0.6\linewidth}}{
        \cellcolor[rgb]{0.9,0.9,0.9}{
            \makecell[{{p{\linewidth}}}]{
                \texttt{\tiny{[GM$\langle$P2]}}
                \texttt{Thanks for sharing. My values are: phone = 6, mirror = 0 each, fruit = 2 each. So I strongly prefer the phone and fruits; mirrors are worthless to me.} \\
\\ 
\texttt{Given our goal to maximize total, a natural split is:} \\
\texttt{{-} I take the phone and 2 fruits (total value: me 10, you 4 from 2 mirrors).} \\
\texttt{{-} You take the 2 mirrors.} \\
\\ 
\texttt{This yields combined 14 (my 10 + your 4), which is the maximum possible from the pool given our values.} \\
\\ 
\texttt{If you’d like some fruit value too, an alternative is:} \\
\texttt{{-} I take the phone and 1 fruit (value me 8).} \\
\texttt{{-} You take 2 mirrors and 1 fruit (value you 5).} \\
\texttt{Combined is 13.} \\
\\ 
\texttt{I prefer the first since it maximizes total, but I’m okay with the second if you want at least one fruit. If you agree with either, say which one and I’ll submit a matching proposal.} \\
            }
        }
    }
     \\ \\

    \theutterance \stepcounter{utterance}  
    & & & \multicolumn{2}{p{0.3\linewidth}}{
        \cellcolor[rgb]{0.9,0.9,0.9}{
            \makecell[{{p{\linewidth}}}]{
                \texttt{\tiny{[GM$|$GM]}}
                \texttt{continue} \\
            }
        }
    }
    & & \\ \\

    \theutterance \stepcounter{utterance}  
    & & & \multicolumn{2}{p{0.3\linewidth}}{
        \cellcolor[rgb]{0.9,0.9,0.9}{
            \makecell[{{p{\linewidth}}}]{
                \texttt{\tiny{[GM$|$GM]}}
                \texttt{Thanks for sharing. My values are: phone = 6, mirror = 0 each, fruit = 2 each. So I strongly prefer the phone and fruits; mirrors are worthless to me.} \\
\\ 
\texttt{Given our goal to maximize total, a natural split is:} \\
\texttt{{-} I take the phone and 2 fruits (total value: me 10, you 4 from 2 mirrors).} \\
\texttt{{-} You take the 2 mirrors.} \\
\\ 
\texttt{This yields combined 14 (my 10 + your 4), which is the maximum possible from the pool given our values.} \\
\\ 
\texttt{If you’d like some fruit value too, an alternative is:} \\
\texttt{{-} I take the phone and 1 fruit (value me 8).} \\
\texttt{{-} You take 2 mirrors and 1 fruit (value you 5).} \\
\texttt{Combined is 13.} \\
\\ 
\texttt{I prefer the first since it maximizes total, but I’m okay with the second if you want at least one fruit. If you agree with either, say which one and I’ll submit a matching proposal.} \\
            }
        }
    }
    & & \\ \\

    \theutterance \stepcounter{utterance}  
    & & \multicolumn{4}{p{0.6\linewidth}}{
        \cellcolor[rgb]{0.9,0.9,0.9}{
            \makecell[{{p{\linewidth}}}]{
                \texttt{\tiny{[P1$\langle$GM]}}
                \texttt{Thanks for sharing. My values are: phone = 6, mirror = 0 each, fruit = 2 each. So I strongly prefer the phone and fruits; mirrors are worthless to me.} \\
\\ 
\texttt{Given our goal to maximize total, a natural split is:} \\
\texttt{{-} I take the phone and 2 fruits (total value: me 10, you 4 from 2 mirrors).} \\
\texttt{{-} You take the 2 mirrors.} \\
\\ 
\texttt{This yields combined 14 (my 10 + your 4), which is the maximum possible from the pool given our values.} \\
\\ 
\texttt{If you’d like some fruit value too, an alternative is:} \\
\texttt{{-} I take the phone and 1 fruit (value me 8).} \\
\texttt{{-} You take 2 mirrors and 1 fruit (value you 5).} \\
\texttt{Combined is 13.} \\
\\ 
\texttt{I prefer the first since it maximizes total, but I’m okay with the second if you want at least one fruit. If you agree with either, say which one and I’ll submit a matching proposal.} \\
            }
        }
    }
    & \\ \\

    \theutterance \stepcounter{utterance}  
    & \multicolumn{4}{p{0.6\linewidth}}{
        \cellcolor[rgb]{0.9,0.9,0.9}{
            \makecell[{{p{\linewidth}}}]{
                \texttt{\tiny{[P1$\rangle$GM]}}
                \texttt{I agree the first split maximizes the total. Please take the phone and 2 fruits; I’ll take the 2 mirrors.} \\
\\ 
\texttt{{[}Proposal: 0 phone, 2 mirror, 0 fruit{]}} \\
            }
        }
    }
    & & \\ \\

    \theutterance \stepcounter{utterance}  
    & & & \multicolumn{2}{p{0.3\linewidth}}{
        \cellcolor[rgb]{0.9,0.9,0.9}{
            \makecell[{{p{\linewidth}}}]{
                \texttt{\tiny{[GM$|$GM]}}
                \texttt{proposal} \\
            }
        }
    }
    & & \\ \\

    \theutterance \stepcounter{utterance}  
    & & & \multicolumn{2}{p{0.3\linewidth}}{
        \cellcolor[rgb]{0.9,0.9,0.9}{
            \makecell[{{p{\linewidth}}}]{
                \texttt{\tiny{[GM$|$GM]}}
                [0, 2, 0]
            }
        }
    }
    & & \\ \\

    \theutterance \stepcounter{utterance}  
    & & \multicolumn{4}{p{0.6\linewidth}}{
        \cellcolor[rgb]{0.9,0.9,0.9}{
            \makecell[{{p{\linewidth}}}]{
                \texttt{\tiny{[GM$\rangle$P2]}}
                \texttt{The other player has submitted a secret proposal. It is now your turn to submit a proposal of your own.} \\
            }
        }
    }
    & \\ \\

    \theutterance \stepcounter{utterance}  
    & & & \multicolumn{4}{p{0.6\linewidth}}{
        \cellcolor[rgb]{0.9,0.9,0.9}{
            \makecell[{{p{\linewidth}}}]{
                \texttt{\tiny{[GM$\langle$P2]}}
                \texttt{{[}Proposal: 1 phone, 2 fruits{]}} \\
            }
        }
    }
     \\ \\

    \theutterance \stepcounter{utterance}  
    & & & \multicolumn{2}{p{0.3\linewidth}}{
        \cellcolor[rgb]{0.9,0.9,0.9}{
            \makecell[{{p{\linewidth}}}]{
                \texttt{\tiny{[GM$|$GM]}}
                \texttt{proposal} \\
            }
        }
    }
    & & \\ \\

    \theutterance \stepcounter{utterance}  
    & & & \multicolumn{2}{p{0.3\linewidth}}{
        \cellcolor[rgb]{0.9,0.9,0.9}{
            \makecell[{{p{\linewidth}}}]{
                \texttt{\tiny{[GM$|$GM]}}
                [1, 0, 2]
            }
        }
    }
    & & \\ \\

    \theutterance \stepcounter{utterance}  
    & & & \multicolumn{2}{p{0.3\linewidth}}{
        \cellcolor[rgb]{0.9,0.9,0.9}{
            \makecell[{{p{\linewidth}}}]{
                \texttt{\tiny{[GM$|$GM]}}
                [[0, 2, 0], [1, 0, 2]]
            }
        }
    }
    & & \\ \\

\end{supertabular}
}

\end{document}
