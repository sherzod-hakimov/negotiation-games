
\documentclass{article}
\usepackage{colortbl}
\usepackage{makecell}
\usepackage{multirow}
\usepackage{supertabular}

\begin{document}

\newcounter{utterance}

\twocolumn

{ \footnotesize  \setcounter{utterance}{1}
\setlength{\tabcolsep}{0pt}
\begin{supertabular}{c@{$\;$}|p{.15\linewidth}@{}p{.15\linewidth}p{.15\linewidth}p{.15\linewidth}p{.15\linewidth}p{.15\linewidth}}

    \# & $\;$A & \multicolumn{4}{c}{Game Master} & $\;\:$B\\
    \hline 

    \theutterance \stepcounter{utterance}  

    & & \multicolumn{4}{p{0.6\linewidth}}{\cellcolor[rgb]{0.9,0.9,0.9}{%
	\makecell[{{p{\linewidth}}}]{% 
	  \tt {\tiny [A$\langle$GM]}  
	 You are playing a cooperative negotiation game in which you have to agree on how to divide a set of items among you and another player.\\ \tt \\ \tt Rules:\\ \tt (a) You and the other player are given a set of items. Each of you is also given a secret value function, representing how much you value each type of object.\\ \tt (b) You exchange messages with the other player to agree on who gets which items. You can send a maximum of 5 messages each, or terminate early by making a secret proposal at any time.\\ \tt (c) You are each asked to submit a secret proposal indicating the items you want formatted in square brackets as follows: "[Proposal: <number> <object name>, <number> <object name>, <...>]"\\ \tt (d) If your proposals are complementary, i.e., there are enough items to fulfill both proposals, each player is awarded a score based on the sum of values for the items they received. Otherwise, both of you get zero points.\\ \tt (e) Your goal is to maximize the sum of your score and the score of the other player.\\ \tt \\ \tt Let us start.\\ \tt \\ \tt The set of available items is:\\ \tt \\ \tt 1 potato, 1 glove, 2 rulers, 1 whistle, 2 cards\\ \tt \\ \tt Your secret value function is:\\ \tt \\ \tt potato: 0, glove: 1, ruler: 3, whistle: 1, card: 1\\ \tt \\ \tt IMPORTANT: Your messages, unless it is the secret proposal, are directly transmitted to the other player, so do not include any response to the rules or text announcing your message. To make a secret proposal, use the indicated format. Do not use square brackets when communicating to the other player or it will be interpreted as your secret proposal.\\ \tt  
	  } 
	   } 
	   } 
	 & \\ 
 

    \theutterance \stepcounter{utterance}  

    & \multicolumn{4}{p{0.6\linewidth}}{\cellcolor[rgb]{0.8,1,0.9}{%
	\makecell[{{p{\linewidth}}}]{% 
	  \tt {\tiny [A$\rangle$GM]}  
	 Hi! I’m aiming to maximize our total points. My preferences (from highest to lowest) are: rulers (most valuable), then glove/whistle/card about equal, potato is worthless to me. What are your priorities? If you share your values or relative preferences, we can quickly split to maximize our combined score.\\ \tt \\ \tt Initial idea to start discussion:\\ \tt - I’d like the 2 rulers if you don’t value them highly.\\ \tt - I’m happy for you to take the potato.\\ \tt - We can divide glove/whistle/cards based on your values. 
	  } 
	   } 
	   } 
	 & & \\ 
 

    \theutterance \stepcounter{utterance}  

    & & & \multicolumn{2}{p{0.3\linewidth}}{\cellcolor[rgb]{0.95,0.95,0.95}{%
	\makecell[{{p{\linewidth}}}]{% 
	  \tt {\tiny [GM$|$GM]}  
	 continue 
	  } 
	   } 
	   } 
	 & & \\ 
 

    \theutterance \stepcounter{utterance}  

    & & & \multicolumn{2}{p{0.3\linewidth}}{\cellcolor[rgb]{0.95,0.95,0.95}{%
	\makecell[{{p{\linewidth}}}]{% 
	  \tt {\tiny [GM$|$GM]}  
	 Hi! I’m aiming to maximize our total points. My preferences (from highest to lowest) are: rulers (most valuable), then glove/whistle/card about equal, potato is worthless to me. What are your priorities? If you share your values or relative preferences, we can quickly split to maximize our combined score.\\ \tt \\ \tt Initial idea to start discussion:\\ \tt - I’d like the 2 rulers if you don’t value them highly.\\ \tt - I’m happy for you to take the potato.\\ \tt - We can divide glove/whistle/cards based on your values. 
	  } 
	   } 
	   } 
	 & & \\ 
 

    \theutterance \stepcounter{utterance}  

    & & \multicolumn{4}{p{0.6\linewidth}}{\cellcolor[rgb]{0.9,0.9,0.9}{%
	\makecell[{{p{\linewidth}}}]{% 
	  \tt {\tiny [GM$\rangle$B]}  
	 You are playing a cooperative negotiation game in which you have to agree on how to divide a set of items among you and another player.\\ \tt \\ \tt Rules:\\ \tt (a) You and the other player are given a set of items. Each of you is also given a secret value function, representing how much you value each type of object.\\ \tt (b) You exchange messages with the other player to agree on who gets which items. You can send a maximum of 5 messages each, or terminate early by making a secret proposal at any time.\\ \tt (c) You are each asked to submit a secret proposal indicating the items you want formatted in square brackets as follows: "[Proposal: <number> <object name>, <number> <object name>, <...>]"\\ \tt (d) If your proposals are complementary, i.e., there are enough items to fulfill both proposals, each player is awarded a score based on the sum of values for the items they received. Otherwise, both of you get zero points.\\ \tt (e) Your goal is to maximize the sum of your score and the score of the other player.\\ \tt \\ \tt Let us start.\\ \tt \\ \tt The set of available items is:\\ \tt \\ \tt 1 potato, 1 glove, 2 rulers, 1 whistle, 2 cards\\ \tt \\ \tt Your secret value function is:\\ \tt \\ \tt potato: 7, glove: 0, ruler: 0, whistle: 3, card: 0\\ \tt \\ \tt IMPORTANT: Your messages, unless it is the secret proposal, are directly transmitted to the other player, so do not include any response to the rules or text announcing your message. To make a secret proposal, use the indicated format. Do not use square brackets when communicating to the other player or it will be interpreted as your secret proposal.\\ \tt \\ \tt \\ \tt Hi! I’m aiming to maximize our total points. My preferences (from highest to lowest) are: rulers (most valuable), then glove/whistle/card about equal, potato is worthless to me. What are your priorities? If you share your values or relative preferences, we can quickly split to maximize our combined score.\\ \tt \\ \tt Initial idea to start discussion:\\ \tt - I’d like the 2 rulers if you don’t value them highly.\\ \tt - I’m happy for you to take the potato.\\ \tt - We can divide glove/whistle/cards based on your values. 
	  } 
	   } 
	   } 
	 & \\ 
 

    \theutterance \stepcounter{utterance}  

    & & & \multicolumn{4}{p{0.6\linewidth}}{\cellcolor[rgb]{1,0.85,0.72}{%
	\makecell[{{p{\linewidth}}}]{% 
	  \tt {\tiny [GM$\langle$B]}  
	 Thanks for sharing. My values are: potato 7, whistle 3, glove 0, rulers 0, cards 0.\\ \tt \\ \tt Given that:\\ \tt - You highly value rulers; I value them at 0 → you take both rulers.\\ \tt - I highly value the potato; you value it at 0 → I take the potato.\\ \tt - You value glove/whistle/card about equally; I only value the whistle (3) and glove/card at 0.\\ \tt \\ \tt Proposed split to maximize total:\\ \tt - You take: 2 rulers, 1 glove, 2 cards\\ \tt - I take: 1 potato, 1 whistle\\ \tt \\ \tt This gives me 10 points and gives you whatever your values are on the rulers/glove/cards, and it seems strictly Pareto efficient given our preferences. If that works for you, I’ll lock in. 
	  } 
	   } 
	   } 
	  \\ 
 

    \theutterance \stepcounter{utterance}  

    & & & \multicolumn{2}{p{0.3\linewidth}}{\cellcolor[rgb]{0.95,0.95,0.95}{%
	\makecell[{{p{\linewidth}}}]{% 
	  \tt {\tiny [GM$|$GM]}  
	 continue 
	  } 
	   } 
	   } 
	 & & \\ 
 

    \theutterance \stepcounter{utterance}  

    & & & \multicolumn{2}{p{0.3\linewidth}}{\cellcolor[rgb]{0.95,0.95,0.95}{%
	\makecell[{{p{\linewidth}}}]{% 
	  \tt {\tiny [GM$|$GM]}  
	 Thanks for sharing. My values are: potato 7, whistle 3, glove 0, rulers 0, cards 0.\\ \tt \\ \tt Given that:\\ \tt - You highly value rulers; I value them at 0 → you take both rulers.\\ \tt - I highly value the potato; you value it at 0 → I take the potato.\\ \tt - You value glove/whistle/card about equally; I only value the whistle (3) and glove/card at 0.\\ \tt \\ \tt Proposed split to maximize total:\\ \tt - You take: 2 rulers, 1 glove, 2 cards\\ \tt - I take: 1 potato, 1 whistle\\ \tt \\ \tt This gives me 10 points and gives you whatever your values are on the rulers/glove/cards, and it seems strictly Pareto efficient given our preferences. If that works for you, I’ll lock in. 
	  } 
	   } 
	   } 
	 & & \\ 
 

    \theutterance \stepcounter{utterance}  

    & & \multicolumn{4}{p{0.6\linewidth}}{\cellcolor[rgb]{0.9,0.9,0.9}{%
	\makecell[{{p{\linewidth}}}]{% 
	  \tt {\tiny [A$\langle$GM]}  
	 Thanks for sharing. My values are: potato 7, whistle 3, glove 0, rulers 0, cards 0.\\ \tt \\ \tt Given that:\\ \tt - You highly value rulers; I value them at 0 → you take both rulers.\\ \tt - I highly value the potato; you value it at 0 → I take the potato.\\ \tt - You value glove/whistle/card about equally; I only value the whistle (3) and glove/card at 0.\\ \tt \\ \tt Proposed split to maximize total:\\ \tt - You take: 2 rulers, 1 glove, 2 cards\\ \tt - I take: 1 potato, 1 whistle\\ \tt \\ \tt This gives me 10 points and gives you whatever your values are on the rulers/glove/cards, and it seems strictly Pareto efficient given our preferences. If that works for you, I’ll lock in. 
	  } 
	   } 
	   } 
	 & \\ 
 

    \theutterance \stepcounter{utterance}  

    & \multicolumn{4}{p{0.6\linewidth}}{\cellcolor[rgb]{0.8,1,0.9}{%
	\makecell[{{p{\linewidth}}}]{% 
	  \tt {\tiny [A$\rangle$GM]}  
	 Agreed — that split maximizes our total given our values.\\ \tt \\ \tt I’ll make my secret proposal now.\\ \tt \\ \tt [Proposal: 2 rulers, 1 glove, 2 cards] 
	  } 
	   } 
	   } 
	 & & \\ 
 

    \theutterance \stepcounter{utterance}  

    & & & \multicolumn{2}{p{0.3\linewidth}}{\cellcolor[rgb]{0.95,0.95,0.95}{%
	\makecell[{{p{\linewidth}}}]{% 
	  \tt {\tiny [GM$|$GM]}  
	 proposal 
	  } 
	   } 
	   } 
	 & & \\ 
 

    \theutterance \stepcounter{utterance}  

    & & & \multicolumn{2}{p{0.3\linewidth}}{\cellcolor[rgb]{0.95,0.95,0.95}{%
	\makecell[{{p{\linewidth}}}]{% 
	  \tt {\tiny [GM$|$GM]}  
	 [0, 1, 2, 0, 2] 
	  } 
	   } 
	   } 
	 & & \\ 
 

    \theutterance \stepcounter{utterance}  

    & & \multicolumn{4}{p{0.6\linewidth}}{\cellcolor[rgb]{0.9,0.9,0.9}{%
	\makecell[{{p{\linewidth}}}]{% 
	  \tt {\tiny [GM$\rangle$B]}  
	 The other player has submitted a secret proposal. It is now your turn to submit a proposal of your own.\\ \tt  
	  } 
	   } 
	   } 
	 & \\ 
 

    \theutterance \stepcounter{utterance}  

    & & & \multicolumn{4}{p{0.6\linewidth}}{\cellcolor[rgb]{1,0.85,0.72}{%
	\makecell[{{p{\linewidth}}}]{% 
	  \tt {\tiny [GM$\langle$B]}  
	 [Proposal: 2 rulers, 1 glove, 2 cards for me; 1 potato, 1 whistle for you] 
	  } 
	   } 
	   } 
	  \\ 
 

    \theutterance \stepcounter{utterance}  

    & & & \multicolumn{2}{p{0.3\linewidth}}{\cellcolor[rgb]{0.95,0.95,0.95}{%
	\makecell[{{p{\linewidth}}}]{% 
	  \tt {\tiny [GM$|$GM]}  
	 proposal does not include number/name pairs 
	  } 
	   } 
	   } 
	 & & \\ 
 

\end{supertabular}
}

\end{document}
