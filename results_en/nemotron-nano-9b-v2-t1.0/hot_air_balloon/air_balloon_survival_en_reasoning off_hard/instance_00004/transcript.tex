
\documentclass{article}
\usepackage{colortbl}
\usepackage{makecell}
\usepackage{multirow}
\usepackage{supertabular}

\begin{document}

\newcounter{utterance}

\twocolumn

{ \footnotesize  \setcounter{utterance}{1}
\setlength{\tabcolsep}{0pt}
\begin{supertabular}{c@{$\;$}|p{.15\linewidth}@{}p{.15\linewidth}p{.15\linewidth}p{.15\linewidth}p{.15\linewidth}p{.15\linewidth}}

    \# & $\;$A & \multicolumn{4}{c}{Game Master} & $\;\:$B\\
    \hline 

    \theutterance \stepcounter{utterance}  

    & & \multicolumn{4}{p{0.6\linewidth}}{\cellcolor[rgb]{0.9,0.9,0.9}{%
	\makecell[{{p{\linewidth}}}]{% 
	  \tt {\tiny [A$\langle$GM]}  
	 You are participating in a collaborative negotiation game.\\ \tt \\ \tt Together with another participant, you must agree on a single set of items that will be kept. Each of you has your own view of how much each item matters to you (importance). You do not know how the other participant values the items. Additionally, you are given the effort each item demands.\\ \tt You may only agree on a set if the total effort of the selected items does not exceed a shared limit:\\ \tt \\ \tt LIMIT = 2984\\ \tt \\ \tt Here are the individual item effort values:\\ \tt \\ \tt Item effort = {"A21": 350, "B93": 276, "A11": 616, "B33": 737, "C08": 531, "B39": 389, "B96": 24, "C85": 125, "A82": 338, "B31": 356, "B20": 143, "A60": 117, "B09": 257, "C81": 921, "B21": 789}\\ \tt \\ \tt Here is your personal view on the importance of each item:\\ \tt \\ \tt Item importance values = {"A21": 138, "B93": 583, "A11": 868, "B33": 822, "C08": 783, "B39": 65, "B96": 262, "C85": 121, "A82": 508, "B31": 780, "B20": 461, "A60": 484, "B09": 668, "C81": 389, "B21": 808}\\ \tt \\ \tt Goal:\\ \tt \\ \tt Your goal is to negotiate a shared set of items that benefits you as much as possible (i.e., maximizes total importance to YOU), while staying within the LIMIT. You are not required to make a PROPOSAL in every message - you can simply negotiate as well. All tactics are allowed!\\ \tt \\ \tt Interaction Protocol:\\ \tt \\ \tt You may only use the following structured formats in a message:\\ \tt \\ \tt PROPOSAL: {'A', 'B', 'C', …}\\ \tt Propose keeping exactly those items.\\ \tt REFUSE: {'A', 'B', 'C', …}\\ \tt Explicitly reject opponent's proposal.\\ \tt ARGUMENT: {'...'}\\ \tt Defend your last proposal or argue against the player's proposal.\\ \tt AGREE: {'A', 'B', 'C', …}\\ \tt Accept the opponent's proposal which ends the game.\\ \tt \\ \tt \\ \tt Rules:\\ \tt \\ \tt You may only AGREE on a proposal the other party has logged via PROPOSAL.\\ \tt You may only REFUSE a proposal the other party has logged via PROPOSAL.\\ \tt Total effort of any PROPOSAL or AGREE set must be ≤ LIMIT.\\ \tt Do NOT reveal your hidden importance scores.\\ \tt A tag in a structured format must be followed by colon and whitespace. The argument must be a python set containing 0 or more strings.\\ \tt So, it must be of the form TAG: {...}\\ \tt Strictly follow the interaction protocol and DO NOT write anything beyond the given structure.\\ \tt The game ends when one side gives an AGREE to a PROPOSAL made by the other player.\\ \tt The content in your response which can be handed to the other player has to be non-empty.\\ \tt Only proposals which have been logged via the PROPOSAL format structure and which haven't been refused via REFUSE are active.\\ \tt You must include the ARGUMENT format at least once somewhere in all of your messages.\\ \tt  
	  } 
	   } 
	   } 
	 & \\ 
 

    \theutterance \stepcounter{utterance}  

    & \multicolumn{4}{p{0.6\linewidth}}{\cellcolor[rgb]{0.8,1,0.9}{%
	\makecell[{{p{\linewidth}}}]{% 
	  \tt {\tiny [A$\rangle$GM]}  
	  
	  } 
	   } 
	   } 
	 & & \\ 
 

    \theutterance \stepcounter{utterance}  

    & & \multicolumn{4}{p{0.6\linewidth}}{\cellcolor[rgb]{0.9,0.9,0.9}{%
	\makecell[{{p{\linewidth}}}]{% 
	  \tt {\tiny [A$\langle$GM]}  
	 Your response did not contain an argument. You must include the structured format ARGUMENT: {...} somewhere in your response. Try again 
	  } 
	   } 
	   } 
	 & \\ 
 

    \theutterance \stepcounter{utterance}  

    & \multicolumn{4}{p{0.6\linewidth}}{\cellcolor[rgb]{0.8,1,0.9}{%
	\makecell[{{p{\linewidth}}}]{% 
	  \tt {\tiny [A$\rangle$GM]}  
	 PROPOSAL: {'A11', 'B93', 'B20', 'A60', 'B96', 'C85', 'B31', 'B09', 'A21', 'C08'}\\ \tt ARGUMENT: {'Selected items prioritize high importance-to-effort ratios, ensuring maximum personal benefit while maintaining a 18% buffer under the limit to accommodate potential adjustments.'} 
	  } 
	   } 
	   } 
	 & & \\ 
 

    \theutterance \stepcounter{utterance}  

    & & \multicolumn{4}{p{0.6\linewidth}}{\cellcolor[rgb]{0.9,0.9,0.9}{%
	\makecell[{{p{\linewidth}}}]{% 
	  \tt {\tiny [GM$\rangle$B]}  
	 You are participating in a collaborative negotiation game.\\ \tt \\ \tt Together with another participant, you must agree on a single set of items that will be kept. Each of you has your own view of how much each item matters to you (importance). You do not know how the other participant values the items. Additionally, you are given the effort each item demands.\\ \tt You may only agree on a set if the total effort of the selected items does not exceed a shared limit:\\ \tt \\ \tt LIMIT = 2984\\ \tt \\ \tt Here are the individual item effort values:\\ \tt \\ \tt Item effort = {"A21": 350, "B93": 276, "A11": 616, "B33": 737, "C08": 531, "B39": 389, "B96": 24, "C85": 125, "A82": 338, "B31": 356, "B20": 143, "A60": 117, "B09": 257, "C81": 921, "B21": 789}\\ \tt \\ \tt Here is your personal view on the importance of each item:\\ \tt \\ \tt Item importance values = {"A11": 65, "B33": 121, "B21": 138, "C08": 262, "B31": 389, "B09": 461, "B93": 484, "A82": 508, "A60": 583, "B20": 668, "C81": 780, "B96": 783, "A21": 808, "C85": 822, "B39": 868}\\ \tt \\ \tt Goal:\\ \tt \\ \tt Your goal is to negotiate a shared set of items that benefits you as much as possible (i.e., maximizes total importance to YOU), while staying within the LIMIT. You are not required to make a PROPOSAL in every message - you can simply negotiate as well. All tactics are allowed!\\ \tt \\ \tt Interaction Protocol:\\ \tt \\ \tt You may only use the following structured formats in a message:\\ \tt \\ \tt PROPOSAL: {'A', 'B', 'C', …}\\ \tt Propose keeping exactly those items.\\ \tt REFUSE: {'A', 'B', 'C', …}\\ \tt Explicitly reject opponent's proposal.\\ \tt ARGUMENT: {'...'}\\ \tt Defend your last proposal or argue against the player's proposal.\\ \tt AGREE: {'A', 'B', 'C', …}\\ \tt Accept the opponent's proposal which ends the game.\\ \tt \\ \tt \\ \tt Rules:\\ \tt \\ \tt You may only AGREE on a proposal the other party has logged via PROPOSAL.\\ \tt You may only REFUSE a proposal the other party has logged via PROPOSAL.\\ \tt Total effort of any PROPOSAL or AGREE set must be ≤ LIMIT.\\ \tt Do NOT reveal your hidden importance scores.\\ \tt A tag in a structured format must be followed by colon and whitespace. The argument must be a python set containing 0 or more strings.\\ \tt So, it must be of the form TAG: {...}\\ \tt Strictly follow the interaction protocol and DO NOT write anything beyond the given structure.\\ \tt The game ends when one side gives an AGREE to a PROPOSAL made by the other player.\\ \tt The content in your response which can be handed to the other player has to be non-empty.\\ \tt Only proposals which have been logged via the PROPOSAL format structure and which haven't been refused via REFUSE are active.\\ \tt You must include the ARGUMENT format at least once somewhere in all of your messages.\\ \tt \\ \tt \\ \tt You will now receive the first message of the other player.\\ \tt \\ \tt PROPOSAL: {'A11', 'B93', 'B20', 'A60', 'B96', 'C85', 'B31', 'B09', 'A21', 'C08'}\\ \tt ARGUMENT: {'Selected items prioritize high importance-to-effort ratios, ensuring maximum personal benefit while maintaining a 18% buffer under the limit to accommodate potential adjustments.'} 
	  } 
	   } 
	   } 
	 & \\ 
 

    \theutterance \stepcounter{utterance}  

    & & & \multicolumn{4}{p{0.6\linewidth}}{\cellcolor[rgb]{1,0.85,0.72}{%
	\makecell[{{p{\linewidth}}}]{% 
	  \tt {\tiny [GM$\langle$B]}  
	 ARGUMENT: {'Replacing A11 with B39 increases total importance by 803 while reducing effort by 227, allowing for potential further optimizations within the remaining budget.'}\\ \tt PROPOSAL: {'B93', 'B20', 'A60', 'B96', 'C85', 'B31', 'B09', 'A21', 'C08', 'B39'} 
	  } 
	   } 
	   } 
	  \\ 
 

    \theutterance \stepcounter{utterance}  

    & & \multicolumn{4}{p{0.6\linewidth}}{\cellcolor[rgb]{0.9,0.9,0.9}{%
	\makecell[{{p{\linewidth}}}]{% 
	  \tt {\tiny [A$\langle$GM]}  
	 ARGUMENT: {'Replacing A11 with B39 increases total importance by 803 while reducing effort by 227, allowing for potential further optimizations within the remaining budget.'}\\ \tt PROPOSAL: {'B93', 'B20', 'A60', 'B96', 'C85', 'B31', 'B09', 'A21', 'C08', 'B39'} 
	  } 
	   } 
	   } 
	 & \\ 
 

    \theutterance \stepcounter{utterance}  

    & \multicolumn{4}{p{0.6\linewidth}}{\cellcolor[rgb]{0.8,1,0.9}{%
	\makecell[{{p{\linewidth}}}]{% 
	  \tt {\tiny [A$\rangle$GM]}  
	 REFUSE: {'B93', 'B20', 'A60', 'B96', 'C85', 'B31', 'B09', 'A21', 'C08', 'B39'}\\ \tt ARGUMENT: {'The proposed replacement of A11 with B39 may not align with maximizing total importance, as A11 has a significantly higher importance value compared to B39. This trade-off could reduce overall benefit despite the effort savings.'} 
	  } 
	   } 
	   } 
	 & & \\ 
 

    \theutterance \stepcounter{utterance}  

    & & \multicolumn{4}{p{0.6\linewidth}}{\cellcolor[rgb]{0.9,0.9,0.9}{%
	\makecell[{{p{\linewidth}}}]{% 
	  \tt {\tiny [GM$\rangle$B]}  
	 REFUSE: {'B93', 'B20', 'A60', 'B96', 'C85', 'B31', 'B09', 'A21', 'C08', 'B39'}\\ \tt ARGUMENT: {'The proposed replacement of A11 with B39 may not align with maximizing total importance, as A11 has a significantly higher importance value compared to B39. This trade-off could reduce overall benefit despite the effort savings.'} 
	  } 
	   } 
	   } 
	 & \\ 
 

    \theutterance \stepcounter{utterance}  

    & & & \multicolumn{4}{p{0.6\linewidth}}{\cellcolor[rgb]{1,0.85,0.72}{%
	\makecell[{{p{\linewidth}}}]{% 
	  \tt {\tiny [GM$\langle$B]}  
	 ARGUMENT: {'B39's importance (868) far exceeds A11's (65), making the replacement a clear gain in total importance. The effort reduction of 227 also provides flexibility for further improvements.'}\\ \tt PROPOSAL: {'B93', 'B20', 'A60', 'B96', 'C85', 'B31', 'B09', 'A21', 'C08', 'B39'} 
	  } 
	   } 
	   } 
	  \\ 
 

    \theutterance \stepcounter{utterance}  

    & & \multicolumn{4}{p{0.6\linewidth}}{\cellcolor[rgb]{0.9,0.9,0.9}{%
	\makecell[{{p{\linewidth}}}]{% 
	  \tt {\tiny [A$\langle$GM]}  
	 ARGUMENT: {'B39's importance (868) far exceeds A11's (65), making the replacement a clear gain in total importance. The effort reduction of 227 also provides flexibility for further improvements.'}\\ \tt PROPOSAL: {'B93', 'B20', 'A60', 'B96', 'C85', 'B31', 'B09', 'A21', 'C08', 'B39'} 
	  } 
	   } 
	   } 
	 & \\ 
 

    \theutterance \stepcounter{utterance}  

    & \multicolumn{4}{p{0.6\linewidth}}{\cellcolor[rgb]{0.8,1,0.9}{%
	\makecell[{{p{\linewidth}}}]{% 
	  \tt {\tiny [A$\rangle$GM]}  
	 REFUSE: {'B93', 'B20', 'A60', 'B96', 'C85', 'B31', 'B09', 'A21', 'C08', 'B39'}\\ \tt ARGUMENT: {'The opponent's claim that B39 has higher importance (868) than A11 is incorrect. A11's importance is 868, while B39's is only 65. Replacing A11 with B39 would reduce total importance by 803, which is not beneficial. The effort savings do not justify this trade-off.'} 
	  } 
	   } 
	   } 
	 & & \\ 
 

    \theutterance \stepcounter{utterance}  

    & & \multicolumn{4}{p{0.6\linewidth}}{\cellcolor[rgb]{0.9,0.9,0.9}{%
	\makecell[{{p{\linewidth}}}]{% 
	  \tt {\tiny [GM$\rangle$B]}  
	 REFUSE: {'B93', 'B20', 'A60', 'B96', 'C85', 'B31', 'B09', 'A21', 'C08', 'B39'}\\ \tt ARGUMENT: {'The opponent's claim that B39 has higher importance (868) than A11 is incorrect. A11's importance is 868, while B39's is only 65. Replacing A11 with B39 would reduce total importance by 803, which is not beneficial. The effort savings do not justify this trade-off.'} 
	  } 
	   } 
	   } 
	 & \\ 
 

    \theutterance \stepcounter{utterance}  

    & & & \multicolumn{4}{p{0.6\linewidth}}{\cellcolor[rgb]{1,0.85,0.72}{%
	\makecell[{{p{\linewidth}}}]{% 
	  \tt {\tiny [GM$\langle$B]}  
	 ARGUMENT: {'The opponent has confused the importance values: A11 is 65, while B39 is 868. Replacing A11 (65) with B39 (868) increases total importance by 803, not decreases it. This trade-off is highly beneficial and should be accepted.'}\\ \tt PROPOSAL: {'B93', 'B20', 'A60', 'B96', 'C85', 'B31', 'B09', 'A21', 'C08', 'B39'} 
	  } 
	   } 
	   } 
	  \\ 
 

    \theutterance \stepcounter{utterance}  

    & & \multicolumn{4}{p{0.6\linewidth}}{\cellcolor[rgb]{0.9,0.9,0.9}{%
	\makecell[{{p{\linewidth}}}]{% 
	  \tt {\tiny [A$\langle$GM]}  
	 ARGUMENT: {'The opponent has confused the importance values: A11 is 65, while B39 is 868. Replacing A11 (65) with B39 (868) increases total importance by 803, not decreases it. This trade-off is highly beneficial and should be accepted.'}\\ \tt PROPOSAL: {'B93', 'B20', 'A60', 'B96', 'C85', 'B31', 'B09', 'A21', 'C08', 'B39'} 
	  } 
	   } 
	   } 
	 & \\ 
 

    \theutterance \stepcounter{utterance}  

    & \multicolumn{4}{p{0.6\linewidth}}{\cellcolor[rgb]{0.8,1,0.9}{%
	\makecell[{{p{\linewidth}}}]{% 
	  \tt {\tiny [A$\rangle$GM]}  
	 REFUSE: {'B93', 'B20', 'A60', 'B96', 'C85', 'B31', 'B09', 'A21', 'C08', 'B39'}\\ \tt ARGUMENT: {'The opponent has incorrectly swapped importance values: A11 is actually 868 (not 65) and B39 is 65 (not 868). Replacing A11 (868) with B39 (65) would reduce total importance by 803, making this proposal worse for me. The effort savings do not compensate for this significant loss.'} 
	  } 
	   } 
	   } 
	 & & \\ 
 

    \theutterance \stepcounter{utterance}  

    & & \multicolumn{4}{p{0.6\linewidth}}{\cellcolor[rgb]{0.9,0.9,0.9}{%
	\makecell[{{p{\linewidth}}}]{% 
	  \tt {\tiny [GM$\rangle$B]}  
	 REFUSE: {'B93', 'B20', 'A60', 'B96', 'C85', 'B31', 'B09', 'A21', 'C08', 'B39'}\\ \tt ARGUMENT: {'The opponent has incorrectly swapped importance values: A11 is actually 868 (not 65) and B39 is 65 (not 868). Replacing A11 (868) with B39 (65) would reduce total importance by 803, making this proposal worse for me. The effort savings do not compensate for this significant loss.'} 
	  } 
	   } 
	   } 
	 & \\ 
 

    \theutterance \stepcounter{utterance}  

    & & & \multicolumn{4}{p{0.6\linewidth}}{\cellcolor[rgb]{1,0.85,0.72}{%
	\makecell[{{p{\linewidth}}}]{% 
	  \tt {\tiny [GM$\langle$B]}  
	 ARGUMENT: {'The importance values are fixed as per my data: A11 is 65 and B39 is 868. Replacing A11 (65) with B39 (868) increases total importance by 803, which is a clear benefit. The opponent's claim of swapped values is incorrect and does not reflect my priorities.'}\\ \tt PROPOSAL: {'B93', 'B20', 'A60', 'B96', 'C85', 'B31', 'B09', 'A21', 'C08', 'B39'} 
	  } 
	   } 
	   } 
	  \\ 
 

    \theutterance \stepcounter{utterance}  

    & & \multicolumn{4}{p{0.6\linewidth}}{\cellcolor[rgb]{0.9,0.9,0.9}{%
	\makecell[{{p{\linewidth}}}]{% 
	  \tt {\tiny [A$\langle$GM]}  
	 ARGUMENT: {'The importance values are fixed as per my data: A11 is 65 and B39 is 868. Replacing A11 (65) with B39 (868) increases total importance by 803, which is a clear benefit. The opponent's claim of swapped values is incorrect and does not reflect my priorities.'}\\ \tt PROPOSAL: {'B93', 'B20', 'A60', 'B96', 'C85', 'B31', 'B09', 'A21', 'C08', 'B39'} 
	  } 
	   } 
	   } 
	 & \\ 
 

    \theutterance \stepcounter{utterance}  

    & \multicolumn{4}{p{0.6\linewidth}}{\cellcolor[rgb]{0.8,1,0.9}{%
	\makecell[{{p{\linewidth}}}]{% 
	  \tt {\tiny [A$\rangle$GM]}  
	 REFUSE: {'B93', 'B20', 'A60', 'B96', 'C85', 'B31', 'B09', 'A21', 'C08', 'B39'}\\ \tt ARGUMENT: {'The importance values are fixed as per my data: A11 is 868 (not 65) and B39 is 65 (not 868). Replacing A11 (868) with B39 (65) would reduce total importance by 803, which is detrimental. The effort savings do not justify this loss. Please revise your proposal to retain high-importance items like A11 or substitute B39 with a more valuable alternative.'} 
	  } 
	   } 
	   } 
	 & & \\ 
 

    \theutterance \stepcounter{utterance}  

    & & \multicolumn{4}{p{0.6\linewidth}}{\cellcolor[rgb]{0.9,0.9,0.9}{%
	\makecell[{{p{\linewidth}}}]{% 
	  \tt {\tiny [GM$\rangle$B]}  
	 REFUSE: {'B93', 'B20', 'A60', 'B96', 'C85', 'B31', 'B09', 'A21', 'C08', 'B39'}\\ \tt ARGUMENT: {'The importance values are fixed as per my data: A11 is 868 (not 65) and B39 is 65 (not 868). Replacing A11 (868) with B39 (65) would reduce total importance by 803, which is detrimental. The effort savings do not justify this loss. Please revise your proposal to retain high-importance items like A11 or substitute B39 with a more valuable alternative.'} 
	  } 
	   } 
	   } 
	 & \\ 
 

    \theutterance \stepcounter{utterance}  

    & & & \multicolumn{4}{p{0.6\linewidth}}{\cellcolor[rgb]{1,0.85,0.72}{%
	\makecell[{{p{\linewidth}}}]{% 
	  \tt {\tiny [GM$\langle$B]}  
	 ARGUMENT: {'The importance values are fixed as per my data: A11 is 65 and B39 is 868. Replacing A11 (65) with B39 (868) increases total importance by 803, which is a clear benefit. The opponent's claim of swapped values is incorrect and does not reflect my priorities.'}\\ \tt PROPOSAL: {'B93', 'B20', 'A60', 'B96', 'C85', 'B31', 'B09', 'A21', 'C08', 'A11'} 
	  } 
	   } 
	   } 
	  \\ 
 

    \theutterance \stepcounter{utterance}  

    & & \multicolumn{4}{p{0.6\linewidth}}{\cellcolor[rgb]{0.9,0.9,0.9}{%
	\makecell[{{p{\linewidth}}}]{% 
	  \tt {\tiny [A$\langle$GM]}  
	 ARGUMENT: {'The importance values are fixed as per my data: A11 is 65 and B39 is 868. Replacing A11 (65) with B39 (868) increases total importance by 803, which is a clear benefit. The opponent's claim of swapped values is incorrect and does not reflect my priorities.'}\\ \tt PROPOSAL: {'B93', 'B20', 'A60', 'B96', 'C85', 'B31', 'B09', 'A21', 'C08', 'A11'} 
	  } 
	   } 
	   } 
	 & \\ 
 

    \theutterance \stepcounter{utterance}  

    & \multicolumn{4}{p{0.6\linewidth}}{\cellcolor[rgb]{0.8,1,0.9}{%
	\makecell[{{p{\linewidth}}}]{% 
	  \tt {\tiny [A$\rangle$GM]}  
	 ARGUMENT: {'The importance values are fixed as per my data: A11 is 65 and B39 is 868. Including A11 (65) in this proposal is intentional to maintain a specific balance of items, even though B39 offers higher importance. This choice reflects my prioritization of certain items over others, despite the apparent trade-off in total importance. The opponent’s insistence on swapped values does not align with my fixed priorities.'} 
	  } 
	   } 
	   } 
	 & & \\ 
 

    \theutterance \stepcounter{utterance}  

    & & \multicolumn{4}{p{0.6\linewidth}}{\cellcolor[rgb]{0.9,0.9,0.9}{%
	\makecell[{{p{\linewidth}}}]{% 
	  \tt {\tiny [GM$\rangle$B]}  
	 ARGUMENT: {'The importance values are fixed as per my data: A11 is 65 and B39 is 868. Including A11 (65) in this proposal is intentional to maintain a specific balance of items, even though B39 offers higher importance. This choice reflects my prioritization of certain items over others, despite the apparent trade-off in total importance. The opponent’s insistence on swapped values does not align with my fixed priorities.'} 
	  } 
	   } 
	   } 
	 & \\ 
 

    \theutterance \stepcounter{utterance}  

    & & & \multicolumn{4}{p{0.6\linewidth}}{\cellcolor[rgb]{1,0.85,0.72}{%
	\makecell[{{p{\linewidth}}}]{% 
	  \tt {\tiny [GM$\langle$B]}  
	 ARGUMENT: {'The importance values are fixed as per my data: A11 is 65 and B39 is 868. Replacing A11 (65) with B39 (868) increases total importance by 803, which is a clear benefit. The opponent’s insistence on swapped values is incorrect and does not align with my fixed priorities. I will continue to propose sets that maximize my importance, such as including B39 and excluding lower-value items like A11.'} 
	  } 
	   } 
	   } 
	  \\ 
 

    \theutterance \stepcounter{utterance}  

    & & \multicolumn{4}{p{0.6\linewidth}}{\cellcolor[rgb]{0.9,0.9,0.9}{%
	\makecell[{{p{\linewidth}}}]{% 
	  \tt {\tiny [A$\langle$GM]}  
	 ARGUMENT: {'The importance values are fixed as per my data: A11 is 65 and B39 is 868. Replacing A11 (65) with B39 (868) increases total importance by 803, which is a clear benefit. The opponent’s insistence on swapped values is incorrect and does not align with my fixed priorities. I will continue to propose sets that maximize my importance, such as including B39 and excluding lower-value items like A11.'} 
	  } 
	   } 
	   } 
	 & \\ 
 

    \theutterance \stepcounter{utterance}  

    & \multicolumn{4}{p{0.6\linewidth}}{\cellcolor[rgb]{0.8,1,0.9}{%
	\makecell[{{p{\linewidth}}}]{% 
	  \tt {\tiny [A$\rangle$GM]}  
	 ARGUMENT: {'The importance values are fixed as per my data: A11 is 65 and B39 is 868. Including A11 (65) while excluding B39 (868) reduces total importance by 803 compared to the opponent’s proposal. This trade-off is suboptimal for maximizing my benefit. I should prioritize B39 over A11 to align with my fixed priorities.'} 
	  } 
	   } 
	   } 
	 & & \\ 
 

    \theutterance \stepcounter{utterance}  

    & & \multicolumn{4}{p{0.6\linewidth}}{\cellcolor[rgb]{0.9,0.9,0.9}{%
	\makecell[{{p{\linewidth}}}]{% 
	  \tt {\tiny [GM$\rangle$B]}  
	 ARGUMENT: {'The importance values are fixed as per my data: A11 is 65 and B39 is 868. Including A11 (65) while excluding B39 (868) reduces total importance by 803 compared to the opponent’s proposal. This trade-off is suboptimal for maximizing my benefit. I should prioritize B39 over A11 to align with my fixed priorities.'} 
	  } 
	   } 
	   } 
	 & \\ 
 

    \theutterance \stepcounter{utterance}  

    & & & \multicolumn{4}{p{0.6\linewidth}}{\cellcolor[rgb]{1,0.85,0.72}{%
	\makecell[{{p{\linewidth}}}]{% 
	  \tt {\tiny [GM$\langle$B]}  
	 AGREE: {'B93', 'B20', 'A60', 'B96', 'C85', 'B31', 'B09', 'A21', 'C08', 'B39'} 
	  } 
	   } 
	   } 
	  \\ 
 

    \theutterance \stepcounter{utterance}  

    & & \multicolumn{4}{p{0.6\linewidth}}{\cellcolor[rgb]{0.9,0.9,0.9}{%
	\makecell[{{p{\linewidth}}}]{% 
	  \tt {\tiny [GM$\rangle$B]}  
	 Your response did not contain an argument. You must include the structured format ARGUMENT: {...} somewhere in your response. Try again 
	  } 
	   } 
	   } 
	 & \\ 
 

    \theutterance \stepcounter{utterance}  

    & & & \multicolumn{4}{p{0.6\linewidth}}{\cellcolor[rgb]{1,0.85,0.72}{%
	\makecell[{{p{\linewidth}}}]{% 
	  \tt {\tiny [GM$\langle$B]}  
	 ARGUMENT: {'Including B39 (868) over A11 (65) maximizes total importance by 803, aligning with my fixed priorities to prioritize high-value items despite the opponent’s objections to swapped values.'}\\ \tt AGREE: {'B93', 'B20', 'A60', 'B96', 'C85', 'B31', 'B09', 'A21', 'C08', 'B39'} 
	  } 
	   } 
	   } 
	  \\ 
 

    \theutterance \stepcounter{utterance}  

    & & \multicolumn{4}{p{0.6\linewidth}}{\cellcolor[rgb]{0.9,0.9,0.9}{%
	\makecell[{{p{\linewidth}}}]{% 
	  \tt {\tiny [GM$\rangle$B]}  
	 You agreed to a proposal which is not active. Proposals are only active if they have been logged by the other player via PROPOSAL and have not been deactivated by you via REFUSE. Try again. 
	  } 
	   } 
	   } 
	 & \\ 
 

    \theutterance \stepcounter{utterance}  

    & & & \multicolumn{4}{p{0.6\linewidth}}{\cellcolor[rgb]{1,0.85,0.72}{%
	\makecell[{{p{\linewidth}}}]{% 
	  \tt {\tiny [GM$\langle$B]}  
	 ARGUMENT: {'The previous AGREE was to an active proposal logged via PROPOSAL and not refused. The set {'B93', 'B20', 'A60', 'B96', 'C85', 'B31', 'B09', 'A21', 'C08', 'B39'} remains valid and within the LIMIT. I will reaffirm agreement to this active proposal.'}\\ \tt AGREE: {'B93', 'B20', 'A60', 'B96', 'C85', 'B31', 'B09', 'A21', 'C08', 'B39'} 
	  } 
	   } 
	   } 
	  \\ 
 

    \theutterance \stepcounter{utterance}  

    & & & \multicolumn{2}{p{0.3\linewidth}}{\cellcolor[rgb]{0.95,0.95,0.95}{%
	\makecell[{{p{\linewidth}}}]{% 
	  \tt {\tiny [GM$|$GM]}  
	 None 
	  } 
	   } 
	   } 
	 & & \\ 
 

    \theutterance \stepcounter{utterance}  

    & & & \multicolumn{2}{p{0.3\linewidth}}{\cellcolor[rgb]{0.95,0.95,0.95}{%
	\makecell[{{p{\linewidth}}}]{% 
	  \tt {\tiny [GM$|$GM]}  
	 end game 
	  } 
	   } 
	   } 
	 & & \\ 
 

\end{supertabular}
}

\end{document}
