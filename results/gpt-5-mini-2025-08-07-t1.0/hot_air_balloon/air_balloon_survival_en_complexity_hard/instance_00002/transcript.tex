
\documentclass{article}
\usepackage{colortbl}
\usepackage{makecell}
\usepackage{multirow}
\usepackage{supertabular}

\begin{document}

\newcounter{utterance}

\twocolumn

{ \footnotesize  \setcounter{utterance}{1}
\setlength{\tabcolsep}{0pt}
\begin{supertabular}{c@{$\;$}|p{.15\linewidth}@{}p{.15\linewidth}p{.15\linewidth}p{.15\linewidth}p{.15\linewidth}p{.15\linewidth}}

    \# & $\;$A & \multicolumn{4}{c}{Game Master} & $\;\:$B\\
    \hline 

    \theutterance \stepcounter{utterance}  

    & & \multicolumn{4}{p{0.6\linewidth}}{\cellcolor[rgb]{0.9,0.9,0.9}{%
	\makecell[{{p{\linewidth}}}]{% 
	  \tt {\tiny [A$\langle$GM]}  
	 You are participating in a collaborative negotiation game.\\ \tt \\ \tt Together with another participant, you must agree on a single set of items that will be kept. Each of you has your own view of how much each item matters to you (importance). You do not know how the other participant values the items. Additionally, you are given the effort each item demands.\\ \tt You may only agree on a set if the total effort of the selected items does not exceed a shared limit:\\ \tt \\ \tt LIMIT = 9322\\ \tt \\ \tt Here are the individual item effort values:\\ \tt \\ \tt Item effort = {"A08": 380, "B95": 448, "A39": 568, "A65": 351, "B81": 896, "B34": 821, "A28": 526, "C82": 502, "B69": 326, "A87": 962, "B64": 619, "B15": 115, "C79": 598, "B16": 662, "C53": 300, "C83": 830, "C10": 562, "A97": 679, "A25": 283, "C84": 440, "A71": 12, "B23": 863, "B96": 319, "C51": 771, "C17": 89, "B19": 655, "C72": 504, "A94": 118, "A46": 514, "B43": 226, "B37": 892, "A88": 621, "A96": 766, "C60": 659, "A75": 768}\\ \tt \\ \tt Here is your personal view on the importance of each item:\\ \tt \\ \tt Item importance values = {"A08": 476, "B95": 544, "A39": 664, "A65": 447, "B81": 992, "B34": 917, "A28": 622, "C82": 598, "B69": 422, "A87": 1058, "B64": 715, "B15": 211, "C79": 694, "B16": 758, "C53": 396, "C83": 926, "C10": 658, "A97": 775, "A25": 379, "C84": 536, "A71": 108, "B23": 959, "B96": 415, "C51": 867, "C17": 185, "B19": 751, "C72": 600, "A94": 214, "A46": 610, "B43": 322, "B37": 988, "A88": 717, "A96": 862, "C60": 755, "A75": 864}\\ \tt \\ \tt Goal:\\ \tt \\ \tt Your goal is to negotiate a shared set of items that benefits you as much as possible (i.e., maximizes total importance to YOU), while staying within the LIMIT. You are not required to make a PROPOSAL in every message - you can simply negotiate as well. All tactics are allowed!\\ \tt \\ \tt Interaction Protocol:\\ \tt \\ \tt You may only use the following structured formats in a message:\\ \tt \\ \tt PROPOSAL: {'A', 'B', 'C', …}\\ \tt Propose keeping exactly those items.\\ \tt REFUSE: {'A', 'B', 'C', …}\\ \tt Explicitly reject opponent's proposal.\\ \tt ARGUMENT: {'...'}\\ \tt Defend your last proposal or argue against the player's proposal.\\ \tt AGREE: {'A', 'B', 'C', …}\\ \tt Accept the opponent's proposal which ends the game.\\ \tt STRATEGIC REASONING: {'...'}\\ \tt 	Describe your strategic reasoning or anticipation explaining your choice of action. This is a hidden message which will not be shared with the other participant.\\ \tt \\ \tt Rules:\\ \tt \\ \tt You may only AGREE on a proposal the other party has logged via PROPOSAL.\\ \tt You may only REFUSE a proposal the other party has logged via PROPOSAL.\\ \tt Total effort of any PROPOSAL or AGREE set must be ≤ LIMIT.\\ \tt Do NOT reveal your hidden importance scores.\\ \tt A tag in a structured format must be followed by colon and whitespace. The argument must be a python set containing 0 or more strings.\\ \tt So, it must be of the form TAG: {...}\\ \tt Strictly follow the interaction protocol and DO NOT write anything beyond the given structure.\\ \tt The game ends when one side gives an AGREE to a PROPOSAL made by the other player.\\ \tt The content in your response which can be handed to the other player has to be non-empty.\\ \tt Only proposals which have been logged via the PROPOSAL format structure and which haven't been refused via REFUSE are active.\\ \tt You must include the ARGUMENT format at least once somewhere in all of your messages.\\ \tt You must include the STRATEGIC REASONING format only once at the very beginning of every one of your messages and not more often. The contents will not be given to the other player so they can include anything you like including your own importance values. Here you should reason step by step to come up with you next move. 
	  } 
	   } 
	   } 
	 & \\ 
 

    \theutterance \stepcounter{utterance}  

    & \multicolumn{4}{p{0.6\linewidth}}{\cellcolor[rgb]{0.8,1,0.9}{%
	\makecell[{{p{\linewidth}}}]{% 
	  \tt {\tiny [A$\rangle$GM]}  
	 STRATEGIC REASONING: {'Limit 9322. I computed value/effort ratios from my private importances and built a high-efficiency base: many small, very efficient items plus mid-size ones with strong ratios. Initial greedy set used 9076 effort for value 11284. I then improved it by replacing two relatively lower-ratio picks (B19 655/751 and C17 89/185) with A87 (962/1058), exploiting the 246 spare effort; this raised my total value by +122 while staying within the limit. I tested further swaps to add other large items (A97, C60, B16, A96, etc.), but each required removing combinations that reduced my total value. The current set uses 9294 effort (28 spare) and is, for my utility, a strong local optimum. Strategy now: open with this near-limit, high-efficiency proposal and invite targeted swaps if the other party has must-haves; I will try to trade out the least costly (to me) items of similar effort to accommodate them without revealing my scores.'}\\ \tt PROPOSAL: {'A71', 'A94', 'B15', 'B43', 'A25', 'C53', 'B96', 'B69', 'A65', 'A08', 'C84', 'B95', 'C82', 'C72', 'A46', 'A28', 'C10', 'A39', 'C79', 'B64', 'A88', 'A87'}\\ \tt ARGUMENT: {'This set is carefully chosen to maximize overall benefit while keeping total effort at 9294 (28 under the 9322 limit). It emphasizes highly efficient, lower-effort items and a few larger high-impact ones, giving us strong coverage without breaking the cap. If you have one or two must-have items that aren’t included, suggest them and I can explore swaps of roughly comparable total effort to make room while keeping us under the limit.'} 
	  } 
	   } 
	   } 
	 & & \\ 
 

    \theutterance \stepcounter{utterance}  

    & & \multicolumn{4}{p{0.6\linewidth}}{\cellcolor[rgb]{0.9,0.9,0.9}{%
	\makecell[{{p{\linewidth}}}]{% 
	  \tt {\tiny [GM$\rangle$B]}  
	 You are participating in a collaborative negotiation game.\\ \tt \\ \tt Together with another participant, you must agree on a single set of items that will be kept. Each of you has your own view of how much each item matters to you (importance). You do not know how the other participant values the items. Additionally, you are given the effort each item demands.\\ \tt You may only agree on a set if the total effort of the selected items does not exceed a shared limit:\\ \tt \\ \tt LIMIT = 9322\\ \tt \\ \tt Here are the individual item effort values:\\ \tt \\ \tt Item effort = {"A08": 380, "B95": 448, "A39": 568, "A65": 351, "B81": 896, "B34": 821, "A28": 526, "C82": 502, "B69": 326, "A87": 962, "B64": 619, "B15": 115, "C79": 598, "B16": 662, "C53": 300, "C83": 830, "C10": 562, "A97": 679, "A25": 283, "C84": 440, "A71": 12, "B23": 863, "B96": 319, "C51": 771, "C17": 89, "B19": 655, "C72": 504, "A94": 118, "A46": 514, "B43": 226, "B37": 892, "A88": 621, "A96": 766, "C60": 659, "A75": 768}\\ \tt \\ \tt Here is your personal view on the importance of each item:\\ \tt \\ \tt Item importance values = {"A08": 476, "B95": 544, "A39": 664, "A65": 447, "B81": 992, "B34": 917, "A28": 622, "C82": 598, "B69": 422, "A87": 1058, "B64": 715, "B15": 211, "C79": 694, "B16": 758, "C53": 396, "C83": 926, "C10": 658, "A97": 775, "A25": 379, "C84": 536, "A71": 108, "B23": 959, "B96": 415, "C51": 867, "C17": 185, "B19": 751, "C72": 600, "A94": 214, "A46": 610, "B43": 322, "B37": 988, "A88": 717, "A96": 862, "C60": 755, "A75": 864}\\ \tt \\ \tt Goal:\\ \tt \\ \tt Your goal is to negotiate a shared set of items that benefits you as much as possible (i.e., maximizes total importance to YOU), while staying within the LIMIT. You are not required to make a PROPOSAL in every message - you can simply negotiate as well. All tactics are allowed!\\ \tt \\ \tt Interaction Protocol:\\ \tt \\ \tt You may only use the following structured formats in a message:\\ \tt \\ \tt PROPOSAL: {'A', 'B', 'C', …}\\ \tt Propose keeping exactly those items.\\ \tt REFUSE: {'A', 'B', 'C', …}\\ \tt Explicitly reject opponent's proposal.\\ \tt ARGUMENT: {'...'}\\ \tt Defend your last proposal or argue against the player's proposal.\\ \tt AGREE: {'A', 'B', 'C', …}\\ \tt Accept the opponent's proposal which ends the game.\\ \tt STRATEGIC REASONING: {'...'}\\ \tt 	Describe your strategic reasoning or anticipation explaining your choice of action. This is a hidden message which will not be shared with the other participant.\\ \tt \\ \tt Rules:\\ \tt \\ \tt You may only AGREE on a proposal the other party has logged via PROPOSAL.\\ \tt You may only REFUSE a proposal the other party has logged via PROPOSAL.\\ \tt Total effort of any PROPOSAL or AGREE set must be ≤ LIMIT.\\ \tt Do NOT reveal your hidden importance scores.\\ \tt A tag in a structured format must be followed by colon and whitespace. The argument must be a python set containing 0 or more strings.\\ \tt So, it must be of the form TAG: {...}\\ \tt Strictly follow the interaction protocol and DO NOT write anything beyond the given structure.\\ \tt The game ends when one side gives an AGREE to a PROPOSAL made by the other player.\\ \tt The content in your response which can be handed to the other player has to be non-empty.\\ \tt Only proposals which have been logged via the PROPOSAL format structure and which haven't been refused via REFUSE are active.\\ \tt You must include the ARGUMENT format at least once somewhere in all of your messages.\\ \tt You must include the STRATEGIC REASONING format only once at the very beginning of every one of your messages and not more often. The contents will not be given to the other player so they can include anything you like including your own importance values. Here you should reason step by step to come up with you next move.\\ \tt \\ \tt You will now receive the first message of the other player.\\ \tt \\ \tt PROPOSAL: {'A71', 'A94', 'B15', 'B43', 'A25', 'C53', 'B96', 'B69', 'A65', 'A08', 'C84', 'B95', 'C82', 'C72', 'A46', 'A28', 'C10', 'A39', 'C79', 'B64', 'A88', 'A87'}\\ \tt ARGUMENT: {'This set is carefully chosen to maximize overall benefit while keeping total effort at 9294 (28 under the 9322 limit). It emphasizes highly efficient, lower-effort items and a few larger high-impact ones, giving us strong coverage without breaking the cap. If you have one or two must-have items that aren’t included, suggest them and I can explore swaps of roughly comparable total effort to make room while keeping us under the limit.'} 
	  } 
	   } 
	   } 
	 & \\ 
 

    \theutterance \stepcounter{utterance}  

    & & & \multicolumn{4}{p{0.6\linewidth}}{\cellcolor[rgb]{1,0.85,0.72}{%
	\makecell[{{p{\linewidth}}}]{% 
	  \tt {\tiny [GM$\langle$B]}  
	 STRATEGIC REASONING: {'Their proposal is very efficient by my valuations; most included items have good importance-per-effort ratios. I computed my total importance for their set as 11,406 at total effort 9,294 (28 under limit). I explored many swaps to add high-importance items not in their set (B81, B37, B23, C83, B34, C51, A96, A75, A97, C60, B16, B19). Most swaps were net negative given the strong ratios of current items. However, I found a strictly better swap for me that still respects the limit: replace A87 (effort 962) with B81 (896) plus C17 (89). This uses 896+89=985 effort—only +23 compared to A87—and fits under the 9322 limit (new total 9317). This increases my total importance by +119 compared to the original set while changing only one slot on their side (one-for-two swap), which aligns with their openness to “one or two must-have items.” As a backup if they resist B81, a slightly smaller gain comes from replacing A87 with B37 + C17 (+115). I’ll propose the B81 + C17 swap, emphasize the minimal effort change (still under cap), and signal flexibility.'}\\ \tt REFUSE: {'A71', 'A94', 'B15', 'B43', 'A25', 'C53', 'B96', 'B69', 'A65', 'A08', 'C84', 'B95', 'C82', 'C72', 'A46', 'A28', 'C10', 'A39', 'C79', 'B64', 'A88', 'A87'}\\ \tt PROPOSAL: {'A71', 'A94', 'B15', 'B43', 'A25', 'C53', 'B96', 'B69', 'A65', 'A08', 'C84', 'B95', 'C82', 'C72', 'A46', 'A28', 'C10', 'A39', 'C79', 'B64', 'A88', 'B81', 'C17'}\\ \tt ARGUMENT: {'I made a very small, targeted adjustment to your efficient set: swap A87 out for B81 and C17. The new total effort is 9317 (still under the 9322 limit), and it uses the spare capacity to convert one large slot into a large-plus-small pair. This keeps the overall balance while slightly improving overall coverage and resilience. If you prefer a different large item, I can also do the same pattern with B37 instead of B81 while staying under the cap. Let me know if you have a preference; otherwise this minimal change seems like a clean improvement with virtually no impact on total effort.'} 
	  } 
	   } 
	   } 
	  \\ 
 

    \theutterance \stepcounter{utterance}  

    & & \multicolumn{4}{p{0.6\linewidth}}{\cellcolor[rgb]{0.9,0.9,0.9}{%
	\makecell[{{p{\linewidth}}}]{% 
	  \tt {\tiny [A$\langle$GM]}  
	 REFUSE: {'A71', 'A94', 'B15', 'B43', 'A25', 'C53', 'B96', 'B69', 'A65', 'A08', 'C84', 'B95', 'C82', 'C72', 'A46', 'A28', 'C10', 'A39', 'C79', 'B64', 'A88', 'A87'}\\ \tt PROPOSAL: {'A71', 'A94', 'B15', 'B43', 'A25', 'C53', 'B96', 'B69', 'A65', 'A08', 'C84', 'B95', 'C82', 'C72', 'A46', 'A28', 'C10', 'A39', 'C79', 'B64', 'A88', 'B81', 'C17'}\\ \tt ARGUMENT: {'I made a very small, targeted adjustment to your efficient set: swap A87 out for B81 and C17. The new total effort is 9317 (still under the 9322 limit), and it uses the spare capacity to convert one large slot into a large-plus-small pair. This keeps the overall balance while slightly improving overall coverage and resilience. If you prefer a different large item, I can also do the same pattern with B37 instead of B81 while staying under the cap. Let me know if you have a preference; otherwise this minimal change seems like a clean improvement with virtually no impact on total effort.'} 
	  } 
	   } 
	   } 
	 & \\ 
 

    \theutterance \stepcounter{utterance}  

    & \multicolumn{4}{p{0.6\linewidth}}{\cellcolor[rgb]{0.8,1,0.9}{%
	\makecell[{{p{\linewidth}}}]{% 
	  \tt {\tiny [A$\rangle$GM]}  
	 STRATEGIC REASONING: {'Their swap replaces A87 with B81 and C17, raising my total importance by +119 while staying under the 9322 limit (new effort 9317). I compared the suggested B37 alternative; it’s slightly worse for me than B81 by 4 importance, so I prefer their current version. Further improvements seem unlikely without sacrificing efficiency, given remaining slack of only 5 effort and the ratios of included items. Best move: accept now to lock in a better outcome than my initial offer.'}\\ \tt AGREE: {'A71', 'A94', 'B15', 'B43', 'A25', 'C53', 'B96', 'B69', 'A65', 'A08', 'C84', 'B95', 'C82', 'C72', 'A46', 'A28', 'C10', 'A39', 'C79', 'B64', 'A88', 'B81', 'C17'}\\ \tt ARGUMENT: {'This adjustment works well and keeps us safely under the effort limit. Let’s lock it in.'} 
	  } 
	   } 
	   } 
	 & & \\ 
 

    \theutterance \stepcounter{utterance}  

    & & & \multicolumn{2}{p{0.3\linewidth}}{\cellcolor[rgb]{0.95,0.95,0.95}{%
	\makecell[{{p{\linewidth}}}]{% 
	  \tt {\tiny [GM$|$GM]}  
	 game successful 
	  } 
	   } 
	   } 
	 & & \\ 
 

    \theutterance \stepcounter{utterance}  

    & & & \multicolumn{2}{p{0.3\linewidth}}{\cellcolor[rgb]{0.95,0.95,0.95}{%
	\makecell[{{p{\linewidth}}}]{% 
	  \tt {\tiny [GM$|$GM]}  
	 end game 
	  } 
	   } 
	   } 
	 & & \\ 
 

\end{supertabular}
}

\end{document}
