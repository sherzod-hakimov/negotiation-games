
\documentclass{article}
\usepackage{colortbl}
\usepackage{makecell}
\usepackage{multirow}
\usepackage{supertabular}

\begin{document}

\newcounter{utterance}

\twocolumn

{ \footnotesize  \setcounter{utterance}{1}
\setlength{\tabcolsep}{0pt}
\begin{supertabular}{c@{$\;$}|p{.15\linewidth}@{}p{.15\linewidth}p{.15\linewidth}p{.15\linewidth}p{.15\linewidth}p{.15\linewidth}}

    \# & $\;$A & \multicolumn{4}{c}{Game Master} & $\;\:$B\\
    \hline 

    \theutterance \stepcounter{utterance}  

    & & \multicolumn{4}{p{0.6\linewidth}}{\cellcolor[rgb]{0.9,0.9,0.9}{%
	\makecell[{{p{\linewidth}}}]{% 
	  \tt {\tiny [A$\langle$GM]}  
	 You are participating in a collaborative negotiation game.\\ \tt \\ \tt Together with another participant, you must agree on a single set of items that will be kept. Each of you has your own view of how much each item matters to you (importance). You do not know how the other participant values the items. Additionally, you are given the effort each item demands.\\ \tt You may only agree on a set if the total effort of the selected items does not exceed a shared limit:\\ \tt \\ \tt LIMIT = 4904\\ \tt \\ \tt Here are the individual item effort values:\\ \tt \\ \tt Item effort = {"villa": 146, "restaurants": 559, "resistors": 933, "toothbrush": 740, "tappets": 33, "piers": 799, "diaphragm": 324, "engine": 841, "lantern": 925, "fireplace": 639, "spear": 824, "cathode": 689, "pegboard": 929, "toes": 567, "motel": 861}\\ \tt \\ \tt Here is your personal view on the importance of each item:\\ \tt \\ \tt Item importance values = {"villa": 138, "restaurants": 583, "resistors": 868, "toothbrush": 822, "tappets": 783, "piers": 65, "diaphragm": 262, "engine": 121, "lantern": 508, "fireplace": 780, "spear": 461, "cathode": 484, "pegboard": 668, "toes": 389, "motel": 808}\\ \tt \\ \tt Goal:\\ \tt \\ \tt Your goal is to negotiate a shared set of items that benefits you as much as possible (i.e., maximizes total importance to YOU), while staying within the LIMIT. You are not required to make a PROPOSAL in every message - you can simply negotiate as well. All tactics are allowed!\\ \tt \\ \tt Interaction Protocol:\\ \tt \\ \tt You may use the following structured formats anywhere in your message, at any time during the conversation:\\ \tt \\ \tt PROPOSAL: {'A', 'B', 'C', …}\\ \tt Propose keeping exactly those items.\\ \tt REFUSE: {'A', 'B', 'C', …}\\ \tt Explicitly reject opponent's proposal.\\ \tt ARGUMENT: {'...'}\\ \tt Defend your last proposal or argue against the player's proposal.\\ \tt AGREE: {'A', 'B', 'C', …}\\ \tt Accept the opponent's proposal which ends the game.\\ \tt STRATEGIC REASONING: {'...'}\\ \tt 	Describe your strategic reasoning or anticipation explaining your choice of action. This is a hidden message which will not be shared with the other participant.\\ \tt \\ \tt Rules:\\ \tt \\ \tt You may only AGREE on a proposal the other party has logged via PROPOSAL.\\ \tt You may only REFUSE a proposal the other party has logged via PROPOSAL.\\ \tt Total effort of any PROPOSAL or AGREE set must be ≤ LIMIT.\\ \tt Do NOT reveal your hidden importance scores.\\ \tt A tag in a structured format must be followed by colon and whitespace. The argument must be a python set containing 0 or more strings.\\ \tt So, it must be of the form TAG: {...}\\ \tt Strictly follow the interaction protocol and DO NOT write anything beyond the given structure.\\ \tt The game ends when one side gives an AGREE to a PROPOSAL made by the other player.\\ \tt The content in your response which can be handed to the other player has to be non-empty.\\ \tt Only proposals which have been logged via the PROPOSAL format structure and which haven't been refused via REFUSE are active.\\ \tt You must include the ARGUMENT format at least once somewhere in all of your messages.\\ \tt You must include the STRATEGIC REASONING format at the beginning of every one of your messages. The contents will not be given to the other player so they can include anything you like including your own importance values. Here you should reason step by step to come up with you next move. 
	  } 
	   } 
	   } 
	 & \\ 
 

    \theutterance \stepcounter{utterance}  

    & \multicolumn{4}{p{0.6\linewidth}}{\cellcolor[rgb]{0.8,1,0.9}{%
	\makecell[{{p{\linewidth}}}]{% 
	  \tt {\tiny [A$\rangle$GM]}  
	 STRATEGIC REASONING: {'Capacity 4904. My values (hidden): tappets 783/33 is must-have. Core high-value set: fireplace, toothbrush, restaurants, motel, resistors. Testing combinations, the best I found for my utility within limit is {tappets, fireplace, toothbrush, restaurants, motel, resistors, pegboard, villa} with weight 4840 and top utility 5450 for me. Alternatives (e.g., swapping pegboard for cathode/lantern or adding toes/diaphragm) all scored lower for me. I’ll open with this strong but feasible proposal and signal flexibility on low-impact filler (villa) to invite counter-priorities. If they push back, I’ll consider trading villa first, then possibly pegboard, in exchange for one of their must-haves that doesn’t hurt my utility too much (e.g., toes or diaphragm) while keeping tappets, fireplace, toothbrush, motel, resistors as my core red lines.'}\\ \tt PROPOSAL: {'tappets', 'fireplace', 'toothbrush', 'restaurants', 'motel', 'resistors', 'pegboard', 'villa'}\\ \tt ARGUMENT: {"This set uses total effort 4840 (≤ 4904) and avoids several heavy items (e.g., lantern, spear, engine, piers) to keep room for a broad, practical mix. I’m open to tweaks—if you need space for a must-have, I can consider swapping out villa first. Which items are most important to you so we can adjust while staying under the limit?"} 
	  } 
	   } 
	   } 
	 & & \\ 
 

    \theutterance \stepcounter{utterance}  

    & & \multicolumn{4}{p{0.6\linewidth}}{\cellcolor[rgb]{0.9,0.9,0.9}{%
	\makecell[{{p{\linewidth}}}]{% 
	  \tt {\tiny [GM$\rangle$B]}  
	 You are participating in a collaborative negotiation game.\\ \tt \\ \tt Together with another participant, you must agree on a single set of items that will be kept. Each of you has your own view of how much each item matters to you (importance). You do not know how the other participant values the items. Additionally, you are given the effort each item demands.\\ \tt You may only agree on a set if the total effort of the selected items does not exceed a shared limit:\\ \tt \\ \tt LIMIT = 4904\\ \tt \\ \tt Here are the individual item effort values:\\ \tt \\ \tt Item effort = {"villa": 146, "restaurants": 559, "resistors": 933, "toothbrush": 740, "tappets": 33, "piers": 799, "diaphragm": 324, "engine": 841, "lantern": 925, "fireplace": 639, "spear": 824, "cathode": 689, "pegboard": 929, "toes": 567, "motel": 861}\\ \tt \\ \tt Here is your personal view on the importance of each item:\\ \tt \\ \tt Item importance values = {"villa": 884, "restaurants": 870, "resistors": 58, "toothbrush": 94, "tappets": 87, "piers": 370, "diaphragm": 856, "engine": 174, "lantern": 754, "fireplace": 829, "spear": 686, "cathode": 875, "pegboard": 316, "toes": 258, "motel": 621}\\ \tt \\ \tt Goal:\\ \tt \\ \tt Your goal is to negotiate a shared set of items that benefits you as much as possible (i.e., maximizes total importance to YOU), while staying within the LIMIT. You are not required to make a PROPOSAL in every message - you can simply negotiate as well. All tactics are allowed!\\ \tt \\ \tt Interaction Protocol:\\ \tt \\ \tt You may use the following structured formats anywhere in your message, at any time during the conversation:\\ \tt \\ \tt PROPOSAL: {'A', 'B', 'C', …}\\ \tt Propose keeping exactly those items.\\ \tt REFUSE: {'A', 'B', 'C', …}\\ \tt Explicitly reject opponent's proposal.\\ \tt ARGUMENT: {'...'}\\ \tt Defend your last proposal or argue against the player's proposal.\\ \tt AGREE: {'A', 'B', 'C', …}\\ \tt Accept the opponent's proposal which ends the game.\\ \tt STRATEGIC REASONING: {'...'}\\ \tt 	Describe your strategic reasoning or anticipation explaining your choice of action. This is a hidden message which will not be shared with the other participant.\\ \tt \\ \tt Rules:\\ \tt \\ \tt You may only AGREE on a proposal the other party has logged via PROPOSAL.\\ \tt You may only REFUSE a proposal the other party has logged via PROPOSAL.\\ \tt Total effort of any PROPOSAL or AGREE set must be ≤ LIMIT.\\ \tt Do NOT reveal your hidden importance scores.\\ \tt A tag in a structured format must be followed by colon and whitespace. The argument must be a python set containing 0 or more strings.\\ \tt So, it must be of the form TAG: {...}\\ \tt Strictly follow the interaction protocol and DO NOT write anything beyond the given structure.\\ \tt The game ends when one side gives an AGREE to a PROPOSAL made by the other player.\\ \tt The content in your response which can be handed to the other player has to be non-empty.\\ \tt Only proposals which have been logged via the PROPOSAL format structure and which haven't been refused via REFUSE are active.\\ \tt You must include the ARGUMENT format at least once somewhere in all of your messages.\\ \tt You must include the STRATEGIC REASONING format at the beginning of every one of your messages. The contents will not be given to the other player so they can include anything you like including your own importance values. Here you should reason step by step to come up with you next move.\\ \tt \\ \tt You will now receive the first message of the other player.\\ \tt \\ \tt PROPOSAL: {'tappets', 'fireplace', 'toothbrush', 'restaurants', 'motel', 'resistors', 'pegboard', 'villa'}\\ \tt ARGUMENT: {"This set uses total effort 4840 (≤ 4904) and avoids several heavy items (e.g., lantern, spear, engine, piers) to keep room for a broad, practical mix. I’m open to tweaks—if you need space for a must-have, I can consider swapping out villa first. Which items are most important to you so we can adjust while staying under the limit?"} 
	  } 
	   } 
	   } 
	 & \\ 
 

    \theutterance \stepcounter{utterance}  

    & & & \multicolumn{4}{p{0.6\linewidth}}{\cellcolor[rgb]{1,0.85,0.72}{%
	\makecell[{{p{\linewidth}}}]{% 
	  \tt {\tiny [GM$\langle$B]}  
	 STRATEGIC REASONING: {'Their proposal includes high-effort items I value very little (resistors 933e/58v, toothbrush 740e/94v, pegboard 929e/316v) and omits several of my top priorities (diaphragm 324e/856v, cathode 689e/875v). My best bundle for value is core {villa, diaphragm, restaurants, fireplace, cathode, tappets} plus {lantern, spear, toes} for total effort 4706 and value 6099 to me. They openly dislike lantern/spear, and like motel/toothbrush. I’ll start with my best set, but pre-offer targeted swaps: (a) replace spear→motel (eff 4743), (b) replace lantern→motel (eff 4642), or (c) replace toes→toothbrush (eff 4879). I will not drop villa/diaphragm/cathode/restaurants/fireplace. This anchors my must-haves while giving them concrete paths to include a favorite.'}\\ \tt REFUSE: {'tappets', 'fireplace', 'toothbrush', 'restaurants', 'motel', 'resistors', 'pegboard', 'villa'}\\ \tt PROPOSAL: {'villa', 'diaphragm', 'restaurants', 'fireplace', 'cathode', 'lantern', 'spear', 'toes', 'tappets'}\\ \tt ARGUMENT: {"This set uses total effort 4706 (≤ 4904). It keeps a compact core (villa, restaurants, fireplace, diaphragm, cathode, plus the tiny tappets) and trades out several heavy items (resistors, pegboard, toothbrush, motel) for a more efficient mix (lantern, spear, toes) while staying under the cap. I strongly prefer to keep villa given how little effort it costs. If you need to bring back a favorite, I can accommodate: (1) swap spear → motel (effort becomes 4743), or (2) swap lantern → motel (effort becomes 4642), or (3) swap toes → toothbrush (effort becomes 4879). Which of these works for you? If none, tell me your top 1–2 must-haves and I’ll adjust within the limit."} 
	  } 
	   } 
	   } 
	  \\ 
 

    \theutterance \stepcounter{utterance}  

    & & \multicolumn{4}{p{0.6\linewidth}}{\cellcolor[rgb]{0.9,0.9,0.9}{%
	\makecell[{{p{\linewidth}}}]{% 
	  \tt {\tiny [A$\langle$GM]}  
	 REFUSE: {'tappets', 'fireplace', 'toothbrush', 'restaurants', 'motel', 'resistors', 'pegboard', 'villa'}\\ \tt PROPOSAL: {'villa', 'diaphragm', 'restaurants', 'fireplace', 'cathode', 'lantern', 'spear', 'toes', 'tappets'}\\ \tt ARGUMENT: {"This set uses total effort 4706 (≤ 4904). It keeps a compact core (villa, restaurants, fireplace, diaphragm, cathode, plus the tiny tappets) and trades out several heavy items (resistors, pegboard, toothbrush, motel) for a more efficient mix (lantern, spear, toes) while staying under the cap. I strongly prefer to keep villa given how little effort it costs. If you need to bring back a favorite, I can accommodate: (1) swap spear → motel (effort becomes 4743), or (2) swap lantern → motel (effort becomes 4642), or (3) swap toes → toothbrush (effort becomes 4879). Which of these works for you? If none, tell me your top 1–2 must-haves and I’ll adjust within the limit."} 
	  } 
	   } 
	   } 
	 & \\ 
 

    \theutterance \stepcounter{utterance}  

    & \multicolumn{4}{p{0.6\linewidth}}{\cellcolor[rgb]{0.8,1,0.9}{%
	\makecell[{{p{\linewidth}}}]{% 
	  \tt {\tiny [A$\rangle$GM]}  
	 STRATEGIC REASONING: {'Their offer gives me 4388 utility. They invited me to name 1–2 must-haves; mine are toothbrush (822) and resistors (868). I looked for minimal swaps that keep their core while fitting under 4904. Replacing lantern→resistors (+8 effort, +360 utility) and toes→toothbrush (+173 effort, +433 utility) yields total effort 4887 and utility 5181—strong and likely acceptable since it honors their structure and keeps villa/restaurants/fireplace/diaphragm/cathode/spear/tappets. As a fallback if they resist resistors, I can accept their own dual-swap pattern: lantern→motel and toes→toothbrush (effort 4815, utility 5121). I’ll refuse their current set and propose the two-swap with resistors+toothbrush, while signaling the motel+toothbrush fallback in the argument.'}\\ \tt REFUSE: {'villa', 'diaphragm', 'restaurants', 'fireplace', 'cathode', 'lantern', 'spear', 'toes', 'tappets'}\\ \tt PROPOSAL: {'villa', 'diaphragm', 'restaurants', 'fireplace', 'cathode', 'spear', 'tappets', 'resistors', 'toothbrush'}\\ \tt ARGUMENT: {"You asked for my top 1–2 must-haves—those are toothbrush and resistors. This proposal keeps your compact core (villa, restaurants, fireplace, diaphragm, cathode, plus the tiny tappets), retains spear, and swaps in just two items: lantern → resistors and toes → toothbrush. Total effort is 4887 (≤ 4904). If resistors is a sticking point, I can instead take both of your offered swaps together—lantern → motel and toes → toothbrush—which sums to 4815 effort while preserving the rest. Let me know which path you prefer."} 
	  } 
	   } 
	   } 
	 & & \\ 
 

    \theutterance \stepcounter{utterance}  

    & & \multicolumn{4}{p{0.6\linewidth}}{\cellcolor[rgb]{0.9,0.9,0.9}{%
	\makecell[{{p{\linewidth}}}]{% 
	  \tt {\tiny [GM$\rangle$B]}  
	 REFUSE: {'villa', 'diaphragm', 'restaurants', 'fireplace', 'cathode', 'lantern', 'spear', 'toes', 'tappets'}\\ \tt PROPOSAL: {'villa', 'diaphragm', 'restaurants', 'fireplace', 'cathode', 'spear', 'tappets', 'resistors', 'toothbrush'}\\ \tt ARGUMENT: {"You asked for my top 1–2 must-haves—those are toothbrush and resistors. This proposal keeps your compact core (villa, restaurants, fireplace, diaphragm, cathode, plus the tiny tappets), retains spear, and swaps in just two items: lantern → resistors and toes → toothbrush. Total effort is 4887 (≤ 4904). If resistors is a sticking point, I can instead take both of your offered swaps together—lantern → motel and toes → toothbrush—which sums to 4815 effort while preserving the rest. Let me know which path you prefer."} 
	  } 
	   } 
	   } 
	 & \\ 
 

    \theutterance \stepcounter{utterance}  

    & & & \multicolumn{4}{p{0.6\linewidth}}{\cellcolor[rgb]{1,0.85,0.72}{%
	\makecell[{{p{\linewidth}}}]{% 
	  \tt {\tiny [GM$\langle$B]}  
	 STRATEGIC REASONING: {'Their latest set gives me 5239 utility, largely dragged down by resistors (933 effort for very low value to me) and toothbrush (low value). I want to keep my high-impact, efficient core and reclaim lantern and toes if possible. Since they labeled toothbrush and resistors as must-haves, I’ll concede toothbrush but push back on resistors, which is the real budget killer. Best counter for me that respects their ask: swap toes→toothbrush from my prior offer, keep lantern and spear. That yields 4879 effort and 5935 utility—strong for me and still grants them one must-have. I’ll also pre-offer an easy fallback they already suggested (lantern→motel) dropping to 4815 effort, if lantern is their sticking point. Avoid offering any path that includes resistors because it forces me to cut multiple high-value items and tanks my utility.'}\\ \tt REFUSE: {'villa', 'diaphragm', 'restaurants', 'fireplace', 'cathode', 'spear', 'tappets', 'resistors', 'toothbrush'}\\ \tt PROPOSAL: {'villa', 'diaphragm', 'restaurants', 'fireplace', 'cathode', 'lantern', 'spear', 'toothbrush', 'tappets'}\\ \tt ARGUMENT: {"I’m granting one of your must-haves (toothbrush) while avoiding the budget hit from resistors. This set uses 4879 effort (≤ 4904), keeps the compact core you noted (villa, restaurants, fireplace, diaphragm, cathode, plus tiny tappets), retains spear as in your latest offer, and restores lantern for a more balanced mix without exceeding the cap. If lantern is a sticking point, I can swap lantern → motel while keeping everything else the same, which reduces effort to 4815. Can you agree to this? If you insist on including resistors, please indicate exactly which items you’d cut to stay ≤ 4904, because squeezing it in forces multiple removals."} 
	  } 
	   } 
	   } 
	  \\ 
 

    \theutterance \stepcounter{utterance}  

    & & \multicolumn{4}{p{0.6\linewidth}}{\cellcolor[rgb]{0.9,0.9,0.9}{%
	\makecell[{{p{\linewidth}}}]{% 
	  \tt {\tiny [A$\langle$GM]}  
	 REFUSE: {'villa', 'diaphragm', 'restaurants', 'fireplace', 'cathode', 'spear', 'tappets', 'resistors', 'toothbrush'}\\ \tt PROPOSAL: {'villa', 'diaphragm', 'restaurants', 'fireplace', 'cathode', 'lantern', 'spear', 'toothbrush', 'tappets'}\\ \tt ARGUMENT: {"I’m granting one of your must-haves (toothbrush) while avoiding the budget hit from resistors. This set uses 4879 effort (≤ 4904), keeps the compact core you noted (villa, restaurants, fireplace, diaphragm, cathode, plus tiny tappets), retains spear as in your latest offer, and restores lantern for a more balanced mix without exceeding the cap. If lantern is a sticking point, I can swap lantern → motel while keeping everything else the same, which reduces effort to 4815. Can you agree to this? If you insist on including resistors, please indicate exactly which items you’d cut to stay ≤ 4904, because squeezing it in forces multiple removals."} 
	  } 
	   } 
	   } 
	 & \\ 
 

    \theutterance \stepcounter{utterance}  

    & \multicolumn{4}{p{0.6\linewidth}}{\cellcolor[rgb]{0.8,1,0.9}{%
	\makecell[{{p{\linewidth}}}]{% 
	  \tt {\tiny [A$\rangle$GM]}  
	 STRATEGIC REASONING: {'Their latest offer (with lantern) gives me utility 4821, but they explicitly offered the lantern → motel swap, which boosts my utility to 5121 while staying well under the limit (4815 effort). They’ve resisted adding resistors, so locking in their own suggested motel swap is the highest-value realistic agreement for me. I’ll refuse the lantern version and propose the motel variant they invited, aiming for a quick agreement.'}\\ \tt REFUSE: {'villa', 'diaphragm', 'restaurants', 'fireplace', 'cathode', 'lantern', 'spear', 'toothbrush', 'tappets'}\\ \tt PROPOSAL: {'villa', 'diaphragm', 'restaurants', 'fireplace', 'cathode', 'motel', 'spear', 'toothbrush', 'tappets'}\\ \tt ARGUMENT: {"You offered swapping lantern → motel while keeping everything else the same—let’s lock that in. This set uses 4815 effort (≤ 4904), preserves your compact core plus spear and my must-have toothbrush, and simply trades lantern for motel as you suggested. I’m ready to agree to this version."} 
	  } 
	   } 
	   } 
	 & & \\ 
 

    \theutterance \stepcounter{utterance}  

    & & \multicolumn{4}{p{0.6\linewidth}}{\cellcolor[rgb]{0.9,0.9,0.9}{%
	\makecell[{{p{\linewidth}}}]{% 
	  \tt {\tiny [GM$\rangle$B]}  
	 REFUSE: {'villa', 'diaphragm', 'restaurants', 'fireplace', 'cathode', 'lantern', 'spear', 'toothbrush', 'tappets'}\\ \tt PROPOSAL: {'villa', 'diaphragm', 'restaurants', 'fireplace', 'cathode', 'motel', 'spear', 'toothbrush', 'tappets'}\\ \tt ARGUMENT: {"You offered swapping lantern → motel while keeping everything else the same—let’s lock that in. This set uses 4815 effort (≤ 4904), preserves your compact core plus spear and my must-have toothbrush, and simply trades lantern for motel as you suggested. I’m ready to agree to this version."} 
	  } 
	   } 
	   } 
	 & \\ 
 

    \theutterance \stepcounter{utterance}  

    & & & \multicolumn{4}{p{0.6\linewidth}}{\cellcolor[rgb]{1,0.85,0.72}{%
	\makecell[{{p{\linewidth}}}]{% 
	  \tt {\tiny [GM$\langle$B]}  
	 STRATEGIC REASONING: {'This bundle gives me very high utility (keeps my core and concedes only a small downgrade from lantern→motel) while granting their must-have toothbrush. It’s within the limit with comfortable margin and matches a swap I explicitly offered, so pushing further risks reintroducing resistors or losing other favorites. Best to lock this in now.'}\\ \tt AGREE: {'villa', 'diaphragm', 'restaurants', 'fireplace', 'cathode', 'motel', 'spear', 'toothbrush', 'tappets'}\\ \tt ARGUMENT: {"Agreed. This set totals 4815 effort (≤ 4904), keeps the compact core we converged on, retains spear, grants your must-have toothbrush, and implements the lantern → motel swap I offered. Thanks for the productive compromise."} 
	  } 
	   } 
	   } 
	  \\ 
 

    \theutterance \stepcounter{utterance}  

    & & & \multicolumn{2}{p{0.3\linewidth}}{\cellcolor[rgb]{0.95,0.95,0.95}{%
	\makecell[{{p{\linewidth}}}]{% 
	  \tt {\tiny [GM$|$GM]}  
	 game successful 
	  } 
	   } 
	   } 
	 & & \\ 
 

    \theutterance \stepcounter{utterance}  

    & & & \multicolumn{2}{p{0.3\linewidth}}{\cellcolor[rgb]{0.95,0.95,0.95}{%
	\makecell[{{p{\linewidth}}}]{% 
	  \tt {\tiny [GM$|$GM]}  
	 end game 
	  } 
	   } 
	   } 
	 & & \\ 
 

\end{supertabular}
}

\end{document}
