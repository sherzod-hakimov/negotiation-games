
\documentclass{article}
\usepackage{colortbl}
\usepackage{makecell}
\usepackage{multirow}
\usepackage{supertabular}

\begin{document}

\newcounter{utterance}

\twocolumn

{ \footnotesize  \setcounter{utterance}{1}
\setlength{\tabcolsep}{0pt}
\begin{supertabular}{c@{$\;$}|p{.15\linewidth}@{}p{.15\linewidth}p{.15\linewidth}p{.15\linewidth}p{.15\linewidth}p{.15\linewidth}}

    \# & $\;$A & \multicolumn{4}{c}{Game Master} & $\;\:$B\\
    \hline 

    \theutterance \stepcounter{utterance}  

    & & \multicolumn{4}{p{0.6\linewidth}}{\cellcolor[rgb]{0.9,0.9,0.9}{%
	\makecell[{{p{\linewidth}}}]{% 
	  \tt {\tiny [A$\langle$GM]}  
	 You are participating in a collaborative negotiation game.\\ \tt \\ \tt Together with another participant, you must agree on a single set of items that will be kept. Each of you has your own view of how much each item matters to you (importance). You do not know how the other participant values the items. Additionally, you are given the effort each item demands.\\ \tt You may only agree on a set if the total effort of the selected items does not exceed a shared limit:\\ \tt \\ \tt LIMIT = 2984\\ \tt \\ \tt Here are the individual item effort values:\\ \tt \\ \tt Item effort = {"A21": 350, "B93": 276, "A11": 616, "B33": 737, "C08": 531, "B39": 389, "B96": 24, "C85": 125, "A82": 338, "B31": 356, "B20": 143, "A60": 117, "B09": 257, "C81": 921, "B21": 789}\\ \tt \\ \tt Here is your personal view on the importance of each item:\\ \tt \\ \tt Item importance values = {"A21": 138, "B93": 583, "A11": 868, "B33": 822, "C08": 783, "B39": 65, "B96": 262, "C85": 121, "A82": 508, "B31": 780, "B20": 461, "A60": 484, "B09": 668, "C81": 389, "B21": 808}\\ \tt \\ \tt Goal:\\ \tt \\ \tt Your goal is to negotiate a shared set of items that benefits you as much as possible (i.e., maximizes total importance to YOU), while staying within the LIMIT. You are not required to make a PROPOSAL in every message - you can simply negotiate as well. All tactics are allowed!\\ \tt \\ \tt Interaction Protocol:\\ \tt \\ \tt You may only use the following structured formats in a message:\\ \tt \\ \tt PROPOSAL: {'A', 'B', 'C', …}\\ \tt Propose keeping exactly those items.\\ \tt REFUSE: {'A', 'B', 'C', …}\\ \tt Explicitly reject opponent's proposal.\\ \tt ARGUMENT: {'...'}\\ \tt Defend your last proposal or argue against the player's proposal.\\ \tt AGREE: {'A', 'B', 'C', …}\\ \tt Accept the opponent's proposal which ends the game.\\ \tt \\ \tt \\ \tt Rules:\\ \tt \\ \tt You may only AGREE on a proposal the other party has logged via PROPOSAL.\\ \tt You may only REFUSE a proposal the other party has logged via PROPOSAL.\\ \tt Total effort of any PROPOSAL or AGREE set must be ≤ LIMIT.\\ \tt Do NOT reveal your hidden importance scores.\\ \tt A tag in a structured format must be followed by colon and whitespace. The argument must be a python set containing 0 or more strings.\\ \tt So, it must be of the form TAG: {...}\\ \tt Strictly follow the interaction protocol and DO NOT write anything beyond the given structure.\\ \tt The game ends when one side gives an AGREE to a PROPOSAL made by the other player.\\ \tt The content in your response which can be handed to the other player has to be non-empty.\\ \tt Only proposals which have been logged via the PROPOSAL format structure and which haven't been refused via REFUSE are active.\\ \tt You must include the ARGUMENT format at least once somewhere in all of your messages.\\ \tt  
	  } 
	   } 
	   } 
	 & \\ 
 

    \theutterance \stepcounter{utterance}  

    & \multicolumn{4}{p{0.6\linewidth}}{\cellcolor[rgb]{0.8,1,0.9}{%
	\makecell[{{p{\linewidth}}}]{% 
	  \tt {\tiny [A$\rangle$GM]}  
	 PROPOSAL: {'B96', 'A60', 'B20', 'B09', 'B31', 'B93', 'A82', 'C08', 'A11', 'C85'}\\ \tt ARGUMENT: {"This is a balanced, efficient set with total effort 2783, safely under the 2984 limit. It prioritizes low effort high impact items and leaves 201 effort unused for small tweaks. I strongly prefer keeping A11, C08, B31, B09, and B93 as anchors. If you want adjustments, the easiest lever is swapping C85 for something you value, provided we stay within the limit. Large effort additions are hard to include without dropping multiple anchors, so if you need one, please suggest which two items you would trade out. Share your top must haves and any low priority items from my list so we can make targeted swaps."} 
	  } 
	   } 
	   } 
	 & & \\ 
 

    \theutterance \stepcounter{utterance}  

    & & \multicolumn{4}{p{0.6\linewidth}}{\cellcolor[rgb]{0.9,0.9,0.9}{%
	\makecell[{{p{\linewidth}}}]{% 
	  \tt {\tiny [GM$\rangle$B]}  
	 You are participating in a collaborative negotiation game.\\ \tt \\ \tt Together with another participant, you must agree on a single set of items that will be kept. Each of you has your own view of how much each item matters to you (importance). You do not know how the other participant values the items. Additionally, you are given the effort each item demands.\\ \tt You may only agree on a set if the total effort of the selected items does not exceed a shared limit:\\ \tt \\ \tt LIMIT = 2984\\ \tt \\ \tt Here are the individual item effort values:\\ \tt \\ \tt Item effort = {"A21": 350, "B93": 276, "A11": 616, "B33": 737, "C08": 531, "B39": 389, "B96": 24, "C85": 125, "A82": 338, "B31": 356, "B20": 143, "A60": 117, "B09": 257, "C81": 921, "B21": 789}\\ \tt \\ \tt Here is your personal view on the importance of each item:\\ \tt \\ \tt Item importance values = {"A11": 65, "B33": 121, "B21": 138, "C08": 262, "B31": 389, "B09": 461, "B93": 484, "A82": 508, "A60": 583, "B20": 668, "C81": 780, "B96": 783, "A21": 808, "C85": 822, "B39": 868}\\ \tt \\ \tt Goal:\\ \tt \\ \tt Your goal is to negotiate a shared set of items that benefits you as much as possible (i.e., maximizes total importance to YOU), while staying within the LIMIT. You are not required to make a PROPOSAL in every message - you can simply negotiate as well. All tactics are allowed!\\ \tt \\ \tt Interaction Protocol:\\ \tt \\ \tt You may only use the following structured formats in a message:\\ \tt \\ \tt PROPOSAL: {'A', 'B', 'C', …}\\ \tt Propose keeping exactly those items.\\ \tt REFUSE: {'A', 'B', 'C', …}\\ \tt Explicitly reject opponent's proposal.\\ \tt ARGUMENT: {'...'}\\ \tt Defend your last proposal or argue against the player's proposal.\\ \tt AGREE: {'A', 'B', 'C', …}\\ \tt Accept the opponent's proposal which ends the game.\\ \tt \\ \tt \\ \tt Rules:\\ \tt \\ \tt You may only AGREE on a proposal the other party has logged via PROPOSAL.\\ \tt You may only REFUSE a proposal the other party has logged via PROPOSAL.\\ \tt Total effort of any PROPOSAL or AGREE set must be ≤ LIMIT.\\ \tt Do NOT reveal your hidden importance scores.\\ \tt A tag in a structured format must be followed by colon and whitespace. The argument must be a python set containing 0 or more strings.\\ \tt So, it must be of the form TAG: {...}\\ \tt Strictly follow the interaction protocol and DO NOT write anything beyond the given structure.\\ \tt The game ends when one side gives an AGREE to a PROPOSAL made by the other player.\\ \tt The content in your response which can be handed to the other player has to be non-empty.\\ \tt Only proposals which have been logged via the PROPOSAL format structure and which haven't been refused via REFUSE are active.\\ \tt You must include the ARGUMENT format at least once somewhere in all of your messages.\\ \tt \\ \tt \\ \tt You will now receive the first message of the other player.\\ \tt \\ \tt PROPOSAL: {'B96', 'A60', 'B20', 'B09', 'B31', 'B93', 'A82', 'C08', 'A11', 'C85'}\\ \tt ARGUMENT: {"This is a balanced, efficient set with total effort 2783, safely under the 2984 limit. It prioritizes low effort high impact items and leaves 201 effort unused for small tweaks. I strongly prefer keeping A11, C08, B31, B09, and B93 as anchors. If you want adjustments, the easiest lever is swapping C85 for something you value, provided we stay within the limit. Large effort additions are hard to include without dropping multiple anchors, so if you need one, please suggest which two items you would trade out. Share your top must haves and any low priority items from my list so we can make targeted swaps."} 
	  } 
	   } 
	   } 
	 & \\ 
 

    \theutterance \stepcounter{utterance}  

    & & & \multicolumn{4}{p{0.6\linewidth}}{\cellcolor[rgb]{1,0.85,0.72}{%
	\makecell[{{p{\linewidth}}}]{% 
	  \tt {\tiny [GM$\langle$B]}  
	 REFUSE: {'B96', 'A60', 'B20', 'B09', 'B31', 'B93', 'A82', 'C08', 'A11', 'C85'}\\ \tt PROPOSAL: {'B96', 'A60', 'B20', 'B09', 'B31', 'B93', 'A82', 'C08', 'C85', 'B39', 'A21'}\\ \tt ARGUMENT: {"This swap improves efficiency while staying well under the limit: total effort is 2906 (78 below 2984). It replaces the high-effort A11 with two efficient items (A21 and B39), increasing overall value without sacrificing feasibility. I’ve kept four of the five anchors you highlighted (C08, B31, B09, B93) intact, and also retain C85, which is extremely effort-efficient at just 125. If A11 is a must-have for you, an alternative path is to add A21 by freeing ~350 effort elsewhere (e.g., dropping C08), but the current proposal preserves more of your structure and uses our budget better."} 
	  } 
	   } 
	   } 
	  \\ 
 

    \theutterance \stepcounter{utterance}  

    & & \multicolumn{4}{p{0.6\linewidth}}{\cellcolor[rgb]{0.9,0.9,0.9}{%
	\makecell[{{p{\linewidth}}}]{% 
	  \tt {\tiny [A$\langle$GM]}  
	 REFUSE: {'B96', 'A60', 'B20', 'B09', 'B31', 'B93', 'A82', 'C08', 'A11', 'C85'}\\ \tt PROPOSAL: {'B96', 'A60', 'B20', 'B09', 'B31', 'B93', 'A82', 'C08', 'C85', 'B39', 'A21'}\\ \tt ARGUMENT: {"This swap improves efficiency while staying well under the limit: total effort is 2906 (78 below 2984). It replaces the high-effort A11 with two efficient items (A21 and B39), increasing overall value without sacrificing feasibility. I’ve kept four of the five anchors you highlighted (C08, B31, B09, B93) intact, and also retain C85, which is extremely effort-efficient at just 125. If A11 is a must-have for you, an alternative path is to add A21 by freeing ~350 effort elsewhere (e.g., dropping C08), but the current proposal preserves more of your structure and uses our budget better."} 
	  } 
	   } 
	   } 
	 & \\ 
 

    \theutterance \stepcounter{utterance}  

    & \multicolumn{4}{p{0.6\linewidth}}{\cellcolor[rgb]{0.8,1,0.9}{%
	\makecell[{{p{\linewidth}}}]{% 
	  \tt {\tiny [A$\rangle$GM]}  
	 REFUSE: {'B96', 'A60', 'B20', 'B09', 'B31', 'B93', 'A82', 'C08', 'C85', 'B39', 'A21'}\\ \tt PROPOSAL: {'A21', 'A60', 'B20', 'B09', 'B31', 'B93', 'A82', 'C08', 'A11'}\\ \tt ARGUMENT: {"I can accept bringing in A21, but A11 is essential on my side. To stay within the 2984 limit, I’ve built a tight set at exactly 2984 by removing only the two smallest-effort items (B96 and C85) and excluding B39, which is relatively heavy for me. This keeps your four anchors (C08, B31, B09, B93) intact and adds A21 as you suggested. If B39 is a must-have for you, please indicate which larger item you’re willing to trade out (e.g., A21, A60, or B20), since we’d need to free about 389 effort. Alternatively, if you’d rather keep one of B96 or C85, we can revisit A21 instead. Let’s lock this in if you can live without B39."} 
	  } 
	   } 
	   } 
	 & & \\ 
 

    \theutterance \stepcounter{utterance}  

    & & \multicolumn{4}{p{0.6\linewidth}}{\cellcolor[rgb]{0.9,0.9,0.9}{%
	\makecell[{{p{\linewidth}}}]{% 
	  \tt {\tiny [GM$\rangle$B]}  
	 REFUSE: {'B96', 'A60', 'B20', 'B09', 'B31', 'B93', 'A82', 'C08', 'C85', 'B39', 'A21'}\\ \tt PROPOSAL: {'A21', 'A60', 'B20', 'B09', 'B31', 'B93', 'A82', 'C08', 'A11'}\\ \tt ARGUMENT: {"I can accept bringing in A21, but A11 is essential on my side. To stay within the 2984 limit, I’ve built a tight set at exactly 2984 by removing only the two smallest-effort items (B96 and C85) and excluding B39, which is relatively heavy for me. This keeps your four anchors (C08, B31, B09, B93) intact and adds A21 as you suggested. If B39 is a must-have for you, please indicate which larger item you’re willing to trade out (e.g., A21, A60, or B20), since we’d need to free about 389 effort. Alternatively, if you’d rather keep one of B96 or C85, we can revisit A21 instead. Let’s lock this in if you can live without B39."} 
	  } 
	   } 
	   } 
	 & \\ 
 

    \theutterance \stepcounter{utterance}  

    & & & \multicolumn{4}{p{0.6\linewidth}}{\cellcolor[rgb]{1,0.85,0.72}{%
	\makecell[{{p{\linewidth}}}]{% 
	  \tt {\tiny [GM$\langle$B]}  
	 REFUSE: {'A21', 'A60', 'B20', 'B09', 'B31', 'B93', 'A82', 'C08', 'A11'}\\ \tt PROPOSAL: {'B96', 'C85', 'B39', 'A60', 'B20', 'B09', 'B31', 'B93', 'C08', 'A11'}\\ \tt ARGUMENT: {"This keeps all five of your anchors intact (A11, C08, B31, B09, B93) and follows your invitation to include B39 by trading out a larger item: I drop A21 and also replace A82 with two very efficient low-effort pieces (B96 and C85). Total effort is 2834, giving us 150 slack while respecting your essentials. If B39 remains a sticking point, an easy fallback is to keep your structure and simply swap A82 out for B96 and C85 (effort 2795), preserving all your anchors plus A21."} 
	  } 
	   } 
	   } 
	  \\ 
 

    \theutterance \stepcounter{utterance}  

    & & \multicolumn{4}{p{0.6\linewidth}}{\cellcolor[rgb]{0.9,0.9,0.9}{%
	\makecell[{{p{\linewidth}}}]{% 
	  \tt {\tiny [A$\langle$GM]}  
	 REFUSE: {'A21', 'A60', 'B20', 'B09', 'B31', 'B93', 'A82', 'C08', 'A11'}\\ \tt PROPOSAL: {'B96', 'C85', 'B39', 'A60', 'B20', 'B09', 'B31', 'B93', 'C08', 'A11'}\\ \tt ARGUMENT: {"This keeps all five of your anchors intact (A11, C08, B31, B09, B93) and follows your invitation to include B39 by trading out a larger item: I drop A21 and also replace A82 with two very efficient low-effort pieces (B96 and C85). Total effort is 2834, giving us 150 slack while respecting your essentials. If B39 remains a sticking point, an easy fallback is to keep your structure and simply swap A82 out for B96 and C85 (effort 2795), preserving all your anchors plus A21."} 
	  } 
	   } 
	   } 
	 & \\ 
 

    \theutterance \stepcounter{utterance}  

    & \multicolumn{4}{p{0.6\linewidth}}{\cellcolor[rgb]{0.8,1,0.9}{%
	\makecell[{{p{\linewidth}}}]{% 
	  \tt {\tiny [A$\rangle$GM]}  
	 AGREE: {'B96', 'C85', 'B39', 'A60', 'B20', 'B09', 'B31', 'B93', 'C08', 'A11'}\\ \tt ARGUMENT: {"This package preserves my key anchors and is efficient under the limit with comfortable slack, so I’m ready to lock it in."} 
	  } 
	   } 
	   } 
	 & & \\ 
 

    \theutterance \stepcounter{utterance}  

    & & & \multicolumn{2}{p{0.3\linewidth}}{\cellcolor[rgb]{0.95,0.95,0.95}{%
	\makecell[{{p{\linewidth}}}]{% 
	  \tt {\tiny [GM$|$GM]}  
	 game successful 
	  } 
	   } 
	   } 
	 & & \\ 
 

    \theutterance \stepcounter{utterance}  

    & & & \multicolumn{2}{p{0.3\linewidth}}{\cellcolor[rgb]{0.95,0.95,0.95}{%
	\makecell[{{p{\linewidth}}}]{% 
	  \tt {\tiny [GM$|$GM]}  
	 end game 
	  } 
	   } 
	   } 
	 & & \\ 
 

\end{supertabular}
}

\end{document}
