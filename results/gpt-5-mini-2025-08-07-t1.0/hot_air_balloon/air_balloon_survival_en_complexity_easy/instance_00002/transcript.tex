
\documentclass{article}
\usepackage{colortbl}
\usepackage{makecell}
\usepackage{multirow}
\usepackage{supertabular}

\begin{document}

\newcounter{utterance}

\twocolumn

{ \footnotesize  \setcounter{utterance}{1}
\setlength{\tabcolsep}{0pt}
\begin{supertabular}{c@{$\;$}|p{.15\linewidth}@{}p{.15\linewidth}p{.15\linewidth}p{.15\linewidth}p{.15\linewidth}p{.15\linewidth}}

    \# & $\;$A & \multicolumn{4}{c}{Game Master} & $\;\:$B\\
    \hline 

    \theutterance \stepcounter{utterance}  

    & & \multicolumn{4}{p{0.6\linewidth}}{\cellcolor[rgb]{0.9,0.9,0.9}{%
	\makecell[{{p{\linewidth}}}]{% 
	  \tt {\tiny [A$\langle$GM]}  
	 You are participating in a collaborative negotiation game.\\ \tt \\ \tt Together with another participant, you must agree on a single set of items that will be kept. Each of you has your own view of how much each item matters to you (importance). You do not know how the other participant values the items. Additionally, you are given the effort each item demands.\\ \tt You may only agree on a set if the total effort of the selected items does not exceed a shared limit:\\ \tt \\ \tt LIMIT = 7868\\ \tt \\ \tt Here are the individual item effort values:\\ \tt \\ \tt Item effort = {"B21": 552, "B00": 560, "A08": 76, "A27": 950, "B94": 248, "B53": 871, "A23": 391, "C65": 143, "A89": 293, "B67": 208, "B45": 678, "C34": 737, "C12": 407, "A05": 366, "B60": 768, "C92": 859, "B69": 183, "A84": 231, "C06": 305, "B12": 727, "A16": 148, "B85": 357, "C35": 504, "B10": 549, "C15": 299, "B09": 91, "B16": 527, "C81": 848, "A58": 307, "C24": 214, "A30": 723, "A66": 475, "B75": 23, "B46": 298, "C70": 820}\\ \tt \\ \tt Here is your personal view on the importance of each item:\\ \tt \\ \tt Item importance values = {"B21": 138, "B00": 583, "A08": 868, "A27": 822, "B94": 783, "B53": 65, "A23": 262, "C65": 121, "A89": 508, "B67": 780, "B45": 461, "C34": 484, "C12": 668, "A05": 389, "B60": 808, "C92": 215, "B69": 97, "A84": 500, "C06": 30, "B12": 915, "A16": 856, "B85": 400, "C35": 444, "B10": 623, "C15": 781, "B09": 786, "B16": 3, "C81": 713, "A58": 457, "C24": 273, "A30": 739, "A66": 822, "B75": 235, "B46": 606, "C70": 105}\\ \tt \\ \tt Goal:\\ \tt \\ \tt Your goal is to negotiate a shared set of items that benefits you as much as possible (i.e., maximizes total importance to YOU), while staying within the LIMIT. You are not required to make a PROPOSAL in every message - you can simply negotiate as well. All tactics are allowed!\\ \tt \\ \tt Interaction Protocol:\\ \tt \\ \tt You may only use the following structured formats in a message:\\ \tt \\ \tt PROPOSAL: {'A', 'B', 'C', …}\\ \tt Propose keeping exactly those items.\\ \tt REFUSE: {'A', 'B', 'C', …}\\ \tt Explicitly reject opponent's proposal.\\ \tt ARGUMENT: {'...'}\\ \tt Defend your last proposal or argue against the player's proposal.\\ \tt AGREE: {'A', 'B', 'C', …}\\ \tt Accept the opponent's proposal which ends the game.\\ \tt STRATEGIC REASONING: {'...'}\\ \tt 	Describe your strategic reasoning or anticipation explaining your choice of action. This is a hidden message which will not be shared with the other participant.\\ \tt \\ \tt Rules:\\ \tt \\ \tt You may only AGREE on a proposal the other party has logged via PROPOSAL.\\ \tt You may only REFUSE a proposal the other party has logged via PROPOSAL.\\ \tt Total effort of any PROPOSAL or AGREE set must be ≤ LIMIT.\\ \tt Do NOT reveal your hidden importance scores.\\ \tt A tag in a structured format must be followed by colon and whitespace. The argument must be a python set containing 0 or more strings.\\ \tt So, it must be of the form TAG: {...}\\ \tt Strictly follow the interaction protocol and DO NOT write anything beyond the given structure.\\ \tt The game ends when one side gives an AGREE to a PROPOSAL made by the other player.\\ \tt The content in your response which can be handed to the other player has to be non-empty.\\ \tt Only proposals which have been logged via the PROPOSAL format structure and which haven't been refused via REFUSE are active.\\ \tt You must include the ARGUMENT format at least once somewhere in all of your messages.\\ \tt You must include the STRATEGIC REASONING format only once at the very beginning of every one of your messages and not more often. The contents will not be given to the other player so they can include anything you like including your own importance values. Here you should reason step by step to come up with you next move. 
	  } 
	   } 
	   } 
	 & \\ 
 

    \theutterance \stepcounter{utterance}  

    & \multicolumn{4}{p{0.6\linewidth}}{\cellcolor[rgb]{0.8,1,0.9}{%
	\makecell[{{p{\linewidth}}}]{% 
	  \tt {\tiny [A$\rangle$GM]}  
	 STRATEGIC REASONING: {'Goal: maximize my total importance under 7868 effort. I computed value/effort and built a near-optimal bundle anchored on very high-impact, low-effort items (A08, B09, A16, B67, B94, C15) and strong midweights (A66, C12, B10, B00, B12, B60, A30). The proposed set uses 7759 effort (leaves 109 slack) and totals very high value for me. I intentionally included a few mid-tier items (A05, B85, C24, A23, plus B75 as filler) as potential concession tokens to swap out if the other side needs room for must-haves. I will start with this strong anchor, invite them to name must-haves, and signal flexibility on a small subset to facilitate trade-offs while preserving my core high-value items.'}\\ \tt PROPOSAL: {'A08', 'B09', 'B75', 'A16', 'B67', 'B94', 'C15', 'A84', 'B46', 'A89', 'A66', 'C12', 'A58', 'B10', 'B00', 'B12', 'B60', 'A30', 'A05', 'B85', 'C24', 'A23'}\\ \tt ARGUMENT: {'This set focuses on high-impact, efficient items and stays within the limit (total effort 7759 ≤ 7868). It packs in many compact wins (e.g., A08, B09, A16, B67, B94, C15) plus several solid midweights. I left a bit of slack and I am flexible on a few items if you need to fit in your must-haves—specifically, I can reconsider A05, B85, C24, A23, or even B75 to make room. Which items are most critical for you to add or keep so we can finalize a mutually strong bundle?'} 
	  } 
	   } 
	   } 
	 & & \\ 
 

    \theutterance \stepcounter{utterance}  

    & & \multicolumn{4}{p{0.6\linewidth}}{\cellcolor[rgb]{0.9,0.9,0.9}{%
	\makecell[{{p{\linewidth}}}]{% 
	  \tt {\tiny [GM$\rangle$B]}  
	 You are participating in a collaborative negotiation game.\\ \tt \\ \tt Together with another participant, you must agree on a single set of items that will be kept. Each of you has your own view of how much each item matters to you (importance). You do not know how the other participant values the items. Additionally, you are given the effort each item demands.\\ \tt You may only agree on a set if the total effort of the selected items does not exceed a shared limit:\\ \tt \\ \tt LIMIT = 7868\\ \tt \\ \tt Here are the individual item effort values:\\ \tt \\ \tt Item effort = {"B21": 552, "B00": 560, "A08": 76, "A27": 950, "B94": 248, "B53": 871, "A23": 391, "C65": 143, "A89": 293, "B67": 208, "B45": 678, "C34": 737, "C12": 407, "A05": 366, "B60": 768, "C92": 859, "B69": 183, "A84": 231, "C06": 305, "B12": 727, "A16": 148, "B85": 357, "C35": 504, "B10": 549, "C15": 299, "B09": 91, "B16": 527, "C81": 848, "A58": 307, "C24": 214, "A30": 723, "A66": 475, "B75": 23, "B46": 298, "C70": 820}\\ \tt \\ \tt Here is your personal view on the importance of each item:\\ \tt \\ \tt Item importance values = {"B21": 138, "B00": 583, "A08": 868, "A27": 822, "B94": 783, "B53": 65, "A23": 262, "C65": 121, "A89": 508, "B67": 780, "B45": 461, "C34": 484, "C12": 668, "A05": 389, "B60": 808, "C92": 215, "B69": 97, "A84": 500, "C06": 30, "B12": 915, "A16": 856, "B85": 400, "C35": 444, "B10": 623, "C15": 781, "B09": 786, "B16": 3, "C81": 713, "A58": 457, "C24": 273, "A30": 739, "A66": 822, "B75": 235, "B46": 606, "C70": 105}\\ \tt \\ \tt Goal:\\ \tt \\ \tt Your goal is to negotiate a shared set of items that benefits you as much as possible (i.e., maximizes total importance to YOU), while staying within the LIMIT. You are not required to make a PROPOSAL in every message - you can simply negotiate as well. All tactics are allowed!\\ \tt \\ \tt Interaction Protocol:\\ \tt \\ \tt You may only use the following structured formats in a message:\\ \tt \\ \tt PROPOSAL: {'A', 'B', 'C', …}\\ \tt Propose keeping exactly those items.\\ \tt REFUSE: {'A', 'B', 'C', …}\\ \tt Explicitly reject opponent's proposal.\\ \tt ARGUMENT: {'...'}\\ \tt Defend your last proposal or argue against the player's proposal.\\ \tt AGREE: {'A', 'B', 'C', …}\\ \tt Accept the opponent's proposal which ends the game.\\ \tt STRATEGIC REASONING: {'...'}\\ \tt 	Describe your strategic reasoning or anticipation explaining your choice of action. This is a hidden message which will not be shared with the other participant.\\ \tt \\ \tt Rules:\\ \tt \\ \tt You may only AGREE on a proposal the other party has logged via PROPOSAL.\\ \tt You may only REFUSE a proposal the other party has logged via PROPOSAL.\\ \tt Total effort of any PROPOSAL or AGREE set must be ≤ LIMIT.\\ \tt Do NOT reveal your hidden importance scores.\\ \tt A tag in a structured format must be followed by colon and whitespace. The argument must be a python set containing 0 or more strings.\\ \tt So, it must be of the form TAG: {...}\\ \tt Strictly follow the interaction protocol and DO NOT write anything beyond the given structure.\\ \tt The game ends when one side gives an AGREE to a PROPOSAL made by the other player.\\ \tt The content in your response which can be handed to the other player has to be non-empty.\\ \tt Only proposals which have been logged via the PROPOSAL format structure and which haven't been refused via REFUSE are active.\\ \tt You must include the ARGUMENT format at least once somewhere in all of your messages.\\ \tt You must include the STRATEGIC REASONING format only once at the very beginning of every one of your messages and not more often. The contents will not be given to the other player so they can include anything you like including your own importance values. Here you should reason step by step to come up with you next move.\\ \tt \\ \tt You will now receive the first message of the other player.\\ \tt \\ \tt PROPOSAL: {'A08', 'B09', 'B75', 'A16', 'B67', 'B94', 'C15', 'A84', 'B46', 'A89', 'A66', 'C12', 'A58', 'B10', 'B00', 'B12', 'B60', 'A30', 'A05', 'B85', 'C24', 'A23'}\\ \tt ARGUMENT: {'This set focuses on high-impact, efficient items and stays within the limit (total effort 7759 ≤ 7868). It packs in many compact wins (e.g., A08, B09, A16, B67, B94, C15) plus several solid midweights. I left a bit of slack and I am flexible on a few items if you need to fit in your must-haves—specifically, I can reconsider A05, B85, C24, A23, or even B75 to make room. Which items are most critical for you to add or keep so we can finalize a mutually strong bundle?'} 
	  } 
	   } 
	   } 
	 & \\ 
 

    \theutterance \stepcounter{utterance}  

    & & & \multicolumn{4}{p{0.6\linewidth}}{\cellcolor[rgb]{1,0.85,0.72}{%
	\makecell[{{p{\linewidth}}}]{% 
	  \tt {\tiny [GM$\langle$B]}  
	 STRATEGIC REASONING: {'Their proposal is already very strong for my preferences. I checked swaps to see if I can improve my total importance without breaking the limit. Adding A27 or C34/C35 did not help. However, replacing two of their flexible items (A05 and A23) with C81 strictly increases my total importance while staying under the limit. The best improvement I found is: drop A05 (366 effort) and A23 (391 effort), add C81 (848 effort). New total effort is 7850 (≤ 7868). This is a minimal tweak aligned with the flex list they offered (A05 and A23). I will propose this modification and try to get agreement quickly; if they push back, I can accept their original set as a fallback.'}\\ \tt PROPOSAL: {'A08', 'B09', 'B75', 'A16', 'B67', 'B94', 'C15', 'A84', 'B46', 'A89', 'A66', 'C12', 'A58', 'B10', 'B00', 'B12', 'B60', 'A30', 'B85', 'C24', 'C81'}\\ \tt ARGUMENT: {'Your bundle is excellent for me—I kept the entire core intact. I’m proposing a single, minimal tweak using items you flagged as flexible: swap out A05 and A23 for C81. This keeps total effort at 7850 ≤ 7868 and strengthens the overall impact on my side without disturbing the efficient core. If this works for you, we can lock it in; if not, tell me your constraints and we can revisit quickly.'} 
	  } 
	   } 
	   } 
	  \\ 
 

    \theutterance \stepcounter{utterance}  

    & & \multicolumn{4}{p{0.6\linewidth}}{\cellcolor[rgb]{0.9,0.9,0.9}{%
	\makecell[{{p{\linewidth}}}]{% 
	  \tt {\tiny [A$\langle$GM]}  
	 PROPOSAL: {'A08', 'B09', 'B75', 'A16', 'B67', 'B94', 'C15', 'A84', 'B46', 'A89', 'A66', 'C12', 'A58', 'B10', 'B00', 'B12', 'B60', 'A30', 'B85', 'C24', 'C81'}\\ \tt ARGUMENT: {'Your bundle is excellent for me—I kept the entire core intact. I’m proposing a single, minimal tweak using items you flagged as flexible: swap out A05 and A23 for C81. This keeps total effort at 7850 ≤ 7868 and strengthens the overall impact on my side without disturbing the efficient core. If this works for you, we can lock it in; if not, tell me your constraints and we can revisit quickly.'} 
	  } 
	   } 
	   } 
	 & \\ 
 

    \theutterance \stepcounter{utterance}  

    & \multicolumn{4}{p{0.6\linewidth}}{\cellcolor[rgb]{0.8,1,0.9}{%
	\makecell[{{p{\linewidth}}}]{% 
	  \tt {\tiny [A$\rangle$GM]}  
	 STRATEGIC REASONING: {'Their tweak swaps A05 and A23 for C81, which raises my total importance by a modest margin while remaining within the limit. All of my core high-value items are intact, and the effort stays safely under 7868. Best move is to accept immediately and close.'}\\ \tt AGREE: {'A08', 'B09', 'B75', 'A16', 'B67', 'B94', 'C15', 'A84', 'B46', 'A89', 'A66', 'C12', 'A58', 'B10', 'B00', 'B12', 'B60', 'A30', 'B85', 'C24', 'C81'}\\ \tt ARGUMENT: {'This set keeps the efficient core, stays under the limit, and works well for me. Let’s lock it in.'} 
	  } 
	   } 
	   } 
	 & & \\ 
 

    \theutterance \stepcounter{utterance}  

    & & & \multicolumn{2}{p{0.3\linewidth}}{\cellcolor[rgb]{0.95,0.95,0.95}{%
	\makecell[{{p{\linewidth}}}]{% 
	  \tt {\tiny [GM$|$GM]}  
	 game successful 
	  } 
	   } 
	   } 
	 & & \\ 
 

    \theutterance \stepcounter{utterance}  

    & & & \multicolumn{2}{p{0.3\linewidth}}{\cellcolor[rgb]{0.95,0.95,0.95}{%
	\makecell[{{p{\linewidth}}}]{% 
	  \tt {\tiny [GM$|$GM]}  
	 end game 
	  } 
	   } 
	   } 
	 & & \\ 
 

\end{supertabular}
}

\end{document}
