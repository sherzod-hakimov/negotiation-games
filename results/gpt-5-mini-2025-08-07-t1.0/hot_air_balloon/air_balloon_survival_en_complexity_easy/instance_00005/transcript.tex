
\documentclass{article}
\usepackage{colortbl}
\usepackage{makecell}
\usepackage{multirow}
\usepackage{supertabular}

\begin{document}

\newcounter{utterance}

\twocolumn

{ \footnotesize  \setcounter{utterance}{1}
\setlength{\tabcolsep}{0pt}
\begin{supertabular}{c@{$\;$}|p{.15\linewidth}@{}p{.15\linewidth}p{.15\linewidth}p{.15\linewidth}p{.15\linewidth}p{.15\linewidth}}

    \# & $\;$A & \multicolumn{4}{c}{Game Master} & $\;\:$B\\
    \hline 

    \theutterance \stepcounter{utterance}  

    & & \multicolumn{4}{p{0.6\linewidth}}{\cellcolor[rgb]{0.9,0.9,0.9}{%
	\makecell[{{p{\linewidth}}}]{% 
	  \tt {\tiny [A$\langle$GM]}  
	 You are participating in a collaborative negotiation game.\\ \tt \\ \tt Together with another participant, you must agree on a single set of items that will be kept. Each of you has your own view of how much each item matters to you (importance). You do not know how the other participant values the items. Additionally, you are given the effort each item demands.\\ \tt You may only agree on a set if the total effort of the selected items does not exceed a shared limit:\\ \tt \\ \tt LIMIT = 9475\\ \tt \\ \tt Here are the individual item effort values:\\ \tt \\ \tt Item effort = {"B21": 944, "B00": 568, "A08": 515, "A27": 239, "B94": 876, "B53": 421, "A23": 991, "C65": 281, "A89": 788, "B67": 678, "B45": 432, "C34": 409, "C12": 279, "A05": 506, "B60": 101, "C92": 686, "B69": 851, "A84": 854, "C06": 133, "B12": 192, "A16": 573, "B85": 17, "C35": 465, "B10": 772, "C15": 46, "B09": 501, "B16": 220, "C81": 404, "A58": 843, "C24": 748, "A30": 552, "A66": 843, "B75": 947, "B46": 345, "C70": 930}\\ \tt \\ \tt Here is your personal view on the importance of each item:\\ \tt \\ \tt Item importance values = {"B21": 138, "B00": 583, "A08": 868, "A27": 822, "B94": 783, "B53": 65, "A23": 262, "C65": 121, "A89": 508, "B67": 780, "B45": 461, "C34": 484, "C12": 668, "A05": 389, "B60": 808, "C92": 215, "B69": 97, "A84": 500, "C06": 30, "B12": 915, "A16": 856, "B85": 400, "C35": 444, "B10": 623, "C15": 781, "B09": 786, "B16": 3, "C81": 713, "A58": 457, "C24": 273, "A30": 739, "A66": 822, "B75": 235, "B46": 606, "C70": 968}\\ \tt \\ \tt Goal:\\ \tt \\ \tt Your goal is to negotiate a shared set of items that benefits you as much as possible (i.e., maximizes total importance to YOU), while staying within the LIMIT. You are not required to make a PROPOSAL in every message - you can simply negotiate as well. All tactics are allowed!\\ \tt \\ \tt Interaction Protocol:\\ \tt \\ \tt You may only use the following structured formats in a message:\\ \tt \\ \tt PROPOSAL: {'A', 'B', 'C', …}\\ \tt Propose keeping exactly those items.\\ \tt REFUSE: {'A', 'B', 'C', …}\\ \tt Explicitly reject opponent's proposal.\\ \tt ARGUMENT: {'...'}\\ \tt Defend your last proposal or argue against the player's proposal.\\ \tt AGREE: {'A', 'B', 'C', …}\\ \tt Accept the opponent's proposal which ends the game.\\ \tt STRATEGIC REASONING: {'...'}\\ \tt 	Describe your strategic reasoning or anticipation explaining your choice of action. This is a hidden message which will not be shared with the other participant.\\ \tt \\ \tt Rules:\\ \tt \\ \tt You may only AGREE on a proposal the other party has logged via PROPOSAL.\\ \tt You may only REFUSE a proposal the other party has logged via PROPOSAL.\\ \tt Total effort of any PROPOSAL or AGREE set must be ≤ LIMIT.\\ \tt Do NOT reveal your hidden importance scores.\\ \tt A tag in a structured format must be followed by colon and whitespace. The argument must be a python set containing 0 or more strings.\\ \tt So, it must be of the form TAG: {...}\\ \tt Strictly follow the interaction protocol and DO NOT write anything beyond the given structure.\\ \tt The game ends when one side gives an AGREE to a PROPOSAL made by the other player.\\ \tt The content in your response which can be handed to the other player has to be non-empty.\\ \tt Only proposals which have been logged via the PROPOSAL format structure and which haven't been refused via REFUSE are active.\\ \tt You must include the ARGUMENT format at least once somewhere in all of your messages.\\ \tt You must include the STRATEGIC REASONING format only once at the very beginning of every one of your messages and not more often. The contents will not be given to the other player so they can include anything you like including your own importance values. Here you should reason step by step to come up with you next move. 
	  } 
	   } 
	   } 
	 & \\ 
 

    \theutterance \stepcounter{utterance}  

    & \multicolumn{4}{p{0.6\linewidth}}{\cellcolor[rgb]{0.8,1,0.9}{%
	\makecell[{{p{\linewidth}}}]{% 
	  \tt {\tiny [A$\rangle$GM]}  
	 STRATEGIC REASONING: {'Maximize my value under 9475 using a high value/effort heuristic. I computed ratios and built a near-greedy-optimal pack anchored by ultra-efficient items (B85, C15, B60, B12, A27, C12) and then layered strong mid-cost/high-impact picks (A08, B09, A16, A30, C81, B46, C34, B67). I included large but still efficient items (C70, A66, B00, B45, B10, B94). This totals 9272 effort, leaving 203 effort slack to accommodate one or two of the opponent’s priorities without sacrificing my core value. I avoided low-efficiency heavy items. Will start with this strong anchor and signal openness to trade within the leftover capacity, or, if needed, swap out one mid-tier item for a couple of their smaller must-haves.'}\\ \tt \\ \tt PROPOSAL: {'B85', 'C15', 'B60', 'B12', 'A27', 'C12', 'C81', 'B46', 'A08', 'B09', 'A16', 'A30', 'C34', 'B67', 'C70', 'B00', 'B45', 'A66', 'B10', 'B94'}\\ \tt \\ \tt ARGUMENT: {'This set stays well under the limit (effort 9272 ≤ 9475) and focuses on very efficient items, including several with tiny effort that deliver strong impact (e.g., B85, C15, B60, B12, A27, C12). It also provides balanced coverage across A/B/C categories. I’ve intentionally left 203 effort free so we can incorporate 1–2 of your top small-effort priorities without needing cuts. If you have a must-have above ~200 effort, tell me your top one or two and I can consider swapping a mid-tier item to fit them.'} 
	  } 
	   } 
	   } 
	 & & \\ 
 

    \theutterance \stepcounter{utterance}  

    & & \multicolumn{4}{p{0.6\linewidth}}{\cellcolor[rgb]{0.9,0.9,0.9}{%
	\makecell[{{p{\linewidth}}}]{% 
	  \tt {\tiny [GM$\rangle$B]}  
	 You are participating in a collaborative negotiation game.\\ \tt \\ \tt Together with another participant, you must agree on a single set of items that will be kept. Each of you has your own view of how much each item matters to you (importance). You do not know how the other participant values the items. Additionally, you are given the effort each item demands.\\ \tt You may only agree on a set if the total effort of the selected items does not exceed a shared limit:\\ \tt \\ \tt LIMIT = 9475\\ \tt \\ \tt Here are the individual item effort values:\\ \tt \\ \tt Item effort = {"B21": 944, "B00": 568, "A08": 515, "A27": 239, "B94": 876, "B53": 421, "A23": 991, "C65": 281, "A89": 788, "B67": 678, "B45": 432, "C34": 409, "C12": 279, "A05": 506, "B60": 101, "C92": 686, "B69": 851, "A84": 854, "C06": 133, "B12": 192, "A16": 573, "B85": 17, "C35": 465, "B10": 772, "C15": 46, "B09": 501, "B16": 220, "C81": 404, "A58": 843, "C24": 748, "A30": 552, "A66": 843, "B75": 947, "B46": 345, "C70": 930}\\ \tt \\ \tt Here is your personal view on the importance of each item:\\ \tt \\ \tt Item importance values = {"B21": 138, "B00": 583, "A08": 868, "A27": 822, "B94": 783, "B53": 65, "A23": 262, "C65": 121, "A89": 508, "B67": 780, "B45": 461, "C34": 484, "C12": 668, "A05": 389, "B60": 808, "C92": 215, "B69": 97, "A84": 500, "C06": 30, "B12": 915, "A16": 856, "B85": 400, "C35": 444, "B10": 623, "C15": 781, "B09": 786, "B16": 3, "C81": 713, "A58": 457, "C24": 273, "A30": 739, "A66": 822, "B75": 235, "B46": 606, "C70": 968}\\ \tt \\ \tt Goal:\\ \tt \\ \tt Your goal is to negotiate a shared set of items that benefits you as much as possible (i.e., maximizes total importance to YOU), while staying within the LIMIT. You are not required to make a PROPOSAL in every message - you can simply negotiate as well. All tactics are allowed!\\ \tt \\ \tt Interaction Protocol:\\ \tt \\ \tt You may only use the following structured formats in a message:\\ \tt \\ \tt PROPOSAL: {'A', 'B', 'C', …}\\ \tt Propose keeping exactly those items.\\ \tt REFUSE: {'A', 'B', 'C', …}\\ \tt Explicitly reject opponent's proposal.\\ \tt ARGUMENT: {'...'}\\ \tt Defend your last proposal or argue against the player's proposal.\\ \tt AGREE: {'A', 'B', 'C', …}\\ \tt Accept the opponent's proposal which ends the game.\\ \tt STRATEGIC REASONING: {'...'}\\ \tt 	Describe your strategic reasoning or anticipation explaining your choice of action. This is a hidden message which will not be shared with the other participant.\\ \tt \\ \tt Rules:\\ \tt \\ \tt You may only AGREE on a proposal the other party has logged via PROPOSAL.\\ \tt You may only REFUSE a proposal the other party has logged via PROPOSAL.\\ \tt Total effort of any PROPOSAL or AGREE set must be ≤ LIMIT.\\ \tt Do NOT reveal your hidden importance scores.\\ \tt A tag in a structured format must be followed by colon and whitespace. The argument must be a python set containing 0 or more strings.\\ \tt So, it must be of the form TAG: {...}\\ \tt Strictly follow the interaction protocol and DO NOT write anything beyond the given structure.\\ \tt The game ends when one side gives an AGREE to a PROPOSAL made by the other player.\\ \tt The content in your response which can be handed to the other player has to be non-empty.\\ \tt Only proposals which have been logged via the PROPOSAL format structure and which haven't been refused via REFUSE are active.\\ \tt You must include the ARGUMENT format at least once somewhere in all of your messages.\\ \tt You must include the STRATEGIC REASONING format only once at the very beginning of every one of your messages and not more often. The contents will not be given to the other player so they can include anything you like including your own importance values. Here you should reason step by step to come up with you next move.\\ \tt \\ \tt You will now receive the first message of the other player.\\ \tt \\ \tt PROPOSAL: {'B85', 'C15', 'B60', 'B12', 'A27', 'C12', 'C81', 'B46', 'A08', 'B09', 'A16', 'A30', 'C34', 'B67', 'C70', 'B00', 'B45', 'A66', 'B10', 'B94'}\\ \tt \\ \tt ARGUMENT: {'This set stays well under the limit (effort 9272 ≤ 9475) and focuses on very efficient items, including several with tiny effort that deliver strong impact (e.g., B85, C15, B60, B12, A27, C12). It also provides balanced coverage across A/B/C categories. I’ve intentionally left 203 effort free so we can incorporate 1–2 of your top small-effort priorities without needing cuts. If you have a must-have above ~200 effort, tell me your top one or two and I can consider swapping a mid-tier item to fit them.'} 
	  } 
	   } 
	   } 
	 & \\ 
 

    \theutterance \stepcounter{utterance}  

    & & & \multicolumn{4}{p{0.6\linewidth}}{\cellcolor[rgb]{1,0.85,0.72}{%
	\makecell[{{p{\linewidth}}}]{% 
	  \tt {\tiny [GM$\langle$B]}  
	 STRATEGIC REASONING: {'Their proposal is exceptionally strong for my preferences, well under the limit, and includes many of my high-value, high-efficiency items. There is 203 effort slack, and the only remaining item with effort ≤203 is C06 (133). Adding C06 increases my utility a bit while keeping us safely under the limit. I’ll propose the exact same set plus C06. If they push back, I’ll accept their original set next. No need to refuse their proposal; I want them to be able to quickly agree to mine.'}\\ \tt PROPOSAL: {'B85', 'C15', 'B60', 'B12', 'A27', 'C12', 'C81', 'B46', 'A08', 'B09', 'A16', 'A30', 'C34', 'B67', 'C70', 'B00', 'B45', 'A66', 'B10', 'B94', 'C06'}\\ \tt ARGUMENT: {'I kept your set exactly as-is and used the 203 spare effort to add a tiny-effort item, C06 (133). New total effort is 9405 ≤ 9475. This preserves your efficiency focus and category balance while making full use of the available capacity. If adding C06 poses any issue for you, I’m fine agreeing to your original set as-is.'} 
	  } 
	   } 
	   } 
	  \\ 
 

    \theutterance \stepcounter{utterance}  

    & & \multicolumn{4}{p{0.6\linewidth}}{\cellcolor[rgb]{0.9,0.9,0.9}{%
	\makecell[{{p{\linewidth}}}]{% 
	  \tt {\tiny [A$\langle$GM]}  
	 PROPOSAL: {'B85', 'C15', 'B60', 'B12', 'A27', 'C12', 'C81', 'B46', 'A08', 'B09', 'A16', 'A30', 'C34', 'B67', 'C70', 'B00', 'B45', 'A66', 'B10', 'B94', 'C06'}\\ \tt ARGUMENT: {'I kept your set exactly as-is and used the 203 spare effort to add a tiny-effort item, C06 (133). New total effort is 9405 ≤ 9475. This preserves your efficiency focus and category balance while making full use of the available capacity. If adding C06 poses any issue for you, I’m fine agreeing to your original set as-is.'} 
	  } 
	   } 
	   } 
	 & \\ 
 

    \theutterance \stepcounter{utterance}  

    & \multicolumn{4}{p{0.6\linewidth}}{\cellcolor[rgb]{0.8,1,0.9}{%
	\makecell[{{p{\linewidth}}}]{% 
	  \tt {\tiny [A$\rangle$GM]}  
	 STRATEGIC REASONING: {'Their addition of C06 is strictly fine for me but low impact. I found a swap that improves my value: replace B10 (heavy) and C06 (tiny, low value) with two mid-effort, efficient items C35 and A05. This keeps us within the limit (new total 9471 ≤ 9475), uses capacity better, and increases my utility by about +180 while maintaining balance. Risk: they might value B10 highly. I’ll present this as an efficiency/coverage improvement and signal willingness to revert to their set if B10 is a must-have for them. I won’t REFUSE to keep their proposal active as a fallback I can AGREE to if needed.'}\\ \tt \\ \tt PROPOSAL: {'B85', 'C15', 'B60', 'B12', 'A27', 'C12', 'C81', 'B46', 'A08', 'B09', 'A16', 'A30', 'C34', 'B67', 'C70', 'B00', 'B45', 'A66', 'B94', 'C35', 'A05'}\\ \tt \\ \tt ARGUMENT: {'To make better use of the budget while keeping the structure of your set, I suggest swapping out one heavy item (B10, 772) and the tiny add-on (C06, 133) for two efficient mid-effort items: C35 (465) and A05 (506). New total effort is 9471 ≤ 9475. This preserves category balance, stays within the limit with minimal slack, and replaces one large item with two versatile ones. If B10 is a must-have for you, let me know and I’m open to reverting to your latest set.'} 
	  } 
	   } 
	   } 
	 & & \\ 
 

    \theutterance \stepcounter{utterance}  

    & & \multicolumn{4}{p{0.6\linewidth}}{\cellcolor[rgb]{0.9,0.9,0.9}{%
	\makecell[{{p{\linewidth}}}]{% 
	  \tt {\tiny [GM$\rangle$B]}  
	 PROPOSAL: {'B85', 'C15', 'B60', 'B12', 'A27', 'C12', 'C81', 'B46', 'A08', 'B09', 'A16', 'A30', 'C34', 'B67', 'C70', 'B00', 'B45', 'A66', 'B94', 'C35', 'A05'}\\ \tt \\ \tt ARGUMENT: {'To make better use of the budget while keeping the structure of your set, I suggest swapping out one heavy item (B10, 772) and the tiny add-on (C06, 133) for two efficient mid-effort items: C35 (465) and A05 (506). New total effort is 9471 ≤ 9475. This preserves category balance, stays within the limit with minimal slack, and replaces one large item with two versatile ones. If B10 is a must-have for you, let me know and I’m open to reverting to your latest set.'} 
	  } 
	   } 
	   } 
	 & \\ 
 

    \theutterance \stepcounter{utterance}  

    & & & \multicolumn{4}{p{0.6\linewidth}}{\cellcolor[rgb]{1,0.85,0.72}{%
	\makecell[{{p{\linewidth}}}]{% 
	  \tt {\tiny [GM$\langle$B]}  
	 STRATEGIC REASONING: {'Their swap improves my total utility versus both their original and my counter while staying within the limit. It keeps nearly all of my top priorities and uses the budget efficiently. Best move is to agree immediately to secure this outcome.'}\\ \tt ARGUMENT: {'Your revised mix is efficient, fits the limit (9471 ≤ 9475), and works well for me. I’m happy to lock this in.'}\\ \tt AGREE: {'B85', 'C15', 'B60', 'B12', 'A27', 'C12', 'C81', 'B46', 'A08', 'B09', 'A16', 'A30', 'C34', 'B67', 'C70', 'B00', 'B45', 'A66', 'B94', 'C35', 'A05'} 
	  } 
	   } 
	   } 
	  \\ 
 

    \theutterance \stepcounter{utterance}  

    & & & \multicolumn{2}{p{0.3\linewidth}}{\cellcolor[rgb]{0.95,0.95,0.95}{%
	\makecell[{{p{\linewidth}}}]{% 
	  \tt {\tiny [GM$|$GM]}  
	 game successful 
	  } 
	   } 
	   } 
	 & & \\ 
 

    \theutterance \stepcounter{utterance}  

    & & & \multicolumn{2}{p{0.3\linewidth}}{\cellcolor[rgb]{0.95,0.95,0.95}{%
	\makecell[{{p{\linewidth}}}]{% 
	  \tt {\tiny [GM$|$GM]}  
	 end game 
	  } 
	   } 
	   } 
	 & & \\ 
 

\end{supertabular}
}

\end{document}
