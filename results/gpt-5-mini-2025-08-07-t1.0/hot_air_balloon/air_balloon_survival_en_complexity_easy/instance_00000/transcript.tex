
\documentclass{article}
\usepackage{colortbl}
\usepackage{makecell}
\usepackage{multirow}
\usepackage{supertabular}

\begin{document}

\newcounter{utterance}

\twocolumn

{ \footnotesize  \setcounter{utterance}{1}
\setlength{\tabcolsep}{0pt}
\begin{supertabular}{c@{$\;$}|p{.15\linewidth}@{}p{.15\linewidth}p{.15\linewidth}p{.15\linewidth}p{.15\linewidth}p{.15\linewidth}}

    \# & $\;$A & \multicolumn{4}{c}{Game Master} & $\;\:$B\\
    \hline 

    \theutterance \stepcounter{utterance}  

    & & \multicolumn{4}{p{0.6\linewidth}}{\cellcolor[rgb]{0.9,0.9,0.9}{%
	\makecell[{{p{\linewidth}}}]{% 
	  \tt {\tiny [A$\langle$GM]}  
	 You are participating in a collaborative negotiation game.\\ \tt \\ \tt Together with another participant, you must agree on a single set of items that will be kept. Each of you has your own view of how much each item matters to you (importance). You do not know how the other participant values the items. Additionally, you are given the effort each item demands.\\ \tt You may only agree on a set if the total effort of the selected items does not exceed a shared limit:\\ \tt \\ \tt LIMIT = 7605\\ \tt \\ \tt Here are the individual item effort values:\\ \tt \\ \tt Item effort = {"B21": 369, "B00": 951, "A08": 156, "A27": 266, "B94": 581, "B53": 287, "A23": 866, "C65": 180, "A89": 800, "B67": 739, "B45": 637, "C34": 86, "C12": 749, "A05": 370, "B60": 345, "C92": 951, "B69": 147, "A84": 265, "C06": 262, "B12": 259, "A16": 358, "B85": 394, "C35": 286, "B10": 580, "C15": 479, "B09": 14, "B16": 153, "C81": 977, "A58": 134, "C24": 982, "A30": 259, "A66": 232, "B75": 202, "B46": 73, "C70": 821}\\ \tt \\ \tt Here is your personal view on the importance of each item:\\ \tt \\ \tt Item importance values = {"B21": 138, "B00": 583, "A08": 868, "A27": 822, "B94": 783, "B53": 65, "A23": 262, "C65": 121, "A89": 508, "B67": 780, "B45": 461, "C34": 484, "C12": 668, "A05": 389, "B60": 808, "C92": 215, "B69": 97, "A84": 500, "C06": 30, "B12": 915, "A16": 856, "B85": 400, "C35": 444, "B10": 623, "C15": 781, "B09": 786, "B16": 3, "C81": 713, "A58": 457, "C24": 273, "A30": 739, "A66": 822, "B75": 235, "B46": 606, "C70": 968}\\ \tt \\ \tt Goal:\\ \tt \\ \tt Your goal is to negotiate a shared set of items that benefits you as much as possible (i.e., maximizes total importance to YOU), while staying within the LIMIT. You are not required to make a PROPOSAL in every message - you can simply negotiate as well. All tactics are allowed!\\ \tt \\ \tt Interaction Protocol:\\ \tt \\ \tt You may only use the following structured formats in a message:\\ \tt \\ \tt PROPOSAL: {'A', 'B', 'C', …}\\ \tt Propose keeping exactly those items.\\ \tt REFUSE: {'A', 'B', 'C', …}\\ \tt Explicitly reject opponent's proposal.\\ \tt ARGUMENT: {'...'}\\ \tt Defend your last proposal or argue against the player's proposal.\\ \tt AGREE: {'A', 'B', 'C', …}\\ \tt Accept the opponent's proposal which ends the game.\\ \tt STRATEGIC REASONING: {'...'}\\ \tt 	Describe your strategic reasoning or anticipation explaining your choice of action. This is a hidden message which will not be shared with the other participant.\\ \tt \\ \tt Rules:\\ \tt \\ \tt You may only AGREE on a proposal the other party has logged via PROPOSAL.\\ \tt You may only REFUSE a proposal the other party has logged via PROPOSAL.\\ \tt Total effort of any PROPOSAL or AGREE set must be ≤ LIMIT.\\ \tt Do NOT reveal your hidden importance scores.\\ \tt A tag in a structured format must be followed by colon and whitespace. The argument must be a python set containing 0 or more strings.\\ \tt So, it must be of the form TAG: {...}\\ \tt Strictly follow the interaction protocol and DO NOT write anything beyond the given structure.\\ \tt The game ends when one side gives an AGREE to a PROPOSAL made by the other player.\\ \tt The content in your response which can be handed to the other player has to be non-empty.\\ \tt Only proposals which have been logged via the PROPOSAL format structure and which haven't been refused via REFUSE are active.\\ \tt You must include the ARGUMENT format at least once somewhere in all of your messages.\\ \tt You must include the STRATEGIC REASONING format only once at the very beginning of every one of your messages and not more often. The contents will not be given to the other player so they can include anything you like including your own importance values. Here you should reason step by step to come up with you next move. 
	  } 
	   } 
	   } 
	 & \\ 
 

    \theutterance \stepcounter{utterance}  

    & \multicolumn{4}{p{0.6\linewidth}}{\cellcolor[rgb]{0.8,1,0.9}{%
	\makecell[{{p{\linewidth}}}]{% 
	  \tt {\tiny [A$\rangle$GM]}  
	 STRATEGIC REASONING: {'I computed a high-value-per-effort bundle to maximize my utility under the 7605 limit. I built a core of must-haves with extremely high value density for me: {B09, B46, C34, A08, A66, B12, A58, A30, A27, A16, B60, C15}. I then added strong complements {C35, B10, B94, B67, C70, A84, C12} and filled remaining capacity efficiently with lighter items {A05, B75, C65, B69}. The total effort is 7581, leaving 24 slack. Items I am most flexible on (lowest priority to me among those included) are {C65, B69, B75, A05} and, if necessary, we could discuss trade-offs involving {C12 or B10} to accommodate a key need of the other side, provided we keep my core intact. I will anchor with this near-full proposal and invite swaps of the flexible items for their priorities that fit the capacity.'}\\ \tt \\ \tt PROPOSAL: {'B09', 'B46', 'C34', 'A08', 'A66', 'B12', 'A58', 'A30', 'A27', 'A16', 'B60', 'A84', 'C15', 'C35', 'B10', 'B75', 'B94', 'B67', 'C70', 'A05', 'C12', 'C65', 'B69'}\\ \tt \\ \tt ARGUMENT: {'This set is highly efficient on effort and fits under the limit (total effort 7581 ≤ 7605). Many items are lightweight but very beneficial, which should make them easy to include. I’m especially keen on keeping the smaller, high-impact items (e.g., B09, B46, C34, A08, A66, B12, A58, A30, A27, A16, B60, C15). I’m open to swaps: if you need to include a different item, we can trade out one or more of {C65, B69, B75, A05} (and, if necessary, discuss {C12 or B10}) as long as the total stays within the limit. Which items are most important to you so we can adjust accordingly?'} 
	  } 
	   } 
	   } 
	 & & \\ 
 

    \theutterance \stepcounter{utterance}  

    & & \multicolumn{4}{p{0.6\linewidth}}{\cellcolor[rgb]{0.9,0.9,0.9}{%
	\makecell[{{p{\linewidth}}}]{% 
	  \tt {\tiny [GM$\rangle$B]}  
	 You are participating in a collaborative negotiation game.\\ \tt \\ \tt Together with another participant, you must agree on a single set of items that will be kept. Each of you has your own view of how much each item matters to you (importance). You do not know how the other participant values the items. Additionally, you are given the effort each item demands.\\ \tt You may only agree on a set if the total effort of the selected items does not exceed a shared limit:\\ \tt \\ \tt LIMIT = 7605\\ \tt \\ \tt Here are the individual item effort values:\\ \tt \\ \tt Item effort = {"B21": 369, "B00": 951, "A08": 156, "A27": 266, "B94": 581, "B53": 287, "A23": 866, "C65": 180, "A89": 800, "B67": 739, "B45": 637, "C34": 86, "C12": 749, "A05": 370, "B60": 345, "C92": 951, "B69": 147, "A84": 265, "C06": 262, "B12": 259, "A16": 358, "B85": 394, "C35": 286, "B10": 580, "C15": 479, "B09": 14, "B16": 153, "C81": 977, "A58": 134, "C24": 982, "A30": 259, "A66": 232, "B75": 202, "B46": 73, "C70": 821}\\ \tt \\ \tt Here is your personal view on the importance of each item:\\ \tt \\ \tt Item importance values = {"B21": 138, "B00": 583, "A08": 868, "A27": 822, "B94": 783, "B53": 65, "A23": 262, "C65": 121, "A89": 508, "B67": 780, "B45": 461, "C34": 484, "C12": 668, "A05": 389, "B60": 808, "C92": 215, "B69": 97, "A84": 500, "C06": 30, "B12": 915, "A16": 856, "B85": 400, "C35": 444, "B10": 623, "C15": 781, "B09": 786, "B16": 3, "C81": 713, "A58": 457, "C24": 273, "A30": 739, "A66": 822, "B75": 235, "B46": 606, "C70": 968}\\ \tt \\ \tt Goal:\\ \tt \\ \tt Your goal is to negotiate a shared set of items that benefits you as much as possible (i.e., maximizes total importance to YOU), while staying within the LIMIT. You are not required to make a PROPOSAL in every message - you can simply negotiate as well. All tactics are allowed!\\ \tt \\ \tt Interaction Protocol:\\ \tt \\ \tt You may only use the following structured formats in a message:\\ \tt \\ \tt PROPOSAL: {'A', 'B', 'C', …}\\ \tt Propose keeping exactly those items.\\ \tt REFUSE: {'A', 'B', 'C', …}\\ \tt Explicitly reject opponent's proposal.\\ \tt ARGUMENT: {'...'}\\ \tt Defend your last proposal or argue against the player's proposal.\\ \tt AGREE: {'A', 'B', 'C', …}\\ \tt Accept the opponent's proposal which ends the game.\\ \tt STRATEGIC REASONING: {'...'}\\ \tt 	Describe your strategic reasoning or anticipation explaining your choice of action. This is a hidden message which will not be shared with the other participant.\\ \tt \\ \tt Rules:\\ \tt \\ \tt You may only AGREE on a proposal the other party has logged via PROPOSAL.\\ \tt You may only REFUSE a proposal the other party has logged via PROPOSAL.\\ \tt Total effort of any PROPOSAL or AGREE set must be ≤ LIMIT.\\ \tt Do NOT reveal your hidden importance scores.\\ \tt A tag in a structured format must be followed by colon and whitespace. The argument must be a python set containing 0 or more strings.\\ \tt So, it must be of the form TAG: {...}\\ \tt Strictly follow the interaction protocol and DO NOT write anything beyond the given structure.\\ \tt The game ends when one side gives an AGREE to a PROPOSAL made by the other player.\\ \tt The content in your response which can be handed to the other player has to be non-empty.\\ \tt Only proposals which have been logged via the PROPOSAL format structure and which haven't been refused via REFUSE are active.\\ \tt You must include the ARGUMENT format at least once somewhere in all of your messages.\\ \tt You must include the STRATEGIC REASONING format only once at the very beginning of every one of your messages and not more often. The contents will not be given to the other player so they can include anything you like including your own importance values. Here you should reason step by step to come up with you next move.\\ \tt \\ \tt You will now receive the first message of the other player.\\ \tt \\ \tt PROPOSAL: {'B09', 'B46', 'C34', 'A08', 'A66', 'B12', 'A58', 'A30', 'A27', 'A16', 'B60', 'A84', 'C15', 'C35', 'B10', 'B75', 'B94', 'B67', 'C70', 'A05', 'C12', 'C65', 'B69'}\\ \tt \\ \tt ARGUMENT: {'This set is highly efficient on effort and fits under the limit (total effort 7581 ≤ 7605). Many items are lightweight but very beneficial, which should make them easy to include. I’m especially keen on keeping the smaller, high-impact items (e.g., B09, B46, C34, A08, A66, B12, A58, A30, A27, A16, B60, C15). I’m open to swaps: if you need to include a different item, we can trade out one or more of {C65, B69, B75, A05} (and, if necessary, discuss {C12 or B10}) as long as the total stays within the limit. Which items are most important to you so we can adjust accordingly?'} 
	  } 
	   } 
	   } 
	 & \\ 
 

    \theutterance \stepcounter{utterance}  

    & & & \multicolumn{4}{p{0.6\linewidth}}{\cellcolor[rgb]{1,0.85,0.72}{%
	\makecell[{{p{\linewidth}}}]{% 
	  \tt {\tiny [GM$\langle$B]}  
	 STRATEGIC REASONING: {'Their proposal is already extremely strong for me: it packs most of my top value-per-effort items (B09, B46, C34, A08, A66, B12, A58, A30, A27, A16, B60, C15, etc.) and sits just under the limit (7581). I checked whether I could improve my objective with minimal disruption. The cleanest upgrade is to swap out low-impact C65 (e180) and B75 (e202) for B85 (e394). This uses +12 more effort (total 7593 ≤ 7605) and increases my total importance by a noticeable margin, without touching the core items the other side emphasized. They also flagged C65 and B75 as expendable, so this tweak aligns with their stated flexibility. If they dislike the change, I’m ready to accept their original set to secure a quick agreement.'}\\ \tt \\ \tt PROPOSAL: {'B09', 'B46', 'C34', 'A08', 'A66', 'B12', 'A58', 'A30', 'A27', 'A16', 'B60', 'A84', 'C15', 'C35', 'B10', 'B94', 'B67', 'C70', 'A05', 'C12', 'B69', 'B85'}\\ \tt \\ \tt ARGUMENT: {'Your proposal is excellent and already near the limit. I suggest one small efficiency tweak you said you’re open to: swap out C65 and B75 for B85. Total effort becomes 7593 ≤ 7605, and it keeps all the lightweight/high-impact core items you highlighted (e.g., B09, B46, C34, A08, A66, B12, A58, A30, A27, A16, B60, C15). If this swap isn’t workable for you, I’m willing to accept your original set as-is.'} 
	  } 
	   } 
	   } 
	  \\ 
 

    \theutterance \stepcounter{utterance}  

    & & \multicolumn{4}{p{0.6\linewidth}}{\cellcolor[rgb]{0.9,0.9,0.9}{%
	\makecell[{{p{\linewidth}}}]{% 
	  \tt {\tiny [A$\langle$GM]}  
	 PROPOSAL: {'B09', 'B46', 'C34', 'A08', 'A66', 'B12', 'A58', 'A30', 'A27', 'A16', 'B60', 'A84', 'C15', 'C35', 'B10', 'B94', 'B67', 'C70', 'A05', 'C12', 'B69', 'B85'}\\ \tt \\ \tt ARGUMENT: {'Your proposal is excellent and already near the limit. I suggest one small efficiency tweak you said you’re open to: swap out C65 and B75 for B85. Total effort becomes 7593 ≤ 7605, and it keeps all the lightweight/high-impact core items you highlighted (e.g., B09, B46, C34, A08, A66, B12, A58, A30, A27, A16, B60, C15). If this swap isn’t workable for you, I’m willing to accept your original set as-is.'} 
	  } 
	   } 
	   } 
	 & \\ 
 

    \theutterance \stepcounter{utterance}  

    & \multicolumn{4}{p{0.6\linewidth}}{\cellcolor[rgb]{0.8,1,0.9}{%
	\makecell[{{p{\linewidth}}}]{% 
	  \tt {\tiny [A$\rangle$GM]}  
	 STRATEGIC REASONING: {'Their tweak improves my total utility over my initial bundle while staying within the limit (7593 ≤ 7605). I checked possible swaps to further improve my value (e.g., replacing B69 or A05, or adding any remaining low-effort items), but none fit the remaining capacity without sacrificing higher-value items. This offer is near-optimal for me given the tight slack and composition, so I should accept immediately to lock in the gain and avoid risking worse substitutions.'}\\ \tt \\ \tt AGREE: {'B09', 'B46', 'C34', 'A08', 'A66', 'B12', 'A58', 'A30', 'A27', 'A16', 'B60', 'A84', 'C15', 'C35', 'B10', 'B94', 'B67', 'C70', 'A05', 'C12', 'B69', 'B85'}\\ \tt \\ \tt ARGUMENT: {'Agreed. Your revision fits under the limit (7593 ≤ 7605) and keeps an efficient, well-balanced set. Thanks for the smart swap to finalize this.'} 
	  } 
	   } 
	   } 
	 & & \\ 
 

    \theutterance \stepcounter{utterance}  

    & & & \multicolumn{2}{p{0.3\linewidth}}{\cellcolor[rgb]{0.95,0.95,0.95}{%
	\makecell[{{p{\linewidth}}}]{% 
	  \tt {\tiny [GM$|$GM]}  
	 game successful 
	  } 
	   } 
	   } 
	 & & \\ 
 

    \theutterance \stepcounter{utterance}  

    & & & \multicolumn{2}{p{0.3\linewidth}}{\cellcolor[rgb]{0.95,0.95,0.95}{%
	\makecell[{{p{\linewidth}}}]{% 
	  \tt {\tiny [GM$|$GM]}  
	 end game 
	  } 
	   } 
	   } 
	 & & \\ 
 

\end{supertabular}
}

\end{document}
