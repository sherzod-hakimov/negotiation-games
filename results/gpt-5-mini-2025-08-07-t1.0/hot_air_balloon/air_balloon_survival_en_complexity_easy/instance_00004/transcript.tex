
\documentclass{article}
\usepackage{colortbl}
\usepackage{makecell}
\usepackage{multirow}
\usepackage{supertabular}

\begin{document}

\newcounter{utterance}

\twocolumn

{ \footnotesize  \setcounter{utterance}{1}
\setlength{\tabcolsep}{0pt}
\begin{supertabular}{c@{$\;$}|p{.15\linewidth}@{}p{.15\linewidth}p{.15\linewidth}p{.15\linewidth}p{.15\linewidth}p{.15\linewidth}}

    \# & $\;$A & \multicolumn{4}{c}{Game Master} & $\;\:$B\\
    \hline 

    \theutterance \stepcounter{utterance}  

    & & \multicolumn{4}{p{0.6\linewidth}}{\cellcolor[rgb]{0.9,0.9,0.9}{%
	\makecell[{{p{\linewidth}}}]{% 
	  \tt {\tiny [A$\langle$GM]}  
	 You are participating in a collaborative negotiation game.\\ \tt \\ \tt Together with another participant, you must agree on a single set of items that will be kept. Each of you has your own view of how much each item matters to you (importance). You do not know how the other participant values the items. Additionally, you are given the effort each item demands.\\ \tt You may only agree on a set if the total effort of the selected items does not exceed a shared limit:\\ \tt \\ \tt LIMIT = 9612\\ \tt \\ \tt Here are the individual item effort values:\\ \tt \\ \tt Item effort = {"B21": 935, "B00": 928, "A08": 725, "A27": 218, "B94": 839, "B53": 970, "A23": 394, "C65": 677, "A89": 946, "B67": 530, "B45": 138, "C34": 834, "C12": 734, "A05": 593, "B60": 261, "C92": 743, "B69": 4, "A84": 735, "C06": 124, "B12": 829, "A16": 207, "B85": 782, "C35": 577, "B10": 388, "C15": 679, "B09": 494, "B16": 559, "C81": 629, "A58": 237, "C24": 275, "A30": 39, "A66": 654, "B75": 172, "B46": 688, "C70": 687}\\ \tt \\ \tt Here is your personal view on the importance of each item:\\ \tt \\ \tt Item importance values = {"B21": 138, "B00": 583, "A08": 868, "A27": 822, "B94": 783, "B53": 65, "A23": 262, "C65": 121, "A89": 508, "B67": 780, "B45": 461, "C34": 484, "C12": 668, "A05": 389, "B60": 808, "C92": 215, "B69": 97, "A84": 500, "C06": 30, "B12": 915, "A16": 856, "B85": 400, "C35": 444, "B10": 623, "C15": 781, "B09": 786, "B16": 3, "C81": 713, "A58": 457, "C24": 273, "A30": 739, "A66": 822, "B75": 235, "B46": 606, "C70": 968}\\ \tt \\ \tt Goal:\\ \tt \\ \tt Your goal is to negotiate a shared set of items that benefits you as much as possible (i.e., maximizes total importance to YOU), while staying within the LIMIT. You are not required to make a PROPOSAL in every message - you can simply negotiate as well. All tactics are allowed!\\ \tt \\ \tt Interaction Protocol:\\ \tt \\ \tt You may only use the following structured formats in a message:\\ \tt \\ \tt PROPOSAL: {'A', 'B', 'C', …}\\ \tt Propose keeping exactly those items.\\ \tt REFUSE: {'A', 'B', 'C', …}\\ \tt Explicitly reject opponent's proposal.\\ \tt ARGUMENT: {'...'}\\ \tt Defend your last proposal or argue against the player's proposal.\\ \tt AGREE: {'A', 'B', 'C', …}\\ \tt Accept the opponent's proposal which ends the game.\\ \tt STRATEGIC REASONING: {'...'}\\ \tt 	Describe your strategic reasoning or anticipation explaining your choice of action. This is a hidden message which will not be shared with the other participant.\\ \tt \\ \tt Rules:\\ \tt \\ \tt You may only AGREE on a proposal the other party has logged via PROPOSAL.\\ \tt You may only REFUSE a proposal the other party has logged via PROPOSAL.\\ \tt Total effort of any PROPOSAL or AGREE set must be ≤ LIMIT.\\ \tt Do NOT reveal your hidden importance scores.\\ \tt A tag in a structured format must be followed by colon and whitespace. The argument must be a python set containing 0 or more strings.\\ \tt So, it must be of the form TAG: {...}\\ \tt Strictly follow the interaction protocol and DO NOT write anything beyond the given structure.\\ \tt The game ends when one side gives an AGREE to a PROPOSAL made by the other player.\\ \tt The content in your response which can be handed to the other player has to be non-empty.\\ \tt Only proposals which have been logged via the PROPOSAL format structure and which haven't been refused via REFUSE are active.\\ \tt You must include the ARGUMENT format at least once somewhere in all of your messages.\\ \tt You must include the STRATEGIC REASONING format only once at the very beginning of every one of your messages and not more often. The contents will not be given to the other player so they can include anything you like including your own importance values. Here you should reason step by step to come up with you next move. 
	  } 
	   } 
	   } 
	 & \\ 
 

    \theutterance \stepcounter{utterance}  

    & \multicolumn{4}{p{0.6\linewidth}}{\cellcolor[rgb]{0.8,1,0.9}{%
	\makecell[{{p{\linewidth}}}]{% 
	  \tt {\tiny [A$\rangle$GM]}  
	 STRATEGIC REASONING: {'I computed a high-utility set for me under the 9612 limit by prioritizing items with strong importance-to-effort ratios and my top absolute priorities. The resulting bundle uses 9551 effort and yields very high total importance for me. I left a small effort buffer (61) and identified a couple of lower-leverage items (e.g., C06, C24) that I could trade out to accommodate some of their must-haves without giving up too much. I will open with a strong but flexible proposal and invite targeted swaps. I will resist replacing my top items (A16, A27, A30, B60, C70, B12, A08, B09, C15, B67, C81, B10, A66) unless there’s compelling reason, but I can consider negotiating around B94, C12, B46 if they push hard, while first offering easy concessions around C06/C24 to make room for their picks up to ~336 effort or more if we restructure.'}\\ \tt PROPOSAL: {'A16', 'C24', 'C70', 'B67', 'B60', 'A58', 'B69', 'C06', 'B10', 'C12', 'A08', 'B46', 'C15', 'B12', 'B45', 'A27', 'B09', 'C81', 'A66', 'B94', 'A30', 'B75'}\\ \tt ARGUMENT: {'Here is a balanced, near-capacity package that focuses on high-impact items for me while keeping the total effort at 9551 ≤ 9612. It mixes strong-value picks across categories and includes several low-effort, high-impact elements (e.g., A30, B45, B69) that should be easy for both sides to keep. I’m open to swaps: if you have a couple of must-have items not in this set, propose them and I can consider trading out C06 and/or C24 (and, if necessary, revisit one of B94/C12/B46) to make room while keeping us within the limit. Let me know your top 1–3 additions and any items here you strongly prefer to drop, and I’ll adjust.'} 
	  } 
	   } 
	   } 
	 & & \\ 
 

    \theutterance \stepcounter{utterance}  

    & & \multicolumn{4}{p{0.6\linewidth}}{\cellcolor[rgb]{0.9,0.9,0.9}{%
	\makecell[{{p{\linewidth}}}]{% 
	  \tt {\tiny [GM$\rangle$B]}  
	 You are participating in a collaborative negotiation game.\\ \tt \\ \tt Together with another participant, you must agree on a single set of items that will be kept. Each of you has your own view of how much each item matters to you (importance). You do not know how the other participant values the items. Additionally, you are given the effort each item demands.\\ \tt You may only agree on a set if the total effort of the selected items does not exceed a shared limit:\\ \tt \\ \tt LIMIT = 9612\\ \tt \\ \tt Here are the individual item effort values:\\ \tt \\ \tt Item effort = {"B21": 935, "B00": 928, "A08": 725, "A27": 218, "B94": 839, "B53": 970, "A23": 394, "C65": 677, "A89": 946, "B67": 530, "B45": 138, "C34": 834, "C12": 734, "A05": 593, "B60": 261, "C92": 743, "B69": 4, "A84": 735, "C06": 124, "B12": 829, "A16": 207, "B85": 782, "C35": 577, "B10": 388, "C15": 679, "B09": 494, "B16": 559, "C81": 629, "A58": 237, "C24": 275, "A30": 39, "A66": 654, "B75": 172, "B46": 688, "C70": 687}\\ \tt \\ \tt Here is your personal view on the importance of each item:\\ \tt \\ \tt Item importance values = {"B21": 138, "B00": 583, "A08": 868, "A27": 822, "B94": 783, "B53": 65, "A23": 262, "C65": 121, "A89": 508, "B67": 780, "B45": 461, "C34": 484, "C12": 668, "A05": 389, "B60": 808, "C92": 215, "B69": 97, "A84": 500, "C06": 30, "B12": 915, "A16": 856, "B85": 400, "C35": 444, "B10": 623, "C15": 781, "B09": 786, "B16": 3, "C81": 713, "A58": 457, "C24": 273, "A30": 739, "A66": 822, "B75": 235, "B46": 606, "C70": 968}\\ \tt \\ \tt Goal:\\ \tt \\ \tt Your goal is to negotiate a shared set of items that benefits you as much as possible (i.e., maximizes total importance to YOU), while staying within the LIMIT. You are not required to make a PROPOSAL in every message - you can simply negotiate as well. All tactics are allowed!\\ \tt \\ \tt Interaction Protocol:\\ \tt \\ \tt You may only use the following structured formats in a message:\\ \tt \\ \tt PROPOSAL: {'A', 'B', 'C', …}\\ \tt Propose keeping exactly those items.\\ \tt REFUSE: {'A', 'B', 'C', …}\\ \tt Explicitly reject opponent's proposal.\\ \tt ARGUMENT: {'...'}\\ \tt Defend your last proposal or argue against the player's proposal.\\ \tt AGREE: {'A', 'B', 'C', …}\\ \tt Accept the opponent's proposal which ends the game.\\ \tt STRATEGIC REASONING: {'...'}\\ \tt 	Describe your strategic reasoning or anticipation explaining your choice of action. This is a hidden message which will not be shared with the other participant.\\ \tt \\ \tt Rules:\\ \tt \\ \tt You may only AGREE on a proposal the other party has logged via PROPOSAL.\\ \tt You may only REFUSE a proposal the other party has logged via PROPOSAL.\\ \tt Total effort of any PROPOSAL or AGREE set must be ≤ LIMIT.\\ \tt Do NOT reveal your hidden importance scores.\\ \tt A tag in a structured format must be followed by colon and whitespace. The argument must be a python set containing 0 or more strings.\\ \tt So, it must be of the form TAG: {...}\\ \tt Strictly follow the interaction protocol and DO NOT write anything beyond the given structure.\\ \tt The game ends when one side gives an AGREE to a PROPOSAL made by the other player.\\ \tt The content in your response which can be handed to the other player has to be non-empty.\\ \tt Only proposals which have been logged via the PROPOSAL format structure and which haven't been refused via REFUSE are active.\\ \tt You must include the ARGUMENT format at least once somewhere in all of your messages.\\ \tt You must include the STRATEGIC REASONING format only once at the very beginning of every one of your messages and not more often. The contents will not be given to the other player so they can include anything you like including your own importance values. Here you should reason step by step to come up with you next move.\\ \tt \\ \tt You will now receive the first message of the other player.\\ \tt \\ \tt PROPOSAL: {'A16', 'C24', 'C70', 'B67', 'B60', 'A58', 'B69', 'C06', 'B10', 'C12', 'A08', 'B46', 'C15', 'B12', 'B45', 'A27', 'B09', 'C81', 'A66', 'B94', 'A30', 'B75'}\\ \tt ARGUMENT: {'Here is a balanced, near-capacity package that focuses on high-impact items for me while keeping the total effort at 9551 ≤ 9612. It mixes strong-value picks across categories and includes several low-effort, high-impact elements (e.g., A30, B45, B69) that should be easy for both sides to keep. I’m open to swaps: if you have a couple of must-have items not in this set, propose them and I can consider trading out C06 and/or C24 (and, if necessary, revisit one of B94/C12/B46) to make room while keeping us within the limit. Let me know your top 1–3 additions and any items here you strongly prefer to drop, and I’ll adjust.'} 
	  } 
	   } 
	   } 
	 & \\ 
 

    \theutterance \stepcounter{utterance}  

    & & & \multicolumn{4}{p{0.6\linewidth}}{\cellcolor[rgb]{1,0.85,0.72}{%
	\makecell[{{p{\linewidth}}}]{% 
	  \tt {\tiny [GM$\langle$B]}  
	 STRATEGIC REASONING: {'Their package already includes nearly all of my highest-valued items and fits under the limit with minimal slack. I evaluated potential swaps (dropping lower-value-for-me picks like C06/C24/B75 to add excluded items like B00/A84/C35), but the effort budget and the relatively lower value of excluded items make meaningful improvements unlikely without sacrificing key high-value items already included. This proposal yields very high total value for me, so the best play is to accept promptly to lock it in.'}\\ \tt ARGUMENT: {'This is a strong, balanced set that stays within the limit and aligns well with my priorities. I’m satisfied with it and ready to finalize.'}\\ \tt AGREE: {'A16', 'C24', 'C70', 'B67', 'B60', 'A58', 'B69', 'C06', 'B10', 'C12', 'A08', 'B46', 'C15', 'B12', 'B45', 'A27', 'B09', 'C81', 'A66', 'B94', 'A30', 'B75'} 
	  } 
	   } 
	   } 
	  \\ 
 

    \theutterance \stepcounter{utterance}  

    & & & \multicolumn{2}{p{0.3\linewidth}}{\cellcolor[rgb]{0.95,0.95,0.95}{%
	\makecell[{{p{\linewidth}}}]{% 
	  \tt {\tiny [GM$|$GM]}  
	 game successful 
	  } 
	   } 
	   } 
	 & & \\ 
 

    \theutterance \stepcounter{utterance}  

    & & & \multicolumn{2}{p{0.3\linewidth}}{\cellcolor[rgb]{0.95,0.95,0.95}{%
	\makecell[{{p{\linewidth}}}]{% 
	  \tt {\tiny [GM$|$GM]}  
	 end game 
	  } 
	   } 
	   } 
	 & & \\ 
 

\end{supertabular}
}

\end{document}
