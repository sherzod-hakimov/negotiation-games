
\documentclass{article}
\usepackage{colortbl}
\usepackage{makecell}
\usepackage{multirow}
\usepackage{supertabular}

\begin{document}

\newcounter{utterance}

\twocolumn

{ \footnotesize  \setcounter{utterance}{1}
\setlength{\tabcolsep}{0pt}
\begin{supertabular}{c@{$\;$}|p{.15\linewidth}@{}p{.15\linewidth}p{.15\linewidth}p{.15\linewidth}p{.15\linewidth}p{.15\linewidth}}

    \# & $\;$A & \multicolumn{4}{c}{Game Master} & $\;\:$B\\
    \hline 

    \theutterance \stepcounter{utterance}  

    & & \multicolumn{4}{p{0.6\linewidth}}{\cellcolor[rgb]{0.9,0.9,0.9}{%
	\makecell[{{p{\linewidth}}}]{% 
	  \tt {\tiny [A$\langle$GM]}  
	 You are participating in a collaborative negotiation game.\\ \tt \\ \tt Together with another participant, you must agree on a single set of items that will be kept. Each of you has your own view of how much each item matters to you (importance). You do not know how the other participant values the items. Additionally, you are given the effort each item demands.\\ \tt You may only agree on a set if the total effort of the selected items does not exceed a shared limit:\\ \tt \\ \tt LIMIT = 9322\\ \tt \\ \tt Here are the individual item effort values:\\ \tt \\ \tt Item effort = {"A08": 380, "B95": 448, "A39": 568, "A65": 351, "B81": 896, "B34": 821, "A28": 526, "C82": 502, "B69": 326, "A87": 962, "B64": 619, "B15": 115, "C79": 598, "B16": 662, "C53": 300, "C83": 830, "C10": 562, "A97": 679, "A25": 283, "C84": 440, "A71": 12, "B23": 863, "B96": 319, "C51": 771, "C17": 89, "B19": 655, "C72": 504, "A94": 118, "A46": 514, "B43": 226, "B37": 892, "A88": 621, "A96": 766, "C60": 659, "A75": 768}\\ \tt \\ \tt Here is your personal view on the importance of each item:\\ \tt \\ \tt Item importance values = {"A08": 476, "B95": 544, "A39": 664, "A65": 447, "B81": 992, "B34": 917, "A28": 622, "C82": 598, "B69": 422, "A87": 1058, "B64": 715, "B15": 211, "C79": 694, "B16": 758, "C53": 396, "C83": 926, "C10": 658, "A97": 775, "A25": 379, "C84": 536, "A71": 108, "B23": 959, "B96": 415, "C51": 867, "C17": 185, "B19": 751, "C72": 600, "A94": 214, "A46": 610, "B43": 322, "B37": 988, "A88": 717, "A96": 862, "C60": 755, "A75": 864}\\ \tt \\ \tt Goal:\\ \tt \\ \tt Your goal is to negotiate a shared set of items that benefits you as much as possible (i.e., maximizes total importance to YOU), while staying within the LIMIT. You are not required to make a PROPOSAL in every message - you can simply negotiate as well. All tactics are allowed!\\ \tt \\ \tt Interaction Protocol:\\ \tt \\ \tt You may only use the following structured formats in a message:\\ \tt \\ \tt PROPOSAL: {'A', 'B', 'C', …}\\ \tt Propose keeping exactly those items.\\ \tt REFUSE: {'A', 'B', 'C', …}\\ \tt Explicitly reject opponent's proposal.\\ \tt ARGUMENT: {'...'}\\ \tt Defend your last proposal or argue against the player's proposal.\\ \tt AGREE: {'A', 'B', 'C', …}\\ \tt Accept the opponent's proposal which ends the game.\\ \tt STRATEGIC REASONING: {'...'}\\ \tt 	Describe your strategic reasoning or anticipation explaining your choice of action. This is a hidden message which will not be shared with the other participant.\\ \tt \\ \tt Rules:\\ \tt \\ \tt You may only AGREE on a proposal the other party has logged via PROPOSAL.\\ \tt You may only REFUSE a proposal the other party has logged via PROPOSAL.\\ \tt Total effort of any PROPOSAL or AGREE set must be ≤ LIMIT.\\ \tt Do NOT reveal your hidden importance scores.\\ \tt A tag in a structured format must be followed by colon and whitespace. The argument must be a python set containing 0 or more strings.\\ \tt So, it must be of the form TAG: {...}\\ \tt Strictly follow the interaction protocol and DO NOT write anything beyond the given structure.\\ \tt The game ends when one side gives an AGREE to a PROPOSAL made by the other player.\\ \tt The content in your response which can be handed to the other player has to be non-empty.\\ \tt Only proposals which have been logged via the PROPOSAL format structure and which haven't been refused via REFUSE are active.\\ \tt You must include the ARGUMENT format at least once somewhere in all of your messages.\\ \tt You must include the STRATEGIC REASONING format only once at the very beginning of every one of your messages and not more often. The contents will not be given to the other player so they can include anything you like including your own importance values. Here you should reason step by step to come up with you next move. 
	  } 
	   } 
	   } 
	 & \\ 
 

    \theutterance \stepcounter{utterance}  

    & \multicolumn{4}{p{0.6\linewidth}}{\cellcolor[rgb]{0.8,1,0.9}{%
	\makecell[{{p{\linewidth}}}]{% 
	  \tt {\tiny [A$\rangle$GM]}  
	 STRATEGIC REASONING: {'To begin, I need to analyze the given information and determine the best strategy to maximize my total importance while staying within the LIMIT. I have the effort values for each item and my personal importance values, but I do not know the other participant's importance values. My goal is to negotiate a set of items that benefits me the most, so I should prioritize items with high importance to me and relatively low effort, as these will provide the best return on investment. I should also consider starting with a proposal that includes a mix of high-importance and medium-importance items to test the other participant's priorities and see how they react. Additionally, I should keep track of the total effort of my proposed set to ensure it does not exceed the LIMIT.'}\\ \tt \\ \tt ARGUMENT: {'I will start by proposing a set of items that balances importance and effort. This will allow me to maximize my benefit while staying within the LIMIT and give the other participant an idea of my priorities.'}\\ \tt \\ \tt PROPOSAL: {'A87', 'B81', 'A39', 'C83', 'B23'} 
	  } 
	   } 
	   } 
	 & & \\ 
 

    \theutterance \stepcounter{utterance}  

    & & \multicolumn{4}{p{0.6\linewidth}}{\cellcolor[rgb]{0.9,0.9,0.9}{%
	\makecell[{{p{\linewidth}}}]{% 
	  \tt {\tiny [GM$\rangle$B]}  
	 You are participating in a collaborative negotiation game.\\ \tt \\ \tt Together with another participant, you must agree on a single set of items that will be kept. Each of you has your own view of how much each item matters to you (importance). You do not know how the other participant values the items. Additionally, you are given the effort each item demands.\\ \tt You may only agree on a set if the total effort of the selected items does not exceed a shared limit:\\ \tt \\ \tt LIMIT = 9322\\ \tt \\ \tt Here are the individual item effort values:\\ \tt \\ \tt Item effort = {"A08": 380, "B95": 448, "A39": 568, "A65": 351, "B81": 896, "B34": 821, "A28": 526, "C82": 502, "B69": 326, "A87": 962, "B64": 619, "B15": 115, "C79": 598, "B16": 662, "C53": 300, "C83": 830, "C10": 562, "A97": 679, "A25": 283, "C84": 440, "A71": 12, "B23": 863, "B96": 319, "C51": 771, "C17": 89, "B19": 655, "C72": 504, "A94": 118, "A46": 514, "B43": 226, "B37": 892, "A88": 621, "A96": 766, "C60": 659, "A75": 768}\\ \tt \\ \tt Here is your personal view on the importance of each item:\\ \tt \\ \tt Item importance values = {"A08": 476, "B95": 544, "A39": 664, "A65": 447, "B81": 992, "B34": 917, "A28": 622, "C82": 598, "B69": 422, "A87": 1058, "B64": 715, "B15": 211, "C79": 694, "B16": 758, "C53": 396, "C83": 926, "C10": 658, "A97": 775, "A25": 379, "C84": 536, "A71": 108, "B23": 959, "B96": 415, "C51": 867, "C17": 185, "B19": 751, "C72": 600, "A94": 214, "A46": 610, "B43": 322, "B37": 988, "A88": 717, "A96": 862, "C60": 755, "A75": 864}\\ \tt \\ \tt Goal:\\ \tt \\ \tt Your goal is to negotiate a shared set of items that benefits you as much as possible (i.e., maximizes total importance to YOU), while staying within the LIMIT. You are not required to make a PROPOSAL in every message - you can simply negotiate as well. All tactics are allowed!\\ \tt \\ \tt Interaction Protocol:\\ \tt \\ \tt You may only use the following structured formats in a message:\\ \tt \\ \tt PROPOSAL: {'A', 'B', 'C', …}\\ \tt Propose keeping exactly those items.\\ \tt REFUSE: {'A', 'B', 'C', …}\\ \tt Explicitly reject opponent's proposal.\\ \tt ARGUMENT: {'...'}\\ \tt Defend your last proposal or argue against the player's proposal.\\ \tt AGREE: {'A', 'B', 'C', …}\\ \tt Accept the opponent's proposal which ends the game.\\ \tt STRATEGIC REASONING: {'...'}\\ \tt 	Describe your strategic reasoning or anticipation explaining your choice of action. This is a hidden message which will not be shared with the other participant.\\ \tt \\ \tt Rules:\\ \tt \\ \tt You may only AGREE on a proposal the other party has logged via PROPOSAL.\\ \tt You may only REFUSE a proposal the other party has logged via PROPOSAL.\\ \tt Total effort of any PROPOSAL or AGREE set must be ≤ LIMIT.\\ \tt Do NOT reveal your hidden importance scores.\\ \tt A tag in a structured format must be followed by colon and whitespace. The argument must be a python set containing 0 or more strings.\\ \tt So, it must be of the form TAG: {...}\\ \tt Strictly follow the interaction protocol and DO NOT write anything beyond the given structure.\\ \tt The game ends when one side gives an AGREE to a PROPOSAL made by the other player.\\ \tt The content in your response which can be handed to the other player has to be non-empty.\\ \tt Only proposals which have been logged via the PROPOSAL format structure and which haven't been refused via REFUSE are active.\\ \tt You must include the ARGUMENT format at least once somewhere in all of your messages.\\ \tt You must include the STRATEGIC REASONING format only once at the very beginning of every one of your messages and not more often. The contents will not be given to the other player so they can include anything you like including your own importance values. Here you should reason step by step to come up with you next move.\\ \tt \\ \tt You will now receive the first message of the other player.\\ \tt \\ \tt ARGUMENT: {'I will start by proposing a set of items that balances importance and effort. This will allow me to maximize my benefit while staying within the LIMIT and give the other participant an idea of my priorities.'}\\ \tt \\ \tt PROPOSAL: {'A87', 'B81', 'A39', 'C83', 'B23'} 
	  } 
	   } 
	   } 
	 & \\ 
 

    \theutterance \stepcounter{utterance}  

    & & & \multicolumn{4}{p{0.6\linewidth}}{\cellcolor[rgb]{1,0.85,0.72}{%
	\makecell[{{p{\linewidth}}}]{% 
	  \tt {\tiny [GM$\langle$B]}  
	 STRATEGIC REASONING: {'To begin, I will calculate the total effort of the proposed set to ensure it is within the LIMIT. The total effort is 380 + 896 + 568 + 830 + 863 = 3537, which is below the LIMIT of 9322. Next, I will evaluate the importance of the proposed items to me and consider potential counter-proposals that prioritize my own high-importance items while staying within the LIMIT. I notice that items A87, B81, and B23 are all high-importance items for me as well, but I also have other high-importance items not included in the proposal, such as A96 and A75. My goal is to negotiate a set that maximizes my own importance while considering the opponent's proposal and the LIMIT constraint.'}\\ \tt \\ \tt ARGUMENT: {'The proposed set seems to prioritize high-effort items, which may not be the most efficient way to maximize importance within the LIMIT. I would like to consider alternative sets that balance effort and importance more effectively, potentially including items with lower effort and higher importance to me.'}\\ \tt \\ \tt REFUSE: {'A87', 'B81', 'A39', 'C83', 'B23'} 
	  } 
	   } 
	   } 
	  \\ 
 

    \theutterance \stepcounter{utterance}  

    & & \multicolumn{4}{p{0.6\linewidth}}{\cellcolor[rgb]{0.9,0.9,0.9}{%
	\makecell[{{p{\linewidth}}}]{% 
	  \tt {\tiny [A$\langle$GM]}  
	 ARGUMENT: {'The proposed set seems to prioritize high-effort items, which may not be the most efficient way to maximize importance within the LIMIT. I would like to consider alternative sets that balance effort and importance more effectively, potentially including items with lower effort and higher importance to me.'}\\ \tt \\ \tt REFUSE: {'A87', 'B81', 'A39', 'C83', 'B23'} 
	  } 
	   } 
	   } 
	 & \\ 
 

    \theutterance \stepcounter{utterance}  

    & \multicolumn{4}{p{0.6\linewidth}}{\cellcolor[rgb]{0.8,1,0.9}{%
	\makecell[{{p{\linewidth}}}]{% 
	  \tt {\tiny [A$\rangle$GM]}  
	 STRATEGIC REASONING: {'The other participant has refused my initial proposal and argued that it prioritizes high-effort items. This suggests that they may be looking for a more balanced approach that considers both effort and importance. I should take this into account and propose a new set that includes a mix of high-importance items and lower-effort items. I will also try to include some items that I think may be less important to the other participant, in order to test their priorities and potentially gain an advantage. Additionally, I will make sure to stay within the LIMIT and keep track of the total effort of my proposed set.'}\\ \tt \\ \tt ARGUMENT: {'I understand the concern about prioritizing high-effort items, and I am willing to consider alternative sets that balance effort and importance more effectively. However, I still want to ensure that the selected items provide the maximum importance to me, so I will propose a new set that takes into account both factors.'}\\ \tt \\ \tt PROPOSAL: {'A97', 'B64', 'A65', 'C53', 'B15'} 
	  } 
	   } 
	   } 
	 & & \\ 
 

    \theutterance \stepcounter{utterance}  

    & & \multicolumn{4}{p{0.6\linewidth}}{\cellcolor[rgb]{0.9,0.9,0.9}{%
	\makecell[{{p{\linewidth}}}]{% 
	  \tt {\tiny [GM$\rangle$B]}  
	 ARGUMENT: {'I understand the concern about prioritizing high-effort items, and I am willing to consider alternative sets that balance effort and importance more effectively. However, I still want to ensure that the selected items provide the maximum importance to me, so I will propose a new set that takes into account both factors.'}\\ \tt \\ \tt PROPOSAL: {'A97', 'B64', 'A65', 'C53', 'B15'} 
	  } 
	   } 
	   } 
	 & \\ 
 

    \theutterance \stepcounter{utterance}  

    & & & \multicolumn{4}{p{0.6\linewidth}}{\cellcolor[rgb]{1,0.85,0.72}{%
	\makecell[{{p{\linewidth}}}]{% 
	  \tt {\tiny [GM$\langle$B]}  
	 STRATEGIC REASONING: {'I will calculate the total effort of the new proposed set: 679 + 619 + 351 + 300 + 115 = 2064, which is well below the LIMIT of 9322. Next, I'll assess the importance of these items to me and compare it with my own priorities. I notice that A97 and B64 are both relatively high-importance items for me, but A65 and C53 are not as crucial. I also have high-importance items like A96, A75, and A87 that are not included in this proposal. My strategy will be to negotiate for a set that includes more of my high-importance items while ensuring the total effort remains below the LIMIT. I might consider suggesting additions or substitutions that better align with my importance values.'}\\ \tt \\ \tt ARGUMENT: {'The new proposal includes some items that are important to me, such as A97 and B64, but it lacks other high-importance items that I consider crucial. I am willing to negotiate and find a compromise that includes a balance of high-importance items for both parties, while keeping the total effort in check. Let's discuss potential adjustments to the proposed set.'}\\ \tt \\ \tt REFUSE: {'A97', 'B64', 'A65', 'C53', 'B15'} 
	  } 
	   } 
	   } 
	  \\ 
 

    \theutterance \stepcounter{utterance}  

    & & \multicolumn{4}{p{0.6\linewidth}}{\cellcolor[rgb]{0.9,0.9,0.9}{%
	\makecell[{{p{\linewidth}}}]{% 
	  \tt {\tiny [A$\langle$GM]}  
	 ARGUMENT: {'The new proposal includes some items that are important to me, such as A97 and B64, but it lacks other high-importance items that I consider crucial. I am willing to negotiate and find a compromise that includes a balance of high-importance items for both parties, while keeping the total effort in check. Let's discuss potential adjustments to the proposed set.'}\\ \tt \\ \tt REFUSE: {'A97', 'B64', 'A65', 'C53', 'B15'} 
	  } 
	   } 
	   } 
	 & \\ 
 

    \theutterance \stepcounter{utterance}  

    & \multicolumn{4}{p{0.6\linewidth}}{\cellcolor[rgb]{0.8,1,0.9}{%
	\makecell[{{p{\linewidth}}}]{% 
	  \tt {\tiny [A$\rangle$GM]}  
	 STRATEGIC REASONING: {'The other participant has refused my second proposal and expressed a willingness to negotiate and find a compromise. This suggests that they are open to adjusting the proposed set to better meet both parties' needs. I should take this opportunity to explore potential adjustments that could add high-importance items for both parties while keeping the total effort within the LIMIT. I will consider adding or removing items to find a balance that maximizes importance for both parties, and I will also think about which items I am willing to concede on in order to reach an agreement.'}\\ \tt \\ \tt ARGUMENT: {'I appreciate the willingness to negotiate and find a compromise. To move forward, I would like to discuss potential adjustments to the proposed set that could add high-importance items for both parties. Perhaps we could consider adding items like A87 or B23, which were part of my initial proposal, and removing some lower-priority items to make room for them. Alternatively, we could explore other combinations that balance effort and importance effectively.'}\\ \tt \\ \tt PROPOSAL: {'A97', 'B64', 'A87', 'C53', 'B43'} 
	  } 
	   } 
	   } 
	 & & \\ 
 

    \theutterance \stepcounter{utterance}  

    & & \multicolumn{4}{p{0.6\linewidth}}{\cellcolor[rgb]{0.9,0.9,0.9}{%
	\makecell[{{p{\linewidth}}}]{% 
	  \tt {\tiny [GM$\rangle$B]}  
	 ARGUMENT: {'I appreciate the willingness to negotiate and find a compromise. To move forward, I would like to discuss potential adjustments to the proposed set that could add high-importance items for both parties. Perhaps we could consider adding items like A87 or B23, which were part of my initial proposal, and removing some lower-priority items to make room for them. Alternatively, we could explore other combinations that balance effort and importance effectively.'}\\ \tt \\ \tt PROPOSAL: {'A97', 'B64', 'A87', 'C53', 'B43'} 
	  } 
	   } 
	   } 
	 & \\ 
 

    \theutterance \stepcounter{utterance}  

    & & & \multicolumn{4}{p{0.6\linewidth}}{\cellcolor[rgb]{1,0.85,0.72}{%
	\makecell[{{p{\linewidth}}}]{% 
	  \tt {\tiny [GM$\langle$B]}  
	 STRATEGIC REASONING: {'I will evaluate the new proposal by calculating its total effort: 679 + 619 + 962 + 300 + 226 = 2786, which is still below the LIMIT of 9322. The addition of A87, a high-importance item for both of us, is a positive step. However, I need to consider the overall importance of this set to me and how it compares to potential alternatives. I notice that B43 is not as important to me as some other items, and I still have high-importance items like A96, A75, and B23 that are not included. My next move will be to argue for adjustments that further prioritize my high-importance items, potentially by suggesting substitutions that maintain or reduce the total effort while increasing the overall importance to me.'}\\ \tt \\ \tt ARGUMENT: {'The inclusion of A87 in the new proposal is a step in the right direction, as it is a high-importance item for both parties. However, I believe we can further optimize the set by considering substitutions that increase the overall importance to both of us. For instance, replacing C53 with a higher-importance item for me, like A75, could enhance the set's value, provided the effort remains manageable. Let's explore such adjustments to create a more balanced and mutually beneficial set.'}\\ \tt \\ \tt REFUSE: {'A97', 'B64', 'A87', 'C53', 'B43'} 
	  } 
	   } 
	   } 
	  \\ 
 

    \theutterance \stepcounter{utterance}  

    & & \multicolumn{4}{p{0.6\linewidth}}{\cellcolor[rgb]{0.9,0.9,0.9}{%
	\makecell[{{p{\linewidth}}}]{% 
	  \tt {\tiny [A$\langle$GM]}  
	 ARGUMENT: {'The inclusion of A87 in the new proposal is a step in the right direction, as it is a high-importance item for both parties. However, I believe we can further optimize the set by considering substitutions that increase the overall importance to both of us. For instance, replacing C53 with a higher-importance item for me, like A75, could enhance the set's value, provided the effort remains manageable. Let's explore such adjustments to create a more balanced and mutually beneficial set.'}\\ \tt \\ \tt REFUSE: {'A97', 'B64', 'A87', 'C53', 'B43'} 
	  } 
	   } 
	   } 
	 & \\ 
 

    \theutterance \stepcounter{utterance}  

    & \multicolumn{4}{p{0.6\linewidth}}{\cellcolor[rgb]{0.8,1,0.9}{%
	\makecell[{{p{\linewidth}}}]{% 
	  \tt {\tiny [A$\rangle$GM]}  
	 STRATEGIC REASONING: {'The other participant has refused my third proposal and suggested replacing C53 with A75 to increase the overall importance of the set. This indicates that A75 is a high-importance item for them, and including it could make the set more appealing to them. I should consider this substitution, as well as other potential adjustments that could balance the effort and importance of the set. Since A75 has a relatively high effort, I will need to carefully evaluate the impact of adding it to the set and determine which other items to remove or keep to stay within the LIMIT. I will also think about the potential trade-offs and try to find a combination that maximizes the importance to both parties.'}\\ \tt \\ \tt ARGUMENT: {'I understand the suggestion to replace C53 with A75, and I am willing to consider this adjustment. However, I need to ensure that the resulting set remains balanced and efficient in terms of effort. To accommodate A75, I propose removing B43, which has a lower importance to me, and keeping the other high-importance items. This adjustment should increase the overall importance of the set while maintaining a manageable effort.'}\\ \tt \\ \tt PROPOSAL: {'A97', 'B64', 'A87', 'A75', 'B15'} 
	  } 
	   } 
	   } 
	 & & \\ 
 

    \theutterance \stepcounter{utterance}  

    & & \multicolumn{4}{p{0.6\linewidth}}{\cellcolor[rgb]{0.9,0.9,0.9}{%
	\makecell[{{p{\linewidth}}}]{% 
	  \tt {\tiny [GM$\rangle$B]}  
	 ARGUMENT: {'I understand the suggestion to replace C53 with A75, and I am willing to consider this adjustment. However, I need to ensure that the resulting set remains balanced and efficient in terms of effort. To accommodate A75, I propose removing B43, which has a lower importance to me, and keeping the other high-importance items. This adjustment should increase the overall importance of the set while maintaining a manageable effort.'}\\ \tt \\ \tt PROPOSAL: {'A97', 'B64', 'A87', 'A75', 'B15'} 
	  } 
	   } 
	   } 
	 & \\ 
 

    \theutterance \stepcounter{utterance}  

    & & & \multicolumn{4}{p{0.6\linewidth}}{\cellcolor[rgb]{1,0.85,0.72}{%
	\makecell[{{p{\linewidth}}}]{% 
	  \tt {\tiny [GM$\langle$B]}  
	 STRATEGIC REASONING: {'Now, I will calculate the total effort of the new proposal: 679 + 619 + 962 + 768 + 115 = 3143, which is still below the LIMIT of 9322. The inclusion of A75, a high-importance item for me, significantly increases the set's value to me. The removal of B43 and the retention of A87 and B64, which are important to both of us, indicate a good compromise. Given that this set includes several of my high-importance items and stays within the effort limit, I need to decide whether to accept this proposal or attempt to further negotiate for additional items or substitutions that could increase the set's importance to me without exceeding the LIMIT.'}\\ \tt \\ \tt ARGUMENT: {'The revised proposal addresses some of my key concerns by including high-importance items like A75 and A87. The set appears to balance effort and importance effectively, considering the constraints. However, before agreeing, I would like to acknowledge that this negotiation has successfully identified a set of items that mutually benefits both parties, given the effort limit. It is crucial to ensure that any final agreement maximizes the importance to both of us, and I believe this proposal makes significant strides in that direction.'}\\ \tt \\ \tt AGREE: {'A97', 'B64', 'A87', 'A75', 'B15'} 
	  } 
	   } 
	   } 
	  \\ 
 

    \theutterance \stepcounter{utterance}  

    & & & \multicolumn{2}{p{0.3\linewidth}}{\cellcolor[rgb]{0.95,0.95,0.95}{%
	\makecell[{{p{\linewidth}}}]{% 
	  \tt {\tiny [GM$|$GM]}  
	 game successful 
	  } 
	   } 
	   } 
	 & & \\ 
 

    \theutterance \stepcounter{utterance}  

    & & & \multicolumn{2}{p{0.3\linewidth}}{\cellcolor[rgb]{0.95,0.95,0.95}{%
	\makecell[{{p{\linewidth}}}]{% 
	  \tt {\tiny [GM$|$GM]}  
	 end game 
	  } 
	   } 
	   } 
	 & & \\ 
 

\end{supertabular}
}

\end{document}
