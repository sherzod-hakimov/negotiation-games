
\documentclass{article}
\usepackage{colortbl}
\usepackage{makecell}
\usepackage{multirow}
\usepackage{supertabular}

\begin{document}

\newcounter{utterance}

\twocolumn

{ \footnotesize  \setcounter{utterance}{1}
\setlength{\tabcolsep}{0pt}
\begin{supertabular}{c@{$\;$}|p{.15\linewidth}@{}p{.15\linewidth}p{.15\linewidth}p{.15\linewidth}p{.15\linewidth}p{.15\linewidth}}

    \# & $\;$A & \multicolumn{4}{c}{Game Master} & $\;\:$B\\
    \hline 

    \theutterance \stepcounter{utterance}  

    & & \multicolumn{4}{p{0.6\linewidth}}{\cellcolor[rgb]{0.9,0.9,0.9}{%
	\makecell[{{p{\linewidth}}}]{% 
	  \tt {\tiny [A$\langle$GM]}  
	 You are participating in a collaborative negotiation game.\\ \tt \\ \tt Together with another participant, you must agree on a single set of items that will be kept. Each of you has your own view of how much each item matters to you (importance). You do not know how the other participant values the items. Additionally, you are given the effort each item demands.\\ \tt You may only agree on a set if the total effort of the selected items does not exceed a shared limit:\\ \tt \\ \tt LIMIT = 3386\\ \tt \\ \tt Here are the individual item effort values:\\ \tt \\ \tt Item effort = {"C17": 313, "B41": 720, "A71": 175, "C00": 461, "B35": 635, "A77": 685, "A03": 544, "A04": 203, "C54": 369, "C01": 539, "A87": 4, "B32": 695, "B86": 399, "B33": 594, "A10": 437}\\ \tt \\ \tt Here is your personal view on the importance of each item:\\ \tt \\ \tt Item importance values = {"C17": 138, "B41": 583, "A71": 65, "C00": 262, "B35": 121, "A77": 508, "A03": 461, "A04": 484, "C54": 668, "C01": 389, "A87": 215, "B32": 97, "B86": 500, "B33": 30, "A10": 400}\\ \tt \\ \tt Goal:\\ \tt \\ \tt Your goal is to negotiate a shared set of items that benefits you as much as possible (i.e., maximizes total importance to YOU), while staying within the LIMIT. You are not required to make a PROPOSAL in every message - you can simply negotiate as well. All tactics are allowed!\\ \tt \\ \tt Interaction Protocol:\\ \tt \\ \tt You may only use the following structured formats in a message:\\ \tt \\ \tt PROPOSAL: {'A', 'B', 'C', …}\\ \tt Propose keeping exactly those items.\\ \tt REFUSE: {'A', 'B', 'C', …}\\ \tt Explicitly reject opponent's proposal.\\ \tt ARGUMENT: {'...'}\\ \tt Defend your last proposal or argue against the player's proposal.\\ \tt AGREE: {'A', 'B', 'C', …}\\ \tt Accept the opponent's proposal which ends the game.\\ \tt STRATEGIC REASONING: {'...'}\\ \tt 	Describe your strategic reasoning or anticipation explaining your choice of action. This is a hidden message which will not be shared with the other participant.\\ \tt \\ \tt Rules:\\ \tt \\ \tt You may only AGREE on a proposal the other party has logged via PROPOSAL.\\ \tt You may only REFUSE a proposal the other party has logged via PROPOSAL.\\ \tt Total effort of any PROPOSAL or AGREE set must be ≤ LIMIT.\\ \tt Do NOT reveal your hidden importance scores.\\ \tt A tag in a structured format must be followed by colon and whitespace. The argument must be a python set containing 0 or more strings.\\ \tt So, it must be of the form TAG: {...}\\ \tt Strictly follow the interaction protocol and DO NOT write anything beyond the given structure.\\ \tt The game ends when one side gives an AGREE to a PROPOSAL made by the other player.\\ \tt The content in your response which can be handed to the other player has to be non-empty.\\ \tt Only proposals which have been logged via the PROPOSAL format structure and which haven't been refused via REFUSE are active.\\ \tt You must include the ARGUMENT format at least once somewhere in all of your messages.\\ \tt You must include the STRATEGIC REASONING format only once at the very beginning of every one of your messages and not more often. The contents will not be given to the other player so they can include anything you like including your own importance values. Here you should reason step by step to come up with you next move. 
	  } 
	   } 
	   } 
	 & \\ 
 

    \theutterance \stepcounter{utterance}  

    & \multicolumn{4}{p{0.6\linewidth}}{\cellcolor[rgb]{0.8,1,0.9}{%
	\makecell[{{p{\linewidth}}}]{% 
	  \tt {\tiny [A$\rangle$GM]}  
	 STRATEGIC REASONING: {'To start the negotiation, I will analyze the given data and try to find the most important items for me with the least effort. I will sort the items based on their importance-to-effort ratio and select the top items that fit within the LIMIT. Since I do not know the other participant\'s importance values, I will try to make a reasonable proposal that benefits me while leaving room for negotiation.'}\\ \tt \\ \tt ARGUMENT: {'I will start by proposing a set of items that I believe are the most valuable to me, while considering the effort limit. I will select items with high importance and relatively low effort, in order to maximize the total importance while staying within the LIMIT.'}\\ \tt \\ \tt PROPOSAL: {'C54', 'A04', 'A77', 'A03', 'C01', 'B41', 'B86', 'A10'} 
	  } 
	   } 
	   } 
	 & & \\ 
 

    \theutterance \stepcounter{utterance}  

    & & \multicolumn{4}{p{0.6\linewidth}}{\cellcolor[rgb]{0.9,0.9,0.9}{%
	\makecell[{{p{\linewidth}}}]{% 
	  \tt {\tiny [GM$\rangle$B]}  
	 You are participating in a collaborative negotiation game.\\ \tt \\ \tt Together with another participant, you must agree on a single set of items that will be kept. Each of you has your own view of how much each item matters to you (importance). You do not know how the other participant values the items. Additionally, you are given the effort each item demands.\\ \tt You may only agree on a set if the total effort of the selected items does not exceed a shared limit:\\ \tt \\ \tt LIMIT = 3386\\ \tt \\ \tt Here are the individual item effort values:\\ \tt \\ \tt Item effort = {"C17": 313, "B41": 720, "A71": 175, "C00": 461, "B35": 635, "A77": 685, "A03": 544, "A04": 203, "C54": 369, "C01": 539, "A87": 4, "B32": 695, "B86": 399, "B33": 594, "A10": 437}\\ \tt \\ \tt Here is your personal view on the importance of each item:\\ \tt \\ \tt Item importance values = {"C54": 30, "B41": 65, "A77": 97, "B86": 121, "A04": 138, "A03": 215, "A10": 262, "C01": 389, "C00": 400, "A87": 461, "C17": 484, "B35": 500, "B32": 508, "A71": 583, "B33": 668}\\ \tt \\ \tt Goal:\\ \tt \\ \tt Your goal is to negotiate a shared set of items that benefits you as much as possible (i.e., maximizes total importance to YOU), while staying within the LIMIT. You are not required to make a PROPOSAL in every message - you can simply negotiate as well. All tactics are allowed!\\ \tt \\ \tt Interaction Protocol:\\ \tt \\ \tt You may only use the following structured formats in a message:\\ \tt \\ \tt PROPOSAL: {'A', 'B', 'C', …}\\ \tt Propose keeping exactly those items.\\ \tt REFUSE: {'A', 'B', 'C', …}\\ \tt Explicitly reject opponent's proposal.\\ \tt ARGUMENT: {'...'}\\ \tt Defend your last proposal or argue against the player's proposal.\\ \tt AGREE: {'A', 'B', 'C', …}\\ \tt Accept the opponent's proposal which ends the game.\\ \tt STRATEGIC REASONING: {'...'}\\ \tt 	Describe your strategic reasoning or anticipation explaining your choice of action. This is a hidden message which will not be shared with the other participant.\\ \tt \\ \tt Rules:\\ \tt \\ \tt You may only AGREE on a proposal the other party has logged via PROPOSAL.\\ \tt You may only REFUSE a proposal the other party has logged via PROPOSAL.\\ \tt Total effort of any PROPOSAL or AGREE set must be ≤ LIMIT.\\ \tt Do NOT reveal your hidden importance scores.\\ \tt A tag in a structured format must be followed by colon and whitespace. The argument must be a python set containing 0 or more strings.\\ \tt So, it must be of the form TAG: {...}\\ \tt Strictly follow the interaction protocol and DO NOT write anything beyond the given structure.\\ \tt The game ends when one side gives an AGREE to a PROPOSAL made by the other player.\\ \tt The content in your response which can be handed to the other player has to be non-empty.\\ \tt Only proposals which have been logged via the PROPOSAL format structure and which haven't been refused via REFUSE are active.\\ \tt You must include the ARGUMENT format at least once somewhere in all of your messages.\\ \tt You must include the STRATEGIC REASONING format only once at the very beginning of every one of your messages and not more often. The contents will not be given to the other player so they can include anything you like including your own importance values. Here you should reason step by step to come up with you next move.\\ \tt \\ \tt You will now receive the first message of the other player.\\ \tt \\ \tt ARGUMENT: {'I will start by proposing a set of items that I believe are the most valuable to me, while considering the effort limit. I will select items with high importance and relatively low effort, in order to maximize the total importance while staying within the LIMIT.'}\\ \tt \\ \tt PROPOSAL: {'C54', 'A04', 'A77', 'A03', 'C01', 'B41', 'B86', 'A10'} 
	  } 
	   } 
	   } 
	 & \\ 
 

    \theutterance \stepcounter{utterance}  

    & & & \multicolumn{4}{p{0.6\linewidth}}{\cellcolor[rgb]{1,0.85,0.72}{%
	\makecell[{{p{\linewidth}}}]{% 
	  \tt {\tiny [GM$\langle$B]}  
	 STRATEGIC REASONING: {'To begin, I will calculate the total effort of the proposed set to ensure it is within the LIMIT. I will then evaluate the importance of each item in the proposal based on my own importance values. Next, I will consider potential alternatives that could increase the total importance to me while staying within the LIMIT. My goal is to maximize my own importance values, so I will look for items with high importance and relatively low effort. I will also consider the potential for negotiation and the possibility of adding or removing items to reach a mutually beneficial agreement.'}\\ \tt \\ \tt ARGUMENT: {'The proposed set includes a mix of items with varying importance and effort values. However, upon reviewing the proposal, I notice that it excludes some items that I consider highly important, such as C17 and B35. I will need to carefully evaluate the trade-offs between the proposed items and my own preferences to determine the best course of action.'}\\ \tt \\ \tt REFUSE: {'C54', 'A04', 'A77', 'A03', 'C01', 'B41', 'B86', 'A10'} 
	  } 
	   } 
	   } 
	  \\ 
 

    \theutterance \stepcounter{utterance}  

    & & \multicolumn{4}{p{0.6\linewidth}}{\cellcolor[rgb]{0.9,0.9,0.9}{%
	\makecell[{{p{\linewidth}}}]{% 
	  \tt {\tiny [A$\langle$GM]}  
	 ARGUMENT: {'The proposed set includes a mix of items with varying importance and effort values. However, upon reviewing the proposal, I notice that it excludes some items that I consider highly important, such as C17 and B35. I will need to carefully evaluate the trade-offs between the proposed items and my own preferences to determine the best course of action.'}\\ \tt \\ \tt REFUSE: {'C54', 'A04', 'A77', 'A03', 'C01', 'B41', 'B86', 'A10'} 
	  } 
	   } 
	   } 
	 & \\ 
 

    \theutterance \stepcounter{utterance}  

    & \multicolumn{4}{p{0.6\linewidth}}{\cellcolor[rgb]{0.8,1,0.9}{%
	\makecell[{{p{\linewidth}}}]{% 
	  \tt {\tiny [A$\rangle$GM]}  
	 STRATEGIC REASONING: {'The opponent has refused my initial proposal, indicating that they have different priorities. They mentioned the exclusion of C17 and B35, which suggests that these items are important to them. I will try to understand their perspective and find a compromise that includes some of their preferred items while still meeting my own goals. I will re-evaluate the importance and effort of each item and consider alternative combinations that could be more acceptable to both parties.'}\\ \tt \\ \tt ARGUMENT: {'I understand that the initial proposal did not meet the opponent's expectations, and I am willing to reconsider the selection of items. I will take into account the opponent's hints about C17 and B35 being important to them and try to find a balance between our preferences. Perhaps including some of their preferred items and removing others that are less important to me could lead to a mutually beneficial agreement.'}\\ \tt \\ \tt PROPOSAL: {'C17', 'B35', 'A04', 'A77', 'C01', 'B41', 'A10'} 
	  } 
	   } 
	   } 
	 & & \\ 
 

    \theutterance \stepcounter{utterance}  

    & & \multicolumn{4}{p{0.6\linewidth}}{\cellcolor[rgb]{0.9,0.9,0.9}{%
	\makecell[{{p{\linewidth}}}]{% 
	  \tt {\tiny [GM$\rangle$B]}  
	 ARGUMENT: {'I understand that the initial proposal did not meet the opponent's expectations, and I am willing to reconsider the selection of items. I will take into account the opponent's hints about C17 and B35 being important to them and try to find a balance between our preferences. Perhaps including some of their preferred items and removing others that are less important to me could lead to a mutually beneficial agreement.'}\\ \tt \\ \tt PROPOSAL: {'C17', 'B35', 'A04', 'A77', 'C01', 'B41', 'A10'} 
	  } 
	   } 
	   } 
	 & \\ 
 

    \theutterance \stepcounter{utterance}  

    & & & \multicolumn{4}{p{0.6\linewidth}}{\cellcolor[rgb]{1,0.85,0.72}{%
	\makecell[{{p{\linewidth}}}]{% 
	  \tt {\tiny [GM$\langle$B]}  
	 STRATEGIC REASONING: {'I will calculate the total effort of the new proposal to ensure it is within the LIMIT. Then, I will assess the importance of each item in the proposal based on my own importance values. I notice that the opponent has included C17 and B35, which are indeed highly important to me. However, I will also consider the items that were removed from the initial proposal and evaluate whether the new proposal better aligns with my goals. Additionally, I will think about potential counter-proposals that could further increase the total importance to me, while taking into account the opponent's willingness to compromise.'}\\ \tt \\ \tt ARGUMENT: {'The revised proposal shows a better understanding of my preferences, as it includes C17 and B35. However, I still see opportunities for improvement. For instance, I consider A87 to be highly important, and its effort value is relatively low. I may propose an alternative that incorporates A87 and potentially other items that align with my importance values, while ensuring the total effort remains within the LIMIT.'}\\ \tt \\ \tt PROPOSAL: {'C17', 'B35', 'A04', 'A77', 'A87', 'C01'} 
	  } 
	   } 
	   } 
	  \\ 
 

    \theutterance \stepcounter{utterance}  

    & & \multicolumn{4}{p{0.6\linewidth}}{\cellcolor[rgb]{0.9,0.9,0.9}{%
	\makecell[{{p{\linewidth}}}]{% 
	  \tt {\tiny [A$\langle$GM]}  
	 ARGUMENT: {'The revised proposal shows a better understanding of my preferences, as it includes C17 and B35. However, I still see opportunities for improvement. For instance, I consider A87 to be highly important, and its effort value is relatively low. I may propose an alternative that incorporates A87 and potentially other items that align with my importance values, while ensuring the total effort remains within the LIMIT.'}\\ \tt \\ \tt PROPOSAL: {'C17', 'B35', 'A04', 'A77', 'A87', 'C01'} 
	  } 
	   } 
	   } 
	 & \\ 
 

    \theutterance \stepcounter{utterance}  

    & \multicolumn{4}{p{0.6\linewidth}}{\cellcolor[rgb]{0.8,1,0.9}{%
	\makecell[{{p{\linewidth}}}]{% 
	  \tt {\tiny [A$\rangle$GM]}  
	 STRATEGIC REASONING: {'The opponent has made a counter-proposal that includes A87, which they consider highly important. I will evaluate this proposal and consider the trade-offs between the included items. I notice that A87 has a relatively low effort value, which makes it an attractive addition. However, I will also review the importance and effort of the other items in the proposal to determine whether this combination meets my goals. I may need to negotiate further to optimize the selection of items.'}\\ \tt \\ \tt ARGUMENT: {'I appreciate the opponent's proposal, which includes A87 and other items that were part of my initial proposal. However, I would like to discuss the inclusion of C01, which has a relatively high effort value. I am willing to consider alternative items that could replace C01, potentially allowing us to include other important items while staying within the LIMIT. Perhaps we can find a more efficient combination that benefits both parties.'}\\ \tt \\ \tt PROPOSAL: {'C17', 'B35', 'A04', 'A77', 'A87', 'B41'} 
	  } 
	   } 
	   } 
	 & & \\ 
 

    \theutterance \stepcounter{utterance}  

    & & \multicolumn{4}{p{0.6\linewidth}}{\cellcolor[rgb]{0.9,0.9,0.9}{%
	\makecell[{{p{\linewidth}}}]{% 
	  \tt {\tiny [GM$\rangle$B]}  
	 ARGUMENT: {'I appreciate the opponent's proposal, which includes A87 and other items that were part of my initial proposal. However, I would like to discuss the inclusion of C01, which has a relatively high effort value. I am willing to consider alternative items that could replace C01, potentially allowing us to include other important items while staying within the LIMIT. Perhaps we can find a more efficient combination that benefits both parties.'}\\ \tt \\ \tt PROPOSAL: {'C17', 'B35', 'A04', 'A77', 'A87', 'B41'} 
	  } 
	   } 
	   } 
	 & \\ 
 

    \theutterance \stepcounter{utterance}  

    & & & \multicolumn{4}{p{0.6\linewidth}}{\cellcolor[rgb]{1,0.85,0.72}{%
	\makecell[{{p{\linewidth}}}]{% 
	  \tt {\tiny [GM$\langle$B]}  
	 STRATEGIC REASONING: {'I will evaluate the new proposal and consider the trade-offs involved in replacing C01 with B41. I notice that B41 has a higher effort value than C01, but it is also an item that I consider relatively important. I will calculate the total effort and importance of the new proposal to determine whether it better aligns with my goals. Additionally, I will think about potential counter-proposals that could further improve the agreement, taking into account the opponent's willingness to compromise and the remaining items that could be included.'}\\ \tt \\ \tt ARGUMENT: {'The revised proposal addresses my concerns about C01, and the inclusion of B41 is a reasonable compromise. However, I would like to explore the possibility of adding one more item to the proposal, potentially A10, which has a moderate effort value and is relatively important to me. I will consider the potential impact on the total effort and importance, and determine whether this addition could lead to a more mutually beneficial agreement.'}\\ \tt \\ \tt PROPOSAL: {'C17', 'B35', 'A04', 'A77', 'A87', 'B41', 'A10'} 
	  } 
	   } 
	   } 
	  \\ 
 

    \theutterance \stepcounter{utterance}  

    & & \multicolumn{4}{p{0.6\linewidth}}{\cellcolor[rgb]{0.9,0.9,0.9}{%
	\makecell[{{p{\linewidth}}}]{% 
	  \tt {\tiny [A$\langle$GM]}  
	 ARGUMENT: {'The revised proposal addresses my concerns about C01, and the inclusion of B41 is a reasonable compromise. However, I would like to explore the possibility of adding one more item to the proposal, potentially A10, which has a moderate effort value and is relatively important to me. I will consider the potential impact on the total effort and importance, and determine whether this addition could lead to a more mutually beneficial agreement.'}\\ \tt \\ \tt PROPOSAL: {'C17', 'B35', 'A04', 'A77', 'A87', 'B41', 'A10'} 
	  } 
	   } 
	   } 
	 & \\ 
 

    \theutterance \stepcounter{utterance}  

    & \multicolumn{4}{p{0.6\linewidth}}{\cellcolor[rgb]{0.8,1,0.9}{%
	\makecell[{{p{\linewidth}}}]{% 
	  \tt {\tiny [A$\rangle$GM]}  
	 STRATEGIC REASONING: {'The opponent has proposed adding A10 to the previous proposal, which would increase the total effort. I will calculate the total effort of this new proposal and determine whether it stays within the LIMIT. If it does, I will evaluate the potential benefits and drawbacks of including A10, considering both the importance and effort values. I may need to weigh the advantages of adding A10 against potential alternatives or adjustments to the proposal.'}\\ \tt \\ \tt ARGUMENT: {'I have reviewed the revised proposal, and I am pleased to see that it includes a diverse set of items that meet both our needs. The addition of A10 increases the total effort, but it appears to be within the acceptable range. I believe that this proposal strikes a reasonable balance between our preferences and the effort constraint. I am willing to accept this proposal, as it seems to be a mutually beneficial agreement.'}\\ \tt \\ \tt AGREE: {'C17', 'B35', 'A04', 'A77', 'A87', 'B41', 'A10'} 
	  } 
	   } 
	   } 
	 & & \\ 
 

    \theutterance \stepcounter{utterance}  

    & & & \multicolumn{2}{p{0.3\linewidth}}{\cellcolor[rgb]{0.95,0.95,0.95}{%
	\makecell[{{p{\linewidth}}}]{% 
	  \tt {\tiny [GM$|$GM]}  
	 game successful 
	  } 
	   } 
	   } 
	 & & \\ 
 

    \theutterance \stepcounter{utterance}  

    & & & \multicolumn{2}{p{0.3\linewidth}}{\cellcolor[rgb]{0.95,0.95,0.95}{%
	\makecell[{{p{\linewidth}}}]{% 
	  \tt {\tiny [GM$|$GM]}  
	 end game 
	  } 
	   } 
	   } 
	 & & \\ 
 

\end{supertabular}
}

\end{document}
