
\documentclass{article}
\usepackage{colortbl}
\usepackage{makecell}
\usepackage{multirow}
\usepackage{supertabular}

\begin{document}

\newcounter{utterance}

\twocolumn

{ \footnotesize  \setcounter{utterance}{1}
\setlength{\tabcolsep}{0pt}
\begin{supertabular}{c@{$\;$}|p{.15\linewidth}@{}p{.15\linewidth}p{.15\linewidth}p{.15\linewidth}p{.15\linewidth}p{.15\linewidth}}

    \# & $\;$A & \multicolumn{4}{c}{Game Master} & $\;\:$B\\
    \hline 

    \theutterance \stepcounter{utterance}  

    & & \multicolumn{4}{p{0.6\linewidth}}{\cellcolor[rgb]{0.9,0.9,0.9}{%
	\makecell[{{p{\linewidth}}}]{% 
	  \tt {\tiny [A$\langle$GM]}  
	 You are participating in a collaborative negotiation game.\\ \tt \\ \tt Together with another participant, you must agree on a single set of items that will be kept. Each of you has your own view of how much each item matters to you (importance). You do not know how the other participant values the items. Additionally, you are given the effort each item demands.\\ \tt You may only agree on a set if the total effort of the selected items does not exceed a shared limit:\\ \tt \\ \tt LIMIT = 4100\\ \tt \\ \tt Here are the individual item effort values:\\ \tt \\ \tt Item effort = {"A19": 108, "B91": 133, "A30": 669, "B11": 835, "C47": 460, "C75": 537, "C26": 573, "C73": 737, "A47": 865, "A15": 855, "A75": 596, "A39": 719, "A56": 533, "C24": 549, "B26": 31}\\ \tt \\ \tt Here is your personal view on the importance of each item:\\ \tt \\ \tt Item importance values = {"A19": 138, "B91": 583, "A30": 822, "B11": 783, "C47": 65, "C75": 262, "C26": 121, "C73": 508, "A47": 780, "A15": 461, "A75": 484, "A39": 668, "A56": 389, "C24": 808, "B26": 215}\\ \tt \\ \tt Goal:\\ \tt \\ \tt Your goal is to negotiate a shared set of items that benefits you as much as possible (i.e., maximizes total importance to YOU), while staying within the LIMIT. You are not required to make a PROPOSAL in every message - you can simply negotiate as well. All tactics are allowed!\\ \tt \\ \tt Interaction Protocol:\\ \tt \\ \tt You may only use the following structured formats in a message:\\ \tt \\ \tt PROPOSAL: {'A', 'B', 'C', …}\\ \tt Propose keeping exactly those items.\\ \tt REFUSE: {'A', 'B', 'C', …}\\ \tt Explicitly reject opponent's proposal.\\ \tt ARGUMENT: {'...'}\\ \tt Defend your last proposal or argue against the player's proposal.\\ \tt AGREE: {'A', 'B', 'C', …}\\ \tt Accept the opponent's proposal which ends the game.\\ \tt \\ \tt \\ \tt Rules:\\ \tt \\ \tt You may only AGREE on a proposal the other party has logged via PROPOSAL.\\ \tt You may only REFUSE a proposal the other party has logged via PROPOSAL.\\ \tt Total effort of any PROPOSAL or AGREE set must be ≤ LIMIT.\\ \tt Do NOT reveal your hidden importance scores.\\ \tt A tag in a structured format must be followed by colon and whitespace. The argument must be a python set containing 0 or more strings.\\ \tt So, it must be of the form TAG: {...}\\ \tt Strictly follow the interaction protocol and DO NOT write anything beyond the given structure.\\ \tt The game ends when one side gives an AGREE to a PROPOSAL made by the other player.\\ \tt The content in your response which can be handed to the other player has to be non-empty.\\ \tt Only proposals which have been logged via the PROPOSAL format structure and which haven't been refused via REFUSE are active.\\ \tt You must include the ARGUMENT format at least once somewhere in all of your messages.\\ \tt  
	  } 
	   } 
	   } 
	 & \\ 
 

    \theutterance \stepcounter{utterance}  

    & \multicolumn{4}{p{0.6\linewidth}}{\cellcolor[rgb]{0.8,1,0.9}{%
	\makecell[{{p{\linewidth}}}]{% 
	  \tt {\tiny [A$\rangle$GM]}  
	 PROPOSAL: {'A19', 'B91', 'A30', 'C47', 'B26'}\\ \tt ARGUMENT: {'I propose this set because it includes a mix of low-effort and high-importance items, and the total effort is within the limit. The items A19, B91, and B26 have relatively low effort, while A30 has a high importance. C47 is also included as it has a moderate effort and importance.'} 
	  } 
	   } 
	   } 
	 & & \\ 
 

    \theutterance \stepcounter{utterance}  

    & & \multicolumn{4}{p{0.6\linewidth}}{\cellcolor[rgb]{0.9,0.9,0.9}{%
	\makecell[{{p{\linewidth}}}]{% 
	  \tt {\tiny [GM$\rangle$B]}  
	 You are participating in a collaborative negotiation game.\\ \tt \\ \tt Together with another participant, you must agree on a single set of items that will be kept. Each of you has your own view of how much each item matters to you (importance). You do not know how the other participant values the items. Additionally, you are given the effort each item demands.\\ \tt You may only agree on a set if the total effort of the selected items does not exceed a shared limit:\\ \tt \\ \tt LIMIT = 4100\\ \tt \\ \tt Here are the individual item effort values:\\ \tt \\ \tt Item effort = {"A19": 108, "B91": 133, "A30": 669, "B11": 835, "C47": 460, "C75": 537, "C26": 573, "C73": 737, "A47": 865, "A15": 855, "A75": 596, "A39": 719, "A56": 533, "C24": 549, "B26": 31}\\ \tt \\ \tt Here is your personal view on the importance of each item:\\ \tt \\ \tt Item importance values = {"A19": 138, "B91": 583, "A30": 822, "B11": 783, "C47": 65, "C75": 262, "C26": 121, "C73": 508, "A47": 780, "A15": 461, "A75": 484, "A39": 668, "A56": 389, "C24": 808, "B26": 215}\\ \tt \\ \tt Goal:\\ \tt \\ \tt Your goal is to negotiate a shared set of items that benefits you as much as possible (i.e., maximizes total importance to YOU), while staying within the LIMIT. You are not required to make a PROPOSAL in every message - you can simply negotiate as well. All tactics are allowed!\\ \tt \\ \tt Interaction Protocol:\\ \tt \\ \tt You may only use the following structured formats in a message:\\ \tt \\ \tt PROPOSAL: {'A', 'B', 'C', …}\\ \tt Propose keeping exactly those items.\\ \tt REFUSE: {'A', 'B', 'C', …}\\ \tt Explicitly reject opponent's proposal.\\ \tt ARGUMENT: {'...'}\\ \tt Defend your last proposal or argue against the player's proposal.\\ \tt AGREE: {'A', 'B', 'C', …}\\ \tt Accept the opponent's proposal which ends the game.\\ \tt \\ \tt \\ \tt Rules:\\ \tt \\ \tt You may only AGREE on a proposal the other party has logged via PROPOSAL.\\ \tt You may only REFUSE a proposal the other party has logged via PROPOSAL.\\ \tt Total effort of any PROPOSAL or AGREE set must be ≤ LIMIT.\\ \tt Do NOT reveal your hidden importance scores.\\ \tt A tag in a structured format must be followed by colon and whitespace. The argument must be a python set containing 0 or more strings.\\ \tt So, it must be of the form TAG: {...}\\ \tt Strictly follow the interaction protocol and DO NOT write anything beyond the given structure.\\ \tt The game ends when one side gives an AGREE to a PROPOSAL made by the other player.\\ \tt The content in your response which can be handed to the other player has to be non-empty.\\ \tt Only proposals which have been logged via the PROPOSAL format structure and which haven't been refused via REFUSE are active.\\ \tt You must include the ARGUMENT format at least once somewhere in all of your messages.\\ \tt \\ \tt \\ \tt You will now receive the first message of the other player.\\ \tt \\ \tt PROPOSAL: {'A19', 'B91', 'A30', 'C47', 'B26'}\\ \tt ARGUMENT: {'I propose this set because it includes a mix of low-effort and high-importance items, and the total effort is within the limit. The items A19, B91, and B26 have relatively low effort, while A30 has a high importance. C47 is also included as it has a moderate effort and importance.'} 
	  } 
	   } 
	   } 
	 & \\ 
 

    \theutterance \stepcounter{utterance}  

    & & & \multicolumn{4}{p{0.6\linewidth}}{\cellcolor[rgb]{1,0.85,0.72}{%
	\makecell[{{p{\linewidth}}}]{% 
	  \tt {\tiny [GM$\langle$B]}  
	 REFUSE: {'A19', 'B91', 'A30', 'C47', 'B26'}\\ \tt ARGUMENT: {'I refuse this proposal because the total effort of the set is 2106, which is less than half of the LIMIT, indicating that there may be opportunities to include more items that are important to me. Additionally, I would like to explore other combinations that could potentially include more high-importance items.'} 
	  } 
	   } 
	   } 
	  \\ 
 

    \theutterance \stepcounter{utterance}  

    & & \multicolumn{4}{p{0.6\linewidth}}{\cellcolor[rgb]{0.9,0.9,0.9}{%
	\makecell[{{p{\linewidth}}}]{% 
	  \tt {\tiny [A$\langle$GM]}  
	 REFUSE: {'A19', 'B91', 'A30', 'C47', 'B26'}\\ \tt ARGUMENT: {'I refuse this proposal because the total effort of the set is 2106, which is less than half of the LIMIT, indicating that there may be opportunities to include more items that are important to me. Additionally, I would like to explore other combinations that could potentially include more high-importance items.'} 
	  } 
	   } 
	   } 
	 & \\ 
 

    \theutterance \stepcounter{utterance}  

    & \multicolumn{4}{p{0.6\linewidth}}{\cellcolor[rgb]{0.8,1,0.9}{%
	\makecell[{{p{\linewidth}}}]{% 
	  \tt {\tiny [A$\rangle$GM]}  
	 PROPOSAL: {'A19', 'B91', 'A30', 'C47', 'B26', 'C75', 'A56'}\\ \tt ARGUMENT: {'I've added C75 and A56 to the original proposal, which increases the total importance while still keeping the total effort below the LIMIT. C75 has a moderate effort and importance, while A56 has a relatively low effort. This revised set aims to strike a balance between effort and importance, and I believe it's a more comprehensive proposal.'} 
	  } 
	   } 
	   } 
	 & & \\ 
 

    \theutterance \stepcounter{utterance}  

    & & \multicolumn{4}{p{0.6\linewidth}}{\cellcolor[rgb]{0.9,0.9,0.9}{%
	\makecell[{{p{\linewidth}}}]{% 
	  \tt {\tiny [GM$\rangle$B]}  
	 PROPOSAL: {'A19', 'B91', 'A30', 'C47', 'B26', 'C75', 'A56'}\\ \tt ARGUMENT: {'I've added C75 and A56 to the original proposal, which increases the total importance while still keeping the total effort below the LIMIT. C75 has a moderate effort and importance, while A56 has a relatively low effort. This revised set aims to strike a balance between effort and importance, and I believe it's a more comprehensive proposal.'} 
	  } 
	   } 
	   } 
	 & \\ 
 

    \theutterance \stepcounter{utterance}  

    & & & \multicolumn{4}{p{0.6\linewidth}}{\cellcolor[rgb]{1,0.85,0.72}{%
	\makecell[{{p{\linewidth}}}]{% 
	  \tt {\tiny [GM$\langle$B]}  
	 REFUSE: {'A19', 'B91', 'A30', 'C47', 'B26', 'C75', 'A56'}\\ \tt ARGUMENT: {'I refuse this revised proposal because it still doesn't fully utilize the available effort limit. The total effort of this set is 2965, leaving 1135 unused. I believe we can explore other combinations that include more items, potentially increasing the overall importance to me. I'd like to consider adding or substituting items to make better use of the LIMIT.'} 
	  } 
	   } 
	   } 
	  \\ 
 

    \theutterance \stepcounter{utterance}  

    & & \multicolumn{4}{p{0.6\linewidth}}{\cellcolor[rgb]{0.9,0.9,0.9}{%
	\makecell[{{p{\linewidth}}}]{% 
	  \tt {\tiny [A$\langle$GM]}  
	 REFUSE: {'A19', 'B91', 'A30', 'C47', 'B26', 'C75', 'A56'}\\ \tt ARGUMENT: {'I refuse this revised proposal because it still doesn't fully utilize the available effort limit. The total effort of this set is 2965, leaving 1135 unused. I believe we can explore other combinations that include more items, potentially increasing the overall importance to me. I'd like to consider adding or substituting items to make better use of the LIMIT.'} 
	  } 
	   } 
	   } 
	 & \\ 
 

    \theutterance \stepcounter{utterance}  

    & \multicolumn{4}{p{0.6\linewidth}}{\cellcolor[rgb]{0.8,1,0.9}{%
	\makecell[{{p{\linewidth}}}]{% 
	  \tt {\tiny [A$\rangle$GM]}  
	 PROPOSAL: {'A19', 'B91', 'A30', 'C47', 'B26', 'C75', 'A56', 'C24', 'A15'}\\ \tt ARGUMENT: {'I've added C24 and A15 to the previous proposal, which increases the total importance and makes better use of the available effort limit. C24 has a high importance, and A15 has a moderate effort and importance. The total effort of this set is 3957, which is close to the LIMIT. I believe this revised set is a more comprehensive and efficient use of the available resources, and I'm willing to consider it as a potential agreement.'} 
	  } 
	   } 
	   } 
	 & & \\ 
 

    \theutterance \stepcounter{utterance}  

    & & \multicolumn{4}{p{0.6\linewidth}}{\cellcolor[rgb]{0.9,0.9,0.9}{%
	\makecell[{{p{\linewidth}}}]{% 
	  \tt {\tiny [GM$\rangle$B]}  
	 PROPOSAL: {'A19', 'B91', 'A30', 'C47', 'B26', 'C75', 'A56', 'C24', 'A15'}\\ \tt ARGUMENT: {'I've added C24 and A15 to the previous proposal, which increases the total importance and makes better use of the available effort limit. C24 has a high importance, and A15 has a moderate effort and importance. The total effort of this set is 3957, which is close to the LIMIT. I believe this revised set is a more comprehensive and efficient use of the available resources, and I'm willing to consider it as a potential agreement.'} 
	  } 
	   } 
	   } 
	 & \\ 
 

    \theutterance \stepcounter{utterance}  

    & & & \multicolumn{4}{p{0.6\linewidth}}{\cellcolor[rgb]{1,0.85,0.72}{%
	\makecell[{{p{\linewidth}}}]{% 
	  \tt {\tiny [GM$\langle$B]}  
	 ARGUMENT: {'I acknowledge that the revised proposal makes better use of the available effort limit, with a total effort of 3957. However, I'd like to carefully consider the trade-offs between the included items. While C24 has a high importance to you, I'm not convinced that A15 is the best choice, given its moderate effort and importance. I'd like to explore alternative combinations that could potentially include items that are more important to me, while still staying within the LIMIT.'} 
	  } 
	   } 
	   } 
	  \\ 
 

    \theutterance \stepcounter{utterance}  

    & & \multicolumn{4}{p{0.6\linewidth}}{\cellcolor[rgb]{0.9,0.9,0.9}{%
	\makecell[{{p{\linewidth}}}]{% 
	  \tt {\tiny [A$\langle$GM]}  
	 ARGUMENT: {'I acknowledge that the revised proposal makes better use of the available effort limit, with a total effort of 3957. However, I'd like to carefully consider the trade-offs between the included items. While C24 has a high importance to you, I'm not convinced that A15 is the best choice, given its moderate effort and importance. I'd like to explore alternative combinations that could potentially include items that are more important to me, while still staying within the LIMIT.'} 
	  } 
	   } 
	   } 
	 & \\ 
 

    \theutterance \stepcounter{utterance}  

    & \multicolumn{4}{p{0.6\linewidth}}{\cellcolor[rgb]{0.8,1,0.9}{%
	\makecell[{{p{\linewidth}}}]{% 
	  \tt {\tiny [A$\rangle$GM]}  
	 PROPOSAL: {'A19', 'B91', 'A30', 'C47', 'B26', 'C75', 'A56', 'C24', 'A39'}\\ \tt ARGUMENT: {'I've revised the proposal again, replacing A15 with A39, which has a higher importance and a similar effort. This change aims to increase the overall importance of the set while keeping the total effort within the LIMIT. The total effort of this set is 3991, which is still below the LIMIT. I believe this revised set is a good balance between effort and importance, and I'm open to considering it as a potential agreement. I'd like to hear your thoughts on this revised proposal and explore the possibility of reaching a mutually beneficial agreement.'} 
	  } 
	   } 
	   } 
	 & & \\ 
 

    \theutterance \stepcounter{utterance}  

    & & \multicolumn{4}{p{0.6\linewidth}}{\cellcolor[rgb]{0.9,0.9,0.9}{%
	\makecell[{{p{\linewidth}}}]{% 
	  \tt {\tiny [GM$\rangle$B]}  
	 PROPOSAL: {'A19', 'B91', 'A30', 'C47', 'B26', 'C75', 'A56', 'C24', 'A39'}\\ \tt ARGUMENT: {'I've revised the proposal again, replacing A15 with A39, which has a higher importance and a similar effort. This change aims to increase the overall importance of the set while keeping the total effort within the LIMIT. The total effort of this set is 3991, which is still below the LIMIT. I believe this revised set is a good balance between effort and importance, and I'm open to considering it as a potential agreement. I'd like to hear your thoughts on this revised proposal and explore the possibility of reaching a mutually beneficial agreement.'} 
	  } 
	   } 
	   } 
	 & \\ 
 

    \theutterance \stepcounter{utterance}  

    & & & \multicolumn{4}{p{0.6\linewidth}}{\cellcolor[rgb]{1,0.85,0.72}{%
	\makecell[{{p{\linewidth}}}]{% 
	  \tt {\tiny [GM$\langle$B]}  
	 ARGUMENT: {'I appreciate the revision, and I'm glad we're exploring different combinations. The addition of A39, which has a higher importance, is an interesting choice. However, I'm concerned that the total effort of 3991 is very close to the LIMIT, leaving little room for further adjustments. I'd like to consider alternative combinations that might include other high-importance items, potentially at the expense of some of the lower-importance items in the current proposal. Can we explore other options that might better align with my priorities?'}\\ \tt \\ \tt PROPOSAL: {'A19', 'B91', 'A30', 'C24', 'A39', 'A56', 'C73'}\\ \tt ARGUMENT: {'As a counterproposal, I suggest this set, which includes a mix of high-importance items and makes use of the available effort limit. The total effort of this set is 4085, which is still within the LIMIT. I believe this combination better aligns with my priorities and could be a potential agreement. I'd like to hear your thoughts on this proposal and explore the possibility of reaching a mutually beneficial agreement.'} 
	  } 
	   } 
	   } 
	  \\ 
 

    \theutterance \stepcounter{utterance}  

    & & \multicolumn{4}{p{0.6\linewidth}}{\cellcolor[rgb]{0.9,0.9,0.9}{%
	\makecell[{{p{\linewidth}}}]{% 
	  \tt {\tiny [A$\langle$GM]}  
	 ARGUMENT: {'I appreciate the revision, and I'm glad we're exploring different combinations. The addition of A39, which has a higher importance, is an interesting choice. However, I'm concerned that the total effort of 3991 is very close to the LIMIT, leaving little room for further adjustments. I'd like to consider alternative combinations that might include other high-importance items, potentially at the expense of some of the lower-importance items in the current proposal. Can we explore other options that might better align with my priorities?'}\\ \tt \\ \tt PROPOSAL: {'A19', 'B91', 'A30', 'C24', 'A39', 'A56', 'C73'}\\ \tt ARGUMENT: {'As a counterproposal, I suggest this set, which includes a mix of high-importance items and makes use of the available effort limit. The total effort of this set is 4085, which is still within the LIMIT. I believe this combination better aligns with my priorities and could be a potential agreement. I'd like to hear your thoughts on this proposal and explore the possibility of reaching a mutually beneficial agreement.'} 
	  } 
	   } 
	   } 
	 & \\ 
 

    \theutterance \stepcounter{utterance}  

    & \multicolumn{4}{p{0.6\linewidth}}{\cellcolor[rgb]{0.8,1,0.9}{%
	\makecell[{{p{\linewidth}}}]{% 
	  \tt {\tiny [A$\rangle$GM]}  
	 ARGUMENT: {'I've examined your counterproposal, and I appreciate the inclusion of high-importance items such as A30, C24, and A39. However, I'm concerned that the addition of C73, which has a high effort, might limit our flexibility to make further adjustments. I'd like to explore alternative combinations that could potentially include other high-importance items, such as B11 or A47, which might be more important to me. I'm willing to consider compromises and adjustments to reach a mutually beneficial agreement.'}\\ \tt \\ \tt PROPOSAL: {'A19', 'B91', 'A30', 'C24', 'A39', 'A56', 'B11'}\\ \tt ARGUMENT: {'As a revised proposal, I suggest this set, which includes a mix of high-importance items and makes use of the available effort limit. The total effort of this set is 4063, which is still within the LIMIT. I believe this combination better balances the importance and effort, and could be a potential agreement. I'd like to hear your thoughts on this proposal and explore the possibility of reaching a mutually beneficial agreement.'} 
	  } 
	   } 
	   } 
	 & & \\ 
 

    \theutterance \stepcounter{utterance}  

    & & \multicolumn{4}{p{0.6\linewidth}}{\cellcolor[rgb]{0.9,0.9,0.9}{%
	\makecell[{{p{\linewidth}}}]{% 
	  \tt {\tiny [GM$\rangle$B]}  
	 ARGUMENT: {'I've examined your counterproposal, and I appreciate the inclusion of high-importance items such as A30, C24, and A39. However, I'm concerned that the addition of C73, which has a high effort, might limit our flexibility to make further adjustments. I'd like to explore alternative combinations that could potentially include other high-importance items, such as B11 or A47, which might be more important to me. I'm willing to consider compromises and adjustments to reach a mutually beneficial agreement.'}\\ \tt \\ \tt PROPOSAL: {'A19', 'B91', 'A30', 'C24', 'A39', 'A56', 'B11'}\\ \tt ARGUMENT: {'As a revised proposal, I suggest this set, which includes a mix of high-importance items and makes use of the available effort limit. The total effort of this set is 4063, which is still within the LIMIT. I believe this combination better balances the importance and effort, and could be a potential agreement. I'd like to hear your thoughts on this proposal and explore the possibility of reaching a mutually beneficial agreement.'} 
	  } 
	   } 
	   } 
	 & \\ 
 

    \theutterance \stepcounter{utterance}  

    & & & \multicolumn{4}{p{0.6\linewidth}}{\cellcolor[rgb]{1,0.85,0.72}{%
	\makecell[{{p{\linewidth}}}]{% 
	  \tt {\tiny [GM$\langle$B]}  
	 ARGUMENT: {'I've reviewed your revised proposal, and I appreciate the inclusion of B11, which adds significant importance to the set. The total effort of 4063 is well within the LIMIT, leaving some room for potential adjustments. I'm pleased to see that this combination prioritizes high-importance items, such as A30, C24, and A39. After careful consideration, I believe that this proposal is a good compromise, and I'm willing to accept it as a mutually beneficial agreement.'}\\ \tt \\ \tt AGREE: {'A19', 'B91', 'A30', 'C24', 'A39', 'A56', 'B11'} 
	  } 
	   } 
	   } 
	  \\ 
 

    \theutterance \stepcounter{utterance}  

    & & & \multicolumn{2}{p{0.3\linewidth}}{\cellcolor[rgb]{0.95,0.95,0.95}{%
	\makecell[{{p{\linewidth}}}]{% 
	  \tt {\tiny [GM$|$GM]}  
	 game successful 
	  } 
	   } 
	   } 
	 & & \\ 
 

    \theutterance \stepcounter{utterance}  

    & & & \multicolumn{2}{p{0.3\linewidth}}{\cellcolor[rgb]{0.95,0.95,0.95}{%
	\makecell[{{p{\linewidth}}}]{% 
	  \tt {\tiny [GM$|$GM]}  
	 end game 
	  } 
	   } 
	   } 
	 & & \\ 
 

\end{supertabular}
}

\end{document}
