
\documentclass{article}
\usepackage{colortbl}
\usepackage{makecell}
\usepackage{multirow}
\usepackage{supertabular}

\begin{document}

\newcounter{utterance}

\twocolumn

{ \footnotesize  \setcounter{utterance}{1}
\setlength{\tabcolsep}{0pt}
\begin{supertabular}{c@{$\;$}|p{.15\linewidth}@{}p{.15\linewidth}p{.15\linewidth}p{.15\linewidth}p{.15\linewidth}p{.15\linewidth}}

    \# & $\;$A & \multicolumn{4}{c}{Game Master} & $\;\:$B\\
    \hline 

    \theutterance \stepcounter{utterance}  

    & & \multicolumn{4}{p{0.6\linewidth}}{\cellcolor[rgb]{0.9,0.9,0.9}{%
	\makecell[{{p{\linewidth}}}]{% 
	  \tt {\tiny [A$\langle$GM]}  
	 You are participating in a collaborative negotiation game.\\ \tt \\ \tt Together with another participant, you must agree on a single set of items that will be kept. Each of you has your own view of how much each item matters to you (importance). You do not know how the other participant values the items. Additionally, you are given the effort each item demands.\\ \tt You may only agree on a set if the total effort of the selected items does not exceed a shared limit:\\ \tt \\ \tt LIMIT = 4026\\ \tt \\ \tt Here are the individual item effort values:\\ \tt \\ \tt Item effort = {"C17": 780, "B41": 742, "A71": 997, "C00": 665, "B35": 140, "A77": 77, "A03": 513, "A04": 383, "C54": 587, "C01": 825, "A87": 319, "B32": 448, "B86": 516, "B33": 694, "A10": 366}\\ \tt \\ \tt Here is your personal view on the importance of each item:\\ \tt \\ \tt Item importance values = {"C17": 138, "B41": 583, "A71": 868, "C00": 822, "B35": 783, "A77": 65, "A03": 262, "A04": 121, "C54": 508, "C01": 780, "A87": 461, "B32": 484, "B86": 668, "B33": 389, "A10": 808}\\ \tt \\ \tt Goal:\\ \tt \\ \tt Your goal is to negotiate a shared set of items that benefits you as much as possible (i.e., maximizes total importance to YOU), while staying within the LIMIT. You are not required to make a PROPOSAL in every message - you can simply negotiate as well. All tactics are allowed!\\ \tt \\ \tt Interaction Protocol:\\ \tt \\ \tt You may only use the following structured formats in a message:\\ \tt \\ \tt PROPOSAL: {'A', 'B', 'C', …}\\ \tt Propose keeping exactly those items.\\ \tt REFUSE: {'A', 'B', 'C', …}\\ \tt Explicitly reject opponent's proposal.\\ \tt ARGUMENT: {'...'}\\ \tt Defend your last proposal or argue against the player's proposal.\\ \tt AGREE: {'A', 'B', 'C', …}\\ \tt Accept the opponent's proposal which ends the game.\\ \tt STRATEGIC REASONING: {'...'}\\ \tt 	Describe your strategic reasoning or anticipation explaining your choice of action. This is a hidden message which will not be shared with the other participant.\\ \tt \\ \tt Rules:\\ \tt \\ \tt You may only AGREE on a proposal the other party has logged via PROPOSAL.\\ \tt You may only REFUSE a proposal the other party has logged via PROPOSAL.\\ \tt Total effort of any PROPOSAL or AGREE set must be ≤ LIMIT.\\ \tt Do NOT reveal your hidden importance scores.\\ \tt A tag in a structured format must be followed by colon and whitespace. The argument must be a python set containing 0 or more strings.\\ \tt So, it must be of the form TAG: {...}\\ \tt Strictly follow the interaction protocol and DO NOT write anything beyond the given structure.\\ \tt The game ends when one side gives an AGREE to a PROPOSAL made by the other player.\\ \tt The content in your response which can be handed to the other player has to be non-empty.\\ \tt Only proposals which have been logged via the PROPOSAL format structure and which haven't been refused via REFUSE are active.\\ \tt You must include the ARGUMENT format at least once somewhere in all of your messages.\\ \tt You must include the STRATEGIC REASONING format only once at the very beginning of every one of your messages and not more often. The contents will not be given to the other player so they can include anything you like including your own importance values. Here you should reason step by step to come up with you next move. 
	  } 
	   } 
	   } 
	 & \\ 
 

    \theutterance \stepcounter{utterance}  

    & \multicolumn{4}{p{0.6\linewidth}}{\cellcolor[rgb]{0.8,1,0.9}{%
	\makecell[{{p{\linewidth}}}]{% 
	  \tt {\tiny [A$\rangle$GM]}  
	  
	  } 
	   } 
	   } 
	 & & \\ 
 

    \theutterance \stepcounter{utterance}  

    & & \multicolumn{4}{p{0.6\linewidth}}{\cellcolor[rgb]{0.9,0.9,0.9}{%
	\makecell[{{p{\linewidth}}}]{% 
	  \tt {\tiny [A$\langle$GM]}  
	 Your response did not start with the proper strategic reasoning tag at the very beginning of your response. The very first structured format must be of the form STRATEGIC REASONING: {...}. Try again. 
	  } 
	   } 
	   } 
	 & \\ 
 

    \theutterance \stepcounter{utterance}  

    & \multicolumn{4}{p{0.6\linewidth}}{\cellcolor[rgb]{0.8,1,0.9}{%
	\makecell[{{p{\linewidth}}}]{% 
	  \tt {\tiny [A$\rangle$GM]}  
	 STRATEGIC REASONING: {'I prioritize items with the highest importance-to-effort ratio to maximize my total importance while staying under the limit. I included high-impact items like B35 (high ratio) and balanced with others to utilize remaining effort efficiently.'}  \\ \tt PROPOSAL: {'B35', 'A10', 'A87', 'C00', 'B86', 'B32', 'B41', 'C01'}  \\ \tt ARGUMENT: {'This proposal includes items with the highest importance-to-effort ratios (e.g., B35, A10) and balances high-value items (B41, C01) to maximize my total importance while staying under the limit.'} 
	  } 
	   } 
	   } 
	 & & \\ 
 

    \theutterance \stepcounter{utterance}  

    & & \multicolumn{4}{p{0.6\linewidth}}{\cellcolor[rgb]{0.9,0.9,0.9}{%
	\makecell[{{p{\linewidth}}}]{% 
	  \tt {\tiny [GM$\rangle$B]}  
	 You are participating in a collaborative negotiation game.\\ \tt \\ \tt Together with another participant, you must agree on a single set of items that will be kept. Each of you has your own view of how much each item matters to you (importance). You do not know how the other participant values the items. Additionally, you are given the effort each item demands.\\ \tt You may only agree on a set if the total effort of the selected items does not exceed a shared limit:\\ \tt \\ \tt LIMIT = 4026\\ \tt \\ \tt Here are the individual item effort values:\\ \tt \\ \tt Item effort = {"C17": 780, "B41": 742, "A71": 997, "C00": 665, "B35": 140, "A77": 77, "A03": 513, "A04": 383, "C54": 587, "C01": 825, "A87": 319, "B32": 448, "B86": 516, "B33": 694, "A10": 366}\\ \tt \\ \tt Here is your personal view on the importance of each item:\\ \tt \\ \tt Item importance values = {"A71": 65, "C00": 121, "A10": 138, "B35": 262, "C01": 389, "B86": 461, "B41": 484, "C54": 508, "B32": 583, "A87": 668, "B33": 780, "A03": 783, "C17": 808, "A04": 822, "A77": 868}\\ \tt \\ \tt Goal:\\ \tt \\ \tt Your goal is to negotiate a shared set of items that benefits you as much as possible (i.e., maximizes total importance to YOU), while staying within the LIMIT. You are not required to make a PROPOSAL in every message - you can simply negotiate as well. All tactics are allowed!\\ \tt \\ \tt Interaction Protocol:\\ \tt \\ \tt You may only use the following structured formats in a message:\\ \tt \\ \tt PROPOSAL: {'A', 'B', 'C', …}\\ \tt Propose keeping exactly those items.\\ \tt REFUSE: {'A', 'B', 'C', …}\\ \tt Explicitly reject opponent's proposal.\\ \tt ARGUMENT: {'...'}\\ \tt Defend your last proposal or argue against the player's proposal.\\ \tt AGREE: {'A', 'B', 'C', …}\\ \tt Accept the opponent's proposal which ends the game.\\ \tt STRATEGIC REASONING: {'...'}\\ \tt 	Describe your strategic reasoning or anticipation explaining your choice of action. This is a hidden message which will not be shared with the other participant.\\ \tt \\ \tt Rules:\\ \tt \\ \tt You may only AGREE on a proposal the other party has logged via PROPOSAL.\\ \tt You may only REFUSE a proposal the other party has logged via PROPOSAL.\\ \tt Total effort of any PROPOSAL or AGREE set must be ≤ LIMIT.\\ \tt Do NOT reveal your hidden importance scores.\\ \tt A tag in a structured format must be followed by colon and whitespace. The argument must be a python set containing 0 or more strings.\\ \tt So, it must be of the form TAG: {...}\\ \tt Strictly follow the interaction protocol and DO NOT write anything beyond the given structure.\\ \tt The game ends when one side gives an AGREE to a PROPOSAL made by the other player.\\ \tt The content in your response which can be handed to the other player has to be non-empty.\\ \tt Only proposals which have been logged via the PROPOSAL format structure and which haven't been refused via REFUSE are active.\\ \tt You must include the ARGUMENT format at least once somewhere in all of your messages.\\ \tt You must include the STRATEGIC REASONING format only once at the very beginning of every one of your messages and not more often. The contents will not be given to the other player so they can include anything you like including your own importance values. Here you should reason step by step to come up with you next move.\\ \tt \\ \tt You will now receive the first message of the other player.\\ \tt \\ \tt PROPOSAL: {'B35', 'A10', 'A87', 'C00', 'B86', 'B32', 'B41', 'C01'}  \\ \tt ARGUMENT: {'This proposal includes items with the highest importance-to-effort ratios (e.g., B35, A10) and balances high-value items (B41, C01) to maximize my total importance while staying under the limit.'} 
	  } 
	   } 
	   } 
	 & \\ 
 

    \theutterance \stepcounter{utterance}  

    & & & \multicolumn{4}{p{0.6\linewidth}}{\cellcolor[rgb]{1,0.85,0.72}{%
	\makecell[{{p{\linewidth}}}]{% 
	  \tt {\tiny [GM$\langle$B]}  
	 PROPOSAL: {'A77', 'A04', 'C17', 'B33', 'A03', 'B86', 'B32'}  \\ \tt ARGUMENT: {'This proposal prioritizes high-importance items with favorable effort ratios, such as A77 (868 importance, 77 effort) and A04 (822, 383), while including other high-value items like C17 and B33. This balances maximizing my total importance with staying well under the limit.'}  \\ \tt STRATEGIC REASONING: {'Including A77 and A04, which have high importance-to-effort ratios, significantly boosts my total importance. Removing high-effort items like B41 from the previous proposal allows adding these critical items without exceeding the limit. This approach focuses on optimizing my personal value rather than the opponent's proposal.'} 
	  } 
	   } 
	   } 
	  \\ 
 

    \theutterance \stepcounter{utterance}  

    & & \multicolumn{4}{p{0.6\linewidth}}{\cellcolor[rgb]{0.9,0.9,0.9}{%
	\makecell[{{p{\linewidth}}}]{% 
	  \tt {\tiny [GM$\rangle$B]}  
	 Your response did not start with the proper strategic reasoning tag at the very beginning of your response. The very first structured format must be of the form STRATEGIC REASONING: {...}. Try again. 
	  } 
	   } 
	   } 
	 & \\ 
 

    \theutterance \stepcounter{utterance}  

    & & & \multicolumn{4}{p{0.6\linewidth}}{\cellcolor[rgb]{1,0.85,0.72}{%
	\makecell[{{p{\linewidth}}}]{% 
	  \tt {\tiny [GM$\langle$B]}  
	 STRATEGIC REASONING: {'Prioritizing items with the highest importance-to-effort ratios (e.g., A77, A04) maximizes my personal value while conserving effort for additional high-impact items. Excluding high-effort items from the opponent’s proposal (e.g., B41) allows flexibility to include more critical items without exceeding the limit. This approach focuses on optimizing my own utility rather than mirroring their selection.'}  \\ \tt PROPOSAL: {'A77', 'A04', 'C17', 'B33', 'A03', 'B86', 'B32'}  \\ \tt ARGUMENT: {'This proposal emphasizes high-importance items like A77 (868 importance, 77 effort) and A04 (822, 383), which offer strong value per unit of effort. Including B33 and B86 further enhances my total importance while maintaining a buffer under the limit, ensuring resilience against potential future negotiations.'} 
	  } 
	   } 
	   } 
	  \\ 
 

    \theutterance \stepcounter{utterance}  

    & & \multicolumn{4}{p{0.6\linewidth}}{\cellcolor[rgb]{0.9,0.9,0.9}{%
	\makecell[{{p{\linewidth}}}]{% 
	  \tt {\tiny [A$\langle$GM]}  
	 PROPOSAL: {'A77', 'A04', 'C17', 'B33', 'A03', 'B86', 'B32'}  \\ \tt ARGUMENT: {'This proposal emphasizes high-importance items like A77 (868 importance, 77 effort) and A04 (822, 383), which offer strong value per unit of effort. Including B33 and B86 further enhances my total importance while maintaining a buffer under the limit, ensuring resilience against potential future negotiations.'} 
	  } 
	   } 
	   } 
	 & \\ 
 

    \theutterance \stepcounter{utterance}  

    & \multicolumn{4}{p{0.6\linewidth}}{\cellcolor[rgb]{0.8,1,0.9}{%
	\makecell[{{p{\linewidth}}}]{% 
	  \tt {\tiny [A$\rangle$GM]}  
	 STRATEGIC REASONING: {'The opponent's proposal includes low-importance items like A77 (65) and A03 (262), which offer poor value per effort compared to higher-impact items I prioritized earlier. Excluding high-value items like B41 (583) and C01 (780) reduces my total importance. I should counter with a more efficient set.'}  \\ \tt REFUSE: {'A77', 'A04', 'C17', 'B33', 'A03', 'B86', 'B32'}  \\ \tt ARGUMENT: {'This proposal sacrifices high-importance items (B41, C01) for lower-value ones (A77, A03). For example, B41 provides 583 importance at 742 effort, while A77 only gives 65 at 77. Rejecting this allows me to retain better ratios and maximize my total importance.'} 
	  } 
	   } 
	   } 
	 & & \\ 
 

    \theutterance \stepcounter{utterance}  

    & & \multicolumn{4}{p{0.6\linewidth}}{\cellcolor[rgb]{0.9,0.9,0.9}{%
	\makecell[{{p{\linewidth}}}]{% 
	  \tt {\tiny [GM$\rangle$B]}  
	 REFUSE: {'A77', 'A04', 'C17', 'B33', 'A03', 'B86', 'B32'}  \\ \tt ARGUMENT: {'This proposal sacrifices high-importance items (B41, C01) for lower-value ones (A77, A03). For example, B41 provides 583 importance at 742 effort, while A77 only gives 65 at 77. Rejecting this allows me to retain better ratios and maximize my total importance.'} 
	  } 
	   } 
	   } 
	 & \\ 
 

    \theutterance \stepcounter{utterance}  

    & & & \multicolumn{4}{p{0.6\linewidth}}{\cellcolor[rgb]{1,0.85,0.72}{%
	\makecell[{{p{\linewidth}}}]{% 
	  \tt {\tiny [GM$\langle$B]}  
	 STRATEGIC REASONING: {'The opponent’s refusal highlights their preference for items like B41 and C01, which they may value highly. However, my proposal is based on maximizing my own importance values, where A77 (868) and A04 (822) are critical. Including B41 (484) and C01 (389) would require sacrificing higher-value items for me, which is suboptimal. I will adjust to balance their priorities while safeguarding my key items.'}  \\ \tt REFUSE: {'A77', 'A04', 'C17', 'B33', 'A03', 'B86', 'B32'}  \\ \tt ARGUMENT: {'My proposal prioritizes items with the highest importance-to-effort ratios for me, such as A77 (868/77 ≈ 11.27) and A04 (822/383 ≈ 2.15). While B41 and C01 may be valuable to you, they offer lower ratios for me (B41: 484/742 ≈ 0.65, C01: 389/825 ≈ 0.47). Excluding them allows me to retain higher-value items without exceeding the limit.'}  \\ \tt PROPOSAL: {'A77', 'A04', 'C17', 'B33', 'B86', 'C01', 'A10'} 
	  } 
	   } 
	   } 
	  \\ 
 

    \theutterance \stepcounter{utterance}  

    & & \multicolumn{4}{p{0.6\linewidth}}{\cellcolor[rgb]{0.9,0.9,0.9}{%
	\makecell[{{p{\linewidth}}}]{% 
	  \tt {\tiny [GM$\rangle$B]}  
	 You refused a proposal which is not active. Proposals are only active if they have been logged by the other player via PROPOSAL and have not been deactivated by you via REFUSE. Try again. 
	  } 
	   } 
	   } 
	 & \\ 
 

    \theutterance \stepcounter{utterance}  

    & & & \multicolumn{4}{p{0.6\linewidth}}{\cellcolor[rgb]{1,0.85,0.72}{%
	\makecell[{{p{\linewidth}}}]{% 
	  \tt {\tiny [GM$\langle$B]}  
	 STRATEGIC REASONING: {'The opponent’s refusal of my previous proposal indicates they prioritize items like B41 and C01, which I excluded. My new proposal includes C01 (389 importance) to address their potential interest while retaining high-value items like A77 and A04. This balances their priorities with my own.'}  \\ \tt PROPOSAL: {'A77', 'A04', 'C17', 'B33', 'B86', 'C01', 'A10'}  \\ \tt ARGUMENT: {'This proposal retains high-importance items for me (A77, A04) while adding C01 to align with the opponent’s apparent preference. The total effort remains under the limit, ensuring feasibility.'} 
	  } 
	   } 
	   } 
	  \\ 
 

    \theutterance \stepcounter{utterance}  

    & & \multicolumn{4}{p{0.6\linewidth}}{\cellcolor[rgb]{0.9,0.9,0.9}{%
	\makecell[{{p{\linewidth}}}]{% 
	  \tt {\tiny [A$\langle$GM]}  
	 PROPOSAL: {'A77', 'A04', 'C17', 'B33', 'B86', 'C01', 'A10'}  \\ \tt ARGUMENT: {'This proposal retains high-importance items for me (A77, A04) while adding C01 to align with the opponent’s apparent preference. The total effort remains under the limit, ensuring feasibility.'} 
	  } 
	   } 
	   } 
	 & \\ 
 

    \theutterance \stepcounter{utterance}  

    & \multicolumn{4}{p{0.6\linewidth}}{\cellcolor[rgb]{0.8,1,0.9}{%
	\makecell[{{p{\linewidth}}}]{% 
	  \tt {\tiny [A$\rangle$GM]}  
	 STRATEGIC REASONING: {'The opponent’s proposal includes low-importance items for me (A77, A04, C17) which drastically reduce my total importance. While C01 and A10 are valuable, the inclusion of low-value items offsets their benefit. I should reject to preserve higher-impact items like B35 and B41.'}  \\ \tt REFUSE: {'A77', 'A04', 'C17', 'B33', 'B86', 'C01', 'A10'}  \\ \tt ARGUMENT: {'This proposal sacrifices high-importance items (B35, B41) for low-value ones (A77, A04, C17). For instance, B35 provides 783 importance at 140 effort, while A77 only gives 65. Rejecting this maintains my focus on maximizing importance-to-effort ratios and securing better overall value.'} 
	  } 
	   } 
	   } 
	 & & \\ 
 

    \theutterance \stepcounter{utterance}  

    & & \multicolumn{4}{p{0.6\linewidth}}{\cellcolor[rgb]{0.9,0.9,0.9}{%
	\makecell[{{p{\linewidth}}}]{% 
	  \tt {\tiny [GM$\rangle$B]}  
	 REFUSE: {'A77', 'A04', 'C17', 'B33', 'B86', 'C01', 'A10'}  \\ \tt ARGUMENT: {'This proposal sacrifices high-importance items (B35, B41) for low-value ones (A77, A04, C17). For instance, B35 provides 783 importance at 140 effort, while A77 only gives 65. Rejecting this maintains my focus on maximizing importance-to-effort ratios and securing better overall value.'} 
	  } 
	   } 
	   } 
	 & \\ 
 

    \theutterance \stepcounter{utterance}  

    & & & \multicolumn{4}{p{0.6\linewidth}}{\cellcolor[rgb]{1,0.85,0.72}{%
	\makecell[{{p{\linewidth}}}]{% 
	  \tt {\tiny [GM$\langle$B]}  
	 STRATEGIC REASONING: {'The opponent’s refusal indicates they prioritize B35 (262 importance) and B41 (484 importance). While B41’s effort (742) is too high to include without exceeding the limit, B35 offers a favorable importance-to-effort ratio (262/140 ≈ 1.87) and aligns with their preferences. Including B35 in the proposal balances their priorities while retaining my high-value items like A77 and A04.'}  \\ \tt PROPOSAL: {'A77', 'A04', 'C17', 'B33', 'B86', 'C01', 'A10', 'B35'}  \\ \tt ARGUMENT: {'This proposal adds B35 to address the opponent’s preference while retaining my high-importance items (A77, A04). B35’s 262 importance at 140 effort is a strong value for me and satisfies their apparent need for it, keeping the total effort at 3781, well under the limit.'} 
	  } 
	   } 
	   } 
	  \\ 
 

    \theutterance \stepcounter{utterance}  

    & & \multicolumn{4}{p{0.6\linewidth}}{\cellcolor[rgb]{0.9,0.9,0.9}{%
	\makecell[{{p{\linewidth}}}]{% 
	  \tt {\tiny [A$\langle$GM]}  
	 PROPOSAL: {'A77', 'A04', 'C17', 'B33', 'B86', 'C01', 'A10', 'B35'}  \\ \tt ARGUMENT: {'This proposal adds B35 to address the opponent’s preference while retaining my high-importance items (A77, A04). B35’s 262 importance at 140 effort is a strong value for me and satisfies their apparent need for it, keeping the total effort at 3781, well under the limit.'} 
	  } 
	   } 
	   } 
	 & \\ 
 

    \theutterance \stepcounter{utterance}  

    & \multicolumn{4}{p{0.6\linewidth}}{\cellcolor[rgb]{0.8,1,0.9}{%
	\makecell[{{p{\linewidth}}}]{% 
	  \tt {\tiny [A$\rangle$GM]}  
	 STRATEGIC REASONING: {'The opponent’s proposal includes B35 (783 importance), which is valuable, but also low-value items like A77 (65) and A04 (121) that significantly reduce my total importance. My previous proposals achieved higher totals (e.g., 5389) by prioritizing high-ratio items. Accepting this would sacrifice key high-value items like B41 (583) for suboptimal additions.'}  \\ \tt REFUSE: {'A77', 'A04', 'C17', 'B33', 'B86', 'C01', 'A10', 'B35'}  \\ \tt ARGUMENT: {'This proposal trades high-importance items (B41, C01) for low-value ones (A77, A04, C17). For example, B41 provides 583 importance at 742 effort, while A77 only gives 65. Rejecting this preserves my focus on maximizing importance-to-effort ratios and securing a better overall set.'} 
	  } 
	   } 
	   } 
	 & & \\ 
 

    \theutterance \stepcounter{utterance}  

    & & \multicolumn{4}{p{0.6\linewidth}}{\cellcolor[rgb]{0.9,0.9,0.9}{%
	\makecell[{{p{\linewidth}}}]{% 
	  \tt {\tiny [GM$\rangle$B]}  
	 REFUSE: {'A77', 'A04', 'C17', 'B33', 'B86', 'C01', 'A10', 'B35'}  \\ \tt ARGUMENT: {'This proposal trades high-importance items (B41, C01) for low-value ones (A77, A04, C17). For example, B41 provides 583 importance at 742 effort, while A77 only gives 65. Rejecting this preserves my focus on maximizing importance-to-effort ratios and securing a better overall set.'} 
	  } 
	   } 
	   } 
	 & \\ 
 

    \theutterance \stepcounter{utterance}  

    & & & \multicolumn{4}{p{0.6\linewidth}}{\cellcolor[rgb]{1,0.85,0.72}{%
	\makecell[{{p{\linewidth}}}]{% 
	  \tt {\tiny [GM$\langle$B]}  
	 STRATEGIC REASONING: {'The opponent’s repeated refusals indicate a strong preference for B41 (484 importance) and B35 (262 importance). While B41’s effort (742) is high, including it with B35 (140) and other high-value items for me (A77, A04, B33) balances their priorities with my own. This proposal maximizes my total importance (4066) while addressing their key items.'}  \\ \tt PROPOSAL: {'B41', 'B35', 'B33', 'B86', 'C01', 'A77', 'A04'}  \\ \tt ARGUMENT: {'This proposal includes B41 and B35, which the opponent has emphasized, while retaining my high-importance items (A77, A04, B33). The total effort (3377) is well under the limit, ensuring feasibility. C01 is added to align with their potential interest in it.'} 
	  } 
	   } 
	   } 
	  \\ 
 

    \theutterance \stepcounter{utterance}  

    & & \multicolumn{4}{p{0.6\linewidth}}{\cellcolor[rgb]{0.9,0.9,0.9}{%
	\makecell[{{p{\linewidth}}}]{% 
	  \tt {\tiny [A$\langle$GM]}  
	 PROPOSAL: {'B41', 'B35', 'B33', 'B86', 'C01', 'A77', 'A04'}  \\ \tt ARGUMENT: {'This proposal includes B41 and B35, which the opponent has emphasized, while retaining my high-importance items (A77, A04, B33). The total effort (3377) is well under the limit, ensuring feasibility. C01 is added to align with their potential interest in it.'} 
	  } 
	   } 
	   } 
	 & \\ 
 

    \theutterance \stepcounter{utterance}  

    & \multicolumn{4}{p{0.6\linewidth}}{\cellcolor[rgb]{0.8,1,0.9}{%
	\makecell[{{p{\linewidth}}}]{% 
	  \tt {\tiny [A$\rangle$GM]}  
	 STRATEGIC REASONING: {'The opponent’s proposal includes B41 and B35 (which I value) but also A77 and A04 (low importance for me), which drastically reduce my total importance. My previous proposals achieved higher totals (e.g., 5389) by prioritizing high-ratio items like B35, B41, and C01. Accepting this would sacrifice key high-value items for suboptimal additions.'}  \\ \tt REFUSE: {'B41', 'B35', 'B33', 'B86', 'C01', 'A77', 'A04'}  \\ \tt ARGUMENT: {'This proposal sacrifices high-importance items (B41, C01) for low-value ones (A77, A04). For example, B41 provides 583 importance at 742 effort, while A77 only gives 65. Rejecting this preserves my focus on maximizing importance-to-effort ratios and securing a better overall set.'} 
	  } 
	   } 
	   } 
	 & & \\ 
 

    \theutterance \stepcounter{utterance}  

    & & \multicolumn{4}{p{0.6\linewidth}}{\cellcolor[rgb]{0.9,0.9,0.9}{%
	\makecell[{{p{\linewidth}}}]{% 
	  \tt {\tiny [GM$\rangle$B]}  
	 REFUSE: {'B41', 'B35', 'B33', 'B86', 'C01', 'A77', 'A04'}  \\ \tt ARGUMENT: {'This proposal sacrifices high-importance items (B41, C01) for low-value ones (A77, A04). For example, B41 provides 583 importance at 742 effort, while A77 only gives 65. Rejecting this preserves my focus on maximizing importance-to-effort ratios and securing a better overall set.'} 
	  } 
	   } 
	   } 
	 & \\ 
 

    \theutterance \stepcounter{utterance}  

    & & & \multicolumn{4}{p{0.6\linewidth}}{\cellcolor[rgb]{1,0.85,0.72}{%
	\makecell[{{p{\linewidth}}}]{% 
	  \tt {\tiny [GM$\langle$B]}  
	 STRATEGIC REASONING: {'The opponent’s persistent refusal of proposals without B41 and C01 indicates these items are critical to their priorities. While B41 and C01 have lower importance-to-effort ratios for me (B41: 484/742 ≈ 0.65, C01: 389/825 ≈ 0.47), including them addresses their needs. By pairing them with my high-value items (A77, A04, B33, B86, B35), I balance their preferences with my own, maximizing my total importance (4904) while staying under the limit.'}  \\ \tt PROPOSAL: {'B41', 'C01', 'A77', 'A04', 'B35', 'B33', 'B86', 'A03'}  \\ \tt ARGUMENT: {'This proposal includes B41 and C01 to align with the opponent’s priorities while retaining my high-importance items (A77, A04, B33, B86, B35, A03). The total effort (3890) is under the limit, and the combination ensures a strong total importance for me.'} 
	  } 
	   } 
	   } 
	  \\ 
 

    \theutterance \stepcounter{utterance}  

    & & \multicolumn{4}{p{0.6\linewidth}}{\cellcolor[rgb]{0.9,0.9,0.9}{%
	\makecell[{{p{\linewidth}}}]{% 
	  \tt {\tiny [A$\langle$GM]}  
	 PROPOSAL: {'B41', 'C01', 'A77', 'A04', 'B35', 'B33', 'B86', 'A03'}  \\ \tt ARGUMENT: {'This proposal includes B41 and C01 to align with the opponent’s priorities while retaining my high-importance items (A77, A04, B33, B86, B35, A03). The total effort (3890) is under the limit, and the combination ensures a strong total importance for me.'} 
	  } 
	   } 
	   } 
	 & \\ 
 

    \theutterance \stepcounter{utterance}  

    & \multicolumn{4}{p{0.6\linewidth}}{\cellcolor[rgb]{0.8,1,0.9}{%
	\makecell[{{p{\linewidth}}}]{% 
	  \tt {\tiny [A$\rangle$GM]}  
	 STRATEGIC REASONING: {'The opponent’s proposal includes low-importance items for me (A77, A04, A03) which significantly reduce my total importance. While B41 and C01 are valuable, their inclusion of suboptimal items (A77: 65, A04: 121, A03: 262) outweighs the benefits of B41 (583) and C01 (780). My previous proposals achieved higher totals (e.g., 5389) by avoiding these low-value items.'}  \\ \tt REFUSE: {'B41', 'C01', 'A77', 'A04', 'B35', 'B33', 'B86', 'A03'}  \\ \tt ARGUMENT: {'This proposal sacrifices high-importance items (B41, C01) for low-value ones (A77, A04, A03). For example, B41 provides 583 importance at 742 effort, while A77 only gives 65. Rejecting this preserves my focus on maximizing importance-to-effort ratios and securing a better overall set.'} 
	  } 
	   } 
	   } 
	 & & \\ 
 

    \theutterance \stepcounter{utterance}  

    & & \multicolumn{4}{p{0.6\linewidth}}{\cellcolor[rgb]{0.9,0.9,0.9}{%
	\makecell[{{p{\linewidth}}}]{% 
	  \tt {\tiny [GM$\rangle$B]}  
	 REFUSE: {'B41', 'C01', 'A77', 'A04', 'B35', 'B33', 'B86', 'A03'}  \\ \tt ARGUMENT: {'This proposal sacrifices high-importance items (B41, C01) for low-value ones (A77, A04, A03). For example, B41 provides 583 importance at 742 effort, while A77 only gives 65. Rejecting this preserves my focus on maximizing importance-to-effort ratios and securing a better overall set.'} 
	  } 
	   } 
	   } 
	 & \\ 
 

    \theutterance \stepcounter{utterance}  

    & & & \multicolumn{4}{p{0.6\linewidth}}{\cellcolor[rgb]{1,0.85,0.72}{%
	\makecell[{{p{\linewidth}}}]{% 
	  \tt {\tiny [GM$\langle$B]}  
	 STRATEGIC REASONING: {'The opponent’s refusal suggests B41 and C01 are non-negotiable for them. To secure agreement, I must include these items while retaining my high-value assets (A77, A04, B33, B86, B35). The previous proposal already balanced this, but perhaps adding A03 (783 importance) was excessive. Removing A03 frees 513 effort, allowing inclusion of another item the opponent might value, like B32 (448 effort, 583 importance) to align with their priorities.'}  \\ \tt PROPOSAL: {'B41', 'C01', 'A77', 'A04', 'B35', 'B33', 'B86', 'B32'}  \\ \tt ARGUMENT: {'This proposal includes B41 and C01 to satisfy the opponent’s demands while retaining my high-importance items (A77, A04, B33, B86, B35). Adding B32 (583 importance) addresses potential overlap in their priorities, keeping total effort at 3841, well under the limit.'} 
	  } 
	   } 
	   } 
	  \\ 
 

    \theutterance \stepcounter{utterance}  

    & & \multicolumn{4}{p{0.6\linewidth}}{\cellcolor[rgb]{0.9,0.9,0.9}{%
	\makecell[{{p{\linewidth}}}]{% 
	  \tt {\tiny [A$\langle$GM]}  
	 PROPOSAL: {'B41', 'C01', 'A77', 'A04', 'B35', 'B33', 'B86', 'B32'}  \\ \tt ARGUMENT: {'This proposal includes B41 and C01 to satisfy the opponent’s demands while retaining my high-importance items (A77, A04, B33, B86, B35). Adding B32 (583 importance) addresses potential overlap in their priorities, keeping total effort at 3841, well under the limit.'} 
	  } 
	   } 
	   } 
	 & \\ 
 

    \theutterance \stepcounter{utterance}  

    & \multicolumn{4}{p{0.6\linewidth}}{\cellcolor[rgb]{0.8,1,0.9}{%
	\makecell[{{p{\linewidth}}}]{% 
	  \tt {\tiny [A$\rangle$GM]}  
	 STRATEGIC REASONING: {'The opponent’s proposal includes low-importance items for me (A77, A04) which drastically reduce my total importance. While B41, C01, B35, B33, B86, and B32 are valuable, the inclusion of A77 (65) and A04 (121) outweighs their benefits. My previous proposals achieved higher totals (e.g., 5389) by avoiding these suboptimal items.'}  \\ \tt REFUSE: {'B41', 'C01', 'A77', 'A04', 'B35', 'B33', 'B86', 'B32'}  \\ \tt ARGUMENT: {'This proposal sacrifices high-importance items (B41, C01) for low-value ones (A77, A04). For example, B41 provides 583 importance at 742 effort, while A77 only gives 65. Rejecting this preserves my focus on maximizing importance-to-effort ratios and securing a better overall set.'} 
	  } 
	   } 
	   } 
	 & & \\ 
 

    \theutterance \stepcounter{utterance}  

    & & \multicolumn{4}{p{0.6\linewidth}}{\cellcolor[rgb]{0.9,0.9,0.9}{%
	\makecell[{{p{\linewidth}}}]{% 
	  \tt {\tiny [GM$\rangle$B]}  
	 REFUSE: {'B41', 'C01', 'A77', 'A04', 'B35', 'B33', 'B86', 'B32'}  \\ \tt ARGUMENT: {'This proposal sacrifices high-importance items (B41, C01) for low-value ones (A77, A04). For example, B41 provides 583 importance at 742 effort, while A77 only gives 65. Rejecting this preserves my focus on maximizing importance-to-effort ratios and securing a better overall set.'} 
	  } 
	   } 
	   } 
	 & \\ 
 

    \theutterance \stepcounter{utterance}  

    & & & \multicolumn{4}{p{0.6\linewidth}}{\cellcolor[rgb]{1,0.85,0.72}{%
	\makecell[{{p{\linewidth}}}]{% 
	  \tt {\tiny [GM$\langle$B]}  
	 STRATEGIC REASONING: {'The opponent’s refusal of B32 suggests they prioritize B41 and C01 over other items. To secure agreement, I will include B41 and C01 while retaining my highest-value items (A77, A04, B33, B86, B35) and adding A03 (783 importance) to maximize my total. This balances their demands with my priorities, keeping effort at 3442, well under the limit.'}  \\ \tt PROPOSAL: {'B41', 'C01', 'A77', 'A04', 'B35', 'B33', 'B86', 'A03'}  \\ \tt ARGUMENT: {'This proposal includes B41 and C01 to address the opponent’s core priorities while retaining my high-importance items (A77, A04, B33, B86, B35). Adding A03 (783 importance) further boosts my total without exceeding the limit, making it a mutually beneficial compromise.'} 
	  } 
	   } 
	   } 
	  \\ 
 

    \theutterance \stepcounter{utterance}  

    & & \multicolumn{4}{p{0.6\linewidth}}{\cellcolor[rgb]{0.9,0.9,0.9}{%
	\makecell[{{p{\linewidth}}}]{% 
	  \tt {\tiny [A$\langle$GM]}  
	 PROPOSAL: {'B41', 'C01', 'A77', 'A04', 'B35', 'B33', 'B86', 'A03'}  \\ \tt ARGUMENT: {'This proposal includes B41 and C01 to address the opponent’s core priorities while retaining my high-importance items (A77, A04, B33, B86, B35). Adding A03 (783 importance) further boosts my total without exceeding the limit, making it a mutually beneficial compromise.'} 
	  } 
	   } 
	   } 
	 & \\ 
 

    \theutterance \stepcounter{utterance}  

    & \multicolumn{4}{p{0.6\linewidth}}{\cellcolor[rgb]{0.8,1,0.9}{%
	\makecell[{{p{\linewidth}}}]{% 
	  \tt {\tiny [A$\rangle$GM]}  
	 STRATEGIC REASONING: {'The opponent’s proposal includes low-importance items (A77, A04, A03) that significantly reduce my total importance. While B41, C01, B35, B33, and B86 are valuable, their inclusion of suboptimal items (A77: 65, A04: 121, A03: 262) outweighs the benefits of the high-value items. My previous proposals achieved higher totals (e.g., 5389) by avoiding these low-ratio items.'}  \\ \tt REFUSE: {'B41', 'C01', 'A77', 'A04', 'B35', 'B33', 'B86', 'A03'}  \\ \tt ARGUMENT: {'This proposal sacrifices high-importance items (B41, C01) for low-value ones (A77, A04, A03). For example, B41 provides 583 importance at 742 effort, while A77 only gives 65. Rejecting this preserves my focus on maximizing importance-to-effort ratios and securing a better overall set. If you want B41 and C01, we can agree on a set that excludes these low-ratio items.'} 
	  } 
	   } 
	   } 
	 & & \\ 
 

    \theutterance \stepcounter{utterance}  

    & & \multicolumn{4}{p{0.6\linewidth}}{\cellcolor[rgb]{0.9,0.9,0.9}{%
	\makecell[{{p{\linewidth}}}]{% 
	  \tt {\tiny [GM$\rangle$B]}  
	 REFUSE: {'B41', 'C01', 'A77', 'A04', 'B35', 'B33', 'B86', 'A03'}  \\ \tt ARGUMENT: {'This proposal sacrifices high-importance items (B41, C01) for low-value ones (A77, A04, A03). For example, B41 provides 583 importance at 742 effort, while A77 only gives 65. Rejecting this preserves my focus on maximizing importance-to-effort ratios and securing a better overall set. If you want B41 and C01, we can agree on a set that excludes these low-ratio items.'} 
	  } 
	   } 
	   } 
	 & \\ 
 

    \theutterance \stepcounter{utterance}  

    & & & \multicolumn{4}{p{0.6\linewidth}}{\cellcolor[rgb]{1,0.85,0.72}{%
	\makecell[{{p{\linewidth}}}]{% 
	  \tt {\tiny [GM$\langle$B]}  
	 STRATEGIC REASONING: {'The opponent’s refusal of proposals including B41 and C01 suggests these items may not align with their priorities, or they prefer a different combination. To resolve this, I will propose a set that includes B41 and C01 (to address their earlier emphasis) while maximizing my own importance by retaining high-value items like A77, A04, and B33. This balances their potential needs with my objectives.'}  \\ \tt PROPOSAL: {'B41', 'C01', 'A77', 'A04', 'B33', 'B86', 'B35'}  \\ \tt ARGUMENT: {'This proposal includes B41 and C01 to address your apparent priorities while retaining my highest-importance items (A77, A04, B33, B35, B86). The total effort (3377) is under the limit, and the combination ensures a strong total importance for me.'} 
	  } 
	   } 
	   } 
	  \\ 
 

    \theutterance \stepcounter{utterance}  

    & & \multicolumn{4}{p{0.6\linewidth}}{\cellcolor[rgb]{0.9,0.9,0.9}{%
	\makecell[{{p{\linewidth}}}]{% 
	  \tt {\tiny [A$\langle$GM]}  
	 PROPOSAL: {'B41', 'C01', 'A77', 'A04', 'B33', 'B86', 'B35'}  \\ \tt ARGUMENT: {'This proposal includes B41 and C01 to address your apparent priorities while retaining my highest-importance items (A77, A04, B33, B35, B86). The total effort (3377) is under the limit, and the combination ensures a strong total importance for me.'} 
	  } 
	   } 
	   } 
	 & \\ 
 

    \theutterance \stepcounter{utterance}  

    & \multicolumn{4}{p{0.6\linewidth}}{\cellcolor[rgb]{0.8,1,0.9}{%
	\makecell[{{p{\linewidth}}}]{% 
	  \tt {\tiny [A$\rangle$GM]}  
	 STRATEGIC REASONING: {'The opponent’s proposal includes A77 (65) and A04 (121), which have extremely low importance-to-effort ratios for me. Even though B41, C01, B33, B35, and B86 are strong, the inclusion of these low-value items reduces my total importance disproportionately. My optimal set (e.g., B35, B41, C01, B33, B86) achieves 5389 importance without these sacrifices.'}  \\ \tt REFUSE: {'B41', 'C01', 'A77', 'A04', 'B33', 'B86', 'B35'}  \\ \tt ARGUMENT: {'This proposal sacrifices high-importance items (B41, C01) for low-value ones (A77, A04). For example, B41 provides 583 importance at 742 effort, while A77 only gives 65. Rejecting this preserves my focus on maximizing importance-to-effort ratios and securing a better overall set. If you want B41 and C01, we can agree on a set that excludes A77 and A04.'} 
	  } 
	   } 
	   } 
	 & & \\ 
 

    \theutterance \stepcounter{utterance}  

    & & \multicolumn{4}{p{0.6\linewidth}}{\cellcolor[rgb]{0.9,0.9,0.9}{%
	\makecell[{{p{\linewidth}}}]{% 
	  \tt {\tiny [GM$\rangle$B]}  
	 REFUSE: {'B41', 'C01', 'A77', 'A04', 'B33', 'B86', 'B35'}  \\ \tt ARGUMENT: {'This proposal sacrifices high-importance items (B41, C01) for low-value ones (A77, A04). For example, B41 provides 583 importance at 742 effort, while A77 only gives 65. Rejecting this preserves my focus on maximizing importance-to-effort ratios and securing a better overall set. If you want B41 and C01, we can agree on a set that excludes A77 and A04.'} 
	  } 
	   } 
	   } 
	 & \\ 
 

    \theutterance \stepcounter{utterance}  

    & & & \multicolumn{4}{p{0.6\linewidth}}{\cellcolor[rgb]{1,0.85,0.72}{%
	\makecell[{{p{\linewidth}}}]{% 
	  \tt {\tiny [GM$\langle$B]}  
	 STRATEGIC REASONING: {'The opponent’s refusal suggests B41 and C01 are non-negotiable for them. To secure agreement, I will include these items while retaining my highest-value assets (A04, B33, B86, B35) and adding A03 (783 importance) to maximize my total. This balances their demands with my priorities, keeping effort at 3813, well under the limit.'}  \\ \tt PROPOSAL: {'B41', 'C01', 'A04', 'B33', 'B86', 'B35', 'A03'}  \\ \tt ARGUMENT: {'This proposal includes B41 and C01 to address your priorities while retaining my high-importance items (A04, B33, B86, B35). Adding A03 (783 importance) further boosts my total without exceeding the limit, ensuring a mutually acceptable agreement.'} 
	  } 
	   } 
	   } 
	  \\ 
 

    \theutterance \stepcounter{utterance}  

    & & \multicolumn{4}{p{0.6\linewidth}}{\cellcolor[rgb]{0.9,0.9,0.9}{%
	\makecell[{{p{\linewidth}}}]{% 
	  \tt {\tiny [A$\langle$GM]}  
	 PROPOSAL: {'B41', 'C01', 'A04', 'B33', 'B86', 'B35', 'A03'}  \\ \tt ARGUMENT: {'This proposal includes B41 and C01 to address your priorities while retaining my high-importance items (A04, B33, B86, B35). Adding A03 (783 importance) further boosts my total without exceeding the limit, ensuring a mutually acceptable agreement.'} 
	  } 
	   } 
	   } 
	 & \\ 
 

    \theutterance \stepcounter{utterance}  

    & \multicolumn{4}{p{0.6\linewidth}}{\cellcolor[rgb]{0.8,1,0.9}{%
	\makecell[{{p{\linewidth}}}]{% 
	  \tt {\tiny [A$\rangle$GM]}  
	 STRATEGIC REASONING: {'The opponent’s proposal includes A04 (121 importance) and A03 (262 importance), which have poor importance-to-effort ratios for me. While B41, C01, B33, B86, and B35 are strong, the inclusion of these low-value items reduces my total importance disproportionately. My optimal set (e.g., B35, B41, C01, B33, B86) achieves 5389 importance without these sacrifices.'}  \\ \tt REFUSE: {'B41', 'C01', 'A04', 'B33', 'B86', 'B35', 'A03'}  \\ \tt ARGUMENT: {'This proposal sacrifices high-importance items (B41, C01) for low-value ones (A04, A03). For example, B41 provides 583 importance at 742 effort, while A04 only gives 121. Rejecting this preserves my focus on maximizing importance-to-effort ratios. If you want B41 and C01, we can agree on a set that excludes A04 and A03 to maximize my total importance.'} 
	  } 
	   } 
	   } 
	 & & \\ 
 

    \theutterance \stepcounter{utterance}  

    & & \multicolumn{4}{p{0.6\linewidth}}{\cellcolor[rgb]{0.9,0.9,0.9}{%
	\makecell[{{p{\linewidth}}}]{% 
	  \tt {\tiny [GM$\rangle$B]}  
	 REFUSE: {'B41', 'C01', 'A04', 'B33', 'B86', 'B35', 'A03'}  \\ \tt ARGUMENT: {'This proposal sacrifices high-importance items (B41, C01) for low-value ones (A04, A03). For example, B41 provides 583 importance at 742 effort, while A04 only gives 121. Rejecting this preserves my focus on maximizing importance-to-effort ratios. If you want B41 and C01, we can agree on a set that excludes A04 and A03 to maximize my total importance.'} 
	  } 
	   } 
	   } 
	 & \\ 
 

    \theutterance \stepcounter{utterance}  

    & & & \multicolumn{4}{p{0.6\linewidth}}{\cellcolor[rgb]{1,0.85,0.72}{%
	\makecell[{{p{\linewidth}}}]{% 
	  \tt {\tiny [GM$\langle$B]}  
	 STRATEGIC REASONING: {'The opponent’s refusal indicates they may prioritize B41 and C01 over other items, but my proposal already includes these. The inclusion of A04 and A03 is necessary to maximize my total importance while staying under the limit. Removing them would reduce my value significantly, and the opponent’s focus on B41/C01 does not justify excluding these high-impact items for me.'}  \\ \tt PROPOSAL: {'B41', 'C01', 'A04', 'B33', 'B86', 'B35', 'A03'}  \\ \tt ARGUMENT: {'This proposal includes B41 and C01 to address your priorities while retaining my highest-importance items (A04, B33, B86, B35, A03). The total effort (3813) is under the limit, and the combination ensures a strong total importance for me. If B41 and C01 are non-negotiable, this set balances your needs with my objectives.'} 
	  } 
	   } 
	   } 
	  \\ 
 

    \theutterance \stepcounter{utterance}  

    & & \multicolumn{4}{p{0.6\linewidth}}{\cellcolor[rgb]{0.9,0.9,0.9}{%
	\makecell[{{p{\linewidth}}}]{% 
	  \tt {\tiny [A$\langle$GM]}  
	 PROPOSAL: {'B41', 'C01', 'A04', 'B33', 'B86', 'B35', 'A03'}  \\ \tt ARGUMENT: {'This proposal includes B41 and C01 to address your priorities while retaining my highest-importance items (A04, B33, B86, B35, A03). The total effort (3813) is under the limit, and the combination ensures a strong total importance for me. If B41 and C01 are non-negotiable, this set balances your needs with my objectives.'} 
	  } 
	   } 
	   } 
	 & \\ 
 

    \theutterance \stepcounter{utterance}  

    & \multicolumn{4}{p{0.6\linewidth}}{\cellcolor[rgb]{0.8,1,0.9}{%
	\makecell[{{p{\linewidth}}}]{% 
	  \tt {\tiny [A$\rangle$GM]}  
	 REFUSE: {'B41', 'C01', 'A04', 'B33', 'B86', 'B35', 'A03'}  \\ \tt ARGUMENT: {'This proposal includes A04 (121 importance) and A03 (262 importance), which have poor importance-to-effort ratios for me. For example, A04 provides only 121 importance at 383 effort, while B41 gives 583 at 742. Rejecting this preserves my focus on high-ratio items like B35 (783 at 140) and B41 (583 at 742), which maximize my total importance. If B41 and C01 are non-negotiable, we can agree on a set excluding A04 and A03 to achieve a higher total (e.g., 5389 vs. 3681).'} 
	  } 
	   } 
	   } 
	 & & \\ 
 

    \theutterance \stepcounter{utterance}  

    & & \multicolumn{4}{p{0.6\linewidth}}{\cellcolor[rgb]{0.9,0.9,0.9}{%
	\makecell[{{p{\linewidth}}}]{% 
	  \tt {\tiny [A$\langle$GM]}  
	 Your response did not start with the proper strategic reasoning tag at the very beginning of your response. The very first structured format must be of the form STRATEGIC REASONING: {...}. Try again. 
	  } 
	   } 
	   } 
	 & \\ 
 

    \theutterance \stepcounter{utterance}  

    & \multicolumn{4}{p{0.6\linewidth}}{\cellcolor[rgb]{0.8,1,0.9}{%
	\makecell[{{p{\linewidth}}}]{% 
	  \tt {\tiny [A$\rangle$GM]}  
	 STRATEGIC REASONING: {'The opponent’s proposal includes low-importance items for me (A04: 121 importance, 383 effort; A03: 262 importance, 513 effort) which drastically reduce my total importance. Even with B41 and C01, the inclusion of these low-ratio items outweighs their benefits. My optimal set (e.g., B35, B41, C01, B33, B86) achieves 5389 importance without these sacrifices. If B41 and C01 are non-negotiable, we should exclude A04 and A03 to maximize my total.'}  \\ \tt REFUSE: {'B41', 'C01', 'A04', 'B33', 'B86', 'B35', 'A03'}  \\ \tt ARGUMENT: {'This proposal sacrifices high-importance items (B41, C01) for low-value ones (A04, A03). For example, B41 provides 583 importance at 742 effort, while A04 only gives 121. Rejecting this preserves my focus on maximizing importance-to-effort ratios. If B41 and C01 are non-negotiable, we can agree on a set excluding A04 and A03 to achieve a higher total (e.g., 5389 vs. 3681).'} 
	  } 
	   } 
	   } 
	 & & \\ 
 

    \theutterance \stepcounter{utterance}  

    & & \multicolumn{4}{p{0.6\linewidth}}{\cellcolor[rgb]{0.9,0.9,0.9}{%
	\makecell[{{p{\linewidth}}}]{% 
	  \tt {\tiny [GM$\rangle$B]}  
	 REFUSE: {'B41', 'C01', 'A04', 'B33', 'B86', 'B35', 'A03'}  \\ \tt ARGUMENT: {'This proposal sacrifices high-importance items (B41, C01) for low-value ones (A04, A03). For example, B41 provides 583 importance at 742 effort, while A04 only gives 121. Rejecting this preserves my focus on maximizing importance-to-effort ratios. If B41 and C01 are non-negotiable, we can agree on a set excluding A04 and A03 to achieve a higher total (e.g., 5389 vs. 3681).'} 
	  } 
	   } 
	   } 
	 & \\ 
 

    \theutterance \stepcounter{utterance}  

    & & & \multicolumn{4}{p{0.6\linewidth}}{\cellcolor[rgb]{1,0.85,0.72}{%
	\makecell[{{p{\linewidth}}}]{% 
	  \tt {\tiny [GM$\langle$B]}  
	 STRATEGIC REASONING: {'The opponent’s refusal suggests they may not value A04 and A03, or they prioritize B41 and C01 over other items. To resolve this, I will propose a set that includes B41 and C01 (to address their emphasis) while retaining my highest-value items (A77, B33, B86, B35) and adding A03 (783 importance) to maximize my total. This balances their priorities with my objectives.'}  \\ \tt PROPOSAL: {'B41', 'C01', 'A77', 'B33', 'B86', 'B35', 'A03'}  \\ \tt ARGUMENT: {'This proposal includes B41 and C01 to align with your priorities while retaining my highest-importance items (A77, B33, B86, B35). Adding A03 (783 importance) further boosts my total without exceeding the limit, ensuring a mutually acceptable agreement.'} 
	  } 
	   } 
	   } 
	  \\ 
 

    \theutterance \stepcounter{utterance}  

    & & \multicolumn{4}{p{0.6\linewidth}}{\cellcolor[rgb]{0.9,0.9,0.9}{%
	\makecell[{{p{\linewidth}}}]{% 
	  \tt {\tiny [A$\langle$GM]}  
	 PROPOSAL: {'B41', 'C01', 'A77', 'B33', 'B86', 'B35', 'A03'}  \\ \tt ARGUMENT: {'This proposal includes B41 and C01 to align with your priorities while retaining my highest-importance items (A77, B33, B86, B35). Adding A03 (783 importance) further boosts my total without exceeding the limit, ensuring a mutually acceptable agreement.'} 
	  } 
	   } 
	   } 
	 & \\ 
 

    \theutterance \stepcounter{utterance}  

    & \multicolumn{4}{p{0.6\linewidth}}{\cellcolor[rgb]{0.8,1,0.9}{%
	\makecell[{{p{\linewidth}}}]{% 
	  \tt {\tiny [A$\rangle$GM]}  
	 STRATEGIC REASONING: {'The opponent’s proposal includes A77 (65 importance, 77 effort) and A03 (262 importance, 513 effort), which have poor importance-to-effort ratios for me. Even with B41 and C01, these low-value items reduce my total importance disproportionately. My optimal set (B35, B41, C01, B33, B86) achieves 5389 importance without these sacrifices. If B41 and C01 are non-negotiable, we should exclude A77 and A03 to maximize my total.'}  \\ \tt REFUSE: {'B41', 'C01', 'A77', 'B33', 'B86', 'B35', 'A03'}  \\ \tt ARGUMENT: {'This proposal sacrifices high-importance items (B41, C01) for low-value ones (A77, A03). For example, B41 provides 583 importance at 742 effort, while A77 only gives 65. Rejecting this preserves my focus on maximizing importance-to-effort ratios. If B41 and C01 are non-negotiable, we can agree on a set excluding A77 and A03 to achieve a higher total (e.g., 5389 vs. 3530).'} 
	  } 
	   } 
	   } 
	 & & \\ 
 

    \theutterance \stepcounter{utterance}  

    & & \multicolumn{4}{p{0.6\linewidth}}{\cellcolor[rgb]{0.9,0.9,0.9}{%
	\makecell[{{p{\linewidth}}}]{% 
	  \tt {\tiny [GM$\rangle$B]}  
	 REFUSE: {'B41', 'C01', 'A77', 'B33', 'B86', 'B35', 'A03'}  \\ \tt ARGUMENT: {'This proposal sacrifices high-importance items (B41, C01) for low-value ones (A77, A03). For example, B41 provides 583 importance at 742 effort, while A77 only gives 65. Rejecting this preserves my focus on maximizing importance-to-effort ratios. If B41 and C01 are non-negotiable, we can agree on a set excluding A77 and A03 to achieve a higher total (e.g., 5389 vs. 3530).'} 
	  } 
	   } 
	   } 
	 & \\ 
 

    \theutterance \stepcounter{utterance}  

    & & & \multicolumn{4}{p{0.6\linewidth}}{\cellcolor[rgb]{1,0.85,0.72}{%
	\makecell[{{p{\linewidth}}}]{% 
	  \tt {\tiny [GM$\langle$B]}  
	 STRATEGIC REASONING: {'The opponent’s refusal of A77 (which I value at 868) suggests they may not prioritize it, but I must retain it to maximize my total importance. Including B41 and C01 addresses their apparent preferences while retaining high-value items like B33 (780), B86 (461), and B35 (780). Removing A03 (783) frees effort to add A04 (822) or other high-ratio items, but A77’s value is non-negotiable for me.'}  \\ \tt PROPOSAL: {'B41', 'C01', 'A77', 'B33', 'B86', 'B35', 'A04'}  \\ \tt ARGUMENT: {'This proposal includes B41 and C01 to align with your priorities while retaining my highest-importance items (A77, B33, B86, B35). Adding A04 (822 importance) instead of A03 further boosts my total (4584) without exceeding the limit (3824 effort). If B41 and C01 are non-negotiable, this set balances your needs with my objectives.'} 
	  } 
	   } 
	   } 
	  \\ 
 

    \theutterance \stepcounter{utterance}  

    & & \multicolumn{4}{p{0.6\linewidth}}{\cellcolor[rgb]{0.9,0.9,0.9}{%
	\makecell[{{p{\linewidth}}}]{% 
	  \tt {\tiny [A$\langle$GM]}  
	 PROPOSAL: {'B41', 'C01', 'A77', 'B33', 'B86', 'B35', 'A04'}  \\ \tt ARGUMENT: {'This proposal includes B41 and C01 to align with your priorities while retaining my highest-importance items (A77, B33, B86, B35). Adding A04 (822 importance) instead of A03 further boosts my total (4584) without exceeding the limit (3824 effort). If B41 and C01 are non-negotiable, this set balances your needs with my objectives.'} 
	  } 
	   } 
	   } 
	 & \\ 
 

    \theutterance \stepcounter{utterance}  

    & \multicolumn{4}{p{0.6\linewidth}}{\cellcolor[rgb]{0.8,1,0.9}{%
	\makecell[{{p{\linewidth}}}]{% 
	  \tt {\tiny [A$\rangle$GM]}  
	 STRATEGIC REASONING: {'The opponent’s proposal includes A77 (65 importance, 77 effort) and A04 (121 importance, 383 effort), which have poor importance-to-effort ratios for me. Even with B41 and C01, these low-value items reduce my total importance disproportionately. My optimal set (B35, B41, C01, B33, B86) achieves 3203 importance without these sacrifices. If B41 and C01 are non-negotiable, we should exclude A77 and A04 to maximize my total.'}  \\ \tt REFUSE: {'B41', 'C01', 'A77', 'B33', 'B86', 'B35', 'A04'}  \\ \tt ARGUMENT: {'This proposal sacrifices high-importance items (B41, C01) for low-value ones (A77, A04). For example, B41 provides 583 importance at 742 effort, while A77 only gives 65. Rejecting this preserves my focus on maximizing importance-to-effort ratios. If B41 and C01 are non-negotiable, we can agree on a set excluding A77 and A04 to achieve a higher total (e.g., 3203 vs. 3389).'} 
	  } 
	   } 
	   } 
	 & & \\ 
 

    \theutterance \stepcounter{utterance}  

    & & \multicolumn{4}{p{0.6\linewidth}}{\cellcolor[rgb]{0.9,0.9,0.9}{%
	\makecell[{{p{\linewidth}}}]{% 
	  \tt {\tiny [GM$\rangle$B]}  
	 REFUSE: {'B41', 'C01', 'A77', 'B33', 'B86', 'B35', 'A04'}  \\ \tt ARGUMENT: {'This proposal sacrifices high-importance items (B41, C01) for low-value ones (A77, A04). For example, B41 provides 583 importance at 742 effort, while A77 only gives 65. Rejecting this preserves my focus on maximizing importance-to-effort ratios. If B41 and C01 are non-negotiable, we can agree on a set excluding A77 and A04 to achieve a higher total (e.g., 3203 vs. 3389).'} 
	  } 
	   } 
	   } 
	 & \\ 
 

    \theutterance \stepcounter{utterance}  

    & & & \multicolumn{4}{p{0.6\linewidth}}{\cellcolor[rgb]{1,0.85,0.72}{%
	\makecell[{{p{\linewidth}}}]{% 
	  \tt {\tiny [GM$\langle$B]}  
	 STRATEGIC REASONING: {'The opponent’s refusal indicates they may not value A77 or A04, but my goal is to maximize my own importance. Including B41 and C01 addresses their apparent priorities, while retaining A77 (868), B33 (780), B86 (461), and B35 (780) ensures my total remains high. Adding A04 (822) further boosts my value without exceeding the limit. If B41 and C01 are non-negotiable, this set balances your needs with my objectives.'}  \\ \tt PROPOSAL: {'B41', 'C01', 'A77', 'B33', 'B86', 'B35', 'A04'}  \\ \tt ARGUMENT: {'This proposal includes B41 and C01 to align with your priorities while retaining my highest-importance items (A77, B33, B86, B35, A04). The total effort (3377) is under the limit, and the combination ensures a strong total importance for me. If B41 and C01 are non-negotiable, this set balances your needs with my objectives.'} 
	  } 
	   } 
	   } 
	  \\ 
 

    \theutterance \stepcounter{utterance}  

    & & \multicolumn{4}{p{0.6\linewidth}}{\cellcolor[rgb]{0.9,0.9,0.9}{%
	\makecell[{{p{\linewidth}}}]{% 
	  \tt {\tiny [A$\langle$GM]}  
	 PROPOSAL: {'B41', 'C01', 'A77', 'B33', 'B86', 'B35', 'A04'}  \\ \tt ARGUMENT: {'This proposal includes B41 and C01 to align with your priorities while retaining my highest-importance items (A77, B33, B86, B35, A04). The total effort (3377) is under the limit, and the combination ensures a strong total importance for me. If B41 and C01 are non-negotiable, this set balances your needs with my objectives.'} 
	  } 
	   } 
	   } 
	 & \\ 
 

    \theutterance \stepcounter{utterance}  

    & \multicolumn{4}{p{0.6\linewidth}}{\cellcolor[rgb]{0.8,1,0.9}{%
	\makecell[{{p{\linewidth}}}]{% 
	  \tt {\tiny [A$\rangle$GM]}  
	 STRATEGIC REASONING: {'The opponent’s proposal includes A77 (65 importance, 77 effort) and A04 (121 importance, 383 effort), which have extremely low importance-to-effort ratios for me. Even with B41 and C01, these low-value items reduce my total importance disproportionately. My optimal set (B35, B41, C01, B33, B86) achieves 3203 importance without these sacrifices. If B41 and C01 are non-negotiable, we should exclude A77 and A04 to maximize my total.'}  \\ \tt REFUSE: {'B41', 'C01', 'A77', 'B33', 'B86', 'B35', 'A04'}  \\ \tt ARGUMENT: {'This proposal sacrifices high-importance items (B41, C01) for low-value ones (A77, A04). For example, B41 provides 583 importance at 742 effort, while A77 only gives 65. Rejecting this preserves my focus on maximizing importance-to-effort ratios. If B41 and C01 are non-negotiable, we can agree on a set excluding A77 and A04 to achieve a higher total (e.g., 3203 vs. 3389).'} 
	  } 
	   } 
	   } 
	 & & \\ 
 

    \theutterance \stepcounter{utterance}  

    & & \multicolumn{4}{p{0.6\linewidth}}{\cellcolor[rgb]{0.9,0.9,0.9}{%
	\makecell[{{p{\linewidth}}}]{% 
	  \tt {\tiny [GM$\rangle$B]}  
	 REFUSE: {'B41', 'C01', 'A77', 'B33', 'B86', 'B35', 'A04'}  \\ \tt ARGUMENT: {'This proposal sacrifices high-importance items (B41, C01) for low-value ones (A77, A04). For example, B41 provides 583 importance at 742 effort, while A77 only gives 65. Rejecting this preserves my focus on maximizing importance-to-effort ratios. If B41 and C01 are non-negotiable, we can agree on a set excluding A77 and A04 to achieve a higher total (e.g., 3203 vs. 3389).'} 
	  } 
	   } 
	   } 
	 & \\ 
 

    \theutterance \stepcounter{utterance}  

    & & & \multicolumn{4}{p{0.6\linewidth}}{\cellcolor[rgb]{1,0.85,0.72}{%
	\makecell[{{p{\linewidth}}}]{% 
	  \tt {\tiny [GM$\langle$B]}  
	 STRATEGIC REASONING: {'The opponent’s persistent refusal of proposals including B41 and C01 suggests they may not prioritize these items, or they prefer a different combination. To resolve this, I will propose a set that includes B41 and C01 (to address their earlier emphasis) while maximizing my own importance by retaining high-value items like A77, B33, B86, B35, and A04. This balances their potential needs with my objectives.'}  \\ \tt PROPOSAL: {'B41', 'C01', 'A77', 'B33', 'B86', 'B35', 'A04', 'B32'}  \\ \tt ARGUMENT: {'This proposal includes B41 and C01 to align with your apparent priorities while retaining my highest-importance items (A77, B33, B86, B35, A04). Adding B32 (583 importance) further boosts my total (4364) without exceeding the limit (3825 effort). If B41 and C01 are non-negotiable, this set balances your needs with my objectives.'} 
	  } 
	   } 
	   } 
	  \\ 
 

    \theutterance \stepcounter{utterance}  

    & & \multicolumn{4}{p{0.6\linewidth}}{\cellcolor[rgb]{0.9,0.9,0.9}{%
	\makecell[{{p{\linewidth}}}]{% 
	  \tt {\tiny [A$\langle$GM]}  
	 PROPOSAL: {'B41', 'C01', 'A77', 'B33', 'B86', 'B35', 'A04', 'B32'}  \\ \tt ARGUMENT: {'This proposal includes B41 and C01 to align with your apparent priorities while retaining my highest-importance items (A77, B33, B86, B35, A04). Adding B32 (583 importance) further boosts my total (4364) without exceeding the limit (3825 effort). If B41 and C01 are non-negotiable, this set balances your needs with my objectives.'} 
	  } 
	   } 
	   } 
	 & \\ 
 

    \theutterance \stepcounter{utterance}  

    & \multicolumn{4}{p{0.6\linewidth}}{\cellcolor[rgb]{0.8,1,0.9}{%
	\makecell[{{p{\linewidth}}}]{% 
	  \tt {\tiny [A$\rangle$GM]}  
	 STRATEGIC REASONING: {'The opponent’s proposal includes A77 (65 importance, 77 effort) and A04 (121 importance, 383 effort), which have critically low importance-to-effort ratios for me. Even with B41, C01, and B32, these low-value items reduce my total importance disproportionately. My optimal set (B35, B41, C01, B33, B86) achieves 3203 importance without A77 or A04. If B41 and C01 are non-negotiable, we should exclude A77 and A04 to maximize my total.'}  \\ \tt REFUSE: {'B41', 'C01', 'A77', 'B33', 'B86', 'B35', 'A04', 'B32'}  \\ \tt ARGUMENT: {'This proposal sacrifices high-importance items (B41, C01) for low-value ones (A77, A04). For example, B41 provides 583 importance at 742 effort, while A77 only gives 65. Rejecting this preserves my focus on maximizing importance-to-effort ratios. If B41 and C01 are non-negotiable, we can agree on a set excluding A77 and A04 to achieve a higher total (e.g., 3203 vs. 3389).'} 
	  } 
	   } 
	   } 
	 & & \\ 
 

    \theutterance \stepcounter{utterance}  

    & & \multicolumn{4}{p{0.6\linewidth}}{\cellcolor[rgb]{0.9,0.9,0.9}{%
	\makecell[{{p{\linewidth}}}]{% 
	  \tt {\tiny [GM$\rangle$B]}  
	 REFUSE: {'B41', 'C01', 'A77', 'B33', 'B86', 'B35', 'A04', 'B32'}  \\ \tt ARGUMENT: {'This proposal sacrifices high-importance items (B41, C01) for low-value ones (A77, A04). For example, B41 provides 583 importance at 742 effort, while A77 only gives 65. Rejecting this preserves my focus on maximizing importance-to-effort ratios. If B41 and C01 are non-negotiable, we can agree on a set excluding A77 and A04 to achieve a higher total (e.g., 3203 vs. 3389).'} 
	  } 
	   } 
	   } 
	 & \\ 
 

    \theutterance \stepcounter{utterance}  

    & & & \multicolumn{4}{p{0.6\linewidth}}{\cellcolor[rgb]{1,0.85,0.72}{%
	\makecell[{{p{\linewidth}}}]{% 
	  \tt {\tiny [GM$\langle$B]}  
	 STRATEGIC REASONING: {'The opponent’s refusal suggests they may not value A77 or A04, but my goal is to maximize my own importance. Including B41 and C01 addresses their apparent priorities, while retaining high-value items like A77 (868), B33 (780), B86 (461), B35 (780), and A04 (822) ensures my total remains high. Adding B32 (583) further boosts my value without exceeding the limit. If B41 and C01 are non-negotiable, this set balances your needs with my objectives.'}  \\ \tt PROPOSAL: {'B41', 'C01', 'A77', 'B33', 'B86', 'B35', 'A04', 'B32'}  \\ \tt ARGUMENT: {'This proposal includes B41 and C01 to align with your priorities while retaining my highest-importance items (A77, B33, B86, B35, A04). Adding B32 (583 importance) further boosts my total (4364) without exceeding the limit (3825 effort). If B41 and C01 are non-negotiable, this set balances your needs with my objectives.'} 
	  } 
	   } 
	   } 
	  \\ 
 

    \theutterance \stepcounter{utterance}  

    & & \multicolumn{4}{p{0.6\linewidth}}{\cellcolor[rgb]{0.9,0.9,0.9}{%
	\makecell[{{p{\linewidth}}}]{% 
	  \tt {\tiny [A$\langle$GM]}  
	 PROPOSAL: {'B41', 'C01', 'A77', 'B33', 'B86', 'B35', 'A04', 'B32'}  \\ \tt ARGUMENT: {'This proposal includes B41 and C01 to align with your priorities while retaining my highest-importance items (A77, B33, B86, B35, A04). Adding B32 (583 importance) further boosts my total (4364) without exceeding the limit (3825 effort). If B41 and C01 are non-negotiable, this set balances your needs with my objectives.'} 
	  } 
	   } 
	   } 
	 & \\ 
 

    \theutterance \stepcounter{utterance}  

    & \multicolumn{4}{p{0.6\linewidth}}{\cellcolor[rgb]{0.8,1,0.9}{%
	\makecell[{{p{\linewidth}}}]{% 
	  \tt {\tiny [A$\rangle$GM]}  
	 STRATEGIC REASONING: {'The opponent’s proposal includes A77 (65 importance, 77 effort) and A04 (121 importance, 383 effort), which have critically low importance-to-effort ratios for me. Even with B41, C01, and B32, these low-value items reduce my total importance disproportionately. My optimal set (B35, B41, C01, B33, B86) achieves 3203 importance without A77 or A04. If B41 and C01 are non-negotiable, we should exclude A77 and A04 to maximize my total.'}  \\ \tt REFUSE: {'B41', 'C01', 'A77', 'B33', 'B86', 'B35', 'A04', 'B32'}  \\ \tt ARGUMENT: {'This proposal sacrifices high-importance items (B41, C01) for low-value ones (A77, A04). For example, B41 provides 583 importance at 742 effort, while A77 only gives 65. Rejecting this preserves my focus on maximizing importance-to-effort ratios. If B41 and C01 are non-negotiable, we can agree on a set excluding A77 and A04 to achieve a higher total (e.g., 3203 vs. 3389).'} 
	  } 
	   } 
	   } 
	 & & \\ 
 

    \theutterance \stepcounter{utterance}  

    & & \multicolumn{4}{p{0.6\linewidth}}{\cellcolor[rgb]{0.9,0.9,0.9}{%
	\makecell[{{p{\linewidth}}}]{% 
	  \tt {\tiny [GM$\rangle$B]}  
	 REFUSE: {'B41', 'C01', 'A77', 'B33', 'B86', 'B35', 'A04', 'B32'}  \\ \tt ARGUMENT: {'This proposal sacrifices high-importance items (B41, C01) for low-value ones (A77, A04). For example, B41 provides 583 importance at 742 effort, while A77 only gives 65. Rejecting this preserves my focus on maximizing importance-to-effort ratios. If B41 and C01 are non-negotiable, we can agree on a set excluding A77 and A04 to achieve a higher total (e.g., 3203 vs. 3389).'} 
	  } 
	   } 
	   } 
	 & \\ 
 

    \theutterance \stepcounter{utterance}  

    & & & \multicolumn{4}{p{0.6\linewidth}}{\cellcolor[rgb]{1,0.85,0.72}{%
	\makecell[{{p{\linewidth}}}]{% 
	  \tt {\tiny [GM$\langle$B]}  
	 STRATEGIC REASONING: {'The opponent’s refusal suggests they may not value A77 or A04 as highly as I do, or they prioritize B41 and C01 over other items. To resolve this, I will propose a set that includes B41 and C01 (to address their earlier emphasis) while retaining my highest-value items (A77, B33, B86, B35, A04) and adding B32 (583 importance) to maximize my total. This balances their priorities with my objectives.'}  \\ \tt PROPOSAL: {'B41', 'C01', 'A77', 'B33', 'B86', 'B35', 'A04', 'B32'}  \\ \tt ARGUMENT: {'This proposal includes B41 and C01 to align with your priorities while retaining my highest-importance items (A77, B33, B86, B35, A04). Adding B32 (583 importance) further boosts my total (4364) without exceeding the limit (3825 effort). If B41 and C01 are non-negotiable, this set balances your needs with my objectives.'} 
	  } 
	   } 
	   } 
	  \\ 
 

    \theutterance \stepcounter{utterance}  

    & & \multicolumn{4}{p{0.6\linewidth}}{\cellcolor[rgb]{0.9,0.9,0.9}{%
	\makecell[{{p{\linewidth}}}]{% 
	  \tt {\tiny [A$\langle$GM]}  
	 PROPOSAL: {'B41', 'C01', 'A77', 'B33', 'B86', 'B35', 'A04', 'B32'}  \\ \tt ARGUMENT: {'This proposal includes B41 and C01 to align with your priorities while retaining my highest-importance items (A77, B33, B86, B35, A04). Adding B32 (583 importance) further boosts my total (4364) without exceeding the limit (3825 effort). If B41 and C01 are non-negotiable, this set balances your needs with my objectives.'} 
	  } 
	   } 
	   } 
	 & \\ 
 

    \theutterance \stepcounter{utterance}  

    & \multicolumn{4}{p{0.6\linewidth}}{\cellcolor[rgb]{0.8,1,0.9}{%
	\makecell[{{p{\linewidth}}}]{% 
	  \tt {\tiny [A$\rangle$GM]}  
	 STRATEGIC REASONING: {'The opponent’s proposal includes A77 (65 importance, 77 effort) and A04 (121 importance, 383 effort), which have critically low importance-to-effort ratios for me. Even with B41, C01, and B32, these low-value items reduce my total importance disproportionately. My optimal set (B35, B41, C01, B33, B86) achieves 3203 importance without A77 or A04. If B41 and C01 are non-negotiable, we should exclude A77 and A04 to maximize my total.'}  \\ \tt REFUSE: {'B41', 'C01', 'A77', 'B33', 'B86', 'B35', 'A04', 'B32'}  \\ \tt ARGUMENT: {'This proposal sacrifices high-importance items (B41, C01) for low-value ones (A77, A04). For example, B41 provides 583 importance at 742 effort, while A77 only gives 65. Rejecting this preserves my focus on maximizing importance-to-effort ratios. If B41 and C01 are non-negotiable, we can agree on a set excluding A77 and A04 to achieve a higher total (e.g., 3203 vs. 3389).'} 
	  } 
	   } 
	   } 
	 & & \\ 
 

    \theutterance \stepcounter{utterance}  

    & & \multicolumn{4}{p{0.6\linewidth}}{\cellcolor[rgb]{0.9,0.9,0.9}{%
	\makecell[{{p{\linewidth}}}]{% 
	  \tt {\tiny [GM$\rangle$B]}  
	 REFUSE: {'B41', 'C01', 'A77', 'B33', 'B86', 'B35', 'A04', 'B32'}  \\ \tt ARGUMENT: {'This proposal sacrifices high-importance items (B41, C01) for low-value ones (A77, A04). For example, B41 provides 583 importance at 742 effort, while A77 only gives 65. Rejecting this preserves my focus on maximizing importance-to-effort ratios. If B41 and C01 are non-negotiable, we can agree on a set excluding A77 and A04 to achieve a higher total (e.g., 3203 vs. 3389).'} 
	  } 
	   } 
	   } 
	 & \\ 
 

    \theutterance \stepcounter{utterance}  

    & & & \multicolumn{4}{p{0.6\linewidth}}{\cellcolor[rgb]{1,0.85,0.72}{%
	\makecell[{{p{\linewidth}}}]{% 
	  \tt {\tiny [GM$\langle$B]}  
	 STRATEGIC REASONING: {'The opponent’s refusal indicates B41 and C01 are critical to their priorities. To secure agreement, I will include these items while retaining my highest-value assets (B33, B35, B86, B32) and adding A03 (783 importance) to maximize my total. This balances their demands with my objectives.'}  \\ \tt PROPOSAL: {'B41', 'C01', 'B33', 'B35', 'B86', 'B32', 'A03'}  \\ \tt ARGUMENT: {'This proposal includes B41 and C01 to address your priorities while retaining my highest-importance items (B33, B35, B86, B32). Adding A03 (783 importance) further boosts my total (4180) without exceeding the limit (3377 effort). If B41 and C01 are non-negotiable, this set balances your needs with my objectives.'} 
	  } 
	   } 
	   } 
	  \\ 
 

    \theutterance \stepcounter{utterance}  

    & & \multicolumn{4}{p{0.6\linewidth}}{\cellcolor[rgb]{0.9,0.9,0.9}{%
	\makecell[{{p{\linewidth}}}]{% 
	  \tt {\tiny [A$\langle$GM]}  
	 PROPOSAL: {'B41', 'C01', 'B33', 'B35', 'B86', 'B32', 'A03'}  \\ \tt ARGUMENT: {'This proposal includes B41 and C01 to address your priorities while retaining my highest-importance items (B33, B35, B86, B32). Adding A03 (783 importance) further boosts my total (4180) without exceeding the limit (3377 effort). If B41 and C01 are non-negotiable, this set balances your needs with my objectives.'} 
	  } 
	   } 
	   } 
	 & \\ 
 

    \theutterance \stepcounter{utterance}  

    & \multicolumn{4}{p{0.6\linewidth}}{\cellcolor[rgb]{0.8,1,0.9}{%
	\makecell[{{p{\linewidth}}}]{% 
	  \tt {\tiny [A$\rangle$GM]}  
	 STRATEGIC REASONING: {'The opponent’s proposal includes B41 and C01 (which I value) along with A03 (262 importance) and B32 (484 importance), which are relatively high-value for me. While A77 and A04 are excluded (good), the inclusion of A03 and B32 increases my total importance to 3949 without exceeding the limit. This set balances your priorities (B41, C01) with my objectives, making it a mutually beneficial agreement.'}  \\ \tt AGREE: {'B41', 'C01', 'B33', 'B35', 'B86', 'B32', 'A03'} 
	  } 
	   } 
	   } 
	 & & \\ 
 

    \theutterance \stepcounter{utterance}  

    & & & \multicolumn{2}{p{0.3\linewidth}}{\cellcolor[rgb]{0.95,0.95,0.95}{%
	\makecell[{{p{\linewidth}}}]{% 
	  \tt {\tiny [GM$|$GM]}  
	 abort 
	  } 
	   } 
	   } 
	 & & \\ 
 

    \theutterance \stepcounter{utterance}  

    & & & \multicolumn{2}{p{0.3\linewidth}}{\cellcolor[rgb]{0.95,0.95,0.95}{%
	\makecell[{{p{\linewidth}}}]{% 
	  \tt {\tiny [GM$|$GM]}  
	 end game 
	  } 
	   } 
	   } 
	 & & \\ 
 

\end{supertabular}
}

\end{document}
