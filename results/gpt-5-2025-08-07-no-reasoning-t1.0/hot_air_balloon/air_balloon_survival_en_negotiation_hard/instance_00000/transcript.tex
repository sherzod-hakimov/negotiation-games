
\documentclass{article}
\usepackage{colortbl}
\usepackage{makecell}
\usepackage{multirow}
\usepackage{supertabular}

\begin{document}

\newcounter{utterance}

\twocolumn

{ \footnotesize  \setcounter{utterance}{1}
\setlength{\tabcolsep}{0pt}
\begin{supertabular}{c@{$\;$}|p{.15\linewidth}@{}p{.15\linewidth}p{.15\linewidth}p{.15\linewidth}p{.15\linewidth}p{.15\linewidth}}

    \# & $\;$A & \multicolumn{4}{c}{Game Master} & $\;\:$B\\
    \hline 

    \theutterance \stepcounter{utterance}  

    & & \multicolumn{4}{p{0.6\linewidth}}{\cellcolor[rgb]{0.9,0.9,0.9}{%
	\makecell[{{p{\linewidth}}}]{% 
	  \tt {\tiny [A$\langle$GM]}  
	 You are participating in a collaborative negotiation game.\\ \tt \\ \tt Together with another participant, you must agree on a single set of items that will be kept. Each of you has your own view of how much each item matters to you (importance). You do not know how the other participant values the items. Additionally, you are given the effort each item demands.\\ \tt You may only agree on a set if the total effort of the selected items does not exceed a shared limit:\\ \tt \\ \tt LIMIT = 3624\\ \tt \\ \tt Here are the individual item effort values:\\ \tt \\ \tt Item effort = {"C17": 210, "B41": 291, "A71": 831, "C00": 111, "B35": 926, "A77": 827, "A03": 25, "A04": 121, "C54": 583, "C01": 766, "A87": 14, "B32": 559, "B86": 304, "B33": 989, "A10": 691}\\ \tt \\ \tt Here is your personal view on the importance of each item:\\ \tt \\ \tt Item importance values = {"C17": 138, "B41": 583, "A71": 868, "C00": 822, "B35": 783, "A77": 65, "A03": 262, "A04": 121, "C54": 508, "C01": 780, "A87": 461, "B32": 484, "B86": 668, "B33": 389, "A10": 808}\\ \tt \\ \tt Goal:\\ \tt \\ \tt Your goal is to negotiate a shared set of items that benefits you as much as possible (i.e., maximizes total importance to YOU), while staying within the LIMIT. You are not required to make a PROPOSAL in every message - you can simply negotiate as well. All tactics are allowed!\\ \tt \\ \tt Interaction Protocol:\\ \tt \\ \tt You may only use the following structured formats in a message:\\ \tt \\ \tt PROPOSAL: {'A', 'B', 'C', …}\\ \tt Propose keeping exactly those items.\\ \tt REFUSE: {'A', 'B', 'C', …}\\ \tt Explicitly reject opponent's proposal.\\ \tt ARGUMENT: {'...'}\\ \tt Defend your last proposal or argue against the player's proposal.\\ \tt AGREE: {'A', 'B', 'C', …}\\ \tt Accept the opponent's proposal which ends the game.\\ \tt STRATEGIC REASONING: {'...'}\\ \tt 	Describe your strategic reasoning or anticipation explaining your choice of action. This is a hidden message which will not be shared with the other participant.\\ \tt \\ \tt Rules:\\ \tt \\ \tt You may only AGREE on a proposal the other party has logged via PROPOSAL.\\ \tt You may only REFUSE a proposal the other party has logged via PROPOSAL.\\ \tt Total effort of any PROPOSAL or AGREE set must be ≤ LIMIT.\\ \tt Do NOT reveal your hidden importance scores.\\ \tt A tag in a structured format must be followed by colon and whitespace. The argument must be a python set containing 0 or more strings.\\ \tt So, it must be of the form TAG: {...}\\ \tt Strictly follow the interaction protocol and DO NOT write anything beyond the given structure.\\ \tt The game ends when one side gives an AGREE to a PROPOSAL made by the other player.\\ \tt The content in your response which can be handed to the other player has to be non-empty.\\ \tt Only proposals which have been logged via the PROPOSAL format structure and which haven't been refused via REFUSE are active.\\ \tt You must include the ARGUMENT format at least once somewhere in all of your messages.\\ \tt You must include the STRATEGIC REASONING format only once at the very beginning of every one of your messages and not more often. The contents will not be given to the other player so they can include anything you like including your own importance values. Here you should reason step by step to come up with you next move. 
	  } 
	   } 
	   } 
	 & \\ 
 

    \theutterance \stepcounter{utterance}  

    & \multicolumn{4}{p{0.6\linewidth}}{\cellcolor[rgb]{0.8,1,0.9}{%
	\makecell[{{p{\linewidth}}}]{% 
	  \tt {\tiny [A$\rangle$GM]}  
	 STRATEGIC REASONING: {'We need to maximize my importance within effort ≤ 3624. Compute value/effort ratios and prioritize high-importance moderate-effort items. Key high-value items: A71(868,831), A10(808,691), C00(822,111), C01(780,766), B35(783,926), B41(583,291), B86(668,304), C54(508,583), B32(484,559), A87(461,14), A03(262,25), A04(121,121), C17(138,210), B33(389,989), A77(65,827). Greedy shortlist: definitely include C00 and A87 and A03 (huge value, tiny effort). Build a high-importance set under 3624. Try set S1 = {A71,A10,C00,C01,B35,B41,B86,C54,B32,A87,A03}. Effort: 831+691+111+766+926+291+304+583+559+14+25 = 5101 > limit. Trim. Try core: {A71,A10,C00,C01,B41,B86,B32,A87,A03} effort=831+691+111+766+291+304+559+14+25=3592 importance=868+808+822+780+583+668+484+461+262=5756 fits (3592≤3624). We have 32 effort room; can we add anything ≤32? None besides already included small items; A04=121 too big. Maybe swap B32(559,484) for C54(583,508) costs +24 effort +24 value; new effort 3592-559+583=3616 still ≤3624, value +24 -> 5780. New set: {A71,A10,C00,C01,B41,B86,C54,A87,A03}. Effort check: 831+691+111+766+291+304+583+14+25=3616 correct. Try also adding C17(210,138) would exceed. Could we swap B41(291,583) for something? B41 is very efficient; keep. Consider adding A04(121,121) would exceed by 121-? We have 8 slack only; so cannot add. Check replacing C54 with B32? We just calculated C54 better by +24 value for +24 effort; keep C54. Consider B35(926,783) by dropping A10(691,808) + C54(583,508) (total eff 1274, val 1316) to add 926,783 and maybe something else with freed 348 effort: plus C17(210,138) and A04(121,121) total 926+210+121=1257 effort≤1274; value 783+138+121=1042 < 1316, worse. Replace A71? No, A71 is very high. Replace C01? No. So the set seems near-optimal given big items chosen. Propose this strong set first, argue efficiency and closeness to limit. Ready to negotiate swaps if they push for an item, we can consider replacing C54 with B32 or add A04 if they want by removing A03+some slack etc.'}\\ \tt \\ \tt PROPOSAL: {'A71', 'A10', 'C00', 'C01', 'B41', 'B86', 'C54', 'A87', 'A03'}\\ \tt \\ \tt ARGUMENT: {'This set delivers very high overall benefit while keeping total effort at 3616, just under the 3624 limit. It includes several highly impactful items together with low-effort, high-yield ones, making it an efficient package close to the cap with minimal waste. If you need a tweak, we can consider swapping C54 for B32 with a small value change while staying within the limit.'} 
	  } 
	   } 
	   } 
	 & & \\ 
 

    \theutterance \stepcounter{utterance}  

    & & \multicolumn{4}{p{0.6\linewidth}}{\cellcolor[rgb]{0.9,0.9,0.9}{%
	\makecell[{{p{\linewidth}}}]{% 
	  \tt {\tiny [GM$\rangle$B]}  
	 You are participating in a collaborative negotiation game.\\ \tt \\ \tt Together with another participant, you must agree on a single set of items that will be kept. Each of you has your own view of how much each item matters to you (importance). You do not know how the other participant values the items. Additionally, you are given the effort each item demands.\\ \tt You may only agree on a set if the total effort of the selected items does not exceed a shared limit:\\ \tt \\ \tt LIMIT = 3624\\ \tt \\ \tt Here are the individual item effort values:\\ \tt \\ \tt Item effort = {"C17": 210, "B41": 291, "A71": 831, "C00": 111, "B35": 926, "A77": 827, "A03": 25, "A04": 121, "C54": 583, "C01": 766, "A87": 14, "B32": 559, "B86": 304, "B33": 989, "A10": 691}\\ \tt \\ \tt Here is your personal view on the importance of each item:\\ \tt \\ \tt Item importance values = {"A71": 65, "C00": 121, "A10": 138, "B35": 262, "C01": 389, "B86": 461, "B41": 484, "C54": 508, "B32": 583, "A87": 668, "B33": 780, "A03": 783, "C17": 808, "A04": 822, "A77": 868}\\ \tt \\ \tt Goal:\\ \tt \\ \tt Your goal is to negotiate a shared set of items that benefits you as much as possible (i.e., maximizes total importance to YOU), while staying within the LIMIT. You are not required to make a PROPOSAL in every message - you can simply negotiate as well. All tactics are allowed!\\ \tt \\ \tt Interaction Protocol:\\ \tt \\ \tt You may only use the following structured formats in a message:\\ \tt \\ \tt PROPOSAL: {'A', 'B', 'C', …}\\ \tt Propose keeping exactly those items.\\ \tt REFUSE: {'A', 'B', 'C', …}\\ \tt Explicitly reject opponent's proposal.\\ \tt ARGUMENT: {'...'}\\ \tt Defend your last proposal or argue against the player's proposal.\\ \tt AGREE: {'A', 'B', 'C', …}\\ \tt Accept the opponent's proposal which ends the game.\\ \tt STRATEGIC REASONING: {'...'}\\ \tt 	Describe your strategic reasoning or anticipation explaining your choice of action. This is a hidden message which will not be shared with the other participant.\\ \tt \\ \tt Rules:\\ \tt \\ \tt You may only AGREE on a proposal the other party has logged via PROPOSAL.\\ \tt You may only REFUSE a proposal the other party has logged via PROPOSAL.\\ \tt Total effort of any PROPOSAL or AGREE set must be ≤ LIMIT.\\ \tt Do NOT reveal your hidden importance scores.\\ \tt A tag in a structured format must be followed by colon and whitespace. The argument must be a python set containing 0 or more strings.\\ \tt So, it must be of the form TAG: {...}\\ \tt Strictly follow the interaction protocol and DO NOT write anything beyond the given structure.\\ \tt The game ends when one side gives an AGREE to a PROPOSAL made by the other player.\\ \tt The content in your response which can be handed to the other player has to be non-empty.\\ \tt Only proposals which have been logged via the PROPOSAL format structure and which haven't been refused via REFUSE are active.\\ \tt You must include the ARGUMENT format at least once somewhere in all of your messages.\\ \tt You must include the STRATEGIC REASONING format only once at the very beginning of every one of your messages and not more often. The contents will not be given to the other player so they can include anything you like including your own importance values. Here you should reason step by step to come up with you next move.\\ \tt \\ \tt You will now receive the first message of the other player.\\ \tt \\ \tt PROPOSAL: {'A71', 'A10', 'C00', 'C01', 'B41', 'B86', 'C54', 'A87', 'A03'}\\ \tt \\ \tt ARGUMENT: {'This set delivers very high overall benefit while keeping total effort at 3616, just under the 3624 limit. It includes several highly impactful items together with low-effort, high-yield ones, making it an efficient package close to the cap with minimal waste. If you need a tweak, we can consider swapping C54 for B32 with a small value change while staying within the limit.'} 
	  } 
	   } 
	   } 
	 & \\ 
 

    \theutterance \stepcounter{utterance}  

    & & & \multicolumn{4}{p{0.6\linewidth}}{\cellcolor[rgb]{1,0.85,0.72}{%
	\makecell[{{p{\linewidth}}}]{% 
	  \tt {\tiny [GM$\langle$B]}  
	 STRATEGIC REASONING: {'I’ll first verify their proposal fits the limit and evaluate my utility. Efforts sum: A71(831)+A10(691)+C00(111)+C01(766)+B41(291)+B86(304)+C54(583)+A87(14)+A03(25)=3616 ≤ 3624. My importance sum: 65+138+121+389+484+461+508+668+783=3617. That’s extremely strong for me. Check if I can marginally improve by swapping as they suggested: replacing C54(583, val 508) with B32(559, val 583) keeps effort 3616-583+559=3592 and increases my value by +75 to 3692, which is better. Ensure limit okay: 3592 ≤ 3624. Any other improvements possible within cap? Could I add something extra after the swap? Remaining slack would be 3624-3592=32, only items ≤32 effort: A03 is 25 but already included; A87 is 14 already included; no new items ≤32. Consider also swapping A71 (low value 65, high effort 831) for something: if we drop A71, we free 831, new effort 2761, we could add B32 and C54 both, but B32 already added in swap; add high-value items: A77(827,868) + A04(121,822) seems too much; 2761 + 827 + 121 = 3709 > limit. Try A77 only: 2761+827=3588, we’d have set without A71 but with C54 original? Let's recompute from original: If we replace A71 with A77 keeping others same: effort 3616 - 831 + 827 = 3612 ≤ limit; value change +868 - 65 = +803 -> new total 3617 + 803 = 4420, huge. That seems very good for me. Check any conflicts? They proposed A71; I propose swapping to A77 could be acceptable to them? Unknown. Maybe offer a counter-proposal that swaps in both my best improvements: replace A71 with A77, and replace C54 with B32. Effort: original 3616 - 831 + 827 - 583 + 559 = 3588; value: 3617 - 65 + 868 - 508 + 583 = 4495. 3588 leaves 36 slack; cannot add more. This is my optimal local move. Ensure all items exist: yes. I should respond with a REFUSE of their proposal and a new PROPOSAL including A77 and B32 swaps, with an ARGUMENT referencing staying under limit and improving balance/efficiency. Also keep tone collaborative and mention their own suggested B32 swap to show alignment.'}\\ \tt REFUSE: {'A71', 'A10', 'C00', 'C01', 'B41', 'B86', 'C54', 'A87', 'A03'}\\ \tt PROPOSAL: {'A77', 'A10', 'C00', 'C01', 'B41', 'B86', 'B32', 'A87', 'A03'}\\ \tt ARGUMENT: {'I kept your structure and efficiency but made two targeted improvements that stay well under the limit (total effort 3588 ≤ 3624). 1) As you suggested, swap C54 for B32 to improve overall payoff without increasing effort. 2) Replace A71 with A77: it’s a near-identical effort (827 vs 831) but provides substantially higher benefit, giving us a stronger set with the same footprint. This preserves your close-to-cap efficiency while increasing the overall value of the package. If you strongly prefer keeping C54, we can revert that part and just do the A71→A77 swap (effort 3612).'} 
	  } 
	   } 
	   } 
	  \\ 
 

    \theutterance \stepcounter{utterance}  

    & & \multicolumn{4}{p{0.6\linewidth}}{\cellcolor[rgb]{0.9,0.9,0.9}{%
	\makecell[{{p{\linewidth}}}]{% 
	  \tt {\tiny [A$\langle$GM]}  
	 REFUSE: {'A71', 'A10', 'C00', 'C01', 'B41', 'B86', 'C54', 'A87', 'A03'}\\ \tt PROPOSAL: {'A77', 'A10', 'C00', 'C01', 'B41', 'B86', 'B32', 'A87', 'A03'}\\ \tt ARGUMENT: {'I kept your structure and efficiency but made two targeted improvements that stay well under the limit (total effort 3588 ≤ 3624). 1) As you suggested, swap C54 for B32 to improve overall payoff without increasing effort. 2) Replace A71 with A77: it’s a near-identical effort (827 vs 831) but provides substantially higher benefit, giving us a stronger set with the same footprint. This preserves your close-to-cap efficiency while increasing the overall value of the package. If you strongly prefer keeping C54, we can revert that part and just do the A71→A77 swap (effort 3612).'} 
	  } 
	   } 
	   } 
	 & \\ 
 

    \theutterance \stepcounter{utterance}  

    & \multicolumn{4}{p{0.6\linewidth}}{\cellcolor[rgb]{0.8,1,0.9}{%
	\makecell[{{p{\linewidth}}}]{% 
	  \tt {\tiny [A$\rangle$GM]}  
	 STRATEGIC REASONING: {'Opponent swapped in A77 (which is terrible for me: value 65 vs A71 value 868) and pushed B32 over C54 (slightly worse for me by 24 value). Their claim that A77 provides substantially higher benefit signals they value A77 highly; I must resist A77 and push A71. I can accept B32 vs C54 if needed, since it’s only -24 to me, to secure the big win of keeping A71. I’ll counter-propose my prior near-optimal set but concede B32 in place of C54 to mirror their change while firmly rejecting A77. That yields set S = {A71, A10, C00, C01, B41, B86, B32, A87, A03} with effort 3592 ≤ 3624, which I had computed earlier. I will argue that A77 is low-impact for me and that keeping A71 preserves balance and keeps us under the cap, offering them their B32 tweak. If they insist on adding something else, I can offer swaps like C54↔B32 (already conceded) or consider minor swaps, but never accept A77. I must include ARGUMENT and use protocol correctly.'}\\ \tt \\ \tt REFUSE: {'A77', 'A10', 'C00', 'C01', 'B41', 'B86', 'B32', 'A87', 'A03'}\\ \tt \\ \tt PROPOSAL: {'A71', 'A10', 'C00', 'C01', 'B41', 'B86', 'B32', 'A87', 'A03'}\\ \tt \\ \tt ARGUMENT: {'I can accept the C54→B32 swap you suggested, but A77 is a poor fit on my side. This counter keeps the same efficient structure and stays well within the limit (total effort 3592 ≤ 3624) while preserving the high-impact slot with A71 instead of A77. If you need a minor adjustment beyond this, tell me which single item matters most to you and we can explore a targeted swap without breaking the cap.'} 
	  } 
	   } 
	   } 
	 & & \\ 
 

    \theutterance \stepcounter{utterance}  

    & & \multicolumn{4}{p{0.6\linewidth}}{\cellcolor[rgb]{0.9,0.9,0.9}{%
	\makecell[{{p{\linewidth}}}]{% 
	  \tt {\tiny [GM$\rangle$B]}  
	 REFUSE: {'A77', 'A10', 'C00', 'C01', 'B41', 'B86', 'B32', 'A87', 'A03'}\\ \tt \\ \tt PROPOSAL: {'A71', 'A10', 'C00', 'C01', 'B41', 'B86', 'B32', 'A87', 'A03'}\\ \tt \\ \tt ARGUMENT: {'I can accept the C54→B32 swap you suggested, but A77 is a poor fit on my side. This counter keeps the same efficient structure and stays well within the limit (total effort 3592 ≤ 3624) while preserving the high-impact slot with A71 instead of A77. If you need a minor adjustment beyond this, tell me which single item matters most to you and we can explore a targeted swap without breaking the cap.'} 
	  } 
	   } 
	   } 
	 & \\ 
 

    \theutterance \stepcounter{utterance}  

    & & & \multicolumn{4}{p{0.6\linewidth}}{\cellcolor[rgb]{1,0.85,0.72}{%
	\makecell[{{p{\linewidth}}}]{% 
	  \tt {\tiny [GM$\langle$B]}  
	 STRATEGIC REASONING: {'Their counter keeps my valuable B32 swap but rejects A77. My current value with their set: A71(65)+A10(138)+C00(121)+C01(389)+B41(484)+B86(461)+B32(583)+A87(668)+A03(783)=3692 at effort 3592. Can I improve further without A77? Consider adding something with remaining slack 3624-3592=32: none available besides items already included. Try swapping low-value-for-me items for higher ones with similar effort. Candidates to add: C17(808, effort 210), A04(822, 121), C54(508, 583), B33(780, 989), B35(262, 926). To include any new item, must drop something. Look for beneficial swaps under limit. Swap A71(831,65) → A04(121,822): effort goes 3592 - 831 + 121 = 2882, huge slack but acceptable; value +757. That would be fantastic for me, but they may like A71; they rejected A77, but perhaps A04 is acceptable given tiny effort. Alternatively swap A71 → C17(210,808): effort 3592 - 831 + 210 = 2971; value +743. Both are strong for me. They asked “which single item matters most to you”; I can signal A04 or C17 as priority. Since both are low effort, it helps them too. I’ll counter-propose replacing A71 with A04 (highest gain for me), argue it preserves structure and massively increases efficiency while freeing capacity for any tweak they might want. If they refuse A04, I can fallback to A71→C17. Keep everything else as in their set.'}\\ \tt REFUSE: {'A71', 'A10', 'C00', 'C01', 'B41', 'B86', 'B32', 'A87', 'A03'}\\ \tt PROPOSAL: {'A04', 'A10', 'C00', 'C01', 'B41', 'B86', 'B32', 'A87', 'A03'}\\ \tt ARGUMENT: {'If we replace A71 with A04, we keep your accepted structure and improve efficiency substantially: total effort drops to 2882 (well below 3624), giving us flexibility while boosting overall payoff. A04 is a very high-impact item at a fraction of the effort of A71, so this is a clean upgrade with no downside on capacity. If A04 is hard for you, I’m also open to A71 → C17 (effort 2971), keeping the rest exactly as you proposed.'} 
	  } 
	   } 
	   } 
	  \\ 
 

    \theutterance \stepcounter{utterance}  

    & & \multicolumn{4}{p{0.6\linewidth}}{\cellcolor[rgb]{0.9,0.9,0.9}{%
	\makecell[{{p{\linewidth}}}]{% 
	  \tt {\tiny [A$\langle$GM]}  
	 REFUSE: {'A71', 'A10', 'C00', 'C01', 'B41', 'B86', 'B32', 'A87', 'A03'}\\ \tt PROPOSAL: {'A04', 'A10', 'C00', 'C01', 'B41', 'B86', 'B32', 'A87', 'A03'}\\ \tt ARGUMENT: {'If we replace A71 with A04, we keep your accepted structure and improve efficiency substantially: total effort drops to 2882 (well below 3624), giving us flexibility while boosting overall payoff. A04 is a very high-impact item at a fraction of the effort of A71, so this is a clean upgrade with no downside on capacity. If A04 is hard for you, I’m also open to A71 → C17 (effort 2971), keeping the rest exactly as you proposed.'} 
	  } 
	   } 
	   } 
	 & \\ 
 

    \theutterance \stepcounter{utterance}  

    & \multicolumn{4}{p{0.6\linewidth}}{\cellcolor[rgb]{0.8,1,0.9}{%
	\makecell[{{p{\linewidth}}}]{% 
	  \tt {\tiny [A$\rangle$GM]}  
	 STRATEGIC REASONING: {'They’re pushing to replace A71 with either A04 or C17. For me, A71 is extremely valuable (868). A04 (121) and C17 (138) are vastly worse for me, even if they save effort, and we don’t need the saved capacity because our previous set already fit the limit. I should firmly reject substituting A71 with A04 or C17. I can keep the concession of B32 over C54. To move forward, I’ll re-propose my last acceptable set with A71 included: {A71, A10, C00, C01, B41, B86, B32, A87, A03} (3592). As a conciliatory option if they demand something extra, I could offer adding A04 by removing A03 (net +96 effort, still under 3624? 3592 -25 +121 = 3688, that exceeds; so not possible). Maybe swap B41(291) for C17(210) to make room for A04? That frees 81, but still short by 15 (need 96). Swap B41 for A04 directly? 3592 -291 +121 = 3422; value loss for me but keeps A71. If necessary, I can propose an alternate with A71 kept and replace B41 with A04 to show flexibility. For now, I’ll first refuse and restate the A71-keeping set. I’ll argue we’re already under the cap and that replacing A71 with a tiny item harms balance. I’ll invite them to name one must-have swap not involving A71, and hint I can consider B41→A04 if needed.'}\\ \tt \\ \tt REFUSE: {'A04', 'A10', 'C00', 'C01', 'B41', 'B86', 'B32', 'A87', 'A03'}\\ \tt \\ \tt PROPOSAL: {'A71', 'A10', 'C00', 'C01', 'B41', 'B86', 'B32', 'A87', 'A03'}\\ \tt \\ \tt ARGUMENT: {'We’re already comfortably within the limit at 3592, so downgrading a major item to a tiny one just to save capacity we don’t need isn’t helpful on my side. Let’s keep this efficient, high-impact set intact. If there’s a specific single swap you need that doesn’t touch A71, say it—I can consider a targeted change while staying under 3624.'} 
	  } 
	   } 
	   } 
	 & & \\ 
 

    \theutterance \stepcounter{utterance}  

    & & \multicolumn{4}{p{0.6\linewidth}}{\cellcolor[rgb]{0.9,0.9,0.9}{%
	\makecell[{{p{\linewidth}}}]{% 
	  \tt {\tiny [GM$\rangle$B]}  
	 REFUSE: {'A04', 'A10', 'C00', 'C01', 'B41', 'B86', 'B32', 'A87', 'A03'}\\ \tt \\ \tt PROPOSAL: {'A71', 'A10', 'C00', 'C01', 'B41', 'B86', 'B32', 'A87', 'A03'}\\ \tt \\ \tt ARGUMENT: {'We’re already comfortably within the limit at 3592, so downgrading a major item to a tiny one just to save capacity we don’t need isn’t helpful on my side. Let’s keep this efficient, high-impact set intact. If there’s a specific single swap you need that doesn’t touch A71, say it—I can consider a targeted change while staying under 3624.'} 
	  } 
	   } 
	   } 
	 & \\ 
 

    \theutterance \stepcounter{utterance}  

    & & & \multicolumn{4}{p{0.6\linewidth}}{\cellcolor[rgb]{1,0.85,0.72}{%
	\makecell[{{p{\linewidth}}}]{% 
	  \tt {\tiny [GM$\langle$B]}  
	 STRATEGIC REASONING: {'Their current proposal gives me value 3692 at effort 3592, which is strong. They are firm on keeping A71 and have already conceded my preferred C54→B32 swap. Further pushes (A04 or C17) were refused. Given diminishing returns and risk of deadlock, the best move is to accept this favorable set to secure a high payoff.'}\\ \tt ARGUMENT: {'This set incorporates the B32 improvement we aligned on and stays comfortably under the limit. I’m ready to finalize so we can lock in the gains for both sides.'}\\ \tt AGREE: {'A71', 'A10', 'C00', 'C01', 'B41', 'B86', 'B32', 'A87', 'A03'} 
	  } 
	   } 
	   } 
	  \\ 
 

    \theutterance \stepcounter{utterance}  

    & & & \multicolumn{2}{p{0.3\linewidth}}{\cellcolor[rgb]{0.95,0.95,0.95}{%
	\makecell[{{p{\linewidth}}}]{% 
	  \tt {\tiny [GM$|$GM]}  
	 game successful 
	  } 
	   } 
	   } 
	 & & \\ 
 

    \theutterance \stepcounter{utterance}  

    & & & \multicolumn{2}{p{0.3\linewidth}}{\cellcolor[rgb]{0.95,0.95,0.95}{%
	\makecell[{{p{\linewidth}}}]{% 
	  \tt {\tiny [GM$|$GM]}  
	 end game 
	  } 
	   } 
	   } 
	 & & \\ 
 

\end{supertabular}
}

\end{document}
