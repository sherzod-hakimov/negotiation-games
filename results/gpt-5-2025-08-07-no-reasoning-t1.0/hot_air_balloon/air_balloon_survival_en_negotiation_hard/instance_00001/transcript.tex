
\documentclass{article}
\usepackage{colortbl}
\usepackage{makecell}
\usepackage{multirow}
\usepackage{supertabular}

\begin{document}

\newcounter{utterance}

\twocolumn

{ \footnotesize  \setcounter{utterance}{1}
\setlength{\tabcolsep}{0pt}
\begin{supertabular}{c@{$\;$}|p{.15\linewidth}@{}p{.15\linewidth}p{.15\linewidth}p{.15\linewidth}p{.15\linewidth}p{.15\linewidth}}

    \# & $\;$A & \multicolumn{4}{c}{Game Master} & $\;\:$B\\
    \hline 

    \theutterance \stepcounter{utterance}  

    & & \multicolumn{4}{p{0.6\linewidth}}{\cellcolor[rgb]{0.9,0.9,0.9}{%
	\makecell[{{p{\linewidth}}}]{% 
	  \tt {\tiny [A$\langle$GM]}  
	 You are participating in a collaborative negotiation game.\\ \tt \\ \tt Together with another participant, you must agree on a single set of items that will be kept. Each of you has your own view of how much each item matters to you (importance). You do not know how the other participant values the items. Additionally, you are given the effort each item demands.\\ \tt You may only agree on a set if the total effort of the selected items does not exceed a shared limit:\\ \tt \\ \tt LIMIT = 4026\\ \tt \\ \tt Here are the individual item effort values:\\ \tt \\ \tt Item effort = {"C17": 780, "B41": 742, "A71": 997, "C00": 665, "B35": 140, "A77": 77, "A03": 513, "A04": 383, "C54": 587, "C01": 825, "A87": 319, "B32": 448, "B86": 516, "B33": 694, "A10": 366}\\ \tt \\ \tt Here is your personal view on the importance of each item:\\ \tt \\ \tt Item importance values = {"C17": 138, "B41": 583, "A71": 868, "C00": 822, "B35": 783, "A77": 65, "A03": 262, "A04": 121, "C54": 508, "C01": 780, "A87": 461, "B32": 484, "B86": 668, "B33": 389, "A10": 808}\\ \tt \\ \tt Goal:\\ \tt \\ \tt Your goal is to negotiate a shared set of items that benefits you as much as possible (i.e., maximizes total importance to YOU), while staying within the LIMIT. You are not required to make a PROPOSAL in every message - you can simply negotiate as well. All tactics are allowed!\\ \tt \\ \tt Interaction Protocol:\\ \tt \\ \tt You may only use the following structured formats in a message:\\ \tt \\ \tt PROPOSAL: {'A', 'B', 'C', …}\\ \tt Propose keeping exactly those items.\\ \tt REFUSE: {'A', 'B', 'C', …}\\ \tt Explicitly reject opponent's proposal.\\ \tt ARGUMENT: {'...'}\\ \tt Defend your last proposal or argue against the player's proposal.\\ \tt AGREE: {'A', 'B', 'C', …}\\ \tt Accept the opponent's proposal which ends the game.\\ \tt STRATEGIC REASONING: {'...'}\\ \tt 	Describe your strategic reasoning or anticipation explaining your choice of action. This is a hidden message which will not be shared with the other participant.\\ \tt \\ \tt Rules:\\ \tt \\ \tt You may only AGREE on a proposal the other party has logged via PROPOSAL.\\ \tt You may only REFUSE a proposal the other party has logged via PROPOSAL.\\ \tt Total effort of any PROPOSAL or AGREE set must be ≤ LIMIT.\\ \tt Do NOT reveal your hidden importance scores.\\ \tt A tag in a structured format must be followed by colon and whitespace. The argument must be a python set containing 0 or more strings.\\ \tt So, it must be of the form TAG: {...}\\ \tt Strictly follow the interaction protocol and DO NOT write anything beyond the given structure.\\ \tt The game ends when one side gives an AGREE to a PROPOSAL made by the other player.\\ \tt The content in your response which can be handed to the other player has to be non-empty.\\ \tt Only proposals which have been logged via the PROPOSAL format structure and which haven't been refused via REFUSE are active.\\ \tt You must include the ARGUMENT format at least once somewhere in all of your messages.\\ \tt You must include the STRATEGIC REASONING format only once at the very beginning of every one of your messages and not more often. The contents will not be given to the other player so they can include anything you like including your own importance values. Here you should reason step by step to come up with you next move. 
	  } 
	   } 
	   } 
	 & \\ 
 

    \theutterance \stepcounter{utterance}  

    & \multicolumn{4}{p{0.6\linewidth}}{\cellcolor[rgb]{0.8,1,0.9}{%
	\makecell[{{p{\linewidth}}}]{% 
	  \tt {\tiny [A$\rangle$GM]}  
	 STRATEGIC REASONING: {'Maximize my total importance under effort 4026. Compute value/effort ratios and try to include my highest-importance items first while watching the cap. My top values: A71(868,997), A10(808,366), C01(780,825), B35(783,140), C00(822,665), B86(668,516), B41(583,742), C54(508,587), A87(461,319), B32(484,448), B33(389,694), A03(262,513), A04(121,383), A77(65,77), C17(138,780). Greedy by value/effort highlights B35 (5.59), A10 (2.21), A87 (1.45), C00 (1.24), A71 (0.87), C01 (0.95), B86 (1.29), B32 (1.08), C54 (0.87), B41 (0.79). Build a bundle: Start with B35, A10, A87, C00, B86, C01, A71 might bust the cap; check combinations. Try high-value set: {A71,A10,C01,B35,C00,B86} efforts 997+366+825+140+665+516=3509, value 868+808+780+783+822+668=4729; add B32(448)=>3957 value +484=5213; remaining 69 so A77(77) doesn't fit, none small enough. This looks excellent and under 4026. Propose that. Keep some flexibility to swap B32 for C54(587) or B41(742) depending on their interests if needed. Argue efficiency and balanced effort. I won’t reveal values.'}\\ \tt \\ \tt PROPOSAL: {'A71', 'A10', 'C01', 'B35', 'C00', 'B86', 'B32'}\\ \tt \\ \tt ARGUMENT: {'This set stays within the effort limit (total 3957 ≤ 4026) and includes a balanced mix across categories while making very efficient use of the available effort. If you have a strong preference for an alternative in place of B32, I’m open to swapping that slot while keeping the total within the limit.'} 
	  } 
	   } 
	   } 
	 & & \\ 
 

    \theutterance \stepcounter{utterance}  

    & & \multicolumn{4}{p{0.6\linewidth}}{\cellcolor[rgb]{0.9,0.9,0.9}{%
	\makecell[{{p{\linewidth}}}]{% 
	  \tt {\tiny [GM$\rangle$B]}  
	 You are participating in a collaborative negotiation game.\\ \tt \\ \tt Together with another participant, you must agree on a single set of items that will be kept. Each of you has your own view of how much each item matters to you (importance). You do not know how the other participant values the items. Additionally, you are given the effort each item demands.\\ \tt You may only agree on a set if the total effort of the selected items does not exceed a shared limit:\\ \tt \\ \tt LIMIT = 4026\\ \tt \\ \tt Here are the individual item effort values:\\ \tt \\ \tt Item effort = {"C17": 780, "B41": 742, "A71": 997, "C00": 665, "B35": 140, "A77": 77, "A03": 513, "A04": 383, "C54": 587, "C01": 825, "A87": 319, "B32": 448, "B86": 516, "B33": 694, "A10": 366}\\ \tt \\ \tt Here is your personal view on the importance of each item:\\ \tt \\ \tt Item importance values = {"A71": 65, "C00": 121, "A10": 138, "B35": 262, "C01": 389, "B86": 461, "B41": 484, "C54": 508, "B32": 583, "A87": 668, "B33": 780, "A03": 783, "C17": 808, "A04": 822, "A77": 868}\\ \tt \\ \tt Goal:\\ \tt \\ \tt Your goal is to negotiate a shared set of items that benefits you as much as possible (i.e., maximizes total importance to YOU), while staying within the LIMIT. You are not required to make a PROPOSAL in every message - you can simply negotiate as well. All tactics are allowed!\\ \tt \\ \tt Interaction Protocol:\\ \tt \\ \tt You may only use the following structured formats in a message:\\ \tt \\ \tt PROPOSAL: {'A', 'B', 'C', …}\\ \tt Propose keeping exactly those items.\\ \tt REFUSE: {'A', 'B', 'C', …}\\ \tt Explicitly reject opponent's proposal.\\ \tt ARGUMENT: {'...'}\\ \tt Defend your last proposal or argue against the player's proposal.\\ \tt AGREE: {'A', 'B', 'C', …}\\ \tt Accept the opponent's proposal which ends the game.\\ \tt STRATEGIC REASONING: {'...'}\\ \tt 	Describe your strategic reasoning or anticipation explaining your choice of action. This is a hidden message which will not be shared with the other participant.\\ \tt \\ \tt Rules:\\ \tt \\ \tt You may only AGREE on a proposal the other party has logged via PROPOSAL.\\ \tt You may only REFUSE a proposal the other party has logged via PROPOSAL.\\ \tt Total effort of any PROPOSAL or AGREE set must be ≤ LIMIT.\\ \tt Do NOT reveal your hidden importance scores.\\ \tt A tag in a structured format must be followed by colon and whitespace. The argument must be a python set containing 0 or more strings.\\ \tt So, it must be of the form TAG: {...}\\ \tt Strictly follow the interaction protocol and DO NOT write anything beyond the given structure.\\ \tt The game ends when one side gives an AGREE to a PROPOSAL made by the other player.\\ \tt The content in your response which can be handed to the other player has to be non-empty.\\ \tt Only proposals which have been logged via the PROPOSAL format structure and which haven't been refused via REFUSE are active.\\ \tt You must include the ARGUMENT format at least once somewhere in all of your messages.\\ \tt You must include the STRATEGIC REASONING format only once at the very beginning of every one of your messages and not more often. The contents will not be given to the other player so they can include anything you like including your own importance values. Here you should reason step by step to come up with you next move.\\ \tt \\ \tt You will now receive the first message of the other player.\\ \tt \\ \tt PROPOSAL: {'A71', 'A10', 'C01', 'B35', 'C00', 'B86', 'B32'}\\ \tt \\ \tt ARGUMENT: {'This set stays within the effort limit (total 3957 ≤ 4026) and includes a balanced mix across categories while making very efficient use of the available effort. If you have a strong preference for an alternative in place of B32, I’m open to swapping that slot while keeping the total within the limit.'} 
	  } 
	   } 
	   } 
	 & \\ 
 

    \theutterance \stepcounter{utterance}  

    & & & \multicolumn{4}{p{0.6\linewidth}}{\cellcolor[rgb]{1,0.85,0.72}{%
	\makecell[{{p{\linewidth}}}]{% 
	  \tt {\tiny [GM$\langle$B]}  
	 STRATEGIC REASONING: {'Compute my values and consider swaps that increase my total importance without exceeding 4026. Opponent proposed set S0 = {A71, A10, C01, B35, C00, B86, B32}. Efforts: A71 997, A10 366, C01 825, B35 140, C00 665, B86 516, B32 448. Total effort = 3957 (as stated). My importances: A71 65, A10 138, C01 389, B35 262, C00 121, B86 461, B32 583. Total importance(S0)=2019. I should try to replace low-importance A71 (65) and C00 (121) with high-importance items among remaining: A77 868 (77 effort), A04 822 (383), C17 808 (780), A03 783 (513), B33 780 (694), A87 668 (319), C54 508 (587), B41 484 (742), B32 already included, B86 included, A10 included, C01 included, B35 included. Try adding A77 first (huge value, tiny effort). Check if we can remove A71 (997) and add A77 (77), freeing 920 effort and +803 importance. New set S1 = S0 - {A71} + {A77} => effort 3957-997+77=3037; importance 2019-65+868=2822. With 989 effort headroom. Use that to add A04 (383, +822): S2 effort 3420, importance 3644. Headroom 606. Add C17 (780) too big; B33 (694) too big; C54 (587, +508) fits: S3 effort 3420+587=4007 ≤ 4026, importance 3644+508=4152. Headroom 19. This is very strong. Could we also swap out C00 (121, 665 effort) for A87 (319 effort, +547 importance net change if we replace C00 with A87)? Wait difference: Remove C00: -665 effort, -121 importance. Add A87: +319 effort, +668 importance. Net effort -346; net importance +547. Starting from S3 (4007 effort), removing C00 and adding A87 gives 4007-665+319=3661 and importance 4152-121+668=470 - wrong arithmetic; correct: 4152-121+668=4699. That’s even better and still under limit. Could we then add more, maybe B41 (742, +484) with new headroom 4026-3661=365, so B41 doesn’t fit. A03 (513) no; C17 (780) no; B33 (694) no; A04 already in; C54 already in; maybe add A71 back? 997 too big. Maybe add C00 back alongside A87? Would exceed again? Current S after swap: {A10, C01, B35, B86, B32, A77, A04, C54, A87}. Effort tally: A10 366, C01 825, B35 140, B86 516, B32 448, A77 77, A04 383, C54 587, A87 319 = 366+825+140+516+448+77+383+587+319 = 366+825=1191; +140=1331; +516=1847; +448=2295; +77=2372; +383=2755; +587=3342; +319=3661. Yes. Headroom 365. What high-value items under 365 effort remain? None except maybe A77 already included; A10 366 is just over by 1 for extra; B35 140 already in; Could add nothing new. Consider if we can replace something to fit C17 (780, +808) by dropping low-importance C00 already dropped; Next low are A10 138 (366 effort) and B35 262 (140 effort) and C01 389 (825) high value; B86 461; B32 583; A04 822; A77 868; C54 508; A87 668. To fit C17 780, need 415 more effort freed (since headroom 365). Dropping A10 (366) gives headroom 731; then add C17 (780) would exceed by 49. If drop A10 (366) and B35 (140) headroom 365+366+140=871; add C17 780 leaves 91. Importance change: -A10 138 -B35 262 +C17 808 = +408. New importance would be 4699 - 138 - 262 + 808 = 5107. Effort would be 3661 - 366 - 140 + 780 = 3935. That’s even better. Maybe we can also add something with remaining 91 effort—nothing. Could we also try swapping C54 (587, 508) with B33 (694, 780) if headroom allows: current headroom 4026-3935=91, need +107 more to replace 587 with 694 would require 107 extra, cannot. What about swapping B32 (448, 583) with B33 (694, 780) need +246 effort, not available. Swap A04 (383, 822) with B33 needs +311, no. Swap C01 (825, 389) with B33: effort change -131 (since 694 instead of 825) so would free effort and improve importance by +391, but C01 is currently included; to do that we could add B33 and drop C01: effort becomes 3935 - 825 + 694 = 3804; importance 5107 - 389 + 780 = 5498. That’s better. With new headroom 222, maybe add A10 back (366) too big; B35 back (140) feasible: Add B35 would be effort 3804 + 140 = 3944; importance 5498 + 262 = 5760. Headroom 82. Could we also add A10 then? No. Could we add C00 (665) no; C54 already in? Wait after swap we still have C54 yes; Check current set: Starting from S* = {A10, C01, B35, B86, B32, A77, A04, C54, A87} with effort 3661, importance 4699. Then we did drop A10 and B35, add C17: -> S1 = {C01, B86, B32, A77, A04, C54, A87, C17} effort 3935, importance 5107. Then swap C01 for B33: S2 = {B86, B32, A77, A04, C54, A87, C17, B33} effort 3804, importance 5498. Then add B35 back: S3 = {B86, B32, A77, A04, C54, A87, C17, B33, B35} effort 3944, importance 5760. Headroom 82 left. Can we add A10 (366) no; A71 (997) no; C01 (825) no; C00 (665) no; C54 already; A03 513 maybe better than something? Could we replace C54 (587, 508) with A03 (513, 783): effort -74, importance +275. That would be even better and still under limit. Do that: Replace C54 with A03: S4 = {B86, B32, A77, A04, A87, C17, B33, B35, A03}. Effort new: 3944 - 587 + 513 = 3870. Importance new: 5760 - 508 + 783 = 6035. Headroom 156. Any further improvement? Consider adding A10 (366) no; C54 back no; C01 (825) no; C00 (665) no; B41 742 no; C17 already; A71 low value; Could we replace B32 (448, 583) with something better but heavier within headroom: If we drop B32 we free 448+156=604 new capacity, to add maybe C54 587 (+508) would be worse than B32 583; Could add A10 366 and something else? But we'd need to ensure overall importance increases. Replace B32 (583) with A10 (138) is much worse. Replace B32 with C54 (508) worse. Replace B35 (262) with something: If drop B35 (140 effort, 262 importance) we free 296 headroom, total 452; still cannot fit B41 (742) or C01 (825) or C00 (665) or A10 366 could fit but importance 138 < 262 so worse; C54 508 > 262 but effort 587 > 296 freed? With total headroom 156 + 587 - 140 = 603, wait you add C54 by dropping B35 gives net effort +447, not correct; We need to see if we can add any additional item with current headroom 156: Items with effort ≤ 156: B35 is 140 already included; A77 77 already included; nothing else under 156. Consider swapping to fit another big one like B41 (742, 484). Need 742-156=586 extra by dropping something with low importance/efficiency. Maybe swap B32 (583, 448) plus B35 (262, 140) total effort 588, importance 845, to add B41 484 importance would be net -361 importance, bad. To fit C01 (825, 389) similarly bad. C54 587, 508 replacing B35 (262) would need 431 more effort; could also drop A10 not in set. Not promising. Our S4 seems near-optimal for my values while within limit. I will propose S4 and justify with efficiency and willingness to consider tweaks if they have must-haves, especially offering to swap B32 or B35 if needed as they are mid value for me compared to my top picks.'}\\ \tt \\ \tt PROPOSAL: {'B86', 'B32', 'A77', 'A04', 'A87', 'C17', 'B33', 'B35', 'A03'}\\ \tt \\ \tt ARGUMENT: {'I appreciate your efficient baseline. I can keep the total well within the limit while substantially increasing overall value by replacing A71 with A77 and pivoting toward higher-impact items. This set totals 3870 effort (≤ 4026) and keeps a good balance across categories. If you have a must-have from your original proposal, I’m open to exploring a swap that stays within the limit—B32 or B35 would be the most flexible slots on my side.'}\\ \tt \\ \tt REFUSE: {'A71', 'A10', 'C01', 'B35', 'C00', 'B86', 'B32'} 
	  } 
	   } 
	   } 
	  \\ 
 

    \theutterance \stepcounter{utterance}  

    & & \multicolumn{4}{p{0.6\linewidth}}{\cellcolor[rgb]{0.9,0.9,0.9}{%
	\makecell[{{p{\linewidth}}}]{% 
	  \tt {\tiny [A$\langle$GM]}  
	 PROPOSAL: {'B86', 'B32', 'A77', 'A04', 'A87', 'C17', 'B33', 'B35', 'A03'}\\ \tt \\ \tt ARGUMENT: {'I appreciate your efficient baseline. I can keep the total well within the limit while substantially increasing overall value by replacing A71 with A77 and pivoting toward higher-impact items. This set totals 3870 effort (≤ 4026) and keeps a good balance across categories. If you have a must-have from your original proposal, I’m open to exploring a swap that stays within the limit—B32 or B35 would be the most flexible slots on my side.'}\\ \tt \\ \tt REFUSE: {'A71', 'A10', 'C01', 'B35', 'C00', 'B86', 'B32'} 
	  } 
	   } 
	   } 
	 & \\ 
 

    \theutterance \stepcounter{utterance}  

    & \multicolumn{4}{p{0.6\linewidth}}{\cellcolor[rgb]{0.8,1,0.9}{%
	\makecell[{{p{\linewidth}}}]{% 
	  \tt {\tiny [A$\rangle$GM]}  
	 STRATEGIC REASONING: {'Their proposal omits several of my highest-value items (A71, A10, C01, C00) and includes some low-value-to-me picks (C17, A04, A03, B33). I should reject and counter with a package that keeps my top items while offering some overlap and flexibility. My best set found earlier: {A71, A10, C01, B35, C00, B86, B32} at effort 3957. To accommodate them, consider swaps in flexible slots: B32 seems negotiable for them, and they like B35 too but called it flexible; keep B35 since it’s extremely valuable to me. Try adding one of their must-haves while staying under 4026: If I add A77(77) to my set, total becomes 4034 (too high). Swap B32(448) out and add A77(77) and A87(319): new set {A71, A10, C01, B35, C00, B86, A77, A87} effort 997+366+825+140+665+516+77+319=3905, strong value for me and includes two of their items (A77, A87) plus B86 which they proposed. This offers them 3 overlaps (B86, A77, A87) and keeps within limit, making it more palatable. I’ll refuse their proposal and counter with this, and indicate willingness to swap one slot (perhaps A10 or C01) for one of their priorities if they insist, as long as we stay under cap. I’ll argue we’ve increased overlaps and kept efficiency.'}\\ \tt \\ \tt REFUSE: {'B86', 'B32', 'A77', 'A04', 'A87', 'C17', 'B33', 'B35', 'A03'}\\ \tt \\ \tt PROPOSAL: {'A71', 'A10', 'C01', 'B35', 'C00', 'B86', 'A77', 'A87'}\\ \tt \\ \tt ARGUMENT: {'This set totals 3905 effort (≤ 4026) and increases our overlap: it keeps B86 and adds A77 and A87 from your side, while maintaining an efficient overall mix. If you need one more swap to reflect a key priority of yours, I’m open to discussing a single substitution that stays within the limit.'} 
	  } 
	   } 
	   } 
	 & & \\ 
 

    \theutterance \stepcounter{utterance}  

    & & \multicolumn{4}{p{0.6\linewidth}}{\cellcolor[rgb]{0.9,0.9,0.9}{%
	\makecell[{{p{\linewidth}}}]{% 
	  \tt {\tiny [GM$\rangle$B]}  
	 REFUSE: {'B86', 'B32', 'A77', 'A04', 'A87', 'C17', 'B33', 'B35', 'A03'}\\ \tt \\ \tt PROPOSAL: {'A71', 'A10', 'C01', 'B35', 'C00', 'B86', 'A77', 'A87'}\\ \tt \\ \tt ARGUMENT: {'This set totals 3905 effort (≤ 4026) and increases our overlap: it keeps B86 and adds A77 and A87 from your side, while maintaining an efficient overall mix. If you need one more swap to reflect a key priority of yours, I’m open to discussing a single substitution that stays within the limit.'} 
	  } 
	   } 
	   } 
	 & \\ 
 

    \theutterance \stepcounter{utterance}  

    & & & \multicolumn{4}{p{0.6\linewidth}}{\cellcolor[rgb]{1,0.85,0.72}{%
	\makecell[{{p{\linewidth}}}]{% 
	  \tt {\tiny [GM$\langle$B]}  
	 STRATEGIC REASONING: {'Evaluate their new proposal S_op = {A71, A10, C01, B35, C00, B86, A77, A87}. Effort 3905 confirmed. My importance: A71 65, A10 138, C01 389, B35 262, C00 121, B86 461, A77 868, A87 668. Total = 2972. My last proposal S_me had value 6035, huge gap. Try to counter with a nearby set they might accept by keeping some of their picks: they seem to want A71, A10, C01, C00, B35, B86, and accepted A77, A87. Which of theirs are worst for me? A71 (65) and C00 (121) are very weak. Propose swapping out A71 and C00 for B32 (583, 448 effort) and A04 (822, 383 effort) while keeping the rest they want plus A77, A87. Check feasibility: Start from S_op effort 3905. Remove A71 (997) and C00 (665): effort -1662 → 2243. Importance -65 -121 = -186 → 2786 remaining. Add B32 (448, +583) → effort 2691, importance 3369. Add A04 (383, +822) → effort 3074, importance 4191. We still have room 952. With remaining room, can we also upgrade further: add C17 (780, +808) to maximize value: effort 3074+780=3854 ≤ 4026, importance 4191+808=4999. Current set would be {A10, C01, B35, B86, A77, A87, B32, A04, C17}. This is basically my prior core minus A03 and B33; value 4999, still strong and closer to their structure since it keeps many of their items (A10, C01, B35, B86, A77, A87) and only swaps out A71 and C00. They previously refused my larger set but might accept this compromise. I’ll propose this, emphasize it keeps 6 of their 8 items and stays within limit, and offer flexibility on B32 vs A03 if they want a different slot. If they insist on one more swap, I might trade B32 for A10 or consider C54 instead, but B32 is good for me. I’ll refuse their proposal and counter.'}\\ \tt \\ \tt REFUSE: {'A71', 'A10', 'C01', 'B35', 'C00', 'B86', 'A77', 'A87'}\\ \tt \\ \tt PROPOSAL: {'A10', 'C01', 'B35', 'B86', 'A77', 'A87', 'B32', 'A04', 'C17'}\\ \tt \\ \tt ARGUMENT: {'This keeps 6 of your 8 items (A10, C01, B35, B86, A77, A87) and simply upgrades two low-impact slots (A71, C00) to stronger, still-efficient options. Total effort is 3854 (≤ 4026). If you have a must-have alternative for B32 or A04, I’m open to a single swap that preserves the limit.'} 
	  } 
	   } 
	   } 
	  \\ 
 

    \theutterance \stepcounter{utterance}  

    & & \multicolumn{4}{p{0.6\linewidth}}{\cellcolor[rgb]{0.9,0.9,0.9}{%
	\makecell[{{p{\linewidth}}}]{% 
	  \tt {\tiny [A$\langle$GM]}  
	 REFUSE: {'A71', 'A10', 'C01', 'B35', 'C00', 'B86', 'A77', 'A87'}\\ \tt \\ \tt PROPOSAL: {'A10', 'C01', 'B35', 'B86', 'A77', 'A87', 'B32', 'A04', 'C17'}\\ \tt \\ \tt ARGUMENT: {'This keeps 6 of your 8 items (A10, C01, B35, B86, A77, A87) and simply upgrades two low-impact slots (A71, C00) to stronger, still-efficient options. Total effort is 3854 (≤ 4026). If you have a must-have alternative for B32 or A04, I’m open to a single swap that preserves the limit.'} 
	  } 
	   } 
	   } 
	 & \\ 
 

    \theutterance \stepcounter{utterance}  

    & \multicolumn{4}{p{0.6\linewidth}}{\cellcolor[rgb]{0.8,1,0.9}{%
	\makecell[{{p{\linewidth}}}]{% 
	  \tt {\tiny [A$\rangle$GM]}  
	 STRATEGIC REASONING: {'Their counter keeps many of my strong picks but still excludes two of my highest-value items to me (A71 and especially C00). They added C17 and A04, which are relatively low-value to me. I should try to reinsert at least one of my must-haves while staying within the limit and preserving several of their preferences to appear cooperative. Current proposal effort 3854 has slack 172. Replacing low-value items: C17(780) and A04(383) are the heaviest among their adds. If I swap C17(780) → C00(665), effort drops by 115 to 3739, good for me; value increases significantly. I also want A71(997), but to add it I must remove something of ~≥997 effort. Candidates: B32(448)+A04(383)+A77(77)=908 combined; still short by 89, but we have 3739 baseline if we did C17→C00. Starting from their set: Effort 3854. Do swap C17→C00: new effort 3739. To add A71 (+997) we must remove at least 710 effort to stay ≤4026: 3739+997=4736; need -710. Options: remove B32(448) and A04(383) = -831, new effort 3905, which works and keeps many overlaps they like (A10, C01, B35, B86, A77, A87). This yields set {A10, C01, B35, B86, A77, A87, C00, A71} which coincidentally is my previous counter; they refused it earlier, but now framed as a modification of their own proposal it might be more palatable since it removes A04 and C17 (their adds) and inserts my must-haves. They might resist A71, but I should at least push for C00 which is easier (only one swap). As a concession path, propose a middle option: swap just C17→C00 and A04→A71 is larger; maybe propose two alternative packages: primary with both A71 and C00 (effort 3905) and a fallback with only C00 (effort 3739) plus keep B32 and A04. But protocol allows only one PROPOSAL. Better propose the stronger set {A10, C01, B35, B86, A77, A87, C00, A71} with argument that it keeps 6 of theirs and just optimizes two slots, mirroring their language. If they refuse, be ready to offer a single-swap fallback: accept their set with C17→C00.'}\\ \tt \\ \tt REFUSE: {'A10', 'C01', 'B35', 'B86', 'A77', 'A87', 'B32', 'A04', 'C17'}\\ \tt \\ \tt PROPOSAL: {'A10', 'C01', 'B35', 'B86', 'A77', 'A87', 'C00', 'A71'}\\ \tt \\ \tt ARGUMENT: {'To streamline your solid baseline, I’ve just optimized the two least efficient slots, replacing C17 with C00 and A04 with A71. This keeps 6 of your 9 items unchanged and remains within the limit at 3905 effort (≤ 4026). If A71 is a sticking point, I’m willing to accept your set with a single swap of C17 → C00, which would lower the total to 3739 while staying aligned with your structure.'} 
	  } 
	   } 
	   } 
	 & & \\ 
 

    \theutterance \stepcounter{utterance}  

    & & \multicolumn{4}{p{0.6\linewidth}}{\cellcolor[rgb]{0.9,0.9,0.9}{%
	\makecell[{{p{\linewidth}}}]{% 
	  \tt {\tiny [GM$\rangle$B]}  
	 REFUSE: {'A10', 'C01', 'B35', 'B86', 'A77', 'A87', 'B32', 'A04', 'C17'}\\ \tt \\ \tt PROPOSAL: {'A10', 'C01', 'B35', 'B86', 'A77', 'A87', 'C00', 'A71'}\\ \tt \\ \tt ARGUMENT: {'To streamline your solid baseline, I’ve just optimized the two least efficient slots, replacing C17 with C00 and A04 with A71. This keeps 6 of your 9 items unchanged and remains within the limit at 3905 effort (≤ 4026). If A71 is a sticking point, I’m willing to accept your set with a single swap of C17 → C00, which would lower the total to 3739 while staying aligned with your structure.'} 
	  } 
	   } 
	   } 
	 & \\ 
 

    \theutterance \stepcounter{utterance}  

    & & & \multicolumn{4}{p{0.6\linewidth}}{\cellcolor[rgb]{1,0.85,0.72}{%
	\makecell[{{p{\linewidth}}}]{% 
	  \tt {\tiny [GM$\langle$B]}  
	 STRATEGIC REASONING: {'Their pattern shows strong insistence on A71 and C00 and rejection of C17, A04, B32, B33, A03. They keep A10, C01, B35, B86, A77, A87 consistently. My values make A71 (65) and C00 (121) very poor. I should try a minimal-change counter that swaps out exactly one of their weak picks for a high-value item they have been less resistant to. They have not explicitly rejected C54 yet; they only targeted C17, A04, B32, B33, A03. Propose replacing A71 with C54 while keeping their remaining 7 items; check effort: S = {A10, C01, B35, B86, A77, A87, C00, C54}. Effort = 366+825+140+516+77+319+665+587=3495 ≤ 4026. My importance = 138+389+262+461+868+668+121+508=3415, much better than accepting theirs (2972) and close to a plausible compromise. If they refuse, offer their suggested single-swap variant but in my favor: swap A71→A04 (still modest effort change), or A71→B32. Check A71→A04 on their set: 3905-997+383=3291 effort; importance gain +757. That could be a next step. For now, counter with C54 replacing A71, justify as improving efficiency while keeping 7/8 of their items, and note willingness for a single alternative swap (A71→A04) if they prefer.'}\\ \tt \\ \tt REFUSE: {'A10', 'C01', 'B35', 'B86', 'A77', 'A87', 'C00', 'A71'}\\ \tt \\ \tt PROPOSAL: {'A10', 'C01', 'B35', 'B86', 'A77', 'A87', 'C00', 'C54'}\\ \tt \\ \tt ARGUMENT: {'This keeps 7 of your 8 items exactly as proposed and simply upgrades A71 to a more efficient option while staying well within the limit (3495 ≤ 4026). If you’d prefer a different single swap, I can instead replace A71 with A04 and keep everything else unchanged, which would also remain comfortably within the limit.'} 
	  } 
	   } 
	   } 
	  \\ 
 

    \theutterance \stepcounter{utterance}  

    & & \multicolumn{4}{p{0.6\linewidth}}{\cellcolor[rgb]{0.9,0.9,0.9}{%
	\makecell[{{p{\linewidth}}}]{% 
	  \tt {\tiny [A$\langle$GM]}  
	 REFUSE: {'A10', 'C01', 'B35', 'B86', 'A77', 'A87', 'C00', 'A71'}\\ \tt \\ \tt PROPOSAL: {'A10', 'C01', 'B35', 'B86', 'A77', 'A87', 'C00', 'C54'}\\ \tt \\ \tt ARGUMENT: {'This keeps 7 of your 8 items exactly as proposed and simply upgrades A71 to a more efficient option while staying well within the limit (3495 ≤ 4026). If you’d prefer a different single swap, I can instead replace A71 with A04 and keep everything else unchanged, which would also remain comfortably within the limit.'} 
	  } 
	   } 
	   } 
	 & \\ 
 

    \theutterance \stepcounter{utterance}  

    & \multicolumn{4}{p{0.6\linewidth}}{\cellcolor[rgb]{0.8,1,0.9}{%
	\makecell[{{p{\linewidth}}}]{% 
	  \tt {\tiny [A$\rangle$GM]}  
	 STRATEGIC REASONING: {'Their latest proposal includes many of my top-valued items: A10(808), C01(780), B35(783), B86(668), A77(65), A87(461), and crucially C00(822). They swapped out A71(868) for C54(508). My value for the proposed set {A10, C01, B35, B86, A77, A87, C00, C54} is 808+780+783+668+65+461+822+508 = 4895. Effort is 366+825+140+516+77+319+665+587 = 3497 (they stated 3495; by my calc 3497, still ≤4026). Compare to my best previously accepted set value 4729–5213 candidates; this 4895 is very strong and within limit, and the opponent seems firm against A71. Accepting now secures a high-utility outcome without risking further concessions.'}\\ \tt \\ \tt AGREE: {'A10', 'C01', 'B35', 'B86', 'A77', 'A87', 'C00', 'C54'}\\ \tt \\ \tt ARGUMENT: {'Agreed. This set stays well within the limit and preserves the core strengths we both converged on.'} 
	  } 
	   } 
	   } 
	 & & \\ 
 

    \theutterance \stepcounter{utterance}  

    & & & \multicolumn{2}{p{0.3\linewidth}}{\cellcolor[rgb]{0.95,0.95,0.95}{%
	\makecell[{{p{\linewidth}}}]{% 
	  \tt {\tiny [GM$|$GM]}  
	 game successful 
	  } 
	   } 
	   } 
	 & & \\ 
 

    \theutterance \stepcounter{utterance}  

    & & & \multicolumn{2}{p{0.3\linewidth}}{\cellcolor[rgb]{0.95,0.95,0.95}{%
	\makecell[{{p{\linewidth}}}]{% 
	  \tt {\tiny [GM$|$GM]}  
	 end game 
	  } 
	   } 
	   } 
	 & & \\ 
 

\end{supertabular}
}

\end{document}
