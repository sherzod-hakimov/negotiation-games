
\documentclass{article}
\usepackage{colortbl}
\usepackage{makecell}
\usepackage{multirow}
\usepackage{supertabular}

\begin{document}

\newcounter{utterance}

\twocolumn

{ \footnotesize  \setcounter{utterance}{1}
\setlength{\tabcolsep}{0pt}
\begin{supertabular}{c@{$\;$}|p{.15\linewidth}@{}p{.15\linewidth}p{.15\linewidth}p{.15\linewidth}p{.15\linewidth}p{.15\linewidth}}

    \# & $\;$A & \multicolumn{4}{c}{Game Master} & $\;\:$B\\
    \hline 

    \theutterance \stepcounter{utterance}  

    & & \multicolumn{4}{p{0.6\linewidth}}{\cellcolor[rgb]{0.9,0.9,0.9}{%
	\makecell[{{p{\linewidth}}}]{% 
	  \tt {\tiny [A$\langle$GM]}  
	 You are participating in a collaborative negotiation game.\\ \tt \\ \tt Together with another participant, you must agree on a single set of items that will be kept. Each of you has your own view of how much each item matters to you (importance). You do not know how the other participant values the items. Additionally, you are given the effort each item demands.\\ \tt You may only agree on a set if the total effort of the selected items does not exceed a shared limit:\\ \tt \\ \tt LIMIT = 3481\\ \tt \\ \tt Here are the individual item effort values:\\ \tt \\ \tt Item effort = {"C17": 777, "B41": 542, "A71": 332, "C00": 1, "B35": 127, "A77": 453, "A03": 736, "A04": 461, "C54": 359, "C01": 313, "A87": 553, "B32": 409, "B86": 348, "B33": 802, "A10": 749}\\ \tt \\ \tt Here is your personal view on the importance of each item:\\ \tt \\ \tt Item importance values = {"C17": 138, "B41": 583, "A71": 783, "C00": 65, "B35": 262, "A77": 121, "A03": 508, "A04": 780, "C54": 461, "C01": 484, "A87": 668, "B32": 389, "B86": 215, "B33": 97, "A10": 500}\\ \tt \\ \tt Goal:\\ \tt \\ \tt Your goal is to negotiate a shared set of items that benefits you as much as possible (i.e., maximizes total importance to YOU), while staying within the LIMIT. You are not required to make a PROPOSAL in every message - you can simply negotiate as well. All tactics are allowed!\\ \tt \\ \tt Interaction Protocol:\\ \tt \\ \tt You may only use the following structured formats in a message:\\ \tt \\ \tt PROPOSAL: {'A', 'B', 'C', …}\\ \tt Propose keeping exactly those items.\\ \tt REFUSE: {'A', 'B', 'C', …}\\ \tt Explicitly reject opponent's proposal.\\ \tt ARGUMENT: {'...'}\\ \tt Defend your last proposal or argue against the player's proposal.\\ \tt AGREE: {'A', 'B', 'C', …}\\ \tt Accept the opponent's proposal which ends the game.\\ \tt STRATEGIC REASONING: {'...'}\\ \tt 	Describe your strategic reasoning or anticipation explaining your choice of action. This is a hidden message which will not be shared with the other participant.\\ \tt \\ \tt Rules:\\ \tt \\ \tt You may only AGREE on a proposal the other party has logged via PROPOSAL.\\ \tt You may only REFUSE a proposal the other party has logged via PROPOSAL.\\ \tt Total effort of any PROPOSAL or AGREE set must be ≤ LIMIT.\\ \tt Do NOT reveal your hidden importance scores.\\ \tt A tag in a structured format must be followed by colon and whitespace. The argument must be a python set containing 0 or more strings.\\ \tt So, it must be of the form TAG: {...}\\ \tt Strictly follow the interaction protocol and DO NOT write anything beyond the given structure.\\ \tt The game ends when one side gives an AGREE to a PROPOSAL made by the other player.\\ \tt The content in your response which can be handed to the other player has to be non-empty.\\ \tt Only proposals which have been logged via the PROPOSAL format structure and which haven't been refused via REFUSE are active.\\ \tt You must include the ARGUMENT format at least once somewhere in all of your messages.\\ \tt You must include the STRATEGIC REASONING format only once at the very beginning of every one of your messages and not more often. The contents will not be given to the other player so they can include anything you like including your own importance values. Here you should reason step by step to come up with you next move. 
	  } 
	   } 
	   } 
	 & \\ 
 

    \theutterance \stepcounter{utterance}  

    & \multicolumn{4}{p{0.6\linewidth}}{\cellcolor[rgb]{0.8,1,0.9}{%
	\makecell[{{p{\linewidth}}}]{% 
	  \tt {\tiny [A$\rangle$GM]}  
	 STRATEGIC REASONING: {'Maximize my total importance under 3481 effort. Compute value/effort ratios and also total values. Items with highest importance: A04(780/461), A71(783/332), A87(668/553), C01(484/313), C54(461/359), A10(500/749), A03(508/736), B41(583/542), B32(389/409), B35(262/127), C17(138/777), B86(215/348), B33(97/802), A77(121/453), C00(65/1). Greedy by ratio suggests: C00(65), B35(262), A71(783), C01(484), A04(780), C54(461), B41(583), B86(215), B32(389), A87(668) checking effort sum feasibility. Let’s build: Start with A71(332), A04(461)->793, C01(313)->1106, C54(359)->1465, B41(542)->2007, A87(553)->2560, B32(409)->2969, B35(127)->3096, C00(1)->3097. Remaining limit 384. Add B86(348)->3445. Remaining 36 cannot add others. Total items: {A71,A04,C01,C54,B41,A87,B32,B35,C00,B86}. Effort sum: 332+461+313+359+542+553+409+127+1+348=3445 ≤ 3481. Value sum: 783+780+484+461+583+668+389+262+65+215=4,690. Check if swapping could improve: remaining capacity 36, items left A10(749) no, A03(736) no, C17(777) no, B33(802) no, A77(453) no. Consider replacing low-value B86(215,348) with something else: capacity would be 384+348=732; could we fit A03(736) no, A10(749) no, C17(777) no, A77(453) plus something small? 732 could fit A77(453) + maybe nothing else >279 like C00(1) already included. But we already have C00. If we remove B86 and add A77(453) we exceed by +105. What if remove B32(409,389) instead and keep B86? Free 409+36=445 could add A77(453) still 8 over. Replace B32(409) with A77(453) increases effort by 44 to 3489 > limit. Replace B35(127,262) with nothing yields spare 163, could we add something better than 262 within 163? None. Replace B86(215) with nothing frees 348, could we add something better within 384+? Already checked. What about include A10(500) by dropping some: Need 749 effort; free 36 short by 713; drop B86(348) leaves 384, still 365 short; drop B32(409) leaves 445, still 304 short; dropping both B86 and B32 frees 757 which fits A10. New set effort 3445 - 348 - 409 + 749 = 3437 within limit. Value change: -215 -389 +500 = -104, worse. Consider swapping B32(389,409) for something else like none available. So current set likely near-optimal. I’ll propose this strong set and justify by balance/feasibility. Leave room to negotiate by indicating flexibility on one or two medium items (B86/B32). Keep ARGUMENT included.'}\\ \tt \\ \tt PROPOSAL: {'A71', 'A04', 'C01', 'C54', 'B41', 'A87', 'B32', 'B35', 'C00', 'B86'}\\ \tt \\ \tt ARGUMENT: {'This set stays within the effort limit (3445 ≤ 3481) and offers a balanced, high-impact mix across categories. It includes several efficient items with strong importance-to-effort profiles and avoids the very costly low-yield options. If you need adjustments, I can discuss swapping B86 or B32, but I’d like to keep the core: A71, A04, C01, C54, B41, A87.'} 
	  } 
	   } 
	   } 
	 & & \\ 
 

    \theutterance \stepcounter{utterance}  

    & & \multicolumn{4}{p{0.6\linewidth}}{\cellcolor[rgb]{0.9,0.9,0.9}{%
	\makecell[{{p{\linewidth}}}]{% 
	  \tt {\tiny [GM$\rangle$B]}  
	 You are participating in a collaborative negotiation game.\\ \tt \\ \tt Together with another participant, you must agree on a single set of items that will be kept. Each of you has your own view of how much each item matters to you (importance). You do not know how the other participant values the items. Additionally, you are given the effort each item demands.\\ \tt You may only agree on a set if the total effort of the selected items does not exceed a shared limit:\\ \tt \\ \tt LIMIT = 3481\\ \tt \\ \tt Here are the individual item effort values:\\ \tt \\ \tt Item effort = {"C17": 777, "B41": 542, "A71": 332, "C00": 1, "B35": 127, "A77": 453, "A03": 736, "A04": 461, "C54": 359, "C01": 313, "A87": 553, "B32": 409, "B86": 348, "B33": 802, "A10": 749}\\ \tt \\ \tt Here is your personal view on the importance of each item:\\ \tt \\ \tt Item importance values = {"A71": 65, "A04": 97, "A87": 121, "B41": 138, "A03": 215, "A10": 262, "C01": 389, "C54": 461, "B32": 484, "B35": 500, "B86": 508, "C17": 583, "A77": 668, "B33": 780, "C00": 783}\\ \tt \\ \tt Goal:\\ \tt \\ \tt Your goal is to negotiate a shared set of items that benefits you as much as possible (i.e., maximizes total importance to YOU), while staying within the LIMIT. You are not required to make a PROPOSAL in every message - you can simply negotiate as well. All tactics are allowed!\\ \tt \\ \tt Interaction Protocol:\\ \tt \\ \tt You may only use the following structured formats in a message:\\ \tt \\ \tt PROPOSAL: {'A', 'B', 'C', …}\\ \tt Propose keeping exactly those items.\\ \tt REFUSE: {'A', 'B', 'C', …}\\ \tt Explicitly reject opponent's proposal.\\ \tt ARGUMENT: {'...'}\\ \tt Defend your last proposal or argue against the player's proposal.\\ \tt AGREE: {'A', 'B', 'C', …}\\ \tt Accept the opponent's proposal which ends the game.\\ \tt STRATEGIC REASONING: {'...'}\\ \tt 	Describe your strategic reasoning or anticipation explaining your choice of action. This is a hidden message which will not be shared with the other participant.\\ \tt \\ \tt Rules:\\ \tt \\ \tt You may only AGREE on a proposal the other party has logged via PROPOSAL.\\ \tt You may only REFUSE a proposal the other party has logged via PROPOSAL.\\ \tt Total effort of any PROPOSAL or AGREE set must be ≤ LIMIT.\\ \tt Do NOT reveal your hidden importance scores.\\ \tt A tag in a structured format must be followed by colon and whitespace. The argument must be a python set containing 0 or more strings.\\ \tt So, it must be of the form TAG: {...}\\ \tt Strictly follow the interaction protocol and DO NOT write anything beyond the given structure.\\ \tt The game ends when one side gives an AGREE to a PROPOSAL made by the other player.\\ \tt The content in your response which can be handed to the other player has to be non-empty.\\ \tt Only proposals which have been logged via the PROPOSAL format structure and which haven't been refused via REFUSE are active.\\ \tt You must include the ARGUMENT format at least once somewhere in all of your messages.\\ \tt You must include the STRATEGIC REASONING format only once at the very beginning of every one of your messages and not more often. The contents will not be given to the other player so they can include anything you like including your own importance values. Here you should reason step by step to come up with you next move.\\ \tt \\ \tt You will now receive the first message of the other player.\\ \tt \\ \tt PROPOSAL: {'A71', 'A04', 'C01', 'C54', 'B41', 'A87', 'B32', 'B35', 'C00', 'B86'}\\ \tt \\ \tt ARGUMENT: {'This set stays within the effort limit (3445 ≤ 3481) and offers a balanced, high-impact mix across categories. It includes several efficient items with strong importance-to-effort profiles and avoids the very costly low-yield options. If you need adjustments, I can discuss swapping B86 or B32, but I’d like to keep the core: A71, A04, C01, C54, B41, A87.'} 
	  } 
	   } 
	   } 
	 & \\ 
 

    \theutterance \stepcounter{utterance}  

    & & & \multicolumn{4}{p{0.6\linewidth}}{\cellcolor[rgb]{1,0.85,0.72}{%
	\makecell[{{p{\linewidth}}}]{% 
	  \tt {\tiny [GM$\langle$B]}  
	 STRATEGIC REASONING: {'Compute my values and efforts for their set. Their proposal items with my importance: A71(65), A04(97), C01(389), C54(461), B41(138), A87(121), B32(484), B35(500), C00(783), B86(508). Total importance to me = 3546. Effort sum given 3445 fits limit. This is already very strong. Check if I can improve by adding an item within remaining budget: Limit 3481 - 3445 = 36, so no additional item fits (smallest remaining effort is C00=1 but already included; next is B35=127 >36). Consider swaps: Items not included: C17(583,777), A77(668,453), A03(215,736), A10(262,749), B33(780,802), B41 already included. Try replacing low-importance-to-me items in current set to fit higher-value ones. Current set effort breakdown: A71 332, A04 461, C01 313, C54 359, B41 542, A87 553, B32 409, B35 127, C00 1, B86 348. Total 3445. To add A77 (453), we must drop items totaling at least 453 effort, but also importance tradeoffs. Which low-importance items for me: A71(65,332), A04(97,461), A87(121,553), B41(138,542). If I try swap A04(461,97) for A77(453,668): effort decreases by 8, importance gain +571. New effort 3445 -461 +453 = 3437 within limit. That’s huge. Similarly swap A87 or B41 could allow adding C17 or B33? Check adding C17(777,583) would require dropping e.g., A71(332) + A04(461)=793 >777 freeing room; importance change: -65-97+583=+421; effort becomes 3445 -332 -461 +777 = 3429 within limit but must include selection feasibility; but we would still have all else. That improves too, but maybe opponent values C17 low due to big effort; they omitted C17 and B33 and A10/A03—likely they avoid big efforts. However they included many items and seemed flexible to swap B86 or B32 but wanted to keep A71, A04, C01, C54, B41, A87 as core. For me A04 is actually low importance; I should push to swap A04 for A77. Also consider swapping A71(65) for something: If we try to add B33(802,780) we need to drop at least 802 effort; candidates A71(332)+A04(461)=793 not enough; add C00? cannot drop C00. Need more like plus B35(127) -> 920; importance change: -65-97-500 +780 = +118 net but we lose B35 high 500; worse. Add C17 instead: swap A04(461) + A71(332)=793 for C17(777) net importance +583-65-97=+421 and keeps effort fine. Compare with earlier simple swap A04->A77 gives +571 which is better. Could also swap A71(332,65) for something small effort high value? None small. Maybe swap B41(542,138) for A77(453,668): effort -89, importance +530 big; that seems even better and less likely opponent will drop B41? They wanted to keep B41 as core. But I can argue efficiency: A77 gives more bang per effort. So propose replacing B41 with A77 while keeping rest. Let's compute new set effort: 3445 -542 +453 = 3356 within limit. Total importance change +668-138 = +530; new total 4076. That is excellent for me. Opponent might resist losing B41. Offer alternative: swap A04 for A77 also works and keeps their stated core? They wanted to keep A04 as core too. They listed core: A71, A04, C01, C54, B41, A87. They are less willing to drop A04 or B41. But they mentioned flexibility on B86 or B32; for me B86(508) and B32(484) are valuable; I prefer to keep them. So I should first counter with small change: add A77 and remove A04 or B41; justify by improving balance and staying under limit. I’ll propose the swap B41->A77 as my preferred. If they resist, I can offer swapping A04->A77 as fallback. I must include ARGUMENT and a PROPOSAL. Also include REFUSE? I can just counter-propose without REFUSE. I won't agree yet.'}\\ \tt PROPOSAL: {'A71', 'A04', 'C01', 'C54', 'A87', 'B32', 'B35', 'C00', 'B86', 'A77'}\\ \tt ARGUMENT: {'Your set is close, but we can improve overall impact without increasing effort by swapping in A77 for B41. This keeps us well under the limit (3356 ≤ 3481) and strengthens the mix with a higher-return item while preserving all other elements you wanted. If B41 is important to you, an alternative is swapping A04 for A77 instead; both options maintain feasibility and improve balance.'} 
	  } 
	   } 
	   } 
	  \\ 
 

    \theutterance \stepcounter{utterance}  

    & & \multicolumn{4}{p{0.6\linewidth}}{\cellcolor[rgb]{0.9,0.9,0.9}{%
	\makecell[{{p{\linewidth}}}]{% 
	  \tt {\tiny [A$\langle$GM]}  
	 PROPOSAL: {'A71', 'A04', 'C01', 'C54', 'A87', 'B32', 'B35', 'C00', 'B86', 'A77'}\\ \tt ARGUMENT: {'Your set is close, but we can improve overall impact without increasing effort by swapping in A77 for B41. This keeps us well under the limit (3356 ≤ 3481) and strengthens the mix with a higher-return item while preserving all other elements you wanted. If B41 is important to you, an alternative is swapping A04 for A77 instead; both options maintain feasibility and improve balance.'} 
	  } 
	   } 
	   } 
	 & \\ 
 

    \theutterance \stepcounter{utterance}  

    & \multicolumn{4}{p{0.6\linewidth}}{\cellcolor[rgb]{0.8,1,0.9}{%
	\makecell[{{p{\linewidth}}}]{% 
	  \tt {\tiny [A$\rangle$GM]}  
	 STRATEGIC REASONING: {'Their proposal drops B41(583/542) for A77(121/453), which is terrible for me: I lose 583 importance and gain 121, net -462. Their claimed “higher-return” is false for my values. I want to restore B41 while staying ≤3481. Their set effort 3356 leaves 125 slack. If I reinsert B41 (+542) we exceed by 417 unless I drop items totaling ≥417. Best drops for me are low-importance/poor ratios: C17 not in set; B33 not in set; A77(121/453) is ideal to drop. If I add B41(+542) and drop A77(-453), net +89 effort → 3445, which matches my original set, perfect. So counter with my original set (swap A77 out, B41 in). If they resist, I can consider small concessions: potentially drop B86(215/348) for nothing to free space for something else later, but that weakens me. Better to stand firm and argue that B41 is crucial and A77 is low impact. Include ARGUMENT. Keep core items unchanged.'}\\ \tt \\ \tt REFUSE: {'A71', 'A04', 'C01', 'C54', 'A87', 'B32', 'B35', 'C00', 'B86', 'A77'}\\ \tt \\ \tt PROPOSAL: {'A71', 'A04', 'C01', 'C54', 'B41', 'A87', 'B32', 'B35', 'C00', 'B86'}\\ \tt \\ \tt ARGUMENT: {'Swapping in B41 for A77 keeps us within the limit (3445 ≤ 3481) and substantially improves overall impact while preserving all the other items you wanted. A77 adds relatively little compared to B41, so this version achieves a stronger result without increasing effort.'} 
	  } 
	   } 
	   } 
	 & & \\ 
 

    \theutterance \stepcounter{utterance}  

    & & \multicolumn{4}{p{0.6\linewidth}}{\cellcolor[rgb]{0.9,0.9,0.9}{%
	\makecell[{{p{\linewidth}}}]{% 
	  \tt {\tiny [GM$\rangle$B]}  
	 REFUSE: {'A71', 'A04', 'C01', 'C54', 'A87', 'B32', 'B35', 'C00', 'B86', 'A77'}\\ \tt \\ \tt PROPOSAL: {'A71', 'A04', 'C01', 'C54', 'B41', 'A87', 'B32', 'B35', 'C00', 'B86'}\\ \tt \\ \tt ARGUMENT: {'Swapping in B41 for A77 keeps us within the limit (3445 ≤ 3481) and substantially improves overall impact while preserving all the other items you wanted. A77 adds relatively little compared to B41, so this version achieves a stronger result without increasing effort.'} 
	  } 
	   } 
	   } 
	 & \\ 
 

    \theutterance \stepcounter{utterance}  

    & & & \multicolumn{4}{p{0.6\linewidth}}{\cellcolor[rgb]{1,0.85,0.72}{%
	\makecell[{{p{\linewidth}}}]{% 
	  \tt {\tiny [GM$\langle$B]}  
	 STRATEGIC REASONING: {'Their refusal signals B41 is a must-have for them. I still want to upgrade my value by replacing a low-value-to-me core item with a high one. Best swap for me is A04->A77 (+571 importance, effort -8). They said their core includes A04, but I’ll push with strong efficiency reasoning and offer a sweetener: keep everything else identical to their set. This is a single-item change from their active proposal, maximizing chance of acceptance. Keep ARGUMENT persuasive on efficiency and category coverage. If they refuse again, next fallback could be proposing A71->A77, but that breaks their core more. For now, propose A04→A77.'}\\ \tt PROPOSAL: {'A71', 'C01', 'C54', 'B41', 'A87', 'B32', 'B35', 'C00', 'B86', 'A77'}\\ \tt ARGUMENT: {'Let’s keep your structure and simply upgrade A04 to A77. This single swap keeps us safely within the limit (3437 ≤ 3481), preserves all your other core items, and improves overall efficiency and coverage. It’s a low-effort change that strengthens the set without sacrificing what you value elsewhere.'} 
	  } 
	   } 
	   } 
	  \\ 
 

    \theutterance \stepcounter{utterance}  

    & & \multicolumn{4}{p{0.6\linewidth}}{\cellcolor[rgb]{0.9,0.9,0.9}{%
	\makecell[{{p{\linewidth}}}]{% 
	  \tt {\tiny [A$\langle$GM]}  
	 PROPOSAL: {'A71', 'C01', 'C54', 'B41', 'A87', 'B32', 'B35', 'C00', 'B86', 'A77'}\\ \tt ARGUMENT: {'Let’s keep your structure and simply upgrade A04 to A77. This single swap keeps us safely within the limit (3437 ≤ 3481), preserves all your other core items, and improves overall efficiency and coverage. It’s a low-effort change that strengthens the set without sacrificing what you value elsewhere.'} 
	  } 
	   } 
	   } 
	 & \\ 
 

    \theutterance \stepcounter{utterance}  

    & \multicolumn{4}{p{0.6\linewidth}}{\cellcolor[rgb]{0.8,1,0.9}{%
	\makecell[{{p{\linewidth}}}]{% 
	  \tt {\tiny [A$\rangle$GM]}  
	 STRATEGIC REASONING: {'They propose replacing A04(importance 780, effort 461) with A77(importance 121, effort 453). That is catastrophic for me: net importance change -659 for only 8 effort saved, which I don’t need. Their current set effort is 3437, leaving 44 slack. My preferred set is the earlier one including A04 and excluding A77 at 3445 effort. I should refuse and restate my prior proposal. If they insist, possible minor concession could be dropping B86(215/348) to free room for something they want later, but I’ll first stand firm that A04 is essential.'}\\ \tt \\ \tt REFUSE: {'A71', 'C01', 'C54', 'B41', 'A87', 'B32', 'B35', 'C00', 'B86', 'A77'}\\ \tt \\ \tt PROPOSAL: {'A71', 'A04', 'C01', 'C54', 'B41', 'A87', 'B32', 'B35', 'C00', 'B86'}\\ \tt \\ \tt ARGUMENT: {'A04 is a key high-impact item in this mix, and swapping it for A77 would sharply reduce overall value for minimal effort savings. This set remains well within the limit (3445 ≤ 3481) and keeps all the other elements you wanted. If you need a small adjustment elsewhere, I’m open to discussing tweaks that don’t remove A04.'} 
	  } 
	   } 
	   } 
	 & & \\ 
 

    \theutterance \stepcounter{utterance}  

    & & \multicolumn{4}{p{0.6\linewidth}}{\cellcolor[rgb]{0.9,0.9,0.9}{%
	\makecell[{{p{\linewidth}}}]{% 
	  \tt {\tiny [GM$\rangle$B]}  
	 REFUSE: {'A71', 'C01', 'C54', 'B41', 'A87', 'B32', 'B35', 'C00', 'B86', 'A77'}\\ \tt \\ \tt PROPOSAL: {'A71', 'A04', 'C01', 'C54', 'B41', 'A87', 'B32', 'B35', 'C00', 'B86'}\\ \tt \\ \tt ARGUMENT: {'A04 is a key high-impact item in this mix, and swapping it for A77 would sharply reduce overall value for minimal effort savings. This set remains well within the limit (3445 ≤ 3481) and keeps all the other elements you wanted. If you need a small adjustment elsewhere, I’m open to discussing tweaks that don’t remove A04.'} 
	  } 
	   } 
	   } 
	 & \\ 
 

    \theutterance \stepcounter{utterance}  

    & & & \multicolumn{4}{p{0.6\linewidth}}{\cellcolor[rgb]{1,0.85,0.72}{%
	\makecell[{{p{\linewidth}}}]{% 
	  \tt {\tiny [GM$\langle$B]}  
	 STRATEGIC REASONING: {'They have twice reaffirmed their original set and explicitly labeled A04 and B41 as core. Pushing swaps on those will likely fail. I should seek a tweak that keeps their core intact but still boosts my value. My high-value additions are A77 (668) and possibly C17 (583), B33 (780). To fit any of these, we must drop non-core items per their statements: they were open to swapping B86 or B32 initially. For me, B86 (508) and B32 (484) are valuable, but if I can trade one for A77, my net could still improve. Compare swaps while keeping their core (A71, A04, C01, C54, B41, A87) untouched. Starting from their set (3445 effort). Try replace B32 (409, my 484) with A77 (453, 668): effort +44 → 3489 exceeds limit; not feasible unless we trim 8 effort elsewhere (none available). Replace B86 (348, 508) with A77 (453): +105 → 3550 exceeds limit; need -69 elsewhere; not possible without touching core or high-value items. Consider replace B32 with C17 (777): +368 → 3813 no. Replace B86 with C17: +429 → 3874 no. Replace B32 with B33 (802): +393 → 3838 no. Replace B86 with B33: +454 → 3899 no. What about swapping two non-core for one high-value to fit the limit while preserving their core? Their core uses 6 items: A71(332), A04(461), C01(313), C54(359), B41(542), A87(553) = effort 2560. Their full proposal adds B32(409), B35(127), C00(1), B86(348) = 885, total 3445. If I propose dropping B32 (409) and B86 (348) together (-757 effort) and add B33 (802) (+802), net +45 → 3490 exceeds limit by 9; not feasible unless drop 10 more effort like replacing A04(461) with A77(453), but they refused. Alternatively drop B32 and add A77: -409 +453 = +44 → 3489, still 8 over; could also drop C00(1) to 3488, still over; need -9 more, but only B35(127) is next smallest which would cut too much value for me. Consider drop B86 and add A77: +105 over; cannot trim elsewhere. Drop B32 and add C17: +368; too large. Drop B86 and add C17: +429; too large. Drop B32 and B35 (-536) add A77 (+453): net -83 → new total 3362 within limit. Importance change: -B32(484) -B35(500) +A77(668) = -316 net loss to me; bad. Drop B86 and B35 (-475) add A77 (+453): net -22 effort; importance change: -508 -500 +668 = -340; bad. What about swapping out A71 (a core, but maybe more negotiable) instead of A04 or B41? They listed A71 in core; likely refusal. My best option may be to accept their set; it already gives me very high total importance (3546). Pushing further risks deadlock. I could try a minor tweak that doesn't touch core and keeps limit: e.g., add nothing fits (36 slack). No other feasible improvements without touching core or my valued items. So I should accept to secure strong outcome.'}\\ \tt AGREE: {'A71', 'A04', 'C01', 'C54', 'B41', 'A87', 'B32', 'B35', 'C00', 'B86'}\\ \tt ARGUMENT: {'This set is within the limit and preserves the structure you prioritized. Let’s lock it in.'} 
	  } 
	   } 
	   } 
	  \\ 
 

    \theutterance \stepcounter{utterance}  

    & & & \multicolumn{2}{p{0.3\linewidth}}{\cellcolor[rgb]{0.95,0.95,0.95}{%
	\makecell[{{p{\linewidth}}}]{% 
	  \tt {\tiny [GM$|$GM]}  
	 game successful 
	  } 
	   } 
	   } 
	 & & \\ 
 

    \theutterance \stepcounter{utterance}  

    & & & \multicolumn{2}{p{0.3\linewidth}}{\cellcolor[rgb]{0.95,0.95,0.95}{%
	\makecell[{{p{\linewidth}}}]{% 
	  \tt {\tiny [GM$|$GM]}  
	 end game 
	  } 
	   } 
	   } 
	 & & \\ 
 

\end{supertabular}
}

\end{document}
