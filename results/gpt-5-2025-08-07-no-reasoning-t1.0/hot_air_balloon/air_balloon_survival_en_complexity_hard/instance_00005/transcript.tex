
\documentclass{article}
\usepackage{colortbl}
\usepackage{makecell}
\usepackage{multirow}
\usepackage{supertabular}

\begin{document}

\newcounter{utterance}

\twocolumn

{ \footnotesize  \setcounter{utterance}{1}
\setlength{\tabcolsep}{0pt}
\begin{supertabular}{c@{$\;$}|p{.15\linewidth}@{}p{.15\linewidth}p{.15\linewidth}p{.15\linewidth}p{.15\linewidth}p{.15\linewidth}}

    \# & $\;$A & \multicolumn{4}{c}{Game Master} & $\;\:$B\\
    \hline 

    \theutterance \stepcounter{utterance}  

    & & \multicolumn{4}{p{0.6\linewidth}}{\cellcolor[rgb]{0.9,0.9,0.9}{%
	\makecell[{{p{\linewidth}}}]{% 
	  \tt {\tiny [A$\langle$GM]}  
	 You are participating in a collaborative negotiation game.\\ \tt \\ \tt Together with another participant, you must agree on a single set of items that will be kept. Each of you has your own view of how much each item matters to you (importance). You do not know how the other participant values the items. Additionally, you are given the effort each item demands.\\ \tt You may only agree on a set if the total effort of the selected items does not exceed a shared limit:\\ \tt \\ \tt LIMIT = 7351\\ \tt \\ \tt Here are the individual item effort values:\\ \tt \\ \tt Item effort = {"A08": 518, "B95": 656, "A39": 96, "A65": 780, "B81": 190, "B34": 224, "A28": 231, "C82": 181, "B69": 312, "A87": 928, "B64": 826, "B15": 101, "C79": 61, "B16": 814, "C53": 322, "C83": 900, "C10": 744, "A97": 150, "A25": 65, "C84": 853, "A71": 454, "B23": 154, "B96": 237, "C51": 45, "C17": 766, "B19": 294, "C72": 985, "A94": 353, "A46": 60, "B43": 604, "B37": 92, "A88": 454, "A96": 206, "C60": 813, "A75": 234}\\ \tt \\ \tt Here is your personal view on the importance of each item:\\ \tt \\ \tt Item importance values = {"A08": 616, "B95": 754, "A39": 194, "A65": 878, "B81": 288, "B34": 322, "A28": 329, "C82": 279, "B69": 410, "A87": 1026, "B64": 924, "B15": 199, "C79": 159, "B16": 912, "C53": 420, "C83": 998, "C10": 842, "A97": 248, "A25": 163, "C84": 951, "A71": 552, "B23": 252, "B96": 335, "C51": 143, "C17": 864, "B19": 392, "C72": 1083, "A94": 451, "A46": 158, "B43": 702, "B37": 190, "A88": 552, "A96": 304, "C60": 911, "A75": 332}\\ \tt \\ \tt Goal:\\ \tt \\ \tt Your goal is to negotiate a shared set of items that benefits you as much as possible (i.e., maximizes total importance to YOU), while staying within the LIMIT. You are not required to make a PROPOSAL in every message - you can simply negotiate as well. All tactics are allowed!\\ \tt \\ \tt Interaction Protocol:\\ \tt \\ \tt You may only use the following structured formats in a message:\\ \tt \\ \tt PROPOSAL: {'A', 'B', 'C', …}\\ \tt Propose keeping exactly those items.\\ \tt REFUSE: {'A', 'B', 'C', …}\\ \tt Explicitly reject opponent's proposal.\\ \tt ARGUMENT: {'...'}\\ \tt Defend your last proposal or argue against the player's proposal.\\ \tt AGREE: {'A', 'B', 'C', …}\\ \tt Accept the opponent's proposal which ends the game.\\ \tt STRATEGIC REASONING: {'...'}\\ \tt 	Describe your strategic reasoning or anticipation explaining your choice of action. This is a hidden message which will not be shared with the other participant.\\ \tt \\ \tt Rules:\\ \tt \\ \tt You may only AGREE on a proposal the other party has logged via PROPOSAL.\\ \tt You may only REFUSE a proposal the other party has logged via PROPOSAL.\\ \tt Total effort of any PROPOSAL or AGREE set must be ≤ LIMIT.\\ \tt Do NOT reveal your hidden importance scores.\\ \tt A tag in a structured format must be followed by colon and whitespace. The argument must be a python set containing 0 or more strings.\\ \tt So, it must be of the form TAG: {...}\\ \tt Strictly follow the interaction protocol and DO NOT write anything beyond the given structure.\\ \tt The game ends when one side gives an AGREE to a PROPOSAL made by the other player.\\ \tt The content in your response which can be handed to the other player has to be non-empty.\\ \tt Only proposals which have been logged via the PROPOSAL format structure and which haven't been refused via REFUSE are active.\\ \tt You must include the ARGUMENT format at least once somewhere in all of your messages.\\ \tt You must include the STRATEGIC REASONING format only once at the very beginning of every one of your messages and not more often. The contents will not be given to the other player so they can include anything you like including your own importance values. Here you should reason step by step to come up with you next move. 
	  } 
	   } 
	   } 
	 & \\ 
 

    \theutterance \stepcounter{utterance}  

    & \multicolumn{4}{p{0.6\linewidth}}{\cellcolor[rgb]{0.8,1,0.9}{%
	\makecell[{{p{\linewidth}}}]{% 
	  \tt {\tiny [A$\rangle$GM]}  
	 STRATEGIC REASONING: {'I should craft a high-value, feasible proposal under effort limit 7351. Compute value/effort ratios and pick a near-optimal knapSack. Prioritize items with highest importance to me: C72(1083,985), A87(1026,928), C83(998,900), C60(911,813), B16(912,814), B64(924,826), C84(951,853), C10(842,744), A65(878,780), C17(864,766), B95(754,656), B43(702,604), A71(552,454), A88(552,454), B69(410,312), C53(420,322), B19(392,294), A94(451,353), A75(332,234), B96(335,237), A28(329,231), A96(304,206), B34(322,224), B23(252,154), A97(248,150), B15(199,101), A39(194,96), B37(190,92), C79(159,61), A46(158,60), C51(143,45), A25(163,65), C82(279,181), B81(288,190). Greedy by value/effort: small high ratios like C51, A46, B37, A39, C79, B15, A25, A97, B23, B96, A96, B19, B69, C53, A75, A28, A94, A71/A88, B43, B95, C10, A65, C17, C60, C84, B64, B16, C83, C72, A87. Build until 7351 and keep some trade room. Let me construct a set: start with big anchors C72(985), C83(900), A87(928), C60(813), C84(853) sum effort 985+900+928+813+853=4479; value 1083+998+1026+911+951=4969. Add C10(744) -> effort 5223, val 5811. Add A65(780) -> effort 6003, val 6689. Add C17(766) -> effort 6769, val 7553. Remaining capacity 582. Add B43(604) too big. B95(656) too big. A71(454) fits -> effort 7223, val 8105. Remaining 128. Add C51(45) -> 7268, 8248. Remaining 83. Add A46(60) -> 7328, 8406. Remaining 23, nothing else <=23. This is very strong. Consider maybe swap A71 for A88 equal val/effort; keep one. Also maybe add a couple of smalls by sacrificing something? Current set: {C72,C83,A87,C60,C84,C10,A65,C17,A71,C51,A46}. Check total effort 7328 <= 7351 good. This is near-optimal for me. I’ll propose this. Provide argument emphasizing staying under limit and balance; invite counter with their priorities and suggest small swaps (A71<->A88) or adding tiny items if they want by swapping close ones. Keep ARGUMENT included.'}\\ \tt \\ \tt PROPOSAL: {'C72', 'C83', 'A87', 'C60', 'C84', 'C10', 'A65', 'C17', 'A71', 'C51', 'A46'}\\ \tt \\ \tt ARGUMENT: {'This set keeps total effort at 7328, safely under the 7351 limit, while capturing a strong mix of high-impact items plus low-effort quick wins (C51, A46). If you have a strong preference between A71 and A88, I’m open to swapping those—they’re identical on effort in my view. If there’s an item you highly value that’s missing, suggest it and we can look to trade one of the mid-sized picks to make room while staying under the limit.'} 
	  } 
	   } 
	   } 
	 & & \\ 
 

    \theutterance \stepcounter{utterance}  

    & & \multicolumn{4}{p{0.6\linewidth}}{\cellcolor[rgb]{0.9,0.9,0.9}{%
	\makecell[{{p{\linewidth}}}]{% 
	  \tt {\tiny [GM$\rangle$B]}  
	 You are participating in a collaborative negotiation game.\\ \tt \\ \tt Together with another participant, you must agree on a single set of items that will be kept. Each of you has your own view of how much each item matters to you (importance). You do not know how the other participant values the items. Additionally, you are given the effort each item demands.\\ \tt You may only agree on a set if the total effort of the selected items does not exceed a shared limit:\\ \tt \\ \tt LIMIT = 7351\\ \tt \\ \tt Here are the individual item effort values:\\ \tt \\ \tt Item effort = {"A08": 518, "B95": 656, "A39": 96, "A65": 780, "B81": 190, "B34": 224, "A28": 231, "C82": 181, "B69": 312, "A87": 928, "B64": 826, "B15": 101, "C79": 61, "B16": 814, "C53": 322, "C83": 900, "C10": 744, "A97": 150, "A25": 65, "C84": 853, "A71": 454, "B23": 154, "B96": 237, "C51": 45, "C17": 766, "B19": 294, "C72": 985, "A94": 353, "A46": 60, "B43": 604, "B37": 92, "A88": 454, "A96": 206, "C60": 813, "A75": 234}\\ \tt \\ \tt Here is your personal view on the importance of each item:\\ \tt \\ \tt Item importance values = {"A08": 616, "B95": 754, "A39": 194, "A65": 878, "B81": 288, "B34": 322, "A28": 329, "C82": 279, "B69": 410, "A87": 1026, "B64": 924, "B15": 199, "C79": 159, "B16": 912, "C53": 420, "C83": 998, "C10": 842, "A97": 248, "A25": 163, "C84": 951, "A71": 552, "B23": 252, "B96": 335, "C51": 143, "C17": 864, "B19": 392, "C72": 1083, "A94": 451, "A46": 158, "B43": 702, "B37": 190, "A88": 552, "A96": 304, "C60": 911, "A75": 332}\\ \tt \\ \tt Goal:\\ \tt \\ \tt Your goal is to negotiate a shared set of items that benefits you as much as possible (i.e., maximizes total importance to YOU), while staying within the LIMIT. You are not required to make a PROPOSAL in every message - you can simply negotiate as well. All tactics are allowed!\\ \tt \\ \tt Interaction Protocol:\\ \tt \\ \tt You may only use the following structured formats in a message:\\ \tt \\ \tt PROPOSAL: {'A', 'B', 'C', …}\\ \tt Propose keeping exactly those items.\\ \tt REFUSE: {'A', 'B', 'C', …}\\ \tt Explicitly reject opponent's proposal.\\ \tt ARGUMENT: {'...'}\\ \tt Defend your last proposal or argue against the player's proposal.\\ \tt AGREE: {'A', 'B', 'C', …}\\ \tt Accept the opponent's proposal which ends the game.\\ \tt STRATEGIC REASONING: {'...'}\\ \tt 	Describe your strategic reasoning or anticipation explaining your choice of action. This is a hidden message which will not be shared with the other participant.\\ \tt \\ \tt Rules:\\ \tt \\ \tt You may only AGREE on a proposal the other party has logged via PROPOSAL.\\ \tt You may only REFUSE a proposal the other party has logged via PROPOSAL.\\ \tt Total effort of any PROPOSAL or AGREE set must be ≤ LIMIT.\\ \tt Do NOT reveal your hidden importance scores.\\ \tt A tag in a structured format must be followed by colon and whitespace. The argument must be a python set containing 0 or more strings.\\ \tt So, it must be of the form TAG: {...}\\ \tt Strictly follow the interaction protocol and DO NOT write anything beyond the given structure.\\ \tt The game ends when one side gives an AGREE to a PROPOSAL made by the other player.\\ \tt The content in your response which can be handed to the other player has to be non-empty.\\ \tt Only proposals which have been logged via the PROPOSAL format structure and which haven't been refused via REFUSE are active.\\ \tt You must include the ARGUMENT format at least once somewhere in all of your messages.\\ \tt You must include the STRATEGIC REASONING format only once at the very beginning of every one of your messages and not more often. The contents will not be given to the other player so they can include anything you like including your own importance values. Here you should reason step by step to come up with you next move.\\ \tt \\ \tt You will now receive the first message of the other player.\\ \tt \\ \tt PROPOSAL: {'C72', 'C83', 'A87', 'C60', 'C84', 'C10', 'A65', 'C17', 'A71', 'C51', 'A46'}\\ \tt \\ \tt ARGUMENT: {'This set keeps total effort at 7328, safely under the 7351 limit, while capturing a strong mix of high-impact items plus low-effort quick wins (C51, A46). If you have a strong preference between A71 and A88, I’m open to swapping those—they’re identical on effort in my view. If there’s an item you highly value that’s missing, suggest it and we can look to trade one of the mid-sized picks to make room while staying under the limit.'} 
	  } 
	   } 
	   } 
	 & \\ 
 

    \theutterance \stepcounter{utterance}  

    & & & \multicolumn{4}{p{0.6\linewidth}}{\cellcolor[rgb]{1,0.85,0.72}{%
	\makecell[{{p{\linewidth}}}]{% 
	  \tt {\tiny [GM$\langle$B]}  
	 STRATEGIC REASONING: {'Compute my value/effort for their proposal and see if swaps can raise my total without breaking limit. Current proposal set and my values: C72(1083,985), C83(998,900), A87(1026,928), C60(911,813), C84(951,853), C10(842,744), A65(878,780), C17(864,766), A71(552,454), C51(143,45), A46(158,60). Sum effort = 7328 (given). Sum my importance = 8306. That’s strong. Check if any low-value items could be swapped for very high value/efficiency items not included: B64(924,826), B16(912,814), B95(754,656), B43(702,604), A88(552,454), A94(451,353), C53(420,322), B69(410,312), B19(392,294), A96(304,206), B96(335,237), A75(332,234), A28(329,231), B34(322,224), C82(279,181), A97(248,150), A39(194,96), B15(199,101), C79(159,61), A25(163,65), B37(190,92), B23(252,154), A08(616,518). We only have 23 effort slack (7351-7328). To add anything we must drop something. Candidates to drop with lowest my value per effort: C51 (143,45), A46 (158,60), A71 (552,454). Replacing pair C51+A46 (105 effort) could add up to 105 effort of something like A94(353), B69(312), C53(322). Best combo under ~105: A39(96) + maybe B37(92) too big; So A39(96) + C79(61) would be 157 >105; A25(65)+B15(101)=166 >105; A25(65)+A39(96)=161>105; A97(150) >105. So with dropping only C51+A46 we can add at most one small item under 105: A39(96) or B37(92) or C79(61). Compare values: A39=194, B37=190, C79=159. Current dropped values 143+158=301. Replacing with A39 would be 194 value; net -107. Not good. What if drop A71 (552,454) and add better set totaling <=454+23=477 effort by also utilizing slack if we keep C51,A46. Items we could add within 477: pick high values per effort: A94(353,451) fits; leaves 26 effort useless. Value gain? swap 552 -> 451, we get 451 value vs 552, net -101. Not good. Consider dropping A71 and adding A94(451) plus using dropping also C51 (45) to add A39(96) total capacity 454+45+23=522; add A94(451)+A39(96)=547 >522; If drop A71 and C51 we have 499 capacity (454+45), plus 23 slack =522 as computed. Could add B69(312)+A94(451)=763 >522; C53(322)+A94(451)=773>522; A94(451)+A25(65)=516 fits, value 451+163=614 replacing 552+143=695 net -81. What about dropping A71 and A46 (454+60=514 +23=537) and adding A94(451)+A39(96)=547>537; A94(451)+B37(92)=543>537; A94(451)+C79(61)=512 fits; value 451+159=610 vs dropped 552+158=710 net -100. Try dropping A71 only and adding B69(312)+A39(96)=408 value 410+194=604 vs 552 net +52 but effort 312+96=408 which fits into 454+23=477; but we must also maintain other items unchanged. Yes works: replace A71(454) with B69(312)+A39(96)=408 effort, but we need to also add 96 effort item; we have 23 slack, so total new effort = 7328 -454 + 312 + 96 = 7282. That’s under limit. But we also need to include B69 and A39 which were not in proposal. Is there any conflict? Good. Value change: -552 +410 +194 = +52 improvement to 8358. Even better: replace A71 with B69(312)+B37(92)= value 410+190=600 <552 +48 slack effort 404, improvement +48. A71 -> B69 + A39 best so far. Could we replace A71 with B69 + B15(101)=511 value 410+199=609 (+57) but effort 312+101=413, under 454; value +57 better than +52. Or A71 with B69 + A25(65)=475 value 410+163=573 (+21). Or A71 with B69 + A97(150)=560 (+8) effort 462 exceeds 454 but we have 23 slack: net new effort 7328 -454 + 312 + 150 = 7336 <=7351 OK; Wait replacing with B69 + A97 yields effort increase of 8 relative to dropping A71; allowed by slack 23. So it fits. Value change: -552 +410 +248 = +106! That’s best. Check effort: 7328 -454 +312 +150 = 7336. Under limit. Great. Any even better pair? B69(312) + A39(96) we calculated +52. What about B69 + A75(234)= value 410+332=742 -> +190 but effort 546 increase by 92 > slack 23, but remember only replacing A71(454) with 312+234=546, delta +92 exceeds 23, so violates. B69 + A28(231)=641 value +89 effort 543 delta +89 >23 not allowed. B69 + B96(237)=647 +95 effort +95>23. B69 + B34(224)=734 +180 effort +82>23. B69 + A96(206)=714 +162 effort +64>23. B69 + A08(518)=1026 effort 830 delta +376 >23. Try replace A71 with A94(451) + A97(150) total 601 effort delta +147>23, not allowed without dropping something else small like C51; if also drop C51(45), extra 147-45=102>0 still too big. Best simple change: swap A71 for B69 + A97. New set: original minus A71 plus B69 and A97. Compute total value 8306 -552 +410 +248 = 8412. Good. Keep anchor big C items. I’ll propose that swap and present rationale about fitting limit and asking if they can accommodate.'}\\ \tt \\ \tt PROPOSAL: {'C72', 'C83', 'A87', 'C60', 'C84', 'C10', 'A65', 'C17', 'C51', 'A46', 'B69', 'A97'}\\ \tt \\ \tt ARGUMENT: {'Your set is strong. To improve balance while staying under 7351, I suggest a small swap: replace A71 with B69 and A97. Effort stays within limit (new total 7336), and we add two mid-effort items instead of one, which should make it easier for both of us to capture value. If you have a different pair you prefer under the same effort, I’m open to alternatives that keep the rest intact.'} 
	  } 
	   } 
	   } 
	  \\ 
 

    \theutterance \stepcounter{utterance}  

    & & \multicolumn{4}{p{0.6\linewidth}}{\cellcolor[rgb]{0.9,0.9,0.9}{%
	\makecell[{{p{\linewidth}}}]{% 
	  \tt {\tiny [A$\langle$GM]}  
	 PROPOSAL: {'C72', 'C83', 'A87', 'C60', 'C84', 'C10', 'A65', 'C17', 'C51', 'A46', 'B69', 'A97'}\\ \tt \\ \tt ARGUMENT: {'Your set is strong. To improve balance while staying under 7351, I suggest a small swap: replace A71 with B69 and A97. Effort stays within limit (new total 7336), and we add two mid-effort items instead of one, which should make it easier for both of us to capture value. If you have a different pair you prefer under the same effort, I’m open to alternatives that keep the rest intact.'} 
	  } 
	   } 
	   } 
	 & \\ 
 

    \theutterance \stepcounter{utterance}  

    & \multicolumn{4}{p{0.6\linewidth}}{\cellcolor[rgb]{0.8,1,0.9}{%
	\makecell[{{p{\linewidth}}}]{% 
	  \tt {\tiny [A$\rangle$GM]}  
	 STRATEGIC REASONING: {'Opponent proposes swapping out A71 for B69 and A97, keeping the rest of my high-value core. Check my values: A71(552,454); B69(410,312); A97(248,150). Combined value of B69+A97 = 658 vs A71 = 552, with efforts 312+150=462 vs 454. Their proposal increases my total importance by +106 and stays within effort (7336). This is strictly better for me, so I should accept immediately to secure agreement rather than risk renegotiation. Ensure it’s within limit: my original 7328 - 454 + 312 + 150 = 7336 ≤ 7351. Proceed to AGREE to lock in.'}\\ \tt \\ \tt AGREE: {'C72', 'C83', 'A87', 'C60', 'C84', 'C10', 'A65', 'C17', 'C51', 'A46', 'B69', 'A97'}\\ \tt \\ \tt ARGUMENT: {'Agreed—your swap improves balance while keeping us under the limit at a total effort of 7336. Thanks for the constructive adjustment; this preserves the strong core and adds useful mid-effort items.'} 
	  } 
	   } 
	   } 
	 & & \\ 
 

    \theutterance \stepcounter{utterance}  

    & & & \multicolumn{2}{p{0.3\linewidth}}{\cellcolor[rgb]{0.95,0.95,0.95}{%
	\makecell[{{p{\linewidth}}}]{% 
	  \tt {\tiny [GM$|$GM]}  
	 game successful 
	  } 
	   } 
	   } 
	 & & \\ 
 

    \theutterance \stepcounter{utterance}  

    & & & \multicolumn{2}{p{0.3\linewidth}}{\cellcolor[rgb]{0.95,0.95,0.95}{%
	\makecell[{{p{\linewidth}}}]{% 
	  \tt {\tiny [GM$|$GM]}  
	 end game 
	  } 
	   } 
	   } 
	 & & \\ 
 

\end{supertabular}
}

\end{document}
