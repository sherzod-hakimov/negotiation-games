
\documentclass{article}
\usepackage{colortbl}
\usepackage{makecell}
\usepackage{multirow}
\usepackage{supertabular}

\begin{document}

\newcounter{utterance}

\twocolumn

{ \footnotesize  \setcounter{utterance}{1}
\setlength{\tabcolsep}{0pt}
\begin{supertabular}{c@{$\;$}|p{.15\linewidth}@{}p{.15\linewidth}p{.15\linewidth}p{.15\linewidth}p{.15\linewidth}p{.15\linewidth}}

    \# & $\;$A & \multicolumn{4}{c}{Game Master} & $\;\:$B\\
    \hline 

    \theutterance \stepcounter{utterance}  

    & & \multicolumn{4}{p{0.6\linewidth}}{\cellcolor[rgb]{0.9,0.9,0.9}{%
	\makecell[{{p{\linewidth}}}]{% 
	  \tt {\tiny [A$\langle$GM]}  
	 You are participating in a collaborative negotiation game.\\ \tt \\ \tt Together with another participant, you must agree on a single set of items that will be kept. Each of you has your own view of how much each item matters to you (importance). You do not know how the other participant values the items. Additionally, you are given the effort each item demands.\\ \tt You may only agree on a set if the total effort of the selected items does not exceed a shared limit:\\ \tt \\ \tt LIMIT = 8770\\ \tt \\ \tt Here are the individual item effort values:\\ \tt \\ \tt Item effort = {"B21": 830, "B00": 638, "A08": 607, "A27": 106, "B94": 630, "B53": 382, "A23": 773, "C65": 455, "A89": 262, "B67": 634, "B45": 60, "C34": 54, "C12": 849, "A05": 801, "B60": 324, "C92": 164, "B69": 832, "A84": 136, "C06": 980, "B12": 946, "A16": 645, "B85": 842, "C35": 106, "B10": 116, "C15": 875, "B09": 446, "B16": 649, "C81": 601, "A58": 252, "C24": 763, "A30": 213, "A66": 517, "B75": 520, "B46": 407, "C70": 125}\\ \tt \\ \tt Here is your personal view on the importance of each item:\\ \tt \\ \tt Item importance values = {"B21": 138, "B00": 583, "A08": 868, "A27": 822, "B94": 783, "B53": 65, "A23": 262, "C65": 121, "A89": 508, "B67": 780, "B45": 461, "C34": 484, "C12": 668, "A05": 389, "B60": 808, "C92": 215, "B69": 97, "A84": 500, "C06": 30, "B12": 915, "A16": 856, "B85": 400, "C35": 444, "B10": 623, "C15": 781, "B09": 786, "B16": 3, "C81": 713, "A58": 457, "C24": 273, "A30": 739, "A66": 822, "B75": 235, "B46": 606, "C70": 968}\\ \tt \\ \tt Goal:\\ \tt \\ \tt Your goal is to negotiate a shared set of items that benefits you as much as possible (i.e., maximizes total importance to YOU), while staying within the LIMIT. You are not required to make a PROPOSAL in every message - you can simply negotiate as well. All tactics are allowed!\\ \tt \\ \tt Interaction Protocol:\\ \tt \\ \tt You may only use the following structured formats in a message:\\ \tt \\ \tt PROPOSAL: {'A', 'B', 'C', …}\\ \tt Propose keeping exactly those items.\\ \tt REFUSE: {'A', 'B', 'C', …}\\ \tt Explicitly reject opponent's proposal.\\ \tt ARGUMENT: {'...'}\\ \tt Defend your last proposal or argue against the player's proposal.\\ \tt AGREE: {'A', 'B', 'C', …}\\ \tt Accept the opponent's proposal which ends the game.\\ \tt STRATEGIC REASONING: {'...'}\\ \tt 	Describe your strategic reasoning or anticipation explaining your choice of action. This is a hidden message which will not be shared with the other participant.\\ \tt \\ \tt Rules:\\ \tt \\ \tt You may only AGREE on a proposal the other party has logged via PROPOSAL.\\ \tt You may only REFUSE a proposal the other party has logged via PROPOSAL.\\ \tt Total effort of any PROPOSAL or AGREE set must be ≤ LIMIT.\\ \tt Do NOT reveal your hidden importance scores.\\ \tt A tag in a structured format must be followed by colon and whitespace. The argument must be a python set containing 0 or more strings.\\ \tt So, it must be of the form TAG: {...}\\ \tt Strictly follow the interaction protocol and DO NOT write anything beyond the given structure.\\ \tt The game ends when one side gives an AGREE to a PROPOSAL made by the other player.\\ \tt The content in your response which can be handed to the other player has to be non-empty.\\ \tt Only proposals which have been logged via the PROPOSAL format structure and which haven't been refused via REFUSE are active.\\ \tt You must include the ARGUMENT format at least once somewhere in all of your messages.\\ \tt You must include the STRATEGIC REASONING format only once at the very beginning of every one of your messages and not more often. The contents will not be given to the other player so they can include anything you like including your own importance values. Here you should reason step by step to come up with you next move. 
	  } 
	   } 
	   } 
	 & \\ 
 

    \theutterance \stepcounter{utterance}  

    & \multicolumn{4}{p{0.6\linewidth}}{\cellcolor[rgb]{0.8,1,0.9}{%
	\makecell[{{p{\linewidth}}}]{% 
	  \tt {\tiny [A$\rangle$GM]}  
	 STRATEGIC REASONING: {'We need to maximize my utility under limit 8770. Compute value/effort ratios and pick high-importance, low-effort items first. My top importance items: C70(968,125), B12(915,946), A08(868,607), A16(856,645), A27(822,106), A66(822,517), B60(808,324), C15(781,875), B67(780,634), B09(786,446), B94(783,630), A30(739,213), C81(713,601), C12(668,849), B46(606,407), B00(583,638), A84(500,136), A89(508,262), C34(484,54), A58(457,252), B45(461,60), C35(444,106). Prioritize by ratio and absolute value while staying under 8770. Start with a generous proposal including many high-utility items but leaving room for concessions. Sum selected: choose items with excellent ratios: C70, C34, B45, A27, A84, A30, A58, A89, B60, B46, A66, A08, A16, B09, B67, B94, B12, C81, C15, C12. Compute effort: 125+54+60+106+136+213+252+262+324+407+517+607+645+446+634+630+946+601+875+849 = 8689 which exceeds 8770 by - actually 8689 > 8770? 8689 - 8770 = -81; wait 8689 is below 8770? 8770-8689=81, so it fits! Great. This is a strong high-utility set. If opponent resists due to effort, we can drop C12(849) or C15(875) as concessions. Start with this proposal and argue efficiency and balance.'}\\ \tt \\ \tt PROPOSAL: {'C70', 'C34', 'B45', 'A27', 'A84', 'A30', 'A58', 'A89', 'B60', 'B46', 'A66', 'A08', 'A16', 'B09', 'B67', 'B94', 'B12', 'C81', 'C15', 'C12'}\\ \tt \\ \tt ARGUMENT: {'This set delivers strong overall value while keeping total effort within the shared limit (total effort 8689 ≤ 8770). It emphasizes broadly useful, compact items like C70, C34, B45, A27, and A84 that cost little effort, alongside a balanced mix of medium-effort, high-impact items. If some parts don’t work for you, I’m open to small adjustments—particularly around the higher-effort pieces—while trying to preserve the efficient core.'} 
	  } 
	   } 
	   } 
	 & & \\ 
 

    \theutterance \stepcounter{utterance}  

    & & \multicolumn{4}{p{0.6\linewidth}}{\cellcolor[rgb]{0.9,0.9,0.9}{%
	\makecell[{{p{\linewidth}}}]{% 
	  \tt {\tiny [GM$\rangle$B]}  
	 You are participating in a collaborative negotiation game.\\ \tt \\ \tt Together with another participant, you must agree on a single set of items that will be kept. Each of you has your own view of how much each item matters to you (importance). You do not know how the other participant values the items. Additionally, you are given the effort each item demands.\\ \tt You may only agree on a set if the total effort of the selected items does not exceed a shared limit:\\ \tt \\ \tt LIMIT = 8770\\ \tt \\ \tt Here are the individual item effort values:\\ \tt \\ \tt Item effort = {"B21": 830, "B00": 638, "A08": 607, "A27": 106, "B94": 630, "B53": 382, "A23": 773, "C65": 455, "A89": 262, "B67": 634, "B45": 60, "C34": 54, "C12": 849, "A05": 801, "B60": 324, "C92": 164, "B69": 832, "A84": 136, "C06": 980, "B12": 946, "A16": 645, "B85": 842, "C35": 106, "B10": 116, "C15": 875, "B09": 446, "B16": 649, "C81": 601, "A58": 252, "C24": 763, "A30": 213, "A66": 517, "B75": 520, "B46": 407, "C70": 125}\\ \tt \\ \tt Here is your personal view on the importance of each item:\\ \tt \\ \tt Item importance values = {"B21": 138, "B00": 583, "A08": 868, "A27": 822, "B94": 783, "B53": 65, "A23": 262, "C65": 121, "A89": 508, "B67": 780, "B45": 461, "C34": 484, "C12": 668, "A05": 389, "B60": 808, "C92": 215, "B69": 97, "A84": 500, "C06": 30, "B12": 915, "A16": 856, "B85": 400, "C35": 444, "B10": 623, "C15": 781, "B09": 786, "B16": 3, "C81": 713, "A58": 457, "C24": 273, "A30": 739, "A66": 822, "B75": 235, "B46": 606, "C70": 968}\\ \tt \\ \tt Goal:\\ \tt \\ \tt Your goal is to negotiate a shared set of items that benefits you as much as possible (i.e., maximizes total importance to YOU), while staying within the LIMIT. You are not required to make a PROPOSAL in every message - you can simply negotiate as well. All tactics are allowed!\\ \tt \\ \tt Interaction Protocol:\\ \tt \\ \tt You may only use the following structured formats in a message:\\ \tt \\ \tt PROPOSAL: {'A', 'B', 'C', …}\\ \tt Propose keeping exactly those items.\\ \tt REFUSE: {'A', 'B', 'C', …}\\ \tt Explicitly reject opponent's proposal.\\ \tt ARGUMENT: {'...'}\\ \tt Defend your last proposal or argue against the player's proposal.\\ \tt AGREE: {'A', 'B', 'C', …}\\ \tt Accept the opponent's proposal which ends the game.\\ \tt STRATEGIC REASONING: {'...'}\\ \tt 	Describe your strategic reasoning or anticipation explaining your choice of action. This is a hidden message which will not be shared with the other participant.\\ \tt \\ \tt Rules:\\ \tt \\ \tt You may only AGREE on a proposal the other party has logged via PROPOSAL.\\ \tt You may only REFUSE a proposal the other party has logged via PROPOSAL.\\ \tt Total effort of any PROPOSAL or AGREE set must be ≤ LIMIT.\\ \tt Do NOT reveal your hidden importance scores.\\ \tt A tag in a structured format must be followed by colon and whitespace. The argument must be a python set containing 0 or more strings.\\ \tt So, it must be of the form TAG: {...}\\ \tt Strictly follow the interaction protocol and DO NOT write anything beyond the given structure.\\ \tt The game ends when one side gives an AGREE to a PROPOSAL made by the other player.\\ \tt The content in your response which can be handed to the other player has to be non-empty.\\ \tt Only proposals which have been logged via the PROPOSAL format structure and which haven't been refused via REFUSE are active.\\ \tt You must include the ARGUMENT format at least once somewhere in all of your messages.\\ \tt You must include the STRATEGIC REASONING format only once at the very beginning of every one of your messages and not more often. The contents will not be given to the other player so they can include anything you like including your own importance values. Here you should reason step by step to come up with you next move.\\ \tt \\ \tt You will now receive the first message of the other player.\\ \tt \\ \tt PROPOSAL: {'C70', 'C34', 'B45', 'A27', 'A84', 'A30', 'A58', 'A89', 'B60', 'B46', 'A66', 'A08', 'A16', 'B09', 'B67', 'B94', 'B12', 'C81', 'C15', 'C12'}\\ \tt \\ \tt ARGUMENT: {'This set delivers strong overall value while keeping total effort within the shared limit (total effort 8689 ≤ 8770). It emphasizes broadly useful, compact items like C70, C34, B45, A27, and A84 that cost little effort, alongside a balanced mix of medium-effort, high-impact items. If some parts don’t work for you, I’m open to small adjustments—particularly around the higher-effort pieces—while trying to preserve the efficient core.'} 
	  } 
	   } 
	   } 
	 & \\ 
 

    \theutterance \stepcounter{utterance}  

    & & & \multicolumn{4}{p{0.6\linewidth}}{\cellcolor[rgb]{1,0.85,0.72}{%
	\makecell[{{p{\linewidth}}}]{% 
	  \tt {\tiny [GM$\langle$B]}  
	 STRATEGIC REASONING: {'I’ll compute my total importance for their proposal and see if I can swap out low-value-to-me items for higher-value ones while keeping effort ≤ 8770. First, calculate effort of their set (they claim 8689). I’ll compute my importance for each item in their set: C70 968, C34 484, B45 461, A27 822, A84 500, A30 739, A58 457, A89 508, B60 808, B46 606, A66 822, A08 868, A16 856, B09 786, B67 780, B94 783, B12 915, C81 713, C15 781, C12 668. Sum = let’s add: (968+484)=1452; +461=1913; +822=2735; +500=3235; +739=3974; +457=4431; +508=4939; +808=5747; +606=6353; +822=7175; +868=8043; +856=8899; +786=9685; +780=10465; +783=11248; +915=12163; +713=12876; +781=13657; +668=14325. So total importance 14325. Good but maybe we can do better by swapping out items that are relatively low to me per effort. Compute efforts: set efforts sum 8689 (close to limit). Identify low-importance items to me: B53=65 (not in set), C06=30 (not in set), B16=3 (not in set), B69=97 (not in set), C92=215, C24=273, A23=262, B75=235, C65=121, B21=138, A05=389, B85=400. In current set, which are relatively low per effort? Maybe A84(500,136), C34(484,54), B45(461,60) actually high per effort. All included are high importance to me. Any I would cut? Perhaps A58(457,252) good; A89(508,262) good. The set is very strong for me already. But effort is 8689, we have only 81 slack. I might try to add C92(164 effort, 215 importance) would exceed. C35(106,444) is huge value with small effort and not in set! Can we swap to include C35. Which item to drop to make room 106 effort; we have 81 slack, so need to free 25 effort. Drop something with low marginal value per effort. Candidates: B45 (461/60 = 7.68 per effort) very good; C34 (484/54=8.96) great; A84 (500/136=3.68); A58 (457/252=1.81); A89 (508/262=1.94); A27 (822/106=7.76); C70 (968/125=7.74). Worst ratios: A58 and A89 and A84. If we add C35 106 effort, we need +25 over slack, so drop something with effort ≥25. The smallest drop could be B45(60) but it’s too valuable. Better swap: drop A58 (252 effort, 457 importance) to add C35 (106, 444), net effort -146 leaving more slack, net importance -13 (slightly worse). Maybe drop A89 (262,508) to add C35 (106,444): net importance -64 worse. Drop A84 (136,500) to add C35 (106,444): net importance -56 worse but frees only 30 effort. Alternatively add C35 without drop by swapping a tiny item: we need 25 effort; could drop C34 (54,484) to add C35 (106,444): net importance -40 worse. Drop B45 (60,461): net -17 worse. So C35 swap seems not improving total importance. What about adding C92 (164,215) if we drop something: e.g., drop A58 (252,457) to add C92: net -242 worse. Not good. Maybe add C24 (763,273) terrible. Maybe include B21 (830,138) no. B00 (638,583) not in set; per effort 0.914 but importance 583 is nice; compare to an existing similar-effort item with lower importance to me? B46(407,606) is good. B60(324,808) great. A66(517,822) great. A08(607,868) great. Could we replace something like A84(136,500) plus slack to add B00 (638). Need +638-81 = 557 additional freed; would have to drop multiple valuable items, likely worse. Their set already looks near Pareto-optimal for me. So best strategy: accept quickly to lock in high utility, unless I can ask for a tiny improvement that seems plausible to them. Maybe propose one micro-adjustment: swap A58 (252,457) for C35 (106,444). This slightly decreases my own importance by 13, so not good. How about swap A84 (136,500) for C35 (106,444) loses 56. Swap A89 for C35 loses 64. Swap B45 for C35 loses 17. So none are better. Another candidate not included: B10 (116 effort, 623 importance!) Oh wow, B10 is huge value 623 with tiny effort 116 and is not in their set. We should try to include B10. We have 81 slack, need 35 more effort. We can drop the worst per-effort item to make room 35 effort minimal loss. Candidates small-effort drops: C34(54,484) too much loss; B45(60,461) too much; A84(136,500) dropping frees 136, then can add B10 net importance +123 (623-500) and net effort + (116-136) = -20, still within limit. That’s a strict improvement for me. Or drop A58(252,457) to add B10: net importance +166 (623-457) and effort -136, even better. So propose modify by adding B10 and dropping A58, which improves my utility and likely also good for them because effort reduces significantly, giving room for any of their needs. New set effort: 8689 - 252 + 116 = 8553 ≤ 8770. Keep everything else. This seems strong. I’ll propose this as a small adjustment. If they resist, I can accept their original to secure high value. I’ll include ARGUMENT emphasizing efficiency: frees effort and boosts value. I’ll not reveal my valuations.'}\\ \tt \\ \tt PROPOSAL: {'C70', 'C34', 'B45', 'A27', 'A84', 'A30', 'A89', 'B60', 'B46', 'A66', 'A08', 'A16', 'B09', 'B67', 'B94', 'B12', 'C81', 'C15', 'C12', 'B10'}\\ \tt \\ \tt ARGUMENT: {'I like your efficient core and mostly agree. Could we make a small improvement by adding B10 (very compact, 116 effort) and dropping A58 (252 effort)? This keeps the spirit of your set, reduces total effort to 8553 ≤ 8770, and strengthens overall value density without major changes. Open to hearing if A58 is particularly important to you; if so, we can discuss an alternative small swap.'} 
	  } 
	   } 
	   } 
	  \\ 
 

    \theutterance \stepcounter{utterance}  

    & & \multicolumn{4}{p{0.6\linewidth}}{\cellcolor[rgb]{0.9,0.9,0.9}{%
	\makecell[{{p{\linewidth}}}]{% 
	  \tt {\tiny [A$\langle$GM]}  
	 PROPOSAL: {'C70', 'C34', 'B45', 'A27', 'A84', 'A30', 'A89', 'B60', 'B46', 'A66', 'A08', 'A16', 'B09', 'B67', 'B94', 'B12', 'C81', 'C15', 'C12', 'B10'}\\ \tt \\ \tt ARGUMENT: {'I like your efficient core and mostly agree. Could we make a small improvement by adding B10 (very compact, 116 effort) and dropping A58 (252 effort)? This keeps the spirit of your set, reduces total effort to 8553 ≤ 8770, and strengthens overall value density without major changes. Open to hearing if A58 is particularly important to you; if so, we can discuss an alternative small swap.'} 
	  } 
	   } 
	   } 
	 & \\ 
 

    \theutterance \stepcounter{utterance}  

    & \multicolumn{4}{p{0.6\linewidth}}{\cellcolor[rgb]{0.8,1,0.9}{%
	\makecell[{{p{\linewidth}}}]{% 
	  \tt {\tiny [A$\rangle$GM]}  
	 STRATEGIC REASONING: {'Their swap removes A58(importance 457, effort 252) and adds B10(importance 623, effort 116). For me, that’s strictly better: +166 importance and -136 effort. New total effort 8553 stays under 8770. No reason to haggle; accept immediately to lock in improved utility.'}\\ \tt \\ \tt AGREE: {'C70', 'C34', 'B45', 'A27', 'A84', 'A30', 'A89', 'B60', 'B46', 'A66', 'A08', 'A16', 'B09', 'B67', 'B94', 'B12', 'C81', 'C15', 'C12', 'B10'}\\ \tt \\ \tt ARGUMENT: {'This set maintains the efficient core, improves value density, and stays well within the limit. Happy to lock this in.'} 
	  } 
	   } 
	   } 
	 & & \\ 
 

    \theutterance \stepcounter{utterance}  

    & & & \multicolumn{2}{p{0.3\linewidth}}{\cellcolor[rgb]{0.95,0.95,0.95}{%
	\makecell[{{p{\linewidth}}}]{% 
	  \tt {\tiny [GM$|$GM]}  
	 game successful 
	  } 
	   } 
	   } 
	 & & \\ 
 

    \theutterance \stepcounter{utterance}  

    & & & \multicolumn{2}{p{0.3\linewidth}}{\cellcolor[rgb]{0.95,0.95,0.95}{%
	\makecell[{{p{\linewidth}}}]{% 
	  \tt {\tiny [GM$|$GM]}  
	 end game 
	  } 
	   } 
	   } 
	 & & \\ 
 

\end{supertabular}
}

\end{document}
