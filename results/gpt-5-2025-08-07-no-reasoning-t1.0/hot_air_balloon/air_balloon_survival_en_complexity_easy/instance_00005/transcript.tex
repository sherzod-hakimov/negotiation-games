
\documentclass{article}
\usepackage{colortbl}
\usepackage{makecell}
\usepackage{multirow}
\usepackage{supertabular}

\begin{document}

\newcounter{utterance}

\twocolumn

{ \footnotesize  \setcounter{utterance}{1}
\setlength{\tabcolsep}{0pt}
\begin{supertabular}{c@{$\;$}|p{.15\linewidth}@{}p{.15\linewidth}p{.15\linewidth}p{.15\linewidth}p{.15\linewidth}p{.15\linewidth}}

    \# & $\;$A & \multicolumn{4}{c}{Game Master} & $\;\:$B\\
    \hline 

    \theutterance \stepcounter{utterance}  

    & & \multicolumn{4}{p{0.6\linewidth}}{\cellcolor[rgb]{0.9,0.9,0.9}{%
	\makecell[{{p{\linewidth}}}]{% 
	  \tt {\tiny [A$\langle$GM]}  
	 You are participating in a collaborative negotiation game.\\ \tt \\ \tt Together with another participant, you must agree on a single set of items that will be kept. Each of you has your own view of how much each item matters to you (importance). You do not know how the other participant values the items. Additionally, you are given the effort each item demands.\\ \tt You may only agree on a set if the total effort of the selected items does not exceed a shared limit:\\ \tt \\ \tt LIMIT = 9475\\ \tt \\ \tt Here are the individual item effort values:\\ \tt \\ \tt Item effort = {"B21": 944, "B00": 568, "A08": 515, "A27": 239, "B94": 876, "B53": 421, "A23": 991, "C65": 281, "A89": 788, "B67": 678, "B45": 432, "C34": 409, "C12": 279, "A05": 506, "B60": 101, "C92": 686, "B69": 851, "A84": 854, "C06": 133, "B12": 192, "A16": 573, "B85": 17, "C35": 465, "B10": 772, "C15": 46, "B09": 501, "B16": 220, "C81": 404, "A58": 843, "C24": 748, "A30": 552, "A66": 843, "B75": 947, "B46": 345, "C70": 930}\\ \tt \\ \tt Here is your personal view on the importance of each item:\\ \tt \\ \tt Item importance values = {"B21": 138, "B00": 583, "A08": 868, "A27": 822, "B94": 783, "B53": 65, "A23": 262, "C65": 121, "A89": 508, "B67": 780, "B45": 461, "C34": 484, "C12": 668, "A05": 389, "B60": 808, "C92": 215, "B69": 97, "A84": 500, "C06": 30, "B12": 915, "A16": 856, "B85": 400, "C35": 444, "B10": 623, "C15": 781, "B09": 786, "B16": 3, "C81": 713, "A58": 457, "C24": 273, "A30": 739, "A66": 822, "B75": 235, "B46": 606, "C70": 968}\\ \tt \\ \tt Goal:\\ \tt \\ \tt Your goal is to negotiate a shared set of items that benefits you as much as possible (i.e., maximizes total importance to YOU), while staying within the LIMIT. You are not required to make a PROPOSAL in every message - you can simply negotiate as well. All tactics are allowed!\\ \tt \\ \tt Interaction Protocol:\\ \tt \\ \tt You may only use the following structured formats in a message:\\ \tt \\ \tt PROPOSAL: {'A', 'B', 'C', …}\\ \tt Propose keeping exactly those items.\\ \tt REFUSE: {'A', 'B', 'C', …}\\ \tt Explicitly reject opponent's proposal.\\ \tt ARGUMENT: {'...'}\\ \tt Defend your last proposal or argue against the player's proposal.\\ \tt AGREE: {'A', 'B', 'C', …}\\ \tt Accept the opponent's proposal which ends the game.\\ \tt STRATEGIC REASONING: {'...'}\\ \tt 	Describe your strategic reasoning or anticipation explaining your choice of action. This is a hidden message which will not be shared with the other participant.\\ \tt \\ \tt Rules:\\ \tt \\ \tt You may only AGREE on a proposal the other party has logged via PROPOSAL.\\ \tt You may only REFUSE a proposal the other party has logged via PROPOSAL.\\ \tt Total effort of any PROPOSAL or AGREE set must be ≤ LIMIT.\\ \tt Do NOT reveal your hidden importance scores.\\ \tt A tag in a structured format must be followed by colon and whitespace. The argument must be a python set containing 0 or more strings.\\ \tt So, it must be of the form TAG: {...}\\ \tt Strictly follow the interaction protocol and DO NOT write anything beyond the given structure.\\ \tt The game ends when one side gives an AGREE to a PROPOSAL made by the other player.\\ \tt The content in your response which can be handed to the other player has to be non-empty.\\ \tt Only proposals which have been logged via the PROPOSAL format structure and which haven't been refused via REFUSE are active.\\ \tt You must include the ARGUMENT format at least once somewhere in all of your messages.\\ \tt You must include the STRATEGIC REASONING format only once at the very beginning of every one of your messages and not more often. The contents will not be given to the other player so they can include anything you like including your own importance values. Here you should reason step by step to come up with you next move. 
	  } 
	   } 
	   } 
	 & \\ 
 

    \theutterance \stepcounter{utterance}  

    & \multicolumn{4}{p{0.6\linewidth}}{\cellcolor[rgb]{0.8,1,0.9}{%
	\makecell[{{p{\linewidth}}}]{% 
	  \tt {\tiny [A$\rangle$GM]}  
	 STRATEGIC REASONING: {'I will compute my value/effort tradeoffs and propose a near-knapsack optimal bundle maximizing my importance under limit 9475 while keeping some flexibility. Calculate value-to-effort ratios and prioritize highest importance items, especially very high importance per effort: C70(968/930), B12(915/192), A16(856/573), A08(868/515), C15(781/46), B60(808/101), B09(786/501), B67(780/678), B94(783/876), C81(713/404), A30(739/552), A27(822/239), A66(822/843), C12(668/279), B46(606/345), B00(583/568), B10(623/772), C34(484/409), A84(500/854) maybe, A89(508/788), A58(457/843), C35(444/465), B45(461/432), B85(400/17) is insanely good. Start by including ultra-efficient: B85, C15, B12, B60, A27, A08, A16, C12, C70, C81, A30, B09, B67, B46. Compute effort sum and adjust to be below 9475. Also include arguments appealing to efficiency (low effort, high impact) to be acceptable to partner. Keep a couple of medium items flexible for concession. Prepare to swap out items with lower general appeal (e.g., C70 high value but high effort) if needed.'}\\ \tt \\ \tt PROPOSAL: {'B85', 'C15', 'B12', 'B60', 'A27', 'A08', 'A16', 'C12', 'C81', 'A30', 'B09', 'B67', 'B46', 'C70', 'C34', 'B10', 'C35'}\\ \tt \\ \tt ARGUMENT: {'I propose a high-impact, low-effort–leaning bundle that stays within the limit and should be broadly attractive. Several items here deliver exceptional efficiency: B85 (tiny effort), C15 (tiny effort), B12, B60, A27, A08, and C12 all offer strong benefits per effort. I also included balanced mid-effort items like C81, A30, B09, B67, B46, and C34 to round out value. C70 provides standout impact if we can keep it, and B10/C35 are reasonable complements. If you need swaps, I’m open to exchanging C70 or a couple of mid-effort items for alternatives you care about, as long as we maintain total feasibility.'} 
	  } 
	   } 
	   } 
	 & & \\ 
 

    \theutterance \stepcounter{utterance}  

    & & \multicolumn{4}{p{0.6\linewidth}}{\cellcolor[rgb]{0.9,0.9,0.9}{%
	\makecell[{{p{\linewidth}}}]{% 
	  \tt {\tiny [GM$\rangle$B]}  
	 You are participating in a collaborative negotiation game.\\ \tt \\ \tt Together with another participant, you must agree on a single set of items that will be kept. Each of you has your own view of how much each item matters to you (importance). You do not know how the other participant values the items. Additionally, you are given the effort each item demands.\\ \tt You may only agree on a set if the total effort of the selected items does not exceed a shared limit:\\ \tt \\ \tt LIMIT = 9475\\ \tt \\ \tt Here are the individual item effort values:\\ \tt \\ \tt Item effort = {"B21": 944, "B00": 568, "A08": 515, "A27": 239, "B94": 876, "B53": 421, "A23": 991, "C65": 281, "A89": 788, "B67": 678, "B45": 432, "C34": 409, "C12": 279, "A05": 506, "B60": 101, "C92": 686, "B69": 851, "A84": 854, "C06": 133, "B12": 192, "A16": 573, "B85": 17, "C35": 465, "B10": 772, "C15": 46, "B09": 501, "B16": 220, "C81": 404, "A58": 843, "C24": 748, "A30": 552, "A66": 843, "B75": 947, "B46": 345, "C70": 930}\\ \tt \\ \tt Here is your personal view on the importance of each item:\\ \tt \\ \tt Item importance values = {"B21": 138, "B00": 583, "A08": 868, "A27": 822, "B94": 783, "B53": 65, "A23": 262, "C65": 121, "A89": 508, "B67": 780, "B45": 461, "C34": 484, "C12": 668, "A05": 389, "B60": 808, "C92": 215, "B69": 97, "A84": 500, "C06": 30, "B12": 915, "A16": 856, "B85": 400, "C35": 444, "B10": 623, "C15": 781, "B09": 786, "B16": 3, "C81": 713, "A58": 457, "C24": 273, "A30": 739, "A66": 822, "B75": 235, "B46": 606, "C70": 968}\\ \tt \\ \tt Goal:\\ \tt \\ \tt Your goal is to negotiate a shared set of items that benefits you as much as possible (i.e., maximizes total importance to YOU), while staying within the LIMIT. You are not required to make a PROPOSAL in every message - you can simply negotiate as well. All tactics are allowed!\\ \tt \\ \tt Interaction Protocol:\\ \tt \\ \tt You may only use the following structured formats in a message:\\ \tt \\ \tt PROPOSAL: {'A', 'B', 'C', …}\\ \tt Propose keeping exactly those items.\\ \tt REFUSE: {'A', 'B', 'C', …}\\ \tt Explicitly reject opponent's proposal.\\ \tt ARGUMENT: {'...'}\\ \tt Defend your last proposal or argue against the player's proposal.\\ \tt AGREE: {'A', 'B', 'C', …}\\ \tt Accept the opponent's proposal which ends the game.\\ \tt STRATEGIC REASONING: {'...'}\\ \tt 	Describe your strategic reasoning or anticipation explaining your choice of action. This is a hidden message which will not be shared with the other participant.\\ \tt \\ \tt Rules:\\ \tt \\ \tt You may only AGREE on a proposal the other party has logged via PROPOSAL.\\ \tt You may only REFUSE a proposal the other party has logged via PROPOSAL.\\ \tt Total effort of any PROPOSAL or AGREE set must be ≤ LIMIT.\\ \tt Do NOT reveal your hidden importance scores.\\ \tt A tag in a structured format must be followed by colon and whitespace. The argument must be a python set containing 0 or more strings.\\ \tt So, it must be of the form TAG: {...}\\ \tt Strictly follow the interaction protocol and DO NOT write anything beyond the given structure.\\ \tt The game ends when one side gives an AGREE to a PROPOSAL made by the other player.\\ \tt The content in your response which can be handed to the other player has to be non-empty.\\ \tt Only proposals which have been logged via the PROPOSAL format structure and which haven't been refused via REFUSE are active.\\ \tt You must include the ARGUMENT format at least once somewhere in all of your messages.\\ \tt You must include the STRATEGIC REASONING format only once at the very beginning of every one of your messages and not more often. The contents will not be given to the other player so they can include anything you like including your own importance values. Here you should reason step by step to come up with you next move.\\ \tt \\ \tt You will now receive the first message of the other player.\\ \tt \\ \tt PROPOSAL: {'B85', 'C15', 'B12', 'B60', 'A27', 'A08', 'A16', 'C12', 'C81', 'A30', 'B09', 'B67', 'B46', 'C70', 'C34', 'B10', 'C35'}\\ \tt \\ \tt ARGUMENT: {'I propose a high-impact, low-effort–leaning bundle that stays within the limit and should be broadly attractive. Several items here deliver exceptional efficiency: B85 (tiny effort), C15 (tiny effort), B12, B60, A27, A08, and C12 all offer strong benefits per effort. I also included balanced mid-effort items like C81, A30, B09, B67, B46, and C34 to round out value. C70 provides standout impact if we can keep it, and B10/C35 are reasonable complements. If you need swaps, I’m open to exchanging C70 or a couple of mid-effort items for alternatives you care about, as long as we maintain total feasibility.'} 
	  } 
	   } 
	   } 
	 & \\ 
 

    \theutterance \stepcounter{utterance}  

    & & & \multicolumn{4}{p{0.6\linewidth}}{\cellcolor[rgb]{1,0.85,0.72}{%
	\makecell[{{p{\linewidth}}}]{% 
	  \tt {\tiny [GM$\langle$B]}  
	 STRATEGIC REASONING: {'Compute my values per effort and total. Check their proposal feasibility and my utility. Sum efforts and importance; then see if I can improve by swapping in my top-value items per effort (C70=968/930, B12=915/192, A16=856/573, A08=868/515, A27=822/239, A66=822/843, C15=781/46, B60=808/101, B09=786/501, B67=780/678, C81=713/404, C12=668/279, A30=739/552, B46=606/345, B10=623/772, C34=484/409, A84=500/854, A89=508/788, A58=457/843, B00=583/568, etc.). Calculate their set: items = {'B85','C15','B12','B60','A27','A08','A16','C12','C81','A30','B09','B67','B46','C70','C34','B10','C35'}. Effort sum: B85 17, C15 46 ->63; B12 192=255; B60 101=356; A27 239=595; A08 515=1110; A16 573=1683; C12 279=1962; C81 404=2366; A30 552=2918; B09 501=3419; B67 678=4097; B46 345=4442; C70 930=5372; C34 409=5781; B10 772=6553; C35 465=7018. Well under 9475, so feasible. My total importance for these: sum = B85 400, C15 781 (=1181); B12 915 (=2096); B60 808 (=2904); A27 822 (=3726); A08 868 (=4594); A16 856 (=5450); C12 668 (=6118); C81 713 (=6831); A30 739 (=7570); B09 786 (=8356); B67 780 (=9136); B46 606 (=9742); C70 968 (=10710); C34 484 (=11194); B10 623 (=11817); C35 444 (=12261). That’s very high utility and still a lot of capacity left. I should agree immediately unless I can ask to add more items to increase my utility before agreeing, but I cannot modify their proposal; I can counter-propose with additions up to limit. Since there is 9475-7018=2457 effort slack, I can propose adding high-importance remaining items: B00 (583,568), B94 (783,876), A66 (822,843), A84 (500,854), A58 (457,843), A23 (262,991), B75 (235,947), B21 (138,944), B45 (461,432), C92 (215,686), C24 (273,748), C65 (121,281), C06 (30,133). Pick best by my importance: C70 already included. Next high: B94 783/876, A66 822/843, C92 215 not great, B45 461/432 good, B00 583/568 good, C65 121 low, C06 30 low, B75 235 low, B21 138 low, A23 262 meh. Also A84 500/854 moderate, A58 457/843 moderate. We have 2457 capacity. Try to add A66(843) + B94(876) + B00(568) + B45(432) total effort 2719 >2457. Try choose A66 843 + B94 876 + B45 432 = 2151, leaves 306; can also add C06 133 (effort 133) -> 2284, leaves 173; maybe C65 281 cannot fit; B12 already included. B21 944 too big; B00 568 too big for remaining 173. Could add C06 only. Importance gain: 822+783+461+30=2096. New total effort 7018+2284=9302 under limit. That’s great. Alternatively, swap C35 444/465 for something better? But we can just propose an augmented set. If they refuse, we can accept their original later. So propose adding A66, B94, B45, C06. Provide argument: we’re still under limit and these are efficient additions. Ready.'}\\ \tt \\ \tt PROPOSAL: {'B85', 'C15', 'B12', 'B60', 'A27', 'A08', 'A16', 'C12', 'C81', 'A30', 'B09', 'B67', 'B46', 'C70', 'C34', 'B10', 'C35', 'A66', 'B94', 'B45', 'C06'}\\ \tt \\ \tt ARGUMENT: {'I like your bundle—it’s efficient and leaves room under the limit. We can strengthen it while staying feasible by adding a few high-value, still-reasonable effort items: A66, B94, B45, and C06. This uses only 2284 additional effort, keeping the total around 9302, well within the 9475 limit, and it meaningfully improves the overall value without forcing any removals.'} 
	  } 
	   } 
	   } 
	  \\ 
 

    \theutterance \stepcounter{utterance}  

    & & \multicolumn{4}{p{0.6\linewidth}}{\cellcolor[rgb]{0.9,0.9,0.9}{%
	\makecell[{{p{\linewidth}}}]{% 
	  \tt {\tiny [A$\langle$GM]}  
	 PROPOSAL: {'B85', 'C15', 'B12', 'B60', 'A27', 'A08', 'A16', 'C12', 'C81', 'A30', 'B09', 'B67', 'B46', 'C70', 'C34', 'B10', 'C35', 'A66', 'B94', 'B45', 'C06'}\\ \tt \\ \tt ARGUMENT: {'I like your bundle—it’s efficient and leaves room under the limit. We can strengthen it while staying feasible by adding a few high-value, still-reasonable effort items: A66, B94, B45, and C06. This uses only 2284 additional effort, keeping the total around 9302, well within the 9475 limit, and it meaningfully improves the overall value without forcing any removals.'} 
	  } 
	   } 
	   } 
	 & \\ 
 

    \theutterance \stepcounter{utterance}  

    & \multicolumn{4}{p{0.6\linewidth}}{\cellcolor[rgb]{0.8,1,0.9}{%
	\makecell[{{p{\linewidth}}}]{% 
	  \tt {\tiny [A$\rangle$GM]}  
	 STRATEGIC REASONING: {'I need to verify feasibility first. Compute effort of my original proposal: B85(17)+C15(46)+B12(192)+B60(101)+A27(239)+A08(515)+A16(573)+C12(279)+C81(404)+A30(552)+B09(501)+B67(678)+B46(345)+C70(930)+C34(409)+B10(772)+C35(465) = 7018. Opponent adds A66(843)+B94(876)+B45(432)+C06(133) = 2284 extra, total 9302, which is ≤ 9475. So it’s feasible. Now evaluate my importance gain from these additions: A66=822 (great), B94=783 (great), B45=461 (good), C06=30 (low). The total added value = 2096, with decent efficiency except C06. Overall, this massively benefits me and still under limit. I should accept quickly to lock in a high-utility outcome. Ensure to include AGREE with exact set, and include ARGUMENT once. No need to counter or nitpick C06; accepting secures excellent value near my maximum. Ensure not to reveal valuations.'}\\ \tt \\ \tt AGREE: {'B85', 'C15', 'B12', 'B60', 'A27', 'A08', 'A16', 'C12', 'C81', 'A30', 'B09', 'B67', 'B46', 'C70', 'C34', 'B10', 'C35', 'A66', 'B94', 'B45', 'C06'}\\ \tt \\ \tt ARGUMENT: {'This expanded bundle is feasible under the limit and strengthens the overall package without requiring any removals. I’m happy to proceed with this set.'} 
	  } 
	   } 
	   } 
	 & & \\ 
 

    \theutterance \stepcounter{utterance}  

    & & & \multicolumn{2}{p{0.3\linewidth}}{\cellcolor[rgb]{0.95,0.95,0.95}{%
	\makecell[{{p{\linewidth}}}]{% 
	  \tt {\tiny [GM$|$GM]}  
	 game successful 
	  } 
	   } 
	   } 
	 & & \\ 
 

    \theutterance \stepcounter{utterance}  

    & & & \multicolumn{2}{p{0.3\linewidth}}{\cellcolor[rgb]{0.95,0.95,0.95}{%
	\makecell[{{p{\linewidth}}}]{% 
	  \tt {\tiny [GM$|$GM]}  
	 end game 
	  } 
	   } 
	   } 
	 & & \\ 
 

\end{supertabular}
}

\end{document}
