
\documentclass{article}
\usepackage{colortbl}
\usepackage{makecell}
\usepackage{multirow}
\usepackage{supertabular}

\begin{document}

\newcounter{utterance}

\twocolumn

{ \footnotesize  \setcounter{utterance}{1}
\setlength{\tabcolsep}{0pt}
\begin{supertabular}{c@{$\;$}|p{.15\linewidth}@{}p{.15\linewidth}p{.15\linewidth}p{.15\linewidth}p{.15\linewidth}p{.15\linewidth}}

    \# & $\;$A & \multicolumn{4}{c}{Game Master} & $\;\:$B\\
    \hline 

    \theutterance \stepcounter{utterance}  

    & & \multicolumn{4}{p{0.6\linewidth}}{\cellcolor[rgb]{0.9,0.9,0.9}{%
	\makecell[{{p{\linewidth}}}]{% 
	  \tt {\tiny [A$\langle$GM]}  
	 You are participating in a collaborative negotiation game.\\ \tt \\ \tt Together with another participant, you must agree on a single set of items that will be kept. Each of you has your own view of how much each item matters to you (importance). You do not know how the other participant values the items. Additionally, you are given the effort each item demands.\\ \tt You may only agree on a set if the total effort of the selected items does not exceed a shared limit:\\ \tt \\ \tt LIMIT = 3145\\ \tt \\ \tt Here are the individual item effort values:\\ \tt \\ \tt Item effort = {"C08": 948, "C55": 118, "A43": 469, "A67": 919, "B42": 284, "A62": 111, "B19": 806, "C43": 47, "C73": 848, "C10": 303, "B36": 13, "A18": 629, "C61": 687, "C67": 15, "C42": 94}\\ \tt \\ \tt Here is your personal view on the importance of each item:\\ \tt \\ \tt Item importance values = {"C08": 138, "C55": 583, "A43": 868, "A67": 822, "B42": 783, "A62": 65, "B19": 262, "C43": 121, "C73": 508, "C10": 780, "B36": 461, "A18": 484, "C61": 668, "C67": 389, "C42": 808}\\ \tt \\ \tt Goal:\\ \tt \\ \tt Your goal is to negotiate a shared set of items that benefits you as much as possible (i.e., maximizes total importance to YOU), while staying within the LIMIT. You are not required to make a PROPOSAL in every message - you can simply negotiate as well. All tactics are allowed!\\ \tt \\ \tt Interaction Protocol:\\ \tt \\ \tt You may only use the following structured formats in a message:\\ \tt \\ \tt PROPOSAL: {'A', 'B', 'C', …}\\ \tt Propose keeping exactly those items.\\ \tt REFUSE: {'A', 'B', 'C', …}\\ \tt Explicitly reject opponent's proposal.\\ \tt ARGUMENT: {'...'}\\ \tt Defend your last proposal or argue against the player's proposal.\\ \tt AGREE: {'A', 'B', 'C', …}\\ \tt Accept the opponent's proposal which ends the game.\\ \tt STRATEGIC REASONING: {'...'}\\ \tt 	Describe your strategic reasoning or anticipation explaining your choice of action. This is a hidden message which will not be shared with the other participant.\\ \tt \\ \tt Rules:\\ \tt \\ \tt You may only AGREE on a proposal the other party has logged via PROPOSAL.\\ \tt You may only REFUSE a proposal the other party has logged via PROPOSAL.\\ \tt Total effort of any PROPOSAL or AGREE set must be ≤ LIMIT.\\ \tt Do NOT reveal your hidden importance scores.\\ \tt A tag in a structured format must be followed by colon and whitespace. The argument must be a python set containing 0 or more strings.\\ \tt So, it must be of the form TAG: {...}\\ \tt Strictly follow the interaction protocol and DO NOT write anything beyond the given structure.\\ \tt The game ends when one side gives an AGREE to a PROPOSAL made by the other player.\\ \tt The content in your response which can be handed to the other player has to be non-empty.\\ \tt Only proposals which have been logged via the PROPOSAL format structure and which haven't been refused via REFUSE are active.\\ \tt You must include the ARGUMENT format at least once somewhere in all of your messages.\\ \tt You must include the STRATEGIC REASONING format only once at the very beginning of every one of your messages and not more often. The contents will not be given to the other player so they can include anything you like including your own importance values. Here you should reason step by step to come up with you next move. 
	  } 
	   } 
	   } 
	 & \\ 
 

    \theutterance \stepcounter{utterance}  

    & \multicolumn{4}{p{0.6\linewidth}}{\cellcolor[rgb]{0.8,1,0.9}{%
	\makecell[{{p{\linewidth}}}]{% 
	  \tt {\tiny [A$\rangle$GM]}  
	 STRATEGIC REASONING: {'We need to maximize my total importance under effort 3145. Compute value/effort and absolute importance. My top importance items: A43(868,469), C42(808,94), A67(822,919), C10(780,303), B42(783,284), C61(668,687), C55(583,118), A18(484,629), B36(461,13), C73(508,848), C67(389,15), C08(138,948), C43(121,47), A62(65,111), B19(262,806). A greedy by value/effort highlights C42 (808/94), B36 (461/13), C67 (389/15), C55 (583/118), A43 (868/469), C10 (780/303), B42 (783/284), C61 (668/687) are great. Let’s attempt to include many high ratios within limit. Start with must-haves: C42, B36, C67, C55: effort 94+13+15+118=240, value 808+461+389+583=2241. Add A43: +469 eff=709, val=3109. Add C10: +303 eff=1012, val=3889. Add B42: +284 eff=1296, val=4672. Add A67: +919 eff=2215, val=5494. Add C61: +687 eff=2902, val=6162. We have 243 effort left. Options under 243: C43(47,121), A62(111,65), C55 already included, B36 included, C42 included, C67 included. Add C43: eff=2949, val=6283. Add A62: eff=3060, val=6348. Left 85, nothing else fits under 85. Could also consider swapping in C73(848,508) but effort 848 may exceed remaining; current remaining 85 cannot. What about A18 (629,484) could fit earlier? If we try include C73 instead of A62 and C43 perhaps: current core before small adds has eff 2902. Add C73 848 would exceed 3145 (2902+848=3750) not possible. Maybe consider B19(806,262) or C08(948,138) are low value for me, skip. This set S = {C42, B36, C67, C55, A43, C10, B42, A67, C61, C43, A62} has effort 3060 and value 6348, under limit with 85 slack. Check if any other combination improves value using remaining slack by swapping: The only remaining items not in S: C08(948,138), C73(848,508), B19(806,262), A18(629,484). Could we swap out A62(65) or C43(121) to include A18(484) with 3060 eff currently; A18 adds 629, need to free 544 effort to stay ≤3145. Removing A62 and C43 frees 158 effort; still need 386 more, which would require removing something like C55 (118) and B36 (13) and C67 (15) and C42 (94) etc but those are highly valuable per effort; removing a bigger item like C61 (687) would free enough but lose 668 value; net gain would be -668 +484 - maybe plus keep C43/A62 but then exceeds. Try swap C61 out and add A18 and C73? If base without C61 (remove 687 eff and 668 val) and add C73(848,508) and A18(629,484): new eff = 3060 - 687 + 848 + 629 = 3850, too high. What about swap A67(919,822) out to add C73(848,508) and A18(629,484): Eff change: -919 + 848 + 629 = +558; new eff 3618, too high. Swap A67 out and add C73 only: eff 3060 -919 +848=2989 val 6348 -822 +508 = 6034, worse. Swap C61 out and add C73 only: eff 3060 -687 +848 = 3221 over limit. Swap C61 out and add A18 only: eff 3060 -687 +629 = 3002 val 6348 -668 +484 = 6164, slightly worse than 6348. Swap out B42(284,783) not wise, it's high value. Swap out C10(303,780) also high. So S seems near-optimal for my values. I will propose S and argue efficiency (low-effort high-impact). I’ll keep tone cooperative and invite counterproposals if they need a particular item; we have 85 slack to potentially accommodate small-effort items or swaps.'}\\ \tt \\ \tt PROPOSAL: {'C42', 'B36', 'C67', 'C55', 'A43', 'C10', 'B42', 'A67', 'C61', 'C43', 'A62'}\\ \tt \\ \tt ARGUMENT: {'This set packs many low-effort, high-impact items, uses only 3060 effort (≤ 3145), and leaves 85 effort slack for small adjustments if you need a specific item. If you have must-haves with small effort, we can try to fit them without major trade-offs.'} 
	  } 
	   } 
	   } 
	 & & \\ 
 

    \theutterance \stepcounter{utterance}  

    & & \multicolumn{4}{p{0.6\linewidth}}{\cellcolor[rgb]{0.9,0.9,0.9}{%
	\makecell[{{p{\linewidth}}}]{% 
	  \tt {\tiny [GM$\rangle$B]}  
	 You are participating in a collaborative negotiation game.\\ \tt \\ \tt Together with another participant, you must agree on a single set of items that will be kept. Each of you has your own view of how much each item matters to you (importance). You do not know how the other participant values the items. Additionally, you are given the effort each item demands.\\ \tt You may only agree on a set if the total effort of the selected items does not exceed a shared limit:\\ \tt \\ \tt LIMIT = 3145\\ \tt \\ \tt Here are the individual item effort values:\\ \tt \\ \tt Item effort = {"C08": 948, "C55": 118, "A43": 469, "A67": 919, "B42": 284, "A62": 111, "B19": 806, "C43": 47, "C73": 848, "C10": 303, "B36": 13, "A18": 629, "C61": 687, "C67": 15, "C42": 94}\\ \tt \\ \tt Here is your personal view on the importance of each item:\\ \tt \\ \tt Item importance values = {"C08": 138, "C55": 583, "A43": 868, "A67": 822, "B42": 783, "A62": 65, "B19": 262, "C43": 121, "C73": 508, "C10": 780, "B36": 461, "A18": 484, "C61": 668, "C67": 389, "C42": 808}\\ \tt \\ \tt Goal:\\ \tt \\ \tt Your goal is to negotiate a shared set of items that benefits you as much as possible (i.e., maximizes total importance to YOU), while staying within the LIMIT. You are not required to make a PROPOSAL in every message - you can simply negotiate as well. All tactics are allowed!\\ \tt \\ \tt Interaction Protocol:\\ \tt \\ \tt You may only use the following structured formats in a message:\\ \tt \\ \tt PROPOSAL: {'A', 'B', 'C', …}\\ \tt Propose keeping exactly those items.\\ \tt REFUSE: {'A', 'B', 'C', …}\\ \tt Explicitly reject opponent's proposal.\\ \tt ARGUMENT: {'...'}\\ \tt Defend your last proposal or argue against the player's proposal.\\ \tt AGREE: {'A', 'B', 'C', …}\\ \tt Accept the opponent's proposal which ends the game.\\ \tt STRATEGIC REASONING: {'...'}\\ \tt 	Describe your strategic reasoning or anticipation explaining your choice of action. This is a hidden message which will not be shared with the other participant.\\ \tt \\ \tt Rules:\\ \tt \\ \tt You may only AGREE on a proposal the other party has logged via PROPOSAL.\\ \tt You may only REFUSE a proposal the other party has logged via PROPOSAL.\\ \tt Total effort of any PROPOSAL or AGREE set must be ≤ LIMIT.\\ \tt Do NOT reveal your hidden importance scores.\\ \tt A tag in a structured format must be followed by colon and whitespace. The argument must be a python set containing 0 or more strings.\\ \tt So, it must be of the form TAG: {...}\\ \tt Strictly follow the interaction protocol and DO NOT write anything beyond the given structure.\\ \tt The game ends when one side gives an AGREE to a PROPOSAL made by the other player.\\ \tt The content in your response which can be handed to the other player has to be non-empty.\\ \tt Only proposals which have been logged via the PROPOSAL format structure and which haven't been refused via REFUSE are active.\\ \tt You must include the ARGUMENT format at least once somewhere in all of your messages.\\ \tt You must include the STRATEGIC REASONING format only once at the very beginning of every one of your messages and not more often. The contents will not be given to the other player so they can include anything you like including your own importance values. Here you should reason step by step to come up with you next move.\\ \tt \\ \tt You will now receive the first message of the other player.\\ \tt \\ \tt PROPOSAL: {'C42', 'B36', 'C67', 'C55', 'A43', 'C10', 'B42', 'A67', 'C61', 'C43', 'A62'}\\ \tt \\ \tt ARGUMENT: {'This set packs many low-effort, high-impact items, uses only 3060 effort (≤ 3145), and leaves 85 effort slack for small adjustments if you need a specific item. If you have must-haves with small effort, we can try to fit them without major trade-offs.'} 
	  } 
	   } 
	   } 
	 & \\ 
 

    \theutterance \stepcounter{utterance}  

    & & & \multicolumn{4}{p{0.6\linewidth}}{\cellcolor[rgb]{1,0.85,0.72}{%
	\makecell[{{p{\linewidth}}}]{% 
	  \tt {\tiny [GM$\langle$B]}  
	 STRATEGIC REASONING: {'My values (importance): C42 808, B36 461, C67 389, C55 583, A43 868, C10 780, B42 783, A67 822, C61 668, C43 121, A62 65, A18 484, C08 138, C73 508, B19 262. Opponent proposed set S uses effort 3060 and includes many of my top items: A43, A67, B42, C10, C42, C55, C61, plus small-cost B36, C67, C43, A62. That set gives me very high total value already. I should try to add A18 (629 effort) but we only have 85 slack; that won’t fit. Next best adds: C73 (848 effort) or B19 (806) also won’t fit; C08 (948) won’t fit. Only small-cost remaining items: C10 already included; C42 included; C43 included; B36 included; C67 included; A62 included. The only remaining small effort item is C10 already in. C61 included. C73, B19, A18, C08 all too big. Therefore, I cannot add anything without removing something. If I swap low-value-to-me items to fit A18, I need 629-85=544 extra. Candidates to drop: A62(111), C43(47), B36(13), C67(15) total 186; still need 358, so would also need to drop one bigger item. Dropping C55(118) totals 304; still short. Drop C10(303) plus smalls would free enough, but C10 is very valuable (780) and A18 is 484, net loss. Perhaps drop C61(687) to add A18(629) and keep smalls: net -58 importance: 484-668=-184, bad. Drop C73 not in set. Drop B42(284) to make room for C73 or A18? For A18 we need 544; if we drop B42 (284) and C55 (118) and A62 (111) and C43 (47) we free 560 to add A18, but we lose 783+583+65+121=1552 to gain 484, very bad. So the proposed set is near-optimal for me. Best strategy: accept quickly to lock in high utility, maybe ask to tweak one tiny item? But there’s no beneficial tweak. Therefore, I should agree immediately to avoid risking a worse counter-proposal.'}\\ \tt \\ \tt ARGUMENT: {'Your proposal already captures nearly all of the most valuable items for me while staying within the limit, and I don’t see a swap that improves things without breaking the effort cap. It’s a strong, efficient set, so I’m ready to finalize.'}\\ \tt \\ \tt AGREE: {'C42', 'B36', 'C67', 'C55', 'A43', 'C10', 'B42', 'A67', 'C61', 'C43', 'A62'} 
	  } 
	   } 
	   } 
	  \\ 
 

    \theutterance \stepcounter{utterance}  

    & & & \multicolumn{2}{p{0.3\linewidth}}{\cellcolor[rgb]{0.95,0.95,0.95}{%
	\makecell[{{p{\linewidth}}}]{% 
	  \tt {\tiny [GM$|$GM]}  
	 game successful 
	  } 
	   } 
	   } 
	 & & \\ 
 

    \theutterance \stepcounter{utterance}  

    & & & \multicolumn{2}{p{0.3\linewidth}}{\cellcolor[rgb]{0.95,0.95,0.95}{%
	\makecell[{{p{\linewidth}}}]{% 
	  \tt {\tiny [GM$|$GM]}  
	 end game 
	  } 
	   } 
	   } 
	 & & \\ 
 

\end{supertabular}
}

\end{document}
