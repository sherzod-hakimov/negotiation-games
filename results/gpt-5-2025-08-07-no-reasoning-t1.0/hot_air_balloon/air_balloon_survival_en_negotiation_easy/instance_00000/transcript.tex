
\documentclass{article}
\usepackage{colortbl}
\usepackage{makecell}
\usepackage{multirow}
\usepackage{supertabular}

\begin{document}

\newcounter{utterance}

\twocolumn

{ \footnotesize  \setcounter{utterance}{1}
\setlength{\tabcolsep}{0pt}
\begin{supertabular}{c@{$\;$}|p{.15\linewidth}@{}p{.15\linewidth}p{.15\linewidth}p{.15\linewidth}p{.15\linewidth}p{.15\linewidth}}

    \# & $\;$A & \multicolumn{4}{c}{Game Master} & $\;\:$B\\
    \hline 

    \theutterance \stepcounter{utterance}  

    & & \multicolumn{4}{p{0.6\linewidth}}{\cellcolor[rgb]{0.9,0.9,0.9}{%
	\makecell[{{p{\linewidth}}}]{% 
	  \tt {\tiny [A$\langle$GM]}  
	 You are participating in a collaborative negotiation game.\\ \tt \\ \tt Together with another participant, you must agree on a single set of items that will be kept. Each of you has your own view of how much each item matters to you (importance). You do not know how the other participant values the items. Additionally, you are given the effort each item demands.\\ \tt You may only agree on a set if the total effort of the selected items does not exceed a shared limit:\\ \tt \\ \tt LIMIT = 3358\\ \tt \\ \tt Here are the individual item effort values:\\ \tt \\ \tt Item effort = {"C08": 171, "C55": 933, "A43": 552, "A67": 219, "B42": 275, "A62": 778, "B19": 341, "C43": 615, "C73": 519, "C10": 862, "B36": 262, "A18": 377, "C61": 347, "C67": 349, "C42": 117}\\ \tt \\ \tt Here is your personal view on the importance of each item:\\ \tt \\ \tt Item importance values = {"C08": 138, "C55": 583, "A43": 868, "A67": 822, "B42": 783, "A62": 65, "B19": 262, "C43": 121, "C73": 508, "C10": 780, "B36": 461, "A18": 484, "C61": 668, "C67": 389, "C42": 808}\\ \tt \\ \tt Goal:\\ \tt \\ \tt Your goal is to negotiate a shared set of items that benefits you as much as possible (i.e., maximizes total importance to YOU), while staying within the LIMIT. You are not required to make a PROPOSAL in every message - you can simply negotiate as well. All tactics are allowed!\\ \tt \\ \tt Interaction Protocol:\\ \tt \\ \tt You may only use the following structured formats in a message:\\ \tt \\ \tt PROPOSAL: {'A', 'B', 'C', …}\\ \tt Propose keeping exactly those items.\\ \tt REFUSE: {'A', 'B', 'C', …}\\ \tt Explicitly reject opponent's proposal.\\ \tt ARGUMENT: {'...'}\\ \tt Defend your last proposal or argue against the player's proposal.\\ \tt AGREE: {'A', 'B', 'C', …}\\ \tt Accept the opponent's proposal which ends the game.\\ \tt STRATEGIC REASONING: {'...'}\\ \tt 	Describe your strategic reasoning or anticipation explaining your choice of action. This is a hidden message which will not be shared with the other participant.\\ \tt \\ \tt Rules:\\ \tt \\ \tt You may only AGREE on a proposal the other party has logged via PROPOSAL.\\ \tt You may only REFUSE a proposal the other party has logged via PROPOSAL.\\ \tt Total effort of any PROPOSAL or AGREE set must be ≤ LIMIT.\\ \tt Do NOT reveal your hidden importance scores.\\ \tt A tag in a structured format must be followed by colon and whitespace. The argument must be a python set containing 0 or more strings.\\ \tt So, it must be of the form TAG: {...}\\ \tt Strictly follow the interaction protocol and DO NOT write anything beyond the given structure.\\ \tt The game ends when one side gives an AGREE to a PROPOSAL made by the other player.\\ \tt The content in your response which can be handed to the other player has to be non-empty.\\ \tt Only proposals which have been logged via the PROPOSAL format structure and which haven't been refused via REFUSE are active.\\ \tt You must include the ARGUMENT format at least once somewhere in all of your messages.\\ \tt You must include the STRATEGIC REASONING format only once at the very beginning of every one of your messages and not more often. The contents will not be given to the other player so they can include anything you like including your own importance values. Here you should reason step by step to come up with you next move. 
	  } 
	   } 
	   } 
	 & \\ 
 

    \theutterance \stepcounter{utterance}  

    & \multicolumn{4}{p{0.6\linewidth}}{\cellcolor[rgb]{0.8,1,0.9}{%
	\makecell[{{p{\linewidth}}}]{% 
	  \tt {\tiny [A$\rangle$GM]}  
	 STRATEGIC REASONING: {'We want to maximize our importance under limit 3358. Compute values: items with high importance/effort ratio and absolute importance. Sort by importance: A43 868(552), C42 808(117), C10 780(862), B42 783(275), C61 668(347), A67 822(219), C55 583(933), A18 484(377), C73 508(519), B36 461(262), C67 389(349), B19 262(341), C08 138(171), C43 121(615), A62 65(778). Great picks by ratio: C42 (808/117), A67 (822/219), B42 (783/275), A43 (868/552), B36 (461/262), C61 (668/347), C73 (508/519), A18 (484/377), C10 (780/862), C67 (389/349). Let's try to build near limit with high sum. Start with top ratios: C42(117), A67(219)=336, B42(275)=611, A43(552)=1163, B36(262)=1425, C61(347)=1772, A18(377)=2149, C73(519)=2668, C10(862) would push to 3530>3358 so too big. Try C67(349): 2668+349=3017, remaining 341 can fit B19(341) to exactly 3358! Sum importance: C42 808 + A67 822 + B42 783 + A43 868 + B36 461 + C61 668 + A18 484 + C73 508 + C67 389 + B19 262 = 6053. Effort sum computed: 117+219+275+552+262+347+377+519+349+341=3358 exactly the limit. This is perfect optimal likely. We exclude C10, C55, C08, C43, A62 as low value or high effort. We'll propose this full set. Provide argument: it uses full budget and covers balanced spread and excludes heavy lifts with lower returns. If they counter, we can consider swapping B19/C67 for C10 or C55 if they care, but those are high effort; we have no slack. To include C10 (862) we must drop e.g., B19(341)+C67(349)+C42(117)=807 and more like C08(171) etc; not good for our value. We open firm but can offer to replace B19 (low importance to us) if needed by dropping B19+something to add their item with similar effort.'}\\ \tt \\ \tt PROPOSAL: {'C42', 'A67', 'B42', 'A43', 'B36', 'C61', 'A18', 'C73', 'C67', 'B19'}\\ \tt \\ \tt ARGUMENT: {'This set uses the full effort limit exactly and focuses on efficient, broadly useful items while avoiding heavy-effort, lower-return picks. It balances several categories and should be practical to implement without overloading the budget.'} 
	  } 
	   } 
	   } 
	 & & \\ 
 

    \theutterance \stepcounter{utterance}  

    & & \multicolumn{4}{p{0.6\linewidth}}{\cellcolor[rgb]{0.9,0.9,0.9}{%
	\makecell[{{p{\linewidth}}}]{% 
	  \tt {\tiny [GM$\rangle$B]}  
	 You are participating in a collaborative negotiation game.\\ \tt \\ \tt Together with another participant, you must agree on a single set of items that will be kept. Each of you has your own view of how much each item matters to you (importance). You do not know how the other participant values the items. Additionally, you are given the effort each item demands.\\ \tt You may only agree on a set if the total effort of the selected items does not exceed a shared limit:\\ \tt \\ \tt LIMIT = 3358\\ \tt \\ \tt Here are the individual item effort values:\\ \tt \\ \tt Item effort = {"C08": 171, "C55": 933, "A43": 552, "A67": 219, "B42": 275, "A62": 778, "B19": 341, "C43": 615, "C73": 519, "C10": 862, "B36": 262, "A18": 377, "C61": 347, "C67": 349, "C42": 117}\\ \tt \\ \tt Here is your personal view on the importance of each item:\\ \tt \\ \tt Item importance values = {"C08": 138, "C55": 583, "A43": 868, "A67": 822, "B42": 783, "A62": 65, "B19": 262, "C43": 121, "C73": 508, "C10": 780, "B36": 461, "A18": 484, "C61": 668, "C67": 389, "C42": 808}\\ \tt \\ \tt Goal:\\ \tt \\ \tt Your goal is to negotiate a shared set of items that benefits you as much as possible (i.e., maximizes total importance to YOU), while staying within the LIMIT. You are not required to make a PROPOSAL in every message - you can simply negotiate as well. All tactics are allowed!\\ \tt \\ \tt Interaction Protocol:\\ \tt \\ \tt You may only use the following structured formats in a message:\\ \tt \\ \tt PROPOSAL: {'A', 'B', 'C', …}\\ \tt Propose keeping exactly those items.\\ \tt REFUSE: {'A', 'B', 'C', …}\\ \tt Explicitly reject opponent's proposal.\\ \tt ARGUMENT: {'...'}\\ \tt Defend your last proposal or argue against the player's proposal.\\ \tt AGREE: {'A', 'B', 'C', …}\\ \tt Accept the opponent's proposal which ends the game.\\ \tt STRATEGIC REASONING: {'...'}\\ \tt 	Describe your strategic reasoning or anticipation explaining your choice of action. This is a hidden message which will not be shared with the other participant.\\ \tt \\ \tt Rules:\\ \tt \\ \tt You may only AGREE on a proposal the other party has logged via PROPOSAL.\\ \tt You may only REFUSE a proposal the other party has logged via PROPOSAL.\\ \tt Total effort of any PROPOSAL or AGREE set must be ≤ LIMIT.\\ \tt Do NOT reveal your hidden importance scores.\\ \tt A tag in a structured format must be followed by colon and whitespace. The argument must be a python set containing 0 or more strings.\\ \tt So, it must be of the form TAG: {...}\\ \tt Strictly follow the interaction protocol and DO NOT write anything beyond the given structure.\\ \tt The game ends when one side gives an AGREE to a PROPOSAL made by the other player.\\ \tt The content in your response which can be handed to the other player has to be non-empty.\\ \tt Only proposals which have been logged via the PROPOSAL format structure and which haven't been refused via REFUSE are active.\\ \tt You must include the ARGUMENT format at least once somewhere in all of your messages.\\ \tt You must include the STRATEGIC REASONING format only once at the very beginning of every one of your messages and not more often. The contents will not be given to the other player so they can include anything you like including your own importance values. Here you should reason step by step to come up with you next move.\\ \tt \\ \tt You will now receive the first message of the other player.\\ \tt \\ \tt PROPOSAL: {'C42', 'A67', 'B42', 'A43', 'B36', 'C61', 'A18', 'C73', 'C67', 'B19'}\\ \tt \\ \tt ARGUMENT: {'This set uses the full effort limit exactly and focuses on efficient, broadly useful items while avoiding heavy-effort, lower-return picks. It balances several categories and should be practical to implement without overloading the budget.'} 
	  } 
	   } 
	   } 
	 & \\ 
 

    \theutterance \stepcounter{utterance}  

    & & & \multicolumn{4}{p{0.6\linewidth}}{\cellcolor[rgb]{1,0.85,0.72}{%
	\makecell[{{p{\linewidth}}}]{% 
	  \tt {\tiny [GM$\langle$B]}  
	 STRATEGIC REASONING: {'Compute my values and efforts for their proposal and compare to alternatives. Efforts for their set sum they claim equals limit; verify: C42 117 + A67 219 + B42 275 + A43 552 + B36 262 + C61 347 + A18 377 + C73 519 + C67 349 + B19 341 = 3,358 which equals LIMIT 3358. My total importance for this set: C42 808 + A67 822 + B42 783 + A43 868 + B36 461 + C61 668 + A18 484 + C73 508 + C67 389 + B19 262 = 6,053. That’s very high for me. Check if there is any dominated item I’d swap to increase my value further under limit: Consider high-value items not included: C10 780 (effort 862), C55 583 (933), C08 138 (171), C43 121 (615), A62 65 (778). The current set excludes C10 and C55 which are high value but very costly; to include them we’d need to drop multiple items which may reduce my total value. Try possible swap: replace C73 (508,519) + B19 (262,341)= value 770 effort 860 to include C10 (780,862): effort roughly equal but value +10; but we also need to remove 2 extra effort (since 860 vs 862 +2), we would need to drop more or adjust exact. Actually current total is exact. If we drop C73 and B19 (-860 effort) and add C10 (+862), we exceed by 2, so need to remove at least 2 more effort, meaning remove something small like C42 117 and add C08 171? That worsens value. Another combo: drop B19 (341) and C61 (347) total 688 effort, add C10 862 exceeds by 174; not possible without removing more. Generally seems the proposed set is near Pareto for me given budget. Therefore best move is to accept to lock in high utility. Before agreeing, per rules, I can AGREE only to their proposal. Also must include ARGUMENT at least once. I should directly AGREE to end the game. I will include an ARGUMENT praising balance and fit. No need to counter. '}\\ \tt \\ \tt AGREE: {'C42', 'A67', 'B42', 'A43', 'B36', 'C61', 'A18', 'C73', 'C67', 'B19'}\\ \tt \\ \tt ARGUMENT: {'This selection fits the limit exactly and offers a strong balance across priorities. It keeps several high-impact items while maintaining overall efficiency, so I’m happy to proceed with it.'} 
	  } 
	   } 
	   } 
	  \\ 
 

    \theutterance \stepcounter{utterance}  

    & & & \multicolumn{2}{p{0.3\linewidth}}{\cellcolor[rgb]{0.95,0.95,0.95}{%
	\makecell[{{p{\linewidth}}}]{% 
	  \tt {\tiny [GM$|$GM]}  
	 game successful 
	  } 
	   } 
	   } 
	 & & \\ 
 

    \theutterance \stepcounter{utterance}  

    & & & \multicolumn{2}{p{0.3\linewidth}}{\cellcolor[rgb]{0.95,0.95,0.95}{%
	\makecell[{{p{\linewidth}}}]{% 
	  \tt {\tiny [GM$|$GM]}  
	 end game 
	  } 
	   } 
	   } 
	 & & \\ 
 

\end{supertabular}
}

\end{document}
