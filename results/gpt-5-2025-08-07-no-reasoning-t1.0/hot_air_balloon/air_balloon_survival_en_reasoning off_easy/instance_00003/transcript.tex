
\documentclass{article}
\usepackage{colortbl}
\usepackage{makecell}
\usepackage{multirow}
\usepackage{supertabular}

\begin{document}

\newcounter{utterance}

\twocolumn

{ \footnotesize  \setcounter{utterance}{1}
\setlength{\tabcolsep}{0pt}
\begin{supertabular}{c@{$\;$}|p{.15\linewidth}@{}p{.15\linewidth}p{.15\linewidth}p{.15\linewidth}p{.15\linewidth}p{.15\linewidth}}

    \# & $\;$A & \multicolumn{4}{c}{Game Master} & $\;\:$B\\
    \hline 

    \theutterance \stepcounter{utterance}  

    & & \multicolumn{4}{p{0.6\linewidth}}{\cellcolor[rgb]{0.9,0.9,0.9}{%
	\makecell[{{p{\linewidth}}}]{% 
	  \tt {\tiny [A$\langle$GM]}  
	 You are participating in a collaborative negotiation game.\\ \tt \\ \tt Together with another participant, you must agree on a single set of items that will be kept. Each of you has your own view of how much each item matters to you (importance). You do not know how the other participant values the items. Additionally, you are given the effort each item demands.\\ \tt You may only agree on a set if the total effort of the selected items does not exceed a shared limit:\\ \tt \\ \tt LIMIT = 3207\\ \tt \\ \tt Here are the individual item effort values:\\ \tt \\ \tt Item effort = {"A19": 67, "B91": 45, "A30": 308, "B11": 835, "C47": 819, "C75": 667, "C26": 547, "C73": 322, "A47": 428, "A15": 306, "A75": 327, "A39": 362, "A56": 280, "C24": 334, "B26": 767}\\ \tt \\ \tt Here is your personal view on the importance of each item:\\ \tt \\ \tt Item importance values = {"A19": 138, "B91": 583, "A30": 822, "B11": 783, "C47": 65, "C75": 262, "C26": 121, "C73": 508, "A47": 780, "A15": 461, "A75": 484, "A39": 668, "A56": 389, "C24": 808, "B26": 215}\\ \tt \\ \tt Goal:\\ \tt \\ \tt Your goal is to negotiate a shared set of items that benefits you as much as possible (i.e., maximizes total importance to YOU), while staying within the LIMIT. You are not required to make a PROPOSAL in every message - you can simply negotiate as well. All tactics are allowed!\\ \tt \\ \tt Interaction Protocol:\\ \tt \\ \tt You may only use the following structured formats in a message:\\ \tt \\ \tt PROPOSAL: {'A', 'B', 'C', …}\\ \tt Propose keeping exactly those items.\\ \tt REFUSE: {'A', 'B', 'C', …}\\ \tt Explicitly reject opponent's proposal.\\ \tt ARGUMENT: {'...'}\\ \tt Defend your last proposal or argue against the player's proposal.\\ \tt AGREE: {'A', 'B', 'C', …}\\ \tt Accept the opponent's proposal which ends the game.\\ \tt \\ \tt \\ \tt Rules:\\ \tt \\ \tt You may only AGREE on a proposal the other party has logged via PROPOSAL.\\ \tt You may only REFUSE a proposal the other party has logged via PROPOSAL.\\ \tt Total effort of any PROPOSAL or AGREE set must be ≤ LIMIT.\\ \tt Do NOT reveal your hidden importance scores.\\ \tt A tag in a structured format must be followed by colon and whitespace. The argument must be a python set containing 0 or more strings.\\ \tt So, it must be of the form TAG: {...}\\ \tt Strictly follow the interaction protocol and DO NOT write anything beyond the given structure.\\ \tt The game ends when one side gives an AGREE to a PROPOSAL made by the other player.\\ \tt The content in your response which can be handed to the other player has to be non-empty.\\ \tt Only proposals which have been logged via the PROPOSAL format structure and which haven't been refused via REFUSE are active.\\ \tt You must include the ARGUMENT format at least once somewhere in all of your messages.\\ \tt  
	  } 
	   } 
	   } 
	 & \\ 
 

    \theutterance \stepcounter{utterance}  

    & \multicolumn{4}{p{0.6\linewidth}}{\cellcolor[rgb]{0.8,1,0.9}{%
	\makecell[{{p{\linewidth}}}]{% 
	  \tt {\tiny [A$\rangle$GM]}  
	 PROPOSAL: {'A30', 'B11', 'C73', 'A47', 'A15', 'A75', 'A39', 'A56', 'C24'}\\ \tt \\ \tt ARGUMENT: {'This set stays within the limit (total effort 3201) and offers a strong, well-rounded selection across categories. It prioritizes high-impact items while keeping several mid-effort pieces, leaving a small buffer for feasibility. If you have concerns about any specific item, I’m open to targeted swaps that don’t increase total effort.'} 
	  } 
	   } 
	   } 
	 & & \\ 
 

    \theutterance \stepcounter{utterance}  

    & & \multicolumn{4}{p{0.6\linewidth}}{\cellcolor[rgb]{0.9,0.9,0.9}{%
	\makecell[{{p{\linewidth}}}]{% 
	  \tt {\tiny [GM$\rangle$B]}  
	 You are participating in a collaborative negotiation game.\\ \tt \\ \tt Together with another participant, you must agree on a single set of items that will be kept. Each of you has your own view of how much each item matters to you (importance). You do not know how the other participant values the items. Additionally, you are given the effort each item demands.\\ \tt You may only agree on a set if the total effort of the selected items does not exceed a shared limit:\\ \tt \\ \tt LIMIT = 3207\\ \tt \\ \tt Here are the individual item effort values:\\ \tt \\ \tt Item effort = {"A19": 67, "B91": 45, "A30": 308, "B11": 835, "C47": 819, "C75": 667, "C26": 547, "C73": 322, "A47": 428, "A15": 306, "A75": 327, "A39": 362, "A56": 280, "C24": 334, "B26": 767}\\ \tt \\ \tt Here is your personal view on the importance of each item:\\ \tt \\ \tt Item importance values = {"A19": 138, "B91": 583, "A30": 822, "B11": 783, "C47": 65, "C75": 262, "C26": 121, "C73": 508, "A47": 780, "A15": 461, "A75": 484, "A39": 668, "A56": 389, "C24": 808, "B26": 215}\\ \tt \\ \tt Goal:\\ \tt \\ \tt Your goal is to negotiate a shared set of items that benefits you as much as possible (i.e., maximizes total importance to YOU), while staying within the LIMIT. You are not required to make a PROPOSAL in every message - you can simply negotiate as well. All tactics are allowed!\\ \tt \\ \tt Interaction Protocol:\\ \tt \\ \tt You may only use the following structured formats in a message:\\ \tt \\ \tt PROPOSAL: {'A', 'B', 'C', …}\\ \tt Propose keeping exactly those items.\\ \tt REFUSE: {'A', 'B', 'C', …}\\ \tt Explicitly reject opponent's proposal.\\ \tt ARGUMENT: {'...'}\\ \tt Defend your last proposal or argue against the player's proposal.\\ \tt AGREE: {'A', 'B', 'C', …}\\ \tt Accept the opponent's proposal which ends the game.\\ \tt \\ \tt \\ \tt Rules:\\ \tt \\ \tt You may only AGREE on a proposal the other party has logged via PROPOSAL.\\ \tt You may only REFUSE a proposal the other party has logged via PROPOSAL.\\ \tt Total effort of any PROPOSAL or AGREE set must be ≤ LIMIT.\\ \tt Do NOT reveal your hidden importance scores.\\ \tt A tag in a structured format must be followed by colon and whitespace. The argument must be a python set containing 0 or more strings.\\ \tt So, it must be of the form TAG: {...}\\ \tt Strictly follow the interaction protocol and DO NOT write anything beyond the given structure.\\ \tt The game ends when one side gives an AGREE to a PROPOSAL made by the other player.\\ \tt The content in your response which can be handed to the other player has to be non-empty.\\ \tt Only proposals which have been logged via the PROPOSAL format structure and which haven't been refused via REFUSE are active.\\ \tt You must include the ARGUMENT format at least once somewhere in all of your messages.\\ \tt \\ \tt \\ \tt You will now receive the first message of the other player.\\ \tt \\ \tt PROPOSAL: {'A30', 'B11', 'C73', 'A47', 'A15', 'A75', 'A39', 'A56', 'C24'}\\ \tt \\ \tt ARGUMENT: {'This set stays within the limit (total effort 3201) and offers a strong, well-rounded selection across categories. It prioritizes high-impact items while keeping several mid-effort pieces, leaving a small buffer for feasibility. If you have concerns about any specific item, I’m open to targeted swaps that don’t increase total effort.'} 
	  } 
	   } 
	   } 
	 & \\ 
 

    \theutterance \stepcounter{utterance}  

    & & & \multicolumn{4}{p{0.6\linewidth}}{\cellcolor[rgb]{1,0.85,0.72}{%
	\makecell[{{p{\linewidth}}}]{% 
	  \tt {\tiny [GM$\langle$B]}  
	 REFUSE: {'A30', 'B11', 'C73', 'A47', 'A15', 'A75', 'A39', 'A56', 'C24'}\\ \tt \\ \tt PROPOSAL: {'A30', 'C73', 'A47', 'A15', 'A75', 'A39', 'A56', 'C24', 'B91'}\\ \tt \\ \tt ARGUMENT: {'Your proposal is close on effort and structure, but it allocates a large chunk to one item that’s hard to justify for me. Swapping in B91 (low effort, high impact for me) in place of B11 keeps us within limit: current total 3201; remove B11 (835) and add B91 (45) brings total to 2411. With the freed capacity we keep the rest intact and gain a lot of value on my side without pushing the effort. If you want to use some of the 796 remaining effort for a specific addition you care about, I’m open to consider it as a follow-up tweak.'} 
	  } 
	   } 
	   } 
	  \\ 
 

    \theutterance \stepcounter{utterance}  

    & & \multicolumn{4}{p{0.6\linewidth}}{\cellcolor[rgb]{0.9,0.9,0.9}{%
	\makecell[{{p{\linewidth}}}]{% 
	  \tt {\tiny [A$\langle$GM]}  
	 REFUSE: {'A30', 'B11', 'C73', 'A47', 'A15', 'A75', 'A39', 'A56', 'C24'}\\ \tt \\ \tt PROPOSAL: {'A30', 'C73', 'A47', 'A15', 'A75', 'A39', 'A56', 'C24', 'B91'}\\ \tt \\ \tt ARGUMENT: {'Your proposal is close on effort and structure, but it allocates a large chunk to one item that’s hard to justify for me. Swapping in B91 (low effort, high impact for me) in place of B11 keeps us within limit: current total 3201; remove B11 (835) and add B91 (45) brings total to 2411. With the freed capacity we keep the rest intact and gain a lot of value on my side without pushing the effort. If you want to use some of the 796 remaining effort for a specific addition you care about, I’m open to consider it as a follow-up tweak.'} 
	  } 
	   } 
	   } 
	 & \\ 
 

    \theutterance \stepcounter{utterance}  

    & \multicolumn{4}{p{0.6\linewidth}}{\cellcolor[rgb]{0.8,1,0.9}{%
	\makecell[{{p{\linewidth}}}]{% 
	  \tt {\tiny [A$\rangle$GM]}  
	 PROPOSAL: {'A30', 'C73', 'A47', 'A15', 'A75', 'A39', 'A56', 'C24', 'B91', 'C75', 'C26'}\\ \tt \\ \tt ARGUMENT: {'I accept swapping in B91. Using the freed capacity, I’m proposing to add C75 and C26—both fit comfortably within the remaining effort and round out the set without overshooting the limit. This keeps your preference for B91 while making efficient use of the budget to strengthen the overall package. Total effort now is 3125, leaving a small buffer.'} 
	  } 
	   } 
	   } 
	 & & \\ 
 

    \theutterance \stepcounter{utterance}  

    & & \multicolumn{4}{p{0.6\linewidth}}{\cellcolor[rgb]{0.9,0.9,0.9}{%
	\makecell[{{p{\linewidth}}}]{% 
	  \tt {\tiny [GM$\rangle$B]}  
	 PROPOSAL: {'A30', 'C73', 'A47', 'A15', 'A75', 'A39', 'A56', 'C24', 'B91', 'C75', 'C26'}\\ \tt \\ \tt ARGUMENT: {'I accept swapping in B91. Using the freed capacity, I’m proposing to add C75 and C26—both fit comfortably within the remaining effort and round out the set without overshooting the limit. This keeps your preference for B91 while making efficient use of the budget to strengthen the overall package. Total effort now is 3125, leaving a small buffer.'} 
	  } 
	   } 
	   } 
	 & \\ 
 

    \theutterance \stepcounter{utterance}  

    & & & \multicolumn{4}{p{0.6\linewidth}}{\cellcolor[rgb]{1,0.85,0.72}{%
	\makecell[{{p{\linewidth}}}]{% 
	  \tt {\tiny [GM$\langle$B]}  
	 ARGUMENT: {'Thanks for incorporating B91. I like most additions, but C26 and C75 together don’t work for me relative to their effort. I can support adding C75 (solid fit) but would prefer to drop C26 and instead include A19 and C47. This keeps us under limit and improves the balance for me without a big effort hit.'}\\ \tt \\ \tt REFUSE: {'A30', 'C73', 'A47', 'A15', 'A75', 'A39', 'A56', 'C24', 'B91', 'C75', 'C26'}\\ \tt \\ \tt PROPOSAL: {'A30', 'C73', 'A47', 'A15', 'A75', 'A39', 'A56', 'C24', 'B91', 'C75', 'A19', 'C47'} 
	  } 
	   } 
	   } 
	  \\ 
 

    \theutterance \stepcounter{utterance}  

    & & \multicolumn{4}{p{0.6\linewidth}}{\cellcolor[rgb]{0.9,0.9,0.9}{%
	\makecell[{{p{\linewidth}}}]{% 
	  \tt {\tiny [A$\langle$GM]}  
	 ARGUMENT: {'Thanks for incorporating B91. I like most additions, but C26 and C75 together don’t work for me relative to their effort. I can support adding C75 (solid fit) but would prefer to drop C26 and instead include A19 and C47. This keeps us under limit and improves the balance for me without a big effort hit.'}\\ \tt \\ \tt REFUSE: {'A30', 'C73', 'A47', 'A15', 'A75', 'A39', 'A56', 'C24', 'B91', 'C75', 'C26'}\\ \tt \\ \tt PROPOSAL: {'A30', 'C73', 'A47', 'A15', 'A75', 'A39', 'A56', 'C24', 'B91', 'C75', 'A19', 'C47'} 
	  } 
	   } 
	   } 
	 & \\ 
 

    \theutterance \stepcounter{utterance}  

    & \multicolumn{4}{p{0.6\linewidth}}{\cellcolor[rgb]{0.8,1,0.9}{%
	\makecell[{{p{\linewidth}}}]{% 
	  \tt {\tiny [A$\rangle$GM]}  
	 ARGUMENT: {'Appreciate the flexibility. Let’s verify feasibility: The set you proposed has total effort 3,126, which exceeds the 3,207 limit when including C26, but with your swap (dropping C26 for A19 and C47), the total becomes 3,126 as well (since C26 wasn’t in your new set). However, your latest proposal still sums to 3,126, which is over the limit of 3,207 only if we had miscounted earlier—let me recalc precisely to ensure we are compliant.'}\\ \tt \\ \tt PROPOSAL: {'A30', 'C73', 'A47', 'A15', 'A75', 'A39', 'A56', 'C24', 'B91', 'C75', 'A19'}\\ \tt \\ \tt ARGUMENT: {'Here’s a corrected set that fits the limit tightly while aligning with your preferences: it includes C75 and A19 (as you wanted) but omits C47 to keep us within budget. Total effort is 3,059, which is safely under 3,207. If C47 is important to you, we could consider swapping A15 (306) or A56 (280) for C47 (819), but that would require another compensating drop to stay under the limit.'} 
	  } 
	   } 
	   } 
	 & & \\ 
 

    \theutterance \stepcounter{utterance}  

    & & \multicolumn{4}{p{0.6\linewidth}}{\cellcolor[rgb]{0.9,0.9,0.9}{%
	\makecell[{{p{\linewidth}}}]{% 
	  \tt {\tiny [GM$\rangle$B]}  
	 ARGUMENT: {'Appreciate the flexibility. Let’s verify feasibility: The set you proposed has total effort 3,126, which exceeds the 3,207 limit when including C26, but with your swap (dropping C26 for A19 and C47), the total becomes 3,126 as well (since C26 wasn’t in your new set). However, your latest proposal still sums to 3,126, which is over the limit of 3,207 only if we had miscounted earlier—let me recalc precisely to ensure we are compliant.'}\\ \tt \\ \tt PROPOSAL: {'A30', 'C73', 'A47', 'A15', 'A75', 'A39', 'A56', 'C24', 'B91', 'C75', 'A19'}\\ \tt \\ \tt ARGUMENT: {'Here’s a corrected set that fits the limit tightly while aligning with your preferences: it includes C75 and A19 (as you wanted) but omits C47 to keep us within budget. Total effort is 3,059, which is safely under 3,207. If C47 is important to you, we could consider swapping A15 (306) or A56 (280) for C47 (819), but that would require another compensating drop to stay under the limit.'} 
	  } 
	   } 
	   } 
	 & \\ 
 

    \theutterance \stepcounter{utterance}  

    & & & \multicolumn{4}{p{0.6\linewidth}}{\cellcolor[rgb]{1,0.85,0.72}{%
	\makecell[{{p{\linewidth}}}]{% 
	  \tt {\tiny [GM$\langle$B]}  
	 ARGUMENT: {'Thanks for recalculating and adjusting. Your corrected set keeps the spirit of our exchange and stays well under the limit while preserving most of the core items we’ve converged on. I can accept this tighter bundle without C47.'}\\ \tt \\ \tt AGREE: {'A30', 'C73', 'A47', 'A15', 'A75', 'A39', 'A56', 'C24', 'B91', 'C75', 'A19'} 
	  } 
	   } 
	   } 
	  \\ 
 

    \theutterance \stepcounter{utterance}  

    & & & \multicolumn{2}{p{0.3\linewidth}}{\cellcolor[rgb]{0.95,0.95,0.95}{%
	\makecell[{{p{\linewidth}}}]{% 
	  \tt {\tiny [GM$|$GM]}  
	 game successful 
	  } 
	   } 
	   } 
	 & & \\ 
 

    \theutterance \stepcounter{utterance}  

    & & & \multicolumn{2}{p{0.3\linewidth}}{\cellcolor[rgb]{0.95,0.95,0.95}{%
	\makecell[{{p{\linewidth}}}]{% 
	  \tt {\tiny [GM$|$GM]}  
	 end game 
	  } 
	   } 
	   } 
	 & & \\ 
 

\end{supertabular}
}

\end{document}
