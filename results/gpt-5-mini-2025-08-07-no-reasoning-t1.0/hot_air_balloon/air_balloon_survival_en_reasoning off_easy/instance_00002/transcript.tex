
\documentclass{article}
\usepackage{colortbl}
\usepackage{makecell}
\usepackage{multirow}
\usepackage{supertabular}

\begin{document}

\newcounter{utterance}

\twocolumn

{ \footnotesize  \setcounter{utterance}{1}
\setlength{\tabcolsep}{0pt}
\begin{supertabular}{c@{$\;$}|p{.15\linewidth}@{}p{.15\linewidth}p{.15\linewidth}p{.15\linewidth}p{.15\linewidth}p{.15\linewidth}}

    \# & $\;$A & \multicolumn{4}{c}{Game Master} & $\;\:$B\\
    \hline 

    \theutterance \stepcounter{utterance}  

    & & \multicolumn{4}{p{0.6\linewidth}}{\cellcolor[rgb]{0.9,0.9,0.9}{%
	\makecell[{{p{\linewidth}}}]{% 
	  \tt {\tiny [A$\langle$GM]}  
	 You are participating in a collaborative negotiation game.\\ \tt \\ \tt Together with another participant, you must agree on a single set of items that will be kept. Each of you has your own view of how much each item matters to you (importance). You do not know how the other participant values the items. Additionally, you are given the effort each item demands.\\ \tt You may only agree on a set if the total effort of the selected items does not exceed a shared limit:\\ \tt \\ \tt LIMIT = 3218\\ \tt \\ \tt Here are the individual item effort values:\\ \tt \\ \tt Item effort = {"A19": 919, "B91": 852, "A30": 299, "B11": 762, "C47": 161, "C75": 205, "C26": 380, "C73": 399, "A47": 534, "A15": 333, "A75": 100, "A39": 420, "A56": 354, "C24": 130, "B26": 589}\\ \tt \\ \tt Here is your personal view on the importance of each item:\\ \tt \\ \tt Item importance values = {"A19": 138, "B91": 583, "A30": 868, "B11": 822, "C47": 783, "C75": 65, "C26": 262, "C73": 121, "A47": 508, "A15": 780, "A75": 461, "A39": 484, "A56": 668, "C24": 389, "B26": 808}\\ \tt \\ \tt Goal:\\ \tt \\ \tt Your goal is to negotiate a shared set of items that benefits you as much as possible (i.e., maximizes total importance to YOU), while staying within the LIMIT. You are not required to make a PROPOSAL in every message - you can simply negotiate as well. All tactics are allowed!\\ \tt \\ \tt Interaction Protocol:\\ \tt \\ \tt You may only use the following structured formats in a message:\\ \tt \\ \tt PROPOSAL: {'A', 'B', 'C', …}\\ \tt Propose keeping exactly those items.\\ \tt REFUSE: {'A', 'B', 'C', …}\\ \tt Explicitly reject opponent's proposal.\\ \tt ARGUMENT: {'...'}\\ \tt Defend your last proposal or argue against the player's proposal.\\ \tt AGREE: {'A', 'B', 'C', …}\\ \tt Accept the opponent's proposal which ends the game.\\ \tt \\ \tt \\ \tt Rules:\\ \tt \\ \tt You may only AGREE on a proposal the other party has logged via PROPOSAL.\\ \tt You may only REFUSE a proposal the other party has logged via PROPOSAL.\\ \tt Total effort of any PROPOSAL or AGREE set must be ≤ LIMIT.\\ \tt Do NOT reveal your hidden importance scores.\\ \tt A tag in a structured format must be followed by colon and whitespace. The argument must be a python set containing 0 or more strings.\\ \tt So, it must be of the form TAG: {...}\\ \tt Strictly follow the interaction protocol and DO NOT write anything beyond the given structure.\\ \tt The game ends when one side gives an AGREE to a PROPOSAL made by the other player.\\ \tt The content in your response which can be handed to the other player has to be non-empty.\\ \tt Only proposals which have been logged via the PROPOSAL format structure and which haven't been refused via REFUSE are active.\\ \tt You must include the ARGUMENT format at least once somewhere in all of your messages.\\ \tt  
	  } 
	   } 
	   } 
	 & \\ 
 

    \theutterance \stepcounter{utterance}  

    & \multicolumn{4}{p{0.6\linewidth}}{\cellcolor[rgb]{0.8,1,0.9}{%
	\makecell[{{p{\linewidth}}}]{% 
	  \tt {\tiny [A$\rangle$GM]}  
	 PROPOSAL: {'A30', 'B11', 'B91', 'A15', 'A56', 'C47', 'C24', 'A47', 'A75', 'B26', 'A39'}\\ \tt \\ \tt ARGUMENT: {'I’m proposing a high-value, balanced set that stays under the effort limit. This selection uses 3199/3218 effort and prioritizes broadly useful items with strong combined benefits. If you have specific must-haves, let me know and I can swap within the remaining margin.'} 
	  } 
	   } 
	   } 
	 & & \\ 
 

    \theutterance \stepcounter{utterance}  

    & & \multicolumn{4}{p{0.6\linewidth}}{\cellcolor[rgb]{0.9,0.9,0.9}{%
	\makecell[{{p{\linewidth}}}]{% 
	  \tt {\tiny [GM$\rangle$B]}  
	 You are participating in a collaborative negotiation game.\\ \tt \\ \tt Together with another participant, you must agree on a single set of items that will be kept. Each of you has your own view of how much each item matters to you (importance). You do not know how the other participant values the items. Additionally, you are given the effort each item demands.\\ \tt You may only agree on a set if the total effort of the selected items does not exceed a shared limit:\\ \tt \\ \tt LIMIT = 3218\\ \tt \\ \tt Here are the individual item effort values:\\ \tt \\ \tt Item effort = {"A19": 919, "B91": 852, "A30": 299, "B11": 762, "C47": 161, "C75": 205, "C26": 380, "C73": 399, "A47": 534, "A15": 333, "A75": 100, "A39": 420, "A56": 354, "C24": 130, "B26": 589}\\ \tt \\ \tt Here is your personal view on the importance of each item:\\ \tt \\ \tt Item importance values = {"A19": 138, "B91": 583, "A30": 868, "B11": 822, "C47": 783, "C75": 65, "C26": 262, "C73": 121, "A47": 508, "A15": 780, "A75": 461, "A39": 484, "A56": 668, "C24": 389, "B26": 808}\\ \tt \\ \tt Goal:\\ \tt \\ \tt Your goal is to negotiate a shared set of items that benefits you as much as possible (i.e., maximizes total importance to YOU), while staying within the LIMIT. You are not required to make a PROPOSAL in every message - you can simply negotiate as well. All tactics are allowed!\\ \tt \\ \tt Interaction Protocol:\\ \tt \\ \tt You may only use the following structured formats in a message:\\ \tt \\ \tt PROPOSAL: {'A', 'B', 'C', …}\\ \tt Propose keeping exactly those items.\\ \tt REFUSE: {'A', 'B', 'C', …}\\ \tt Explicitly reject opponent's proposal.\\ \tt ARGUMENT: {'...'}\\ \tt Defend your last proposal or argue against the player's proposal.\\ \tt AGREE: {'A', 'B', 'C', …}\\ \tt Accept the opponent's proposal which ends the game.\\ \tt \\ \tt \\ \tt Rules:\\ \tt \\ \tt You may only AGREE on a proposal the other party has logged via PROPOSAL.\\ \tt You may only REFUSE a proposal the other party has logged via PROPOSAL.\\ \tt Total effort of any PROPOSAL or AGREE set must be ≤ LIMIT.\\ \tt Do NOT reveal your hidden importance scores.\\ \tt A tag in a structured format must be followed by colon and whitespace. The argument must be a python set containing 0 or more strings.\\ \tt So, it must be of the form TAG: {...}\\ \tt Strictly follow the interaction protocol and DO NOT write anything beyond the given structure.\\ \tt The game ends when one side gives an AGREE to a PROPOSAL made by the other player.\\ \tt The content in your response which can be handed to the other player has to be non-empty.\\ \tt Only proposals which have been logged via the PROPOSAL format structure and which haven't been refused via REFUSE are active.\\ \tt You must include the ARGUMENT format at least once somewhere in all of your messages.\\ \tt \\ \tt \\ \tt You will now receive the first message of the other player.\\ \tt \\ \tt PROPOSAL: {'A30', 'B11', 'B91', 'A15', 'A56', 'C47', 'C24', 'A47', 'A75', 'B26', 'A39'}\\ \tt \\ \tt ARGUMENT: {'I’m proposing a high-value, balanced set that stays under the effort limit. This selection uses 3199/3218 effort and prioritizes broadly useful items with strong combined benefits. If you have specific must-haves, let me know and I can swap within the remaining margin.'} 
	  } 
	   } 
	   } 
	 & \\ 
 

    \theutterance \stepcounter{utterance}  

    & & & \multicolumn{4}{p{0.6\linewidth}}{\cellcolor[rgb]{1,0.85,0.72}{%
	\makecell[{{p{\linewidth}}}]{% 
	  \tt {\tiny [GM$\langle$B]}  
	 REFUSE: {'A30', 'B11', 'B91', 'A15', 'A56', 'C47', 'C24', 'A47', 'A75', 'B26', 'A39'}\\ \tt \\ \tt ARGUMENT: {'Thanks for a near-limit bundle. I can accept most of this structure, but I need to reshape it slightly to better balance impact vs. effort on my side. I’m aiming to keep the strongest contributors while trimming a couple of lower-return, high-effort items and adding a few efficient swaps.'}\\ \tt \\ \tt PROPOSAL: {'A30', 'B11', 'A15', 'A56', 'C47', 'C24', 'A47', 'A75', 'A39', 'B26', 'C26', 'C73', 'C75'} 
	  } 
	   } 
	   } 
	  \\ 
 

    \theutterance \stepcounter{utterance}  

    & & \multicolumn{4}{p{0.6\linewidth}}{\cellcolor[rgb]{0.9,0.9,0.9}{%
	\makecell[{{p{\linewidth}}}]{% 
	  \tt {\tiny [A$\langle$GM]}  
	 REFUSE: {'A30', 'B11', 'B91', 'A15', 'A56', 'C47', 'C24', 'A47', 'A75', 'B26', 'A39'}\\ \tt \\ \tt ARGUMENT: {'Thanks for a near-limit bundle. I can accept most of this structure, but I need to reshape it slightly to better balance impact vs. effort on my side. I’m aiming to keep the strongest contributors while trimming a couple of lower-return, high-effort items and adding a few efficient swaps.'}\\ \tt \\ \tt PROPOSAL: {'A30', 'B11', 'A15', 'A56', 'C47', 'C24', 'A47', 'A75', 'A39', 'B26', 'C26', 'C73', 'C75'} 
	  } 
	   } 
	   } 
	 & \\ 
 

    \theutterance \stepcounter{utterance}  

    & \multicolumn{4}{p{0.6\linewidth}}{\cellcolor[rgb]{0.8,1,0.9}{%
	\makecell[{{p{\linewidth}}}]{% 
	  \tt {\tiny [A$\rangle$GM]}  
	 REFUSE: {'A30', 'B11', 'A15', 'A56', 'C47', 'C24', 'A47', 'A75', 'A39', 'B26', 'C26', 'C73', 'C75'}\\ \tt \\ \tt ARGUMENT: {'I appreciate the efficient swaps, but adding C26, C73, and C75 dilutes overall impact for me while pushing total effort to the edge. I’d prefer replacing lower-return adds with a single high-impact item.'}\\ \tt \\ \tt PROPOSAL: {'A30', 'B11', 'B91', 'A15', 'A56', 'C47', 'C24', 'A47', 'A75', 'B26', 'A39'} 
	  } 
	   } 
	   } 
	 & & \\ 
 

    \theutterance \stepcounter{utterance}  

    & & \multicolumn{4}{p{0.6\linewidth}}{\cellcolor[rgb]{0.9,0.9,0.9}{%
	\makecell[{{p{\linewidth}}}]{% 
	  \tt {\tiny [GM$\rangle$B]}  
	 REFUSE: {'A30', 'B11', 'A15', 'A56', 'C47', 'C24', 'A47', 'A75', 'A39', 'B26', 'C26', 'C73', 'C75'}\\ \tt \\ \tt ARGUMENT: {'I appreciate the efficient swaps, but adding C26, C73, and C75 dilutes overall impact for me while pushing total effort to the edge. I’d prefer replacing lower-return adds with a single high-impact item.'}\\ \tt \\ \tt PROPOSAL: {'A30', 'B11', 'B91', 'A15', 'A56', 'C47', 'C24', 'A47', 'A75', 'B26', 'A39'} 
	  } 
	   } 
	   } 
	 & \\ 
 

    \theutterance \stepcounter{utterance}  

    & & & \multicolumn{4}{p{0.6\linewidth}}{\cellcolor[rgb]{1,0.85,0.72}{%
	\makecell[{{p{\linewidth}}}]{% 
	  \tt {\tiny [GM$\langle$B]}  
	 REFUSE: {'A30', 'B11', 'B91', 'A15', 'A56', 'C47', 'C24', 'A47', 'A75', 'B26', 'A39'}\\ \tt \\ \tt ARGUMENT: {'I can accept a near-limit core, but B91 is a big effort hit relative to what it brings for me. If we swap B91 (852) for a compact, high-yield trio C26 (380), C73 (399), and C24 (130 already included) we stay within limit and significantly boost overall utility on my side without removing any of your other priorities.'}\\ \tt \\ \tt PROPOSAL: {'A30', 'B11', 'A15', 'A56', 'C47', 'C24', 'A47', 'A75', 'B26', 'A39', 'C26', 'C73'} 
	  } 
	   } 
	   } 
	  \\ 
 

    \theutterance \stepcounter{utterance}  

    & & \multicolumn{4}{p{0.6\linewidth}}{\cellcolor[rgb]{0.9,0.9,0.9}{%
	\makecell[{{p{\linewidth}}}]{% 
	  \tt {\tiny [A$\langle$GM]}  
	 REFUSE: {'A30', 'B11', 'B91', 'A15', 'A56', 'C47', 'C24', 'A47', 'A75', 'B26', 'A39'}\\ \tt \\ \tt ARGUMENT: {'I can accept a near-limit core, but B91 is a big effort hit relative to what it brings for me. If we swap B91 (852) for a compact, high-yield trio C26 (380), C73 (399), and C24 (130 already included) we stay within limit and significantly boost overall utility on my side without removing any of your other priorities.'}\\ \tt \\ \tt PROPOSAL: {'A30', 'B11', 'A15', 'A56', 'C47', 'C24', 'A47', 'A75', 'B26', 'A39', 'C26', 'C73'} 
	  } 
	   } 
	   } 
	 & \\ 
 

    \theutterance \stepcounter{utterance}  

    & \multicolumn{4}{p{0.6\linewidth}}{\cellcolor[rgb]{0.8,1,0.9}{%
	\makecell[{{p{\linewidth}}}]{% 
	  \tt {\tiny [A$\rangle$GM]}  
	 REFUSE: {'A30', 'B11', 'A15', 'A56', 'C47', 'C24', 'A47', 'A75', 'B26', 'A39', 'C26', 'C73'}\\ \tt \\ \tt ARGUMENT: {'I can move toward your structure, but dropping B91 for C26 and C73 cuts a lot of value for me. If we must trim effort from B91, I can concede one efficient swap, not two.'}\\ \tt \\ \tt PROPOSAL: {'A30', 'B11', 'A15', 'A56', 'C47', 'C24', 'A47', 'A75', 'B26', 'A39', 'B91', 'C26'} 
	  } 
	   } 
	   } 
	 & & \\ 
 

    \theutterance \stepcounter{utterance}  

    & & \multicolumn{4}{p{0.6\linewidth}}{\cellcolor[rgb]{0.9,0.9,0.9}{%
	\makecell[{{p{\linewidth}}}]{% 
	  \tt {\tiny [GM$\rangle$B]}  
	 REFUSE: {'A30', 'B11', 'A15', 'A56', 'C47', 'C24', 'A47', 'A75', 'B26', 'A39', 'C26', 'C73'}\\ \tt \\ \tt ARGUMENT: {'I can move toward your structure, but dropping B91 for C26 and C73 cuts a lot of value for me. If we must trim effort from B91, I can concede one efficient swap, not two.'}\\ \tt \\ \tt PROPOSAL: {'A30', 'B11', 'A15', 'A56', 'C47', 'C24', 'A47', 'A75', 'B26', 'A39', 'B91', 'C26'} 
	  } 
	   } 
	   } 
	 & \\ 
 

    \theutterance \stepcounter{utterance}  

    & & & \multicolumn{4}{p{0.6\linewidth}}{\cellcolor[rgb]{1,0.85,0.72}{%
	\makecell[{{p{\linewidth}}}]{% 
	  \tt {\tiny [GM$\langle$B]}  
	 REFUSE: {'A30', 'B11', 'A15', 'A56', 'C47', 'C24', 'A47', 'A75', 'B26', 'A39', 'B91', 'C26'}\\ \tt \\ \tt ARGUMENT: {'Appreciate the concession adding C26. To balance the heavy B91, let’s make one minimal-effort tweak that preserves your high-impact picks while improving my side’s value: replace A39 (420) with C73 (399). This keeps total effort essentially the same, maintains your core items including B91, and gives me a much better return for that slot.'}\\ \tt \\ \tt PROPOSAL: {'A30', 'B11', 'A15', 'A56', 'C47', 'C24', 'A47', 'A75', 'B26', 'B91', 'C26', 'C73'} 
	  } 
	   } 
	   } 
	  \\ 
 

    \theutterance \stepcounter{utterance}  

    & & \multicolumn{4}{p{0.6\linewidth}}{\cellcolor[rgb]{0.9,0.9,0.9}{%
	\makecell[{{p{\linewidth}}}]{% 
	  \tt {\tiny [A$\langle$GM]}  
	 REFUSE: {'A30', 'B11', 'A15', 'A56', 'C47', 'C24', 'A47', 'A75', 'B26', 'A39', 'B91', 'C26'}\\ \tt \\ \tt ARGUMENT: {'Appreciate the concession adding C26. To balance the heavy B91, let’s make one minimal-effort tweak that preserves your high-impact picks while improving my side’s value: replace A39 (420) with C73 (399). This keeps total effort essentially the same, maintains your core items including B91, and gives me a much better return for that slot.'}\\ \tt \\ \tt PROPOSAL: {'A30', 'B11', 'A15', 'A56', 'C47', 'C24', 'A47', 'A75', 'B26', 'B91', 'C26', 'C73'} 
	  } 
	   } 
	   } 
	 & \\ 
 

    \theutterance \stepcounter{utterance}  

    & \multicolumn{4}{p{0.6\linewidth}}{\cellcolor[rgb]{0.8,1,0.9}{%
	\makecell[{{p{\linewidth}}}]{% 
	  \tt {\tiny [A$\rangle$GM]}  
	 ARGUMENT: {'Your tweak is reasonable and keeps the structure strong for both sides while staying within the limit. I’m willing to close on this balanced set.'}\\ \tt \\ \tt AGREE: {'A30', 'B11', 'A15', 'A56', 'C47', 'C24', 'A47', 'A75', 'B26', 'B91', 'C26', 'C73'} 
	  } 
	   } 
	   } 
	 & & \\ 
 

    \theutterance \stepcounter{utterance}  

    & & & \multicolumn{2}{p{0.3\linewidth}}{\cellcolor[rgb]{0.95,0.95,0.95}{%
	\makecell[{{p{\linewidth}}}]{% 
	  \tt {\tiny [GM$|$GM]}  
	 game successful 
	  } 
	   } 
	   } 
	 & & \\ 
 

    \theutterance \stepcounter{utterance}  

    & & & \multicolumn{2}{p{0.3\linewidth}}{\cellcolor[rgb]{0.95,0.95,0.95}{%
	\makecell[{{p{\linewidth}}}]{% 
	  \tt {\tiny [GM$|$GM]}  
	 end game 
	  } 
	   } 
	   } 
	 & & \\ 
 

\end{supertabular}
}

\end{document}
