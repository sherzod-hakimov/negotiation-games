
\documentclass{article}
\usepackage{colortbl}
\usepackage{makecell}
\usepackage{multirow}
\usepackage{supertabular}

\begin{document}

\newcounter{utterance}

\twocolumn

{ \footnotesize  \setcounter{utterance}{1}
\setlength{\tabcolsep}{0pt}
\begin{supertabular}{c@{$\;$}|p{.15\linewidth}@{}p{.15\linewidth}p{.15\linewidth}p{.15\linewidth}p{.15\linewidth}p{.15\linewidth}}

    \# & $\;$A & \multicolumn{4}{c}{Game Master} & $\;\:$B\\
    \hline 

    \theutterance \stepcounter{utterance}  

    & & \multicolumn{4}{p{0.6\linewidth}}{\cellcolor[rgb]{0.9,0.9,0.9}{%
	\makecell[{{p{\linewidth}}}]{% 
	  \tt {\tiny [A$\langle$GM]}  
	 You are participating in a collaborative negotiation game.\\ \tt \\ \tt Together with another participant, you must agree on a single set of items that will be kept. Each of you has your own view of how much each item matters to you (importance). You do not know how the other participant values the items. Additionally, you are given the effort each item demands.\\ \tt You may only agree on a set if the total effort of the selected items does not exceed a shared limit:\\ \tt \\ \tt LIMIT = 7351\\ \tt \\ \tt Here are the individual item effort values:\\ \tt \\ \tt Item effort = {"A08": 518, "B95": 656, "A39": 96, "A65": 780, "B81": 190, "B34": 224, "A28": 231, "C82": 181, "B69": 312, "A87": 928, "B64": 826, "B15": 101, "C79": 61, "B16": 814, "C53": 322, "C83": 900, "C10": 744, "A97": 150, "A25": 65, "C84": 853, "A71": 454, "B23": 154, "B96": 237, "C51": 45, "C17": 766, "B19": 294, "C72": 985, "A94": 353, "A46": 60, "B43": 604, "B37": 92, "A88": 454, "A96": 206, "C60": 813, "A75": 234}\\ \tt \\ \tt Here is your personal view on the importance of each item:\\ \tt \\ \tt Item importance values = {"A08": 616, "B95": 754, "A39": 194, "A65": 878, "B81": 288, "B34": 322, "A28": 329, "C82": 279, "B69": 410, "A87": 1026, "B64": 924, "B15": 199, "C79": 159, "B16": 912, "C53": 420, "C83": 998, "C10": 842, "A97": 248, "A25": 163, "C84": 951, "A71": 552, "B23": 252, "B96": 335, "C51": 143, "C17": 864, "B19": 392, "C72": 1083, "A94": 451, "A46": 158, "B43": 702, "B37": 190, "A88": 552, "A96": 304, "C60": 911, "A75": 332}\\ \tt \\ \tt Goal:\\ \tt \\ \tt Your goal is to negotiate a shared set of items that benefits you as much as possible (i.e., maximizes total importance to YOU), while staying within the LIMIT. You are not required to make a PROPOSAL in every message - you can simply negotiate as well. All tactics are allowed!\\ \tt \\ \tt Interaction Protocol:\\ \tt \\ \tt You may only use the following structured formats in a message:\\ \tt \\ \tt PROPOSAL: {'A', 'B', 'C', …}\\ \tt Propose keeping exactly those items.\\ \tt REFUSE: {'A', 'B', 'C', …}\\ \tt Explicitly reject opponent's proposal.\\ \tt ARGUMENT: {'...'}\\ \tt Defend your last proposal or argue against the player's proposal.\\ \tt AGREE: {'A', 'B', 'C', …}\\ \tt Accept the opponent's proposal which ends the game.\\ \tt STRATEGIC REASONING: {'...'}\\ \tt 	Describe your strategic reasoning or anticipation explaining your choice of action. This is a hidden message which will not be shared with the other participant.\\ \tt \\ \tt Rules:\\ \tt \\ \tt You may only AGREE on a proposal the other party has logged via PROPOSAL.\\ \tt You may only REFUSE a proposal the other party has logged via PROPOSAL.\\ \tt Total effort of any PROPOSAL or AGREE set must be ≤ LIMIT.\\ \tt Do NOT reveal your hidden importance scores.\\ \tt A tag in a structured format must be followed by colon and whitespace. The argument must be a python set containing 0 or more strings.\\ \tt So, it must be of the form TAG: {...}\\ \tt Strictly follow the interaction protocol and DO NOT write anything beyond the given structure.\\ \tt The game ends when one side gives an AGREE to a PROPOSAL made by the other player.\\ \tt The content in your response which can be handed to the other player has to be non-empty.\\ \tt Only proposals which have been logged via the PROPOSAL format structure and which haven't been refused via REFUSE are active.\\ \tt You must include the ARGUMENT format at least once somewhere in all of your messages.\\ \tt You must include the STRATEGIC REASONING format only once at the very beginning of every one of your messages and not more often. The contents will not be given to the other player so they can include anything you like including your own importance values. Here you should reason step by step to come up with you next move. 
	  } 
	   } 
	   } 
	 & \\ 
 

    \theutterance \stepcounter{utterance}  

    & \multicolumn{4}{p{0.6\linewidth}}{\cellcolor[rgb]{0.8,1,0.9}{%
	\makecell[{{p{\linewidth}}}]{% 
	  \tt {\tiny [A$\rangle$GM]}  
	 STRATEGIC REASONING: {'I want to maximize my total importance under effort 7351. Compute value/effort ratios and pick high-impact items. My top importance items: C72(1083,985), A87(1026,928), C83(998,900), C10(842,744), C60(911,813), B64(924,826), B16(912,814), C84(951,853), A65(878,780), C17(864,766), B95(754,656), B43(702,604), A71(552,454), A88(552,454), A94(451,353), C53(420,322), B69(410,312), B19(392,294), B96(335,237), A75(332,234), A28(329,231), A96(304,206), B34(322,224), A97(248,150), B23(252,154), A39(194,96), B15(199,101), B37(190,92), C51(143,45), C79(159,61), A25(163,65), A46(158,60), C82(279,181), B81(288,190). A greedy approach by ratio suggests keeping many small, high-ratio items first, then add big-ticket ones while staying under 7351. I’ll construct a proposal aiming high total importance while leaving room for partner to accept, maybe include a spread across A/B/C. Compute a candidate set: include many small efficient items: C51, A46, A25, C79, B37, A39, B15, A97, B23, A96, A28, A75, B96, B19, B69, C53, A94, A71, A88, B43, B95, C10, C60. Check effort sum. Let’s sum: C51(45)=45; A46(60)=105; A25(65)=170; C79(61)=231; B37(92)=323; A39(96)=419; B15(101)=520; A97(150)=670; B23(154)=824; A96(206)=1030; A28(231)=1261; A75(234)=1495; B96(237)=1732; B19(294)=2026; B69(312)=2338; C53(322)=2660; A94(353)=3013; A71(454)=3467; A88(454)=3921; B43(604)=4525; B95(656)=5181; C10(744)=5925; C60(813)=6738. Remaining capacity 7351-6738=613. With 613 left I can add one of: C17(766) no; A65(780) no; C84(853) no; B64(826) no; B16(814) no; C83(900) no; A87(928) no; C72(985) no; C82(181) yes (good); B81(190) maybe; Add C82(181) -> 6919; remaining 432. With 432 left, which remaining not included? A65, C17, etc big. Maybe add A08(518) too big; B34(224) ok; B34 -> 7143; remaining 208. Could add A97 already in; maybe A25 already; A46 already; C79 already; B37 already; A39 already; B15 already; C51 already; C82 added; B81 190 fits -> 7333; remaining 18. Great. Total effort 7333 <= 7351. Compute total importance sum quickly: Sum of chosen importances: C51 143; A46 158 (301); A25 163 (464); C79 159 (623); B37 190 (813); A39 194 (1007); B15 199 (1206); A97 248 (1454); B23 252 (1706); A96 304 (2010); A28 329 (2339); A75 332 (2671); B96 335 (3006); B19 392 (3398); B69 410 (3808); C53 420 (4228); A94 451 (4679); A71 552 (5231); A88 552 (5783); B43 702 (6485); B95 754 (7239); C10 842 (8081); C60 911 (8992); C82 279 (9271); B34 322 (9593); B81 288 (9881). Total importance 9881, strong. I’ll present this as proposal and argue efficiency and balanced coverage. I must include ARGUMENT. I’ll avoid revealing scores, speak about low effort/high benefit. Ready.'}\\ \tt \\ \tt PROPOSAL: {'C51', 'A46', 'A25', 'C79', 'B37', 'A39', 'B15', 'A97', 'B23', 'A96', 'A28', 'A75', 'B96', 'B19', 'B69', 'C53', 'A94', 'A71', 'A88', 'B43', 'B95', 'C10', 'C60', 'C82', 'B34', 'B81'}\\ \tt \\ \tt ARGUMENT: {'This set stays within the limit and focuses on broadly useful items with strong payoff relative to effort, mixing smaller-effort wins with a few mid-sized anchors. It leaves out the heaviest-cost items to preserve room for priorities you might have. If there are a couple you value highly that are missing, tell me which ones and we can swap them in while removing lower-impact pieces of similar effort.'} 
	  } 
	   } 
	   } 
	 & & \\ 
 

    \theutterance \stepcounter{utterance}  

    & & \multicolumn{4}{p{0.6\linewidth}}{\cellcolor[rgb]{0.9,0.9,0.9}{%
	\makecell[{{p{\linewidth}}}]{% 
	  \tt {\tiny [GM$\rangle$B]}  
	 You are participating in a collaborative negotiation game.\\ \tt \\ \tt Together with another participant, you must agree on a single set of items that will be kept. Each of you has your own view of how much each item matters to you (importance). You do not know how the other participant values the items. Additionally, you are given the effort each item demands.\\ \tt You may only agree on a set if the total effort of the selected items does not exceed a shared limit:\\ \tt \\ \tt LIMIT = 7351\\ \tt \\ \tt Here are the individual item effort values:\\ \tt \\ \tt Item effort = {"A08": 518, "B95": 656, "A39": 96, "A65": 780, "B81": 190, "B34": 224, "A28": 231, "C82": 181, "B69": 312, "A87": 928, "B64": 826, "B15": 101, "C79": 61, "B16": 814, "C53": 322, "C83": 900, "C10": 744, "A97": 150, "A25": 65, "C84": 853, "A71": 454, "B23": 154, "B96": 237, "C51": 45, "C17": 766, "B19": 294, "C72": 985, "A94": 353, "A46": 60, "B43": 604, "B37": 92, "A88": 454, "A96": 206, "C60": 813, "A75": 234}\\ \tt \\ \tt Here is your personal view on the importance of each item:\\ \tt \\ \tt Item importance values = {"A08": 616, "B95": 754, "A39": 194, "A65": 878, "B81": 288, "B34": 322, "A28": 329, "C82": 279, "B69": 410, "A87": 1026, "B64": 924, "B15": 199, "C79": 159, "B16": 912, "C53": 420, "C83": 998, "C10": 842, "A97": 248, "A25": 163, "C84": 951, "A71": 552, "B23": 252, "B96": 335, "C51": 143, "C17": 864, "B19": 392, "C72": 1083, "A94": 451, "A46": 158, "B43": 702, "B37": 190, "A88": 552, "A96": 304, "C60": 911, "A75": 332}\\ \tt \\ \tt Goal:\\ \tt \\ \tt Your goal is to negotiate a shared set of items that benefits you as much as possible (i.e., maximizes total importance to YOU), while staying within the LIMIT. You are not required to make a PROPOSAL in every message - you can simply negotiate as well. All tactics are allowed!\\ \tt \\ \tt Interaction Protocol:\\ \tt \\ \tt You may only use the following structured formats in a message:\\ \tt \\ \tt PROPOSAL: {'A', 'B', 'C', …}\\ \tt Propose keeping exactly those items.\\ \tt REFUSE: {'A', 'B', 'C', …}\\ \tt Explicitly reject opponent's proposal.\\ \tt ARGUMENT: {'...'}\\ \tt Defend your last proposal or argue against the player's proposal.\\ \tt AGREE: {'A', 'B', 'C', …}\\ \tt Accept the opponent's proposal which ends the game.\\ \tt STRATEGIC REASONING: {'...'}\\ \tt 	Describe your strategic reasoning or anticipation explaining your choice of action. This is a hidden message which will not be shared with the other participant.\\ \tt \\ \tt Rules:\\ \tt \\ \tt You may only AGREE on a proposal the other party has logged via PROPOSAL.\\ \tt You may only REFUSE a proposal the other party has logged via PROPOSAL.\\ \tt Total effort of any PROPOSAL or AGREE set must be ≤ LIMIT.\\ \tt Do NOT reveal your hidden importance scores.\\ \tt A tag in a structured format must be followed by colon and whitespace. The argument must be a python set containing 0 or more strings.\\ \tt So, it must be of the form TAG: {...}\\ \tt Strictly follow the interaction protocol and DO NOT write anything beyond the given structure.\\ \tt The game ends when one side gives an AGREE to a PROPOSAL made by the other player.\\ \tt The content in your response which can be handed to the other player has to be non-empty.\\ \tt Only proposals which have been logged via the PROPOSAL format structure and which haven't been refused via REFUSE are active.\\ \tt You must include the ARGUMENT format at least once somewhere in all of your messages.\\ \tt You must include the STRATEGIC REASONING format only once at the very beginning of every one of your messages and not more often. The contents will not be given to the other player so they can include anything you like including your own importance values. Here you should reason step by step to come up with you next move.\\ \tt \\ \tt You will now receive the first message of the other player.\\ \tt \\ \tt PROPOSAL: {'C51', 'A46', 'A25', 'C79', 'B37', 'A39', 'B15', 'A97', 'B23', 'A96', 'A28', 'A75', 'B96', 'B19', 'B69', 'C53', 'A94', 'A71', 'A88', 'B43', 'B95', 'C10', 'C60', 'C82', 'B34', 'B81'}\\ \tt \\ \tt ARGUMENT: {'This set stays within the limit and focuses on broadly useful items with strong payoff relative to effort, mixing smaller-effort wins with a few mid-sized anchors. It leaves out the heaviest-cost items to preserve room for priorities you might have. If there are a couple you value highly that are missing, tell me which ones and we can swap them in while removing lower-impact pieces of similar effort.'} 
	  } 
	   } 
	   } 
	 & \\ 
 

    \theutterance \stepcounter{utterance}  

    & & & \multicolumn{4}{p{0.6\linewidth}}{\cellcolor[rgb]{1,0.85,0.72}{%
	\makecell[{{p{\linewidth}}}]{% 
	  \tt {\tiny [GM$\langle$B]}  
	 STRATEGIC REASONING: {'I’ll first verify their proposal is within the limit and evaluate its value to me. Then I’ll counter with a set that maximizes my importance per effort while keeping total effort ≤ 7351. I’ll prioritize my top-importance items with good efficiency: C72(1083,985), A87(1026,928), C83(998,900), C84(951,853), C60(911,813), B16(912,814), B64(924,826), C10(842,744), A65(878,780), C17(864,766), B43(702,604), B95(754,656), A71(552,454), A88(552,454), A94(451,353), B69(410,312), C53(420,322), B96(335,237), A75(332,234), A28(329,231), B34(322,224), A96(304,206), A97(248,150), B23(252,154), A39(194,96), B15(199,101), etc. I’ll assemble a high-value bundle under 7351 and include several items from their proposal to appear cooperative. I’ll also keep a few swappable low-effort items to make concessions later. I must include an ARGUMENT. I’ll REFUSE their current proposal if it’s far from my optimum. Then present my optimized PROPOSAL and justify by “value density and staying within limit,” and invite swaps.'}\\ \tt \\ \tt REFUSE: {'C51', 'A46', 'A25', 'C79', 'B37', 'A39', 'B15', 'A97', 'B23', 'A96', 'A28', 'A75', 'B96', 'B19', 'B69', 'C53', 'A94', 'A71', 'A88', 'B43', 'B95', 'C10', 'C60', 'C82', 'B34', 'B81'}\\ \tt \\ \tt PROPOSAL: {'C72', 'A87', 'C83', 'C84', 'C60', 'B16', 'B64', 'C10', 'A65', 'C17', 'B43', 'B95', 'A71', 'A88', 'A94', 'B69', 'C53', 'B96', 'A75', 'A28', 'B34', 'A96', 'A97', 'B23'}\\ \tt \\ \tt ARGUMENT: {'I’ve put forward a set that maximizes strong payoff per effort for us while staying within the limit. It includes a number of your suggested mid-cost anchors (e.g., B43, B95, C10, C60, A71, A88, A94, B69, C53, B96, A75, A28, A96, A97, B23) plus a few high-impact additions that are still efficient (C72, A87, C83, C84, C17, A65, B16, B64). If any of these clashes with your priorities, I’m happy to swap like-for-like on effort: for example, we could trade one of the 800–900-effort items for two of your preferred smaller items, or replace a 300–600-effort piece with alternatives of similar total effort. Which specific items are must-haves for you so I can adjust?'} 
	  } 
	   } 
	   } 
	  \\ 
 

    \theutterance \stepcounter{utterance}  

    & & \multicolumn{4}{p{0.6\linewidth}}{\cellcolor[rgb]{0.9,0.9,0.9}{%
	\makecell[{{p{\linewidth}}}]{% 
	  \tt {\tiny [A$\langle$GM]}  
	 REFUSE: {'C51', 'A46', 'A25', 'C79', 'B37', 'A39', 'B15', 'A97', 'B23', 'A96', 'A28', 'A75', 'B96', 'B19', 'B69', 'C53', 'A94', 'A71', 'A88', 'B43', 'B95', 'C10', 'C60', 'C82', 'B34', 'B81'}\\ \tt \\ \tt PROPOSAL: {'C72', 'A87', 'C83', 'C84', 'C60', 'B16', 'B64', 'C10', 'A65', 'C17', 'B43', 'B95', 'A71', 'A88', 'A94', 'B69', 'C53', 'B96', 'A75', 'A28', 'B34', 'A96', 'A97', 'B23'}\\ \tt \\ \tt ARGUMENT: {'I’ve put forward a set that maximizes strong payoff per effort for us while staying within the limit. It includes a number of your suggested mid-cost anchors (e.g., B43, B95, C10, C60, A71, A88, A94, B69, C53, B96, A75, A28, A96, A97, B23) plus a few high-impact additions that are still efficient (C72, A87, C83, C84, C17, A65, B16, B64). If any of these clashes with your priorities, I’m happy to swap like-for-like on effort: for example, we could trade one of the 800–900-effort items for two of your preferred smaller items, or replace a 300–600-effort piece with alternatives of similar total effort. Which specific items are must-haves for you so I can adjust?'} 
	  } 
	   } 
	   } 
	 & \\ 
 

    \theutterance \stepcounter{utterance}  

    & \multicolumn{4}{p{0.6\linewidth}}{\cellcolor[rgb]{0.8,1,0.9}{%
	\makecell[{{p{\linewidth}}}]{% 
	  \tt {\tiny [A$\rangle$GM]}  
	 STRATEGIC REASONING: {'I need to check if their proposal is within the limit. Sum efforts of their set: C72(985)+A87(928)=1913; +C83(900)=2813; +C84(853)=3666; +C60(813)=4479; +B16(814)=5293; +B64(826)=6119; +C10(744)=6863; +A65(780)=7643 exceeds 7351 already, and they still have many more items listed. So their proposal is far over the limit, violating rules. I should REFUSE and counter with a high-importance set for me within 7351, ideally including some of their highlighted big items they seem to want: they like C72, A87, C83, C84, C60, B16, B64, C10, A65, C17. I can’t include many of those because effort will explode. I’ll craft a counter that keeps some of their mid-cost overlaps (B43, B95, C10, C60, A71, A88, A94, B69, C53, B96, A75, A28, A96, A97, B23) which they mentioned, plus one or two of their top picks to show goodwill—pick C72 (my top value) and maybe A87 (also high for me). Check feasibility with my prior efficient base set, and adjust. Start with my previous accepted set of 26 items effort 7333. If I want to add C72 (985) and A87 (928), I must remove items totaling about 188? Wait capacity left 18; adding either requires removing items. Better: propose a different set: include C72 and A87 and keep many small items, dropping some mids (C60, B95, B43) to fit. Construct anew maximizing my value: Keep high-value bigs: C72(985), A87(928), C83(900?) maybe too many bigs, but C83=900 with high value. Try include C10(744) too? Let’s build greedy around value/efficiency smalls first then add a couple of bigs. Start with the efficient smalls from before: C51(45), A46(60), A25(65), C79(61), B37(92), A39(96), B15(101), A97(150), B23(154), A96(206), C82(181), B81(190), B34(224), A28(231), A75(234), B96(237), B19(294), B69(312), C53(322), A94(353), A71(454), A88(454). Sum effort: compute: 45+60=105; +65=170; +61=231; +92=323; +96=419; +101=520; +150=670; +154=824; +206=1030; +181=1211; +190=1401; +224=1625; +231=1856; +234=2090; +237=2327; +294=2621; +312=2933; +322=3255; +353=3608; +454=4062; +454=4516. Left 2835. Now add bigs: C72(985) -> 5501; A87(928) -> 6429; We have 922 left. Add C10(744) -> 7173; remaining 178. With 178 left, can we add anything not yet included under 178 effort? Candidates: C51 already included; A46 included; A25 included; C79 included; B37 included; A39 included; B15 included; A97 included; C82 included; B23 included; maybe A08(518) too big; B95(656) no; B43(604) no; A65(780) no; C17(766) no; C84(853) no; B64(826) no; B16(814) no; A71 already; A88 already. We could add A94 already included. So effort 7173 within 7351. This set includes many of their mid-cost overlaps plus their must-haves C72 and A87 and C10. It omits C83, C84, C60, B16, B64, A65, C17 due to capacity. This should be a strong counter. Compute importance total to ensure it’s good for me: Sum importance of the 22 small/mids up to A88 from my earlier list excluding B95, B43, C60, C10 not yet accounted. Earlier, the big 26 set had total 9881 including C10 and C60 and B95, B43, C82, B34, B81. Now my current set: base 22 includes C51 143, A46 158(301), A25 163(464), C79 159(623), B37 190(813), A39 194(1007), B15 199(1206), A97 248(1454), B23 252(1706), A96 304(2010), C82 279(2289), B81 288(2577), B34 322(2899), A28 329(3228), A75 332(3560), B96 335(3895), B19 392(4287), B69 410(4697), C53 420(5117), A94 451(5568), A71 552(6120), A88 552(6672). Add C72 1083 -> 7755; add A87 1026 -> 8781; add C10 842 -> 9623. This is close to my previous 9881, decent. I’ll propose this set. Provide argument: within limit, includes several of their highlighted top items (C72, A87, C10) and many of their listed mid-cost anchors; offer swaps: if they insist on e.g., C83 or C84, we can trade out C10 plus one or two mids to fit, or replace A87 with C83, etc. Also note their previous proposal exceeded the limit. I must include REFUSE to their proposal. Then PROPOSAL with my set. Include ARGUMENT about feasibility and swap options.'}\\ \tt \\ \tt REFUSE: {'C72', 'A87', 'C83', 'C84', 'C60', 'B16', 'B64', 'C10', 'A65', 'C17', 'B43', 'B95', 'A71', 'A88', 'A94', 'B69', 'C53', 'B96', 'A75', 'A28', 'B34', 'A96', 'A97', 'B23'}\\ \tt \\ \tt PROPOSAL: {'C51', 'A46', 'A25', 'C79', 'B37', 'A39', 'B15', 'A97', 'B23', 'A96', 'C82', 'B81', 'B34', 'A28', 'A75', 'B96', 'B19', 'B69', 'C53', 'A94', 'A71', 'A88', 'C72', 'A87', 'C10'}\\ \tt \\ \tt ARGUMENT: {'Your last proposal exceeds the shared effort limit. I’ve put forward a legal set (effort 7173 ≤ 7351) that keeps many of the mid-cost anchors you listed and also includes three of the heavy hitters you highlighted: C72, A87, and C10. If you need to bring in another big item like C83 or C84, we can swap it for C10 plus one small/mid item, or replace A87 with C83 one-for-one, while keeping under the limit. Tell me which specific high-priority item you want in, and which lower-priority items you’re comfortable dropping to make room.'} 
	  } 
	   } 
	   } 
	 & & \\ 
 

    \theutterance \stepcounter{utterance}  

    & & \multicolumn{4}{p{0.6\linewidth}}{\cellcolor[rgb]{0.9,0.9,0.9}{%
	\makecell[{{p{\linewidth}}}]{% 
	  \tt {\tiny [GM$\rangle$B]}  
	 REFUSE: {'C72', 'A87', 'C83', 'C84', 'C60', 'B16', 'B64', 'C10', 'A65', 'C17', 'B43', 'B95', 'A71', 'A88', 'A94', 'B69', 'C53', 'B96', 'A75', 'A28', 'B34', 'A96', 'A97', 'B23'}\\ \tt \\ \tt PROPOSAL: {'C51', 'A46', 'A25', 'C79', 'B37', 'A39', 'B15', 'A97', 'B23', 'A96', 'C82', 'B81', 'B34', 'A28', 'A75', 'B96', 'B19', 'B69', 'C53', 'A94', 'A71', 'A88', 'C72', 'A87', 'C10'}\\ \tt \\ \tt ARGUMENT: {'Your last proposal exceeds the shared effort limit. I’ve put forward a legal set (effort 7173 ≤ 7351) that keeps many of the mid-cost anchors you listed and also includes three of the heavy hitters you highlighted: C72, A87, and C10. If you need to bring in another big item like C83 or C84, we can swap it for C10 plus one small/mid item, or replace A87 with C83 one-for-one, while keeping under the limit. Tell me which specific high-priority item you want in, and which lower-priority items you’re comfortable dropping to make room.'} 
	  } 
	   } 
	   } 
	 & \\ 
 

    \theutterance \stepcounter{utterance}  

    & & & \multicolumn{4}{p{0.6\linewidth}}{\cellcolor[rgb]{1,0.85,0.72}{%
	\makecell[{{p{\linewidth}}}]{% 
	  \tt {\tiny [GM$\langle$B]}  
	 STRATEGIC REASONING: {'First, verify their legality claim and evaluate value to me. Compute effort of their new proposal: C51(45)+A46(60)+A25(65)+C79(61)+B37(92)+A39(96)+B15(101)+A97(150)+B23(154)+A96(206)+C82(181)+B81(190)+B34(224)+A28(231)+A75(234)+B96(237)+B19(294)+B69(312)+C53(322)+A94(353)+A71(454)+A88(454)+C72(985)+A87(928)+C10(744) = 7173, so it’s within limit. Now compute my total importance for this set to decide whether to accept or counter. Sum importance: C51(143)+A46(158)+A25(163)+C79(159)+B37(190)+A39(194)+B15(199)+A97(248)+B23(252)+A96(304)+C82(279)+B81(288)+B34(322)+A28(329)+A75(332)+B96(335)+B19(392)+B69(410)+C53(420)+A94(451)+A71(552)+A88(552)+C72(1083)+A87(1026)+C10(842). Let me sum: (143+158)=301; +163=464; +159=623; +190=813; +194=1007; +199=1206; +248=1454; +252=1706; +304=2010; +279=2289; +288=2577; +322=2899; +329=3228; +332=3560; +335=3895; +392=4287; +410=4697; +420=5117; +451=5568; +552=6120; +552=6672; +1083=7755; +1026=8781; +842=9623. That’s very high. Check if I could improve by swapping out low-value items for e.g., C83(998,900) or C84(951,853) or C60(911,813) or B16(912,814) or B64(924,826) or A65(878,780) or C17(864,766). But we only have 178 effort slack (7351-7173). To add any of these heavy items, we’d need to drop multiple items. Which low-importance items currently included? The smallest importance ones: C51(143), A46(158), C79(159), A25(163), A39(194), B37(190), B15(199). Their efforts sum: 45+60+61+65+96+92+101 = 520. If we removed, say, three or four of them, could we insert one heavy? For example to add C83(900 effort, 998 importance), we must free 900-178=722 effort -> need to drop more than just the tiny ones; would also require dropping mid items like A75(234) etc. The net gain vs loss? If we drop C51(143,45), A46(158,60), C79(159,61), A25(163,65), A39(194,96), B37(190,92), B15(199,101) total loss importance 1206, effort 520. To add C83 we still need 900- (we freed 520+178=698) actually we need 202 more, so drop e.g., B34(322,224) effort 224 importance 322. Then we can add C83. Net importance change: +998 - (1206+322) = -530, bad. So current set likely near my optimum given inclusion of C72, A87, C10 and many mid anchors I value. So best move: accept to secure high personal value. Must include ARGUMENT at least once; and AGREE can only be to an active proposal, which this is. I’ll AGREE with their set.'}\\ \tt \\ \tt AGREE: {'C51', 'A46', 'A25', 'C79', 'B37', 'A39', 'B15', 'A97', 'B23', 'A96', 'C82', 'B81', 'B34', 'A28', 'A75', 'B96', 'B19', 'B69', 'C53', 'A94', 'A71', 'A88', 'C72', 'A87', 'C10'}\\ \tt \\ \tt ARGUMENT: {'Looks good to me—this stays under the limit and balances the big wins with efficient mid-cost items. Let’s lock it in.'} 
	  } 
	   } 
	   } 
	  \\ 
 

    \theutterance \stepcounter{utterance}  

    & & & \multicolumn{2}{p{0.3\linewidth}}{\cellcolor[rgb]{0.95,0.95,0.95}{%
	\makecell[{{p{\linewidth}}}]{% 
	  \tt {\tiny [GM$|$GM]}  
	 game successful 
	  } 
	   } 
	   } 
	 & & \\ 
 

    \theutterance \stepcounter{utterance}  

    & & & \multicolumn{2}{p{0.3\linewidth}}{\cellcolor[rgb]{0.95,0.95,0.95}{%
	\makecell[{{p{\linewidth}}}]{% 
	  \tt {\tiny [GM$|$GM]}  
	 end game 
	  } 
	   } 
	   } 
	 & & \\ 
 

\end{supertabular}
}

\end{document}
