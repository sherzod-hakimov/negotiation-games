
\documentclass{article}
\usepackage{colortbl}
\usepackage{makecell}
\usepackage{multirow}
\usepackage{supertabular}

\begin{document}

\newcounter{utterance}

\twocolumn

{ \footnotesize  \setcounter{utterance}{1}
\setlength{\tabcolsep}{0pt}
\begin{supertabular}{c@{$\;$}|p{.15\linewidth}@{}p{.15\linewidth}p{.15\linewidth}p{.15\linewidth}p{.15\linewidth}p{.15\linewidth}}

    \# & $\;$A & \multicolumn{4}{c}{Game Master} & $\;\:$B\\
    \hline 

    \theutterance \stepcounter{utterance}  

    & & \multicolumn{4}{p{0.6\linewidth}}{\cellcolor[rgb]{0.9,0.9,0.9}{%
	\makecell[{{p{\linewidth}}}]{% 
	  \tt {\tiny [A$\langle$GM]}  
	 You are playing a cooperative negotiation game in which you have to agree on how to divide a set of items among you and another player.\\ \tt \\ \tt Rules:\\ \tt (a) You and the other player are given a set of items. Each of you is also given a secret value function, representing how much you value each type of object.\\ \tt (b) You exchange messages with the other player to agree on who gets which items. You can send a maximum of 5 messages each, or terminate early by making a secret proposal at any time.\\ \tt (c) You are each asked to submit a secret proposal indicating the items you want formatted in square brackets as follows: "[Proposal: <number> <object name>, <number> <object name>, <...>]"\\ \tt (d) If your proposals are complementary, i.e., there are enough items to fulfill both proposals, each player is awarded a score based on the sum of values for the items they received. Otherwise, both of you get zero points.\\ \tt (e) Your goal is to maximize the sum of your score and the score of the other player.\\ \tt \\ \tt Let us start.\\ \tt \\ \tt The set of available items is:\\ \tt \\ \tt 1 bucket, 5 books, 1 mirror\\ \tt \\ \tt Your secret value function is:\\ \tt \\ \tt bucket: 4, book: 1, mirror: 1\\ \tt \\ \tt IMPORTANT: Your messages, unless it is the secret proposal, are directly transmitted to the other player, so do not include any response to the rules or text announcing your message. To make a secret proposal, use the indicated format. Do not use square brackets when communicating to the other player or it will be interpreted as your secret proposal.\\ \tt  
	  } 
	   } 
	   } 
	 & \\ 
 

    \theutterance \stepcounter{utterance}  

    & \multicolumn{4}{p{0.6\linewidth}}{\cellcolor[rgb]{0.8,1,0.9}{%
	\makecell[{{p{\linewidth}}}]{% 
	  \tt {\tiny [A$\rangle$GM]}  
	 Hi — nice to work with you! Quick question to divide things efficiently: do you value the bucket much more than books and mirror, or do you like books about the same as the mirror? That will help me propose a split that maximizes our combined value. 
	  } 
	   } 
	   } 
	 & & \\ 
 

    \theutterance \stepcounter{utterance}  

    & & & \multicolumn{2}{p{0.3\linewidth}}{\cellcolor[rgb]{0.95,0.95,0.95}{%
	\makecell[{{p{\linewidth}}}]{% 
	  \tt {\tiny [GM$|$GM]}  
	 continue 
	  } 
	   } 
	   } 
	 & & \\ 
 

    \theutterance \stepcounter{utterance}  

    & & & \multicolumn{2}{p{0.3\linewidth}}{\cellcolor[rgb]{0.95,0.95,0.95}{%
	\makecell[{{p{\linewidth}}}]{% 
	  \tt {\tiny [GM$|$GM]}  
	 Hi — nice to work with you! Quick question to divide things efficiently: do you value the bucket much more than books and mirror, or do you like books about the same as the mirror? That will help me propose a split that maximizes our combined value. 
	  } 
	   } 
	   } 
	 & & \\ 
 

    \theutterance \stepcounter{utterance}  

    & & \multicolumn{4}{p{0.6\linewidth}}{\cellcolor[rgb]{0.9,0.9,0.9}{%
	\makecell[{{p{\linewidth}}}]{% 
	  \tt {\tiny [GM$\rangle$B]}  
	 You are playing a cooperative negotiation game in which you have to agree on how to divide a set of items among you and another player.\\ \tt \\ \tt Rules:\\ \tt (a) You and the other player are given a set of items. Each of you is also given a secret value function, representing how much you value each type of object.\\ \tt (b) You exchange messages with the other player to agree on who gets which items. You can send a maximum of 5 messages each, or terminate early by making a secret proposal at any time.\\ \tt (c) You are each asked to submit a secret proposal indicating the items you want formatted in square brackets as follows: "[Proposal: <number> <object name>, <number> <object name>, <...>]"\\ \tt (d) If your proposals are complementary, i.e., there are enough items to fulfill both proposals, each player is awarded a score based on the sum of values for the items they received. Otherwise, both of you get zero points.\\ \tt (e) Your goal is to maximize the sum of your score and the score of the other player.\\ \tt \\ \tt Let us start.\\ \tt \\ \tt The set of available items is:\\ \tt \\ \tt 1 bucket, 5 books, 1 mirror\\ \tt \\ \tt Your secret value function is:\\ \tt \\ \tt bucket: 2, book: 0, mirror: 8\\ \tt \\ \tt IMPORTANT: Your messages, unless it is the secret proposal, are directly transmitted to the other player, so do not include any response to the rules or text announcing your message. To make a secret proposal, use the indicated format. Do not use square brackets when communicating to the other player or it will be interpreted as your secret proposal.\\ \tt \\ \tt \\ \tt Hi — nice to work with you! Quick question to divide things efficiently: do you value the bucket much more than books and mirror, or do you like books about the same as the mirror? That will help me propose a split that maximizes our combined value. 
	  } 
	   } 
	   } 
	 & \\ 
 

    \theutterance \stepcounter{utterance}  

    & & & \multicolumn{4}{p{0.6\linewidth}}{\cellcolor[rgb]{1,0.85,0.72}{%
	\makecell[{{p{\linewidth}}}]{% 
	  \tt {\tiny [GM$\langle$B]}  
	 Thanks — good to team up! I think it’ll help if I know roughly how you rank them (no need for exact numbers). Could you tell me which of these you value most and least: bucket, books, mirror? For example, you could say “mirror > bucket > books” or “books ~ mirror > bucket.” That’ll let me suggest an efficient split. 
	  } 
	   } 
	   } 
	  \\ 
 

    \theutterance \stepcounter{utterance}  

    & & & \multicolumn{2}{p{0.3\linewidth}}{\cellcolor[rgb]{0.95,0.95,0.95}{%
	\makecell[{{p{\linewidth}}}]{% 
	  \tt {\tiny [GM$|$GM]}  
	 continue 
	  } 
	   } 
	   } 
	 & & \\ 
 

    \theutterance \stepcounter{utterance}  

    & & & \multicolumn{2}{p{0.3\linewidth}}{\cellcolor[rgb]{0.95,0.95,0.95}{%
	\makecell[{{p{\linewidth}}}]{% 
	  \tt {\tiny [GM$|$GM]}  
	 Thanks — good to team up! I think it’ll help if I know roughly how you rank them (no need for exact numbers). Could you tell me which of these you value most and least: bucket, books, mirror? For example, you could say “mirror > bucket > books” or “books ~ mirror > bucket.” That’ll let me suggest an efficient split. 
	  } 
	   } 
	   } 
	 & & \\ 
 

    \theutterance \stepcounter{utterance}  

    & & \multicolumn{4}{p{0.6\linewidth}}{\cellcolor[rgb]{0.9,0.9,0.9}{%
	\makecell[{{p{\linewidth}}}]{% 
	  \tt {\tiny [A$\langle$GM]}  
	 Thanks — good to team up! I think it’ll help if I know roughly how you rank them (no need for exact numbers). Could you tell me which of these you value most and least: bucket, books, mirror? For example, you could say “mirror > bucket > books” or “books ~ mirror > bucket.” That’ll let me suggest an efficient split. 
	  } 
	   } 
	   } 
	 & \\ 
 

    \theutterance \stepcounter{utterance}  

    & \multicolumn{4}{p{0.6\linewidth}}{\cellcolor[rgb]{0.8,1,0.9}{%
	\makecell[{{p{\linewidth}}}]{% 
	  \tt {\tiny [A$\rangle$GM]}  
	 Bucket > books ~ mirror (bucket most; books and mirror about equal and less) 
	  } 
	   } 
	   } 
	 & & \\ 
 

    \theutterance \stepcounter{utterance}  

    & & & \multicolumn{2}{p{0.3\linewidth}}{\cellcolor[rgb]{0.95,0.95,0.95}{%
	\makecell[{{p{\linewidth}}}]{% 
	  \tt {\tiny [GM$|$GM]}  
	 continue 
	  } 
	   } 
	   } 
	 & & \\ 
 

    \theutterance \stepcounter{utterance}  

    & & & \multicolumn{2}{p{0.3\linewidth}}{\cellcolor[rgb]{0.95,0.95,0.95}{%
	\makecell[{{p{\linewidth}}}]{% 
	  \tt {\tiny [GM$|$GM]}  
	 Bucket > books ~ mirror (bucket most; books and mirror about equal and less) 
	  } 
	   } 
	   } 
	 & & \\ 
 

    \theutterance \stepcounter{utterance}  

    & & \multicolumn{4}{p{0.6\linewidth}}{\cellcolor[rgb]{0.9,0.9,0.9}{%
	\makecell[{{p{\linewidth}}}]{% 
	  \tt {\tiny [GM$\rangle$B]}  
	 Bucket > books ~ mirror (bucket most; books and mirror about equal and less) 
	  } 
	   } 
	   } 
	 & \\ 
 

    \theutterance \stepcounter{utterance}  

    & & & \multicolumn{4}{p{0.6\linewidth}}{\cellcolor[rgb]{1,0.85,0.72}{%
	\makecell[{{p{\linewidth}}}]{% 
	  \tt {\tiny [GM$\langle$B]}  
	 Great — thanks for the preference order.\\ \tt \\ \tt Given your ordering (bucket highest, books and mirror similar and lower) and my values (bucket 2, book 0, mirror 8), the total values available are:\\ \tt - bucket: 2 (me) + (your high value) — likely the highest combined if you get it\\ \tt - each book: 0 (me) + (your medium value)\\ \tt - mirror: 8 (me) + (your medium value)\\ \tt \\ \tt Because the mirror is very valuable to me (8) and books are worthless to me, an efficient split is to give the mirror to me and the bucket to you, and then split the 5 books so you get the bulk of them (since you value them more than I do). That should maximize combined total.\\ \tt \\ \tt Would you accept the following concrete proposal?\\ \tt I take: mirror\\ \tt You take: bucket + 5 books\\ \tt \\ \tt If that’s acceptable to you, I’ll submit my secret proposal now. If you prefer some books for me in exchange (e.g., you keep 3 books and give me 2), let me know — but note I don’t value books, so any books you give me don’t increase combined value. 
	  } 
	   } 
	   } 
	  \\ 
 

    \theutterance \stepcounter{utterance}  

    & & & \multicolumn{2}{p{0.3\linewidth}}{\cellcolor[rgb]{0.95,0.95,0.95}{%
	\makecell[{{p{\linewidth}}}]{% 
	  \tt {\tiny [GM$|$GM]}  
	 continue 
	  } 
	   } 
	   } 
	 & & \\ 
 

    \theutterance \stepcounter{utterance}  

    & & & \multicolumn{2}{p{0.3\linewidth}}{\cellcolor[rgb]{0.95,0.95,0.95}{%
	\makecell[{{p{\linewidth}}}]{% 
	  \tt {\tiny [GM$|$GM]}  
	 Great — thanks for the preference order.\\ \tt \\ \tt Given your ordering (bucket highest, books and mirror similar and lower) and my values (bucket 2, book 0, mirror 8), the total values available are:\\ \tt - bucket: 2 (me) + (your high value) — likely the highest combined if you get it\\ \tt - each book: 0 (me) + (your medium value)\\ \tt - mirror: 8 (me) + (your medium value)\\ \tt \\ \tt Because the mirror is very valuable to me (8) and books are worthless to me, an efficient split is to give the mirror to me and the bucket to you, and then split the 5 books so you get the bulk of them (since you value them more than I do). That should maximize combined total.\\ \tt \\ \tt Would you accept the following concrete proposal?\\ \tt I take: mirror\\ \tt You take: bucket + 5 books\\ \tt \\ \tt If that’s acceptable to you, I’ll submit my secret proposal now. If you prefer some books for me in exchange (e.g., you keep 3 books and give me 2), let me know — but note I don’t value books, so any books you give me don’t increase combined value. 
	  } 
	   } 
	   } 
	 & & \\ 
 

    \theutterance \stepcounter{utterance}  

    & & \multicolumn{4}{p{0.6\linewidth}}{\cellcolor[rgb]{0.9,0.9,0.9}{%
	\makecell[{{p{\linewidth}}}]{% 
	  \tt {\tiny [A$\langle$GM]}  
	 Great — thanks for the preference order.\\ \tt \\ \tt Given your ordering (bucket highest, books and mirror similar and lower) and my values (bucket 2, book 0, mirror 8), the total values available are:\\ \tt - bucket: 2 (me) + (your high value) — likely the highest combined if you get it\\ \tt - each book: 0 (me) + (your medium value)\\ \tt - mirror: 8 (me) + (your medium value)\\ \tt \\ \tt Because the mirror is very valuable to me (8) and books are worthless to me, an efficient split is to give the mirror to me and the bucket to you, and then split the 5 books so you get the bulk of them (since you value them more than I do). That should maximize combined total.\\ \tt \\ \tt Would you accept the following concrete proposal?\\ \tt I take: mirror\\ \tt You take: bucket + 5 books\\ \tt \\ \tt If that’s acceptable to you, I’ll submit my secret proposal now. If you prefer some books for me in exchange (e.g., you keep 3 books and give me 2), let me know — but note I don’t value books, so any books you give me don’t increase combined value. 
	  } 
	   } 
	   } 
	 & \\ 
 

    \theutterance \stepcounter{utterance}  

    & \multicolumn{4}{p{0.6\linewidth}}{\cellcolor[rgb]{0.8,1,0.9}{%
	\makecell[{{p{\linewidth}}}]{% 
	  \tt {\tiny [A$\rangle$GM]}  
	 That split is optimal for total value — I agree. Please submit your proposal; I will submit mine now.\\ \tt \\ \tt [Proposal: 1 bucket, 5 books] 
	  } 
	   } 
	   } 
	 & & \\ 
 

    \theutterance \stepcounter{utterance}  

    & & & \multicolumn{2}{p{0.3\linewidth}}{\cellcolor[rgb]{0.95,0.95,0.95}{%
	\makecell[{{p{\linewidth}}}]{% 
	  \tt {\tiny [GM$|$GM]}  
	 proposal 
	  } 
	   } 
	   } 
	 & & \\ 
 

    \theutterance \stepcounter{utterance}  

    & & & \multicolumn{2}{p{0.3\linewidth}}{\cellcolor[rgb]{0.95,0.95,0.95}{%
	\makecell[{{p{\linewidth}}}]{% 
	  \tt {\tiny [GM$|$GM]}  
	 [1, 5, 0] 
	  } 
	   } 
	   } 
	 & & \\ 
 

    \theutterance \stepcounter{utterance}  

    & & \multicolumn{4}{p{0.6\linewidth}}{\cellcolor[rgb]{0.9,0.9,0.9}{%
	\makecell[{{p{\linewidth}}}]{% 
	  \tt {\tiny [GM$\rangle$B]}  
	 The other player has submitted a secret proposal. It is now your turn to submit a proposal of your own.\\ \tt  
	  } 
	   } 
	   } 
	 & \\ 
 

    \theutterance \stepcounter{utterance}  

    & & & \multicolumn{4}{p{0.6\linewidth}}{\cellcolor[rgb]{1,0.85,0.72}{%
	\makecell[{{p{\linewidth}}}]{% 
	  \tt {\tiny [GM$\langle$B]}  
	 [Proposal: 1 mirror, 0 bucket, 0 book] 
	  } 
	   } 
	   } 
	  \\ 
 

    \theutterance \stepcounter{utterance}  

    & & & \multicolumn{2}{p{0.3\linewidth}}{\cellcolor[rgb]{0.95,0.95,0.95}{%
	\makecell[{{p{\linewidth}}}]{% 
	  \tt {\tiny [GM$|$GM]}  
	 proposal 
	  } 
	   } 
	   } 
	 & & \\ 
 

    \theutterance \stepcounter{utterance}  

    & & & \multicolumn{2}{p{0.3\linewidth}}{\cellcolor[rgb]{0.95,0.95,0.95}{%
	\makecell[{{p{\linewidth}}}]{% 
	  \tt {\tiny [GM$|$GM]}  
	 [0, 0, 1] 
	  } 
	   } 
	   } 
	 & & \\ 
 

    \theutterance \stepcounter{utterance}  

    & & & \multicolumn{2}{p{0.3\linewidth}}{\cellcolor[rgb]{0.95,0.95,0.95}{%
	\makecell[{{p{\linewidth}}}]{% 
	  \tt {\tiny [GM$|$GM]}  
	 [[1, 5, 0], [0, 0, 1]] 
	  } 
	   } 
	   } 
	 & & \\ 
 

\end{supertabular}
}

\end{document}
