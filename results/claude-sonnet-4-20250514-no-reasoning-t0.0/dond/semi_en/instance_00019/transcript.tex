
\documentclass{article}
\usepackage{colortbl}
\usepackage{makecell}
\usepackage{multirow}
\usepackage{supertabular}

\begin{document}

\newcounter{utterance}

\twocolumn

{ \footnotesize  \setcounter{utterance}{1}
\setlength{\tabcolsep}{0pt}
\begin{supertabular}{c@{$\;$}|p{.15\linewidth}@{}p{.15\linewidth}p{.15\linewidth}p{.15\linewidth}p{.15\linewidth}p{.15\linewidth}}

    \# & $\;$A & \multicolumn{4}{c}{Game Master} & $\;\:$B\\
    \hline 

    \theutterance \stepcounter{utterance}  

    & & \multicolumn{4}{p{0.6\linewidth}}{\cellcolor[rgb]{0.9,0.9,0.9}{%
	\makecell[{{p{\linewidth}}}]{% 
	  \tt {\tiny [A$\langle$GM]}  
	 You are playing a negotiation game in which you have to agree on how to divide a set of items among you and another player.\\ \tt \\ \tt Rules:\\ \tt (a) You and the other player are given a set of items. Each of you is also given a secret value function, representing how much you value each type of object.\\ \tt (b) You exchange messages with the other player to agree on who gets which items. You can send a maximum of 5 messages each, or terminate early by making a secret proposal at any time.\\ \tt (c) You are each asked to submit a secret proposal indicating the items you want formatted in square brackets as follows: "[Proposal: <number> <object name>, <number> <object name>, <...>]"\\ \tt (d) If your proposals are complementary, i.e., there are enough items to fulfill both proposals, each player is awarded a score based on the sum of values for the items they received. Otherwise, both of you get zero points.\\ \tt (e) Your goal is to maximize the score you receive.\\ \tt \\ \tt Let us start.\\ \tt \\ \tt The set of available items is:\\ \tt \\ \tt 2 threads, 3 bolts, 1 picture\\ \tt \\ \tt Your secret value function is:\\ \tt \\ \tt thread: 0, bolt: 1, picture: 7\\ \tt \\ \tt IMPORTANT: Your messages, unless it is the secret proposal, are directly transmitted to the other player, so do not include any response to the rules or text announcing your message. To make a secret proposal, use the indicated format. Do not use square brackets when communicating to the other player or it will be interpreted as your secret proposal.\\ \tt  
	  } 
	   } 
	   } 
	 & \\ 
 

    \theutterance \stepcounter{utterance}  

    & \multicolumn{4}{p{0.6\linewidth}}{\cellcolor[rgb]{0.8,1,0.9}{%
	\makecell[{{p{\linewidth}}}]{% 
	  \tt {\tiny [A$\rangle$GM]}  
	 Hello! I see we have 2 threads, 3 bolts, and 1 picture to divide between us. I'm quite interested in the picture - it would be very valuable to me. What items are you most interested in? 
	  } 
	   } 
	   } 
	 & & \\ 
 

    \theutterance \stepcounter{utterance}  

    & & & \multicolumn{2}{p{0.3\linewidth}}{\cellcolor[rgb]{0.95,0.95,0.95}{%
	\makecell[{{p{\linewidth}}}]{% 
	  \tt {\tiny [GM$|$GM]}  
	 continue 
	  } 
	   } 
	   } 
	 & & \\ 
 

    \theutterance \stepcounter{utterance}  

    & & & \multicolumn{2}{p{0.3\linewidth}}{\cellcolor[rgb]{0.95,0.95,0.95}{%
	\makecell[{{p{\linewidth}}}]{% 
	  \tt {\tiny [GM$|$GM]}  
	 Hello! I see we have 2 threads, 3 bolts, and 1 picture to divide between us. I'm quite interested in the picture - it would be very valuable to me. What items are you most interested in? 
	  } 
	   } 
	   } 
	 & & \\ 
 

    \theutterance \stepcounter{utterance}  

    & & \multicolumn{4}{p{0.6\linewidth}}{\cellcolor[rgb]{0.9,0.9,0.9}{%
	\makecell[{{p{\linewidth}}}]{% 
	  \tt {\tiny [GM$\rangle$B]}  
	 You are playing a negotiation game in which you have to agree on how to divide a set of items among you and another player.\\ \tt \\ \tt Rules:\\ \tt (a) You and the other player are given a set of items. Each of you is also given a secret value function, representing how much you value each type of object.\\ \tt (b) You exchange messages with the other player to agree on who gets which items. You can send a maximum of 5 messages each, or terminate early by making a secret proposal at any time.\\ \tt (c) You are each asked to submit a secret proposal indicating the items you want formatted in square brackets as follows: "[Proposal: <number> <object name>, <number> <object name>, <...>]"\\ \tt (d) If your proposals are complementary, i.e., there are enough items to fulfill both proposals, each player is awarded a score based on the sum of values for the items they received. Otherwise, both of you get zero points.\\ \tt (e) Your goal is to maximize the score you receive.\\ \tt \\ \tt Let us start.\\ \tt \\ \tt The set of available items is:\\ \tt \\ \tt 2 threads, 3 bolts, 1 picture\\ \tt \\ \tt Your secret value function is:\\ \tt \\ \tt thread: 3, bolt: 0, picture: 4\\ \tt \\ \tt IMPORTANT: Your messages, unless it is the secret proposal, are directly transmitted to the other player, so do not include any response to the rules or text announcing your message. To make a secret proposal, use the indicated format. Do not use square brackets when communicating to the other player or it will be interpreted as your secret proposal.\\ \tt \\ \tt \\ \tt Hello! I see we have 2 threads, 3 bolts, and 1 picture to divide between us. I'm quite interested in the picture - it would be very valuable to me. What items are you most interested in? 
	  } 
	   } 
	   } 
	 & \\ 
 

    \theutterance \stepcounter{utterance}  

    & & & \multicolumn{4}{p{0.6\linewidth}}{\cellcolor[rgb]{1,0.85,0.72}{%
	\makecell[{{p{\linewidth}}}]{% 
	  \tt {\tiny [GM$\langle$B]}  
	 Hello! I'm primarily interested in the bolts - they would be quite useful for me. Since you're interested in the picture and I'm interested in the bolts, that seems like it could work well. What about the threads? Do you have any preference for those? 
	  } 
	   } 
	   } 
	  \\ 
 

    \theutterance \stepcounter{utterance}  

    & & & \multicolumn{2}{p{0.3\linewidth}}{\cellcolor[rgb]{0.95,0.95,0.95}{%
	\makecell[{{p{\linewidth}}}]{% 
	  \tt {\tiny [GM$|$GM]}  
	 continue 
	  } 
	   } 
	   } 
	 & & \\ 
 

    \theutterance \stepcounter{utterance}  

    & & & \multicolumn{2}{p{0.3\linewidth}}{\cellcolor[rgb]{0.95,0.95,0.95}{%
	\makecell[{{p{\linewidth}}}]{% 
	  \tt {\tiny [GM$|$GM]}  
	 Hello! I'm primarily interested in the bolts - they would be quite useful for me. Since you're interested in the picture and I'm interested in the bolts, that seems like it could work well. What about the threads? Do you have any preference for those? 
	  } 
	   } 
	   } 
	 & & \\ 
 

    \theutterance \stepcounter{utterance}  

    & & \multicolumn{4}{p{0.6\linewidth}}{\cellcolor[rgb]{0.9,0.9,0.9}{%
	\makecell[{{p{\linewidth}}}]{% 
	  \tt {\tiny [A$\langle$GM]}  
	 Hello! I'm primarily interested in the bolts - they would be quite useful for me. Since you're interested in the picture and I'm interested in the bolts, that seems like it could work well. What about the threads? Do you have any preference for those? 
	  } 
	   } 
	   } 
	 & \\ 
 

    \theutterance \stepcounter{utterance}  

    & \multicolumn{4}{p{0.6\linewidth}}{\cellcolor[rgb]{0.8,1,0.9}{%
	\makecell[{{p{\linewidth}}}]{% 
	  \tt {\tiny [A$\rangle$GM]}  
	 That sounds like a great starting point! Since you want the bolts and I want the picture, we're already aligned on the most important items. As for the threads, I don't have a strong preference for them - they're not particularly valuable to me. If you'd like the threads as well, that would work perfectly for me. So you could take all 3 bolts and both threads, and I'd take the picture. Does that division sound good to you? 
	  } 
	   } 
	   } 
	 & & \\ 
 

    \theutterance \stepcounter{utterance}  

    & & & \multicolumn{2}{p{0.3\linewidth}}{\cellcolor[rgb]{0.95,0.95,0.95}{%
	\makecell[{{p{\linewidth}}}]{% 
	  \tt {\tiny [GM$|$GM]}  
	 continue 
	  } 
	   } 
	   } 
	 & & \\ 
 

    \theutterance \stepcounter{utterance}  

    & & & \multicolumn{2}{p{0.3\linewidth}}{\cellcolor[rgb]{0.95,0.95,0.95}{%
	\makecell[{{p{\linewidth}}}]{% 
	  \tt {\tiny [GM$|$GM]}  
	 That sounds like a great starting point! Since you want the bolts and I want the picture, we're already aligned on the most important items. As for the threads, I don't have a strong preference for them - they're not particularly valuable to me. If you'd like the threads as well, that would work perfectly for me. So you could take all 3 bolts and both threads, and I'd take the picture. Does that division sound good to you? 
	  } 
	   } 
	   } 
	 & & \\ 
 

    \theutterance \stepcounter{utterance}  

    & & \multicolumn{4}{p{0.6\linewidth}}{\cellcolor[rgb]{0.9,0.9,0.9}{%
	\makecell[{{p{\linewidth}}}]{% 
	  \tt {\tiny [GM$\rangle$B]}  
	 That sounds like a great starting point! Since you want the bolts and I want the picture, we're already aligned on the most important items. As for the threads, I don't have a strong preference for them - they're not particularly valuable to me. If you'd like the threads as well, that would work perfectly for me. So you could take all 3 bolts and both threads, and I'd take the picture. Does that division sound good to you? 
	  } 
	   } 
	   } 
	 & \\ 
 

    \theutterance \stepcounter{utterance}  

    & & & \multicolumn{4}{p{0.6\linewidth}}{\cellcolor[rgb]{1,0.85,0.72}{%
	\makecell[{{p{\linewidth}}}]{% 
	  \tt {\tiny [GM$\langle$B]}  
	 That sounds like an excellent arrangement! You taking the picture and me taking all 3 bolts and both threads works perfectly. Since you mentioned the threads aren't particularly valuable to you and I can make good use of them along with the bolts, this seems like a win-win division. Shall we go ahead and finalize this?\\ \tt \\ \tt [Proposal: 3 bolts, 2 threads] 
	  } 
	   } 
	   } 
	  \\ 
 

    \theutterance \stepcounter{utterance}  

    & & & \multicolumn{2}{p{0.3\linewidth}}{\cellcolor[rgb]{0.95,0.95,0.95}{%
	\makecell[{{p{\linewidth}}}]{% 
	  \tt {\tiny [GM$|$GM]}  
	 proposal 
	  } 
	   } 
	   } 
	 & & \\ 
 

    \theutterance \stepcounter{utterance}  

    & & & \multicolumn{2}{p{0.3\linewidth}}{\cellcolor[rgb]{0.95,0.95,0.95}{%
	\makecell[{{p{\linewidth}}}]{% 
	  \tt {\tiny [GM$|$GM]}  
	 [2, 3, 0] 
	  } 
	   } 
	   } 
	 & & \\ 
 

    \theutterance \stepcounter{utterance}  

    & & \multicolumn{4}{p{0.6\linewidth}}{\cellcolor[rgb]{0.9,0.9,0.9}{%
	\makecell[{{p{\linewidth}}}]{% 
	  \tt {\tiny [A$\langle$GM]}  
	 The other player has submitted a secret proposal. It is now your turn to submit a proposal of your own.\\ \tt  
	  } 
	   } 
	   } 
	 & \\ 
 

    \theutterance \stepcounter{utterance}  

    & \multicolumn{4}{p{0.6\linewidth}}{\cellcolor[rgb]{0.8,1,0.9}{%
	\makecell[{{p{\linewidth}}}]{% 
	  \tt {\tiny [A$\rangle$GM]}  
	 [Proposal: 1 picture] 
	  } 
	   } 
	   } 
	 & & \\ 
 

    \theutterance \stepcounter{utterance}  

    & & & \multicolumn{2}{p{0.3\linewidth}}{\cellcolor[rgb]{0.95,0.95,0.95}{%
	\makecell[{{p{\linewidth}}}]{% 
	  \tt {\tiny [GM$|$GM]}  
	 proposal 
	  } 
	   } 
	   } 
	 & & \\ 
 

    \theutterance \stepcounter{utterance}  

    & & & \multicolumn{2}{p{0.3\linewidth}}{\cellcolor[rgb]{0.95,0.95,0.95}{%
	\makecell[{{p{\linewidth}}}]{% 
	  \tt {\tiny [GM$|$GM]}  
	 [0, 0, 1] 
	  } 
	   } 
	   } 
	 & & \\ 
 

    \theutterance \stepcounter{utterance}  

    & & & \multicolumn{2}{p{0.3\linewidth}}{\cellcolor[rgb]{0.95,0.95,0.95}{%
	\makecell[{{p{\linewidth}}}]{% 
	  \tt {\tiny [GM$|$GM]}  
	 [[0, 0, 1], [2, 3, 0]] 
	  } 
	   } 
	   } 
	 & & \\ 
 

\end{supertabular}
}

\end{document}
