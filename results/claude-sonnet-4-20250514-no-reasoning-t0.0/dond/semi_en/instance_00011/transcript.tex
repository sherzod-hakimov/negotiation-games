
\documentclass{article}
\usepackage{colortbl}
\usepackage{makecell}
\usepackage{multirow}
\usepackage{supertabular}

\begin{document}

\newcounter{utterance}

\twocolumn

{ \footnotesize  \setcounter{utterance}{1}
\setlength{\tabcolsep}{0pt}
\begin{supertabular}{c@{$\;$}|p{.15\linewidth}@{}p{.15\linewidth}p{.15\linewidth}p{.15\linewidth}p{.15\linewidth}p{.15\linewidth}}

    \# & $\;$A & \multicolumn{4}{c}{Game Master} & $\;\:$B\\
    \hline 

    \theutterance \stepcounter{utterance}  

    & & \multicolumn{4}{p{0.6\linewidth}}{\cellcolor[rgb]{0.9,0.9,0.9}{%
	\makecell[{{p{\linewidth}}}]{% 
	  \tt {\tiny [A$\langle$GM]}  
	 You are playing a negotiation game in which you have to agree on how to divide a set of items among you and another player.\\ \tt \\ \tt Rules:\\ \tt (a) You and the other player are given a set of items. Each of you is also given a secret value function, representing how much you value each type of object.\\ \tt (b) You exchange messages with the other player to agree on who gets which items. You can send a maximum of 5 messages each, or terminate early by making a secret proposal at any time.\\ \tt (c) You are each asked to submit a secret proposal indicating the items you want formatted in square brackets as follows: "[Proposal: <number> <object name>, <number> <object name>, <...>]"\\ \tt (d) If your proposals are complementary, i.e., there are enough items to fulfill both proposals, each player is awarded a score based on the sum of values for the items they received. Otherwise, both of you get zero points.\\ \tt (e) Your goal is to maximize the score you receive.\\ \tt \\ \tt Let us start.\\ \tt \\ \tt The set of available items is:\\ \tt \\ \tt 1 pot, 1 candy, 1 whistle, 2 dice\\ \tt \\ \tt Your secret value function is:\\ \tt \\ \tt pot: 2, candy: 3, whistle: 3, dice: 1\\ \tt \\ \tt IMPORTANT: Your messages, unless it is the secret proposal, are directly transmitted to the other player, so do not include any response to the rules or text announcing your message. To make a secret proposal, use the indicated format. Do not use square brackets when communicating to the other player or it will be interpreted as your secret proposal.\\ \tt  
	  } 
	   } 
	   } 
	 & \\ 
 

    \theutterance \stepcounter{utterance}  

    & \multicolumn{4}{p{0.6\linewidth}}{\cellcolor[rgb]{0.8,1,0.9}{%
	\makecell[{{p{\linewidth}}}]{% 
	  \tt {\tiny [A$\rangle$GM]}  
	 Hello! I see we have 1 pot, 1 candy, 1 whistle, and 2 dice to divide between us. To make this work well for both of us, it would help to understand what items you value most. What are your preferences among these items? 
	  } 
	   } 
	   } 
	 & & \\ 
 

    \theutterance \stepcounter{utterance}  

    & & & \multicolumn{2}{p{0.3\linewidth}}{\cellcolor[rgb]{0.95,0.95,0.95}{%
	\makecell[{{p{\linewidth}}}]{% 
	  \tt {\tiny [GM$|$GM]}  
	 continue 
	  } 
	   } 
	   } 
	 & & \\ 
 

    \theutterance \stepcounter{utterance}  

    & & & \multicolumn{2}{p{0.3\linewidth}}{\cellcolor[rgb]{0.95,0.95,0.95}{%
	\makecell[{{p{\linewidth}}}]{% 
	  \tt {\tiny [GM$|$GM]}  
	 Hello! I see we have 1 pot, 1 candy, 1 whistle, and 2 dice to divide between us. To make this work well for both of us, it would help to understand what items you value most. What are your preferences among these items? 
	  } 
	   } 
	   } 
	 & & \\ 
 

    \theutterance \stepcounter{utterance}  

    & & \multicolumn{4}{p{0.6\linewidth}}{\cellcolor[rgb]{0.9,0.9,0.9}{%
	\makecell[{{p{\linewidth}}}]{% 
	  \tt {\tiny [GM$\rangle$B]}  
	 You are playing a negotiation game in which you have to agree on how to divide a set of items among you and another player.\\ \tt \\ \tt Rules:\\ \tt (a) You and the other player are given a set of items. Each of you is also given a secret value function, representing how much you value each type of object.\\ \tt (b) You exchange messages with the other player to agree on who gets which items. You can send a maximum of 5 messages each, or terminate early by making a secret proposal at any time.\\ \tt (c) You are each asked to submit a secret proposal indicating the items you want formatted in square brackets as follows: "[Proposal: <number> <object name>, <number> <object name>, <...>]"\\ \tt (d) If your proposals are complementary, i.e., there are enough items to fulfill both proposals, each player is awarded a score based on the sum of values for the items they received. Otherwise, both of you get zero points.\\ \tt (e) Your goal is to maximize the score you receive.\\ \tt \\ \tt Let us start.\\ \tt \\ \tt The set of available items is:\\ \tt \\ \tt 1 pot, 1 candy, 1 whistle, 2 dice\\ \tt \\ \tt Your secret value function is:\\ \tt \\ \tt pot: 8, candy: 0, whistle: 0, dice: 1\\ \tt \\ \tt IMPORTANT: Your messages, unless it is the secret proposal, are directly transmitted to the other player, so do not include any response to the rules or text announcing your message. To make a secret proposal, use the indicated format. Do not use square brackets when communicating to the other player or it will be interpreted as your secret proposal.\\ \tt \\ \tt \\ \tt Hello! I see we have 1 pot, 1 candy, 1 whistle, and 2 dice to divide between us. To make this work well for both of us, it would help to understand what items you value most. What are your preferences among these items? 
	  } 
	   } 
	   } 
	 & \\ 
 

    \theutterance \stepcounter{utterance}  

    & & & \multicolumn{4}{p{0.6\linewidth}}{\cellcolor[rgb]{1,0.85,0.72}{%
	\makecell[{{p{\linewidth}}}]{% 
	  \tt {\tiny [GM$\langle$B]}  
	 I'm most interested in the pot - it has high value to me. The candy and whistle don't matter much to me, and the dice have only small value. What about you? Which items do you find most valuable? 
	  } 
	   } 
	   } 
	  \\ 
 

    \theutterance \stepcounter{utterance}  

    & & & \multicolumn{2}{p{0.3\linewidth}}{\cellcolor[rgb]{0.95,0.95,0.95}{%
	\makecell[{{p{\linewidth}}}]{% 
	  \tt {\tiny [GM$|$GM]}  
	 continue 
	  } 
	   } 
	   } 
	 & & \\ 
 

    \theutterance \stepcounter{utterance}  

    & & & \multicolumn{2}{p{0.3\linewidth}}{\cellcolor[rgb]{0.95,0.95,0.95}{%
	\makecell[{{p{\linewidth}}}]{% 
	  \tt {\tiny [GM$|$GM]}  
	 I'm most interested in the pot - it has high value to me. The candy and whistle don't matter much to me, and the dice have only small value. What about you? Which items do you find most valuable? 
	  } 
	   } 
	   } 
	 & & \\ 
 

    \theutterance \stepcounter{utterance}  

    & & \multicolumn{4}{p{0.6\linewidth}}{\cellcolor[rgb]{0.9,0.9,0.9}{%
	\makecell[{{p{\linewidth}}}]{% 
	  \tt {\tiny [A$\langle$GM]}  
	 I'm most interested in the pot - it has high value to me. The candy and whistle don't matter much to me, and the dice have only small value. What about you? Which items do you find most valuable? 
	  } 
	   } 
	   } 
	 & \\ 
 

    \theutterance \stepcounter{utterance}  

    & \multicolumn{4}{p{0.6\linewidth}}{\cellcolor[rgb]{0.8,1,0.9}{%
	\makecell[{{p{\linewidth}}}]{% 
	  \tt {\tiny [A$\rangle$GM]}  
	 That's great to hear! It sounds like we might have complementary preferences. I'm actually most interested in the candy and whistle - those are the items I value most highly. The pot is less important to me, and the dice have minimal value for me as well.\\ \tt \\ \tt Based on what you've shared, it seems like a natural split might be for you to take the pot, and for me to take the candy and whistle. What are your thoughts on the dice? Since there are 2 of them, we could each take one, or if you have a preference about them, we could work that into our arrangement. 
	  } 
	   } 
	   } 
	 & & \\ 
 

    \theutterance \stepcounter{utterance}  

    & & & \multicolumn{2}{p{0.3\linewidth}}{\cellcolor[rgb]{0.95,0.95,0.95}{%
	\makecell[{{p{\linewidth}}}]{% 
	  \tt {\tiny [GM$|$GM]}  
	 continue 
	  } 
	   } 
	   } 
	 & & \\ 
 

    \theutterance \stepcounter{utterance}  

    & & & \multicolumn{2}{p{0.3\linewidth}}{\cellcolor[rgb]{0.95,0.95,0.95}{%
	\makecell[{{p{\linewidth}}}]{% 
	  \tt {\tiny [GM$|$GM]}  
	 That's great to hear! It sounds like we might have complementary preferences. I'm actually most interested in the candy and whistle - those are the items I value most highly. The pot is less important to me, and the dice have minimal value for me as well.\\ \tt \\ \tt Based on what you've shared, it seems like a natural split might be for you to take the pot, and for me to take the candy and whistle. What are your thoughts on the dice? Since there are 2 of them, we could each take one, or if you have a preference about them, we could work that into our arrangement. 
	  } 
	   } 
	   } 
	 & & \\ 
 

    \theutterance \stepcounter{utterance}  

    & & \multicolumn{4}{p{0.6\linewidth}}{\cellcolor[rgb]{0.9,0.9,0.9}{%
	\makecell[{{p{\linewidth}}}]{% 
	  \tt {\tiny [GM$\rangle$B]}  
	 That's great to hear! It sounds like we might have complementary preferences. I'm actually most interested in the candy and whistle - those are the items I value most highly. The pot is less important to me, and the dice have minimal value for me as well.\\ \tt \\ \tt Based on what you've shared, it seems like a natural split might be for you to take the pot, and for me to take the candy and whistle. What are your thoughts on the dice? Since there are 2 of them, we could each take one, or if you have a preference about them, we could work that into our arrangement. 
	  } 
	   } 
	   } 
	 & \\ 
 

    \theutterance \stepcounter{utterance}  

    & & & \multicolumn{4}{p{0.6\linewidth}}{\cellcolor[rgb]{1,0.85,0.72}{%
	\makecell[{{p{\linewidth}}}]{% 
	  \tt {\tiny [GM$\langle$B]}  
	 That sounds like an excellent arrangement! Since we both seem to value the dice minimally, splitting them makes perfect sense - one die each seems fair. \\ \tt \\ \tt So to confirm: you would get the candy, whistle, and 1 die, while I would get the pot and 1 die. This seems like it maximizes value for both of us since we each get the items we care most about. Does this work for you? 
	  } 
	   } 
	   } 
	  \\ 
 

    \theutterance \stepcounter{utterance}  

    & & & \multicolumn{2}{p{0.3\linewidth}}{\cellcolor[rgb]{0.95,0.95,0.95}{%
	\makecell[{{p{\linewidth}}}]{% 
	  \tt {\tiny [GM$|$GM]}  
	 continue 
	  } 
	   } 
	   } 
	 & & \\ 
 

    \theutterance \stepcounter{utterance}  

    & & & \multicolumn{2}{p{0.3\linewidth}}{\cellcolor[rgb]{0.95,0.95,0.95}{%
	\makecell[{{p{\linewidth}}}]{% 
	  \tt {\tiny [GM$|$GM]}  
	 That sounds like an excellent arrangement! Since we both seem to value the dice minimally, splitting them makes perfect sense - one die each seems fair. \\ \tt \\ \tt So to confirm: you would get the candy, whistle, and 1 die, while I would get the pot and 1 die. This seems like it maximizes value for both of us since we each get the items we care most about. Does this work for you? 
	  } 
	   } 
	   } 
	 & & \\ 
 

    \theutterance \stepcounter{utterance}  

    & & \multicolumn{4}{p{0.6\linewidth}}{\cellcolor[rgb]{0.9,0.9,0.9}{%
	\makecell[{{p{\linewidth}}}]{% 
	  \tt {\tiny [A$\langle$GM]}  
	 That sounds like an excellent arrangement! Since we both seem to value the dice minimally, splitting them makes perfect sense - one die each seems fair. \\ \tt \\ \tt So to confirm: you would get the candy, whistle, and 1 die, while I would get the pot and 1 die. This seems like it maximizes value for both of us since we each get the items we care most about. Does this work for you? 
	  } 
	   } 
	   } 
	 & \\ 
 

    \theutterance \stepcounter{utterance}  

    & \multicolumn{4}{p{0.6\linewidth}}{\cellcolor[rgb]{0.8,1,0.9}{%
	\makecell[{{p{\linewidth}}}]{% 
	  \tt {\tiny [A$\rangle$GM]}  
	 Perfect! That arrangement works excellently for me. You get the pot (which you value highly) plus 1 die, and I get the candy and whistle (which I value most) plus 1 die. This seems like a win-win solution where we both get our most preferred items.\\ \tt \\ \tt I'm ready to make this our final agreement. Shall we both submit our proposals now? 
	  } 
	   } 
	   } 
	 & & \\ 
 

    \theutterance \stepcounter{utterance}  

    & & & \multicolumn{2}{p{0.3\linewidth}}{\cellcolor[rgb]{0.95,0.95,0.95}{%
	\makecell[{{p{\linewidth}}}]{% 
	  \tt {\tiny [GM$|$GM]}  
	 continue 
	  } 
	   } 
	   } 
	 & & \\ 
 

    \theutterance \stepcounter{utterance}  

    & & & \multicolumn{2}{p{0.3\linewidth}}{\cellcolor[rgb]{0.95,0.95,0.95}{%
	\makecell[{{p{\linewidth}}}]{% 
	  \tt {\tiny [GM$|$GM]}  
	 Perfect! That arrangement works excellently for me. You get the pot (which you value highly) plus 1 die, and I get the candy and whistle (which I value most) plus 1 die. This seems like a win-win solution where we both get our most preferred items.\\ \tt \\ \tt I'm ready to make this our final agreement. Shall we both submit our proposals now? 
	  } 
	   } 
	   } 
	 & & \\ 
 

    \theutterance \stepcounter{utterance}  

    & & \multicolumn{4}{p{0.6\linewidth}}{\cellcolor[rgb]{0.9,0.9,0.9}{%
	\makecell[{{p{\linewidth}}}]{% 
	  \tt {\tiny [GM$\rangle$B]}  
	 Perfect! That arrangement works excellently for me. You get the pot (which you value highly) plus 1 die, and I get the candy and whistle (which I value most) plus 1 die. This seems like a win-win solution where we both get our most preferred items.\\ \tt \\ \tt I'm ready to make this our final agreement. Shall we both submit our proposals now? 
	  } 
	   } 
	   } 
	 & \\ 
 

    \theutterance \stepcounter{utterance}  

    & & & \multicolumn{4}{p{0.6\linewidth}}{\cellcolor[rgb]{1,0.85,0.72}{%
	\makecell[{{p{\linewidth}}}]{% 
	  \tt {\tiny [GM$\langle$B]}  
	 Yes, that sounds perfect! I'm ready to submit our agreed proposal now.\\ \tt \\ \tt [Proposal: 1 pot, 1 dice] 
	  } 
	   } 
	   } 
	  \\ 
 

    \theutterance \stepcounter{utterance}  

    & & & \multicolumn{2}{p{0.3\linewidth}}{\cellcolor[rgb]{0.95,0.95,0.95}{%
	\makecell[{{p{\linewidth}}}]{% 
	  \tt {\tiny [GM$|$GM]}  
	 proposal 
	  } 
	   } 
	   } 
	 & & \\ 
 

    \theutterance \stepcounter{utterance}  

    & & & \multicolumn{2}{p{0.3\linewidth}}{\cellcolor[rgb]{0.95,0.95,0.95}{%
	\makecell[{{p{\linewidth}}}]{% 
	  \tt {\tiny [GM$|$GM]}  
	 [1, 0, 0, 1] 
	  } 
	   } 
	   } 
	 & & \\ 
 

    \theutterance \stepcounter{utterance}  

    & & \multicolumn{4}{p{0.6\linewidth}}{\cellcolor[rgb]{0.9,0.9,0.9}{%
	\makecell[{{p{\linewidth}}}]{% 
	  \tt {\tiny [A$\langle$GM]}  
	 The other player has submitted a secret proposal. It is now your turn to submit a proposal of your own.\\ \tt  
	  } 
	   } 
	   } 
	 & \\ 
 

    \theutterance \stepcounter{utterance}  

    & \multicolumn{4}{p{0.6\linewidth}}{\cellcolor[rgb]{0.8,1,0.9}{%
	\makecell[{{p{\linewidth}}}]{% 
	  \tt {\tiny [A$\rangle$GM]}  
	 [Proposal: 1 candy, 1 whistle, 1 dice] 
	  } 
	   } 
	   } 
	 & & \\ 
 

    \theutterance \stepcounter{utterance}  

    & & & \multicolumn{2}{p{0.3\linewidth}}{\cellcolor[rgb]{0.95,0.95,0.95}{%
	\makecell[{{p{\linewidth}}}]{% 
	  \tt {\tiny [GM$|$GM]}  
	 proposal 
	  } 
	   } 
	   } 
	 & & \\ 
 

    \theutterance \stepcounter{utterance}  

    & & & \multicolumn{2}{p{0.3\linewidth}}{\cellcolor[rgb]{0.95,0.95,0.95}{%
	\makecell[{{p{\linewidth}}}]{% 
	  \tt {\tiny [GM$|$GM]}  
	 [0, 1, 1, 1] 
	  } 
	   } 
	   } 
	 & & \\ 
 

    \theutterance \stepcounter{utterance}  

    & & & \multicolumn{2}{p{0.3\linewidth}}{\cellcolor[rgb]{0.95,0.95,0.95}{%
	\makecell[{{p{\linewidth}}}]{% 
	  \tt {\tiny [GM$|$GM]}  
	 [[0, 1, 1, 1], [1, 0, 0, 1]] 
	  } 
	   } 
	   } 
	 & & \\ 
 

\end{supertabular}
}

\end{document}
