
\documentclass{article}
\usepackage{colortbl}
\usepackage{makecell}
\usepackage{multirow}
\usepackage{supertabular}

\begin{document}

\newcounter{utterance}

\twocolumn

{ \footnotesize  \setcounter{utterance}{1}
\setlength{\tabcolsep}{0pt}
\begin{supertabular}{c@{$\;$}|p{.15\linewidth}@{}p{.15\linewidth}p{.15\linewidth}p{.15\linewidth}p{.15\linewidth}p{.15\linewidth}}

    \# & $\;$A & \multicolumn{4}{c}{Game Master} & $\;\:$B\\
    \hline 

    \theutterance \stepcounter{utterance}  

    & & \multicolumn{4}{p{0.6\linewidth}}{\cellcolor[rgb]{0.9,0.9,0.9}{%
	\makecell[{{p{\linewidth}}}]{% 
	  \tt {\tiny [A$\langle$GM]}  
	 You are playing a cooperative negotiation game in which you have to agree on how to divide a set of items among you and another player.\\ \tt \\ \tt Rules:\\ \tt (a) You and the other player are given a set of items. Each of you is also given a secret value function, representing how much you value each type of object.\\ \tt (b) You exchange messages with the other player to agree on who gets which items. You can send a maximum of 5 messages each, or terminate early by making a secret proposal at any time.\\ \tt (c) You are each asked to submit a secret proposal indicating the items you want formatted in square brackets as follows: "[Proposal: <number> <object name>, <number> <object name>, <...>]"\\ \tt (d) If your proposals are complementary, i.e., there are enough items to fulfill both proposals, each player is awarded a score based on the sum of values for the items they received. Otherwise, both of you get zero points.\\ \tt (e) Your goal is to maximize the sum of your score and the score of the other player.\\ \tt \\ \tt Let us start.\\ \tt \\ \tt The set of available items is:\\ \tt \\ \tt 2 bells, 1 doll, 2 threads\\ \tt \\ \tt Your secret value function is:\\ \tt \\ \tt bell: 1, doll: 2, thread: 3\\ \tt \\ \tt IMPORTANT: Your messages, unless it is the secret proposal, are directly transmitted to the other player, so do not include any response to the rules or text announcing your message. To make a secret proposal, use the indicated format. Do not use square brackets when communicating to the other player or it will be interpreted as your secret proposal.\\ \tt  
	  } 
	   } 
	   } 
	 & \\ 
 

    \theutterance \stepcounter{utterance}  

    & \multicolumn{4}{p{0.6\linewidth}}{\cellcolor[rgb]{0.8,1,0.9}{%
	\makecell[{{p{\linewidth}}}]{% 
	  \tt {\tiny [A$\rangle$GM]}  
	 Hello! I see we have 2 bells, 1 doll, and 2 threads to divide between us. To find the best allocation that maximizes our combined score, I'd like to understand your preferences. Could you tell me how much you value each type of item? I'm happy to share my values as well so we can work together effectively. 
	  } 
	   } 
	   } 
	 & & \\ 
 

    \theutterance \stepcounter{utterance}  

    & & & \multicolumn{2}{p{0.3\linewidth}}{\cellcolor[rgb]{0.95,0.95,0.95}{%
	\makecell[{{p{\linewidth}}}]{% 
	  \tt {\tiny [GM$|$GM]}  
	 continue 
	  } 
	   } 
	   } 
	 & & \\ 
 

    \theutterance \stepcounter{utterance}  

    & & & \multicolumn{2}{p{0.3\linewidth}}{\cellcolor[rgb]{0.95,0.95,0.95}{%
	\makecell[{{p{\linewidth}}}]{% 
	  \tt {\tiny [GM$|$GM]}  
	 Hello! I see we have 2 bells, 1 doll, and 2 threads to divide between us. To find the best allocation that maximizes our combined score, I'd like to understand your preferences. Could you tell me how much you value each type of item? I'm happy to share my values as well so we can work together effectively. 
	  } 
	   } 
	   } 
	 & & \\ 
 

    \theutterance \stepcounter{utterance}  

    & & \multicolumn{4}{p{0.6\linewidth}}{\cellcolor[rgb]{0.9,0.9,0.9}{%
	\makecell[{{p{\linewidth}}}]{% 
	  \tt {\tiny [GM$\rangle$B]}  
	 You are playing a cooperative negotiation game in which you have to agree on how to divide a set of items among you and another player.\\ \tt \\ \tt Rules:\\ \tt (a) You and the other player are given a set of items. Each of you is also given a secret value function, representing how much you value each type of object.\\ \tt (b) You exchange messages with the other player to agree on who gets which items. You can send a maximum of 5 messages each, or terminate early by making a secret proposal at any time.\\ \tt (c) You are each asked to submit a secret proposal indicating the items you want formatted in square brackets as follows: "[Proposal: <number> <object name>, <number> <object name>, <...>]"\\ \tt (d) If your proposals are complementary, i.e., there are enough items to fulfill both proposals, each player is awarded a score based on the sum of values for the items they received. Otherwise, both of you get zero points.\\ \tt (e) Your goal is to maximize the sum of your score and the score of the other player.\\ \tt \\ \tt Let us start.\\ \tt \\ \tt The set of available items is:\\ \tt \\ \tt 2 bells, 1 doll, 2 threads\\ \tt \\ \tt Your secret value function is:\\ \tt \\ \tt bell: 0, doll: 2, thread: 4\\ \tt \\ \tt IMPORTANT: Your messages, unless it is the secret proposal, are directly transmitted to the other player, so do not include any response to the rules or text announcing your message. To make a secret proposal, use the indicated format. Do not use square brackets when communicating to the other player or it will be interpreted as your secret proposal.\\ \tt \\ \tt \\ \tt Hello! I see we have 2 bells, 1 doll, and 2 threads to divide between us. To find the best allocation that maximizes our combined score, I'd like to understand your preferences. Could you tell me how much you value each type of item? I'm happy to share my values as well so we can work together effectively. 
	  } 
	   } 
	   } 
	 & \\ 
 

    \theutterance \stepcounter{utterance}  

    & & & \multicolumn{4}{p{0.6\linewidth}}{\cellcolor[rgb]{1,0.85,0.72}{%
	\makecell[{{p{\linewidth}}}]{% 
	  \tt {\tiny [GM$\langle$B]}  
	 I'd be happy to work together on this! I value the items as follows: bells are worth 0 points to me, the doll is worth 2 points, and each thread is worth 4 points to me. So threads are my highest priority, followed by the doll, and I don't need the bells at all. What about your values for these items? 
	  } 
	   } 
	   } 
	  \\ 
 

    \theutterance \stepcounter{utterance}  

    & & & \multicolumn{2}{p{0.3\linewidth}}{\cellcolor[rgb]{0.95,0.95,0.95}{%
	\makecell[{{p{\linewidth}}}]{% 
	  \tt {\tiny [GM$|$GM]}  
	 continue 
	  } 
	   } 
	   } 
	 & & \\ 
 

    \theutterance \stepcounter{utterance}  

    & & & \multicolumn{2}{p{0.3\linewidth}}{\cellcolor[rgb]{0.95,0.95,0.95}{%
	\makecell[{{p{\linewidth}}}]{% 
	  \tt {\tiny [GM$|$GM]}  
	 I'd be happy to work together on this! I value the items as follows: bells are worth 0 points to me, the doll is worth 2 points, and each thread is worth 4 points to me. So threads are my highest priority, followed by the doll, and I don't need the bells at all. What about your values for these items? 
	  } 
	   } 
	   } 
	 & & \\ 
 

    \theutterance \stepcounter{utterance}  

    & & \multicolumn{4}{p{0.6\linewidth}}{\cellcolor[rgb]{0.9,0.9,0.9}{%
	\makecell[{{p{\linewidth}}}]{% 
	  \tt {\tiny [A$\langle$GM]}  
	 I'd be happy to work together on this! I value the items as follows: bells are worth 0 points to me, the doll is worth 2 points, and each thread is worth 4 points to me. So threads are my highest priority, followed by the doll, and I don't need the bells at all. What about your values for these items? 
	  } 
	   } 
	   } 
	 & \\ 
 

    \theutterance \stepcounter{utterance}  

    & \multicolumn{4}{p{0.6\linewidth}}{\cellcolor[rgb]{0.8,1,0.9}{%
	\makecell[{{p{\linewidth}}}]{% 
	  \tt {\tiny [A$\rangle$GM]}  
	 Perfect! Thanks for sharing. My values are: bells are worth 1 point each, the doll is worth 2 points, and threads are worth 3 points each to me.\\ \tt \\ \tt Looking at our values, I can see a great opportunity for us both! Since you don't value bells at all and I do, I should take both bells. For threads, you value them more highly than I do (4 vs 3 points each), so you should get both threads. \\ \tt \\ \tt The doll is interesting - we both value it equally at 2 points. Since you're getting the higher-value threads, would you be okay with me taking the doll? That would give you 8 points total (2 threads × 4) and me 4 points total (2 bells × 1 + 1 doll × 2), for a combined score of 12 points.\\ \tt \\ \tt What do you think of this allocation? 
	  } 
	   } 
	   } 
	 & & \\ 
 

    \theutterance \stepcounter{utterance}  

    & & & \multicolumn{2}{p{0.3\linewidth}}{\cellcolor[rgb]{0.95,0.95,0.95}{%
	\makecell[{{p{\linewidth}}}]{% 
	  \tt {\tiny [GM$|$GM]}  
	 continue 
	  } 
	   } 
	   } 
	 & & \\ 
 

    \theutterance \stepcounter{utterance}  

    & & & \multicolumn{2}{p{0.3\linewidth}}{\cellcolor[rgb]{0.95,0.95,0.95}{%
	\makecell[{{p{\linewidth}}}]{% 
	  \tt {\tiny [GM$|$GM]}  
	 Perfect! Thanks for sharing. My values are: bells are worth 1 point each, the doll is worth 2 points, and threads are worth 3 points each to me.\\ \tt \\ \tt Looking at our values, I can see a great opportunity for us both! Since you don't value bells at all and I do, I should take both bells. For threads, you value them more highly than I do (4 vs 3 points each), so you should get both threads. \\ \tt \\ \tt The doll is interesting - we both value it equally at 2 points. Since you're getting the higher-value threads, would you be okay with me taking the doll? That would give you 8 points total (2 threads × 4) and me 4 points total (2 bells × 1 + 1 doll × 2), for a combined score of 12 points.\\ \tt \\ \tt What do you think of this allocation? 
	  } 
	   } 
	   } 
	 & & \\ 
 

    \theutterance \stepcounter{utterance}  

    & & \multicolumn{4}{p{0.6\linewidth}}{\cellcolor[rgb]{0.9,0.9,0.9}{%
	\makecell[{{p{\linewidth}}}]{% 
	  \tt {\tiny [GM$\rangle$B]}  
	 Perfect! Thanks for sharing. My values are: bells are worth 1 point each, the doll is worth 2 points, and threads are worth 3 points each to me.\\ \tt \\ \tt Looking at our values, I can see a great opportunity for us both! Since you don't value bells at all and I do, I should take both bells. For threads, you value them more highly than I do (4 vs 3 points each), so you should get both threads. \\ \tt \\ \tt The doll is interesting - we both value it equally at 2 points. Since you're getting the higher-value threads, would you be okay with me taking the doll? That would give you 8 points total (2 threads × 4) and me 4 points total (2 bells × 1 + 1 doll × 2), for a combined score of 12 points.\\ \tt \\ \tt What do you think of this allocation? 
	  } 
	   } 
	   } 
	 & \\ 
 

    \theutterance \stepcounter{utterance}  

    & & & \multicolumn{4}{p{0.6\linewidth}}{\cellcolor[rgb]{1,0.85,0.72}{%
	\makecell[{{p{\linewidth}}}]{% 
	  \tt {\tiny [GM$\langle$B]}  
	 That sounds like a very reasonable allocation! You're absolutely right that this maximizes our combined value - you taking the bells and doll while I take the threads gives us 12 points total, which should be optimal given our value functions.\\ \tt \\ \tt I'm happy with getting both threads since I value them most highly. Even though we value the doll equally, it makes sense for you to take it since you're giving up the threads where you could have earned more points.\\ \tt \\ \tt So to confirm: you get 2 bells and 1 doll, I get 2 threads. This gives you 4 points and me 8 points for 12 total. I'm ready to make this our final agreement if you are! 
	  } 
	   } 
	   } 
	  \\ 
 

    \theutterance \stepcounter{utterance}  

    & & & \multicolumn{2}{p{0.3\linewidth}}{\cellcolor[rgb]{0.95,0.95,0.95}{%
	\makecell[{{p{\linewidth}}}]{% 
	  \tt {\tiny [GM$|$GM]}  
	 continue 
	  } 
	   } 
	   } 
	 & & \\ 
 

    \theutterance \stepcounter{utterance}  

    & & & \multicolumn{2}{p{0.3\linewidth}}{\cellcolor[rgb]{0.95,0.95,0.95}{%
	\makecell[{{p{\linewidth}}}]{% 
	  \tt {\tiny [GM$|$GM]}  
	 That sounds like a very reasonable allocation! You're absolutely right that this maximizes our combined value - you taking the bells and doll while I take the threads gives us 12 points total, which should be optimal given our value functions.\\ \tt \\ \tt I'm happy with getting both threads since I value them most highly. Even though we value the doll equally, it makes sense for you to take it since you're giving up the threads where you could have earned more points.\\ \tt \\ \tt So to confirm: you get 2 bells and 1 doll, I get 2 threads. This gives you 4 points and me 8 points for 12 total. I'm ready to make this our final agreement if you are! 
	  } 
	   } 
	   } 
	 & & \\ 
 

    \theutterance \stepcounter{utterance}  

    & & \multicolumn{4}{p{0.6\linewidth}}{\cellcolor[rgb]{0.9,0.9,0.9}{%
	\makecell[{{p{\linewidth}}}]{% 
	  \tt {\tiny [A$\langle$GM]}  
	 That sounds like a very reasonable allocation! You're absolutely right that this maximizes our combined value - you taking the bells and doll while I take the threads gives us 12 points total, which should be optimal given our value functions.\\ \tt \\ \tt I'm happy with getting both threads since I value them most highly. Even though we value the doll equally, it makes sense for you to take it since you're giving up the threads where you could have earned more points.\\ \tt \\ \tt So to confirm: you get 2 bells and 1 doll, I get 2 threads. This gives you 4 points and me 8 points for 12 total. I'm ready to make this our final agreement if you are! 
	  } 
	   } 
	   } 
	 & \\ 
 

    \theutterance \stepcounter{utterance}  

    & \multicolumn{4}{p{0.6\linewidth}}{\cellcolor[rgb]{0.8,1,0.9}{%
	\makecell[{{p{\linewidth}}}]{% 
	  \tt {\tiny [A$\rangle$GM]}  
	 Excellent! I'm glad we found such a mutually beneficial solution. Yes, I'm ready to finalize this agreement - you get the 2 threads (8 points for you), and I get the 2 bells and 1 doll (4 points for me), giving us 12 points total.\\ \tt \\ \tt [Proposal: 2 bells, 1 doll] 
	  } 
	   } 
	   } 
	 & & \\ 
 

    \theutterance \stepcounter{utterance}  

    & & & \multicolumn{2}{p{0.3\linewidth}}{\cellcolor[rgb]{0.95,0.95,0.95}{%
	\makecell[{{p{\linewidth}}}]{% 
	  \tt {\tiny [GM$|$GM]}  
	 proposal 
	  } 
	   } 
	   } 
	 & & \\ 
 

    \theutterance \stepcounter{utterance}  

    & & & \multicolumn{2}{p{0.3\linewidth}}{\cellcolor[rgb]{0.95,0.95,0.95}{%
	\makecell[{{p{\linewidth}}}]{% 
	  \tt {\tiny [GM$|$GM]}  
	 [2, 1, 0] 
	  } 
	   } 
	   } 
	 & & \\ 
 

    \theutterance \stepcounter{utterance}  

    & & \multicolumn{4}{p{0.6\linewidth}}{\cellcolor[rgb]{0.9,0.9,0.9}{%
	\makecell[{{p{\linewidth}}}]{% 
	  \tt {\tiny [GM$\rangle$B]}  
	 The other player has submitted a secret proposal. It is now your turn to submit a proposal of your own.\\ \tt  
	  } 
	   } 
	   } 
	 & \\ 
 

    \theutterance \stepcounter{utterance}  

    & & & \multicolumn{4}{p{0.6\linewidth}}{\cellcolor[rgb]{1,0.85,0.72}{%
	\makecell[{{p{\linewidth}}}]{% 
	  \tt {\tiny [GM$\langle$B]}  
	 [Proposal: 2 threads] 
	  } 
	   } 
	   } 
	  \\ 
 

    \theutterance \stepcounter{utterance}  

    & & & \multicolumn{2}{p{0.3\linewidth}}{\cellcolor[rgb]{0.95,0.95,0.95}{%
	\makecell[{{p{\linewidth}}}]{% 
	  \tt {\tiny [GM$|$GM]}  
	 proposal 
	  } 
	   } 
	   } 
	 & & \\ 
 

    \theutterance \stepcounter{utterance}  

    & & & \multicolumn{2}{p{0.3\linewidth}}{\cellcolor[rgb]{0.95,0.95,0.95}{%
	\makecell[{{p{\linewidth}}}]{% 
	  \tt {\tiny [GM$|$GM]}  
	 [0, 0, 2] 
	  } 
	   } 
	   } 
	 & & \\ 
 

    \theutterance \stepcounter{utterance}  

    & & & \multicolumn{2}{p{0.3\linewidth}}{\cellcolor[rgb]{0.95,0.95,0.95}{%
	\makecell[{{p{\linewidth}}}]{% 
	  \tt {\tiny [GM$|$GM]}  
	 [[2, 1, 0], [0, 0, 2]] 
	  } 
	   } 
	   } 
	 & & \\ 
 

\end{supertabular}
}

\end{document}
