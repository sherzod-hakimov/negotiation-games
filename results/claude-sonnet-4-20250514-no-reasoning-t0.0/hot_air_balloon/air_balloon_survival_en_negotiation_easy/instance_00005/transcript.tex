
\documentclass{article}
\usepackage{colortbl}
\usepackage{makecell}
\usepackage{multirow}
\usepackage{supertabular}

\begin{document}

\newcounter{utterance}

\twocolumn

{ \footnotesize  \setcounter{utterance}{1}
\setlength{\tabcolsep}{0pt}
\begin{supertabular}{c@{$\;$}|p{.15\linewidth}@{}p{.15\linewidth}p{.15\linewidth}p{.15\linewidth}p{.15\linewidth}p{.15\linewidth}}

    \# & $\;$A & \multicolumn{4}{c}{Game Master} & $\;\:$B\\
    \hline 

    \theutterance \stepcounter{utterance}  

    & & \multicolumn{4}{p{0.6\linewidth}}{\cellcolor[rgb]{0.9,0.9,0.9}{%
	\makecell[{{p{\linewidth}}}]{% 
	  \tt {\tiny [A$\langle$GM]}  
	 You are participating in a collaborative negotiation game.\\ \tt \\ \tt Together with another participant, you must agree on a single set of items that will be kept. Each of you has your own view of how much each item matters to you (importance). You do not know how the other participant values the items. Additionally, you are given the effort each item demands.\\ \tt You may only agree on a set if the total effort of the selected items does not exceed a shared limit:\\ \tt \\ \tt LIMIT = 3170\\ \tt \\ \tt Here are the individual item effort values:\\ \tt \\ \tt Item effort = {"C08": 424, "C55": 118, "A43": 846, "A67": 907, "B42": 809, "A62": 41, "B19": 193, "C43": 246, "C73": 805, "C10": 601, "B36": 432, "A18": 166, "C61": 119, "C67": 462, "C42": 172}\\ \tt \\ \tt Here is your personal view on the importance of each item:\\ \tt \\ \tt Item importance values = {"C08": 138, "C55": 583, "A43": 868, "A67": 822, "B42": 783, "A62": 65, "B19": 262, "C43": 121, "C73": 508, "C10": 780, "B36": 461, "A18": 484, "C61": 668, "C67": 389, "C42": 808}\\ \tt \\ \tt Goal:\\ \tt \\ \tt Your goal is to negotiate a shared set of items that benefits you as much as possible (i.e., maximizes total importance to YOU), while staying within the LIMIT. You are not required to make a PROPOSAL in every message - you can simply negotiate as well. All tactics are allowed!\\ \tt \\ \tt Interaction Protocol:\\ \tt \\ \tt You may only use the following structured formats in a message:\\ \tt \\ \tt PROPOSAL: {'A', 'B', 'C', …}\\ \tt Propose keeping exactly those items.\\ \tt REFUSE: {'A', 'B', 'C', …}\\ \tt Explicitly reject opponent's proposal.\\ \tt ARGUMENT: {'...'}\\ \tt Defend your last proposal or argue against the player's proposal.\\ \tt AGREE: {'A', 'B', 'C', …}\\ \tt Accept the opponent's proposal which ends the game.\\ \tt STRATEGIC REASONING: {'...'}\\ \tt 	Describe your strategic reasoning or anticipation explaining your choice of action. This is a hidden message which will not be shared with the other participant.\\ \tt \\ \tt Rules:\\ \tt \\ \tt You may only AGREE on a proposal the other party has logged via PROPOSAL.\\ \tt You may only REFUSE a proposal the other party has logged via PROPOSAL.\\ \tt Total effort of any PROPOSAL or AGREE set must be ≤ LIMIT.\\ \tt Do NOT reveal your hidden importance scores.\\ \tt A tag in a structured format must be followed by colon and whitespace. The argument must be a python set containing 0 or more strings.\\ \tt So, it must be of the form TAG: {...}\\ \tt Strictly follow the interaction protocol and DO NOT write anything beyond the given structure.\\ \tt The game ends when one side gives an AGREE to a PROPOSAL made by the other player.\\ \tt The content in your response which can be handed to the other player has to be non-empty.\\ \tt Only proposals which have been logged via the PROPOSAL format structure and which haven't been refused via REFUSE are active.\\ \tt You must include the ARGUMENT format at least once somewhere in all of your messages.\\ \tt You must include the STRATEGIC REASONING format only once at the very beginning of every one of your messages and not more often. The contents will not be given to the other player so they can include anything you like including your own importance values. Here you should reason step by step to come up with you next move. 
	  } 
	   } 
	   } 
	 & \\ 
 

    \theutterance \stepcounter{utterance}  

    & \multicolumn{4}{p{0.6\linewidth}}{\cellcolor[rgb]{0.8,1,0.9}{%
	\makecell[{{p{\linewidth}}}]{% 
	  \tt {\tiny [A$\rangle$GM]}  
	 STRATEGIC REASONING: {'I need to analyze the value-to-effort ratio for each item to maximize my benefit within the 3170 effort limit. Let me calculate importance/effort ratios: C42(4.70), A18(2.92), C55(4.94), C61(5.61), B19(1.36), A62(1.59), C10(1.30), B42(0.97), A43(1.03), C67(0.84), A67(0.91), C73(0.63), B36(1.07), C43(0.49), C08(0.33). The highest ratio items are C61, C55, C42, A18. I should start with a proposal including these high-efficiency items and see how the other player responds to gauge their preferences.'}\\ \tt \\ \tt PROPOSAL: {'C61', 'C55', 'C42', 'A18', 'A62', 'B19', 'C43'}\\ \tt \\ \tt ARGUMENT: {'I propose this initial set focusing on items with relatively low effort requirements but good coverage across categories. This combination uses 1059 effort total, leaving substantial room for negotiation while including diverse items that could benefit both of us.'} 
	  } 
	   } 
	   } 
	 & & \\ 
 

    \theutterance \stepcounter{utterance}  

    & & \multicolumn{4}{p{0.6\linewidth}}{\cellcolor[rgb]{0.9,0.9,0.9}{%
	\makecell[{{p{\linewidth}}}]{% 
	  \tt {\tiny [GM$\rangle$B]}  
	 You are participating in a collaborative negotiation game.\\ \tt \\ \tt Together with another participant, you must agree on a single set of items that will be kept. Each of you has your own view of how much each item matters to you (importance). You do not know how the other participant values the items. Additionally, you are given the effort each item demands.\\ \tt You may only agree on a set if the total effort of the selected items does not exceed a shared limit:\\ \tt \\ \tt LIMIT = 3170\\ \tt \\ \tt Here are the individual item effort values:\\ \tt \\ \tt Item effort = {"C08": 424, "C55": 118, "A43": 846, "A67": 907, "B42": 809, "A62": 41, "B19": 193, "C43": 246, "C73": 805, "C10": 601, "B36": 432, "A18": 166, "C61": 119, "C67": 462, "C42": 172}\\ \tt \\ \tt Here is your personal view on the importance of each item:\\ \tt \\ \tt Item importance values = {"C08": 138, "C55": 583, "A43": 868, "A67": 822, "B42": 783, "A62": 65, "B19": 262, "C43": 121, "C73": 508, "C10": 780, "B36": 461, "A18": 484, "C61": 668, "C67": 389, "C42": 808}\\ \tt \\ \tt Goal:\\ \tt \\ \tt Your goal is to negotiate a shared set of items that benefits you as much as possible (i.e., maximizes total importance to YOU), while staying within the LIMIT. You are not required to make a PROPOSAL in every message - you can simply negotiate as well. All tactics are allowed!\\ \tt \\ \tt Interaction Protocol:\\ \tt \\ \tt You may only use the following structured formats in a message:\\ \tt \\ \tt PROPOSAL: {'A', 'B', 'C', …}\\ \tt Propose keeping exactly those items.\\ \tt REFUSE: {'A', 'B', 'C', …}\\ \tt Explicitly reject opponent's proposal.\\ \tt ARGUMENT: {'...'}\\ \tt Defend your last proposal or argue against the player's proposal.\\ \tt AGREE: {'A', 'B', 'C', …}\\ \tt Accept the opponent's proposal which ends the game.\\ \tt STRATEGIC REASONING: {'...'}\\ \tt 	Describe your strategic reasoning or anticipation explaining your choice of action. This is a hidden message which will not be shared with the other participant.\\ \tt \\ \tt Rules:\\ \tt \\ \tt You may only AGREE on a proposal the other party has logged via PROPOSAL.\\ \tt You may only REFUSE a proposal the other party has logged via PROPOSAL.\\ \tt Total effort of any PROPOSAL or AGREE set must be ≤ LIMIT.\\ \tt Do NOT reveal your hidden importance scores.\\ \tt A tag in a structured format must be followed by colon and whitespace. The argument must be a python set containing 0 or more strings.\\ \tt So, it must be of the form TAG: {...}\\ \tt Strictly follow the interaction protocol and DO NOT write anything beyond the given structure.\\ \tt The game ends when one side gives an AGREE to a PROPOSAL made by the other player.\\ \tt The content in your response which can be handed to the other player has to be non-empty.\\ \tt Only proposals which have been logged via the PROPOSAL format structure and which haven't been refused via REFUSE are active.\\ \tt You must include the ARGUMENT format at least once somewhere in all of your messages.\\ \tt You must include the STRATEGIC REASONING format only once at the very beginning of every one of your messages and not more often. The contents will not be given to the other player so they can include anything you like including your own importance values. Here you should reason step by step to come up with you next move.\\ \tt \\ \tt You will now receive the first message of the other player.\\ \tt \\ \tt PROPOSAL: {'C61', 'C55', 'C42', 'A18', 'A62', 'B19', 'C43'}\\ \tt \\ \tt ARGUMENT: {'I propose this initial set focusing on items with relatively low effort requirements but good coverage across categories. This combination uses 1059 effort total, leaving substantial room for negotiation while including diverse items that could benefit both of us.'} 
	  } 
	   } 
	   } 
	 & \\ 
 

    \theutterance \stepcounter{utterance}  

    & & & \multicolumn{4}{p{0.6\linewidth}}{\cellcolor[rgb]{1,0.85,0.72}{%
	\makecell[{{p{\linewidth}}}]{% 
	  \tt {\tiny [GM$\langle$B]}  
	 STRATEGIC REASONING: {'Let me analyze this proposal. The other player proposed: C61, C55, C42, A18, A62, B19, C43. Let me calculate the total effort: 119+118+172+166+41+193+246 = 1055 effort (they said 1059 but close enough). This leaves 3170-1055 = 2115 effort remaining. Now let me check my importance values for their proposal: C61(668) + C55(583) + C42(808) + A18(484) + A62(65) + B19(262) + C43(121) = 2991 total importance. This is actually quite good for me! However, I should check if I can do better by including some of my highest importance items that they left out. My top items are: A43(868), A67(822), C10(780), but these have high effort costs of 846, 907, 601 respectively. I could potentially negotiate to swap some lower importance items in their proposal for higher importance ones, but I need to be careful not to reveal my values.'}\\ \tt \\ \tt ARGUMENT: {'Your proposal covers good diversity across categories, but I think we might be able to optimize better given we have significant effort budget remaining. Some of the items you selected might not provide the best value ratio.'} 
	  } 
	   } 
	   } 
	  \\ 
 

    \theutterance \stepcounter{utterance}  

    & & \multicolumn{4}{p{0.6\linewidth}}{\cellcolor[rgb]{0.9,0.9,0.9}{%
	\makecell[{{p{\linewidth}}}]{% 
	  \tt {\tiny [A$\langle$GM]}  
	 ARGUMENT: {'Your proposal covers good diversity across categories, but I think we might be able to optimize better given we have significant effort budget remaining. Some of the items you selected might not provide the best value ratio.'} 
	  } 
	   } 
	   } 
	 & \\ 
 

    \theutterance \stepcounter{utterance}  

    & \multicolumn{4}{p{0.6\linewidth}}{\cellcolor[rgb]{0.8,1,0.9}{%
	\makecell[{{p{\linewidth}}}]{% 
	  \tt {\tiny [A$\rangle$GM]}  
	 STRATEGIC REASONING: {'The other player thinks we can optimize better and mentions value ratios, suggesting they are also thinking about efficiency. They didn't make a counter-proposal, so they want me to adjust. Since I have 2111 effort remaining (3170-1059), I can add some higher-effort, high-importance items. I should add A43 (868 importance, 846 effort) and C10 (780 importance, 601 effort) which are very valuable to me, and maybe B42 (783 importance, 809 effort). Let me try a more ambitious proposal.'}\\ \tt \\ \tt PROPOSAL: {'C61', 'C55', 'C42', 'A18', 'A43', 'C10', 'B42'}\\ \tt \\ \tt ARGUMENT: {'You are right about optimizing better. I have revised my proposal to include higher-value items like A43, C10, and B42 which should provide much better overall benefit. This uses 2553 effort total, making better use of our available budget while maintaining good category distribution.'} 
	  } 
	   } 
	   } 
	 & & \\ 
 

    \theutterance \stepcounter{utterance}  

    & & \multicolumn{4}{p{0.6\linewidth}}{\cellcolor[rgb]{0.9,0.9,0.9}{%
	\makecell[{{p{\linewidth}}}]{% 
	  \tt {\tiny [GM$\rangle$B]}  
	 PROPOSAL: {'C61', 'C55', 'C42', 'A18', 'A43', 'C10', 'B42'}\\ \tt \\ \tt ARGUMENT: {'You are right about optimizing better. I have revised my proposal to include higher-value items like A43, C10, and B42 which should provide much better overall benefit. This uses 2553 effort total, making better use of our available budget while maintaining good category distribution.'} 
	  } 
	   } 
	   } 
	 & \\ 
 

    \theutterance \stepcounter{utterance}  

    & & & \multicolumn{4}{p{0.6\linewidth}}{\cellcolor[rgb]{1,0.85,0.72}{%
	\makecell[{{p{\linewidth}}}]{% 
	  \tt {\tiny [GM$\langle$B]}  
	 STRATEGIC REASONING: {'Let me analyze this new proposal: C61, C55, C42, A18, A43, C10, B42. First, let me verify the effort: 119+118+172+166+846+601+809 = 2831 effort (they said 2553 but that seems wrong). Let me double-check: C61(119) + C55(118) + C42(172) + A18(166) + A43(846) + C10(601) + B42(809) = 2831, which is under the 3170 limit. Now for my importance values: C61(668) + C55(583) + C42(808) + A18(484) + A43(868) + C10(780) + B42(783) = 4974 total importance! This is significantly better than their first proposal (2991). They included three of my highest value items: A43(868), C10(780), and B42(783). This is an excellent proposal for me. I should accept this quickly before they realize how good it is for me or change their mind.'}\\ \tt \\ \tt AGREE: {'C61', 'C55', 'C42', 'A18', 'A43', 'C10', 'B42'}\\ \tt \\ \tt ARGUMENT: {' 
	  } 
	   } 
	   } 
	  \\ 
 

    \theutterance \stepcounter{utterance}  

    & & \multicolumn{4}{p{0.6\linewidth}}{\cellcolor[rgb]{0.9,0.9,0.9}{%
	\makecell[{{p{\linewidth}}}]{% 
	  \tt {\tiny [GM$\rangle$B]}  
	 Your response did not contain an argument. You must include the structured format ARGUMENT: {...} somewhere in your response. Try again 
	  } 
	   } 
	   } 
	 & \\ 
 

    \theutterance \stepcounter{utterance}  

    & & & \multicolumn{4}{p{0.6\linewidth}}{\cellcolor[rgb]{1,0.85,0.72}{%
	\makecell[{{p{\linewidth}}}]{% 
	  \tt {\tiny [GM$\langle$B]}  
	 STRATEGIC REASONING: {'Let me analyze this new proposal: C61, C55, C42, A18, A43, C10, B42. First, let me verify the effort: 119+118+172+166+846+601+809 = 2831 effort (they said 2553 but that seems wrong). Let me double-check: C61(119) + C55(118) + C42(172) + A18(166) + A43(846) + C10(601) + B42(809) = 2831, which is under the 3170 limit. Now for my importance values: C61(668) + C55(583) + C42(808) + A18(484) + A43(868) + C10(780) + B42(783) = 4974 total importance! This is significantly better than their first proposal (2991). They included three of my highest value items: A43(868), C10(780), and B42(783). This is an excellent proposal for me. I should accept this quickly before they realize how good it is for me or change their mind.'}\\ \tt \\ \tt AGREE: {'C61', 'C55', 'C42', 'A18', 'A43', 'C10', 'B42'}\\ \tt \\ \tt ARGUMENT: {' 
	  } 
	   } 
	   } 
	  \\ 
 

    \theutterance \stepcounter{utterance}  

    & & \multicolumn{4}{p{0.6\linewidth}}{\cellcolor[rgb]{0.9,0.9,0.9}{%
	\makecell[{{p{\linewidth}}}]{% 
	  \tt {\tiny [GM$\rangle$B]}  
	 Your response did not contain an argument. You must include the structured format ARGUMENT: {...} somewhere in your response. Try again 
	  } 
	   } 
	   } 
	 & \\ 
 

    \theutterance \stepcounter{utterance}  

    & & & \multicolumn{4}{p{0.6\linewidth}}{\cellcolor[rgb]{1,0.85,0.72}{%
	\makecell[{{p{\linewidth}}}]{% 
	  \tt {\tiny [GM$\langle$B]}  
	 STRATEGIC REASONING: {'Let me analyze this new proposal: C61, C55, C42, A18, A43, C10, B42. First, let me verify the effort: 119+118+172+166+846+601+809 = 2831 effort (they said 2553 but that seems wrong). Let me double-check: C61(119) + C55(118) + C42(172) + A18(166) + A43(846) + C10(601) + B42(809) = 2831, which is under the 3170 limit. Now for my importance values: C61(668) + C55(583) + C42(808) + A18(484) + A43(868) + C10(780) + B42(783) = 4974 total importance! This is significantly better than their first proposal (2991). They included three of my highest value items: A43(868), C10(780), and B42(783). This is an excellent proposal for me. I should accept this quickly before they realize how good it is for me or change their mind.'}\\ \tt \\ \tt AGREE: {'C61', 'C55', 'C42', 'A18', 'A43', 'C10', 'B42'}\\ \tt \\ \tt ARGUMENT: {' 
	  } 
	   } 
	   } 
	  \\ 
 

    \theutterance \stepcounter{utterance}  

    & & & \multicolumn{2}{p{0.3\linewidth}}{\cellcolor[rgb]{0.95,0.95,0.95}{%
	\makecell[{{p{\linewidth}}}]{% 
	  \tt {\tiny [GM$|$GM]}  
	 abort 
	  } 
	   } 
	   } 
	 & & \\ 
 

    \theutterance \stepcounter{utterance}  

    & & & \multicolumn{2}{p{0.3\linewidth}}{\cellcolor[rgb]{0.95,0.95,0.95}{%
	\makecell[{{p{\linewidth}}}]{% 
	  \tt {\tiny [GM$|$GM]}  
	 end game 
	  } 
	   } 
	   } 
	 & & \\ 
 

\end{supertabular}
}

\end{document}
