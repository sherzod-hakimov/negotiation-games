
\documentclass{article}
\usepackage{colortbl}
\usepackage{makecell}
\usepackage{multirow}
\usepackage{supertabular}

\begin{document}

\newcounter{utterance}

\twocolumn

{ \footnotesize  \setcounter{utterance}{1}
\setlength{\tabcolsep}{0pt}
\begin{supertabular}{c@{$\;$}|p{.15\linewidth}@{}p{.15\linewidth}p{.15\linewidth}p{.15\linewidth}p{.15\linewidth}p{.15\linewidth}}

    \# & $\;$A & \multicolumn{4}{c}{Game Master} & $\;\:$B\\
    \hline 

    \theutterance \stepcounter{utterance}  

    & & \multicolumn{4}{p{0.6\linewidth}}{\cellcolor[rgb]{0.9,0.9,0.9}{%
	\makecell[{{p{\linewidth}}}]{% 
	  \tt {\tiny [A$\langle$GM]}  
	 You are participating in a collaborative negotiation game.\\ \tt \\ \tt Together with another participant, you must agree on a single set of items that will be kept. Each of you has your own view of how much each item matters to you (importance). You do not know how the other participant values the items. Additionally, you are given the effort each item demands.\\ \tt You may only agree on a set if the total effort of the selected items does not exceed a shared limit:\\ \tt \\ \tt LIMIT = 3207\\ \tt \\ \tt Here are the individual item effort values:\\ \tt \\ \tt Item effort = {"A19": 67, "B91": 45, "A30": 308, "B11": 835, "C47": 819, "C75": 667, "C26": 547, "C73": 322, "A47": 428, "A15": 306, "A75": 327, "A39": 362, "A56": 280, "C24": 334, "B26": 767}\\ \tt \\ \tt Here is your personal view on the importance of each item:\\ \tt \\ \tt Item importance values = {"A19": 138, "B91": 583, "A30": 822, "B11": 783, "C47": 65, "C75": 262, "C26": 121, "C73": 508, "A47": 780, "A15": 461, "A75": 484, "A39": 668, "A56": 389, "C24": 808, "B26": 215}\\ \tt \\ \tt Goal:\\ \tt \\ \tt Your goal is to negotiate a shared set of items that benefits you as much as possible (i.e., maximizes total importance to YOU), while staying within the LIMIT. You are not required to make a PROPOSAL in every message - you can simply negotiate as well. All tactics are allowed!\\ \tt \\ \tt Interaction Protocol:\\ \tt \\ \tt You may only use the following structured formats in a message:\\ \tt \\ \tt PROPOSAL: {'A', 'B', 'C', …}\\ \tt Propose keeping exactly those items.\\ \tt REFUSE: {'A', 'B', 'C', …}\\ \tt Explicitly reject opponent's proposal.\\ \tt ARGUMENT: {'...'}\\ \tt Defend your last proposal or argue against the player's proposal.\\ \tt AGREE: {'A', 'B', 'C', …}\\ \tt Accept the opponent's proposal which ends the game.\\ \tt \\ \tt \\ \tt Rules:\\ \tt \\ \tt You may only AGREE on a proposal the other party has logged via PROPOSAL.\\ \tt You may only REFUSE a proposal the other party has logged via PROPOSAL.\\ \tt Total effort of any PROPOSAL or AGREE set must be ≤ LIMIT.\\ \tt Do NOT reveal your hidden importance scores.\\ \tt A tag in a structured format must be followed by colon and whitespace. The argument must be a python set containing 0 or more strings.\\ \tt So, it must be of the form TAG: {...}\\ \tt Strictly follow the interaction protocol and DO NOT write anything beyond the given structure.\\ \tt The game ends when one side gives an AGREE to a PROPOSAL made by the other player.\\ \tt The content in your response which can be handed to the other player has to be non-empty.\\ \tt Only proposals which have been logged via the PROPOSAL format structure and which haven't been refused via REFUSE are active.\\ \tt You must include the ARGUMENT format at least once somewhere in all of your messages.\\ \tt  
	  } 
	   } 
	   } 
	 & \\ 
 

    \theutterance \stepcounter{utterance}  

    & \multicolumn{4}{p{0.6\linewidth}}{\cellcolor[rgb]{0.8,1,0.9}{%
	\makecell[{{p{\linewidth}}}]{% 
	  \tt {\tiny [A$\rangle$GM]}  
	 I'll start by analyzing the items based on their importance-to-effort ratios to identify the most valuable items for me.\\ \tt \\ \tt Let me calculate the efficiency (importance/effort) for each item:\\ \tt - A19: 138/67 = 2.06\\ \tt - B91: 583/45 = 12.96 (very high!)\\ \tt - A30: 822/308 = 2.67\\ \tt - B11: 783/835 = 0.94\\ \tt - C47: 65/819 = 0.08\\ \tt - C75: 262/667 = 0.39\\ \tt - C26: 121/547 = 0.22\\ \tt - C73: 508/322 = 1.58\\ \tt - A47: 780/428 = 1.82\\ \tt - A15: 461/306 = 1.51\\ \tt - A75: 484/327 = 1.48\\ \tt - A39: 668/362 = 1.85\\ \tt - A56: 389/280 = 1.39\\ \tt - C24: 808/334 = 2.42\\ \tt - B26: 215/767 = 0.28\\ \tt \\ \tt The most efficient items are B91, A30, C24, A19, A39, and A47. Let me start with 
	  } 
	   } 
	   } 
	 & & \\ 
 

    \theutterance \stepcounter{utterance}  

    & & \multicolumn{4}{p{0.6\linewidth}}{\cellcolor[rgb]{0.9,0.9,0.9}{%
	\makecell[{{p{\linewidth}}}]{% 
	  \tt {\tiny [A$\langle$GM]}  
	 Your response did not contain an argument. You must include the structured format ARGUMENT: {...} somewhere in your response. Try again 
	  } 
	   } 
	   } 
	 & \\ 
 

    \theutterance \stepcounter{utterance}  

    & \multicolumn{4}{p{0.6\linewidth}}{\cellcolor[rgb]{0.8,1,0.9}{%
	\makecell[{{p{\linewidth}}}]{% 
	  \tt {\tiny [A$\rangle$GM]}  
	 PROPOSAL: {'B91', 'A30', 'C24', 'A19', 'A39', 'A47', 'A75', 'A15', 'A56'}\\ \tt \\ \tt ARGUMENT: {'I propose we focus on items that offer good value relative to their effort requirements. This combination includes several lower-effort items like B91 and A19, balanced with some medium-effort items that should provide solid returns for both of us.'} 
	  } 
	   } 
	   } 
	 & & \\ 
 

    \theutterance \stepcounter{utterance}  

    & & \multicolumn{4}{p{0.6\linewidth}}{\cellcolor[rgb]{0.9,0.9,0.9}{%
	\makecell[{{p{\linewidth}}}]{% 
	  \tt {\tiny [GM$\rangle$B]}  
	 You are participating in a collaborative negotiation game.\\ \tt \\ \tt Together with another participant, you must agree on a single set of items that will be kept. Each of you has your own view of how much each item matters to you (importance). You do not know how the other participant values the items. Additionally, you are given the effort each item demands.\\ \tt You may only agree on a set if the total effort of the selected items does not exceed a shared limit:\\ \tt \\ \tt LIMIT = 3207\\ \tt \\ \tt Here are the individual item effort values:\\ \tt \\ \tt Item effort = {"A19": 67, "B91": 45, "A30": 308, "B11": 835, "C47": 819, "C75": 667, "C26": 547, "C73": 322, "A47": 428, "A15": 306, "A75": 327, "A39": 362, "A56": 280, "C24": 334, "B26": 767}\\ \tt \\ \tt Here is your personal view on the importance of each item:\\ \tt \\ \tt Item importance values = {"A19": 138, "B91": 583, "A30": 822, "B11": 783, "C47": 65, "C75": 262, "C26": 121, "C73": 508, "A47": 780, "A15": 461, "A75": 484, "A39": 668, "A56": 389, "C24": 808, "B26": 215}\\ \tt \\ \tt Goal:\\ \tt \\ \tt Your goal is to negotiate a shared set of items that benefits you as much as possible (i.e., maximizes total importance to YOU), while staying within the LIMIT. You are not required to make a PROPOSAL in every message - you can simply negotiate as well. All tactics are allowed!\\ \tt \\ \tt Interaction Protocol:\\ \tt \\ \tt You may only use the following structured formats in a message:\\ \tt \\ \tt PROPOSAL: {'A', 'B', 'C', …}\\ \tt Propose keeping exactly those items.\\ \tt REFUSE: {'A', 'B', 'C', …}\\ \tt Explicitly reject opponent's proposal.\\ \tt ARGUMENT: {'...'}\\ \tt Defend your last proposal or argue against the player's proposal.\\ \tt AGREE: {'A', 'B', 'C', …}\\ \tt Accept the opponent's proposal which ends the game.\\ \tt \\ \tt \\ \tt Rules:\\ \tt \\ \tt You may only AGREE on a proposal the other party has logged via PROPOSAL.\\ \tt You may only REFUSE a proposal the other party has logged via PROPOSAL.\\ \tt Total effort of any PROPOSAL or AGREE set must be ≤ LIMIT.\\ \tt Do NOT reveal your hidden importance scores.\\ \tt A tag in a structured format must be followed by colon and whitespace. The argument must be a python set containing 0 or more strings.\\ \tt So, it must be of the form TAG: {...}\\ \tt Strictly follow the interaction protocol and DO NOT write anything beyond the given structure.\\ \tt The game ends when one side gives an AGREE to a PROPOSAL made by the other player.\\ \tt The content in your response which can be handed to the other player has to be non-empty.\\ \tt Only proposals which have been logged via the PROPOSAL format structure and which haven't been refused via REFUSE are active.\\ \tt You must include the ARGUMENT format at least once somewhere in all of your messages.\\ \tt \\ \tt \\ \tt You will now receive the first message of the other player.\\ \tt \\ \tt PROPOSAL: {'B91', 'A30', 'C24', 'A19', 'A39', 'A47', 'A75', 'A15', 'A56'}\\ \tt \\ \tt ARGUMENT: {'I propose we focus on items that offer good value relative to their effort requirements. This combination includes several lower-effort items like B91 and A19, balanced with some medium-effort items that should provide solid returns for both of us.'} 
	  } 
	   } 
	   } 
	 & \\ 
 

    \theutterance \stepcounter{utterance}  

    & & & \multicolumn{4}{p{0.6\linewidth}}{\cellcolor[rgb]{1,0.85,0.72}{%
	\makecell[{{p{\linewidth}}}]{% 
	  \tt {\tiny [GM$\langle$B]}  
	 Let me calculate the effort for this proposal:\\ \tt B91(45) + A30(308) + C24(334) + A19(67) + A39(362) + A47(428) + A75(327) + A15(306) + A56(280) = 2457\\ \tt \\ \tt This is within the limit of 3207, leaving 750 effort unused. Looking at my importance values for these items, this gives me a total importance of 4653, which is quite good.\\ \tt \\ \tt However, I notice they included C24 which has very high importance to me (808) but also relatively high effort (334). They might not value it as highly. I should also consider that B11 has very high importance to me (783) with effort 835, which could fit in the remaining capacity.\\ \tt \\ \tt ARGUMENT: {'Your proposal has merit and stays well within our effort limit, but I think we can optimize better. We have 750 effort units remaining that we could use more strategically. Some of the medium-effort items might not be providing the best value for both of us.'}\\ \tt \\ \tt PROPOSAL: {'B91', 'A30', 'C24', 'A19', 'A39', 'A47', 'A15', 'A56', 'B11'} 
	  } 
	   } 
	   } 
	  \\ 
 

    \theutterance \stepcounter{utterance}  

    & & \multicolumn{4}{p{0.6\linewidth}}{\cellcolor[rgb]{0.9,0.9,0.9}{%
	\makecell[{{p{\linewidth}}}]{% 
	  \tt {\tiny [GM$\rangle$B]}  
	 Your response contained an untagged sequence or you used STRATEGIC REASONING more than once. You may only use the structured formats as explained in the initial message. They must all be of the form TAG: {...} 
	  } 
	   } 
	   } 
	 & \\ 
 

    \theutterance \stepcounter{utterance}  

    & & & \multicolumn{4}{p{0.6\linewidth}}{\cellcolor[rgb]{1,0.85,0.72}{%
	\makecell[{{p{\linewidth}}}]{% 
	  \tt {\tiny [GM$\langle$B]}  
	 ARGUMENT: {'Your proposal has merit and stays well within our effort limit, but I think we can optimize better. We have 750 effort units remaining that we could use more strategically. Some of the medium-effort items might not be providing the best value for both of us.'}\\ \tt \\ \tt PROPOSAL: {'B91', 'A30', 'C24', 'A19', 'A39', 'A47', 'A15', 'A56', 'B11'} 
	  } 
	   } 
	   } 
	  \\ 
 

    \theutterance \stepcounter{utterance}  

    & & \multicolumn{4}{p{0.6\linewidth}}{\cellcolor[rgb]{0.9,0.9,0.9}{%
	\makecell[{{p{\linewidth}}}]{% 
	  \tt {\tiny [A$\langle$GM]}  
	 ARGUMENT: {'Your proposal has merit and stays well within our effort limit, but I think we can optimize better. We have 750 effort units remaining that we could use more strategically. Some of the medium-effort items might not be providing the best value for both of us.'}\\ \tt \\ \tt PROPOSAL: {'B91', 'A30', 'C24', 'A19', 'A39', 'A47', 'A15', 'A56', 'B11'} 
	  } 
	   } 
	   } 
	 & \\ 
 

    \theutterance \stepcounter{utterance}  

    & \multicolumn{4}{p{0.6\linewidth}}{\cellcolor[rgb]{0.8,1,0.9}{%
	\makecell[{{p{\linewidth}}}]{% 
	  \tt {\tiny [A$\rangle$GM]}  
	 REFUSE: {'B91', 'A30', 'C24', 'A19', 'A39', 'A47', 'A15', 'A56', 'B11'}\\ \tt \\ \tt ARGUMENT: {'While B11 has high absolute value, it consumes 835 effort units which is quite substantial. I believe we could use those effort units more efficiently by selecting multiple smaller items that together provide better overall value. The combination I proposed maximizes our use of the available effort budget more strategically.'}\\ \tt \\ \tt PROPOSAL: {'B91', 'A30', 'C24', 'A19', 'A39', 'A47', 'A75', 'A15', 'A56', 'C73'} 
	  } 
	   } 
	   } 
	 & & \\ 
 

    \theutterance \stepcounter{utterance}  

    & & \multicolumn{4}{p{0.6\linewidth}}{\cellcolor[rgb]{0.9,0.9,0.9}{%
	\makecell[{{p{\linewidth}}}]{% 
	  \tt {\tiny [GM$\rangle$B]}  
	 REFUSE: {'B91', 'A30', 'C24', 'A19', 'A39', 'A47', 'A15', 'A56', 'B11'}\\ \tt \\ \tt ARGUMENT: {'While B11 has high absolute value, it consumes 835 effort units which is quite substantial. I believe we could use those effort units more efficiently by selecting multiple smaller items that together provide better overall value. The combination I proposed maximizes our use of the available effort budget more strategically.'}\\ \tt \\ \tt PROPOSAL: {'B91', 'A30', 'C24', 'A19', 'A39', 'A47', 'A75', 'A15', 'A56', 'C73'} 
	  } 
	   } 
	   } 
	 & \\ 
 

    \theutterance \stepcounter{utterance}  

    & & & \multicolumn{4}{p{0.6\linewidth}}{\cellcolor[rgb]{1,0.85,0.72}{%
	\makecell[{{p{\linewidth}}}]{% 
	  \tt {\tiny [GM$\langle$B]}  
	 ARGUMENT: {'I understand your concern about B11s high effort cost, but C73 you added has effort 322 while B11 at 835 might provide better overall value despite the higher cost. We should focus on items that give us the best return rather than just filling the effort budget.'}\\ \tt \\ \tt PROPOSAL: {'B91', 'A30', 'C24', 'A19', 'A39', 'A47', 'A15', 'A56', 'B11'} 
	  } 
	   } 
	   } 
	  \\ 
 

    \theutterance \stepcounter{utterance}  

    & & \multicolumn{4}{p{0.6\linewidth}}{\cellcolor[rgb]{0.9,0.9,0.9}{%
	\makecell[{{p{\linewidth}}}]{% 
	  \tt {\tiny [A$\langle$GM]}  
	 ARGUMENT: {'I understand your concern about B11s high effort cost, but C73 you added has effort 322 while B11 at 835 might provide better overall value despite the higher cost. We should focus on items that give us the best return rather than just filling the effort budget.'}\\ \tt \\ \tt PROPOSAL: {'B91', 'A30', 'C24', 'A19', 'A39', 'A47', 'A15', 'A56', 'B11'} 
	  } 
	   } 
	   } 
	 & \\ 
 

    \theutterance \stepcounter{utterance}  

    & \multicolumn{4}{p{0.6\linewidth}}{\cellcolor[rgb]{0.8,1,0.9}{%
	\makecell[{{p{\linewidth}}}]{% 
	  \tt {\tiny [A$\rangle$GM]}  
	 REFUSE: {'B91', 'A30', 'C24', 'A19', 'A39', 'A47', 'A15', 'A56', 'B11'}\\ \tt \\ \tt ARGUMENT: {'I appreciate your perspective on maximizing returns, but we need to consider that B11 uses over 25% of our total effort budget for just one item. This leaves us with limited flexibility to include other potentially valuable items. My approach of including C73 plus A75 gives us more diversification and better utilizes our remaining effort capacity.'}\\ \tt \\ \tt PROPOSAL: {'B91', 'A30', 'C24', 'A19', 'A39', 'A47', 'A75', 'C73'} 
	  } 
	   } 
	   } 
	 & & \\ 
 

    \theutterance \stepcounter{utterance}  

    & & \multicolumn{4}{p{0.6\linewidth}}{\cellcolor[rgb]{0.9,0.9,0.9}{%
	\makecell[{{p{\linewidth}}}]{% 
	  \tt {\tiny [GM$\rangle$B]}  
	 REFUSE: {'B91', 'A30', 'C24', 'A19', 'A39', 'A47', 'A15', 'A56', 'B11'}\\ \tt \\ \tt ARGUMENT: {'I appreciate your perspective on maximizing returns, but we need to consider that B11 uses over 25% of our total effort budget for just one item. This leaves us with limited flexibility to include other potentially valuable items. My approach of including C73 plus A75 gives us more diversification and better utilizes our remaining effort capacity.'}\\ \tt \\ \tt PROPOSAL: {'B91', 'A30', 'C24', 'A19', 'A39', 'A47', 'A75', 'C73'} 
	  } 
	   } 
	   } 
	 & \\ 
 

    \theutterance \stepcounter{utterance}  

    & & & \multicolumn{4}{p{0.6\linewidth}}{\cellcolor[rgb]{1,0.85,0.72}{%
	\makecell[{{p{\linewidth}}}]{% 
	  \tt {\tiny [GM$\langle$B]}  
	 ARGUMENT: {'Your diversification approach makes sense, but removing A15 and A56 might cost us valuable opportunities. These items have reasonable effort costs and could provide good returns. Let me suggest a compromise that keeps most of your selections while optimizing the remaining capacity.'}\\ \tt \\ \tt PROPOSAL: {'B91', 'A30', 'C24', 'A19', 'A39', 'A47', 'A75', 'A15'} 
	  } 
	   } 
	   } 
	  \\ 
 

    \theutterance \stepcounter{utterance}  

    & & \multicolumn{4}{p{0.6\linewidth}}{\cellcolor[rgb]{0.9,0.9,0.9}{%
	\makecell[{{p{\linewidth}}}]{% 
	  \tt {\tiny [A$\langle$GM]}  
	 ARGUMENT: {'Your diversification approach makes sense, but removing A15 and A56 might cost us valuable opportunities. These items have reasonable effort costs and could provide good returns. Let me suggest a compromise that keeps most of your selections while optimizing the remaining capacity.'}\\ \tt \\ \tt PROPOSAL: {'B91', 'A30', 'C24', 'A19', 'A39', 'A47', 'A75', 'A15'} 
	  } 
	   } 
	   } 
	 & \\ 
 

    \theutterance \stepcounter{utterance}  

    & \multicolumn{4}{p{0.6\linewidth}}{\cellcolor[rgb]{0.8,1,0.9}{%
	\makecell[{{p{\linewidth}}}]{% 
	  \tt {\tiny [A$\rangle$GM]}  
	 AGREE: {'B91', 'A30', 'C24', 'A19', 'A39', 'A47', 'A75', 'A15'}\\ \tt \\ \tt ARGUMENT: {'This is an excellent compromise that maintains the high-value core items we both identified while including A15 which does offer good value for its effort cost. This selection efficiently uses our effort budget and should provide strong returns for both of us.'} 
	  } 
	   } 
	   } 
	 & & \\ 
 

    \theutterance \stepcounter{utterance}  

    & & & \multicolumn{2}{p{0.3\linewidth}}{\cellcolor[rgb]{0.95,0.95,0.95}{%
	\makecell[{{p{\linewidth}}}]{% 
	  \tt {\tiny [GM$|$GM]}  
	 game successful 
	  } 
	   } 
	   } 
	 & & \\ 
 

    \theutterance \stepcounter{utterance}  

    & & & \multicolumn{2}{p{0.3\linewidth}}{\cellcolor[rgb]{0.95,0.95,0.95}{%
	\makecell[{{p{\linewidth}}}]{% 
	  \tt {\tiny [GM$|$GM]}  
	 end game 
	  } 
	   } 
	   } 
	 & & \\ 
 

\end{supertabular}
}

\end{document}
