
\documentclass{article}
\usepackage{colortbl}
\usepackage{makecell}
\usepackage{multirow}
\usepackage{supertabular}

\begin{document}

\newcounter{utterance}

\twocolumn

{ \footnotesize  \setcounter{utterance}{1}
\setlength{\tabcolsep}{0pt}
\begin{supertabular}{c@{$\;$}|p{.15\linewidth}@{}p{.15\linewidth}p{.15\linewidth}p{.15\linewidth}p{.15\linewidth}p{.15\linewidth}}

    \# & $\;$A & \multicolumn{4}{c}{Game Master} & $\;\:$B\\
    \hline 

    \theutterance \stepcounter{utterance}  

    & & \multicolumn{4}{p{0.6\linewidth}}{\cellcolor[rgb]{0.9,0.9,0.9}{%
	\makecell[{{p{\linewidth}}}]{% 
	  \tt {\tiny [A$\langle$GM]}  
	 You are playing a cooperative negotiation game in which you have to agree on how to divide a set of items among you and another player.\\ \tt \\ \tt Rules:\\ \tt (a) You and the other player are given a set of items. Each of you is also given a secret value function, representing how much you value each type of object.\\ \tt (b) You exchange messages with the other player to agree on who gets which items. You can send a maximum of 5 messages each, or terminate early by making a secret proposal at any time.\\ \tt (c) You are each asked to submit a secret proposal indicating the items you want formatted in square brackets as follows: "[Proposal: <number> <object name>, <number> <object name>, <...>]"\\ \tt (d) If your proposals are complementary, i.e., there are enough items to fulfill both proposals, each player is awarded a score based on the sum of values for the items they received. Otherwise, both of you get zero points.\\ \tt (e) Your goal is to maximize the sum of your score and the score of the other player.\\ \tt \\ \tt Let us start.\\ \tt \\ \tt The set of available items is:\\ \tt \\ \tt 1 pot, 3 magnets, 1 plate\\ \tt \\ \tt Your secret value function is:\\ \tt \\ \tt pot: 5, magnet: 0, plate: 5\\ \tt \\ \tt IMPORTANT: Your messages, unless it is the secret proposal, are directly transmitted to the other player, so do not include any response to the rules or text announcing your message. To make a secret proposal, use the indicated format. Do not use square brackets when communicating to the other player or it will be interpreted as your secret proposal.\\ \tt  
	  } 
	   } 
	   } 
	 & \\ 
 

    \theutterance \stepcounter{utterance}  

    & \multicolumn{4}{p{0.6\linewidth}}{\cellcolor[rgb]{0.8,1,0.9}{%
	\makecell[{{p{\linewidth}}}]{% 
	  \tt {\tiny [A$\rangle$GM]}  
	 Hi — we should split the items so each of us gets value and total sum is maximized. For me, pot and plate are valuable (I value pot=5, plate=5), magnets are worthless to me. How do you value them? If you also value pot or plate highly, maybe I can take magnets and one of pot/plate. What's your preference? 
	  } 
	   } 
	   } 
	 & & \\ 
 

    \theutterance \stepcounter{utterance}  

    & & & \multicolumn{2}{p{0.3\linewidth}}{\cellcolor[rgb]{0.95,0.95,0.95}{%
	\makecell[{{p{\linewidth}}}]{% 
	  \tt {\tiny [GM$|$GM]}  
	 continue 
	  } 
	   } 
	   } 
	 & & \\ 
 

    \theutterance \stepcounter{utterance}  

    & & & \multicolumn{2}{p{0.3\linewidth}}{\cellcolor[rgb]{0.95,0.95,0.95}{%
	\makecell[{{p{\linewidth}}}]{% 
	  \tt {\tiny [GM$|$GM]}  
	 Hi — we should split the items so each of us gets value and total sum is maximized. For me, pot and plate are valuable (I value pot=5, plate=5), magnets are worthless to me. How do you value them? If you also value pot or plate highly, maybe I can take magnets and one of pot/plate. What's your preference? 
	  } 
	   } 
	   } 
	 & & \\ 
 

    \theutterance \stepcounter{utterance}  

    & & \multicolumn{4}{p{0.6\linewidth}}{\cellcolor[rgb]{0.9,0.9,0.9}{%
	\makecell[{{p{\linewidth}}}]{% 
	  \tt {\tiny [GM$\rangle$B]}  
	 You are playing a cooperative negotiation game in which you have to agree on how to divide a set of items among you and another player.\\ \tt \\ \tt Rules:\\ \tt (a) You and the other player are given a set of items. Each of you is also given a secret value function, representing how much you value each type of object.\\ \tt (b) You exchange messages with the other player to agree on who gets which items. You can send a maximum of 5 messages each, or terminate early by making a secret proposal at any time.\\ \tt (c) You are each asked to submit a secret proposal indicating the items you want formatted in square brackets as follows: "[Proposal: <number> <object name>, <number> <object name>, <...>]"\\ \tt (d) If your proposals are complementary, i.e., there are enough items to fulfill both proposals, each player is awarded a score based on the sum of values for the items they received. Otherwise, both of you get zero points.\\ \tt (e) Your goal is to maximize the sum of your score and the score of the other player.\\ \tt \\ \tt Let us start.\\ \tt \\ \tt The set of available items is:\\ \tt \\ \tt 1 pot, 3 magnets, 1 plate\\ \tt \\ \tt Your secret value function is:\\ \tt \\ \tt pot: 4, magnet: 2, plate: 0\\ \tt \\ \tt IMPORTANT: Your messages, unless it is the secret proposal, are directly transmitted to the other player, so do not include any response to the rules or text announcing your message. To make a secret proposal, use the indicated format. Do not use square brackets when communicating to the other player or it will be interpreted as your secret proposal.\\ \tt \\ \tt \\ \tt Hi — we should split the items so each of us gets value and total sum is maximized. For me, pot and plate are valuable (I value pot=5, plate=5), magnets are worthless to me. How do you value them? If you also value pot or plate highly, maybe I can take magnets and one of pot/plate. What's your preference? 
	  } 
	   } 
	   } 
	 & \\ 
 

    \theutterance \stepcounter{utterance}  

    & & & \multicolumn{4}{p{0.6\linewidth}}{\cellcolor[rgb]{1,0.85,0.72}{%
	\makecell[{{p{\linewidth}}}]{% 
	  \tt {\tiny [GM$\langle$B]}  
	 Thanks — good idea to maximize total. I value pot at 4, each magnet at 2, plate at 0. So my total value for all items is 4 + 3*2 + 0 = 10.\\ \tt \\ \tt Given your values (pot=5, plate=5, magnets=0), your total for all items is 10 as well.\\ \tt \\ \tt To maximize combined value, we should give each item to the person who values it more. You value plate much more (5 vs my 0), and you value pot slightly more (5 vs my 4). I value magnets (2 each) and you value them 0.\\ \tt \\ \tt So the efficient division is:\\ \tt - You get pot and plate.\\ \tt - I get the 3 magnets.\\ \tt \\ \tt This yields scores: you 5+5=10, me 3*2=6, combined 16. Any other distribution gives a lower total (for example if you took magnets instead of one high-value item the total would drop).\\ \tt \\ \tt Does that division work for you? If you'd prefer a more balanced split of points rather than strictly maximizing total, tell me a trade (e.g., one magnet to you in exchange for something) and we can adjust. 
	  } 
	   } 
	   } 
	  \\ 
 

    \theutterance \stepcounter{utterance}  

    & & & \multicolumn{2}{p{0.3\linewidth}}{\cellcolor[rgb]{0.95,0.95,0.95}{%
	\makecell[{{p{\linewidth}}}]{% 
	  \tt {\tiny [GM$|$GM]}  
	 continue 
	  } 
	   } 
	   } 
	 & & \\ 
 

    \theutterance \stepcounter{utterance}  

    & & & \multicolumn{2}{p{0.3\linewidth}}{\cellcolor[rgb]{0.95,0.95,0.95}{%
	\makecell[{{p{\linewidth}}}]{% 
	  \tt {\tiny [GM$|$GM]}  
	 Thanks — good idea to maximize total. I value pot at 4, each magnet at 2, plate at 0. So my total value for all items is 4 + 3*2 + 0 = 10.\\ \tt \\ \tt Given your values (pot=5, plate=5, magnets=0), your total for all items is 10 as well.\\ \tt \\ \tt To maximize combined value, we should give each item to the person who values it more. You value plate much more (5 vs my 0), and you value pot slightly more (5 vs my 4). I value magnets (2 each) and you value them 0.\\ \tt \\ \tt So the efficient division is:\\ \tt - You get pot and plate.\\ \tt - I get the 3 magnets.\\ \tt \\ \tt This yields scores: you 5+5=10, me 3*2=6, combined 16. Any other distribution gives a lower total (for example if you took magnets instead of one high-value item the total would drop).\\ \tt \\ \tt Does that division work for you? If you'd prefer a more balanced split of points rather than strictly maximizing total, tell me a trade (e.g., one magnet to you in exchange for something) and we can adjust. 
	  } 
	   } 
	   } 
	 & & \\ 
 

    \theutterance \stepcounter{utterance}  

    & & \multicolumn{4}{p{0.6\linewidth}}{\cellcolor[rgb]{0.9,0.9,0.9}{%
	\makecell[{{p{\linewidth}}}]{% 
	  \tt {\tiny [A$\langle$GM]}  
	 Thanks — good idea to maximize total. I value pot at 4, each magnet at 2, plate at 0. So my total value for all items is 4 + 3*2 + 0 = 10.\\ \tt \\ \tt Given your values (pot=5, plate=5, magnets=0), your total for all items is 10 as well.\\ \tt \\ \tt To maximize combined value, we should give each item to the person who values it more. You value plate much more (5 vs my 0), and you value pot slightly more (5 vs my 4). I value magnets (2 each) and you value them 0.\\ \tt \\ \tt So the efficient division is:\\ \tt - You get pot and plate.\\ \tt - I get the 3 magnets.\\ \tt \\ \tt This yields scores: you 5+5=10, me 3*2=6, combined 16. Any other distribution gives a lower total (for example if you took magnets instead of one high-value item the total would drop).\\ \tt \\ \tt Does that division work for you? If you'd prefer a more balanced split of points rather than strictly maximizing total, tell me a trade (e.g., one magnet to you in exchange for something) and we can adjust. 
	  } 
	   } 
	   } 
	 & \\ 
 

    \theutterance \stepcounter{utterance}  

    & \multicolumn{4}{p{0.6\linewidth}}{\cellcolor[rgb]{0.8,1,0.9}{%
	\makecell[{{p{\linewidth}}}]{% 
	  \tt {\tiny [A$\rangle$GM]}  
	 That division works for me — giving you the pot and plate and you giving me the 3 magnets achieves a high total (you 10, me 6, combined 16). I'm happy to accept that.\\ \tt \\ \tt I'll make the secret proposal now: [Proposal: 1 pot, 3 magnets, 1 plate] 
	  } 
	   } 
	   } 
	 & & \\ 
 

    \theutterance \stepcounter{utterance}  

    & & & \multicolumn{2}{p{0.3\linewidth}}{\cellcolor[rgb]{0.95,0.95,0.95}{%
	\makecell[{{p{\linewidth}}}]{% 
	  \tt {\tiny [GM$|$GM]}  
	 proposal 
	  } 
	   } 
	   } 
	 & & \\ 
 

    \theutterance \stepcounter{utterance}  

    & & & \multicolumn{2}{p{0.3\linewidth}}{\cellcolor[rgb]{0.95,0.95,0.95}{%
	\makecell[{{p{\linewidth}}}]{% 
	  \tt {\tiny [GM$|$GM]}  
	 [1, 3, 1] 
	  } 
	   } 
	   } 
	 & & \\ 
 

    \theutterance \stepcounter{utterance}  

    & & \multicolumn{4}{p{0.6\linewidth}}{\cellcolor[rgb]{0.9,0.9,0.9}{%
	\makecell[{{p{\linewidth}}}]{% 
	  \tt {\tiny [GM$\rangle$B]}  
	 The other player has submitted a secret proposal. It is now your turn to submit a proposal of your own.\\ \tt  
	  } 
	   } 
	   } 
	 & \\ 
 

    \theutterance \stepcounter{utterance}  

    & & & \multicolumn{4}{p{0.6\linewidth}}{\cellcolor[rgb]{1,0.85,0.72}{%
	\makecell[{{p{\linewidth}}}]{% 
	  \tt {\tiny [GM$\langle$B]}  
	 [Proposal: 1 pot, 3 magnets, 1 plate] 
	  } 
	   } 
	   } 
	  \\ 
 

    \theutterance \stepcounter{utterance}  

    & & & \multicolumn{2}{p{0.3\linewidth}}{\cellcolor[rgb]{0.95,0.95,0.95}{%
	\makecell[{{p{\linewidth}}}]{% 
	  \tt {\tiny [GM$|$GM]}  
	 proposal 
	  } 
	   } 
	   } 
	 & & \\ 
 

    \theutterance \stepcounter{utterance}  

    & & & \multicolumn{2}{p{0.3\linewidth}}{\cellcolor[rgb]{0.95,0.95,0.95}{%
	\makecell[{{p{\linewidth}}}]{% 
	  \tt {\tiny [GM$|$GM]}  
	 [1, 3, 1] 
	  } 
	   } 
	   } 
	 & & \\ 
 

    \theutterance \stepcounter{utterance}  

    & & & \multicolumn{2}{p{0.3\linewidth}}{\cellcolor[rgb]{0.95,0.95,0.95}{%
	\makecell[{{p{\linewidth}}}]{% 
	  \tt {\tiny [GM$|$GM]}  
	 [[1, 3, 1], [1, 3, 1]] 
	  } 
	   } 
	   } 
	 & & \\ 
 

\end{supertabular}
}

\end{document}
