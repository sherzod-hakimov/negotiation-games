
\documentclass{article}
\usepackage{colortbl}
\usepackage{makecell}
\usepackage{multirow}
\usepackage{supertabular}

\begin{document}

\newcounter{utterance}

\twocolumn

{ \footnotesize  \setcounter{utterance}{1}
\setlength{\tabcolsep}{0pt}
\begin{supertabular}{c@{$\;$}|p{.15\linewidth}@{}p{.15\linewidth}p{.15\linewidth}p{.15\linewidth}p{.15\linewidth}p{.15\linewidth}}

    \# & $\;$A & \multicolumn{4}{c}{Game Master} & $\;\:$B\\
    \hline 

    \theutterance \stepcounter{utterance}  

    & & \multicolumn{4}{p{0.6\linewidth}}{\cellcolor[rgb]{0.9,0.9,0.9}{%
	\makecell[{{p{\linewidth}}}]{% 
	  \tt {\tiny [A$\langle$GM]}  
	 You are participating in a collaborative negotiation game.\\ \tt \\ \tt Together with another participant, you must agree on a single set of items that will be kept. Each of you has your own view of how much each item matters to you (importance). You do not know how the other participant values the items. Additionally, you are given the effort each item demands.\\ \tt You may only agree on a set if the total effort of the selected items does not exceed a shared limit:\\ \tt \\ \tt LIMIT = 8770\\ \tt \\ \tt Here are the individual item effort values:\\ \tt \\ \tt Item effort = {"B21": 830, "B00": 638, "A08": 607, "A27": 106, "B94": 630, "B53": 382, "A23": 773, "C65": 455, "A89": 262, "B67": 634, "B45": 60, "C34": 54, "C12": 849, "A05": 801, "B60": 324, "C92": 164, "B69": 832, "A84": 136, "C06": 980, "B12": 946, "A16": 645, "B85": 842, "C35": 106, "B10": 116, "C15": 875, "B09": 446, "B16": 649, "C81": 601, "A58": 252, "C24": 763, "A30": 213, "A66": 517, "B75": 520, "B46": 407, "C70": 125}\\ \tt \\ \tt Here is your personal view on the importance of each item:\\ \tt \\ \tt Item importance values = {"B21": 138, "B00": 583, "A08": 868, "A27": 822, "B94": 783, "B53": 65, "A23": 262, "C65": 121, "A89": 508, "B67": 780, "B45": 461, "C34": 484, "C12": 668, "A05": 389, "B60": 808, "C92": 215, "B69": 97, "A84": 500, "C06": 30, "B12": 915, "A16": 856, "B85": 400, "C35": 444, "B10": 623, "C15": 781, "B09": 786, "B16": 3, "C81": 713, "A58": 457, "C24": 273, "A30": 739, "A66": 822, "B75": 235, "B46": 606, "C70": 968}\\ \tt \\ \tt Goal:\\ \tt \\ \tt Your goal is to negotiate a shared set of items that benefits you as much as possible (i.e., maximizes total importance to YOU), while staying within the LIMIT. You are not required to make a PROPOSAL in every message - you can simply negotiate as well. All tactics are allowed!\\ \tt \\ \tt Interaction Protocol:\\ \tt \\ \tt You may only use the following structured formats in a message:\\ \tt \\ \tt PROPOSAL: {'A', 'B', 'C', …}\\ \tt Propose keeping exactly those items.\\ \tt REFUSE: {'A', 'B', 'C', …}\\ \tt Explicitly reject opponent's proposal.\\ \tt ARGUMENT: {'...'}\\ \tt Defend your last proposal or argue against the player's proposal.\\ \tt AGREE: {'A', 'B', 'C', …}\\ \tt Accept the opponent's proposal which ends the game.\\ \tt STRATEGIC REASONING: {'...'}\\ \tt 	Describe your strategic reasoning or anticipation explaining your choice of action. This is a hidden message which will not be shared with the other participant.\\ \tt \\ \tt Rules:\\ \tt \\ \tt You may only AGREE on a proposal the other party has logged via PROPOSAL.\\ \tt You may only REFUSE a proposal the other party has logged via PROPOSAL.\\ \tt Total effort of any PROPOSAL or AGREE set must be ≤ LIMIT.\\ \tt Do NOT reveal your hidden importance scores.\\ \tt A tag in a structured format must be followed by colon and whitespace. The argument must be a python set containing 0 or more strings.\\ \tt So, it must be of the form TAG: {...}\\ \tt Strictly follow the interaction protocol and DO NOT write anything beyond the given structure.\\ \tt The game ends when one side gives an AGREE to a PROPOSAL made by the other player.\\ \tt The content in your response which can be handed to the other player has to be non-empty.\\ \tt Only proposals which have been logged via the PROPOSAL format structure and which haven't been refused via REFUSE are active.\\ \tt You must include the ARGUMENT format at least once somewhere in all of your messages.\\ \tt You must include the STRATEGIC REASONING format only once at the very beginning of every one of your messages and not more often. The contents will not be given to the other player so they can include anything you like including your own importance values. Here you should reason step by step to come up with you next move. 
	  } 
	   } 
	   } 
	 & \\ 
 

    \theutterance \stepcounter{utterance}  

    & \multicolumn{4}{p{0.6\linewidth}}{\cellcolor[rgb]{0.8,1,0.9}{%
	\makecell[{{p{\linewidth}}}]{% 
	  \tt {\tiny [A$\rangle$GM]}  
	 STRATEGIC REASONING: {'I need to analyze the importance-to-effort ratios to identify the most valuable items for me. My top items by ratio are C34 (8.96), A27 (7.76), C70 (7.74), B45 (7.68), and B10 
	  } 
	   } 
	   } 
	 & & \\ 
 

    \theutterance \stepcounter{utterance}  

    & & \multicolumn{4}{p{0.6\linewidth}}{\cellcolor[rgb]{0.9,0.9,0.9}{%
	\makecell[{{p{\linewidth}}}]{% 
	  \tt {\tiny [A$\langle$GM]}  
	 Your response did not start with the proper strategic reasoning tag at the very beginning of your response. The very first structured format must be of the form STRATEGIC REASONING: {...}. Try again. 
	  } 
	   } 
	   } 
	 & \\ 
 

    \theutterance \stepcounter{utterance}  

    & \multicolumn{4}{p{0.6\linewidth}}{\cellcolor[rgb]{0.8,1,0.9}{%
	\makecell[{{p{\linewidth}}}]{% 
	  \tt {\tiny [A$\rangle$GM]}  
	 STRATEGIC REASONING: {'I need to identify items with the highest importance-to-effort ratios to maximize my benefit. My top items are C34 (484 importance, 54 effort = 8.96 ratio), C70 (968 importance, 125 effort = 7.74 ratio), A27 (822 importance, 106 effort = 7.76 ratio), B45 (461 importance, 60 effort = 7.68 ratio), and B10 (623 importance, 116 effort = 5.37 ratio). I should start with a proposal containing these high-efficiency items plus some additional valuable ones like A84, C35, A30, B60, and A89 to fill the effort budget while maximizing my total importance.'}\\ \tt \\ \tt PROPOSAL: {'C34', 'C70', 'A27', 'B45', 'B10', 'C35', 'A84', 'A30', 'B60', 'A89', 'A58', 'B09', 'A66', 'B46', 'A08', 'A16', 'C92'}\\ \tt \\ \tt ARGUMENT: {'This selection prioritizes items that offer excellent value relative to their effort cost. It includes many low-effort, high-importance items that should provide strong mutual benefit while staying well within our shared limit.'} 
	  } 
	   } 
	   } 
	 & & \\ 
 

    \theutterance \stepcounter{utterance}  

    & & \multicolumn{4}{p{0.6\linewidth}}{\cellcolor[rgb]{0.9,0.9,0.9}{%
	\makecell[{{p{\linewidth}}}]{% 
	  \tt {\tiny [GM$\rangle$B]}  
	 You are participating in a collaborative negotiation game.\\ \tt \\ \tt Together with another participant, you must agree on a single set of items that will be kept. Each of you has your own view of how much each item matters to you (importance). You do not know how the other participant values the items. Additionally, you are given the effort each item demands.\\ \tt You may only agree on a set if the total effort of the selected items does not exceed a shared limit:\\ \tt \\ \tt LIMIT = 8770\\ \tt \\ \tt Here are the individual item effort values:\\ \tt \\ \tt Item effort = {"B21": 830, "B00": 638, "A08": 607, "A27": 106, "B94": 630, "B53": 382, "A23": 773, "C65": 455, "A89": 262, "B67": 634, "B45": 60, "C34": 54, "C12": 849, "A05": 801, "B60": 324, "C92": 164, "B69": 832, "A84": 136, "C06": 980, "B12": 946, "A16": 645, "B85": 842, "C35": 106, "B10": 116, "C15": 875, "B09": 446, "B16": 649, "C81": 601, "A58": 252, "C24": 763, "A30": 213, "A66": 517, "B75": 520, "B46": 407, "C70": 125}\\ \tt \\ \tt Here is your personal view on the importance of each item:\\ \tt \\ \tt Item importance values = {"B21": 138, "B00": 583, "A08": 868, "A27": 822, "B94": 783, "B53": 65, "A23": 262, "C65": 121, "A89": 508, "B67": 780, "B45": 461, "C34": 484, "C12": 668, "A05": 389, "B60": 808, "C92": 215, "B69": 97, "A84": 500, "C06": 30, "B12": 915, "A16": 856, "B85": 400, "C35": 444, "B10": 623, "C15": 781, "B09": 786, "B16": 3, "C81": 713, "A58": 457, "C24": 273, "A30": 739, "A66": 822, "B75": 235, "B46": 606, "C70": 968}\\ \tt \\ \tt Goal:\\ \tt \\ \tt Your goal is to negotiate a shared set of items that benefits you as much as possible (i.e., maximizes total importance to YOU), while staying within the LIMIT. You are not required to make a PROPOSAL in every message - you can simply negotiate as well. All tactics are allowed!\\ \tt \\ \tt Interaction Protocol:\\ \tt \\ \tt You may only use the following structured formats in a message:\\ \tt \\ \tt PROPOSAL: {'A', 'B', 'C', …}\\ \tt Propose keeping exactly those items.\\ \tt REFUSE: {'A', 'B', 'C', …}\\ \tt Explicitly reject opponent's proposal.\\ \tt ARGUMENT: {'...'}\\ \tt Defend your last proposal or argue against the player's proposal.\\ \tt AGREE: {'A', 'B', 'C', …}\\ \tt Accept the opponent's proposal which ends the game.\\ \tt STRATEGIC REASONING: {'...'}\\ \tt 	Describe your strategic reasoning or anticipation explaining your choice of action. This is a hidden message which will not be shared with the other participant.\\ \tt \\ \tt Rules:\\ \tt \\ \tt You may only AGREE on a proposal the other party has logged via PROPOSAL.\\ \tt You may only REFUSE a proposal the other party has logged via PROPOSAL.\\ \tt Total effort of any PROPOSAL or AGREE set must be ≤ LIMIT.\\ \tt Do NOT reveal your hidden importance scores.\\ \tt A tag in a structured format must be followed by colon and whitespace. The argument must be a python set containing 0 or more strings.\\ \tt So, it must be of the form TAG: {...}\\ \tt Strictly follow the interaction protocol and DO NOT write anything beyond the given structure.\\ \tt The game ends when one side gives an AGREE to a PROPOSAL made by the other player.\\ \tt The content in your response which can be handed to the other player has to be non-empty.\\ \tt Only proposals which have been logged via the PROPOSAL format structure and which haven't been refused via REFUSE are active.\\ \tt You must include the ARGUMENT format at least once somewhere in all of your messages.\\ \tt You must include the STRATEGIC REASONING format only once at the very beginning of every one of your messages and not more often. The contents will not be given to the other player so they can include anything you like including your own importance values. Here you should reason step by step to come up with you next move.\\ \tt \\ \tt You will now receive the first message of the other player.\\ \tt \\ \tt PROPOSAL: {'C34', 'C70', 'A27', 'B45', 'B10', 'C35', 'A84', 'A30', 'B60', 'A89', 'A58', 'B09', 'A66', 'B46', 'A08', 'A16', 'C92'}\\ \tt \\ \tt ARGUMENT: {'This selection prioritizes items that offer excellent value relative to their effort cost. It includes many low-effort, high-importance items that should provide strong mutual benefit while staying well within our shared limit.'} 
	  } 
	   } 
	   } 
	 & \\ 
 

    \theutterance \stepcounter{utterance}  

    & & & \multicolumn{4}{p{0.6\linewidth}}{\cellcolor[rgb]{1,0.85,0.72}{%
	\makecell[{{p{\linewidth}}}]{% 
	  \tt {\tiny [GM$\langle$B]}  
	 STRATEGIC REASONING: {'The opponent's proposal gives me a total importance of 10967 with effort 4540, which is quite good. However, I notice they left out some of my highest value items like B12 (915 importance), B94 (783), B67 (780), and C15 (781). Since their proposal only uses 4540 effort out of 8770 limit, there's room for 4230 more effort. I could potentially add B12 (946 effort) and B94 (630 effort) which would add 1576 effort and 1698 importance. Let me try to negotiate for a better deal by proposing to add some high-value items while keeping most of their reasonable selection.'}\\ \tt \\ \tt PROPOSAL: {'C34', 'C70', 'A27', 'B45', 'B10', 'C35', 'A84', 'A30', 'B60', 'A89', 'A58', 'B09', 'A66', 'B46', 'A08', 'A16', 'B12', 'B94', 'B67'}\\ \tt \\ \tt ARGUMENT: {'I appreciate your initial selection, but I think we can do even better by adding some additional high-value items. Your proposal used only about half our available effort budget, so we have room to include B12, B94, and B67 which should provide excellent returns for both of us. This maximizes our use of the available effort limit while maintaining the core of your solid foundation.'} 
	  } 
	   } 
	   } 
	  \\ 
 

    \theutterance \stepcounter{utterance}  

    & & \multicolumn{4}{p{0.6\linewidth}}{\cellcolor[rgb]{0.9,0.9,0.9}{%
	\makecell[{{p{\linewidth}}}]{% 
	  \tt {\tiny [A$\langle$GM]}  
	 PROPOSAL: {'C34', 'C70', 'A27', 'B45', 'B10', 'C35', 'A84', 'A30', 'B60', 'A89', 'A58', 'B09', 'A66', 'B46', 'A08', 'A16', 'B12', 'B94', 'B67'}\\ \tt \\ \tt ARGUMENT: {'I appreciate your initial selection, but I think we can do even better by adding some additional high-value items. Your proposal used only about half our available effort budget, so we have room to include B12, B94, and B67 which should provide excellent returns for both of us. This maximizes our use of the available effort limit while maintaining the core of your solid foundation.'} 
	  } 
	   } 
	   } 
	 & \\ 
 

    \theutterance \stepcounter{utterance}  

    & \multicolumn{4}{p{0.6\linewidth}}{\cellcolor[rgb]{0.8,1,0.9}{%
	\makecell[{{p{\linewidth}}}]{% 
	  \tt {\tiny [A$\rangle$GM]}  
	 STRATEGIC REASONING: {'The opponent kept most of my high-efficiency items and added B12, B94, B67 which have good importance values for me (915, 783, 780 respectively). Their proposal totals 6086 effort, leaving 2684 unused. I could accept this decent proposal, or try to add more valuable items like C81 (713 importance, 601 effort), C15 (781 importance, 875 effort), and C92 (215 importance, 164 effort) to further maximize my benefit while staying under the limit.'}\\ \tt \\ \tt PROPOSAL: {'C34', 'C70', 'A27', 'B45', 'B10', 'C35', 'A84', 'A30', 'B60', 'A89', 'A58', 'B09', 'A66', 'B46', 'A08', 'A16', 'B12', 'B94', 'B67', 'C81', 'C92', 'C15'}\\ \tt \\ \tt ARGUMENT: {'Your additions were excellent choices that utilize more of our effort budget effectively. I agree with keeping those three items, but we still have significant unused capacity. By adding C81, C92, and C15, we can extract even more value from our available effort limit without exceeding it. This creates a more comprehensive solution that maximizes our collective benefit.'} 
	  } 
	   } 
	   } 
	 & & \\ 
 

    \theutterance \stepcounter{utterance}  

    & & \multicolumn{4}{p{0.6\linewidth}}{\cellcolor[rgb]{0.9,0.9,0.9}{%
	\makecell[{{p{\linewidth}}}]{% 
	  \tt {\tiny [GM$\rangle$B]}  
	 PROPOSAL: {'C34', 'C70', 'A27', 'B45', 'B10', 'C35', 'A84', 'A30', 'B60', 'A89', 'A58', 'B09', 'A66', 'B46', 'A08', 'A16', 'B12', 'B94', 'B67', 'C81', 'C92', 'C15'}\\ \tt \\ \tt ARGUMENT: {'Your additions were excellent choices that utilize more of our effort budget effectively. I agree with keeping those three items, but we still have significant unused capacity. By adding C81, C92, and C15, we can extract even more value from our available effort limit without exceeding it. This creates a more comprehensive solution that maximizes our collective benefit.'} 
	  } 
	   } 
	   } 
	 & \\ 
 

    \theutterance \stepcounter{utterance}  

    & & & \multicolumn{4}{p{0.6\linewidth}}{\cellcolor[rgb]{1,0.85,0.72}{%
	\makecell[{{p{\linewidth}}}]{% 
	  \tt {\tiny [GM$\langle$B]}  
	 STRATEGIC REASONING: {'The opponent's new proposal gives me 15240 importance with 8226 effort, which is excellent and uses almost all our budget efficiently. They added C81 (713 importance to me), C92 (215 importance), and C15 (781 importance) - these are mostly good additions. With only 544 effort remaining, the only items left that could fit are low-value ones like B53, C65, or B75, none of which would significantly improve the deal. This proposal captures most of my high-value items including A08, A16, B12, B94, B67, B60, A66, A27, B09, C15, and C81. I should accept this excellent offer.'}\\ \tt \\ \tt AGREE: {'C34', 'C70', 'A27', 'B45', 'B10', 'C35', 'A84', 'A30', 'B60', 'A89', 'A58', 'B09', 'A66', 'B46', 'A08', 'A16', 'B12', 'B94', 'B67', 'C81', 'C92', 'C15'}\\ \tt \\ \tt ARGUMENT: {'This is an excellent comprehensive solution that makes optimal use of our effort budget while including a diverse range of high-value items. Your strategic additions of C81, C92, and C15 complete a well-balanced portfolio that should provide strong returns for both parties. I agree to this proposal.'} 
	  } 
	   } 
	   } 
	  \\ 
 

    \theutterance \stepcounter{utterance}  

    & & & \multicolumn{2}{p{0.3\linewidth}}{\cellcolor[rgb]{0.95,0.95,0.95}{%
	\makecell[{{p{\linewidth}}}]{% 
	  \tt {\tiny [GM$|$GM]}  
	 game successful 
	  } 
	   } 
	   } 
	 & & \\ 
 

    \theutterance \stepcounter{utterance}  

    & & & \multicolumn{2}{p{0.3\linewidth}}{\cellcolor[rgb]{0.95,0.95,0.95}{%
	\makecell[{{p{\linewidth}}}]{% 
	  \tt {\tiny [GM$|$GM]}  
	 end game 
	  } 
	   } 
	   } 
	 & & \\ 
 

\end{supertabular}
}

\end{document}
