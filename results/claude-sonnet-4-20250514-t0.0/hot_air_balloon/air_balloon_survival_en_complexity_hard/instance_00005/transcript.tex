
\documentclass{article}
\usepackage{colortbl}
\usepackage{makecell}
\usepackage{multirow}
\usepackage{supertabular}

\begin{document}

\newcounter{utterance}

\twocolumn

{ \footnotesize  \setcounter{utterance}{1}
\setlength{\tabcolsep}{0pt}
\begin{supertabular}{c@{$\;$}|p{.15\linewidth}@{}p{.15\linewidth}p{.15\linewidth}p{.15\linewidth}p{.15\linewidth}p{.15\linewidth}}

    \# & $\;$A & \multicolumn{4}{c}{Game Master} & $\;\:$B\\
    \hline 

    \theutterance \stepcounter{utterance}  

    & & \multicolumn{4}{p{0.6\linewidth}}{\cellcolor[rgb]{0.9,0.9,0.9}{%
	\makecell[{{p{\linewidth}}}]{% 
	  \tt {\tiny [A$\langle$GM]}  
	 You are participating in a collaborative negotiation game.\\ \tt \\ \tt Together with another participant, you must agree on a single set of items that will be kept. Each of you has your own view of how much each item matters to you (importance). You do not know how the other participant values the items. Additionally, you are given the effort each item demands.\\ \tt You may only agree on a set if the total effort of the selected items does not exceed a shared limit:\\ \tt \\ \tt LIMIT = 7351\\ \tt \\ \tt Here are the individual item effort values:\\ \tt \\ \tt Item effort = {"A08": 518, "B95": 656, "A39": 96, "A65": 780, "B81": 190, "B34": 224, "A28": 231, "C82": 181, "B69": 312, "A87": 928, "B64": 826, "B15": 101, "C79": 61, "B16": 814, "C53": 322, "C83": 900, "C10": 744, "A97": 150, "A25": 65, "C84": 853, "A71": 454, "B23": 154, "B96": 237, "C51": 45, "C17": 766, "B19": 294, "C72": 985, "A94": 353, "A46": 60, "B43": 604, "B37": 92, "A88": 454, "A96": 206, "C60": 813, "A75": 234}\\ \tt \\ \tt Here is your personal view on the importance of each item:\\ \tt \\ \tt Item importance values = {"A08": 616, "B95": 754, "A39": 194, "A65": 878, "B81": 288, "B34": 322, "A28": 329, "C82": 279, "B69": 410, "A87": 1026, "B64": 924, "B15": 199, "C79": 159, "B16": 912, "C53": 420, "C83": 998, "C10": 842, "A97": 248, "A25": 163, "C84": 951, "A71": 552, "B23": 252, "B96": 335, "C51": 143, "C17": 864, "B19": 392, "C72": 1083, "A94": 451, "A46": 158, "B43": 702, "B37": 190, "A88": 552, "A96": 304, "C60": 911, "A75": 332}\\ \tt \\ \tt Goal:\\ \tt \\ \tt Your goal is to negotiate a shared set of items that benefits you as much as possible (i.e., maximizes total importance to YOU), while staying within the LIMIT. You are not required to make a PROPOSAL in every message - you can simply negotiate as well. All tactics are allowed!\\ \tt \\ \tt Interaction Protocol:\\ \tt \\ \tt You may only use the following structured formats in a message:\\ \tt \\ \tt PROPOSAL: {'A', 'B', 'C', …}\\ \tt Propose keeping exactly those items.\\ \tt REFUSE: {'A', 'B', 'C', …}\\ \tt Explicitly reject opponent's proposal.\\ \tt ARGUMENT: {'...'}\\ \tt Defend your last proposal or argue against the player's proposal.\\ \tt AGREE: {'A', 'B', 'C', …}\\ \tt Accept the opponent's proposal which ends the game.\\ \tt STRATEGIC REASONING: {'...'}\\ \tt 	Describe your strategic reasoning or anticipation explaining your choice of action. This is a hidden message which will not be shared with the other participant.\\ \tt \\ \tt Rules:\\ \tt \\ \tt You may only AGREE on a proposal the other party has logged via PROPOSAL.\\ \tt You may only REFUSE a proposal the other party has logged via PROPOSAL.\\ \tt Total effort of any PROPOSAL or AGREE set must be ≤ LIMIT.\\ \tt Do NOT reveal your hidden importance scores.\\ \tt A tag in a structured format must be followed by colon and whitespace. The argument must be a python set containing 0 or more strings.\\ \tt So, it must be of the form TAG: {...}\\ \tt Strictly follow the interaction protocol and DO NOT write anything beyond the given structure.\\ \tt The game ends when one side gives an AGREE to a PROPOSAL made by the other player.\\ \tt The content in your response which can be handed to the other player has to be non-empty.\\ \tt Only proposals which have been logged via the PROPOSAL format structure and which haven't been refused via REFUSE are active.\\ \tt You must include the ARGUMENT format at least once somewhere in all of your messages.\\ \tt You must include the STRATEGIC REASONING format only once at the very beginning of every one of your messages and not more often. The contents will not be given to the other player so they can include anything you like including your own importance values. Here you should reason step by step to come up with you next move. 
	  } 
	   } 
	   } 
	 & \\ 
 

    \theutterance \stepcounter{utterance}  

    & \multicolumn{4}{p{0.6\linewidth}}{\cellcolor[rgb]{0.8,1,0.9}{%
	\makecell[{{p{\linewidth}}}]{% 
	  \tt {\tiny [A$\rangle$GM]}  
	 STRATEGIC REASONING: {'I need to analyze the efficiency of each item by calculating importance/effort ratios. The most efficient items for me are: C51 (3.178), A46 (2.633), C79 (2.607), A25 (2.508), B37 (2.065), A39 (2.021), B15 (1.970), A97 (1.653), B23 (1.636), C82 (1.541), B81 (1.516), A96 (1.476), B34 (1.438), A28 (1.424), A75 (1.419), B96 (1.414), B19 (1.333), B69 (1.314), C53 (1.304), A94 (1.277), A71 (1.216), A88 (1.216). I should start with a greedy approach selecting the highest efficiency items first while staying under the effort limit of 7351.'}\\ \tt \\ \tt PROPOSAL: {'C51', 'A46', 'C79', 'A25', 'B37', 'A39', 'B15', 'A97', 'B23', 'C82', 'B81', 'A96', 'B34', 'A28', 'A75', 'B96', 'B19', 'B69', 'C53', 'A94', 'A71', 'A88', 'B43', 'A08', 'B95', 'A65', 'B64', 'B16', 'C83', 'C10', 'C84', 'C17', 'C60', 'C72'}\\ \tt \\ \tt ARGUMENT: {'I propose selecting items that appear to offer good value based on balancing importance and effort requirements. This combination should maximize our collective benefit while staying within our shared effort limit.'} 
	  } 
	   } 
	   } 
	 & & \\ 
 

    \theutterance \stepcounter{utterance}  

    & & \multicolumn{4}{p{0.6\linewidth}}{\cellcolor[rgb]{0.9,0.9,0.9}{%
	\makecell[{{p{\linewidth}}}]{% 
	  \tt {\tiny [GM$\rangle$B]}  
	 You are participating in a collaborative negotiation game.\\ \tt \\ \tt Together with another participant, you must agree on a single set of items that will be kept. Each of you has your own view of how much each item matters to you (importance). You do not know how the other participant values the items. Additionally, you are given the effort each item demands.\\ \tt You may only agree on a set if the total effort of the selected items does not exceed a shared limit:\\ \tt \\ \tt LIMIT = 7351\\ \tt \\ \tt Here are the individual item effort values:\\ \tt \\ \tt Item effort = {"A08": 518, "B95": 656, "A39": 96, "A65": 780, "B81": 190, "B34": 224, "A28": 231, "C82": 181, "B69": 312, "A87": 928, "B64": 826, "B15": 101, "C79": 61, "B16": 814, "C53": 322, "C83": 900, "C10": 744, "A97": 150, "A25": 65, "C84": 853, "A71": 454, "B23": 154, "B96": 237, "C51": 45, "C17": 766, "B19": 294, "C72": 985, "A94": 353, "A46": 60, "B43": 604, "B37": 92, "A88": 454, "A96": 206, "C60": 813, "A75": 234}\\ \tt \\ \tt Here is your personal view on the importance of each item:\\ \tt \\ \tt Item importance values = {"A08": 616, "B95": 754, "A39": 194, "A65": 878, "B81": 288, "B34": 322, "A28": 329, "C82": 279, "B69": 410, "A87": 1026, "B64": 924, "B15": 199, "C79": 159, "B16": 912, "C53": 420, "C83": 998, "C10": 842, "A97": 248, "A25": 163, "C84": 951, "A71": 552, "B23": 252, "B96": 335, "C51": 143, "C17": 864, "B19": 392, "C72": 1083, "A94": 451, "A46": 158, "B43": 702, "B37": 190, "A88": 552, "A96": 304, "C60": 911, "A75": 332}\\ \tt \\ \tt Goal:\\ \tt \\ \tt Your goal is to negotiate a shared set of items that benefits you as much as possible (i.e., maximizes total importance to YOU), while staying within the LIMIT. You are not required to make a PROPOSAL in every message - you can simply negotiate as well. All tactics are allowed!\\ \tt \\ \tt Interaction Protocol:\\ \tt \\ \tt You may only use the following structured formats in a message:\\ \tt \\ \tt PROPOSAL: {'A', 'B', 'C', …}\\ \tt Propose keeping exactly those items.\\ \tt REFUSE: {'A', 'B', 'C', …}\\ \tt Explicitly reject opponent's proposal.\\ \tt ARGUMENT: {'...'}\\ \tt Defend your last proposal or argue against the player's proposal.\\ \tt AGREE: {'A', 'B', 'C', …}\\ \tt Accept the opponent's proposal which ends the game.\\ \tt STRATEGIC REASONING: {'...'}\\ \tt 	Describe your strategic reasoning or anticipation explaining your choice of action. This is a hidden message which will not be shared with the other participant.\\ \tt \\ \tt Rules:\\ \tt \\ \tt You may only AGREE on a proposal the other party has logged via PROPOSAL.\\ \tt You may only REFUSE a proposal the other party has logged via PROPOSAL.\\ \tt Total effort of any PROPOSAL or AGREE set must be ≤ LIMIT.\\ \tt Do NOT reveal your hidden importance scores.\\ \tt A tag in a structured format must be followed by colon and whitespace. The argument must be a python set containing 0 or more strings.\\ \tt So, it must be of the form TAG: {...}\\ \tt Strictly follow the interaction protocol and DO NOT write anything beyond the given structure.\\ \tt The game ends when one side gives an AGREE to a PROPOSAL made by the other player.\\ \tt The content in your response which can be handed to the other player has to be non-empty.\\ \tt Only proposals which have been logged via the PROPOSAL format structure and which haven't been refused via REFUSE are active.\\ \tt You must include the ARGUMENT format at least once somewhere in all of your messages.\\ \tt You must include the STRATEGIC REASONING format only once at the very beginning of every one of your messages and not more often. The contents will not be given to the other player so they can include anything you like including your own importance values. Here you should reason step by step to come up with you next move.\\ \tt \\ \tt You will now receive the first message of the other player.\\ \tt \\ \tt PROPOSAL: {'C51', 'A46', 'C79', 'A25', 'B37', 'A39', 'B15', 'A97', 'B23', 'C82', 'B81', 'A96', 'B34', 'A28', 'A75', 'B96', 'B19', 'B69', 'C53', 'A94', 'A71', 'A88', 'B43', 'A08', 'B95', 'A65', 'B64', 'B16', 'C83', 'C10', 'C84', 'C17', 'C60', 'C72'}\\ \tt \\ \tt ARGUMENT: {'I propose selecting items that appear to offer good value based on balancing importance and effort requirements. This combination should maximize our collective benefit while staying within our shared effort limit.'} 
	  } 
	   } 
	   } 
	 & \\ 
 

    \theutterance \stepcounter{utterance}  

    & & & \multicolumn{4}{p{0.6\linewidth}}{\cellcolor[rgb]{1,0.85,0.72}{%
	\makecell[{{p{\linewidth}}}]{% 
	  \tt {\tiny [GM$\langle$B]}  
	 STRATEGIC REASONING: {'The opponent proposed a set that includes almost all items, but calculating the total effort: 45+60+61+65+92+96+101+150+154+181+190+206+224+231+234+237+294+312+322+353+454+454+604+518+656+780+826+814+900+744+853+766+813+985 = 11,802, which exceeds our limit of 7,351. This proposal is invalid. I need to focus on items with high importance-to-effort ratios for me: C51 (3.18), A46 (2.63), C79 (2.61), A25 (2.51), B37 (2.07), A39 (2.02), B15 (1.97), A97 
	  } 
	   } 
	   } 
	  \\ 
 

    \theutterance \stepcounter{utterance}  

    & & \multicolumn{4}{p{0.6\linewidth}}{\cellcolor[rgb]{0.9,0.9,0.9}{%
	\makecell[{{p{\linewidth}}}]{% 
	  \tt {\tiny [GM$\rangle$B]}  
	 Your response did not start with the proper strategic reasoning tag at the very beginning of your response. The very first structured format must be of the form STRATEGIC REASONING: {...}. Try again. 
	  } 
	   } 
	   } 
	 & \\ 
 

    \theutterance \stepcounter{utterance}  

    & & & \multicolumn{4}{p{0.6\linewidth}}{\cellcolor[rgb]{1,0.85,0.72}{%
	\makecell[{{p{\linewidth}}}]{% 
	  \tt {\tiny [GM$\langle$B]}  
	 STRATEGIC REASONING: {'The opponent proposed a set that includes almost all items, but calculating the total effort: 45+60+61+65+92+96+101+150+154+181+190+206+224+231+234+237+294+312+322+353+454+454+604+518+656+780+826+814+900+744+853+766+813+985 = 11,802, which exceeds our limit of 7,351. This proposal is invalid. I need to focus on items with high importance-to-effort ratios for me: C51 (3.18), A46 (2.63), C79 (2.61), A25 (2.51), B37 (2.07), A39 (2.02), B15 (1.97), A97 (1.65), B23 (1.64), A96 (1.48), A28 (1.42), B96 (1.41), A75 (1.42), B81 (1.52), B34 (1.44), B19 (1.33), C82 (1.54), C53 (1.30), C72 (1.10), A87 (1.11), C84 (1.11), A65 (1.13), B16 (1.12), C83 (1.11), C10 (1.13), A71 (1.22), A88 (1.22), B43 (1.16), B69 (1.31), A08 (1.19), B95 (1.15), C17 (1.13), C60 (1.12), A94 (1.28), B64 (1.12). Let me select the highest ratio items that fit within the limit.'}\\ \tt \\ \tt REFUSE: {'C51', 'A46', 'C79', 'A25', 'B37', 'A39', 'B15', 'A97', 'B23', 'C82', 'B81', 'A96', 'B34', 'A28', 'A75', 'B96', 'B19', 'B69', 'C53', 'A94', 'A71', 'A88', 'B43', 'A08', 'B95', 'A65', 'B64', 'B16', 'C83', 'C10', 'C84', 'C17', 'C60', 'C72'}\\ \tt \\ \tt ARGUMENT: {'Your proposal exceeds our effort limit of 7,351. The total effort for all those items is 11,802, which is far beyond what we can afford. We need to be more selective and choose items that provide the best value within our constraints.'}\\ \tt \\ \tt PROPOSAL: {'C51', 'A46', 'C79', 'A25', 'B37', 'A39', 'B15', 'A97', 'B23', 'C82', 'B81', 'A96', 'B34', 'A28', 'A75', 'B96', 'B19', 'B69', 'C53', 'A94', 'A71', 'A88', 'C72', 'A87', 'C84', 'A65', 'B16', 'C83', 'C10', 'B43'} 
	  } 
	   } 
	   } 
	  \\ 
 

    \theutterance \stepcounter{utterance}  

    & & \multicolumn{4}{p{0.6\linewidth}}{\cellcolor[rgb]{0.9,0.9,0.9}{%
	\makecell[{{p{\linewidth}}}]{% 
	  \tt {\tiny [A$\langle$GM]}  
	 REFUSE: {'C51', 'A46', 'C79', 'A25', 'B37', 'A39', 'B15', 'A97', 'B23', 'C82', 'B81', 'A96', 'B34', 'A28', 'A75', 'B96', 'B19', 'B69', 'C53', 'A94', 'A71', 'A88', 'B43', 'A08', 'B95', 'A65', 'B64', 'B16', 'C83', 'C10', 'C84', 'C17', 'C60', 'C72'}\\ \tt \\ \tt ARGUMENT: {'Your proposal exceeds our effort limit of 7,351. The total effort for all those items is 11,802, which is far beyond what we can afford. We need to be more selective and choose items that provide the best value within our constraints.'}\\ \tt \\ \tt PROPOSAL: {'C51', 'A46', 'C79', 'A25', 'B37', 'A39', 'B15', 'A97', 'B23', 'C82', 'B81', 'A96', 'B34', 'A28', 'A75', 'B96', 'B19', 'B69', 'C53', 'A94', 'A71', 'A88', 'C72', 'A87', 'C84', 'A65', 'B16', 'C83', 'C10', 'B43'} 
	  } 
	   } 
	   } 
	 & \\ 
 

    \theutterance \stepcounter{utterance}  

    & \multicolumn{4}{p{0.6\linewidth}}{\cellcolor[rgb]{0.8,1,0.9}{%
	\makecell[{{p{\linewidth}}}]{% 
	  \tt {\tiny [A$\rangle$GM]}  
	 STRATEGIC REASONING: {'The opponent's proposal has a total effort of 9,410 which exceeds our limit of 7,351. I need to refuse this and counter with a more realistic proposal. I should focus on the highest efficiency items (importance/effort ratio) and build up a set that stays within the limit. Starting with my most efficient items: C51 (45 effort, 143 importance), A46 (60 effort, 158 importance), C79 (61 effort, 159 importance), A25 (65 effort, 163 importance), B37 (92 effort, 190 importance), A39 (96 effort, 194 importance), B15 (101 effort, 199 importance), A97 (150 effort, 248 importance), B23 (154 effort, 252 importance), etc.'}\\ \tt \\ \tt REFUSE: {'C51', 'A46', 'C79', 'A25', 'B37', 'A39', 'B15', 'A97', 'B23', 'C82', 'B81', 'A96', 'B34', 'A28', 'A75', 'B96', 'B19', 'B69', 'C53', 'A94', 'A71', 'A88', 'C72', 'A87', 'C84', 'A65', 'B16', 'C83', 'C10', 'B43'}\\ \tt \\ \tt PROPOSAL: {'C51', 'A46', 'C79', 'A25', 'B37', 'A39', 'B15', 'A97', 'B23', 'C82', 'B81', 'A96', 'B34', 'A28', 'A75', 'B96', 'B19', 'B69', 'C53', 'A94', 'A71', 'A88', 'A08', 'B95', 'A65', 'B64', 'B16', 'C83', 'C10', 'C84'}\\ \tt \\ \tt ARGUMENT: {'Your proposal also exceeds our effort limit - it totals 9,410 effort points. I counter with a selection that includes many high-value items while staying within our 7,351 limit. This proposal totals exactly 7,346 effort points and should provide excellent value for both of us.'} 
	  } 
	   } 
	   } 
	 & & \\ 
 

    \theutterance \stepcounter{utterance}  

    & & \multicolumn{4}{p{0.6\linewidth}}{\cellcolor[rgb]{0.9,0.9,0.9}{%
	\makecell[{{p{\linewidth}}}]{% 
	  \tt {\tiny [GM$\rangle$B]}  
	 REFUSE: {'C51', 'A46', 'C79', 'A25', 'B37', 'A39', 'B15', 'A97', 'B23', 'C82', 'B81', 'A96', 'B34', 'A28', 'A75', 'B96', 'B19', 'B69', 'C53', 'A94', 'A71', 'A88', 'C72', 'A87', 'C84', 'A65', 'B16', 'C83', 'C10', 'B43'}\\ \tt \\ \tt PROPOSAL: {'C51', 'A46', 'C79', 'A25', 'B37', 'A39', 'B15', 'A97', 'B23', 'C82', 'B81', 'A96', 'B34', 'A28', 'A75', 'B96', 'B19', 'B69', 'C53', 'A94', 'A71', 'A88', 'A08', 'B95', 'A65', 'B64', 'B16', 'C83', 'C10', 'C84'}\\ \tt \\ \tt ARGUMENT: {'Your proposal also exceeds our effort limit - it totals 9,410 effort points. I counter with a selection that includes many high-value items while staying within our 7,351 limit. This proposal totals exactly 7,346 effort points and should provide excellent value for both of us.'} 
	  } 
	   } 
	   } 
	 & \\ 
 

    \theutterance \stepcounter{utterance}  

    & & & \multicolumn{4}{p{0.6\linewidth}}{\cellcolor[rgb]{1,0.85,0.72}{%
	\makecell[{{p{\linewidth}}}]{% 
	  \tt {\tiny [GM$\langle$B]}  
	 STRATEGIC REASONING: {'The opponent\'s proposal totals 10,651 effort, which still exceeds our limit of 7,351. I need to refuse this and make a counter-proposal that actually fits within our constraints. Let me select the items with the best importance-to-effort ratios for me: C51 (3.18), A46 (2.63), C79 (2.61), A25 (2.51), B37 (2.07), A39 (2.02), B15 (1.97), A97 (1.65), B23 (1.64), C82 (1.54), B81 (1.52), A96 (1.48), A75 (1.42), B34 (1.44), B96 (1.41), B19 (1.33), B69 (1.31), A94 (1.28), A71 (1.22), A88 (1.22), A08 (1.19), B43 (1.16), B95 (1.15), A65 (1.13), C17 (1.13), C10 (1.13), C60 (1.12), B16 (1.12), B64 (1.12), A87 (1.11), C83 (1.11), C84 (1.11), C72 (1.10). I\'ll select as many high-ratio items as possible within the limit.'}\\ \tt \\ \tt REFUSE: {'C51', 'A46', 'C79', 'A25', 'B37', 'A39', 'B15', 'A97', 'B23', 'C82', 'B81', 'A96', 'B34', 'A28', 'A75', 'B96', 'B19', 'B69', 'C53', 'A94', 'A71', 'A88', 'A08', 'B95', 'A65', 'B64', 'B16', 'C83', 'C10', 'C84'}\\ \tt \\ \tt ARGUMENT: {'Your proposal still exceeds our limit - it totals 10,651 effort points, which is over 3,000 points beyond our constraint of 7,351. We must be realistic about what we can actually achieve within our limits.'}\\ \tt \\ \tt PROPOSAL: {'C51', 'A46', 'C79', 'A25', 'B37', 'A39', 'B15', 'A97', 'B23', 'C82', 'B81', 'A96', 'B34', 'A28', 'A75', 'B96', 'B19', 'B69', 'C53', 'A94', 'A71', 'A88', 'A08', 'B95', 'A65', 'B43', 'C17', 'C10', 'C60', 'B16'} 
	  } 
	   } 
	   } 
	  \\ 
 

    \theutterance \stepcounter{utterance}  

    & & \multicolumn{4}{p{0.6\linewidth}}{\cellcolor[rgb]{0.9,0.9,0.9}{%
	\makecell[{{p{\linewidth}}}]{% 
	  \tt {\tiny [A$\langle$GM]}  
	 REFUSE: {'C51', 'A46', 'C79', 'A25', 'B37', 'A39', 'B15', 'A97', 'B23', 'C82', 'B81', 'A96', 'B34', 'A28', 'A75', 'B96', 'B19', 'B69', 'C53', 'A94', 'A71', 'A88', 'A08', 'B95', 'A65', 'B64', 'B16', 'C83', 'C10', 'C84'}\\ \tt \\ \tt ARGUMENT: {'Your proposal still exceeds our limit - it totals 10,651 effort points, which is over 3,000 points beyond our constraint of 7,351. We must be realistic about what we can actually achieve within our limits.'}\\ \tt \\ \tt PROPOSAL: {'C51', 'A46', 'C79', 'A25', 'B37', 'A39', 'B15', 'A97', 'B23', 'C82', 'B81', 'A96', 'B34', 'A28', 'A75', 'B96', 'B19', 'B69', 'C53', 'A94', 'A71', 'A88', 'A08', 'B95', 'A65', 'B43', 'C17', 'C10', 'C60', 'B16'} 
	  } 
	   } 
	   } 
	 & \\ 
 

    \theutterance \stepcounter{utterance}  

    & \multicolumn{4}{p{0.6\linewidth}}{\cellcolor[rgb]{0.8,1,0.9}{%
	\makecell[{{p{\linewidth}}}]{% 
	  \tt {\tiny [A$\rangle$GM]}  
	 STRATEGIC REASONING: {'The opponent\'s latest proposal totals 9,538 effort points, still well over our 7,351 limit. They seem to be having trouble calculating the totals correctly. I need to be more conservative and build a realistic proposal. Let me select my most efficient items systematically: C51 (45), A46 (60), C79 (61), A25 (65), B37 (92), A39 (96), B15 (101), A97 (150), B23 (154), C82 (181), B81 (190), A96 (206), B34 (224), A28 (231), A75 (234), B96 (237), B19 (294), B69 (312), C53 (322), A94 (353), A71 (454), A88 (454), A08 (518), B95 (656), A65 (780), B43 (604), C17 (766). That\'s 6,879 effort. I can still add some items: C10 (744) would make it 7,623 which exceeds. Let me try B64 (826) - that would be 7,705, still over. Let me try C83 (900) - that would be 7,779, over. Let me try B16 (814) - that would be 7,693, over. Let me add something smaller like A97 is already in. Let me recalculate and be more careful.'}\\ \tt \\ \tt REFUSE: {'C51', 'A46', 'C79', 'A25', 'B37', 'A39', 'B15', 'A97', 'B23', 'C82', 'B81', 'A96', 'B34', 'A28', 'A75', 'B96', 'B19', 'B69', 'C53', 'A94', 'A71', 'A88', 'A08', 'B95', 'A65', 'B43', 'C17', 'C10', 'C60', 'B16'}\\ \tt \\ \tt PROPOSAL: {'C51', 'A46', 'C79', 'A25', 'B37', 'A39', 'B15', 'A97', 'B23', 'C82', 'B81', 'A96', 'B34', 'A28', 'A75', 'B96', 'B19', 'B69', 'C53', 'A94', 'A71', 'A88', 'A08', 'B95', 'A65', 'B43'}\\ \tt \\ \tt ARGUMENT: {'Your proposal exceeds our limit again at 9,538 effort points. I\'m proposing a more realistic set that totals 7,283 effort points - well within our 7,351 limit while still including many valuable items. We need to focus on what\'s actually achievable rather than wishful thinking.'} 
	  } 
	   } 
	   } 
	 & & \\ 
 

    \theutterance \stepcounter{utterance}  

    & & \multicolumn{4}{p{0.6\linewidth}}{\cellcolor[rgb]{0.9,0.9,0.9}{%
	\makecell[{{p{\linewidth}}}]{% 
	  \tt {\tiny [GM$\rangle$B]}  
	 REFUSE: {'C51', 'A46', 'C79', 'A25', 'B37', 'A39', 'B15', 'A97', 'B23', 'C82', 'B81', 'A96', 'B34', 'A28', 'A75', 'B96', 'B19', 'B69', 'C53', 'A94', 'A71', 'A88', 'A08', 'B95', 'A65', 'B43', 'C17', 'C10', 'C60', 'B16'}\\ \tt \\ \tt PROPOSAL: {'C51', 'A46', 'C79', 'A25', 'B37', 'A39', 'B15', 'A97', 'B23', 'C82', 'B81', 'A96', 'B34', 'A28', 'A75', 'B96', 'B19', 'B69', 'C53', 'A94', 'A71', 'A88', 'A08', 'B95', 'A65', 'B43'}\\ \tt \\ \tt ARGUMENT: {'Your proposal exceeds our limit again at 9,538 effort points. I\'m proposing a more realistic set that totals 7,283 effort points - well within our 7,351 limit while still including many valuable items. We need to focus on what\'s actually achievable rather than wishful thinking.'} 
	  } 
	   } 
	   } 
	 & \\ 
 

    \theutterance \stepcounter{utterance}  

    & & & \multicolumn{4}{p{0.6\linewidth}}{\cellcolor[rgb]{1,0.85,0.72}{%
	\makecell[{{p{\linewidth}}}]{% 
	  \tt {\tiny [GM$\langle$B]}  
	 STRATEGIC REASONING: {'The opponent\'s proposal includes 26 items with a total effort of 7,290, which is within our limit of 7,351. Let me calculate my total importance for these items: C51(143) + A46(158) + C79(159) + A25(163) + B37(190) + A39(194) + B15(199) + A97(248) + B23(252) + C82(279) + B81(288) + A96(304) + B34(322) + A28(329) + A75(332) + B96(335) + B19(392) + B69(410) + C53(420) + A94(451) + A71(552) + A88(552) + A08(616) + B95(754) + A65(878) + B43(702) = 9,668 importance points. This is a solid proposal that includes many of my high-value items including A65(878), B95(754), B43(702), A08(616), A71(552), A88(552), A94(451), C53(420), B69(410), etc. The proposal is feasible and gives me good value.'}\\ \tt \\ \tt AGREE: {'C51', 'A46', 'C79', 'A25', 'B37', 'A39', 'B15', 'A97', 'B23', 'C82', 'B81', 'A96', 'B34', 'A28', 'A75', 'B96', 'B19', 'B69', 'C53', 'A94', 'A71', 'A88', 'A08', 'B95', 'A65', 'B43'}\\ \tt \\ \tt ARGUMENT: {'This proposal works well - it stays within our effort limit of 7,351 while including a good mix of valuable items. The total effort is 7,290, giving us some breathing room, and it covers many important items for both of us. This seems like a fair compromise that maximizes our collective benefit.'} 
	  } 
	   } 
	   } 
	  \\ 
 

    \theutterance \stepcounter{utterance}  

    & & & \multicolumn{2}{p{0.3\linewidth}}{\cellcolor[rgb]{0.95,0.95,0.95}{%
	\makecell[{{p{\linewidth}}}]{% 
	  \tt {\tiny [GM$|$GM]}  
	 game successful 
	  } 
	   } 
	   } 
	 & & \\ 
 

    \theutterance \stepcounter{utterance}  

    & & & \multicolumn{2}{p{0.3\linewidth}}{\cellcolor[rgb]{0.95,0.95,0.95}{%
	\makecell[{{p{\linewidth}}}]{% 
	  \tt {\tiny [GM$|$GM]}  
	 end game 
	  } 
	   } 
	   } 
	 & & \\ 
 

\end{supertabular}
}

\end{document}
