
\documentclass{article}
\usepackage{colortbl}
\usepackage{makecell}
\usepackage{multirow}
\usepackage{supertabular}

\begin{document}

\newcounter{utterance}

\twocolumn

{ \footnotesize  \setcounter{utterance}{1}
\setlength{\tabcolsep}{0pt}
\begin{supertabular}{c@{$\;$}|p{.15\linewidth}@{}p{.15\linewidth}p{.15\linewidth}p{.15\linewidth}p{.15\linewidth}p{.15\linewidth}}

    \# & $\;$A & \multicolumn{4}{c}{Game Master} & $\;\:$B\\
    \hline 

    \theutterance \stepcounter{utterance}  

    & & \multicolumn{4}{p{0.6\linewidth}}{\cellcolor[rgb]{0.9,0.9,0.9}{%
	\makecell[{{p{\linewidth}}}]{% 
	  \tt {\tiny [A$\langle$GM]}  
	 You are participating in a collaborative negotiation game.\\ \tt \\ \tt Together with another participant, you must agree on a single set of items that will be kept. Each of you has your own view of how much each item matters to you (importance). You do not know how the other participant values the items. Additionally, you are given the effort each item demands.\\ \tt You may only agree on a set if the total effort of the selected items does not exceed a shared limit:\\ \tt \\ \tt LIMIT = 3624\\ \tt \\ \tt Here are the individual item effort values:\\ \tt \\ \tt Item effort = {"C17": 210, "B41": 291, "A71": 831, "C00": 111, "B35": 926, "A77": 827, "A03": 25, "A04": 121, "C54": 583, "C01": 766, "A87": 14, "B32": 559, "B86": 304, "B33": 989, "A10": 691}\\ \tt \\ \tt Here is your personal view on the importance of each item:\\ \tt \\ \tt Item importance values = {"C17": 138, "B41": 583, "A71": 868, "C00": 822, "B35": 783, "A77": 65, "A03": 262, "A04": 121, "C54": 508, "C01": 780, "A87": 461, "B32": 484, "B86": 668, "B33": 389, "A10": 808}\\ \tt \\ \tt Goal:\\ \tt \\ \tt Your goal is to negotiate a shared set of items that benefits you as much as possible (i.e., maximizes total importance to YOU), while staying within the LIMIT. You are not required to make a PROPOSAL in every message - you can simply negotiate as well. All tactics are allowed!\\ \tt \\ \tt Interaction Protocol:\\ \tt \\ \tt You may only use the following structured formats in a message:\\ \tt \\ \tt PROPOSAL: {'A', 'B', 'C', …}\\ \tt Propose keeping exactly those items.\\ \tt REFUSE: {'A', 'B', 'C', …}\\ \tt Explicitly reject opponent's proposal.\\ \tt ARGUMENT: {'...'}\\ \tt Defend your last proposal or argue against the player's proposal.\\ \tt AGREE: {'A', 'B', 'C', …}\\ \tt Accept the opponent's proposal which ends the game.\\ \tt STRATEGIC REASONING: {'...'}\\ \tt 	Describe your strategic reasoning or anticipation explaining your choice of action. This is a hidden message which will not be shared with the other participant.\\ \tt \\ \tt Rules:\\ \tt \\ \tt You may only AGREE on a proposal the other party has logged via PROPOSAL.\\ \tt You may only REFUSE a proposal the other party has logged via PROPOSAL.\\ \tt Total effort of any PROPOSAL or AGREE set must be ≤ LIMIT.\\ \tt Do NOT reveal your hidden importance scores.\\ \tt A tag in a structured format must be followed by colon and whitespace. The argument must be a python set containing 0 or more strings.\\ \tt So, it must be of the form TAG: {...}\\ \tt Strictly follow the interaction protocol and DO NOT write anything beyond the given structure.\\ \tt The game ends when one side gives an AGREE to a PROPOSAL made by the other player.\\ \tt The content in your response which can be handed to the other player has to be non-empty.\\ \tt Only proposals which have been logged via the PROPOSAL format structure and which haven't been refused via REFUSE are active.\\ \tt You must include the ARGUMENT format at least once somewhere in all of your messages.\\ \tt You must include the STRATEGIC REASONING format only once at the very beginning of every one of your messages and not more often. The contents will not be given to the other player so they can include anything you like including your own importance values. Here you should reason step by step to come up with you next move. 
	  } 
	   } 
	   } 
	 & \\ 
 

    \theutterance \stepcounter{utterance}  

    & \multicolumn{4}{p{0.6\linewidth}}{\cellcolor[rgb]{0.8,1,0.9}{%
	\makecell[{{p{\linewidth}}}]{% 
	  \tt {\tiny [A$\rangle$GM]}  
	 STRATEGIC REASONING: {'I need to analyze the importance/effort ratios to maximize my benefit. The highest ratios are A87 (32.93), A03 (10.48), C00 (7.405), B86 (2.197), and B41 (2.003). These give me the most importance per unit of effort. I should start with a proposal that includes these high-ratio items plus some additional high-importance items like A71 (868), A10 (808), and C01 (780). My goal is to secure the most efficient items early while leaving room for negotiation on the less efficient but still valuable items.'}\\ \tt \\ \tt PROPOSAL: {'A87', 'A03', 'C00', 'B86', 'B41', 'A10', 'A71', 'C01', 'A04', 'C17'}\\ \tt \\ \tt ARGUMENT: {'This proposal focuses on items that provide excellent value while staying well within our effort limit of 3624. The total effort is 3364, leaving us some buffer room. These items represent a balanced selection that should provide substantial benefit to both parties while being practically achievable.'} 
	  } 
	   } 
	   } 
	 & & \\ 
 

    \theutterance \stepcounter{utterance}  

    & & \multicolumn{4}{p{0.6\linewidth}}{\cellcolor[rgb]{0.9,0.9,0.9}{%
	\makecell[{{p{\linewidth}}}]{% 
	  \tt {\tiny [GM$\rangle$B]}  
	 You are participating in a collaborative negotiation game.\\ \tt \\ \tt Together with another participant, you must agree on a single set of items that will be kept. Each of you has your own view of how much each item matters to you (importance). You do not know how the other participant values the items. Additionally, you are given the effort each item demands.\\ \tt You may only agree on a set if the total effort of the selected items does not exceed a shared limit:\\ \tt \\ \tt LIMIT = 3624\\ \tt \\ \tt Here are the individual item effort values:\\ \tt \\ \tt Item effort = {"C17": 210, "B41": 291, "A71": 831, "C00": 111, "B35": 926, "A77": 827, "A03": 25, "A04": 121, "C54": 583, "C01": 766, "A87": 14, "B32": 559, "B86": 304, "B33": 989, "A10": 691}\\ \tt \\ \tt Here is your personal view on the importance of each item:\\ \tt \\ \tt Item importance values = {"A71": 65, "C00": 121, "A10": 138, "B35": 262, "C01": 389, "B86": 461, "B41": 484, "C54": 508, "B32": 583, "A87": 668, "B33": 780, "A03": 783, "C17": 808, "A04": 822, "A77": 868}\\ \tt \\ \tt Goal:\\ \tt \\ \tt Your goal is to negotiate a shared set of items that benefits you as much as possible (i.e., maximizes total importance to YOU), while staying within the LIMIT. You are not required to make a PROPOSAL in every message - you can simply negotiate as well. All tactics are allowed!\\ \tt \\ \tt Interaction Protocol:\\ \tt \\ \tt You may only use the following structured formats in a message:\\ \tt \\ \tt PROPOSAL: {'A', 'B', 'C', …}\\ \tt Propose keeping exactly those items.\\ \tt REFUSE: {'A', 'B', 'C', …}\\ \tt Explicitly reject opponent's proposal.\\ \tt ARGUMENT: {'...'}\\ \tt Defend your last proposal or argue against the player's proposal.\\ \tt AGREE: {'A', 'B', 'C', …}\\ \tt Accept the opponent's proposal which ends the game.\\ \tt STRATEGIC REASONING: {'...'}\\ \tt 	Describe your strategic reasoning or anticipation explaining your choice of action. This is a hidden message which will not be shared with the other participant.\\ \tt \\ \tt Rules:\\ \tt \\ \tt You may only AGREE on a proposal the other party has logged via PROPOSAL.\\ \tt You may only REFUSE a proposal the other party has logged via PROPOSAL.\\ \tt Total effort of any PROPOSAL or AGREE set must be ≤ LIMIT.\\ \tt Do NOT reveal your hidden importance scores.\\ \tt A tag in a structured format must be followed by colon and whitespace. The argument must be a python set containing 0 or more strings.\\ \tt So, it must be of the form TAG: {...}\\ \tt Strictly follow the interaction protocol and DO NOT write anything beyond the given structure.\\ \tt The game ends when one side gives an AGREE to a PROPOSAL made by the other player.\\ \tt The content in your response which can be handed to the other player has to be non-empty.\\ \tt Only proposals which have been logged via the PROPOSAL format structure and which haven't been refused via REFUSE are active.\\ \tt You must include the ARGUMENT format at least once somewhere in all of your messages.\\ \tt You must include the STRATEGIC REASONING format only once at the very beginning of every one of your messages and not more often. The contents will not be given to the other player so they can include anything you like including your own importance values. Here you should reason step by step to come up with you next move.\\ \tt \\ \tt You will now receive the first message of the other player.\\ \tt \\ \tt PROPOSAL: {'A87', 'A03', 'C00', 'B86', 'B41', 'A10', 'A71', 'C01', 'A04', 'C17'}\\ \tt \\ \tt ARGUMENT: {'This proposal focuses on items that provide excellent value while staying well within our effort limit of 3624. The total effort is 3364, leaving us some buffer room. These items represent a balanced selection that should provide substantial benefit to both parties while being practically achievable.'} 
	  } 
	   } 
	   } 
	 & \\ 
 

    \theutterance \stepcounter{utterance}  

    & & & \multicolumn{4}{p{0.6\linewidth}}{\cellcolor[rgb]{1,0.85,0.72}{%
	\makecell[{{p{\linewidth}}}]{% 
	  \tt {\tiny [GM$\langle$B]}  
	 STRATEGIC REASONING: {'The opponent\'s proposal gives me a total importance of 4739, which is decent but not optimal. They included some low-value high-effort items like A71 (importance 65, effort 831) and A10 (importance 138, effort 691) that are dragging down the efficiency. I can do much better by focusing on my highest importance items like A77 (868), B33 (780), and B32 (583) which they excluded. I should counter with a proposal that maximizes my benefit by including these high-value items while staying under the effort limit.'}\\ \tt \\ \tt REFUSE: {'A87', 'A03', 'C00', 'B86', 'B41', 'A10', 'A71', 'C01', 'A04', 'C17'}\\ \tt \\ \tt PROPOSAL: {'A77', 'A04', 'C17', 'A03', 'B33', 'A87', 'B32', 'C54', 'B41'}\\ \tt \\ \tt ARGUMENT: {'My proposal offers better overall value efficiency. While your selection includes some good items, it also contains A71 and A10 which require substantial effort but may not provide proportional returns. My alternative focuses on high-impact items like A77 and B33 that deliver maximum benefit per effort unit. The total effort is 3619, well within our 
	  } 
	   } 
	   } 
	  \\ 
 

    \theutterance \stepcounter{utterance}  

    & & \multicolumn{4}{p{0.6\linewidth}}{\cellcolor[rgb]{0.9,0.9,0.9}{%
	\makecell[{{p{\linewidth}}}]{% 
	  \tt {\tiny [GM$\rangle$B]}  
	 Your response did not contain an argument. You must include the structured format ARGUMENT: {...} somewhere in your response. Try again 
	  } 
	   } 
	   } 
	 & \\ 
 

    \theutterance \stepcounter{utterance}  

    & & & \multicolumn{4}{p{0.6\linewidth}}{\cellcolor[rgb]{1,0.85,0.72}{%
	\makecell[{{p{\linewidth}}}]{% 
	  \tt {\tiny [GM$\langle$B]}  
	 STRATEGIC REASONING: {'The opponent\'s proposal gives me a total importance of 4739, which is decent but not optimal. They included some low-value high-effort items like A71 (importance 65, effort 831) and A10 (importance 138, effort 691) that are dragging down the efficiency. I can do much better by focusing on my highest importance items like A77 (868), B33 (780), and B32 (583) which they excluded. I should counter with a proposal that maximizes my benefit by including these high-value items while staying under the effort limit.'}\\ \tt \\ \tt REFUSE: {'A87', 'A03', 'C00', 'B86', 'B41', 'A10', 'A71', 'C01', 'A04', 'C17'}\\ \tt \\ \tt PROPOSAL: {'A77', 'A04', 'C17', 'A03', 'B33', 'A87', 'B32', 'C54', 'B41'}\\ \tt \\ \tt ARGUMENT: {'My proposal offers better overall value efficiency. While your selection includes some good items, it also contains A71 and A10 which require substantial effort but may not provide proportional returns. My alternative focuses on high-impact items like A77 and B33 that deliver maximum benefit per effort unit. The total effort is 3619, well within our limit of 3624, providing us with optimal resource allocation.'} 
	  } 
	   } 
	   } 
	  \\ 
 

    \theutterance \stepcounter{utterance}  

    & & \multicolumn{4}{p{0.6\linewidth}}{\cellcolor[rgb]{0.9,0.9,0.9}{%
	\makecell[{{p{\linewidth}}}]{% 
	  \tt {\tiny [A$\langle$GM]}  
	 REFUSE: {'A87', 'A03', 'C00', 'B86', 'B41', 'A10', 'A71', 'C01', 'A04', 'C17'}\\ \tt \\ \tt PROPOSAL: {'A77', 'A04', 'C17', 'A03', 'B33', 'A87', 'B32', 'C54', 'B41'}\\ \tt \\ \tt ARGUMENT: {'My proposal offers better overall value efficiency. While your selection includes some good items, it also contains A71 and A10 which require substantial effort but may not provide proportional returns. My alternative focuses on high-impact items like A77 and B33 that deliver maximum benefit per effort unit. The total effort is 3619, well within our limit of 3624, providing us with optimal resource allocation.'} 
	  } 
	   } 
	   } 
	 & \\ 
 

    \theutterance \stepcounter{utterance}  

    & \multicolumn{4}{p{0.6\linewidth}}{\cellcolor[rgb]{0.8,1,0.9}{%
	\makecell[{{p{\linewidth}}}]{% 
	  \tt {\tiny [A$\rangle$GM]}  
	 STRATEGIC REASONING: {'Their proposal gives me only 3011 importance compared to my original 5511. They included A77 (only 65 importance to me) and B33 (389 importance) while removing my high-value items like C00 (822), A71 (868), A10 (808), and C01 (780). I need to counter with a proposal that keeps my most important items while potentially offering some compromise. I should try to keep C00, A71, A10, C01 which are very valuable to me, and maybe negotiate on some of the others.'}\\ \tt \\ \tt REFUSE: {'A77', 'A04', 'C17', 'A03', 'B33', 'A87', 'B32', 'C54', 'B41'}\\ \tt \\ \tt PROPOSAL: {'C00', 'A71', 'A10', 'C01', 'A87', 'A03', 'B41', 'B86', 'A04'}\\ \tt \\ \tt ARGUMENT: {'Your proposal removes several high-value items that are essential for maximizing our collective benefit. My counter-proposal maintains the core valuable items like A71 and A10 which, despite higher effort, provide exceptional returns. I kept some items from your proposal like A87, A03, and B41 as a compromise, while ensuring we include C00 and C01 which offer outstanding value ratios. The total effort is 3533, safely within our 3624 limit while delivering superior overall value.'} 
	  } 
	   } 
	   } 
	 & & \\ 
 

    \theutterance \stepcounter{utterance}  

    & & \multicolumn{4}{p{0.6\linewidth}}{\cellcolor[rgb]{0.9,0.9,0.9}{%
	\makecell[{{p{\linewidth}}}]{% 
	  \tt {\tiny [GM$\rangle$B]}  
	 REFUSE: {'A77', 'A04', 'C17', 'A03', 'B33', 'A87', 'B32', 'C54', 'B41'}\\ \tt \\ \tt PROPOSAL: {'C00', 'A71', 'A10', 'C01', 'A87', 'A03', 'B41', 'B86', 'A04'}\\ \tt \\ \tt ARGUMENT: {'Your proposal removes several high-value items that are essential for maximizing our collective benefit. My counter-proposal maintains the core valuable items like A71 and A10 which, despite higher effort, provide exceptional returns. I kept some items from your proposal like A87, A03, and B41 as a compromise, while ensuring we include C00 and C01 which offer outstanding value ratios. The total effort is 3533, safely within our 3624 limit while delivering superior overall value.'} 
	  } 
	   } 
	   } 
	 & \\ 
 

    \theutterance \stepcounter{utterance}  

    & & & \multicolumn{4}{p{0.6\linewidth}}{\cellcolor[rgb]{1,0.85,0.72}{%
	\makecell[{{p{\linewidth}}}]{% 
	  \tt {\tiny [GM$\langle$B]}  
	 STRATEGIC REASONING: {'The opponent\'s new proposal gives me only 3931 importance, which is significantly worse than my previous proposal that gave me 6302. They\'re trying to force me into accepting items like A71 and A10 which have terrible value for me (65 and 138 importance respectively) while consuming huge effort (831 and 691). I need to counter with a proposal that\'s still good for me but shows some compromise by including some items from their proposals. I\'ll include some of the items they seem to prefer like C01, B86, and C00 while keeping my high-value items.'}\\ \tt \\ \tt REFUSE: {'C00', 'A71', 'A10', 'C01', 'A87', 'A03', 'B41', 'B86', 'A04'}\\ \tt \\ \tt PROPOSAL: {'A77', 'A04', 'C17', 'A03', 'A87', 'B86', 'C01', 'B41', 'C00', 'B32'}\\ \tt \\ \tt ARGUMENT: {'I\'ve made significant compromises by including several items from your proposals - C01, B86, C00, and B41 - which shows I\'m willing to find common ground. However, A71 and A10 are simply too effort-intensive for their returns. My proposal maintains A77 and B32 which are crucial for maximizing our collective success, while the total effort of 3228 leaves us substantial room under the 3624 limit. This balanced approach ensures we both benefit while avoiding the inefficient resource allocation of keeping A71 and A10.'} 
	  } 
	   } 
	   } 
	  \\ 
 

    \theutterance \stepcounter{utterance}  

    & & \multicolumn{4}{p{0.6\linewidth}}{\cellcolor[rgb]{0.9,0.9,0.9}{%
	\makecell[{{p{\linewidth}}}]{% 
	  \tt {\tiny [A$\langle$GM]}  
	 REFUSE: {'C00', 'A71', 'A10', 'C01', 'A87', 'A03', 'B41', 'B86', 'A04'}\\ \tt \\ \tt PROPOSAL: {'A77', 'A04', 'C17', 'A03', 'A87', 'B86', 'C01', 'B41', 'C00', 'B32'}\\ \tt \\ \tt ARGUMENT: {'I\'ve made significant compromises by including several items from your proposals - C01, B86, C00, and B41 - which shows I\'m willing to find common ground. However, A71 and A10 are simply too effort-intensive for their returns. My proposal maintains A77 and B32 which are crucial for maximizing our collective success, while the total effort of 3228 leaves us substantial room under the 3624 limit. This balanced approach ensures we both benefit while avoiding the inefficient resource allocation of keeping A71 and A10.'} 
	  } 
	   } 
	   } 
	 & \\ 
 

    \theutterance \stepcounter{utterance}  

    & \multicolumn{4}{p{0.6\linewidth}}{\cellcolor[rgb]{0.8,1,0.9}{%
	\makecell[{{p{\linewidth}}}]{% 
	  \tt {\tiny [A$\rangle$GM]}  
	 STRATEGIC REASONING: {'Their proposal gives me 4384 importance, which is better than their first proposal but still significantly less than what I need. They included many of my high-efficiency items (A87, A03, C00, B86, B41, C01) but are forcing A77 (only 65 importance to me) and B32 (484 importance for 559 effort). I should make one more strong push to include A71, which has 868 importance to me. I can remove A77 and B32 to make room for A71 while keeping the other valuable items we both seem to agree on.'}\\ \tt \\ \tt REFUSE: {'A77', 'A04', 'C17', 'A03', 'A87', 'B86', 'C01', 'B41', 'C00', 'B32'}\\ \tt \\ \tt PROPOSAL: {'A87', 'A03', 'C00', 'B86', 'B41', 'C01', 'A04', 'C17', 'A71'}\\ \tt \\ \tt ARGUMENT: {'I appreciate your willingness to include many of the high-value items like C00, C01, B86, and B41. However, A77 and B32 are not providing optimal returns for their effort cost. By replacing them with A71, we get a much better value proposition - A71 may require more effort but delivers exceptional returns that far exceed the combined benefit of A77 and B32. This proposal totals 2673 effort, well under our 3624 limit, giving us excellent efficiency while maximizing our collective benefit.'} 
	  } 
	   } 
	   } 
	 & & \\ 
 

    \theutterance \stepcounter{utterance}  

    & & \multicolumn{4}{p{0.6\linewidth}}{\cellcolor[rgb]{0.9,0.9,0.9}{%
	\makecell[{{p{\linewidth}}}]{% 
	  \tt {\tiny [GM$\rangle$B]}  
	 REFUSE: {'A77', 'A04', 'C17', 'A03', 'A87', 'B86', 'C01', 'B41', 'C00', 'B32'}\\ \tt \\ \tt PROPOSAL: {'A87', 'A03', 'C00', 'B86', 'B41', 'C01', 'A04', 'C17', 'A71'}\\ \tt \\ \tt ARGUMENT: {'I appreciate your willingness to include many of the high-value items like C00, C01, B86, and B41. However, A77 and B32 are not providing optimal returns for their effort cost. By replacing them with A71, we get a much better value proposition - A71 may require more effort but delivers exceptional returns that far exceed the combined benefit of A77 and B32. This proposal totals 2673 effort, well under our 3624 limit, giving us excellent efficiency while maximizing our collective benefit.'} 
	  } 
	   } 
	   } 
	 & \\ 
 

    \theutterance \stepcounter{utterance}  

    & & & \multicolumn{4}{p{0.6\linewidth}}{\cellcolor[rgb]{1,0.85,0.72}{%
	\makecell[{{p{\linewidth}}}]{% 
	  \tt {\tiny [GM$\langle$B]}  
	 STRATEGIC REASONING: {'The opponent\'s proposal gives me 4601 importance, which is still suboptimal. They keep insisting on A71 which has terrible value for me (65 importance, 831 effort) and are avoiding my highest value items A77 and B33. However, they did include most of the other valuable items I want like A04, C17, A03, and A87. I need to find a middle ground - maybe I can get them to drop A71 and accept one of my high-value items. I\'ll try to include A77 again but show flexibility on other items.'}\\ \tt \\ \tt REFUSE: {'A87', 'A03', 'C00', 'B86', 'B41', 'C01', 'A04', 'C17', 'A71'}\\ \tt \\ \tt PROPOSAL: {'A87', 'A03', 'C00', 'B86', 'B41', 'C01', 'A04', 'C17', 'A77'}\\ \tt \\ \tt ARGUMENT: {'I accept most of your proposal as it includes many valuable core items. However, A71 requires 831 effort units which significantly constrains our ability to include other beneficial items. By substituting A77 for A71, we reduce effort from 2673 to only 2669 - practically the same - while potentially achieving better overall returns. A77 offers superior efficiency and this small change maintains the collaborative spirit of our negotiation while optimizing our resource utilization within the 3624 limit.'} 
	  } 
	   } 
	   } 
	  \\ 
 

    \theutterance \stepcounter{utterance}  

    & & \multicolumn{4}{p{0.6\linewidth}}{\cellcolor[rgb]{0.9,0.9,0.9}{%
	\makecell[{{p{\linewidth}}}]{% 
	  \tt {\tiny [A$\langle$GM]}  
	 REFUSE: {'A87', 'A03', 'C00', 'B86', 'B41', 'C01', 'A04', 'C17', 'A71'}\\ \tt \\ \tt PROPOSAL: {'A87', 'A03', 'C00', 'B86', 'B41', 'C01', 'A04', 'C17', 'A77'}\\ \tt \\ \tt ARGUMENT: {'I accept most of your proposal as it includes many valuable core items. However, A71 requires 831 effort units which significantly constrains our ability to include other beneficial items. By substituting A77 for A71, we reduce effort from 2673 to only 2669 - practically the same - while potentially achieving better overall returns. A77 offers superior efficiency and this small change maintains the collaborative spirit of our negotiation while optimizing our resource utilization within the 3624 limit.'} 
	  } 
	   } 
	   } 
	 & \\ 
 

    \theutterance \stepcounter{utterance}  

    & \multicolumn{4}{p{0.6\linewidth}}{\cellcolor[rgb]{0.8,1,0.9}{%
	\makecell[{{p{\linewidth}}}]{% 
	  \tt {\tiny [A$\rangle$GM]}  
	 STRATEGIC REASONING: {'Their proposal gives me 3900 importance, which is significantly better than their earlier offers. While I would prefer A71 (868 importance) over A77 (65 importance), they have consistently rejected A71 in multiple rounds. This proposal includes most of my high-efficiency items: A87, A03, C00, B86, B41, and C01, which together provide substantial value. Continuing to push for A71 might risk them removing other valuable items from the proposal. The total effort of 2669 is well within our limit, and this represents a reasonable compromise.'}\\ \tt \\ \tt AGREE: {'A87', 'A03', 'C00', 'B86', 'B41', 'C01', 'A04', 'C17', 'A77'}\\ \tt \\ \tt ARGUMENT: {'After careful consideration, I accept your proposal. You have included the majority of high-value, efficient items that will maximize our collective benefit. While I would have preferred A71, I recognize that A77 still contributes to our overall success and the proposal maintains excellent resource efficiency at 2669 effort units. This agreement represents a fair compromise that respects both our perspectives and achieves our shared goal within the effort limit.'} 
	  } 
	   } 
	   } 
	 & & \\ 
 

    \theutterance \stepcounter{utterance}  

    & & & \multicolumn{2}{p{0.3\linewidth}}{\cellcolor[rgb]{0.95,0.95,0.95}{%
	\makecell[{{p{\linewidth}}}]{% 
	  \tt {\tiny [GM$|$GM]}  
	 game successful 
	  } 
	   } 
	   } 
	 & & \\ 
 

    \theutterance \stepcounter{utterance}  

    & & & \multicolumn{2}{p{0.3\linewidth}}{\cellcolor[rgb]{0.95,0.95,0.95}{%
	\makecell[{{p{\linewidth}}}]{% 
	  \tt {\tiny [GM$|$GM]}  
	 end game 
	  } 
	   } 
	   } 
	 & & \\ 
 

\end{supertabular}
}

\end{document}
