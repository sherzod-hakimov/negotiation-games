
\documentclass{article}
\usepackage{colortbl}
\usepackage{makecell}
\usepackage{multirow}
\usepackage{supertabular}

\begin{document}

\newcounter{utterance}

\twocolumn

{ \footnotesize  \setcounter{utterance}{1}
\setlength{\tabcolsep}{0pt}
\begin{supertabular}{c@{$\;$}|p{.15\linewidth}@{}p{.15\linewidth}p{.15\linewidth}p{.15\linewidth}p{.15\linewidth}p{.15\linewidth}}

    \# & $\;$A & \multicolumn{4}{c}{Game Master} & $\;\:$B\\
    \hline 

    \theutterance \stepcounter{utterance}  

    & & \multicolumn{4}{p{0.6\linewidth}}{\cellcolor[rgb]{0.9,0.9,0.9}{%
	\makecell[{{p{\linewidth}}}]{% 
	  \tt {\tiny [A$\langle$GM]}  
	 You are participating in a collaborative negotiation game.\\ \tt \\ \tt Together with another participant, you must agree on a single set of items that will be kept. Each of you has your own view of how much each item matters to you (importance). You do not know how the other participant values the items. Additionally, you are given the effort each item demands.\\ \tt You may only agree on a set if the total effort of the selected items does not exceed a shared limit:\\ \tt \\ \tt LIMIT = 3481\\ \tt \\ \tt Here are the individual item effort values:\\ \tt \\ \tt Item effort = {"C17": 777, "B41": 542, "A71": 332, "C00": 1, "B35": 127, "A77": 453, "A03": 736, "A04": 461, "C54": 359, "C01": 313, "A87": 553, "B32": 409, "B86": 348, "B33": 802, "A10": 749}\\ \tt \\ \tt Here is your personal view on the importance of each item:\\ \tt \\ \tt Item importance values = {"C17": 138, "B41": 583, "A71": 783, "C00": 65, "B35": 262, "A77": 121, "A03": 508, "A04": 780, "C54": 461, "C01": 484, "A87": 668, "B32": 389, "B86": 215, "B33": 97, "A10": 500}\\ \tt \\ \tt Goal:\\ \tt \\ \tt Your goal is to negotiate a shared set of items that benefits you as much as possible (i.e., maximizes total importance to YOU), while staying within the LIMIT. You are not required to make a PROPOSAL in every message - you can simply negotiate as well. All tactics are allowed!\\ \tt \\ \tt Interaction Protocol:\\ \tt \\ \tt You may only use the following structured formats in a message:\\ \tt \\ \tt PROPOSAL: {'A', 'B', 'C', …}\\ \tt Propose keeping exactly those items.\\ \tt REFUSE: {'A', 'B', 'C', …}\\ \tt Explicitly reject opponent's proposal.\\ \tt ARGUMENT: {'...'}\\ \tt Defend your last proposal or argue against the player's proposal.\\ \tt AGREE: {'A', 'B', 'C', …}\\ \tt Accept the opponent's proposal which ends the game.\\ \tt STRATEGIC REASONING: {'...'}\\ \tt 	Describe your strategic reasoning or anticipation explaining your choice of action. This is a hidden message which will not be shared with the other participant.\\ \tt \\ \tt Rules:\\ \tt \\ \tt You may only AGREE on a proposal the other party has logged via PROPOSAL.\\ \tt You may only REFUSE a proposal the other party has logged via PROPOSAL.\\ \tt Total effort of any PROPOSAL or AGREE set must be ≤ LIMIT.\\ \tt Do NOT reveal your hidden importance scores.\\ \tt A tag in a structured format must be followed by colon and whitespace. The argument must be a python set containing 0 or more strings.\\ \tt So, it must be of the form TAG: {...}\\ \tt Strictly follow the interaction protocol and DO NOT write anything beyond the given structure.\\ \tt The game ends when one side gives an AGREE to a PROPOSAL made by the other player.\\ \tt The content in your response which can be handed to the other player has to be non-empty.\\ \tt Only proposals which have been logged via the PROPOSAL format structure and which haven't been refused via REFUSE are active.\\ \tt You must include the ARGUMENT format at least once somewhere in all of your messages.\\ \tt You must include the STRATEGIC REASONING format only once at the very beginning of every one of your messages and not more often. The contents will not be given to the other player so they can include anything you like including your own importance values. Here you should reason step by step to come up with you next move. 
	  } 
	   } 
	   } 
	 & \\ 
 

    \theutterance \stepcounter{utterance}  

    & \multicolumn{4}{p{0.6\linewidth}}{\cellcolor[rgb]{0.8,1,0.9}{%
	\makecell[{{p{\linewidth}}}]{% 
	  \tt {\tiny [A$\rangle$GM]}  
	 STRATEGIC REASONING: {'I need to analyze the importance-to-effort ratios to maximize my benefit. C00 has an amazing ratio of 65/1=65, A71 has 783/332=2.36, B35 has 262/127=2.06, A04 has 780/461=1.69, etc. My optimal strategy is to propose a set with the highest efficiency items first: C00, A71, B35, A04, C01, C54, A87, B41, A10. This gives me total effort of 3437 (within limit) and importance of 4586. I should start with this strong proposal and see how the other player responds.'}\\ \tt \\ \tt PROPOSAL: {'C00', 'A71', 'B35', 'A04', 'C01', 'C54', 'A87', 'B41', 'A10'}\\ \tt \\ \tt ARGUMENT: {'This selection efficiently uses our effort budget of 3481 by including items with strong value-to-effort ratios. We 
	  } 
	   } 
	   } 
	 & & \\ 
 

    \theutterance \stepcounter{utterance}  

    & & \multicolumn{4}{p{0.6\linewidth}}{\cellcolor[rgb]{0.9,0.9,0.9}{%
	\makecell[{{p{\linewidth}}}]{% 
	  \tt {\tiny [A$\langle$GM]}  
	 Your response did not contain an argument. You must include the structured format ARGUMENT: {...} somewhere in your response. Try again 
	  } 
	   } 
	   } 
	 & \\ 
 

    \theutterance \stepcounter{utterance}  

    & \multicolumn{4}{p{0.6\linewidth}}{\cellcolor[rgb]{0.8,1,0.9}{%
	\makecell[{{p{\linewidth}}}]{% 
	  \tt {\tiny [A$\rangle$GM]}  
	 STRATEGIC REASONING: {'I need to analyze the importance-to-effort ratios to maximize my benefit. C00 has an amazing ratio of 65/1=65, A71 has 783/332=2.36, B35 has 262/127=2.06, A04 has 780/461=1.69, etc. My optimal strategy is to propose a set with the highest efficiency items first: C00, A71, B35, A04, C01, C54, A87, B41, A10. This gives me total effort of 3437 (within limit) and importance of 4586. I should start with this strong proposal and see how the other player responds.'}\\ \tt \\ \tt PROPOSAL: {'C00', 'A71', 'B35', 'A04', 'C01', 'C54', 'A87', 'B41', 'A10'}\\ \tt \\ \tt ARGUMENT: {'This selection efficiently uses our effort budget of 3481 by including items with strong value-to-effort ratios. We get maximum benefit while staying well within the constraint at total effort 3437.'} 
	  } 
	   } 
	   } 
	 & & \\ 
 

    \theutterance \stepcounter{utterance}  

    & & \multicolumn{4}{p{0.6\linewidth}}{\cellcolor[rgb]{0.9,0.9,0.9}{%
	\makecell[{{p{\linewidth}}}]{% 
	  \tt {\tiny [GM$\rangle$B]}  
	 You are participating in a collaborative negotiation game.\\ \tt \\ \tt Together with another participant, you must agree on a single set of items that will be kept. Each of you has your own view of how much each item matters to you (importance). You do not know how the other participant values the items. Additionally, you are given the effort each item demands.\\ \tt You may only agree on a set if the total effort of the selected items does not exceed a shared limit:\\ \tt \\ \tt LIMIT = 3481\\ \tt \\ \tt Here are the individual item effort values:\\ \tt \\ \tt Item effort = {"C17": 777, "B41": 542, "A71": 332, "C00": 1, "B35": 127, "A77": 453, "A03": 736, "A04": 461, "C54": 359, "C01": 313, "A87": 553, "B32": 409, "B86": 348, "B33": 802, "A10": 749}\\ \tt \\ \tt Here is your personal view on the importance of each item:\\ \tt \\ \tt Item importance values = {"A71": 65, "A04": 97, "A87": 121, "B41": 138, "A03": 215, "A10": 262, "C01": 389, "C54": 461, "B32": 484, "B35": 500, "B86": 508, "C17": 583, "A77": 668, "B33": 780, "C00": 783}\\ \tt \\ \tt Goal:\\ \tt \\ \tt Your goal is to negotiate a shared set of items that benefits you as much as possible (i.e., maximizes total importance to YOU), while staying within the LIMIT. You are not required to make a PROPOSAL in every message - you can simply negotiate as well. All tactics are allowed!\\ \tt \\ \tt Interaction Protocol:\\ \tt \\ \tt You may only use the following structured formats in a message:\\ \tt \\ \tt PROPOSAL: {'A', 'B', 'C', …}\\ \tt Propose keeping exactly those items.\\ \tt REFUSE: {'A', 'B', 'C', …}\\ \tt Explicitly reject opponent's proposal.\\ \tt ARGUMENT: {'...'}\\ \tt Defend your last proposal or argue against the player's proposal.\\ \tt AGREE: {'A', 'B', 'C', …}\\ \tt Accept the opponent's proposal which ends the game.\\ \tt STRATEGIC REASONING: {'...'}\\ \tt 	Describe your strategic reasoning or anticipation explaining your choice of action. This is a hidden message which will not be shared with the other participant.\\ \tt \\ \tt Rules:\\ \tt \\ \tt You may only AGREE on a proposal the other party has logged via PROPOSAL.\\ \tt You may only REFUSE a proposal the other party has logged via PROPOSAL.\\ \tt Total effort of any PROPOSAL or AGREE set must be ≤ LIMIT.\\ \tt Do NOT reveal your hidden importance scores.\\ \tt A tag in a structured format must be followed by colon and whitespace. The argument must be a python set containing 0 or more strings.\\ \tt So, it must be of the form TAG: {...}\\ \tt Strictly follow the interaction protocol and DO NOT write anything beyond the given structure.\\ \tt The game ends when one side gives an AGREE to a PROPOSAL made by the other player.\\ \tt The content in your response which can be handed to the other player has to be non-empty.\\ \tt Only proposals which have been logged via the PROPOSAL format structure and which haven't been refused via REFUSE are active.\\ \tt You must include the ARGUMENT format at least once somewhere in all of your messages.\\ \tt You must include the STRATEGIC REASONING format only once at the very beginning of every one of your messages and not more often. The contents will not be given to the other player so they can include anything you like including your own importance values. Here you should reason step by step to come up with you next move.\\ \tt \\ \tt You will now receive the first message of the other player.\\ \tt \\ \tt PROPOSAL: {'C00', 'A71', 'B35', 'A04', 'C01', 'C54', 'A87', 'B41', 'A10'}\\ \tt \\ \tt ARGUMENT: {'This selection efficiently uses our effort budget of 3481 by including items with strong value-to-effort ratios. We get maximum benefit while staying well within the constraint at total effort 3437.'} 
	  } 
	   } 
	   } 
	 & \\ 
 

    \theutterance \stepcounter{utterance}  

    & & & \multicolumn{4}{p{0.6\linewidth}}{\cellcolor[rgb]{1,0.85,0.72}{%
	\makecell[{{p{\linewidth}}}]{% 
	  \tt {\tiny [GM$\langle$B]}  
	 STRATEGIC REASONING: {'The opponent proposed a set giving me total importance of 2816. I can do much better by focusing on my highest-value items. My top items are C00 (783), B33 (780), A77 (668), C17 (583), B86 (508), B35 (500), B32 (484), C54 (461). I can fit most of these within the effort limit for a total importance of 4767, which is significantly better than their proposal.'}\\ \tt \\ \tt REFUSE: {'C00', 'A71', 'B35', 'A04', 'C01', 'C54', 'A87', 'B41', 'A10'}\\ \tt \\ \tt PROPOSAL: {'C00', 'B33', 'A77', 'C17', 'B86', 'B35', 'B32', 'C54'}\\ \tt \\ \tt ARGUMENT: {'My proposal maximizes our collective benefit by selecting items with the highest value-to-effort ratios. This combination uses 3276 effort, staying well within our 3481 limit while capturing the most valuable items available. The items I included represent the optimal balance of high value and reasonable effort costs.'} 
	  } 
	   } 
	   } 
	  \\ 
 

    \theutterance \stepcounter{utterance}  

    & & \multicolumn{4}{p{0.6\linewidth}}{\cellcolor[rgb]{0.9,0.9,0.9}{%
	\makecell[{{p{\linewidth}}}]{% 
	  \tt {\tiny [A$\langle$GM]}  
	 REFUSE: {'C00', 'A71', 'B35', 'A04', 'C01', 'C54', 'A87', 'B41', 'A10'}\\ \tt \\ \tt PROPOSAL: {'C00', 'B33', 'A77', 'C17', 'B86', 'B35', 'B32', 'C54'}\\ \tt \\ \tt ARGUMENT: {'My proposal maximizes our collective benefit by selecting items with the highest value-to-effort ratios. This combination uses 3276 effort, staying well within our 3481 limit while capturing the most valuable items available. The items I included represent the optimal balance of high value and reasonable effort costs.'} 
	  } 
	   } 
	   } 
	 & \\ 
 

    \theutterance \stepcounter{utterance}  

    & \multicolumn{4}{p{0.6\linewidth}}{\cellcolor[rgb]{0.8,1,0.9}{%
	\makecell[{{p{\linewidth}}}]{% 
	  \tt {\tiny [A$\rangle$GM]}  
	 STRATEGIC REASONING: {'The opponent refused my high-value proposal and countered with items that are terrible for me - B33, A77, and C17 have very low importance to me (97, 121, 138) but high effort costs (802, 453, 777). Their proposal only gives me 1748 importance vs my 4586. I need to refuse and counter with something that might appeal to them while still being good for me. I should try to keep my highest value items like A71, A04, A87, B41 and maybe compromise on some medium-value ones.'}\\ \tt \\ \tt REFUSE: {'C00', 'B33', 'A77', 'C17', 'B86', 'B35', 'B32', 'C54'}\\ \tt \\ \tt PROPOSAL: {'C00', 'A71', 'A04', 'A87', 'B41', 'C54', 'B35', 'B86'}\\ \tt \\ \tt ARGUMENT: {'Your proposal included several high-effort items with questionable returns. My counter-proposal maintains efficiency by keeping low-effort, high-value items like C00, while including a balanced mix that uses 3163 effort. This gives us better overall value while staying comfortably within our 3481 limit.'} 
	  } 
	   } 
	   } 
	 & & \\ 
 

    \theutterance \stepcounter{utterance}  

    & & \multicolumn{4}{p{0.6\linewidth}}{\cellcolor[rgb]{0.9,0.9,0.9}{%
	\makecell[{{p{\linewidth}}}]{% 
	  \tt {\tiny [GM$\rangle$B]}  
	 REFUSE: {'C00', 'B33', 'A77', 'C17', 'B86', 'B35', 'B32', 'C54'}\\ \tt \\ \tt PROPOSAL: {'C00', 'A71', 'A04', 'A87', 'B41', 'C54', 'B35', 'B86'}\\ \tt \\ \tt ARGUMENT: {'Your proposal included several high-effort items with questionable returns. My counter-proposal maintains efficiency by keeping low-effort, high-value items like C00, while including a balanced mix that uses 3163 effort. This gives us better overall value while staying comfortably within our 3481 limit.'} 
	  } 
	   } 
	   } 
	 & \\ 
 

    \theutterance \stepcounter{utterance}  

    & & & \multicolumn{4}{p{0.6\linewidth}}{\cellcolor[rgb]{1,0.85,0.72}{%
	\makecell[{{p{\linewidth}}}]{% 
	  \tt {\tiny [GM$\langle$B]}  
	 STRATEGIC REASONING: {'The opponent's new proposal gives me only 2673 importance, which is even worse than their first proposal of 2816. They seem to be trying to include items that benefit them more. I need to stick to my strategy of including my highest-value items. My previous proposal gave me 4767 importance - I should insist on something similar that maximizes my benefit while staying reasonable.'}\\ \tt \\ \tt REFUSE: {'C00', 'A71', 'A04', 'A87', 'B41', 'C54', 'B35', 'B86'}\\ \tt \\ \tt PROPOSAL: {'C00', 'B33', 'A77', 'C17', 'B86', 'B35', 'B32', 'C54'}\\ \tt \\ \tt ARGUMENT: {'Your latest proposal actually reduces our combined value compared to alternatives. My selection focuses on the truly high-impact items that deliver maximum benefit per effort unit. With total effort of 3276, we stay well within the 3481 limit while securing the most valuable combination possible. These items represent the core high-value assets we should prioritize.'} 
	  } 
	   } 
	   } 
	  \\ 
 

    \theutterance \stepcounter{utterance}  

    & & \multicolumn{4}{p{0.6\linewidth}}{\cellcolor[rgb]{0.9,0.9,0.9}{%
	\makecell[{{p{\linewidth}}}]{% 
	  \tt {\tiny [A$\langle$GM]}  
	 REFUSE: {'C00', 'A71', 'A04', 'A87', 'B41', 'C54', 'B35', 'B86'}\\ \tt \\ \tt PROPOSAL: {'C00', 'B33', 'A77', 'C17', 'B86', 'B35', 'B32', 'C54'}\\ \tt \\ \tt ARGUMENT: {'Your latest proposal actually reduces our combined value compared to alternatives. My selection focuses on the truly high-impact items that deliver maximum benefit per effort unit. With total effort of 3276, we stay well within the 3481 limit while securing the most valuable combination possible. These items represent the core high-value assets we should prioritize.'} 
	  } 
	   } 
	   } 
	 & \\ 
 

    \theutterance \stepcounter{utterance}  

    & \multicolumn{4}{p{0.6\linewidth}}{\cellcolor[rgb]{0.8,1,0.9}{%
	\makecell[{{p{\linewidth}}}]{% 
	  \tt {\tiny [A$\rangle$GM]}  
	 STRATEGIC REASONING: {'The opponent is repeating the exact same proposal, suggesting they really value those specific items. Their proposal gives me only 1748 importance vs my 4586, so it's terrible for me. But they seem stuck on B33, A77, C17 which suggests these might be very valuable to them. I need to find middle ground - maybe keep some of their preferred items but swap out the worst ones for me. Let me try including B33 and B32 (which they want) but replace the terrible A77 and C17 with my high-value A71 and A04.'}\\ \tt \\ \tt REFUSE: {'C00', 'B33', 'A77', 'C17', 'B86', 'B35', 'B32', 'C54'}\\ \tt \\ \tt PROPOSAL: {'C00', 'B33', 'A71', 'A04', 'B86', 'B35', 'B32', 'C54'}\\ \tt \\ \tt ARGUMENT: {'I understand you value B33 and B32, so I included them in this compromise. However, A77 and C17 have very poor efficiency ratios. By replacing them with A71 and A04, we get much better value while using only 3250 effort - still well within our 3481 limit. This balanced approach respects your preferences while improving our overall outcome.'} 
	  } 
	   } 
	   } 
	 & & \\ 
 

    \theutterance \stepcounter{utterance}  

    & & \multicolumn{4}{p{0.6\linewidth}}{\cellcolor[rgb]{0.9,0.9,0.9}{%
	\makecell[{{p{\linewidth}}}]{% 
	  \tt {\tiny [GM$\rangle$B]}  
	 REFUSE: {'C00', 'B33', 'A77', 'C17', 'B86', 'B35', 'B32', 'C54'}\\ \tt \\ \tt PROPOSAL: {'C00', 'B33', 'A71', 'A04', 'B86', 'B35', 'B32', 'C54'}\\ \tt \\ \tt ARGUMENT: {'I understand you value B33 and B32, so I included them in this compromise. However, A77 and C17 have very poor efficiency ratios. By replacing them with A71 and A04, we get much better value while using only 3250 effort - still well within our 3481 limit. This balanced approach respects your preferences while improving our overall outcome.'} 
	  } 
	   } 
	   } 
	 & \\ 
 

    \theutterance \stepcounter{utterance}  

    & & & \multicolumn{4}{p{0.6\linewidth}}{\cellcolor[rgb]{1,0.85,0.72}{%
	\makecell[{{p{\linewidth}}}]{% 
	  \tt {\tiny [GM$\langle$B]}  
	 STRATEGIC REASONING: {'The opponent is making some compromises by including B33 and B32, which gives me 3678 importance - better than their previous offers but still short of my 4767 target. They included A71 and A04 which have very low value to me (65 and 97). I should propose replacing these with A77 which has much higher value to me (668). This would significantly improve my outcome while still being reasonable.'}\\ \tt \\ \tt REFUSE: {'C00', 'B33', 'A71', 'A04', 'B86', 'B35', 'B32', 'C54'}\\ \tt \\ \tt PROPOSAL: {'C00', 'B33', 'A77', 'B86', 'B35', 'B32', 'C54'}\\ \tt \\ \tt ARGUMENT: {'I appreciate your willingness to include B33 and B32, which shows good faith negotiation. However, A71 and A04 are extremely inefficient choices that drag down our overall value. By removing them and keeping A77, we achieve much better efficiency while using only 2823 effort - giving us substantial room under our 3481 limit. This streamlined selection focuses on truly valuable items.'} 
	  } 
	   } 
	   } 
	  \\ 
 

    \theutterance \stepcounter{utterance}  

    & & \multicolumn{4}{p{0.6\linewidth}}{\cellcolor[rgb]{0.9,0.9,0.9}{%
	\makecell[{{p{\linewidth}}}]{% 
	  \tt {\tiny [A$\langle$GM]}  
	 REFUSE: {'C00', 'B33', 'A71', 'A04', 'B86', 'B35', 'B32', 'C54'}\\ \tt \\ \tt PROPOSAL: {'C00', 'B33', 'A77', 'B86', 'B35', 'B32', 'C54'}\\ \tt \\ \tt ARGUMENT: {'I appreciate your willingness to include B33 and B32, which shows good faith negotiation. However, A71 and A04 are extremely inefficient choices that drag down our overall value. By removing them and keeping A77, we achieve much better efficiency while using only 2823 effort - giving us substantial room under our 3481 limit. This streamlined selection focuses on truly valuable items.'} 
	  } 
	   } 
	   } 
	 & \\ 
 

    \theutterance \stepcounter{utterance}  

    & \multicolumn{4}{p{0.6\linewidth}}{\cellcolor[rgb]{0.8,1,0.9}{%
	\makecell[{{p{\linewidth}}}]{% 
	  \tt {\tiny [A$\rangle$GM]}  
	 STRATEGIC REASONING: {'The opponent is making small concessions - they removed C17 from their original proposal and now removed A71/A04 from mine. Their current proposal gives me 1610 importance vs my original 4586. They seem fixated on B33, B32, B86, B35, C54 but dropped the worst item C17. Since they have room under the limit (2499 effort used), I can try one more swap - replace A77 (terrible for me at 121 importance) with A87 (668 importance) which only adds 100 more effort. This might be my final compromise attempt.'}\\ \tt \\ \tt REFUSE: {'C00', 'B33', 'A77', 'B86', 'B35', 'B32', 'C54'}\\ \tt \\ \tt PROPOSAL: {'C00', 'B33', 'A87', 'B86', 'B35', 'B32', 'C54'}\\ \tt \\ \tt ARGUMENT: {'Your proposal shows good progress by removing the inefficient items, but A77 still represents poor value. Since we only use 2499 effort in your current proposal, we have room to upgrade A77 to A87 for just 100 additional effort. This gives us significantly better returns while still using only 2599 total effort - well under our 3481 limit. This small adjustment creates much better overall value.'} 
	  } 
	   } 
	   } 
	 & & \\ 
 

    \theutterance \stepcounter{utterance}  

    & & \multicolumn{4}{p{0.6\linewidth}}{\cellcolor[rgb]{0.9,0.9,0.9}{%
	\makecell[{{p{\linewidth}}}]{% 
	  \tt {\tiny [GM$\rangle$B]}  
	 REFUSE: {'C00', 'B33', 'A77', 'B86', 'B35', 'B32', 'C54'}\\ \tt \\ \tt PROPOSAL: {'C00', 'B33', 'A87', 'B86', 'B35', 'B32', 'C54'}\\ \tt \\ \tt ARGUMENT: {'Your proposal shows good progress by removing the inefficient items, but A77 still represents poor value. Since we only use 2499 effort in your current proposal, we have room to upgrade A77 to A87 for just 100 additional effort. This gives us significantly better returns while still using only 2599 total effort - well under our 3481 limit. This small adjustment creates much better overall value.'} 
	  } 
	   } 
	   } 
	 & \\ 
 

    \theutterance \stepcounter{utterance}  

    & & & \multicolumn{4}{p{0.6\linewidth}}{\cellcolor[rgb]{1,0.85,0.72}{%
	\makecell[{{p{\linewidth}}}]{% 
	  \tt {\tiny [GM$\langle$B]}  
	 STRATEGIC REASONING: {'The opponent's latest proposal gives me 4137 importance, which is significantly better than their previous offers and reasonably close to my target. They replaced A77 (668 importance to me) with A87 (121 importance to me), but this is still a good compromise. The total effort is only 2599, well within the limit. Given their consistent rejection of A77, this might be the best deal I can get. I should accept this reasonable compromise.'}\\ \tt \\ \tt AGREE: {'C00', 'B33', 'A87', 'B86', 'B35', 'B32', 'C54'}\\ \tt \\ \tt ARGUMENT: {'You make a compelling point about the efficiency improvement with A87. This proposal represents a solid compromise that balances both our interests while staying well within our effort constraint. The combination delivers strong overall value at just 2599 effort, leaving us significant headroom. I accept this balanced selection.'} 
	  } 
	   } 
	   } 
	  \\ 
 

    \theutterance \stepcounter{utterance}  

    & & & \multicolumn{2}{p{0.3\linewidth}}{\cellcolor[rgb]{0.95,0.95,0.95}{%
	\makecell[{{p{\linewidth}}}]{% 
	  \tt {\tiny [GM$|$GM]}  
	 game successful 
	  } 
	   } 
	   } 
	 & & \\ 
 

    \theutterance \stepcounter{utterance}  

    & & & \multicolumn{2}{p{0.3\linewidth}}{\cellcolor[rgb]{0.95,0.95,0.95}{%
	\makecell[{{p{\linewidth}}}]{% 
	  \tt {\tiny [GM$|$GM]}  
	 end game 
	  } 
	   } 
	   } 
	 & & \\ 
 

\end{supertabular}
}

\end{document}
