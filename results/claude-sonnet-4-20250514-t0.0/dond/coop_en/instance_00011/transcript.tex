
\documentclass{article}
\usepackage{colortbl}
\usepackage{makecell}
\usepackage{multirow}
\usepackage{supertabular}

\begin{document}

\newcounter{utterance}

\twocolumn

{ \footnotesize  \setcounter{utterance}{1}
\setlength{\tabcolsep}{0pt}
\begin{supertabular}{c@{$\;$}|p{.15\linewidth}@{}p{.15\linewidth}p{.15\linewidth}p{.15\linewidth}p{.15\linewidth}p{.15\linewidth}}

    \# & $\;$A & \multicolumn{4}{c}{Game Master} & $\;\:$B\\
    \hline 

    \theutterance \stepcounter{utterance}  

    & & \multicolumn{4}{p{0.6\linewidth}}{\cellcolor[rgb]{0.9,0.9,0.9}{%
	\makecell[{{p{\linewidth}}}]{% 
	  \tt {\tiny [A$\langle$GM]}  
	 You are playing a cooperative negotiation game in which you have to agree on how to divide a set of items among you and another player.\\ \tt \\ \tt Rules:\\ \tt (a) You and the other player are given a set of items. Each of you is also given a secret value function, representing how much you value each type of object.\\ \tt (b) You exchange messages with the other player to agree on who gets which items. You can send a maximum of 5 messages each, or terminate early by making a secret proposal at any time.\\ \tt (c) You are each asked to submit a secret proposal indicating the items you want formatted in square brackets as follows: "[Proposal: <number> <object name>, <number> <object name>, <...>]"\\ \tt (d) If your proposals are complementary, i.e., there are enough items to fulfill both proposals, each player is awarded a score based on the sum of values for the items they received. Otherwise, both of you get zero points.\\ \tt (e) Your goal is to maximize the sum of your score and the score of the other player.\\ \tt \\ \tt Let us start.\\ \tt \\ \tt The set of available items is:\\ \tt \\ \tt 1 ring, 3 fruits, 1 chalk\\ \tt \\ \tt Your secret value function is:\\ \tt \\ \tt ring: 6, fruit: 1, chalk: 1\\ \tt \\ \tt IMPORTANT: Your messages, unless it is the secret proposal, are directly transmitted to the other player, so do not include any response to the rules or text announcing your message. To make a secret proposal, use the indicated format. Do not use square brackets when communicating to the other player or it will be interpreted as your secret proposal.\\ \tt  
	  } 
	   } 
	   } 
	 & \\ 
 

    \theutterance \stepcounter{utterance}  

    & \multicolumn{4}{p{0.6\linewidth}}{\cellcolor[rgb]{0.8,1,0.9}{%
	\makecell[{{p{\linewidth}}}]{% 
	  \tt {\tiny [A$\rangle$GM]}  
	 Hello! I'm excited to work together to maximize our combined score. To start our negotiation, I'd like to understand what items are most important to you. I have some preferences myself, but I think we can find a mutually beneficial division if we share information about what we value most. What items would you most like to have from this set? 
	  } 
	   } 
	   } 
	 & & \\ 
 

    \theutterance \stepcounter{utterance}  

    & & & \multicolumn{2}{p{0.3\linewidth}}{\cellcolor[rgb]{0.95,0.95,0.95}{%
	\makecell[{{p{\linewidth}}}]{% 
	  \tt {\tiny [GM$|$GM]}  
	 continue 
	  } 
	   } 
	   } 
	 & & \\ 
 

    \theutterance \stepcounter{utterance}  

    & & & \multicolumn{2}{p{0.3\linewidth}}{\cellcolor[rgb]{0.95,0.95,0.95}{%
	\makecell[{{p{\linewidth}}}]{% 
	  \tt {\tiny [GM$|$GM]}  
	 Hello! I'm excited to work together to maximize our combined score. To start our negotiation, I'd like to understand what items are most important to you. I have some preferences myself, but I think we can find a mutually beneficial division if we share information about what we value most. What items would you most like to have from this set? 
	  } 
	   } 
	   } 
	 & & \\ 
 

    \theutterance \stepcounter{utterance}  

    & & \multicolumn{4}{p{0.6\linewidth}}{\cellcolor[rgb]{0.9,0.9,0.9}{%
	\makecell[{{p{\linewidth}}}]{% 
	  \tt {\tiny [GM$\rangle$B]}  
	 You are playing a cooperative negotiation game in which you have to agree on how to divide a set of items among you and another player.\\ \tt \\ \tt Rules:\\ \tt (a) You and the other player are given a set of items. Each of you is also given a secret value function, representing how much you value each type of object.\\ \tt (b) You exchange messages with the other player to agree on who gets which items. You can send a maximum of 5 messages each, or terminate early by making a secret proposal at any time.\\ \tt (c) You are each asked to submit a secret proposal indicating the items you want formatted in square brackets as follows: "[Proposal: <number> <object name>, <number> <object name>, <...>]"\\ \tt (d) If your proposals are complementary, i.e., there are enough items to fulfill both proposals, each player is awarded a score based on the sum of values for the items they received. Otherwise, both of you get zero points.\\ \tt (e) Your goal is to maximize the sum of your score and the score of the other player.\\ \tt \\ \tt Let us start.\\ \tt \\ \tt The set of available items is:\\ \tt \\ \tt 1 ring, 3 fruits, 1 chalk\\ \tt \\ \tt Your secret value function is:\\ \tt \\ \tt ring: 6, fruit: 1, chalk: 1\\ \tt \\ \tt IMPORTANT: Your messages, unless it is the secret proposal, are directly transmitted to the other player, so do not include any response to the rules or text announcing your message. To make a secret proposal, use the indicated format. Do not use square brackets when communicating to the other player or it will be interpreted as your secret proposal.\\ \tt \\ \tt \\ \tt Hello! I'm excited to work together to maximize our combined score. To start our negotiation, I'd like to understand what items are most important to you. I have some preferences myself, but I think we can find a mutually beneficial division if we share information about what we value most. What items would you most like to have from this set? 
	  } 
	   } 
	   } 
	 & \\ 
 

    \theutterance \stepcounter{utterance}  

    & & & \multicolumn{4}{p{0.6\linewidth}}{\cellcolor[rgb]{1,0.85,0.72}{%
	\makecell[{{p{\linewidth}}}]{% 
	  \tt {\tiny [GM$\langle$B]}  
	 I'd be happy to share! The ring is quite valuable to me, so I'd really appreciate getting that if possible. The fruits and chalk are less important to me - I'd be willing to be flexible with those. \\ \tt \\ \tt What about you? Which items do you value most? If you're not as interested in the ring, maybe we can work out a trade where I get the ring and you get more of the other items. What would work best for you? 
	  } 
	   } 
	   } 
	  \\ 
 

    \theutterance \stepcounter{utterance}  

    & & & \multicolumn{2}{p{0.3\linewidth}}{\cellcolor[rgb]{0.95,0.95,0.95}{%
	\makecell[{{p{\linewidth}}}]{% 
	  \tt {\tiny [GM$|$GM]}  
	 continue 
	  } 
	   } 
	   } 
	 & & \\ 
 

    \theutterance \stepcounter{utterance}  

    & & & \multicolumn{2}{p{0.3\linewidth}}{\cellcolor[rgb]{0.95,0.95,0.95}{%
	\makecell[{{p{\linewidth}}}]{% 
	  \tt {\tiny [GM$|$GM]}  
	 I'd be happy to share! The ring is quite valuable to me, so I'd really appreciate getting that if possible. The fruits and chalk are less important to me - I'd be willing to be flexible with those. \\ \tt \\ \tt What about you? Which items do you value most? If you're not as interested in the ring, maybe we can work out a trade where I get the ring and you get more of the other items. What would work best for you? 
	  } 
	   } 
	   } 
	 & & \\ 
 

    \theutterance \stepcounter{utterance}  

    & & \multicolumn{4}{p{0.6\linewidth}}{\cellcolor[rgb]{0.9,0.9,0.9}{%
	\makecell[{{p{\linewidth}}}]{% 
	  \tt {\tiny [A$\langle$GM]}  
	 I'd be happy to share! The ring is quite valuable to me, so I'd really appreciate getting that if possible. The fruits and chalk are less important to me - I'd be willing to be flexible with those. \\ \tt \\ \tt What about you? Which items do you value most? If you're not as interested in the ring, maybe we can work out a trade where I get the ring and you get more of the other items. What would work best for you? 
	  } 
	   } 
	   } 
	 & \\ 
 

    \theutterance \stepcounter{utterance}  

    & \multicolumn{4}{p{0.6\linewidth}}{\cellcolor[rgb]{0.8,1,0.9}{%
	\makecell[{{p{\linewidth}}}]{% 
	  \tt {\tiny [A$\rangle$GM]}  
	 I appreciate your openness! Actually, the ring is also quite valuable to me, so we both seem to have similar preferences there. However, I'm definitely interested in finding the solution that maximizes our combined benefit.\\ \tt \\ \tt Since we both value the ring highly, let me ask - would you be satisfied with getting the ring if I took the 3 fruits and the chalk? That way you'd get your most important item, and I'd get the remaining items. \\ \tt \\ \tt Or do you have a sense of roughly how much more valuable the ring is to you compared to the fruits and chalk? If your valuation of the ring is significantly higher than mine, it might make sense for you to have it. What do you think would give us the best total outcome? 
	  } 
	   } 
	   } 
	 & & \\ 
 

    \theutterance \stepcounter{utterance}  

    & & & \multicolumn{2}{p{0.3\linewidth}}{\cellcolor[rgb]{0.95,0.95,0.95}{%
	\makecell[{{p{\linewidth}}}]{% 
	  \tt {\tiny [GM$|$GM]}  
	 continue 
	  } 
	   } 
	   } 
	 & & \\ 
 

    \theutterance \stepcounter{utterance}  

    & & & \multicolumn{2}{p{0.3\linewidth}}{\cellcolor[rgb]{0.95,0.95,0.95}{%
	\makecell[{{p{\linewidth}}}]{% 
	  \tt {\tiny [GM$|$GM]}  
	 I appreciate your openness! Actually, the ring is also quite valuable to me, so we both seem to have similar preferences there. However, I'm definitely interested in finding the solution that maximizes our combined benefit.\\ \tt \\ \tt Since we both value the ring highly, let me ask - would you be satisfied with getting the ring if I took the 3 fruits and the chalk? That way you'd get your most important item, and I'd get the remaining items. \\ \tt \\ \tt Or do you have a sense of roughly how much more valuable the ring is to you compared to the fruits and chalk? If your valuation of the ring is significantly higher than mine, it might make sense for you to have it. What do you think would give us the best total outcome? 
	  } 
	   } 
	   } 
	 & & \\ 
 

    \theutterance \stepcounter{utterance}  

    & & \multicolumn{4}{p{0.6\linewidth}}{\cellcolor[rgb]{0.9,0.9,0.9}{%
	\makecell[{{p{\linewidth}}}]{% 
	  \tt {\tiny [GM$\rangle$B]}  
	 I appreciate your openness! Actually, the ring is also quite valuable to me, so we both seem to have similar preferences there. However, I'm definitely interested in finding the solution that maximizes our combined benefit.\\ \tt \\ \tt Since we both value the ring highly, let me ask - would you be satisfied with getting the ring if I took the 3 fruits and the chalk? That way you'd get your most important item, and I'd get the remaining items. \\ \tt \\ \tt Or do you have a sense of roughly how much more valuable the ring is to you compared to the fruits and chalk? If your valuation of the ring is significantly higher than mine, it might make sense for you to have it. What do you think would give us the best total outcome? 
	  } 
	   } 
	   } 
	 & \\ 
 

    \theutterance \stepcounter{utterance}  

    & & & \multicolumn{4}{p{0.6\linewidth}}{\cellcolor[rgb]{1,0.85,0.72}{%
	\makecell[{{p{\linewidth}}}]{% 
	  \tt {\tiny [GM$\langle$B]}  
	 That's a great way to think about it! Let me be specific about my valuations to help us find the best solution.\\ \tt \\ \tt For me, the ring is worth significantly more than the other items combined - about 1.5 times as much as getting all 3 fruits plus the chalk together. So while I'd prefer the ring, if you value the ring much more highly relative to the other items than I do, then you should definitely take it for our best combined outcome.\\ \tt \\ \tt How do your valuations compare? Is the ring worth much more to you than the 3 fruits and chalk combined, or are they closer in value? If the ring is worth way more to you proportionally, then you should get it. But if our relative preferences are similar, then maybe I should take the ring since I expressed interest first, and you'd get good value from the other items.\\ \tt \\ \tt What do you think would maximize our total points? 
	  } 
	   } 
	   } 
	  \\ 
 

    \theutterance \stepcounter{utterance}  

    & & & \multicolumn{2}{p{0.3\linewidth}}{\cellcolor[rgb]{0.95,0.95,0.95}{%
	\makecell[{{p{\linewidth}}}]{% 
	  \tt {\tiny [GM$|$GM]}  
	 continue 
	  } 
	   } 
	   } 
	 & & \\ 
 

    \theutterance \stepcounter{utterance}  

    & & & \multicolumn{2}{p{0.3\linewidth}}{\cellcolor[rgb]{0.95,0.95,0.95}{%
	\makecell[{{p{\linewidth}}}]{% 
	  \tt {\tiny [GM$|$GM]}  
	 That's a great way to think about it! Let me be specific about my valuations to help us find the best solution.\\ \tt \\ \tt For me, the ring is worth significantly more than the other items combined - about 1.5 times as much as getting all 3 fruits plus the chalk together. So while I'd prefer the ring, if you value the ring much more highly relative to the other items than I do, then you should definitely take it for our best combined outcome.\\ \tt \\ \tt How do your valuations compare? Is the ring worth much more to you than the 3 fruits and chalk combined, or are they closer in value? If the ring is worth way more to you proportionally, then you should get it. But if our relative preferences are similar, then maybe I should take the ring since I expressed interest first, and you'd get good value from the other items.\\ \tt \\ \tt What do you think would maximize our total points? 
	  } 
	   } 
	   } 
	 & & \\ 
 

    \theutterance \stepcounter{utterance}  

    & & \multicolumn{4}{p{0.6\linewidth}}{\cellcolor[rgb]{0.9,0.9,0.9}{%
	\makecell[{{p{\linewidth}}}]{% 
	  \tt {\tiny [A$\langle$GM]}  
	 That's a great way to think about it! Let me be specific about my valuations to help us find the best solution.\\ \tt \\ \tt For me, the ring is worth significantly more than the other items combined - about 1.5 times as much as getting all 3 fruits plus the chalk together. So while I'd prefer the ring, if you value the ring much more highly relative to the other items than I do, then you should definitely take it for our best combined outcome.\\ \tt \\ \tt How do your valuations compare? Is the ring worth much more to you than the 3 fruits and chalk combined, or are they closer in value? If the ring is worth way more to you proportionally, then you should get it. But if our relative preferences are similar, then maybe I should take the ring since I expressed interest first, and you'd get good value from the other items.\\ \tt \\ \tt What do you think would maximize our total points? 
	  } 
	   } 
	   } 
	 & \\ 
 

    \theutterance \stepcounter{utterance}  

    & \multicolumn{4}{p{0.6\linewidth}}{\cellcolor[rgb]{0.8,1,0.9}{%
	\makecell[{{p{\linewidth}}}]{% 
	  \tt {\tiny [A$\rangle$GM]}  
	 That's really helpful information! It sounds like we have very similar relative preferences - the ring is worth about 1.5 times the other items combined for me as well. \\ \tt \\ \tt Given that our ratios seem quite similar, I think either division would give us roughly the same total value. Since we both expressed interest in the ring and have comparable relative valuations, let me suggest this: I'll take the ring, and you get the 3 fruits and chalk. This seems fair given our similar preference structures, and should give us a good combined score.\\ \tt \\ \tt Does this arrangement work for you? If you're comfortable with this split, I'm ready to make it official.\\ \tt \\ \tt [Proposal: 1 ring] 
	  } 
	   } 
	   } 
	 & & \\ 
 

    \theutterance \stepcounter{utterance}  

    & & & \multicolumn{2}{p{0.3\linewidth}}{\cellcolor[rgb]{0.95,0.95,0.95}{%
	\makecell[{{p{\linewidth}}}]{% 
	  \tt {\tiny [GM$|$GM]}  
	 proposal 
	  } 
	   } 
	   } 
	 & & \\ 
 

    \theutterance \stepcounter{utterance}  

    & & & \multicolumn{2}{p{0.3\linewidth}}{\cellcolor[rgb]{0.95,0.95,0.95}{%
	\makecell[{{p{\linewidth}}}]{% 
	  \tt {\tiny [GM$|$GM]}  
	 [1, 0, 0] 
	  } 
	   } 
	   } 
	 & & \\ 
 

    \theutterance \stepcounter{utterance}  

    & & \multicolumn{4}{p{0.6\linewidth}}{\cellcolor[rgb]{0.9,0.9,0.9}{%
	\makecell[{{p{\linewidth}}}]{% 
	  \tt {\tiny [GM$\rangle$B]}  
	 The other player has submitted a secret proposal. It is now your turn to submit a proposal of your own.\\ \tt  
	  } 
	   } 
	   } 
	 & \\ 
 

    \theutterance \stepcounter{utterance}  

    & & & \multicolumn{4}{p{0.6\linewidth}}{\cellcolor[rgb]{1,0.85,0.72}{%
	\makecell[{{p{\linewidth}}}]{% 
	  \tt {\tiny [GM$\langle$B]}  
	 [Proposal: 1 ring] 
	  } 
	   } 
	   } 
	  \\ 
 

    \theutterance \stepcounter{utterance}  

    & & & \multicolumn{2}{p{0.3\linewidth}}{\cellcolor[rgb]{0.95,0.95,0.95}{%
	\makecell[{{p{\linewidth}}}]{% 
	  \tt {\tiny [GM$|$GM]}  
	 proposal 
	  } 
	   } 
	   } 
	 & & \\ 
 

    \theutterance \stepcounter{utterance}  

    & & & \multicolumn{2}{p{0.3\linewidth}}{\cellcolor[rgb]{0.95,0.95,0.95}{%
	\makecell[{{p{\linewidth}}}]{% 
	  \tt {\tiny [GM$|$GM]}  
	 [1, 0, 0] 
	  } 
	   } 
	   } 
	 & & \\ 
 

    \theutterance \stepcounter{utterance}  

    & & & \multicolumn{2}{p{0.3\linewidth}}{\cellcolor[rgb]{0.95,0.95,0.95}{%
	\makecell[{{p{\linewidth}}}]{% 
	  \tt {\tiny [GM$|$GM]}  
	 [[1, 0, 0], [1, 0, 0]] 
	  } 
	   } 
	   } 
	 & & \\ 
 

\end{supertabular}
}

\end{document}
