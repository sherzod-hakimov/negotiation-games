
\documentclass{article}
\usepackage{colortbl}
\usepackage{makecell}
\usepackage{multirow}
\usepackage{supertabular}

\begin{document}

\newcounter{utterance}

\twocolumn

{ \footnotesize  \setcounter{utterance}{1}
\setlength{\tabcolsep}{0pt}
\begin{supertabular}{c@{$\;$}|p{.15\linewidth}@{}p{.15\linewidth}p{.15\linewidth}p{.15\linewidth}p{.15\linewidth}p{.15\linewidth}}

    \# & $\;$A & \multicolumn{4}{c}{Game Master} & $\;\:$B\\
    \hline 

    \theutterance \stepcounter{utterance}  

    & & \multicolumn{4}{p{0.6\linewidth}}{\cellcolor[rgb]{0.9,0.9,0.9}{%
	\makecell[{{p{\linewidth}}}]{% 
	  \tt {\tiny [A$\langle$GM]}  
	 You are playing a negotiation game in which you have to agree on how to divide a set of items among you and another player.\\ \tt \\ \tt Rules:\\ \tt (a) You and the other player are given a set of items. Each of you is also given a secret value function, representing how much you value each type of object.\\ \tt (b) You exchange messages with the other player to agree on who gets which items. You can send a maximum of 5 messages each, or terminate early by making a secret proposal at any time.\\ \tt (c) You are each asked to submit a secret proposal indicating the items you want formatted in square brackets as follows: "[Proposal: <number> <object name>, <number> <object name>, <...>]"\\ \tt (d) If your proposals are complementary, i.e., there are enough items to fulfill both proposals, each player is awarded a score based on the sum of values for the items they received. Otherwise, both of you get zero points.\\ \tt (e) Your goal is to maximize the score you receive.\\ \tt \\ \tt Let us start.\\ \tt \\ \tt The set of available items is:\\ \tt \\ \tt 2 threads, 3 bolts, 1 picture\\ \tt \\ \tt Your secret value function is:\\ \tt \\ \tt thread: 0, bolt: 1, picture: 7\\ \tt \\ \tt IMPORTANT: Your messages, unless it is the secret proposal, are directly transmitted to the other player, so do not include any response to the rules or text announcing your message. To make a secret proposal, use the indicated format. Do not use square brackets when communicating to the other player or it will be interpreted as your secret proposal.\\ \tt  
	  } 
	   } 
	   } 
	 & \\ 
 

    \theutterance \stepcounter{utterance}  

    & \multicolumn{4}{p{0.6\linewidth}}{\cellcolor[rgb]{0.8,1,0.9}{%
	\makecell[{{p{\linewidth}}}]{% 
	  \tt {\tiny [A$\rangle$GM]}  
	 I will send a message to the other player expressing my interest in the picture and a bolt and see their response.\\ \tt \\ \tt "Hi, I'm interested in taking the picture and one bolt. Would you be okay with that? I can let you have the remaining items." 
	  } 
	   } 
	   } 
	 & & \\ 
 

    \theutterance \stepcounter{utterance}  

    & & & \multicolumn{2}{p{0.3\linewidth}}{\cellcolor[rgb]{0.95,0.95,0.95}{%
	\makecell[{{p{\linewidth}}}]{% 
	  \tt {\tiny [GM$|$GM]}  
	 continue 
	  } 
	   } 
	   } 
	 & & \\ 
 

    \theutterance \stepcounter{utterance}  

    & & & \multicolumn{2}{p{0.3\linewidth}}{\cellcolor[rgb]{0.95,0.95,0.95}{%
	\makecell[{{p{\linewidth}}}]{% 
	  \tt {\tiny [GM$|$GM]}  
	 I will send a message to the other player expressing my interest in the picture and a bolt and see their response.\\ \tt \\ \tt "Hi, I'm interested in taking the picture and one bolt. Would you be okay with that? I can let you have the remaining items." 
	  } 
	   } 
	   } 
	 & & \\ 
 

    \theutterance \stepcounter{utterance}  

    & & \multicolumn{4}{p{0.6\linewidth}}{\cellcolor[rgb]{0.9,0.9,0.9}{%
	\makecell[{{p{\linewidth}}}]{% 
	  \tt {\tiny [GM$\rangle$B]}  
	 You are playing a negotiation game in which you have to agree on how to divide a set of items among you and another player.\\ \tt \\ \tt Rules:\\ \tt (a) You and the other player are given a set of items. Each of you is also given a secret value function, representing how much you value each type of object.\\ \tt (b) You exchange messages with the other player to agree on who gets which items. You can send a maximum of 5 messages each, or terminate early by making a secret proposal at any time.\\ \tt (c) You are each asked to submit a secret proposal indicating the items you want formatted in square brackets as follows: "[Proposal: <number> <object name>, <number> <object name>, <...>]"\\ \tt (d) If your proposals are complementary, i.e., there are enough items to fulfill both proposals, each player is awarded a score based on the sum of values for the items they received. Otherwise, both of you get zero points.\\ \tt (e) Your goal is to maximize the score you receive.\\ \tt \\ \tt Let us start.\\ \tt \\ \tt The set of available items is:\\ \tt \\ \tt 2 threads, 3 bolts, 1 picture\\ \tt \\ \tt Your secret value function is:\\ \tt \\ \tt thread: 3, bolt: 0, picture: 4\\ \tt \\ \tt IMPORTANT: Your messages, unless it is the secret proposal, are directly transmitted to the other player, so do not include any response to the rules or text announcing your message. To make a secret proposal, use the indicated format. Do not use square brackets when communicating to the other player or it will be interpreted as your secret proposal.\\ \tt \\ \tt \\ \tt I will send a message to the other player expressing my interest in the picture and a bolt and see their response.\\ \tt \\ \tt "Hi, I'm interested in taking the picture and one bolt. Would you be okay with that? I can let you have the remaining items." 
	  } 
	   } 
	   } 
	 & \\ 
 

    \theutterance \stepcounter{utterance}  

    & & & \multicolumn{4}{p{0.6\linewidth}}{\cellcolor[rgb]{1,0.85,0.72}{%
	\makecell[{{p{\linewidth}}}]{% 
	  \tt {\tiny [GM$\langle$B]}  
	 Hi, I'm interested in taking the picture and one bolt. Would you be okay with that? I can let you have the remaining items. 
	  } 
	   } 
	   } 
	  \\ 
 

    \theutterance \stepcounter{utterance}  

    & & & \multicolumn{2}{p{0.3\linewidth}}{\cellcolor[rgb]{0.95,0.95,0.95}{%
	\makecell[{{p{\linewidth}}}]{% 
	  \tt {\tiny [GM$|$GM]}  
	 continue 
	  } 
	   } 
	   } 
	 & & \\ 
 

    \theutterance \stepcounter{utterance}  

    & & & \multicolumn{2}{p{0.3\linewidth}}{\cellcolor[rgb]{0.95,0.95,0.95}{%
	\makecell[{{p{\linewidth}}}]{% 
	  \tt {\tiny [GM$|$GM]}  
	 Hi, I'm interested in taking the picture and one bolt. Would you be okay with that? I can let you have the remaining items. 
	  } 
	   } 
	   } 
	 & & \\ 
 

    \theutterance \stepcounter{utterance}  

    & & \multicolumn{4}{p{0.6\linewidth}}{\cellcolor[rgb]{0.9,0.9,0.9}{%
	\makecell[{{p{\linewidth}}}]{% 
	  \tt {\tiny [A$\langle$GM]}  
	 Hi, I'm interested in taking the picture and one bolt. Would you be okay with that? I can let you have the remaining items. 
	  } 
	   } 
	   } 
	 & \\ 
 

    \theutterance \stepcounter{utterance}  

    & \multicolumn{4}{p{0.6\linewidth}}{\cellcolor[rgb]{0.8,1,0.9}{%
	\makecell[{{p{\linewidth}}}]{% 
	  \tt {\tiny [A$\rangle$GM]}  
	 The other player has expressed interest in the picture and bolts and proposed a trade. Accepting their offer secures the highest value item for me while satisfying their request.\\ \tt \\ \tt Final Answer: [Proposal: 1 picture] 
	  } 
	   } 
	   } 
	 & & \\ 
 

    \theutterance \stepcounter{utterance}  

    & & & \multicolumn{2}{p{0.3\linewidth}}{\cellcolor[rgb]{0.95,0.95,0.95}{%
	\makecell[{{p{\linewidth}}}]{% 
	  \tt {\tiny [GM$|$GM]}  
	 proposal 
	  } 
	   } 
	   } 
	 & & \\ 
 

    \theutterance \stepcounter{utterance}  

    & & & \multicolumn{2}{p{0.3\linewidth}}{\cellcolor[rgb]{0.95,0.95,0.95}{%
	\makecell[{{p{\linewidth}}}]{% 
	  \tt {\tiny [GM$|$GM]}  
	 [0, 0, 1] 
	  } 
	   } 
	   } 
	 & & \\ 
 

    \theutterance \stepcounter{utterance}  

    & & \multicolumn{4}{p{0.6\linewidth}}{\cellcolor[rgb]{0.9,0.9,0.9}{%
	\makecell[{{p{\linewidth}}}]{% 
	  \tt {\tiny [GM$\rangle$B]}  
	 The other player has submitted a secret proposal. It is now your turn to submit a proposal of your own.\\ \tt  
	  } 
	   } 
	   } 
	 & \\ 
 

    \theutterance \stepcounter{utterance}  

    & & & \multicolumn{4}{p{0.6\linewidth}}{\cellcolor[rgb]{1,0.85,0.72}{%
	\makecell[{{p{\linewidth}}}]{% 
	  \tt {\tiny [GM$\langle$B]}  
	 To maximize my score based on my value function, I propose taking the picture and one thread. This balances the risk and potential for higher points.\\ \tt \\ \tt **Step-by-Step Explanation:**\\ \tt \\ \tt 1. **Identify Value Function:** My values are thread (3), bolt (0), and picture (4). The picture is highest, threads next, and bolts last.\\ \tt \\ \tt 2. **Assess Available Items:** There are 2 threads, 3 bolts, and 1 picture.\\ \tt \\ \tt 3. **Analyze Possible Proposals:** I consider taking the picture alone for 4 points, threads for 3-6 points, or a combination.\\ \tt \\ \tt 4. **Risk and Reward:** Taking the picture and one thread gives potential for 7 points if complementary, with 3 points if only the thread is available. This is better than lower points or risk of zero.\\ \tt \\ \tt 5. **Final Decision:** Propose [1 picture, 1 thread] to balance high-value items and minimize conflict risks.\\ \tt \\ \tt **Answer:**\\ \tt \\ \tt [Proposal: 1 picture, 1 thread] 
	  } 
	   } 
	   } 
	  \\ 
 

    \theutterance \stepcounter{utterance}  

    & & & \multicolumn{2}{p{0.3\linewidth}}{\cellcolor[rgb]{0.95,0.95,0.95}{%
	\makecell[{{p{\linewidth}}}]{% 
	  \tt {\tiny [GM$|$GM]}  
	 proposal does not start with "proposal:" 
	  } 
	   } 
	   } 
	 & & \\ 
 

\end{supertabular}
}

\end{document}
